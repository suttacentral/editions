\documentclass[12pt,openany]{book}%
\usepackage{lastpage}%
%
\usepackage[inner=1in, outer=1in, top=.7in, bottom=1in, papersize={6in,9in}, headheight=13pt]{geometry}
\usepackage{polyglossia}
\usepackage[12pt]{moresize}
\usepackage{soul}%
\usepackage{microtype}
\usepackage{tocbasic}
\usepackage{realscripts}
\usepackage{epigraph}%
\usepackage{setspace}%
\usepackage{sectsty}
\usepackage{fontspec}
\usepackage{marginnote}
\usepackage[bottom]{footmisc}
\usepackage{enumitem}
\usepackage{fancyhdr}
\usepackage{extramarks}
\usepackage{graphicx}
\usepackage{verse}
\usepackage{relsize}
\usepackage{etoolbox}
\usepackage[a-3u]{pdfx}

\hypersetup{
colorlinks=true,
urlcolor=black,
linkcolor=black,
citecolor=black
}

% use a small amount of tracking on small caps
\SetTracking[ spacing = {25*,166, } ]{ encoding = *, shape = sc }{ 25 }

% add a blank page
\newcommand{\blankpage}{
\newpage
\thispagestyle{empty}
\mbox{}
\newpage
}

% define languages
\setdefaultlanguage[]{english}
\setotherlanguage[script=Latin]{sanskrit}

%\usepackage{pagegrid}
%\pagegridsetup{top-left, step=.25in}

% define fonts
% use if arno sanskrit is unavailable
%\setmainfont{Gentium Plus}
%\newfontfamily\Semiboldsubheadfont[]{Gentium Plus}
%\newfontfamily\Semiboldnormalfont[]{Gentium Plus}
%\newfontfamily\Lightfont[]{Gentium Plus}
%\newfontfamily\Marginalfont[]{Gentium Plus}
%\newfontfamily\Allsmallcapsfont[RawFeature=+c2sc]{Gentium Plus}
%\newfontfamily\Noligaturefont[Renderer=Basic]{Gentium Plus}
%\newfontfamily\Noligaturecaptionfont[Renderer=Basic]{Gentium Plus}
%\newfontfamily\Fleuronfont[Ornament=1]{Gentium Plus}

% use if arno sanskrit is available. display is applied to \chapter and \part, subhead to \section and \subsection. When specifying semibold, the italic must be defined.
\setmainfont[Numbers=OldStyle]{Arno Pro}
\newfontfamily\Semibolddisplayfont[BoldItalicFont = Arno Pro Semibold Italic Display]{Arno Pro Semibold Display} %
\newfontfamily\Semiboldsubheadfont[BoldItalicFont = Arno Pro Semibold Italic Subhead]{Arno Pro Semibold Subhead}
\newfontfamily\Semiboldnormalfont[BoldItalicFont = Arno Pro Semibold Italic]{Arno Pro Semibold}
\newfontfamily\Marginalfont[RawFeature=+subs]{Arno Pro Regular}
\newfontfamily\Allsmallcapsfont[RawFeature=+c2sc]{Arno Pro}
\newfontfamily\Noligaturefont[Renderer=Basic]{Arno Pro}
\newfontfamily\Noligaturecaptionfont[Renderer=Basic]{Arno Pro Caption}

% chinese fonts
\newfontfamily\cjk{Noto Serif TC}
\newcommand*{\langlzh}[1]{\cjk{#1}\normalfont}%

% logo
\newfontfamily\Logofont{sclogo.ttf}
\newcommand*{\sclogo}[1]{\large\Logofont{#1}}

% use subscript numerals for margin notes
\renewcommand*{\marginfont}{\Marginalfont}

% ensure margin notes have consistent vertical alignment
\renewcommand*{\marginnotevadjust}{-.17em}

% use compact lists
\setitemize{noitemsep,leftmargin=1em}
\setenumerate{noitemsep,leftmargin=1em}
\setdescription{noitemsep, style=unboxed, leftmargin=0em}

% style ToC
\DeclareTOCStyleEntries[
  raggedentrytext,
  linefill=\hfill,
  pagenumberwidth=.5in,
  pagenumberformat=\normalfont,
  entryformat=\normalfont
]{tocline}{chapter,section}


  \setlength\topsep{0pt}%
  \setlength\parskip{0pt}%

% define new \centerpars command for use in ToC. This ensures centering, proper wrapping, and no page break after
\def\startcenter{%
  \par
  \begingroup
  \leftskip=0pt plus 1fil
  \rightskip=\leftskip
  \parindent=0pt
  \parfillskip=0pt
}
\def\stopcenter{%
  \par
  \endgroup
}
\long\def\centerpars#1{\startcenter#1\stopcenter}

% redefine part, so that it adds a toc entry without page number
\let\oldcontentsline\contentsline
\newcommand{\nopagecontentsline}[3]{\oldcontentsline{#1}{#2}{}}

    \makeatletter
\renewcommand*\l@part[2]{%
  \ifnum \c@tocdepth >-2\relax
    \addpenalty{-\@highpenalty}%
    \addvspace{0em \@plus\p@}%
    \setlength\@tempdima{3em}%
    \begingroup
      \parindent \z@ \rightskip \@pnumwidth
      \parfillskip -\@pnumwidth
      {\leavevmode
       \setstretch{.85}\large\scshape\centerpars{#1}\vspace*{-1em}\llap{#2}}\par
       \nobreak
         \global\@nobreaktrue
         \everypar{\global\@nobreakfalse\everypar{}}%
    \endgroup
  \fi}
\makeatother

\makeatletter
\def\@pnumwidth{2em}
\makeatother

% define new sectioning command, which is only used in volumes where the pannasa is found in some parts but not others, especially in an and sn

\newcommand*{\pannasa}[1]{\clearpage\thispagestyle{empty}\begin{center}\vspace*{14em}\setstretch{.85}\huge\itshape\scshape\MakeLowercase{#1}\end{center}}

    \makeatletter
\newcommand*\l@pannasa[2]{%
  \ifnum \c@tocdepth >-2\relax
    \addpenalty{-\@highpenalty}%
    \addvspace{.5em \@plus\p@}%
    \setlength\@tempdima{3em}%
    \begingroup
      \parindent \z@ \rightskip \@pnumwidth
      \parfillskip -\@pnumwidth
      {\leavevmode
       \setstretch{.85}\large\itshape\scshape\lowercase{\centerpars{#1}}\vspace*{-1em}\llap{#2}}\par
       \nobreak
         \global\@nobreaktrue
         \everypar{\global\@nobreakfalse\everypar{}}%
    \endgroup
  \fi}
\makeatother

% don't put page number on first page of toc (relies on etoolbox)
\patchcmd{\chapter}{plain}{empty}{}{}

% global line height
\setstretch{1.05}

% allow linebreak after em-dash
\catcode`\—=13
\protected\def—{\unskip\textemdash\allowbreak}

% style headings with secsty. chapter and section are defined per-edition
\partfont{\setstretch{.85}\normalfont\centering\textsc}
\subsectionfont{\setstretch{.85}\Semiboldsubheadfont}%
\subsubsectionfont{\setstretch{.85}\Semiboldnormalfont}

% style elements of suttatitle
\newcommand*{\suttatitleacronym}[1]{\smaller[2]{#1}\vspace*{.3em}}
\newcommand*{\suttatitletranslation}[1]{\linebreak{#1}}
\newcommand*{\suttatitleroot}[1]{\linebreak\smaller[2]\itshape{#1}}

\DeclareTOCStyleEntries[
  indent=3.3em,
  dynindent,
  beforeskip=.2em plus -2pt minus -1pt,
]{tocline}{section}

\DeclareTOCStyleEntries[
  indent=0em,
  dynindent,
  beforeskip=.4em plus -2pt minus -1pt,
]{tocline}{chapter}

\newcommand*{\tocacronym}[1]{\hspace*{-3.3em}{#1}\quad}
\newcommand*{\toctranslation}[1]{#1}
\newcommand*{\tocroot}[1]{(\textit{#1})}
\newcommand*{\tocchapterline}[1]{\bfseries\itshape{#1}}


% redefine paragraph and subparagraph headings to not be inline
\makeatletter
% Change the style of paragraph headings %
\renewcommand\paragraph{\@startsection{paragraph}{4}{\z@}%
            {-2.5ex\@plus -1ex \@minus -.25ex}%
            {1.25ex \@plus .25ex}%
            {\noindent\Semiboldnormalfont\normalsize}}

% Change the style of subparagraph headings %
\renewcommand\subparagraph{\@startsection{subparagraph}{5}{\z@}%
            {-2.5ex\@plus -1ex \@minus -.25ex}%
            {1.25ex \@plus .25ex}%
            {\noindent\Semiboldnormalfont\small}}
\makeatother

% use etoolbox to suppress page numbers on \part
\patchcmd{\part}{\thispagestyle{plain}}{\thispagestyle{empty}}
  {}{\errmessage{Cannot patch \string\part}}

% and to reduce margins on quotation
\patchcmd{\quotation}{\rightmargin}{\leftmargin 1.2em \rightmargin}{}{}
\AtBeginEnvironment{quotation}{\small}

% titlepage
\newcommand*{\titlepageTranslationTitle}[1]{{\begin{center}\begin{large}{#1}\end{large}\end{center}}}
\newcommand*{\titlepageCreatorName}[1]{{\begin{center}\begin{normalsize}{#1}\end{normalsize}\end{center}}}

% halftitlepage
\newcommand*{\halftitlepageTranslationTitle}[1]{\setstretch{2.5}{\begin{Huge}\uppercase{\so{#1}}\end{Huge}}}
\newcommand*{\halftitlepageTranslationSubtitle}[1]{\setstretch{1.2}{\begin{large}{#1}\end{large}}}
\newcommand*{\halftitlepageFleuron}[1]{{\begin{large}\Fleuronfont{{#1}}\end{large}}}
\newcommand*{\halftitlepageByline}[1]{{\begin{normalsize}\textit{{#1}}\end{normalsize}}}
\newcommand*{\halftitlepageCreatorName}[1]{{\begin{LARGE}{\textsc{#1}}\end{LARGE}}}
\newcommand*{\halftitlepageVolumeNumber}[1]{{\begin{normalsize}{\Allsmallcapsfont{\textsc{#1}}}\end{normalsize}}}
\newcommand*{\halftitlepageVolumeAcronym}[1]{{\begin{normalsize}{#1}\end{normalsize}}}
\newcommand*{\halftitlepageVolumeTranslationTitle}[1]{{\begin{Large}{\textsc{#1}}\end{Large}}}
\newcommand*{\halftitlepageVolumeRootTitle}[1]{{\begin{normalsize}{\Allsmallcapsfont{\textsc{\itshape #1}}}\end{normalsize}}}
\newcommand*{\halftitlepagePublisher}[1]{{\begin{large}{\Noligaturecaptionfont\textsc{#1}}\end{large}}}

% epigraph
\renewcommand{\epigraphflush}{center}
\renewcommand*{\epigraphwidth}{.85\textwidth}
\newcommand*{\epigraphTranslatedTitle}[1]{\vspace*{.5em}\footnotesize\textsc{#1}\\}%
\newcommand*{\epigraphRootTitle}[1]{\footnotesize\textit{#1}\\}%
\newcommand*{\epigraphReference}[1]{\footnotesize{#1}}%

% custom commands for html styling classes
\newcommand*{\scnamo}[1]{\begin{center}\textit{#1}\end{center}}
\newcommand*{\scendsection}[1]{\begin{center}\textit{#1}\end{center}}
\newcommand*{\scendsutta}[1]{\begin{center}\textit{#1}\end{center}}
\newcommand*{\scendbook}[1]{\begin{center}\uppercase{#1}\end{center}}
\newcommand*{\scendkanda}[1]{\begin{center}\textbf{#1}\end{center}}
\newcommand*{\scend}[1]{\begin{center}\textit{#1}\end{center}}
\newcommand*{\scuddanaintro}[1]{\textit{#1}}
\newcommand*{\scendvagga}[1]{\begin{center}\textbf{#1}\end{center}}
\newcommand*{\scrule}[1]{\textbf{#1}}
\newcommand*{\scadd}[1]{\textit{#1}}
\newcommand*{\scevam}[1]{\textsc{#1}}
\newcommand*{\scspeaker}[1]{\hspace{2em}\textit{#1}}
\newcommand*{\scbyline}[1]{\begin{flushright}\textit{#1}\end{flushright}\bigskip}

% custom command for thematic break = hr
\newcommand*{\thematicbreak}{\begin{center}\rule[.5ex]{6em}{.4pt}\begin{normalsize}\quad\Fleuronfont{•}\quad\end{normalsize}\rule[.5ex]{6em}{.4pt}\end{center}}

% manage and style page header and footer. "fancy" has header and footer, "plain" has footer only

\pagestyle{fancy}
\fancyhf{}
\fancyfoot[RE,LO]{\thepage}
\fancyfoot[LE,RO]{\footnotesize\lastleftxmark}
\fancyhead[CE]{\setstretch{.85}\Noligaturefont\MakeLowercase{\textsc{\firstrightmark}}}
\fancyhead[CO]{\setstretch{.85}\Noligaturefont\MakeLowercase{\textsc{\firstleftmark}}}
\renewcommand{\headrulewidth}{0pt}
\fancypagestyle{plain}{ %
\fancyhf{} % remove everything
\fancyfoot[RE,LO]{\thepage}
\fancyfoot[LE,RO]{\footnotesize\lastleftxmark}
\renewcommand{\headrulewidth}{0pt}
\renewcommand{\footrulewidth}{0pt}}

% style footnotes
\setlength{\skip\footins}{1em}

\makeatletter
\newcommand{\@makefntextcustom}[1]{%
    \parindent 0em%
    \thefootnote.\enskip #1%
}
\renewcommand{\@makefntext}[1]{\@makefntextcustom{#1}}
\makeatother

% hang quotes (requires microtype)
\microtypesetup{
  protrusion = true,
  expansion  = true,
  tracking   = true,
  factor     = 1000,
  patch      = all,
  final
}

% Custom protrusion rules to allow hanging punctuation
\SetProtrusion
{ encoding = *}
{
% char   right left
  {-} = {    , 500 },
  % Double Quotes
  \textquotedblleft
      = {1000,     },
  \textquotedblright
      = {    , 1000},
  \quotedblbase
      = {1000,     },
  % Single Quotes
  \textquoteleft
      = {1000,     },
  \textquoteright
      = {    , 1000},
  \quotesinglbase
      = {1000,     }
}

% make latex use actual font em for parindent, not Computer Modern Roman
\AtBeginDocument{\setlength{\parindent}{1em}}%
%

% Default values; a bit sloppier than normal
\tolerance 1414
\hbadness 1414
\emergencystretch 1.5em
\hfuzz 0.3pt
\clubpenalty = 10000
\widowpenalty = 10000
\displaywidowpenalty = 10000
\hfuzz \vfuzz
 \raggedbottom%

\title{Middle Discourses}
\author{Bhikkhu Sujato}
\date{}%
% define a different fleuron for each edition
\newfontfamily\Fleuronfont[Ornament=4]{Arno Pro}

% Define heading styles per edition for chapter, section, and subsection. Suttatitle can be any one of these, depending on the volume. 

\let\oldfrontmatter\frontmatter
\renewcommand{\frontmatter}{%
\chapterfont{\setstretch{.85}\normalfont\centering}%
\sectionfont{\setstretch{.85}\Semiboldsubheadfont}%
\oldfrontmatter}

\let\oldmainmatter\mainmatter
\renewcommand{\mainmatter}{%
\chapterfont{\thispagestyle{empty}\vspace*{4em}\setstretch{.85}\LARGE\normalfont\itshape\scshape\centering\MakeLowercase}
\sectionfont{\clearpage\thispagestyle{plain}\vspace*{2em}\setstretch{.85}\normalfont\centering}%
\oldmainmatter}

\let\oldbackmatter\backmatter
\renewcommand{\backmatter}{%
\chapterfont{\setstretch{.85}\normalfont\centering}%
\sectionfont{\setstretch{.85}\Semiboldsubheadfont}%
\oldbackmatter}
%
%
\begin{document}%
\normalsize%
\frontmatter%
\setlength{\parindent}{0cm}

\pagestyle{empty}

\maketitle

\blankpage%
\begin{center}

\vspace*{2.2em}

\halftitlepageTranslationTitle{Middle Discourses}

\vspace*{1em}

\halftitlepageTranslationSubtitle{A lucid translation of the Majjhima Nikāya}

\vspace*{2em}

\halftitlepageFleuron{•}

\vspace*{2em}

\halftitlepageByline{translated and introduced by}

\vspace*{.5em}

\halftitlepageCreatorName{Bhikkhu Sujato}

\vspace*{4em}

\halftitlepageVolumeNumber{Volume 3}

\smallskip

\halftitlepageVolumeAcronym{MN 101–152}

\smallskip

\halftitlepageVolumeTranslationTitle{The Final Fifty}

\smallskip

\halftitlepageVolumeRootTitle{Uparipaṇṇāsa}

\vspace*{\fill}

\sclogo{0}
 \halftitlepagePublisher{SuttaCentral}

\end{center}

\newpage
%
\setstretch{1.05}

\begin{footnotesize}

\textit{Middle Discourses} is a translation of the Majjhimanikāya by Bhikkhu Sujato.

\medskip

Creative Commons Zero (CC0)

To the extent possible under law, Bhikkhu Sujato has waived all copyright and related or neighboring rights to \textit{Middle Discourses}.

\medskip

This work is published from Australia.

\begin{center}
\textit{This translation is an expression of an ancient spiritual text that has been passed down by the Buddhist tradition for the benefit of all sentient beings. It is dedicated to the public domain via Creative Commons Zero (CC0). You are encouraged to copy, reproduce, adapt, alter, or otherwise make use of this translation. The translator respectfully requests that any use be in accordance with the values and principles of the Buddhist community.}
\end{center}

\medskip

\begin{description}
    \item[Web publication date] 2018
    \item[This edition] 2022-11-22 08:17:58
    \item[Publication type] paperback
    \item[Edition] ed5
    \item[Number of volumes] 3
    \item[Publication ISBN] 978-1-76132-065-1
    \item[Publication URL] https://suttacentral.net/editions/mn/en/sujato
    \item[Source URL] https://github.com/suttacentral/bilara-data/tree/published/translation/en/sujato/sutta/mn
    \item[Publication number] scpub3
\end{description}

\medskip

Published by SuttaCentral

\medskip

\textit{SuttaCentral,\\
c/o Alwis \& Alwis Pty Ltd\\
Kaurna Country,\\
Suite 12,\\
198 Greenhill Road,\\
Eastwood,\\
SA 5063,\\
Australia}

\end{footnotesize}

\newpage

\setlength{\parindent}{1.5em}%%
\tableofcontents
\newpage
\pagestyle{fancy}
%
\chapter*{Summary of Contents}
\addcontentsline{toc}{chapter}{Summary of Contents}
\markboth{Summary of Contents}{Summary of Contents}

\begin{description}%
\item[\href{\#mn{-}devadahavagga}{None}] Diverse teachings.%
\item[\href{\#mn101}{None}] The Buddha tackles a group of Jain ascetics, pressing them on their claim to be practicing to end all suffering by self-mortification. He points out a series of fallacies in their logic, and explains his own middle way.%
\item[\href{\#mn102}{None}] A middle length version of the more famous Brahmajala Sutta (DN1), this surveys a range of speculative views and dismisses them all.%
\item[\href{\#mn103}{None}] The Buddha teaches the monks to not dispute about the fundamental teachings, but to always strive for harmony.%
\item[\href{\#mn104}{None}] Hearing of the death of the Jain leader \textsanskrit{Nigaṇṭha} \textsanskrit{Nātaputta}, the Buddha encourages the \textsanskrit{Saṅgha} to swiftly resolve any disputes. He lays down a series of seven methods for resolving disputes. These form the foundation for the monastic code.%
\item[\href{\#mn105}{None}] Not all of those who claim to be awakened are genuine. The Buddha teaches how true spiritual progress depends on an irreversible letting go of the forces that lead to suffering.%
\item[\href{\#mn106}{None}] Beginning with profound meditation absorption, the Buddha goes on to deeper and deeper levels, showing how insight on this basis leads to the detaching of consciousness from any form of rebirth.%
\item[\href{\#mn107}{None}] The Buddha compares the training of an accountant with the step by step spiritual path of his followers. But even with such a well explained path, the Buddha can only show the way, and it is up to us to walk it.%
\item[\href{\#mn108}{None}] Amid rising military tensions after the Buddha’s death, Venerable Ānanda is questioned about how the \textsanskrit{Saṅgha} planned to continue in their teacher’s absence. As the Buddha refused to appoint a successor, the teaching and practice that he laid down become the teacher, and the \textsanskrit{Saṅgha} resolves issues by consensus.%
\item[\href{\#mn109}{None}] On a lovely full moon night, one of the mendicants presents the Buddha with a series of questions that go to the heart of the teaching. But when he hears of the doctrine of not-self, another mendicant is unable to grasp the meaning.%
\item[\href{\#mn110}{None}] A good person is able to understand a bad person, but not vice versa.%
\item[\href{\#mn{-}anupadavagga}{None}] Many of the discourses in this chapter delve into complex and analytical presentations of core teachings. It includes important discourses on meditation.%
\item[\href{\#mn111}{None}] The Buddha describes in technical detail the process of insight of Venerable \textsanskrit{Sāriputta}. Many ideas and terms in this text anticipate the Abhidhamma.%
\item[\href{\#mn112}{None}] If someone claims to be awakened, their claim should be interrogated with a detailed series of detailed questions. Only if they can answer them clearly should the claim be accepted.%
\item[\href{\#mn113}{None}] The Buddha explains that a truly good person does not disparage others or feel superior because of their attainment.%
\item[\href{\#mn114}{None}] The Buddha sets up a framework on things to be cultivated or avoided, and Venerable \textsanskrit{Sāriputta} volunteers to elaborate.%
\item[\href{\#mn115}{None}] Beginning by praising a wise person, the Buddha goes on to explain that one becomes wise by inquiring into the elements, sense fields, dependent origination, and what is possible and impossible.%
\item[\href{\#mn116}{None}] Reflecting on the changes that even geographical features undergo, the Buddha then recounts the names of sages of the past who have lived in Mount Isigili near \textsanskrit{Rājagaha}.%
\item[\href{\#mn117}{None}] A discourse on the prerequisites of right \textsanskrit{samādhi} that emphasizes the interrelationship and mutual support of all the factors of the eightfold path.%
\item[\href{\#mn118}{None}] Surrounded by many well-practiced mendicants, the Buddha teaches mindfulness of breathing in detail, showing how they relate to the four kinds of mindfulness meditation.%
\item[\href{\#mn119}{None}] This focuses on the first aspect of mindfulness meditation, the observation of the body. This set of practices, simple as they seem, have far-reaching benefits.%
\item[\href{\#mn120}{None}] The Buddha explains how one can make a wish to be reborn in different realms.%
\item[\href{\#mn{-}sunnatavagga}{None}] Named after the first discourses, which deal with emptiness, this chapter presents less analytical and more narrative texts.%
\item[\href{\#mn121}{None}] The Buddha describes his own practice of the meditation on emptiness.%
\item[\href{\#mn122}{None}] A group of mendicants have taken to socializing too much, so the Buddha teaches on the importance of seclusion in order to enter fully into emptiness.%
\item[\href{\#mn123}{None}] Venerable Ānanda is invited by the Buddha to speak on the Buddha’s amazing qualities, and proceeds to list a series of apparently miraculous events accompanying his birth. The Buddha caps it off by explaining what he thinks is really amazing about himself.%
\item[\href{\#mn124}{None}] Venerable Bakkula, regarded as the healthiest of the mendicants, explains to an old friend his strict and austere practice. The unusual form of this discourse suggests it was added to the canon some time after the Buddha’s death.%
\item[\href{\#mn125}{None}] A young monk is unable to persuade a prince of the blessings of peace of mind. The Buddha offers similes based on training an elephant that would have been successful, as this was a field the prince was familiar with.%
\item[\href{\#mn126}{None}] Success in the spiritual life does not depend on any vows you may or may not make, but on whether you practice well.%
\item[\href{\#mn127}{None}] A lay person becomes confused when encouraged to develop the “limitless” and “expansive” liberations, and asks Venerable Anuruddha to explain whether they are the same or different.%
\item[\href{\#mn128}{None}] A second discourse set at the quarrel of Kosambi, this depicts the Buddha, having failed to achieve reconciliation between the disputing mendicants, leaving the monastery. He spends time in the wilderness before encountering an inspiring community of practicing monks. There he discusses in detail obstacles to meditation that he encountered before awakening.%
\item[\href{\#mn129}{None}] A fool suffers both in this life and the next, while the astute benefits in both respects.%
\item[\href{\#mn130}{None}] Expanding on the previous, this discourse contains the most detailed descriptions of the horrors of hell.%
\item[\href{\#mn{-}vibhangavagga}{None}] A series of discourses presented as technical analyses of shorter teachings.%
\item[\href{\#mn131}{None}] This discourse opens with a short but powerful set of verses extolling the benefits of insight into the here and now, followed by an explanation.%
\item[\href{\#mn132}{None}] The same discourse as MN 131, but spoken by Venerable Ānanda.%
\item[\href{\#mn133}{None}] The verses from MN 131 are explained in a different way by Venerable \textsanskrit{Mahakaccāna}.%
\item[\href{\#mn134}{None}] A monk who does not know the verses from MN 131 is encouraged by a deity to learn them.%
\item[\href{\#mn135}{None}] The Buddha explains to a brahmin how your deeds in past lives affect you in this life.%
\item[\href{\#mn136}{None}] Confronted with an overly simplistic version of his own teachings, the Buddha emphasizes the often overlooked nuances and qualifications in how karma plays out.%
\item[\href{\#mn137}{None}] A detailed analysis of the six senses and the relation to emotional and cognitive processes.%
\item[\href{\#mn138}{None}] The Buddha gives a brief and enigmatic statement on the ways consciousness may become attached. Venerable \textsanskrit{Mahākaccāna} is invited by the mendicants to draw out the implications.%
\item[\href{\#mn139}{None}] Achieving peace is no simple matter. The Buddha explains how to avoid conflict through contentment, right speech, understanding pleasure, and not insisting on local conventions.%
\item[\href{\#mn140}{None}] While staying overnight in a potter’s workshop, the Buddha has a chance encounter with a monk who does not recognize him. They have a long and profound discussion based on the four elements. This is one of the most insightful and moving discourses in the canon.%
\item[\href{\#mn141}{None}] Expanding on the Buddha’s first sermon, Venerable \textsanskrit{Sāriputta} gives a detailed explanation of the four noble truths.%
\item[\href{\#mn142}{None}] When his step-mother \textsanskrit{Mahāpajāpatī} wishes to offer him a robe for his personal use, the Buddha encourages her to offer it to the entire \textsanskrit{Saṅgha} instead. He goes on to explain that the best kind of offering to the \textsanskrit{Saṅgha} is one given to the dual community of monks and nuns, headed by the Buddha.%
\item[\href{\#mn{-}salayatanavagga}{None}] Most discourses in this chapter deal with the six sense fields.%
\item[\href{\#mn143}{None}] As the great lay disciple \textsanskrit{Anāthapiṇḍika} lies dying, Venerable \textsanskrit{Sāriputta} visits him and gives a powerful teaching on non-attachment.%
\item[\href{\#mn144}{None}] The monk Channa is suffering a painful terminal illness and wishes to take his own life.%
\item[\href{\#mn145}{None}] On the eve of his departure to a distant country, full of wild and unpredictable people, Venerable \textsanskrit{Puṇṇa} is asked by the Buddha how he would respond if attacked there.%
\item[\href{\#mn146}{None}] When asked to teach the nuns, Venerable Nandaka proceeds by inviting them to engage with his discourse and ask if there is anything that needs further explanation.%
\item[\href{\#mn147}{None}] The Buddha takes \textsanskrit{Rāhula} with him to a secluded spot in order to lead him on to liberation.%
\item[\href{\#mn148}{None}] The Buddha analyzes the six senses from six perspectives, and demonstrates the emptiness of all of them.%
\item[\href{\#mn149}{None}] Explains how insight into the six senses is integrated with the eightfold path and leads to liberation.%
\item[\href{\#mn150}{None}] In discussion with a group of householders, the Buddha helps them to distinguish those spiritual practitioners who are truly worthy of respect.%
\item[\href{\#mn151}{None}] The Buddha notices Venerable \textsanskrit{Sāriputta}’s glowing complexion, which is the result of his deep meditation. He then presents a series of reflections by which a mendicant can be sure that they are worthy of their alms-food.%
\item[\href{\#mn152}{None}] A brahmin teacher advocates that purification of the senses consists in simply avoiding seeing and hearing things. The Buddha explains that it is not about avoiding sense experience, but understanding it and learning to not be affected by sense experience.%
\end{description}

%
\mainmatter%
\pagestyle{fancy}%
\addtocontents{toc}{\let\protect\contentsline\protect\nopagecontentsline}
\part*{The Final Fifty }
\addcontentsline{toc}{part}{The Final Fifty }
\markboth{}{}
\addtocontents{toc}{\let\protect\contentsline\protect\oldcontentsline}

%
\addtocontents{toc}{\let\protect\contentsline\protect\nopagecontentsline}
\chapter*{The Chapter Beginning With Devadaha }
\addcontentsline{toc}{chapter}{\tocchapterline{The Chapter Beginning With Devadaha }}
\addtocontents{toc}{\let\protect\contentsline\protect\oldcontentsline}

%
\section*{{\suttatitleacronym MN 101}{\suttatitletranslation At Devadaha }{\suttatitleroot Devadahasutta}}
\addcontentsline{toc}{section}{\tocacronym{MN 101} \toctranslation{At Devadaha } \tocroot{Devadahasutta}}
\markboth{At Devadaha }{Devadahasutta}
\extramarks{MN 101}{MN 101}

\scevam{So\marginnote{1.1} I have heard. }At one time the Buddha was staying in the land of the Sakyans, near the Sakyan town named Devadaha. There the Buddha addressed the mendicants, “Mendicants!” 

“Venerable\marginnote{2.2} sir,” they replied. The Buddha said this: 

“Mendicants,\marginnote{2.4} there are some ascetics and brahmins who have this doctrine and view: ‘Everything this individual experiences—pleasurable, painful, or neutral—is because of past deeds. So, due to eliminating past deeds by mortification, and not doing any new deeds, there’s nothing to come up in the future. With nothing to come up in the future, deeds end. With the ending of deeds, suffering ends. With the ending of suffering, feeling ends. And with the ending of feeling, all suffering will have been worn away.’ Such is the doctrine of the Jain ascetics. 

I’ve\marginnote{3.1} gone up to the Jain ascetics who say this and said, ‘Is it really true that this is the venerables’ view?’ They admitted that it is. 

I\marginnote{4.1} said to them, ‘But reverends, do you know for sure that you existed in the past, and it is not the case that you didn’t exist?’ 

‘No\marginnote{4.4} we don’t, reverend.’ 

‘But\marginnote{4.5} reverends, do you know for sure that you did bad deeds in the past?’ 

‘No\marginnote{4.7} we don’t, reverend.’ 

‘But\marginnote{4.8} reverends, do you know that you did such and such bad deeds?’ 

‘No\marginnote{4.10} we don’t, reverend.’ 

‘But\marginnote{4.11} reverends, do you know that so much suffering has already been worn away? Or that so much suffering still remains to be worn away? Or that when so much suffering is worn away all suffering will have been worn away?’ 

‘No\marginnote{4.13} we don’t, reverend.’ 

‘But\marginnote{4.14} reverends, do you know about giving up unskillful qualities in the present life and embracing skillful qualities?’ 

‘No\marginnote{4.16} we don’t, reverend.’ 

‘So\marginnote{5.1} it seems that you don’t know any of these things. In that case, it’s not appropriate for the Jain venerables to declare this. 

Now,\marginnote{6.1} supposing you did know these things. In that case, it would be appropriate for the Jain venerables to declare this. 

Suppose\marginnote{7.1} a man was struck by an arrow thickly smeared with poison, causing painful feelings, sharp and severe. Their friends and colleagues, relatives and kin would get a field surgeon to treat them. The surgeon would cut open the wound with a scalpel, causing painful feelings, sharp and severe. They’d probe for the arrow, causing painful feelings, sharp and severe. They’d extract the arrow, causing painful feelings, sharp and severe. They’d apply cauterizing medicine to the wound, causing painful feelings, sharp and severe. After some time that wound would be healed and the skin regrown. They’d be healthy, happy, autonomous, master of themselves, able to go where they wanted. 

They’d\marginnote{7.13} think, “Earlier I was struck by an arrow thickly smeared with poison, causing painful feelings, sharp and severe. My friends and colleagues, relatives and kin got a field surgeon to treat me. At each step, the treatment was painful. But these days that wound is healed and the skin regrown. I’m healthy, happy, autonomous, my own master, able to go where I want.” 

In\marginnote{8.1} the same way, reverends, if you knew about these things, it would be appropriate for the Jain venerables to declare this. 

But\marginnote{9.1} since you don’t know any of these things, it’s not appropriate for the Jain venerables to declare this.’ 

When\marginnote{10.1} I said this, those Jain ascetics said to me, ‘Reverend, the Jain leader \textsanskrit{Nāṭaputta} claims to be all-knowing and all-seeing, to know and see everything without exception, thus: “Knowledge and vision are constantly and continually present to me, while walking, standing, sleeping, and waking.” 

He\marginnote{10.4} says: “O reverend Jain ascetics, you have done bad deeds in a past life. Wear them away with these severe and grueling austerities. And when you refrain from such deeds in the present by way of body, speech, and mind, you’re not doing any bad deeds for the future. So, due to eliminating past deeds by mortification, and not doing any new deeds, there’s nothing to come up in the future. With nothing to come up in the future, deeds end. With the ending of deeds, suffering ends. With the ending of suffering, feeling ends. And with the ending of feeling, all suffering will have been worn away.” We like and accept this, and we are satisfied with it.’ 

When\marginnote{11.1} they said this, I said to them, ‘These five things can be seen to turn out in two different ways. What five? Faith, preference, oral tradition, reasoned contemplation, and acceptance of a view after consideration. These are the five things that can be seen to turn out in two different ways. In this case, what faith in your teacher do you have when it comes to the past? What preference, oral tradition, reasoned contemplation, or acceptance of a view after consideration?’ When I said this, I did not see any legitimate defense of their doctrine from the Jains. 

Furthermore,\marginnote{12.1} I said to those Jain ascetics, ‘What do you think, reverends? At a time of intense exertion and striving do you suffer painful, sharp, severe, acute feelings due to overexertion? Whereas at a time without intense exertion and striving do you not suffer painful, sharp, severe, acute feelings due to overexertion?’ 

‘Reverend\marginnote{12.5} Gotama, at a time of intense exertion we suffer painful, sharp feelings due to overexertion, not without intense exertion.’ 

‘So\marginnote{13.1} it seems that only at a time of intense exertion do you suffer painful, sharp feelings due to overexertion, not without intense exertion. In that case, it’s not appropriate for the Jain venerables to declare: “Everything this individual experiences—pleasurable, painful, or neutral—is because of past deeds. …” 

If\marginnote{14.1} at a time of intense exertion you did not suffer painful, sharp feelings due to overexertion, and if without intense exertion you did experience such feelings, it would be appropriate for the Jain venerables to declare this. 

But\marginnote{15.1} since this is not the case, aren’t you experiencing painful, sharp feelings due only to your own exertion, which out of ignorance, unknowing, and confusion you misconstrue to imply: “Everything this individual experiences—pleasurable, painful, or neutral—is because of past deeds. …”?’ When I said this, I did not see any legitimate defense of their doctrine from the Jains. 

Furthermore,\marginnote{16.1} I said to those Jain ascetics, ‘What do you think, reverends? If a deed is to be experienced in this life, can exertion make it be experienced in lives to come?’ 

‘No,\marginnote{16.3} reverend.’ 

‘But\marginnote{16.4} if a deed is to be experienced in lives to come, can exertion make it be experienced in this life?’ 

‘No,\marginnote{16.5} reverend.’ 

‘What\marginnote{17.1} do you think, reverends? If a deed is to be experienced as pleasure, can exertion make it be experienced as pain?’ 

‘No,\marginnote{17.2} reverend.’ 

‘But\marginnote{17.3} if a deed is to be experienced as pain, can exertion make it be experienced as pleasure?’ 

‘No,\marginnote{17.4} reverend.’ 

‘What\marginnote{18.1} do you think, reverends? If a deed is to be experienced when fully ripened, can exertion make it be experienced when not fully ripened?’ 

‘No,\marginnote{18.2} reverend.’ 

‘But\marginnote{18.3} if a deed is to be experienced when not fully ripened, can exertion make it be experienced when fully ripened?’ 

‘No,\marginnote{18.4} reverend.’ 

‘What\marginnote{19.1} do you think, reverends? If a deed is to be experienced strongly, can exertion make it be experienced weakly?’ 

‘No,\marginnote{19.2} reverend.’ 

‘But\marginnote{19.3} if a deed is to be experienced weakly, can exertion make it be experienced strongly?’ 

‘No,\marginnote{19.4} reverend.’ 

‘What\marginnote{20.1} do you think, reverends? If a deed is to be experienced, can exertion make it not be experienced?’ 

‘No,\marginnote{20.2} reverend.’ 

‘But\marginnote{20.3} if a deed is not to be experienced, can exertion make it be experienced?’ 

‘No,\marginnote{20.4} reverend.’ 

‘So\marginnote{21.1} it seems that exertion cannot change the way deeds are experienced in any of these ways. This being so, your exertion and striving are fruitless.’ 

Such\marginnote{22.1} is the doctrine of the Jain ascetics. Saying this, the Jain ascetics deserve rebuke and criticism on ten legitimate grounds. 

If\marginnote{22.3} sentient beings experience pleasure and pain because of past deeds, clearly the Jains have done bad deeds in the past, since they now experience such intense pain. If sentient beings experience pleasure and pain because of the Lord God’s creation, clearly the Jains were created by a bad God, since they now experience such intense pain. If sentient beings experience pleasure and pain because of circumstance and nature, clearly the Jains arise from bad circumstances, since they now experience such intense pain. If sentient beings experience pleasure and pain because of the class of rebirth, clearly the Jains have been reborn in a bad class, since they now experience such intense pain. If sentient beings experience pleasure and pain because of exertion in the present, clearly the Jains exert themselves badly in the present, since they now experience such intense pain. 

The\marginnote{22.13} Jains deserve criticism whether or not sentient beings experience pleasure and pain because of past deeds, or the Lord God’s creation, or circumstance and nature, or class of rebirth, or exertion in the present. Such is the doctrine of the Jain ascetics. The Jain ascetics who say this deserve rebuke and criticism on these ten legitimate grounds. That’s how exertion and striving is fruitless. 

And\marginnote{23.1} how is exertion and striving fruitful? It’s when a mendicant doesn’t bring suffering upon themselves; and they don’t give up legitimate pleasure, but they’re not besotted with that pleasure. They understand: ‘When I actively strive I become dispassionate towards this source of suffering. But when I develop equanimity I become dispassionate towards this other source of suffering.’ So they either actively strive or develop equanimity as appropriate. Through active striving they become dispassionate towards that specific source of suffering, and so that suffering is worn away. Through developing equanimity they become dispassionate towards that other source of suffering, and so that suffering is worn away. 

Suppose\marginnote{24.1} a man is in love with a woman, full of intense desire and lust. Then he sees her standing together with another man, chatting, giggling, and laughing. 

What\marginnote{24.3} do you think, mendicants? Would that give rise to sorrow, lamentation, pain, sadness, and distress for him?” 

“Yes,\marginnote{24.5} sir. Why is that? Because that man is in love that woman, full of intense desire and lust.” 

“Then\marginnote{25.1} that man might think: ‘I’m in love with that woman, full of intense desire and lust. When I saw her standing together with another man, chatting, giggling, and laughing, it gave rise to sorrow, lamentation, pain, sadness, and distress for me. Why don’t I give up that desire and lust for that woman?’ So that’s what he did. Some time later he sees her again standing together with another man, chatting, giggling, and laughing. 

What\marginnote{25.7} do you think, mendicants? Would that give rise to sorrow, lamentation, pain, sadness, and distress for him?” 

“No,\marginnote{25.9} sir. Why is that? Because he no longer desires that woman.” 

“In\marginnote{26.1} the same way, a mendicant doesn’t bring suffering upon themselves; and they don’t give up legitimate pleasure, but they’re not besotted with that pleasure. They understand: ‘When I actively strive I become dispassionate towards this source of suffering. But when I develop equanimity I become dispassionate towards this other source of suffering.’ So they either actively strive or develop equanimity as appropriate. Through active striving they become dispassionate towards that specific source of suffering, and so that suffering is worn away. Through developing equanimity they become dispassionate towards that other source of suffering, and so that suffering is worn away. That’s how exertion and striving is fruitful. 

Furthermore,\marginnote{27.1} a mendicant reflects: ‘When I live as I please, unskillful qualities grow and skillful qualities decline. But when I strive painfully, unskillful qualities decline and skillful qualities grow. Why don’t I strive painfully?’ So that’s what they do, and as they do so unskillful qualities decline and skillful qualities grow. After some time, they no longer strive painfully. Why is that? Because they have accomplished the goal for which they strived painfully. 

Suppose\marginnote{28.1} an arrowsmith was heating an arrow shaft between two firebrands, making it straight and fit for use. After it’s been made straight and fit for use, they’d no longer heat it to make it straight and fit for use. Why is that? Because they have accomplished the goal for which they heated it. 

In\marginnote{29.5} the same way, a mendicant reflects: ‘When I live as I please, unskillful qualities grow and skillful qualities decline. But when I strive painfully, unskillful qualities decline and skillful qualities grow. Why don’t I strive painfully?’ … After some time, they no longer strive painfully. That too is how exertion and striving is fruitful. 

Furthermore,\marginnote{30.1} a Realized One arises in the world, perfected, a fully awakened Buddha, accomplished in knowledge and conduct, holy, knower of the world, supreme guide for those who wish to train, teacher of gods and humans, awakened, blessed. He has realized with his own insight this world—with its gods, \textsanskrit{Māras} and \textsanskrit{Brahmās}, this population with its ascetics and brahmins, gods and humans—and he makes it known to others. He teaches Dhamma that’s good in the beginning, good in the middle, and good in the end, meaningful and well-phrased. And he reveals a spiritual practice that’s entirely full and pure. 

A\marginnote{31.1} householder hears that teaching, or a householder’s child, or someone reborn in some clan. They gain faith in the Realized One, and reflect: ‘Living in a house is cramped and dirty, but the life of one gone forth is wide open. It’s not easy for someone living at home to lead the spiritual life utterly full and pure, like a polished shell. Why don’t I shave off my hair and beard, dress in ocher robes, and go forth from the lay life to homelessness?’ After some time they give up a large or small fortune, and a large or small family circle. They shave off hair and beard, dress in ocher robes, and go forth from the lay life to homelessness. 

Once\marginnote{32.1} they’ve gone forth, they take up the training and livelihood of the mendicants. They give up killing living creatures, renouncing the rod and the sword. They’re scrupulous and kind, living full of compassion for all living beings. They give up stealing. They take only what’s given, and expect only what’s given. They keep themselves clean by not thieving. They give up unchastity. They are celibate, set apart, avoiding the common practice of sex. They give up lying. They speak the truth and stick to the truth. They’re honest and trustworthy, and don’t trick the world with their words. They give up divisive speech. They don’t repeat in one place what they heard in another so as to divide people against each other. Instead, they reconcile those who are divided, supporting unity, delighting in harmony, loving harmony, speaking words that promote harmony. They give up harsh speech. They speak in a way that’s mellow, pleasing to the ear, lovely, going to the heart, polite, likable and agreeable to the people. They give up talking nonsense. Their words are timely, true, and meaningful, in line with the teaching and training. They say things at the right time which are valuable, reasonable, succinct, and beneficial. They avoid injuring plants and seeds. They eat in one part of the day, abstaining from eating at night and food at the wrong time. They avoid dancing, singing, music, and seeing shows. They avoid beautifying and adorning themselves with garlands, perfumes, and makeup. They avoid high and luxurious beds. They avoid receiving gold and money, raw grains, raw meat, women and girls, male and female bondservants, goats and sheep, chickens and pigs, elephants, cows, horses, and mares, and fields and land. They avoid running errands and messages; buying and selling; falsifying weights, metals, or measures; bribery, fraud, cheating, and duplicity; mutilation, murder, abduction, banditry, plunder, and violence. 

They’re\marginnote{33.1} content with robes to look after the body and almsfood to look after the belly. Wherever they go, they set out taking only these things. They’re like a bird: wherever it flies, wings are its only burden. In the same way, a mendicant is content with robes to look after the body and almsfood to look after the belly. Wherever they go, they set out taking only these things. When they have this entire spectrum of noble ethics, they experience a blameless happiness inside themselves. 

When\marginnote{34.1} they see a sight with their eyes, they don’t get caught up in the features and details. If the faculty of sight were left unrestrained, bad unskillful qualities of desire and aversion would become overwhelming. For this reason, they practice restraint, protecting the faculty of sight, and achieving its restraint. When they hear a sound with their ears … When they smell an odor with their nose … When they taste a flavor with their tongue … When they feel a touch with their body … When they know a thought with their mind, they don’t get caught up in the features and details. If the faculty of mind were left unrestrained, bad unskillful qualities of desire and aversion would become overwhelming. For this reason, they practice restraint, protecting the faculty of mind, and achieving its restraint. When they have this noble sense restraint, they experience an unsullied bliss inside themselves. 

They\marginnote{35.1} act with situational awareness when going out and coming back; when looking ahead and aside; when bending and extending the limbs; when bearing the outer robe, bowl and robes; when eating, drinking, chewing, and tasting; when urinating and defecating; when walking, standing, sitting, sleeping, waking, speaking, and keeping silent. 

When\marginnote{36.1} they have this noble spectrum of ethics, this noble sense restraint, and this noble mindfulness and situational awareness, they frequent a secluded lodging—a wilderness, the root of a tree, a hill, a ravine, a mountain cave, a charnel ground, a forest, the open air, a heap of straw. After the meal, they return from almsround, sit down cross-legged with their body straight, and establish mindfulness right there. 

Giving\marginnote{37.1} up desire for the world, they meditate with a heart rid of desire, cleansing the mind of desire. Giving up ill will and malevolence, they meditate with a mind rid of ill will, full of compassion for all living beings, cleansing the mind of ill will. Giving up dullness and drowsiness, they meditate with a mind rid of dullness and drowsiness, perceiving light, mindful and aware, cleansing the mind of dullness and drowsiness. Giving up restlessness and remorse, they meditate without restlessness, their mind peaceful inside, cleansing the mind of restlessness and remorse. Giving up doubt, they meditate having gone beyond doubt, not undecided about skillful qualities, cleansing the mind of doubt. 

They\marginnote{38.1} give up these five hindrances, corruptions of the heart that weaken wisdom. Then, quite secluded from sensual pleasures, secluded from unskillful qualities, they enter and remain in the first absorption, which has the rapture and bliss born of seclusion, while placing the mind and keeping it connected. That too is how exertion and striving is fruitful. 

Furthermore,\marginnote{39.1} as the placing of the mind and keeping it connected are stilled, they enter and remain in the second absorption, which has the rapture and bliss born of immersion, with internal clarity and confidence, and unified mind, without placing the mind and keeping it connected. That too is how exertion and striving is fruitful. 

Furthermore,\marginnote{40.1} with the fading away of rapture, a mendicant enters and remains in the third absorption, where they meditate with equanimity, mindful and aware, personally experiencing the bliss of which the noble ones declare, ‘Equanimous and mindful, one meditates in bliss.’ That too is how exertion and striving is fruitful. 

Furthermore,\marginnote{41.1} giving up pleasure and pain, and ending former happiness and sadness, they enter and remain in the fourth absorption, without pleasure or pain, with pure equanimity and mindfulness. That too is how exertion and striving is fruitful. 

When\marginnote{42.1} their mind has become immersed in \textsanskrit{samādhi} like this—purified, bright, flawless, rid of corruptions, pliable, workable, steady, and imperturbable—they extend it toward recollection of past lives. They recollect many kinds of past lives, that is, one, two, three, four, five, ten, twenty, thirty, forty, fifty, a hundred, a thousand, a hundred thousand rebirths; many eons of the world contracting, many eons of the world expanding, many eons of the world contracting and expanding. They remember: ‘There, I was named this, my clan was that, I looked like this, and that was my food. This was how I felt pleasure and pain, and that was how my life ended. When I passed away from that place I was reborn somewhere else. There, too, I was named this, my clan was that, I looked like this, and that was my food. This was how I felt pleasure and pain, and that was how my life ended. When I passed away from that place I was reborn here.’ And so they recollect their many kinds of past lives, with features and details. That too is how exertion and striving is fruitful. 

When\marginnote{43.1} their mind has become immersed in \textsanskrit{samādhi} like this—purified, bright, flawless, rid of corruptions, pliable, workable, steady, and imperturbable—they extend it toward knowledge of the death and rebirth of sentient beings. With clairvoyance that is purified and superhuman, they see sentient beings passing away and being reborn—inferior and superior, beautiful and ugly, in a good place or a bad place. They understood how sentient beings are reborn according to their deeds: ‘These dear beings did bad things by way of body, speech, and mind. They spoke ill of the noble ones; they had wrong view; and they chose to act out of that wrong view. When their body breaks up, after death, they’re reborn in a place of loss, a bad place, the underworld, hell. These dear beings, however, did good things by way of body, speech, and mind. They never spoke ill of the noble ones; they had right view; and they chose to act out of that right view. When their body breaks up, after death, they’re reborn in a good place, a heavenly realm.’ And so, with clairvoyance that is purified and superhuman, they see sentient beings passing away and being reborn—inferior and superior, beautiful and ugly, in a good place or a bad place. They understand how sentient beings are reborn according to their deeds. That too is how exertion and striving is fruitful. 

When\marginnote{44.1} their mind has become immersed in \textsanskrit{samādhi} like this—purified, bright, flawless, rid of corruptions, pliable, workable, steady, and imperturbable—they extend it toward knowledge of the ending of defilements. They truly understand: ‘This is suffering’ … ‘This is the origin of suffering’ … ‘This is the cessation of suffering’ … ‘This is the practice that leads to the cessation of suffering’. They truly understand: ‘These are defilements’ … ‘This is the origin of defilements’ … ‘This is the cessation of defilements’ … ‘This is the practice that leads to the cessation of defilements’. 

Knowing\marginnote{45.1} and seeing like this, their mind is freed from the defilements of sensuality, desire to be reborn, and ignorance. When they’re freed, they know they’re freed. 

They\marginnote{45.3} understand: ‘Rebirth is ended, the spiritual journey has been completed, what had to be done has been done, there is no return to any state of existence.’ That too is how exertion and striving is fruitful. Such is the doctrine of the Realized One. Saying this, the Realized One deserves praise on ten legitimate grounds. 

If\marginnote{46.3} sentient beings experience pleasure and pain because of past deeds, clearly the Realized One has done good deeds in the past, since he now experiences such undefiled pleasure. If sentient beings experience pleasure and pain because of the Lord God’s creation, clearly the Realized One was created by a good God, since he now experiences such undefiled pleasure. If sentient beings experience pleasure and pain because of circumstance and nature, clearly the Realized One arises from good circumstances, since he now experiences such undefiled pleasure. If sentient beings experience pleasure and pain because of the class of rebirth, clearly the Realized One was reborn in a good class, since he now experiences such undefiled pleasure. If sentient beings experience pleasure and pain because of exertion in the present, clearly the Realized One exerts himself well in the present, since he now experiences such undefiled pleasure. 

The\marginnote{46.13} Realized One deserves praise whether or not sentient beings experience pleasure and pain because of past deeds, or the Lord God’s creation, or circumstance and nature, or class of rebirth, or exertion in the present. Such is the doctrine of the Realized One. Saying this, the Realized One deserves praise on these ten legitimate grounds.” 

That\marginnote{46.25} is what the Buddha said. Satisfied, the mendicants were happy with what the Buddha said. 

%
\section*{{\suttatitleacronym MN 102}{\suttatitletranslation The Five and Three }{\suttatitleroot Pañcattayasutta}}
\addcontentsline{toc}{section}{\tocacronym{MN 102} \toctranslation{The Five and Three } \tocroot{Pañcattayasutta}}
\markboth{The Five and Three }{Pañcattayasutta}
\extramarks{MN 102}{MN 102}

\scevam{So\marginnote{1.1} I have heard. }At one time the Buddha was staying near \textsanskrit{Sāvatthī} in Jeta’s Grove, \textsanskrit{Anāthapiṇḍika}’s monastery. There the Buddha addressed the mendicants, “Mendicants!” 

“Venerable\marginnote{1.5} sir,” they replied. The Buddha said this: 

“Mendicants,\marginnote{2.1} there are some ascetics and brahmins who theorize about the future, and assert various hypotheses concerning the future. Some propose this: ‘The self is percipient and is well after death.’ Some propose this: ‘The self is non-percipient and is well after death.’ Some propose this: ‘The self is neither percipient nor non-percipient and is well after death.’ But some assert the annihilation, eradication, and obliteration of an existing being, while others propose extinguishment in the present life. Thus they assert an existent self that is well after death; or they assert the annihilation of an existing being; or they propose extinguishment in the present life. In this way five become three, and three become five. This is the passage for recitation of the five and three. 

Now,\marginnote{3.1} the ascetics and brahmins who assert a self that is percipient and well after death describe it as having form, or being formless, or both having form and being formless, or neither having form nor being formless. Or they describe it as of unified perception, or of diverse perception, or of limited perception, or of limitless perception. Or some among those who go beyond this propose universal consciousness, limitless and imperturbable. 

The\marginnote{4.1} Realized One understands this as follows. There are ascetics and brahmins who assert a self that is percipient and well after death, describing it as having form, or being formless, or both having form and being formless, or neither having form nor being formless. Or they describe it as of unified perception, or of diverse perception, or of limited perception, or of limitless perception. Or some, aware that ‘there is nothing at all’, propose the dimension of nothingness, limitless and imperturbable. They declare that this is the purest, highest, best, and supreme of all those perceptions, whether of form or of formlessness or of unity or of diversity. ‘All that is conditioned and coarse. But there is the cessation of conditions—\emph{that} is real.’ Understanding thus and seeing the escape from it, the Realized One has gone beyond all that. 

Now,\marginnote{5.1} the ascetics and brahmins who assert a self that is non-percipient and well after death describe it as having form, or being formless, or both having form and being formless, or neither having form nor being formless. 

So\marginnote{6.1} they reject those who assert a self that is percipient and well after death. Why is that? Because they believe that perception is a disease, a boil, a dart, and that the state of non-perception is peaceful and sublime. 

The\marginnote{7.1} Realized One understands this as follows. There are ascetics and brahmins who assert a self that is non-percipient and well after death, describing it as having form, or being formless, or both having form and being formless, or neither having form nor being formless. But if any ascetic or brahmin should say this: ‘Apart from form, feeling, perception, and choices, I will describe the coming and going of consciousness, its passing away and reappearing, its growth, increase, and maturity.’ That is not possible. ‘All that is conditioned and coarse. But there is the cessation of conditions—\emph{that} is real.’ Understanding this and seeing the escape from it, the Realized One has gone beyond all that. 

Now,\marginnote{8.1} the ascetics and brahmins who assert a self that is neither percipient nor non-percipient and well after death describe it as having form, or being formless, or both having form and being formless, or neither having form nor being formless. 

So\marginnote{9.1} they reject those who assert a self that is percipient and well after death, as well as those who assert a self that is non-percipient and sound after death. Why is that? Because they believe that perception is a disease, a boil, a dart, and that the state of neither perception nor non-perception is peaceful and sublime. 

The\marginnote{10.1} Realized One understands this as follows. There are ascetics and brahmins who assert a self that is neither percipient nor non-percipient and well after death, describing it as having form, or being formless, or both having form and being formless, or neither having form nor being formless. Some ascetics or brahmins assert the embracing of that dimension merely through the conditioned phenomena of what is seen, heard, thought, and known. But that is said to be a disastrous approach. For that dimension is said to be not attainable by means of conditioned phenomena, but only with a residue of conditioned phenomena. ‘All that is conditioned and coarse. But there is the cessation of conditions—\emph{that} is real.’ Understanding this and seeing the escape from it, the Realized One has gone beyond all that. 

Now,\marginnote{11.1} the ascetics and brahmins who assert the annihilation, eradication, and obliteration of an existing being reject those who assert a self that is well after death, whether percipient or non-percipient or neither percipient non-percipient. Why is that? Because all of those ascetics and brahmins only assert their attachment to heading upstream: ‘After death we shall be like this! After death we shall be like that!’ 

Suppose\marginnote{12.1} a trader was going to market, thinking: ‘With this, that shall be mine! This way, I shall get that!’ In the same way, those ascetics and brahmins seem to be like traders when they say: ‘After death we shall be like this! After death we shall be like that!’ The Realized One understands this as follows. The ascetics and brahmins who assert the annihilation, eradication, and obliteration of an existing being; from fear and disgust with identity, they just keep running and circling around identity. Suppose a dog on a leash was tethered to a strong post or pillar. It would just keeping running and circling around that post or pillar. In the same way, those ascetics and brahmins, from fear and disgust with identity, just keep running and circling around identity. ‘All that is conditioned and coarse. But there is the cessation of conditions—\emph{that} is real.’ Understanding this and seeing the escape from it, the Realized One has gone beyond all that. 

Whatever\marginnote{13.1} ascetics and brahmins theorize about the future, and propose various hypotheses concerning the future, all of them propose one or other of these five theses. 

There\marginnote{14.1} are some ascetics and brahmins who theorize about the past, and propose various hypotheses concerning the past. They propose the following, each insisting that theirs is the only truth and that everything else is wrong. ‘The self and the cosmos are eternal.’ ‘The self and the cosmos are not eternal.’ ‘The self and the cosmos are both eternal and not eternal.’ ‘The self and the cosmos are neither eternal nor not eternal.’ ‘The self and the cosmos are finite.’ ‘The self and the cosmos are infinite.’ ‘The self and the cosmos are both finite and infinite.’ ‘The self and the cosmos are neither finite nor infinite.’ ‘The self and the cosmos are unified in perception.’ ‘The self and the cosmos are diverse in perception.’ ‘The self and the cosmos have limited perception.’ ‘The self and the cosmos have limitless perception.’ ‘The self and the cosmos experience nothing but happiness.’ ‘The self and the cosmos experience nothing but suffering.’ ‘The self and the cosmos experience both happiness and suffering.’ ‘The self and the cosmos experience neither happiness nor suffering.’ 

Now,\marginnote{15.1} consider the ascetics and brahmins whose view is as follows. ‘The self and the cosmos are eternal. This is the only truth, other ideas are silly.’ It’s simply not possible for them to have purified and clear personal knowledge of this, apart from faith, preference, oral tradition, reasoned contemplation, or acceptance of a view after consideration. And in the absence of such knowledge, even the partial knowledge that they are clear about is said to be grasping on their part. ‘All that is conditioned and coarse. But there is the cessation of conditions—\emph{that} is real.’ Understanding this and seeing the escape from it, the Realized One has gone beyond all that. 

Now,\marginnote{16.1} consider the ascetics and brahmins whose view is as follows. The self and the cosmos are not eternal, or both eternal and not eternal, or neither eternal nor not-eternal, or finite, or infinite, or both finite and infinite, or neither finite nor infinite, or of unified perception, or of diverse perception, or of limited perception, or of limitless perception, or experience nothing but happiness, or experience nothing but suffering, or experience both happiness and suffering, or experience neither happiness nor suffering. It’s simply not possible for them to have purified and clear personal knowledge of this, apart from faith, preference, oral tradition, reasoned contemplation, or acceptance of a view after consideration. And in the absence of such knowledge, even the partial knowledge that they are clear about is said to be grasping on their part. ‘All that is conditioned and coarse. But there is the cessation of conditions—\emph{that} is real.’ Understanding this and seeing the escape from it, the Realized One has gone beyond all that. 

Now,\marginnote{17.1} some ascetics and brahmins, letting go of theories about the past and the future, shedding the fetters of sensuality, enter and remain in the rapture of seclusion: ‘This is peaceful, this is sublime, that is, entering and remaining in the rapture of seclusion.’ But that rapture of seclusion of theirs ceases. When the rapture of seclusion ceases, sadness arises; and when sadness ceases, the rapture of seclusion arises. 

It’s\marginnote{18.1} like how the sunlight fills the space when the shadow leaves, or the shadow fills the space when the sunshine leaves. In the same way, when the rapture of seclusion ceases, sadness arises; and when sadness ceases, the rapture of seclusion arises. The Realized One understands this as follows. This good ascetic or brahmin, letting go of theories about the past and the future, shedding the fetters of sensuality, enters and remains in the rapture of seclusion: ‘This is peaceful, this is sublime, that is, entering and remaining in the rapture of seclusion.’ But that rapture of seclusion of theirs ceases. When the rapture of seclusion ceases, sadness arises; and when sadness ceases, the rapture of seclusion arises. ‘All that is conditioned and coarse. But there is the cessation of conditions—\emph{that} is real.’ Understanding this and seeing the escape from it, the Realized One has gone beyond all that. 

Now,\marginnote{19.1} some ascetics and brahmins, letting go of theories about the past and the future, shedding the fetters of sensuality, going beyond the rapture of seclusion, enter and remain in spiritual bliss. ‘This is peaceful, this is sublime, that is, entering and remaining in spiritual bliss.’ But that spiritual bliss of theirs ceases. When spiritual bliss ceases, the rapture of seclusion arises; and when the rapture of seclusion ceases, spiritual bliss arises. 

It’s\marginnote{20.1} like how the sunlight fills the space when the shadow leaves, or the shadow fills the space when the sunshine leaves. … The Realized One understands this as follows. This good ascetic or brahmin, letting go of theories about the past and the future, shedding the fetters of sensuality, going beyond the rapture of seclusion, enters and remains in spiritual bliss. ‘This is peaceful, this is sublime, that is, entering and remaining in spiritual bliss.’ But that spiritual bliss of theirs ceases. When spiritual bliss ceases, the rapture of seclusion arises; and when the rapture of seclusion ceases, spiritual bliss arises. ‘All that is conditioned and coarse. But there is the cessation of conditions—\emph{that} is real.’ Understanding this and seeing the escape from it, the Realized One has gone beyond all that. 

Now,\marginnote{21.1} some ascetics and brahmins, letting go of theories about the past and the future, shedding the fetters of sensuality, going beyond the rapture of seclusion and spiritual bliss, enter and remain in neutral feeling. ‘This is peaceful, this is sublime, that is, entering and remaining in neutral feeling.’ Then that neutral feeling ceases. When neutral feeling ceases, spiritual bliss arises; and when spiritual bliss ceases, neutral feelings arises. 

It’s\marginnote{22.1} like how the sunlight fills the space when the shadow leaves, or the shadow fills the space when the sunshine leaves. … The Realized One understands this as follows. This good ascetic or brahmin, letting go of theories about the past and the future, shedding the fetters of sensuality, going beyond the rapture of seclusion and spiritual bliss, enters and remains in neutral feeling. ‘This is peaceful, this is sublime, that is, entering and remaining in neutral feeling.’ Then that neutral feeling ceases. When neutral feeling ceases, spiritual bliss arises; and when spiritual bliss ceases, neutral feelings arises. ‘All that is conditioned and coarse. But there is the cessation of conditions—\emph{that} is real.’ Understanding this and seeing the escape from it, the Realized One has gone beyond all that. 

Now,\marginnote{23.1} some ascetics and brahmins, letting go of theories about the past and the future, shedding the fetters of sensuality, go beyond the rapture of seclusion, spiritual bliss, and neutral feeling. They regard themselves like this: ‘I am at peace; I am extinguished; I am free of grasping.’ 

The\marginnote{24.1} Realized One understands this as follows. This good ascetic or brahmin, letting go of theories about the past and the future, shedding the fetters of sensuality, goes beyond the rapture of seclusion, spiritual bliss, and neutral feeling. They regard themselves like this: ‘I am at peace; I am extinguished; I am free of grasping.’ Clearly this venerable speaks of a practice that’s conducive to extinguishment. Nevertheless, they still grasp at theories about the past or the future, or the fetters of sensuality, or the rapture of seclusion, or spiritual bliss, or neutral feeling. And when they regard themselves like this: ‘I am at peace; I am extinguished; I am free of grasping,’ that’s also said to be grasping on their part. ‘All that is conditioned and coarse. But there is the cessation of conditions—\emph{that} is real.’ Understanding this and seeing the escape from it, the Realized One has gone beyond all that. 

But\marginnote{25.1} the Realized One has awakened to the supreme state of sublime peace, that is, liberation by not grasping after truly understanding the six fields of contact’s origin, ending, gratification, drawback, and escape.” 

That\marginnote{25.3} is what the Buddha said. Satisfied, the mendicants were happy with what the Buddha said. 

%
\section*{{\suttatitleacronym MN 103}{\suttatitletranslation Is This What You Think Of Me? }{\suttatitleroot Kintisutta}}
\addcontentsline{toc}{section}{\tocacronym{MN 103} \toctranslation{Is This What You Think Of Me? } \tocroot{Kintisutta}}
\markboth{Is This What You Think Of Me? }{Kintisutta}
\extramarks{MN 103}{MN 103}

\scevam{So\marginnote{1.1} I have heard. }At one time the Buddha was staying near \textsanskrit{Kusināra}, in the Forest of Offerings. There the Buddha addressed the mendicants, “Mendicants!” 

“Venerable\marginnote{1.5} sir,” they replied. The Buddha said this: 

“Mendicants,\marginnote{2.1} is this what you think of me? ‘The ascetic Gotama teaches the Dhamma for the sake of robes, almsfood, lodgings, or rebirth in this or that state.’” 

“No\marginnote{2.3} sir, we don’t think of you that way.” 

“If\marginnote{2.5} you don’t think of me that way, then what exactly do you think of me?” 

“We\marginnote{2.9} think of you this way: ‘The Buddha is compassionate and wants what’s best for us. He teaches out of compassion.’” 

“So\marginnote{2.12} it seems you think that I teach out of compassion. 

In\marginnote{3.1} that case, each and every one of you should train in the things I have taught from my direct knowledge, that is: the four kinds of mindfulness meditation, the four right efforts, the four bases of psychic power, the five faculties, the five powers, the seven awakening factors, and the noble eightfold path. You should train in these things in harmony, appreciating each other, without quarreling. 

As\marginnote{4.1} you do so, it may happen that two mendicants disagree about the teaching. 

Now,\marginnote{5.1} you might think, ‘These two venerables disagree on both the meaning and the phrasing.’ So you should approach whichever mendicant you think is most amenable and say to them: ‘The venerables disagree on the meaning and the phrasing. But the venerables should know that this is how such disagreement on the meaning and the phrasing comes to be. Please don’t get into a dispute about this.’ Then they should approach whichever mendicant they think is most amenable among those who side with the other party and say to them: ‘The venerables disagree on the meaning and the phrasing. But the venerables should know that this is how such disagreement on the meaning and the phrasing comes to be. Please don’t get into a dispute about this.’ So you should remember what has been incorrectly memorized as incorrectly memorized and what has been correctly memorized as correctly memorized. Remembering this, you should speak on the teaching and the training. 

Now,\marginnote{6.1} you might think, ‘These two venerables disagree on the meaning but agree on the phrasing.’ So you should approach whichever mendicant you think is most amenable and say to them: ‘The venerables disagree on the meaning but agree on the phrasing. But the venerables should know that this is how such disagreement on the meaning and agreement on the phrasing comes to be. Please don’t get into a dispute about this.’ Then they should approach whichever mendicant they think is most amenable among those who side with the other party and say to them: ‘The venerables disagree on the meaning but agree on the phrasing. But the venerables should know that this is how such disagreement on the meaning and agreement on the phrasing comes to be. Please don’t get into a dispute about this.’ So you should remember what has been incorrectly memorized as incorrectly memorized and what has been correctly memorized as correctly memorized. Remembering this, you should speak on the teaching and the training. 

Now,\marginnote{7.1} you might think, ‘These two venerables agree on the meaning but disagree on the phrasing.’ So you should approach whichever mendicant you think is most amenable and say to them: ‘The venerables agree on the meaning but disagree on the phrasing. But the venerables should know that this is how such agreement on the meaning and disagreement on the phrasing comes to be. But the phrasing is a minor matter. Please don’t get into a dispute about something so minor.’ Then they should approach whichever mendicant they think is most amenable among those who side with the other party and say to them: ‘The venerables agree on the meaning but disagree on the phrasing. But the venerables should know that this is how such agreement on the meaning and disagreement on the phrasing comes to be. But the phrasing is a minor matter. Please don’t get into a dispute about something so minor.’ So you should remember what has been correctly memorized as correctly memorized and what has been incorrectly memorized as incorrectly memorized. Remembering this, you should speak on the teaching and the training. 

Now,\marginnote{8.1} you might think, ‘These two venerables agree on both the meaning and the phrasing.’ So you should approach whichever mendicant you think is most amenable and say to them: ‘The venerables agree on both the meaning and the phrasing. But the venerables should know that this is how they come to agree on the meaning and the phrasing. Please don’t get into a dispute about this.’ Then they should approach whichever mendicant they think is most amenable among those who side with the other party and say to them: ‘The venerables agree on both the meaning and the phrasing. But the venerables should know that this is how they come to agree on the meaning and the phrasing. Please don’t get into a dispute about this.’ So you should remember what has been correctly memorized as correctly memorized. Remembering this, you should speak on the teaching and the training. 

As\marginnote{9.1} you train in harmony, appreciating each other, without quarreling, one of the mendicants might commit an offense or transgression. In such a case, you should not be in a hurry to accuse them. The individual should be examined like this: ‘I won’t be troubled and the other individual won’t be hurt, for they’re not irritable and hostile. They don’t hold fast to their views, but let them go easily. I can draw them away from the unskillful and establish them in the skillful.’ If that’s what you think, then it’s appropriate to speak to them. 

But\marginnote{11.1} suppose you think this: ‘I will be troubled and the other individual will be hurt, for they’re irritable and hostile. However, they don’t hold fast to their views, but let them go easily. I can draw them away from the unskillful and establish them in the skillful. But for the other individual to get hurt is a minor matter. It’s more important that I can draw them away from the unskillful and establish them in the skillful.’ If that’s what you think, then it’s appropriate to speak to them. 

But\marginnote{12.1} suppose you think this: ‘I will be troubled but the other individual won’t be hurt, for they’re not irritable and hostile. However, they hold fast to their views, refusing to let go. Nevertheless, I can draw them away from the unskillful and establish them in the skillful. But for me to be troubled is a minor matter. It’s more important that I can draw them away from the unskillful and establish them in the skillful.’ If that’s what you think, then it’s appropriate to speak to them. 

But\marginnote{13.1} suppose you think this: ‘I will be troubled and the other individual will be hurt, for they’re irritable and hostile. And they hold fast to their views, refusing to let go. Nevertheless, I can draw them away from the unskillful and establish them in the skillful. But for me to be troubled and the other individual to get hurt is a minor matter. It’s more important that I can draw them away from the unskillful and establish them in the skillful.’ If that’s what you think, then it’s appropriate to speak to them. 

But\marginnote{14.1} suppose you think this: ‘I will be troubled and the other individual will be hurt, for they’re irritable and hostile. And they hold fast to their views, refusing to let go. I cannot draw them away from the unskillful and establish them in the skillful.’ Don’t underestimate the value of equanimity for such a person. 

As\marginnote{15.1} you train in harmony, appreciating each other, without quarreling, mutual tale-bearing might come up, with contempt for each other’s views, resentful, bitter, and exasperated. In this case you should approach whichever mendicant you think is most amenable among those who side with one party and say to them: ‘Reverend, as we were training, mutual tale-bearing came up. If the Ascetic knew about this, would he rebuke it?’ Answering rightly, the mendicant should say: ‘Yes, reverend, he would.’ ‘But without giving that up, reverend, can one realize extinguishment?’ Answering rightly, the mendicant should say: ‘No, reverend, one cannot.’ 

Then\marginnote{16.1} they should approach whichever mendicant they think is most amenable among those who side with the other party and say to them: ‘Reverend, as we were training, mutual tale-bearing came up. If the Ascetic knew about this, would he rebuke it?’ Answering rightly, the mendicant should say: ‘Yes, reverend, he would.’ ‘But without giving that up, reverend, can one realize extinguishment?’ Answering rightly, the mendicant should say: ‘No, reverend, one cannot.’ 

If\marginnote{17.1} others should ask that mendicant: ‘Were you the venerable who drew those mendicants away from the unskillful and established them in the skillful?’ Answering rightly, the mendicant should say: ‘Well, reverends, I approached the Buddha. He taught me the Dhamma. After hearing that teaching I explained it to those mendicants. When those mendicants heard that teaching they were drawn away from the unskillful and established in the skillful.’ Answering in this way, that mendicant doesn’t glorify themselves or put others down. They answer in line with the teaching, with no legitimate grounds for rebuke and criticism.” 

That\marginnote{17.7} is what the Buddha said. Satisfied, the mendicants were happy with what the Buddha said. 

%
\section*{{\suttatitleacronym MN 104}{\suttatitletranslation At Sāmagāma }{\suttatitleroot Sāmagāmasutta}}
\addcontentsline{toc}{section}{\tocacronym{MN 104} \toctranslation{At Sāmagāma } \tocroot{Sāmagāmasutta}}
\markboth{At Sāmagāma }{Sāmagāmasutta}
\extramarks{MN 104}{MN 104}

\scevam{So\marginnote{1.1} I have heard. }At one time the Buddha was staying among the Sakyans near the village of \textsanskrit{Sāma}. 

Now\marginnote{2.1} at that time the \textsanskrit{Nigaṇṭha} \textsanskrit{Nātaputta} had recently passed away at \textsanskrit{Pāvā}. With his passing the Jain ascetics split, dividing into two factions, arguing, quarreling, and disputing, continually wounding each other with barbed words: “You don’t understand this teaching and training. I understand this teaching and training. What, you understand this teaching and training? You’re practicing wrong. I’m practicing right. I stay on topic, you don’t. You said last what you should have said first. You said first what you should have said last. What you’ve thought so much about has been disproved. Your doctrine is refuted. Go on, save your doctrine! You’re trapped; get yourself out of this—if you can!” You’d think there was nothing but slaughter going on among the Jain ascetics. And the \textsanskrit{Nigaṇṭha} \textsanskrit{Nātaputta}’s white-clothed lay disciples were disillusioned, dismayed, and disappointed in the Jain ascetics. They were equally disappointed with a teaching and training so poorly explained and poorly propounded, not emancipating, not leading to peace, proclaimed by someone who is not a fully awakened Buddha, with broken monument and without a refuge. 

And\marginnote{3.1} then, after completing the rainy season residence near \textsanskrit{Pāvā}, the novice Cunda went to see Venerable Ānanda at \textsanskrit{Sāma} village. He bowed, sat down to one side, and told him what had happened. 

Ānanda\marginnote{4.1} said to him, “Reverend Cunda, we should see the Buddha about this matter. Come, let’s go to the Buddha and inform him about this.” 

“Yes,\marginnote{4.4} sir,” replied Cunda. 

Then\marginnote{4.5} Ānanda and Cunda went to the Buddha, bowed, sat down to one side, and Ānanda informed him of what Cunda had said. He went on to say, “Sir, it occurs to me: ‘When the Buddha has passed away, let no dispute arise in the \textsanskrit{Saṅgha}. For such a dispute would be for the hurt and unhappiness of the people, for the harm, hurt, and suffering of gods and humans.’” 

“What\marginnote{5.1} do you think, Ānanda? Do you see even two mendicants who disagree regarding the things I have taught from my direct knowledge, that is, the four kinds of mindfulness meditation, the four right efforts, the four bases of psychic power, the five faculties, the five powers, the seven awakening factors, and the noble eightfold path?” 

“No,\marginnote{5.4} sir, I do not. Nevertheless, there are some individuals who appear to live obedient to the Buddha, but when the Buddha has passed away they might create a dispute in the \textsanskrit{Saṅgha} regarding livelihood or the monastic code. Such a dispute would be for the hurt and unhappiness of the people, for the harm, hurt, and suffering of gods and humans.” 

“Ānanda,\marginnote{5.8} dispute about livelihood or the monastic code is a minor matter. But should a dispute arise in the \textsanskrit{Saṅgha} concerning the path or the practice, that would be for the hurt and unhappiness of the people, for the harm, hurt, and suffering of gods and humans. 

Ānanda,\marginnote{6.1} there are these six roots of arguments. What six? Firstly, a mendicant is irritable and hostile. Such a mendicant lacks respect and reverence for the teacher, the teaching, and the \textsanskrit{Saṅgha}, and they don’t fulfill the training. They create a dispute in the \textsanskrit{Saṅgha}, which is for the hurt and unhappiness of the people, for the harm, hurt, and suffering of gods and humans. If you see such a root of arguments in yourselves or others, you should try to give up this bad thing. If you don’t see it, you should practice so that it doesn’t come up in the future. That’s how to give up this bad root of arguments, so it doesn’t come up in the future. 

Furthermore,\marginnote{7{-}11.1} a mendicant is offensive and contemptuous … They’re jealous and stingy … They’re devious and deceitful … They have wicked desires and wrong view … They’re attached to their own views, holding them tight, and refusing to let go. Such a mendicant lacks respect and reverence for the teacher, the teaching, and the \textsanskrit{Saṅgha}, and they don’t fulfill the training. They create a dispute in the \textsanskrit{Saṅgha}, which is for the hurt and unhappiness of the people, for the harm, hurt, and suffering of gods and humans. If you see such a root of arguments in yourselves or others, you should try to give up this bad thing. If you don’t see it, you should practice so that it doesn’t come up in the future. That’s how to give up this bad root of arguments, so it doesn’t come up in the future. These are the six roots of arguments. 

There\marginnote{12.1} are four kinds of disciplinary issues. What four? Disciplinary issues due to disputes, accusations, offenses, or proceedings. These are the four kinds of disciplinary issues. There are seven methods for the settlement of any disciplinary issues that might arise. Removal in the presence of those concerned is applicable. Removal by accurate recollection is applicable. Removal due to recovery from madness is applicable. The offense should be acknowledged. The decision of a majority. A verdict of aggravated misconduct. Covering over with grass. 

And\marginnote{14.1} how is there removal in the presence of those concerned? It’s when mendicants are disputing: ‘This is the teaching,’ ‘This is not the teaching,’ ‘This is the monastic law,’ ‘This is not the monastic law.’ Those mendicants should all sit together in harmony and thoroughly go over the guidelines of the teaching. They should settle that disciplinary issue in agreement with the guidelines. That’s how there is removal in the presence of those concerned. And that’s how certain disciplinary issues are settled, that is, by removal in the presence of those concerned. 

And\marginnote{15.1} how is there the decision of a majority? If those mendicants are not able to settle that issue in that monastery, they should go to another monastery with more mendicants. There they should all sit together in harmony and thoroughly go over the guidelines of the teaching. They should settle that disciplinary issue in agreement with the guidelines. That’s how there is the decision of a majority. And that’s how certain disciplinary issues are settled, that is, by decision of a majority. 

And\marginnote{16.1} how is there removal by accurate recollection? It’s when mendicants accuse a mendicant of a serious offense; one entailing expulsion, or close to it: ‘Venerable, do you recall committing the kind of serious offense that entails expulsion or close to it?’ They say: ‘No, reverends, I don’t recall committing such an offense.’ The removal by accurate recollection is applicable to them. That’s how there is the removal by accurate recollection. And that’s how certain disciplinary issues are settled, that is, by removal by accurate recollection. 

And\marginnote{17.1} how is there removal by recovery from madness? It’s when mendicants accuse a mendicant of the kind of serious offense that entails expulsion, or close to it: ‘Venerable, do you recall committing the kind of serious offense that entails expulsion or close to it?’ They say: ‘No, reverends, I don’t recall committing such an offense.’ But though they try to get out of it, the mendicants pursue the issue: ‘Surely the venerable must know perfectly well if you recall committing an offense that entails expulsion or close to it!’ They say: ‘Reverends, I had gone mad, I was out of my mind. And while I was mad I did and said many things that are not proper for an ascetic. I don’t remember any of that, I was mad when I did it.’ The removal by recovery from madness is applicable to them. That’s how there is the removal by recovery from madness. And that’s how certain disciplinary issues are settled, that is, by recovery from madness. 

And\marginnote{18.1} how is there the acknowledging of an offense? It’s when a mendicant, whether accused or not, recalls an offense and clarifies it and reveals it. After approaching a more senior mendicant, that mendicant should arrange his robe over one shoulder, bow to that mendicant’s feet, squat on their heels, raise their joined palms, and say: ‘Sir, I have fallen into such-and-such an offense. I confess it.’ The senior mendicant says: ‘Do you see it?’ ‘Yes, I see it.’ ‘Then restrain yourself in future.’ ‘I shall restrain myself.’ That’s how there is the acknowledging of an offense. And that’s how certain disciplinary issues are settled, that is, by acknowledging an offense. 

And\marginnote{19.1} how is there a verdict of aggravated misconduct? It’s when a mendicant accuses a mendicant of the kind of serious offense that entails expulsion, or close to it: ‘Venerable, do you recall committing the kind of serious offense that entails expulsion or close to it?’ They say: ‘No, reverends, I don’t recall committing such an offense.’ But though they try to get out of it, the mendicants pursue the issue: ‘Surely the venerable must know perfectly well if you recall committing an offense that entails expulsion or close to it!’ They say: ‘Reverends, I don’t recall committing a serious offense of that nature. But I do recall committing a light offense.’ But though they try to get out of it, the mendicants pursue the issue: ‘Surely the venerable must know perfectly well if you recall committing an offense that entails expulsion or close to it!’ They say: ‘Reverends, I’ll go so far as to acknowledge this light offense even when not asked. Why wouldn’t I acknowledge a serious offense when asked?’ They say: ‘You wouldn’t have acknowledged that light offense without being asked, so why would you acknowledge a serious offense? Surely the venerable must know perfectly well if you recall committing an offense that entails expulsion or close to it!’ They say: ‘Reverend, I do recall committing the kind of serious offense that entails expulsion or close to it. I spoke too hastily when I said that I didn’t recall it.’ That’s how there is a verdict of aggravated misconduct. And that’s how certain disciplinary issues are settled, that is, by a verdict of aggravated misconduct. 

And\marginnote{20.1} how is there the covering over with grass? It’s when the mendicants continually argue, quarrel, and dispute, doing and saying many things that are not proper for an ascetic. Those mendicants should all sit together in harmony. A competent mendicant of one party, having got up from their seat, arranged their robe over one shoulder, and raised their joined palms, should inform the \textsanskrit{Saṅgha}: 

‘Sir,\marginnote{20.5} let the \textsanskrit{Saṅgha} listen to me. We have been continually arguing, quarreling, and disputing, doing and saying many things that are not proper for an ascetic. If it seems appropriate to the \textsanskrit{Saṅgha}, then—for the benefit of these venerables and myself—I disclose in the middle of the \textsanskrit{Saṅgha} by means of covering over with grass any offenses committed by these venerables and by myself, excepting only those that are gravely blameworthy and those connected with laypeople.’ 

Then\marginnote{20.8} a competent mendicant of the other party, having got up from their seat, arranged their robe over one shoulder, and raising their joined palms, should inform the \textsanskrit{Saṅgha}: 

‘Sir,\marginnote{20.9} let the \textsanskrit{Saṅgha} listen to me. We have been continually arguing, quarreling, and disputing, doing and saying many things that are not proper for an ascetic. If it seems appropriate to the \textsanskrit{Saṅgha}, then—for the benefit of these venerables and myself—I disclose in the middle of the \textsanskrit{Saṅgha} by means of covering over with grass any offenses committed by these venerables and by myself, excepting only those that are gravely blameworthy and those connected with laypeople.’ 

That’s\marginnote{20.12} how there is the covering over with grass. And that’s how certain disciplinary issues are settled, that is, by covering over with grass. 

Ānanda,\marginnote{21.1} these six warm-hearted qualities make for fondness and respect, conducing to inclusion, harmony, and unity, without quarreling. What six? Firstly, a mendicant consistently treats their spiritual companions with bodily kindness, both in public and in private. This warm-hearted quality makes for fondness and respect, conducing to inclusion, harmony, and unity, without quarreling. 

Furthermore,\marginnote{21.5} a mendicant consistently treats their spiritual companions with verbal kindness … This too is a warm-hearted quality. 

Furthermore,\marginnote{21.7} a mendicant consistently treats their spiritual companions with mental kindness … This too is a warm-hearted quality. 

Furthermore,\marginnote{21.9} a mendicant shares without reservation any material possessions they have gained by legitimate means, even the food placed in the alms-bowl, using them in common with their ethical spiritual companions. This too is a warm-hearted quality. 

Furthermore,\marginnote{21.11} a mendicant lives according to the precepts shared with their spiritual companions, both in public and in private. Those precepts are unbroken, impeccable, spotless, and unmarred, liberating, praised by sensible people, not mistaken, and leading to immersion. This too is a warm-hearted quality. 

Furthermore,\marginnote{21.13} a mendicant lives according to the view shared with their spiritual companions, both in public and in private. That view is noble and emancipating, and leads one who practices it to the complete ending of suffering. This too is a warm-hearted quality. 

These\marginnote{21.15} six warm-hearted qualities make for fondness and respect, conducing to inclusion, harmony, and unity, without quarreling. 

If\marginnote{22.1} you should undertake and follow these six warm-hearted qualities, do you see any criticism, large or small, that you could not endure?” 

“No,\marginnote{22.2} sir.” 

“That’s\marginnote{22.3} why, Ānanda, you should undertake and follow these six warm-hearted qualities. That will be for your lasting welfare and happiness.” 

That\marginnote{22.5} is what the Buddha said. Satisfied, Venerable Ānanda was happy with what the Buddha said. 

%
\section*{{\suttatitleacronym MN 105}{\suttatitletranslation With Sunakkhatta }{\suttatitleroot Sunakkhattasutta}}
\addcontentsline{toc}{section}{\tocacronym{MN 105} \toctranslation{With Sunakkhatta } \tocroot{Sunakkhattasutta}}
\markboth{With Sunakkhatta }{Sunakkhattasutta}
\extramarks{MN 105}{MN 105}

\scevam{So\marginnote{1.1} I have heard. }At one time the Buddha was staying near \textsanskrit{Vesālī}, at the Great Wood, in the hall with the peaked roof. 

Now\marginnote{2.1} at that time several mendicants had declared their enlightenment in the Buddha’s presence: 

“We\marginnote{2.2} understand: ‘Rebirth is ended, the spiritual journey has been completed, what had to be done has been done, there is no return to any state of existence.’” 

Sunakkhatta\marginnote{2.3} the Licchavi heard about this. 

He\marginnote{3.1} went to the Buddha, bowed, sat down to one side, and said to him, “Sir, I have heard that several mendicants have declared their enlightenment in the Buddha’s presence. I trust they did so rightly—or are there some who declared enlightenment out of overestimation?” 

“Some\marginnote{5.1} of them did so rightly, Sunakkhatta, while others did so out of overestimation. Now, when mendicants declare enlightenment rightly, that’s how it is for them. But when mendicants declare enlightenment out of overestimation, the Realized One thinks: ‘I should teach them the Dhamma.’ If the Realized One thinks he should teach them the Dhamma, but then certain foolish men, having carefully planned a question, approach the Realized One and ask it, then the Realized One changes his mind.” 

“Now\marginnote{6.1} is the time, Blessed One! Now is the time, Holy One! Let the Buddha teach the Dhamma. The mendicants will listen and remember it.” 

“Well\marginnote{6.3} then, Sunakkhatta, listen and pay close attention, I will speak.” 

“Yes,\marginnote{6.4} sir,” replied Sunakkhatta. The Buddha said this: 

“Sunakkhatta,\marginnote{7.1} there are these five kinds of sensual stimulation. What five? Sights known by the eye that are likable, desirable, agreeable, pleasant, sensual, and arousing. Sounds known by the ear … Smells known by the nose … Tastes known by the tongue … Touches known by the body that are likable, desirable, agreeable, pleasant, sensual, and arousing. These are the five kinds of sensual stimulation. 

It’s\marginnote{8.1} possible that a certain individual may be intent on material pleasures. Such an individual engages in pertinent conversation, thinking and considering in line with that. They associate with that kind of person, and they find it satisfying. But when talk connected with the imperturbable is going on they don’t want to listen. They don’t lend an ear or apply their minds to understand it. They don’t associate with that kind of person, and they don’t find it satisfying. 

Suppose\marginnote{9.1} a person had left their own village or town long ago, and they saw another person who had only recently left there. They would ask about whether their village was safe, with plenty of food and little disease, and the other person would tell them the news. What do you think, Sunakkhatta? Would that person want to listen to that other person? Would they lend an ear and apply their minds to understand? Would they associate with that person, and find it satisfying?” 

“Yes,\marginnote{9.7} sir.” 

“In\marginnote{9.8} the same way, it’s possible that a certain individual may be intent on material pleasures. Such an individual engages in pertinent conversation, thinking and considering in line with that. They associate with that kind of person, and they find it satisfying. But when talk connected with the imperturbable is going on they don’t want to listen. They don’t lend an ear or apply their minds to understand it. They don’t associate with that kind of person, and they don’t find it satisfying. You should know of them: ‘That individual is intent on material pleasures, for they’re detached from things connected with the imperturbable.’ 

It’s\marginnote{10.1} possible that a certain individual may be intent on the imperturbable. Such an individual engages in pertinent conversation, thinking and considering in line with that. They associate with that kind of person, and they find it satisfying. But when talk connected with material pleasures is going on they don’t want to listen. They don’t lend an ear or apply their minds to understand it. They don’t associate with that kind of person, and they don’t find it satisfying. 

Suppose\marginnote{11.1} there was a fallen, withered leaf. It’s incapable of becoming green again. In the same way, an individual intent on the imperturbable has dropped the connection with material pleasures. You should know of them: ‘That individual is intent on the imperturbable, for they’re detached from things connected with material pleasures.’ 

It’s\marginnote{12.1} possible that a certain individual may be intent on the dimension of nothingness. Such an individual engages in pertinent conversation, thinking and considering in line with that. They associate with that kind of person, and they find it satisfying. But when talk connected with the imperturbable is going on they don’t want to listen. They don’t lend an ear or apply their minds to understand it. They don’t associate with that kind of person, and they don’t find it satisfying. 

Suppose\marginnote{13.1} there was a broad rock that had been broken in half, so that it could not be put back together again. In the same way, an individual intent on the dimension of nothingness has broken the connection with the imperturbable. You should know of them: ‘That individual is intent on the dimension of nothingness, for they’re detached from things connected with the imperturbable.’ 

It’s\marginnote{14.1} possible that a certain individual may be intent on the dimension of neither perception nor non-perception. Such an individual engages in pertinent conversation, thinking and considering in line with that. They associate with that kind of person, and they find it satisfying. But when talk connected with the dimension of nothingness is going on they don’t want to listen. They don’t lend an ear or apply their minds to understand it. They don’t associate with that kind of person, and they don’t find it satisfying. 

Suppose\marginnote{15.1} someone had eaten some delectable food and thrown it up. What do you think, Sunakkhatta? Would that person want to eat that food again?” 

“No,\marginnote{15.4} sir. Why is that? Because that food is considered repulsive.” 

“In\marginnote{15.7} the same way, an individual intent on the dimension of neither perception nor non-perception has vomited the connection with the dimension of nothingness. You should know of them: ‘That individual is intent on the dimension of neither perception nor non-perception, for they’re detached from things connected with the dimension of nothingness.’ 

It’s\marginnote{16.1} possible that a certain individual may be rightly intent on extinguishment. Such an individual engages in pertinent conversation, thinking and considering in line with that. They associate with that kind of person, and they find it satisfying. But when talk connected with the dimension of neither perception nor non-perception is going on they don’t want to listen. They don’t lend an ear or apply their minds to understand it. They don’t associate with that kind of person, and they don’t find it satisfying. 

Suppose\marginnote{17.1} there was a palm tree with its crown cut off. It’s incapable of further growth. In the same way, an individual rightly intent on extinguishment has cut off the connection with the dimension of neither perception nor non-perception at the root, made it like a palm stump, obliterated it, so it’s unable to arise in the future. You should know of them: ‘That individual is rightly intent on extinguishment, for they’re detached from things connected with the dimension of neither perception nor non-perception.’ 

It’s\marginnote{18.1} possible that a certain mendicant might think: ‘The Ascetic has said that craving is a dart; and that the poison of ignorance is inflicted by desire and ill will. I have given up the dart of craving and expelled the poison of ignorance; I am rightly intent on extinguishment.’ Having such conceit, though it’s not based in fact, they would engage in things unconducive to extinguishment: unsuitable sights, sounds, smells, tastes, touches, and thoughts. Doing so, lust infects their mind, resulting in death or deadly pain. 

Suppose\marginnote{19.1} a man was struck by an arrow thickly smeared with poison. Their friends and colleagues, relatives and kin would get a field surgeon to treat them. The surgeon would cut open the wound with a scalpel, probe for the arrow, extract it, and expel the poison, leaving some residue behind. Thinking that no residue remained, the surgeon would say: ‘My good man, the dart has been extracted and the poison expelled without residue. It’s not capable of harming you. Eat only suitable food. Don’t eat unsuitable food, or else the wound may get infected. Regularly wash the wound and anoint the opening, or else it’ll get covered with pus and blood. Don’t walk too much in the wind and sun, or else dust and dirt will infect the wound. Take care of the wound, my good sir, heal it.’ 

They’d\marginnote{20.1} think: ‘The dart has been extracted and the poison expelled without residue. It’s not capable of harming me.’ They’d eat unsuitable food, and the wound would get infected. And they wouldn’t regularly wash and anoint the opening, so it would get covered in pus and blood. And they’d walk too much in the wind and sun, so dust and dirt infected the wound. And they wouldn’t take care of the wound or heal it. Then both because they did what was unsuitable, and because of the residue of unclean poison, the wound would spread, resulting in death or deadly pain. 

In\marginnote{21.1} the same way, it’s possible that a certain mendicant might think: ‘The Ascetic has said that craving is a dart; and that the poison of ignorance is inflicted by desire and ill will. I have given up the dart of craving and expelled the poison of ignorance; I am rightly intent on extinguishment.’ Having such conceit, though it’s not based in fact, they would engage in things unconducive to extinguishment: unsuitable sights, sounds, smells, tastes, touches, and thoughts. Doing so, lust infects their mind, resulting in death or deadly pain. 

For\marginnote{22.1} it is death in the training of the Noble One to resign the training and return to a lesser life. And it is deadly pain to commit one of the corrupt offenses. 

It’s\marginnote{23.1} possible that a certain mendicant might think: ‘The Ascetic has said that craving is a dart; and that the poison of ignorance is inflicted by desire and ill will. I have given up the dart of craving and expelled the poison of ignorance; I am rightly intent on extinguishment.’ Being rightly intent on extinguishment, they wouldn’t engage in things unconducive to extinguishment: unsuitable sights, sounds, smells, tastes, touches, and thoughts. Doing so, lust wouldn’t infect their mind, so no death or deadly pain would result. 

Suppose\marginnote{24.1} a man was struck by an arrow thickly smeared with poison. Their friends and colleagues, relatives and kin would get a field surgeon to treat them. The surgeon would cut open the wound with a scalpel, probe for the arrow, extract it, and expel the poison, leaving no residue behind. Knowing that no residue remained, the surgeon would say: ‘My good man, the dart has been extracted and the poison expelled without residue. It’s not capable of harming you. Eat only suitable food. Don’t eat unsuitable food, or else the wound may get infected. Regularly wash the wound and anoint the opening, or else it’ll get covered with pus and blood. Don’t walk too much in the wind and sun, or else dust and dirt will infect the wound. Take care of the wound, my good sir, heal it.’ 

They’d\marginnote{25.1} think: ‘The dart has been extracted and the poison expelled without residue. It’s not capable of harming me.’ They’d eat suitable food, and the wound wouldn’t get infected. And they’d regularly wash and anoint the opening, so it wouldn’t get covered in pus and blood. And they wouldn’t walk too much in the wind and sun, so dust and dirt wouldn’t infect the wound. And they’d take care of the wound and heal it. Then both because they did what was suitable, and the unclean poison had left no residue, the wound would heal, and no death or deadly pain would result. 

In\marginnote{26.1} the same way, it’s possible that a certain mendicant might think: ‘The Ascetic has said that craving is a dart; and that the poison of ignorance is inflicted by desire and ill will. I have given up the dart of craving and expelled the poison of ignorance; I am rightly intent on extinguishment.’ Being rightly intent on extinguishment, they wouldn’t engage in things unconducive to extinguishment: unsuitable sights, sounds, smells, tastes, touches, and thoughts. Doing so, lust wouldn’t infect their mind, so no death or deadly pain would result. 

I’ve\marginnote{27.1} made up this simile to make a point. And this is the point: ‘Wound’ is a term for the six interior sense fields. ‘Poison’ is a term for ignorance. ‘Dart’ is a term for craving. ‘Probing’ is a term for mindfulness. ‘Scalpel’ is a term for noble wisdom. ‘Field surgeon’ is a term for the Realized One, the perfected one, the fully awakened Buddha. 

Truly,\marginnote{28.1} Sunakkhatta, that mendicant practices restraint regarding the six fields of contact. Understanding that attachment is the root of suffering, they are freed with the ending of attachments. It’s not possible that they would apply their body or interest their mind in any attachment. 

Suppose\marginnote{29.1} there was a bronze cup of beverage that had a nice color, aroma, and flavor. But it was mixed with poison. Then a person would come along who wants to live and doesn’t want to die, who wants to be happy and recoils from pain. What do you think, Sunakkhatta? Would that person drink that beverage if they knew that it would result in death or deadly suffering?” 

“No,\marginnote{29.7} sir.” 

“In\marginnote{29.8} the same way, Sunakkhatta, that mendicant practices restraint regarding the six fields of contact. Understanding that attachment is the root of suffering, they are freed with the ending of attachments. It’s not possible that they would apply their body or interest their mind in any attachment. 

Suppose\marginnote{30.1} there was a lethal viper. Then a person would come along who wants to live and doesn’t want to die, who wants to be happy and recoils from pain. What do you think, Sunakkhatta? Would that person give that lethal viper their hand or finger if they knew that it would result in death or deadly suffering?” 

“No,\marginnote{30.6} sir.” 

“In\marginnote{30.7} the same way, Sunakkhatta, that mendicant practices restraint regarding the six fields of contact. Understanding that attachment is the root of suffering, they are freed with the ending of attachments. It’s not possible that they would apply their body or interest their mind in any attachment.” 

That\marginnote{30.10} is what the Buddha said. Satisfied, Sunakkhatta of the Licchavi clan was happy with what the Buddha said. 

%
\section*{{\suttatitleacronym MN 106}{\suttatitletranslation Conducive to the Imperturbable }{\suttatitleroot Āneñjasappāyasutta}}
\addcontentsline{toc}{section}{\tocacronym{MN 106} \toctranslation{Conducive to the Imperturbable } \tocroot{Āneñjasappāyasutta}}
\markboth{Conducive to the Imperturbable }{Āneñjasappāyasutta}
\extramarks{MN 106}{MN 106}

\scevam{So\marginnote{1.1} I have heard. }At one time the Buddha was staying in the land of the Kurus, near the Kuru town named \textsanskrit{Kammāsadamma}. There the Buddha addressed the mendicants, “Mendicants!” 

“Venerable\marginnote{1.5} sir,” they replied. The Buddha said this: 

“Mendicants,\marginnote{2.1} sensual pleasures are impermanent, hollow, false, and deceptive, made by illusion, cooed over by fools. Sensual pleasures in this life and in lives to come, sensual perceptions in this life and in lives to come; both of these are \textsanskrit{Māra}’s domain, \textsanskrit{Māra}’s realm, and \textsanskrit{Māra}’s territory. They conduce to bad, unskillful qualities such as desire, ill will, and aggression. And they create an obstacle for a noble disciple training here. 

A\marginnote{3.1} noble disciple reflects on this: ‘Sensual pleasures in this life and in lives to come, sensual perceptions in this life and in lives to come; both of these are \textsanskrit{Māra}’s domain, \textsanskrit{Māra}’s realm, and \textsanskrit{Māra}’s territory. They conduce to bad, unskillful qualities such as desire, ill will, and aggression. And they create an obstacle for a noble disciple training here. Why don’t I meditate with an abundant, expansive heart, having mastered the world and stabilized the mind? Then I will have no more bad, unskillful qualities such as desire, ill will, and aggression. And by giving them up my mind, no longer limited, will become limitless and well developed.’ 

Practicing\marginnote{3.10} in this way and meditating on it often their mind becomes confident in this dimension. Being confident, they either attain the imperturbable now, or are freed by wisdom. When their body breaks up, after death, it’s possible that the consciousness headed that way will be reborn in the imperturbable. This is said to be the first way of practice suitable for attaining the imperturbable. 

Furthermore,\marginnote{4.1} a noble disciple reflects: ‘Sensual pleasures in this life and in lives to come, sensual perceptions in this life and in lives to come; whatever is form, all form is the four primary elements, or form derived from the four primary elements.’ Practicing in this way and meditating on it often their mind becomes confident in this dimension. Being confident, they either attain the imperturbable now, or are freed by wisdom. When their body breaks up, after death, it’s possible that the consciousness headed that way will be reborn in the imperturbable. This is said to be the second way of practice suitable for attaining the imperturbable. 

Furthermore,\marginnote{5.1} a noble disciple reflects: ‘Sensual pleasures in this life and in lives to come, sensual perceptions in this life and in lives to come, visions in this life and in lives to come, perceptions of visions in this life and in lives to come; all of these are impermanent. And what’s impermanent is not worth approving, welcoming, or clinging to.’ Practicing in this way and meditating on it often their mind becomes confident in this dimension. Being confident, they either attain the imperturbable now, or are freed by wisdom. When their body breaks up, after death, it’s possible that the consciousness headed that way will be reborn in the imperturbable. This is said to be the third way of practice suitable for attaining the imperturbable. 

Furthermore,\marginnote{6.1} a noble disciple reflects: ‘Sensual pleasures in this life and in lives to come, sensual perceptions in this life and in lives to come, visions in this life and in lives to come, perceptions of visions in this life and in lives to come, and perceptions of the imperturbable; all are perceptions. Where they cease without anything left over, that is peaceful, that is sublime, namely the dimension of nothingness.’ Practicing in this way and meditating on it often their mind becomes confident in this dimension. Being confident, they either attain the dimension of nothingness now, or are freed by wisdom. When their body breaks up, after death, it’s possible that the consciousness headed that way will be reborn in the dimension of nothingness. This is said to be the first way of practice suitable for attaining the dimension of nothingness. 

Furthermore,\marginnote{7.1} a noble disciple has gone to a wilderness, or to the root of a tree, or to an empty hut, and reflects like this: ‘This is empty of a self or what belongs to a self.’ Practicing in this way and meditating on it often their mind becomes confident in this dimension. Being confident, they either attain the dimension of nothingness now, or are freed by wisdom. When their body breaks up, after death, it’s possible that the consciousness headed that way will be reborn in the dimension of nothingness. This is said to be the second way of practice suitable for attaining the dimension of nothingness. 

Furthermore,\marginnote{8.1} a noble disciple reflects: ‘I don’t belong to anyone anywhere! And nothing belongs to me anywhere!’ Practicing in this way and meditating on it often their mind becomes confident in this dimension. Being confident, they either attain the dimension of nothingness now, or are freed by wisdom. When their body breaks up, after death, it’s possible that the consciousness headed that way will be reborn in the dimension of nothingness. This is said to be the third way of practice suitable for attaining the dimension of nothingness. 

Furthermore,\marginnote{9.1} a noble disciple reflects: ‘Sensual pleasures in this life and in lives to come, sensual perceptions in this life and in lives to come, visions in this life and in lives to come, perceptions of visions in this life and in lives to come, perceptions of the imperturbable, and perceptions of the dimension of nothingness; all are perceptions. Where they cease without anything left over, that is peaceful, that is sublime, namely the dimension of neither perception nor non-perception.’ Practicing in this way and meditating on it often their mind becomes confident in this dimension. Being confident, they either attain the dimension of neither perception nor non-perception now, or are freed by wisdom. When their body breaks up, after death, it’s possible that the consciousness headed that way will be reborn in the dimension of neither perception nor non-perception. This is said to be the way of practice suitable for attaining the dimension of neither perception nor non-perception.” 

When\marginnote{10.1} he said this, Venerable Ānanda said to the Buddha: “Sir, take a mendicant who practices like this: ‘It might not be, and it might not be mine. It will not be, and it will not be mine. I am giving up what exists, what has come to be.’ In this way they gain equanimity. Would that mendicant become extinguished or not?” 

“One\marginnote{10.6} such mendicant might become extinguished, Ānanda, while another might not.” 

“What\marginnote{10.7} is the cause, sir, what is the reason for this?” 

“Ānanda,\marginnote{10.8} take a mendicant who practices like this: ‘It might not be, and it might not be mine. It will not be, and it will not be mine. I am giving up what exists, what has come to be.’ In this way they gain equanimity. They approve, welcome, and keep clinging to that equanimity. Their consciousness relies on that and grasps it. A mendicant with grasping does not become extinguished.” 

“But\marginnote{11.1} sir, what is that mendicant grasping?” 

“The\marginnote{11.2} dimension of neither perception nor non-perception.” 

“Sir,\marginnote{11.3} it seems that mendicant is grasping the best thing to grasp!” 

“Indeed,\marginnote{11.4} Ānanda. For the best thing to grasp is the dimension of neither perception nor non-perception. 

Take\marginnote{12.1} a mendicant who practices like this: ‘It might not be, and it might not be mine. It will not be, and it will not be mine. I am giving up what exists, what has come to be.’ In this way they gain equanimity. They don’t approve, welcome, or keep clinging to that equanimity. So their consciousness doesn’t rely on that and grasp it. A mendicant free of grasping becomes extinguished.” 

“It’s\marginnote{13.1} incredible, sir, it’s amazing! The Buddha has explained to us how to cross over the flood by relying on one support or the other. But sir, what is noble liberation?” 

“Ananda,\marginnote{13.4} it’s when a mendicant reflects like this: ‘Sensual pleasures in this life and in lives to come, sensual perceptions in this life and in lives to come, visions in this life and in lives to come, perceptions of visions in this life and in lives to come, perceptions of the imperturbable, perceptions of the dimension of nothingness, perceptions of the dimension of neither perception nor non-perception; that is identity as far as identity extends. This is the deathless, namely the liberation of the mind through not grasping. 

So,\marginnote{14.1} Ānanda, I have taught the ways of practice suitable for attaining the imperturbable, the dimension of nothingness, and the dimension of neither perception nor non-perception. I have taught how to cross the flood by relying on one support or the other, and I have taught noble liberation. 

Out\marginnote{15.1} of compassion, I’ve done what a teacher should do who wants what’s best for their disciples. Here are these roots of trees, and here are these empty huts. Practice absorption, Ānanda! Don’t be negligent! Don’t regret it later! This is my instruction to you.” 

That\marginnote{15.3} is what the Buddha said. Satisfied, Venerable Ānanda was happy with what the Buddha said. 

%
\section*{{\suttatitleacronym MN 107}{\suttatitletranslation With Moggallāna the Accountant }{\suttatitleroot Gaṇakamoggallānasutta}}
\addcontentsline{toc}{section}{\tocacronym{MN 107} \toctranslation{With Moggallāna the Accountant } \tocroot{Gaṇakamoggallānasutta}}
\markboth{With Moggallāna the Accountant }{Gaṇakamoggallānasutta}
\extramarks{MN 107}{MN 107}

\scevam{So\marginnote{1.1} I have heard. }At one time the Buddha was staying near \textsanskrit{Sāvatthī} in the Eastern Monastery, the stilt longhouse of \textsanskrit{Migāra}’s mother. Then the brahmin \textsanskrit{Moggallāna} the Accountant went up to the Buddha, and exchanged greetings with him. When the greetings and polite conversation were over, he sat down to one side and said to the Buddha: 

“Master\marginnote{2.1} Gotama, in this stilt longhouse we can see gradual progress down to the last step of the staircase. Among the brahmins we can see gradual progress in learning the chants. Among archers we can see gradual progress in archery. Among us accountants, who earn a living by accounting, we can see gradual progress in mathematics. For when we get an apprentice we first make them count: ‘One one, two twos, three threes, four fours, five fives, six sixes, seven sevens, eight eights, nine nines, ten tens.’ We even make them count up to a hundred. Is it possible to similarly describe a gradual training, gradual progress, and gradual practice in this teaching and training?” 

“It\marginnote{3.1} is possible, brahmin. Suppose a deft horse trainer were to obtain a fine thoroughbred. First of all he’d make it get used to wearing the bit. In the same way, when the Realized One gets a person for training they first guide them like this: ‘Come, mendicant, be ethical and restrained in the monastic code, conducting yourself well and seeking alms in suitable places. Seeing danger in the slightest fault, keep the rules you’ve undertaken.’ 

When\marginnote{4.1} they have ethical conduct, the Realized One guides them further: ‘Come, mendicant, guard your sense doors. When you see a sight with your eyes, don’t get caught up in the features and details. If the faculty of sight were left unrestrained, bad unskillful qualities of desire and aversion would become overwhelming. For this reason, practice restraint, protect the faculty of sight, and achieve restraint over it. When you hear a sound with your ears … When you smell an odor with your nose … When you taste a flavor with your tongue … When you feel a touch with your body … When you know a thought with your mind, don’t get caught up in the features and details. If the faculty of mind were left unrestrained, bad unskillful qualities of desire and aversion would become overwhelming. For this reason, practice restraint, protect the faculty of mind, and achieve its restraint.’ 

When\marginnote{5.1} they guard their sense doors, the Realized One guides them further: ‘Come, mendicant, eat in moderation. Reflect properly on the food that you eat: ‘Not for fun, indulgence, adornment, or decoration, but only to sustain this body, to avoid harm, and to support spiritual practice. In this way, I shall put an end to old discomfort and not give rise to new discomfort, and I will live blamelessly and at ease.’ 

When\marginnote{6.1} they eat in moderation, the Realized One guides them further: ‘Come, mendicant, be committed to wakefulness. Practice walking and sitting meditation by day, purifying your mind from obstacles. In the evening, continue to practice walking and sitting meditation. In the middle of the night, lie down in the lion’s posture—on the right side, placing one foot on top of the other—mindful and aware, and focused on the time of getting up. In the last part of the night, get up and continue to practice walking and sitting meditation, purifying your mind from obstacles.’ 

When\marginnote{7.1} they are committed to wakefulness, the Realized One guides them further: ‘Come, mendicant, have mindfulness and situational awareness. Act with situational awareness when going out and coming back; when looking ahead and aside; when bending and extending the limbs; when bearing the outer robe, bowl and robes; when eating, drinking, chewing, and tasting; when urinating and defecating; when walking, standing, sitting, sleeping, waking, speaking, and keeping silent.’ 

When\marginnote{8.1} they have mindfulness and situational awareness, the Realized One guides them further: ‘Come, mendicant, frequent a secluded lodging—a wilderness, the root of a tree, a hill, a ravine, a mountain cave, a charnel ground, a forest, the open air, a heap of straw.’ And they do so. 

After\marginnote{9.2} the meal, they return from almsround, sit down cross-legged with their body straight, and establish mindfulness right there. Giving up desire for the world, they meditate with a heart rid of desire, cleansing the mind of desire. Giving up ill will and malevolence, they meditate with a mind rid of ill will, full of compassion for all living beings, cleansing the mind of ill will. Giving up dullness and drowsiness, they meditate with a mind rid of dullness and drowsiness, perceiving light, mindful and aware, cleansing the mind of dullness and drowsiness. Giving up restlessness and remorse, they meditate without restlessness, their mind peaceful inside, cleansing the mind of restlessness and remorse. Giving up doubt, they meditate having gone beyond doubt, not undecided about skillful qualities, cleansing the mind of doubt. 

They\marginnote{10.1} give up these five hindrances, corruptions of the heart that weaken wisdom. Then, quite secluded from sensual pleasures, secluded from unskillful qualities, they enter and remain in the first absorption, which has the rapture and bliss born of seclusion, while placing the mind and keeping it connected. As the placing of the mind and keeping it connected are stilled, they enter and remain in the second absorption, which has the rapture and bliss born of immersion, with internal clarity and confidence, and unified mind, without placing the mind and keeping it connected. And with the fading away of rapture, they enter and remain in the third absorption, where they meditate with equanimity, mindful and aware, personally experiencing the bliss of which the noble ones declare, ‘Equanimous and mindful, one meditates in bliss.’ Giving up pleasure and pain, and ending former happiness and sadness, they enter and remain in the fourth absorption, without pleasure or pain, with pure equanimity and mindfulness. 

That’s\marginnote{11.1} how I instruct the mendicants who are trainees—who haven’t achieved their heart’s desire, but live aspiring to the supreme sanctuary. But for those mendicants who are perfected—who have ended the defilements, completed the spiritual journey, done what had to be done, laid down the burden, achieved their own goal, utterly ended the fetters of rebirth, and are rightly freed through enlightenment—these things lead to blissful meditation in the present life, and to mindfulness and awareness.” 

When\marginnote{12.1} he had spoken, \textsanskrit{Moggallāna} the Accountant said to the Buddha, “When his disciples are instructed and advised like this by Master Gotama, do all of them achieve the ultimate goal, extinguishment, or do some of them fail?” 

“Some\marginnote{12.3} succeed, while others fail.” 

“What\marginnote{13.1} is the cause, Master Gotama, what is the reason why, though extinguishment is present, the path leading to extinguishment is present, and Master Gotama is present to encourage them, still some succeed while others fail?” 

“Well\marginnote{14.1} then, brahmin, I’ll ask you about this in return, and you can answer as you like. What do you think, brahmin? Are you skilled in the road to \textsanskrit{Rājagaha}?” 

“Yes,\marginnote{14.4} I am.” 

“What\marginnote{14.5} do you think, brahmin? Suppose a person was to come along who wanted to go to \textsanskrit{Rājagaha}. He’d approach you and say: ‘Sir, I wish to go to \textsanskrit{Rājagaha}. Please point out the road to \textsanskrit{Rājagaha}.’ Then you’d say to them: ‘Here, mister, this road goes to \textsanskrit{Rājagaha}. Go along it for a while, and you’ll see a certain village. Go along a while further, and you’ll see a certain town. Go along a while further and you’ll see \textsanskrit{Rājagaha} with its delightful parks, woods, meadows, and lotus ponds.’ Instructed like this by you, they might still take the wrong road, heading west. But a second person might come with the same question and receive the same instructions. Instructed by you, they might safely arrive at \textsanskrit{Rājagaha}. What is the cause, brahmin, what is the reason why, though \textsanskrit{Rājagaha} is present, the path leading to \textsanskrit{Rājagaha} is present, and you are there to encourage them, one person takes the wrong path and heads west, while another arrives safely at \textsanskrit{Rājagaha}?” 

“What\marginnote{14.27} can I do about that, Master Gotama? I am the one who shows the way.” 

“In\marginnote{14.29} the same way, though extinguishment is present, the path leading to extinguishment is present, and I am present to encourage them, still some of my disciples, instructed and advised like this, achieve the ultimate goal, extinguishment, while some of them fail. What can I do about that, brahmin? The Realized One is the one who shows the way.” 

When\marginnote{15.1} he had spoken, \textsanskrit{Moggallāna} the Accountant said to the Buddha, “Master Gotama, there are those faithless people who went forth from the lay life to homelessness not out of faith but to earn a livelihood. They’re devious, deceitful, and sneaky. They’re restless, insolent, fickle, scurrilous, and loose-tongued. They do not guard their sense doors or eat in moderation, and they are not committed to wakefulness. They don’t care about the ascetic life, and don’t keenly respect the training. They’re indulgent and slack, leaders in backsliding, neglecting seclusion, lazy, and lacking energy. They’re unmindful, lacking situational awareness and immersion, with straying minds, witless and stupid. Master Gotama doesn’t live together with these. 

But\marginnote{15.3} there are those gentlemen who went forth from the lay life to homelessness out of faith. They’re not devious, deceitful, and sneaky. They’re not restless, insolent, fickle, scurrilous, and loose-tongued. They guard their sense doors and eat in moderation, and they are committed to wakefulness. They care about the ascetic life, and keenly respect the training. They’re not indulgent or slack, nor are they leaders in backsliding, neglecting seclusion. They’re energetic and determined. They’re mindful, with situational awareness, immersion, and unified minds; wise, not stupid. Master Gotama does live together with these. 

Of\marginnote{16.1} all kinds of fragrant root, spikenard is said to be the best. Of all kinds of fragrant heartwood, red sandalwood is said to be the best. Of all kinds of fragrant flower, jasmine is said to be the best. In the same way, Master Gotama’s advice is the best of contemporary teachings. 

Excellent,\marginnote{17.1} Master Gotama! Excellent! As if he were righting the overturned, or revealing the hidden, or pointing out the path to the lost, or lighting a lamp in the dark so people with good eyes can see what’s there, Master Gotama has made the Teaching clear in many ways. I go for refuge to Master Gotama, to the teaching, and to the mendicant \textsanskrit{Saṅgha}. From this day forth, may Master Gotama remember me as a lay follower who has gone for refuge for life.” 

%
\section*{{\suttatitleacronym MN 108}{\suttatitletranslation With Moggallāna the Guardian }{\suttatitleroot Gopakamoggallānasutta}}
\addcontentsline{toc}{section}{\tocacronym{MN 108} \toctranslation{With Moggallāna the Guardian } \tocroot{Gopakamoggallānasutta}}
\markboth{With Moggallāna the Guardian }{Gopakamoggallānasutta}
\extramarks{MN 108}{MN 108}

\scevam{So\marginnote{1.1} I have heard. }At one time Venerable Ānanda was staying near \textsanskrit{Rājagaha}, in the Bamboo Grove, the squirrels’ feeding ground. It was not long after the Buddha had become fully extinguished. 

Now\marginnote{2.1} at that time King \textsanskrit{Ajātasattu} Vedehiputta of Magadha, being suspicious of King Pajjota, was having \textsanskrit{Rājagaha} fortified. 

Then\marginnote{3.1} Venerable Ānanda robed up in the morning and, taking his bowl and robe, entered \textsanskrit{Rājagaha} for alms. 

Then\marginnote{3.2} it occurred to him, “It’s too early to wander for alms in \textsanskrit{Rājagaha}. Why don’t I go to see the brahmin \textsanskrit{Moggallāna} the Guardian at his place of work?” 

So\marginnote{4.1} that’s what he did. \textsanskrit{Moggallāna} the Guardian saw Ānanda coming off in the distance and said to him, “Come, Master Ānanda! Welcome, Master Ānanda! It’s been a long time since you took the opportunity to come here. Please, sir, sit down, this seat is ready.” 

Ānanda\marginnote{4.8} sat down on the seat spread out, while \textsanskrit{Moggallāna} took a low seat and sat to one side. Then he said to Ānanda, “Master Ānanda, is there even a single mendicant who has all the same qualities in each and every way as possessed by Master Gotama, the perfected one, the fully awakened Buddha?” 

“No,\marginnote{5.2} brahmin, there is not. For the Blessed One gave rise to the unarisen path, gave birth to the unborn path, and explained the unexplained path. He is the knower of the path, the discoverer of the path, the expert on the path. And now the disciples live following the path; they acquire it later.” 

But\marginnote{6.1} this conversation between Ānanda and \textsanskrit{Moggallāna} the Guardian was left unfinished. 

For\marginnote{6.2} just then the brahmin \textsanskrit{Vassakāra}, a chief minister of Magadha, while supervising the work at \textsanskrit{Rājagaha}, approached Ānanda at \textsanskrit{Moggallāna}’s place of work and exchanged greetings with him. When the greetings and polite conversation were over, he sat down to one side and said to Ānanda, “Master Ānanda, what were you sitting talking about just now? What conversation was left unfinished?” 

So\marginnote{6.5} Ānanda told him of the conversation that they were having when \textsanskrit{Vassakāra} arrived. \textsanskrit{Vassakāra} said: 

“Master\marginnote{7.1} Ānanda, is there even a single mendicant who was appointed by Master Gotama, saying: ‘This one will be your refuge when I have passed away,’ to whom you now turn?” 

“No,\marginnote{7.3} there is not.” 

“But\marginnote{8.1} is there even a single mendicant who has been elected to such a position by the \textsanskrit{Saṅgha} and appointed by several senior mendicants?” 

“No,\marginnote{8.3} there is not.” 

“But\marginnote{9.1} since you lack a refuge, Master Ānanda, what’s the reason for your harmony?” 

“We\marginnote{9.2} don’t lack a refuge, brahmin, we have a refuge. The teaching is our refuge.” 

“But\marginnote{10.1} Master Ānanda, when asked whether there was even a single mendicant—either appointed by the Buddha, or elected by the \textsanskrit{Saṅgha} and appointed by several senior mendicants—who serves as your refuge after the Buddha passed away, to whom you now turn, you replied, ‘No, there is not.’ But you say that the reason for your harmony is that you have the teaching as a refuge. How should I see the meaning of this statement?” 

“The\marginnote{10.14} Blessed One, who knows and sees, the perfected one, the fully awakened Buddha laid down training rules and recited the monastic code for the mendicants. On the day of the sabbath all of us who live in dependence on one village district gather together as one. We invite one who has freshly rehearsed the code to recite it. If anyone remembers an offense or transgression while they’re reciting, we make them act in line with the teachings and in line with the instructions. It’s not the venerables that make us act, it’s the teaching that makes us act.” 

“Master\marginnote{11.1} Ānanda, is there even a single mendicant who you honor, respect, revere, venerate, and rely on?” 

“There\marginnote{11.2} is, brahmin.” 

“But\marginnote{11.3} Master Ānanda, when asked whether there was even a single mendicant—either appointed by the Buddha, or elected by the \textsanskrit{Saṅgha} and appointed by several senior mendicants—who serves as your refuge after the Buddha passed away, to whom you now turn, you replied, ‘No, there is not.’ But when asked whether there is even a single mendicant who you honor, respect, revere, venerate, and rely on, you replied, ‘There is.’ How should I see the meaning of this statement?” 

“There\marginnote{13.1} are ten inspiring things explained by the Blessed One, who knows and sees, the perfected one, the fully awakened Buddha. We honor anyone in whom these things are found. What ten? 

It’s\marginnote{14.1} when a mendicant is ethical, restrained in the monastic code, conducting themselves well and seeking alms in suitable places. Seeing danger in the slightest fault, they keep the rules they’ve undertaken. 

They’re\marginnote{15.1} very learned, remembering and keeping what they’ve learned. These teachings are good in the beginning, good in the middle, and good in the end, meaningful and well-phrased, describing a spiritual practice that’s entirely full and pure. They are very learned in such teachings, remembering them, reinforcing them by recitation, mentally scrutinizing them, and comprehending them theoretically. 

They’re\marginnote{16.1} content with robes, almsfood, lodgings, and medicines and supplies for the sick. 

They\marginnote{17.1} get the four absorptions—blissful meditations in the present life that belong to the higher mind—when they want, without trouble or difficulty. 

They\marginnote{18.1} wield the many kinds of psychic power: multiplying themselves and becoming one again; appearing and disappearing; going unimpeded through a wall, a rampart, or a mountain as if through space; diving in and out of the earth as if it were water; walking on water as if it were earth; flying cross-legged through the sky like a bird; touching and stroking with the hand the sun and moon, so mighty and powerful. They control the body as far as the \textsanskrit{Brahmā} realm. 

With\marginnote{19.1} clairaudience that is purified and superhuman, they hear both kinds of sounds, human and divine, whether near or far. 

They\marginnote{20.1} understand the minds of other beings and individuals, having comprehended them with their own mind. They understand mind with greed as ‘mind with greed’, and mind without greed as ‘mind without greed’. They understand mind with hate … mind without hate … mind with delusion … mind without delusion … constricted mind … scattered mind … expansive mind … unexpansive mind … mind that is not supreme … mind that is supreme … mind immersed in \textsanskrit{samādhi} … mind not immersed in \textsanskrit{samādhi} … freed mind … They understand unfreed mind as ‘unfreed mind’. 

They\marginnote{21.1} recollect many kinds of past lives. That is: one, two, three, four, five, ten, twenty, thirty, forty, fifty, a hundred, a thousand, a hundred thousand rebirths; many eons of the world contracting, many eons of the world expanding, many eons of the world contracting and expanding. They remember: ‘There, I was named this, my clan was that, I looked like this, and that was my food. This was how I felt pleasure and pain, and that was how my life ended. When I passed away from that place I was reborn somewhere else. There, too, I was named this, my clan was that, I looked like this, and that was my food. This was how I felt pleasure and pain, and that was how my life ended. When I passed away from that place I was reborn here.’ And so they recollect their many kinds of past lives, with features and details. 

With\marginnote{22.1} clairvoyance that is purified and superhuman, they see sentient beings passing away and being reborn—inferior and superior, beautiful and ugly, in a good place or a bad place. They understand how sentient beings are reborn according to their deeds. 

They\marginnote{23.1} realize the undefiled freedom of heart and freedom by wisdom in this very life. And they live having realized it with their own insight due to the ending of defilements. 

These\marginnote{23.2} are the ten inspiring things explained by the Blessed One, who knows and sees, the perfected one, the fully awakened Buddha. We honor anyone in whom these things are found, and rely on them.” 

When\marginnote{24.1} he had spoken, \textsanskrit{Vassakāra} addressed General Upananda, “What do you think, general? Do these venerables honor, respect, revere, and venerate those who are worthy?” 

“Indeed\marginnote{24.4} they do. For if these venerables were not to honor, respect, revere, and venerate such a person, then who exactly would they honor?” 

Then\marginnote{25.1} \textsanskrit{Vassakāra} said to Ānanda, “Where are you staying at present?” 

“In\marginnote{25.3} the Bamboo Grove, brahmin.” 

“I\marginnote{25.4} hope the Bamboo Grove is delightful, quiet and still, far from the madding crowd, remote from human settlements, and fit for retreat?” 

“Indeed\marginnote{25.5} it is, brahmin. And it is like that owing to such protectors and guardians as yourself.” 

“Surely,\marginnote{25.6} Master Ānanda, it is owing to the venerables who meditate, making a habit of meditating. For the venerables do in fact meditate and make a habit of meditating. 

This\marginnote{25.8} one time, Master Ānanda, Master Gotama was staying near \textsanskrit{Vesālī}, at the Great Wood, in the hall with the peaked roof. So I went there to see him. And there he spoke about meditation in many ways. He meditated, and made a habit of meditating. And he praised all kinds of meditation.” 

“No,\marginnote{26.1} brahmin, the Buddha did not praise all kinds of meditation, nor did he dispraise all kinds of meditation. And what kind of meditation did he not praise? It’s when someone’s heart is overcome and mired in sensual desire, and they don’t truly understand the escape from sensual desire that has arisen. Harboring sensual desire within they meditate and concentrate and contemplate and ruminate. Their heart is overcome and mired in ill will … dullness and drowsiness … restlessness and remorse … doubt, and they don’t truly know and see the escape from doubt that has arisen. Harboring doubt within they meditate and concentrate and contemplate and ruminate. The Buddha didn’t praise this kind of meditation. 

And\marginnote{27.1} what kind of meditation did he praise? It’s when a mendicant, quite secluded from sensual pleasures, secluded from unskillful qualities, enters and remains in the first absorption, which has the rapture and bliss born of seclusion, while placing the mind and keeping it connected. As the placing of the mind and keeping it connected are stilled, they enter and remain in the second absorption, which has the rapture and bliss born of immersion, with internal clarity and confidence, and unified mind, without placing the mind and keeping it connected. And with the fading away of rapture, they enter and remain in the third absorption, where they meditate with equanimity, mindful and aware, personally experiencing the bliss of which the noble ones declare, ‘Equanimous and mindful, one meditates in bliss.’ Giving up pleasure and pain, and ending former happiness and sadness, they enter and remain in the fourth absorption, without pleasure or pain, with pure equanimity and mindfulness. The Buddha praised this kind of meditation.” 

“Well,\marginnote{28.1} Master Ānanda, it seems that Master Gotama criticized the kind of meditation that deserves criticism and praised that deserving of praise. Well, now, Master Ānanda, I must go. I have many duties, and much to do.” 

“Please,\marginnote{28.4} brahmin, go at your convenience.” 

Then\marginnote{28.5} \textsanskrit{Vassakāra} the brahmin, having approved and agreed with what Venerable Ānanda said, got up from his seat and left. 

Soon\marginnote{29.1} after he had left, \textsanskrit{Moggallāna} the Guardian said to Ānanda, “Master Ānanda, you still haven’t answered my question.” 

“But\marginnote{29.3} brahmin, didn’t I say: ‘There is no single mendicant who has all the same qualities in each and every way as possessed by Master Gotama, the perfected one, the fully awakened Buddha. For the Blessed One gave rise to the unarisen path, gave birth to the unborn path, and explained the unexplained path. He is the knower of the path, the discoverer of the path, the expert on the path. And now the disciples live following the path; they acquire it later.’” 

%
\section*{{\suttatitleacronym MN 109}{\suttatitletranslation The Longer Discourse on the Full-Moon Night }{\suttatitleroot Mahāpuṇṇamasutta}}
\addcontentsline{toc}{section}{\tocacronym{MN 109} \toctranslation{The Longer Discourse on the Full-Moon Night } \tocroot{Mahāpuṇṇamasutta}}
\markboth{The Longer Discourse on the Full-Moon Night }{Mahāpuṇṇamasutta}
\extramarks{MN 109}{MN 109}

\scevam{So\marginnote{1.1} I have heard. }At one time the Buddha was staying near \textsanskrit{Sāvatthī} in the Eastern Monastery, the stilt longhouse of \textsanskrit{Migāra}’s mother. 

Now,\marginnote{2.1} at that time it was the sabbath—the full moon on the fifteenth day—and the Buddha was sitting in the open surrounded by the \textsanskrit{Saṅgha} of monks. 

Then\marginnote{3.1} one of the mendicants got up from their seat, arranged their robe over one shoulder, raised their joined palms toward the Buddha, and said, “I’d like to ask the Buddha about a certain point, if you’d take the time to answer.” 

“Well\marginnote{3.3} then, mendicant, take your own seat and ask what you wish.” 

That\marginnote{3.4} mendicant took his seat and said to the Buddha: 

“Sir,\marginnote{4.1} are these the five grasping aggregates: form, feeling, perception, choices, and consciousness?” 

“Yes,\marginnote{4.3} they are,” replied the Buddha. 

Saying\marginnote{4.5} “Good, sir”, that mendicant approved and agreed with what the Buddha said. Then he asked another question: 

“But\marginnote{5.1} sir, what is the root of these five grasping aggregates?” 

“These\marginnote{5.2} five grasping aggregates are rooted in desire.” 

“But\marginnote{6.1} sir, is that grasping the exact same thing as the five grasping aggregates? Or is grasping one thing and the five grasping aggregates another?” 

“Neither.\marginnote{6.2} Rather, the desire and greed for them is the grasping there.” 

“But\marginnote{7.1} sir, can there be different kinds of desire and greed for the five grasping aggregates?” 

“There\marginnote{7.2} can,” said the Buddha. “It’s when someone thinks: ‘In the future, may I be of such form, such feeling, such perception, such choices, and such consciousness!’ That’s how there can be different kinds of desire and greed for the five grasping aggregates.” 

“Sir,\marginnote{8.1} what is the scope of the term ‘aggregates’ as applied to the aggregates?” 

“Any\marginnote{8.2} kind of form at all—past, future, or present; internal or external; coarse or fine; inferior or superior; far or near: this is called the aggregate of form. Any kind of feeling at all … Any kind of perception at all … Any kind of choices at all … Any kind of consciousness at all—past, future, or present; internal or external; coarse or fine; inferior or superior; far or near: this is called the aggregate of consciousness. That’s the scope of the term ‘aggregates’ as applied to the aggregates.” 

“What\marginnote{9.1} is the cause, sir, what is the reason why the aggregate of form is found? What is the cause, what is the reason why the aggregate of feeling … perception … choices … consciousness is found?” 

“The\marginnote{9.6} four primary elements are the reason why the aggregate of form is found. Contact is the reason why the aggregates of feeling … perception … and choices are found. Name and form are the reasons why the aggregate of consciousness is found.” 

“But\marginnote{10.1} sir, how does identity view come about?” 

“It’s\marginnote{10.2} when an unlearned ordinary person has not seen the noble ones, and is neither skilled nor trained in the teaching of the noble ones. They’ve not seen good persons, and are neither skilled nor trained in the teaching of the good persons. They regard form as self, self as having form, form in self, or self in form. They regard feeling as self, self as having feeling, feeling in self, or self in feeling. They regard perception as self, self as having perception, perception in self, or self in perception. They regard choices as self, self as having choices, choices in self, or self in choices. They regard consciousness as self, self as having consciousness, consciousness in self, or self in consciousness. That’s how identity view comes about.” 

“But\marginnote{11.1} sir, how does identity view not come about?” 

“It’s\marginnote{11.2} when a learned noble disciple has seen the noble ones, and is skilled and trained in the teaching of the noble ones. They’ve seen good persons, and are skilled and trained in the teaching of the good persons. They don’t regard form as self, self as having form, form in self, or self in form. They don’t regard feeling as self, self as having feeling, feeling in self, or self in feeling. They don’t regard perception as self, self as having perception, perception in self, or self in perception. They don’t regard choices as self, self as having choices, choices in self, or self in choices. They don’t regard consciousness as self, self as having consciousness, consciousness in self, or self in consciousness. That’s how identity view does not come about.” 

“Sir,\marginnote{12.1} what’s the gratification, the drawback, and the escape when it comes to form, feeling, perception, choices, and consciousness?” 

“The\marginnote{12.6} pleasure and happiness that arise from form: this is its gratification. That form is impermanent, suffering, and perishable: this is its drawback. Removing and giving up desire and greed for form: this is its escape. The pleasure and happiness that arise from feeling … perception … choices … consciousness: this is its gratification. That consciousness is impermanent, suffering, and perishable: this is its drawback. Removing and giving up desire and greed for consciousness: this is its escape.” 

“Sir,\marginnote{13.1} how does one know and see so that there’s no ego, possessiveness, or underlying tendency to conceit for this conscious body and all external stimuli?” 

“One\marginnote{13.2} truly sees any kind of form at all—past, future, or present; internal or external; coarse or fine; inferior or superior; far or near: \emph{all} form—with right understanding: ‘This is not mine, I am not this, this is not my self.’ One truly sees any kind of feeling … perception … choices … consciousness at all—past, future, or present; internal or external; coarse or fine; inferior or superior; far or near, \emph{all} consciousness—with right understanding: ‘This is not mine, I am not this, this is not my self.’ That’s how to know and see so that there’s no ego, possessiveness, or underlying tendency to conceit for this conscious body and all external stimuli.” 

Now\marginnote{14.1} at that time one of the mendicants had the thought, “So it seems, good sir, that form, feeling, perception, choices, and consciousness are not-self. Then what self will the deeds done by not-self affect?” 

But\marginnote{14.4} the Buddha, knowing what that monk was thinking, addressed the mendicants: “It’s possible that some foolish person here—unknowing and ignorant, their mind dominated by craving—thinks they can overstep the teacher’s instructions. They think: ‘So it seems, good sir, that form, feeling, perception, choices, and consciousness are not-self. Then what self will the deeds done by not-self affect?’ Now, mendicants, you have been educated by me in questioning with regard to all these things in all such cases. 

What\marginnote{15.1} do you think, mendicants? Is form permanent or impermanent?” 

“Impermanent,\marginnote{15.3} sir.” 

“But\marginnote{15.4} if it’s impermanent, is it suffering or happiness?” 

“Suffering,\marginnote{15.5} sir.” 

“But\marginnote{15.6} if it’s impermanent, suffering, and perishable, is it fit to be regarded thus: ‘This is mine, I am this, this is my self’?” 

“No,\marginnote{15.8} sir.” 

“What\marginnote{16.1} do you think, mendicants? Is feeling … perception … choices … consciousness permanent or impermanent?” 

“Impermanent,\marginnote{16.6} sir.” 

“But\marginnote{16.7} if it’s impermanent, is it suffering or happiness?” 

“Suffering,\marginnote{16.8} sir.” 

“But\marginnote{16.9} if it’s impermanent, suffering, and perishable, is it fit to be regarded thus: ‘This is mine, I am this, this is my self’?” 

“No,\marginnote{16.11} sir.” 

“So\marginnote{16.12} you should truly see any kind of form at all—past, future, or present; internal or external; coarse or fine; inferior or superior; far or near: \emph{all} form—with right understanding: ‘This is not mine, I am not this, this is not my self.’ 

You\marginnote{17.1} should truly see any kind of feeling … perception … choices … consciousness at all—past, future, or present; internal or external; coarse or fine; inferior or superior; far or near, \emph{all} consciousness—with right understanding: ‘This is not mine, I am not this, this is not my self.’ 

Seeing\marginnote{17.5} this, a learned noble disciple grows disillusioned with form, feeling, perception, choices, and consciousness. 

Being\marginnote{18.1} disillusioned, desire fades away. When desire fades away they’re freed. When they’re freed, they know they’re freed. 

They\marginnote{18.2} understand: ‘Rebirth is ended, the spiritual journey has been completed, what had to be done has been done, there is no return to any state of existence.’” 

That\marginnote{18.3} is what the Buddha said. Satisfied, the mendicants were happy with what the Buddha said. And while this discourse was being spoken, the minds of sixty mendicants were freed from defilements by not grasping. 

%
\section*{{\suttatitleacronym MN 110}{\suttatitletranslation The Shorter Discourse on the Full-Moon Night }{\suttatitleroot Cūḷapuṇṇamasutta}}
\addcontentsline{toc}{section}{\tocacronym{MN 110} \toctranslation{The Shorter Discourse on the Full-Moon Night } \tocroot{Cūḷapuṇṇamasutta}}
\markboth{The Shorter Discourse on the Full-Moon Night }{Cūḷapuṇṇamasutta}
\extramarks{MN 110}{MN 110}

\scevam{So\marginnote{1.1} I have heard. }At one time the Buddha was staying near \textsanskrit{Sāvatthī} in the Eastern Monastery, the stilt longhouse of \textsanskrit{Migāra}’s mother. 

Now,\marginnote{2.1} at that time it was the sabbath—the full moon on the fifteenth day—and the Buddha was sitting in the open surrounded by the \textsanskrit{Saṅgha} of monks. Then the Buddha looked around the \textsanskrit{Saṅgha} of monks, who were so very silent. He addressed them, “Mendicants, could a bad person know of a bad person: ‘This fellow is a bad person’?” 

“No,\marginnote{3.3} sir.” 

“Good,\marginnote{3.4} mendicants! It’s impossible, it can’t happen, that a bad person could know of a bad person: ‘This fellow is a bad person.’ But could a bad person know of a good person: ‘This fellow is a good person’?” 

“No,\marginnote{3.9} sir.” 

“Good,\marginnote{3.10} mendicants! That too is impossible. A bad person has bad qualities, associates with bad people, and has the intentions, counsel, speech, actions, views, and giving of a bad person. 

And\marginnote{5.1} how does a bad person have bad qualities? It’s when a bad person is faithless, shameless, imprudent, unlearned, lazy, unmindful, and witless. That’s how a bad person has bad qualities. 

And\marginnote{6.1} how does a bad person associate with bad people? It’s when a bad person is a friend and companion of ascetics and brahmins who are faithless, shameless, imprudent, unlearned, lazy, unmindful, and witless. That’s how a bad person associates with bad people. 

And\marginnote{7.1} how does a bad person have the intentions of a bad person? It’s when a bad person intends to hurt themselves, hurt others, and hurt both. That’s how a bad person has the intentions of a bad person. 

And\marginnote{8.1} how does a bad person offer the counsel of a bad person? It’s when a bad person offers counsel that hurts themselves, hurts others, and hurts both. That’s how a bad person offers the counsel of a bad person. 

And\marginnote{9.1} how does a bad person have the speech of a bad person? It’s when a bad person uses speech that’s false, divisive, harsh, and nonsensical. That’s how a bad person has the speech of a bad person. 

And\marginnote{10.1} how does a bad person have the action of a bad person? It’s when a bad person kills living creatures, steals, and commits sexual misconduct. That’s how a bad person has the actions of a bad person. 

And\marginnote{11.1} how does a bad person have the view of a bad person? It’s when a bad person has such a view: ‘There’s no meaning in giving, sacrifice, or offerings. There’s no fruit or result of good and bad deeds. There’s no afterlife. There’s no such thing as mother and father, or beings that are reborn spontaneously. And there’s no ascetic or brahmin who is well attained and practiced, and who describes the afterlife after realizing it with their own insight.’ That’s how a bad person has the view of a bad person. 

And\marginnote{12.1} how does a bad person give the gifts of a bad person? It’s when a bad person gives a gift carelessly, not with their own hand, and thoughtlessly. They give the dregs, and they give without consideration for consequences. That’s how a bad person gives the gifts of a bad person. 

That\marginnote{13.1} bad person—who has such bad qualities, frequents bad people, and has the intentions, counsel, speech, actions, views, and giving of a bad person—when their body breaks up, after death, is reborn in the place where bad people are reborn. And what is the place where bad people are reborn? Hell or the animal realm. 

Mendicants,\marginnote{14.1} could a good person know of a good person: ‘This fellow is a good person’?” 

“Yes,\marginnote{14.3} sir.” 

“Good,\marginnote{14.4} mendicants! It is possible that a good person could know of a good person: ‘This fellow is a good person.’ But could a good person know of a bad person: ‘This fellow is a bad person’?” 

“Yes,\marginnote{14.9} sir.” 

“Good,\marginnote{15.1} mendicants! That too is possible. A good person has good qualities, associates with good people, and has the intentions, counsel, speech, actions, views, and giving of a good person. 

And\marginnote{16.1} how does a good person have good qualities? It’s when a good person is faithful, conscientious, prudent, learned, energetic, mindful, and wise. That’s how a good person has good qualities. 

And\marginnote{17.1} how does a good person associate with good people? It’s when a good person is a friend and companion of ascetics and brahmins who are faithful, conscientious, prudent, learned, energetic, mindful, and wise. That’s how a good person associates with good people. 

And\marginnote{18.1} how does a good person have the intentions of a good person? It’s when a good person doesn’t intend to hurt themselves, hurt others, and hurt both. That’s how a good person has the intentions of a good person. 

And\marginnote{19.1} how does a good person offer the counsel of a good person? It’s when a good person offers counsel that doesn’t hurt themselves, hurt others, and hurt both. That’s how a good person offers the counsel of a good person. 

And\marginnote{20.1} how does a good person have the speech of a good person? It’s when a good person refrains from speech that’s false, divisive, harsh, or nonsensical. That’s how a good person has the speech of a good person. 

And\marginnote{21.1} how does a good person have the action of a good person? It’s when a good person refrains from killing living creatures, stealing, and committing sexual misconduct. That’s how a good person has the action of a good person. 

And\marginnote{22.1} how does a good person have the view of a good person? It’s when a good person has such a view: ‘There is meaning in giving, sacrifice, and offerings. There are fruits and results of good and bad deeds. There is an afterlife. There are such things as mother and father, and beings that are reborn spontaneously. And there are ascetics and brahmins who are well attained and practiced, and who describe the afterlife after realizing it with their own insight.’ That’s how a good person has the view of a good person. 

And\marginnote{23.1} how does a good person give the gifts of a good person? It’s when a good person gives a gift carefully, with their own hand, and thoughtfully. They don’t give the dregs, and they give with consideration for consequences. That’s how a good person gives the gifts of a good person. 

That\marginnote{24.1} good person—who has such good qualities, associates with good people, and has the intentions, counsel, speech, actions, views, and giving of a good person—when their body breaks up, after death, is reborn in the place where good people are reborn. And what is the place where good people are reborn? A state of greatness among gods or humans.” 

That\marginnote{25.1} is what the Buddha said. Satisfied, the mendicants were happy with what the Buddha said. 

%
\addtocontents{toc}{\let\protect\contentsline\protect\nopagecontentsline}
\chapter*{The Chapter Beginning with One By One }
\addcontentsline{toc}{chapter}{\tocchapterline{The Chapter Beginning with One By One }}
\addtocontents{toc}{\let\protect\contentsline\protect\oldcontentsline}

%
\section*{{\suttatitleacronym MN 111}{\suttatitletranslation One by One }{\suttatitleroot Anupadasutta}}
\addcontentsline{toc}{section}{\tocacronym{MN 111} \toctranslation{One by One } \tocroot{Anupadasutta}}
\markboth{One by One }{Anupadasutta}
\extramarks{MN 111}{MN 111}

\scevam{So\marginnote{1.1} I have heard. }At one time the Buddha was staying near \textsanskrit{Sāvatthī} in Jeta’s Grove, \textsanskrit{Anāthapiṇḍika}’s monastery. There the Buddha addressed the mendicants, “Mendicants!” 

“Venerable\marginnote{1.5} sir,” they replied. The Buddha said this: 

“\textsanskrit{Sāriputta}\marginnote{2.1} is astute, mendicants. He has great wisdom, widespread wisdom, laughing wisdom, swift wisdom, sharp wisdom, and penetrating wisdom. For a fortnight he practiced discernment of phenomena one by one. And this is how he did it. 

Quite\marginnote{3.1} secluded from sensual pleasures, secluded from unskillful qualities, he entered and remained in the first absorption, which has the rapture and bliss born of seclusion, while placing the mind and keeping it connected. And he distinguished the phenomena in the first absorption one by one: placing and keeping and rapture and bliss and unification of mind; contact, feeling, perception, intention, mind, enthusiasm, decision, energy, mindfulness, equanimity, and attention. He knew those phenomena as they arose, as they remained, and as they went away. He understood: ‘So it seems that these phenomena, not having been, come to be; and having come to be, they flit away.’ And he meditated without attraction or repulsion for those phenomena; independent, untied, liberated, detached, his mind free of limits. He understood: ‘There is an escape beyond.’ And by repeated practice he knew for sure that there is. 

Furthermore,\marginnote{5.1} as the placing of the mind and keeping it connected were stilled, he entered and remained in the second absorption, which has the rapture and bliss born of immersion, with internal clarity and confidence, and unified mind, without placing the mind and keeping it connected. 

And\marginnote{6.1} he distinguished the phenomena in the second absorption one by one: internal confidence and rapture and bliss and unification of mind; contact, feeling, perception, intention, mind, enthusiasm, decision, energy, mindfulness, equanimity, and attention. He knew those phenomena as they arose, as they remained, and as they went away. He understood: ‘So it seems that these phenomena, not having been, come to be; and having come to be, they flit away.’ And he meditated without attraction or repulsion for those phenomena; independent, untied, liberated, detached, his mind free of limits. He understood: ‘There is an escape beyond.’ And by repeated practice he knew for sure that there is. 

Furthermore,\marginnote{7.1} with the fading away of rapture, he entered and remained in the third absorption, where he meditated with equanimity, mindful and aware, personally experiencing the bliss of which the noble ones declare, ‘Equanimous and mindful, one meditates in bliss.’ 

And\marginnote{8.1} he distinguished the phenomena in the third absorption one by one: bliss and mindfulness and awareness and unification of mind; contact, feeling, perception, intention, mind, enthusiasm, decision, energy, mindfulness, equanimity, and attention. He knew those phenomena as they arose, as they remained, and as they went away. He understood: ‘So it seems that these phenomena, not having been, come to be; and having come to be, they flit away.’ And he meditated without attraction or repulsion for those phenomena; independent, untied, liberated, detached, his mind free of limits. He understood: ‘There is an escape beyond.’ And by repeated practice he knew for sure that there is. 

Furthermore,\marginnote{9.1} with the giving up of pleasure and pain, and the ending of former happiness and sadness, he entered and remained in the fourth absorption, without pleasure or pain, with pure equanimity and mindfulness. 

And\marginnote{10.1} he distinguished the phenomena in the fourth absorption one by one: equanimity and neutral feeling and mental unconcern due to tranquility and pure mindfulness and unification of mind; contact, feeling, perception, intention, mind, enthusiasm, decision, energy, mindfulness, equanimity, and attention. He knew those phenomena as they arose, as they remained, and as they went away. He understood: ‘So it seems that these phenomena, not having been, come to be; and having come to be, they flit away.’ And he meditated without attraction or repulsion for those phenomena; independent, untied, liberated, detached, his mind free of limits. He understood: ‘There is an escape beyond.’ And by repeated practice he knew for sure that there is. 

Furthermore,\marginnote{11.1} going totally beyond perceptions of form, with the ending of perceptions of impingement, not focusing on perceptions of diversity, aware that ‘space is infinite’, he entered and remained in the dimension of infinite space. 

And\marginnote{12.1} he distinguished the phenomena in the dimension of infinite space one by one: the perception of the dimension of infinite space and unification of mind; contact, feeling, perception, intention, mind, enthusiasm, decision, energy, mindfulness, equanimity, and attention. He knew those phenomena as they arose, as they remained, and as they went away. He understood: ‘So it seems that these phenomena, not having been, come to be; and having come to be, they flit away.’ And he meditated without attraction or repulsion for those phenomena; independent, untied, liberated, detached, his mind free of limits. He understood: ‘There is an escape beyond.’ And by repeated practice he knew for sure that there is. 

Furthermore,\marginnote{13.1} going totally beyond the dimension of infinite space, aware that ‘consciousness is infinite’, he entered and remained in the dimension of infinite consciousness. 

And\marginnote{14.1} he distinguished the phenomena in the dimension of infinite consciousness one by one: the perception of the dimension of infinite consciousness and unification of mind; contact, feeling, perception, intention, mind, enthusiasm, decision, energy, mindfulness, equanimity, and attention. He knew those phenomena as they arose, as they remained, and as they went away. He understood: ‘So it seems that these phenomena, not having been, come to be; and having come to be, they flit away.’ And he meditated without attraction or repulsion for those phenomena; independent, untied, liberated, detached, his mind free of limits. He understood: ‘There is an escape beyond.’ And by repeated practice he knew for sure that there is. 

Furthermore,\marginnote{15.1} going totally beyond the dimension of infinite consciousness, aware that ‘there is nothing at all’, he entered and remained in the dimension of nothingness. 

And\marginnote{16.1} he distinguished the phenomena in the dimension of nothingness one by one: the perception of the dimension of nothingness and unification of mind; contact, feeling, perception, intention, mind, enthusiasm, decision, energy, mindfulness, equanimity, and attention. He knew those phenomena as they arose, as they remained, and as they went away. He understood: ‘So it seems that these phenomena, not having been, come to be; and having come to be, they flit away.’ And he meditated without attraction or repulsion for those phenomena; independent, untied, liberated, detached, his mind free of limits. He understood: ‘There is an escape beyond.’ And by repeated practice he knew for sure that there is. 

Furthermore,\marginnote{17.1} going totally beyond the dimension of nothingness, he entered and remained in the dimension of neither perception nor non-perception. 

And\marginnote{18.1} he emerged from that attainment with mindfulness. Then he contemplated the phenomena in that attainment that had passed, ceased, and perished: ‘So it seems that these phenomena, not having been, come to be; and having come to be, they flit away.’ And he meditated without attraction or repulsion for those phenomena; independent, untied, liberated, detached, his mind free of limits. He understood: ‘There is an escape beyond.’ And by repeated practice he knew for sure that there is. 

Furthermore,\marginnote{19.1} going totally beyond the dimension of neither perception nor non-perception, he entered and remained in the cessation of perception and feeling. And, having seen with wisdom, his defilements came to an end. 

And\marginnote{20.1} he emerged from that attainment with mindfulness. Then he contemplated the phenomena in that attainment that had passed, ceased, and perished: ‘So it seems that these phenomena, not having been, come to be; and having come to be, they flit away.’ And he meditated without attraction or repulsion for those phenomena; independent, untied, liberated, detached, his mind free of limits. He understood: ‘There is no escape beyond.’ And by repeated practice he knew for sure that there is not. 

And\marginnote{21.1} if there’s anyone of whom it may be rightly said that they have attained mastery and perfection in noble ethics, immersion, wisdom, and freedom, it’s \textsanskrit{Sāriputta}. 

And\marginnote{22.1} if there’s anyone of whom it may be rightly said that they’re the Buddha’s true-born child, born from his mouth, born of the teaching, created by the teaching, heir to the teaching, not the heir in material things, it’s \textsanskrit{Sāriputta}. 

\textsanskrit{Sāriputta}\marginnote{23.1} rightly keeps rolling the supreme Wheel of Dhamma that was rolled forth by the Realized One.” 

That\marginnote{23.2} is what the Buddha said. Satisfied, the mendicants were happy with what the Buddha said. 

%
\section*{{\suttatitleacronym MN 112}{\suttatitletranslation The Sixfold Purification }{\suttatitleroot Chabbisodhanasutta}}
\addcontentsline{toc}{section}{\tocacronym{MN 112} \toctranslation{The Sixfold Purification } \tocroot{Chabbisodhanasutta}}
\markboth{The Sixfold Purification }{Chabbisodhanasutta}
\extramarks{MN 112}{MN 112}

\scevam{So\marginnote{1.1} I have heard. }At one time the Buddha was staying near \textsanskrit{Sāvatthī} in Jeta’s Grove, \textsanskrit{Anāthapiṇḍika}’s monastery. There the Buddha addressed the mendicants, “Mendicants!” 

“Venerable\marginnote{1.5} sir,” they replied. The Buddha said this: 

“Take\marginnote{2.1} a mendicant who declares enlightenment: ‘I understand: “Rebirth is ended, the spiritual journey has been completed, what had to be done has been done, there is no return to any state of existence.”’ 

You\marginnote{3.1} should neither approve nor dismiss that mendicant’s statement. Rather, you should question them: ‘Reverend, these four kinds of expression have been rightly explained by the Blessed One, who knows and sees, the perfected one, the fully awakened Buddha. What four? One speaks of the seen as seen, the heard as heard, the thought as thought, and the known as known. These are the four kinds of expression rightly explained by the Blessed One, who knows and sees, the perfected one, the fully awakened Buddha. How does the venerable know and see regarding these four kinds of expression so that your mind is freed from defilements by not grasping?’ 

For\marginnote{4.1} a mendicant with defilements ended—who has completed the spiritual journey, done what had to be done, laid down the burden, achieved their own goal, utterly ended the fetters of rebirth, and is rightly freed through enlightenment—it is in line with the teaching to answer: ‘Reverends, I live without attraction or repulsion for what is seen; independent, untied, liberated, detached, my mind free of limits. I live without attraction or repulsion for what is heard … thought … or known; independent, untied, liberated, detached, my mind free of limits. That is how I know and see regarding these four kinds of expression so that my mind is freed from defilements by not grasping.’ 

Saying\marginnote{5.1} ‘Good!’ you should applaud and cheer that mendicant’s statement, then ask a further question: 

‘Reverend,\marginnote{5.3} these five grasping aggregates have been rightly explained by the Buddha. What five? That is: the grasping aggregates of form, feeling, perception, choices, and consciousness. These are the five grasping aggregates that have been rightly explained by the Buddha. How does the venerable know and see regarding these five grasping aggregates so that your mind is freed from defilements by not grasping?’ 

For\marginnote{6.1} a mendicant with defilements ended it is in line with the teaching to answer: ‘Reverends, knowing that form is powerless, fading, and unreliable, I understand that my mind is freed through the ending, fading away, cessation, giving away, and letting go of attraction, grasping, mental fixation, insistence, and underlying tendency for form. Knowing that feeling … perception … choices … consciousness is powerless, fading, and unreliable, I understand that my mind is freed through the ending, fading away, cessation, giving away, and letting go of attraction, grasping, mental fixation, insistence, and underlying tendency for consciousness. That is how I know and see regarding these five grasping aggregates so that my mind is freed from defilements by not grasping.’ 

Saying\marginnote{7.1} ‘Good!’ you should applaud and cheer that mendicant’s statement, then ask a further question: 

‘Reverend,\marginnote{7.3} these six elements have been rightly explained by the Buddha. What six? The elements of earth, water, fire, air, space, and consciousness. These are the six elements that have been rightly explained by the Buddha. How does the venerable know and see regarding these six elements so that your mind is freed from defilements by not grasping?’ 

For\marginnote{8.1} a mendicant with defilements ended it is in line with the teaching to answer: ‘Reverends, I’ve not taken the earth element as self, nor is there a self based on the earth element. And I understand that my mind is freed through the ending, fading away, cessation, giving away, and letting go of attraction, grasping, mental fixation, insistence, and underlying tendency based on the earth element. I’ve not taken the water element … fire element … air element … space element … consciousness element as self, nor is there a self based on the consciousness element. And I understand that my mind is freed through the ending of attraction based on the consciousness element. That is how I know and see regarding these six elements so that my mind is freed from defilements by not grasping.’ 

Saying\marginnote{9.1} ‘Good!’ you should applaud and cheer that mendicant’s statement, then ask a further question: 

‘Reverend,\marginnote{9.3} these six interior and exterior sense fields have been rightly explained by the Buddha. What six? The eye and sights, the ear and sounds, the nose and smells, the tongue and tastes, the body and touches, and the mind and thoughts. These are the six interior and exterior sense fields that have been rightly explained by the Buddha. How does the venerable know and see regarding these six interior and exterior sense fields so that your mind is freed from defilements by not grasping?’ 

For\marginnote{10.1} a mendicant with defilements ended it is in line with the teaching to answer: ‘I understand that my mind is freed through the ending, fading away, cessation, giving away, and letting go of desire and greed and relishing and craving; attraction, grasping, mental fixation, insistence, and underlying tendency for the eye, sights, eye consciousness, and things knowable by eye consciousness. I understand that my mind is freed through the ending of desire for the ear … nose … tongue … body … mind, thoughts, mind consciousness, and things knowable by mind consciousness. That is how I know and see regarding these six interior and exterior sense fields so that my mind is freed from defilements by not grasping.’ 

Saying\marginnote{11.1} ‘Good!’ you should applaud and cheer that mendicant’s statement, then ask a further question: 

‘Sir,\marginnote{11.3} how does the venerable know and see so that he has eradicated ego, possessiveness, and underlying tendency to conceit for this conscious body and all external stimuli?’ 

For\marginnote{12.1} a mendicant with defilements ended it is in line with the teaching to answer: ‘Formerly, reverends, when I was still a layperson, I was ignorant. Then the Realized One or one of his disciples taught me the Dhamma. I gained faith in the Realized One, and reflected: 

“Living\marginnote{12.6} in a house is cramped and dirty, but the life of one gone forth is wide open. It’s not easy for someone living at home to lead the spiritual life utterly full and pure, like a polished shell. Why don’t I shave off my hair and beard, dress in ocher robes, and go forth from lay life to homelessness?” 

After\marginnote{13.1} some time I gave up a large or small fortune, and a large or small family circle. I shaved off hair and beard, dressed in ocher robes, and went forth from the lay life to homelessness. Once I had gone forth, I took up the training and livelihood of the mendicants. I gave up killing living creatures, renouncing the rod and the sword. I was scrupulous and kind, living full of compassion for all living beings. I gave up stealing. I took only what’s given, and expected only what’s given. I kept myself clean by not thieving. I gave up unchastity. I became celibate, set apart, avoiding the common practice of sex. I gave up lying. I spoke the truth and stuck to the truth. I was honest and trustworthy, not tricking the world with my words. I gave up divisive speech. I didn’t repeat in one place what I heard in another so as to divide people against each other. Instead, I reconciled those who are divided, supporting unity, delighting in harmony, loving harmony, speaking words that promote harmony. I gave up harsh speech. I spoke in a way that’s mellow, pleasing to the ear, lovely, going to the heart, polite, likable and agreeable to the people. I gave up talking nonsense. My words were timely, true, and meaningful, in line with the teaching and training. I said things at the right time which are valuable, reasonable, succinct, and beneficial. 

I\marginnote{14.1} avoided injuring plants and seeds. I ate in one part of the day, abstaining from eating at night and food at the wrong time. I avoided dancing, singing, music, and seeing shows. I avoided beautifying and adorning myself with garlands, perfumes, and makeup. I avoided high and luxurious beds. I avoided receiving gold and money, raw grains, raw meat, women and girls, male and female bondservants, goats and sheep, chicken and pigs, elephants, cows, horses, and mares, and fields and land. I avoided running errands and messages; buying and selling; falsifying weights, metals, or measures; bribery, fraud, cheating, and duplicity; mutilation, murder, abduction, banditry, plunder, and violence. 

I\marginnote{14.19} became content with robes to look after the body and almsfood to look after the belly. Wherever I went, I set out taking only these things. Like a bird: wherever it flies, wings are its only burden. In the same way, I became content with robes to look after the body and almsfood to look after the belly. Wherever I went, I set out taking only these things. When I had this entire spectrum of noble ethics, I experienced a blameless happiness inside myself. 

When\marginnote{15.1} I saw a sight with my eyes, I didn’t get caught up in the features and details. If the faculty of sight were left unrestrained, bad unskillful qualities of desire and aversion would become overwhelming. For this reason, I practiced restraint, protecting the faculty of sight, and achieving its restraint. When I heard a sound with my ears … When I smelled an odor with my nose … When I tasted a flavor with my tongue … When I felt a touch with my body … When I knew a thought with my mind, I didn’t get caught up in the features and details. If the faculty of the mind were left unrestrained, bad unskillful qualities of desire and aversion would become overwhelming. For this reason, I practiced restraint, protecting the faculty of the mind, and achieving its restraint. When I had this noble sense restraint, I experienced an unsullied bliss inside myself. 

I\marginnote{16.1} acted with situational awareness when going out and coming back; when looking ahead and aside; when bending and extending the limbs; when bearing the outer robe, bowl and robes; when eating, drinking, chewing, and tasting; when urinating and defecating; when walking, standing, sitting, sleeping, waking, speaking, and keeping silent. 

When\marginnote{16.2} I had this noble spectrum of ethics, this noble sense restraint, and this noble mindfulness and situational awareness, I frequented a secluded lodging—a wilderness, the root of a tree, a hill, a ravine, a mountain cave, a charnel ground, a forest, the open air, a heap of straw. After the meal, I returned from almsround, sat down cross-legged with my body straight, and established mindfulness right there. 

Giving\marginnote{17.1} up desire for the world, I meditated with a heart rid of desire, cleansing the mind of desire. Giving up ill will and malevolence, I meditated with a mind rid of ill will, full of compassion for all living beings, cleansing the mind of ill will. Giving up dullness and drowsiness, I meditated with a mind rid of dullness and drowsiness, perceiving light, mindful and aware, cleansing the mind of dullness and drowsiness. Giving up restlessness and remorse, I meditated without restlessness, my mind peaceful inside, cleansing the mind of restlessness and remorse. Giving up doubt, I meditated having gone beyond doubt, not undecided about skillful qualities, cleansing the mind of doubt. 

I\marginnote{18.1} gave up these five hindrances, corruptions of the heart that weaken wisdom. Then, quite secluded from sensual pleasures, secluded from unskillful qualities, I entered and remained in the first absorption, which has the rapture and bliss born of seclusion, while placing the mind and keeping it connected. As the placing of the mind and keeping it connected were stilled, I entered and remained in the second absorption … third absorption … fourth absorption. 

When\marginnote{19.1} my mind had immersed in \textsanskrit{samādhi} like this—purified, bright, flawless, rid of corruptions, pliable, workable, steady, and imperturbable—I extended it toward knowledge of the ending of defilements. I truly understood: “This is suffering” … “This is the origin of suffering” … “This is the cessation of suffering” … “This is the practice that leads to the cessation of suffering”. I truly understood: “These are defilements”… “This is the origin of defilements” … “This is the cessation of defilements” … “This is the practice that leads to the cessation of defilements”. 

Knowing\marginnote{20.1} and seeing like this, my mind was freed from the defilements of sensuality, desire to be reborn, and ignorance. When it was freed, I knew it was freed. I understood: “Rebirth is ended; the spiritual journey has been completed; what had to be done has been done; there is no return to any state of existence.” That is how I know and see so that I have eradicated ego, possessiveness, and underlying tendency to conceit for this conscious body and all external stimuli.’ 

Saying\marginnote{21.1} ‘Good!’ you should applaud and cheer that mendicant’s statement, and then say to them: ‘We are fortunate, reverend, so very fortunate to see a venerable such as yourself as one of our spiritual companions!’” 

That\marginnote{21.5} is what the Buddha said. Satisfied, the mendicants were happy with what the Buddha said. 

%
\section*{{\suttatitleacronym MN 113}{\suttatitletranslation A Good Person }{\suttatitleroot Sappurisasutta}}
\addcontentsline{toc}{section}{\tocacronym{MN 113} \toctranslation{A Good Person } \tocroot{Sappurisasutta}}
\markboth{A Good Person }{Sappurisasutta}
\extramarks{MN 113}{MN 113}

\scevam{So\marginnote{1.1} I have heard. }At one time the Buddha was staying near \textsanskrit{Sāvatthī} in Jeta’s Grove, \textsanskrit{Anāthapiṇḍika}’s monastery. There the Buddha addressed the mendicants, “Mendicants!” 

“Venerable\marginnote{1.5} sir,” they replied. The Buddha said this: 

“Mendicants,\marginnote{2.1} I will teach you the qualities of a good person and the qualities of a bad person. Listen and pay close attention, I will speak.” 

“Yes,\marginnote{2.3} sir,” they replied. The Buddha said this: 

“And\marginnote{3.1} what is a quality of a bad person? Take a bad person who has gone forth from an eminent family. They reflect: ‘I have gone forth from an eminent family, unlike these other mendicants.’ And they glorify themselves and put others down on account of that. This is a quality of a bad person. A good person reflects: ‘It’s not because of one’s eminent family that thoughts of greed, hate, or delusion come to an end. Even if someone has not gone forth from an eminent family, if they practice in line with the teaching, practice properly, and live in line with the teaching, they are worthy of honor and praise for that.’ Keeping only the practice close to their heart, they don’t glorify themselves and put others down on account of their eminent family. This is a quality of a good person. 

Furthermore,\marginnote{4{-}6.1} take a bad person who has gone forth from a great family … from a wealthy family … from an extremely wealthy family. They reflect: ‘I have gone forth from an extremely wealthy family, unlike these other mendicants.’ And they glorify themselves and put others down on account of that. This too is a quality of a bad person. A good person reflects: ‘It’s not because of one’s extremely wealthy family that thoughts of greed, hate, or delusion come to an end. Even if someone has not gone forth from an extremely wealthy family, if they practice in line with the teaching, practice properly, and live in line with the teaching, they are worthy of honor and praise for that.’ Keeping only the practice close to their heart, they don’t glorify themselves and put others down on account of their extremely wealthy family. This too is a quality of a good person. 

Furthermore,\marginnote{7.1} take a bad person who is well-known and famous. They reflect: ‘I’m well-known and famous. These other mendicants are obscure and insignificant.’ And they glorify themselves and put others down on account of that. This too is a quality of a bad person. A good person reflects: ‘It’s not because of one’s fame that thoughts of greed, hate, or delusion come to an end. Even if someone is not well-known and famous, if they practice in line with the teaching, practice properly, and live in line with the teaching, they are worthy of honor and praise for that.’ Keeping only the practice close to their heart, they don’t glorify themselves and put others down on account of their fame. This too is a quality of a good person. 

Furthermore,\marginnote{8.1} take a bad person who receives robes, almsfood, lodgings, and medicines and supplies for the sick. They reflect: ‘I receive robes, almsfood, lodgings, and medicines and supplies for the sick, unlike these other mendicants.’ And they glorify themselves and put others down on account of that. This too is a quality of a bad person. A good person reflects: ‘It’s not because of one’s material possessions that thoughts of greed, hate, or delusion come to an end. Even if someone doesn’t receive robes, almsfood, lodgings, and medicines and supplies for the sick, if they practice in line with the teaching, practice properly, and live in line with the teaching, they are worthy of honor and praise for that.’ Keeping only the practice close to their heart, they don’t glorify themselves and put others down on account of their material possessions. This too is a quality of a good person. 

Furthermore,\marginnote{9.1} take a bad person who is very learned … who is an expert in the monastic law … who is a Dhamma teacher … who dwells in the wilderness … who is a rag robe wearer … who eats only almsfood … who stays at the root of a tree … who stays in a charnel ground … who stays in the open air … who never lies down … who sleeps wherever they lay their mat … who eats in one sitting per day. They reflect: ‘I eat in one sitting per day, unlike these other mendicants.’ And they glorify themselves and put others down on account of that. This too is a quality of a bad person. A good person reflects: ‘It’s not because of eating in one sitting per day that thoughts of greed, hate, or delusion come to an end. Even if someone eats in more than one sitting per day, if they practice in line with the teaching, practice properly, and live in line with the teaching, they are worthy of honor and praise for that.’ Keeping only the practice close to their heart, they don’t glorify themselves and put others down on account of their eating in one sitting per day. This too is a quality of a good person. 

Furthermore,\marginnote{21.1} take a bad person who, quite secluded from sensual pleasures, secluded from unskillful qualities, enters and remains in the first absorption, which has the rapture and bliss born of seclusion, while placing the mind and keeping it connected. They reflect: ‘I have attained the first absorption, unlike these other mendicants.’ And they glorify themselves and put others down on account of that. This too is a quality of a bad person. A good person reflects: ‘The Buddha has spoken of not identifying even with the attainment of the first absorption. For whatever they imagine it is, it turns out to be something else.’ Keeping only non-identification close to their heart, they don’t glorify themselves and put others down on account of their attainment of the first absorption. This too is a quality of a good person. 

Furthermore,\marginnote{22{-}24.1} take a bad person who, as the placing of the mind and keeping it connected are stilled, enters and remains in the second absorption … third absorption … fourth absorption. They reflect: ‘I have attained the fourth absorption, unlike these other mendicants.’ And they glorify themselves and put others down on account of that. This too is a quality of a bad person. A good person reflects: ‘The Buddha has spoken of not identifying even with the attainment of the fourth absorption. For whatever they imagine it is, it turns out to be something else.’ Keeping only non-identification close to their heart, they don’t glorify themselves and put others down on account of their attainment of the fourth absorption. This too is a quality of a good person. 

Furthermore,\marginnote{25.1} take someone who, going totally beyond perceptions of form, with the ending of perceptions of impingement, not focusing on perceptions of diversity, aware that ‘space is infinite’, enters and remains in the dimension of infinite space … the dimension of infinite consciousness … the dimension of nothingness … the dimension of neither perception nor non-perception. They reflect: ‘I have attained the dimension of neither perception nor non-perception, unlike these other mendicants.’ And they glorify themselves and put others down on account of that. This too is a quality of a bad person. A good person reflects: ‘The Buddha has spoken of not identifying even with the attainment of the dimension of neither perception nor non-perception. For whatever they imagine it is, it turns out to be something else.’ Keeping only non-identification close to their heart, they don’t glorify themselves and put others down on account of their attainment of the dimension of neither perception nor non-perception. This too is a quality of a good person. 

Furthermore,\marginnote{29.1} take a good person who, going totally beyond the dimension of neither perception nor non-perception, enters and remains in the cessation of perception and feeling. And, having seen with wisdom, their defilements come to an end. This is a mendicant who does not identify with anything, does not identify regarding anything, does not identify through anything.” 

That\marginnote{29.3} is what the Buddha said. Satisfied, the mendicants were happy with what the Buddha said. 

%
\section*{{\suttatitleacronym MN 114}{\suttatitletranslation What Should and Should Not Be Cultivated }{\suttatitleroot Sevitabbāsevitabbasutta}}
\addcontentsline{toc}{section}{\tocacronym{MN 114} \toctranslation{What Should and Should Not Be Cultivated } \tocroot{Sevitabbāsevitabbasutta}}
\markboth{What Should and Should Not Be Cultivated }{Sevitabbāsevitabbasutta}
\extramarks{MN 114}{MN 114}

\scevam{So\marginnote{1.1} I have heard. }At one time the Buddha was staying near \textsanskrit{Sāvatthī} in Jeta’s Grove, \textsanskrit{Anāthapiṇḍika}’s monastery. There the Buddha addressed the mendicants, “Mendicants!” 

“Venerable\marginnote{1.5} sir,” they replied. The Buddha said this: 

“Mendicants,\marginnote{2.1} I will teach you an exposition of the teaching on what should and should not be cultivated. Listen and pay close attention, I will speak.” 

“Yes,\marginnote{2.3} sir,” they replied. The Buddha said this: 

“I\marginnote{3.1} say that there are two kinds of bodily behavior: that which you should cultivate, and that which you should not cultivate. And each of these is a kind of behavior. 

I\marginnote{3.4} say that there are two kinds of verbal behavior: that which you should cultivate, and that which you should not cultivate. And each of these is a kind of behavior. 

I\marginnote{3.7} say that there are two kinds of mental behavior: that which you should cultivate, and that which you should not cultivate. And each of these is a kind of behavior. 

I\marginnote{3.10} say that there are two ways of giving rise to a thought: that which you should cultivate, and that which you should not cultivate. And each of these is a way of giving rise to a thought. 

I\marginnote{3.13} say that there are two ways of acquiring perception: that which you should cultivate, and that which you should not cultivate. And each of these is a way of acquiring perception. 

I\marginnote{3.16} say that there are two ways of acquiring views: that which you should cultivate, and that which you should not cultivate. And each of these is a way of acquiring views. 

I\marginnote{3.19} say that there are two ways of reincarnating: that which you should cultivate, and that which you should not cultivate. And each of these is a way of reincarnating.” 

When\marginnote{4.1} he said this, Venerable \textsanskrit{Sāriputta} said to the Buddha, “Sir, this is how I understand the detailed meaning of the Buddha’s brief statement. 

‘I\marginnote{5.1} say that there are two kinds of bodily behavior: that which you should cultivate, and that which you should not cultivate. And each of these is a kind of bodily behavior.’ That’s what the Buddha said, but why did he say it? You should not cultivate the kind of bodily behavior which causes unskillful qualities to grow while skillful qualities decline. And you should cultivate the kind of bodily behavior which causes unskillful qualities to decline while skillful qualities grow. 

And\marginnote{5.8} what kind of bodily behavior causes unskillful qualities to grow while skillful qualities decline? It’s when someone kills living creatures. They’re violent, bloody-handed, a hardened killer, merciless to living beings. They steal. With the intention to commit theft, they take the wealth or belongings of others from village or wilderness. They commit sexual misconduct. They have sexual relations with women who have their mother, father, both mother and father, brother, sister, relatives, or clan as guardian. They have sexual relations with a woman who is protected on principle, or who has a husband, or whose violation is punishable by law, or even one who has been garlanded as a token of betrothal. That kind of bodily behavior causes unskillful qualities to grow while skillful qualities decline. 

And\marginnote{5.13} what kind of bodily behavior causes unskillful qualities to decline while skillful qualities grow? It’s when someone gives up killing living creatures. They renounce the rod and the sword. They’re scrupulous and kind, living full of compassion for all living beings. They give up stealing. They don’t, with the intention to commit theft, take the wealth or belongings of others from village or wilderness. They give up sexual misconduct. They don’t have sexual relations with women who have their mother, father, both mother and father, brother, sister, relatives, or clan as guardian. They don’t have sexual relations with a woman who is protected on principle, or who has a husband, or whose violation is punishable by law, or even one who has been garlanded as a token of betrothal. That kind of bodily behavior causes unskillful qualities to decline while skillful qualities grow. ‘I say that there are two kinds of bodily behavior: that which you should cultivate, and that which you should not cultivate. And each of these is a kind of bodily behavior.’ That’s what the Buddha said, and this is why he said it. 

‘I\marginnote{6.1} say that there are two kinds of verbal behavior: that which you should cultivate, and that which you should not cultivate. And each of these is a kind of verbal behavior.’ That’s what the Buddha said, but why did he say it? You should not cultivate the kind of verbal behavior which causes unskillful qualities to grow while skillful qualities decline. And you should cultivate the kind of verbal behavior which causes unskillful qualities to decline while skillful qualities grow. 

And\marginnote{6.8} what kind of verbal behavior causes unskillful qualities to grow while skillful qualities decline? It’s when someone lies. They’re summoned to a council, an assembly, a family meeting, a guild, or to the royal court, and asked to bear witness: ‘Please, mister, say what you know.’ Not knowing, they say ‘I know.’ Knowing, they say ‘I don’t know.’ Not seeing, they say ‘I see.’ And seeing, they say ‘I don’t see.’ So they deliberately lie for the sake of themselves or another, or for some trivial worldly reason. They speak divisively. They repeat in one place what they heard in another so as to divide people against each other. And so they divide those who are harmonious, supporting division, delighting in division, loving division, speaking words that promote division. They speak harshly. They use the kinds of words that are cruel, nasty, hurtful, offensive, bordering on anger, not leading to immersion. They talk nonsense. Their speech is untimely, and is neither factual nor beneficial. It has nothing to do with the teaching or the training. Their words have no value, and are untimely, unreasonable, rambling, and pointless. That kind of verbal behavior causes unskillful qualities to grow while skillful qualities decline. 

And\marginnote{6.14} what kind of verbal behavior causes unskillful qualities to decline while skillful qualities grow? It’s when a certain person gives up lying. They’re summoned to a council, an assembly, a family meeting, a guild, or to the royal court, and asked to bear witness: ‘Please, mister, say what you know.’ Not knowing, they say ‘I don’t know.’ Knowing, they say ‘I know.’ Not seeing, they say ‘I don’t see.’ And seeing, they say ‘I see.’ So they don’t deliberately lie for the sake of themselves or another, or for some trivial worldly reason. They give up divisive speech. They don’t repeat in one place what they heard in another so as to divide people against each other. Instead, they reconcile those who are divided, supporting unity, delighting in harmony, loving harmony, speaking words that promote harmony. They give up harsh speech. They speak in a way that’s mellow, pleasing to the ear, lovely, going to the heart, polite, likable and agreeable to the people. They give up talking nonsense. Their words are timely, true, and meaningful, in line with the teaching and training. They say things at the right time which are valuable, reasonable, succinct, and beneficial. That kind of verbal behavior causes unskillful qualities to decline while skillful qualities grow. ‘I say that there are two kinds of verbal behavior: that which you should cultivate, and that which you should not cultivate. And each of these is a kind of verbal behavior.’ That’s what the Buddha said, and this is why he said it. 

‘I\marginnote{7.1} say that there are two kinds of mental behavior: that which you should cultivate, and that which you should not cultivate. And each of these is a kind of mental behavior.’ That’s what the Buddha said, but why did he say it? You should not cultivate the kind of mental behavior which causes unskillful qualities to grow while skillful qualities decline. And you should cultivate the kind of mental behavior which causes unskillful qualities to decline while skillful qualities grow. 

And\marginnote{7.8} what kind of mental behavior causes unskillful qualities to grow while skillful qualities decline? It’s when someone is covetous. They covet the wealth and belongings of others: ‘Oh, if only their belongings were mine!’ They have ill will and malicious intentions: ‘May these sentient beings be killed, slaughtered, slain, destroyed, or annihilated!’ That kind of mental behavior causes unskillful qualities to grow while skillful qualities decline. 

And\marginnote{7.12} what kind of mental behavior causes unskillful qualities to decline while skillful qualities grow? It’s when someone is content. They don’t covet the wealth and belongings of others: ‘Oh, if only their belongings were mine!’ They have a kind heart and loving intentions: ‘May these sentient beings live free of enmity and ill will, untroubled and happy!’ That kind of mental behavior causes unskillful qualities to decline while skillful qualities grow. ‘I say that there are two kinds of mental behavior: that which you should cultivate, and that which you should not cultivate. And each of these is a kind of mental behavior.’ That’s what the Buddha said, and this is why he said it. 

‘I\marginnote{8.1} say that there are two ways of giving rise to a thought: that which you should cultivate, and that which you should not cultivate. And each of these is a way of giving rise to a thought.’ That’s what the Buddha said, but why did he say it? You should not cultivate the way of giving rise to a thought which causes unskillful qualities to grow while skillful qualities decline. And you should cultivate the way of giving rise to a thought which causes unskillful qualities to decline while skillful qualities grow. 

And\marginnote{8.8} what way of giving rise to a thought causes unskillful qualities to grow while skillful qualities decline? It’s when someone is covetous, and lives with their heart full of covetousness. They are malicious, and live with their heart full of ill will. They’re hurtful, and live with their heart intent on harm. That way of giving rise to a thought causes unskillful qualities to grow while skillful qualities decline. 

And\marginnote{8.13} what way of giving rise to a thought causes unskillful qualities to decline while skillful qualities grow? It’s when someone is content, and lives with their heart full of contentment. They have good will, and live with their heart full of good will. They’re kind, and live with their heart full of kindness. That way of giving rise to a thought causes unskillful qualities to decline while skillful qualities grow. ‘I say that there are two ways of giving rise to a thought: that which you should cultivate, and that which you should not cultivate. And each of these is a way of giving rise to a thought.’ That’s what the Buddha said, and this is why he said it. 

‘I\marginnote{9.1} say that there are two ways of acquiring perception: that which you should cultivate, and that which you should not cultivate. And each of these is a way of acquiring perception.’ That’s what the Buddha said, but why did he say it? You should not cultivate the way of acquiring perception which causes unskillful qualities to grow while skillful qualities decline. And you should cultivate the way of acquiring perception which causes unskillful qualities to decline while skillful qualities grow. 

And\marginnote{9.8} what way of acquiring perception causes unskillful qualities to grow while skillful qualities decline? It’s when someone is covetous, and lives with their perception full of covetousness. They are malicious, and live with their perception full of ill will. They’re hurtful, and live with their perception intent on harm. That way of acquiring perception causes unskillful qualities to grow while skillful qualities decline. 

And\marginnote{9.13} what way of acquiring perception causes unskillful qualities to decline while skillful qualities grow? It’s when someone is content, and lives with their perception full of contentment. They have good will, and live with their perception full of good will. They’re kind, and live with their perception full of kindness. That way of acquiring perception causes unskillful qualities to decline while skillful qualities grow. ‘I say that there are two ways of acquiring perception: that which you should cultivate, and that which you should not cultivate. And each of these is a way of acquiring perception.’ That’s what the Buddha said, and this is why he said it. 

‘I\marginnote{10.1} say that there are two ways of acquiring views: that which you should cultivate, and that which you should not cultivate. And each of these is a way of acquiring views.’ That’s what the Buddha said, but why did he say it? You should not cultivate the way of acquiring views which causes unskillful qualities to grow while skillful qualities decline. And you should cultivate the way of acquiring views which causes unskillful qualities to decline while skillful qualities grow. 

And\marginnote{10.8} what way of acquiring views causes unskillful qualities to grow while skillful qualities decline? It’s when someone has such a view: ‘There’s no meaning in giving, sacrifice, or offerings. There’s no fruit or result of good and bad deeds. There’s no afterlife. There’s no such thing as mother and father, or beings that are reborn spontaneously. And there’s no ascetic or brahmin who is well attained and practiced, and who describes the afterlife after realizing it with their own insight.’ That way of acquiring views causes unskillful qualities to grow while skillful qualities decline. 

And\marginnote{10.12} what way of acquiring views causes unskillful qualities to decline while skillful qualities grow? It’s when someone has such a view: ‘There is meaning in giving, sacrifice, and offerings. There are fruits and results of good and bad deeds. There is an afterlife. There are such things as mother and father, and beings that are reborn spontaneously. And there are ascetics and brahmins who are well attained and practiced, and who describe the afterlife after realizing it with their own insight.’ That way of acquiring views causes unskillful qualities to decline while skillful qualities grow. ‘I say that there are two ways of acquiring views: that which you should cultivate, and that which you should not cultivate. And each of these is a way of acquiring views.’ That’s what the Buddha said, and this is why he said it. 

‘I\marginnote{11.1} say that there are two ways of reincarnating: that which you should cultivate, and that which you should not cultivate. And each of these is a way of reincarnating.’ That’s what the Buddha said, but why did he say it? The way of reincarnating that causes unskillful qualities to grow while skillful qualities decline: you should not cultivate that way of reincarnating. The way of reincarnating that causes unskillful qualities to decline while skillful qualities grow: you should cultivate that way of reincarnating. 

And\marginnote{11.10} what way of reincarnating causes unskillful qualities to grow while skillful qualities decline? Generating rebirth in a hurtful reincarnation, which because of its unpreparedness causes unskillful qualities to grow while skillful qualities decline. And what way of reincarnating causes unskillful qualities to decline while skillful qualities grow? Generating rebirth in a pleasing reincarnation, which because of its preparedness causes unskillful qualities to decline while skillful qualities grow. ‘I say that there are two ways of reincarnating: that which you should cultivate, and that which you should not cultivate. And each of these is a way of reincarnating.’ That’s what the Buddha said, and this is why he said it. 

Sir,\marginnote{12.1} that’s how I understand the detailed meaning of the Buddha’s brief statement.” 

“Good,\marginnote{13.1} good, \textsanskrit{Sāriputta}! It’s good that you understand the detailed meaning of my brief statement in this way.” 

And\marginnote{14.1} the Buddha went on to repeat and endorse Venerable \textsanskrit{Sāriputta}’s explanation in full. Then he went on to explain further: 

“I\marginnote{22.1} say that there are two kinds of sight known by the eye: that which you should cultivate, and that which you should not cultivate. I say that there are two kinds of sound known by the ear … two kinds of smell known by the nose … two kinds of taste known by the tongue … two kinds of touch known by the body … two kinds of thought known by the mind: that which you should cultivate, and that which you should not cultivate.” 

When\marginnote{23.1} he said this, Venerable \textsanskrit{Sāriputta} said to the Buddha: 

“Sir,\marginnote{23.2} this is how I understand the detailed meaning of the Buddha’s brief statement. 

‘I\marginnote{24.1} say that there are two kinds of sight known by the eye: that which you should cultivate, and that which you should not cultivate.’ That’s what the Buddha said, but why did he say it? You should not cultivate the kind of sight known by the eye which causes unskillful qualities to grow while skillful qualities decline. And you should cultivate the kind of sight known by the eye which causes unskillful qualities to decline while skillful qualities grow. ‘I say that there are two kinds of sight known by the eye: that which you should cultivate, and that which you should not cultivate.’ That’s what the Buddha said, and this is why he said it. 

‘I\marginnote{25.1} say that there are two kinds of sound known by the ear … two kinds of smell known by the nose … two kinds of taste known by the tongue … two kinds of touch known by the body … two kinds of thought known by the mind: that which you should cultivate, and that which you should not cultivate.’ That’s what the Buddha said, but why did he say it? 

You\marginnote{30.1} should not cultivate the kind of thought known by the mind which causes unskillful qualities to grow while skillful qualities decline. And you should cultivate the kind of thought known by the mind which causes unskillful qualities to decline while skillful qualities grow. ‘I say that there are two kinds of thought known by the mind: that which you should cultivate, and that which you should not cultivate.’ That’s what the Buddha said, and this is why he said it. Sir, that’s how I understand the detailed meaning of the Buddha’s brief statement.” 

“Good,\marginnote{31.1} good, \textsanskrit{Sāriputta}! It’s good that you understand the detailed meaning of my brief statement in this way.” 

And\marginnote{32{-}36.1} the Buddha went on to repeat and endorse Venerable \textsanskrit{Sāriputta}’s explanation in full. Then he went on to explain further: 

“I\marginnote{39.1} say that there are two kinds of robes: that which you should cultivate, and that which you should not cultivate. I say that there are two kinds of almsfood … lodging … village … town … city … country … person: that which you should cultivate, and that which you should not cultivate.” 

When\marginnote{40.1} he said this, Venerable \textsanskrit{Sāriputta} said to the Buddha: 

“Sir,\marginnote{40.2} this is how I understand the detailed meaning of the Buddha’s brief statement. ‘I say that there are two kinds of robes … almsfood … lodging … village … town … city … country … person: that which you should cultivate, and that which you should not cultivate.’ That’s what the Buddha said, but why did he say it? You should not cultivate the kind of person who causes unskillful qualities to grow while skillful qualities decline. And you should cultivate the kind of person who causes unskillful qualities to decline while skillful qualities grow. ‘I say that there are two kinds of person: those who you should cultivate, and those who you should not cultivate.’ That’s what the Buddha said, and this is why he said it. 

Sir,\marginnote{49.1} that’s how I understand the detailed meaning of the Buddha’s brief statement.” 

“Good,\marginnote{50.1} good, \textsanskrit{Sāriputta}! It’s good that you understand the detailed meaning of my brief statement in this way.” 

And\marginnote{51{-}58.1} the Buddha went on to repeat and endorse Venerable \textsanskrit{Sāriputta}’s explanation in full. Then he added: 

“If\marginnote{60.1} all the aristocrats, brahmins, merchants, and workers were to understand the detailed meaning of my brief statement in this way, it would be for their lasting welfare and happiness. If the whole world—with its gods, \textsanskrit{Māras} and \textsanskrit{Brahmās}, this population with its ascetics and brahmins, gods and humans—was to understand the detailed meaning of my brief statement in this way, it would be for the whole world’s lasting welfare and happiness.” 

That\marginnote{61.1} is what the Buddha said. Satisfied, Venerable \textsanskrit{Sāriputta} was happy with what the Buddha said. 

%
\section*{{\suttatitleacronym MN 115}{\suttatitletranslation Many Elements }{\suttatitleroot Bahudhātukasutta}}
\addcontentsline{toc}{section}{\tocacronym{MN 115} \toctranslation{Many Elements } \tocroot{Bahudhātukasutta}}
\markboth{Many Elements }{Bahudhātukasutta}
\extramarks{MN 115}{MN 115}

\scevam{So\marginnote{1.1} I have heard. }At one time the Buddha was staying near \textsanskrit{Sāvatthī} in Jeta’s Grove, \textsanskrit{Anāthapiṇḍika}’s monastery. There the Buddha addressed the mendicants, “Mendicants!” 

“Venerable\marginnote{1.5} sir,” they replied. The Buddha said this: 

“Whatever\marginnote{2.1} dangers there are, all come from the foolish, not from the astute. Whatever perils there are, all come from the foolish, not from the astute. Whatever hazards there are, all come from the foolish, not from the astute. It’s like a fire that spreads from a hut made of reeds or grass, and burns down even a bungalow, plastered inside and out, draft-free, with latches fastened and windows shuttered. In the same way, whatever dangers there are, all come from the foolish, not from the astute. Whatever perils there are, all come from the foolish, not from the astute. Whatever hazards there are, all come from the foolish, not from the astute. So, the fool is dangerous, but the astute person is safe. The fool is perilous, but the astute person is not. The fool is hazardous, but the astute person is not. There’s no danger, peril, or hazard that comes from the astute. So you should train like this: ‘We shall be astute, we shall be inquirers.’” 

When\marginnote{3.1} he said this, Venerable Ānanda said to the Buddha, “Sir, how is a mendicant qualified to be called ‘astute, an inquirer’?” 

“Ānanda,\marginnote{3.3} it’s when a mendicant is skilled in the elements, in the sense fields, in dependent origination, and in the possible and the impossible. That’s how a mendicant is qualified to be called ‘astute, an inquirer’.” 

“But\marginnote{4.1} sir, how is a mendicant qualified to be called ‘skilled in the elements’?” 

“There\marginnote{4.2} are, Ānanda, these eighteen elements: the elements of the eye, sights, and eye consciousness; the ear, sounds, and ear consciousness; the nose, smells, and nose consciousness; the tongue, tastes, and tongue consciousness; the body, touches, and body consciousness; the mind, thoughts, and mind consciousness. When a mendicant knows and sees these eighteen elements, they’re qualified to be called ‘skilled in the elements’.” 

“But\marginnote{5.1} sir, could there be another way in which a mendicant is qualified to be called ‘skilled in the elements’?” 

“There\marginnote{5.2} could, Ānanda. There are these six elements: the elements of earth, water, fire, air, space, and consciousness. When a mendicant knows and sees these six elements, they’re qualified to be called ‘skilled in the elements’.” 

“But\marginnote{6.1} sir, could there be another way in which a mendicant is qualified to be called ‘skilled in the elements’?” 

“There\marginnote{6.2} could, Ānanda. There are these six elements: the elements of pleasure, pain, happiness, sadness, equanimity, and ignorance. When a mendicant knows and sees these six elements, they’re qualified to be called ‘skilled in the elements’.” 

“But\marginnote{7.1} sir, could there be another way in which a mendicant is qualified to be called ‘skilled in the elements’?” 

“There\marginnote{7.2} could, Ānanda. There are these six elements: the elements of sensuality and renunciation, malice and good will, and cruelty and harmlessness. When a mendicant knows and sees these six elements, they’re qualified to be called ‘skilled in the elements’.” 

“But\marginnote{8.1} sir, could there be another way in which a mendicant is qualified to be called ‘skilled in the elements’?” 

“There\marginnote{8.2} could, Ānanda. There are these three elements: the elements of the sensual realm, the realm of luminous form, and the formless realm. When a mendicant knows and sees these three elements, they’re qualified to be called ‘skilled in the elements’.” 

“But\marginnote{9.1} sir, could there be another way in which a mendicant is qualified to be called ‘skilled in the elements’?” 

“There\marginnote{9.2} could, Ānanda. There are these two elements: the conditioned element and the unconditioned element. When a mendicant knows and sees these two elements, they’re qualified to be called ‘skilled in the elements’.” 

“But\marginnote{10.1} sir, how is a mendicant qualified to be called ‘skilled in the sense fields’?” 

“There\marginnote{10.2} are these six interior and exterior sense fields: the eye and sights, the ear and sounds, the nose and smells, the tongue and tastes, the body and touches, and the mind and thoughts. When a mendicant knows and sees these six interior and exterior sense fields, they’re qualified to be called ‘skilled in the sense fields’.” 

“But\marginnote{11.1} sir, how is a mendicant qualified to be called ‘skilled in dependent origination’?” 

“It’s\marginnote{11.2} when a mendicant understands: ‘When this exists, that is; due to the arising of this, that arises. When this doesn’t exist, that is not; due to the cessation of this, that ceases. That is: ignorance is a condition for choices. Choices are conditions for consciousness. Consciousness is a condition for name and form. Name and form are conditions for the six sense fields. The six sense fields are conditions for contact. Contact is a condition for feeling. Feeling is a condition for craving. Craving is a condition for grasping. Grasping is a condition for continued existence. Continued existence is a condition for rebirth. Rebirth is a condition for old age and death, sorrow, lamentation, pain, sadness, and distress to come to be. That is how this entire mass of suffering originates. When ignorance fades away and ceases with nothing left over, choices cease. When choices cease, consciousness ceases. When consciousness ceases, name and form cease. When name and form cease, the six sense fields cease. When the six sense fields cease, contact ceases. When contact ceases, feeling ceases. When feeling ceases, craving ceases. When craving ceases, grasping ceases. When grasping ceases, continued existence ceases. When continued existence ceases, rebirth ceases. When rebirth ceases, old age and death, sorrow, lamentation, pain, sadness, and distress cease. That is how this entire mass of suffering ceases.’ That’s how a mendicant is qualified to be called ‘skilled in dependent origination’.” 

“But\marginnote{12.1} sir, how is a mendicant qualified to be called ‘skilled in the possible and impossible’?” 

“It’s\marginnote{12.2} when a mendicant understands: ‘It’s impossible for a person accomplished in view to take any condition as permanent. That is not possible. But it’s possible for an ordinary person to take some condition as permanent. That is possible.’ They understand: ‘It’s impossible for a person accomplished in view to take any condition as pleasant. But it’s possible for an ordinary person to take some condition as pleasant.’ They understand: ‘It’s impossible for a person accomplished in view to take anything as self. But it’s possible for an ordinary person to take something as self.’ 

They\marginnote{13.1} understand: ‘It’s impossible for a person accomplished in view to murder their mother. But it’s possible for an ordinary person to murder their mother.’ They understand: ‘It’s impossible for a person accomplished in view to murder their father … or murder a perfected one. But it’s possible for an ordinary person to murder their father … or a perfected one.’ They understand: ‘It’s impossible for a person accomplished in view to injure a Realized One with malicious intent. But it’s possible for an ordinary person to injure a Realized One with malicious intent.’ They understand: ‘It’s impossible for a person accomplished in view to cause a schism in the \textsanskrit{Saṅgha}. But it’s possible for an ordinary person to cause a schism in the \textsanskrit{Saṅgha}.’ They understand: ‘It’s impossible for a person accomplished in view to acknowledge another teacher. But it’s possible for an ordinary person to acknowledge another teacher.’ 

They\marginnote{14.1} understand: ‘It’s impossible for two perfected ones, fully awakened Buddhas to arise in the same solar system at the same time. But it is possible for just one perfected one, a fully awakened Buddha, to arise in one solar system.’ They understand: ‘It’s impossible for two wheel-turning monarchs to arise in the same solar system at the same time. But it is possible for just one wheel-turning monarch to arise in one solar system.’ 

They\marginnote{15.1} understand: ‘It’s impossible for a woman to be a perfected one, a fully awakened Buddha. But it is possible for a man to be a perfected one, a fully awakened Buddha.’ They understand: ‘It’s impossible for a woman to be a wheel-turning monarch. But it is possible for a man to be a wheel-turning monarch.’ They understand: ‘It’s impossible for a woman to perform the role of Sakka, \textsanskrit{Māra}, or \textsanskrit{Brahmā}. But it is possible for a man to perform the role of Sakka, \textsanskrit{Māra}, or \textsanskrit{Brahmā}.’ 

They\marginnote{16.1} understand: ‘It’s impossible for a likable, desirable, agreeable result to come from bad conduct of body, speech, and mind. But it is possible for an unlikable, undesirable, disagreeable result to come from bad conduct of body, speech, and mind.’ 

They\marginnote{17.1} understand: ‘It’s impossible for an unlikable, undesirable, disagreeable result to come from good conduct of body, speech, and mind. But it is possible for a likable, desirable, agreeable result to come from good conduct of body, speech, and mind.’ 

They\marginnote{18.1} understand: ‘It’s impossible that someone who has engaged in bad conduct of body, speech, and mind, could for that reason alone, when their body breaks up, after death, be reborn in a good place, a heavenly realm. But it is possible that someone who has engaged in bad conduct of body, speech, and mind could, for that reason alone, when their body breaks up, after death, be reborn in a place of loss, a bad place, the underworld, hell.’ 

They\marginnote{19.1} understand: ‘It’s impossible that someone who has engaged in good conduct of body, speech, and mind could, for that reason alone, when their body breaks up, after death, be reborn in a place of loss, the underworld, a lower realm, hell. But it is possible that someone who has engaged in good conduct of body, speech, and mind could, for that reason alone, when their body breaks up, after death, be reborn in a good place, a heavenly realm.’ That’s how a mendicant is qualified to be called ‘skilled in the possible and impossible’.” 

When\marginnote{20.1} he said this, Venerable Ānanda said to the Buddha, “It’s incredible, sir, it’s amazing! What is the name of this exposition of the teaching?” 

“In\marginnote{20.4} that case, Ānanda, you may remember this exposition of the teaching as ‘The Many Elements’, or else ‘The Four Cycles’, or else ‘The Mirror of the Teaching’, or else ‘The Drum of the Deathless’, or else ‘The Supreme Victory in Battle’.” 

That\marginnote{20.5} is what the Buddha said. Satisfied, Venerable Ānanda was happy with what the Buddha said. 

%
\section*{{\suttatitleacronym MN 116}{\suttatitletranslation At Isigili }{\suttatitleroot Isigilisutta}}
\addcontentsline{toc}{section}{\tocacronym{MN 116} \toctranslation{At Isigili } \tocroot{Isigilisutta}}
\markboth{At Isigili }{Isigilisutta}
\extramarks{MN 116}{MN 116}

\scevam{So\marginnote{1.1} I have heard. }At one time the Buddha was staying near \textsanskrit{Rājagaha}, on the Isigili Mountain. There the Buddha addressed the mendicants, “Mendicants!” 

“Venerable\marginnote{1.5} sir,” they replied. The Buddha said this: 

“Mendicants,\marginnote{2.1} do you see that Mount \textsanskrit{Vebhāra}?” 

“Yes,\marginnote{2.2} sir.” 

“It\marginnote{2.3} used to have a different label and description. Do you see that Mount \textsanskrit{Paṇḍava}?” 

“Yes,\marginnote{2.5} sir.” 

“It\marginnote{2.6} too used to have a different label and description. Do you see that Mount Vepulla?” 

“Yes,\marginnote{2.8} sir.” 

“It\marginnote{2.9} too used to have a different label and description. Do you see that Mount Vulture’s Peak?” 

“Yes,\marginnote{2.11} sir.” 

“It\marginnote{2.12} too used to have a different label and description. Do you see that Mount Isigili?” 

“Yes,\marginnote{2.14} sir.” 

“It\marginnote{3.1} used to have exactly the same label and description. 

Once\marginnote{3.2} upon a time, five hundred Buddhas awakened for themselves dwelt for a long time on this Isigili. They were seen entering the mountain, but after entering were seen no more. When people noticed this they said: ‘That mountain swallows these hermits!’ That’s how it came to be known as Isigili. 

I\marginnote{3.7} shall declare the names of the Buddhas awakened for themselves; I shall extol the names of the Buddhas awakened for themselves; I shall teach the names of the Buddhas awakened for themselves. Listen and pay close attention, I will speak.” 

“Yes,\marginnote{3.11} sir,” they replied. The Buddha said this: 

“The\marginnote{4.1} Buddhas awakened for themselves who dwelt for a long time on this Isigili were named \textsanskrit{Ariṭṭha}, \textsanskrit{Upariṭṭha}, \textsanskrit{Tagarasikhī}, Yasassin, Sudassana, Piyadassin, \textsanskrit{Gandhāra}, \textsanskrit{Piṇḍola}, \textsanskrit{Upāsabha}, \textsanskrit{Nītha}, Tatha, \textsanskrit{Sutavā}, and \textsanskrit{Bhāvitatta}. 

\begin{verse}%
Those\marginnote{5.1} saintly beings, untroubled, with no need for hope, \\
who each achieved awakening by themselves; \\
hear me extol their names, \\
the supreme persons, free of thorns. 

\textsanskrit{Ariṭṭha},\marginnote{5.5} \textsanskrit{Upariṭṭha}, \textsanskrit{Tagarasikhī}, Yasassin, \\
Sudassana, and Piyadassin the awakened; \\
\textsanskrit{Gandhāra}, \textsanskrit{Piṇḍola}, and \textsanskrit{Upāsabha}, \\
\textsanskrit{Nītha}, Tatha, \textsanskrit{Sutavā}, and \textsanskrit{Bhāvitatta}. 

Sumbha,\marginnote{6.1} Subha, Methula, and \textsanskrit{Aṭṭhama}, \\
and Assumegha, \textsanskrit{Anīgha}, and \textsanskrit{Sudāṭha}, \\
awakened for themselves, enders of the conduit to rebirth. \\
\textsanskrit{Hiṅgū}, and \textsanskrit{Hiṅga} the mighty. 

Two\marginnote{6.5} sages named \textsanskrit{Jāli}, and \textsanskrit{Aṭṭhaka}. \\
Then the Buddha Kosala and \textsanskrit{Subāhu}; \\
Upanemi, Nemi, and Santacitta, \\
right and true, stainless and astute. 

\textsanskrit{Kāḷa}\marginnote{6.9} and \textsanskrit{Upakāḷa}, Vijita and Jita, \\
\textsanskrit{Aṅga} and \textsanskrit{Paṅga}, and Guttijita too; \\
Passin gave up attachment, suffering’s root, \\
while \textsanskrit{Aparājita} defeated \textsanskrit{Māra}’s power. 

Satthar,\marginnote{6.13} Pavattar, \textsanskrit{Sarabhaṅga}, \textsanskrit{Lomahaṁsa}, \\
\textsanskrit{Uccaṅgamāya}, Asita, \textsanskrit{Anāsava}, \\
Manomaya, and Bandhumant the cutter of conceit, \\
and \textsanskrit{Tadādhimutta} the immaculate and resplendent. 

\textsanskrit{Ketumbarāga},\marginnote{6.17} \textsanskrit{Mātaṅga}, and Ariya, \\
then Accuta, \textsanskrit{Accutagāma}, and \textsanskrit{Byāmaka}, \\
\textsanskrit{Sumaṅgala}, Dabbila, \textsanskrit{Supatiṭṭhita}, \\
Asayha, \textsanskrit{Khemābhirata}, and Sorata. 

Durannaya,\marginnote{6.21} \textsanskrit{Saṅgha}, and also Ujjaya, \\
another sage, Sayha of peerless effort. \\
There are twelve Ānandas, Nandas, and Upanandas, \\
and \textsanskrit{Bhāradvāja}, bearing his final body. 

Bodhi,\marginnote{6.25} also \textsanskrit{Mahānāma} the supreme, \\
\textsanskrit{Kesī}, \textsanskrit{Sikhī}, Sundara, and \textsanskrit{Bhāradvāja}, \\
Tissa and Upatissa, who’ve both cut the bonds to rebirth, \\
\textsanskrit{Upasīdarin} and \textsanskrit{Sīdarin}, who’ve both cut off craving. 

\textsanskrit{Maṅgala}\marginnote{6.29} was awakened, free of greed, \\
Usabha cut the net, the root of suffering, \\
\textsanskrit{Upanīta} who attained the state of peace, \\
Uposatha, Sundara, and \textsanskrit{Saccanāma}. 

Jeta,\marginnote{6.33} Jayanta, Paduma, and Uppala; \\
Padumuttara, Rakkhita, and Pabbata, \\
\textsanskrit{Mānatthaddha}, beautiful and free of greed, \\
and the Buddha \textsanskrit{Kaṇha}, his mind well freed. 

These\marginnote{7.1} and other mighty ones awakened for themselves, \\
enders of the conduit to rebirth—\\
honor these great hermits who are fully extinguished, \\
having slipped all chains, limitless.” 

%
\end{verse}

%
\section*{{\suttatitleacronym MN 117}{\suttatitletranslation The Great Forty }{\suttatitleroot Mahācattārīsakasutta}}
\addcontentsline{toc}{section}{\tocacronym{MN 117} \toctranslation{The Great Forty } \tocroot{Mahācattārīsakasutta}}
\markboth{The Great Forty }{Mahācattārīsakasutta}
\extramarks{MN 117}{MN 117}

\scevam{So\marginnote{1.1} I have heard. }At one time the Buddha was staying near \textsanskrit{Sāvatthī} in Jeta’s Grove, \textsanskrit{Anāthapiṇḍika}’s monastery. There the Buddha addressed the mendicants, “Mendicants!” 

“Venerable\marginnote{1.5} sir,” they replied. The Buddha said this: 

“Mendicants,\marginnote{2.1} I will teach you noble right immersion with its vital conditions and its prerequisites. Listen and pay close attention, I will speak.” 

“Yes,\marginnote{2.3} sir,” they replied. The Buddha said this: 

“And\marginnote{3.1} what is noble right immersion with its vital conditions and its prerequisites? They are: right view, right thought, right speech, right action, right livelihood, right effort, and right mindfulness. Unification of mind with these seven factors as prerequisites is called noble right immersion with its vital conditions and also with its prerequisites. 

In\marginnote{4.1} this context, right view comes first. And how does right view come first? When you understand wrong view as wrong view and right view as right view, that’s your right view. 

And\marginnote{5.1} what is wrong view? ‘There’s no meaning in giving, sacrifice, or offerings. There’s no fruit or result of good and bad deeds. There’s no afterlife. There’s no such thing as mother and father, or beings that are reborn spontaneously. And there’s no ascetic or brahmin who is well attained and practiced, and who describes the afterlife after realizing it with their own insight.’ This is wrong view. 

And\marginnote{6.1} what is right view? Right view is twofold, I say. There is right view that is accompanied by defilements, has the attributes of good deeds, and ripens in attachment. And there is right view that is noble, undefiled, transcendent, a factor of the path. 

And\marginnote{7.1} what is right view that is accompanied by defilements, has the attributes of good deeds, and ripens in attachment? ‘There is meaning in giving, sacrifice, and offerings. There are fruits and results of good and bad deeds. There is an afterlife. There are such things as mother and father, and beings that are reborn spontaneously. And there are ascetics and brahmins who are well attained and practiced, and who describe the afterlife after realizing it with their own insight.’ This is right view that is accompanied by defilements, has the attributes of good deeds, and ripens in attachment. 

And\marginnote{8.1} what is right view that is noble, undefiled, transcendent, a factor of the path? It’s the wisdom—the faculty of wisdom, the power of wisdom, the awakening factor of investigation of principles, and right view as a factor of the path—in one of noble mind and undefiled mind, who possesses the noble path and develops the noble path. This is called right view that is noble, undefiled, transcendent, a factor of the path. 

They\marginnote{9.1} make an effort to give up wrong view and embrace right view: that’s their right effort. Mindfully they give up wrong view and take up right view: that’s their right mindfulness. So these three things keep running and circling around right view, namely: right view, right effort, and right mindfulness. 

In\marginnote{10.1} this context, right view comes first. And how does right view come first? When you understand wrong thought as wrong thought and right thought as right thought, that’s your right view. 

And\marginnote{11.1} what is wrong thought? Thoughts of sensuality, of malice, and of cruelty. This is wrong thought. 

And\marginnote{12.1} what is right thought? Right thought is twofold, I say. There is right thought that is accompanied by defilements, has the attributes of good deeds, and ripens in attachment. And there is right thought that is noble, undefiled, transcendent, a factor of the path. 

And\marginnote{13.1} what is right thought that is accompanied by defilements, has the attributes of good deeds, and ripens in attachment? Thoughts of renunciation, good will, and harmlessness. This is right thought that is accompanied by defilements. 

And\marginnote{14.1} what is right thought that is noble, undefiled, transcendent, a factor of the path? It’s the thinking—the placing of the mind, thought, applying, application, implanting of the mind, verbal processes—in one of noble mind and undefiled mind, who possesses the noble path and develops the noble path. This is right thought that is noble. 

They\marginnote{15.1} make an effort to give up wrong thought and embrace right thought: that’s their right effort. Mindfully they give up wrong thought and take up right thought: that’s their right mindfulness. So these three things keep running and circling around right thought, namely: right view, right effort, and right mindfulness. 

In\marginnote{16.1} this context, right view comes first. And how does right view come first? When you understand wrong speech as wrong speech and right speech as right speech, that’s your right view. 

And\marginnote{17.1} what is wrong speech? Speech that’s false, divisive, harsh, or nonsensical. This is wrong speech. 

And\marginnote{18.1} what is right speech? Right speech is twofold, I say. There is right speech that is accompanied by defilements, has the attributes of good deeds, and ripens in attachment. And there is right speech that is noble, undefiled, transcendent, a factor of the path. 

And\marginnote{19.1} what is right speech that is accompanied by defilements, has the attributes of good deeds, and ripens in attachment? The refraining from lying, divisive speech, harsh speech, and talking nonsense. This is right speech that is accompanied by defilements. 

And\marginnote{20.1} what is right speech that is noble, undefiled, transcendent, a factor of the path? It’s the desisting, abstaining, abstinence, and refraining from the four kinds of bad verbal conduct in one of noble mind and undefiled mind, who possesses the noble path and develops the noble path. This is right speech that is noble. 

They\marginnote{21.1} make an effort to give up wrong speech and embrace right speech: that’s their right effort. Mindfully they give up wrong speech and take up right speech: that’s their right mindfulness. So these three things keep running and circling around right speech, namely: right view, right effort, and right mindfulness. 

In\marginnote{22.1} this context, right view comes first. And how does right view come first? When you understand wrong action as wrong action and right action as right action, that’s your right view. 

And\marginnote{23.1} what is wrong action? Killing living creatures, stealing, and sexual misconduct. This is wrong action. 

And\marginnote{24.1} what is right action? Right action is twofold, I say. There is right action that is accompanied by defilements, has the attributes of good deeds, and ripens in attachment. And there is right action that is noble, undefiled, transcendent, a factor of the path. 

And\marginnote{25.1} what is right action that is accompanied by defilements, has the attributes of good deeds, and ripens in attachment? Refraining from killing living creatures, stealing, and sexual misconduct. This is right action that is accompanied by defilements. 

And\marginnote{26.1} what is right action that is noble, undefiled, transcendent, a factor of the path? It’s the desisting, abstaining, abstinence, and refraining from the three kinds of bad bodily conduct in one of noble mind and undefiled mind, who possesses the noble path and develops the noble path. This is right action that is noble. 

They\marginnote{27.1} make an effort to give up wrong action and embrace right action: that’s their right effort. Mindfully they give up wrong action and take up right action: that’s their right mindfulness. So these three things keep running and circling around right action, namely: right view, right effort, and right mindfulness. 

In\marginnote{28.1} this context, right view comes first. And how does right view come first? When you understand wrong livelihood as wrong livelihood and right livelihood as right livelihood, that’s your right view. 

And\marginnote{29.1} what is wrong livelihood? Deceit, flattery, hinting, and belittling, and using material possessions to chase after other material possessions. This is wrong livelihood. 

And\marginnote{30.1} what is right livelihood? Right livelihood is twofold, I say. There is right livelihood that is accompanied by defilements, has the attributes of good deeds, and ripens in attachment. And there is right livelihood that is noble, undefiled, transcendent, a factor of the path. 

And\marginnote{31.1} what is right livelihood that is accompanied by defilements, has the attributes of good deeds, and ripens in attachment? It’s when a noble disciple gives up wrong livelihood and earns a living by right livelihood. This is right livelihood that is accompanied by defilements. 

And\marginnote{32.1} what is right livelihood that is noble, undefiled, transcendent, a factor of the path? It’s the desisting, abstaining, abstinence, and refraining from wrong livelihood in one of noble mind and undefiled mind, who possesses the noble path and develops the noble path. This is right livelihood that is noble. 

They\marginnote{33.1} make an effort to give up wrong livelihood and embrace right livelihood: that’s their right effort. Mindfully they give up wrong livelihood and take up right livelihood: that’s their right mindfulness. So these three things keep running and circling around right livelihood, namely: right view, right effort, and right mindfulness. 

In\marginnote{34.1} this context, right view comes first. And how does right view come first? Right view gives rise to right thought. Right thought gives rise to right speech. Right speech gives rise to right action. Right action gives rise to right livelihood. Right livelihood gives rise to right effort. Right effort gives rise to right mindfulness. Right mindfulness gives rise to right immersion. Right immersion gives rise to right knowledge. Right knowledge gives rise to right freedom. So the trainee has eight factors, while the perfected one has ten factors. And here too, the eradication of many bad, unskillful qualities is fully developed due to right knowledge. 

In\marginnote{35.1} this context, right view comes first. And how does right view come first? For one of right view, wrong view is worn away. And the many bad, unskillful qualities that arise because of wrong view are worn away. And because of right view, many skillful qualities are fully developed. For one of right thought, wrong thought is worn away. … For one of right speech, wrong speech is worn away. … For one of right action, wrong action is worn away. … For one of right livelihood, wrong livelihood is worn away. … For one of right effort, wrong effort is worn away. … For one of right mindfulness, wrong mindfulness is worn away. … For one of right immersion, wrong immersion is worn away. … For one of right knowledge, wrong knowledge is worn away. … For one of right freedom, wrong freedom is worn away. And the many bad, unskillful qualities that arise because of wrong freedom are worn away. And because of right freedom, many skillful qualities are fully developed. 

So\marginnote{36.1} there are twenty on the side of the skillful, and twenty on the side of the unskillful. This exposition of the teaching on the Great Forty has been rolled forth. And it cannot be rolled back by any ascetic or brahmin or god or \textsanskrit{Māra} or \textsanskrit{Brahmā} or by anyone in the world. 

If\marginnote{37.1} any ascetic or brahmin imagines they can criticize and reject the exposition of the teaching on the Great Forty, they deserve rebuke and criticism on ten legitimate grounds in the present life. If such a gentleman criticizes right view, they praise and honor the ascetics and brahmins who have wrong view. If they criticize right thought … right speech … right action … right livelihood … right effort … right mindfulness … right immersion … right knowledge … right freedom, they praise and honor the ascetics and brahmins who have wrong freedom. If any ascetic or brahmin imagines they can criticize and reject the exposition of the teaching on the Great Forty, they deserve rebuke and criticism on these ten legitimate grounds in the present life. 

Even\marginnote{38.1} those wanderers of the past, Vassa and \textsanskrit{Bhañña} of \textsanskrit{Ukkalā}, who taught the doctrines of no-cause, inaction, and nihilism, didn’t imagine that the Great Forty should be criticized or rejected. Why is that? For fear of blame, attack, and condemnation.” 

That\marginnote{38.4} is what the Buddha said. Satisfied, the mendicants were happy with what the Buddha said. 

%
\section*{{\suttatitleacronym MN 118}{\suttatitletranslation Mindfulness of Breathing }{\suttatitleroot Ānāpānassatisutta}}
\addcontentsline{toc}{section}{\tocacronym{MN 118} \toctranslation{Mindfulness of Breathing } \tocroot{Ānāpānassatisutta}}
\markboth{Mindfulness of Breathing }{Ānāpānassatisutta}
\extramarks{MN 118}{MN 118}

\scevam{So\marginnote{1.1} I have heard. }At one time the Buddha was staying near \textsanskrit{Sāvatthī} in the Eastern Monastery, the stilt longhouse of \textsanskrit{Migāra}’s mother, together with several well-known senior disciples, such as the venerables \textsanskrit{Sāriputta}, \textsanskrit{Mahāmoggallāna}, \textsanskrit{Mahākassapa}, \textsanskrit{Mahākaccāna}, \textsanskrit{Mahākoṭṭhita}, \textsanskrit{Mahākappina}, \textsanskrit{Mahācunda}, Anuruddha, Revata, Ānanda, and others. 

Now\marginnote{2.1} at that time the senior mendicants were advising and instructing the junior mendicants. Some senior mendicants instructed ten mendicants, while some instructed twenty, thirty, or forty. Being instructed by the senior mendicants, the junior mendicants realized a higher distinction than they had before. 

Now,\marginnote{3.1} at that time it was the sabbath—the full moon on the fifteenth day—and the Buddha was sitting surrounded by the \textsanskrit{Saṅgha} of monks for the invitation to admonish. Then the Buddha looked around the \textsanskrit{Saṅgha} of monks, who were so very silent. He addressed them: 

“I\marginnote{4.1} am satisfied, mendicants, with this practice. My heart is satisfied with this practice. So you should rouse up even more energy for attaining the unattained, achieving the unachieved, and realizing the unrealized. I will wait here in \textsanskrit{Sāvatthī} for the Komudi full moon of the fourth month.” 

Mendicants\marginnote{5.1} from around the country heard about this, and came down to \textsanskrit{Sāvatthī} to see the Buddha. 

And\marginnote{6.1} those senior mendicants instructed the junior mendicants even more. Some senior mendicants instructed ten mendicants, while some instructed twenty, thirty, or forty. Being instructed by the senior mendicants, the junior mendicants realized a higher distinction than they had before. 

Now,\marginnote{7.1} at that time it was the sabbath—the Komudi full moon on the fifteenth day of the fourth month—and the Buddha was sitting in the open surrounded by the \textsanskrit{Saṅgha} of monks. Then the Buddha looked around the \textsanskrit{Saṅgha} of monks, who were so very silent. He addressed them: 

“This\marginnote{8.1} assembly has no nonsense, mendicants, it’s free of nonsense. It consists purely of the essential core. Such is this \textsanskrit{Saṅgha} of monks, such is this assembly! An assembly such as this is worthy of offerings dedicated to the gods, worthy of hospitality, worthy of a religious donation, worthy of greeting with joined palms, and is the supreme field of merit for the world. Such is this \textsanskrit{Saṅgha} of monks, such is this assembly! Even a small gift to an assembly such as this is fruitful, while giving more is even more fruitful. Such is this \textsanskrit{Saṅgha} of monks, such is this assembly! An assembly such as this is rarely seen in the world. Such is this \textsanskrit{Saṅgha} of monks, such is this assembly! An assembly such as this is worth traveling many leagues to see, even if you have to carry your own provisions in a shoulder bag. 

For\marginnote{9.1} in this \textsanskrit{Saṅgha} there are perfected mendicants, who have ended the defilements, completed the spiritual journey, done what had to be done, laid down the burden, achieved their own goal, utterly ended the fetters of rebirth, and are rightly freed through enlightenment. There are such mendicants in this \textsanskrit{Saṅgha}. 

In\marginnote{10.1} this \textsanskrit{Saṅgha} there are mendicants who, with the ending of the five lower fetters are reborn spontaneously. They are extinguished there, and are not liable to return from that world. There are such mendicants in this \textsanskrit{Saṅgha}. 

In\marginnote{11.1} this \textsanskrit{Saṅgha} there are mendicants who, with the ending of three fetters, and the weakening of greed, hate, and delusion, are once-returners. They come back to this world once only, then make an end of suffering. There are such mendicants in this \textsanskrit{Saṅgha}. 

In\marginnote{12.1} this \textsanskrit{Saṅgha} there are mendicants who, with the ending of three fetters are stream-enterers, not liable to be reborn in the underworld, bound for awakening. There are such mendicants in this \textsanskrit{Saṅgha}. 

In\marginnote{13.1} this \textsanskrit{Saṅgha} there are mendicants who are committed to developing the four kinds of mindfulness meditation … the four right efforts … the four bases of psychic power … the five faculties … the five powers … the seven awakening factors … the noble eightfold path. There are such mendicants in this \textsanskrit{Saṅgha}. In this \textsanskrit{Saṅgha} there are mendicants who are committed to developing the meditation on love … compassion … rejoicing … equanimity … ugliness … impermanence. There are such mendicants in this \textsanskrit{Saṅgha}. In this \textsanskrit{Saṅgha} there are mendicants who are committed to developing the meditation on mindfulness of breathing. 

Mendicants,\marginnote{15.1} when mindfulness of breathing is developed and cultivated it is very fruitful and beneficial. Mindfulness of breathing, when developed and cultivated, fulfills the four kinds of mindfulness meditation. The four kinds of mindfulness meditation, when developed and cultivated, fulfill the seven awakening factors. And the seven awakening factors, when developed and cultivated, fulfill knowledge and freedom. 

And\marginnote{16.1} how is mindfulness of breathing developed and cultivated to be very fruitful and beneficial? 

It’s\marginnote{17.1} when a mendicant has gone to a wilderness, or to the root of a tree, or to an empty hut. They sit down cross-legged, with their body straight, and establish mindfulness right there. Just mindful, they breathe in. Mindful, they breathe out. 

When\marginnote{18.1} breathing in heavily they know: ‘I’m breathing in heavily.’ When breathing out heavily they know: ‘I’m breathing out heavily.’ When breathing in lightly they know: ‘I’m breathing in lightly.’ When breathing out lightly they know: ‘I’m breathing out lightly.’ They practice breathing in experiencing the whole body. They practice breathing out experiencing the whole body. They practice breathing in stilling the body’s motion. They practice breathing out stilling the body’s motion. 

They\marginnote{19.1} practice breathing in experiencing rapture. They practice breathing out experiencing rapture. They practice breathing in experiencing bliss. They practice breathing out experiencing bliss. They practice breathing in experiencing these emotions. They practice breathing out experiencing these emotions. They practice breathing in stilling these emotions. They practice breathing out stilling these emotions. 

They\marginnote{20.1} practice breathing in experiencing the mind. They practice breathing out experiencing the mind. They practice breathing in gladdening the mind. They practice breathing out gladdening the mind. They practice breathing in immersing the mind in \textsanskrit{samādhi}. They practice breathing out immersing the mind in \textsanskrit{samādhi}. They practice breathing in freeing the mind. They practice breathing out freeing the mind. 

They\marginnote{21.1} practice breathing in observing impermanence. They practice breathing out observing impermanence. They practice breathing in observing fading away. They practice breathing out observing fading away. They practice breathing in observing cessation. They practice breathing out observing cessation. They practice breathing in observing letting go. They practice breathing out observing letting go. 

Mindfulness\marginnote{22.1} of breathing, when developed and cultivated in this way, is very fruitful and beneficial. 

And\marginnote{23.1} how is mindfulness of breathing developed and cultivated so as to fulfill the four kinds of mindfulness meditation? 

Whenever\marginnote{24.1} a mendicant knows that they breathe heavily, or lightly, or experiencing the whole body, or stilling the body’s motion—at that time they’re meditating by observing an aspect of the body—keen, aware, and mindful, rid of desire and aversion for the world. For I say that the in-breaths and out-breaths are an aspect of the body. That’s why at that time a mendicant is meditating by observing an aspect of the body—keen, aware, and mindful, rid of desire and aversion for the world. 

Whenever\marginnote{25.1} a mendicant practices breathing while experiencing rapture, or experiencing bliss, or experiencing these emotions, or stilling these emotions—at that time they meditate observing an aspect of feelings—keen, aware, and mindful, rid of desire and aversion for the world. For I say that close attention to the in-breaths and out-breaths is an aspect of feelings. That’s why at that time a mendicant is meditating by observing an aspect of feelings—keen, aware, and mindful, rid of desire and aversion for the world. 

Whenever\marginnote{26.1} a mendicant practices breathing while experiencing the mind, or gladdening the mind, or immersing the mind in \textsanskrit{samādhi}, or freeing the mind—at that time they meditate observing an aspect of the mind—keen, aware, and mindful, rid of desire and aversion for the world. There is no development of mindfulness of breathing for someone who is unmindful and lacks awareness, I say. That’s why at that time a mendicant is meditating by observing an aspect of the mind—keen, aware, and mindful, rid of desire and aversion for the world. 

Whenever\marginnote{27.1} a mendicant practices breathing while observing impermanence, or observing fading away, or observing cessation, or observing letting go—at that time they meditate observing an aspect of principles—keen, aware, and mindful, rid of desire and aversion for the world. Having seen with wisdom the giving up of desire and aversion, they watch over closely with equanimity. That’s why at that time a mendicant is meditating by observing an aspect of principles—keen, aware, and mindful, rid of desire and aversion for the world. 

That’s\marginnote{28.1} how mindfulness of breathing, when developed and cultivated, fulfills the four kinds of mindfulness meditation. 

And\marginnote{29.1} how are the four kinds of mindfulness meditation developed and cultivated so as to fulfill the seven awakening factors? 

Whenever\marginnote{30.1} a mendicant meditates by observing an aspect of the body, at that time their mindfulness is established and lucid. At such a time, a mendicant has activated the awakening factor of mindfulness; they develop it and perfect it. 

As\marginnote{31.1} they live mindfully in this way they investigate, explore, and inquire into that principle with wisdom. At such a time, a mendicant has activated the awakening factor of investigation of principles; they develop it and perfect it. 

As\marginnote{32.1} they investigate principles with wisdom in this way their energy is roused up and unflagging. At such a time, a mendicant has activated the awakening factor of energy; they develop it and perfect it. 

When\marginnote{33.1} they’re energetic, spiritual rapture arises. At such a time, a mendicant has activated the awakening factor of rapture; they develop it and perfect it. 

When\marginnote{34.1} the mind is full of rapture, the body and mind become tranquil. At such a time, a mendicant has activated the awakening factor of tranquility; they develop it and perfect it. 

When\marginnote{35.1} the body is tranquil and they feel bliss, the mind becomes immersed in \textsanskrit{samādhi}. At such a time, a mendicant has activated the awakening factor of immersion; they develop it and perfect it. 

They\marginnote{36.1} closely watch over that mind immersed in \textsanskrit{samādhi}. At such a time, a mendicant has activated the awakening factor of equanimity; they develop it and perfect it. 

Whenever\marginnote{37.1} a mendicant meditates by observing an aspect of feelings … mind … principles, at that time their mindfulness is established and lucid. At such a time, a mendicant has activated the awakening factor of mindfulness … investigation of principles … energy … rapture … tranquility … immersion … equanimity. 

That’s\marginnote{40.1} how the four kinds of mindfulness meditation, when developed and cultivated, fulfill the seven awakening factors. 

And\marginnote{41.1} how are the seven awakening factors developed and cultivated so as to fulfill knowledge and freedom? 

It’s\marginnote{42.1} when a mendicant develops the awakening factors of mindfulness, investigation of principles, energy, rapture, tranquility, immersion, 

and\marginnote{42.7} equanimity, which rely on seclusion, fading away, and cessation, and ripen as letting go. 

That’s\marginnote{43.1} how the seven awakening factors, when developed and cultivated, fulfill knowledge and freedom.” 

That\marginnote{43.2} is what the Buddha said. Satisfied, the mendicants were happy with what the Buddha said. 

%
\section*{{\suttatitleacronym MN 119}{\suttatitletranslation Mindfulness of the Body }{\suttatitleroot Kāyagatāsatisutta}}
\addcontentsline{toc}{section}{\tocacronym{MN 119} \toctranslation{Mindfulness of the Body } \tocroot{Kāyagatāsatisutta}}
\markboth{Mindfulness of the Body }{Kāyagatāsatisutta}
\extramarks{MN 119}{MN 119}

\scevam{So\marginnote{1.1} I have heard. }At one time the Buddha was staying near \textsanskrit{Sāvatthī} in Jeta’s Grove, \textsanskrit{Anāthapiṇḍika}’s monastery. 

Then\marginnote{2.1} after the meal, on return from almsround, several senior mendicants sat together in the assembly hall and this discussion came up among them. 

“It’s\marginnote{2.2} incredible, reverends, it’s amazing, how the Blessed One, who knows and sees, the perfected one, the fully awakened Buddha has said that mindfulness of the body, when developed and cultivated, is very fruitful and beneficial.” 

But\marginnote{2.4} their conversation was left unfinished. Then the Buddha came out of retreat and went to the pavilion. He sat on the seat spread out and addressed the mendicants, “Mendicants, what were you sitting talking about just now? What conversation was left unfinished?” 

So\marginnote{2.7} the mendicants told him what they had been talking about. The Buddha said: 

“And\marginnote{3.1} how, mendicants, is mindfulness of the body developed and cultivated to be very fruitful and beneficial? 

It’s\marginnote{4.1} when a mendicant has gone to a wilderness, or to the root of a tree, or to an empty hut. They sit down cross-legged, with their body straight, and establish mindfulness right there. Just mindful, they breathe in. Mindful, they breathe out. When breathing in heavily they know: ‘I’m breathing in heavily.’ When breathing out heavily they know: ‘I’m breathing out heavily.’ When breathing in lightly they know: ‘I’m breathing in lightly.’ When breathing out lightly they know: ‘I’m breathing out lightly.’ They practice breathing in experiencing the whole body. They practice breathing out experiencing the whole body. They practice breathing in stilling the body’s motion. They practice breathing out stilling the body’s motion. As they meditate like this—diligent, keen, and resolute—memories and thoughts of the lay life are given up. Their mind becomes stilled internally; it settles, unifies, and becomes immersed in \textsanskrit{samādhi}. That’s how a mendicant develops mindfulness of the body. 

Furthermore,\marginnote{5.1} when a mendicant is walking they know ‘I am walking’. When standing they know ‘I am standing’. When sitting they know ‘I am sitting’. And when lying down they know ‘I am lying down’. Whatever posture their body is in, they know it. As they meditate like this—diligent, keen, and resolute—memories and thoughts of the lay life are given up. Their mind becomes stilled internally; it settles, unifies, and becomes immersed in \textsanskrit{samādhi}. That too is how a mendicant develops mindfulness of the body. 

Furthermore,\marginnote{6.1} a mendicant acts with situational awareness when going out and coming back; when looking ahead and aside; when bending and extending the limbs; when bearing the outer robe, bowl and robes; when eating, drinking, chewing, and tasting; when urinating and defecating; when walking, standing, sitting, sleeping, waking, speaking, and keeping silent. As they meditate like this—diligent, keen, and resolute—memories and thoughts of the lay life are given up. Their mind becomes stilled internally; it settles, unifies, and becomes immersed in \textsanskrit{samādhi}. That too is how a mendicant develops mindfulness of the body. 

Furthermore,\marginnote{7.1} a mendicant examines their own body, up from the soles of the feet and down from the tips of the hairs, wrapped in skin and full of many kinds of filth. ‘In this body there is head hair, body hair, nails, teeth, skin, flesh, sinews, bones, bone marrow, kidneys, heart, liver, diaphragm, spleen, lungs, intestines, mesentery, undigested food, feces, bile, phlegm, pus, blood, sweat, fat, tears, grease, saliva, snot, synovial fluid, urine.’ 

It’s\marginnote{7.3} as if there were a bag with openings at both ends, filled with various kinds of grains, such as fine rice, wheat, mung beans, peas, sesame, and ordinary rice. And someone with good eyesight were to open it and examine the contents: ‘These grains are fine rice, these are wheat, these are mung beans, these are peas, these are sesame, and these are ordinary rice.’ In the same way, a mendicant examines their own body, up from the soles of the feet and down from the tips of the hairs, wrapped in skin and full of many kinds of filth. … As they meditate like this—diligent, keen, and resolute—memories and thoughts of the lay life are given up. Their mind becomes stilled internally; it settles, unifies, and becomes immersed in \textsanskrit{samādhi}. That too is how a mendicant develops mindfulness of the body. 

Furthermore,\marginnote{8.1} a mendicant examines their own body, whatever its placement or posture, according to the elements: ‘In this body there is the earth element, the water element, the fire element, and the air element.’ 

It’s\marginnote{8.3} as if a deft butcher or butcher’s apprentice were to kill a cow and sit down at the crossroads with the meat cut into portions. In the same way, a mendicant examines their own body, whatever its placement or posture, according to the elements: ‘In this body there is the earth element, the water element, the fire element, and the air element.’ As they meditate like this—diligent, keen, and resolute—memories and thoughts of the lay life are given up. Their mind becomes stilled internally; it settles, unifies, and becomes immersed in \textsanskrit{samādhi}. That too is how a mendicant develops mindfulness of the body. 

Furthermore,\marginnote{9.1} suppose a mendicant were to see a corpse discarded in a charnel ground. And it had been dead for one, two, or three days, bloated, livid, and festering. They’d compare it with their own body: ‘This body is also of that same nature, that same kind, and cannot go beyond that.’ As they meditate like this—diligent, keen, and resolute—memories and thoughts of the lay life are given up. Their mind becomes stilled internally; it settles, unifies, and becomes immersed in \textsanskrit{samādhi}. That too is how a mendicant develops mindfulness of the body. 

Or\marginnote{10.1} suppose they were to see a corpse discarded in a charnel ground being devoured by crows, hawks, vultures, herons, dogs, tigers, leopards, jackals, and many kinds of little creatures. They’d compare it with their own body: ‘This body is also of that same nature, that same kind, and cannot go beyond that.’ That too is how a mendicant develops mindfulness of the body. 

Furthermore,\marginnote{11{-}14.1} suppose they were to see a corpse discarded in a charnel ground, a skeleton with flesh and blood, held together by sinews … A skeleton without flesh but smeared with blood, and held together by sinews … A skeleton rid of flesh and blood, held together by sinews … Bones rid of sinews scattered in every direction. Here a hand-bone, there a foot-bone, here a shin-bone, there a thigh-bone, here a hip-bone, there a rib-bone, here a back-bone, there an arm-bone, here a neck-bone, there a jaw-bone, here a tooth, there the skull … 

White\marginnote{15{-}17.1} bones, the color of shells … Decrepit bones, heaped in a pile … Bones rotted and crumbled to powder. They’d compare it with their own body: ‘This body is also of that same nature, that same kind, and cannot go beyond that.’ As they meditate like this—diligent, keen, and resolute—memories and thoughts of the lay life are given up. Their mind becomes stilled internally; it settles, unifies, and becomes immersed in \textsanskrit{samādhi}. That too is how a mendicant develops mindfulness of the body. 

Furthermore,\marginnote{18.1} a mendicant, quite secluded from sensual pleasures, secluded from unskillful qualities, enters and remains in the first absorption, which has the rapture and bliss born of seclusion, while placing the mind and keeping it connected. They drench, steep, fill, and spread their body with rapture and bliss born of seclusion. There’s no part of the body that’s not spread with rapture and bliss born of seclusion. It’s like when a deft bathroom attendant or their apprentice pours bath powder into a bronze dish, sprinkling it little by little with water. They knead it until the ball of bath powder is soaked and saturated with moisture, spread through inside and out; yet no moisture oozes out. In the same way, they drench, steep, fill, and spread their body with rapture and bliss born of seclusion. There’s no part of the body that’s not spread with rapture and bliss born of seclusion. As they meditate like this—diligent, keen, and resolute—memories and thoughts of the lay life are given up. Their mind becomes stilled internally; it settles, unifies, and becomes immersed in \textsanskrit{samādhi}. That too is how a mendicant develops mindfulness of the body. 

Furthermore,\marginnote{19.1} as the placing of the mind and keeping it connected are stilled, a mendicant enters and remains in the second absorption, which has the rapture and bliss born of immersion, with internal clarity and confidence, and unified mind, without placing the mind and keeping it connected. They drench, steep, fill, and spread their body with rapture and bliss born of immersion. There’s no part of the body that’s not spread with rapture and bliss born of immersion. It’s like a deep lake fed by spring water. There’s no inlet to the east, west, north, or south, and no rainfall to replenish it from time to time. But the stream of cool water welling up in the lake drenches, steeps, fills, and spreads throughout the lake. There’s no part of the lake that’s not spread through with cool water. In the same way, a mendicant drenches, steeps, fills, and spreads their body with rapture and bliss born of immersion. There’s no part of the body that’s not spread with rapture and bliss born of immersion. That too is how a mendicant develops mindfulness of the body. 

Furthermore,\marginnote{20.1} with the fading away of rapture, a mendicant enters and remains in the third absorption. They meditate with equanimity, mindful and aware, personally experiencing the bliss of which the noble ones declare, ‘Equanimous and mindful, one meditates in bliss.’ They drench, steep, fill, and spread their body with bliss free of rapture. There’s no part of the body that’s not spread with bliss free of rapture. It’s like a pool with blue water lilies, or pink or white lotuses. Some of them sprout and grow in the water without rising above it, thriving underwater. From the tip to the root they’re drenched, steeped, filled, and soaked with cool water. There’s no part of them that’s not soaked with cool water. In the same way, a mendicant drenches, steeps, fills, and spreads their body with bliss free of rapture. There’s no part of the body that’s not spread with bliss free of rapture. That too is how a mendicant develops mindfulness of the body. 

Furthermore,\marginnote{21.1} a mendicant, giving up pleasure and pain, and ending former happiness and sadness, enters and remains in the fourth absorption, without pleasure or pain, with pure equanimity and mindfulness. They sit spreading their body through with pure bright mind. There’s no part of the body that’s not filled with pure bright mind. It’s like someone sitting wrapped from head to foot with white cloth. There’s no part of the body that’s not spread over with white cloth. In the same way, they sit spreading their body through with pure bright mind. There’s no part of the body that’s not filled with pure bright mind. As they meditate like this—diligent, keen, and resolute—memories and thoughts of the lay life are given up. Their mind becomes stilled internally; it settles, unifies, and becomes immersed in \textsanskrit{samādhi}. That too is how a mendicant develops mindfulness of the body. 

Anyone\marginnote{22.1} who has developed and cultivated mindfulness of the body includes all of the skillful qualities that play a part in realization. Anyone who brings into their mind the great ocean includes all of the streams that run down into it. In the same way, anyone who has developed and cultivated mindfulness of the body includes all of the skillful qualities that play a part in realization. 

When\marginnote{23.1} a mendicant has not developed or cultivated mindfulness of the body, \textsanskrit{Māra} finds a vulnerability and gets hold of them. Suppose a person were to throw a heavy stone ball on a mound of wet clay. 

What\marginnote{23.3} do you think, mendicants? Would that heavy stone ball find an entry into that mound of wet clay?” 

“Yes,\marginnote{23.5} sir.” 

“In\marginnote{23.6} the same way, when a mendicant has not developed or cultivated mindfulness of the body, \textsanskrit{Māra} finds a vulnerability and gets hold of them. 

Suppose\marginnote{24.1} there was a dried up, withered log. Then a person comes along with a drill-stick, thinking to light a fire and produce heat. 

What\marginnote{24.4} do you think, mendicants? By drilling the stick against that dried up, withered log on dry land far from water, could they light a fire and produce heat?” 

“Yes,\marginnote{24.6} sir.” 

“In\marginnote{24.7} the same way, when a mendicant has not developed or cultivated mindfulness of the body, \textsanskrit{Māra} finds a vulnerability and gets hold of them. 

Suppose\marginnote{25.1} a water jar was placed on a stand, empty and hollow. Then a person comes along with a load of water. 

What\marginnote{25.3} do you think, mendicants? Could that person pour water into the jar?” 

“Yes,\marginnote{25.5} sir.” 

“In\marginnote{25.6} the same way, when a mendicant has not developed or cultivated mindfulness of the body, \textsanskrit{Māra} finds a vulnerability and gets hold of them. 

When\marginnote{26.1} a mendicant has developed and cultivated mindfulness of the body, \textsanskrit{Māra} cannot find a vulnerability and doesn’t get hold of them. 

Suppose\marginnote{26.2} a person were to throw a light ball of string at a door-panel made entirely of hardwood. 

What\marginnote{26.3} do you think, mendicants? Would that light ball of string find an entry into that door-panel made entirely of hardwood?” 

“No,\marginnote{26.5} sir.” 

“In\marginnote{26.6} the same way, when a mendicant has developed and cultivated mindfulness of the body, \textsanskrit{Māra} cannot find a vulnerability and doesn’t get hold of them. 

Suppose\marginnote{27.1} there was a green, sappy log. Then a person comes along with a drill-stick, thinking to light a fire and produce heat. 

What\marginnote{27.4} do you think, mendicants? By drilling the stick against that green, sappy log on dry land far from water, could they light a fire and produce heat?” 

“No,\marginnote{27.6} sir.” 

“In\marginnote{27.7} the same way, when a mendicant has developed and cultivated mindfulness of the body, \textsanskrit{Māra} cannot find a vulnerability and doesn’t get hold of them. Suppose a water jar was placed on a stand, full to the brim so a crow could drink from it. Then a person comes along with a load of water. 

What\marginnote{28.3} do you think, mendicants? Could that person pour water into the jar?” 

“No,\marginnote{28.5} sir.” 

“In\marginnote{28.6} the same way, when a mendicant has developed and cultivated mindfulness of the body, \textsanskrit{Māra} cannot find a vulnerability and doesn’t get hold of them. 

When\marginnote{29.1} a mendicant has developed and cultivated mindfulness of the body, they become capable of realizing anything that can be realized by insight to which they extend the mind, in each and every case. 

Suppose\marginnote{29.2} a water jar was placed on a stand, full to the brim so a crow could drink from it. If a strong man was to pour it on any side, would water pour out?” 

“Yes,\marginnote{29.4} sir.” 

“In\marginnote{29.5} the same way, when a mendicant has developed and cultivated mindfulness of the body, they become capable of realizing anything that can be realized by insight to which they extend the mind, in each and every case. 

Suppose\marginnote{30.1} there was a square, walled lotus pond on level ground, full to the brim so a crow could drink from it. If a strong man was to open the wall on any side, would water pour out?” 

“Yes,\marginnote{30.3} sir.” 

“In\marginnote{30.4} the same way, when a mendicant has developed and cultivated mindfulness of the body, they become capable of realizing anything that can be realized by insight to which they extend the mind, in each and every case. Suppose a chariot stood harnessed to thoroughbreds at a level crossroads, with a goad ready. Then a deft horse trainer, a master charioteer, might mount the chariot, taking the reins in his right hand and goad in the left. He’d drive out and back wherever he wishes, whenever he wishes. In the same way, when a mendicant has developed and cultivated mindfulness of the body, they become capable of realizing anything that can be realized by insight to which they extend the mind, in each and every case. 

You\marginnote{32.1} can expect ten benefits when mindfulness of the body has been cultivated, developed, and practiced, made a vehicle and a basis, kept up, consolidated, and properly implemented. 

They\marginnote{33.1} prevail over desire and discontent, and live having mastered desire and discontent whenever they arose. 

They\marginnote{34.1} prevail over fear and dread, and live having mastered fear and dread whenever they arose. 

They\marginnote{35.1} endure cold, heat, hunger, and thirst; the touch of flies, mosquitoes, wind, sun, and reptiles; rude and unwelcome criticism; and put up with physical pain—sharp, severe, acute, unpleasant, disagreeable, and life-threatening. 

They\marginnote{36.1} get the four absorptions—blissful meditations in the present life that belong to the higher mind—when they want, without trouble or difficulty. 

They\marginnote{37.1} wield the many kinds of psychic power: multiplying themselves and becoming one again … They control the body as far as the \textsanskrit{Brahmā} realm. 

With\marginnote{38.1} clairaudience that is purified and superhuman, they hear both kinds of sounds, human and divine, whether near or far. … 

They\marginnote{39.1} understand the minds of other beings and individuals, having comprehended them with their own mind. … 

They\marginnote{40.1} recollect many kinds of past lives, with features and details. 

With\marginnote{41.1} clairvoyance that is purified and superhuman, they see sentient beings passing away and being reborn—inferior and superior, beautiful and ugly, in a good place or a bad place. They understand how sentient beings are reborn according to their deeds. 

They\marginnote{42.1} realize the undefiled freedom of heart and freedom by wisdom in this very life. And they live having realized it with their own insight due to the ending of defilements. 

You\marginnote{43.1} can expect these ten benefits when mindfulness of the body has been cultivated, developed, and practiced, made a vehicle and a basis, kept up, consolidated, and properly implemented.” 

That\marginnote{43.2} is what the Buddha said. Satisfied, the mendicants were happy with what the Buddha said. 

%
\section*{{\suttatitleacronym MN 120}{\suttatitletranslation Rebirth by Choice }{\suttatitleroot Saṅkhārupapattisutta}}
\addcontentsline{toc}{section}{\tocacronym{MN 120} \toctranslation{Rebirth by Choice } \tocroot{Saṅkhārupapattisutta}}
\markboth{Rebirth by Choice }{Saṅkhārupapattisutta}
\extramarks{MN 120}{MN 120}

\scevam{So\marginnote{1.1} I have heard. }At one time the Buddha was staying near \textsanskrit{Sāvatthī} in Jeta’s Grove, \textsanskrit{Anāthapiṇḍika}’s monastery. There the Buddha addressed the mendicants, “Mendicants!” 

“Venerable\marginnote{1.5} sir,” they replied. The Buddha said this: 

“I\marginnote{2.1} shall teach you rebirth by choice. Listen and pay close attention, I will speak.” 

“Yes,\marginnote{2.3} sir,” they replied. The Buddha said this: 

“Take\marginnote{3.1} a mendicant who has faith, ethics, learning, generosity, and wisdom. They think: ‘If only, when my body breaks up, after death, I would be reborn in the company of well-to-do aristocrats!’ They settle on that thought, stabilize it and develop it. Those choices and meditations of theirs, developed and cultivated like this, lead to rebirth there. This is the path and the practice that leads to rebirth there. 

Furthermore,\marginnote{4{-}5.1} take a mendicant who has faith, ethics, learning, generosity, and wisdom. They think: ‘If only, when my body breaks up, after death, I would be reborn in the company of well-to-do brahmins … well-to-do householders.’ They settle on that thought, stabilize it and develop it. Those choices and meditations of theirs, developed and cultivated like this, lead to rebirth there. This is the path and the practice that leads to rebirth there. 

Furthermore,\marginnote{6.1} take a mendicant who has faith, ethics, learning, generosity, and wisdom. And they’ve heard: ‘The Gods of the Four Great Kings are long-lived, beautiful, and very happy.’ They think: ‘If only, when my body breaks up, after death, I would be reborn in the company of the Gods of the Four Great Kings!’ They settle on that thought, stabilize it and develop it. Those choices and meditations of theirs, developed and cultivated like this, lead to rebirth there. This is the path and the practice that leads to rebirth there. 

Furthermore,\marginnote{7{-}11.1} take a mendicant who has faith, ethics, learning, generosity, and wisdom. And they’ve heard: ‘The Gods of the Thirty-Three … the Gods of Yama … the Joyful Gods … the Gods Who Love to Create … the Gods Who Control the Creations of Others are long-lived, beautiful, and very happy.’ They think: ‘If only, when my body breaks up, after death, I would be reborn in the company of the Gods Who Control the Creations of Others!’ They settle on that thought, stabilize it and develop it. Those choices and meditations of theirs, developed and cultivated like this, lead to rebirth there. This is the path and the practice that leads to rebirth there. 

Furthermore,\marginnote{12.1} take a mendicant who has faith, ethics, learning, generosity, and wisdom. And they’ve heard: ‘The \textsanskrit{Brahmā} of a thousand is long-lived, beautiful, and very happy.’ Now the \textsanskrit{Brahmā} of a thousand meditates determined on pervading a galaxy of a thousand solar systems, as well as the sentient beings reborn there. As a person might pick up a gallnut in their hand and examine it, so too the \textsanskrit{Brahmā} of a thousand meditates determined on pervading a galaxy of a thousand solar systems, as well as the sentient beings reborn there. They think: ‘If only, when my body breaks up, after death, I would be reborn in the company of the \textsanskrit{Brahmā} of a thousand!’ They settle on that thought, stabilize it and develop it. Those choices and meditations of theirs, developed and cultivated like this, lead to rebirth there. This is the path and the practice that leads to rebirth there. 

Furthermore,\marginnote{13{-}16.1} take a mendicant who has faith, ethics, learning, generosity, and wisdom. And they’ve heard: ‘The \textsanskrit{Brahmā} of two thousand … the \textsanskrit{Brahmā} of three thousand … the \textsanskrit{Brahmā} of four thousand … the \textsanskrit{Brahmā} of five thousand is long-lived, beautiful, and very happy.’ Now the \textsanskrit{Brahmā} of five thousand meditates determined on pervading a galaxy of five thousand solar systems, as well as the sentient beings reborn there. As a person might pick up five gallnuts in their hand and examine them, so too the \textsanskrit{Brahmā} of five thousand meditates determined on pervading a galaxy of five thousand solar systems, as well as the sentient beings reborn there. They think: ‘If only, when my body breaks up, after death, I would be reborn in the company of the \textsanskrit{Brahmā} of five thousand!’ They settle on that thought, stabilize it and develop it. Those choices and meditations of theirs, developed and cultivated like this, lead to rebirth there. This is the path and the practice that leads to rebirth there. 

Furthermore,\marginnote{17.1} take a mendicant who has faith, ethics, learning, generosity, and wisdom. And they’ve heard: ‘The \textsanskrit{Brahmā} of ten thousand is long-lived, beautiful, and very happy.’ Now the \textsanskrit{Brahmā} of ten thousand meditates determined on pervading a galaxy of ten thousand solar systems, as well as the sentient beings reborn there. Suppose there was a beryl gem that was naturally beautiful, eight-faceted, well-worked. When placed on a cream rug it would shine and glow and radiate. In the same way the \textsanskrit{Brahmā} of ten thousand meditates determined on pervading a galaxy of ten thousand solar systems, as well as the sentient beings reborn there. They think: ‘If only, when my body breaks up, after death, I would be reborn in the company of the \textsanskrit{Brahmā} of ten thousand!’ They settle on that thought, stabilize it and develop it. Those choices and meditations of theirs, developed and cultivated like this, lead to rebirth there. This is the path and the practice that leads to rebirth there. 

Furthermore,\marginnote{18.1} take a mendicant who has faith, ethics, learning, generosity, and wisdom. And they’ve heard: ‘The \textsanskrit{Brahmā} of a hundred thousand is long-lived, beautiful, and very happy.’ Now the \textsanskrit{Brahmā} of a hundred thousand meditates determined on pervading a galaxy of a hundred thousand solar systems, as well as the sentient beings reborn there. Suppose there was a pendant of river gold, fashioned by an expert smith, well wrought in the forge. When placed on a cream rug it would shine and glow and radiate. In the same way the \textsanskrit{Brahmā} of a hundred thousand meditates determined on pervading a galaxy of a hundred thousand solar systems, as well as the sentient beings reborn there. They think: ‘If only, when my body breaks up, after death, I would be reborn in the company of the \textsanskrit{Brahmā} of a hundred thousand!’ They settle on that thought, stabilize it and develop it. Those choices and meditations of theirs, developed and cultivated like this, lead to rebirth there. This is the path and the practice that leads to rebirth there. 

Furthermore,\marginnote{19.1} take a mendicant who has faith, ethics, learning, generosity, and wisdom. And they’ve heard: ‘The Radiant Gods … the Gods of Limited Radiance … the Gods of Limitless Radiance … the Gods of Streaming Radiance … the Gods of Limited Glory … the Gods of Limitless Glory … the Gods Replete with Glory … the Gods of Abundant Fruit … the Gods of Aviha … the Gods of Atappa … the Gods Fair to See … the Fair Seeing Gods … the Gods of \textsanskrit{Akaniṭṭha} … the gods of the dimension of infinite space … the gods of the dimension of infinite consciousness … the gods of the dimension of nothingness … the gods of the dimension of neither perception nor non-perception are long-lived, beautiful, and very happy.’ They think: ‘If only, when my body breaks up, after death, I would be reborn in the company of the gods of the dimension of neither perception nor non-perception!’ They settle on that thought, stabilize it and develop it. Those choices and meditations of theirs, developed and cultivated like this, lead to rebirth there. This is the path and the practice that leads to rebirth there. 

Furthermore,\marginnote{37.1} take a mendicant who has faith, ethics, learning, generosity, and wisdom. They think: ‘If only I might realize the undefiled freedom of heart and freedom by wisdom in this very life, and live having realized it with my own insight due to the ending of defilements.’ They realize the undefiled freedom of heart and freedom by wisdom in this very life. And they live having realized it with their own insight due to the ending of defilements. And, mendicants, that mendicant is not reborn anywhere.” 

That\marginnote{37.9} is what the Buddha said. Satisfied, the mendicants were happy with what the Buddha said. 

%
\addtocontents{toc}{\let\protect\contentsline\protect\nopagecontentsline}
\chapter*{The Chapter Beginning with Emptiness }
\addcontentsline{toc}{chapter}{\tocchapterline{The Chapter Beginning with Emptiness }}
\addtocontents{toc}{\let\protect\contentsline\protect\oldcontentsline}

%
\section*{{\suttatitleacronym MN 121}{\suttatitletranslation The Shorter Discourse on Emptiness }{\suttatitleroot Cūḷasuññatasutta}}
\addcontentsline{toc}{section}{\tocacronym{MN 121} \toctranslation{The Shorter Discourse on Emptiness } \tocroot{Cūḷasuññatasutta}}
\markboth{The Shorter Discourse on Emptiness }{Cūḷasuññatasutta}
\extramarks{MN 121}{MN 121}

\scevam{So\marginnote{1.1} I have heard. }At one time the Buddha was staying near \textsanskrit{Sāvatthī} in the Eastern Monastery, the stilt longhouse of \textsanskrit{Migāra}’s mother. 

Then\marginnote{2.1} in the late afternoon, Venerable Ānanda came out of retreat and went to the Buddha. He bowed, sat down to one side, and said to him: 

“Sir,\marginnote{3.1} this one time the Buddha was staying in the land of the Sakyans where they have a town named Townsville. There I heard and learned this in the presence of the Buddha: ‘Ānanda, these days I usually practice the meditation on emptiness.’ I trust I properly heard, learned, attended, and remembered that from the Buddha?” 

“Indeed,\marginnote{3.5} Ānanda, you properly heard, learned, attended, and remembered that. Now, as before, I usually practice the meditation on emptiness. 

Consider\marginnote{4.1} this stilt longhouse of \textsanskrit{Migāra}’s mother. It’s empty of elephants, cows, horses, and mares; of gold and money; and of gatherings of men and women. There is only this that is not emptiness, namely, the oneness dependent on the mendicant \textsanskrit{Saṅgha}. In the same way, a mendicant—ignoring the perception of the village and the perception of people—focuses on the oneness dependent on the perception of wilderness. Their mind becomes eager, confident, settled, and decided in that perception of wilderness. They understand: ‘Here there is no stress due to the perception of village or the perception of people. There is only this modicum of stress, namely the oneness dependent on the perception of wilderness.’ They understand: ‘This field of perception is empty of the perception of the village. It is empty of the perception of people. There is only this that is not emptiness, namely the oneness dependent on the perception of wilderness.’ And so they regard it as empty of what is not there, but as to what remains they understand that it is present. That’s how emptiness is born in them—genuine, undistorted, and pure. 

Furthermore,\marginnote{5.1} a mendicant—ignoring the perception of people and the perception of wilderness—focuses on the oneness dependent on the perception of earth. Their mind becomes eager, confident, settled, and decided in that perception of earth. As a bull’s hide is rid of folds when fully stretched out by a hundred pegs, so too, ignoring the hilly terrain, inaccessible riverlands, stumps and thorns, and rugged mountains, they focus on the oneness dependent on the perception of earth. Their mind becomes eager, confident, settled, and decided in that perception of earth. They understand: ‘Here there is no stress due to the perception of people or the perception of wilderness. There is only this modicum of stress, namely the oneness dependent on the perception of earth.’ They understand: ‘This field of perception is empty of the perception of people. It is empty of the perception of wilderness. There is only this that is not emptiness, namely the oneness dependent on the perception of earth.’ And so they regard it as empty of what is not there, but as to what remains they understand that it is present. That’s how emptiness is born in them—genuine, undistorted, and pure. 

Furthermore,\marginnote{6.1} a mendicant—ignoring the perception of wilderness and the perception of earth—focuses on the oneness dependent on the perception of the dimension of infinite space. Their mind becomes eager, confident, settled, and decided in that perception of the dimension of infinite space. They understand: ‘Here there is no stress due to the perception of wilderness or the perception of earth. There is only this modicum of stress, namely the oneness dependent on the perception of the dimension of infinite space.’ They understand: ‘This field of perception is empty of the perception of wilderness. It is empty of the perception of earth. There is only this that is not emptiness, namely the oneness dependent on the perception of the dimension of infinite space.’ And so they regard it as empty of what is not there, but as to what remains they understand that it is present. That’s how emptiness is born in them—genuine, undistorted, and pure. 

Furthermore,\marginnote{7.1} a mendicant—ignoring the perception of earth and the perception of the dimension of infinite space—focuses on the oneness dependent on the perception of the dimension of infinite consciousness. Their mind becomes eager, confident, settled, and decided in that perception of the dimension of infinite consciousness. They understand: ‘Here there is no stress due to the perception of earth or the perception of the dimension of infinite space. There is only this modicum of stress, namely the oneness dependent on the perception of the dimension of infinite consciousness.’ They understand: ‘This field of perception is empty of the perception of earth. It is empty of the perception of the dimension of infinite space. There is only this modicum of stress, namely the oneness dependent on the perception of the dimension of infinite consciousness.’ And so they regard it as empty of what is not there, but as to what remains they understand that it is present. That’s how emptiness is born in them—genuine, undistorted, and pure. 

Furthermore,\marginnote{8.1} a mendicant—ignoring the perception of the dimension of infinite space and the perception of the dimension of infinite consciousness—focuses on the oneness dependent on the perception of the dimension of nothingness. Their mind becomes eager, confident, settled, and decided in that perception of the dimension of nothingness. They understand: ‘Here there is no stress due to the perception of the dimension of infinite space or the perception of the dimension of infinite consciousness. There is only this modicum of stress, namely the oneness dependent on the perception of the dimension of nothingness.’ They understand: ‘This field of perception is empty of the perception of the dimension of infinite space. It is empty of the perception of the dimension of infinite consciousness. There is only this that is not emptiness, namely the oneness dependent on the perception of the dimension of nothingness.’ And so they regard it as empty of what is not there, but as to what remains they understand that it is present. That’s how emptiness is born in them—genuine, undistorted, and pure. 

Furthermore,\marginnote{9.1} a mendicant—ignoring the perception of the dimension of infinite consciousness and the perception of the dimension of nothingness—focuses on the oneness dependent on the perception of the dimension of neither perception nor non-perception. Their mind becomes eager, confident, settled, and decided in that perception of the dimension of neither perception nor non-perception. They understand: ‘Here there is no stress due to the perception of the dimension of infinite consciousness or the perception of the dimension of nothingness. There is only this modicum of stress, namely the oneness dependent on the perception of the dimension of neither perception nor non-perception.’ They understand: ‘This field of perception is empty of the perception of the dimension of infinite consciousness. It is empty of the perception of the dimension of nothingness. There is only this that is not emptiness, namely the oneness dependent on the perception of the dimension of neither perception nor non-perception.’ And so they regard it as empty of what is not there, but as to what remains they understand that it is present. That’s how emptiness is born in them—genuine, undistorted, and pure. 

Furthermore,\marginnote{10.1} a mendicant—ignoring the perception of the dimension of nothingness and the perception of the dimension of neither perception nor non-perception—focuses on the oneness dependent on the signless immersion of the heart. Their mind becomes eager, confident, settled, and decided in that signless immersion of the heart. They understand: ‘Here there is no stress due to the perception of the dimension of nothingness or the perception of the dimension of neither perception nor non-perception. There is only this modicum of stress, namely that associated with the six sense fields dependent on this body and conditioned by life.’ They understand: ‘This field of perception is empty of the perception of the dimension of nothingness. It is empty of the perception of the dimension of neither perception nor non-perception. There is only this that is not emptiness, namely that associated with the six sense fields dependent on this body and conditioned by life.’ And so they regard it as empty of what is not there, but as to what remains they understand that it is present. That’s how emptiness is born in them—genuine, undistorted, and pure. 

Furthermore,\marginnote{11.1} a mendicant—ignoring the perception of the dimension of nothingness and the perception of the dimension of neither perception nor non-perception—focuses on the oneness dependent on the signless immersion of the heart. Their mind becomes eager, confident, settled, and decided in that signless immersion of the heart. They understand: ‘Even this signless immersion of the heart is produced by choices and intentions.’ They understand: ‘But whatever is produced by choices and intentions is impermanent and liable to cessation.’ Knowing and seeing like this, their mind is freed from the defilements of sensuality, desire to be reborn, and ignorance. When they’re freed, they know they’re freed. 

They\marginnote{11.8} understand: ‘Rebirth is ended, the spiritual journey has been completed, what had to be done has been done, there is no return to any state of existence.’ 

They\marginnote{12.1} understand: ‘Here there is no stress due to the defilements of sensuality, desire to be reborn, or ignorance. There is only this modicum of stress, namely that associated with the six sense fields dependent on this body and conditioned by life.’ They understand: ‘This field of perception is empty of the perception of the defilements of sensuality, desire to be reborn, and ignorance. There is only this that is not emptiness, namely that associated with the six sense fields dependent on this body and conditioned by life.’ And so they regard it as empty of what is not there, but as to what remains they understand that it is present. That’s how emptiness is born in them—genuine, undistorted, and pure. 

Whatever\marginnote{13.1} ascetics and brahmins enter and remain in the pure, ultimate, supreme emptiness—whether in the past, future, or present—all of them enter and remain in this same pure, ultimate, supreme emptiness. So, Ānanda, you should train like this: ‘We will enter and remain in the pure, ultimate, supreme emptiness.’ That’s how you should train.” 

That\marginnote{13.6} is what the Buddha said. Satisfied, Venerable Ānanda was happy with what the Buddha said. 

%
\section*{{\suttatitleacronym MN 122}{\suttatitletranslation The Longer Discourse on Emptiness }{\suttatitleroot Mahāsuññatasutta}}
\addcontentsline{toc}{section}{\tocacronym{MN 122} \toctranslation{The Longer Discourse on Emptiness } \tocroot{Mahāsuññatasutta}}
\markboth{The Longer Discourse on Emptiness }{Mahāsuññatasutta}
\extramarks{MN 122}{MN 122}

\scevam{So\marginnote{1.1} I have heard. }At one time the Buddha was staying in the land of the Sakyans, near Kapilavatthu in the Banyan Tree Monastery. 

Then\marginnote{2.1} the Buddha robed up in the morning and, taking his bowl and robe, entered Kapilavatthu for alms. He wandered for alms in Kapilavatthu. After the meal, on his return from almsround, he went to the dwelling of Kā\textsanskrit{ḷakhemaka} the Sakyan for the day’s meditation. 

Now\marginnote{2.3} at that time many resting places had been spread out at Kā\textsanskrit{ḷakhemaka}’s dwelling. The Buddha saw this, and wondered, “Many resting places have been spread out; are there many mendicants living here?” 

Now\marginnote{2.8} at that time Venerable Ānanda, together with many other mendicants, was making robes in \textsanskrit{Ghaṭa} the Sakyan’s dwelling. Then in the late afternoon, the Buddha came out of retreat and went to \textsanskrit{Ghaṭa}’s dwelling, where he sat on the seat spread out and said to Venerable Ānanda, “Many resting places have been spread out at \textsanskrit{Kāḷakhemaka}’s dwelling; are many mendicants living there?” 

“Indeed\marginnote{2.14} there are, sir. It’s currently the time for making robes.” 

“Ānanda,\marginnote{3.1} a mendicant doesn’t shine who enjoys company and groups, who loves them and likes to enjoy them. It’s simply not possible that such a mendicant will get the pleasure of renunciation, the pleasure of seclusion, the pleasure of peace, the pleasure of awakening when they want, without trouble or difficulty. But you should expect that a mendicant who lives alone, withdrawn from the group, will get the pleasure of renunciation, the pleasure of seclusion, the pleasure of peace, the pleasure of awakening when they want, without trouble or difficulty. That is possible. 

Indeed,\marginnote{4.1} Ānanda, it is not possible that a mendicant who enjoys company will enter and remain in the freedom of heart—either that which is temporary and pleasant, or that which is irreversible and unshakable. But it is possible that a mendicant who lives alone, withdrawn from the group will enter and remain in the freedom of heart—either that which is temporary and pleasant, or that which is irreversible and unshakable. 

Ānanda,\marginnote{5.1} I do not see even a single sight which, with its decay and perishing, would not give rise to sorrow, lamentation, pain, sadness, and distress in someone who has desire and lust for it. 

But\marginnote{6.1} the Realized One woke up to this meditation, namely to enter and remain in emptiness internally by not focusing on any signs. Now, suppose that while the Realized One is practicing this meditation, monks, nuns, laymen, laywomen, rulers and their ministers, founders of religious sects and their disciples go to visit him. In that case, with a mind slanting, sloping, and inclining to seclusion, withdrawn, and loving renunciation, he invariably gives each of them a talk emphasizing the topic of dismissal. 

Therefore,\marginnote{7.1} if a mendicant might wish: ‘May I enter and remain in emptiness internally!’ So they should still, settle, unify, and immerse their mind in \textsanskrit{samādhi} internally. 

And\marginnote{7.3} how does a mendicant still, settle, unify, and immerse their mind in \textsanskrit{samādhi} internally? 

It’s\marginnote{8.1} when a mendicant, quite secluded from sensual pleasures, secluded from unskillful qualities, enters and remains in the first absorption … second absorption … third absorption … fourth absorption. That’s how a mendicant stills, settles, unifies, and immerses their mind in \textsanskrit{samādhi} internally. 

They\marginnote{9.1} focus on emptiness internally, but their mind isn’t eager, confident, settled, and decided. In that case, they understand: ‘I am focusing on emptiness internally, but my mind isn’t eager, confident, settled, and decided.’ In this way they are aware of the situation. They focus on emptiness externally … They focus on emptiness internally and externally … They focus on the imperturbable, but their mind isn’t eager, confident, settled, and decided. In that case, they understand: ‘I am focusing on the imperturbable internally, but my mind isn’t eager, confident, settled, and decided.’ In this way they are aware of the situation. 

Then\marginnote{10.1} that mendicant should still, settle, unify, and immerse their mind in \textsanskrit{samādhi} internally using the same meditation subject as a basis of immersion that they used before. They focus on emptiness internally, and their mind is eager, confident, settled, and decided. In that case, they understand: ‘I am focusing on emptiness internally, and my mind is eager, confident, settled, and decided.’ In this way they are aware of the situation. They focus on emptiness externally … They focus on emptiness internally and externally … They focus on the imperturbable, and their mind is eager, confident, settled, and decided. In that case, they understand: ‘I am focusing on the imperturbable, and my mind is eager, confident, settled, and decided.’ In this way they are aware of the situation. 

While\marginnote{11.1} a mendicant is practicing such meditation, if their mind inclines to walking, they walk, thinking: ‘While I’m walking, bad, unskillful qualities of desire and aversion will not overwhelm me.’ In this way they are aware of the situation. While a mendicant is practicing such meditation, if their mind inclines to standing, they stand, thinking: ‘While I’m standing, bad, unskillful qualities of desire and aversion will not overwhelm me.’ In this way they are aware of the situation. While a mendicant is practicing such meditation, if their mind inclines to sitting, they sit, thinking: ‘While I’m sitting, bad, unskillful qualities of desire and aversion will not overwhelm me.’ In this way they are aware of the situation. While a mendicant is practicing such meditation, if their mind inclines to lying down, they lie down, thinking: ‘While I’m lying down, bad, unskillful qualities of desire and aversion will not overwhelm me.’ In this way they are aware of the situation. 

While\marginnote{12.1} a mendicant is practicing such meditation, if their mind inclines to talking, they think: ‘I will not engage in the kind of speech that is low, crude, ordinary, ignoble, and pointless. Such speech doesn’t lead to disillusionment, dispassion, cessation, peace, insight, awakening, and extinguishment. Namely: talk about kings, bandits, and ministers; talk about armies, threats, and wars; talk about food, drink, clothes, and beds; talk about garlands and fragrances; talk about family, vehicles, villages, towns, cities, and countries; talk about women and heroes; street talk and well talk; talk about the departed; motley talk; tales of land and sea; and talk about being reborn in this or that state of existence.’ In this way they are aware of the situation. ‘But I will take part in talk about self-effacement that helps open the heart and leads solely to disillusionment, dispassion, cessation, peace, insight, awakening, and extinguishment. That is, talk about fewness of wishes, contentment, seclusion, aloofness, arousing energy, ethics, immersion, wisdom, freedom, and the knowledge and vision of freedom.’ In this way they are aware of the situation. 

While\marginnote{13.1} a mendicant is practicing such meditation, if their mind inclines to thinking, they think: ‘I will not think the kind of thought that is low, crude, ordinary, ignoble, and pointless. Such thoughts don’t lead to disillusionment, dispassion, cessation, peace, insight, awakening, and extinguishment. That is, sensual, malicious, or cruel thoughts.’ In this way they are aware of the situation. ‘But I will think the kind of thought that is noble and emancipating, and brings one who practices it to the complete ending of suffering. That is, thoughts of renunciation, good will, and harmlessness.’ In this way they are aware of the situation. 

There\marginnote{14.1} are these five kinds of sensual stimulation. What five? Sights known by the eye that are likable, desirable, agreeable, pleasant, sensual, and arousing. Sounds known by the ear … Smells known by the nose … Tastes known by the tongue … Touches known by the body that are likable, desirable, agreeable, pleasant, sensual, and arousing. These are the five kinds of sensual stimulation. 

So\marginnote{15.1} you should regularly check your own mind: ‘Does my mind take an interest in any of these five kinds of sensual stimulation?’ Suppose that, upon checking, a mendicant knows this: ‘My mind does take an interest.’ In that case, they understand: ‘I have not given up desire and greed for the five kinds of sensual stimulation.’ In this way they are aware of the situation. But suppose that, upon checking, a mendicant knows this: ‘My mind does not take an interest.’ In that case, they understand: ‘I have given up desire and greed for the five kinds of sensual stimulation.’ In this way they are aware of the situation. 

A\marginnote{16.1} mendicant should meditate observing rise and fall in these five grasping aggregates: ‘Such is form, such is the origin of form, such is the ending of form. Such is feeling … Such is perception … Such are choices … Such is consciousness, such is the origin of consciousness, such is the ending of consciousness.’ 

As\marginnote{17.1} they do so, they give up the conceit ‘I am’ regarding the five grasping aggregates. In that case, they understand: ‘I have given up the conceit “I am” regarding the five grasping aggregates.’ In this way they are aware of the situation. 

These\marginnote{18.1} principles are entirely skillful, with skillful outcomes; they are noble, transcendent, and inaccessible to the Wicked One. 

What\marginnote{19.1} do you think, Ānanda? For what reason would a disciple value following the Teacher, even if sent away?” 

“Our\marginnote{19.3} teachings are rooted in the Buddha. He is our guide and our refuge. Sir, may the Buddha himself please clarify the meaning of this. The mendicants will listen and remember it.” 

“A\marginnote{20.1} disciple should not value following the Teacher for the sake of statements, songs, or discussions. Why is that? Because for a long time you have learned the teachings, remembering them, reciting them, mentally scrutinizing them, and comprehending them theoretically. But a disciple should value following the Teacher, even if asked to go away, for the sake of talk about self-effacement that helps open the heart and leads solely to disillusionment, dispassion, cessation, peace, insight, awakening, and extinguishment. That is, talk about fewness of wishes, contentment, seclusion, aloofness, arousing energy, ethics, immersion, wisdom, freedom, and the knowledge and vision of freedom. 

This\marginnote{21.1} being so, Ānanda, there is a peril for the teacher, a peril for the student, and a peril for a spiritual practitioner. 

And\marginnote{22.1} how is there a peril for the teacher? It’s when some teacher frequents a secluded lodging—a wilderness, the root of a tree, a hill, a ravine, a mountain cave, a charnel ground, a forest, the open air, a heap of straw. While meditating withdrawn, they’re visited by a stream of brahmins and householders of the city and country. When this happens, they enjoy infatuation, fall into greed, and return to indulgence. This teacher is said to be imperiled by the teacher’s peril. They’re ruined by bad, unskillful qualities that are corrupting, leading to future lives, hurtful, resulting in suffering and future rebirth, old age, and death. That’s how there is a peril for the teacher. 

And\marginnote{23.1} how is there a peril for the student? It’s when the student of a teacher, emulating their teacher’s fostering of seclusion, frequents a secluded lodging—a wilderness, the root of a tree, a hill, a ravine, a mountain cave, a charnel ground, a forest, the open air, a heap of straw. While meditating withdrawn, they’re visited by a stream of brahmins and householders of the city and country. When this happens, they enjoy infatuation, fall into greed, and return to indulgence. This student is said to be imperiled by the student’s peril. They’re ruined by bad, unskillful qualities that are corrupting, leading to future lives, hurtful, resulting in suffering and future rebirth, old age, and death. That’s how there is a peril for the student. 

And\marginnote{24.1} how is there a peril for a spiritual practitioner? It’s when a Realized One arises in the world, perfected, a fully awakened Buddha, accomplished in knowledge and conduct, holy, knower of the world, supreme guide for those who wish to train, teacher of gods and humans, awakened, blessed. He frequents a secluded lodging—a wilderness, the root of a tree, a hill, a ravine, a mountain cave, a charnel ground, a forest, the open air, a heap of straw. While meditating withdrawn, he’s visited by a stream of brahmins and householders of the city and country. When this happens, he doesn’t enjoy infatuation, fall into greed, and return to indulgence. But a disciple of this teacher, emulating their teacher’s fostering of seclusion, frequents a secluded lodging—a wilderness, the root of a tree, a hill, a ravine, a mountain cave, a charnel ground, a forest, the open air, a heap of straw. While meditating withdrawn, they’re visited by a stream of brahmins and householders of the city and country. When this happens, they enjoy infatuation, fall into greed, and return to indulgence. This spiritual practitioner is said to be imperiled by the spiritual practitioner’s peril. They’re ruined by bad, unskillful qualities that are corrupting, leading to future lives, hurtful, resulting in suffering and future rebirth, old age, and death. That’s how there is a peril for the spiritual practitioner. 

And\marginnote{24.13} in this context, Ānanda, as compared to the peril of the teacher or the student, the peril of the spiritual practitioner has more painful, bitter results, and even leads to the underworld. 

So,\marginnote{25.1} Ānanda, treat me as a friend, not as an enemy. That will be for your lasting welfare and happiness. 

And\marginnote{25.3} how do disciples treat their Teacher as an enemy, not a friend? It’s when the Teacher teaches the Dhamma out of kindness and compassion: ‘This is for your welfare. This is for your happiness.’ But their disciples don’t want to listen. They don’t pay attention or apply their minds to understand. They proceed having turned away from the Teacher’s instruction. That’s how the disciples treat their Teacher as an enemy, not a friend. 

And\marginnote{26.1} how do disciples treat their Teacher as a friend, not an enemy? It’s when the Teacher teaches the Dhamma out of kindness and compassion: ‘This is for your welfare. This is for your happiness.’ And their disciples want to listen. They pay attention and apply their minds to understand. They don’t proceed having turned away from the Teacher’s instruction. That’s how the disciples treat their Teacher as a friend, not an enemy. 

So,\marginnote{26.6} Ānanda, treat me as a friend, not as an enemy. That will be for your lasting welfare and happiness. I shall not mollycoddle you like a potter with their damp, unfired pots. I shall speak, correcting you again and again, pressing you again and again. The core will stand the test.” 

That\marginnote{27.5} is what the Buddha said. Satisfied, Venerable Ānanda was happy with what the Buddha said. 

%
\section*{{\suttatitleacronym MN 123}{\suttatitletranslation Incredible and Amazing }{\suttatitleroot Acchariyaabbhutasutta}}
\addcontentsline{toc}{section}{\tocacronym{MN 123} \toctranslation{Incredible and Amazing } \tocroot{Acchariyaabbhutasutta}}
\markboth{Incredible and Amazing }{Acchariyaabbhutasutta}
\extramarks{MN 123}{MN 123}

\scevam{So\marginnote{1.1} I have heard. }At one time the Buddha was staying near \textsanskrit{Sāvatthī} in Jeta’s Grove, \textsanskrit{Anāthapiṇḍika}’s monastery. 

Then\marginnote{2.1} after the meal, on return from almsround, several senior mendicants sat together in the assembly hall and this discussion came up among them: 

“It’s\marginnote{2.2} incredible, reverends, it’s amazing, the power and might of a Realized One! For he is able to know the Buddhas of the past who have become completely extinguished, cut off proliferation, cut off the track, finished off the cycle, and transcended suffering. He knows the caste they were born in, and also their names, clans, conduct, teaching, wisdom, meditation, and freedom.” 

When\marginnote{2.5} they said this, Venerable Ānanda said, “The Realized Ones are incredible, reverends, and they have incredible qualities. They’re amazing, and they have amazing qualities.” But this conversation among those mendicants was left unfinished. 

Then\marginnote{2.9} in the late afternoon, the Buddha came out of retreat, went to the assembly hall, sat down on the seat spread out, and addressed the mendicants: “Mendicants, what were you sitting talking about just now? What conversation was left unfinished?” 

So\marginnote{2.12} the mendicants told him what they had been talking about. The Buddha said, “Well then, Ānanda, say some more about the incredible and amazing qualities of the Realized One.” 

“Sir,\marginnote{3.1} I have heard and learned this in the presence of the Buddha: ‘Mindful and aware, the being intent on awakening was reborn in the host of Joyful Gods.’ This I remember as an incredible quality of the Buddha. 

I\marginnote{4.1} have learned this in the presence of the Buddha: ‘Mindful and aware, the being intent on awakening remained in the host of Joyful Gods.’ This too I remember as an incredible quality of the Buddha. 

I\marginnote{5.1} have learned this in the presence of the Buddha: ‘For the whole of that life, the being intent on awakening remained in the host of Joyful Gods.’ This too I remember as an incredible quality of the Buddha. 

I\marginnote{6.1} have learned this in the presence of the Buddha: ‘Mindful and aware, the being intent on awakening passed away from the host of Joyful Gods and was conceived in his mother’s womb.’ This too I remember as an incredible quality of the Buddha. 

I\marginnote{7.1} have learned this in the presence of the Buddha: ‘When the being intent on awakening passes away from the host of Joyful Gods, he is conceived in his mother’s womb. And then—in this world with its gods, \textsanskrit{Māras} and \textsanskrit{Brahmās}, this population with its ascetics and brahmins, gods and humans—an immeasurable, magnificent light appears, surpassing the glory of the gods. Even in the boundless desolation of interstellar space—so utterly dark that even the light of the moon and the sun, so mighty and powerful, makes no impression—an immeasurable, magnificent light appears, surpassing the glory of the gods. And even the sentient beings reborn there recognize each other by that light: “So, it seems other sentient beings have been reborn here!” And this galaxy shakes and rocks and trembles. And an immeasurable, magnificent light appears in the world, surpassing the glory of the gods.’ This too I remember as an incredible quality of the Buddha. 

I\marginnote{8.1} have learned this in the presence of the Buddha: ‘When the being intent on awakening is conceived in his mother’s belly, four deities approach to guard the four quarters, so that no human or non-human or anyone at all shall harm the being intent on awakening or his mother.’ This too I remember as an incredible quality of the Buddha. 

I\marginnote{9.1} have learned this in the presence of the Buddha: ‘When the being intent on awakening is conceived in his mother’s belly, she becomes naturally ethical. She refrains from killing living creatures, stealing, sexual misconduct, lying, and alcoholic drinks that cause negligence.’ This too I remember as an incredible quality of the Buddha. 

I\marginnote{10.1} have learned this in the presence of the Buddha: ‘When the being intent on awakening is conceived in his mother’s belly, she no longer feels sexual desire for men, and she cannot be violated by a man of lustful intent.’ This too I remember as an incredible quality of the Buddha. 

I\marginnote{11.1} have learned this in the presence of the Buddha: ‘When the being intent on awakening is conceived in his mother’s belly, she obtains the five kinds of sensual stimulation and amuses herself, supplied and provided with them.’ This too I remember as an incredible quality of the Buddha. 

I\marginnote{12.1} have learned this in the presence of the Buddha: ‘When the being intent on awakening is conceived in his mother’s belly, no afflictions beset her. She’s happy and free of bodily fatigue. And she sees the being intent on awakening in her womb, complete with all his various parts, not deficient in any faculty. Suppose there was a beryl gem that was naturally beautiful, eight-faceted, well-worked. And it was strung with a thread of blue, yellow, red, white, or golden brown. And someone with good eyesight were to take it in their hand and examine it: “This beryl gem is naturally beautiful, eight-faceted, well-worked. And it’s strung with a thread of blue, yellow, red, white, or golden brown.” 

In\marginnote{12.5} the same way, when the being intent on awakening is conceived in his mother’s belly, no afflictions beset her. She’s happy and free of bodily fatigue. And she sees the being intent on awakening in her womb, complete with all his various parts, not deficient in any faculty.’ This too I remember as an incredible quality of the Buddha. 

I\marginnote{13.1} have learned this in the presence of the Buddha: ‘Seven days after the being intent on awakening is born, his mother passes away and is reborn in the host of Joyful Gods.’ This too I remember as an incredible quality of the Buddha. 

I\marginnote{14.1} have learned this in the presence of the Buddha: ‘Other women carry the infant in the womb for nine or ten months before giving birth. Not so the mother of the being intent on awakening. She gives birth after exactly ten months.’ This too I remember as an incredible quality of the Buddha. 

I\marginnote{15.1} have learned this in the presence of the Buddha: ‘Other women give birth while sitting or lying down. Not so the mother of the being intent on awakening. She only gives birth standing up.’ This too I remember as an incredible quality of the Buddha. 

I\marginnote{16.1} have learned this in the presence of the Buddha: ‘When the being intent on awakening emerges from his mother’s womb, gods receive him first, then humans.’ This too I remember as an incredible quality of the Buddha. 

I\marginnote{17.1} have learned this in the presence of the Buddha: ‘When the being intent on awakening emerges from his mother’s womb, before he reaches the ground, four deities receive him and place him before his mother, saying: “Rejoice, O Queen! An illustrious son is born to you.”’ This too I remember as an incredible quality of the Buddha. 

I\marginnote{18.1} have learned this in the presence of the Buddha: ‘When the being intent on awakening emerges from his mother’s womb, he emerges already clean, unsoiled by waters, mucus, blood, or any other kind of impurity, pure and clean. Suppose a jewel-treasure was placed on a cloth from \textsanskrit{Kāsī}. The jewel would not soil the cloth, nor would the cloth soil the jewel. Why is that? Because of the cleanliness of them both. 

In\marginnote{18.6} the same way, when the being intent on awakening emerges from his mother’s womb, he emerges already clean, unsoiled by waters, mucus, blood, or any other kind of impurity, pure and clean.’ This too I remember as an incredible quality of the Buddha. 

I\marginnote{19.1} have learned this in the presence of the Buddha: ‘When the being intent on awakening emerges from his mother’s womb, two streams of water appear in the sky, one cool, one warm, for bathing the being intent on awakening and his mother.’ This too I remember as an incredible quality of the Buddha. 

I\marginnote{20.1} have learned this in the presence of the Buddha: ‘As soon as he’s born, the being intent on awakening stands firm with his own feet on the ground. Facing north, he takes seven strides with a white parasol held above him, surveys all quarters, and makes this dramatic proclamation: “I am the foremost in the world! I am the eldest in the world! I am the first in the world! This is my last rebirth. Now there are no more future lives.”’ This too I remember as an incredible quality of the Buddha. 

I\marginnote{21.1} have learned this in the presence of the Buddha: ‘When the being intent on awakening emerges from his mother’s womb, then—in this world with its gods, \textsanskrit{Māras} and \textsanskrit{Brahmās}, this population with its ascetics and brahmins, gods and humans—an immeasurable, magnificent light appears, surpassing the glory of the gods. Even in the boundless desolation of interstellar space—so utterly dark that even the light of the moon and the sun, so mighty and powerful, makes no impression—an immeasurable, magnificent light appears, surpassing the glory of the gods. And the sentient beings reborn there recognize each other by that light: “So, it seems other sentient beings have been reborn here!” And this galaxy shakes and rocks and trembles. And an immeasurable, magnificent light appears in the world, surpassing the glory of the gods.’ This too I remember as an incredible and amazing quality of the Buddha.” 

“Well\marginnote{22.1} then, Ānanda, you should also remember this as an incredible and amazing quality of the Realized One. It’s that the Realized One knows feelings as they arise, as they remain, and as they go away. He knows perceptions as they arise, as they remain, and as they go away. He knows thoughts as they arise, as they remain, and as they go away. This too you should remember as an incredible and amazing quality of the Realized One.” 

“Sir,\marginnote{23.1} the Buddha knows feelings as they arise, as they remain, and as they go away. He knows perceptions as they arise, as they remain, and as they go away. He knows thoughts as they arise, as they remain, and as they go away. This too I remember as an incredible and amazing quality of the Buddha.” 

That’s\marginnote{23.5} what Ānanda said, and the teacher approved. Satisfied, those mendicants were happy with what Venerable Ānanda said. 

%
\section*{{\suttatitleacronym MN 124}{\suttatitletranslation With Bakkula }{\suttatitleroot Bākulasutta}}
\addcontentsline{toc}{section}{\tocacronym{MN 124} \toctranslation{With Bakkula } \tocroot{Bākulasutta}}
\markboth{With Bakkula }{Bākulasutta}
\extramarks{MN 124}{MN 124}

\scevam{So\marginnote{1.1} I have heard. }At one time Venerable Bakkula was staying near \textsanskrit{Rājagaha}, in the Bamboo Grove, the squirrels’ feeding ground. 

Then\marginnote{2.1} the naked ascetic Kassapa, who had been a friend of Bakkula’s in the lay life, approached him, and exchanged greetings with him. When the greetings and polite conversation were over, he sat down to one side and said to Venerable Bakkula, “Reverend Bakkula, how long has it been since you went forth?” 

“It\marginnote{3.2} has been eighty years, reverend.” 

“But\marginnote{3.3} in these eighty years, how many times have you had sex?” 

“You\marginnote{3.4} shouldn’t ask me such a question. Rather, you should ask me this: ‘But in these eighty years, how many times have sensual perceptions ever arisen in you?’” 

“But\marginnote{3.8} in these eighty years, how many times have sensual perceptions ever arisen in you?” 

“In\marginnote{3.9} these eighty years, I don’t recall that any sensual perception has ever arisen in me.” 

“This\marginnote{3.10} we remember as an incredible quality of Venerable Bakkula.” 

“In\marginnote{4{-}5.1} these eighty years, I don’t recall that any perception of ill will … or cruelty has ever arisen in me.” 

“This\marginnote{4{-}5.3} too we remember as an incredible quality of Venerable Bakkula.” 

“In\marginnote{6.1} these eighty years, I don’t recall that any thought of sensuality … ill will … or cruelty has ever arisen in me.” 

“This\marginnote{7{-}8.3} too we remember as an incredible quality of Venerable Bakkula.” 

“In\marginnote{9.1} these eighty years, I don’t recall accepting a robe from a householder … cutting a robe with a knife … sewing a robe with a needle … dying a robe … sewing a robe during the robe-making ceremony … looking for robe material for my companions in the spiritual life when they are making robes … accepting an invitation … having such a thought: ‘If only someone would invite me!’ … sitting down inside a house … eating inside a house … getting caught up in the details of female’s appearance … teaching a female, even so much as a four line verse … going to the nuns’ quarters … teaching the nuns … teaching the trainee nuns … teaching the novice nuns … giving the going forth … giving the ordination … giving dependence … being looked after by a novice … bathing in the sauna … bathing with bath powder … looking for a massage from my companions in the spiritual life … being ill, even for as long as it takes to pull a cow’s udder … being presented with medicine, even as much as a bit of yellow myrobalan … leaning on a headrest … preparing a cot …” 

“This\marginnote{30{-}36.8} too we remember as an incredible quality of Venerable Bakkula.” 

“In\marginnote{37.1} these eighty years, I don’t recall commencing the rainy season residence within a village.” 

“This\marginnote{37.2} too we remember as an incredible quality of Venerable Bakkula.” 

“Reverend,\marginnote{38.1} for seven days I ate the nation’s almsfood as a debtor. Then on the eighth day I became enlightened.” 

“This\marginnote{38.3} too we remember as an incredible quality of Venerable Bakkula. 

Reverend\marginnote{39.1} Bakkula, may I receive the going forth, the ordination in this teaching and training?” And the naked ascetic Kassapa received the going forth, the ordination in this teaching and training. 

Not\marginnote{39.3} long after his ordination, Venerable Kassapa, living alone, withdrawn, diligent, keen, and resolute, soon realized the supreme end of the spiritual path in this very life. He lived having achieved with his own insight the goal for which gentlemen rightly go forth from the lay life to homelessness. 

He\marginnote{39.4} understood: “Rebirth is ended; the spiritual journey has been completed; what had to be done has been done; there is no return to any state of existence.” And Venerable Kassapa became one of the perfected. 

Then\marginnote{40.1} some time later Venerable Bakkula took a key and went from dwelling to dwelling, saying, “Come forth, venerables, come forth! Today will be my final extinguishment.” 

“This\marginnote{40.3} too we remember as an incredible quality of Venerable Bakkula.” 

And\marginnote{41.1} Venerable Bakkula became fully extinguished while sitting right in the middle of the \textsanskrit{Saṅgha}. 

“This\marginnote{41.2} too we remember as an incredible quality of Venerable Bakkula.” 

%
\section*{{\suttatitleacronym MN 125}{\suttatitletranslation The Level of the Tamed }{\suttatitleroot Dantabhūmisutta}}
\addcontentsline{toc}{section}{\tocacronym{MN 125} \toctranslation{The Level of the Tamed } \tocroot{Dantabhūmisutta}}
\markboth{The Level of the Tamed }{Dantabhūmisutta}
\extramarks{MN 125}{MN 125}

\scevam{So\marginnote{1.1} I have heard. }At one time the Buddha was staying near \textsanskrit{Rājagaha}, in the Bamboo Grove, the squirrels’ feeding ground. 

Now\marginnote{2.1} at that time the novice Aciravata was staying in a wilderness hut. Then as Prince Jayasena was going for a walk he approached Aciravata, and exchanged greetings with him. 

When\marginnote{2.3} the greetings and polite conversation were over, he sat down to one side and said to Aciravata, “Master Aggivessana, I have heard that a mendicant who meditates diligently, keenly, and resolutely can experience unification of mind.” 

“That’s\marginnote{2.6} so true, Prince! That’s so true! A mendicant who meditates diligently, keenly, and resolutely can experience unification of mind.” 

“Master\marginnote{3.1} Aggivessana, please teach me the Dhamma as you have learned and memorized it.” 

“I’m\marginnote{3.2} not competent to do so, Prince. For if I were to teach you the Dhamma as I have learned and memorized it, you might not understand the meaning, which would be wearying and troublesome for me.” 

“Master\marginnote{4.1} Aggivessana, please teach me the Dhamma as you have learned and memorized it. Hopefully I will understand the meaning of what you say.” 

“Then\marginnote{4.3} I shall teach you. If you understand the meaning of what I say, that’s good. If not, then leave each to his own, and do not question me about it further.” 

“Master\marginnote{4.6} Aggivessana, please teach me the Dhamma as you have learned and memorized it. If I understand the meaning of what you say, that’s good. If not, then I will leave each to his own, and not question you about it further.” 

Then\marginnote{5.1} the novice Aciravata taught Prince Jayasena the Dhamma as he had learned and memorized it. When he had spoken, Jayasena said to him, “It is impossible, Master Aggivessana, it cannot happen that a mendicant who meditates diligently, keenly, and resolutely can experience unification of mind.” Having declared that this was impossible, Jayasena got up from his seat and left. 

Not\marginnote{6.1} long after he had left, Aciravata went to the Buddha, bowed, sat down to one side, and informed the Buddha of all they had discussed. 

When\marginnote{7.1} he had spoken, the Buddha said to him, 

“How\marginnote{7.2} could it possibly be otherwise, Aggivessana? Prince Jayasena dwells in the midst of sensual pleasures, enjoying them, consumed by thoughts of them, burning with fever for them, and eagerly seeking more. It’s simply impossible for him to know or see or realize what can only be known, seen, and realized by renunciation. 

Suppose\marginnote{8.1} there was a pair of elephants or horse or oxen in training who were well tamed and well trained. And there was a pair who were not tamed or trained. What do you think, Aggivessana? Wouldn’t the pair that was well tamed and well trained perform the tasks of the tamed and reach the level of the tamed?” 

“Yes,\marginnote{8.4} sir.” 

“But\marginnote{8.5} would the pair that was not tamed and trained perform the tasks of the tamed and reach the level of the tamed, just like the tamed pair?” 

“No,\marginnote{8.6} sir.” 

“In\marginnote{8.7} the same way, Prince Jayasena dwells in the midst of sensual pleasures, enjoying them, consumed by thoughts of them, burning with fever for them, and eagerly seeking more. It’s simply impossible for him to know or see or realize what can only be known, seen, and realized by renunciation. 

Suppose\marginnote{9.1} there was a big mountain not far from a town or village. And two friends set out from that village or town, lending each other a hand up to the mountain. Once there, one friend would remain at the foot of the mountain, while the other would climb to the peak. Then the one standing at the foot would say to the one at the peak, ‘My friend, what do you see, standing there at the peak?’ They’d reply, ‘Standing at the peak, I see delightful parks, woods, meadows, and lotus ponds!’ 

But\marginnote{9.7} the other would say, ‘It’s impossible, it cannot happen that, standing at the peak, you can see delightful parks, woods, meadows, and lotus ponds.’ So their friend would come down from the peak, take their friend by the arm, and make them climb to the peak. After giving them a moment to catch their breath, they’d say, ‘My friend, what do you see, standing here at the peak?’ They’d reply, ‘Standing at the peak, I see delightful parks, woods, meadows, and lotus ponds!’ 

They’d\marginnote{9.13} say, ‘Just now I understood you to say: “It’s impossible, it cannot happen that, standing at the peak, you can see delightful parks, woods, meadows, and lotus ponds.” But now you say: “Standing at the peak, I see delightful parks, woods, meadows, and lotus ponds!”’ They’d say, ‘But my friend, it was because I was obstructed by this big mountain that I didn’t see what could be seen.’ 

But\marginnote{10.1} bigger than that is the mass of ignorance by which Prince Jayasena is veiled, shrouded, covered, and engulfed. Prince Jayasena dwells in the midst of sensual pleasures, enjoying them, consumed by thoughts of them, burning with fever for them, and eagerly seeking more. It’s simply impossible for him to know or see or realize what can only be known, seen, and realized by renunciation. It wouldn’t be surprising if, had these two similes occurred to you, Prince Jayasena would have gained confidence in you and shown his confidence.” 

“But\marginnote{11.2} sir, how could these two similes have occurred to me as they did to the Buddha, since they were neither supernaturally inspired, nor learned before in the past?” 

“Suppose,\marginnote{12.1} Aggivessana, an anointed aristocratic king was to address his elephant tracker, ‘Please, my good elephant tracker, mount the royal bull elephant and enter the elephant wood. When you see a wild bull elephant, tether it by the neck to the royal elephant.’ ‘Yes, Your Majesty,’ replied the elephant tracker, and did as he was asked. The royal elephant leads the wild elephant out into the open; and it’s only then that it comes out into the open, for a wild bull elephant clings to the elephant wood. Then the elephant tracker informs the king, ‘Sire, the wild elephant has come out into the open.’ Then the king addresses his elephant trainer, ‘Please, my good elephant trainer, tame the wild bull elephant. Subdue its wild behaviors, its wild memories and thoughts, and its wild stress, weariness, and fever. Make it happy to be within a village, and instill behaviors congenial to humans.’ 

‘Yes,\marginnote{12.11} Your Majesty,’ replied the elephant trainer. He dug a large post into the earth and tethered the elephant to it by the neck, so as to subdue its wild behaviors, its wild memories and thoughts, and its wild stress, weariness, and fever, and to make it happy to be within a village, and instill behaviors congenial to humans. He spoke in a way that’s mellow, pleasing to the ear, lovely, going to the heart, polite, likable and agreeable to the people. Spoken to in such a way by the elephant trainer, the wild elephant wanted to listen. It leant an ear and applied its mind to understand. So the elephant trainer rewards it with grass, fodder, and water. 

When\marginnote{12.15} the wild elephant accepts the grass, fodder, and water, the trainer knows, ‘Now the wild elephant will survive!’ Then he sets it a further task: ‘Pick it up, sir! Put it down, sir!’ When the wild elephant picks up and puts down when the trainer says, following instructions, the trainer sets it a further task: ‘Forward, sir! Back, sir!’ When the wild elephant goes forward and back when the trainer says, following instructions, the trainer sets it a further task: ‘Stand, sir! Sit, sir!’ 

When\marginnote{12.23} the wild elephant stands and sits when the trainer says, following instructions, the trainer sets the task called imperturbability. He fastens a large plank to its trunk; a lancer sits on its neck; other lancers surround it on all sides; and the trainer himself stands in front with a long lance. While practicing this task, it doesn’t budge its fore-feet or hind-feet, its fore-quarters or hind-quarters, its head, ears, tusks, tail, or trunk. The wild bull elephant endures being struck by spears, swords, arrows, and axes; it endures the thunder of the drums, kettledrums, horns, and cymbals. Rid of all crooks and flaws, and purged of defects, it is worthy of a king, fit to serve a king, and considered a factor of kingship. 

In\marginnote{13.1} the same way, Aggivessana, a Realized One arises in the world, perfected, a fully awakened Buddha, accomplished in knowledge and conduct, holy, knower of the world, supreme guide for those who wish to train, teacher of gods and humans, awakened, blessed. He realizes with his own insight this world—with its gods, \textsanskrit{Māras} and \textsanskrit{Brahmās}, this population with its ascetics and brahmins, gods and humans—and he makes it known to others. He teaches Dhamma that’s good in the beginning, good in the middle, and good in the end, meaningful and well-phrased. And he reveals a spiritual practice that’s entirely full and pure. 

A\marginnote{14.1} householder hears that teaching, or a householder’s child, or someone reborn in some clan. They gain faith in the Realized One, and reflect, ‘Living in a house is cramped and dirty, but the life of one gone forth is wide open. It’s not easy for someone living at home to lead the spiritual life utterly full and pure, like a polished shell. Why don’t I shave off my hair and beard, dress in ocher robes, and go forth from the lay life to homelessness?’ 

After\marginnote{14.7} some time they give up a large or small fortune, and a large or small family circle. They shave off hair and beard, dress in ocher robes, and go forth from the lay life to homelessness. And it’s only then that a noble disciple comes out into the open, for gods and humans cling to the five kinds of sensual stimulation. 

Then\marginnote{15.1} the Realized One guides them further: ‘Come, mendicant, be ethical and restrained in the monastic code, conducting yourself well and seeking alms in suitable places. Seeing danger in the slightest fault, keep the rules you’ve undertaken.’ 

When\marginnote{16.1} they have ethical conduct, the Realized One guides them further: ‘Come, mendicant, guard your sense doors. When you see a sight with your eyes, don’t get caught up in the features and details. … 

(This\marginnote{17{-}21.1} should be expanded as in MN 107, the Discourse with \textsanskrit{Moggallāna} the Accountant.) 

They\marginnote{22.1} give up these five hindrances, corruptions of the heart that weaken wisdom. Then they meditate observing an aspect of the body—keen, aware, and mindful, rid of desire and aversion for the world. 

They\marginnote{23.1} meditate observing an aspect of feelings … mind … principles—keen, aware, and mindful, rid of desire and aversion for the world. It’s like when the elephant trainer dug a large post into the earth and tethered the elephant to it by the neck, so as to subdue its wild behaviors, its wild memories and thoughts, and its wild stress, weariness, and fever, and to make it happy to be within a village, and instill behaviors congenial to humans. In the same way, a noble disciple has these four kinds of mindfulness meditation as tethers for the mind so as to subdue behaviors of the lay life, memories and thoughts of the lay life, the stress, weariness, and fever of the lay life, to end the cycle of suffering and to realize extinguishment. 

Then\marginnote{24.1} the Realized One guides them further: ‘Come, mendicant, meditate observing an aspect of the body, but don’t think thoughts connected with sensual pleasures. Meditate observing an aspect of feelings … mind … principles, but don’t think thoughts connected with sensual pleasures.’ 

As\marginnote{25.1} the placing of the mind and keeping it connected are stilled, they enter and remain in the second absorption … third absorption … fourth absorption. 

When\marginnote{26.1} their mind has become immersed in \textsanskrit{samādhi} like this—purified, bright, flawless, rid of corruptions, pliable, workable, steady, and imperturbable—they extend it toward recollection of past lives. They recollect many kinds of past lives. That is: one, two, three, four, five, ten, twenty, thirty, forty, fifty, a hundred, a thousand, a hundred thousand rebirths; many eons of the world contracting, many eons of the world expanding, many eons of the world contracting and expanding. And so they recollect their many kinds of past lives, with features and details. 

When\marginnote{27{-}28.1} their mind has become immersed in \textsanskrit{samādhi} like this—purified, bright, flawless, rid of corruptions, pliable, workable, steady, and imperturbable—they extend it toward knowledge of the death and rebirth of sentient beings. With clairvoyance that is purified and superhuman, they see sentient beings passing away and being reborn—inferior and superior, beautiful and ugly, in a good place or a bad place. They understand how sentient beings are reborn according to their deeds. 

When\marginnote{29.1} their mind has become immersed in \textsanskrit{samādhi} like this—purified, bright, flawless, rid of corruptions, pliable, workable, steady, and imperturbable—they extend it toward knowledge of the ending of defilements. They truly understand: ‘This is suffering’ … ‘This is the origin of suffering’ … ‘This is the cessation of suffering’ … ‘This is the practice that leads to the cessation of suffering’. They truly understand: ‘These are defilements’ … ‘This is the origin of defilements’ … ‘This is the cessation of defilements’ … ‘This is the practice that leads to the cessation of defilements’. Knowing and seeing like this, their mind is freed from the defilements of sensuality, desire to be reborn, and ignorance. When they’re freed, they know they’re freed. 

They\marginnote{29.6} understand: ‘Rebirth is ended, the spiritual journey has been completed, what had to be done has been done, there is no return to any state of existence.’ 

Such\marginnote{30.1} a mendicant endures cold, heat, hunger, and thirst; the touch of flies, mosquitoes, wind, sun, and reptiles; rude and unwelcome criticism; and puts up with physical pain—sharp, severe, acute, unpleasant, disagreeable, and life-threatening. Rid of all greed, hate, and delusion, and purged of defects, they are worthy of offerings dedicated to the gods, worthy of hospitality, worthy of a religious donation, worthy of greeting with joined palms, and are the supreme field of merit for the world. 

If\marginnote{31.1} a royal bull elephant passes away untamed and untrained—whether in their old age, middle age, or youth—they’re considered a royal bull elephant who passed away untamed. In the same way, if a mendicant passes away without having ended the defilements—whether as a senior, middle, or junior—they’re considered as a mendicant who passed away untamed. 

If\marginnote{32.1} a royal bull elephant passes away tamed and trained—whether in their old age, middle age, or youth—they’re considered a royal bull elephant who passed away tamed. In the same way, if a mendicant passes away having ended the defilements—whether as a senior, middle, or junior—they’re considered as a mendicant who passed away tamed.” 

That\marginnote{32.7} is what the Buddha said. Satisfied, the novice Aciravata was happy with what the Buddha said. 

%
\section*{{\suttatitleacronym MN 126}{\suttatitletranslation With Bhūmija }{\suttatitleroot Bhūmijasutta}}
\addcontentsline{toc}{section}{\tocacronym{MN 126} \toctranslation{With Bhūmija } \tocroot{Bhūmijasutta}}
\markboth{With Bhūmija }{Bhūmijasutta}
\extramarks{MN 126}{MN 126}

\scevam{So\marginnote{1.1} I have heard. }At one time the Buddha was staying near \textsanskrit{Rājagaha}, in the Bamboo Grove, the squirrels’ feeding ground. 

Then\marginnote{2.1} Venerable \textsanskrit{Bhūmija} robed up in the morning and, taking his bowl and robe, went to the home of Prince Jayasena, where he sat on the seat spread out. 

Then\marginnote{3.1} Jayasena approached and exchanged greetings with him. When the greetings and polite conversation were over, he sat down to one side and said to \textsanskrit{Bhūmija}: 

“Master\marginnote{3.3} \textsanskrit{Bhūmija}, there are some ascetics and brahmins who have this doctrine and view: ‘If you make a wish and lead the spiritual life, you can’t win the fruit. If you don’t make a wish and lead the spiritual life, you can’t win the fruit. If you both make a wish and don’t make a wish and lead the spiritual life, you can’t win the fruit. If you neither make a wish nor don’t make a wish and lead the spiritual life, you can’t win the fruit.’ What does Master \textsanskrit{Bhūmija}’s Teacher say about this? How does he explain it?” 

“Prince,\marginnote{4.1} I haven’t heard and learned this in the presence of the Buddha. But it’s possible that he might explain it like this: ‘If you lead the spiritual life irrationally, you can’t win the fruit, regardless of whether you make a wish, you don’t make a wish, you both do and do not make a wish, or you neither do nor don’t make a wish. But if you lead the spiritual life rationally, you can win the fruit, regardless of whether you make a wish, you don’t make a wish, you both do and do not make a wish, or you neither do nor don’t make a wish.’ I haven’t heard and learned this in the presence of the Buddha. But it’s possible that he might explain it like that.” 

“If\marginnote{5.1} that’s what your teacher says, Master \textsanskrit{Bhūmija}, he clearly stands head and shoulders above all the various other ascetics and brahmins.” Then Prince Jayasena served Venerable \textsanskrit{Bhūmija} from his own dish. 

Then\marginnote{7.1} after the meal, on his return from almsround, \textsanskrit{Bhūmija} went to the Buddha, bowed, sat down to one side, and told him all that had happened, adding: “Answering this way, I trust that I repeated what the Buddha has said, and didn’t misrepresent him with an untruth. I trust my explanation was in line with the teaching, and that there are no legitimate grounds for rebuke or criticism.” 

“Indeed,\marginnote{8.1} \textsanskrit{Bhūmija}, in answering this way you repeated what I’ve said, and didn’t misrepresent me with an untruth. Your explanation was in line with the teaching, and there are no legitimate grounds for rebuke or criticism. 

There\marginnote{9.1} are some ascetics and brahmins who have wrong view, wrong thought, wrong speech, wrong action, wrong livelihood, wrong effort, wrong mindfulness, and wrong immersion. If they lead the spiritual life, they can’t win the fruit, regardless of whether they make a wish, they don’t make a wish, they both do and do not make a wish, or they neither do nor don’t make a wish. Why is that? Because that’s an irrational way to win the fruit. 

Suppose\marginnote{10.1} there was a person in need of oil. While wandering in search of oil, they tried heaping sand in a bucket, sprinkling it thoroughly with water, and pressing it out. But by doing this, they couldn’t extract any oil, regardless of whether they made a wish, didn’t make a wish, both did and did not make a wish, or neither did nor did not make a wish. Why is that? Because that’s an irrational way to extract oil. 

And\marginnote{10.8} so it is for any ascetics and brahmins who have wrong view, wrong thought, wrong speech, wrong action, wrong livelihood, wrong effort, wrong mindfulness, and wrong immersion. If they lead the spiritual life, they can’t win the fruit, regardless of whether or not they make a wish. Why is that? Because that’s an irrational way to win the fruit. 

Suppose\marginnote{11.1} there was a person in need of milk. While wandering in search of milk, they tried pulling the horn of a newly-calved cow. But by doing this, they couldn’t get any milk, regardless of whether they made a wish, didn’t make a wish, both did and did not make a wish, or neither did nor did not make a wish. Why is that? Because that’s an irrational way to get milk. 

And\marginnote{11.8} so it is for any ascetics and brahmins who have wrong view … Because that’s an irrational way to win the fruit. 

Suppose\marginnote{12.1} there was a person in need of butter. While wandering in search of butter, they tried pouring water into a pot and churning it with a stick. But by doing this, they couldn’t produce any butter, regardless of whether they made a wish, didn’t make a wish, both did and did not make a wish, or neither did nor did not make a wish. Why is that? Because that’s an irrational way to produce butter. 

And\marginnote{12.8} so it is for any ascetics and brahmins who have wrong view … Because that’s an irrational way to win the fruit. 

Suppose\marginnote{13.1} there was a person in need of fire. While wandering in search of fire, they tried drilling a green, sappy log with a drill-stick. But by doing this, they couldn’t start a fire, regardless of whether they made a wish, didn’t make a wish, both did and did not make a wish, or neither did nor did not make a wish. Why is that? Because that’s an irrational way to start a fire. 

And\marginnote{13.8} so it is for any ascetics and brahmins who have wrong view … Because that’s an irrational way to win the fruit. 

There\marginnote{14.1} are some ascetics and brahmins who have right view, right thought, right speech, right action, right livelihood, right effort, right mindfulness, and right immersion. If they lead the spiritual life, they can win the fruit, regardless of whether they make a wish, they don’t make a wish, they both do and do not make a wish, or they neither do nor do not make a wish. Why is that? Because that’s a rational way to win the fruit. 

Suppose\marginnote{15.1} there was a person in need of oil. While wandering in search of oil, they tried heaping sesame flour in a bucket, sprinkling it thoroughly with water, and pressing it out. By doing this, they could extract oil, regardless of whether they made a wish, didn’t make a wish, both did and did not make a wish, or neither did nor did not make a wish. Why is that? Because that’s a rational way to extract oil. 

And\marginnote{15.8} so it is for any ascetics and brahmins who have right view … Because that’s a rational way to win the fruit. 

Suppose\marginnote{16.1} there was a person in need of milk. While wandering in search of milk, they tried pulling the udder of a newly-calved cow. By doing this, they could get milk, regardless of whether they made a wish, didn’t make a wish, both did and did not make a wish, or neither did nor did not make a wish. Why is that? Because that’s a rational way to get milk. 

And\marginnote{16.8} so it is for any ascetics and brahmins who have right view … Because that’s a rational way to win the fruit. 

Suppose\marginnote{17.1} there was a person in need of butter. While wandering in search of butter, they tried pouring curds into a pot and churning them with a stick. By doing this, they could produce butter, regardless of whether they made a wish, didn’t make a wish, both did and did not make a wish, or neither did nor did not make a wish. Why is that? Because that’s a rational way to produce butter. 

And\marginnote{17.8} so it is for any ascetics and brahmins who have right view … Because that’s a rational way to win the fruit. 

Suppose\marginnote{18.1} there was a person in need of fire. While wandering in search of fire, they tried drilling a dried up, withered log with a drill-stick. By doing this, they could start a fire, regardless of whether they made a wish, didn’t make a wish, both did and did not make a wish, or neither did nor did not make a wish. Why is that? Because that’s a rational way to start a fire. 

And\marginnote{18.8} so it is for any ascetics and brahmins who have right view … Because that’s a rational way to win the fruit. 

\textsanskrit{Bhūmija},\marginnote{19.1} it wouldn’t be surprising if, had these four similes occurred to you, Prince Jayasena would have gained confidence in you and shown his confidence.” 

“But\marginnote{19.2} sir, how could these four similes have occurred to me as they did to the Buddha, since they were neither supernaturally inspired, nor learned before in the past?” 

That\marginnote{19.3} is what the Buddha said. Satisfied, Venerable \textsanskrit{Bhūmija} was happy with what the Buddha said. 

%
\section*{{\suttatitleacronym MN 127}{\suttatitletranslation With Anuruddha }{\suttatitleroot Anuruddhasutta}}
\addcontentsline{toc}{section}{\tocacronym{MN 127} \toctranslation{With Anuruddha } \tocroot{Anuruddhasutta}}
\markboth{With Anuruddha }{Anuruddhasutta}
\extramarks{MN 127}{MN 127}

\scevam{So\marginnote{1.1} I have heard. }At one time the Buddha was staying near \textsanskrit{Sāvatthī} in Jeta’s Grove, \textsanskrit{Anāthapiṇḍika}’s monastery. 

And\marginnote{2.1} then the master builder \textsanskrit{Pañcakaṅga} addressed a man, “Please, mister, go to Venerable Anuruddha, and in my name bow with your head to his feet. Say to him, ‘Sir, the master builder \textsanskrit{Pañcakaṅga} bows with his head to your feet.’ And then ask him whether he might please accept tomorrow’s meal from \textsanskrit{Pañcakaṅga} together with the mendicant \textsanskrit{Saṅgha}. And ask whether he might please come earlier than usual, for \textsanskrit{Pañcakaṅga} has many duties, and much work to do for the king.” 

“Yes,\marginnote{2.8} sir,” that man replied. He did as \textsanskrit{Pañcakaṅga} asked, and Venerable Anuruddha consented in silence. 

Then\marginnote{3.1} when the night had passed, Anuruddha robed up in the morning and, taking his bowl and robe, went to \textsanskrit{Pañcakaṅga}’s home, where he sat on the seat spread out. Then \textsanskrit{Pañcakaṅga} served and satisfied Anuruddha with his own hands with a variety of delicious foods. When Anuruddha had eaten and washed his hands and bowl, \textsanskrit{Pañcakaṅga} took a low seat, sat to one side, and said to him: 

“Sir,\marginnote{4.1} some senior mendicants have come to me and said, ‘Householder, develop the limitless release of heart.’ Others have said, ‘Householder, develop the expansive release of heart.’ Now, the limitless release of the heart and the expansive release of the heart: do these things differ in both meaning and phrasing? Or do they mean the same thing, and differ only in the phrasing?” 

“Well\marginnote{5.1} then, householder, let me know what you think about this. Afterwards you’ll get it for sure.” 

“Sir,\marginnote{5.2} this is what I think. The limitless release of the heart and the expansive release of the heart mean the same thing, and differ only in the phrasing.” 

“The\marginnote{6.1} limitless release of the heart and the expansive release of the heart differ in both meaning and phrasing. This is a way to understand how these things differ in both meaning and phrasing. 

And\marginnote{7.1} what is the limitless release of the heart? It’s when a mendicant meditates spreading a heart full of love to one direction, and to the second, and to the third, and to the fourth. In the same way above, below, across, everywhere, all around, they spread a heart full of love to the whole world—abundant, expansive, limitless, free of enmity and ill will. They meditate spreading a heart full of compassion … They meditate spreading a heart full of rejoicing … They meditate spreading a heart full of equanimity to one direction, and to the second, and to the third, and to the fourth. In the same way above, below, across, everywhere, all around, they spread a heart full of equanimity to the whole world—abundant, expansive, limitless, free of enmity and ill will. This is called the limitless release of the heart. 

And\marginnote{8.1} what is the expansive release of the heart? It’s when a mendicant meditates determined on pervading the extent of a single tree root as expansive. This is called the expansive release of the heart. Also, a mendicant meditates determined on pervading the extent of two or three tree roots … a single village district … two or three village districts … a single kingdom … two or three kingdoms … this land surrounded by ocean. This too is called the expansive release of the heart. This is a way to understand how these things differ in both meaning and phrasing. 

Householder,\marginnote{9.1} there are these four kinds of rebirth in a future life. What four? Take someone who meditates determined on pervading ‘limited radiance’. When their body breaks up, after death, they’re reborn in the company of the gods of limited radiance. Next, take someone who meditates determined on pervading ‘limitless radiance’. When their body breaks up, after death, they’re reborn in the company of the gods of limitless radiance. Next, take someone who meditates determined on pervading ‘corrupted radiance’. When their body breaks up, after death, they’re reborn in the company of the gods of corrupted radiance. Next, take someone who meditates determined on pervading ‘pure radiance’. When their body breaks up, after death, they’re reborn in the company of the gods of pure radiance. These are the four kinds of rebirth in a future life. 

There\marginnote{10.1} comes a time, householder, when the deities gather together as one. When they do so, a difference in their color is evident, but not in their radiance. It’s like when a person brings several oil lamps into one house. You can detect a difference in their flames, but not in their radiance. In the same way, when the deities gather together as one, a difference in their color is evident, but not in their radiance. 

There\marginnote{11.1} comes a time when those deities go their separate ways. When they do so, a difference both in their color and also in their radiance is evident. It’s like when a person takes those several oil lamps out of that house. You can detect a difference both in their flames and also in their radiance. In the same way, when the deities go their separate ways, a difference both in their color and also in their radiance is evident. 

It’s\marginnote{12.1} not that those deities think, ‘What we have is permanent, lasting, and eternal.’ Rather, wherever those deities cling, that’s where they take pleasure. It’s like when flies are being carried along on a carrying-pole or basket. It’s not that they think, ‘What we have is permanent, lasting, and eternal.’ Rather, wherever those flies cling, that’s where they take pleasure. In the same way, it’s not that those deities think, ‘What we have is permanent, lasting, and eternal.’ Rather, wherever those deities cling, that’s where they take pleasure.” 

When\marginnote{13.1} he had spoken, Venerable Sabhiya \textsanskrit{Kaccāna} said to Venerable Anuruddha: 

“Good,\marginnote{13.2} Venerable Anuruddha! I have a further question about this. Do all the radiant deities have limited radiance, or do some there have limitless radiance?” 

“In\marginnote{13.5} that respect, Reverend \textsanskrit{Kaccāna}, some deities there have limited radiance, while some have limitless radiance.” 

“What\marginnote{14.1} is the cause, Venerable Anuruddha, what is the reason why, when those deities have been reborn in a single order of gods, some deities there have limited radiance, while some have limitless radiance?” 

“Well\marginnote{14.2} then, Reverend \textsanskrit{Kaccāna}, I’ll ask you about this in return, and you can answer as you like. What do you think, Reverend \textsanskrit{Kaccāna}? Which of these two kinds of mental development is more expansive: when a mendicant meditates determined on pervading as expansive the extent of a single tree root, or two or three tree roots?” 

“When\marginnote{14.6} a mendicant meditates on two or three tree roots.” 

“What\marginnote{14.8} do you think, Reverend \textsanskrit{Kaccāna}? Which of these two kinds of mental development is more expansive: when a mendicant meditates determined on pervading as expansive the extent of two or three tree roots, or a single village district … two or three village districts … a single kingdom … two or three kingdoms … this land surrounded by ocean?” 

“When\marginnote{14.27} a mendicant meditates on this land surrounded by ocean.” 

“This\marginnote{14.29} is the cause, Reverend \textsanskrit{Kaccāna}, this is the reason why, when those deities have been reborn in a single order of gods, some deities there have limited radiance, while some have limitless radiance.” 

“Good,\marginnote{15.1} Venerable Anuruddha! I have a further question about this. Do all the radiant deities have corrupted radiance, or do some there have pure radiance?” 

“In\marginnote{15.4} that respect, Reverend \textsanskrit{Kaccāna}, some deities there have corrupted radiance, while some have pure radiance.” 

“What\marginnote{16.1} is the cause, Venerable Anuruddha, what is the reason why, when those deities have been reborn in a single order of gods, some deities there have corrupted radiance, while some have pure radiance?” 

“Well\marginnote{16.2} then, Reverend \textsanskrit{Kaccāna}, I shall give you a simile. For by means of a simile some sensible people understand the meaning of what is said. Suppose an oil lamp was burning with impure oil and impure wick. Because of the impurity of the oil and the wick it burns dimly, as it were. 

In\marginnote{16.6} the same way, take some mendicant who meditates determined on pervading ‘corrupted radiance’. Their physical discomfort is not completely settled, their dullness and drowsiness is not completely eradicated, and their restlessness and remorse is not completely eliminated. Because of this they practice absorption dimly, as it were. When their body breaks up, after death, they’re reborn in the company of the gods of corrupted radiance. 

Suppose\marginnote{16.10} an oil lamp was burning with pure oil and pure wick. Because of the purity of the oil and the wick it doesn’t burn dimly, as it were. 

In\marginnote{16.12} the same way, take some mendicant who meditates determined on pervading ‘pure radiance’. Their physical discomfort is completely settled, their dullness and drowsiness is completely eradicated, and their restlessness and remorse is completely eliminated. Because of this they don’t practice absorption dimly, as it were. When their body breaks up, after death, they’re reborn in the company of the gods of pure radiance. 

“This\marginnote{16.16} is the cause, Reverend \textsanskrit{Kaccāna}, this is the reason why, when those deities have been reborn in a single order of gods, some deities there have corrupted radiance, while some have pure radiance.” 

When\marginnote{17.1} he had spoken, Venerable Sabhiya \textsanskrit{Kaccāna} said to Venerable Anuruddha, “Good, Venerable Anuruddha! 

Venerable\marginnote{17.3} Anuruddha, you don’t say, ‘So I have heard’ or ‘It ought to be like this.’ Rather, you say: ‘These deities are like this, those deities are like that.’ Sir, it occurs to me, ‘Clearly, Venerable Anuruddha has previously lived together with those deities, conversed, and engaged in discussion.’” 

“Your\marginnote{17.8} words are clearly invasive and intrusive, Reverend \textsanskrit{Kaccāna}. Nevertheless, I will answer you. For a long time I have previously lived together with those deities, conversed, and engaged in discussion.” 

When\marginnote{18.1} he had spoken, Venerable Sabhiya \textsanskrit{Kaccāna} said to \textsanskrit{Pañcakaṅga} the master builder, “You’re fortunate, householder, so very fortunate, to have given up your state of uncertainty, and to have got the chance to listen to this exposition of the teaching.” 

%
\section*{{\suttatitleacronym MN 128}{\suttatitletranslation Corruptions }{\suttatitleroot Upakkilesasutta}}
\addcontentsline{toc}{section}{\tocacronym{MN 128} \toctranslation{Corruptions } \tocroot{Upakkilesasutta}}
\markboth{Corruptions }{Upakkilesasutta}
\extramarks{MN 128}{MN 128}

\scevam{So\marginnote{1.1} I have heard. }At one time the Buddha was staying near Kosambi, in Ghosita’s Monastery. 

Now\marginnote{2.1} at that time the mendicants of Kosambi were arguing, quarreling, and disputing, continually wounding each other with barbed words. 

Then\marginnote{3.1} a mendicant went up to the Buddha, bowed, stood to one side, and told him what was happening, adding: “Please, sir go to those mendicants out of compassion.” The Buddha consented in silence. 

Then\marginnote{4.1} the Buddha went up to those mendicants and said, “Enough, mendicants! Stop arguing, quarreling, and disputing.” 

When\marginnote{4.3} he said this, one of the mendicants said to the Buddha, “Wait, sir! Let the Buddha, the Lord of the Dhamma, remain passive, dwelling in blissful meditation in the present life. We will be known for this arguing, quarreling, and disputing.” 

For\marginnote{4.8} a second time … and a third time the Buddha said to those mendicants, “Enough, mendicants! Stop arguing, quarreling, and disputing.” 

For\marginnote{4.17} a third time that mendicant said to the Buddha, “Wait, sir! Let the Buddha, the Lord of the Dhamma, remain passive, dwelling in blissful meditation in the present life. We will be known for this arguing, quarreling, and disputing.” 

Then\marginnote{5.1} the Buddha robed up in the morning and, taking his bowl and robe, entered Kosambi for alms. After the meal, on his return from almsround, he set his lodgings in order. Taking his bowl and robe, he recited these verses while standing right there: 

\begin{verse}%
“When\marginnote{6.1} many voices shout at once, \\
no-one thinks that they’re a fool! \\
While the \textsanskrit{Saṅgha}’s being split, \\
none thought another to be better. 

Dolts\marginnote{6.5} pretending to be astute, \\
they talk, their words right out of bounds. \\
They blab at will, their mouths agape, \\
and no-one knows what leads them on. 

“They\marginnote{6.9} abused me, they hit me! \\
They beat me, they robbed me!” \\
For those who bear such a grudge, \\
hatred never ends. 

“They\marginnote{6.13} abused me, they hit me! \\
They beat me, they robbed me!” \\
For those who bear no such grudge, \\
hatred has an end. 

For\marginnote{6.17} never is hatred \\
settled by hate, \\
it’s only settled by love: \\
this is an eternal truth. 

Others\marginnote{6.21} don’t understand \\
that here we need to be restrained. \\
But those who do understand this, \\
being clever, settle their conflicts. 

Breakers\marginnote{6.25} of bones and takers of life, \\
thieves of cattle, horses, wealth, \\
those who plunder the nation: \\
even they can come together, \\
so why on earth can’t you? 

If\marginnote{6.30} you find an alert companion, \\
a wise and virtuous friend, \\
then, overcoming all adversities, \\
wander with them, joyful and mindful. 

If\marginnote{6.34} you find no alert companion, \\
no wise and virtuous friend, \\
then, like a king who flees his conquered realm, \\
wander alone like a tusker in the wilds. 

It’s\marginnote{6.38} better to wander alone, \\
there’s no fellowship with fools. \\
Wander alone and do no wrong, \\
at ease like a tusker in the wilds.” 

%
\end{verse}

After\marginnote{7.1} speaking these verses while standing, the Buddha went to the village of the child salt-miners, where Venerable Bhagu was staying at the time. Bhagu saw the Buddha coming off in the distance, so he spread out a seat and placed water for washing the feet. The Buddha sat on the seat spread out, and washed his feet. Bhagu bowed to the Buddha and sat down to one side. 

The\marginnote{7.8} Buddha said to him, “I hope you’re keeping well, mendicant; I hope you’re all right. And I hope you’re having no trouble getting almsfood.” 

“I’m\marginnote{7.10} keeping well, sir; I’m all right. And I’m having no trouble getting almsfood.” 

Then\marginnote{7.11} the Buddha educated, encouraged, fired up, and inspired Bhagu with a Dhamma talk, after which he got up from his seat and set out for the Eastern Bamboo Park. 

Now\marginnote{8.1} at that time the venerables Anuruddha, Nandiya, and Kimbila were staying in the Eastern Bamboo Park. The park keeper saw the Buddha coming off in the distance and said to the Buddha, “Don’t come into this park, ascetic. There are three gentlemen who love themselves staying here. Don’t disturb them.” 

Anuruddha\marginnote{9.1} heard the park keeper conversing with the Buddha, and said to him, “Don’t keep the Buddha out, good park keeper! Our Teacher, the Blessed One, has arrived.” 

Then\marginnote{10.1} Anuruddha went to Nandiya and Kimbila, and said to them, “Come forth, venerables, come forth! Our Teacher, the Blessed One, has arrived!” 

Then\marginnote{10.3} Anuruddha, Nandiya, and Kimbila came out to greet the Buddha. One received his bowl and robe, one spread out a seat, and one set out water for washing his feet. The Buddha sat on the seat spread out and washed his feet. Those venerables bowed and sat down to one side. 

The\marginnote{10.8} Buddha said to Anuruddha, “I hope you’re keeping well, Anuruddha and friends; I hope you’re all right. And I hope you’re having no trouble getting almsfood.” 

“We’re\marginnote{10.10} keeping well, sir; we’re all right. And we’re having no trouble getting almsfood.” 

“I\marginnote{11.1} hope you’re living in harmony, appreciating each other, without quarreling, blending like milk and water, and regarding each other with kindly eyes?” 

“Indeed,\marginnote{11.2} sir, we live in harmony as you say.” 

“But\marginnote{11.3} how do you live this way?” 

“In\marginnote{12.1} this case, sir, I think: ‘I’m fortunate, so very fortunate, to live together with spiritual companions such as these.’ I consistently treat these venerables with kindness by way of body, speech, and mind, both in public and in private. I think: ‘Why don’t I set aside my own ideas and just go along with these venerables’ ideas?’ And that’s what I do. Though we’re different in body, sir, we’re one in mind, it seems to me.” 

And\marginnote{12.11} the venerables Nandiya and Kimbila spoke likewise, and they added: “That’s how we live in harmony, appreciating each other, without quarreling, blending like milk and water, and regarding each other with kindly eyes.” 

“Good,\marginnote{13.1} good, Anuruddha and friends! But I hope you’re living diligently, keen, and resolute?” 

“Indeed,\marginnote{14.1} sir, we live diligently.” 

“But\marginnote{14.2} how do you live this way?” 

“In\marginnote{14.3} this case, sir, whoever returns first from almsround prepares the seats, and puts out the drinking water and the rubbish bin. If there’s anything left over, whoever returns last eats it if they like. Otherwise they throw it out where there is little that grows, or drop it into water that has no living creatures. Then they put away the seats, drinking water, and rubbish bin, and sweep the refectory. If someone sees that the pot of water for washing, drinking, or the toilet is empty they set it up. If he can’t do it, he summons another with a wave of the hand, and they set it up by lifting it with their hands. But we don’t break into speech for that reason. And every five days we sit together for the whole night and discuss the teachings. That’s how we live diligently, keen, and resolute.” 

“Good,\marginnote{15.1} good, Anuruddha and friends! But as you live diligently like this, have you achieved any superhuman distinction in knowledge and vision worthy of the noble ones, a meditation at ease?” 

“Well,\marginnote{15.3} sir, while meditating diligent, keen, and resolute, we perceive both light and vision of forms. But before long the light and the vision of forms vanish. We haven’t worked out the reason for that.” 

“Well,\marginnote{16.1} you should work out the reason for that. Before my awakening—when I was still unawakened but intent on awakening—I too perceived both light and vision of forms. But before long my light and vision of forms vanished. It occurred to me: ‘What’s the cause, what’s the reason why my light and vision of forms vanish?’ It occurred to me: ‘Doubt arose in me, and because of that my immersion fell away. When immersion falls away, the light and vision of forms vanish. I’ll make sure that doubt will not arise in me again.’ 

While\marginnote{17.1} meditating diligent, keen, and resolute, I perceived both light and vision of forms. But before long my light and vision of forms vanished. It occurred to me: ‘What’s the cause, what’s the reason why my light and vision of forms vanish?’ It occurred to me: ‘Loss of focus arose in me, and because of that my immersion fell away. When immersion falls away, the light and vision of forms vanish. I’ll make sure that neither doubt nor loss of focus will arise in me again.’ 

While\marginnote{18.1} meditating … ‘Dullness and drowsiness arose in me … I’ll make sure that neither doubt nor loss of focus nor dullness and drowsiness will arise in me again.’ 

While\marginnote{19.1} meditating … ‘Terror arose in me, and because of that my immersion fell away. When immersion falls away, the light and vision of forms vanish. Suppose a person was traveling along a road, and killers were to spring out at them from both sides. They’d feel terrified because of that. In the same way, terror arose in me … I’ll make sure that neither doubt nor loss of focus nor dullness and drowsiness nor terror will arise in me again.’ 

While\marginnote{20.1} meditating … ‘Excitement arose in me, and because of that my immersion fell away. When immersion falls away, the light and vision of forms vanish. Suppose a person was looking for an entrance to a hidden treasure. And all at once they’d come across five entrances! They’d feel excited because of that. In the same way, excitement arose in me … I’ll make sure that neither doubt nor loss of focus nor dullness and drowsiness nor terror nor excitement will arise in me again.’ 

While\marginnote{21.1} meditating … ‘Discomfort arose in me … I’ll make sure that neither doubt nor loss of focus nor dullness and drowsiness nor terror nor excitement nor discomfort will arise in me again.’ 

While\marginnote{22.1} meditating … ‘Excessive energy arose in me, and because of that my immersion fell away. When immersion falls away, the light and vision of forms vanish. Suppose a person was to grip a quail too tightly in this hands—it would die right there. I’ll make sure that neither doubt nor loss of focus nor dullness and drowsiness nor terror nor excitement nor discomfort nor excessive energy will arise in me again.’ 

While\marginnote{23.1} meditating … ‘Overly lax energy arose in me, and because of that my immersion fell away. When immersion falls away, the light and vision of forms vanish. Suppose a person was to grip a quail too loosely—it would fly out of their hands. I’ll make sure that neither doubt nor loss of focus nor dullness and drowsiness nor terror nor excitement nor discomfort nor excessive energy nor overly lax energy will arise in me again.’ 

While\marginnote{24.1} meditating … ‘Longing arose in me … I’ll make sure that neither doubt nor loss of focus nor dullness and drowsiness nor terror nor excitement nor discomfort nor excessive energy nor overly lax energy nor longing will arise in me again.’ 

While\marginnote{25.1} meditating … ‘Perceptions of diversity arose in me … I’ll make sure that neither doubt nor loss of focus nor dullness and drowsiness nor terror nor excitement nor discomfort nor excessive energy nor overly lax energy nor longing nor perception of diversity will arise in me again.’ 

While\marginnote{26.1} meditating diligent, keen, and resolute, I perceived both light and vision of forms. But before long my light and vision of forms vanished. It occurred to me: ‘What’s the cause, what’s the reason why my light and vision of forms vanish?’ It occurred to me: ‘Excessive concentration on forms arose in me, and because of that my immersion fell away. When immersion falls away, the light and vision of forms vanish. I’ll make sure that neither doubt nor loss of focus nor dullness and drowsiness nor terror nor excitement nor discomfort nor excessive energy nor overly lax energy nor longing nor perception of diversity nor excessive concentration on forms will arise in me again.’ 

When\marginnote{27.1} I understood that doubt is a corruption of the mind, I gave it up. When I understood that loss of focus, dullness and drowsiness, terror, excitement, discomfort, excessive energy, overly lax energy, longing, perception of diversity, and excessive concentration on forms are corruptions of the mind, I gave them up. 

While\marginnote{28.1} meditating diligent, keen, and resolute, I perceived light but did not see forms, or I saw forms, but did not see light. And this went on for a whole night, a whole day, even a whole night and day. I thought: ‘What is the cause, what is the reason for this?’ It occurred to me: ‘When I don’t focus on the foundation of the forms, but focus on the foundation of the light, then I perceive light and do not see forms. But when I don’t focus on the foundation of the light, but focus on the foundation of the forms, then I see forms and do not perceive light. And this goes on for a whole night, a whole day, even a whole night and day.’ 

While\marginnote{29.1} meditating diligent, keen, and resolute, I perceived limited light and saw limited forms, or I perceived limitless light and saw limitless forms. And this went on for a whole night, a whole day, even a whole night and day. I thought: ‘What is the cause, what is the reason for this?’ It occurred to me: ‘When my immersion is limited, then my vision is limited, and with limited vision I perceive limited light and see limited forms. But when my immersion is limitless, then my vision is limitless, and with limitless vision I perceive limitless light and see limitless forms. And this goes on for a whole night, a whole day, even a whole night and day.’ 

After\marginnote{30.1} understanding that doubt, loss of focus, dullness and drowsiness, terror, excitement, discomfort, excessive energy, overly lax energy, longing, perception of diversity, and excessive concentration on forms are corruptions of the mind, I had given them up. 

I\marginnote{31.1} thought: ‘I’ve given up my mental corruptions. Now let me develop immersion in three ways.’ I developed immersion while placing the mind and keeping it connected; without placing the mind, but just keeping it connected; without placing the mind or keeping it connected; with rapture; without rapture; with pleasure; with equanimity. 

When\marginnote{32.1} I had developed immersion in these ways, the knowledge and vision arose in me: ‘My freedom is unshakable; this is my last rebirth; now there are no more future lives.’” 

That\marginnote{32.4} is what the Buddha said. Satisfied, Venerable Anuruddha was happy with what the Buddha said. 

%
\section*{{\suttatitleacronym MN 129}{\suttatitletranslation The Foolish and the Astute }{\suttatitleroot Bālapaṇḍitasutta}}
\addcontentsline{toc}{section}{\tocacronym{MN 129} \toctranslation{The Foolish and the Astute } \tocroot{Bālapaṇḍitasutta}}
\markboth{The Foolish and the Astute }{Bālapaṇḍitasutta}
\extramarks{MN 129}{MN 129}

\scevam{So\marginnote{1.1} I have heard. }At one time the Buddha was staying near \textsanskrit{Sāvatthī} in Jeta’s Grove, \textsanskrit{Anāthapiṇḍika}’s monastery. There the Buddha addressed the mendicants, “Mendicants!” 

“Venerable\marginnote{1.5} sir,” they replied. The Buddha said this: 

“These\marginnote{2.1} are the three characteristics, signs, and manifestations of a fool. What three? A fool thinks poorly, speaks poorly, and acts poorly. If a fool didn’t think poorly, speak poorly, and act poorly, then how would the astute know of them, ‘This fellow is a fool, a bad person’? But since a fool does think poorly, speak poorly, and act poorly, then the astute do know of them, ‘This fellow is a fool, a bad person’. 

A\marginnote{3.1} fool experiences three kinds of suffering and sadness in the present life. 

Suppose\marginnote{3.2} a fool is sitting in a council hall, a street, or a crossroad, where people are discussing what is proper and fitting. And suppose that fool is someone who kills living creatures, steals, commits sexual misconduct, lies, and uses alcoholic drinks that cause negligence. Then that fool thinks, ‘These people are discussing what is proper and fitting. But those bad things are found in me and I am seen in them!’ This is the first kind of suffering and sadness that a fool experiences in the present life. 

Furthermore,\marginnote{4.1} a fool sees that the kings have arrested a bandit, a criminal, and subjected them to various punishments—whipping, caning, and clubbing; cutting off hands or feet, or both; cutting off ears or nose, or both; the ‘porridge pot’, the ‘shell-shave’, the ‘demon’s mouth’, the ‘garland of fire’, the ‘burning hand’, the ‘grass blades’, the ‘bark dress’, the ‘antelope’, the ‘meat hook’, the ‘coins’, the ‘caustic pickle’, the ‘twisting bar’, the ‘straw mat’; being splashed with hot oil, being fed to the dogs, being impaled alive, and being beheaded. Then that fool thinks, ‘The kinds of deeds for which the kings inflict such punishments—those things are found in me and I am seen in them! If the kings find out about me, they will inflict the same kinds of punishments on me!’ This is the second kind of suffering and sadness that a fool experiences in the present life. 

Furthermore,\marginnote{5.1} when a fool is resting on a chair or a bed or on the ground, their past bad deeds—misconduct of body, speech, and mind—settle down upon them, rest down upon them, and lay down upon them. It is like the shadow of a great mountain peak in the evening as it settles down, rests down, and lays down upon the earth. In the same way, when a fool is resting on a chair or a bed or on the ground, their past bad deeds—misconduct of body, speech, and mind—settle down upon them, rest down upon them, and lay down upon them. Then that fool thinks, ‘Well, I haven’t done good and skillful things that keep me safe. And I have done bad, violent, and depraved things. When I depart, I’ll go to the place where people who’ve done such things go.’ They sorrow and wail and lament, beating their breasts and falling into confusion. This is the third kind of suffering and sadness that a fool experiences in the present life. 

Having\marginnote{6.1} done bad things by way of body, speech, and mind, when their body breaks up, after death, they’re reborn in a place of loss, a bad place, the underworld, hell. 

And\marginnote{7.1} if there’s anything of which it may be rightly said that it is utterly unlikable, undesirable, and disagreeable, it is of hell that this should be said. So much so that it’s not easy to give a simile for how painful hell is.” 

When\marginnote{7.5} he said this, one of the mendicants asked the Buddha, “But sir, is it possible to give a simile?” 

“It’s\marginnote{8.1} possible,” said the Buddha. 

“Suppose\marginnote{8.2} they arrest a bandit, a criminal and present him to the king, saying, ‘Your Majesty, this is a bandit, a criminal. Punish him as you will.’ The king would say, ‘Go, my men, and strike this man in the morning with a hundred spears!’ The king’s men did as they were told. Then at midday the king would say, ‘My men, how is that man?’ ‘He’s still alive, Your Majesty.’ The king would say, ‘Go, my men, and strike this man in the midday with a hundred spears!’ The king’s men did as they were told. Then late in the afternoon the king would say, ‘My men, how is that man?’ ‘He’s still alive, Your Majesty.’ The king would say, ‘Go, my men, and strike this man in the late afternoon with a hundred spears!’ The king’s men did as they were told. 

What\marginnote{8.19} do you think, mendicants? Would that man experience pain and distress from being struck with three hundred spears?” 

“Sir,\marginnote{8.21} that man would experience pain and distress from being struck with one spear, let alone three hundred spears!” 

Then\marginnote{9.1} the Buddha, picking up a stone the size of his palm, addressed the mendicants, “What do you think, mendicants? Which is bigger: the stone the size of my palm that I’ve picked up, or the Himalayas, the king of mountains?” 

“Sir,\marginnote{9.4} the stone you’ve picked up is tiny. Compared to the Himalayas, it doesn’t count, it’s not worth a fraction, there’s no comparison.” 

“In\marginnote{9.5} the same way, compared to the suffering in hell, the pain and distress experienced by that man due to being struck with three hundred spears doesn’t count, it’s not worth a fraction, there’s no comparison. 

The\marginnote{10.1} wardens of hell punish them with the five-fold crucifixion. They drive red-hot stakes through the hands and feet, and another in the middle of the chest. And there they feel painful, sharp, severe, acute feelings—but they don’t die until that bad deed is eliminated. 

The\marginnote{11.1} wardens of hell throw them down and hack them with axes. … 

They\marginnote{12.1} hang them upside-down and hack them with hatchets. … 

They\marginnote{13.1} harness them to a chariot, and drive them back and forth across burning ground, blazing and glowing. … 

They\marginnote{14.1} make them climb up and down a huge mountain of burning coals, blazing and glowing. … 

The\marginnote{15.1} wardens of hell turn them upside down and throw them into a red-hot copper pot, burning, blazing, and glowing. There they’re seared in boiling scum, and they’re swept up and down and round and round. And there they feel painful, sharp, severe, acute feelings—but they don’t die until that bad deed is eliminated. 

The\marginnote{16.1} wardens of hell toss them in the Great Hell. Now, about that Great Hell: 

\begin{verse}%
‘Four\marginnote{16.3} are its corners, four its doors, \\
neatly divided in equal parts. \\
Surrounded by an iron wall, \\
of iron is its roof. 

The\marginnote{16.7} ground is even made of iron, \\
it burns with fierce fire. \\
The heat forever radiates \\
a hundred leagues around.’ 

%
\end{verse}

I\marginnote{17.1} could tell you many different things about hell. So much so that it’s not easy to completely describe the suffering in hell. 

There\marginnote{18.1} are, mendicants, animals that feed on grass. They eat by cropping fresh or dried grass with their teeth. And what animals feed on grass? Elephants, horses, cattle, donkeys, goats, deer, and various others. A fool who used to be a glutton here and did bad deeds here, when their body breaks up, after death, is reborn in the company of those sentient beings who feed on grass. 

There\marginnote{19.1} are animals that feed on dung. When they catch a whiff of dung they run to it, thinking, ‘There we’ll eat! There we’ll eat!’ It’s like when brahmins smell a burnt offering, they run to it, thinking, ‘There we’ll eat! There we’ll eat!’ In the same way, there are animals that feed on dung. When they catch a whiff of dung they run to it, thinking, ‘There we’ll eat! There we’ll eat!’ And what animals feed on dung? Chickens, pigs, dogs, jackals, and various others. A fool who used to be a glutton here and did bad deeds here, after death is reborn in the company of those sentient beings who feed on dung. 

There\marginnote{20.1} are animals who are born, live, and die in darkness. And what animals are born, live, and die in darkness? Moths, maggots, earthworms, and various others. A fool who used to be a glutton here and did bad deeds here, after death is reborn in the company of those sentient beings who are born, live, and die in darkness. 

There\marginnote{21.1} are animals who are born, live, and die in water. And what animals are born, live, and die in water? Fish, turtles, crocodiles, and various others. A fool who used to be a glutton here and did bad deeds here, after death is reborn in the company of those sentient beings who are born, live, and die in water. 

There\marginnote{22.1} are animals who are born, live, and die in filth. And what animals are born, live, and die in filth? Those animals that are born, live, and die in a rotten fish, a rotten corpse, rotten porridge, or a sewer. A fool who used to be a glutton here and did bad deeds here, after death is reborn in the company of those sentient beings who are born, live, and die in filth. 

I\marginnote{23.1} could tell you many different things about the animal realm. So much so that it’s not easy to completely describe the suffering in the animal realm. 

Mendicants,\marginnote{24.1} suppose a person were to throw a yoke with a single hole into the ocean. The east wind wafts it west; the west wind wafts it east; the north wind wafts it south; and the south wind wafts it north. And there was a one-eyed turtle who popped up once every hundred years. 

What\marginnote{24.4} do you think, mendicants? Would that one-eyed turtle still poke its neck through the hole in that yoke?” 

“No,\marginnote{24.6} sir. Only after a very long time, sir, if ever.” 

“That\marginnote{24.8} one-eyed turtle would poke its neck through the hole in that yoke sooner than a fool who has fallen to the underworld would be reborn as a human being, I say. Why is that? Because in that place there’s no principled or moral conduct, and no doing what is good and skillful. There they just prey on each other, preying on the weak. 

And\marginnote{25.1} suppose that fool, after a very long time, returned to the human realm. They’d be reborn in a low class family—a family of outcastes, hunters, bamboo-workers, chariot-makers, or waste-collectors. Such families are poor, with little to eat or drink, where life is tough, and food and shelter are hard to find. And they’d be ugly, unsightly, deformed, chronically ill—one-eyed, crippled, lame, or half-paralyzed. They don’t get to have food, drink, clothes, and vehicles; garlands, perfumes, and makeup; or bed, house, and lighting. And they do bad things by way of body, speech, and mind. When their body breaks up, after death, they’re reborn in a place of loss, a bad place, the underworld, hell. 

Suppose\marginnote{26.1} a gambler on the first unlucky throw were to lose his wife and child, all his property, and then get thrown in jail. But such an unlucky throw is trivial compared to the unlucky throw whereby a fool, having done bad things by way of body, speech, and mind, when their body breaks up, after death, is reborn in a place of loss, a bad place, the underworld, hell. This is the total fulfillment of the fool’s level. 

There\marginnote{27.1} are these three characteristics, signs, and manifestations of an astute person. What three? An astute person thinks well, speaks well, and acts well. If an astute person didn’t think well, speak well, and act well, then how would the astute know of them, ‘This fellow is astute, a good person’? 

But\marginnote{28.1} since an astute person does think well, speak well, and act well, then the astute do know of them, ‘This fellow is astute, a good person’. An astute person experiences three kinds of pleasure and happiness in the present life. Suppose an astute person is sitting in a council hall, a street, or a crossroad, where people are discussing about what is proper and fitting. And suppose that astute person is someone who refrains from killing living creatures, stealing, committing sexual misconduct, lying, and alcoholic drinks that cause negligence. Then that astute person thinks, ‘These people are discussing what is proper and fitting. And those good things are found in me and I am seen in them.’ This is the first kind of pleasure and happiness that an astute person experiences in the present life. 

Furthermore,\marginnote{29.1} an astute person sees that the kings have arrested a bandit, a criminal, and subjected them to various punishments—whipping, caning, and clubbing; cutting off hands or feet, or both; cutting off ears or nose, or both; the ‘porridge pot’, the ‘shell-shave’, the ‘demon’s mouth’, the ‘garland of fire’, the ‘burning hand’, the ‘grass blades’, the ‘bark dress’, the ‘antelope’, the ‘meat hook’, the ‘coins’, the ‘caustic pickle’, the ‘twisting bar’, the ‘straw mat’; being splashed with hot oil, being fed to the dogs, being impaled alive, and being beheaded. Then that astute person thinks, ‘The kinds of deeds for which the kings inflict such punishments—those things are not found in me and I am not seen in them!’ This is the second kind of pleasure and happiness that an astute person experiences in the present life. 

Furthermore,\marginnote{30.1} when an astute person is resting on a chair or a bed or on the ground, their past good deeds—good conduct of body, speech, and mind—settle down upon them, rest down upon them, and lay down upon them. It is like the shadow of a great mountain peak in the evening as it settles down, rests down, and lays down upon the earth. In the same way, when an astute person is resting on a chair or a bed or on the ground, their past good deeds—good conduct of body, speech, and mind—settle down upon them, rest down upon them, and lay down upon them. Then that astute person thinks, ‘Well, I haven’t done bad, violent, and depraved things. And I have done good and skillful deeds that keep me safe. When I pass away, I’ll go to the place where people who’ve done such things go.’ So they don’t sorrow and wail and lament, beating their breast and falling into confusion. This is the third kind of pleasure and happiness that an astute person experiences in the present life. 

When\marginnote{31.1} their body breaks up, after death, they’re reborn in a good place, a heavenly realm. 

And\marginnote{32.1} if there’s anything of which it may be rightly said that it is utterly likable, desirable, and agreeable, it is of heaven that this should be said. So much so that it’s not easy to give a simile for how pleasurable heaven is.” 

When\marginnote{32.5} he said this, one of the mendicants asked the Buddha, “But sir, is it possible to give a simile?” 

“It’s\marginnote{33.1} possible,” said the Buddha. 

“Suppose\marginnote{33.2} there was a king, a wheel-turning monarch who possessed seven treasures and four blessings, and experienced pleasure and happiness because of them. 

What\marginnote{34.1} seven? It’s when, on the fifteenth day sabbath, an anointed aristocratic king has bathed his head and gone upstairs in the royal longhouse to observe the sabbath. And the heavenly wheel-treasure appears to him, with a thousand spokes, with rim and hub, complete in every detail. Seeing this, the king thinks, ‘I have heard that when the heavenly wheel-treasure appears to a king in this way, he becomes a wheel-turning monarch. Am I then a wheel-turning monarch?’ 

Then\marginnote{35.1} the anointed aristocratic king, taking a ceremonial vase in his left hand, besprinkled the wheel-treasure with his right hand, saying, ‘Roll forth, O wheel-treasure! Triumph, O wheel-treasure!’ Then the wheel-treasure rolls towards the east. And the king follows it together with his army of four divisions. In whatever place the wheel-treasure stands still, there the king comes to stay together with his army. And any opposing rulers of the eastern quarter come to the wheel-turning monarch and say, ‘Come, great king! Welcome, great king! We are yours, great king, instruct us.’ The wheel-turning monarch says, ‘Do not kill living creatures. Do not steal. Do not commit sexual misconduct. Do not lie. Do not drink alcohol. Maintain the current level of taxation.’ And so the opposing rulers of the eastern quarter become his vassals. 

Then\marginnote{35.9} the wheel-treasure, having plunged into the eastern ocean and emerged again, rolls towards the south. … Having plunged into the southern ocean and emerged again, it rolls towards the west. … Having plunged into the western ocean and emerged again, it rolls towards the north, followed by the king together with his army of four divisions. In whatever place the wheel-treasure stands still, there the king comes to stay together with his army. 

And\marginnote{35.12} any opposing rulers of the northern quarter come to the wheel-turning monarch and say, ‘Come, great king! Welcome, great king! We are yours, great king, instruct us.’ The wheel-turning monarch says, ‘Do not kill living creatures. Do not steal. Do not commit sexual misconduct. Do not lie. Do not drink alcohol. Maintain the current level of taxation.’ And so the rulers of the northern quarter become his vassals. 

And\marginnote{35.17} then the wheel-treasure, having triumphed over this land surrounded by ocean, returns to the royal capital. There it stands still at the gate to the royal compound as if fixed to an axle, illuminating the royal compound. Such is the wheel-treasure that appears to the wheel-turning monarch. 

Next,\marginnote{36.1} the elephant-treasure appears to the wheel-turning monarch. It was an all-white sky-walker with psychic power, touching the ground in seven places, a king of elephants named Sabbath. Seeing him, the king was impressed, ‘This would truly be a fine elephant vehicle, if he would submit to taming.’ Then the elephant-treasure submitted to taming, as if he were a fine thoroughbred elephant that had been tamed for a long time. Once it so happened that the wheel-turning monarch, testing that same elephant-treasure, mounted him in the morning and traversed the land surrounded by ocean before returning to the royal capital in time for breakfast. Such is the elephant-treasure that appears to the wheel-turning monarch. 

Next,\marginnote{37.1} the horse-treasure appears to the wheel-turning monarch. It was an all-white sky-walker with psychic power, with head of black and mane like woven reeds, a royal steed named Thundercloud. Seeing him, the king was impressed, ‘This would truly be a fine horse vehicle, if he would submit to taming.’ Then the horse-treasure submitted to taming, as if he were a fine thoroughbred horse that had been tamed for a long time. Once it so happened that the wheel-turning monarch, testing that same horse-treasure, mounted him in the morning and traversed the land surrounded by ocean before returning to the royal capital in time for breakfast. Such is the horse-treasure that appears to the wheel-turning monarch. 

Next,\marginnote{38.1} the jewel-treasure appears to the wheel-turning monarch. It is a beryl gem that’s naturally beautiful, eight-faceted, well-worked. And the radiance of that jewel spreads all-round for a league. Once it so happened that the wheel-turning monarch, testing that same jewel-treasure, mobilized his army of four divisions and, with the jewel hoisted on his banner, set out in the dark of the night. Then the villagers around them set off to work, thinking that it was day. Such is the jewel-treasure that appears to the wheel-turning monarch. 

Next,\marginnote{39.1} the woman-treasure appears to the wheel-turning monarch. She is attractive, good-looking, lovely, of surpassing beauty. She’s neither too tall nor too short; neither too thin nor too fat; neither too dark nor too light. She outdoes human beauty without reaching divine beauty. And her touch is like a tuft of cotton-wool or kapok. When it’s cool her limbs are warm, and when it’s warm her limbs are cool. The fragrance of sandal floats from her body, and lotus from her mouth. She gets up before the king and goes to bed after him, and is obliging, behaving nicely and speaking politely. The woman-treasure does not betray the wheel-turning monarch even in thought, still less in deed. Such is the woman-treasure who appears to the wheel-turning monarch. 

Next,\marginnote{40.1} the householder-treasure appears to the wheel-turning monarch. The power of clairvoyance manifests in him as a result of past deeds, by which he sees hidden treasure, both owned and ownerless. He approaches the wheel-turning monarch and says, ‘Relax, sire. I will take care of the treasury.’ Once it so happened that the wheel-turning monarch, testing that same householder-treasure, boarded a boat and sailed to the middle of the Ganges river. Then he said to the householder-treasure, ‘Householder, I need gold coins and bullion.’ ‘Well then, great king, draw the boat up to one shore.’ ‘It’s right here, householder, that I need gold coins and bullion.’ Then that householder-treasure, immersing both hands in the water, pulled up a pot full of gold coin and bullion, and said to the king, ‘Is this sufficient, great king? Has enough been done, great king, enough offered?’ The wheel-turning monarch said, ‘That is sufficient, householder. Enough has been done, enough offered.’ Such is the householder-treasure that appears to the wheel-turning monarch. 

Next,\marginnote{41.1} the counselor-treasure appears to the wheel-turning monarch. He is astute, competent, intelligent, and capable of getting the king to appoint who should be appointed, dismiss who should be dismissed, and retain who should be retained. He approaches the wheel-turning monarch and says, ‘Relax, sire. I shall issue instructions.’ Such is the counselor-treasure that appears to the wheel-turning monarch. These are the seven treasures possessed by a wheel-turning monarch. 

And\marginnote{42.1} what are the four blessings? 

A\marginnote{42.2} wheel-turning monarch is attractive, good-looking, lovely, of surpassing beauty, more so than other people. This is the first blessing. 

Furthermore,\marginnote{43.1} he is long-lived, more so than other people. This is the second blessing. 

Furthermore,\marginnote{44.1} he is rarely ill or unwell, and his stomach digests well, being neither too hot nor too cold, more so than other people. This is the third blessing. 

Furthermore,\marginnote{45.1} a wheel-turning monarch is as dear and beloved to the brahmins and householders as a father is to his children. And the brahmins and householders are as dear to the wheel-turning monarch as children are to their father. 

Once\marginnote{45.7} it so happened that a wheel-turning monarch went with his army of four divisions to visit a park. Then the brahmins and householders went up to him and said, ‘Slow down, Your Majesty, so we may see you longer!’ And the king addressed his charioteer, ‘Drive slowly, charioteer, so I can see the brahmins and householders longer!’ This is the fourth blessing. 

These\marginnote{45.13} are the four blessings possessed by a wheel-turning monarch. 

What\marginnote{46.1} do you think, mendicants? Would a wheel-turning monarch who possessed these seven treasures and these four blessings experience pleasure and happiness because of them?” 

“Sir,\marginnote{46.3} a wheel-turning monarch who possessed even a single one of these treasures would experience pleasure and happiness because of that, let alone all seven treasures and four blessings!” 

Then\marginnote{47.1} the Buddha, picking up a stone the size of his palm, addressed the mendicants, “What do you think, mendicants? Which is bigger: the stone the size of my palm that I’ve picked up, or the Himalayas, the king of mountains?” 

“Sir,\marginnote{47.4} the stone you’ve picked up is tiny. Compared to the Himalayas, it doesn’t count, it’s not worth a fraction, there’s no comparison.” 

“In\marginnote{47.5} the same way, compared to the happiness of heaven, the pleasure and happiness experienced by a wheel-turning monarch due to those seven treasures and those four blessings doesn’t even count, it’s not even a fraction, there’s no comparison. 

And\marginnote{48.1} suppose that astute person, after a very long time, returned to the human realm. They’d be reborn in a well-to-do family of aristocrats, brahmins, or householders—rich, affluent, and wealthy, with lots of gold and silver, lots of property and assets, and lots of money and grain. And they’d be attractive, good-looking, lovely, of surpassing beauty. They’d get to have food, drink, clothes, and vehicles; garlands, perfumes, and makeup; and a bed, house, and lighting. And they do good things by way of body, speech, and mind. When their body breaks up, after death, they’re reborn in a good place, a heavenly realm. 

Suppose\marginnote{49.1} a gambler on the first lucky throw was to win a big pile of money. But such a lucky throw is trivial compared to the lucky throw whereby an astute person, when their body breaks up, after death, is reborn in a good place, a heavenly realm. This is the total fulfillment of the astute person’s level.” 

That\marginnote{49.5} is what the Buddha said. Satisfied, the mendicants were happy with what the Buddha said. 

%
\section*{{\suttatitleacronym MN 130}{\suttatitletranslation Messengers of the Gods }{\suttatitleroot Devadūtasutta}}
\addcontentsline{toc}{section}{\tocacronym{MN 130} \toctranslation{Messengers of the Gods } \tocroot{Devadūtasutta}}
\markboth{Messengers of the Gods }{Devadūtasutta}
\extramarks{MN 130}{MN 130}

\scevam{So\marginnote{1.1} I have heard. }At one time the Buddha was staying near \textsanskrit{Sāvatthī} in Jeta’s Grove, \textsanskrit{Anāthapiṇḍika}’s monastery. There the Buddha addressed the mendicants, “Mendicants!” 

“Venerable\marginnote{1.5} sir,” they replied. The Buddha said this: 

“Mendicants,\marginnote{2.1} suppose there were two houses with doors. A person with good eyesight standing in between them would see people entering and leaving a house and wandering to and fro. 

In\marginnote{2.2} the same way, with clairvoyance that is purified and superhuman, I see sentient beings passing away and being reborn—inferior and superior, beautiful and ugly, in a good place or a bad place. I understand how sentient beings are reborn according to their deeds: ‘These dear beings did good things by way of body, speech, and mind. They never spoke ill of the noble ones; they had right view; and they chose to act out of that right view. When their body breaks up, after death, they’re reborn in a good place, a heavenly realm, or among humans. These dear beings did bad things by way of body, speech, and mind. They spoke ill of the noble ones; they had wrong view; and they chose to act out of that wrong view. When their body breaks up, after death, they’re reborn in the ghost realm, the animal realm, or in a lower realm, a bad destination, a world of misery, hell.’ 

Then\marginnote{3.1} the wardens of hell take them by the arms and present them to King Yama, saying, ‘Your Majesty, this person did not pay due respect to their mother and father, ascetics and brahmins, or honor the elders in the family. May Your Majesty punish them!’ 

Then\marginnote{3.4} King Yama pursues, presses, and grills them about the first messenger of the gods. ‘Mister, did you not see the first messenger of the gods that appeared among human beings?’ 

He\marginnote{3.6} says, ‘I saw nothing, sir.’ 

Then\marginnote{4.1} King Yama says, ‘Mister, did you not see among human beings a little baby collapsed in their own urine and feces?’ 

He\marginnote{4.3} says, ‘I saw that, sir.’ 

Then\marginnote{4.5} King Yama says, ‘Mister, did it not occur to you—being sensible and mature—“I, too, am liable to be born. I’m not exempt from rebirth. I’d better do good by way of body, speech, and mind”?’ 

He\marginnote{4.8} says, ‘I couldn’t, sir. I was negligent.’ 

Then\marginnote{4.10} King Yama says, ‘Mister, because you were negligent, you didn’t do good by way of body, speech, and mind. Well, they’ll definitely punish you to fit your negligence. That bad deed wasn’t done by your mother, father, brother, or sister. It wasn’t done by friends and colleagues, by relatives and kin, by ascetics and brahmins, or by the deities. That bad deed was done by you alone, and you alone will experience the result.’ 

Then\marginnote{5.1} King Yama grills them about the second messenger of the gods. ‘Mister, did you not see the second messenger of the gods that appeared among human beings?’ 

He\marginnote{5.3} says, ‘I saw nothing, sir.’ 

Then\marginnote{5.5} King Yama says, ‘Mister, did you not see among human beings an elderly woman or a man—eighty, ninety, or a hundred years old—bent double, crooked, leaning on a staff, trembling as they walk, ailing, past their prime, with teeth broken, hair grey and scanty or bald, skin wrinkled, and limbs blotchy?’ 

He\marginnote{5.7} says, ‘I saw that, sir.’ 

Then\marginnote{5.9} King Yama says, ‘Mister, did it not occur to you—being sensible and mature—“I, too, am liable to grow old. I’m not exempt from old age. I’d better do good by way of body, speech, and mind”?’ 

He\marginnote{5.12} says, ‘I couldn’t, sir. I was negligent.’ 

Then\marginnote{5.14} King Yama says, ‘Mister, because you were negligent, you didn’t do good by way of body, speech, and mind. Well, they’ll definitely punish you to fit your negligence. That bad deed wasn’t done by your mother, father, brother, or sister. It wasn’t done by friends and colleagues, by relatives and kin, by ascetics and brahmins, or by the deities. That bad deed was done by you alone, and you alone will experience the result.’ 

Then\marginnote{6.1} King Yama grills them about the third messenger of the gods. ‘Mister, did you not see the third messenger of the gods that appeared among human beings?’ 

He\marginnote{6.3} says, ‘I saw nothing, sir.’ 

Then\marginnote{6.5} King Yama says, ‘Mister, did you not see among human beings a woman or a man, sick, suffering, gravely ill, collapsed in their own urine and feces, being picked up by some and put down by others?’ 

He\marginnote{6.7} says, ‘I saw that, sir.’ 

Then\marginnote{6.9} King Yama says, ‘Mister, did it not occur to you—being sensible and mature—“I, too, am liable to become sick. I’m not exempt from sickness. I’d better do good by way of body, speech, and mind”?’ He says, ‘I couldn’t, sir. I was negligent.’ 

Then\marginnote{6.14} King Yama says, ‘Mister, because you were negligent, you didn’t do good by way of body, speech, and mind. Well, they’ll definitely punish you to fit your negligence. That bad deed wasn’t done by your mother, father, brother, or sister. It wasn’t done by friends and colleagues, by relatives and kin, by ascetics and brahmins, or by the deities. That bad deed was done by you alone, and you alone will experience the result.’ 

Then\marginnote{7.1} King Yama grills them about the fourth messenger of the gods. ‘Mister, did you not see the fourth messenger of the gods that appeared among human beings?’ 

He\marginnote{7.3} says, ‘I saw nothing, sir.’ 

Then\marginnote{7.5} King Yama says, ‘Mister, did you not see among human beings when the rulers arrested a bandit, a criminal, and subjected them to various punishments—whipping, caning, and clubbing; cutting off hands or feet, or both; cutting off ears or nose, or both; the ‘porridge pot’, the ‘shell-shave’, the ‘demon’s mouth’, the ‘garland of fire’, the ‘burning hand’, the ‘grass blades’, the ‘bark dress’, the ‘antelope’, the ‘meat hook’, the ‘coins’, the ‘caustic pickle’, the ‘twisting bar’, the ‘straw mat’; being splashed with hot oil, being fed to the dogs, being impaled alive, and being beheaded?’ 

He\marginnote{7.8} says, ‘I saw that, sir.’ 

Then\marginnote{7.10} King Yama says, ‘Mister, did it not occur to you—being sensible and mature—that if someone who does bad deeds receives such punishment in the present life, what must happen to them in the next; I’d better do good by way of body, speech, and mind”?’ 

He\marginnote{7.13} says, ‘I couldn’t, sir. I was negligent.’ 

Then\marginnote{7.15} King Yama says, ‘Mister, because you were negligent, you didn’t do good by way of body, speech, and mind. Well, they’ll definitely punish you to fit your negligence. That bad deed wasn’t done by your mother, father, brother, or sister. It wasn’t done by friends and colleagues, by relatives and kin, by ascetics and brahmins, or by the deities. That bad deed was done by you alone, and you alone will experience the result.’ 

Then\marginnote{8.1} King Yama grills them about the fifth messenger of the gods. ‘Mister, did you not see the fifth messenger of the gods that appeared among human beings?’ 

He\marginnote{8.3} says, ‘I saw nothing, sir.’ 

Then\marginnote{8.5} King Yama says, ‘Mister, did you not see among human beings a woman or a man, dead for one, two, or three days, bloated, livid, and festering?’ 

He\marginnote{8.7} says, ‘I saw that, sir.’ 

Then\marginnote{8.9} King Yama says, ‘Mister, did it not occur to you—being sensible and mature—“I, too, am liable to die. I’m not exempt from death. I’d better do good by way of body, speech, and mind”?’ 

He\marginnote{8.12} says, ‘I couldn’t, sir. I was negligent.’ 

Then\marginnote{8.14} King Yama says, ‘Mister, because you were negligent, you didn’t do good by way of body, speech, and mind. Well, they’ll definitely punish you to fit your negligence. That bad deed wasn’t done by your mother, father, brother, or sister. It wasn’t done by friends and colleagues, by relatives and kin, by ascetics and brahmins, or by the deities. That bad deed was done by you alone, and you alone will experience the result.’ 

Then,\marginnote{9.1} after grilling them about the fifth messenger of the gods, King Yama falls silent. 

Then\marginnote{10.1} the wardens of hell punish them with the five-fold crucifixion. They drive red-hot stakes through the hands and feet, and another in the middle of the chest. And there they suffer painful, sharp, severe, acute feelings—but they don’t die until that bad deed is eliminated. 

Then\marginnote{11.1} the wardens of hell throw them down and hack them with axes. … 

They\marginnote{12.1} hang them upside-down and hack them with hatchets. … 

They\marginnote{13.1} harness them to a chariot, and drive them back and forth across burning ground, blazing and glowing. … 

They\marginnote{14.1} make them climb up and down a huge mountain of burning coals, blazing and glowing. … 

Then\marginnote{15.1} the wardens of hell turn them upside down and throw them in a red-hot copper pot, burning, blazing, and glowing. There they’re seared in boiling scum, and they’re swept up and down and round and round. And there they suffer painful, sharp, severe, acute feelings—but they don’t die until that bad deed is eliminated. 

Then\marginnote{16.1} the wardens of hell toss them into the Great Hell. Now, about that Great Hell: 

\begin{verse}%
‘Four\marginnote{16.3} are its corners, four its doors, \\
neatly divided in equal parts. \\
Surrounded by an iron wall, \\
of iron is its roof. 

The\marginnote{16.7} ground is even made of iron, \\
it burns with fierce fire. \\
The heat forever radiates \\
a hundred leagues around.’ 

%
\end{verse}

Now\marginnote{17.1} in the Great Hell, flames surge out of the walls and crash into the opposite wall: from east to west, from west to east, from north to south, from south to north, from bottom to top, from top to bottom. And there they suffer painful, sharp, severe, acute feelings—but they don’t die until that bad deed is eliminated. 

There\marginnote{18.1} comes a time when, after a very long period has passed, the eastern gate of the Great Hell is opened. So they run there as fast as they can. And as they run, their outer skin, inner skin, flesh, and sinews burn and even their bones smoke. Such is their escape; but when they’ve managed to make it most of the way, the gate is slammed shut. And there they suffer painful, sharp, severe, acute feelings—but they don’t die until that bad deed is eliminated. 

There\marginnote{18.6} comes a time when, after a very long period has passed, the western gate … northern gate … southern gate of the Great hell is opened. So they run there as fast as they can. And as they run, their outer skin, inner skin, flesh, and sinews burn and even their bones smoke. Such is their escape; but when they’ve managed to make it most of the way, the gate is slammed shut. And there they suffer painful, sharp, severe, acute feelings—but they don’t die until that bad deed is eliminated. 

There\marginnote{19.1} comes a time when, after a very long period has passed, the eastern gate of the Great Hell is opened. So they run there as fast as they can. And as they run, their outer skin, inner skin, flesh, and sinews burn and even their bones smoke. Such is their escape; and they make it out that door. 

Immediately\marginnote{20.1} adjacent to the Great Hell is the vast Dung Hell. And that’s where they fall. In that Dung Hell there are needle-mouthed creatures that bore through the outer skin, the inner skin, the flesh, sinews, and bones, until they reach the marrow and devour it. And there they suffer painful, sharp, severe, acute feelings—but they don’t die until that bad deed is eliminated. 

Immediately\marginnote{21.1} adjacent to the Dung Hell is the vast Hell of Hot Coals. And that’s where they fall. And there they suffer painful, sharp, severe, acute feelings—but they don’t die until that bad deed is eliminated. 

Immediately\marginnote{22.1} adjacent to the Hell of Hot Coals is the vast Hell of the Red Silk-Cotton Wood. It’s a league high, full of sixteen-inch thorns, burning, blazing, and glowing. And there they make them climb up and down. And there they suffer painful, sharp, severe, acute feelings—but they don’t die until that bad deed is eliminated. 

Immediately\marginnote{23.1} adjacent to the Hell of the Red Silk-Cotton Wood is the vast Hell of the Sword-Leaf Wood. They enter that. There the fallen leaves blown by the wind cut their hands, feet, both hands and feet; they cut their ears, nose, both ears and nose. And there they suffer painful, sharp, severe, acute feelings—but they don’t die until that bad deed is eliminated. 

Immediately\marginnote{24.1} adjacent to the Hell of the Sword-Leaf Wood is the vast Acid Hell. And that’s where they fall. There they are swept upstream, swept downstream, and swept both up and down stream. And there they suffer painful, sharp, severe, acute feelings—but they don’t die until that bad deed is eliminated. 

Then\marginnote{25.1} the wardens of hell pull them out with a hook and place them on dry land, and say, ‘Mister, what do you want?’ 

They\marginnote{25.3} say, ‘I’m hungry, sir.’ 

The\marginnote{25.5} wardens of hell force open their mouth with a hot iron spike—burning, blazing, glowing—and shove in a red-hot copper ball, burning, blazing, and glowing. It burns their lips, mouth, tongue, throat, and stomach before coming out below dragging their entrails. And there they feel painful, sharp, severe, acute feelings—but they don’t die until that bad deed is eliminated. 

Then\marginnote{26.1} the wardens of hell say, ‘Mister, what do you want?’ 

They\marginnote{26.3} say, ‘I’m thirsty, sir.’ 

The\marginnote{26.5} wardens of hell force open their mouth with a hot iron spike—burning, blazing, glowing—and pour in molten copper, burning, blazing, and glowing. It burns their lips, mouth, tongue, throat, and stomach before coming out below dragging their entrails. And there they feel painful, sharp, severe, acute feelings—but they don’t die until that bad deed is eliminated. 

Then\marginnote{27.1} the wardens of hell toss them back in the Great Hell. 

Once\marginnote{28.1} upon a time, King Yama thought: ‘Those who do such bad deeds in the world receive these many different punishments. Oh, I hope I may be reborn as a human being! And that a Realized One—a perfected one, a fully awakened Buddha—arises in the world! And that I may pay homage to the Buddha! Then the Buddha can teach me Dhamma, so that I may understand his teaching.’ 

Now,\marginnote{29.1} I don’t say this because I’ve heard it from some other ascetic or brahmin. I only say it because I’ve known, seen, and realized it for myself.” 

That\marginnote{30.1} is what the Buddha said. Then the Holy One, the Teacher, went on to say: 

\begin{verse}%
“Those\marginnote{30.3} people who are negligent, \\
when warned by the gods’ messengers: \\
a long time they sorrow, \\
when they go to that wretched place. 

But\marginnote{30.7} those good and peaceful people, \\
when warned by the gods’ messengers, \\
never neglect \\
the teaching of the noble ones. 

Seeing\marginnote{30.11} the danger in grasping, \\
the origin of birth and death, \\
the unattached are freed \\
with the ending of birth and death. 

Happy,\marginnote{30.15} they’ve come to a safe place, \\
extinguished in this very life. \\
They’ve gone beyond all threats and perils, \\
and risen above all suffering.” 

%
\end{verse}

%
\addtocontents{toc}{\let\protect\contentsline\protect\nopagecontentsline}
\chapter*{The Chapter on Analysis }
\addcontentsline{toc}{chapter}{\tocchapterline{The Chapter on Analysis }}
\addtocontents{toc}{\let\protect\contentsline\protect\oldcontentsline}

%
\section*{{\suttatitleacronym MN 131}{\suttatitletranslation One Fine Night }{\suttatitleroot Bhaddekarattasutta}}
\addcontentsline{toc}{section}{\tocacronym{MN 131} \toctranslation{One Fine Night } \tocroot{Bhaddekarattasutta}}
\markboth{One Fine Night }{Bhaddekarattasutta}
\extramarks{MN 131}{MN 131}

\scevam{So\marginnote{1.1} I have heard. }At one time the Buddha was staying near \textsanskrit{Sāvatthī} in Jeta’s Grove, \textsanskrit{Anāthapiṇḍika}’s monastery. There the Buddha addressed the mendicants, “Mendicants!” 

“Venerable\marginnote{1.5} sir,” they replied. The Buddha said this: 

“I\marginnote{2.1} shall teach you the passage for recitation and the analysis of One Fine Night. Listen and pay close attention, I will speak.” 

“Yes,\marginnote{2.3} sir,” they replied. The Buddha said this: 

\begin{verse}%
“Don’t\marginnote{3.1} run back to the past, \\
don’t hope for the future. \\
What’s past is left behind; \\
the future has not arrived; 

and\marginnote{3.5} phenomena in the present \\
are clearly seen in every case. \\
Knowing this, foster it—\\
unfaltering, unshakable. 

Today’s\marginnote{3.9} the day to keenly work—\\
who knows, tomorrow may bring death! \\
For there is no bargain to be struck \\
with Death and his mighty hordes. 

The\marginnote{3.13} peaceful sage explained it’s those \\
who keenly meditate like this, \\
tireless all night and day, \\
who truly have that one fine night. 

%
\end{verse}

And\marginnote{4.1} how do you run back to the past? You muster delight there, thinking: ‘I had such form in the past.’ … ‘I had such feeling … perception … choice … consciousness in the past.’ That’s how you run back to the past. 

And\marginnote{5.1} how do you not run back to the past? You don’t muster delight there, thinking: ‘I had such form in the past.’ … ‘I had such feeling … perception … choice … consciousness in the past.’ That’s how you don’t run back to the past. 

And\marginnote{6.1} how do you hope for the future? You muster delight there, thinking: ‘May I have such form in the future.’ … ‘May I have such feeling … perception … choice … consciousness in the future.’ That’s how you hope for the future. 

And\marginnote{7.1} how do you not hope for the future? You don’t muster delight there, thinking: ‘May I have such form in the future.’ … ‘May I have such feeling … perception … choice … consciousness in the future.’ That’s how you don’t hope for the future. 

And\marginnote{8.1} how do you falter amid presently arisen phenomena? It’s when an unlearned ordinary person has not seen the noble ones, and is neither skilled nor trained in the teaching of the noble ones. They’ve not seen good persons, and are neither skilled nor trained in the teaching of the good persons. They regard form as self, self as having form, form in self, or self in form. They regard feeling … perception … choices … consciousness as self, self as having consciousness, consciousness in self, or self in consciousness. That’s how you falter amid presently arisen phenomena. 

And\marginnote{9.1} how do you not falter amid presently arisen phenomena? It’s when a learned noble disciple has seen the noble ones, and is skilled and trained in the teaching of the noble ones. They’ve seen good persons, and are skilled and trained in the teaching of the good persons. They don’t regard form as self, self as having form, form in self, or self in form. They don’t regard feeling … perception … choices … consciousness as self, self as having consciousness, consciousness in self, or self in consciousness. That’s how you don’t falter amid presently arisen phenomena. 

\begin{verse}%
Don’t\marginnote{10.1} run back to the past, \\
don’t hope for the future. \\
What’s past is left behind; \\
the future has not arrived; 

and\marginnote{10.5} phenomena in the present \\
are clearly seen in every case. \\
Knowing this, foster it—\\
unfaltering, unshakable. 

Today’s\marginnote{10.9} the day to keenly work—\\
who knows, tomorrow may bring death! \\
For there is no bargain to be struck \\
with Death and his mighty hordes. 

The\marginnote{10.13} peaceful sage explained it’s those \\
who keenly meditate like this, \\
tireless all night and day, \\
who truly have that one fine night. 

%
\end{verse}

And\marginnote{11.1} that’s what I meant when I said: ‘I shall teach you the passage for recitation and the analysis of One Fine Night.’” 

That\marginnote{11.3} is what the Buddha said. Satisfied, the mendicants were happy with what the Buddha said. 

%
\section*{{\suttatitleacronym MN 132}{\suttatitletranslation Ānanda and One Fine Night }{\suttatitleroot Ānandabhaddekarattasutta}}
\addcontentsline{toc}{section}{\tocacronym{MN 132} \toctranslation{Ānanda and One Fine Night } \tocroot{Ānandabhaddekarattasutta}}
\markboth{Ānanda and One Fine Night }{Ānandabhaddekarattasutta}
\extramarks{MN 132}{MN 132}

\scevam{So\marginnote{1.1} I have heard. }At one time the Buddha was staying near \textsanskrit{Sāvatthī} in Jeta’s Grove, \textsanskrit{Anāthapiṇḍika}’s monastery. 

Now\marginnote{2.1} at that time Venerable Ānanda was educating, encouraging, firing up, and inspiring the mendicants in the assembly hall with a Dhamma talk on the topic of the recitation passage and analysis of One Fine Night. 

Then\marginnote{2.2} in the late afternoon, the Buddha came out of retreat, went to the assembly hall, where he sat on the seat spread out, and addressed the mendicants, “Who was inspiring the mendicants with a talk on the recitation passage and analysis of One Fine Night?” 

“It\marginnote{2.5} was Venerable Ānanda, sir.” 

Then\marginnote{2.6} the Buddha said to Venerable Ānanda, “But in what way were you inspiring the mendicants with a talk on the recitation passage and analysis of One Fine Night?” 

“I\marginnote{3.1} was doing so in this way, sir,” replied Ānanda. 

\begin{verse}%
“Don’t\marginnote{3.2} run back to the past, \\
don’t hope for the future. \\
What’s past is left behind; \\
the future has not arrived; 

and\marginnote{3.6} phenomena in the present \\
are clearly seen in every case. \\
Knowing this, foster it—\\
unfaltering, unshakable. 

Today’s\marginnote{3.10} the day to keenly work—\\
who knows, tomorrow may bring death! \\
For there is no bargain to be struck \\
with Death and his mighty hordes. 

The\marginnote{3.14} peaceful sage explained it’s those \\
who keenly meditate like this, \\
tireless all night and day, \\
who truly have that one fine night.” 

%
\end{verse}

(Ānanda\marginnote{4.1} went on to repeat the analysis as in the previous discourse, MN 131.) 

“That’s\marginnote{11.1} how I was inspiring the mendicants with a talk on the recitation passage and analysis of One Fine Night.” 

“Good,\marginnote{11.2} good, Ānanda. It’s good that you were inspiring the mendicants with a talk on the recitation passage and analysis of One Fine Night.” 

(And\marginnote{13{-}19.1} the Buddha repeated the verses and analysis once more.) 

That\marginnote{13{-}19.17} is what the Buddha said. Satisfied, Venerable Ānanda was happy with what the Buddha said. 

%
\section*{{\suttatitleacronym MN 133}{\suttatitletranslation Mahākaccāna and One Fine Night }{\suttatitleroot Mahākaccānabhaddekarattasutta}}
\addcontentsline{toc}{section}{\tocacronym{MN 133} \toctranslation{Mahākaccāna and One Fine Night } \tocroot{Mahākaccānabhaddekarattasutta}}
\markboth{Mahākaccāna and One Fine Night }{Mahākaccānabhaddekarattasutta}
\extramarks{MN 133}{MN 133}

\scevam{So\marginnote{1.1} I have heard. }At one time the Buddha was staying near \textsanskrit{Rājagaha} in the Hot Springs Monastery. 

Then\marginnote{1.3} Venerable Samiddhi rose at the crack of dawn and went to the hot springs to bathe. When he had bathed and emerged from the water he stood in one robe drying himself. 

Then,\marginnote{1.5} late at night, a glorious deity, lighting up the entire hot springs, went up to Samiddhi, stood to one side, and said to Samiddhi: 

“Mendicant,\marginnote{2.1} do you remember the recitation passage and analysis of One Fine Night?” 

“No,\marginnote{2.2} reverend, I do not. Do you?” 

“I\marginnote{2.4} also do not. But do you remember just the verses on One Fine Night?” 

“I\marginnote{2.6} do not. Do you?” 

“I\marginnote{2.8} also do not. Learn the recitation passage and analysis of One Fine Night, mendicant, memorize it, and remember it. It is beneficial and relates to the fundamentals of the spiritual life.” 

That’s\marginnote{2.13} what that deity said, before vanishing right there. 

Then,\marginnote{3.1} when the night had passed, Samiddhi went to the Buddha, bowed, sat down to one side, and told him what had happened. Then he added: 

“Sir,\marginnote{4.13} please teach me the recitation passage and analysis of One Fine night.” 

“Well\marginnote{4.16} then, mendicant, listen and pay close attention, I will speak.” 

“Yes,\marginnote{4.17} sir,” Samiddhi replied. The Buddha said this: 

\begin{verse}%
“Don’t\marginnote{5.1} run back to the past, \\
don’t hope for the future. \\
What’s past is left behind; \\
the future has not arrived; 

and\marginnote{5.5} phenomena in the present \\
are clearly seen in every case. \\
Knowing this, foster it—\\
unfaltering, unshakable. 

Today’s\marginnote{5.9} the day to keenly work—\\
who knows, tomorrow may bring death! \\
For there is no bargain to be struck \\
with Death and his mighty hordes. 

The\marginnote{5.13} peaceful sage explained it’s those \\
who keenly meditate like this, \\
tireless all night and day, \\
who truly have that one fine night.” 

%
\end{verse}

That\marginnote{6.1} is what the Buddha said. When he had spoken, the Holy One got up from his seat and entered his dwelling. 

Soon\marginnote{7.1} after the Buddha left, those mendicants considered, “The Buddha gave this brief passage for recitation, then entered his dwelling without explaining the meaning in detail. … 

Who\marginnote{7.19} can explain in detail the meaning of this brief summary given by the Buddha?” 

Then\marginnote{7.20} those mendicants thought: 

“This\marginnote{7.21} Venerable \textsanskrit{Mahākaccāna} is praised by the Buddha and esteemed by his sensible spiritual companions. He is capable of explaining in detail the meaning of this brief passage for recitation given by the Buddha. Let’s go to him, and ask him about this matter.” 

Then\marginnote{7.24} those mendicants went to \textsanskrit{Mahākaccāna}, and exchanged greetings with him. When the greetings and polite conversation were over, they sat down to one side. They told him what had happened, and said: 

“May\marginnote{8.1} Venerable \textsanskrit{Mahākaccāna} please explain this.” 

“Reverends,\marginnote{9.1} suppose there was a person in need of heartwood. And while wandering in search of heartwood he’d come across a large tree standing with heartwood. But he’d pass over the roots and trunk, imagining that the heartwood should be sought in the branches and leaves. Such is the consequence for the venerables. Though you were face to face with the Buddha, you overlooked him, imagining that you should ask me about this matter. For he is the Buddha, who knows and sees. He is vision, he is knowledge, he is the truth, he is supreme. He is the teacher, the proclaimer, the elucidator of meaning, the bestower of the deathless, the lord of truth, the Realized One. That was the time to approach the Buddha and ask about this matter. You should have remembered it in line with the Buddha’s answer.” 

“Certainly\marginnote{10.1} he is the Buddha, who knows and sees. He is vision, he is knowledge, he is the truth, he is supreme. He is the teacher, the proclaimer, the elucidator of meaning, the bestower of the deathless, the lord of truth, the Realized One. That was the time to approach the Buddha and ask about this matter. We should have remembered it in line with the Buddha’s answer. Still, Venerable \textsanskrit{Mahākaccāna} is praised by the Buddha and esteemed by his sensible spiritual companions. He is capable of explaining in detail the meaning of this brief passage for recitation given by the Buddha. Please explain this, if it’s no trouble.” 

“Well\marginnote{11.1} then, reverends, listen and pay close attention, I will speak.” 

“Yes,\marginnote{11.2} reverend,” they replied. Venerable \textsanskrit{Mahākaccāna} said this: 

“Reverends,\marginnote{12.1} the Buddha gave this brief passage for recitation, then entered his dwelling without explaining the meaning in detail: 

\begin{verse}%
‘Don’t\marginnote{12.2} run back to the past … \\
tireless all night and day, \\
who truly have that one fine night.’ 

%
\end{verse}

And\marginnote{12.6} this is how I understand the detailed meaning of this passage for recitation. 

And\marginnote{13.1} how do you run back to the past? Consciousness gets tied up there with desire and lust, thinking: ‘In the past I had such eyes and such sights.’ So you take pleasure in that, and that’s when you run back to the past. 

Consciousness\marginnote{13.4} gets tied up there with desire and lust, thinking: ‘In the past I had such ears and such sounds … such a nose and such smells … such a tongue and such tastes … such a body and such touches … such a mind and such thoughts.’ So you take pleasure in that, and that’s when you run back to the past. That’s how you run back to the past. 

And\marginnote{14.1} how do you not run back to the past? Consciousness doesn’t get tied up there with desire and lust, thinking: ‘In the past I had such eyes and such sights.’ So you don’t take pleasure in that, and that’s when you no longer run back to the past. 

Consciousness\marginnote{14.4} doesn’t get tied up there with desire and lust, thinking: ‘In the past I had such ears and such sounds … such a nose and such smells … such a tongue and such tastes … such a body and such touches … such a mind and such thoughts.’ So you don’t take pleasure in that, and that’s when you no longer run back to the past. That’s how you don’t run back to the past. 

And\marginnote{15.1} how do you hope for the future? The heart is set on getting what it does not have, thinking: ‘May I have such eyes and such sights in the future.’ So you take pleasure in that, and that’s when you hope for the future. The heart is set on getting what it does not have, thinking: ‘May I have such ears and such sounds … such a nose and such smells … such a tongue and such tastes … such a body and such touches … such a mind and such thoughts in the future.’ So you take pleasure in that, and that’s when you hope for the future. That’s how you hope for the future. 

And\marginnote{16.1} how do you not hope for the future? The heart is not set on getting what it does not have, thinking: ‘May I have such eyes and such sights in the future.’ So you don’t take pleasure in that, and that’s when you no longer hope for the future. The heart is not set on getting what it does not have, thinking: ‘May I have such ears and such sounds … such a nose and such smells … such a tongue and such tastes … such a body and such touches … such a mind and such thoughts in the future.’ So you don’t take pleasure in that, and that’s when you no longer hope for the future. That’s how you don’t hope for the future. 

And\marginnote{17.1} how do you falter amid presently arisen phenomena? Both the eye and sights are presently arisen. If consciousness gets tied up there in the present with desire and lust, you take pleasure in that, and that’s when you falter amid presently arisen phenomena. Both the ear and sounds … nose and smells … tongue and tastes … body and touches … mind and thoughts are presently arisen. If consciousness gets tied up there in the present with desire and lust, you take pleasure in that, and that’s when you falter amid presently arisen phenomena. That’s how you falter amid presently arisen phenomena. 

And\marginnote{18.1} how do you not falter amid presently arisen phenomena? Both the eye and sights are presently arisen. If consciousness doesn’t get tied up there in the present with desire and lust, you don’t take pleasure in that, and that’s when you no longer falter amid presently arisen phenomena. Both the ear and sounds … nose and smells … tongue and tastes … body and touches … mind and thoughts are presently arisen. If consciousness doesn’t get tied up there in the present with desire and lust, you don’t take pleasure in that, and that’s when you no longer falter amid presently arisen phenomena. That’s how you don’t falter amid presently arisen phenomena. 

This\marginnote{19.1} is how I understand the detailed meaning of that brief passage for recitation given by the Buddha. 

If\marginnote{19.6} you wish, you may go to the Buddha and ask him about this. You should remember it in line with the Buddha’s answer.” 

“Yes,\marginnote{20.1} reverend,” said those mendicants, approving and agreeing with what \textsanskrit{Mahākaccāna} said. Then they rose from their seats and went to the Buddha, bowed, sat down to one side, and told him what had happened, adding: 

“\textsanskrit{Mahākaccāna}\marginnote{20.25} clearly explained the meaning to us in this manner, with these words and phrases.” 

“\textsanskrit{Mahākaccāna}\marginnote{21.1} is astute, mendicants, he has great wisdom. If you came to me and asked this question, I would answer it in exactly the same way as \textsanskrit{Mahākaccāna}. That is what it means, and that’s how you should remember it.” 

That\marginnote{21.4} is what the Buddha said. Satisfied, the mendicants were happy with what the Buddha said. 

%
\section*{{\suttatitleacronym MN 134}{\suttatitletranslation Lomasakaṅgiya and One Fine Night }{\suttatitleroot Lomasakaṅgiyabhaddekarattasutta}}
\addcontentsline{toc}{section}{\tocacronym{MN 134} \toctranslation{Lomasakaṅgiya and One Fine Night } \tocroot{Lomasakaṅgiyabhaddekarattasutta}}
\markboth{Lomasakaṅgiya and One Fine Night }{Lomasakaṅgiyabhaddekarattasutta}
\extramarks{MN 134}{MN 134}

\scevam{So\marginnote{1.1} I have heard. }At one time the Buddha was staying near \textsanskrit{Sāvatthī} in Jeta’s Grove, \textsanskrit{Anāthapiṇḍika}’s monastery. 

Now\marginnote{1.3} at that time Venerable \textsanskrit{Lomasakaṅgiya} was staying in the Sakyan country at Kapilavatthu in the Banyan Tree Monastery. 

Then,\marginnote{2.1} late at night, the glorious god Candana, lighting up the entire Banyan Tree Monastery, went up to the Venerable \textsanskrit{Lomasakaṅgiya}, and stood to one side. Standing to one side, he said to \textsanskrit{Lomasakaṅgiya}: 

“Mendicant,\marginnote{2.2} do you remember the recitation passage and analysis of One Fine Night?” 

“No,\marginnote{2.3} reverend, I do not. Do you?” 

“I\marginnote{2.5} also do not. But do you remember just the verses on One Fine Night?” 

“I\marginnote{2.7} do not. Do you?” 

“I\marginnote{2.9} do.” 

“How\marginnote{2.10} do you remember the verses on One Fine Night?” 

“This\marginnote{2.11} one time, the Buddha was staying among the gods of the Thirty-Three at the root of the Shady Orchid Tree on the stone spread with a cream rug. There he taught the recitation passage and analysis of One Fine Night to the gods of the Thirty-Three: 

\begin{verse}%
‘Don’t\marginnote{3.1} run back to the past, \\
don’t hope for the future. \\
What’s past is left behind; \\
the future has not arrived; 

and\marginnote{3.5} phenomena in the present \\
are clearly seen in every case. \\
Knowing this, foster it—\\
unfaltering, unshakable. 

Today’s\marginnote{3.9} the day to keenly work—\\
who knows, tomorrow may bring death! \\
For there is no bargain to be struck \\
with Death and his mighty hordes. 

The\marginnote{3.13} peaceful sage explained it’s those \\
who keenly meditate like this, \\
tireless all night and day, \\
who truly have that one fine night.’ 

%
\end{verse}

That’s\marginnote{4.1} how I remember the verses of One Fine Night. Learn the recitation passage and analysis of One Fine Night, mendicant, memorize it, and remember it. It is beneficial and relates to the fundamentals of the spiritual life.” 

That’s\marginnote{4.6} what the god Candana said before vanishing right there. 

Then\marginnote{5.1} \textsanskrit{Lomasakaṅgiya} set his lodgings in order and, taking his bowl and robe, set out for \textsanskrit{Sāvatthī}. Eventually he came to \textsanskrit{Sāvatthī} and Jeta’s Grove. He went up to the Buddha, bowed, sat down to one side, and told him what had happened. Then he added: 

“Sir,\marginnote{5.26} please teach me the recitation passage and analysis of One Fine night.” 

“But\marginnote{6.1} mendicant, do you know that god?” 

“I\marginnote{6.2} do not, sir.” 

“That\marginnote{6.3} god was named Candana. Candana pays heed, pays attention, engages wholeheartedly, and lends an ear to the teaching. Well then, mendicant, listen and pay close attention, I will speak.” 

“Yes,\marginnote{6.6} sir,” \textsanskrit{Lomasakaṅgiya} replied. The Buddha said this: 

\begin{verse}%
“Don’t\marginnote{7.1} run back to the past, \\
don’t hope for the future. \\
What’s past is left behind; \\
the future has not arrived; 

and\marginnote{7.5} phenomena in the present \\
are clearly seen in every case. \\
Knowing this, foster it—\\
unfaltering, unshakable. 

Today’s\marginnote{7.9} the day to keenly work—\\
who knows, tomorrow may bring death! \\
For there is no bargain to be struck \\
with Death and his mighty hordes. 

The\marginnote{7.13} peaceful sage explained it’s those \\
who keenly meditate like this, \\
tireless all night and day, \\
who truly have that one fine night. 

%
\end{verse}

And\marginnote{8{-}13.1} how do you run back to the past? …” 

(And\marginnote{8{-}13.2} the Buddha repeated the analysis as in MN 131.) 

That\marginnote{14.17} is what the Buddha said. Satisfied, Venerable \textsanskrit{Lomasakaṅgiya} was happy with what the Buddha said. 

%
\section*{{\suttatitleacronym MN 135}{\suttatitletranslation The Shorter Analysis of Deeds }{\suttatitleroot Cūḷakammavibhaṅgasutta}}
\addcontentsline{toc}{section}{\tocacronym{MN 135} \toctranslation{The Shorter Analysis of Deeds } \tocroot{Cūḷakammavibhaṅgasutta}}
\markboth{The Shorter Analysis of Deeds }{Cūḷakammavibhaṅgasutta}
\extramarks{MN 135}{MN 135}

\scevam{So\marginnote{1.1} I have heard. }At one time the Buddha was staying near \textsanskrit{Sāvatthī} in Jeta’s Grove, \textsanskrit{Anāthapiṇḍika}’s monastery. 

Then\marginnote{2.1} the brahmin student Subha, Todeyya’s son, approached the Buddha, and exchanged greetings with him. When the greetings and polite conversation were over, he sat down to one side and said to the Buddha: 

“What\marginnote{3.1} is the cause, Master Gotama, what is the reason why even among those who are human beings some are seen to be inferior and superior? For people are seen who are short-lived and long-lived, sickly and healthy, ugly and beautiful, insignificant and illustrious, poor and rich, from low and eminent families, witless and wise. What is the reason why even among those who are human beings some are seen to be inferior and superior?” 

“Student,\marginnote{4.1} sentient beings are the owners of their deeds and heir to their deeds. Deeds are their womb, their relative, and their refuge. It is deeds that divide beings into inferior and superior.” 

“I\marginnote{4.4} don’t understand the meaning of what Master Gotama has said in brief, without explaining the details. Master Gotama, please teach me this matter in detail so I can understand the meaning.” 

“Well\marginnote{4.6} then, student, listen and pay close attention, I will speak.” 

“Yes,\marginnote{4.7} sir,” replied Subha. The Buddha said this: 

“Take\marginnote{5.1} some woman or man who kills living creatures. They’re violent, bloody-handed, a hardened killer, merciless to living beings. Because of undertaking such deeds, when their body breaks up, after death, they’re reborn in a place of loss, a bad place, the underworld, hell. If they’re not reborn in a place of loss, but return to the human realm, then wherever they’re reborn they’re short-lived. For killing living creatures is the path leading to a short lifespan. 

But\marginnote{6.1} take some woman or man who gives up killing living creatures. They renounce the rod and the sword. They’re scrupulous and kind, living full of compassion for all living beings. Because of undertaking such deeds, when their body breaks up, after death, they’re reborn in a good place, a heavenly realm. If they’re not reborn in a heavenly realm, but return to the human realm, then wherever they’re reborn they’re long-lived. For not killing living creatures is the path leading to a long lifespan. 

Take\marginnote{7.1} some woman or man who habitually hurts living creatures with a fist, stone, rod, or sword. Because of undertaking such deeds, after death they’re reborn in a place of loss … or if they return to the human realm, they’re sickly … 

But\marginnote{8.1} take some woman or man who does not habitually hurt living creatures with a fist, stone, rod, or sword. Because of undertaking such deeds, after death they’re reborn in a heavenly realm … or if they return to the human realm, they’re healthy … 

Take\marginnote{9.1} some woman or man who is irritable and bad-tempered. Even when lightly criticized they lose their temper, becoming annoyed, hostile, and hard-hearted, and displaying annoyance, hate, and bitterness. Because of undertaking such deeds, after death they’re reborn in a place of loss … or if they return to the human realm, they’re ugly … 

But\marginnote{10.1} take some woman or man who isn’t irritable and bad-tempered. Even when heavily criticized, they don’t lose their temper, become annoyed, hostile, and hard-hearted, or display annoyance, hate, and bitterness. Because of undertaking such deeds, after death they’re reborn in a heavenly realm … or if they return to the human realm, they’re lovely … 

Take\marginnote{11.1} some woman or man who is jealous. They envy, resent, and begrudge the possessions, honor, respect, reverence, homage, and veneration given to others. Because of undertaking such deeds, after death they’re reborn in a place of loss … or if they return to the human realm, they’re insignificant … 

But\marginnote{12.1} take some woman or man who is not jealous … Because of undertaking such deeds, after death they’re reborn in a heavenly realm … or if they return to the human realm, they’re illustrious … 

Take\marginnote{13.1} some woman or man who doesn’t give to ascetics or brahmins such things as food, drink, clothing, vehicles; garlands, perfumes, and makeup; and bed, house, and lighting. Because of undertaking such deeds, after death they’re reborn in a place of loss … or if they return to the human realm, they’re poor … 

But\marginnote{14.1} take some woman or man who does give to ascetics or brahmins … Because of undertaking such deeds, after death they’re reborn in a heavenly realm … or if they return to the human realm, they’re rich … 

Take\marginnote{15.1} some woman or man who is obstinate and vain. They don’t bow to those they should bow to. They don’t rise up for them, offer them a seat, make way for them, or honor, respect, esteem, or venerate those who are worthy of such. Because of undertaking such deeds, after death they’re reborn in a place of loss … or if they return to the human realm, they’re reborn in a low class family … 

But\marginnote{16.1} take some woman or man who is not obstinate and vain … Because of undertaking such deeds, after death they’re reborn in a heavenly realm … or if they return to the human realm, they’re reborn in an eminent family … 

Take\marginnote{17.1} some woman or man who doesn’t approach an ascetic or brahmin to ask: ‘Sir, what is skillful and what is unskillful? What is blameworthy and what is blameless? What should be cultivated and what should not be cultivated? What kind of action will lead to my lasting harm and suffering? Or what kind of action will lead to my lasting welfare and happiness?’ Because of undertaking such deeds, after death they’re reborn in a place of loss … or if they return to the human realm, they’re witless … 

But\marginnote{18.1} take some woman or man who does approach an ascetic or brahmin to ask: ‘Sir, what is skillful and what is unskillful? What is blameworthy and what is blameless? What should be cultivated and what should not be cultivated? What kind of action will lead to my lasting harm and suffering? Or what kind of action will lead to my lasting welfare and happiness?’ Because of undertaking such deeds, when their body breaks up, after death, they’re reborn in a good place, a heavenly realm. If they’re not reborn in a heavenly realm, but return to the human realm, then wherever they’re reborn they’re very wise. For asking questions of ascetics or brahmins is the path leading to wisdom. 

So\marginnote{19.1} it is the way people live that makes them how they are, whether short-lived or long-lived, sickly or healthy, ugly or lovely, insignificant or illustrious, poor or rich, in a low class or eminent family, or witless or wise. 

Sentient\marginnote{20.1} beings are the owners of their deeds and heir to their deeds. Deeds are their womb, their relative, and their refuge. It is deeds that divide beings into inferior and superior.” 

When\marginnote{21.1} he had spoken, Subha said to him, “Excellent, Master Gotama! Excellent! As if he were righting the overturned, or revealing the hidden, or pointing out the path to the lost, or lighting a lamp in the dark so people with good eyes can see what’s there, Master Gotama has made the Teaching clear in many ways. I go for refuge to Master Gotama, to the teaching, and to the mendicant \textsanskrit{Saṅgha}. From this day forth, may Master Gotama remember me as a lay follower who has gone for refuge for life.” 

%
\section*{{\suttatitleacronym MN 136}{\suttatitletranslation The Longer Analysis of Deeds }{\suttatitleroot Mahākammavibhaṅgasutta}}
\addcontentsline{toc}{section}{\tocacronym{MN 136} \toctranslation{The Longer Analysis of Deeds } \tocroot{Mahākammavibhaṅgasutta}}
\markboth{The Longer Analysis of Deeds }{Mahākammavibhaṅgasutta}
\extramarks{MN 136}{MN 136}

\scevam{So\marginnote{1.1} I have heard. }At one time the Buddha was staying near \textsanskrit{Rājagaha}, in the Bamboo Grove, the squirrels’ feeding ground. 

Now\marginnote{2.1} at that time Venerable Samiddhi was staying in a wilderness hut. Then as the wanderer Potaliputta was going for a walk he came up to Venerable Samiddhi and exchanged greetings with him. When the greetings and polite conversation were over, he sat down to one side and said to him: 

“Reverend\marginnote{2.4} Samiddhi, I have heard and learned this in the presence of the ascetic Gotama: ‘Deeds by way of body and speech are done in vain. Only mental deeds are real.’ And: ‘There is such an attainment where the one who enters it does not feel anything at all.’” 

“Don’t\marginnote{2.7} say that, Reverend Potaliputta, don’t say that! Don’t misrepresent the Buddha, for misrepresentation of the Buddha is not good. And the Buddha would not say this. But, reverend, there is such an attainment where the one who enters it does not feel anything at all.” 

“Reverend\marginnote{2.10} Samiddhi, how long has it been since you went forth?” 

“Not\marginnote{2.11} long, reverend: three years.” 

“Well\marginnote{2.12} now, what are we to say to the senior mendicants, when even such a junior mendicant imagines their Teacher needs defending? After doing an intentional deed by way of body, speech, or mind, reverend, what does one feel?” 

“After\marginnote{2.14} doing an intentional deed by way of body, speech, or mind, reverend, one feels suffering.” Then, neither approving nor dismissing Samiddhi’s statement, Potaliputta got up from his seat and left. 

Soon\marginnote{3.1} after he had left, Venerable Samiddhi went to Venerable Ānanda, and exchanged greetings with him. When the greetings and polite conversation were over, he sat down to one side, and informed Ānanda of all they had discussed. 

When\marginnote{3.4} he had spoken, Ānanda said to him, “Reverend Samiddhi, we should see the Buddha about this matter. Come, let’s go to the Buddha and inform him about this. As he answers, so we’ll remember it.” 

“Yes,\marginnote{3.8} reverend,” Samiddhi replied. 

Then\marginnote{4.1} Ānanda and Samiddhi went up to the Buddha, bowed, sat down to one side, and told him what had happened. 

When\marginnote{5.1} they had spoken, the Buddha said to Ānanda, “I don’t recall even seeing the wanderer Potaliputta, Ānanda, so how could we have had such a discussion? The wanderer Potaliputta’s question should have been answered after analyzing it, but this foolish person answered definitively.” 

When\marginnote{6.1} he said this, Venerable \textsanskrit{Udāyī} said to him, “But perhaps, sir, Venerable Samiddhi spoke in reference to the statement: ‘Suffering includes whatever is felt.’” 

But\marginnote{6.4} the Buddha said to Venerable Ānanda, “See how this foolish person \textsanskrit{Udāyī} comes up with an idea? I knew that he was going to come up with such an irrational idea. Right from the start Potaliputta asked about the three feelings. Suppose the foolish person Samiddhi had answered the wanderer Potaliputta’s question like this: ‘After doing an intentional deed to be experienced as pleasant by way of body, speech, or mind, one feels pleasure. After doing an intentional deed to be experienced as painful by way of body, speech, or mind, one feels pain. After doing an intentional deed to be experienced as neutral by way of body, speech, or mind, one feels neutral.’ Answering in this way, Samiddhi would have rightly answered Potaliputta. 

Still,\marginnote{6.14} who are those foolish and incompetent wanderers who follow other paths to understand the Realized One’s great analysis of deeds? Ānanda, if only you would all listen to the Realized One’s explanation of the great analysis of deeds.” 

“Now\marginnote{7.1} is the time, Blessed One! Now is the time, Holy One! Let the Buddha explain the great analysis of deeds. The mendicants will listen and remember it.” 

“Well\marginnote{7.3} then, Ānanda, listen and pay close attention, I will speak.” 

“Yes,\marginnote{7.4} sir,” Ānanda replied. The Buddha said this: 

“Ānanda,\marginnote{8.1} these four people are found in the world. What four? Some person here kills living creatures, steals, and commits sexual misconduct. They use speech that’s false, divisive, harsh, or nonsensical. And they’re covetous, malicious, and have wrong view. When their body breaks up, after death, they’re reborn in a place of loss, a bad place, the underworld, hell. 

But\marginnote{8.5} some other person here kills living creatures, steals, and commits sexual misconduct. They use speech that’s false, divisive, harsh, or nonsensical. And they’re covetous, malicious, and have wrong view. When their body breaks up, after death, they’re reborn in a good place, a heavenly realm. 

But\marginnote{8.7} some other person here refrains from killing living creatures, stealing, committing sexual misconduct, or using speech that’s false, divisive, harsh, or nonsensical. And they’re contented, kind-hearted, and have right view. When their body breaks up, after death, they’re reborn in a good place, a heavenly realm. 

But\marginnote{8.9} some other person here refrains from killing living creatures, stealing, committing sexual misconduct, or using speech that’s false, divisive, harsh, or nonsensical. And they’re contented, kind-hearted, and have right view. When their body breaks up, after death, they’re reborn in a place of loss, a bad place, the underworld, hell. 

Now,\marginnote{9.1} some ascetic or brahmin—by dint of keen, resolute, committed, and diligent effort, and right focus—experiences an immersion of the heart of such a kind that it gives rise to clairvoyance that is purified and superhuman. With that clairvoyance they see that person here who killed living creatures, stole, and committed sexual misconduct; who used speech that’s false, divisive, harsh, or nonsensical; and who was covetous, malicious, and had wrong view. And they see that, when their body breaks up, after death, that person is reborn in a place of loss, a bad place, the underworld, hell. They say: ‘It seems that there is such a thing as bad deeds, and the result of bad conduct. For I saw a person here who killed living creatures … and had wrong view. And when their body broke up, after death, they were reborn in a place of loss, a bad place, the underworld, hell.’ They say: ‘It seems that everyone who kills living creatures … and has wrong view is reborn in hell. Those who know this are right. Those who know something else are wrong.’ And so they obstinately stick to what they have known, seen, and understood for themselves, insisting that: ‘This is the only truth, other ideas are silly.’ 

But\marginnote{10.1} some other ascetic or brahmin—by dint of keen, resolute, committed, and diligent effort, and right focus—experiences an immersion of the heart of such a kind that it gives rise to clairvoyance that is purified and superhuman. With that clairvoyance they see that person here who killed living creatures … and had wrong view. And they see that that person is reborn in a heavenly realm. They say: ‘It seems that there is no such thing as bad deeds, and the result of bad conduct. For I have seen a person here who killed living creatures … and had wrong view. And I saw that that person was reborn in a heavenly realm.’ They say: ‘It seems that everyone who kills living creatures … and has wrong view is reborn in a heavenly realm. Those who know this are right. Those who know something else are wrong.’ And so they obstinately stick to what they have known, seen, and understood for themselves, insisting that: ‘This is the only truth, other ideas are silly.’ 

Take\marginnote{11.1} some ascetic or brahmin who with clairvoyance sees a person here who refrained from killing living creatures … and had right view. And they see that that person is reborn in a heavenly realm. They say: ‘It seems that there is such a thing as good deeds, and the result of good conduct. For I have seen a person here who refrained from killing living creatures … and had right view. And I saw that that person was reborn in a heavenly realm.’ They say: ‘It seems that everyone who refrains from killing living creatures … and has right view is reborn in a heavenly realm. Those who know this are right. Those who know something else are wrong.’ And so they obstinately stick to what they have known, seen, and understood for themselves, insisting that: ‘This is the only truth, other ideas are silly.’ 

Take\marginnote{12.1} some ascetic or brahmin who with clairvoyance sees a person here who refrained from killing living creatures … and had right view. And they see that that person is reborn in hell. They say: ‘It seems that there is no such thing as good deeds, and the result of good conduct. For I have seen a person here who refrained from killing living creatures … and had right view. And I saw that that person was reborn in hell.’ They say: ‘It seems that everyone who refrains from killing living creatures … and has right view is reborn in hell. Those who know this are right. Those who know something else are wrong.’ And so they obstinately stick to what they have known, seen, and understood for themselves, insisting that: ‘This is the only truth, other ideas are silly.’ 

In\marginnote{13.1} this case, when an ascetic or brahmin says this: ‘It seems that there is such a thing as bad deeds, and the result of bad conduct,’ I grant them that. And when they say: ‘I have seen a person here who killed living creatures … and had wrong view. And after death, they were reborn in hell,’ I also grant them that. But when they say: ‘It seems that everyone who kills living creatures … and has wrong view is reborn in hell,’ I don’t grant them that. And when they say: ‘Those who know this are right. Those who know something else are wrong,’ I also don’t grant them that. And when they obstinately stick to what they have known, seen, and understood for themselves, insisting that: ‘This is the only truth, other ideas are silly,’ I also don’t grant them that. Why is that? Because the Realized One’s knowledge of the great analysis of deeds is otherwise. 

In\marginnote{14.1} this case, when an ascetic or brahmin says this: ‘It seems that there is no such thing as bad deeds, and the result of bad conduct,’ I don’t grant them that. But when they say: ‘I have seen a person here who killed living creatures … and had wrong view. And I saw that that person was reborn in a heavenly realm,’ I grant them that. But when they say: ‘It seems that everyone who kills living creatures … and has wrong view is reborn in a heavenly realm,’ I don’t grant them that. … Because the Realized One’s knowledge of the great analysis of deeds is otherwise. 

In\marginnote{15.1} this case, when an ascetic or brahmin says this: ‘It seems that there is such a thing as good deeds, and the result of good conduct,’ I grant them that. And when they say: ‘I have seen a person here who refrained from killing living creatures … and had right view. And I saw that that person was reborn in a heavenly realm,’ I grant them that. But when they say: ‘It seems that everyone who refrains from killing living creatures … and has right view is reborn in a heavenly realm,’ I don’t grant them that. … Because the Realized One’s knowledge of the great analysis of deeds is otherwise. 

In\marginnote{16.1} this case, when an ascetic or brahmin says this: ‘It seems that there is no such thing as good deeds, and the result of good conduct,’ I don’t grant them that. But when they say: ‘I have seen a person here who refrained from killing living creatures … and had right view. And after death, they were reborn in hell,’ I grant them that. But when they say: ‘It seems that everyone who refrains from killing living creatures … and has right view is reborn in hell,’ I don’t grant them that. But when they say: ‘Those who know this are right. Those who know something else are wrong,’ I also don’t grant them that. And when they obstinately stick to what they have known, seen, and understood for themselves, insisting that: ‘This is the only truth, other ideas are silly,’ I also don’t grant them that. Why is that? Because the Realized One’s knowledge of the great analysis of deeds is otherwise. 

Now,\marginnote{17.1} Ānanda, take the case of the person here who killed living creatures … and had wrong view, and who, when their body breaks up, after death, is reborn in a place of loss, a bad place, the underworld, hell. They must have done a bad deed to be experienced as painful either previously or later, or else at the time of death they undertook wrong view. And that’s why, when their body breaks up, after death, they’re reborn in a place of loss, a bad place, the underworld, hell. But anyone here who kills living creatures … and has wrong view experiences the result of that in the present life, or in the next life, or in some subsequent period. 

Now,\marginnote{18.1} Ānanda, take the case of the person here who killed living creatures … and had wrong view, and who is reborn in a heavenly realm. They must have done a good deed to be experienced as pleasant either previously or later, or else at the time of death they undertook right view. And that’s why, when their body breaks up, after death, they’re reborn in a good place, a heavenly realm. But anyone here who kills living creatures … and has wrong view experiences the result of that in the present life, or in the next life, or in some subsequent period. 

Now,\marginnote{19.1} Ānanda, take the case of the person here who refrained from killing living creatures … and had right view, and who is reborn in a heavenly realm. They must have done a good deed to be experienced as pleasant either previously or later, or else at the time of death they undertook right view. And that’s why, when their body breaks up, after death, they’re reborn in a good place, a heavenly realm. But anyone here who refrains from killing living creatures … and has right view experiences the result of that in the present life, or in the next life, or in some subsequent period. 

Now,\marginnote{20.1} Ānanda, take the case of the person here who refrained from killing living creatures … and had right view, and who is reborn in hell. They must have done a bad deed to be experienced as painful either previously or later, or else at the time of death they undertook wrong view. And that’s why, when their body breaks up, after death, they’re reborn in a place of loss, a bad place, the underworld, hell. But anyone here who refrains from killing living creatures … and has right view experiences the result of that in the present life, or in the next life, or in some subsequent period. 

So,\marginnote{21.1} Ānanda, there are deeds that are ineffective and appear ineffective. There are deeds that are ineffective but appear effective. There are deeds that are effective and appear effective. And there are deeds that are effective but appear ineffective.” 

That\marginnote{21.2} is what the Buddha said. Satisfied, Venerable Ānanda was happy with what the Buddha said. 

%
\section*{{\suttatitleacronym MN 137}{\suttatitletranslation The Analysis of the Six Sense Fields }{\suttatitleroot Saḷāyatanavibhaṅgasutta}}
\addcontentsline{toc}{section}{\tocacronym{MN 137} \toctranslation{The Analysis of the Six Sense Fields } \tocroot{Saḷāyatanavibhaṅgasutta}}
\markboth{The Analysis of the Six Sense Fields }{Saḷāyatanavibhaṅgasutta}
\extramarks{MN 137}{MN 137}

\scevam{So\marginnote{1.1} I have heard. }At one time the Buddha was staying near \textsanskrit{Sāvatthī} in Jeta’s Grove, \textsanskrit{Anāthapiṇḍika}’s monastery. There the Buddha addressed the mendicants, “Mendicants!” 

“Venerable\marginnote{1.5} sir,” they replied. The Buddha said this: 

“Mendicants,\marginnote{2.1} I shall teach you the analysis of the six sense fields. Listen and pay close attention, I will speak.” 

“Yes,\marginnote{2.3} sir,” they replied. The Buddha said this: 

“‘The\marginnote{3.1} six interior sense fields should be understood. The six exterior sense fields should be understood. The six classes of consciousness should be understood. The six classes of contact should be understood. The eighteen mental preoccupations should be understood. The thirty-six positions of sentient beings should be understood. Therein, relying on this, give up that. The Noble One cultivates the establishment of mindfulness in three cases, by virtue of which they are a Teacher worthy to instruct a group. Of all meditation teachers, it is he that is called the supreme guide for those who wish to train.’ This is the recitation passage for the analysis of the six sense fields. 

‘The\marginnote{4.1} six interior sense fields should be understood.’ That’s what I said, but why did I say it? There are the sense fields of the eye, ear, nose, tongue, body, and mind. ‘The six interior sense fields should be understood.’ That’s what I said, and this is why I said it. 

‘The\marginnote{5.1} six exterior sense fields should be understood.’ That’s what I said, but why did I say it? There are the sense fields of sights, sounds, smells, tastes, touches, and thoughts. ‘The six exterior sense fields should be understood.’ That’s what I said, and this is why I said it. 

‘The\marginnote{6.1} six classes of consciousness should be understood.’ That’s what I said, but why did I say it? There are eye, ear, nose, tongue, body, and mind consciousness. ‘The six classes of consciousness should be understood.’ That’s what I said, and this is why I said it. 

‘The\marginnote{7.1} six classes of contact should be understood.’ That’s what I said, but why did I say it? There is contact through the eye, ear, nose, tongue, body, and mind. ‘The six classes of contact should be understood.’ That’s what I said, and this is why I said it. 

‘The\marginnote{8.1} eighteen mental preoccupations should be understood.’ That’s what I said, but why did I say it? Seeing a sight with the eye, one is preoccupied with a sight that’s a basis for happiness or sadness or equanimity. Hearing a sound with the ear … Smelling an odor with the nose … Tasting a flavor with the tongue … 

Feeling\marginnote{8.7} a touch with the body … Becoming conscious of a thought with the mind, one is preoccupied with a thought that’s a basis for happiness or sadness or equanimity. So there are six preoccupations with happiness, six preoccupations with sadness, and six preoccupations with equanimity. ‘The eighteen mental preoccupations should be understood.’ That’s what I said, and this is why I said it. 

‘The\marginnote{9.1} thirty-six positions of sentient beings should be understood.’ That’s what I said, but why did I say it? There are six kinds of lay happiness and six kinds of renunciate happiness. There are six kinds of lay sadness and six kinds of renunciate sadness. There are six kinds of lay equanimity and six kinds of renunciate equanimity. 

And\marginnote{10.1} in this context what are the six kinds of lay happiness? There are sights known by the eye that are likable, desirable, agreeable, pleasing, connected with the world’s material delights. Happiness arises when you regard it as a gain to obtain such sights, or when you recollect sights you formerly obtained that have passed, ceased, and perished. Such happiness is called lay happiness. There are sounds known by the ear … Smells known by the nose … Tastes known by the tongue … Touches known by the body … Thoughts known by the mind that are likable, desirable, agreeable, pleasing, connected with the world’s material delights. Happiness arises when you regard it as a gain to obtain such thoughts, or when you recollect thoughts you formerly obtained that have passed, ceased, and perished. Such happiness is called lay happiness. These are the six kinds of lay happiness. 

And\marginnote{11.1} in this context what are the six kinds of renunciate happiness? When you’ve understood the impermanence of sights—their perishing, fading away, and cessation—happiness arises as you truly understand through right understanding that both formerly and now all those sights are impermanent, suffering, and perishable. Such happiness is called renunciate happiness. When you’ve understood the impermanence of sounds … smells … tastes … touches … thoughts—their perishing, fading away, and cessation—happiness arises as you truly understand through right understanding that both formerly and now all those thoughts are impermanent, suffering, and perishable. Such happiness is called renunciate happiness. These are the six kinds of renunciate happiness. 

And\marginnote{12.1} in this context what are the six kinds of lay sadness? There are sights known by the eye that are likable, desirable, agreeable, pleasing, connected with the world’s material delights. Sadness arises when you regard it as a loss to lose such sights, or when you recollect sights you formerly lost that have passed, ceased, and perished. Such sadness is called lay sadness. There are sounds known by the ear … There are smells known by the nose … There are tastes known by the tongue … There are touches known by the body … There are thoughts known by the mind that are likable, desirable, agreeable, pleasing, connected with the world’s material delights. Sadness arises when you regard it as a loss to lose such thoughts, or when you recollect thoughts you formerly lost that have passed, ceased, and perished. Such sadness is called lay sadness. These are the six kinds of lay sadness. 

And\marginnote{13.1} in this context what are the six kinds of renunciate sadness? When you’ve understood the impermanence of sights—their perishing, fading away, and cessation—you truly understand through right understanding that both formerly and now all those sights are impermanent, suffering, and perishable. Upon seeing this, you give rise to yearning for the supreme liberations: ‘Oh, when will I enter and remain in the same dimension that the noble ones enter and remain in today?’ When you give rise to yearning for the supreme liberations like this, sadness arises because of the yearning. Such sadness is called renunciate sadness. When you’ve understood the impermanence of sounds … smells … tastes … touches … thoughts—their perishing, fading away, and cessation—you truly understand through right understanding that both formerly and now all those thoughts are impermanent, suffering, and perishable. Upon seeing this, you give rise to yearning for the supreme liberations: ‘Oh, when will I enter and remain in the same dimension that the noble ones enter and remain in today?’ When you give rise to yearning for the supreme liberations like this, sadness arises because of the yearning. Such sadness is called renunciate sadness. These are the six kinds of renunciate sadness. 

And\marginnote{14.1} in this context what are the six kinds of lay equanimity? When seeing a sight with the eye, equanimity arises for the unlearned ordinary person—a foolish ordinary person who has not overcome their limitations and the results of deeds, and is blind to the drawbacks. Such equanimity does not transcend the sight. That’s why it’s called lay equanimity. When hearing a sound with the ear … When smelling an odor with the nose … When tasting a flavor with the tongue … When feeling a touch with the body … When knowing a thought with the mind, equanimity arises for the unlearned ordinary person—a foolish ordinary person who has not overcome their limitations and the results of deeds, and is blind to the drawbacks. Such equanimity does not transcend the thought. That’s why it’s called lay equanimity. These are the six kinds of lay equanimity. 

And\marginnote{15.1} in this context what are the six kinds of renunciate equanimity? When you’ve understood the impermanence of sights—their perishing, fading away, and cessation—equanimity arises as you truly understand through right understanding that both formerly and now all those sights are impermanent, suffering, and perishable. Such equanimity transcends the sight. That’s why it’s called renunciate equanimity. When you’ve understood the impermanence of sounds … smells … tastes … touches … thoughts—their perishing, fading away, and cessation—equanimity arises as you truly understand through right understanding that both formerly and now all those thoughts are impermanent, suffering, and perishable. Such equanimity transcends the thought. That’s why it’s called renunciate equanimity. These are the six kinds of renunciate equanimity. ‘The thirty-six positions of sentient beings should be understood.’ That’s what I said, and this is why I said it. 

‘Therein,\marginnote{16.1} relying on this, give up that.’ That’s what I said, but why did I say it? 

Therein,\marginnote{16.3} by relying and depending on the six kinds of renunciate happiness, give up and go beyond the six kinds of lay happiness. That’s how they are given up. 

Therein,\marginnote{16.5} by relying on the six kinds of renunciate sadness, give up the six kinds of lay sadness. That’s how they are given up. 

Therein,\marginnote{16.7} by relying on the six kinds of renunciate equanimity, give up the six kinds of lay equanimity. That’s how they are given up. 

Therein,\marginnote{16.9} by relying on the six kinds of renunciate happiness, give up the six kinds of renunciate sadness. That’s how they are given up. 

Therein,\marginnote{16.11} by relying on the six kinds of renunciate equanimity, give up the six kinds of renunciate happiness. That’s how they are given up. 

There\marginnote{17.1} is equanimity that is diversified, based on diversity, and equanimity that is unified, based on unity. 

And\marginnote{18.1} what is equanimity based on diversity? There is equanimity towards sights, sounds, smells, tastes, and touches. This is equanimity based on diversity. 

And\marginnote{19.1} what is equanimity based on unity? There is equanimity based on the dimensions of infinite space, infinite consciousness, nothingness, and neither perception nor non-perception. This is equanimity based on unity. 

Therein,\marginnote{20.1} relying on equanimity based on unity, give up equanimity based on diversity. That’s how it is given up. 

Relying\marginnote{20.3} on non-identification, give up equanimity based on unity. That’s how it is given up. ‘Therein, relying on this, give up that.’ That’s what I said, and this is why I said it. 

‘The\marginnote{21.1} Noble One cultivates the establishment of mindfulness in three cases, by virtue of which they are a Teacher worthy to instruct a group.’ That’s what I said, but why did I say it? 

The\marginnote{22.1} first case is when the Teacher teaches the Dhamma out of kindness and compassion: ‘This is for your welfare. This is for your happiness.’ But their disciples don’t want to listen. They don’t pay attention or apply their minds to understand. They proceed having turned away from the Teacher’s instruction. In this case the Realized One is not displeased, he does not feel displeasure. He remains unaffected, mindful and aware. This is the first case in which the Noble One cultivates the establishment of mindfulness. 

The\marginnote{23.1} next case is when the Teacher teaches the Dhamma out of kindness and compassion: ‘This is for your welfare. This is for your happiness.’ And some of their disciples don’t want to listen. They don’t pay attention or apply their minds to understand. They proceed having turned away from the Teacher’s instruction. But some of their disciples do want to listen. They pay attention and apply their minds to understand. They don’t proceed having turned away from the Teacher’s instruction. In this case the Realized One is not displeased, nor is he pleased. Rejecting both displeasure and pleasure, he remains equanimous, mindful and aware. This is the second case in which the Noble One cultivates the establishment of mindfulness. 

The\marginnote{24.1} next case is when the Teacher teaches the Dhamma out of kindness and compassion: ‘This is for your welfare. This is for your happiness.’ And their disciples want to listen. They pay attention and apply their minds to understand. They don’t proceed having turned away from the Teacher’s instruction. In this case the Realized One is not pleased, he does not feel pleasure. He remains unaffected, mindful and aware. This is the third case in which the Noble One cultivates the establishment of mindfulness. ‘The Noble One cultivates the establishment of mindfulness in three cases, by virtue of which they are a Teacher worthy to instruct a group.’ That’s what I said, and this is why I said it. 

‘Of\marginnote{25.1} all meditation teachers, it is he that is called the supreme guide for those who wish to train.’ That’s what I said, but why did I say it? Driven by an elephant trainer, an elephant in training proceeds in just one direction: east, west, north, or south. 

Driven\marginnote{26.1} by a horse trainer, a horse in training proceeds in just one direction: east, west, north, or south. Driven by an ox trainer, an ox in training proceeds in just one direction: east, west, north, or south. But driven by the Realized One, the perfected one, the fully awakened Buddha, a person in training proceeds in eight directions: 

Having\marginnote{27.1} physical form, they see visions. This is the first direction. Not perceiving physical form internally, they see visions externally. This is the second direction. They’re focused only on beauty. This is the third direction. Going totally beyond perceptions of form, with the ending of perceptions of impingement, not focusing on perceptions of diversity, aware that ‘space is infinite’, they enter and remain in the dimension of infinite space. This is the fourth direction. Going totally beyond the dimension of infinite space, aware that ‘consciousness is infinite’, they enter and remain in the dimension of infinite consciousness. This is the fifth direction. Going totally beyond the dimension of infinite consciousness, aware that ‘there is nothing at all’, they enter and remain in the dimension of nothingness. This is the sixth direction. Going totally beyond the dimension of nothingness, they enter and remain in the dimension of neither perception nor non-perception. This is the seventh direction. Going totally beyond the dimension of neither perception nor non-perception, they enter and remain in the cessation of perception and feeling. This is the eighth direction. Driven by the Realized One, the perfected one, the fully awakened Buddha, a person in training proceeds in these eight directions. 

‘Of\marginnote{28.1} all meditation teachers, it is he that is called the supreme guide for those who wish to train.’ That’s what I said, and this is why I said it.” 

That\marginnote{28.3} is what the Buddha said. Satisfied, the mendicants were happy with what the Buddha said. 

%
\section*{{\suttatitleacronym MN 138}{\suttatitletranslation The Analysis of a Recitation Passage }{\suttatitleroot Uddesavibhaṅgasutta}}
\addcontentsline{toc}{section}{\tocacronym{MN 138} \toctranslation{The Analysis of a Recitation Passage } \tocroot{Uddesavibhaṅgasutta}}
\markboth{The Analysis of a Recitation Passage }{Uddesavibhaṅgasutta}
\extramarks{MN 138}{MN 138}

\scevam{So\marginnote{1.1} I have heard. }At one time the Buddha was staying near \textsanskrit{Sāvatthī} in Jeta’s Grove, \textsanskrit{Anāthapiṇḍika}’s monastery. There the Buddha addressed the mendicants, “Mendicants!” 

“Venerable\marginnote{1.5} sir,” they replied. The Buddha said this: 

“Mendicants,\marginnote{2.1} I shall teach you the analysis of a recitation passage. Listen and pay close attention, I will speak.” 

“Yes,\marginnote{2.3} sir,” they replied. The Buddha said this: 

“A\marginnote{3.1} mendicant should examine in any such a way that their consciousness is neither scattered and diffused externally nor stuck internally, and they are not anxious because of grasping. When this is the case and they are no longer anxious, there is for them no coming to be of the origin of suffering—of rebirth, old age, and death in the future.” 

That\marginnote{4.1} is what the Buddha said. When he had spoken, the Holy One got up from his seat and entered his dwelling. 

Soon\marginnote{5.1} after the Buddha left, those mendicants considered, “The Buddha gave this brief passage for recitation, then entered his dwelling without explaining the meaning in detail. Who can explain in detail the meaning of this brief passage for recitation given by the Buddha?” 

Then\marginnote{5.6} those mendicants thought, “This Venerable \textsanskrit{Mahākaccāna} is praised by the Buddha and esteemed by his sensible spiritual companions. He is capable of explaining in detail the meaning of this brief passage for recitation given by the Buddha. Let’s go to him, and ask him about this matter.” 

Then\marginnote{6.1} those mendicants went to \textsanskrit{Mahākaccāna}, and exchanged greetings with him. When the greetings and polite conversation were over, they sat down to one side. They told him what had happened, and said, “May Venerable \textsanskrit{Mahākaccāna} please explain this.” 

“Reverends,\marginnote{7.1} suppose there was a person in need of heartwood. And while wandering in search of heartwood he’d come across a large tree standing with heartwood. But he’d pass over the roots and trunk, imagining that the heartwood should be sought in the branches and leaves. Such is the consequence for the venerables. Though you were face to face with the Buddha, you overlooked him, imagining that you should ask me about this matter. For he is the Buddha, who knows and sees. He is vision, he is knowledge, he is the truth, he is supreme. He is the teacher, the proclaimer, the elucidator of meaning, the bestower of the deathless, the lord of truth, the Realized One. That was the time to approach the Buddha and ask about this matter. You should have remembered it in line with the Buddha’s answer.” 

“Certainly\marginnote{8.1} he is the Buddha, who knows and sees. He is vision, he is knowledge, he is the truth, he is supreme. He is the teacher, the proclaimer, the elucidator of meaning, the bestower of the deathless, the lord of truth, the Realized One. That was the time to approach the Buddha and ask about this matter. We should have remembered it in line with the Buddha’s answer. Still, Venerable \textsanskrit{Mahākaccāna} is praised by the Buddha and esteemed by his sensible spiritual companions. He is capable of explaining in detail the meaning of this brief passage for recitation given by the Buddha. Please explain this, if it’s no trouble.” 

“Well\marginnote{9.1} then, reverends, listen and pay close attention, I will speak.” 

“Yes,\marginnote{9.2} reverend,” they replied. Venerable \textsanskrit{Mahākaccāna} said this: 

“Reverends,\marginnote{9.4} the Buddha gave this brief passage for recitation, then entered his dwelling without explaining the meaning in detail: ‘A mendicant should examine in any such a way that their consciousness is neither scattered and diffused externally nor stuck internally, and they are not anxious because of grasping. When this is the case and they are no longer anxious, there is for them no coming to be of the origin of suffering—of rebirth, old age, and death in the future.’ And this is how I understand the detailed meaning of this passage for recitation. 

And\marginnote{10.1} how is consciousness scattered and diffused externally? Take a mendicant who sees a sight with their eyes. Their consciousness follows after the features of that sight, tied, attached, and fettered to gratification in its features. So their consciousness is said to be scattered and diffused externally. When they hear a sound with their ears … When they smell an odor with their nose … When they taste a flavor with their tongue … When they feel a touch with their body … When they know a thought with their mind, their consciousness follows after the features of that thought, tied, attached, and fettered to gratification in its features. So their consciousness is said to be scattered and diffused externally. That’s how consciousness is scattered and diffused externally. 

And\marginnote{11.1} how is consciousness not scattered and diffused externally? Take a mendicant who sees a sight with their eyes. Their consciousness doesn’t follow after the features of that sight, and is not tied, attached, and fettered to gratification in its features. So their consciousness is said to be not scattered and diffused externally. When they hear a sound with their ears … When they smell an odor with their nose … When they taste a flavor with their tongue … When they feel a touch with their body … When they know a thought with their mind, their consciousness doesn’t follow after the features of that thought, and is not tied, attached, and fettered to gratification in its features. So their consciousness is said to be not scattered and diffused externally. That’s how consciousness is not scattered and diffused externally. 

And\marginnote{12.1} how is their consciousness stuck internally? Take a mendicant who, quite secluded from sensual pleasures, secluded from unskillful qualities, enters and remains in the first absorption, which has the rapture and bliss born of seclusion, while placing the mind and keeping it connected. Their consciousness follows after that rapture and bliss born of seclusion, tied, attached, and fettered to gratification in that rapture and bliss born of seclusion. So their mind is said to be stuck internally. 

Furthermore,\marginnote{13.1} as the placing of the mind and keeping it connected are stilled, a mendicant enters and remains in the second absorption, which has the rapture and bliss born of immersion, with internal clarity and confidence, and unified mind, without placing the mind and keeping it connected. Their consciousness follows after that rapture and bliss born of immersion, tied, attached, and fettered to gratification in that rapture and bliss born of immersion. So their mind is said to be stuck internally. 

Furthermore,\marginnote{14.1} with the fading away of rapture, a mendicant enters and remains in the third absorption, where they meditate with equanimity, mindful and aware, personally experiencing the bliss of which the noble ones declare, ‘Equanimous and mindful, one meditates in bliss.’ Their consciousness follows after that equanimity, tied, attached, and fettered to gratification in that equanimous bliss. So their mind is said to be stuck internally. 

Furthermore,\marginnote{15.1} giving up pleasure and pain, and ending former happiness and sadness, a mendicant enters and remains in the fourth absorption, without pleasure or pain, with pure equanimity and mindfulness. Their consciousness follows after that neutral feeling, tied, attached, and fettered to gratification in that neutral feeling. So their mind is said to be stuck internally. That’s how their consciousness is stuck internally. 

And\marginnote{16.1} how is their consciousness not stuck internally? It’s when a mendicant, quite secluded from sensual pleasures, secluded from unskillful qualities, enters and remains in the first absorption, which has the rapture and bliss born of seclusion, while placing the mind and keeping it connected. Their consciousness doesn’t follow after that rapture and bliss born of seclusion, and is not tied, attached, and fettered to gratification in that rapture and bliss born of seclusion. So their mind is said to be not stuck internally. 

Furthermore,\marginnote{17.1} they enter the second absorption … Their consciousness doesn’t follow after that rapture and bliss born of immersion … 

Furthermore,\marginnote{18.1} they enter and remain in the third absorption … Their consciousness doesn’t follow after that equanimity, and is not tied, attached, and fettered to gratification in that equanimous bliss. So their mind is said to be not stuck internally. 

Furthermore,\marginnote{19.1} they enter and remain in the fourth absorption … Their consciousness doesn’t follow after that neutral feeling, and is not tied, attached, and fettered to gratification in that neutral feeling. So their mind is said to be not stuck internally. That’s how their consciousness is not stuck internally. 

And\marginnote{20.1} how are they anxious because of grasping? It’s when an unlearned ordinary person has not seen the noble ones, and is neither skilled nor trained in the teaching of the noble ones. They’ve not seen good persons, and are neither skilled nor trained in the teaching of the good persons. They regard form as self, self as having form, form in self, or self in form. But that form of theirs decays and perishes, and consciousness latches on to the perishing of form. Anxieties occupy their mind, born of latching on to the perishing of form, and originating in accordance with natural principles. So they become frightened, worried, concerned, and anxious because of grasping. They regard feeling … perception … choices … consciousness as self, self as having consciousness, consciousness in self, or self in consciousness. But that consciousness of theirs decays and perishes, and consciousness latches on to the perishing of consciousness. Anxieties occupy their mind, born of latching on to the perishing of consciousness, and originating in accordance with natural principles. So they become frightened, worried, concerned, and anxious because of grasping. That’s how they are anxious because of grasping. 

And\marginnote{21.1} how are they not anxious because of grasping? It’s when a learned noble disciple has seen the noble ones, and is skilled and trained in the teaching of the noble ones. They’ve seen good persons, and are skilled and trained in the teaching of the good persons. They don’t regard form as self, self as having form, form in self, or self in form. When that form of theirs decays and perishes, consciousness doesn’t latch on to the perishing of form. Anxieties—born of latching on to the perishing of form and originating in accordance with natural principles—don’t occupy their mind. So they don’t become frightened, worried, concerned, or anxious because of grasping. They don’t regard feeling … perception … choices … consciousness as self, self as having consciousness, consciousness in self, or self in consciousness. When that consciousness of theirs decays and perishes, consciousness doesn’t latch on to the perishing of consciousness. Anxieties—born of latching on to the perishing of consciousness and originating in accordance with natural principles—don’t occupy their mind. So they don’t become frightened, worried, concerned, or anxious because of grasping. That’s how they are not anxious because of grasping. 

The\marginnote{22.1} Buddha gave this brief passage for recitation, then entered his dwelling without explaining the meaning in detail: ‘A mendicant should examine in any such a way that their consciousness is neither scattered and diffused externally nor stuck internally, and they are not anxious because of grasping. When this is the case and they are no longer anxious, there is for them no coming to be of the origin of suffering—of rebirth, old age, and death in the future.’ And this is how I understand the detailed meaning of this passage for recitation. If you wish, you may go to the Buddha and ask him about this. You should remember it in line with the Buddha’s answer.” 

“Yes,\marginnote{23.1} reverend,” said those mendicants, approving and agreeing with what \textsanskrit{Mahākaccāna} said. Then they rose from their seats and went to the Buddha, bowed, sat down to one side, and told him what had happened, saying: 

“\textsanskrit{Mahākaccāna}\marginnote{23.14} clearly explained the meaning to us in this manner, with these words and phrases.” 

“\textsanskrit{Mahākaccāna}\marginnote{24.1} is astute, mendicants, he has great wisdom. If you came to me and asked this question, I would answer it in exactly the same way as \textsanskrit{Mahākaccāna}. That is what it means, and that’s how you should remember it.” 

That\marginnote{24.4} is what the Buddha said. Satisfied, the mendicants were happy with what the Buddha said. 

%
\section*{{\suttatitleacronym MN 139}{\suttatitletranslation The Analysis of Non-Conflict }{\suttatitleroot Araṇavibhaṅgasutta}}
\addcontentsline{toc}{section}{\tocacronym{MN 139} \toctranslation{The Analysis of Non-Conflict } \tocroot{Araṇavibhaṅgasutta}}
\markboth{The Analysis of Non-Conflict }{Araṇavibhaṅgasutta}
\extramarks{MN 139}{MN 139}

\scevam{So\marginnote{1.1} I have heard. }At one time the Buddha was staying near \textsanskrit{Sāvatthī} in Jeta’s Grove, \textsanskrit{Anāthapiṇḍika}’s monastery. There the Buddha addressed the mendicants, “Mendicants!” 

“Venerable\marginnote{1.5} sir,” they replied. The Buddha said this: 

“Mendicants,\marginnote{2.1} I shall teach you the analysis of non-conflict. Listen and pay close attention, I will speak.” 

“Yes,\marginnote{2.3} sir,” they replied. The Buddha said this: 

“Don’t\marginnote{3.1} indulge in sensual pleasures, which are low, crude, ordinary, ignoble, and pointless. And don’t indulge in self-mortification, which is painful, ignoble, and pointless. Avoiding these two extremes, the Realized One woke up by understanding the middle way of practice, which gives vision and knowledge, and leads to peace, direct knowledge, awakening, and extinguishment. Know what it means to flatter and to rebuke. Knowing these, avoid them, and just teach Dhamma. Know how to assess different kinds of pleasure. Knowing this, pursue inner bliss. Don’t talk behind people’s backs, and don’t speak sharply in their presence. Don’t speak hurriedly. Don’t insist on local terminology and don’t override normal usage. This is the recitation passage for the analysis of non-conflict. 

‘Don’t\marginnote{4.1} indulge in sensual pleasures, which are low, crude, ordinary, ignoble, and pointless. And don’t indulge in self-mortification, which is painful, ignoble, and pointless.’ That’s what I said, but why did I say it? Pleasure linked to sensuality is low, crude, ordinary, ignoble, and pointless. Indulging in such happiness is a principle beset by pain, harm, stress, and fever, and it is the wrong way. Breaking off such indulgence is a principle free of pain, harm, stress, and fever, and it is the right way. Indulging in self-mortification is painful, ignoble, and pointless. It is a principle beset by pain, harm, stress, and fever, and it is the wrong way. Breaking off such indulgence is a principle free of pain, harm, stress, and fever, and it is the right way. ‘Don’t indulge in sensual pleasures, which are low, crude, ordinary, ignoble, and pointless. And don’t indulge in self-mortification, which is painful, ignoble, and pointless.’ That’s what I said, and this is why I said it. 

‘Avoiding\marginnote{5.1} these two extremes, the Realized One woke up by understanding the middle way of practice, which gives vision and knowledge, and leads to peace, direct knowledge, awakening, and extinguishment.’ That’s what I said, but why did I say it? It is simply this noble eightfold path, that is: right view, right thought, right speech, right action, right livelihood, right effort, right mindfulness, and right immersion. ‘Avoiding these two extremes, the Realized One woke up by understanding the middle way of practice, which gives vision and knowledge, and leads to peace, direct knowledge, awakening, and extinguishment.’ That’s what I said, and this is why I said it. 

‘Know\marginnote{6.1} what it means to flatter and to rebuke. Knowing these, avoid them, and just teach Dhamma.’ That’s what I said, but why did I say it? 

And\marginnote{7.1} how is there flattering and rebuking without teaching Dhamma? In speaking like this, some are rebuked: ‘Pleasure linked to sensuality is low, crude, ordinary, ignoble, and pointless. All those who indulge in such happiness are beset by pain, harm, stress, and fever, and they are practicing the wrong way.’ 

In\marginnote{7.4} speaking like this, some are flattered: ‘Pleasure linked to sensuality is low, crude, ordinary, ignoble, and pointless. All those who have broken off such indulgence are free of pain, harm, stress, and fever, and they are practicing the right way.’ 

In\marginnote{7.6} speaking like this, some are rebuked: ‘Indulging in self-mortification is painful, ignoble, and pointless. All those who indulge in it are beset by pain, harm, stress, and fever, and they are practicing the wrong way.’ 

In\marginnote{7.8} speaking like this, some are flattered: ‘Indulging in self-mortification is painful, ignoble, and pointless. All those who have broken off such indulgence are free of pain, harm, stress, and fever, and they are practicing the right way.’ 

In\marginnote{7.10} speaking like this, some are rebuked: ‘All those who have not given up the fetters of rebirth are beset by pain, harm, stress, and fever, and they are practicing the wrong way.’ 

In\marginnote{7.12} speaking like this, some are flattered: ‘All those who have given up the fetters of rebirth are free of pain, harm, stress, and fever, and they are practicing the right way.’ That’s how there is flattering and rebuking without teaching Dhamma. 

And\marginnote{8.1} how is there neither flattering nor rebuking, and just teaching Dhamma? You don’t say: ‘Pleasure linked to sensuality is low, crude, ordinary, ignoble, and pointless. All those who indulge in such happiness are beset by pain, harm, stress, and fever, and they are practicing the wrong way.’ Rather, by saying this you just teach Dhamma: ‘The indulgence is a principle beset by pain, harm, stress, and fever, and it is the wrong way.’ 

You\marginnote{8.7} don’t say: ‘Pleasure linked to sensuality is low, crude, ordinary, ignoble, and pointless. All those who have broken off such indulgence are free of pain, harm, stress, and fever, and they are practicing the right way.’ Rather, by saying this you just teach Dhamma: ‘Breaking off the indulgence is a principle free of pain, harm, stress, and fever, and it is the right way.’ 

You\marginnote{8.12} don’t say: ‘Indulging in self-mortification is painful, ignoble, and pointless. All those who indulge in it are beset by pain, harm, stress, and fever, and they are practicing the wrong way.’ Rather, by saying this you just teach Dhamma: ‘The indulgence is a principle beset by pain, harm, stress, and fever, and it is the wrong way.’ 

You\marginnote{8.17} don’t say: ‘Indulging in self-mortification is painful, ignoble, and pointless. All those who have broken off such indulgence are free of pain, harm, stress, and fever, and they are practicing the right way.’ Rather, by saying this you just teach Dhamma: ‘Breaking off the indulgence is a principle free of pain, harm, stress, and fever, and it is the right way.’ 

You\marginnote{8.22} don’t say: ‘All those who have not given up the fetters of rebirth are beset by pain, harm, stress, and fever, and they are practicing the wrong way.’ Rather, by saying this you just teach Dhamma: ‘When the fetter of rebirth is not given up, rebirth is also not given up.’ 

You\marginnote{8.26} don’t say: ‘All those who have given up the fetters of rebirth are free of pain, harm, stress, and fever, and they are practicing the right way.’ Rather, by saying this you just teach Dhamma: ‘When the fetter of rebirth is given up, rebirth is also given up.’ That’s how there is neither flattering nor rebuking, and just teaching Dhamma. ‘Know what it means to flatter and to rebuke. Knowing these, avoid them, and just teach Dhamma.’ That’s what I said, and this is why I said it. 

‘Know\marginnote{9.1} how to assess different kinds of pleasure. Knowing this, pursue inner bliss.’ That’s what I said, but why did I say it? There are these five kinds of sensual stimulation. What five? Sights known by the eye that are likable, desirable, agreeable, pleasant, sensual, and arousing. Sounds known by the ear … Smells known by the nose … Tastes known by the tongue … Touches known by the body that are likable, desirable, agreeable, pleasant, sensual, and arousing. These are the five kinds of sensual stimulation. The pleasure and happiness that arise from these five kinds of sensual stimulation is called sensual pleasure—a filthy, common, ignoble pleasure. Such pleasure should not be cultivated or developed, but should be feared, I say. Now, take a mendicant who, quite secluded from sensual pleasures, secluded from unskillful qualities, enters and remains in the first absorption, which has the rapture and bliss born of seclusion, while placing the mind and keeping it connected. As the placing of the mind and keeping it connected are stilled, they enter and remain in the second absorption … third absorption … fourth absorption. This is called the pleasure of renunciation, the pleasure of seclusion, the pleasure of peace, the pleasure of awakening. Such pleasure should be cultivated and developed, and should not be feared, I say. ‘Know how to assess different kinds of pleasure. Knowing this, pursue inner bliss.’ That’s what I said, and this is why I said it. 

‘Don’t\marginnote{10.1} talk behind people’s backs, and don’t speak sharply in their presence.’ That’s what I said, but why did I say it? When you know that what you say behind someone’s back is untrue, false, and harmful, then if at all possible you should not speak. When you know that what you say behind someone’s back is true and correct, but harmful, then you should train yourself not to speak. When you know that what you say behind someone’s back is true, correct, and beneficial, then you should know the right time to speak. When you know that your sharp words in someone’s presence are untrue, false, and harmful, then if at all possible you should not speak. When you know that your sharp words in someone’s presence are true and correct, but harmful, then you should train yourself not to speak. When you know that your sharp words in someone’s presence are true, correct, and beneficial, then you should know the right time to speak. ‘Don’t talk behind people’s backs, and don’t speak sharply in their presence.’ That’s what I said, and this is why I said it. 

‘Don’t\marginnote{11.1} speak hurriedly.’ That’s what I said, but why did I say it? When speaking hurriedly, your body gets tired, your mind gets stressed, your voice gets stressed, your throat gets sore, and your words become unclear and hard to understand. When not speaking hurriedly, your body doesn’t get tired, your mind doesn’t get stressed, your voice doesn’t get stressed, your throat doesn’t get sore, and your words are clear and easy to understand. ‘Don’t speak hurriedly.’ That’s what I said, and this is why I said it. 

‘Don’t\marginnote{12.1} insist on local terminology and don’t override normal usage.’ That’s what I said, but why did I say it? And how do you insist on local terminology and override normal usage? It’s when in different localities the same thing is known as a ‘plate’, a ‘bowl’, a ‘cup’, a ‘dish’, a ‘basin’, a ‘tureen’, or a ‘porringer’. And however it is known in those various localities, you speak accordingly, obstinately sticking to that and insisting: ‘This is the only truth, other ideas are silly.’ That’s how you insist on local terminology and override normal usage. 

And\marginnote{12.8} how do you not insist on local terminology and not override normal usage? It’s when in different localities the same thing is known as a ‘plate’, a ‘bowl’, a ‘cup’, a ‘dish’, a ‘basin’, a ‘tureen’, or a ‘porringer’. And however it is known in those various localities, you speak accordingly, thinking: ‘It seems that the venerables are referring to this.’ That’s how you don’t insist on local terminology and don’t override normal usage. ‘Don’t insist on local terminology and don’t override normal usage.’ That’s what I said, and this is why I said it. 

Now,\marginnote{13.1} mendicants, pleasure linked to sensuality is low, crude, ordinary, ignoble, and pointless. Indulging in such happiness is a principle beset by pain, harm, stress, and fever, and it is the wrong way. That’s why this is a principle beset by conflict. Breaking off such indulgence is a principle free of pain, harm, stress, and fever, and it is the right way. That’s why this is a principle free of conflict. 

Indulging\marginnote{13.7} in self-mortification is painful, ignoble, and pointless. It is a principle beset by pain, harm, stress, and fever, and it is the wrong way. That’s why this is a principle beset by conflict. Breaking off such indulgence is a principle free of pain, harm, stress, and fever, and it is the right way. That’s why this is a principle free of conflict. 

The\marginnote{13.13} middle way of practice by which the Realized One was awakened gives vision and knowledge, and leads to peace, direct knowledge, awakening, and extinguishment. It is a principle free of pain, harm, stress, and fever, and it is the right way. That’s why this is a principle free of conflict. 

Flattering\marginnote{13.16} and rebuking without teaching Dhamma is a principle beset by pain, harm, stress, and fever, and it is the wrong way. That’s why this is a principle beset by conflict. Neither flattering nor rebuking, and just teaching Dhamma is a principle free of pain, harm, stress, and fever, and it is the right way. That’s why this is a principle free of conflict. 

Sensual\marginnote{13.22} pleasure—a filthy, common, ignoble pleasure—is a principle beset by pain, harm, stress, and fever, and it is the wrong way. That’s why this is a principle beset by conflict. The pleasure of renunciation, the pleasure of seclusion, the pleasure of peace, the pleasure of awakening is a principle free of pain, harm, stress, and fever, and it is the right way. That’s why this is a principle free of conflict. 

Saying\marginnote{13.28} untrue, false, and harmful things behind someone’s back is a principle beset by pain, harm, stress, and fever, and it is the wrong way. That’s why this is a principle beset by conflict. Saying true and correct, but harmful things behind someone’s back is a principle beset by pain, harm, stress, and fever, and it is the wrong way. That’s why this is a principle beset by conflict. Saying true, correct, and beneficial things behind someone’s back is a principle free of pain, harm, stress, and fever, and it is the right way. That’s why this is a principle free of conflict. 

Saying\marginnote{13.37} untrue, false, and harmful things in someone’s presence is a principle beset by pain, harm, stress, and fever, and it is the wrong way. That’s why this is a principle beset by conflict. Saying true and correct, but harmful things in someone’s presence is a principle beset by pain, harm, stress, and fever, and it is the wrong way. That’s why this is a principle beset by conflict. Saying true, correct, and beneficial things in someone’s presence is a principle free of pain, harm, stress, and fever, and it is the right way. That’s why this is a principle free of conflict. 

Speaking\marginnote{13.46} hurriedly is a principle beset by pain, harm, stress, and fever, and it is the wrong way. That’s why this is a principle beset by conflict. Speaking unhurriedly is a principle free of pain, harm, stress, and fever, and it is the right way. That’s why this is a principle free of conflict. 

Insisting\marginnote{13.52} on local terminology and overriding normal usage is a principle beset by pain, harm, stress, and fever, and it is the wrong way. That’s why this is a principle beset by conflict. Not insisting on local terminology and not overriding normal usage is a principle free of pain, harm, stress, and fever, and it is the right way. That’s why this is a principle free of conflict. 

So\marginnote{14.1} you should train like this: ‘We shall know the principles beset by conflict and the principles free of conflict. Knowing this, we will practice the way free of conflict.’ 

And,\marginnote{14.3} mendicants, \textsanskrit{Subhūti}, the gentleman, practices the way of non-conflict.” 

That\marginnote{14.4} is what the Buddha said. Satisfied, the mendicants were happy with what the Buddha said. 

%
\section*{{\suttatitleacronym MN 140}{\suttatitletranslation The Analysis of the Elements }{\suttatitleroot Dhātuvibhaṅgasutta}}
\addcontentsline{toc}{section}{\tocacronym{MN 140} \toctranslation{The Analysis of the Elements } \tocroot{Dhātuvibhaṅgasutta}}
\markboth{The Analysis of the Elements }{Dhātuvibhaṅgasutta}
\extramarks{MN 140}{MN 140}

\scevam{So\marginnote{1.1} I have heard. }At one time the Buddha was wandering in the Magadhan lands when he arrived at \textsanskrit{Rājagaha}. He went to see Bhaggava the potter, and said, “Bhaggava, if it is no trouble, I’d like to spend a single night in your workshop.” 

“It’s\marginnote{2.2} no trouble, sir. But there’s a renunciate already staying there. If he allows it, sir, you may stay as long as you like.” 

Now\marginnote{3.1} at that time a gentleman named \textsanskrit{Pukkusāti} had gone forth from the lay life to homelessness out of faith in the Buddha. And it was he who had first taken up residence in the workshop. Then the Buddha approached Venerable \textsanskrit{Pukkusāti} and said, “Mendicant, if it is no trouble, I’d like to spend a single night in the workshop.” 

“The\marginnote{3.5} potter’s workshop is spacious, reverend. Please stay as long as you like.” 

Then\marginnote{4.1} the Buddha entered the workshop and spread out a grass mat to one side. He sat down cross-legged, with his body straight, and established mindfulness right there. He spent most of the night sitting in meditation, and so did \textsanskrit{Pukkusāti}. 

Then\marginnote{4.4} it occurred to the Buddha, “This gentleman’s conduct is impressive. Why don’t I question him?” 

So\marginnote{4.7} the Buddha said to \textsanskrit{Pukkusāti}, “In whose name have you gone forth, reverend? Who is your Teacher? Whose teaching do you believe in?” 

“Reverend,\marginnote{5.2} there is the ascetic Gotama—a Sakyan, gone forth from a Sakyan family. He has this good reputation: ‘That Blessed One is perfected, a fully awakened Buddha, accomplished in knowledge and conduct, holy, knower of the world, supreme guide for those who wish to train, teacher of gods and humans, awakened, blessed.’ I’ve gone forth in his name. That Blessed One is my Teacher, and I believe in his teaching.” 

“But\marginnote{5.8} mendicant, where is the Blessed One at present, the perfected one, the fully awakened Buddha?” 

“In\marginnote{5.9} the northern lands there is a city called \textsanskrit{Sāvatthī}. There the Blessed One is now staying, the perfected one, the fully awakened Buddha.” 

“But\marginnote{5.11} have you ever seen that Buddha? Would you recognize him if you saw him?” 

“No,\marginnote{5.13} I’ve never seen him, and I wouldn’t recognize him if I did.” 

Then\marginnote{6.1} it occurred to the Buddha, “This gentleman has gone forth in my name. Why don’t I teach him the Dhamma?” 

So\marginnote{6.4} the Buddha said to \textsanskrit{Pukkusāti}, “Mendicant, I shall teach you the Dhamma. Listen and pay close attention, I will speak.” 

“Yes,\marginnote{6.7} reverend,” replied \textsanskrit{Pukkusāti}. The Buddha said this: 

“‘This\marginnote{7.1} person has six elements, six fields of contact, eighteen mental preoccupations, and four foundations. Wherever they stand, the streams of identification do not flow. And when the streams of identification do not flow, they are called a sage at peace. Do not neglect wisdom; preserve truth; foster generosity; and train only for peace.’ This is the recitation passage for the analysis of the elements. 

‘This\marginnote{8.1} person has six elements.’ That’s what I said, but why did I say it? There are these six elements: the elements of earth, water, fire, air, space, and consciousness. ‘This person has six elements.’ That’s what I said, and this is why I said it. 

‘This\marginnote{9.1} person has six fields of contact.’ That’s what I said, but why did I say it? The fields of contact of the eye, ear, nose, tongue, body, and mind. ‘This person has six fields of contact.’ That’s what I said, and this is why I said it. 

‘This\marginnote{10.1} person has eighteen mental preoccupations.’ That’s what I said, but why did I say it? Seeing a sight with the eye, one is preoccupied with a sight that’s a basis for happiness or sadness or equanimity. Hearing a sound with the ear … Smelling an odor with the nose … Tasting a flavor with the tongue … 

Feeling\marginnote{10.7} a touch with the body … Becoming conscious of a thought with the mind, one is preoccupied with a thought that’s a basis for happiness or sadness or equanimity. So there are six preoccupations with happiness, six preoccupations with sadness, and six preoccupations with equanimity. ‘This person has eighteen mental preoccupations.’ That’s what I said, and this is why I said it. 

‘This\marginnote{11.1} person has four foundations.’ That’s what I said, but why did I say it? The foundations of wisdom, truth, generosity, and peace. ‘This person has four foundations.’ That’s what I said, and this is why I said it. 

‘Do\marginnote{12.1} not neglect wisdom; preserve truth; foster generosity; and train only for peace.’ That’s what I said, but why did I say it? 

And\marginnote{13.1} how does one not neglect wisdom? There are these six elements: the elements of earth, water, fire, air, space, and consciousness. 

And\marginnote{14.1} what is the earth element? The earth element may be interior or exterior. And what is the interior earth element? Anything hard, solid, and appropriated that’s internal, pertaining to an individual. This includes head hair, body hair, nails, teeth, skin, flesh, sinews, bones, bone marrow, kidneys, heart, liver, diaphragm, spleen, lungs, intestines, mesentery, undigested food, feces, or anything else hard, solid, and appropriated that’s internal, pertaining to an individual. This is called the interior earth element. The interior earth element and the exterior earth element are just the earth element. This should be truly seen with right understanding like this: ‘This is not mine, I am not this, this is not my self.’ When you truly see with right understanding, you reject the earth element, detaching the mind from the earth element. 

And\marginnote{15.1} what is the water element? The water element may be interior or exterior. And what is the interior water element? Anything that’s water, watery, and appropriated that’s internal, pertaining to an individual. This includes bile, phlegm, pus, blood, sweat, fat, tears, grease, saliva, snot, synovial fluid, urine, or anything else that’s water, watery, and appropriated that’s internal, pertaining to an individual. This is called the interior water element. The interior water element and the exterior water element are just the water element. This should be truly seen with right understanding like this: ‘This is not mine, I am not this, this is not my self.’ When you truly see with right understanding, you reject the water element, detaching the mind from the water element. 

And\marginnote{16.1} what is the fire element? The fire element may be interior or exterior. And what is the interior fire element? Anything that’s fire, fiery, and appropriated that’s internal, pertaining to an individual. This includes that which warms, that which ages, that which heats you up when feverish, that which properly digests food and drink, or anything else that’s fire, fiery, and appropriated that’s internal, pertaining to an individual. This is called the interior fire element. The interior fire element and the exterior fire element are just the fire element. This should be truly seen with right understanding like this: ‘This is not mine, I am not this, this is not my self.’ When you truly see with right understanding, you reject the fire element, detaching the mind from the fire element. 

And\marginnote{17.1} what is the air element? The air element may be interior or exterior. And what is the interior air element? Anything that’s air, airy, and appropriated that’s internal, pertaining to an individual. This includes winds that go up or down, winds in the belly or the bowels, winds that flow through the limbs, in-breaths and out-breaths, or anything else that’s air, airy, and appropriated that’s internal, pertaining to an individual. This is called the interior air element. The interior air element and the exterior air element are just the air element. This should be truly seen with right understanding like this: ‘This is not mine, I am not this, this is not my self.’ When you truly see with right understanding, you reject the air element, detaching the mind from the air element. 

And\marginnote{18.1} what is the space element? The space element may be interior or exterior. And what is the interior space element? Anything that’s space, spacious, and appropriated that’s internal, pertaining to an individual. This includes the ear canals, nostrils, and mouth; and the space for swallowing what is eaten and drunk, the space where it stays, and the space for excreting it from the nether regions. This is called the interior space element. The interior space element and the exterior space element are just the space element. This should be truly seen with right understanding like this: ‘This is not mine, I am not this, this is not my self.’ When you truly see with right understanding, you reject the space element, detaching the mind from the space element. 

There\marginnote{19.1} remains only consciousness, pure and bright. And what does that consciousness know? It knows ‘pleasure’ and ‘pain’ and ‘neutral’. Pleasant feeling arises dependent on a contact to be experienced as pleasant. When they feel a pleasant feeling, they know: ‘I feel a pleasant feeling.’ They know: ‘With the cessation of that contact to be experienced as pleasant, the corresponding pleasant feeling ceases and stops.’ 

Painful\marginnote{19.7} feeling arises dependent on a contact to be experienced as painful. When they feel a painful feeling, they know: ‘I feel a painful feeling.’ They know: ‘With the cessation of that contact to be experienced as painful, the corresponding painful feeling ceases and stops.’ 

Neutral\marginnote{19.10} feeling arises dependent on a contact to be experienced as neutral. When they feel a neutral feeling, they know: ‘I feel a neutral feeling.’ They know: ‘With the cessation of that contact to be experienced as neutral, the corresponding neutral feeling ceases and stops.’ 

When\marginnote{19.13} you rub two sticks together, heat is generated and fire is produced. But when you part the sticks and lay them aside, any corresponding heat ceases and stops. In the same way, pleasant feeling arises dependent on a contact to be experienced as pleasant. … 

They\marginnote{19.20} know: ‘With the cessation of that contact to be experienced as neutral, the corresponding neutral feeling ceases and stops.’ 

There\marginnote{20.1} remains only equanimity, pure, bright, pliable, workable, and radiant. It’s like when a goldsmith or a goldsmith’s apprentice prepares a forge, fires the crucible, picks up some gold with tongs and puts it in the crucible. From time to time they fan it, from time to time they sprinkle water on it, and from time to time they just watch over it. That gold becomes pliable, workable, and radiant, not brittle, and is ready to be worked. Then the goldsmith can successfully create any kind of ornament they want, whether a bracelet, earrings, a necklace, or a golden garland. In the same way, there remains only equanimity, pure, bright, pliable, workable, and radiant. 

They\marginnote{21.1} understand: ‘If I were to apply this equanimity, so pure and bright, to the dimension of infinite space, my mind would develop accordingly. And this equanimity of mine, relying on that and grasping it, would remain for a very long time. If I were to apply this equanimity, so pure and bright, to the dimension of infinite consciousness, my mind would develop accordingly. And this equanimity of mine, relying on that and grasping it, would remain for a very long time. If I were to apply this equanimity, so pure and bright, to the dimension of nothingness, my mind would develop accordingly. And this equanimity of mine, relying on that and grasping it, would remain for a very long time. If I were to apply this equanimity, so pure and bright, to the dimension of neither perception nor non-perception, my mind would develop accordingly. And this equanimity of mine, relying on that and grasping it, would remain for a very long time.’ 

They\marginnote{22.1} understand: ‘If I were to apply this equanimity, so pure and bright, to the dimension of infinite space, my mind would develop accordingly. But that is conditioned. If I were to apply this equanimity, so pure and bright, to the dimension of infinite consciousness … nothingness … neither perception nor non-perception, my mind would develop accordingly. But that is conditioned.’ 

They\marginnote{22.10} neither make a choice nor form an intention to continue existence or to end existence. Because of this, they don’t grasp at anything in the world. Not grasping, they’re not anxious. Not being anxious, they personally become extinguished. 

They\marginnote{22.13} understand: ‘Rebirth is ended, the spiritual journey has been completed, what had to be done has been done, there is no return to any state of existence.’ 

If\marginnote{23.1} they feel a pleasant feeling, they understand that it’s impermanent, that they’re not attached to it, and that they don’t take pleasure in it. If they feel a painful feeling, they understand that it’s impermanent, that they’re not attached to it, and that they don’t take pleasure in it. If they feel a neutral feeling, they understand that it’s impermanent, that they’re not attached to it, and that they don’t take pleasure in it. 

If\marginnote{24.1} they feel a pleasant feeling, they feel it detached. If they feel a painful feeling, they feel it detached. If they feel a neutral feeling, they feel it detached. Feeling the end of the body approaching, they understand: ‘I feel the end of the body approaching.’ Feeling the end of life approaching, they understand: ‘I feel the end of life approaching.’ They understand: ‘When my body breaks up and my life has come to an end, everything that’s felt, since I no longer take pleasure in it, will become cool right here.’ 

Suppose\marginnote{24.6} an oil lamp depended on oil and a wick to burn. As the oil and the wick are used up, it would be extinguished due to lack of fuel. In the same way, feeling the end of the body approaching, they understand: ‘I feel the end of the body approaching.’ Feeling the end of life approaching, they understand: ‘I feel the end of life approaching.’ They understand: ‘When my body breaks up and my life has come to an end, everything that’s felt, since I no longer take pleasure in it, will become cool right here.’ 

Therefore\marginnote{25.1} a mendicant thus endowed is endowed with the ultimate foundation of wisdom. For this is the ultimate noble wisdom, namely, the knowledge of the ending of suffering. 

Their\marginnote{26.1} freedom, being founded on truth, is unshakable. For that which is false has a deceptive nature, while that which is true has an undeceptive nature—extinguishment. Therefore a mendicant thus endowed is endowed with the ultimate resolve of truth. For this is the ultimate noble truth, namely, that which has an undeceptive nature—extinguishment. 

In\marginnote{27.1} their ignorance, they used to acquire attachments. Those have been cut off at the root, made like a palm stump, obliterated so they are unable to arise in the future. Therefore a mendicant thus endowed is endowed with the ultimate foundation of generosity. For this is the ultimate noble generosity, namely, letting go of all attachments. 

In\marginnote{28.1} their ignorance, they used to be covetous, full of desire and lust. That has been cut off at the root, made like a palm stump, obliterated so it’s unable to arise in the future. In their ignorance, they used to be contemptuous, full of ill will and malevolence. That has been cut off at the root, made like a palm stump, obliterated so it’s unable to arise in the future. In their ignorance, they used to be ignorant, full of delusion. That has been cut off at the root, made like a palm stump, obliterated so it’s unable to arise in the future. Therefore a mendicant thus endowed is endowed with the ultimate foundation of peace. For this is the ultimate noble peace, namely, the pacification of greed, hate, and delusion. 

‘Do\marginnote{29.1} not neglect wisdom; preserve truth; foster generosity; and train only for peace.’ That’s what I said, and this is why I said it. 

‘Wherever\marginnote{30.1} they stand, the streams of identification do not flow. And when the streams of identification do not flow, they are called a sage at peace.’ That’s what I said, but why did I say it? 

These\marginnote{31.1} are all forms of identifying: ‘I am’, ‘I am this’, ‘I will be’, ‘I will not be’, ‘I will have form’, ‘I will be formless’, ‘I will be percipient’, ‘I will be non-percipient’, ‘I will be neither percipient nor non-percipient.’ Identification is a disease, a boil, a dart. Having gone beyond all identification, one is called a sage at peace. The sage at peace is not reborn, does not grow old, and does not die. They are not shaken, and do not yearn. For they have nothing which would cause them to be reborn. Not being reborn, how could they grow old? Not growing old, how could they die? Not dying, how could they be shaken? Not shaking, for what could they yearn? 

‘Wherever\marginnote{32.1} they stand, the streams of identification do not flow. And when the streams of identification do not flow, they are called a sage at peace.’ That’s what I said, and this is why I said it. Mendicant, you should remember this brief analysis of the six elements.” 

Then\marginnote{33.1} Venerable \textsanskrit{Pukkusāti} thought, “It seems the Teacher has come to me! The Holy One has come to me! The fully awakened Buddha has come to me!” He got up from his seat, arranged his robe over one shoulder, bowed with his head at the Buddha’s feet, and said, “I have made a mistake, sir. It was foolish, stupid, and unskillful of me to presume to address the Buddha as ‘reverend’. Please, sir, accept my mistake for what it is, so I will restrain myself in future.” 

“Indeed,\marginnote{33.5} mendicant, you made a mistake. It was foolish, stupid, and unskillful of you to act in that way. But since you have recognized your mistake for what it is, and have dealt with it properly, I accept it. For it is growth in the training of the Noble One to recognize a mistake for what it is, deal with it properly, and commit to restraint in the future.” 

“Sir,\marginnote{34.1} may I receive the going forth, the ordination in the Buddha’s presence?” 

“But\marginnote{34.2} mendicant, are your bowl and robes complete?” 

“No,\marginnote{34.3} sir, they are not.” 

“The\marginnote{34.4} Realized Ones do not ordain those whose bowl and robes are incomplete.” 

And\marginnote{35.1} then Venerable \textsanskrit{Pukkusāti} approved and agreed with what the Buddha said. He got up from his seat, bowed, and respectfully circled the Buddha, keeping him on his right, before leaving. 

But\marginnote{35.2} while he was wandering in search of a bowl and robes, a stray cow took his life. 

Then\marginnote{36.1} several mendicants went up to the Buddha, bowed, sat down to one side, and said to him, “Sir, the gentleman named \textsanskrit{Pukkusāti}, who was advised in brief by the Buddha, has passed away. Where has he been reborn in his next life?” 

“Mendicants,\marginnote{36.4} \textsanskrit{Pukkusāti} was astute. He practiced in line with the teachings, and did not trouble me about the teachings. With the ending of the five lower fetters, he’s been reborn spontaneously and will become extinguished there, not liable to return from that world.” 

That\marginnote{36.6} is what the Buddha said. Satisfied, the mendicants were happy with what the Buddha said. 

%
\section*{{\suttatitleacronym MN 141}{\suttatitletranslation The Analysis of the Truths }{\suttatitleroot Saccavibhaṅgasutta}}
\addcontentsline{toc}{section}{\tocacronym{MN 141} \toctranslation{The Analysis of the Truths } \tocroot{Saccavibhaṅgasutta}}
\markboth{The Analysis of the Truths }{Saccavibhaṅgasutta}
\extramarks{MN 141}{MN 141}

\scevam{So\marginnote{1.1} I have heard. }At one time the Buddha was staying near Benares, in the deer park at Isipatana. There the Buddha addressed the mendicants, “Mendicants!” 

“Venerable\marginnote{1.5} sir,” they replied. The Buddha said this: 

“Near\marginnote{2.1} Benares, in the deer park at Isipatana, the Realized One, the perfected one, the fully awakened Buddha rolled forth the supreme Wheel of Dhamma. And that wheel cannot be rolled back by any ascetic or brahmin or god or \textsanskrit{Māra} or \textsanskrit{Brahmā} or by anyone in the world. It is the teaching, advocating, establishing, clarifying, analyzing, and revealing of the four noble truths. What four? 

The\marginnote{3.1} noble truths of suffering, the origin of suffering, the cessation of suffering, and the practice that leads to the cessation of suffering. 

Near\marginnote{4.1} Benares, in the deer park at Isipatana, the Realized One, the perfected one, the fully awakened Buddha rolled forth the supreme Wheel of Dhamma. And that wheel cannot be rolled back by any ascetic or brahmin or god or \textsanskrit{Māra} or \textsanskrit{Brahmā} or by anyone in the world. It is the teaching, advocating, establishing, clarifying, analyzing, and revealing of the four noble truths. 

Mendicants,\marginnote{5.1} you should cultivate friendship with \textsanskrit{Sāriputta} and \textsanskrit{Moggallāna}. You should associate with \textsanskrit{Sāriputta} and \textsanskrit{Moggallāna}. They’re astute, and they support their spiritual companions. \textsanskrit{Sāriputta} is just like the mother who gives birth, while \textsanskrit{Moggallāna} is like the one who raises the child. \textsanskrit{Sāriputta} guides people to the fruit of stream-entry, \textsanskrit{Moggallāna} to the highest goal. \textsanskrit{Sāriputta} is able to explain, teach, assert, establish, clarify, analyze, and reveal the four noble truths in detail.” 

That\marginnote{6.1} is what the Buddha said. When he had spoken, the Holy One got up from his seat and entered his dwelling. 

Then\marginnote{7.1} soon after the Buddha left, Venerable \textsanskrit{Sāriputta} said to the mendicants, “Reverends, mendicants!” 

“Reverend,”\marginnote{7.3} they replied. \textsanskrit{Sāriputta} said this: 

“Near\marginnote{8.1} Benares, in the deer park at Isipatana, the Realized One, the perfected one, the fully awakened Buddha rolled forth the supreme Wheel of Dhamma. And that wheel cannot be rolled back by any ascetic or brahmin or god or \textsanskrit{Māra} or \textsanskrit{Brahmā} or by anyone in the world. It is the teaching, advocating, establishing, clarifying, analyzing, and revealing of the four noble truths. What four? 

The\marginnote{9.1} noble truths of suffering, the origin of suffering, the cessation of suffering, and the practice that leads to the cessation of suffering. 

And\marginnote{10.1} what is the noble truth of suffering? Rebirth is suffering; old age is suffering; death is suffering; sorrow, lamentation, pain, sadness, and distress are suffering; not getting what you wish for is suffering. In brief, the five grasping aggregates are suffering. 

And\marginnote{11.1} what is rebirth? The rebirth, inception, conception, reincarnation, manifestation of the aggregates, and acquisition of the sense fields of the various sentient beings in the various orders of sentient beings. This is called rebirth. 

And\marginnote{12.1} what is old age? The old age, decrepitude, broken teeth, grey hair, wrinkly skin, diminished vitality, and failing faculties of the various sentient beings in the various orders of sentient beings. This is called old age. 

And\marginnote{13.1} what is death? The passing away, perishing, disintegration, demise, mortality, death, decease, breaking up of the aggregates, laying to rest of the corpse, and cutting off of the life faculty of the various sentient beings in the various orders of sentient beings. This is called death. 

And\marginnote{14.1} what is sorrow? The sorrow, sorrowing, state of sorrow, inner sorrow, inner deep sorrow in someone who has undergone misfortune, who has experienced suffering. This is called sorrow. 

And\marginnote{15.1} what is lamentation? The wail, lament, wailing, lamenting, state of wailing and lamentation in someone who has undergone misfortune, who has experienced suffering. This is called lamentation. 

And\marginnote{16.1} what is pain? Physical pain, physical displeasure, the painful, unpleasant feeling that’s born from physical contact. This is called pain. 

And\marginnote{17.1} what is sadness? Mental pain, mental displeasure, the painful, unpleasant feeling that’s born from mind contact. This is called sadness. 

And\marginnote{18.1} what is distress? The stress, distress, state of stress and distress in someone who has undergone misfortune, who has experienced suffering. This is called distress. 

And\marginnote{19.1} what is ‘not getting what you wish for is suffering’? In sentient beings who are liable to be reborn, such a wish arises: ‘Oh, if only we were not liable to be reborn! If only rebirth would not come to us!’ But you can’t get that by wishing. This is: ‘not getting what you wish for is suffering.’ In sentient beings who are liable to grow old … fall ill … die … experience sorrow, lamentation, pain, sadness, and distress, such a wish arises: ‘Oh, if only we were not liable to experience sorrow, lamentation, pain, sadness, and distress! If only sorrow, lamentation, pain, sadness, and distress would not come to us!’ But you can’t get that by wishing. This is: ‘not getting what you wish for is suffering.’ 

And\marginnote{20.1} what is ‘in brief, the five grasping aggregates are suffering’? They are the grasping aggregates that consist of form, feeling, perception, choices, and consciousness. This is called ‘in brief, the five grasping aggregates are suffering.’ This is called the noble truth of suffering. 

And\marginnote{21.1} what is the noble truth of the origin of suffering? It’s the craving that leads to future lives, mixed up with relishing and greed, chasing pleasure in various realms. That is, craving for sensual pleasures, craving to continue existence, and craving to end existence. This is called the noble truth of the origin of suffering. 

And\marginnote{22.1} what is the noble truth of the cessation of suffering? It’s the fading away and cessation of that very same craving with nothing left over; giving it away, letting it go, releasing it, and not adhering to it. This is called the noble truth of the cessation of suffering. 

And\marginnote{23.1} what is the noble truth of the practice that leads to the cessation of suffering? It is simply this noble eightfold path, that is: right view, right thought, right speech, right action, right livelihood, right effort, right mindfulness, and right immersion. 

And\marginnote{24.1} what is right view? Knowing about suffering, the origin of suffering, the cessation of suffering, and the practice that leads to the cessation of suffering. This is called right view. 

And\marginnote{25.1} what is right thought? Thoughts of renunciation, good will, and harmlessness. This is called right thought. 

And\marginnote{26.1} what is right speech? Refraining from lying, divisive speech, harsh speech, and talking nonsense. This is called right speech. 

And\marginnote{27.1} what is right action? Refraining from killing living creatures, stealing, and sexual misconduct. This is called right action. 

And\marginnote{28.1} what is right livelihood? It’s when a noble disciple gives up wrong livelihood and earns a living by right livelihood. This is called right livelihood. 

And\marginnote{29.1} what is right effort? It’s when a mendicant generates enthusiasm, tries, makes an effort, exerts the mind, and strives so that bad, unskillful qualities don’t arise. They generate enthusiasm, try, make an effort, exert the mind, and strive so that bad, unskillful qualities that have arisen are given up. They generate enthusiasm, try, make an effort, exert the mind, and strive so that skillful qualities arise. They generate enthusiasm, try, make an effort, exert the mind, and strive so that skillful qualities that have arisen remain, are not lost, but increase, mature, and are completed by development. This is called right effort. 

And\marginnote{30.1} what is right mindfulness? It’s when a mendicant meditates by observing an aspect of the body—keen, aware, and mindful, rid of desire and aversion for the world. They meditate observing an aspect of feelings … mind … principles—keen, aware, and mindful, rid of desire and aversion for the world. This is called right mindfulness. 

And\marginnote{31.1} what is right immersion? It’s when a mendicant, quite secluded from sensual pleasures, secluded from unskillful qualities, enters and remains in the first absorption, which has the rapture and bliss born of seclusion, while placing the mind and keeping it connected. As the placing of the mind and keeping it connected are stilled, they enter and remain in the second absorption, which has the rapture and bliss born of immersion, with internal clarity and confidence, and unified mind, without placing the mind and keeping it connected. And with the fading away of rapture, they enter and remain in the third absorption, where they meditate with equanimity, mindful and aware, personally experiencing the bliss of which the noble ones declare, ‘Equanimous and mindful, one meditates in bliss.’ Giving up pleasure and pain, and ending former happiness and sadness, they enter and remain in the fourth absorption, without pleasure or pain, with pure equanimity and mindfulness. This is called right immersion. This is called the noble truth of the practice that leads to the cessation of suffering. 

Near\marginnote{32.1} Benares, in the deer park at Isipatana, the Realized One, the perfected one, the fully awakened Buddha rolled forth the supreme Wheel of Dhamma. And that wheel cannot be rolled back by any ascetic or brahmin or god or \textsanskrit{Māra} or \textsanskrit{Brahmā} or by anyone in the world. It is the teaching, advocating, establishing, clarifying, analyzing, and revealing of the four noble truths.” 

That’s\marginnote{32.3} what Venerable \textsanskrit{Sāriputta} said. Satisfied, the mendicants were happy with what \textsanskrit{Sāriputta} said. 

%
\section*{{\suttatitleacronym MN 142}{\suttatitletranslation The Analysis of Religious Donations }{\suttatitleroot Dakkhiṇāvibhaṅgasutta}}
\addcontentsline{toc}{section}{\tocacronym{MN 142} \toctranslation{The Analysis of Religious Donations } \tocroot{Dakkhiṇāvibhaṅgasutta}}
\markboth{The Analysis of Religious Donations }{Dakkhiṇāvibhaṅgasutta}
\extramarks{MN 142}{MN 142}

\scevam{So\marginnote{1.1} I have heard. }At one time the Buddha was staying in the land of the Sakyans, near Kapilavatthu in the Banyan Tree Monastery. 

Then\marginnote{2.1} \textsanskrit{Mahāpajāpati} \textsanskrit{Gotamī} approached the Buddha bringing a new pair of garments. She bowed, sat down to one side, and said to the Buddha, “Sir, I have spun and woven this new pair of garments specially for the Buddha. May the Buddha please accept this from me out of compassion.” 

When\marginnote{2.4} she said this, the Buddha said to her, “Give it to the \textsanskrit{Saṅgha}, \textsanskrit{Gotamī}. When you give to the \textsanskrit{Saṅgha}, both the \textsanskrit{Saṅgha} and I will be honored.” 

For\marginnote{2.7} a second time … 

For\marginnote{2.13} a third time, \textsanskrit{Mahāpajāpatī} \textsanskrit{Gotamī} said to the Buddha, “Sir, I have spun and woven this new pair of garments specially for the Buddha. May the Buddha please accept this from me out of compassion.” 

And\marginnote{2.16} for a third time, the Buddha said to her, “Give it to the \textsanskrit{Saṅgha}, \textsanskrit{Gotamī}. When you give to the \textsanskrit{Saṅgha}, both the \textsanskrit{Saṅgha} and I will be honored.” 

When\marginnote{3.1} he said this, Venerable Ānanda said to the Buddha, “Sir, please accept the new pair of garments from \textsanskrit{Mahāpajāpatī} \textsanskrit{Gotamī}. Sir, \textsanskrit{Mahāpajāpatī} was very helpful to the Buddha. As his aunt, she raised him, nurtured him, and gave him her milk. When the Buddha’s birth mother passed away, she nurtured him at her own breast. 

And\marginnote{3.5} the Buddha has been very helpful to \textsanskrit{Mahāpajāpatī}. It is owing to the Buddha that \textsanskrit{Mahāpajāpatī} has gone for refuge to the Buddha, the teaching, and the \textsanskrit{Saṅgha}. It’s owing to the Buddha that she refrains from killing living creatures, stealing, committing sexual misconduct, lying, and taking alcoholic drinks that cause negligence. It’s owing to the Buddha that she has experiential confidence in the Buddha, the teaching, and the \textsanskrit{Saṅgha}, and has the ethics loved by the noble ones. It’s owing to the Buddha that she is free of doubt regarding suffering, its origin, its cessation, and the practice that leads to its cessation. The Buddha has been very helpful to \textsanskrit{Mahāpajāpatī}.” 

“That’s\marginnote{4.1} so true, Ānanda. When someone has enabled you to go for refuge, it’s not easy to repay them by bowing down to them, rising up for them, greeting them with joined palms, and observing proper etiquette for them; or by providing them with robes, almsfood, lodgings, and medicines and supplies for the sick. 

When\marginnote{4.4} someone has enabled you to refrain from killing, stealing, sexual misconduct, lying, and alcoholic drinks that cause negligence, it’s not easy to repay them … 

When\marginnote{4.6} someone has enabled you to have experiential confidence in the Buddha, the teaching, and the \textsanskrit{Saṅgha}, and the ethics loved by the noble ones, it’s not easy to repay them … 

When\marginnote{4.8} someone has enabled you to be free of doubt regarding suffering, its origin, its cessation, and the practice that leads to its cessation, it’s not easy to repay them by bowing down to them, rising up for them, greeting them with joined palms, and observing proper etiquette for them; or by providing them with robes, almsfood, lodgings, and medicines and supplies for the sick. 

Ānanda,\marginnote{5.1} there are these fourteen religious donations to individuals. What fourteen? One gives a gift to the Realized One, the perfected one, the fully awakened Buddha. This is the first religious donation to an individual. One gives a gift to a Buddha awakened for themselves. This is the second religious donation to an individual. One gives a gift to a perfected one. This is the third religious donation to an individual. One gives a gift to someone practicing to realize the fruit of perfection. This is the fourth religious donation to an individual. One gives a gift to a non-returner. This is the fifth religious donation to an individual. One gives a gift to someone practicing to realize the fruit of non-return. This is the sixth religious donation to an individual. One gives a gift to a once-returner. This is the seventh religious donation to an individual. One gives a gift to someone practicing to realize the fruit of once-return. This is the eighth religious donation to an individual. One gives a gift to a stream-enterer. This is the ninth religious donation to an individual. One gives a gift to someone practicing to realize the fruit of stream-entry. This is the tenth religious donation to an individual. One gives a gift to an outsider who is free of sensual desire. This is the eleventh religious donation to an individual. One gives a gift to an ordinary person who has good ethical conduct. This is the twelfth religious donation to an individual. One gives a gift to an ordinary person who has bad ethical conduct. This is the thirteenth religious donation to an individual. One gives a gift to an animal. This is the fourteenth religious donation to an individual. 

Now,\marginnote{6.1} Ānanda, gifts to the following persons may be expected to yield the following returns. To an animal, a hundred times. To an unethical ordinary person, a thousand. To an ethical ordinary person, a hundred thousand. To an outsider free of sensual desire, 10,000,000,000. But a gift to someone practicing to realize the fruit of stream-entry may be expected to yield incalculable, immeasurable returns. How much more so a gift to a stream-enterer, someone practicing to realize the fruit of once-return, a once-returner, someone practicing to realize the fruit of non-return, a non-returner, someone practicing to realize the fruit of perfection, a perfected one, or a Buddha awakened for themselves? How much more so a Realized One, a perfected one, a fully awakened Buddha? 

But\marginnote{7.1} there are, Ānanda, seven religious donations bestowed on a \textsanskrit{Saṅgha}. What seven? One gives a gift to the communities of both monks and nuns headed by the Buddha. This is the first religious donation bestowed on a \textsanskrit{Saṅgha}. One gives a gift to the communities of both monks and nuns after the Buddha has finally become extinguished. This is the second religious donation bestowed on a \textsanskrit{Saṅgha}. One gives a gift to the \textsanskrit{Saṅgha} of monks. This is the third religious donation bestowed on a \textsanskrit{Saṅgha}. One gives a gift to the \textsanskrit{Saṅgha} of nuns. This is the fourth religious donation bestowed on a \textsanskrit{Saṅgha}. One gives a gift, thinking: ‘Appoint this many monks and nuns for me from the \textsanskrit{Saṅgha}.’ This is the fifth religious donation bestowed on a \textsanskrit{Saṅgha}. One gives a gift, thinking: ‘Appoint this many monks for me from the \textsanskrit{Saṅgha}.’ This is the sixth religious donation bestowed on a \textsanskrit{Saṅgha}. One gives a gift, thinking: ‘Appoint this many nuns for me from the \textsanskrit{Saṅgha}.’ This is the seventh religious donation bestowed on a \textsanskrit{Saṅgha}. 

In\marginnote{8.1} times to come there will be members of the spiritual family merely by virtue of wearing ocher cloth around their necks; but they are unethical and of bad character. People will give gifts to those unethical people in the name of the \textsanskrit{Saṅgha}. Even then, I say, a religious donation bestowed on the \textsanskrit{Saṅgha} is incalculable and immeasurable. But I say that there is no way a personal offering can be more fruitful than one bestowed on a \textsanskrit{Saṅgha}. 

Ānanda,\marginnote{9.1} there are these four ways of purifying a religious donation. What four? There’s a religious donation that’s purified by the giver, not the recipient. There’s a religious donation that’s purified by the recipient, not the giver. There’s a religious donation that’s purified by neither the giver nor the recipient. There’s a religious donation that’s purified by both the giver and the recipient. 

And\marginnote{10.1} how is a religious donation purified by the giver, not the recipient? It’s when the giver is ethical, of good character, but the recipient is unethical, of bad character. 

And\marginnote{11.1} how is a religious donation purified by the recipient, not the giver? It’s when the giver is unethical, of bad character, but the recipient is ethical, of good character. 

And\marginnote{12.1} how is a religious donation purified by neither the giver nor the recipient? It’s when both the giver and the recipient are unethical, of bad character. 

And\marginnote{13.1} how is a religious donation purified by both the giver and the recipient? It’s when both the giver and the recipient are ethical, of good character. These are the four ways of purifying a religious donation.” 

That\marginnote{14.1} is what the Buddha said. Then the Holy One, the Teacher, went on to say: 

\begin{verse}%
“When\marginnote{14.3} an ethical person with trusting heart \\
gives a proper gift to unethical persons, \\
trusting in the ample fruit of deeds, \\
that offering is purified by the giver. 

When\marginnote{14.7} an unethical and untrusting person, \\
gives an improper gift to ethical persons, \\
not trusting in the ample fruit of deeds, \\
that offering is purified by the receivers. 

When\marginnote{14.11} an unethical and untrusting person, \\
gives an improper gift to unethical persons, \\
not trusting in the ample fruit of deeds, \\
I declare that gift is not very fruitful. 

When\marginnote{14.15} an ethical person with trusting heart \\
gives a proper gift to ethical persons, \\
trusting in the ample fruit of deeds, \\
I declare that gift is abundantly fruitful. 

But\marginnote{14.19} when a passionless one gives to the passionless \\
a proper gift with trusting heart, \\
trusting in the ample fruit of deeds, \\
that’s truly the best of material gifts.” 

%
\end{verse}

%
\addtocontents{toc}{\let\protect\contentsline\protect\nopagecontentsline}
\chapter*{The Chapter on the Six Senses}
\addcontentsline{toc}{chapter}{\tocchapterline{The Chapter on the Six Senses}}
\addtocontents{toc}{\let\protect\contentsline\protect\oldcontentsline}

%
\section*{{\suttatitleacronym MN 143}{\suttatitletranslation Advice to Anāthapiṇḍika }{\suttatitleroot Anāthapiṇḍikovādasutta}}
\addcontentsline{toc}{section}{\tocacronym{MN 143} \toctranslation{Advice to Anāthapiṇḍika } \tocroot{Anāthapiṇḍikovādasutta}}
\markboth{Advice to Anāthapiṇḍika }{Anāthapiṇḍikovādasutta}
\extramarks{MN 143}{MN 143}

\scevam{So\marginnote{1.1} I have heard. }At one time the Buddha was staying near \textsanskrit{Sāvatthī} in Jeta’s Grove, \textsanskrit{Anāthapiṇḍika}’s monastery. 

Now\marginnote{2.1} at that time the householder \textsanskrit{Anāthapiṇḍika} was sick, suffering, gravely ill. Then he addressed a man, “Please, mister, go to the Buddha, and in my name bow with your head to his feet. Say to him: ‘Sir, the householder \textsanskrit{Anāthapiṇḍika} is sick, suffering, gravely ill. He bows with his head to your feet.’ Then go to Venerable \textsanskrit{Sāriputta}, and in my name bow with your head to his feet. Say to him: ‘Sir, the householder \textsanskrit{Anāthapiṇḍika} is sick, suffering, gravely ill. He bows with his head to your feet.’ And then say: ‘Sir, please visit him at his home out of compassion.’” 

“Yes,\marginnote{2.11} sir,” that man replied. He did as \textsanskrit{Anāthapiṇḍika} asked. \textsanskrit{Sāriputta} consented in silence. 

Then\marginnote{3.1} Venerable \textsanskrit{Sāriputta} robed up in the morning and, taking his bowl and robe, went with Venerable Ānanda as his second monk to \textsanskrit{Anāthapiṇḍika}’s home. He sat down on the seat spread out, and said to \textsanskrit{Anāthapiṇḍika}, “I hope you’re keeping well, householder; I hope you’re alright. And I hope the pain is fading, not growing, that its fading is evident, not its growing.” 

“I’m\marginnote{4.1} not keeping well, Master \textsanskrit{Sāriputta}, I’m not alright. The pain is terrible and growing, not fading, its growing, not its fading, is evident. The winds piercing my head are so severe, it feels like a strong man drilling into my head with a sharp point. The pain in my head is so severe, it feels like a strong man tightening a tough leather strap around my head. The winds slicing my belly are so severe, like a deft butcher or their apprentice were slicing open a cows’s belly open with a meat cleaver. The burning in my body is so severe, it feels like two strong men grabbing a weaker man by the arms to burn and scorch him on a pit of glowing coals. That’s how severe the burning is in my body. I’m not keeping well, Master \textsanskrit{Sāriputta}, I’m not alright. The pain is terrible and growing, not fading, its growing, not its fading, is evident.” 

“That’s\marginnote{5.1} why, householder, you should train like this: ‘I shall not grasp the eye, and there shall be no consciousness of mine dependent on the eye.’ That’s how you should train. 

You\marginnote{5.4} should train like this: ‘I shall not grasp the ear, and there shall be no consciousness of mine dependent on the ear.’ … 

‘I\marginnote{5.7} shall not grasp the nose, and there shall be no consciousness of mine dependent on the nose.’ … 

‘I\marginnote{5.10} shall not grasp the tongue, and there shall be no consciousness of mine dependent on the tongue.’ … 

‘I\marginnote{5.13} shall not grasp the body, and there shall be no consciousness of mine dependent on the body.’ … 

‘I\marginnote{5.16} shall not grasp the mind, and there shall be no consciousness of mine dependent on the mind.’ That’s how you should train. 

You\marginnote{6.1} should train like this: ‘I shall not grasp sight, and there shall be no consciousness of mine dependent on sight.’ … ‘I shall not grasp sound … smell … taste … touch … thought, and there shall be no consciousness of mine dependent on thought.’ That’s how you should train. 

You\marginnote{7.1} should train like this: ‘I shall not grasp eye consciousness, and there shall be no consciousness of mine dependent on eye consciousness.’ … ‘I shall not grasp ear consciousness … nose consciousness … tongue consciousness … body consciousness … mind consciousness, and there shall be no consciousness of mine dependent on mind consciousness.’ That’s how you should train. 

You\marginnote{8.1} should train like this: ‘I shall not grasp eye contact … ear contact … nose contact … tongue contact … body contact … mind contact, and there shall be no consciousness of mine dependent on mind contact.’ That’s how you should train. 

You\marginnote{9.1} should train like this: ‘I shall not grasp feeling born of eye contact … feeling born of ear contact … feeling born of nose contact … feeling born of tongue contact … feeling born of body contact … feeling born of mind contact, and there shall be no consciousness of mine dependent on the feeling born of mind contact.’ That’s how you should train. 

You\marginnote{10.1} should train like this: ‘I shall not grasp the earth element … water element … fire element … air element … space element … consciousness element, and there shall be no consciousness of mine dependent on the consciousness element.’ That’s how you should train. 

You\marginnote{11.1} should train like this: ‘I shall not grasp form … feeling … perception … choices … consciousness, and there shall be no consciousness of mine dependent on consciousness.’ That’s how you should train. 

You\marginnote{12.1} should train like this: ‘I shall not grasp the dimension of infinite space … the dimension of infinite consciousness … the dimension of nothingness … the dimension of neither perception nor non-perception, and there shall be no consciousness of mine dependent on the dimension of neither perception nor non-perception.’ That’s how you should train. 

You\marginnote{13.1} should train like this: ‘I shall not grasp this world, and there shall be no consciousness of mine dependent on this world.’ That’s how you should train. 

You\marginnote{14.1} should train like this: ‘I shall not grasp the other world, and there shall be no consciousness of mine dependent on the other world.’ That’s how you should train. You should train like this: ‘I shall not grasp whatever is seen, heard, thought, known, attained, sought, and explored by my mind, and there shall be no consciousness of mine dependent on that.’ That’s how you should train.” 

When\marginnote{15.1} he said this, \textsanskrit{Anāthapiṇḍika} cried and burst out in tears. Venerable Ānanda said to him, “Are you failing, householder? Are you fading, householder?” 

“No,\marginnote{15.4} sir. But for a long time I have paid homage to the Buddha and the esteemed mendicants. Yet I have never before heard such a Dhamma talk.” 

“Householder,\marginnote{15.7} it does not occur to us to teach such a Dhamma talk to white-clothed laypeople. Rather, we teach like this to those gone forth.” 

“Well\marginnote{15.9} then, Master \textsanskrit{Sāriputta}, let it occur to you to teach such a Dhamma talk to white-clothed laypeople as well! There are gentlemen with little dust in their eyes. They’re in decline because they haven’t heard the teaching. There will be those who understand the teaching!” 

And\marginnote{16.1} when the venerables \textsanskrit{Sāriputta} and Ānanda had given the householder \textsanskrit{Anāthapiṇḍika} this advice they got up from their seat and left. Not long after they had left, \textsanskrit{Anāthapiṇḍika} passed away and was reborn in the host of Joyful Gods. 

Then,\marginnote{17.1} late at night, the glorious god \textsanskrit{Anāthapiṇḍika}, lighting up the entire Jeta’s Grove, went up to the Buddha, bowed, stood to one side, and addressed the Buddha in verse: 

\begin{verse}%
“This\marginnote{17.3} is indeed that Jeta’s Grove, \\
frequented by the \textsanskrit{Saṅgha} of hermits, \\
where the King of Dhamma stayed: \\
it brings me joy! 

Deeds,\marginnote{17.7} knowledge, and principle; \\
ethical conduct, an excellent livelihood; \\
by these are mortals purified, \\
not by clan or wealth. 

That’s\marginnote{17.11} why an astute person, \\
seeing what’s good for themselves, \\
would examine the teaching rationally, \\
and thus be purified in it. 

\textsanskrit{Sāriputta}\marginnote{17.15} has true wisdom, \\
ethics, and also peace. \\
Any mendicant who has crossed over \\
can at best equal him.” 

%
\end{verse}

This\marginnote{18.1} is what the god \textsanskrit{Anāthapiṇḍika} said, and the teacher approved. Then the god \textsanskrit{Anāthapiṇḍika}, knowing that the teacher approved, bowed and respectfully circled the Buddha, keeping him on his right, before vanishing right there. 

Then,\marginnote{19.1} when the night had passed, the Buddha told the mendicants all that had happened. 

When\marginnote{20.1} he had spoken, Venerable Ānanda said to the Buddha: 

“Sir,\marginnote{20.2} that god must surely have been \textsanskrit{Anāthapiṇḍika}. For the householder \textsanskrit{Anāthapiṇḍika} was devoted to Venerable \textsanskrit{Sāriputta}.” 

“Good,\marginnote{20.4} good, Ānanda. You’ve reached the logical conclusion, as far as logic goes. For that was indeed the god \textsanskrit{Anāthapiṇḍika}.” 

That\marginnote{20.7} is what the Buddha said. Satisfied, Venerable Ānanda was happy with what the Buddha said. 

%
\section*{{\suttatitleacronym MN 144}{\suttatitletranslation Advice to Channa }{\suttatitleroot Channovādasutta}}
\addcontentsline{toc}{section}{\tocacronym{MN 144} \toctranslation{Advice to Channa } \tocroot{Channovādasutta}}
\markboth{Advice to Channa }{Channovādasutta}
\extramarks{MN 144}{MN 144}

\scevam{So\marginnote{1.1} I have heard. }At one time the Buddha was staying near \textsanskrit{Rājagaha}, in the Bamboo Grove, the squirrels’ feeding ground. 

Now\marginnote{2.1} at that time the venerables \textsanskrit{Sāriputta}, \textsanskrit{Mahācunda}, and Channa were staying on the Vulture’s Peak Mountain. 

Now\marginnote{3.1} at that time Venerable Channa was sick, suffering, gravely ill. 

Then\marginnote{3.2} in the late afternoon, Venerable \textsanskrit{Sāriputta} came out of retreat, went to Venerable \textsanskrit{Mahācunda} and said to him, “Come, Reverend Cunda, let’s go to see Venerable Channa and ask about his illness.” 

“Yes,\marginnote{3.4} reverend,” replied \textsanskrit{Mahācunda}. 

And\marginnote{4.1} then \textsanskrit{Sāriputta} and \textsanskrit{Mahācunda} went to see Channa and exchanged greetings with him. When the greetings and polite conversation were over, they sat down to one side. Then \textsanskrit{Sāriputta} said to Channa, “I hope you’re keeping well, Reverend Channa; I hope you’re alright. I hope that your pain is fading, not growing, that its fading is evident, not its growing.” 

“Reverend\marginnote{5.1} \textsanskrit{Sāriputta}, I’m not keeping well, I’m not alright. The pain is terrible and growing, not fading; its growing is evident, not its fading. The winds piercing my head are so severe, it feels like a strong man drilling into my head with a sharp point. The pain in my head is so severe, it feels like a strong man tightening a tough leather strap around my head. The winds slicing my belly are so severe, like a deft butcher or their apprentice were slicing open a cows’s belly with a meat cleaver. The burning in my body is so severe, it feels like two strong men grabbing a weaker man by the arms to burn and scorch him on a pit of glowing coals. I’m not keeping well, I’m not alright. The pain is terrible and growing, not fading; its growing is evident, not its fading. Reverend \textsanskrit{Sāriputta}, I will slit my wrists. I don’t wish to live.” 

“Please\marginnote{6.1} don’t slit your wrists! Venerable Channa, keep going! We want you to keep going. If you don’t have any suitable food, we’ll find it for you. If you don’t have suitable medicine, we’ll find it for you. If you don’t have a capable carer, we’ll find one for you. Please don’t slit your wrists! Venerable Channa, keep going! We want you to keep going.” 

“Reverend\marginnote{7.1} \textsanskrit{Sāriputta}, it’s not that I don’t have suitable food, or suitable medicine, or a capable carer. Moreover, for a long time now I have served the Teacher with love, not without love. For it is proper for a disciple to serve the Teacher with love, not without love. You should remember this: ‘The mendicant Channa will slit his wrists blamelessly.’” 

“I’d\marginnote{8.1} like to ask you about a certain point, if you’d take the time to answer.” 

“Ask,\marginnote{8.2} Reverend \textsanskrit{Sāriputta}. When I’ve heard it I’ll know.” 

“Reverend\marginnote{9.1} Channa, do you regard the eye, eye consciousness, and things knowable by eye consciousness in this way: ‘This is mine, I am this, this is my self’? Do you regard the ear … nose … tongue … body … mind, mind consciousness, and things knowable by mind consciousness in this way: ‘This is mine, I am this, this is my self’?” 

“Reverend\marginnote{9.7} \textsanskrit{Sāriputta}, I regard the eye, eye consciousness, and things knowable by eye consciousness in this way: ‘This is not mine, I am not this, this is not my self.’ I regard the ear … nose … tongue … body … mind, mind consciousness, and things knowable by mind consciousness in this way: ‘This is not mine, I am not this, this is not my self’.” 

“Reverend\marginnote{10.1} Channa, what have you seen, what have you directly known in these things that you regard them in this way: ‘This is not mine, I am not this, this is not my self’?” 

“Reverend\marginnote{10.7} \textsanskrit{Sāriputta}, after seeing cessation, after directly knowing cessation in these things I regard them in this way: ‘This is not mine, I am not this, this is not my self’.” 

When\marginnote{11.1} he said this, Venerable \textsanskrit{Mahācunda} said to Venerable Channa: 

“So,\marginnote{11.2} Reverend Channa, you should pay close attention to this instruction of the Buddha whenever you can: ‘For the dependent there is agitation. For the independent there’s no agitation. When there’s no agitation there is tranquility. When there is tranquility there’s no inclination. When there’s no inclination there’s no coming and going. When there’s no coming and going there’s no passing away and reappearing. When there’s no passing away and reappearing there’s no this world or world beyond or between the two. Just this is the end of suffering.’” And when the venerables \textsanskrit{Sāriputta} and \textsanskrit{Mahācunda} had given Venerable Channa this advice they got up from their seat and left. 

Not\marginnote{12.1} long after those venerables had left, Venerable Channa slit his wrists. 

Then\marginnote{13.1} \textsanskrit{Sāriputta} went up to the Buddha, bowed, sat down to one side, and said to him, “Sir, Venerable Channa has slit his wrists. Where has he been reborn in his next life?” 

“\textsanskrit{Sāriputta},\marginnote{13.4} didn’t the mendicant Channa declare his blamelessness to you personally?” 

“Sir,\marginnote{13.5} there is a Vajjian village named Pubbavijjhana where Channa had families with whom he was friendly, intimate, and familiar.” 

“The\marginnote{13.7} mendicant Channa did indeed have such families. But this is not enough for me to call someone ‘blameworthy’. When someone lays down this body and takes up another body, I call them ‘blameworthy’. But the mendicant Channa did no such thing. You should remember this: ‘The mendicant Channa slit his wrists blamelessly.’” 

That\marginnote{13.12} is what the Buddha said. Satisfied, Venerable \textsanskrit{Sāriputta} was happy with what the Buddha said. 

%
\section*{{\suttatitleacronym MN 145}{\suttatitletranslation Advice to Puṇṇa }{\suttatitleroot Puṇṇovādasutta}}
\addcontentsline{toc}{section}{\tocacronym{MN 145} \toctranslation{Advice to Puṇṇa } \tocroot{Puṇṇovādasutta}}
\markboth{Advice to Puṇṇa }{Puṇṇovādasutta}
\extramarks{MN 145}{MN 145}

\scevam{So\marginnote{1.1} I have heard. }At one time the Buddha was staying near \textsanskrit{Sāvatthī} in Jeta’s Grove, \textsanskrit{Anāthapiṇḍika}’s monastery. 

Then\marginnote{1.3} in the late afternoon, Venerable \textsanskrit{Puṇṇa} came out of retreat and went to the Buddha. He bowed, sat down to one side, and said to the Buddha, “Sir, may the Buddha please teach me Dhamma in brief. When I’ve heard it, I’ll live alone, withdrawn, diligent, keen, and resolute.” 

“Well\marginnote{2.2} then, \textsanskrit{Puṇṇa}, listen and pay close attention, I will speak.” 

“Yes,\marginnote{2.3} sir,” replied \textsanskrit{Puṇṇa}. The Buddha said this: 

“\textsanskrit{Puṇṇa},\marginnote{3.1} there are sights known by the eye that are likable, desirable, agreeable, pleasant, sensual, and arousing. If a mendicant approves, welcomes, and keeps clinging to them, this gives rise to relishing. Relishing is the origin of suffering, I say. 

There\marginnote{3.5} are sounds known by the ear … smells known by the nose … tastes known by the tongue … touches known by the body … thoughts known by the mind that are likable, desirable, agreeable, pleasant, sensual, and arousing. If a mendicant approves, welcomes, and keeps clinging to them, this gives rise to relishing. Relishing is the origin of suffering, I say. 

There\marginnote{4.1} are sights known by the eye that are likable, desirable, agreeable, pleasant, sensual, and arousing. If a mendicant doesn’t approve, welcome, and keep clinging to them, relishing ceases. When relishing ceases, suffering ceases, I say. 

There\marginnote{4.5} are sounds known by the ear … smells known by the nose … tastes known by the tongue … touches known by the body … thoughts known by the mind that are likable, desirable, agreeable, pleasant, sensual, and arousing. If a mendicant doesn’t approve, welcome, and keep clinging to them, relishing ceases. When relishing ceases, suffering ceases, I say. 

\textsanskrit{Puṇṇa},\marginnote{5.1} now that I’ve given you this brief advice, what country will you live in?” 

“Sir,\marginnote{5.2} there’s a country named \textsanskrit{Sunāparanta}. I shall live there.” 

“The\marginnote{5.3} people of \textsanskrit{Sunāparanta} are wild and rough, \textsanskrit{Puṇṇa}. If they abuse and insult you, what will you think of them?” 

“If\marginnote{5.6} they abuse and insult me, I will think: ‘These people of \textsanskrit{Sunāparanta} are gracious, truly gracious, since they don’t hit me with their fists.’ That’s what I’ll think, Blessed One. That’s what I’ll think, Holy One.” 

“But\marginnote{5.10} if they do hit you with their fists, what will you think of them then?” 

“If\marginnote{5.11} they hit me with their fists, I’ll think: ‘These people of \textsanskrit{Sunāparanta} are gracious, truly gracious, since they don’t throw stones at me.’ That’s what I’ll think, Blessed One. That’s what I’ll think, Holy One.” 

“But\marginnote{5.15} if they do throw stones at you, what will you think of them then?” 

“If\marginnote{5.16} they throw stones at me, I’ll think: ‘These people of \textsanskrit{Sunāparanta} are gracious, truly gracious, since they don’t beat me with a club.’ That’s what I’ll think, Blessed One. That’s what I’ll think, Holy One.” 

“But\marginnote{5.20} if they do beat you with a club, what will you think of them then?” 

“If\marginnote{5.21} they beat me with a club, I’ll think: ‘These people of \textsanskrit{Sunāparanta} are gracious, truly gracious, since they don’t stab me with a knife.’ That’s what I’ll think, Blessed One. That’s what I’ll think, Holy One.” 

“But\marginnote{5.25} if they do stab you with a knife, what will you think of them then?” 

“If\marginnote{5.26} they stab me with a knife, I’ll think: ‘These people of \textsanskrit{Sunāparanta} are gracious, truly gracious, since they don’t take my life with a sharp knife.’ That’s what I’ll think, Blessed One. That’s what I’ll think, Holy One.” 

“But\marginnote{5.30} if they do take your life with a sharp knife, what will you think of them then?” 

“If\marginnote{5.31} they take my life with a sharp knife, I’ll think: ‘There are disciples of the Buddha who looked for someone to assist with slitting their wrists because they were horrified, repelled, and disgusted with the body and with life. And I have found this without looking!’ That’s what I’ll think, Blessed One. That’s what I’ll think, Holy One.” 

“Good,\marginnote{6.1} good \textsanskrit{Puṇṇa}! Having such self-control and peacefulness, you will be quite capable of living in \textsanskrit{Sunāparanta}. Now, \textsanskrit{Puṇṇa}, go at your convenience.” 

And\marginnote{7.1} then \textsanskrit{Puṇṇa} welcomed and agreed with the Buddha’s words. He got up from his seat, bowed, and respectfully circled the Buddha, keeping him on his right. Then he set his lodgings in order and, taking his bowl and robe, set out for \textsanskrit{Sunāparanta}. Traveling stage by stage, he arrived at \textsanskrit{Sunāparanta}, and stayed there. Within that rainy season he confirmed around five hundred male and five hundred female lay followers. And within that same rainy season he realized the three knowledges. Some time later he became fully extinguished. 

Then\marginnote{8.1} several mendicants went up to the Buddha, bowed, sat down to one side, and said to him, “Sir, the gentleman named \textsanskrit{Puṇṇa}, who was advised in brief by the Buddha, has passed away. Where has he been reborn in his next life?” 

“Mendicants,\marginnote{8.4} \textsanskrit{Puṇṇa} was astute. He practiced in line with the teachings, and did not trouble me about the teachings. \textsanskrit{Puṇṇa} has become completely extinguished.” 

That\marginnote{8.6} is what the Buddha said. Satisfied, the mendicants were happy with what the Buddha said. 

%
\section*{{\suttatitleacronym MN 146}{\suttatitletranslation Advice from Nandaka }{\suttatitleroot Nandakovādasutta}}
\addcontentsline{toc}{section}{\tocacronym{MN 146} \toctranslation{Advice from Nandaka } \tocroot{Nandakovādasutta}}
\markboth{Advice from Nandaka }{Nandakovādasutta}
\extramarks{MN 146}{MN 146}

\scevam{So\marginnote{1.1} I have heard. }At one time the Buddha was staying near \textsanskrit{Sāvatthī} in Jeta’s Grove, \textsanskrit{Anāthapiṇḍika}’s monastery. 

Then\marginnote{2.1} \textsanskrit{Mahāpajāpatī} \textsanskrit{Gotamī} together with around five hundred nuns approached the Buddha, bowed, stood to one side, and said to him, “Sir, may the Buddha please advise and instruct the nuns. Please give the nuns a Dhamma talk.” 

Now\marginnote{3.1} at that time the senior monks were taking turns to advise the nuns. But Venerable Nandaka didn’t want to take his turn. 

Then\marginnote{3.3} the Buddha said to Venerable Ānanda, “Ānanda, whose turn is it to advise the nuns today?” 

“It’s\marginnote{3.5} Nandaka’s turn, sir, but he doesn’t want to do it.” 

Then\marginnote{4.1} the Buddha said to Nandaka, “Nandaka, please advise and instruct the nuns. Please, brahmin, give the nuns a Dhamma talk.” 

“Yes,\marginnote{4.5} sir,” replied Nandaka. Then, in the morning, he robed up and, taking his bowl and robe, entered \textsanskrit{Sāvatthī} for alms. He wandered for alms in \textsanskrit{Sāvatthī}. After the meal, on his return from almsround, he went to the Royal Monastery with a companion. Those nuns saw him coming off in the distance, so they spread out a seat and placed water for washing the feet. Nandaka sat down on the seat spread out, and washed his feet. Those nuns bowed, and sat down to one side. 

Nandaka\marginnote{4.12} said to them, “Sisters, this talk shall be in the form of questions. When you understand, say so. When you don’t understand, say so. If anyone has a doubt or uncertainty, ask me about it: ‘Why, sir, does it say this? What does that mean?’” 

“We’re\marginnote{5.5} already delighted and satisfied with Venerable Nandaka, since he invites us like this.” 

“What\marginnote{6.1} do you think, sisters? Is the eye permanent or impermanent?” 

“Impermanent,\marginnote{6.3} sir.” 

“But\marginnote{6.4} if it’s impermanent, is it suffering or happiness?” 

“Suffering,\marginnote{6.5} sir.” 

“But\marginnote{6.6} if it’s impermanent, suffering, and perishable, is it fit to be regarded thus: ‘This is mine, I am this, this is my self’?” 

“No,\marginnote{6.8} sir.” 

“What\marginnote{6.9} do you think, sisters? Is the ear … nose … tongue … body … mind permanent or impermanent?” 

“Impermanent,\marginnote{6.19} sir.” 

“But\marginnote{6.20} if it’s impermanent, is it suffering or happiness?” 

“Suffering,\marginnote{6.21} sir.” 

“But\marginnote{6.22} if it’s impermanent, suffering, and perishable, is it fit to be regarded thus: ‘This is mine, I am this, this is my self’?” 

“No,\marginnote{6.24} sir. Why is that? Because we have already truly seen this with right wisdom: ‘So these six interior sense fields are impermanent.’” 

“Good,\marginnote{6.28} good, sisters! That’s how it is for a noble disciple who truly sees with right wisdom. 

What\marginnote{7.1} do you think, sisters? Are sights permanent or impermanent?” 

“Impermanent,\marginnote{7.3} sir.” 

“But\marginnote{7.4} if they're impermanent, are they suffering or happiness?” 

“Suffering,\marginnote{7.5} sir.” 

“But\marginnote{7.6} if they're impermanent, suffering, and perishable, are they fit to be regarded thus: ‘This is mine, I am this, this is my self’?” 

“No,\marginnote{7.8} sir.” 

“What\marginnote{7.9} do you think, sisters? Are sounds … smells … tastes … touches … thoughts permanent or impermanent?” 

“Impermanent,\marginnote{7.19} sir.” 

“But\marginnote{7.20} if they're impermanent, are they suffering or happiness?” 

“Suffering,\marginnote{7.21} sir.” 

“But\marginnote{7.22} if they're impermanent, suffering, and perishable, are they fit to be regarded thus: ‘This is mine, I am this, this is my self’?” 

“No,\marginnote{7.24} sir. Why is that? Because we have already truly seen this with right wisdom: ‘So these six exterior sense fields are impermanent.’” 

“Good,\marginnote{7.28} good, sisters! That’s how it is for a noble disciple who truly sees with right wisdom. 

What\marginnote{8.1} do you think, sisters? Is eye consciousness … ear consciousness … nose consciousness … tongue consciousness … body consciousness … mind consciousness permanent or impermanent?” 

“Impermanent,\marginnote{8.18} sir.” 

“But\marginnote{8.19} if it’s impermanent, is it suffering or happiness?” 

“Suffering,\marginnote{8.20} sir.” 

“But\marginnote{8.21} if it’s impermanent, suffering, and perishable, is it fit to be regarded thus: ‘This is mine, I am this, this is my self’?” 

“No,\marginnote{8.23} sir. Why is that? Because we have already truly seen this with right wisdom: ‘So these six classes of consciousness are impermanent.’” 

“Good,\marginnote{8.27} good, sisters! That’s how it is for a noble disciple who truly sees with right wisdom. 

Suppose\marginnote{9.1} there was an oil lamp burning. The oil, wick, flame, and light were all impermanent and perishable. Now, suppose someone was to say: ‘While this oil lamp is burning, the oil, the wick, and the flame are all impermanent and perishable. But the light is permanent, lasting, eternal, and imperishable.’ Would they be speaking rightly?” 

“No,\marginnote{9.6} sir. Why is that? Because that oil lamp’s oil, wick, and flame are all impermanent and perishable, let alone the light.” 

“In\marginnote{9.10} the same way, suppose someone was to say: ‘These six interior sense fields are impermanent. But the feeling—whether pleasant, painful, or neutral—that I experience due to these six interior sense fields is permanent, lasting, eternal, and imperishable.’ Would they be speaking rightly?” 

“No,\marginnote{9.14} sir. Why is that? Because each kind of feeling arises dependent on the corresponding condition. When the corresponding condition ceases, the appropriate feeling ceases.” 

“Good,\marginnote{9.18} good, sisters! That’s how it is for a noble disciple who truly sees with right wisdom. 

Suppose\marginnote{10.1} there was a large tree standing with heartwood. The roots, trunk, branches and leaves, and shadow were all impermanent and perishable. Now, suppose someone was to say: ‘There’s a large tree standing with heartwood. The roots, trunk, and branches and leaves are all impermanent and perishable. But the shadow is permanent, lasting, eternal, and imperishable.’ Would they be speaking rightly?” 

“No,\marginnote{10.5} sir. Why is that? Because that large tree’s roots, trunk, and branches and leaves are all impermanent and perishable, let alone the shadow.” 

“In\marginnote{10.9} the same way, suppose someone was to say: ‘These six exterior sense fields are impermanent. But the feeling—whether pleasant, painful, or neutral—that I experience due to these six exterior sense fields is permanent, lasting, eternal, and imperishable.’ Would they be speaking rightly?” 

“No,\marginnote{10.13} sir. Why is that? Because each kind of feeling arises dependent on the corresponding condition. When the corresponding condition ceases, the appropriate feeling ceases.” 

“Good,\marginnote{10.17} good, sisters! That’s how it is for a noble disciple who truly sees with right wisdom. 

Suppose\marginnote{11.1} a deft butcher or their apprentice was to kill a cow and carve it with a sharp meat cleaver. Without damaging the flesh inside or the hide outside, they’d cut, carve, sever, and slice through the connecting tendons, sinews, and ligaments, and then peel off the outer hide. Then they’d wrap that cow up in that very same hide and say: ‘This cow is joined to its hide just like before.’ Would they be speaking rightly?” 

“No,\marginnote{11.6} sir. Why is that? Because even if they wrap that cow up in that very same hide and say: ‘This cow is joined to its hide just like before,’ still that cow is not joined to that hide.” 

“I’ve\marginnote{12.1} made up this simile to make a point. And this is the point. ‘The inner flesh’ is a term for the six interior sense fields. ‘The outer hide’ is a term for the six exterior sense fields. ‘The connecting tendons, sinews, and ligaments’ is a term for greed and relishing. ‘A sharp meat cleaver’ is a term for noble wisdom. And it is that noble wisdom which cuts, carves, severs, and slices the connecting corruption, fetter, and bond. 

Sisters,\marginnote{13.1} by developing and cultivating these seven awakening factors, a mendicant realizes the undefiled freedom of heart and freedom by wisdom in this very life. And they live having realized it with their own insight due to the ending of defilements. What seven? It’s when a mendicant develops the awakening factors of mindfulness, investigation of principles, energy, rapture, tranquility, immersion, and equanimity, which rely on seclusion, fading away, and cessation, and ripen as letting go. It is by developing and cultivating these seven awakening factors that a mendicant realizes the undefiled freedom of heart and freedom by wisdom in this very life. And they live having realized it with their own insight due to the ending of defilements.” 

Then\marginnote{14.1} after giving this advice to the nuns, Nandaka dismissed them, saying, “Go, sisters, it is time.” 

And\marginnote{14.3} then those nuns approved and agreed with what Nandaka had said. They got up from their seat, bowed, and respectfully circled him, keeping him on their right. Then they went up to the Buddha, bowed, and stood to one side. The Buddha said to them, “Go, nuns, it is time.” 

Then\marginnote{14.5} those nuns bowed to the Buddha respectfully circled him, keeping him on their right, before departing. 

Soon\marginnote{15.1} after those nuns had left, the Buddha addressed the mendicants: “Suppose, mendicants, it was the sabbath of the fourteenth day. You wouldn’t get lots of people wondering whether the moon is full or not, since it is obviously not full. 

In\marginnote{15.4} the same way, those nuns were uplifted by Nandaka’s Dhamma teaching, but they still haven’t found what they’re looking for.” 

Then\marginnote{16.1} the Buddha said to Nandaka, “Well then, Nandaka, tomorrow you should give those nuns the same advice again.” 

“Yes,\marginnote{16.3} sir,” Nandaka replied. And the next day he went to those nuns, and all unfolded just like the previous day. 

Soon\marginnote{27.1} after those nuns had left, the Buddha addressed the mendicants: “Suppose, mendicants, it was the sabbath of the fifteenth day. You wouldn’t get lots of people wondering whether the moon is full or not, since it is obviously full. In the same way, those nuns were uplifted by Nandaka’s Dhamma teaching, and they found what they’re looking for. Even the last of these five hundred nuns is a stream-enterer, not liable to be reborn in the underworld, bound for awakening.” 

That\marginnote{27.6} is what the Buddha said. Satisfied, the mendicants were happy with what the Buddha said. 

%
\section*{{\suttatitleacronym MN 147}{\suttatitletranslation The Shorter Advice to Rāhula }{\suttatitleroot Cūḷarāhulovādasutta}}
\addcontentsline{toc}{section}{\tocacronym{MN 147} \toctranslation{The Shorter Advice to Rāhula } \tocroot{Cūḷarāhulovādasutta}}
\markboth{The Shorter Advice to Rāhula }{Cūḷarāhulovādasutta}
\extramarks{MN 147}{MN 147}

\scevam{So\marginnote{1.1} I have heard. }At one time the Buddha was staying near \textsanskrit{Sāvatthī} in Jeta’s Grove, \textsanskrit{Anāthapiṇḍika}’s monastery. 

Then\marginnote{1.3} as he was in private retreat this thought came to his mind, “The qualities that ripen in freedom have ripened in \textsanskrit{Rāhula}. Why don’t I lead him further to the ending of defilements?” 

Then\marginnote{1.6} the Buddha robed up in the morning and, taking his bowl and robe, entered \textsanskrit{Sāvatthī} for alms. 

Then,\marginnote{1.7} after the meal, on his return from almsround, he addressed Venerable \textsanskrit{Rāhula}, “\textsanskrit{Rāhula}, get your sitting cloth. Let’s go to the Dark Forest for the day’s meditation.” 

“Yes,\marginnote{1.10} sir,” replied \textsanskrit{Rāhula}. Taking his sitting cloth he followed behind the Buddha. 

Now\marginnote{2.1} at that time many thousands of deities followed the Buddha, thinking, “Today the Buddha will lead \textsanskrit{Rāhula} further to the ending of defilements!” 

Then\marginnote{2.3} the Buddha plunged deep into the Dark Forest and sat at the root of a tree on the seat spread out. \textsanskrit{Rāhula} bowed to the Buddha and sat down to one side. The Buddha said to him: 

“What\marginnote{2.6} do you think, \textsanskrit{Rāhula}? Is the eye permanent or impermanent?” 

“Impermanent,\marginnote{2.8} sir.” 

“But\marginnote{2.9} if it’s impermanent, is it suffering or happiness?” 

“Suffering,\marginnote{2.10} sir.” 

“But\marginnote{2.11} if it’s impermanent, suffering, and perishable, is it fit to be regarded thus: ‘This is mine, I am this, this is my self’?” 

“No,\marginnote{2.13} sir.” 

“What\marginnote{3.1} do you think, \textsanskrit{Rāhula}? Are sights permanent or impermanent?” 

“Impermanent,\marginnote{3.3} sir.” 

“But\marginnote{3.4} if they're impermanent, are they suffering or happiness?” 

“Suffering,\marginnote{3.5} sir.” 

“But\marginnote{3.6} if they're impermanent, suffering, and perishable, are they fit to be regarded thus: ‘This is mine, I am this, this is my self’?” 

“No,\marginnote{3.8} sir.” 

“What\marginnote{3.9} do you think, \textsanskrit{Rāhula}? Is eye consciousness permanent or impermanent?” 

“Impermanent,\marginnote{3.11} sir.” 

“But\marginnote{3.12} if it’s impermanent, is it suffering or happiness?” 

“Suffering,\marginnote{3.13} sir.” 

“But\marginnote{3.14} if it’s impermanent, suffering, and perishable, is it fit to be regarded thus: ‘This is mine, I am this, this is my self’?” 

“No,\marginnote{3.16} sir.” 

“What\marginnote{3.17} do you think, \textsanskrit{Rāhula}? Is eye contact permanent or impermanent?” 

“Impermanent,\marginnote{3.19} sir.” 

“But\marginnote{3.20} if it’s impermanent, is it suffering or happiness?” 

“Suffering,\marginnote{3.21} sir.” 

“But\marginnote{3.22} if it’s impermanent, suffering, and perishable, is it fit to be regarded thus: ‘This is mine, I am this, this is my self’?” 

“No,\marginnote{3.24} sir.” 

“What\marginnote{3.25} do you think, \textsanskrit{Rāhula}? Anything included in feeling, perception, choices, and consciousness that arises conditioned by eye contact: is that permanent or impermanent?” 

“Impermanent,\marginnote{3.27} sir.” 

“But\marginnote{3.28} if it’s impermanent, is it suffering or happiness?” 

“Suffering,\marginnote{3.29} sir.” 

“But\marginnote{3.30} if it’s impermanent, suffering, and perishable, is it fit to be regarded thus: ‘This is mine, I am this, this is my self’?” 

“No,\marginnote{3.32} sir.” 

“What\marginnote{4{-}8.1} do you think, \textsanskrit{Rāhula}? Is the ear … nose … tongue … body … mind permanent or impermanent?” 

“Impermanent,\marginnote{4{-}8.10} sir.” 

“But\marginnote{4{-}8.11} if it’s impermanent, is it suffering or happiness?” 

“Suffering,\marginnote{4{-}8.12} sir.” 

“But\marginnote{4{-}8.13} if it’s impermanent, suffering, and perishable, is it fit to be regarded thus: ‘This is mine, I am this, this is my self’?” 

“No,\marginnote{4{-}8.15} sir.” 

“What\marginnote{4{-}8.16} do you think, \textsanskrit{Rāhula}? Are thoughts permanent or impermanent?” 

“Impermanent,\marginnote{4{-}8.17} sir.” 

“But\marginnote{4{-}8.18} if they're impermanent, are they suffering or happiness?” 

“Suffering,\marginnote{4{-}8.19} sir.” 

“But\marginnote{4{-}8.20} if they're impermanent, suffering, and perishable, are they fit to be regarded thus: ‘This is mine, I am this, this is my self’?” 

“No,\marginnote{4{-}8.22} sir.” 

“What\marginnote{4{-}8.23} do you think, \textsanskrit{Rāhula}? Is mind consciousness permanent or impermanent?” 

“Impermanent,\marginnote{4{-}8.24} sir.” 

“But\marginnote{4{-}8.25} if it’s impermanent, is it suffering or happiness?” 

“Suffering,\marginnote{4{-}8.26} sir.” 

“But\marginnote{4{-}8.27} if it’s impermanent, suffering, and perishable, is it fit to be regarded thus: ‘This is mine, I am this, this is my self’?” 

“No,\marginnote{4{-}8.29} sir.” 

“What\marginnote{4{-}8.30} do you think, \textsanskrit{Rāhula}? Is mind contact permanent or impermanent?” 

“Impermanent,\marginnote{4{-}8.31} sir.” 

“But\marginnote{4{-}8.32} if it’s impermanent, is it suffering or happiness?” 

“Suffering,\marginnote{4{-}8.33} sir.” 

“But\marginnote{4{-}8.34} if it’s impermanent, suffering, and perishable, is it fit to be regarded thus: ‘This is mine, I am this, this is my self’?” 

“No,\marginnote{4{-}8.36} sir.” 

“What\marginnote{4{-}8.37} do you think, \textsanskrit{Rāhula}? Anything included in feeling, perception, choices, and consciousness that arises conditioned by mind contact: is that permanent or impermanent?” 

“Impermanent,\marginnote{4{-}8.39} sir.” 

“But\marginnote{4{-}8.40} if it’s impermanent, is it suffering or happiness?” 

“Suffering,\marginnote{4{-}8.41} sir.” 

“But\marginnote{4{-}8.42} if it’s impermanent, suffering, and perishable, is it fit to be regarded thus: ‘This is mine, I am this, this is my self’?” 

“No,\marginnote{4{-}8.44} sir.” 

“Seeing\marginnote{9.1} this, a learned noble disciple grows disillusioned with the eye, sights, eye consciousness, and eye contact. And they grow disillusioned with anything included in feeling, perception, choices, and consciousness that arises conditioned by eye contact. They grow disillusioned with the ear … nose … tongue … body … mind, thoughts, mind consciousness, and mind contact. And they grow disillusioned with anything included in feeling, perception, choices, and consciousness that arises conditioned by mind contact. Being disillusioned, desire fades away. When desire fades away they’re freed. When they’re freed, they know they’re freed. 

They\marginnote{9.8} understand: ‘Rebirth is ended, the spiritual journey has been completed, what had to be done has been done, there is no return to any state of existence.’” 

That\marginnote{9.9} is what the Buddha said. Satisfied, Venerable \textsanskrit{Rāhula} was happy with what the Buddha said. And while this discourse was being spoken, \textsanskrit{Rāhula}’s mind was freed from defilements by not grasping. 

And\marginnote{9.12} the stainless, immaculate vision of the Dhamma arose in those thousands of deities: “Everything that has a beginning has an end.” 

%
\section*{{\suttatitleacronym MN 148}{\suttatitletranslation Six By Six }{\suttatitleroot Chachakkasutta}}
\addcontentsline{toc}{section}{\tocacronym{MN 148} \toctranslation{Six By Six } \tocroot{Chachakkasutta}}
\markboth{Six By Six }{Chachakkasutta}
\extramarks{MN 148}{MN 148}

\scevam{So\marginnote{1.1} I have heard. }At one time the Buddha was staying near \textsanskrit{Sāvatthī} in Jeta’s Grove, \textsanskrit{Anāthapiṇḍika}’s monastery. There the Buddha addressed the mendicants, “Mendicants!” 

“Venerable\marginnote{1.5} sir,” they replied. The Buddha said this: 

“Mendicants,\marginnote{2.1} I shall teach you the Dhamma that’s good in the beginning, good in the middle, and good in the end, meaningful and well-phrased. And I shall reveal a spiritual practice that’s entirely full and pure, namely, the six sets of six. Listen and pay close attention, I will speak.” 

“Yes,\marginnote{2.4} sir,” they replied. The Buddha said this: 

“The\marginnote{3.1} six interior sense fields should be understood. The six exterior sense fields should be understood. The six classes of consciousness should be understood. The six classes of contact should be understood. The six classes of feeling should be understood. The six classes of craving should be understood. 

‘The\marginnote{4.1} six interior sense fields should be understood.’ That’s what I said, but why did I say it? There are the sense fields of the eye, ear, nose, tongue, body, and mind. ‘The six interior sense fields should be understood.’ That’s what I said, and this is why I said it. This is the first set of six. 

‘The\marginnote{5.1} six exterior sense fields should be understood.’ That’s what I said, but why did I say it? There are the sense fields of sights, sounds, smells, tastes, touches, and thoughts. ‘The six exterior sense fields should be understood.’ That’s what I said, and this is why I said it. This is the second set of six. 

‘The\marginnote{6.1} six classes of consciousness should be understood.’ That’s what I said, but why did I say it? Eye consciousness arises dependent on the eye and sights. Ear consciousness arises dependent on the ear and sounds. Nose consciousness arises dependent on the nose and smells. Tongue consciousness arises dependent on the tongue and tastes. Body consciousness arises dependent on the body and touches. Mind consciousness arises dependent on the mind and thoughts. ‘The six classes of consciousness should be understood.’ That’s what I said, and this is why I said it. This is the third set of six. 

‘The\marginnote{7.1} six classes of contact should be understood.’ That’s what I said, but why did I say it? Eye consciousness arises dependent on the eye and sights. The meeting of the three is contact. Ear consciousness arises dependent on the ear and sounds. The meeting of the three is contact. Nose consciousness arises dependent on the nose and smells. The meeting of the three is contact. Tongue consciousness arises dependent on the tongue and tastes. The meeting of the three is contact. Body consciousness arises dependent on the body and touches. The meeting of the three is contact. Mind consciousness arises dependent on the mind and thoughts. The meeting of the three is contact. ‘The six classes of contact should be understood.’ That’s what I said, and this is why I said it. This is the fourth set of six. 

‘The\marginnote{8.1} six classes of feeling should be understood.’ That’s what I said, but why did I say it? Eye consciousness arises dependent on the eye and sights. The meeting of the three is contact. Contact is a condition for feeling. Ear consciousness arises dependent on the ear and sounds. The meeting of the three is contact. Contact is a condition for feeling. Nose consciousness arises dependent on the nose and smells. The meeting of the three is contact. Contact is a condition for feeling. Tongue consciousness arises dependent on the tongue and tastes. The meeting of the three is contact. Contact is a condition for feeling. Body consciousness arises dependent on the body and touches. The meeting of the three is contact. Contact is a condition for feeling. Mind consciousness arises dependent on the mind and thoughts. The meeting of the three is contact. Contact is a condition for feeling. ‘The six classes of feeling should be understood.’ That’s what I said, and this is why I said it. This is the fifth set of six. 

‘The\marginnote{9.1} six classes of craving should be understood.’ That’s what I said, but why did I say it? Eye consciousness arises dependent on the eye and sights. The meeting of the three is contact. Contact is a condition for feeling. Feeling is a condition for craving. Ear consciousness … Nose consciousness … Tongue consciousness … Body consciousness … Mind consciousness arises dependent on the mind and thoughts. The meeting of the three is contact. Contact is a condition for feeling. Feeling is a condition for craving. ‘The six classes of craving should be understood.’ That’s what I said, and this is why I said it. This is the sixth set of six. 

If\marginnote{10.1} anyone says, ‘the eye is self,’ that is not tenable. The arising and vanishing of the eye is evident, so it would follow that one’s self arises and vanishes. That’s why it’s not tenable to claim that the eye is self. So the eye is not self. 

If\marginnote{10.7} anyone says, ‘sights are self,’ that is not tenable. The arising and vanishing of sights is evident, so it would follow that one’s self arises and vanishes. That’s why it’s not tenable to claim that sights are self. So the eye is not self and sights are not self. 

If\marginnote{10.13} anyone says, ‘eye consciousness is self,’ that is not tenable. The arising and vanishing of eye consciousness is evident, so it would follow that one’s self arises and vanishes. That’s why it’s not tenable to claim that eye consciousness is self. So the eye, sights, and eye consciousness are not self. 

If\marginnote{10.19} anyone says, ‘eye contact is self,’ that is not tenable. The arising and vanishing of eye contact is evident, so it would follow that one’s self arises and vanishes. That’s why it’s not tenable to claim that eye contact is self. So the eye, sights, eye consciousness, and eye contact are not self. 

If\marginnote{10.25} anyone says, ‘feeling is self,’ that is not tenable. The arising and vanishing of feeling is evident, so it would follow that one’s self arises and vanishes. That’s why it’s not tenable to claim that feeling is self. So the eye, sights, eye consciousness, eye contact, and feeling are not self. 

If\marginnote{10.31} anyone says, ‘craving is self,’ that is not tenable. The arising and vanishing of craving is evident, so it would follow that one’s self arises and vanishes. That’s why it’s not tenable to claim that craving is self. So the eye, sights, eye consciousness, eye contact, feeling, and craving are not self. 

If\marginnote{11.1} anyone says, ‘the ear is self’ … ‘the nose is self’ … ‘the tongue is self’ … ‘the body is self’ … ‘the mind is self,’ that is not tenable. The arising and vanishing of the mind is evident, so it would follow that one’s self arises and vanishes. That’s why it’s not tenable to claim that the mind is self. So the mind is not self. 

If\marginnote{12.1} anyone says, ‘thoughts are self’ … ‘mind consciousness is self’ … ‘mind contact is self’ … ‘feeling is self’ … ‘craving is self,’ that is not tenable. The arising and vanishing of craving is evident, so it would follow that one’s self arises and vanishes. That’s why it’s not tenable to claim that craving is self. So the mind, thoughts, mind consciousness, mind contact, feeling, and craving are not self. 

Now,\marginnote{16.1} mendicants, this is the way that leads to the origin of identity. You regard the eye like this: ‘This is mine, I am this, this is my self.’ 

You\marginnote{17{-}21.1} regard sights … eye consciousness … eye contact … feeling … craving like this: ‘This is mine, I am this, this is my self.’ You regard the ear … nose … tongue … body … mind … thoughts … mind consciousness … mind contact … feeling … craving like this: ‘This is mine, I am this, this is my self.’ 

But\marginnote{22.1} this is the way that leads to the cessation of identity. You regard the eye like this: ‘This is not mine, I am not this, this is not my self.’ 

You\marginnote{23{-}27.1} regard sights … eye consciousness … eye contact … feeling … craving like this: ‘This is not mine, I am not this, this is not my self.’ You regard the ear … nose … tongue … body … mind like this: ‘This is not mine, I am not this, this is not my self.’ You regard thoughts … mind consciousness … mind contact … feeling … craving like this: ‘This is not mine, I am not this, this is not my self.’ 

Eye\marginnote{28.1} consciousness arises dependent on the eye and sights. The meeting of the three is contact. Contact is a condition for the arising of what is felt as pleasant, painful, or neutral. When you experience a pleasant feeling, if you approve, welcome, and keep clinging to it, the underlying tendency to greed underlies that. When you experience a painful feeling, if you sorrow and wail and lament, beating your breast and falling into confusion, the underlying tendency to repulsion underlies that. When you experience a neutral feeling, if you don’t truly understand that feeling’s origin, ending, gratification, drawback, and escape, the underlying tendency to ignorance underlies that. Mendicants, without giving up the underlying tendency to greed for pleasant feeling, without dispelling the underlying tendency to repulsion towards painful feeling, without eradicating ignorance in the case of neutral feeling, without giving up ignorance and without giving rise to knowledge, it’s simply impossible to make an end of suffering in the present life. 

Ear\marginnote{29{-}33.1} consciousness … Nose consciousness … Tongue consciousness … Body consciousness … Mind consciousness arises dependent on the mind and thoughts. The meeting of the three is contact. Contact is a condition for the arising of what is felt as pleasant, painful, or neutral. When you experience a pleasant feeling, if you approve, welcome, and keep clinging to it, the underlying tendency to greed underlies that. When you experience a painful feeling, if you sorrow and wail and lament, beating your breast and falling into confusion, the underlying tendency to repulsion underlies that. When you experience a neutral feeling, if you don’t truly understand that feeling’s origin, ending, gratification, drawback, and escape, the underlying tendency to ignorance underlies that. Mendicants, without giving up the underlying tendency to greed for pleasant feeling, without dispelling the underlying tendency to repulsion towards painful feeling, without eradicating ignorance in the case of neutral feeling, without giving up ignorance and without giving rise to knowledge, it’s simply impossible to make an end of suffering in the present life. 

Eye\marginnote{34.1} consciousness arises dependent on the eye and sights. The meeting of the three is contact. Contact is a condition for the arising of what is felt as pleasant, painful, or neutral. When you experience a pleasant feeling, if you don’t approve, welcome, and keep clinging to it, the underlying tendency to greed does not underlie that. When you experience a painful feeling, if you don’t sorrow or wail or lament, beating your breast and falling into confusion, the underlying tendency to repulsion does not underlie that. When you experience a neutral feeling, if you truly understand that feeling’s origin, ending, gratification, drawback, and escape, the underlying tendency to ignorance does not underlie that. Mendicants, after giving up the underlying tendency to greed for pleasant feeling, after dispelling the underlying tendency to repulsion towards painful feeling, after eradicating ignorance in the case of neutral feeling, after giving up ignorance and giving rise to knowledge, it’s totally possible to make an end of suffering in the present life. 

Ear\marginnote{35.1} consciousness … 

Nose\marginnote{36.1} consciousness … 

Tongue\marginnote{37.1} consciousness … 

Body\marginnote{38.1} consciousness … 

Mind\marginnote{39.1} consciousness arises dependent on the mind and thoughts. The meeting of the three is contact. Contact is a condition for what is felt as pleasant, painful, or neutral. When you experience a pleasant feeling, if you don’t approve, welcome, and keep clinging to it, the underlying tendency to greed does not underlie that. When you experience a painful feeling, if you don’t sorrow or wail or lament, beating your breast and falling into confusion, the underlying tendency to repulsion does not underlie that. When you experience a neutral feeling, if you truly understand that feeling’s origin, ending, gratification, drawback, and escape, the underlying tendency to ignorance does not underlie that. Mendicants, after giving up the underlying tendency to greed for pleasant feeling, after dispelling the underlying tendency to repulsion towards painful feeling, after eradicating ignorance in the case of neutral feeling, after giving up ignorance and giving rise to knowledge, it’s totally possible to make an end of suffering in the present life. 

Seeing\marginnote{40.1} this, a learned noble disciple grows disillusioned with the eye, sights, eye consciousness, eye contact, feeling, and craving. 

They\marginnote{41.1} grow disillusioned with the ear … nose … tongue … body … mind, thoughts, mind consciousness, mind contact, feeling, and craving. Being disillusioned, desire fades away. When desire fades away they’re freed. When it is freed, they know it is freed. 

They\marginnote{41.7} understand: ‘Rebirth is ended, the spiritual journey has been completed, what had to be done has been done, there is no return to any state of existence.’” 

That\marginnote{41.8} is what the Buddha said. Satisfied, the mendicants were happy with what the Buddha said. And while this discourse was being spoken, the minds of sixty mendicants were freed from defilements by not grasping. 

%
\section*{{\suttatitleacronym MN 149}{\suttatitletranslation The Great Discourse on the Six Sense Fields }{\suttatitleroot Mahāsaḷāyatanikasutta}}
\addcontentsline{toc}{section}{\tocacronym{MN 149} \toctranslation{The Great Discourse on the Six Sense Fields } \tocroot{Mahāsaḷāyatanikasutta}}
\markboth{The Great Discourse on the Six Sense Fields }{Mahāsaḷāyatanikasutta}
\extramarks{MN 149}{MN 149}

\scevam{So\marginnote{1.1} I have heard. }At one time the Buddha was staying near \textsanskrit{Sāvatthī} in Jeta’s Grove, \textsanskrit{Anāthapiṇḍika}’s monastery. There the Buddha addressed the mendicants, “Mendicants!” 

“Venerable\marginnote{1.5} sir,” they replied. The Buddha said this: 

“Mendicants,\marginnote{2.1} I shall teach you the great discourse on the six sense fields. Listen and pay close attention, I will speak.” 

“Yes,\marginnote{2.3} sir,” they replied. The Buddha said this: 

“Mendicants,\marginnote{3.1} when you don’t truly know and see the eye, sights, eye consciousness, eye contact, and what is felt as pleasant, painful, or neutral that arises conditioned by eye contact, you’re aroused by these things. 

Someone\marginnote{3.2} who lives aroused like this—fettered, confused, concentrating on gratification—accumulates the five grasping aggregates for themselves in the future. And their craving—which leads to future lives, mixed up with relishing and greed, chasing pleasure in various realms—grows. Their physical and mental stress, torment, and fever grow. And they experience physical and mental suffering. 

When\marginnote{4{-}7.1} you don’t truly know and see the ear … nose … tongue … body … mind, thoughts, mind consciousness, mind contact, and what is felt as pleasant, painful, or neutral that arises conditioned by mind contact, you’re aroused by these things. 

Someone\marginnote{8.1} who lives aroused like this—fettered, confused, concentrating on gratification—accumulates the five grasping aggregates for themselves in the future. And their craving—which leads to future lives, mixed up with relishing and greed, chasing pleasure in various realms—grows. Their physical and mental stress, torment, and fever grow. And they experience physical and mental suffering. 

When\marginnote{9.1} you do truly know and see the eye, sights, eye consciousness, eye contact, and what is felt as pleasant, painful, or neutral that arises conditioned by eye contact, you’re not aroused by these things. 

Someone\marginnote{9.2} who lives unaroused like this—unfettered, unconfused, concentrating on drawbacks—disperses the the five grasping aggregates for themselves in the future. And their craving—which leads to future lives, mixed up with relishing and greed, chasing pleasure in various realms—is given up. Their physical and mental stress, torment, and fever are given up. And they experience physical and mental pleasure. 

The\marginnote{10.1} view of such a person is right view. Their intention is right intention, their effort is right effort, their mindfulness is right mindfulness, and their immersion is right immersion. And their actions of body and speech have already been fully purified before. So this noble eightfold path is fully developed. 

When\marginnote{10.8} the noble eightfold path is developed, the following are fully developed: the four kinds of mindfulness meditation, the four right efforts, the four bases of psychic power, the five faculties, the five powers, and the seven awakening factors. 

And\marginnote{10.9} these two qualities proceed in conjunction: serenity and discernment. They completely understand by direct knowledge those things that should be completely understood by direct knowledge. They give up by direct knowledge those things that should be given up by direct knowledge. They develop by direct knowledge those things that should be developed by direct knowledge. They realize by direct knowledge those things that should be realized by direct knowledge. 

And\marginnote{11.1} what are the things that should be completely understood by direct knowledge? You should say: ‘The five grasping aggregates.’ That is: form, feeling, perception, choices, and consciousness. These are the things that should be completely understood by direct knowledge. 

And\marginnote{11.5} what are the things that should be given up by direct knowledge? Ignorance and craving for continued existence. These are the things that should be given up by direct knowledge. 

And\marginnote{11.8} what are the things that should be developed by direct knowledge? Serenity and discernment. These are the things that should be developed by direct knowledge. 

And\marginnote{11.11} what are the things that should be realized by direct knowledge? Knowledge and freedom. These are the things that should be realized by direct knowledge. 

When\marginnote{12{-}18.1} you truly know and see the ear … nose … tongue … body … mind, thoughts, mind consciousness, mind contact, and what is felt as pleasant, painful, or neutral that arises conditioned by mind contact, you are not aroused by these things. … 

These\marginnote{26.1} are the things that should be realized by direct knowledge.” 

That\marginnote{26.4} is what the Buddha said. Satisfied, the mendicants were happy with what the Buddha said. 

%
\section*{{\suttatitleacronym MN 150}{\suttatitletranslation With the People of Nagaravinda }{\suttatitleroot Nagaravindeyyasutta}}
\addcontentsline{toc}{section}{\tocacronym{MN 150} \toctranslation{With the People of Nagaravinda } \tocroot{Nagaravindeyyasutta}}
\markboth{With the People of Nagaravinda }{Nagaravindeyyasutta}
\extramarks{MN 150}{MN 150}

\scevam{So\marginnote{1.1} I have heard. }At one time the Buddha was wandering in the land of the Kosalans together with a large \textsanskrit{Saṅgha} of mendicants when he arrived at a village of the Kosalan brahmins named Nagaravinda. 

The\marginnote{2.1} brahmins and householders of Nagaravinda heard, “It seems the ascetic Gotama—a Sakyan, gone forth from a Sakyan family—while wandering in the land of the Kosalans has arrived at Nagaravinda, together with a large \textsanskrit{Saṅgha} of mendicants. He has this good reputation: ‘That Blessed One is perfected, a fully awakened Buddha, accomplished in knowledge and conduct, holy, knower of the world, supreme guide for those who wish to train, teacher of gods and humans, awakened, blessed.’ He has realized with his own insight this world—with its gods, \textsanskrit{Māras} and \textsanskrit{Brahmās}, this population with its ascetics and brahmins, gods and humans—and he makes it known to others. He teaches Dhamma that’s good in the beginning, good in the middle, and good in the end, meaningful and well-phrased. And he reveals a spiritual practice that’s entirely full and pure. It’s good to see such perfected ones.” 

Then\marginnote{3.1} the brahmins and householders of Nagaravinda went up to the Buddha. Before sitting down to one side, some bowed, some exchanged greetings and polite conversation, some held up their joined palms toward the Buddha, some announced their name and clan, while some kept silent. The Buddha said to them: 

“Householders,\marginnote{4.1} if wanderers who follow another path were to ask you: ‘What kind of ascetic or brahmin doesn’t deserve honor, respect, reverence, and veneration?’ You should answer them: ‘There are ascetics and brahmins who are not free of greed, hate, and delusion for sights known by the eye, who are not peaceful inside, and who conduct themselves badly among the good by way of body, speech, and mind. They don’t deserve honor, respect, reverence, and veneration. Why is that? Because we ourselves are not free of these things, so we do not see that they have any higher good conduct than us. That’s why they don’t deserve honor, respect, reverence, and veneration. There are ascetics and brahmins who are not free of greed, hate, and delusion for sounds known by the ear … smells known by the nose … tastes known by the tongue… touches known by the body … thoughts known by the mind, who are not peaceful inside, and who conduct themselves badly among the good by way of body, speech, and mind. They don’t deserve honor, respect, reverence, and veneration. Why is that? Because we ourselves are not free of these things, so we do not see that they have any higher good conduct than us. That’s why they don’t deserve honor, respect, reverence, and veneration.’ When questioned by wanderers who follow other paths, that’s how you should answer them. 

If\marginnote{5.1} wanderers who follow other paths were to ask you: ‘What kind of ascetic or brahmin deserves honor, respect, reverence, and veneration?’ You should answer them: ‘There are ascetics and brahmins who are free of greed, hate, and delusion for sights known by the eye, who are peaceful inside, and who conduct themselves well by way of body, speech, and mind. They deserve honor, respect, reverence, and veneration. Why is that? Because we ourselves are not free of these things, but we see that they have a higher good conduct than us. That’s why they deserve honor, respect, reverence, and veneration. There are ascetics and brahmins who are free of greed, hate, and delusion for sounds known by the ear … smells known by the nose … tastes known by the tongue … touches known by the body … thoughts known by the mind, who are peaceful inside, and who conduct themselves well by way of body, speech, and mind. They deserve honor, respect, reverence, and veneration. Why is that? Because we ourselves are not free of these things, but we see that they have a higher good conduct than us. That’s why they deserve honor, respect, reverence, and veneration. When questioned by wanderers who follow other paths, that’s how you should answer them. 

If\marginnote{6.1} wanderers who follow other paths were to ask you: ‘But what reasons and evidence do you have regarding those venerables that justifies saying, “Clearly those venerables are free of greed, hate, and delusion, or practicing to be free of them”?’ You should answer them: ‘It’s because those venerables frequent remote lodgings in the wilderness and the forest. In such places there are no sights known by the eye to see and enjoy, there are no sounds known by the ear to hear and enjoy, no odors known by the nose to smell and enjoy, no flavors known by the tongue to taste and enjoy, and no touches known by the body to feel and enjoy. These are the reasons and evidence that you have regarding those venerables that justifies saying, “Clearly those venerables are free of greed, hate, and delusion, or practicing to be free of them”.’ When questioned by wanderers who follow other paths, that’s how you should answer them.” 

When\marginnote{7.1} he had spoken, the brahmins and householders of Nagaravinda said to the Buddha, “Excellent, Master Gotama! Excellent! As if he were righting the overturned, or revealing the hidden, or pointing out the path to the lost, or lighting a lamp in the dark so people with good eyes can see what’s there, Master Gotama has made the Teaching clear in many ways. We go for refuge to Master Gotama, to the teaching, and to the mendicant \textsanskrit{Saṅgha}. From this day forth, may Master Gotama remember us as lay followers who have gone for refuge for life.” 

%
\section*{{\suttatitleacronym MN 151}{\suttatitletranslation The Purification of Alms }{\suttatitleroot Piṇḍapātapārisuddhisutta}}
\addcontentsline{toc}{section}{\tocacronym{MN 151} \toctranslation{The Purification of Alms } \tocroot{Piṇḍapātapārisuddhisutta}}
\markboth{The Purification of Alms }{Piṇḍapātapārisuddhisutta}
\extramarks{MN 151}{MN 151}

\scevam{So\marginnote{1.1} I have heard. }At one time the Buddha was staying near \textsanskrit{Rājagaha}, in the Bamboo Grove, the squirrels’ feeding ground. 

Then\marginnote{1.3} in the late afternoon, \textsanskrit{Sāriputta} came out of retreat and went to the Buddha. He bowed and sat down to one side. The Buddha said to him, “\textsanskrit{Sāriputta}, your faculties are so very clear, and your complexion is pure and bright. What kind of meditation are you usually practicing these days?” 

“Sir,\marginnote{2.3} these days I usually practice the meditation on emptiness.” 

“Good,\marginnote{2.4} good, \textsanskrit{Sāriputta}! It seems you usually practice the meditation of a great man. For emptiness is the meditation of a great man. 

Now,\marginnote{3.1} a mendicant might wish: ‘May I usually practice the meditation on emptiness.’ So they should reflect: ‘Along the path that I went for alms, or in the place I wandered for alms, or along the path that I returned from alms, was there any desire or greed or hate or delusion or repulsion in my heart for sights known by the eye?’ Suppose that, upon checking, a mendicant knows that there was such desire or greed or hate or delusion or repulsion in their heart, they should make an effort to give up those unskillful qualities. But suppose that, upon checking, a mendicant knows that there was no such desire or greed or hate or delusion or repulsion in their heart, they should meditate with rapture and joy, training day and night in skillful qualities. 

Furthermore,\marginnote{4{-}8.1} a mendicant should reflect: ‘Along the path that I went for alms, or in the place I wandered for alms, or along the path that I returned from alms, was there any desire or greed or hate or delusion or repulsion in my heart for sounds known by the ear … smells known by the nose … tastes known by the tongue … touches known by the body … thoughts known by the mind?’ Suppose that, upon checking, a mendicant knows that there was such desire or greed or hate or delusion or repulsion in their heart, they should make an effort to give up those unskillful qualities. But suppose that, upon checking, a mendicant knows that there was no such desire or greed or hate or delusion or repulsion in their heart, they should meditate with rapture and joy, training day and night in skillful qualities. 

Furthermore,\marginnote{9.1} a mendicant should reflect: ‘Have I given up the five kinds of sensual stimulation?’ Suppose that, upon checking, a mendicant knows that they have not given them up, they should make an effort to do so. But suppose that, upon checking, a mendicant knows that they have given them up, they should meditate with rapture and joy, training day and night in skillful qualities. 

Furthermore,\marginnote{10.1} a mendicant should reflect: ‘Have I given up the five hindrances?’ Suppose that, upon checking, a mendicant knows that they have not given them up, they should make an effort to do so. But suppose that, upon checking, a mendicant knows that they have given them up, they should meditate with rapture and joy, training day and night in skillful qualities. 

Furthermore,\marginnote{11.1} a mendicant should reflect: ‘Have I completely understood the five grasping aggregates?’ Suppose that, upon checking, a mendicant knows that they have not completely understood them, they should make an effort to do so. But suppose that, upon checking, a mendicant knows that they have completely understood them, they should meditate with rapture and joy, training day and night in skillful qualities. 

Furthermore,\marginnote{12.1} a mendicant should reflect: ‘Have I developed the four kinds of mindfulness meditation?’ Suppose that, upon checking, a mendicant knows that they haven’t developed them, they should make an effort to do so. But suppose that, upon checking, a mendicant knows that they have developed them, they should meditate with rapture and joy, training day and night in skillful qualities. 

Furthermore,\marginnote{13.1} a mendicant should reflect: ‘Have I developed the four right efforts … the four bases of psychic power … the five faculties … the five powers … the seven awakening factors … the noble eightfold path?’ Suppose that, upon checking, a mendicant knows that they haven’t developed it, they should make an effort to do so. But suppose that, upon checking, a mendicant knows that they have developed it, they should meditate with rapture and joy, training day and night in skillful qualities. 

Furthermore,\marginnote{19.1} a mendicant should reflect: ‘Have I developed serenity and discernment?’ Suppose that, upon checking, a mendicant knows that they haven’t developed them, they should make an effort to do so. But suppose that, upon checking, a mendicant knows that they have developed them, they should meditate with rapture and joy, training day and night in skillful qualities. 

Furthermore,\marginnote{20.1} a mendicant should reflect: ‘Have I realized knowledge and freedom?’ Suppose that, upon checking, a mendicant knows that they haven’t realized them, they should make an effort to do so. But suppose that, upon checking, a mendicant knows that they have realized them, they should meditate with rapture and joy, training day and night in skillful qualities. 

Whether\marginnote{21.1} in the past, future, or present, all those who purify their almsfood do so by continually checking in this way. So, \textsanskrit{Sāriputta}, you should all train like this: ‘We shall purify our almsfood by continually checking.’” 

That\marginnote{21.6} is what the Buddha said. Satisfied, Venerable \textsanskrit{Sāriputta} was happy with what the Buddha said. 

%
\section*{{\suttatitleacronym MN 152}{\suttatitletranslation The Development of the Faculties }{\suttatitleroot Indriyabhāvanāsutta}}
\addcontentsline{toc}{section}{\tocacronym{MN 152} \toctranslation{The Development of the Faculties } \tocroot{Indriyabhāvanāsutta}}
\markboth{The Development of the Faculties }{Indriyabhāvanāsutta}
\extramarks{MN 152}{MN 152}

\scevam{So\marginnote{1.1} I have heard. }At one time the Buddha was staying near \textsanskrit{Kajaṅgalā} in a bamboo grove. 

Then\marginnote{2.1} the brahmin student Uttara, a pupil of the brahmin \textsanskrit{Pārāsariya}, approached the Buddha, and exchanged greetings with him. When the greetings and polite conversation were over, he sat down to one side. The Buddha said to him, “Uttara, does \textsanskrit{Pārāsariya} teach his disciples the development of the faculties?” 

“He\marginnote{2.4} does, Master Gotama.” 

“But\marginnote{2.5} how does he teach it?” 

“Master\marginnote{2.6} Gotama, it’s when the eye sees no sight and the ear hears no sound. That’s how \textsanskrit{Pārāsariya} teaches his disciples the development of the faculties.” 

“In\marginnote{2.8} that case, Uttara, a blind person and a deaf person will have developed faculties according to what \textsanskrit{Pārāsariya} says. For a blind person sees no sight with the eye and a deaf person hears no sound with the ear.” When he said this, Uttara sat silent, embarrassed, shoulders drooping, downcast, depressed, with nothing to say. 

Knowing\marginnote{3.1} this, the Buddha addressed Venerable Ānanda, “Ānanda, the development of the faculties taught by \textsanskrit{Pārāsariya} is quite different from the supreme development of the faculties in the training of the Noble One.” 

“Now\marginnote{3.3} is the time, Blessed One! Now is the time, Holy One. Let the Buddha teach the supreme development of the faculties in the training of the Noble One. The mendicants will listen and remember it.” 

“Well\marginnote{3.5} then, Ānanda, listen and pay close attention, I will speak.” 

“Yes,\marginnote{3.6} sir,” Ānanda replied. The Buddha said this: 

“And\marginnote{4.1} how, Ānanda, is there the supreme development of the faculties in the training of the Noble One? When a mendicant sees a sight with their eyes, liking, disliking, and both liking and disliking come up in them. They understand: ‘Liking, disliking, and both liking and disliking have come up in me. That’s conditioned, coarse, and dependently originated. But this is peaceful and sublime, namely equanimity.’ Then the liking, disliking, and both liking and disliking that came up in them cease, and equanimity becomes stabilized. It’s like how a person with good sight might open their eyes then shut them; or might shut their eyes then open them. Such is the speed, the swiftness, the ease with which any liking, disliking, and both liking and disliking at all that came up in them cease, and equanimity becomes stabilized. In the training of the Noble One this is called the supreme development of the faculties regarding sights known by the eye. 

Furthermore,\marginnote{5.1} when a mendicant hears a sound with their ears, liking, disliking, and both liking and disliking come up in them. They understand: ‘Liking, disliking, and both liking and disliking have come up in me. That’s conditioned, coarse, and dependently originated. But this is peaceful and sublime, namely equanimity.’ Then the liking, disliking, and both liking and disliking that came up in them cease, and equanimity becomes stabilized. It’s like how a strong person can effortlessly snap their fingers. Such is the speed, the swiftness, the ease with which any liking, disliking, and both liking and disliking at all that came up in them cease, and equanimity becomes stabilized. In the training of the Noble One this is called the supreme development of the faculties regarding sounds known by the ear. 

Furthermore,\marginnote{6.1} when a mendicant smells an odor with their nose, liking, disliking, and both liking and disliking come up in them. They understand: ‘Liking, disliking, and both liking and disliking have come up in me. That’s conditioned, coarse, and dependently originated. But this is peaceful and sublime, namely equanimity.’ Then the liking, disliking, and both liking and disliking that came up in them cease, and equanimity becomes stabilized. It’s like how a drop of water would roll off a gently sloping lotus leaf, and would not stay there. Such is the speed, the swiftness, the ease with which any liking, disliking, and both liking and disliking at all that came up in them cease, and equanimity becomes stabilized. In the training of the Noble One this is called the supreme development of the faculties regarding smells known by the nose. 

Furthermore,\marginnote{7.1} when a mendicant tastes a flavor with their tongue, liking, disliking, and both liking and disliking come up in them. They understand: ‘Liking, disliking, and both liking and disliking have come up in me. That’s conditioned, coarse, and dependently originated. But this is peaceful and sublime, namely equanimity.’ Then the liking, disliking, and both liking and disliking that came up in them cease, and equanimity becomes stabilized. It’s like how a strong person who’s formed a glob of spit on the tip of their tongue could easily spit it out. Such is the speed, the swiftness, the ease with which any liking, disliking, and both liking and disliking at all that came up in them cease, and equanimity becomes stabilized. In the training of the Noble One this is called the supreme development of the faculties regarding tastes known by the tongue. 

Furthermore,\marginnote{8.1} when a mendicant feels a touch with their body, liking, disliking, and both liking and disliking come up in them. They understand: ‘Liking, disliking, and both liking and disliking have come up in me. That’s conditioned, coarse, and dependently originated. But this is peaceful and sublime, namely equanimity.’ Then the liking, disliking, and both liking and disliking that came up in them cease, and equanimity becomes stabilized. It’s like how a strong person can extend or contract their arm. Such is the speed, the swiftness, the ease with which any liking, disliking, and both liking and disliking at all that came up in them cease, and equanimity becomes stabilized. In the training of the Noble One this is called the supreme development of the faculties regarding touches known by the body. 

Furthermore,\marginnote{9.1} when a mendicant knows a thought with their mind, liking, disliking, and both liking and disliking come up in them. They understand: ‘Liking, disliking, and both liking and disliking have come up in me. That’s conditioned, coarse, and dependently originated. But this is peaceful and sublime, namely equanimity.’ Then the liking, disliking, and both liking and disliking that came up in them cease, and equanimity becomes stabilized. It’s like how a strong person could let two or three drops of water fall onto an iron cauldron that had been heated all day. The drops would be slow to fall, but they’d quickly dry up and evaporate. Such is the speed, the swiftness, the ease with which any liking, disliking, and both liking and disliking at all that came up in them cease, and equanimity becomes stabilized. In the training of the Noble One this is called the supreme development of the faculties regarding thoughts known by the mind. That’s how there is the supreme development of the faculties in the training of the Noble One. 

And\marginnote{10.1} how are they a practicing trainee? When a mendicant sees a sight with their eyes, liking, disliking, and both liking and disliking come up in them. They are horrified, repelled, and disgusted by that. When they hear a sound with their ears … When they smell an odor with their nose … When they taste a flavor with their tongue … When they feel a touch with their body … When they know a thought with their mind, liking, disliking, and both liking and disliking come up in them. They are horrified, repelled, and disgusted by that. That’s how they are a practicing trainee. 

And\marginnote{11{-}15.1} how are they a noble one with developed faculties? When a mendicant sees a sight with their eyes, liking, disliking, and both liking and disliking come up in them. If they wish: ‘May I meditate perceiving the unrepulsive in the repulsive,’ that’s what they do. If they wish: ‘May I meditate perceiving the repulsive in the unrepulsive,’ that’s what they do. If they wish: ‘May I meditate perceiving the unrepulsive in the repulsive and the unrepulsive,’ that’s what they do. If they wish: ‘May I meditate perceiving the repulsive in the unrepulsive and the repulsive,’ that’s what they do. If they wish: ‘May I meditate staying equanimous, mindful and aware, rejecting both the repulsive and the unrepulsive,’ that’s what they do. 

When\marginnote{16.1} they hear a sound with their ear … When they smell an odor with their nose … When they taste a flavor with their tongue … When they feel a touch with their body … When they know a thought with their mind, liking, disliking, and both liking and disliking come up in them. If they wish: ‘May I meditate perceiving the unrepulsive in the repulsive,’ that’s what they do. If they wish: ‘May I meditate perceiving the repulsive in the unrepulsive,’ that’s what they do. If they wish: ‘May I meditate perceiving the unrepulsive in the repulsive and the unrepulsive,’ that’s what they do. If they wish: ‘May I meditate perceiving the repulsive in the unrepulsive and the repulsive,’ that’s what they do. If they wish: ‘May I meditate staying equanimous, mindful and aware, rejecting both the repulsive and the unrepulsive,’ that’s what they do. That’s how they are a noble one with developed faculties. 

So,\marginnote{17.1} Ānanda, I have taught the supreme development of the faculties in the training of the Noble One, I have taught the practicing trainee, and I have taught the noble one with developed faculties. 

Out\marginnote{18.1} of compassion, I’ve done what a teacher should do who wants what’s best for their disciples. Here are these roots of trees, and here are these empty huts. Practice absorption, Ānanda! Don’t be negligent! Don’t regret it later! This is my instruction to you.” 

That\marginnote{18.3} is what the Buddha said. Satisfied, Venerable Ānanda was happy with what the Buddha said. 

\scendbook{The Middle Discourses is completed. }

%
\backmatter%
\chapter*{Colophon}
\addcontentsline{toc}{chapter}{Colophon}
\markboth{Colophon}{Colophon}

\section*{The Translator}

Bhikkhu Sujato was born as Anthony Aidan Best on 4/11/1966 in Perth, Western Australia. He grew up in the pleasant suburbs of Mt Lawley and Attadale alongside his sister Nicola, who was the good child. His mother, Margaret Lorraine Huntsman née Pinder, said “he’ll either be a priest or a poet”, while his father, Anthony Thomas Best, advised him to “never do anything for money”. He attended Aquinas College, a Catholic school, where he decided to become an atheist. At the University of WA he studied philosophy, aiming to learn what he wanted to do with his life. Finding that what he wanted to do was play guitar, he dropped out. His main band was named Martha’s Vineyard, which achieved modest success in the indie circuit. 

A seemingly random encounter with a roadside joey took him to Thailand, where he entered his first meditation retreat at Wat Ram Poeng, Chieng Mai in 1992. Feeling the call to the Buddha’s path, he took full ordination in Wat Pa Nanachat in 1994, where his teachers were Ajahn Pasanno and Ajahn Jayasaro. In 1997 he returned to Perth to study with Ajahn Brahm at Bodhinyana Monastery. 

He spent several years practicing in seclusion in Malaysia and Thailand before establishing Santi Forest Monastery in Bundanoon, NSW, in 2003. There he was instrumental in supporting the establishment of the Theravada bhikkhuni order in Australia and advocating for women’s rights. He continues to teach in Australia and globally, with a special concern for the moral implications of climate change and other forms of environmental destruction. He has published a series of books of original and groundbreaking research on early Buddhism. 

In 2005 he founded SuttaCentral together with Rod Bucknell and John Kelly. In 2015, seeing the need for a complete, accurate, plain English translation of the Pali texts, he undertook the task, spending nearly three years in isolation on the isle of Qi Mei off the coast of the nation of Taiwan. He completed the four main \textsanskrit{Nikāyas} in 2018, and the early books of the Khuddaka \textsanskrit{Nikāya} were complete by 2021. All this work is dedicated to the public domain and is entirely free of copyright encumbrance. 

In 2019 he returned to Sydney where he established Lokanta Vihara (The Monastery at the End of the World). 

\section*{Creation Process}

Primary source was the digital \textsanskrit{Mahāsaṅgīti} edition of the Pali \textsanskrit{Tipiṭaka}. Translated from the Pali, with reference to several English translations, especially those of Bhikkhu Bodhi.

\section*{The Translation}

This translation was part of a project to translate the four Pali \textsanskrit{Nikāyas} with the following aims: plain, approachable English; consistent terminology; accurate rendition of the Pali; free of copyright. It was made during 2016–2018 while Bhikkhu Sujato was staying in Qimei, Taiwan.

\section*{About SuttaCentral}

SuttaCentral publishes early Buddhist texts. Since 2005 we have provided root texts in Pali, Chinese, Sanskrit, Tibetan, and other languages, parallels between these texts, and translations in many modern languages. We build on the work of generations of scholars, and offer our contribution freely.

SuttaCentral is driven by volunteer contributions, and in addition we employ professional developers. We offer a sponsorship program for high quality translations from the original languages. Financial support for SuttaCentral is handled by the SuttaCentral Development Trust, a charitable trust registered in Australia.

\section*{About Bilara}

“Bilara” means “cat” in Pali, and it is the name of our Computer Assisted Translation (CAT) software. Bilara is a web app that enables translators to translate early Buddhist texts into their own language. These translations are published on SuttaCentral with the root text and translation side by side.

\section*{About SuttaCentral Editions}

The SuttaCentral Editions project makes high quality books from selected Bilara translations. These are published in formats including HTML, EPUB, PDF, and print.

If you want to print any of our Editions, please let us know and we will help prepare a file to your specifications.

%
\end{document}