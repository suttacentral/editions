\documentclass[12pt,openany]{book}%
\usepackage{lastpage}%
%
\usepackage[inner=1in, outer=1in, top=.7in, bottom=1in, papersize={6in,9in}, headheight=13pt]{geometry}
\usepackage{polyglossia}
\usepackage[12pt]{moresize}
\usepackage{soul}%
\usepackage{microtype}
\usepackage{tocbasic}
\usepackage{realscripts}
\usepackage{epigraph}%
\usepackage{setspace}%
\usepackage{sectsty}
\usepackage{fontspec}
\usepackage{marginnote}
\usepackage[bottom]{footmisc}
\usepackage{enumitem}
\usepackage{fancyhdr}
\usepackage{extramarks}
\usepackage{graphicx}
\usepackage{verse}
\usepackage{relsize}
\usepackage{etoolbox}
\usepackage[a-3u]{pdfx}

\hypersetup{
colorlinks=true,
urlcolor=black,
linkcolor=black,
citecolor=black
}

% use a small amount of tracking on small caps
\SetTracking[ spacing = {25*,166, } ]{ encoding = *, shape = sc }{ 25 }

% add a blank page
\newcommand{\blankpage}{
\newpage
\thispagestyle{empty}
\mbox{}
\newpage
}

% define languages
\setdefaultlanguage[]{english}
\setotherlanguage[script=Latin]{sanskrit}

%\usepackage{pagegrid}
%\pagegridsetup{top-left, step=.25in}

% define fonts
% use if arno sanskrit is unavailable
%\setmainfont{Gentium Plus}
%\newfontfamily\Semiboldsubheadfont[]{Gentium Plus}
%\newfontfamily\Semiboldnormalfont[]{Gentium Plus}
%\newfontfamily\Lightfont[]{Gentium Plus}
%\newfontfamily\Marginalfont[]{Gentium Plus}
%\newfontfamily\Allsmallcapsfont[RawFeature=+c2sc]{Gentium Plus}
%\newfontfamily\Noligaturefont[Renderer=Basic]{Gentium Plus}
%\newfontfamily\Noligaturecaptionfont[Renderer=Basic]{Gentium Plus}
%\newfontfamily\Fleuronfont[Ornament=1]{Gentium Plus}

% use if arno sanskrit is available. display is applied to \chapter and \part, subhead to \section and \subsection. When specifying semibold, the italic must be defined.
\setmainfont[Numbers=OldStyle]{Arno Pro}
\newfontfamily\Semibolddisplayfont[BoldItalicFont = Arno Pro Semibold Italic Display]{Arno Pro Semibold Display} %
\newfontfamily\Semiboldsubheadfont[BoldItalicFont = Arno Pro Semibold Italic Subhead]{Arno Pro Semibold Subhead}
\newfontfamily\Semiboldnormalfont[BoldItalicFont = Arno Pro Semibold Italic]{Arno Pro Semibold}
\newfontfamily\Marginalfont[RawFeature=+subs]{Arno Pro Regular}
\newfontfamily\Allsmallcapsfont[RawFeature=+c2sc]{Arno Pro}
\newfontfamily\Noligaturefont[Renderer=Basic]{Arno Pro}
\newfontfamily\Noligaturecaptionfont[Renderer=Basic]{Arno Pro Caption}

% chinese fonts
\newfontfamily\cjk{Noto Serif TC}
\newcommand*{\langlzh}[1]{\cjk{#1}\normalfont}%

% logo
\newfontfamily\Logofont{sclogo.ttf}
\newcommand*{\sclogo}[1]{\large\Logofont{#1}}

% use subscript numerals for margin notes
\renewcommand*{\marginfont}{\Marginalfont}

% ensure margin notes have consistent vertical alignment
\renewcommand*{\marginnotevadjust}{-.17em}

% use compact lists
\setitemize{noitemsep,leftmargin=1em}
\setenumerate{noitemsep,leftmargin=1em}
\setdescription{noitemsep, style=unboxed, leftmargin=0em}

% style ToC
\DeclareTOCStyleEntries[
  raggedentrytext,
  linefill=\hfill,
  pagenumberwidth=.5in,
  pagenumberformat=\normalfont,
  entryformat=\normalfont
]{tocline}{chapter,section}


  \setlength\topsep{0pt}%
  \setlength\parskip{0pt}%

% define new \centerpars command for use in ToC. This ensures centering, proper wrapping, and no page break after
\def\startcenter{%
  \par
  \begingroup
  \leftskip=0pt plus 1fil
  \rightskip=\leftskip
  \parindent=0pt
  \parfillskip=0pt
}
\def\stopcenter{%
  \par
  \endgroup
}
\long\def\centerpars#1{\startcenter#1\stopcenter}

% redefine part, so that it adds a toc entry without page number
\let\oldcontentsline\contentsline
\newcommand{\nopagecontentsline}[3]{\oldcontentsline{#1}{#2}{}}

    \makeatletter
\renewcommand*\l@part[2]{%
  \ifnum \c@tocdepth >-2\relax
    \addpenalty{-\@highpenalty}%
    \addvspace{0em \@plus\p@}%
    \setlength\@tempdima{3em}%
    \begingroup
      \parindent \z@ \rightskip \@pnumwidth
      \parfillskip -\@pnumwidth
      {\leavevmode
       \setstretch{.85}\large\scshape\centerpars{#1}\vspace*{-1em}\llap{#2}}\par
       \nobreak
         \global\@nobreaktrue
         \everypar{\global\@nobreakfalse\everypar{}}%
    \endgroup
  \fi}
\makeatother

\makeatletter
\def\@pnumwidth{2em}
\makeatother

% define new sectioning command, which is only used in volumes where the pannasa is found in some parts but not others, especially in an and sn

\newcommand*{\pannasa}[1]{\clearpage\thispagestyle{empty}\begin{center}\vspace*{14em}\setstretch{.85}\huge\itshape\scshape\MakeLowercase{#1}\end{center}}

    \makeatletter
\newcommand*\l@pannasa[2]{%
  \ifnum \c@tocdepth >-2\relax
    \addpenalty{-\@highpenalty}%
    \addvspace{.5em \@plus\p@}%
    \setlength\@tempdima{3em}%
    \begingroup
      \parindent \z@ \rightskip \@pnumwidth
      \parfillskip -\@pnumwidth
      {\leavevmode
       \setstretch{.85}\large\itshape\scshape\lowercase{\centerpars{#1}}\vspace*{-1em}\llap{#2}}\par
       \nobreak
         \global\@nobreaktrue
         \everypar{\global\@nobreakfalse\everypar{}}%
    \endgroup
  \fi}
\makeatother

% don't put page number on first page of toc (relies on etoolbox)
\patchcmd{\chapter}{plain}{empty}{}{}

% global line height
\setstretch{1.05}

% allow linebreak after em-dash
\catcode`\—=13
\protected\def—{\unskip\textemdash\allowbreak}

% style headings with secsty. chapter and section are defined per-edition
\partfont{\setstretch{.85}\normalfont\centering\textsc}
\subsectionfont{\setstretch{.85}\Semiboldsubheadfont}%
\subsubsectionfont{\setstretch{.85}\Semiboldnormalfont}

% style elements of suttatitle
\newcommand*{\suttatitleacronym}[1]{\smaller[2]{#1}\vspace*{.3em}}
\newcommand*{\suttatitletranslation}[1]{\linebreak{#1}}
\newcommand*{\suttatitleroot}[1]{\linebreak\smaller[2]\itshape{#1}}

\DeclareTOCStyleEntries[
  indent=3.3em,
  dynindent,
  beforeskip=.2em plus -2pt minus -1pt,
]{tocline}{section}

\DeclareTOCStyleEntries[
  indent=0em,
  dynindent,
  beforeskip=.4em plus -2pt minus -1pt,
]{tocline}{chapter}

\newcommand*{\tocacronym}[1]{\hspace*{-3.3em}{#1}\quad}
\newcommand*{\toctranslation}[1]{#1}
\newcommand*{\tocroot}[1]{(\textit{#1})}
\newcommand*{\tocchapterline}[1]{\bfseries\itshape{#1}}


% redefine paragraph and subparagraph headings to not be inline
\makeatletter
% Change the style of paragraph headings %
\renewcommand\paragraph{\@startsection{paragraph}{4}{\z@}%
            {-2.5ex\@plus -1ex \@minus -.25ex}%
            {1.25ex \@plus .25ex}%
            {\noindent\Semiboldnormalfont\normalsize}}

% Change the style of subparagraph headings %
\renewcommand\subparagraph{\@startsection{subparagraph}{5}{\z@}%
            {-2.5ex\@plus -1ex \@minus -.25ex}%
            {1.25ex \@plus .25ex}%
            {\noindent\Semiboldnormalfont\small}}
\makeatother

% use etoolbox to suppress page numbers on \part
\patchcmd{\part}{\thispagestyle{plain}}{\thispagestyle{empty}}
  {}{\errmessage{Cannot patch \string\part}}

% and to reduce margins on quotation
\patchcmd{\quotation}{\rightmargin}{\leftmargin 1.2em \rightmargin}{}{}
\AtBeginEnvironment{quotation}{\small}

% titlepage
\newcommand*{\titlepageTranslationTitle}[1]{{\begin{center}\begin{large}{#1}\end{large}\end{center}}}
\newcommand*{\titlepageCreatorName}[1]{{\begin{center}\begin{normalsize}{#1}\end{normalsize}\end{center}}}

% halftitlepage
\newcommand*{\halftitlepageTranslationTitle}[1]{\setstretch{2.5}{\begin{Huge}\uppercase{\so{#1}}\end{Huge}}}
\newcommand*{\halftitlepageTranslationSubtitle}[1]{\setstretch{1.2}{\begin{large}{#1}\end{large}}}
\newcommand*{\halftitlepageFleuron}[1]{{\begin{large}\Fleuronfont{{#1}}\end{large}}}
\newcommand*{\halftitlepageByline}[1]{{\begin{normalsize}\textit{{#1}}\end{normalsize}}}
\newcommand*{\halftitlepageCreatorName}[1]{{\begin{LARGE}{\textsc{#1}}\end{LARGE}}}
\newcommand*{\halftitlepageVolumeNumber}[1]{{\begin{normalsize}{\Allsmallcapsfont{\textsc{#1}}}\end{normalsize}}}
\newcommand*{\halftitlepageVolumeAcronym}[1]{{\begin{normalsize}{#1}\end{normalsize}}}
\newcommand*{\halftitlepageVolumeTranslationTitle}[1]{{\begin{Large}{\textsc{#1}}\end{Large}}}
\newcommand*{\halftitlepageVolumeRootTitle}[1]{{\begin{normalsize}{\Allsmallcapsfont{\textsc{\itshape #1}}}\end{normalsize}}}
\newcommand*{\halftitlepagePublisher}[1]{{\begin{large}{\Noligaturecaptionfont\textsc{#1}}\end{large}}}

% epigraph
\renewcommand{\epigraphflush}{center}
\renewcommand*{\epigraphwidth}{.85\textwidth}
\newcommand*{\epigraphTranslatedTitle}[1]{\vspace*{.5em}\footnotesize\textsc{#1}\\}%
\newcommand*{\epigraphRootTitle}[1]{\footnotesize\textit{#1}\\}%
\newcommand*{\epigraphReference}[1]{\footnotesize{#1}}%

% custom commands for html styling classes
\newcommand*{\scnamo}[1]{\begin{center}\textit{#1}\end{center}}
\newcommand*{\scendsection}[1]{\begin{center}\textit{#1}\end{center}}
\newcommand*{\scendsutta}[1]{\begin{center}\textit{#1}\end{center}}
\newcommand*{\scendbook}[1]{\begin{center}\uppercase{#1}\end{center}}
\newcommand*{\scendkanda}[1]{\begin{center}\textbf{#1}\end{center}}
\newcommand*{\scend}[1]{\begin{center}\textit{#1}\end{center}}
\newcommand*{\scuddanaintro}[1]{\textit{#1}}
\newcommand*{\scendvagga}[1]{\begin{center}\textbf{#1}\end{center}}
\newcommand*{\scrule}[1]{\textbf{#1}}
\newcommand*{\scadd}[1]{\textit{#1}}
\newcommand*{\scevam}[1]{\textsc{#1}}
\newcommand*{\scspeaker}[1]{\hspace{2em}\textit{#1}}
\newcommand*{\scbyline}[1]{\begin{flushright}\textit{#1}\end{flushright}\bigskip}

% custom command for thematic break = hr
\newcommand*{\thematicbreak}{\begin{center}\rule[.5ex]{6em}{.4pt}\begin{normalsize}\quad\Fleuronfont{•}\quad\end{normalsize}\rule[.5ex]{6em}{.4pt}\end{center}}

% manage and style page header and footer. "fancy" has header and footer, "plain" has footer only

\pagestyle{fancy}
\fancyhf{}
\fancyfoot[RE,LO]{\thepage}
\fancyfoot[LE,RO]{\footnotesize\lastleftxmark}
\fancyhead[CE]{\setstretch{.85}\Noligaturefont\MakeLowercase{\textsc{\firstrightmark}}}
\fancyhead[CO]{\setstretch{.85}\Noligaturefont\MakeLowercase{\textsc{\firstleftmark}}}
\renewcommand{\headrulewidth}{0pt}
\fancypagestyle{plain}{ %
\fancyhf{} % remove everything
\fancyfoot[RE,LO]{\thepage}
\fancyfoot[LE,RO]{\footnotesize\lastleftxmark}
\renewcommand{\headrulewidth}{0pt}
\renewcommand{\footrulewidth}{0pt}}

% style footnotes
\setlength{\skip\footins}{1em}

\makeatletter
\newcommand{\@makefntextcustom}[1]{%
    \parindent 0em%
    \thefootnote.\enskip #1%
}
\renewcommand{\@makefntext}[1]{\@makefntextcustom{#1}}
\makeatother

% hang quotes (requires microtype)
\microtypesetup{
  protrusion = true,
  expansion  = true,
  tracking   = true,
  factor     = 1000,
  patch      = all,
  final
}

% Custom protrusion rules to allow hanging punctuation
\SetProtrusion
{ encoding = *}
{
% char   right left
  {-} = {    , 500 },
  % Double Quotes
  \textquotedblleft
      = {1000,     },
  \textquotedblright
      = {    , 1000},
  \quotedblbase
      = {1000,     },
  % Single Quotes
  \textquoteleft
      = {1000,     },
  \textquoteright
      = {    , 1000},
  \quotesinglbase
      = {1000,     }
}

% make latex use actual font em for parindent, not Computer Modern Roman
\AtBeginDocument{\setlength{\parindent}{1em}}%
%

% Default values; a bit sloppier than normal
\tolerance 1414
\hbadness 1414
\emergencystretch 1.5em
\hfuzz 0.3pt
\clubpenalty = 10000
\widowpenalty = 10000
\displaywidowpenalty = 10000
\hfuzz \vfuzz
 \raggedbottom%

\title{Middle Discourses}
\author{Bhikkhu Sujato}
\date{}%
% define a different fleuron for each edition
\newfontfamily\Fleuronfont[Ornament=4]{Arno Pro}

% Define heading styles per edition for chapter, section, and subsection. Suttatitle can be any one of these, depending on the volume. 

\let\oldfrontmatter\frontmatter
\renewcommand{\frontmatter}{%
\chapterfont{\setstretch{.85}\normalfont\centering}%
\sectionfont{\setstretch{.85}\Semiboldsubheadfont}%
\oldfrontmatter}

\let\oldmainmatter\mainmatter
\renewcommand{\mainmatter}{%
\chapterfont{\thispagestyle{empty}\vspace*{4em}\setstretch{.85}\LARGE\normalfont\itshape\scshape\centering\MakeLowercase}
\sectionfont{\clearpage\thispagestyle{plain}\vspace*{2em}\setstretch{.85}\normalfont\centering}%
\oldmainmatter}

\let\oldbackmatter\backmatter
\renewcommand{\backmatter}{%
\chapterfont{\setstretch{.85}\normalfont\centering}%
\sectionfont{\setstretch{.85}\Semiboldsubheadfont}%
\oldbackmatter}
%
%
\begin{document}%
\normalsize%
\frontmatter%
\setlength{\parindent}{0cm}

\pagestyle{empty}

\maketitle

\blankpage%
\begin{center}

\vspace*{2.2em}

\halftitlepageTranslationTitle{Middle Discourses}

\vspace*{1em}

\halftitlepageTranslationSubtitle{A lucid translation of the Majjhima Nikāya}

\vspace*{2em}

\halftitlepageFleuron{•}

\vspace*{2em}

\halftitlepageByline{translated and introduced by}

\vspace*{.5em}

\halftitlepageCreatorName{Bhikkhu Sujato}

\vspace*{4em}

\halftitlepageVolumeNumber{Volume 1}

\smallskip

\halftitlepageVolumeAcronym{MN 1–50}

\smallskip

\halftitlepageVolumeTranslationTitle{The First Fifty}

\smallskip

\halftitlepageVolumeRootTitle{Mūlapaṇṇāsa}

\vspace*{\fill}

\sclogo{0}
 \halftitlepagePublisher{SuttaCentral}

\end{center}

\newpage
%
\setstretch{1.05}

\begin{footnotesize}

\textit{Middle Discourses} is a translation of the Majjhimanikāya by Bhikkhu Sujato.

\medskip

Creative Commons Zero (CC0)

To the extent possible under law, Bhikkhu Sujato has waived all copyright and related or neighboring rights to \textit{Middle Discourses}.

\medskip

This work is published from Australia.

\begin{center}
\textit{This translation is an expression of an ancient spiritual text that has been passed down by the Buddhist tradition for the benefit of all sentient beings. It is dedicated to the public domain via Creative Commons Zero (CC0). You are encouraged to copy, reproduce, adapt, alter, or otherwise make use of this translation. The translator respectfully requests that any use be in accordance with the values and principles of the Buddhist community.}
\end{center}

\medskip

\begin{description}
    \item[Web publication date] 2018
    \item[This edition] 2022-11-30 08:48:21
    \item[Publication type] paperback
    \item[Edition] ed5
    \item[Number of volumes] 3
    \item[Publication ISBN] 978-1-76132-065-1
    \item[Publication URL] https://suttacentral.net/editions/mn/en/sujato
    \item[Source URL] https://github.com/suttacentral/bilara-data/tree/published/translation/en/sujato/sutta/mn
    \item[Publication number] scpub3
\end{description}

\medskip

Published by SuttaCentral

\medskip

\textit{SuttaCentral,\\
c/o Alwis \& Alwis Pty Ltd\\
Kaurna Country,\\
Suite 12,\\
198 Greenhill Road,\\
Eastwood,\\
SA 5063,\\
Australia}

\end{footnotesize}

\newpage

\setlength{\parindent}{1.5em}%%
\newpage

\vspace*{\fill}

\begin{center}
\epigraph{The sage at peace is not reborn, does not grow old, and does not die. They are not shaken, and do not yearn. For they have nothing which would cause them to be reborn. Not being reborn, how could they grow old? Not growing old, how could they die? Not dying, how could they be shaken? Not shaking, for what could they yearn?}
{
\epigraphTranslatedTitle{“The Analysis of the Elements”}
\epigraphRootTitle{\textsanskrit{Dhātuvibhaṅgasutta}}
\epigraphReference{Majjhima \textsanskrit{Nikāya} 140}
}
\end{center}

\vspace*{2in}

\vspace*{\fill}

\blankpage%

\setlength{\parindent}{1em}
%
\tableofcontents
\newpage
\pagestyle{fancy}
%
\chapter*{The SuttaCentral Editions Series}
\addcontentsline{toc}{chapter}{The SuttaCentral Editions Series}
\markboth{The SuttaCentral Editions Series}{The SuttaCentral Editions Series}

Since 2005 SuttaCentral has provided access to the texts, translations, and parallels of early Buddhist texts. In 2018 we started creating and publishing our own translations of these seminal spiritual classics. The “Editions” series now makes selected translations available as books in various forms, including print, PDF, and EPUB.

Editions are selected from our most complete, well-crafted, and reliable translations. They aim to bring these texts to a wider audience in forms that reward mindful reading. Care is taken with every detail of the production, and we aim to meet or exceed professional best standards in every way. These are the core scriptures underlying the entire Buddhist tradition, and we believe that they deserve to be preserved and made available in highest quality without compromise.

SuttaCentral is a charitable organization. Our work is accomplished by volunteers and through the generosity of our donors. Everything we create is offered to all of humanity free of any copyright or licensing restrictions. 

%
\chapter*{Preface to \emph{Middle Discourses}}
\addcontentsline{toc}{chapter}{Preface to \emph{Middle Discourses}}
\markboth{Preface to \emph{Middle Discourses}}{Preface to \emph{Middle Discourses}}

It was in 1992 that I first encountered the Buddha’s words. I was in Chieng Mai, having just completed my very first meditation retreat, a month-long \emph{\textsanskrit{vipassanā}} intensive in the Mahasi style. During the retreat, the guides at Wat Ram Poeng told us not to read any books, but to learn only from our own experience. This was an entirely novel concept for me at the time. The retreat, quite frankly, blew my mind.

When the retreat was over reading was allowed, and as a confirmed  bookworm from childhood, I couldn’t wait to start making sense of all that had happened on my retreat. They gave me some of the then-available books, collections of Dhamma talks and the like. They were fine, but nothing that really sparked my enthusiasm. 

I went back and said, “They talk about these things called ‘the Suttas’. What are they?”

This was the days before the internet and before the excellent editions by Bhikkhu Bodhi. What they had was the selected discourses from the Majjhima \textsanskrit{Nikāya} translated by Bhikkhu \textsanskrit{Ñāṇamoḷi}, edited and compiled by Bhikkhu Khantipalo as \textit{A Treasury of the Buddha’s Words}. Very excited, I took the first volume back to my room and began to read.

Right away I knew. There was something about it. It was not at all what I was expecting. I don’t know what I was expecting, but this wasn’t it. It was so different than my usual reading. At once transparent and obscure; forbidding and intimate. The Buddha’s voice was refreshingly free of the apologetics and mysticism I associated with “spiritual” literature. He was direct, blunt even. He spoke with the quiet authority of someone who actually knew what he was talking about.

I took it slow, reading a Sutta per day. There was much that I didn’t understand, even with the notes and introductions. But I knew that there were great depths there. This was the Buddha! This would be just the start of a long journey. So I read, slowly, going back over it again and again, digesting as much as I could. When I had the time, I’d read an hour a day and meditate a few hours. The teachers at Wat Ram Poeng told us to sit mindfully when learning Dhamma, so I did. I brought as much care and mindful attention to reading as I did to meditation.

Gradually I made my way through the Majjhima \textsanskrit{Nikāya}, learning volumes along the way. I remember some months after that retreat I was staying in Mae Hong Son, at a farm for orphaned children called Buddhakaset. I was there as a volunteer for a while. When the kids had gone to bed, I’d sit in the evening under the shelter of a simple wooden verandah. I’d carefully open my book of Suttas and read the next one, never quite knowing what was to come. There are many things that most Buddhists will encounter and know through Buddhist culture, then rediscover when reading the Suttas. “Ahh, that’s where that came from!” For me it was the opposite: I discovered things in the Suttas, then noticed them in Buddhist culture.

I continued these practices through my early years as a monk. Gradually, I read my way through the Suttas, often in the archaic translations that were then available. But I kept the simple approach I had from the start. Read a bit, meditate a lot. Things that I read then have stayed with me until the present.

I reread the Majjhima \textsanskrit{Nikāya} in other translations. First that by I.B. Horner for the Pali Text Society, which was roughly contemporary to that of \textsanskrit{Ñāṇamoḷi}, but which in publication and language seemed of an earlier generation. Then Bhikkhu Bodhi’s edition came out, a completely revised and seemingly perfected translation based on that of \textsanskrit{Ñāṇamoḷi}. Exciting times! I was beginning my Pali studies, but I never expected to one day be making my own translation.

I hope that you might experience something of the joy and inspiration that I found in my discovery of the Suttas. At this point, having been a monk for twenty-five years, it is redundant to say that these teachings changed my life. They might change yours, too, if you let them.

%
\chapter*{The Middle Discourses: conversations on matters of deep truth}
\addcontentsline{toc}{chapter}{The Middle Discourses: conversations on matters of deep truth}
\markboth{The Middle Discourses: conversations on matters of deep truth}{The Middle Discourses: conversations on matters of deep truth}

\scbyline{Bhikkhu Sujato, 2019}

The Majjhima \textsanskrit{Nikāya} is the second of the four main divisions in the Sutta \textsanskrit{Piṭaka} of the Pali Canon (\textit{\textsanskrit{tipiṭaka}}). It is translated here as \textit{Middle Discourses}, and is sometimes known as the \textit{Middle-Length Discourses}. As the title suggests, its discourses are somewhat shorter than those of the \textsanskrit{Dīgha} \textsanskrit{Nikāya} (\textit{Long Discourses}), but longer than the many short discourses collected in the \textsanskrit{Saṁyutta} \textsanskrit{Nikāya} (\textit{Linked Discourses}) and \textsanskrit{Aṅguttara} \textsanskrit{Nikāya} (\textit{Numbered Discourses}).

In the classic introduction to his translation of the Majjhima \textsanskrit{Nikāya}, Bhikkhu Bodhi described the Majjhima as “the collection that combines the richest variety of contextual settings with the deepest and most comprehensive assortment of teachings”. He went on to situate it among the other \textit{\textsanskrit{nikāyas}}:

\begin{quotation}%
Like the \textsanskrit{Dīgha} \textsanskrit{Nikāya}, the Majjhima is replete with drama and narrative, while lacking much of its predecessor’s tendency towards imaginative embellishment and profusion of legend. Like the \textsanskrit{Saṁyutta}, it contains some of the profoundest discourses in the Canon, disclosing the Buddha’s radical insights into the nature of existence; and like the \textsanskrit{Aṅguttara}, it covers a wide range of topics of practical applicability. In contrast to those two \textsanskrit{Nikāyas}, however, the Majjhima sets forth this material not in the form of short, self-contained utterances, but in the context of a fascinating procession of scenarios that exhibit the Buddha’s resplendence of wisdom, his skill in adapting his teachings to the needs and proclivities of his interlocutors, his wit and gentle humour, his majestic sublimity, and his compassionate humanity.

%
\end{quotation}

In this guide, I describe some of the special features of the Majjhima. After noting some formal and doctrinal features, I focus this essay on the Buddha and the people he encountered, whether his monastic Sangha, his lay followers, or the many non-Buddhists he spoke with. The Majjhima includes many of the most important autobiographical discourses, so it is a natural place to discuss the Buddha’s life and person. And the dialogue format makes the Majjhima a specially rich context to see how the Dhamma emerged through interaction and conversation with people of all kinds.

\section*{How the Majjhima is Organized}

There are 152 discourses in the Majjhima. These are collected into groups of 50 discourses (\textit{\textsanskrit{paṇṇāsa}}), although the final \textit{\textsanskrit{paṇṇāsa}} contains 52.

Within each \textit{\textsanskrit{paṇṇāsa}} is a set of five \textit{vaggas}. As usual, most of the \textit{vaggas} are simply named after their first sutta, but a few exhibit some thematic unity:

\begin{description}%
\item[\textbf{Opamma Vagga (Chapter With Similes)}] All these discourses prominently feature similes.%
\item[\textbf{\textsanskrit{Mahāyamaka} Vagga (Large Chapter With Pairs)}] Paired “long” and “short” discourses. (The following Short Chapter With Pairs only has two sets of pairs.)%
\item[\textbf{Vaggas 6–10}] These collect discourses featuring certain kinds of people: householders, mendicants, wanderers, kings, and brahmins.%
\item[\textbf{\textsanskrit{Vibhaṅga} Vagga (Chapter on Analysis)}] These discourses consist of a lengthy “analysis” of a short saying.%
\item[\textbf{\textsanskrit{Saḷāyatana} Vagga (Chapter on the Six Senses)}] These focus on the six senses.%
\end{description}

\section*{Imagery and Narrative}

The Majjhima includes an astonishing range of imagery, with similes found in almost all discourses. Sometimes these are extended to short parables. \href{https://suttacentral.net/mn21}{MN 21} \textit{The Simile of the Saw} (\textit{\textsanskrit{Kakacūpamasutta}}) tells the memorable story of the bold maid \textsanskrit{Kāḷī} who tested her mistress. In \href{https://suttacentral.net/mn56\#27}{MN 56:27} \textit{With \textsanskrit{Upāli}} we meet the brahmin lady who wanted to not only dye her pet monkey, but press him and wring him out.

While the Buddha himself speaks only short tales, in the background narratives we find more developed narratives, including famous stories such as the taming of the vicious serial-killer \textsanskrit{Aṅgulimāla} (\href{https://suttacentral.net/mn86}{MN 86}) or the uncompromising commitment of the wealthy youth \textsanskrit{Raṭṭhapāla} (\href{https://suttacentral.net/mn82}{MN 82}), who defied his parents’ will to take ordination, and whose discourse finishes with an extraordinary set of teachings for his king.

Some stories are less earth-bound. \href{https://suttacentral.net/mn37}{MN 37} \textit{The Shorter Discourse on the Ending of Craving} (\textit{\textsanskrit{Cūḷataṇhāsaṅkhayasutta}}) depicts \textsanskrit{Moggallāna} ascending to the heaven of Sakka, the lord of gods, to check his indulgence. \href{https://suttacentral.net/mn50}{MN 50} \textit{The Rebuke of \textsanskrit{Māra}} (\textit{\textsanskrit{Māratajjanīyasutta}}) also features \textsanskrit{Moggallāna}, this time in a dialogue with \textsanskrit{Māra}, the lord of deceit and death; and it contains the startling revelation that \textsanskrit{Moggallāna} himself was a \textsanskrit{Māra} in a past life. Such cosmic drama reaches its apex in \href{https://suttacentral.net/mn49}{MN 49} \textit{On the Invitation of \textsanskrit{Brahmā}} (\textit{Brahmanimantanikasutta}), where the Buddha takes on no less that \textsanskrit{Brahmā} himself in a high-level philosophical debate, forcing \textsanskrit{Māra} to reveal himself on the side of \textsanskrit{Brahmā}. From the perspective of early Buddhism, God and the Devil are not so very different.

\section*{Theory \& Practice}

The Majjhima is perhaps the richest of the early Buddhist collections in matters of doctrine. It contains an extraordinary series of discourses that delve into profound topics with a detail and complexity not found elsewhere. It’s worth bearing in mind, though, that such discourses are for advanced students, and fascinating as they are, it is important to get a solid grounding on the fundamental doctrines collected in the \textsanskrit{Saṁyutta}. For this reason I reserve most doctrinal explanations for my guide to the Linked Discourses and make only a few brief remarks here.

The teachings familiar from the \textsanskrit{Saṁyutta} are all found in the Majjhima, and in several cases the Majjhima offers more detailed explanations. These discourses are important and deserve close study, but beware of equating length with importance. In Buddhist texts it’s just as likely that length implies, not that it is something the Buddha regarded as important, but that it is a late compilation.

Such is the case with \href{https://suttacentral.net/mn10}{MN 10} \textit{Mindfulness Meditation} (\textit{\textsanskrit{Satipaṭṭhānasutta}}), which, together with its expanded version at \href{https://suttacentral.net/dn22}{DN 22}, is the most detailed explanation of mindfulness meditation in the canon. Yet critical and comparative analysis reveals that the discourse as found in the Pali has been subject to considerable late development. \href{https://suttacentral.net/mn141}{MN 141} \textit{The Analysis of the Truths} (\textit{\textsanskrit{Saccavibhaṅgasutta}}), the most detailed discourse of the four noble truths, is closely related to \href{https://suttacentral.net/mn10}{MN 10}—in fact \href{https://suttacentral.net/dn22}{DN 22} is virtually a combination of this and \href{https://suttacentral.net/mn10}{MN 10}—and it too might be suspected to have late features. No such doubt attends to \href{https://suttacentral.net/mn111}{MN 111} \textit{One by One} (\textit{Anupadasutta}), which is clearly a late sutta. This is not the place for a complex discussion of text-critical method, but it is common and natural to assume that length implies importance, and it is worth bearing in mind that the situation is more complicated than that.

Majjhima suttas that deal with key doctrinal teachings can be understood as offering in-depth analyses of particular factors of the four noble truths. The first truth of suffering is explored in detail in \href{https://suttacentral.net/mn13}{MN 13} and \href{https://suttacentral.net/mn14}{MN 14} on the “Mass of Suffering”. Various topics under this heading are also treated in detail; the six sense fields are taught in several suttas (\href{https://suttacentral.net/mn18}{MN 18}, \href{https://suttacentral.net/mn137}{MN 137}, \href{https://suttacentral.net/mn138}{MN 138}) and even an entire \textit{vagga} (\href{https://suttacentral.net/mn143}{MN 143}–152), while several suttas investigate the teaching on the elements in great detail, exposing depths that one might not suspect for what appears to be such a simple teaching (\href{https://suttacentral.net/mn28}{MN 28}, \href{https://suttacentral.net/mn115}{MN 115}, \href{https://suttacentral.net/mn140}{MN 140}). The aggregates appear, but are treated in less detail.

The second and third noble truths are featured in \href{https://suttacentral.net/mn38}{MN 38}, a complex and rewarding discourse on dependent origination, as well as \href{https://suttacentral.net/mn135}{MN 135} and \href{https://suttacentral.net/mn136}{MN 136}, two of the most detailed and influential discourses on the topic of \textit{kamma}; see too \href{https://suttacentral.net/mn120}{MN 120}.

But it is the fourth noble truth, the path, that dominates the Majjhima. The vast majority of discourses deal with the path as a central topic. The noble eightfold path is treated in complex detail at \href{https://suttacentral.net/mn117}{MN 117} \textit{The Great Forty} (\textit{\textsanskrit{Mahācattārīsakasutta}}), and several discourses deal with specific path factors. In \href{https://suttacentral.net/mn9}{MN 9}, Venerable \textsanskrit{Sāriputta} presents the topic of right view from a diverse range of perspectives. The second path factor, right thought or right intention, is the special subject of \href{https://suttacentral.net/mn19}{MN 19} and \href{https://suttacentral.net/mn20}{MN 20}, which give advice from the Buddha’s own experience on how to first purify thought and then let it go entirely. The path factors on ethics are treated in too many suttas to be summarized here; some of these are covered in the sections on the \textsanskrit{Saṅgha} and the lay communities. Right effort is featured in many suttas, but is specially emphasized in texts such as \href{https://suttacentral.net/mn29}{MN 29}, \href{https://suttacentral.net/mn30}{MN 30}, and \href{https://suttacentral.net/mn32}{MN 32}. \href{https://suttacentral.net/mn10}{MN 10}, as noted, deals with right mindfulness, and the topic is treated from more specialized angles in \href{https://suttacentral.net/mn118}{MN 118} on mindfulness of breathing, and \href{https://suttacentral.net/mn119}{MN 119} on mindfulness of the body. In \href{https://suttacentral.net/mn77}{MN 77} we find a long list of different presentations of the path, including the topics found as heads in the \textsanskrit{Saṁyutta}, and quite a few more.

Right immersion or \emph{\textsanskrit{jhāna}}, the final factor of the path, is prominent throughout the Majjhima. It is difficult to overstate how central the \textit{\textsanskrit{jhānas}} are to Buddhist meditation. The formula for the four absorptions appears around 50 times in the Majjhima, which is probably more than the formulas for all the other path factors \emph{combined}. Moreover, right immersion appears also in other guises, such as the four “divine meditations” (\textit{\textsanskrit{brahmavihāra}}), where the pure emotions of love, compassion, rejoicing, and equanimity serve as basis for immersion. These appear around 13 times in the Majjhima. The even more subtle “formless” meditations are also a frequent topic, also appearing no less than 13 times. Unlike the absorptions, these are not an absolute requirement for the path to awakening, but clearly they were part of the practice for many talented meditators. Discourses such as \href{https://suttacentral.net/mn52}{MN 52} \textit{The Man From the City of \textsanskrit{Aṭṭhaka}} (\textit{\textsanskrit{Aṭṭhakanāgarasutta}}) combine these three sets of meditations, while advanced texts such as \href{https://suttacentral.net/mn43}{MN 43} \textit{The Great Classification} (\textit{\textsanskrit{Mahāvedallasutta}}), \href{https://suttacentral.net/mn44}{MN 44} \textit{The Shorter Classification} (\textit{\textsanskrit{Cūḷavedallasutta}}), \href{https://suttacentral.net/mn106}{MN 106} \textit{\textsanskrit{Āneñjasappāyasutta}}, \href{https://suttacentral.net/mn121}{MN 121} \textit{The Shorter Discourse on Emptiness} (\textit{\textsanskrit{Cūḷasuññatasutta}}), and \href{https://suttacentral.net/mn122}{MN 122} \textit{The Longer Discourse on Emptiness} (\textit{\textsanskrit{Mahāsuññatasutta}}) deal with rarely-discussed subtleties and refinements pertaining to the most advanced forms of meditation.

\section*{The Buddha}

The story of the Buddha’s life has become a primary vehicle for sharing and passing down Buddhist teachings and values in the Buddhist traditions. Yet there is no coherent biography of the Buddha in the early texts. Such information as we do have is scattered and piecemeal, found in the occasional details shared by the Buddha himself or inferred from the various situations in which we find him.

But perhaps this shouldn’t come as such a surprise. There’s no particular reason for the Buddha to be interested in telling his life story—he had lived it. And his disciples knew him personally. Only after he had passed away did the community feel the need to keep their teacher’s memory alive through vivid and detailed stories, growing more elaborate with every telling.

The broad outlines of the later Buddha legends grew out of the kernels in the early texts, many of which are found in the Majjhima. There we find the Buddha’s birth, his early upbringing, renunciation, practice and awakening, challenges involved in setting up a community, and various encounters while teaching. In the \textsanskrit{Dīgha} we find the longest narrative of early Buddhism, an extensive record of the Buddha’s last days found in \href{https://suttacentral.net/dn16}{DN 16} \textit{\textsanskrit{Mahāparinibbāna}}. And scattered throughout the texts we find isolated events and encounters. In one sense, most of the discourses can be considered as episodes of the Buddha’s life, vignettes in a magnificent myth, each one contributing a little to understanding the man and his message. These texts form our primary source of knowledge for the Buddha’s early life and teaching career. While the tendency towards legend-building is apparent in some places, most of these episodes are simple and realistic.

An exception is \href{https://suttacentral.net/mn123}{MN 123} \textit{Incredible and Amazing} (\textit{Acchariyaabbhutasutta}), where we find Ānanda, the founder of Buddhist biography, recounting an extended series of apparently miraculous events that accompanied the birth of the Buddha-to-be. This is evidently derived from \href{https://suttacentral.net/dn14}{DN 14} \textit{\textsanskrit{Mahāpadāna}} with some expansions. With its devotional tone and emphasis on the extraordinary, this text shows a shift in emphasis towards honoring the person of the Buddha rather than practicing his teachings; a shift that the Buddha resists by pointing out that the truly extraordinary thing is to be aware of one’s own mind.

The Buddha’s immediate family is mentioned only rarely in the \textit{\textsanskrit{nikāyas}}. The Buddha’s wife appears only in the Vinaya and \textsanskrit{Therīgāthā}, where she has some verses. His son \textsanskrit{Rāhula} is prominently featured in several discourses (\href{https://suttacentral.net/mn61}{MN 61}, \href{https://suttacentral.net/mn62}{MN 62}, \href{https://suttacentral.net/mn147}{MN 147}), showing the Buddha’s patient teaching and \textsanskrit{Rāhula}’s eventual awakening. The Buddha’s father is briefly mentioned in the above passage on meditation as a child. Both his parents are mentioned in several Majjhima dicourses (\href{https://suttacentral.net/mn26}{MN 26}, \href{https://suttacentral.net/mn36}{MN 36}, \href{https://suttacentral.net/mn85}{MN 85}, \href{https://suttacentral.net/mn100}{MN 100}), but they are only named elsewhere: Suddhodana his father in \href{https://suttacentral.net/snp3.11}{Snp 3.11} and \href{https://suttacentral.net/pli{-}tv{-}kd1.\#54}{Kd 1:54}; \textsanskrit{Māyā} his mother in \href{https://suttacentral.net/thig6.6}{Thig 6.6}; and both in \href{https://suttacentral.net/dn14}{DN 14} and \href{https://suttacentral.net/thag10.1}{Thag 10.1}. His stepmother \textsanskrit{Mahāpajāpatī} \textsanskrit{Gotamī}, however, appears in a few discourses, notably \href{https://suttacentral.net/mn142}{MN 142} \textit{The Analysis of Religious Donations} (\textit{\textsanskrit{Dakkhiṇāvibhaṅgasutta}}), and \href{https://suttacentral.net/an8.51}{AN 8.51}. More distant relatives include several well-known monastic and lay figures such as Ānanda, Anuruddha, and \textsanskrit{Mahānāma}.

In \href{https://suttacentral.net/mn75\#10}{MN 75:10} \textit{With \textsanskrit{Māgaṇḍiya}} the Buddha recounts the luxuries he enjoyed as a young man (cp. \href{https://suttacentral.net/an3.39}{AN 3.39}), and in \href{https://suttacentral.net/mn26\#13}{MN 26:13} he tells of the painful moment when he left home, though his parents wept in distress. An alternate account of the going forth is found in the distinctive and early \textsanskrit{Attadaṇḍa} Sutta, where going forth is prompted not by the sight of old age, sickness, and death, but by seeing the unceasing conflict and distress of the world (\href{https://suttacentral.net/snp4.15}{Snp 4.15}).

There follows the story of the six years of striving, divided into three periods. It seems that first he practiced under a yogic tradition, probably following the \textsanskrit{Upaniṣadic} philosophy. \href{https://suttacentral.net/mn26}{MN 26} \textit{The Noble Search} (\textit{Ariyapariyesanasutta}, also known as the \textit{\textsanskrit{Pāsarāsisutta}}) tells of his experience under the famed spiritual guides \textsanskrit{Āḷāra} \textsanskrit{Kālāma} and Uddaka \textsanskrit{Rāmaputta}. Under their tutelage he realized deep immersion (\textit{\textsanskrit{samādhi}}) and formless attainments, but he was still unsatisfied, so he left to embark on a more severe ascetic path.

According to the traditions, the bulk of this period was spent enduring harsh austerities, practices that are similar or identical to those observed by the Jains. These are described in detail in \href{https://suttacentral.net/mn12}{MN 12} \textit{The Longer Discourse on the Lion’s Roar} (\textit{\textsanskrit{Mahāsīhanādasutta}}) and \href{https://suttacentral.net/mn36}{MN 36} \textit{The Longer Discourse With Saccaka}. But after many years of such self-mortification he was getting nowhere. So he took some solid food and, recalling his childhood experience of absorption, began the third and final phase of his practice, which led to his awakening.

While it might seem as if the night of awakening followed directly from his rejection of austerities, several discourses indicate that this period involved a rather extensive development of meditation. \href{https://suttacentral.net/mn19}{MN 19} \textit{Two Kinds of Thought} (\textit{\textsanskrit{Dvedhāvitakkasutta}}), for example, tells of his analysis and training in wholesome thought; \href{https://suttacentral.net/mn4}{MN 4} \textit{Fear and Terror} (\textit{Bhayabheravasutta}) depicts his struggles to overcome fear; and \href{https://suttacentral.net/mn128}{MN 128} \textit{Corruptions} (\textit{Upakkilesasutta}) speaks of a series of specific meditative hindrances he encountered (see too \href{https://suttacentral.net/an8.64}{AN 8.64} and \href{https://suttacentral.net/an9.40}{AN 9.40}). While his reflections on pursuing a path of practice are always logical, in \href{https://suttacentral.net/an5.196}{AN 5.196} we read about an extraordinary series of dreams that foretold his awakening; the symbolism of the dream imagery repays careful attention.

It is in the Majjhima (\href{https://suttacentral.net/mn36}{MN 36}, \href{https://suttacentral.net/mn85}{MN 85}, \href{https://suttacentral.net/mn100}{MN 100}) that the Buddha speaks of that crucial moment in his childhood where he spontaneously entered the first absorption, a state of profound peace and stillness in meditation. Much later, when he arrived at a crisis in his spiritual progress, he was to recall this event and realize that meditative absorption was the only true path to awakening (\textit{eso’va maggo \textsanskrit{bodhāya}}).

The Buddha’s awakening is told from several different perspectives. In \href{https://suttacentral.net/mn4}{MN 4} \textit{Bhayabherava}, after describing how he overcame his fears, the Buddha tells how he developed the absorptions and gained the three higher knowledges. In \href{https://suttacentral.net/mn14}{MN 14} \textit{The Shorter Discourse on the Mass of Suffering} (\textit{\textsanskrit{Cūḷadukkhakkhandha}}) he speaks of the escape from suffering by letting go of sensual pleasures; see too \href{https://suttacentral.net/sn35.117}{SN 35.117}. Several discourses employ the stock framework of understanding the three aspects of gratification, drawback, and escape, applying this to the five aggregates (\href{https://suttacentral.net/sn22.26}{SN 22.26}), the elements (\href{https://suttacentral.net/sn14.31}{SN 14.31}), feelings (\href{https://suttacentral.net/sn36.24}{SN 36.24}), or the world (\href{https://suttacentral.net/an3.101}{AN 3.101}). Elsewhere awakening is seen as a result of understanding dependent origination (\href{https://suttacentral.net/sn12.10}{SN 12.10}, \href{https://suttacentral.net/dn14}{DN 14}). Other contexts depict awakening as emerging from different practices, such as the four kinds of mindfulness meditation (\href{https://suttacentral.net/sn47.31}{SN 47.31}), the four bases of psychic powers (\href{https://suttacentral.net/sn51.9}{SN 51.9}), or the abandoning of thoughts (\href{https://suttacentral.net/mn19}{MN 19}). This is far from an exhaustive list, as all of the Buddha's teachings depict different aspects of the wisdom that stems from awakening.

The period after the awakening is told in some detail, recounting the Buddha’s journeys, various encounters along the way, his first conversions, and setting up the Sangha. However the detailed account of this is in the Vinaya (\href{https://suttacentral.net/pli{-}tv{-}kd1}{Kd 1}), and only portions of these events are found in the \textit{\textsanskrit{nikāyas}}, such as the first three sermons (\href{https://suttacentral.net/sn56.11}{SN 56.11}, \href{https://suttacentral.net/sn22.59}{SN 22.59}, \href{https://suttacentral.net/sn35.28}{SN 35.28}).

The Buddha followed the same simple lifestyle as his monastic disciples. His possessions consisted of a single set of three robes, a bowl, and few other sundry items. He stayed for the most part in \textsanskrit{Anāthapiṇḍika}’s monastery near \textsanskrit{Sāvatthī}, although he spent time also in other monasteries. While in a monastery he lived in a simple hut. The year was divided into three seasons—hot, rainy, and cold. During the rainy season he always stayed in a monastery, while for the other seasons he might also wander the countryside.

A typical day would begin with early rising for meditation. The approaching dawn signaled the start of the day’s activities, particularly the daily alms round. Now, while in the monastery, monastics would typically wear only a lower sarong-like robe (and an upper cloth for the nuns). So, some time after dawn, they would dress in the full set of three robes before proceeding to a nearby village or town for alms round. During the alms round, as may be seen in Buddhist lands today, lay folk would put some food in the bowl to be eaten that day. Buddhist monastics are not allowed to receive money. The monastics would eat once or twice a day, but always between dawn and noon.

Normally the mendicants would retire after the morning meal to solitude for the day’s meditation, either in their huts, or perhaps in a nearby forest or some other quiet spot. In the hot season the Buddha would sometimes have a short siesta in the afternoon (\href{https://suttacentral.net/mn36\#46}{MN 36:46}). Towards evening the Buddha would emerge and would often give teachings or answer questions. But this was spontaneous, and not a fixed routine. Then he would meditate until late at night, needing only a few hours sleep in the middle of the night.

\href{https://suttacentral.net/mn91}{MN 91} \textit{With \textsanskrit{Brahmāyu}} describes his behavior in minute detail, recording the tiny nuances of mindfulness and care that he brought to every activity. His followers saw him as the embodiment of the highest truth. Discerning his realization in every detail of his words and acts, they could be moved to exalted praise in passages of joy and high beauty (\href{https://suttacentral.net/mn92}{MN 92} \textit{With Sela}, \href{https://suttacentral.net/an6.43}{AN 6.43}, \href{https://suttacentral.net/an10.30}{AN 10.30}, \href{https://suttacentral.net/sn8.7}{SN 8.7}). However, far from encouraging mindless devotion, the Buddha encouraged his students to investigate him, prescribing a rigorous and detailed set of tests that a good spiritual teacher should pass (\href{https://suttacentral.net/mn47}{MN 47} \textit{The Inquirer}).

This brief description does not do justice to the impact that the Buddha had on those who encountered him. He is constantly praised as the one who arises in the world out of compassion for sentient beings (\href{https://suttacentral.net/mn4}{MN 4}), who is perfected in good qualities (\href{https://suttacentral.net/mn38}{MN 38}), whose arising is the manifestation of a great light (\href{https://suttacentral.net/an1.175}{AN 1.175}–\href{https://suttacentral.net/an1.177}{AN 1.177}), who fully understands whatever is knowable (\href{https://suttacentral.net/mn1}{MN 1}; \href{https://suttacentral.net/an4.23}{AN 4.23}), and possesses complete confidence and courage no matter what the context (\href{https://suttacentral.net/mn12}{MN 12}). He possessed overwhelming physical beauty and charisma, making an unforgettable impression on many of the people he encountered (\href{https://suttacentral.net/mn26}{MN 26}, \href{https://suttacentral.net/an4.36}{AN 4.36}), but he told overzealous devotees to forget about his “putrid body” (\href{https://suttacentral.net/sn22.87}{SN 22.87}).

While as a person he was unique, and unequaled in his time (\href{https://suttacentral.net/mn56}{MN 56}; \href{https://suttacentral.net/an1.174}{AN 1.174}), as the “Realized One” (\textit{\textsanskrit{tathāgatha}}) the Buddha was one of a series of awakened masters of truly cosmic significance. His understanding did not just pertain to the narrow locale of his own time and place, but was equally relevant to all sentient beings in all realms at all times.

Other Buddhas have arisen in the past and will arise again in the future, with the same realization and teachings. A detailed list of past Buddhas is found at \href{https://suttacentral.net/dn14}{DN 14} \textit{\textsanskrit{Mahāpadāna}}. Mentions elsewhere, such as the Buddhas Kakusandha in \href{https://suttacentral.net/mn50}{MN 50} and Kassapa in \href{https://suttacentral.net/mn81}{MN 81}, show that the underlying idea was current throughout the \textit{\textsanskrit{nikāyas}} (See too \href{https://suttacentral.net/sn12.14}{SN 12.14}–10, \href{https://suttacentral.net/an3.80}{AN 3.80}, and \href{https://suttacentral.net/an5.180}{AN 5.180}).

\href{https://suttacentral.net/mn116}{MN 116} \textit{At Isigili} also refers to the so-called “Buddhas awakened for themselves” (\textit{paccekabuddha}), a mysterious kind of sage who has realized the same truths as the “fully awakened Buddha” (\textit{\textsanskrit{sammāsambuddha}}) but does not establish a religious movement.

Despite his exalted and revered status, the Buddha in the \textit{\textsanskrit{nikāyas}} had not yet been elevated to the cosmic divinity who appears in later Buddhism. For all his extraordinary qualities, he remains a very human figure. Nowhere does he say that his practice in past lives led to his awakening in this life; there is no mention of the \textit{\textsanskrit{pāramīs}}. Indeed, in one of the rare occasions when he refers to a past life (\href{https://suttacentral.net/mn81}{MN 81} \textit{With \textsanskrit{Ghaṭikāra}}), he appears decidedly un-enlightened.

In the early texts, the term \textit{bodhisatta} means “one intent on awakening”. It primarily refers to the period after leaving home and before awakening. The Buddha-to-be is not described as following a path that he had started long ago, but as exploring the different options available to him, uncertain as to how awakening may be gained. His crucial insight came, not through vows made in past lives, but when he remembered the time he fell into absorption (\textit{\textsanskrit{jhāna}}) as a child.

The Buddha goes to great lengths to detail the many trials and experiments he undertook before his path was mature. His truly special quality was that he discovered the path through his own efforts, and later taught that same path to his disciples. Even disciples who had realized the same liberation and understanding revered the Buddha as the one of unsurpassed wisdom and compassion who illuminated the path for others.

\section*{The Stages of Awakening}

Spiritual enlightenment or awakening in Buddhism is not seen as a vague or unknown quantity. On the contrary, it is specific, precise, and repeatable. The texts develop a detailed typology of enlightened beings. There are four main stages, subdivided into eight. These are called the “noble disciples” (\textit{\textsanskrit{ariyasāvaka}}) or “good people” (\textit{sappurisa}). These are presented from different perspectives throughout the texts; here is an overview.

The factors of the path are developed until they are all present to a sufficient degree of maturity. At this point one is said to be a “follower of the teachings” (\textit{\textsanskrit{dhammānusāri}}) or “follower by faith” (\textit{\textsanskrit{saddhānusāri}}), depending on whether wisdom or faith is predominant (\href{https://suttacentral.net/mn70\#20}{MN 70:20}). Such a person is also called “one who is on the path to stream-entry” (\href{https://suttacentral.net/mn142\#5}{MN 142:5}, \href{https://suttacentral.net/mn48\#15}{MN 48:15}). They will realize the Dhamma in this life, yet they have still not actually seen a vision of the Dhamma and their understanding is still to a degree conceptual (\href{https://suttacentral.net/sn25.1}{SN 25.1}). How far and fast they proceed depends on the strength of their faculties and their effort.

The texts are unclear as to whether such a person definitively knows that they have reached this point. No such ambiguity attaches to the moment of actually realizing stream-entry, however. It hits like a flash of lightning in the dead of night (\href{https://suttacentral.net/an3.25}{AN 3.25}), and you will remember precisely when and where it happened (\href{https://suttacentral.net/pj4}{Pj 4}). You have a vision of the four noble truths, at which point the conceptual understanding of the path becomes fully experiential (\href{https://suttacentral.net/sn25.1}{SN 25.1}). Letting go of three fetters (\textit{\textsanskrit{saṁyojana}})—doubt, misapprehension of precepts and vows, and any views that identify a self with the aggregates—one reaches the first stage of awakening, known as “stream entry”.

A stream-enterer is bound for awakening, free from any rebirth in lower realms (\href{https://suttacentral.net/mn6}{MN 6}; \href{https://suttacentral.net/sn55.1}{SN 55.1}), and is reborn a maximum of seven times (\href{https://suttacentral.net/an3.88}{AN 3.88}). They have eliminated a huge mass of suffering; what’s left is like seven small pebbles compared to the Himalayas (\href{https://suttacentral.net/sn56.59}{SN 56.59}). They are generous, devoted, and naturally keep the five precepts at minimum (\href{https://suttacentral.net/mn53}{MN 53}; \href{https://suttacentral.net/sn12.41}{SN 12.41}; \href{https://suttacentral.net/an7.6}{AN 7.6}). They understand the four noble truths and dependent origination and have no doubts as to the Buddha, his teaching, or his \textsanskrit{Saṅgha} (\href{https://suttacentral.net/mn7}{MN 7}; \href{https://suttacentral.net/sn12.41}{SN 12.41}). Yet there are still attachments; one may still sorrow when relatives pass away (\href{https://suttacentral.net/ud8.8}{Ud 8.8}), or suffer moments of grief or despair (\href{https://suttacentral.net/dn16\#5}{DN 16:5}.13).

When a stream-enterer further develops the path, they are said to be on the path to once-return; with the lessening of greed and hate they reach the state of a once-returner. At this point one will be reborn in this world once only. Again developing the path one completely eliminates greed and hate, at which point one is considered a non-returner. Such a person is reborn usually in one of the special realms of high divinity known as the Pure Abodes, and from there attains full extinguishment. However in certain cases a non-returner may become fully extinguished before being reborn (\href{https://suttacentral.net/an3.88}{AN 3.88}).

Yet even the exalted state of the non-returner is not entirely free from attachments. They have perfected ethics and immersion, yet their wisdom is still not complete. They still have the five higher fetters: attachment to rebirth in the realms of luminous form and the formless; a restless urgency to reach full awakening; a residual sense of self, or conceit; and ignorance.

Once again they further develop the path until they attain full perfection. A perfected one, or arahant, has fully eliminated all defilements, has made an end of rebirth, and when this life is over will suffer no more. Their liberation is identical with that of the Buddha. Since they have no hindrances or defilements of mind, they can attain deep meditation whenever they want, and their lives are full of joy and peace. They feel no grief, no anxiety, no confusion or doubt. A perfected one lives only in accordance with the Dhamma, and is incapable of reverting to worldly ways, or of indulging in material desires. They live a life of serenity and virtue, preferring seclusion and meditation, but are selfless in service, helping others whenever they can, especially through teaching. They are devoid of fear and have complete equanimity when faced with death (\href{https://suttacentral.net/mn140\#24}{MN 140:24}). They still experience the pain of the body, but have no mental suffering at all (\href{https://suttacentral.net/sn36.6}{SN 36.6}).

\section*{The Monastic Disciples}

Although we think of the suttas as “teachings of the Buddha”, in fact only about a quarter of the discourses in the Majjhima feature the Buddha simply delivering a discourse to the assembly. Over half the discourses consist of dialogues (\textit{\textsanskrit{vyākaraṇa}}). Sometimes the Buddha asks a question and leads the assembly through a process of Socratic inquiry. Sometimes a seeker asks a question or a series of questions. The Buddha may answer in various ways—sometimes directly, sometimes with a counter-question, sometimes with a lengthy analysis—depending on the nature of the question and the questioner. And intriguingly, he sometimes doesn’t answer at all.

And as well as featuring in one way or another in most of the discourses, various disciples serve as primary teachers in about a fifth of the discourses of the Majjhima, continuing the Buddha’s work of exploring and explaining the teachings. In the Majjhima \textsanskrit{Nikāya} we meet a range of skilled and accomplished teachers, many of whom became widely-famed in the Buddhist traditions. From the beginning of his dispensation, the Buddha was eager to empower his followers, encouraging them to share the teaching to the best of their ability.

Nine discourses are spoken by the Buddha’s chief disciple and renowned General of the Dhamma, \textsanskrit{Sāriputta}. These include \href{https://suttacentral.net/mn9}{MN 9}, a wide-ranging exploration of the many facets of right view; \href{https://suttacentral.net/mn28}{MN 28}, which shows how all the teachings can be included in the four noble truths; and \href{https://suttacentral.net/mn141}{MN 141}, giving a fully detailed explanation of the four noble truths based on the Buddha’s first sermon. In these discourses \textsanskrit{Sāriputta} shows his interest in developing a systematic overview of the Dhamma. Such analyses are one of the primary inspirations behind the later development of the Abhidhamma texts, and \textsanskrit{Sāriputta} is rightly regarded as one of the fathers of the Abhidhamma.

\textsanskrit{Mahākaccāna} is another monk of renowned wisdom, whose incisive analytic style, with special focus on the process of sense experience, also influenced the Abhidhamma. He spoke four discourses in the Majjhima. \href{https://suttacentral.net/mn18}{MN 18}, the “Honey Cake” demonstrates his unmatched skill in drawing out subtle implications of brief teachings by the Buddha. This discourse received a detailed treatment in Katukurunde Nyanananda’s \textit{Concept and Reality in Early Buddhist Thought}, one of the most influential monographs in modern Buddhist studies.

\href{https://suttacentral.net/mn24}{MN 24}, a dialogue between \textsanskrit{Sāriputta} and \textsanskrit{Puṇṇa} \textsanskrit{Mantāṇiputta}, speaks of a relay of chariots, introducing the idea of the seven stages of purification that was to greatly influence the concept of stages of insight in Buddhaghosa’s \textit{Visuddhimagga} and, from there, contemporary Theravada meditation.

The nun \textsanskrit{Dhammadinnā} presents the only discourse by a \textit{\textsanskrit{bhikkhunī}} in the Majjhima. It seems that the teachings by women in the suttas, while rare, were included because of their unique and striking wisdom. In \href{https://suttacentral.net/mn44}{MN 44} she responds to a series of questions by her former husband, revealing her depth of meditative accomplishment and wisdom.

Further discourses by disciples include several by \textsanskrit{Moggallāna}, Anuruddha, and others. And we cannot pass a discussion on disciples without mentioning Ānanda, who lived closest to the Buddha, and through whom, according to tradition, all the texts passed. Ānanda is the main teacher in seven discourses, and features in many more. Ānanda has a specially close interest in the personal life of the Buddha, and it is with him that the Buddha’s life story began to take on its familiar form.

\section*{The \textsanskrit{Saṅgha}}

Mighty in wisdom though they were, the enlightened disciples of the Buddha did not exist in isolation. They were part of an organized spiritual community called the \textsanskrit{Saṅgha}, or “Monastic Order”. \textit{\textsanskrit{Saṅgha}} as a religious term is used in two specific senses: the “monastic order” (\textit{\textsanskrit{bhikkhusaṅgha}}) and the “community of noble disciples” (\textit{\textsanskrit{sāvakasaṅgha}}; the expected term \textit{\textsanskrit{ariyasaṅgha}} only appears in one verse at \href{https://suttacentral.net/an6.54}{AN 6.54}). The former term refers to those who have taken ordination and practice as a Buddhist mendicant, while the latter refers to someone who has reached one of the stages of awakening, which may, of course, include lay followers.

The distinction between these two is not as clear-cut as one might imagine. The standard formula describing the \textsanskrit{Saṅgha} in the recollection of the Triple Gem mentions the eight kinds of noble disciple on the path. But it also describes them in terms typical of the monastic order, for example as a “field of merit”. It is difficult to draw a decisive conclusion from this, however, as this formula appears to be a composite one. As a general rule, it is safe to assume that, unless the context specifies the \textsanskrit{Saṅgha} of noble disciples, it is referring to the monastic order.

The monastic community was established shortly after the Buddha’s awakening, as told in the first chapter of the Vinaya Khandhakas, known as the \textsanskrit{Mahākkhandhaka} or the Pabbajjakkhandhaka (\href{https://suttacentral.net/pli{-}tv{-}kd1}{Kd 1}). Portions of that narrative appear in the suttas, but the entire story should really be read.

The early \textsanskrit{Saṅgha} consisted of a small community of advanced spiritual practitioners; according to the texts, they were all perfected ones. Famously, the Buddha urged his followers to wander the countryside, teaching the Dhamma for the benefit of all people (\href{https://suttacentral.net/sn4.5}{SN 4.5}). This sets the example for how the Buddha was to relate to his community. He did not operate as a guru figure who insisted on obedience and treated his followers as dependents. On the contrary, he treated his community as adults who could make their own choices, and he trusted them to make their own contributions. To be sure, at this early stage they were all awakened; but this policy continued throughout his life, even up to his deathbed when this had long ceased to be the case.

\href{https://suttacentral.net/pli{-}tv{-}kd1}{Kd 1} tells the story of how the ordination procedure evolved. Originally the Buddha himself simply invited his followers with the words “Come, mendicant!” Later, as the numbers of candidates grew, the Buddha allowed the mendicants to perform ordination themselves using a simple ceremony of going to the three refuges; this is still used for novice ordinations. Eventually the procedure was formalized in the mature form as a “motion and three announcements”. The mature form of ordination ceremony had various formal and legalistic elements: a set of questions was given to vet the candidates; the questioners were formally appointed by the \textsanskrit{Saṅgha}; the candidates, having been questioned, were assigned a mentor; and the entire \textsanskrit{Saṅgha} gave their assent to the ordination. The entire procedure is straightforward and legalistic, devoid of ritual or embellishment. The core of the procedure is still followed today, although the traditions have adorned the bare bones of the procedure with a colorful range of rituals and celebrations.

This procedure set the template for the proceedings of the Sangha. It is by consensus: the \textsanskrit{Saṅgha} as a whole gives the ordination (\emph{\textsanskrit{saṁgho} \textsanskrit{itthannāmaṁ} \textsanskrit{upasampādeti}}). If even a single mendicant dissents the ordination does not proceed. Contrary to a misunderstanding that is unfortunately common even within the \textsanskrit{Saṅgha}, the mentor—called \textit{\textsanskrit{upajjhāya}} for the monks or \textit{\textsanskrit{pavattinī}} for the nuns—does not perform the ordination; they are appointed by the \textsanskrit{Saṅgha} to support the new monastics.

For legal purposes the \textsanskrit{Saṅgha} is the community within the “boundary” (\textit{\textsanskrit{sīmā}}), which is an arbitrary area formally designated by the local community; typically it would have been the grounds of a monastery, but it could have been much bigger or smaller (see \href{https://suttacentral.net/pli{-}tv{-}kd2}{Kd 2}). This is an important pragmatic point: the \textsanskrit{Saṅgha} is decentralized. It is impractical to expect all mendicants from all over the world to come together to agree, so all procedures are based on the local community.

There is no hierarchy in the \textsanskrit{Saṅgha}: all members have the same say. Respect is owed to seniors on account of their experience and wisdom, but this does not translate to a power of command. No \textsanskrit{Saṅgha} member has the right to force anyone to do anything, and if a senior \textsanskrit{Saṅgha} member, even one’s mentor or teacher, tells one to do something that is against the Dhamma or Vinaya, one is obligated to disobey. As an example of how the mendicants were to make decisions, \href{https://suttacentral.net/mn17}{MN 17} \textit{Jungle Thickets} (\textit{Vanapatthasutta}) gives some guidelines for whether a mendicant should stay in a monastery or leave; there is no question of being ordered to go to one place or the other.

The \textsanskrit{Saṅgha} soon set up monasteries, with relatively settled communities. Typically mendicants would stay in monasteries for part of the year, especially in the three months of the rainy season retreat, while much of the rest of the year may have been spent wandering. They would meet together each fortnight for the “sabbath” (\textit{uposatha}), during which time there would be teachings (\href{https://suttacentral.net/mn109}{MN 109} \textit{The Longer Discourse on the Full-Moon Night}, \textit{\textsanskrit{Mahāpuṇṇamasutta}}), and later, the recitation of the monastic rules (\textit{\textsanskrit{pātimokkha}}).

At some point a community of nuns (\textit{\textsanskrit{bhikkhunīs}}) was set up along the lines of the monks’ order. The traditional account says that this was on the instigation of the Buddha’s step-mother, \textsanskrit{Mahapajāpatī} \textsanskrit{Gotamī}. However, the account as preserved today is deeply problematic both textually and ethically, and cannot be accepted without reservation. In any case, we know that a nuns’ community was established, and that it ran on mostly independent grounds. The nuns built their own monasteries (\href{https://suttacentral.net/pli{-}tv{-}bi{-}vb{-}pj5}{Bi Pj 5}), took their own students (\href{https://suttacentral.net/thig5.11}{Thig 5.11}), studied the texts (\href{https://suttacentral.net/pli{-}tv{-}bi{-}vb{-}pc{-}33}{Bi Pc 33}), developed meditation (\href{https://suttacentral.net/sn47.10}{SN 47.10}), wandered the countryside (\href{https://suttacentral.net/pli{-}tv{-}bi{-}vb{-}pc{-}50}{Bi Pc 50}), and achieved the wisdom of awakening (\href{https://suttacentral.net/mn44}{MN 44} \textit{The Shorter Classification}, \textit{\textsanskrit{Cūḷavedallasutta}}). While it is true that certain of the rules as they exist today discriminate against the nuns, other rules protect them; for example, the monks are forbidden from having a nun wash their robes, thus preventing the monks from treating the nuns like domestic servants. When monks taught the nuns, they did so respectfully, engaging with them as equals (\href{https://suttacentral.net/mn146}{MN 146} \textit{Advice from Nandaka}). The order of nuns gave women of the time a rare opportunity to pursue their own spiritual path, supported by the community. It survived through the years in the East Asian traditions, and in recent years has been revived within the Tibetan and \textsanskrit{Theravādin} schools.

Note that in the suttas, the term \textit{bhikkhu} (masculine gender) is used as a generic term to include both monks and nuns. That nuns were included in the generic masculine is clear from such contexts as \href{https://suttacentral.net/an4.170}{AN 4.170}, where Ānanda begins by referring to “monks and nuns” and continues with just “monks”, or \href{https://suttacentral.net/dn16}{DN 16}, which speaks of “monks and nuns” but uses a masculine pronoun to refer to them both. That the default masculine may refer to women is further confirmed by passages such as \href{https://suttacentral.net/thig16.1}{Thig 16.1}, where the lady \textsanskrit{Sumedhā} is called \textit{putta} by her father. \textit{Putta} as “son” contrasts with \textit{\textsanskrit{dhītā}} as “daughter”, but this passage shows it can be used in the generic sense of “child” as well. In general teaching, it is likely that monks and nuns, as well as lay people, would have been present, yet the texts by convention are addressed to “monks” (\textit{bhikkhave}). Hence I have rendered \textit{bhikkhu} throughout with the gender-neutral “mendicant”, except where it is necessary to distinguish the genders, in which case I use “monk”. In the Vinaya \textsanskrit{Piṭaka}, however, the texts are by default separated by gender, so it is best to use “monk” there. Note too that “mendicant” is what \textit{bhikkhu} actually means: it refers to someone who makes a living by walking for alms.

Originally the \textsanskrit{Saṅgha} followed an informal set of principles considered appropriate for ascetics, which was similar to those followed by other ascetic groups. These are retained in detail in the ethics portion of the Gradual Training (\href{https://suttacentral.net/mn51}{MN 51}; \href{https://suttacentral.net/dn1}{DN 1}, etc.). The Buddha initially refused to set up a formal system of monastic law, but eventually the \textsanskrit{Saṅgha} grew so large that such a system became necessary (\href{https://suttacentral.net/pli{-}tv{-}bu{-}vb{-}pj1}{Bu Pj 1}). This is detailed in the extensive texts of the Vinaya \textsanskrit{Piṭaka}. Note that when the word \textit{vinaya} is used in the four \textit{\textsanskrit{nikāyas}}, it rarely refers to the Vinaya \textsanskrit{Piṭaka}. Normally it is a general term for the practical application of the teaching: \textit{dhammavinaya} means something like “theory and practice”.

Scattered throughout the four \textit{\textsanskrit{nikāyas}} we find a fair number of teachings intended for the monastic community. Sometimes these refer to technical procedures of the Vinaya \textsanskrit{Piṭaka}, probably to make sure that students of the suttas would be familiar with them (\href{https://suttacentral.net/mn104}{MN 104} \textit{At \textsanskrit{Sāmagāma}}). More commonly, however, they were general principles of ethics, laying down guidelines for a harmonious and flourishing spiritual community.

In \href{https://suttacentral.net/mn3}{MN 3} \textit{\textsanskrit{Dhammadāyāda}} the Buddha teaches his mendicants to be his “heirs in the teaching”, inheriting spiritual not material things from their Teacher. In \href{https://suttacentral.net/mn5}{MN 5} \textit{Unblemished} (\textit{\textsanskrit{Anaṅgaṇasutta}}), Venerables \textsanskrit{Sāriputta} and \textsanskrit{Moggallāna} speak of the those who have “blemishes” that spoil their spiritual integrity. When guilty of an offense, rather than clearing it by confession, they hide it. Or they hope the Buddha asks them a question about the teaching, or that they get the best food at meal time. Such things appear small, but over time they corrupt, so they must be polished off diligently, for scrupulous attention to ethics is the foundation for all higher achievements in the spiritual path (\href{https://suttacentral.net/mn6}{MN 6} \textit{One Might Wish}, \textit{\textsanskrit{Ākaṅkheyyasutta}}). It is essential for community members to be open to admonition, for otherwise they cannot identify their flaws and heal them (\href{https://suttacentral.net/mn15}{MN 15} \textit{Measuring Up}, \textit{\textsanskrit{Anumānasutta}}). But a community cannot be based on sniping and criticism, but on love, generosity, and respect (\href{https://suttacentral.net/mn16}{MN 16} \textit{Emotional Barrenness}, \textit{Cetokhilasutta}).

Not all were satisfied with the Buddha’s path. A certain Sunakkhatta—familiar from \href{https://suttacentral.net/dn24}{DN 24}—disrobed, for he wanted to see more miracles, and thought the mere ending of suffering was a poor goal (\href{https://suttacentral.net/mn12}{MN 12}; see too \href{https://suttacentral.net/mn63}{MN 63}). The monk \textsanskrit{Ariṭṭha} went even further, directly contradicting the fundamental principles of the Dhamma by declaring that the things the Buddha said were harmful were not in fact harmful (\href{https://suttacentral.net/mn22}{MN 22} \textit{The Simile of the Snake}, \textit{\textsanskrit{Alagaddūpamasutta}}). This event prompted the establishment of a number of Vinaya rules, showing the interdependence of these bodies of literature (\href{https://suttacentral.net/pli{-}tv{-}bu{-}vb{-}pc68}{Bu Pc 68}, \href{https://suttacentral.net/pli{-}tv{-}bu{-}vb{-}pc69}{Bu Pc 69}; \href{https://suttacentral.net/pli{-}tv{-}kd11}{Kd 11}).

Sometimes problems went beyond just an individual, and a whole community could split apart. \href{https://suttacentral.net/mn48}{MN 48} \textit{The Mendicants of Kosambi}, which also has parallels in the Vinaya, tells of how a community can be split because of an apparently trivial difference. They became so consumed by anger and conceit that they even ignored the Buddha’s attempts at reconciliation. Eventually, though, they came to their senses. Likewise, the monks at \textsanskrit{Cātuma} (\href{https://suttacentral.net/mn67}{MN 67}) were so unruly the Buddha dismissed them, but was persuaded to relent.

Reconciliation and growth is also the message of \href{https://suttacentral.net/mn65}{MN 65} \textit{With \textsanskrit{Bhaddāli}}, where the Buddha requests that the mendicants eat before noon. \textsanskrit{Bhaddāli} refuses, out of greed and stubbornness, but later accepts the Buddha’s ruling and is forgiven. This rule was clearly a big deal, for it is also discussed in \href{https://suttacentral.net/mn66}{MN 66} and \href{https://suttacentral.net/mn70}{MN 70}.

But in general the suttas paint a glowingly positive view of the renunciate life, leaving the Vinaya \textsanskrit{Piṭaka} to deal with the nitty-gritty of human failings. \href{https://suttacentral.net/mn31}{MN 31} \textit{The Shorter Discourse at \textsanskrit{Gosiṅga}} (\textit{\textsanskrit{Cūḷagosiṅgasutta}} depicts an idyllic fellowship of three monks living in gracious and fulfilling harmony, where the simplicity and purity of their lifestyle forms the basis for advanced meditation (see \href{https://suttacentral.net/mn128}{MN 128} \textit{Corruptions}, \textit{Upakkilesasutta}, which shares the same setting). The \textsanskrit{Saṅgha} lives honoring the Buddha not out of fear but from genuine love and respect (\href{https://suttacentral.net/mn77}{MN 77} \textit{The Longer Discourse with \textsanskrit{Sakuludāyī}}). By practicing in line with the Dhamma, they honor their Teacher and become worthy of their alms-food (\href{https://suttacentral.net/mn141}{MN 141} \textit{The Purification of Alms}, \textit{\textsanskrit{Piṇḍapātapārisuddhisutta}}).

\section*{The Wider Community}

The monastics are far from the only people we meet. In his wide wanderings across the Ganges plain, the Buddha interacted with a wide cross-section of the local peoples: learned brahmins and simple villagers; kings and slaves; priests and prostitutes; women and men; children and the elderly; the devout and the skeptical; the sick and the disabled; those seeking to disparage and those sincerely seeking the truth.

It should not be thought that teachings for monastics and lay were completely separated. For example, \href{https://suttacentral.net/mn7}{MN 7} \textit{The Simile of the Cloth} (\textit{Vatthasutta} or \textit{\textsanskrit{Vatthūpamasutta}}) begins with a standard teaching to the mendicants about purifying the mind, leading up to the divine meditations and full awakening. The Buddha calls such a person “bathed with the inner bathing”. At this, the brahmin Sundarika—who happened to be sitting nearby—protested, saying that the brahmins attested to the purifying properties of bathing in the river \textsanskrit{Bāhuka}. This casual example shows that, even when a discourse is addressed to the mendicants, a wide range of people, including non-Buddhists, might be present.

The brahmin from \href{https://suttacentral.net/mn7}{MN 7} ended up taking ordination. However there are also many cases of long-term practitioners who remained in the lay life. In \href{https://suttacentral.net/mn14}{MN 14} \textit{\textsanskrit{Cūḷadukkhakkhandha}}, the Buddha’s relative \textsanskrit{Mahānāma} laments that despite his long years of practice, he still has greed, hate, and delusion. Evidently lay meditators struggled to find deep peace of mind in those days, just like today.

Such cases show how the Buddha spoke to individuals, addressing their specific needs and concerns. Elsewhere he gave more general discourses on ethical principles. In \href{https://suttacentral.net/mn41}{MN 41} \textit{The People of \textsanskrit{Sālā}} (\textit{\textsanskrit{Sāleyyakasutta}}), he teaches a group of non-Buddhist lay people what are commonly called the “ten ways of doing skillful deeds” (\textit{\textsanskrit{dasakusalakammapathā}}; see \href{https://suttacentral.net/an10.176}{AN 10.176}):

\begin{enumerate}%
\item No killing%
\item No stealing%
\item No sexual misconduct%
\item True speech%
\item Harmonious speech%
\item Gentle speech%
\item Meaningful speech%
\item No covetousness%
\item No ill will%
\item Right view%
\end{enumerate}

The first three pertain to the body; the second four to speech, and the final three to the mind. These extend the well-known five precepts, offering a complete guide to ethical living. They are taught in many places in the suttas, and here they are defined in detail.

While in the Majjhima, and everywhere in the canon, formalistic and artificial settings dominate, several discourses have a somewhat messy narrative structure, which seem to preserve the memory of real-life incidents. In \href{https://suttacentral.net/mn51}{MN 51} \textit{With Kandaraka} the Buddha is approached by an elephant driver named Pessa and a wandering ascetic named Kandaraka. Kandaraka is impressed by the silence of the \textsanskrit{Saṅgha}, prompting the Buddha to attribute this to their practice of mindfulness meditation. Pessa intervenes, saying that lay folk also sometimes practice mindfulness, and the Buddha engages him on the topic of people who act for their own harm or benefit, or that of others; a topic familiar from the \textsanskrit{Aṅguttara} (\href{https://suttacentral.net/an4.198}{AN 4.198}). But Pessa has to leave, at which the Buddha says he would have benefited by staying. The mendicants ask the Buddha to finish what he was about to say, at which he gives a lengthy version of the Gradual Training. We don’t hear about what happened to either Pessa or Kandaraka. We do, however, find the Gradual Training taught to lay folk elsewhere, for example by Ānanda on the occasion of opening a new community hall belonging to his clan, the Sakyans (\href{https://suttacentral.net/mn53}{MN 53} \textit{A Trainee}, \textit{Sekhasutta}). Indeed, the gradual progress of a mendicant is compared with a graduated professional education of an accountant (\href{https://suttacentral.net/mn107}{MN 107} \textit{With \textsanskrit{Moggallāna} the Accountant}, \textit{\textsanskrit{Gaṇakamoggallānasutta}}).

In addition to personal problems and general ethical teachings, the Buddha responded to criticisms of his community, not with defensiveness, but by clear explanation. In \href{https://suttacentral.net/mn55}{MN 55} \textit{With \textsanskrit{Jīvaka}}, the Buddha’s physician reports that people are saying that the Buddha eats meat that has been slaughtered on purpose for him. The Buddha denies this, saying that he and his mendicants only accept food that has been freely offered, and will refuse any meat they suspect has been killed for them.

While the critics are not identified in \href{https://suttacentral.net/mn55}{MN 55}, \href{https://suttacentral.net/mn56}{MN 56} \textit{With \textsanskrit{Upāli}} and \href{https://suttacentral.net/mn58}{MN 58} \textit{With Prince Abhaya} (\textit{\textsanskrit{Abhayarājakumārasutta}}) depicts the Jains as deliberately trying to take down the Buddha in debate, and—since these are Buddhist texts—failing. Brahmins attempt the same trick, sending their most precocious scholars against the Buddha with no more success (\href{https://suttacentral.net/mn93}{MN 93}, \href{https://suttacentral.net/mn95}{MN 95}).

A few discourses focus not on the Buddha’s message as such, but on the impact it had on family life. When the wealthy young man \textsanskrit{Raṭṭhapāla} ordains, his parents are highly distressed and try to entice him to disrobe (\href{https://suttacentral.net/mn82}{MN 82}). \href{https://suttacentral.net/mn87}{MN 87} \textit{Born From the Beloved} (\textit{\textsanskrit{Piyajātikasutta}}) gives us a glimpse into the family life of King Pasenadi and Queen \textsanskrit{Mallikā}. They hear of the Buddha’s teaching that our loved ones bring us suffering, and when \textsanskrit{Mallikā} agrees with this, her husband is not pleased at all. \textsanskrit{Mallikā} is careful to first confirm that the teaching was, in fact, what the Buddha said, then she gently and kindly explains the teaching to the King, leading him to announce his faith in the Buddha. He became a devoted follower featured in many discourses. \href{https://suttacentral.net/mn100}{MN 100} \textit{With \textsanskrit{Saṅgārava}} is another case where a wife leads her husband to the Dhamma.

It is not only the great and the good who are in need of teachings. The suttas depict Venerable \textsanskrit{Sāriputta} helping his old friend \textsanskrit{Dhanañjāni}, who was redeemed near the end of his life, despite his long career of corruption (\href{https://suttacentral.net/mn97}{MN 97}). Even the serial killer \textsanskrit{Aṅgulimāla} found redemption and forgiveness (\href{https://suttacentral.net/mn86}{MN 86}; cf. \href{https://suttacentral.net/thag16.8}{Thag 16.8}).

\section*{A Brief Textual History}

The Majjhima \textsanskrit{Nikāya} was edited by V. Trenckner (vol. 1) and Robert Chalmers (vols. 2 and 3) on the basis of manuscripts in Sinhalese, Burmese, and Thai scripts, and published in Latin script by the Pali Text Society from 1888 to 1899. The first translation into English followed in 1926–7 by Robert Chalmers under the title \textit{Further Dialogues of the Buddha}.

Rapid improvements in understanding of Pali and Buddhism during the early 20th century soon made it clear that an improved translation was needed. This was undertaken in the 1950s by I.B. Horner, and was published by the PTS as \textit{The Book of Middle Length Sayings} in 1954–9.

Her translation, while a significant improvement on Chalmers’, was soon eclipsed by that of Bhikkhu \textsanskrit{Ñāṇamoḷi}. \textsanskrit{Ñāṇamoḷi}’s extraordinary career as a Pali scholar and translator was tragically cut short by his early death, and his Majjhima \textsanskrit{Nikāya} translation remained as an unfinished hand-written manuscript. Nevertheless, its value was so clear that it was published, first as a selection of 90 discourses edited by Bhikkhu Khantipalo and published in Bangkok in 1976 as \textit{A Treasury of the Buddha’s Words}, then as a fully edited and updated version by Bhikkhu Bodhi in 1995 under the title \textit{The Middle Length Discourses of the Buddha: A Translation of the Majjhima Nikaya}. The latter version reached a peak of accuracy, consistency, and readability that has become the standard to which all later translations of the \textit{\textsanskrit{nikāyas}} have aspired.

When the Pali was unclear I frequently referred to the earlier work of \textsanskrit{Ñāṇamoḷi} \& Bodhi (both the published work and \textsanskrit{Ñāṇamoḷi}’s hand-written original), and less often to Horner and various translations of specific texts. I also had access to notes by Bhikkhus \textsanskrit{Ñāṇadīpa} and \textsanskrit{Ñāṇatusita} on Bhikkhu Bodhi’s translation. In addition, I consulted the Chinese and other parallel texts, and the detailed studies on these by Bhikkhu \textsanskrit{Anālayo}. However, I found these to be useful for translation in only a few instances, as I believe it is important to preserve the integrity of the different textual lineages. Comparative studies lose value when the underlying texts have already been reconciled.

%
\chapter*{Acknowledgements}
\addcontentsline{toc}{chapter}{Acknowledgements}
\markboth{Acknowledgements}{Acknowledgements}

I remember with gratitude all those from whom I have learned the Dhamma, especially Ajahn Brahm and Bhikkhu Bodhi, the two monks who more than anyone else showed me the depth, meaning, and practical value of the Suttas.

Special thanks to Dustin and Keiko Cheah and family, who sponsored my stay in Qi Mei while I made this translation.

Thanks also for Blake Walshe, who provided essential software support for my translation work.

Throughout the process of translation, I have frequently sought feedback and suggestions from the community on the SuttaCentral community on our forum, “Discuss and Discover”. I want to thank all those who have made suggestions and contributed to my understanding, as well as to the moderators who have made the forum possible. A special thanks is due to \textsanskrit{Sabbamittā}, a true friend of all, who has tirelessly and precisely checked my work.

Finally my everlasting thanks to all those people, far too many to mention, who have supported SuttaCentral, and those who have supported my life as a monastic. None of this would be possible without you.

%
\chapter*{Summary of Contents}
\addcontentsline{toc}{chapter}{Summary of Contents}
\markboth{Summary of Contents}{Summary of Contents}

\begin{description}%
\item[\href{\#mn{-}mulapariyayavagga}{None}] This chapter, though beginning with the abstruse \textsanskrit{Mūlapariyāya} Sutta, mostly contains foundational teachings and can, as a whole, serve as an introduction to the discourses.%
\item[\href{\#mn1}{None}] The Buddha examines how the notion of a permanent self emerges from the process of perception. A wide range of phenomena are considered, embracing both naturalistic and cosmological dimensions. An unawakened person interprets experience in terms of a self, while those more advanced have the same experiences without attachment.%
\item[\href{\#mn2}{None}] The diverse problems of the spiritual journey demand a diverse range of responses. Rather than applying the same solution to every problem, the Buddha outlines seven methods of dealing with defilements, each of which works in certain cases.%
\item[\href{\#mn3}{None}] Some of the Buddha’s students inherit from him only material profits and fame. But his true inheritance is the spiritual path, the way of contentment. Venerable \textsanskrit{Sāriputta} explains how by following the Buddha’s example we can experience the fruits of the path.%
\item[\href{\#mn4}{None}] The Buddha explains the difficulties of living in the wilderness, and how they are overcome by purity of conduct and meditation. He recounts some of the fears and obstacles he faced during his own practice.%
\item[\href{\#mn5}{None}] The Buddha’s chief disciples, \textsanskrit{Sāriputta} and \textsanskrit{Moggallāna}, use a simile of a tarnished bowl to illustrate the blemishes of the mind and conduct. They emphasize how the crucial thing is not so much whether there are blemishes, but whether we are aware of them.%
\item[\href{\#mn6}{None}] According to the Buddha, careful observance of ethical precepts is the foundation of all higher achievements in the spiritual life.%
\item[\href{\#mn7}{None}] The many different kinds of impurities that defile the mind are compared to a dirty cloth. When the mind is clean we find joy, which leads to states of higher consciousness. Finally, the Buddha rejects the Brahmanical notion that purity comes from bathing in sacred rivers.%
\item[\href{\#mn8}{None}] The Buddha differentiates between peaceful meditation and spiritual practices that encompass the whole of life. He lists forty-four aspects, which he explains as “effacement”, the wearing away of conceit.%
\item[\href{\#mn9}{None}] Venerable \textsanskrit{Sāriputta} gives a detailed explanation of right view, the first factor of the noble eightfold path. At the prompting of the other mendicants, he approaches the topic from a wide range of perspectives.%
\item[\href{\#mn10}{None}] Here the Buddha details the seventh factor of the noble eightfold path, mindfulness meditation. This collects many of the meditation teachings found throughout the canon, especially the foundational practices focusing on the body, and is regarded as one of the most important meditation discourses.%
\item[\href{\#mn{-}sihanadavagga}{None}] Beginning with two discourses containing a “lion’s roar”, this chapter deals with suffering, community life, and practical meditation advice.%
\item[\href{\#mn11}{None}] The Buddha declares that only those following his path can genuinely experience the four stages of awakening. This is because, while much is shared with other systems, none of them go so far as to fully reject all attachment to the idea of a self.%
\item[\href{\#mn12}{None}] A disrobed monk, Sunakkhatta, attacks the Buddha’s teaching because it merely leads to the end of suffering. The Buddha counters that this is, in fact, praise, and goes on to enumerate his many profound and powerful achievements.%
\item[\href{\#mn13}{None}] Challenged to show the difference between his teaching and that of other ascetics, the Buddha points out that they speak of letting go, but do not really understand why. He then explains in great detail the suffering that arises from attachment to sensual stimulation.%
\item[\href{\#mn14}{None}] A lay person is puzzled at how, despite their long practice, they still have greedy or hateful thoughts. The Buddha explains the importance of absorption meditation for letting go such attachments. But he also criticizes self-mortification, and recounts a previous dialog with Jain ascetics.%
\item[\href{\#mn15}{None}] Venerable \textsanskrit{Moggallāna} raises the topic of admonishment, without which healthy community is not possible. He lists a number of qualities that will encourage others to think it worthwhile to admonish you in a constructive way.%
\item[\href{\#mn16}{None}] The Buddha explains various ways one can become emotionally cut off from one’s spiritual community.%
\item[\href{\#mn17}{None}] While living in the wilderness is great, not everyone is ready for it. The Buddha encourages meditators to reflect on whether one’s environment is genuinely supporting their meditation practice, and if not, to leave.%
\item[\href{\#mn18}{None}] Challenged by a brahmin, the Buddha gives an enigmatic response on how conflict arises due to proliferation based on perceptions. Venerable \textsanskrit{Kaccāna} draws out the detailed implications of this in one of the most insightful passages in the entire canon.%
\item[\href{\#mn19}{None}] Recounting his own experiences in developing meditation, the Buddha explains how to understand harmful and harmless thoughts, and how to go beyond thought altogether.%
\item[\href{\#mn20}{None}] In a practical meditation teaching, the Buddha describes five different approaches to stopping thoughts.%
\item[\href{\#mn{-}opammavagga}{None}] A diverse chapter including biography, non-violence, not-self, and an influential teaching on the progress of meditation.%
\item[\href{\#mn21}{None}] A discourse full of vibrant and memorable similes, on the importance of patience and love even when faced with abuse and criticism. The Buddha finishes with the simile of the saw, one of the most memorable similes found in the discourses.%
\item[\href{\#mn22}{None}] One of the monks denies that prohibited conduct is really a problem. The monks and then the Buddha subject him to an impressive dressing down. The Buddha compares someone who understands only the letter of the teachings to someone who grabs a snake by the tail, and also invokes the famous simile of the raft.%
\item[\href{\#mn23}{None}] In a curious discourse laden with evocative imagery, a deity presents a riddle to a mendicant, who seeks an answer from the Buddha.%
\item[\href{\#mn24}{None}] Venerable \textsanskrit{Sāriputta} seeks a dialog with an esteemed monk, Venerable \textsanskrit{Puṇṇa} \textsanskrit{Mantāniputta}, and they discuss the stages of purification.%
\item[\href{\#mn25}{None}] The Buddha compares getting trapped by \textsanskrit{Māra} with a deer getting caught in a snare, illustrating the ever more complex strategies employed by hunter and hunted.%
\item[\href{\#mn26}{None}] This is one of the most important biographical discourses, telling the Buddha’s experiences from leaving home to realizing awakening. Throughout, he was driven by the imperative to fully escape from rebirth and suffering.%
\item[\href{\#mn27}{None}] The Buddha cautions against swift conclusions about a teacher’s spiritual accomplishments, comparing it to the care a tracker would use when tracking elephants. He presents the full training of a monastic.%
\item[\href{\#mn28}{None}] \textsanskrit{Sāriputta} gives an elaborate demonstration of how, just as any footprint can fit inside an elephant’s, all the Buddha’s teaching can fit inside the four noble truths. This offers an overall template for organizing the Buddha’s teachings.%
\item[\href{\#mn29}{None}] Following the incident with Devadatta, the Buddha cautions the mendicants against becoming complacent with superficial benefits of spiritual life and points to liberation as the true heart of the teaching.%
\item[\href{\#mn30}{None}] Similar to the previous. After the incident with Devadatta, the Buddha cautions the mendicants against becoming complacent and points to liberation as the true heart of the teaching.%
\item[\href{\#mn{-}mahayamakavagga}{None}] Discourses arranged as pairs of longer and shorter texts.%
\item[\href{\#mn31}{None}] The Buddha comes across three mendicants practicing diligently and harmoniously, and asks them how they do it. Reluctant to disclose their higher attainments, they explain how they deal with the practical affairs of living together. But when pressed by the Buddha, they reveal their meditation attainments.%
\item[\href{\#mn32}{None}] Several senior mendicants, reveling in the beauty of the night, discuss what kind of practitioner would adorn the park. They take their answers to the Buddha, who praises their answers, but gives his own twist.%
\item[\href{\#mn33}{None}] For eleven reasons a cowherd is not able to properly look after a herd. The Buddha compares this to the spiritual growth of a mendicant.%
\item[\href{\#mn34}{None}] Drawing parallels with a cowherd guiding his herd across a dangerous river, the Buddha presents the various kinds of enlightened disciples who cross the stream of transmigration.%
\item[\href{\#mn35}{None}] Saccaka was a debater, who challenged the Buddha to a contest. Despite his bragging, the Buddha is not at all perturbed at his attacks.%
\item[\href{\#mn36}{None}] In a less confrontational meeting, the Buddha and Saccaka discuss the difference between physical and mental development. The Buddha gives a long account of the various practices he did before awakening, detailing the astonishing lengths he took to mortify the body.%
\item[\href{\#mn37}{None}] \textsanskrit{Moggallāna} visits the heaven of Sakka, the lord of gods, to see whether he really understands what the Buddha is teaching.%
\item[\href{\#mn38}{None}] To counter the wrong view that a self-identical consciousness transmigrates from one life to the next, the Buddha teaches dependent origination, showing that consciousness invariably arises dependent on conditions.%
\item[\href{\#mn39}{None}] The Buddha encourages the mendicants to live up to their name, by actually practicing in a way that meets or exceeds the expectations people have for renunciants.%
\item[\href{\#mn40}{None}] The labels of being a spiritual practitioner don’t just come from external trappings, but from sincere inner change.%
\item[\href{\#mn{-}culayamakavagga}{None}] A similar arrangement to the previous.%
\item[\href{\#mn41}{None}] The Buddha explains to a group of brahmins the conduct leading to rebirth in higher or lower states, including detailed explanations of the ten core practices which lay people should undertake, and which also form the basis for liberation.%
\item[\href{\#mn42}{None}] Similar to the previous. The Buddha explains the conduct leading to rebirth in higher or lower states, including detailed explanations of the ten core practices.%
\item[\href{\#mn43}{None}] A series of questions and answers between \textsanskrit{Sāriputta} and \textsanskrit{Mahākoṭṭhita}, examining various subtle and abstruse aspects of the teachings.%
\item[\href{\#mn44}{None}] The layman \textsanskrit{Visākha} asks the nun \textsanskrit{Dhammadinnā} about various difficult matters, including some of the highest meditation attainments. The Buddha fully endorses her answers.%
\item[\href{\#mn45}{None}] The Buddha explains how taking up different practices may have harmful or beneficial results. The memorable simile of the creeper shows how insidious temptations can be.%
\item[\href{\#mn46}{None}] While we all want to be happy, we often find the opposite happens. The Buddha explains why.%
\item[\href{\#mn47}{None}] While some spiritual teachers prefer to remain in obscurity, the Buddha not only encouraged his followers to closely investigate him, but gave them a detailed and demanding method to do so.%
\item[\href{\#mn48}{None}] Despite the Buddha’s presence, the monks of Kosambi fell into a deep and bitter dispute. The Buddha taught the reluctant monks to develop love and harmony, reminding them of the state of peace that they sought.%
\item[\href{\#mn49}{None}] The Buddha ascends to a high heavenly realm where he engages in a cosmic contest with a powerful divinity, who had fallen into the delusion that he was eternal and all-powerful.%
\item[\href{\#mn50}{None}] \textsanskrit{Māra}, the trickster and god of death, tried to annoy \textsanskrit{Moggallāna}. He not only failed but was subject to a stern sermon warning of the dangers of attacking the Buddha’s disciples.%
\end{description}

%
\mainmatter%
\pagestyle{fancy}%
\addtocontents{toc}{\let\protect\contentsline\protect\nopagecontentsline}
\part*{The First Fifty }
\addcontentsline{toc}{part}{The First Fifty }
\markboth{}{}
\addtocontents{toc}{\let\protect\contentsline\protect\oldcontentsline}

%
\addtocontents{toc}{\let\protect\contentsline\protect\nopagecontentsline}
\chapter*{The Chapter on the Root of All Things }
\addcontentsline{toc}{chapter}{\tocchapterline{The Chapter on the Root of All Things }}
\addtocontents{toc}{\let\protect\contentsline\protect\oldcontentsline}

%
\section*{{\suttatitleacronym MN 1}{\suttatitletranslation The Root of All Things }{\suttatitleroot Mūlapariyāyasutta}}
\addcontentsline{toc}{section}{\tocacronym{MN 1} \toctranslation{The Root of All Things } \tocroot{Mūlapariyāyasutta}}
\markboth{The Root of All Things }{Mūlapariyāyasutta}
\extramarks{MN 1}{MN 1}

\scevam{So\marginnote{1.1} I have heard. }At one time the Buddha was staying near \textsanskrit{Ukkaṭṭhā}, in the Subhaga Forest at the root of a magnificent sal tree. There the Buddha addressed the mendicants, “Mendicants!” 

“Venerable\marginnote{1.5} sir,” they replied. The Buddha said this: 

“Mendicants,\marginnote{2.1} I will teach you the explanation of the root of all things. Listen and pay close attention, I will speak.” 

“Yes,\marginnote{2.3} sir,” they replied. The Buddha said this: 

“Take\marginnote{3.1} an unlearned ordinary person who has not seen the noble ones, and is neither skilled nor trained in the teaching of the noble ones. They’ve not seen good persons, and are neither skilled nor trained in the teaching of the good persons. They perceive earth as earth. But then they identify with earth, they identify regarding earth, they identify as earth, they identify that ‘earth is mine’, they take pleasure in earth. Why is that? Because they haven’t completely understood it, I say. 

They\marginnote{4.1} perceive water as water. But then they identify with water … Why is that? Because they haven’t completely understood it, I say. 

They\marginnote{5.1} perceive fire as fire. But then they identify with fire … Why is that? Because they haven’t completely understood it, I say. 

They\marginnote{6.1} perceive air as air. But then they identify with air … Why is that? Because they haven’t completely understood it, I say. 

They\marginnote{7.1} perceive creatures as creatures. But then they identify with creatures … Why is that? Because they haven’t completely understood it, I say. 

They\marginnote{8.1} perceive gods as gods. But then they identify with gods … Why is that? Because they haven’t completely understood it, I say. 

They\marginnote{9.1} perceive the Creator as the Creator. But then they identify with the Creator … Why is that? Because they haven’t completely understood it, I say. 

They\marginnote{10.1} perceive \textsanskrit{Brahmā} as \textsanskrit{Brahmā}. But then they identify with \textsanskrit{Brahmā} … Why is that? Because they haven’t completely understood it, I say. 

They\marginnote{11.1} perceive the gods of streaming radiance as the gods of streaming radiance. But then they identify with the gods of streaming radiance … Why is that? Because they haven’t completely understood it, I say. 

They\marginnote{12.1} perceive the gods replete with glory as the gods replete with glory. But then they identify with the gods replete with glory … Why is that? Because they haven’t completely understood it, I say. 

They\marginnote{13.1} perceive the gods of abundant fruit as the gods of abundant fruit. But then they identify with the gods of abundant fruit … Why is that? Because they haven’t completely understood it, I say. 

They\marginnote{14.1} perceive the Overlord as the Overlord. But then they identify with the Overlord … Why is that? Because they haven’t completely understood it, I say. 

They\marginnote{15.1} perceive the dimension of infinite space as the dimension of infinite space. But then they identify with the dimension of infinite space … Why is that? Because they haven’t completely understood it, I say. 

They\marginnote{16.1} perceive the dimension of infinite consciousness as the dimension of infinite consciousness. But then they identify with the dimension of infinite consciousness … Why is that? Because they haven’t completely understood it, I say. 

They\marginnote{17.1} perceive the dimension of nothingness as the dimension of nothingness. But then they identify with the dimension of nothingness … Why is that? Because they haven’t completely understood it, I say. 

They\marginnote{18.1} perceive the dimension of neither perception nor non-perception as the dimension of neither perception nor non-perception. But then they identify with the dimension of neither perception nor non-perception … Why is that? Because they haven’t completely understood it, I say. 

They\marginnote{19.1} perceive the seen as the seen. But then they identify with the seen … Why is that? Because they haven’t completely understood it, I say. 

They\marginnote{20.1} perceive the heard as the heard. But then they identify with the heard … Why is that? Because they haven’t completely understood it, I say. 

They\marginnote{21.1} perceive the thought as the thought. But then they identify with the thought … Why is that? Because they haven’t completely understood it, I say. 

They\marginnote{22.1} perceive the known as the known. But then they identify with the known … Why is that? Because they haven’t completely understood it, I say. 

They\marginnote{23.1} perceive oneness as oneness. But then they identify with oneness … Why is that? Because they haven’t completely understood it, I say. 

They\marginnote{24.1} perceive diversity as diversity. But then they identify with diversity … Why is that? Because they haven’t completely understood it, I say. 

They\marginnote{25.1} perceive all as all. But then they identify with all … Why is that? Because they haven’t completely understood it, I say. 

They\marginnote{26.1} perceive extinguishment as extinguishment. But then they identify with extinguishment, they identify regarding extinguishment, they identify as extinguishment, they identify that ‘extinguishment is mine’, they take pleasure in extinguishment. Why is that? Because they haven’t completely understood it, I say. 

A\marginnote{27.1} mendicant who is a trainee, who hasn’t achieved their heart’s desire, but lives aspiring to the supreme sanctuary, directly knows earth as earth. But they shouldn’t identify with earth, they shouldn’t identify regarding earth, they shouldn’t identify as earth, they shouldn’t identify that ‘earth is mine’, they shouldn’t take pleasure in earth. Why is that? So that they may completely understand it, I say. 

They\marginnote{28{-}49.1} directly know water … fire … air … creatures … gods … the Creator … \textsanskrit{Brahmā} … the gods of streaming radiance … the gods replete with glory … the gods of abundant fruit … the Overlord … the dimension of infinite space … the dimension of infinite consciousness … the dimension of nothingness … the dimension of neither perception nor non-perception … the seen … the heard … the thought … the known … oneness … diversity … all … They directly know extinguishment as extinguishment. But they shouldn’t identify with extinguishment, they shouldn’t identify regarding extinguishment, they shouldn’t identify as extinguishment, they shouldn’t identify that ‘extinguishment is mine’, they shouldn’t take pleasure in extinguishment. Why is that? So that they may completely understand it, I say. 

A\marginnote{51.1} mendicant who is perfected—with defilements ended, who has completed the spiritual journey, done what had to be done, laid down the burden, achieved their own true goal, utterly ended the fetters of rebirth, and is rightly freed through enlightenment—directly knows earth as earth. But they don’t identify with earth, they don’t identify regarding earth, they don’t identify as earth, they don’t identify that ‘earth is mine’, they don’t take pleasure in earth. Why is that? Because they have completely understood it, I say. 

They\marginnote{52{-}74.1} directly know water … fire … air … creatures … gods … the Creator … \textsanskrit{Brahmā} … the gods of streaming radiance … the gods replete with glory … the gods of abundant fruit … the Overlord … the dimension of infinite space … the dimension of infinite consciousness … the dimension of nothingness … the dimension of neither perception nor non-perception … the seen … the heard … the thought … the known … oneness … diversity … all … They directly know extinguishment as extinguishment. But they don’t identify with extinguishment, they don’t identify regarding extinguishment, they don’t identify as extinguishment, they don’t identify that ‘extinguishment is mine’, they don’t take pleasure in extinguishment. Why is that? Because they have completely understood it, I say. 

A\marginnote{75.1} mendicant who is perfected—with defilements ended, who has completed the spiritual journey, done what had to be done, laid down the burden, achieved their own true goal, utterly ended the fetters of rebirth, and is rightly freed through enlightenment—directly knows earth as earth. But they don’t identify with earth, they don’t identify regarding earth, they don’t identify as earth, they don’t identify that ‘earth is mine’, they don’t take pleasure in earth. Why is that? Because they’re free of greed due to the ending of greed. 

They\marginnote{76{-}98.1} directly know water … fire … air … creatures … gods … the Creator … \textsanskrit{Brahmā} … the gods of streaming radiance … the gods replete with glory … the gods of abundant fruit … the Overlord … the dimension of infinite space … the dimension of infinite consciousness … the dimension of nothingness … the dimension of neither perception nor non-perception … the seen … the heard … the thought … the known … oneness … diversity … all … They directly know extinguishment as extinguishment. But they don’t identify with extinguishment, they don’t identify regarding extinguishment, they don’t identify as extinguishment, they don’t identify that ‘extinguishment is mine’, they don’t take pleasure in extinguishment. Why is that? Because they’re free of greed due to the ending of greed. 

A\marginnote{99.1} mendicant who is perfected—with defilements ended, who has completed the spiritual journey, done what had to be done, laid down the burden, achieved their own true goal, utterly ended the fetters of rebirth, and is rightly freed through enlightenment—directly knows earth as earth. But they don’t identify with earth, they don’t identify regarding earth, they don’t identify as earth, they don’t identify that ‘earth is mine’, they don’t take pleasure in earth. Why is that? Because they’re free of hate due to the ending of hate. 

They\marginnote{100{-}122.1} directly know water … fire … air … creatures … gods … the Creator … \textsanskrit{Brahmā} … the gods of streaming radiance … the gods replete with glory … the gods of abundant fruit … the Overlord … the dimension of infinite space … the dimension of infinite consciousness … the dimension of nothingness … the dimension of neither perception nor non-perception … the seen … the heard … the thought … the known … oneness … diversity … all … They directly know extinguishment as extinguishment. But they don’t identify with extinguishment, they don’t identify regarding extinguishment, they don’t identify as extinguishment, they don’t identify that ‘extinguishment is mine’, they don’t take pleasure in extinguishment. Why is that? Because they’re free of hate due to the ending of hate. 

A\marginnote{123.1} mendicant who is perfected—with defilements ended, who has completed the spiritual journey, done what had to be done, laid down the burden, achieved their own true goal, utterly ended the fetters of rebirth, and is rightly freed through enlightenment—directly knows earth as earth. But they don’t identify with earth, they don’t identify regarding earth, they don’t identify as earth, they don’t identify that ‘earth is mine’, they don’t take pleasure in earth. Why is that? Because they’re free of delusion due to the ending of delusion. 

They\marginnote{124{-}146.1} directly know water … fire … air … creatures … gods … the Creator … \textsanskrit{Brahmā} … the gods of streaming radiance … the gods replete with glory … the gods of abundant fruit … the Overlord … the dimension of infinite space … the dimension of infinite consciousness … the dimension of nothingness … the dimension of neither perception nor non-perception … the seen … the heard … the thought … the known … oneness … diversity … all … They directly know extinguishment as extinguishment. But they don’t identify with extinguishment, they don’t identify regarding extinguishment, they don’t identify as extinguishment, they don’t identify that ‘extinguishment is mine’, they don’t take pleasure in extinguishment. Why is that? Because they’re free of delusion due to the ending of delusion. 

The\marginnote{147.1} Realized One, the perfected one, the fully awakened Buddha directly knows earth as earth. But he doesn’t identify with earth, he doesn’t identify regarding earth, he doesn’t identify as earth, he doesn’t identify that ‘earth is mine’, he doesn’t take pleasure in earth. Why is that? Because the Realized One has completely understood it to the end, I say. 

He\marginnote{148{-}170.1} directly knows water … fire … air … creatures … gods … the Creator … \textsanskrit{Brahmā} … the gods of streaming radiance … the gods replete with glory … the gods of abundant fruit … the Overlord … the dimension of infinite space … the dimension of infinite consciousness … the dimension of nothingness … the dimension of neither perception nor non-perception … the seen … the heard … the thought … the known … oneness … diversity … all … He directly knows extinguishment as extinguishment. But he doesn’t identify with extinguishment, he doesn’t identify regarding extinguishment, he doesn’t identify as extinguishment, he doesn’t identify that ‘extinguishment is mine’, he doesn’t take pleasure in extinguishment. Why is that? Because the Realized One has completely understood it to the end, I say. 

The\marginnote{171.1} Realized One, the perfected one, the fully awakened Buddha directly knows earth as earth. But he doesn’t identify with earth, he doesn’t identify regarding earth, he doesn’t identify as earth, he doesn’t identify that ‘earth is mine’, he doesn’t take pleasure in earth. Why is that? Because he has understood that relishing is the root of suffering, and that rebirth comes from continued existence; whoever has come to be gets old and dies. That’s why the Realized One—with the ending, fading away, cessation, giving up, and letting go of all cravings—has awakened to the supreme perfect Awakening, I say. 

He\marginnote{172{-}194.1} directly knows water … fire … air … creatures … gods … the Creator … \textsanskrit{Brahmā} … the gods of streaming radiance … the gods replete with glory … the gods of abundant fruit … the Overlord … the dimension of infinite space … the dimension of infinite consciousness … the dimension of nothingness … the dimension of neither perception nor non-perception … the seen … the heard … the thought … the known … oneness … diversity … all … He directly knows extinguishment as extinguishment. But he doesn’t identify with extinguishment, he doesn’t identify regarding extinguishment, he doesn’t identify as extinguishment, he doesn’t identify that ‘extinguishment is mine’, he doesn’t take pleasure in extinguishment. Why is that? Because he has understood that relishing is the root of suffering, and that rebirth comes from continued existence; whoever has come to be gets old and dies. That’s why the Realized One—with the ending, fading away, cessation, giving up, and letting go of all cravings—has awakened to the supreme perfect Awakening, I say.” 

That\marginnote{172{-}194.30} is what the Buddha said. But the mendicants were not happy with what the Buddha said. 

%
\section*{{\suttatitleacronym MN 2}{\suttatitletranslation All the Defilements }{\suttatitleroot Sabbāsavasutta}}
\addcontentsline{toc}{section}{\tocacronym{MN 2} \toctranslation{All the Defilements } \tocroot{Sabbāsavasutta}}
\markboth{All the Defilements }{Sabbāsavasutta}
\extramarks{MN 2}{MN 2}

\scevam{So\marginnote{1.1} I have heard. }At one time the Buddha was staying near \textsanskrit{Sāvatthī} in Jeta’s Grove, \textsanskrit{Anāthapiṇḍika}’s monastery. There the Buddha addressed the mendicants, “Mendicants!” 

“Venerable\marginnote{1.5} sir,” they replied. The Buddha said this: 

“Mendicants,\marginnote{2.1} I will teach you the explanation of the restraint of all defilements. Listen and pay close attention, I will speak.” 

“Yes,\marginnote{2.3} sir,” they replied. The Buddha said this: 

“Mendicants,\marginnote{3.1} I say that the ending of defilements is for one who knows and sees, not for one who does not know or see. For one who knows and sees what? Proper attention and improper attention. When you pay improper attention, defilements arise, and once arisen they grow. When you pay proper attention, defilements don’t arise, and those that have already arisen are given up. 

Some\marginnote{4.1} defilements should be given up by seeing, some by restraint, some by using, some by enduring, some by avoiding, some by dispelling, and some by developing. 

\subsection*{1. Defilements Given Up by Seeing }

And\marginnote{5.1} what are the defilements that should be given up by seeing? Take an unlearned ordinary person who has not seen the noble ones, and is neither skilled nor trained in the teaching of the noble ones. They’ve not seen good persons, and are neither skilled nor trained in the teaching of the good persons. They don’t understand to which things they should pay attention and to which things they should not pay attention. So they pay attention to things they shouldn’t and don’t pay attention to things they should. 

And\marginnote{6.1} what are the things to which they pay attention but should not? They are the things that, when attention is paid to them, give rise to unarisen defilements and make arisen defilements grow: the defilements of sensual desire, desire to be reborn, and ignorance. These are the things to which they pay attention but should not. 

And\marginnote{6.6} what are the things to which they do not pay attention but should? They are the things that, when attention is paid to them, do not give rise to unarisen defilements and give up arisen defilements: the defilements of sensual desire, desire to be reborn, and ignorance. These are the things to which they do not pay attention but should. 

Because\marginnote{7.1} of paying attention to what they should not and not paying attention to what they should, unarisen defilements arise and arisen defilements grow. 

This\marginnote{7.2} is how they attend improperly: ‘Did I exist in the past? Did I not exist in the past? What was I in the past? How was I in the past? After being what, what did I become in the past? Will I exist in the future? Will I not exist in the future? What will I be in the future? How will I be in the future? After being what, what will I become in the future?’ Or they are undecided about the present thus: ‘Am I? Am I not? What am I? How am I? This sentient being—where did it come from? And where will it go?’ 

When\marginnote{8.1} they attend improperly in this way, one of the following six views arises in them and is taken as a genuine fact. The view: ‘My self exists in an absolute sense.’ The view: ‘My self does not exist in an absolute sense.’ The view: ‘I perceive the self with the self.’ The view: ‘I perceive what is not-self with the self.’ The view: ‘I perceive the self with what is not-self.’ Or they have such a view: ‘This self of mine is he who speaks and feels and experiences the results of good and bad deeds in all the different realms. This self is permanent, everlasting, eternal, and imperishable, and will last forever and ever.’ This is called a misconception, the thicket of views, the desert of views, the trick of views, the evasiveness of views, the fetter of views. An unlearned ordinary person who is fettered by views is not freed from rebirth, old age, and death, from sorrow, lamentation, pain, sadness, and distress. They’re not freed from suffering, I say. 

But\marginnote{9.1} take a learned noble disciple who has seen the noble ones, and is skilled and trained in the teaching of the noble ones. They’ve seen good persons, and are skilled and trained in the teaching of the good persons. They understand to which things they should pay attention and to which things they should not pay attention. So they pay attention to things they should and don’t pay attention to things they shouldn’t. 

And\marginnote{10.1} what are the things to which they don’t pay attention and should not? They are the things that, when attention is paid to them, give rise to unarisen defilements and make arisen defilements grow: the defilements of sensual desire, desire to be reborn, and ignorance. These are the things to which they don’t pay attention and should not. 

And\marginnote{10.6} what are the things to which they do pay attention and should? They are the things that, when attention is paid to them, do not give rise to unarisen defilements and give up arisen defilements: the defilements of sensual desire, desire to be reborn, and ignorance. These are the things to which they do pay attention and should. 

Because\marginnote{10.11} of not paying attention to what they should not and paying attention to what they should, unarisen defilements don’t arise and arisen defilements are given up. 

They\marginnote{11.1} properly attend: ‘This is suffering’ … ‘This is the origin of suffering’ … ‘This is the cessation of suffering’ … ‘This is the practice that leads to the cessation of suffering’. And as they do so, they give up three fetters: identity view, doubt, and misapprehension of precepts and observances. These are called the defilements that should be given up by seeing. 

\subsection*{2. Defilements Given Up by Restraint }

And\marginnote{12.1} what are the defilements that should be given up by restraint? Take a mendicant who, reflecting properly, lives restraining the faculty of the eye. For the distressing and feverish defilements that might arise in someone who lives without restraint of the eye faculty do not arise when there is such restraint. Reflecting properly, they live restraining the faculty of the ear … the nose … the tongue … the body … the mind. For the distressing and feverish defilements that might arise in someone who lives without restraint of the mind faculty do not arise when there is such restraint. 

For\marginnote{12.10} the distressing and feverish defilements that might arise in someone who lives without restraint do not arise when there is such restraint. These are called the defilements that should be given up by restraint. 

\subsection*{3. Defilements Given Up by Using }

And\marginnote{13.1} what are the defilements that should be given up by using? Take a mendicant who, reflecting properly, makes use of robes: ‘Only for the sake of warding off cold and heat; for warding off the touch of flies, mosquitoes, wind, sun, and reptiles; and for covering up the private parts.’ 

Reflecting\marginnote{14.1} properly, they make use of almsfood: ‘Not for fun, indulgence, adornment, or decoration, but only to sustain this body, to avoid harm, and to support spiritual practice. In this way, I shall put an end to old discomfort and not give rise to new discomfort, and I will live blamelessly and at ease.’ 

Reflecting\marginnote{15.1} properly, they make use of lodgings: ‘Only for the sake of warding off cold and heat; for warding off the touch of flies, mosquitoes, wind, sun, and reptiles; to shelter from harsh weather and to enjoy retreat.’ 

Reflecting\marginnote{16.1} properly, they make use of medicines and supplies for the sick: ‘Only for the sake of warding off the pains of illness and to promote good health.’ 

For\marginnote{17.1} the distressing and feverish defilements that might arise in someone who lives without using these things do not arise when they are used. These are called the defilements that should be given up by using. 

\subsection*{4. Defilements Given Up by Enduring }

And\marginnote{18.1} what are the defilements that should be given up by enduring? Take a mendicant who, reflecting properly, endures cold, heat, hunger, and thirst. They endure the touch of flies, mosquitoes, wind, sun, and reptiles. They endure rude and unwelcome criticism. And they put up with physical pain—sharp, severe, acute, unpleasant, disagreeable, and life-threatening. 

For\marginnote{18.3} the distressing and feverish defilements that might arise in someone who lives without enduring these things do not arise when they are endured. These are called the defilements that should be given up by enduring. 

\subsection*{5. Defilements Given Up by Avoiding }

And\marginnote{19.1} what are the defilements that should be given up by avoiding? Take a mendicant who, reflecting properly, avoids a wild elephant, a wild horse, a wild ox, a wild dog, a snake, a stump, thorny ground, a pit, a cliff, a swamp, and a sewer. Reflecting properly, they avoid sitting on inappropriate seats, walking in inappropriate neighborhoods, and mixing with bad friends—whatever sensible spiritual companions would believe to be a bad setting. 

For\marginnote{19.4} the distressing and feverish defilements that might arise in someone who lives without avoiding these things do not arise when they are avoided. These are called the defilements that should be given up by avoiding. 

\subsection*{6. Defilements Given Up by Dispelling }

And\marginnote{20.1} what are the defilements that should be given up by dispelling? Take a mendicant who, reflecting properly, doesn’t tolerate a sensual, malicious, or cruel thought that has arisen, but gives it up, gets rid of it, eliminates it, and obliterates it. They don’t tolerate any bad, unskillful qualities that have arisen, but give them up, get rid of them, eliminate them, and obliterate them. 

For\marginnote{20.3} the distressing and feverish defilements that might arise in someone who lives without dispelling these things do not arise when they are dispelled. These are called the defilements that should be given up by dispelling. 

\subsection*{7. Defilements Given Up by Developing }

And\marginnote{21.1} what are the defilements that should be given up by developing? It’s when a mendicant, reflecting properly, develops the awakening factors of mindfulness, investigation of principles, energy, rapture, tranquility, immersion, and equanimity, which rely on seclusion, fading away, and cessation, and ripen as letting go. 

For\marginnote{21.9} the distressing and feverish defilements that might arise in someone who lives without developing these things do not arise when they are developed. These are called the defilements that should be given up by developing. 

Now,\marginnote{22.1} take a mendicant who, by seeing, has given up the defilements that should be given up by seeing. By restraint, they’ve given up the defilements that should be given up by restraint. By using, they’ve given up the defilements that should be given up by using. By enduring, they’ve given up the defilements that should be given up by enduring. By avoiding, they’ve given up the defilements that should be given up by avoiding. By dispelling, they’ve given up the defilements that should be given up by dispelling. By developing, they’ve given up the defilements that should be given up by developing. They’re called a mendicant who lives having restrained all defilements, who has cut off craving, untied the fetters, and by rightly comprehending conceit has made an end of suffering.” 

That\marginnote{22.3} is what the Buddha said. Satisfied, the mendicants were happy with what the Buddha said. 

%
\section*{{\suttatitleacronym MN 3}{\suttatitletranslation Heirs in the Teaching }{\suttatitleroot Dhammadāyādasutta}}
\addcontentsline{toc}{section}{\tocacronym{MN 3} \toctranslation{Heirs in the Teaching } \tocroot{Dhammadāyādasutta}}
\markboth{Heirs in the Teaching }{Dhammadāyādasutta}
\extramarks{MN 3}{MN 3}

\scevam{So\marginnote{1.1} I have heard. }At one time the Buddha was staying near \textsanskrit{Sāvatthī} in Jeta’s Grove, \textsanskrit{Anāthapiṇḍika}’s monastery. There the Buddha addressed the mendicants, “Mendicants!” 

“Venerable\marginnote{1.5} sir,” they replied. The Buddha said this: 

“Mendicants,\marginnote{2.1} be my heirs in the teaching, not in material things. Out of compassion for you, I think, ‘How can my disciples become heirs in the teaching, not in material things?’ 

If\marginnote{2.4} you become heirs in material things, not in the teaching, they’ll point to you, saying, ‘The Teacher’s disciples live as heirs in material things, not in the teaching.’ And they’ll point to me, saying, ‘The Teacher’s disciples live as heirs in material things, not in the teaching.’ 

If\marginnote{2.8} you become heirs in the teaching, not in material things, they’ll point to you, saying, ‘The Teacher’s disciples live as heirs in the teaching, not in material things.’ And they’ll point to me, saying, ‘The Teacher’s disciples live as heirs in the teaching, not in material things.’ 

So,\marginnote{2.12} mendicants, be my heirs in the teaching, not in material things. Out of compassion for you, I think, ‘How can my disciples become heirs in the teaching, not in material things?’ 

Suppose\marginnote{3.1} that I had eaten and refused more food, being full, and having had as much as I needed. And there was some extra almsfood that was going to be thrown away. Then two mendicants were to come who were weak with hunger. I’d say to them, ‘Mendicants, I have eaten and refused more food, being full, and having had as much as I need. And there is this extra almsfood that’s going to be thrown away. Eat it if you like. Otherwise I’ll throw it out where there is little that grows, or drop it into water that has no living creatures.’ 

Then\marginnote{3.8} one of those mendicants thought, ‘The Buddha has eaten and refused more food. And he has some extra almsfood that’s going to be thrown away. If we don’t eat it he’ll throw it away. But the Buddha has also said: “Be my heirs in the teaching, not in material things.” And almsfood is a kind of material thing. Instead of eating this almsfood, why don’t I spend this day and night weak with hunger?’ And that’s what they did. 

Then\marginnote{3.17} the second of those mendicants thought, ‘The Buddha has eaten and refused more food. And he has some extra almsfood that’s going to be thrown away. If we don’t eat it he’ll throw it away. Why don’t I eat this almsfood, then spend the day and night having got rid of my hunger and weakness?’ And that’s what they did. 

Even\marginnote{3.23} though that mendicant, after eating the almsfood, spent the day and night rid of hunger and weakness, it is the former mendicant who is more worthy of respect and praise. Why is that? Because for a long time that will conduce to that mendicant being of few wishes, content, self-effacing, unburdensome, and energetic. 

So,\marginnote{3.26} mendicants, be my heirs in the teaching, not in material things. Out of compassion for you, I think, ‘How can my disciples become heirs in the teaching, not in material things?’” 

That\marginnote{4.1} is what the Buddha said. When he had spoken, the Holy One got up from his seat and entered his dwelling. 

Then\marginnote{4.3} soon after the Buddha left, Venerable \textsanskrit{Sāriputta} said to the mendicants, “Reverends, mendicants!” 

“Reverend,”\marginnote{4.5} they replied. \textsanskrit{Sāriputta} said this: 

“Reverends,\marginnote{5.1} how do the disciples of a Teacher who lives in seclusion not train in seclusion? And how do they train in seclusion?” 

“Reverend,\marginnote{5.2} we would travel a long way to learn the meaning of this statement in the presence of Venerable \textsanskrit{Sāriputta}. May Venerable \textsanskrit{Sāriputta} himself please clarify the meaning of this. The mendicants will listen and remember it.” 

“Well\marginnote{5.5} then, reverends, listen and pay close attention, I will speak.” 

“Yes,\marginnote{5.6} reverend,” they replied. \textsanskrit{Sāriputta} said this: 

“Reverends,\marginnote{6.1} how do the disciples of a Teacher who lives in seclusion not train in seclusion? The disciples of a teacher who lives in seclusion do not train in seclusion. They don’t give up what the Teacher tells them to give up. They’re indulgent and slack, leaders in backsliding, neglecting seclusion. In this case, the senior mendicants should be criticized on three grounds. ‘The disciples of a teacher who lives in seclusion do not train in seclusion.’ This is the first ground. ‘They don’t give up what the Teacher tells them to give up.’ This is the second ground. ‘They’re indulgent and slack, leaders in backsliding, neglecting seclusion.’ This is the third ground. The senior mendicants should be criticized on these three grounds. In this case, the middle mendicants and the junior mendicants should be criticized on the same three grounds. This is how the disciples of a Teacher who lives in seclusion do not train in seclusion. 

And\marginnote{7.1} how do the disciples of a teacher who lives in seclusion train in seclusion? The disciples of a teacher who lives in seclusion train in seclusion. They give up what the Teacher tells them to give up. They’re not indulgent and slack, leaders in backsliding, neglecting seclusion. In this case, the senior mendicants should be praised on three grounds. ‘The disciples of a teacher who lives in seclusion train in seclusion.’ This is the first ground. ‘They give up what the Teacher tells them to give up.’ This is the second ground. ‘They’re not indulgent and slack, leaders in backsliding, neglecting seclusion.’ This is the third ground. The senior mendicants should be praised on these three grounds. In this case, the middle mendicants and the junior mendicants should be praised on the same three grounds. This is how the disciples of a Teacher who lives in seclusion train in seclusion. 

The\marginnote{8.1} bad thing here is greed and hate. There is a middle way of practice for giving up greed and hate. It gives vision and knowledge, and leads to peace, direct knowledge, awakening, and extinguishment. And what is that middle way of practice? It is simply this noble eightfold path, that is: right view, right thought, right speech, right action, right livelihood, right effort, right mindfulness, and right immersion. This is that middle way of practice, which gives vision and knowledge, and leads to peace, direct knowledge, awakening, and extinguishment. 

The\marginnote{9{-}15.1} bad thing here is anger and hostility. … disdain and contempt … jealousy and stinginess … deceit and deviousness … obstinacy and aggression … conceit and arrogance … vanity and negligence. There is a middle way of practice for giving up vanity and negligence. It gives vision and knowledge, and leads to peace, direct knowledge, awakening, and extinguishment. And what is that middle way of practice? It is simply this noble eightfold path, that is: right view, right thought, right speech, right action, right livelihood, right effort, right mindfulness, and right immersion. This is that middle way of practice, which gives vision and knowledge, and leads to peace, direct knowledge, awakening, and extinguishment.” 

This\marginnote{9{-}15.13} is what Venerable \textsanskrit{Sāriputta} said. Satisfied, the mendicants were happy with what \textsanskrit{Sāriputta} said. 

%
\section*{{\suttatitleacronym MN 4}{\suttatitletranslation Fear and Dread }{\suttatitleroot Bhayabheravasutta}}
\addcontentsline{toc}{section}{\tocacronym{MN 4} \toctranslation{Fear and Dread } \tocroot{Bhayabheravasutta}}
\markboth{Fear and Dread }{Bhayabheravasutta}
\extramarks{MN 4}{MN 4}

\scevam{So\marginnote{1.1} I have heard. }At one time the Buddha was staying near \textsanskrit{Sāvatthī} in Jeta’s Grove, \textsanskrit{Anāthapiṇḍika}’s monastery. 

Then\marginnote{2.1} the brahmin \textsanskrit{Jāṇussoṇi} went up to the Buddha, and exchanged greetings with him. When the greetings and polite conversation were over, he sat down to one side and said to the Buddha: 

“Master\marginnote{2.3} Gotama, those gentlemen who have gone forth from the lay life to homelessness out of faith in Master Gotama have Master Gotama to lead the way, help them out, and give them encouragement. And those people follow Master Gotama’s example.” 

“That’s\marginnote{2.5} so true, brahmin! Everything you say is true, brahmin!” 

“But\marginnote{2.8} Master Gotama, remote lodgings in the wilderness and the forest are challenging. It’s hard to maintain seclusion and hard to find joy in solitude. The forests seem to rob the mind of a mendicant who isn’t immersed in \textsanskrit{samādhi}.” 

“That’s\marginnote{2.10} so true, brahmin! Everything you say is true, brahmin! 

Before\marginnote{3.1} my awakening—when I was still unawakened but intent on awakening—I too thought, ‘Remote lodgings in the wilderness and the forest are challenging. It’s hard to maintain seclusion and hard to find joy in solitude. The forests seem to rob the mind of a mendicant who isn’t immersed in \textsanskrit{samādhi}.’ 

Then\marginnote{4.1} I thought, ‘There are ascetics and brahmins with unpurified conduct of body, speech, and mind who frequent remote lodgings in the wilderness and the forest. Those ascetics and brahmins summon unskillful fear and dread because of these defects in their conduct. But I don’t frequent remote lodgings in the wilderness and the forest with unpurified conduct of body, speech, and mind. My conduct is purified. I am one of those noble ones who frequent remote lodgings in the wilderness and the forest with purified conduct of body, speech, and mind.’ Seeing this purity of conduct in myself I felt even more unruffled about staying in the forest. 

Then\marginnote{5{-}7.1} I thought, ‘There are ascetics and brahmins with unpurified livelihood who frequent remote lodgings in the wilderness and the forest. Those ascetics and brahmins summon unskillful fear and dread because of these defects in their livelihood. But I don’t frequent remote lodgings in the wilderness and the forest with unpurified livelihood. My livelihood is purified. I am one of those noble ones who frequent remote lodgings in the wilderness and the forest with purified livelihood.’ Seeing this purity of livelihood in myself I felt even more unruffled about staying in the forest. 

Then\marginnote{8.1} I thought, ‘There are ascetics and brahmins full of desire for sensual pleasures, with acute lust … I am not full of desire …’ 

‘There\marginnote{9.1} are ascetics and brahmins full of ill will, with malicious intentions … I have a heart full of love …’ 

‘There\marginnote{10.1} are ascetics and brahmins overcome with dullness and drowsiness … I am free of dullness and drowsiness …’ 

‘There\marginnote{11.1} are ascetics and brahmins who are restless, with no peace of mind … My mind is peaceful …’ 

‘There\marginnote{12.1} are ascetics and brahmins who are doubting and uncertain … I’ve gone beyond doubt …’ 

‘There\marginnote{13.1} are ascetics and brahmins who glorify themselves and put others down … I don’t glorify myself and put others down …’ 

‘There\marginnote{14.1} are ascetics and brahmins who are cowardly and craven … I don’t get startled …’ 

‘There\marginnote{15.1} are ascetics and brahmins who enjoy possessions, honor, and popularity … I have few wishes …’ 

‘There\marginnote{16.1} are ascetics and brahmins who are lazy and lack energy … I am energetic …’ 

‘There\marginnote{17.1} are ascetics and brahmins who are unmindful and lack situational awareness … I am mindful …’ 

‘There\marginnote{18.1} are ascetics and brahmins who lack immersion, with straying minds … I am accomplished in immersion …’ 

‘There\marginnote{19.1} are ascetics and brahmins who are witless and stupid who frequent remote lodgings in the wilderness and the forest. Those ascetics and brahmins summon unskillful fear and dread because of the defects of witlessness and stupidity. But I don’t frequent remote lodgings in the wilderness and the forest witless and stupid. I am accomplished in wisdom. I am one of those noble ones who frequent remote lodgings in the wilderness and the forest accomplished in wisdom.’ Seeing this accomplishment of wisdom in myself I felt even more unruffled about staying in the forest. 

Then\marginnote{20.1} I thought, ‘There are certain nights that are recognized as specially portentous: the fourteenth, fifteenth, and eighth of the fortnight. On such nights, why don’t I stay in awe-inspiring and hair-raising shrines in parks, forests, and trees? In such lodgings, hopefully I might see that fear and dread.’ Some time later, that’s what I did. As I was staying there a deer came by, or a peacock snapped a twig, or the wind rustled the leaves. Then I thought, ‘Is this that fear and dread coming?’ Then I thought, ‘Why do I always meditate expecting that fear and terror to come? Why don’t I get rid of that fear and dread just as it comes, while remaining just as I am?’ Then that fear and dread came upon me as I was walking. I didn’t stand still or sit down or lie down until I had got rid of that fear and dread while walking. Then that fear and dread came upon me as I was standing. I didn’t walk or sit down or lie down until I had got rid of that fear and dread while standing. Then that fear and dread came upon me as I was sitting. I didn’t lie down or stand still or walk until I had got rid of that fear and dread while sitting. Then that fear and dread came upon me as I was lying down. I didn’t sit up or stand still or walk until I had got rid of that fear and dread while lying down. 

There\marginnote{21.1} are some ascetics and brahmins who perceive that it’s day when in fact it’s night, or perceive that it’s night when in fact it’s day. This meditation of theirs is delusional, I say. I perceive that it’s night when in fact it is night, and perceive that it’s day when in fact it is day. And if there’s anyone of whom it may be rightly said that a being not liable to delusion has arisen in the world for the welfare and happiness of the people, out of compassion for the world, for the benefit, welfare, and happiness of gods and humans, it’s of me that this should be said. 

My\marginnote{22{-}26.1} energy was roused up and unflagging, my mindfulness was established and lucid, my body was tranquil and undisturbed, and my mind was immersed in \textsanskrit{samādhi}. Quite secluded from sensual pleasures, secluded from unskillful qualities, I entered and remained in the first absorption, which has the rapture and bliss born of seclusion, while placing the mind and keeping it connected. As the placing of the mind and keeping it connected were stilled, I entered and remained in the second absorption, which has the rapture and bliss born of immersion, with internal clarity and confidence, and unified mind, without placing the mind and keeping it connected. And with the fading away of rapture, I entered and remained in the third absorption, where I meditated with equanimity, mindful and aware, personally experiencing the bliss of which the noble ones declare, ‘Equanimous and mindful, one meditates in bliss.’ With the giving up of pleasure and pain, and the ending of former happiness and sadness, I entered and remained in the fourth absorption, without pleasure or pain, with pure equanimity and mindfulness. 

When\marginnote{27.1} my mind had become immersed in \textsanskrit{samādhi} like this—purified, bright, flawless, rid of corruptions, pliable, workable, steady, and imperturbable—I extended it toward recollection of past lives. I recollected many kinds of past lives. That is: one, two, three, four, five, ten, twenty, thirty, forty, fifty, a hundred, a thousand, a hundred thousand rebirths; many eons of the world contracting, many eons of the world expanding, many eons of the world contracting and expanding. I remembered: ‘There, I was named this, my clan was that, I looked like this, and that was my food. This was how I felt pleasure and pain, and that was how my life ended. When I passed away from that place I was reborn somewhere else. There, too, I was named this, my clan was that, I looked like this, and that was my food. This was how I felt pleasure and pain, and that was how my life ended. When I passed away from that place I was reborn here.’ And so I recollected my many kinds of past lives, with features and details. 

This\marginnote{28.1} was the first knowledge, which I achieved in the first watch of the night. Ignorance was destroyed and knowledge arose; darkness was destroyed and light arose, as happens for a meditator who is diligent, keen, and resolute. 

When\marginnote{29.1} my mind had become immersed in \textsanskrit{samādhi} like this—purified, bright, flawless, rid of corruptions, pliable, workable, steady, and imperturbable—I extended it toward knowledge of the death and rebirth of sentient beings. With clairvoyance that is purified and superhuman, I saw sentient beings passing away and being reborn—inferior and superior, beautiful and ugly, in a good place or a bad place. I understood how sentient beings are reborn according to their deeds: ‘These dear beings did bad things by way of body, speech, and mind. They spoke ill of the noble ones; they had wrong view; and they chose to act out of that wrong view. When their body breaks up, after death, they’re reborn in a place of loss, a bad place, the underworld, hell. These dear beings, however, did good things by way of body, speech, and mind. They never spoke ill of the noble ones; they had right view; and they chose to act out of that right view. When their body breaks up, after death, they’re reborn in a good place, a heavenly realm.’ And so, with clairvoyance that is purified and superhuman, I saw sentient beings passing away and being reborn—inferior and superior, beautiful and ugly, in a good place or a bad place. I understood how sentient beings are reborn according to their deeds. 

This\marginnote{30.1} was the second knowledge, which I achieved in the middle watch of the night. Ignorance was destroyed and knowledge arose; darkness was destroyed and light arose, as happens for a meditator who is diligent, keen, and resolute. 

When\marginnote{31.1} my mind had become immersed in \textsanskrit{samādhi} like this—purified, bright, flawless, rid of corruptions, pliable, workable, steady, and imperturbable—I extended it toward knowledge of the ending of defilements. I truly understood: ‘This is suffering’ … ‘This is the origin of suffering’ … ‘This is the cessation of suffering’ … ‘This is the practice that leads to the cessation of suffering’. I truly understood: ‘These are defilements’ … ‘This is the origin of defilements’ … ‘This is the cessation of defilements’ … ‘This is the practice that leads to the cessation of defilements’. 

Knowing\marginnote{32.1} and seeing like this, my mind was freed from the defilements of sensuality, desire to be reborn, and ignorance. When it was freed, I knew it was freed. 

I\marginnote{32.3} understood: ‘Rebirth is ended; the spiritual journey has been completed; what had to be done has been done; there is no return to any state of existence.’ 

This\marginnote{33.1} was the third knowledge, which I achieved in the final watch of the night. Ignorance was destroyed and knowledge arose; darkness was destroyed and light arose, as happens for a meditator who is diligent, keen, and resolute. 

Brahmin,\marginnote{34.1} you might think: ‘Perhaps the Master Gotama is not free of greed, hate, and delusion even today, and that is why he still frequents remote lodgings in the wilderness and the forest.’ But you should not see it like this. I see two reasons to frequent remote lodgings in the wilderness and the forest. I see a happy life for myself in the present, and I have compassion for future generations.” 

“Indeed,\marginnote{35.1} Master Gotama has compassion for future generations, since he is a perfected one, a fully awakened Buddha. Excellent, Master Gotama! Excellent, Master Gotama! As if he were righting the overturned, or revealing the hidden, or pointing out the path to the lost, or lighting a lamp in the dark so people with good eyes can see what’s there, Master Gotama has made the teaching clear in many ways. I go for refuge to Master Gotama, to the teaching, and to the mendicant \textsanskrit{Saṅgha}. From this day forth, may Master Gotama remember me as a lay follower who has gone for refuge for life.” 

%
\section*{{\suttatitleacronym MN 5}{\suttatitletranslation Unblemished }{\suttatitleroot Anaṅgaṇasutta}}
\addcontentsline{toc}{section}{\tocacronym{MN 5} \toctranslation{Unblemished } \tocroot{Anaṅgaṇasutta}}
\markboth{Unblemished }{Anaṅgaṇasutta}
\extramarks{MN 5}{MN 5}

\scevam{So\marginnote{1.1} I have heard. }At one time the Buddha was staying near \textsanskrit{Sāvatthī} in Jeta’s Grove, \textsanskrit{Anāthapiṇḍika}’s monastery. There \textsanskrit{Sāriputta} addressed the mendicants: “Reverends, mendicants!” 

“Reverend,”\marginnote{1.5} they replied. \textsanskrit{Sāriputta} said this: 

“Mendicants,\marginnote{2.1} these four people are found in the world. What four? One person with a blemish doesn’t truly understand: ‘There is a blemish in me.’ But another person with a blemish does truly understand: ‘There is a blemish in me.’ One person without a blemish doesn’t truly understand: ‘There is no blemish in me.’ But another person without a blemish does truly understand: ‘There is no blemish in me.’ In this case, of the two persons with a blemish, the one who doesn’t understand is said to be worse, while the one who does understand is better. And of the two persons without a blemish, the one who doesn’t understand is said to be worse, while the one who does understand is better.” 

When\marginnote{3.1} he said this, Venerable \textsanskrit{Mahāmoggallāna} said to him: 

“What\marginnote{3.2} is the cause, Reverend \textsanskrit{Sāriputta}, what is the reason why, of the two persons with a blemish, one is said to be worse and one better? And what is the cause, what is the reason why, of the two persons without a blemish, one is said to be worse and one better?” 

“Reverend,\marginnote{4.1} take the case of the person who has a blemish and does not understand it. You can expect that they won’t generate enthusiasm, make an effort, or rouse up energy to give up that blemish. And they will die with greed, hate, and delusion, blemished, with a corrupted mind. Suppose a bronze dish was brought from a shop or smithy covered with dirt or stains. And the owners neither used it or had it cleaned, but kept it in a dirty place. Over time, wouldn’t that bronze dish get even dirtier and more stained?” 

“Yes,\marginnote{4.6} reverend.” 

“In\marginnote{4.7} the same way, take the case of the person who has a blemish and does not understand it. You can expect that … they will die with a corrupted mind. 

Take\marginnote{5.1} the case of the person who has a blemish and does understand it. You can expect that they will generate enthusiasm, make an effort, and rouse up energy to give up that blemish. And they will die without greed, hate, and delusion, unblemished, with an uncorrupted mind. Suppose a bronze dish was brought from a shop or smithy covered with dirt or stains. But the owners used it and had it cleaned, and didn’t keep it in a dirty place. Over time, wouldn’t that bronze dish get cleaner and brighter?” 

“Yes,\marginnote{5.6} reverend.” 

“In\marginnote{5.7} the same way, take the case of the person who has a blemish and does understand it. You can expect that … they will die with an uncorrupted mind. 

Take\marginnote{6.1} the case of the person who doesn’t have a blemish but does not understand it. You can expect that they will focus on the feature of beauty, and because of that, lust will infect their mind. And they will die with greed, hate, and delusion, blemished, with a corrupted mind. Suppose a bronze dish was brought from a shop or smithy clean and bright. And the owners neither used it or had it cleaned, but kept it in a dirty place. Over time, wouldn’t that bronze dish get dirtier and more stained?” 

“Yes,\marginnote{6.6} reverend.” 

“In\marginnote{6.7} the same way, take the case of the person who has no blemish and does not understand it. You can expect that … they will die with a corrupted mind. 

Take\marginnote{7.1} the case of the person who doesn’t have a blemish and does understand it. You can expect that they won’t focus on the feature of beauty, and because of that, lust won’t infect their mind. And they will die without greed, hate, and delusion, unblemished, with an uncorrupted mind. Suppose a bronze dish was brought from a shop or smithy clean and bright. And the owners used it and had it cleaned, and didn’t keep it in a dirty place. Over time, wouldn’t that bronze dish get cleaner and brighter?” 

“Yes,\marginnote{7.6} reverend.” 

“In\marginnote{7.7} the same way, take the case of the person who doesn’t have a blemish and does understand it. You can expect that … they will die with an uncorrupted mind. 

This\marginnote{8.1} is the cause, this is the reason why, of the two persons with a blemish, one is said to be worse and one better. And this is the cause, this is the reason why, of the two persons without a blemish, one is said to be worse and one better.” 

“Reverend,\marginnote{9.1} the word ‘blemish’ is spoken of. But what is ‘blemish’ a term for?” 

“Reverend,\marginnote{9.3} ‘blemish’ is a term for the spheres of bad, unskillful wishes. 

It’s\marginnote{10.1} possible that some mendicant might wish: ‘If I commit an offense, I hope the mendicants don’t find out!’ But it’s possible that the mendicants do find out that that mendicant has committed an offense. Thinking, ‘The mendicants have found out about my offense,’ they get angry and bitter. And that anger and that bitterness are both blemishes. 

It’s\marginnote{11.1} possible that some mendicant might wish: ‘If I commit an offense, I hope the mendicants accuse me in private, not in the middle of the \textsanskrit{Saṅgha}.’ But it’s possible that the mendicants do accuse that mendicant in the middle of the \textsanskrit{Saṅgha} … 

It’s\marginnote{12.1} possible that some mendicant might wish: ‘If I commit an offense, I hope I’m accused by an equal, not by someone who is not an equal.’ But it’s possible that someone who is not an equal accuses that mendicant … 

It’s\marginnote{13.1} possible that some mendicant might wish: ‘Oh, I hope the Teacher will teach the mendicants by repeatedly questioning me alone, not some other mendicant.’ But it’s possible that the Teacher will teach the mendicants by repeatedly questioning some other mendicant … 

It’s\marginnote{14.1} possible that some mendicant might wish: ‘Oh, I hope the mendicants will enter the village for the meal putting me at the very front, not some other mendicant.’ But it’s possible that the mendicants will enter the village for the meal putting some other mendicant at the very front … 

It’s\marginnote{15.1} possible that some mendicant might wish: ‘Oh, I hope that I alone get the best seat, the best drink, and the best almsfood in the refectory, not some other mendicant.’ But it’s possible that some other mendicant gets the best seat, the best drink, and the best almsfood in the refectory … 

It’s\marginnote{16.1} possible that some mendicant might wish: ‘I hope that I alone give the verses of gratitude after eating in the refectory, not some other mendicant.’ But it’s possible that some other mendicant gives the verses of gratitude after eating in the refectory … 

It’s\marginnote{17.1} possible that some mendicant might wish: ‘Oh, I hope that I might teach the Dhamma to the monks, nuns, laymen, and laywomen in the monastery, not some other mendicant.’ 

But\marginnote{18{-}20.1} it’s possible that some other mendicant teaches the Dhamma … 

It’s\marginnote{21.1} possible that some mendicant might wish: ‘Oh, I hope that the monks, nuns, laymen, and laywomen will honor, respect, revere, and venerate me alone, not some other mendicant.’ 

But\marginnote{22{-}24.1} it’s possible that some other mendicant is honored, respected, revered, and venerated … 

It’s\marginnote{25.1} possible that some mendicant might wish: ‘I hope I get the nicest robes, almsfood, lodgings, and medicines and supplies for the sick, not some other mendicant.’ But it’s possible that some other mendicant gets the nicest robes, almsfood, lodgings, and medicines and supplies for the sick … 

Thinking,\marginnote{26{-}27.1} ‘Some other mendicant has got the nicest robes, almsfood, lodgings, and medicines and supplies for the sick’, they get angry and bitter. And that anger and that bitterness are both blemishes. 

‘Blemish’\marginnote{28.1} is a term for these spheres of bad, unskillful wishes. 

Suppose\marginnote{29.1} these spheres of bad, unskillful wishes are seen and heard to be not given up by a mendicant. Even though they dwell in the wilderness, in remote lodgings, eat only almsfood, wander indiscriminately for almsfood, wear rag robes, and wear shabby robes, their spiritual companions don’t honor, respect, revere, and venerate them. Why is that? It’s because these spheres of bad, unskillful wishes are seen and heard to be not given up by that venerable. Suppose a bronze dish was brought from a shop or smithy clean and bright. Then the owners were to prepare it with the carcass of a snake, a dog, or a human, cover it with a bronze lid, and parade it through the market-place. When people saw it they’d say: ‘My good man, what is it that you’re carrying like a precious treasure?’ So they’d open up the lid for people to look inside. But as soon as they saw it they were filled with loathing, revulsion, and disgust. Not even those who were hungry wanted to eat it, let alone those who had eaten. 

In\marginnote{29.11} the same way, when these spheres of bad, unskillful wishes are seen and heard to be not given up by a mendicant … their spiritual companions don’t honor, respect, revere, and venerate them. Why is that? It’s because these spheres of bad, unskillful wishes are seen and heard to be not given up by that venerable. 

Suppose\marginnote{30.1} these spheres of bad, unskillful wishes are seen and heard to be given up by a mendicant. Even though they dwell within a village, accept invitations to a meal, and wear robes offered by householders, their spiritual companions honor, respect, revere, and venerate them. Why is that? It’s because these spheres of bad, unskillful wishes are seen and heard to be given up by that venerable. Suppose a bronze dish was brought from a shop or smithy clean and bright. Then the owners were to prepare it with boiled fine rice with the dark grains picked out and served with many soups and sauces, cover it with a bronze lid, and parade it through the market-place. When people saw it they’d say: ‘My good man, what is it that you’re carrying like a precious treasure?’ So they’d open up the lid for people to look inside. And as soon as they saw it they were filled with liking, attraction, and relish. Even those who had eaten wanted to eat it, let alone those who were hungry. 

In\marginnote{30.11} the same way, when these spheres of bad, unskillful wishes are seen and heard to be given up by a mendicant … their spiritual companions honor, respect, revere, and venerate them. Why is that? It’s because these spheres of bad, unskillful wishes are seen and heard to be given up by that venerable.” 

When\marginnote{31.1} he said this, Venerable \textsanskrit{Mahāmoggallāna} said to him, “Reverend \textsanskrit{Sāriputta}, a simile springs to mind.” 

“Then\marginnote{31.3} speak as you feel inspired,” said \textsanskrit{Sāriputta}. 

“Reverend,\marginnote{31.4} at one time I was staying right here in \textsanskrit{Rājagaha}, the Mountainfold. Then I robed up in the morning and, taking my bowl and robe, entered \textsanskrit{Rājagaha} for alms. Now at that time \textsanskrit{Samīti} the cartwright was planing the rim of a chariot wheel. The \textsanskrit{Ājīvaka} ascetic \textsanskrit{Paṇḍuputta}, who used to be a cartwright, was standing by, and this thought came to his mind: ‘Oh, I hope \textsanskrit{Samīti} the cartwright planes out the crooks, bends, and flaws in this rim. Then the rim will be rid of crooks, bends, and flaws, and consist purely of the essential core.’ And \textsanskrit{Samīti} planed out the flaws in the rim just as \textsanskrit{Paṇḍuputta} thought. Then \textsanskrit{Paṇḍuputta} expressed his gladness: ‘He planes like he knows my heart with his heart!’ 

In\marginnote{32.1} the same way, there are those faithless people who went forth from the lay life to homelessness not out of faith but to earn a livelihood. They’re devious, deceitful, and sneaky. They’re restless, insolent, fickle, scurrilous, and loose-tongued. They do not guard their sense doors or eat in moderation, and they are not dedicated to wakefulness. They don’t care about the ascetic life, and don’t keenly respect the training. They’re indulgent and slack, leaders in backsliding, neglecting seclusion, lazy, and lacking energy. They’re unmindful, lacking situational awareness and immersion, with straying minds, witless and stupid. Venerable \textsanskrit{Sāriputta} planes their faults with this exposition of the teaching as if he knows my heart with his heart! 

But\marginnote{32.2} there are those gentlemen who went forth from the lay life to homelessness out of faith. They’re not devious, deceitful, and sneaky. They’re not restless, insolent, fickle, scurrilous, and loose-tongued. They guard their sense doors and eat in moderation, and they are dedicated to wakefulness. They care about the ascetic life, and keenly respect the training. They’re not indulgent or slack, nor are they leaders in backsliding, neglecting seclusion. They’re energetic and determined. They’re mindful, with situational awareness, immersion, and unified minds; wise, not stupid. Hearing this exposition of the teaching from Venerable \textsanskrit{Sāriputta}, they drink it up and devour it, as it were. And in speech and thought they say: ‘It’s good, sirs, that he draws his spiritual companions away from the unskillful and establishes them in the skillful.’ 

Suppose\marginnote{33.1} there was a woman or man who was young, youthful, and fond of adornments, and had bathed their head. After getting a garland of lotuses, jasmine, or liana flowers, they would take them in both hands and place them on the crown of the head. In the same way, those gentlemen who went forth from the lay life to homelessness out of faith … say: ‘It’s good, sirs, that he draws his spiritual companions away from the unskillful and establishes them in the skillful.’” And so these two spiritual giants agreed with each others’ fine words. 

%
\section*{{\suttatitleacronym MN 6}{\suttatitletranslation One Might Wish }{\suttatitleroot Ākaṅkheyyasutta}}
\addcontentsline{toc}{section}{\tocacronym{MN 6} \toctranslation{One Might Wish } \tocroot{Ākaṅkheyyasutta}}
\markboth{One Might Wish }{Ākaṅkheyyasutta}
\extramarks{MN 6}{MN 6}

\scevam{So\marginnote{1.1} I have heard. }At one time the Buddha was staying near \textsanskrit{Sāvatthī} in Jeta’s Grove, \textsanskrit{Anāthapiṇḍika}’s monastery. There the Buddha addressed the mendicants, “Mendicants!” 

“Venerable\marginnote{1.5} sir,” they replied. The Buddha said this: 

“Mendicants,\marginnote{2.1} live by the ethical precepts and the monastic code. Live restrained in the monastic code, conducting yourselves well and seeking alms in suitable places. Seeing danger in the slightest fault, keep the rules you’ve undertaken. 

A\marginnote{3.1} mendicant might wish: ‘May I be liked and approved by my spiritual companions, respected and admired.’ So let them fulfill their precepts, be committed to inner serenity of the heart, not neglect absorption, be endowed with discernment, and frequent empty huts. 

A\marginnote{4.1} mendicant might wish: ‘May I receive robes, almsfood, lodgings, and medicines and supplies for the sick.’ So let them fulfill their precepts, be committed to inner serenity of the heart, not neglect absorption, be endowed with discernment, and frequent empty huts. 

A\marginnote{5.1} mendicant might wish: ‘May the services of those whose robes, almsfood, lodgings, and medicines and supplies for the sick I enjoy be very fruitful and beneficial for them.’ So let them fulfill their precepts … 

A\marginnote{6.1} mendicant might wish: ‘When deceased family and relatives who have passed away recollect me with a confident mind, may this be very fruitful and beneficial for them.’ So let them fulfill their precepts … 

A\marginnote{7.1} mendicant might wish: ‘May I prevail over desire and discontent, and may desire and discontent not prevail over me. May I live having mastered desire and discontent whenever they arose.’ So let them fulfill their precepts … 

A\marginnote{8.1} mendicant might wish: ‘May I prevail over fear and dread, and may fear and dread not prevail over me. May I live having mastered fear and dread whenever they arose.’ So let them fulfill their precepts … 

A\marginnote{9.1} mendicant might wish: ‘May I get the four absorptions—blissful meditations in the present life that belong to the higher mind—when I want, without trouble or difficulty.’ So let them fulfill their precepts … 

A\marginnote{10.1} mendicant might wish: ‘May I have direct meditative experience of the peaceful liberations that are formless, transcending form.’ So let them fulfill their precepts … 

A\marginnote{11.1} mendicant might wish: ‘May I, with the ending of three fetters, become a stream-enterer, not liable to be reborn in the underworld, bound for awakening.’ So let them fulfill their precepts … 

A\marginnote{12.1} mendicant might wish: ‘May I, with the ending of three fetters, and the weakening of greed, hate, and delusion, become a once-returner, coming back to this world once only, then making an end of suffering.’ So let them fulfill their precepts … 

A\marginnote{13.1} mendicant might wish: ‘May I, with the ending of the five lower fetters, be reborn spontaneously and become extinguished there, not liable to return from that world.’ So let them fulfill their precepts … 

A\marginnote{14.1} mendicant might wish: ‘May I wield the many kinds of psychic power: multiplying myself and becoming one again; appearing and disappearing; going unimpeded through a wall, a rampart, or a mountain as if through space; diving in and out of the earth as if it were water; walking on water as if it were earth; flying cross-legged through the sky like a bird; touching and stroking with my hand the sun and moon, so mighty and powerful; controlling the body as far as the \textsanskrit{Brahmā} realm.’ So let them fulfill their precepts … 

A\marginnote{15.1} mendicant might wish: ‘With clairaudience that is purified and superhuman, may I hear both kinds of sounds, human and divine, whether near or far.’ So let them fulfill their precepts … 

A\marginnote{16.1} mendicant might wish: ‘May I understand the minds of other beings and individuals, having comprehended them with my mind. May I understand mind with greed as “mind with greed”, and mind without greed as “mind without greed”; mind with hate as “mind with hate”, and mind without hate as “mind without hate”; mind with delusion as “mind with delusion”, and mind without delusion as “mind without delusion”; constricted mind as “constricted mind”, and scattered mind as “scattered mind”; expansive mind as “expansive mind”, and unexpansive mind as “unexpansive mind”; mind that is not supreme as “mind that is not supreme”, and mind that is supreme as “mind that is supreme”; mind immersed in \textsanskrit{samādhi} as “mind immersed in \textsanskrit{samādhi}”, and mind not immersed in \textsanskrit{samādhi} as “mind not immersed in \textsanskrit{samādhi}”; freed mind as “freed mind”, and unfreed mind as “unfreed mind”.’ So let them fulfill their precepts … 

A\marginnote{17.1} mendicant might wish: ‘May I recollect many kinds of past lives. That is: one, two, three, four, five, ten, twenty, thirty, forty, fifty, a hundred, a thousand, a hundred thousand rebirths; many eons of the world contracting, many eons of the world expanding, many eons of the world contracting and expanding. May I remember: “There, I was named this, my clan was that, I looked like this, and that was my food. This was how I felt pleasure and pain, and that was how my life ended. When I passed away from that place I was reborn somewhere else. There, too, I was named this, my clan was that, I looked like this, and that was my food. This was how I felt pleasure and pain, and that was how my life ended. When I passed away from that place I was reborn here.” May I thus recollect my many kinds of past lives, with features and details.’ So let them fulfill their precepts … 

A\marginnote{18.1} mendicant might wish: ‘With clairvoyance that is purified and superhuman, may I see sentient beings passing away and being reborn—inferior and superior, beautiful and ugly, in a good place or a bad place—and understand how sentient beings are reborn according to their deeds: “These dear beings did bad things by way of body, speech, and mind. They spoke ill of the noble ones; they had wrong view; and they chose to act out of that wrong view. When their body breaks up, after death, they’re reborn in a place of loss, a bad place, the underworld, hell. These dear beings, however, did good things by way of body, speech, and mind. They never spoke ill of the noble ones; they had right view; and they chose to act out of that right view. When their body breaks up, after death, they’re reborn in a good place, a heavenly realm.” And so, with clairvoyance that is purified and superhuman, may I see sentient beings passing away and being reborn—inferior and superior, beautiful and ugly, in a good place or a bad place. And may I understand how sentient beings are reborn according to their deeds.’ So let them fulfill their precepts … 

A\marginnote{19.1} mendicant might wish: ‘May I realize the undefiled freedom of heart and freedom by wisdom in this very life, and live having realized it with my own insight due to the ending of defilements.’ So let them fulfill their precepts, be committed to inner serenity of the heart, not neglect absorption, be endowed with discernment, and frequent empty huts. 

‘Mendicants,\marginnote{20.1} live by the ethical precepts and the monastic code. Live restrained in the monastic code, conducting yourselves well and seeking alms in suitable places. Seeing danger in the slightest fault, keep the rules you’ve undertaken.’ That’s what I said, and this is why I said it.” 

That\marginnote{20.3} is what the Buddha said. Satisfied, the mendicants were happy with what the Buddha said. 

%
\section*{{\suttatitleacronym MN 7}{\suttatitletranslation The Simile of the Cloth }{\suttatitleroot Vatthasutta}}
\addcontentsline{toc}{section}{\tocacronym{MN 7} \toctranslation{The Simile of the Cloth } \tocroot{Vatthasutta}}
\markboth{The Simile of the Cloth }{Vatthasutta}
\extramarks{MN 7}{MN 7}

\scevam{So\marginnote{1.1} I have heard. }At one time the Buddha was staying near \textsanskrit{Sāvatthī} in Jeta’s Grove, \textsanskrit{Anāthapiṇḍika}’s monastery. There the Buddha addressed the mendicants, “Mendicants!” 

“Venerable\marginnote{1.5} sir,” they replied. The Buddha said this: 

“Suppose,\marginnote{2.1} mendicants, there was a cloth that was dirty and soiled. No matter what dye the dyer applied—whether blue or yellow or red or magenta—it would look poorly dyed and impure in color. Why is that? Because of the impurity of the cloth. 

In\marginnote{2.5} the same way, when the mind is corrupt, a bad destiny is to be expected. Suppose there was a cloth that was pure and clean. No matter what dye the dyer applied—whether blue or yellow or red or magenta—it would look well dyed and pure in color. Why is that? Because of the purity of the cloth. 

In\marginnote{2.10} the same way, when the mind isn’t corrupt, a good destiny is to be expected. 

And\marginnote{3.1} what are the corruptions of the mind? Covetousness and immoral greed, ill will, anger, hostility, disdain, contempt, jealousy, stinginess, deceit, deviousness, obstinacy, aggression, conceit, arrogance, vanity, and negligence are corruptions of the mind. 

A\marginnote{4.1} mendicant who understands that covetousness and immoral greed are corruptions of the mind gives them up. A mendicant who understands that ill will … negligence is a corruption of the mind gives it up. 

When\marginnote{5.1} they have understood these corruptions of the mind for what they are, and have given them up, they have experiential confidence in the Buddha: ‘That Blessed One is perfected, a fully awakened Buddha, accomplished in knowledge and conduct, holy, knower of the world, supreme guide for those who wish to train, teacher of gods and humans, awakened, blessed.’ 

They\marginnote{6.1} have experiential confidence in the teaching: ‘The teaching is well explained by the Buddha—visible in this very life, immediately effective, inviting inspection, relevant, so that sensible people can know it for themselves.’ 

They\marginnote{7.1} have experiential confidence in the \textsanskrit{Saṅgha}: ‘The \textsanskrit{Saṅgha} of the Buddha’s disciples is practicing the way that’s good, direct, methodical, and proper. It consists of the four pairs, the eight individuals. This is the \textsanskrit{Saṅgha} of the Buddha’s disciples that is worthy of offerings dedicated to the gods, worthy of hospitality, worthy of a religious donation, worthy of greeting with joined palms, and is the supreme field of merit for the world.’ 

When\marginnote{8.1} a mendicant has discarded, eliminated, released, given up, and relinquished to this extent, thinking, ‘I have experiential confidence in the Buddha … the teaching … the \textsanskrit{Saṅgha},’ they find inspiration in the meaning and the teaching, and find joy connected with the teaching. Thinking: ‘I have discarded, eliminated, released, given up, and relinquished to this extent,’ they find inspiration in the meaning and the teaching, and find joy connected with the teaching. When they’re joyful, rapture springs up. When the mind is full of rapture, the body becomes tranquil. When the body is tranquil, they feel bliss. And when they’re blissful, the mind becomes immersed in \textsanskrit{samādhi}. 

When\marginnote{12.1} a mendicant of such ethics, such qualities, and such wisdom eats boiled fine rice with the dark grains picked out and served with many soups and sauces, that is no obstacle for them. Compare with cloth that is dirty and soiled; it can be made pure and clean by pure water. Or unrefined gold, which can be made pure and bright by a forge. In the same way, when a mendicant of such ethics, such qualities, and such wisdom eats boiled fine rice with the dark grains picked out and served with many soups and sauces, that is no obstacle for them. 

They\marginnote{13.1} meditate spreading a heart full of love to one direction, and to the second, and to the third, and to the fourth. In the same way above, below, across, everywhere, all around, they spread a heart full of love to the whole world—abundant, expansive, limitless, free of enmity and ill will. They meditate spreading a heart full of compassion to one direction, and to the second, and to the third, and to the fourth. In the same way above, below, across, everywhere, all around, they spread a heart full of compassion to the whole world—abundant, expansive, limitless, free of enmity and ill will. They meditate spreading a heart full of rejoicing to one direction, and to the second, and to the third, and to the fourth. In the same way above, below, across, everywhere, all around, they spread a heart full of rejoicing to the whole world—abundant, expansive, limitless, free of enmity and ill will. They meditate spreading a heart full of equanimity to one direction, and to the second, and to the third, and to the fourth. In the same way above, below, across, everywhere, all around, they spread a heart full of equanimity to the whole world—abundant, expansive, limitless, free of enmity and ill will. 

They\marginnote{17.1} understand: ‘There is this, there is what is worse than this, there is what is better than this, and there is an escape beyond the scope of perception.’ 

Knowing\marginnote{18.1} and seeing like this, their mind is freed from the defilements of sensuality, desire to be reborn, and ignorance. When they’re freed, they know they’re freed. 

They\marginnote{18.3} understand: ‘Rebirth is ended, the spiritual journey has been completed, what had to be done has been done, there is no return to any state of existence.’ This is called a mendicant who is bathed with the inner bathing.” 

Now\marginnote{19.1} at that time the brahmin \textsanskrit{Sundarikabhāradvāja} was sitting not far from the Buddha. He said to the Buddha, “But does Master Gotama go to the river \textsanskrit{Bāhuka} to bathe?” 

“Brahmin,\marginnote{19.4} why go to the river \textsanskrit{Bāhuka}? What can the river \textsanskrit{Bāhuka} do?” 

“Many\marginnote{19.6} people agree that the river \textsanskrit{Bāhuka} bestows cleanliness and merit. And many people wash off their bad deeds in the river \textsanskrit{Bāhuka}.” 

Then\marginnote{20.1} the Buddha addressed \textsanskrit{Sundarikabhāradvāja} in verse: 

\begin{verse}%
“The\marginnote{20.2} \textsanskrit{Bāhuka} and the Adhikakka, \\
the \textsanskrit{Gayā} and the \textsanskrit{Sundarikā} too, \\
\textsanskrit{Sarasvatī} and \textsanskrit{Payāga}, \\
and the river \textsanskrit{Bāhumati}: \\
a fool can constantly plunge into them \\
but it won’t purify their dark deeds. 

What\marginnote{20.8} can the \textsanskrit{Sundarikā} do? \\
What the \textsanskrit{Payāga} or the \textsanskrit{Bāhuka}? \\
They can’t cleanse a cruel and criminal person \\
from their bad deeds. 

For\marginnote{20.12} the pure in heart it’s always \\
the spring festival or the sabbath. \\
For the pure in heart and clean of deed, \\
their vows will always be fulfilled. \\
It’s here alone that you should bathe, brahmin, \\
making yourself a sanctuary for all creatures. 

And\marginnote{20.18} if you speak no lies, \\
nor harm any living creature, \\
nor steal anything not given, \\
and you’re faithful and not stingy: \\
what’s the point of going to \textsanskrit{Gayā}? \\
For any well may be your \textsanskrit{Gayā}!” 

%
\end{verse}

When\marginnote{21.1} he had spoken, the brahmin \textsanskrit{Sundarikabhāradvāja} said to the Buddha, “Excellent, Master Gotama! Excellent! As if he were righting the overturned, or revealing the hidden, or pointing out the path to the lost, or lighting a lamp in the dark so people with good eyes can see what’s there, Master Gotama has made the teaching clear in many ways. I go for refuge to Master Gotama, to the teaching, and to the mendicant \textsanskrit{Saṅgha}. Sir, may I receive the going forth, the ordination in the Buddha’s presence?” 

And\marginnote{22.1} the brahmin \textsanskrit{Sundarikabhāradvāja} received the going forth, the ordination in the Buddha’s presence. Not long after his ordination, Venerable \textsanskrit{Bhāradvāja}, living alone, withdrawn, diligent, keen, and resolute, soon realized the supreme end of the spiritual path in this very life. He lived having achieved with his own insight the goal for which gentlemen rightly go forth from the lay life to homelessness. 

He\marginnote{22.3} understood: “Rebirth is ended; the spiritual journey has been completed; what had to be done has been done; there is no return to any state of existence.” And Venerable \textsanskrit{Bhāradvāja} became one of the perfected. 

%
\section*{{\suttatitleacronym MN 8}{\suttatitletranslation Self-Effacement }{\suttatitleroot Sallekhasutta}}
\addcontentsline{toc}{section}{\tocacronym{MN 8} \toctranslation{Self-Effacement } \tocroot{Sallekhasutta}}
\markboth{Self-Effacement }{Sallekhasutta}
\extramarks{MN 8}{MN 8}

\scevam{So\marginnote{1.1} I have heard. }At one time the Buddha was staying near \textsanskrit{Sāvatthī} in Jeta’s Grove, \textsanskrit{Anāthapiṇḍika}’s monastery. 

Then\marginnote{2.1} in the late afternoon, Venerable \textsanskrit{Mahācunda} came out of retreat and went to the Buddha. He bowed, sat down to one side, and said to the Buddha: 

“Sir,\marginnote{3.1} there are many different views that arise in the world connected with doctrines of the self or with doctrines of the cosmos. How does a mendicant who is focusing on the starting point give up and let go of these views?” 

“Cunda,\marginnote{3.4} there are many different views that arise in the world connected with doctrines of the self or with doctrines of the cosmos. A mendicant gives up and lets go of these views by truly seeing with right wisdom where they arise, where they settle in, and where they operate as: ‘This is not mine, I am not this, this is not my self.’ 

It’s\marginnote{4.1} possible that a certain mendicant, quite secluded from sensual pleasures, secluded from unskillful qualities, might enter and remain in the first absorption, which has the rapture and bliss born of seclusion, while placing the mind and keeping it connected. They might think they’re practicing self-effacement. But in the training of the Noble One these are not called ‘self-effacement’; they’re called ‘blissful meditations in the present life’. 

It’s\marginnote{5.1} possible that some mendicant, as the placing of the mind and keeping it connected are stilled, might enter and remain in the second absorption, which has the rapture and bliss born of immersion, with internal clarity and confidence, and unified mind, without placing the mind and keeping it connected. They might think they’re practicing self-effacement. But in the training of the Noble One these are not called ‘self-effacement’; they’re called ‘blissful meditations in the present life’. 

It’s\marginnote{6.1} possible that some mendicant, with the fading away of rapture, might enter and remain in the third absorption, where they meditate with equanimity, mindful and aware, personally experiencing the bliss of which the noble ones declare, ‘Equanimous and mindful, one meditates in bliss.’ They might think they’re practicing self-effacement. But in the training of the Noble One these are not called ‘self-effacement’; they’re called ‘blissful meditations in the present life’. 

It’s\marginnote{7.1} possible that some mendicant, with the giving up of pleasure and pain, and the ending of former happiness and sadness, might enter and remain in the fourth absorption, without pleasure or pain, with pure equanimity and mindfulness. They might think they’re practicing self-effacement. But in the training of the Noble One these are not called ‘self-effacement’; they’re called ‘blissful meditations in the present life’. 

It’s\marginnote{8.1} possible that some mendicant, going totally beyond perceptions of form, with the ending of perceptions of impingement, not focusing on perceptions of diversity, aware that ‘space is infinite’, might enter and remain in the dimension of infinite space. They might think they’re practicing self-effacement. But in the training of the Noble One these are not called ‘self-effacement’; they’re called ‘peaceful meditations’. 

It’s\marginnote{9.1} possible that some mendicant, going totally beyond the dimension of infinite space, aware that ‘consciousness is infinite’, might enter and remain in the dimension of infinite consciousness. They might think they’re practicing self-effacement. But in the training of the Noble One these are not called ‘self-effacement’; they’re called ‘peaceful meditations’. 

It’s\marginnote{10.1} possible that some mendicant, going totally beyond the dimension of infinite consciousness, aware that ‘there is nothing at all’, might enter and remain in the dimension of nothingness. They might think they’re practicing self-effacement. But in the training of the Noble One these are not called ‘self-effacement’; they’re called ‘peaceful meditations’. 

It’s\marginnote{11.1} possible that some mendicant, going totally beyond the dimension of nothingness, might enter and remain in the dimension of neither perception nor non-perception. They might think they’re practicing self-effacement. But in the training of the Noble One these are not called ‘self-effacement’; they’re called ‘peaceful meditations’. 

\subsection*{1. The Exposition of Self-Effacement }

Now,\marginnote{12.1} Cunda, you should work on self-effacement in each of the following ways. 

‘Others\marginnote{12.2} will be cruel, but here we will not be cruel.’ 

‘Others\marginnote{12.3} will kill living creatures, but here we will not kill living creatures.’ 

‘Others\marginnote{12.4} will steal, but here we will not steal.’ 

‘Others\marginnote{12.5} will be unchaste, but here we will not be unchaste.’ 

‘Others\marginnote{12.6} will lie, but here we will not lie.’ 

‘Others\marginnote{12.7} will speak divisively, but here we will not speak divisively.’ 

‘Others\marginnote{12.8} will speak harshly, but here we will not speak harshly.’ 

‘Others\marginnote{12.9} will talk nonsense, but here we will not talk nonsense.’ 

‘Others\marginnote{12.10} will be covetous, but here we will not be covetous.’ 

‘Others\marginnote{12.11} will have ill will, but here we will not have ill will.’ 

‘Others\marginnote{12.12} will have wrong view, but here we will have right view.’ 

‘Others\marginnote{12.13} will have wrong thought, but here we will have right thought.’ 

‘Others\marginnote{12.14} will have wrong speech, but here we will have right speech.’ 

‘Others\marginnote{12.15} will have wrong action, but here we will have right action.’ 

‘Others\marginnote{12.16} will have wrong livelihood, but here we will have right livelihood.’ 

‘Others\marginnote{12.17} will have wrong effort, but here we will have right effort.’ 

‘Others\marginnote{12.18} will have wrong mindfulness, but here we will have right mindfulness.’ 

‘Others\marginnote{12.19} will have wrong immersion, but here we will have right immersion.’ 

‘Others\marginnote{12.20} will have wrong knowledge, but here we will have right knowledge.’ 

‘Others\marginnote{12.21} will have wrong freedom, but here we will have right freedom.’ 

‘Others\marginnote{12.22} will be overcome with dullness and drowsiness, but here we will be rid of dullness and drowsiness.’ 

‘Others\marginnote{12.23} will be restless, but here we will not be restless.’ 

‘Others\marginnote{12.24} will have doubts, but here we will have gone beyond doubt.’ 

‘Others\marginnote{12.25} will be irritable, but here we will be without anger.’ 

‘Others\marginnote{12.26} will be hostile, but here we will be without hostility.’ 

‘Others\marginnote{12.27} will be offensive, but here we will be inoffensive.’ 

‘Others\marginnote{12.28} will be contemptuous, but here we will be without contempt.’ 

‘Others\marginnote{12.29} will be jealous, but here we will be without jealousy.’ 

‘Others\marginnote{12.30} will be stingy, but here we will be without stinginess.’ 

‘Others\marginnote{12.31} will be devious, but here we will not be devious.’ 

‘Others\marginnote{12.32} will be deceitful, but here we will not be deceitful.’ 

‘Others\marginnote{12.33} will be pompous, but here we will not be pompous.’ 

‘Others\marginnote{12.34} will be arrogant, but here we will not be arrogant.’ 

‘Others\marginnote{12.35} will be hard to admonish, but here we will not be hard to admonish.’ 

‘Others\marginnote{12.36} will have bad friends, but here we will have good friends.’ 

‘Others\marginnote{12.37} will be negligent, but here we will be diligent.’ 

‘Others\marginnote{12.38} will be faithless, but here we will have faith.’ 

‘Others\marginnote{12.39} will be conscienceless, but here we will have a sense of conscience.’ 

‘Others\marginnote{12.40} will be imprudent, but here we will be prudent.’ 

‘Others\marginnote{12.41} will be unlearned, but here we will be well educated.’ 

‘Others\marginnote{12.42} will be lazy, but here we will be energetic.’ 

‘Others\marginnote{12.43} will be unmindful, but here we will be mindful.’ 

‘Others\marginnote{12.44} will be witless, but here we will be accomplished in wisdom.’ 

‘Others\marginnote{12.45} will be attached to their own views, holding them tight, and refusing to let go, but here we will not be attached to our own views, not holding them tight, but will let them go easily.’ 

\subsection*{2. Giving Rise to the Thought }

Cunda,\marginnote{13.1} I say that even giving rise to the thought of skillful qualities is very helpful, let alone following that path in body and speech. That’s why you should give rise to the following thoughts. ‘Others will be cruel, but here we will not be cruel.’ ‘Others will kill living creatures, but here we will not kill living creatures.’ … ‘Others will be attached to their own views, holding them tight, and refusing to let go, but here we will not be attached to our own views, not holding them tight, but will let them go easily.’ 

\subsection*{3. A Way Around }

Cunda,\marginnote{14.1} suppose there was a rough path and another smooth path to get around it. Or suppose there was a rough ford and another smooth ford to get around it. In the same way, a cruel individual gets around it by not being cruel. An individual who kills gets around it by not killing. … 

An\marginnote{14.5} individual who is attached to their own views, holding them tight, and refusing to let go, gets around it by not being attached to their own views, not holding them tight, but letting them go easily. 

\subsection*{4. Going Up }

Cunda,\marginnote{15.1} all unskillful qualities lead downwards, while all skillful qualities lead upwards. In the same way, a cruel individual is led upwards by not being cruel. An individual who kills is led upwards by not killing … An individual who is attached to their own views, holding them tight, and refusing to let go, is led upwards by not being attached to their own views, not holding them tight, but letting them go easily. 

\subsection*{5. The Exposition by Extinguishment }

Truly,\marginnote{16.1} Cunda, if you’re sinking down in the mud you can’t pull out someone else who is also sinking down in the mud. But if you’re not sinking down in the mud you can pull out someone else who is sinking down in the mud. Truly, if you’re not tamed, trained, and extinguished you can’t tame, train, and extinguish someone else. But if you’re tamed, trained, and extinguished you can tame, train, and extinguish someone else. 

In\marginnote{16.5} the same way, a cruel individual extinguishes it by not being cruel. An individual who kills extinguishes it by not killing. … 

An\marginnote{16.24} individual who is attached to their own views, holding them tight, and refusing to let go, extinguishes it by not being attached to their own views, not holding them tight, but letting them go easily. 

So,\marginnote{17.1} Cunda, I’ve taught the expositions by way of self-effacement, giving rise to thought, the way around, going up, and extinguishing. Out of compassion, I’ve done what a teacher should do who wants what’s best for their disciples. Here are these roots of trees, and here are these empty huts. Practice absorption, Cunda! Don’t be negligent! Don’t regret it later! This is my instruction.” 

That\marginnote{17.4} is what the Buddha said. Satisfied, Venerable \textsanskrit{Mahācunda} was happy with what the Buddha said. 

\begin{verse}%
Forty-four\marginnote{17.6} items have been stated, \\
organized into five sections. \\
“Effacement” is the name of this discourse, \\
which is deep as the ocean. 

%
\end{verse}

%
\section*{{\suttatitleacronym MN 9}{\suttatitletranslation Right View }{\suttatitleroot Sammādiṭṭhisutta}}
\addcontentsline{toc}{section}{\tocacronym{MN 9} \toctranslation{Right View } \tocroot{Sammādiṭṭhisutta}}
\markboth{Right View }{Sammādiṭṭhisutta}
\extramarks{MN 9}{MN 9}

\scevam{So\marginnote{1.1} I have heard. }At one time the Buddha was staying near \textsanskrit{Sāvatthī} in Jeta’s Grove, \textsanskrit{Anāthapiṇḍika}’s monastery. There \textsanskrit{Sāriputta} addressed the mendicants: “Reverends, mendicants!” 

“Reverend,”\marginnote{1.5} they replied. \textsanskrit{Sāriputta} said this: 

“Reverends,\marginnote{2.1} they speak of this thing called ‘right view’. How do you define a noble disciple who has right view, whose view is correct, who has experiential confidence in the teaching, and has come to the true teaching?” 

“Reverend,\marginnote{2.3} we would travel a long way to learn the meaning of this statement in the presence of Venerable \textsanskrit{Sāriputta}. May Venerable \textsanskrit{Sāriputta} himself please clarify the meaning of this. The mendicants will listen and remember it.” 

“Well\marginnote{2.6} then, reverends, listen and pay close attention, I will speak.” 

“Yes,\marginnote{2.7} reverend,” they replied. \textsanskrit{Sāriputta} said this: 

“A\marginnote{3.1} noble disciple understands the unskillful and its root, and the skillful and its root. When they’ve done this, they’re defined as a noble disciple who has right view, whose view is correct, who has experiential confidence in the teaching, and has come to the true teaching. 

But\marginnote{4.1} what is the unskillful and what is its root? And what is the skillful and what is its root? Killing living creatures, stealing, and sexual misconduct; speech that’s false, divisive, harsh, or nonsensical; and covetousness, ill will, and wrong view. This is called the unskillful. 

And\marginnote{5.1} what is the root of the unskillful? Greed, hate, and delusion. This is called the root of the unskillful. 

And\marginnote{6.1} what is the skillful? Avoiding killing living creatures, stealing, and sexual misconduct; avoiding speech that’s false, divisive, harsh, or nonsensical; contentment, good will, and right view. This is called the skillful. 

And\marginnote{7.1} what is the root of the skillful? Contentment, love, and understanding. This is called the root of the skillful. 

A\marginnote{8.1} noble disciple understands in this way the unskillful and its root, and the skillful and its root. They’ve completely given up the underlying tendency to greed, got rid of the underlying tendency to repulsion, and eradicated the underlying tendency to the view and conceit ‘I am’. They’ve given up ignorance and given rise to knowledge, and make an end of suffering in this very life. When they’ve done this, they’re defined as a noble disciple who has right view, whose view is correct, who has experiential confidence in the teaching, and has come to the true teaching.” 

Saying\marginnote{9.1} “Good, sir,” those mendicants approved and agreed with what \textsanskrit{Sāriputta} said. Then they asked another question: “But reverend, might there be another way to describe a noble disciple who has right view, whose view is correct, who has experiential confidence in the teaching, and has come to the true teaching?” 

“There\marginnote{10.1} might, reverends. A noble disciple understands fuel, its origin, its cessation, and the practice that leads to its cessation. When they’ve done this, they’re defined as a noble disciple who has right view, whose view is correct, who has experiential confidence in the teaching, and has come to the true teaching. 

But\marginnote{11.1} what is fuel? What is its origin, its cessation, and the practice that leads to its cessation? There are these four fuels. They maintain sentient beings that have been born and help those that are about to be born. What four? Solid food, whether coarse or fine; contact is the second, mental intention the third, and consciousness the fourth. Fuel originates from craving. Fuel ceases when craving ceases. The practice that leads to the cessation of fuel is simply this noble eightfold path, that is: right view, right thought, right speech, right action, right livelihood, right effort, right mindfulness, and right immersion. 

A\marginnote{12.1} noble disciple understands in this way fuel, its origin, its cessation, and the practice that leads to its cessation. They’ve completely given up the underlying tendency to greed, got rid of the underlying tendency to repulsion, and eradicated the underlying tendency to the view and conceit ‘I am’. They’ve given up ignorance and given rise to knowledge, and make an end of suffering in this very life. When they’ve done this, they’re defined as a noble disciple who has right view, whose view is correct, who has experiential confidence in the teaching, and has come to the true teaching.” 

Saying\marginnote{13.1} “Good, sir,” those mendicants … asked another question: “But reverend, might there be another way to describe a noble disciple who … has come to the true teaching?” 

“There\marginnote{14{-}18.1} might, reverends. A noble disciple understands suffering, its origin, its cessation, and the practice that leads to its cessation. When they’ve done this, they’re defined as a noble disciple who … has come to the true teaching. But what is suffering? What is its origin, its cessation, and the practice that leads to its cessation? Rebirth is suffering; old age is suffering; death is suffering; sorrow, lamentation, pain, sadness, and distress are suffering; association with the disliked is suffering; separation from the liked is suffering; not getting what you wish for is suffering. In brief, the five grasping aggregates are suffering. This is called suffering. And what is the origin of suffering? It’s the craving that leads to future lives, mixed up with relishing and greed, chasing pleasure in various realms. That is, craving for sensual pleasures, craving for continued existence, and craving to end existence. This is called the origin of suffering. And what is the cessation of suffering? It’s the fading away and cessation of that very same craving with nothing left over; giving it away, letting it go, releasing it, and not adhering to it. This is called the cessation of suffering. And what is the practice that leads to the cessation of suffering? It is simply this noble eightfold path, that is: right view … right immersion. This is called the practice that leads to the cessation of suffering. 

A\marginnote{19.1} noble disciple understands in this way suffering, its origin, its cessation, and the practice that leads to its cessation. They’ve completely given up the underlying tendency to greed, got rid of the underlying tendency to repulsion, and eradicated the underlying tendency to the view and conceit ‘I am’. They’ve given up ignorance and given rise to knowledge, and make an end of suffering in this very life. When they’ve done this, they’re defined as a noble disciple who has right view, whose view is correct, who has experiential confidence in the teaching, and has come to the true teaching.” 

Saying\marginnote{20.1} “Good, sir,” those mendicants … asked another question: “But reverend, might there be another way to describe a noble disciple who … has come to the true teaching?” 

“There\marginnote{21{-}22.1} might, reverends. A noble disciple understands old age and death, their origin, their cessation, and the practice that leads to their cessation … But what are old age and death? What is their origin, their cessation, and the practice that leads to their cessation? The old age, decrepitude, broken teeth, gray hair, wrinkly skin, diminished vitality, and failing faculties of the various sentient beings in the various orders of sentient beings. This is called old age. And what is death? The passing away, perishing, disintegration, demise, mortality, death, decease, breaking up of the aggregates, laying to rest of the corpse, and cutting off of the life faculty of the various sentient beings in the various orders of sentient beings. This is called death. Such is old age, and such is death. This is called old age and death. Old age and death originate from rebirth. Old age and death cease when rebirth ceases. The practice that leads to the cessation of old age and death is simply this noble eightfold path …” 

“Might\marginnote{24{-}26.1} there be another way to describe a noble disciple?” 

“There\marginnote{24{-}26.3} might, reverends. A noble disciple understands rebirth, its origin, its cessation, and the practice that leads to its cessation … But what is rebirth? What is its origin, its cessation, and the practice that leads to its cessation? The rebirth, inception, conception, reincarnation, manifestation of the aggregates, and acquisition of the sense fields of the various sentient beings in the various orders of sentient beings. This is called rebirth. Rebirth originates from continued existence. Rebirth ceases when continued existence ceases. The practice that leads to the cessation of rebirth is simply this noble eightfold path …” 

“Might\marginnote{28{-}30.1} there be another way to describe a noble disciple?” 

“There\marginnote{28{-}30.3} might, reverends. A noble disciple understands continued existence, its origin, its cessation, and the practice that leads to its cessation. But what is continued existence? What is its origin, its cessation, and the practice that leads to its cessation? There are these three states of continued existence. Existence in the sensual realm, the realm of luminous form, and the formless realm. Continued existence originates from grasping. Continued existence ceases when grasping ceases. The practice that leads to the cessation of continued existence is simply this noble eightfold path …” 

“Might\marginnote{32{-}34.1} there be another way to describe a noble disciple?” 

“There\marginnote{32{-}34.3} might, reverends. A noble disciple understands grasping, its origin, its cessation, and the practice that leads to its cessation … But what is grasping? What is its origin, its cessation, and the practice that leads to its cessation? There are these four kinds of grasping. Grasping at sensual pleasures, views, precepts and observances, and theories of a self. Grasping originates from craving. Grasping ceases when craving ceases. The practice that leads to the cessation of grasping is simply this noble eightfold path …” 

“Might\marginnote{36{-}38.1} there be another way to describe a noble disciple?” 

“There\marginnote{36{-}38.3} might, reverends. A noble disciple understands craving, its origin, its cessation, and the practice that leads to its cessation … But what is craving? What is its origin, its cessation, and the practice that leads to its cessation? There are these six classes of craving. Craving for sights, sounds, smells, tastes, touches, and thoughts. Craving originates from feeling. Craving ceases when feeling ceases. The practice that leads to the cessation of craving is simply this noble eightfold path …” 

“Might\marginnote{40{-}42.1} there be another way to describe a noble disciple?” 

“There\marginnote{40{-}42.3} might, reverends. A noble disciple understands feeling, its origin, its cessation, and the practice that leads to its cessation … But what is feeling? What is its origin, its cessation, and the practice that leads to its cessation? There are these six classes of feeling. Feeling born of contact through the eye, ear, nose, tongue, body, and mind. Feeling originates from contact. Feeling ceases when contact ceases. The practice that leads to the cessation of feeling is simply this noble eightfold path …” 

“Might\marginnote{44{-}46.1} there be another way to describe a noble disciple?” 

“There\marginnote{44{-}46.3} might, reverends. A noble disciple understands contact, its origin, its cessation, and the practice that leads to its cessation … But what is contact? What is its origin, its cessation, and the practice that leads to its cessation? There are these six classes of contact. Contact through the eye, ear, nose, tongue, body, and mind. Contact originates from the six sense fields. Contact ceases when the six sense fields cease. The practice that leads to the cessation of contact is simply this noble eightfold path …” 

“Might\marginnote{48{-}50.1} there be another way to describe a noble disciple?” 

“There\marginnote{48{-}50.3} might, reverends. A noble disciple understands the six sense fields, their origin, their cessation, and the practice that leads to their cessation … But what are the six sense fields? What is their origin, their cessation, and the practice that leads to their cessation? There are these six sense fields. The sense fields of the eye, ear, nose, tongue, body, and mind. The six sense fields originate from name and form. The six sense fields cease when name and form cease. The practice that leads to the cessation of the six sense fields is simply this noble eightfold path …” 

“Might\marginnote{52{-}54.1} there be another way to describe a noble disciple?” 

“There\marginnote{52{-}54.3} might, reverends. A noble disciple understands name and form, their origin, their cessation, and the practice that leads to their cessation … But what are name and form? What is their origin, their cessation, and the practice that leads to their cessation? Feeling, perception, intention, contact, and attention—this is called name. The four primary elements, and form derived from the four primary elements—this is called form. Such is name and such is form. This is called name and form. Name and form originate from consciousness. Name and form cease when consciousness ceases. The practice that leads to the cessation of name and form is simply this noble eightfold path …” 

“Might\marginnote{56{-}58.1} there be another way to describe a noble disciple?” 

“There\marginnote{56{-}58.3} might, reverends. A noble disciple understands consciousness, its origin, its cessation, and the practice that leads to its cessation … But what is consciousness? What is its origin, its cessation, and the practice that leads to its cessation? There are these six classes of consciousness. Eye, ear, nose, tongue, body, and mind consciousness. Consciousness originates from choices. Consciousness ceases when choices cease. The practice that leads to the cessation of consciousness is simply this noble eightfold path …” 

“Might\marginnote{60{-}62.1} there be another way to describe a noble disciple?” 

“There\marginnote{60{-}62.4} might, reverends. A noble disciple understands choices, their origin, their cessation, and the practice that leads to their cessation … But what are choices? What is their origin, their cessation, and the practice that leads to their cessation? There are these three kinds of choice. Choices by way of body, speech, and mind. Choices originate from ignorance. Choices cease when ignorance ceases. The practice that leads to the cessation of choices is simply this noble eightfold path …” 

“Might\marginnote{64{-}66.1} there be another way to describe a noble disciple?” 

“There\marginnote{64{-}66.3} might, reverends. A noble disciple understands ignorance, its origin, its cessation, and the practice that leads to its cessation … But what is ignorance? What is its origin, its cessation, and the practice that leads to its cessation? Not knowing about suffering, the origin of suffering, the cessation of suffering, and the practice that leads to the cessation of suffering. This is called ignorance. Ignorance originates from defilement. Ignorance ceases when defilement ceases. The practice that leads to the cessation of ignorance is simply this noble eightfold path …” 

Saying\marginnote{68.1} “Good, sir,” those mendicants approved and agreed with what \textsanskrit{Sāriputta} said. Then they asked another question: “But reverend, might there be another way to describe a noble disciple who has right view, whose view is correct, who has experiential confidence in the teaching, and has come to the true teaching?” 

“There\marginnote{69.1} might, reverends. A noble disciple understands defilement, its origin, its cessation, and the practice that leads to its cessation. When they’ve done this, they’re defined as a noble disciple who has right view, whose view is correct, who has experiential confidence in the teaching, and has come to the true teaching. 

But\marginnote{70.1} what is defilement? What is its origin, its cessation, and the practice that leads to its cessation? There are these three defilements. The defilements of sensuality, desire to be reborn, and ignorance. Defilement originates from ignorance. Defilement ceases when ignorance ceases. The practice that leads to the cessation of defilement is simply this noble eightfold path, that is: right view, right thought, right speech, right action, right livelihood, right effort, right mindfulness, and right immersion. 

A\marginnote{71.1} noble disciple understands in this way defilement, its origin, its cessation, and the practice that leads to its cessation. They’ve completely given up the underlying tendency to greed, got rid of the underlying tendency to repulsion, and eradicated the underlying tendency to the view and conceit ‘I am’. They’ve given up ignorance and given rise to knowledge, and make an end of suffering in this very life. When they’ve done this, they’re defined as a noble disciple who has right view, whose view is correct, who has experiential confidence in the teaching, and has come to the true teaching.” 

This\marginnote{71.3} is what Venerable \textsanskrit{Sāriputta} said. Satisfied, the mendicants were happy with what \textsanskrit{Sāriputta} said. 

%
\section*{{\suttatitleacronym MN 10}{\suttatitletranslation Mindfulness Meditation }{\suttatitleroot Mahāsatipaṭṭhānasutta}}
\addcontentsline{toc}{section}{\tocacronym{MN 10} \toctranslation{Mindfulness Meditation } \tocroot{Mahāsatipaṭṭhānasutta}}
\markboth{Mindfulness Meditation }{Mahāsatipaṭṭhānasutta}
\extramarks{MN 10}{MN 10}

\scevam{So\marginnote{1.1} I have heard. }At one time the Buddha was staying in the land of the Kurus, near the Kuru town named \textsanskrit{Kammāsadamma}. There the Buddha addressed the mendicants, “Mendicants!” 

“Venerable\marginnote{1.5} sir,” they replied. The Buddha said this: 

“Mendicants,\marginnote{2.1} the four kinds of mindfulness meditation are the path to convergence. They are in order to purify sentient beings, to get past sorrow and crying, to make an end of pain and sadness, to end the cycle of suffering, and to realize extinguishment. 

What\marginnote{3.1} four? It’s when a mendicant meditates by observing an aspect of the body—keen, aware, and mindful, rid of desire and aversion for the world. They meditate observing an aspect of feelings—keen, aware, and mindful, rid of desire and aversion for the world. They meditate observing an aspect of the mind—keen, aware, and mindful, rid of desire and aversion for the world. They meditate observing an aspect of principles—keen, aware, and mindful, rid of desire and aversion for the world. 

\subsection*{1. Observing the Body }

\subsubsection*{1.1. Mindfulness of Breathing }

And\marginnote{4.1} how does a mendicant meditate observing an aspect of the body? 

It’s\marginnote{4.2} when a mendicant—gone to a wilderness, or to the root of a tree, or to an empty hut—sits down cross-legged, with their body straight, and focuses their mindfulness right there. Just mindful, they breathe in. Mindful, they breathe out. 

When\marginnote{4.4} breathing in heavily they know: ‘I’m breathing in heavily.’ When breathing out heavily they know: ‘I’m breathing out heavily.’ 

When\marginnote{4.5} breathing in lightly they know: ‘I’m breathing in lightly.’ When breathing out lightly they know: ‘I’m breathing out lightly.’ 

They\marginnote{4.6} practice breathing in experiencing the whole body. They practice breathing out experiencing the whole body. 

They\marginnote{4.7} practice breathing in stilling the body’s motion. They practice breathing out stilling the body’s motion. 

It’s\marginnote{4.8} like a deft carpenter or carpenter’s apprentice. When making a deep cut they know: ‘I’m making a deep cut,’ and when making a shallow cut they know: ‘I’m making a shallow cut.’ 

And\marginnote{5.1} so they meditate observing an aspect of the body internally, externally, and both internally and externally. They meditate observing the body as liable to originate, as liable to vanish, and as liable to both originate and vanish. Or mindfulness is established that the body exists, to the extent necessary for knowledge and mindfulness. They meditate independent, not grasping at anything in the world. 

That’s\marginnote{5.4} how a mendicant meditates by observing an aspect of the body. 

\subsubsection*{1.2. The Postures }

Furthermore,\marginnote{6.1} when a mendicant is walking they know: ‘I am walking.’ When standing they know: ‘I am standing.’ When sitting they know: ‘I am sitting.’ And when lying down they know: ‘I am lying down.’ Whatever posture their body is in, they know it. 

And\marginnote{7.1} so they meditate observing an aspect of the body internally, externally, and both internally and externally. They meditate observing the body as liable to originate, as liable to vanish, and as liable to both originate and vanish. Or mindfulness is established that the body exists, to the extent necessary for knowledge and mindfulness. They meditate independent, not grasping at anything in the world. 

That\marginnote{7.4} too is how a mendicant meditates by observing an aspect of the body. 

\subsubsection*{1.3. Situational Awareness }

Furthermore,\marginnote{8.1} a mendicant acts with situational awareness when going out and coming back; when looking ahead and aside; when bending and extending the limbs; when bearing the outer robe, bowl and robes; when eating, drinking, chewing, and tasting; when urinating and defecating; when walking, standing, sitting, sleeping, waking, speaking, and keeping silent. 

And\marginnote{9.1} so they meditate observing an aspect of the body internally … 

That\marginnote{9.2} too is how a mendicant meditates by observing an aspect of the body. 

\subsubsection*{1.4. Focusing on the Repulsive }

Furthermore,\marginnote{10.1} a mendicant examines their own body, up from the soles of the feet and down from the tips of the hairs, wrapped in skin and full of many kinds of filth. ‘In this body there is head hair, body hair, nails, teeth, skin, flesh, sinews, bones, bone marrow, kidneys, heart, liver, diaphragm, spleen, lungs, intestines, mesentery, undigested food, feces, bile, phlegm, pus, blood, sweat, fat, tears, grease, saliva, snot, synovial fluid, urine.’ 

It’s\marginnote{10.3} as if there were a bag with openings at both ends, filled with various kinds of grains, such as fine rice, wheat, mung beans, peas, sesame, and ordinary rice. And someone with good eyesight were to open it and examine the contents: ‘These grains are fine rice, these are wheat, these are mung beans, these are peas, these are sesame, and these are ordinary rice.’ 

And\marginnote{11.1} so they meditate observing an aspect of the body internally … 

That\marginnote{11.2} too is how a mendicant meditates by observing an aspect of the body. 

\subsubsection*{1.5. Focusing on the Elements }

Furthermore,\marginnote{12.1} a mendicant examines their own body, whatever its placement or posture, according to the elements: ‘In this body there is the earth element, the water element, the fire element, and the air element.’ 

It’s\marginnote{12.3} as if a deft butcher or butcher’s apprentice were to kill a cow and sit down at the crossroads with the meat cut into portions. 

And\marginnote{13.1} so they meditate observing an aspect of the body internally … 

That\marginnote{13.2} too is how a mendicant meditates by observing an aspect of the body. 

\subsubsection*{1.6. The Charnel Ground Contemplations }

Furthermore,\marginnote{14.1} suppose a mendicant were to see a corpse discarded in a charnel ground. And it had been dead for one, two, or three days, bloated, livid, and festering. They’d compare it with their own body: ‘This body is also of that same nature, that same kind, and cannot go beyond that.’ And so they meditate observing an aspect of the body internally … 

That\marginnote{15.2} too is how a mendicant meditates by observing an aspect of the body. 

Furthermore,\marginnote{16.1} suppose they were to see a corpse discarded in a charnel ground being devoured by crows, hawks, vultures, herons, dogs, tigers, leopards, jackals, and many kinds of little creatures. They’d compare it with their own body: ‘This body is also of that same nature, that same kind, and cannot go beyond that.’ And so they meditate observing an aspect of the body internally … 

That\marginnote{17.2} too is how a mendicant meditates by observing an aspect of the body. 

Furthermore,\marginnote{18{-}23.1} suppose they were to see a corpse discarded in a charnel ground, a skeleton with flesh and blood, held together by sinews … 

A\marginnote{18{-}23.2} skeleton without flesh but smeared with blood, and held together by sinews … 

A\marginnote{18{-}23.3} skeleton rid of flesh and blood, held together by sinews … 

Bones\marginnote{24.1} rid of sinews scattered in every direction. Here a hand-bone, there a foot-bone, here a shin-bone, there a thigh-bone, here a hip-bone, there a rib-bone, here a back-bone, there an arm-bone, here a neck-bone, there a jaw-bone, here a tooth, there the skull … 

White\marginnote{26{-}28.1} bones, the color of shells … 

Decrepit\marginnote{29.1} bones, heaped in a pile … 

Bones\marginnote{30.1} rotted and crumbled to powder. They’d compare it with their own body: ‘This body is also of that same nature, that same kind, and cannot go beyond that.’ 

And\marginnote{31.1} so they meditate observing an aspect of the body internally, externally, and both internally and externally. They meditate observing the body as liable to originate, as liable to vanish, and as liable to both originate and vanish. Or mindfulness is established that the body exists, to the extent necessary for knowledge and mindfulness. They meditate independent, not grasping at anything in the world. 

That\marginnote{31.4} too is how a mendicant meditates by observing an aspect of the body. 

\subsection*{2. Observing the Feelings }

And\marginnote{32.1} how does a mendicant meditate observing an aspect of feelings? 

It’s\marginnote{32.2} when a mendicant who feels a pleasant feeling knows: ‘I feel a pleasant feeling.’ 

When\marginnote{32.3} they feel a painful feeling, they know: ‘I feel a painful feeling.’ 

When\marginnote{32.4} they feel a neutral feeling, they know: ‘I feel a neutral feeling.’ 

When\marginnote{32.5} they feel a material pleasant feeling, they know: ‘I feel a material pleasant feeling.’ 

When\marginnote{32.6} they feel a spiritual pleasant feeling, they know: ‘I feel a spiritual pleasant feeling.’ 

When\marginnote{32.7} they feel a material painful feeling, they know: ‘I feel a material painful feeling.’ 

When\marginnote{32.8} they feel a spiritual painful feeling, they know: ‘I feel a spiritual painful feeling.’ 

When\marginnote{32.9} they feel a material neutral feeling, they know: ‘I feel a material neutral feeling.’ 

When\marginnote{32.10} they feel a spiritual neutral feeling, they know: ‘I feel a spiritual neutral feeling.’ 

And\marginnote{33.1} so they meditate observing an aspect of feelings internally, externally, and both internally and externally. They meditate observing feelings as liable to originate, as liable to vanish, and as liable to both originate and vanish. Or mindfulness is established that feelings exist, to the extent necessary for knowledge and mindfulness. They meditate independent, not grasping at anything in the world. 

That’s\marginnote{33.4} how a mendicant meditates by observing an aspect of feelings. 

\subsection*{3. Observing the Mind }

And\marginnote{34.1} how does a mendicant meditate observing an aspect of the mind? 

It’s\marginnote{34.2} when a mendicant understands mind with greed as ‘mind with greed,’ and mind without greed as ‘mind without greed.’ They understand mind with hate as ‘mind with hate,’ and mind without hate as ‘mind without hate.’ They understand mind with delusion as ‘mind with delusion,’ and mind without delusion as ‘mind without delusion.’ They know constricted mind as ‘constricted mind,’ and scattered mind as ‘scattered mind.’ They know expansive mind as ‘expansive mind,’ and unexpansive mind as ‘unexpansive mind.’ They know mind that is not supreme as ‘mind that is not supreme,’ and mind that is supreme as ‘mind that is supreme.’ They know mind immersed in \textsanskrit{samādhi} as ‘mind immersed in \textsanskrit{samādhi},’ and mind not immersed in \textsanskrit{samādhi} as ‘mind not immersed in \textsanskrit{samādhi}.’ They know freed mind as ‘freed mind,’ and unfreed mind as ‘unfreed mind.’ 

And\marginnote{35.1} so they meditate observing an aspect of the mind internally, externally, and both internally and externally. They meditate observing the mind as liable to originate, as liable to vanish, and as liable to both originate and vanish. Or mindfulness is established that the mind exists, to the extent necessary for knowledge and mindfulness. They meditate independent, not grasping at anything in the world. 

That’s\marginnote{35.4} how a mendicant meditates by observing an aspect of the mind. 

\subsection*{4. Observing Principles }

\subsubsection*{4.1. The Hindrances }

And\marginnote{36.1} how does a mendicant meditate observing an aspect of principles? 

It’s\marginnote{36.2} when a mendicant meditates by observing an aspect of principles with respect to the five hindrances. And how does a mendicant meditate observing an aspect of principles with respect to the five hindrances? 

It’s\marginnote{36.4} when a mendicant who has sensual desire in them understands: ‘I have sensual desire in me.’ When they don’t have sensual desire in them, they understand: ‘I don’t have sensual desire in me.’ They understand how sensual desire arises; how, when it’s already arisen, it’s given up; and how, once it’s given up, it doesn’t arise again in the future. 

When\marginnote{36.5} they have ill will in them, they understand: ‘I have ill will in me.’ When they don’t have ill will in them, they understand: ‘I don’t have ill will in me.’ They understand how ill will arises; how, when it’s already arisen, it’s given up; and how, once it’s given up, it doesn’t arise again in the future. 

When\marginnote{36.6} they have dullness and drowsiness in them, they understand: ‘I have dullness and drowsiness in me.’ When they don’t have dullness and drowsiness in them, they understand: ‘I don’t have dullness and drowsiness in me.’ They understand how dullness and drowsiness arise; how, when they’ve already arisen, they’re given up; and how, once they’re given up, they don’t arise again in the future. 

When\marginnote{36.7} they have restlessness and remorse in them, they understand: ‘I have restlessness and remorse in me.’ When they don’t have restlessness and remorse in them, they understand: ‘I don’t have restlessness and remorse in me.’ They understand how restlessness and remorse arise; how, when they’ve already arisen, they’re given up; and how, once they’re given up, they don’t arise again in the future. 

When\marginnote{36.8} they have doubt in them, they understand: ‘I have doubt in me.’ When they don’t have doubt in them, they understand: ‘I don’t have doubt in me.’ They understand how doubt arises; how, when it’s already arisen, it’s given up; and how, once it’s given up, it doesn’t arise again in the future. 

And\marginnote{37.1} so they meditate observing an aspect of principles internally, externally, and both internally and externally. They meditate observing the principles as liable to originate, as liable to vanish, and as liable to both originate and vanish. Or mindfulness is established that principles exist, to the extent necessary for knowledge and mindfulness. They meditate independent, not grasping at anything in the world. 

That’s\marginnote{37.4} how a mendicant meditates by observing an aspect of principles with respect to the five hindrances. 

\subsubsection*{4.2. The Aggregates }

Furthermore,\marginnote{38.1} a mendicant meditates by observing an aspect of principles with respect to the five grasping aggregates. And how does a mendicant meditate observing an aspect of principles with respect to the five grasping aggregates? It’s when a mendicant contemplates: ‘Such is form, such is the origin of form, such is the ending of form. Such is feeling, such is the origin of feeling, such is the ending of feeling. Such is perception, such is the origin of perception, such is the ending of perception. Such are choices, such is the origin of choices, such is the ending of choices. Such is consciousness, such is the origin of consciousness, such is the ending of consciousness.’ 

And\marginnote{39.1} so they meditate observing an aspect of principles internally … 

That’s\marginnote{39.4} how a mendicant meditates by observing an aspect of principles with respect to the five grasping aggregates. 

\subsubsection*{4.3. The Sense Fields }

Furthermore,\marginnote{40.1} a mendicant meditates by observing an aspect of principles with respect to the six interior and exterior sense fields. And how does a mendicant meditate observing an aspect of principles with respect to the six interior and exterior sense fields? 

It’s\marginnote{40.3} when a mendicant understands the eye, sights, and the fetter that arises dependent on both of these. They understand how the fetter that has not arisen comes to arise; how the arisen fetter comes to be abandoned; and how the abandoned fetter comes to not rise again in the future. 

They\marginnote{40.4} understand the ear, sounds, and the fetter … 

They\marginnote{40.5} understand the nose, smells, and the fetter … 

They\marginnote{40.6} understand the tongue, tastes, and the fetter … 

They\marginnote{40.7} understand the body, touches, and the fetter … 

They\marginnote{40.8} understand the mind, thoughts, and the fetter that arises dependent on both of these. They understand how the fetter that has not arisen comes to arise; how the arisen fetter comes to be abandoned; and how the abandoned fetter comes to not rise again in the future. 

And\marginnote{41.1} so they meditate observing an aspect of principles internally … 

That’s\marginnote{41.4} how a mendicant meditates by observing an aspect of principles with respect to the six internal and external sense fields. 

\subsubsection*{4.4. The Awakening Factors }

Furthermore,\marginnote{42.1} a mendicant meditates by observing an aspect of principles with respect to the seven awakening factors. And how does a mendicant meditate observing an aspect of principles with respect to the seven awakening factors? 

It’s\marginnote{42.3} when a mendicant who has the awakening factor of mindfulness in them understands: ‘I have the awakening factor of mindfulness in me.’ When they don’t have the awakening factor of mindfulness in them, they understand: ‘I don’t have the awakening factor of mindfulness in me.’ They understand how the awakening factor of mindfulness that has not arisen comes to arise; and how the awakening factor of mindfulness that has arisen becomes fulfilled by development. 

When\marginnote{42.4} they have the awakening factor of investigation of principles … energy … rapture … tranquility … immersion … equanimity in them, they understand: ‘I have the awakening factor of equanimity in me.’ When they don’t have the awakening factor of equanimity in them, they understand: ‘I don’t have the awakening factor of equanimity in me.’ They understand how the awakening factor of equanimity that has not arisen comes to arise; and how the awakening factor of equanimity that has arisen becomes fulfilled by development. 

And\marginnote{43.1} so they meditate observing an aspect of principles internally, externally, and both internally and externally. They meditate observing the principles as liable to originate, as liable to vanish, and as liable to both originate and vanish. Or mindfulness is established that principles exist, to the extent necessary for knowledge and mindfulness. They meditate independent, not grasping at anything in the world. 

That’s\marginnote{43.4} how a mendicant meditates by observing an aspect of principles with respect to the seven awakening factors. 

\subsubsection*{4.5. The Truths }

Furthermore,\marginnote{44.1} a mendicant meditates by observing an aspect of principles with respect to the four noble truths. 

And\marginnote{44.2} how does a mendicant meditate observing an aspect of principles with respect to the four noble truths? It’s when a mendicant truly understands: ‘This is suffering’ … ‘This is the origin of suffering’ … ‘This is the cessation of suffering’ … ‘This is the practice that leads to the cessation of suffering.’ 

And\marginnote{45.1} so they meditate observing an aspect of principles internally, externally, and both internally and externally. They meditate observing the principles as liable to originate, as liable to vanish, and as liable to both originate and vanish. Or mindfulness is established that principles exist, to the extent necessary for knowledge and mindfulness. They meditate independent, not grasping at anything in the world. 

That’s\marginnote{45.4} how a mendicant meditates by observing an aspect of principles with respect to the four noble truths. 

Anyone\marginnote{46.1} who develops these four kinds of mindfulness meditation in this way for seven years can expect one of two results: enlightenment in the present life, or if there’s something left over, non-return. 

Let\marginnote{46.3} alone seven years, anyone who develops these four kinds of mindfulness meditation in this way for six years … five years … four years … three years … two years … one year … seven months … six months … five months … four months … three months … two months … one month … a fortnight … Let alone a fortnight, anyone who develops these four kinds of mindfulness meditation in this way for seven days can expect one of two results: enlightenment in the present life, or if there’s something left over, non-return. 

‘The\marginnote{47.1} four kinds of mindfulness meditation are the path to convergence. They are in order to purify sentient beings, to get past sorrow and crying, to make an end of pain and sadness, to end the cycle of suffering, and to realize extinguishment.’ That’s what I said, and this is why I said it.” 

That\marginnote{47.3} is what the Buddha said. Satisfied, the mendicants were happy with what the Buddha said. 

%
\addtocontents{toc}{\let\protect\contentsline\protect\nopagecontentsline}
\chapter*{The Chapter on the Lion’s Roar }
\addcontentsline{toc}{chapter}{\tocchapterline{The Chapter on the Lion’s Roar }}
\addtocontents{toc}{\let\protect\contentsline\protect\oldcontentsline}

%
\section*{{\suttatitleacronym MN 11}{\suttatitletranslation The Shorter Discourse on the Lion’s Roar }{\suttatitleroot Cūḷasīhanādasutta}}
\addcontentsline{toc}{section}{\tocacronym{MN 11} \toctranslation{The Shorter Discourse on the Lion’s Roar } \tocroot{Cūḷasīhanādasutta}}
\markboth{The Shorter Discourse on the Lion’s Roar }{Cūḷasīhanādasutta}
\extramarks{MN 11}{MN 11}

\scevam{So\marginnote{1.1} I have heard. }At one time the Buddha was staying near \textsanskrit{Sāvatthī} in Jeta’s Grove, \textsanskrit{Anāthapiṇḍika}’s monastery. There the Buddha addressed the mendicants, “Mendicants!” 

“Venerable\marginnote{1.5} sir,” they replied. The Buddha said this: 

“‘Only\marginnote{2.1} here is there a true ascetic, here a second ascetic, here a third ascetic, and here a fourth ascetic. Other sects are empty of ascetics.’ This, mendicants, is how you should rightly roar your lion’s roar. 

It’s\marginnote{3.1} possible that wanderers who follow other paths might say: ‘But what is the source of the venerables’ self-confidence and forcefulness that they say this?’ You should say to them: ‘There are four things explained by the Blessed One, who knows and sees, the perfected one, the fully awakened Buddha. Seeing these things in ourselves we say that: 

“Only\marginnote{3.7} here is there a true ascetic, here a second ascetic, here a third ascetic, and here a fourth ascetic. Other sects are empty of ascetics.” What four? We have confidence in the Teacher, we have confidence in the teaching, and we have fulfilled the precepts. And we have love and affection for those who share our path, both laypeople and renunciates. These are the four things.’ 

It’s\marginnote{4.1} possible that wanderers who follow other paths might say: ‘We too have confidence in the Teacher—our Teacher; we have confidence in the teaching—our teaching; and we have fulfilled the precepts—our precepts. And we have love and affection for those who share our path, both laypeople and renunciates. What, then, is the difference between you and us?’ 

You\marginnote{5.1} should say to them: ‘Well, reverends, is the goal one or many?’ Answering rightly, the wanderers would say: ‘The goal is one, reverends, not many.’ 

‘But\marginnote{5.5} is that goal for the greedy or for those free of greed?’ Answering rightly, the wanderers would say: ‘That goal is for those free of greed, not for the greedy.’ 

‘Is\marginnote{5.8} it for the hateful or those free of hate?’ ‘It’s for those free of hate.’ 

‘Is\marginnote{5.11} it for the delusional or those free of delusion?’ ‘It’s for those free of delusion.’ 

‘Is\marginnote{5.14} it for those who crave or those rid of craving?’ ‘It’s for those rid of craving.’ 

‘Is\marginnote{5.17} it for those who grasp or those who don’t grasp?’ ‘It’s for those who don’t grasp.’ 

‘Is\marginnote{5.20} it for the knowledgeable or the ignorant?’ ‘It’s for the knowledgeable.’ 

‘Is\marginnote{5.23} it for those who favor and oppose or for those who don’t favor and oppose?’ ‘It’s for those who don’t favor and oppose.’ 

‘But\marginnote{5.26} is that goal for those who enjoy proliferation or for those who enjoy non-proliferation?’ Answering rightly, the wanderers would say: ‘It’s for those who enjoy non-proliferation, not for those who enjoy proliferation.’ 

Mendicants,\marginnote{6.1} there are these two views: views favoring continued existence and views favoring ending existence. Any ascetics or brahmins who cling, hold, and attach to a view favoring continued existence will oppose a view favoring ending existence. Any ascetics or brahmins who cling, hold, and attach to a view favoring ending existence will oppose a view favoring continued existence. 

There\marginnote{7.1} are some ascetics and brahmins who don’t truly understand these two views’ origin, ending, gratification, drawback, and escape. They’re greedy, hateful, delusional, craving, grasping, and ignorant. They favor and oppose, and they enjoy proliferation. They’re not freed from rebirth, old age, and death, from sorrow, lamentation, pain, sadness, and distress. They’re not freed from suffering, I say. 

There\marginnote{8.1} are some ascetics and brahmins who do truly understand these two views’ origin, ending, gratification, drawback, and escape. They’re rid of greed, hate, delusion, craving, grasping, and ignorance. They don’t favor and oppose, and they enjoy non-proliferation. They’re freed from rebirth, old age, and death, from sorrow, lamentation, pain, sadness, and distress. They’re freed from suffering, I say. 

There\marginnote{9.1} are these four kinds of grasping. What four? Grasping at sensual pleasures, views, precepts and observances, and theories of a self. 

There\marginnote{10.1} are some ascetics and brahmins who claim to propound the complete understanding of all kinds of grasping. But they don’t correctly describe the complete understanding of all kinds of grasping. They describe the complete understanding of grasping at sensual pleasures, but not views, precepts and observances, and theories of a self. Why is that? Because those gentlemen don’t truly understand these three things. That’s why they claim to propound the complete understanding of all kinds of grasping, but they don’t really. 

There\marginnote{11.1} are some other ascetics and brahmins who claim to propound the complete understanding of all kinds of grasping, but they don’t really. They describe the complete understanding of grasping at sensual pleasures and views, but not precepts and observances, and theories of a self. Why is that? Because those gentlemen don’t truly understand these two things. That’s why they claim to propound the complete understanding of all kinds of grasping, but they don’t really. 

There\marginnote{12.1} are some other ascetics and brahmins who claim to propound the complete understanding of all kinds of grasping, but they don’t really. They describe the complete understanding of grasping at sensual pleasures, views, and precepts and observances, but not theories of a self. Why is that? Because those gentlemen don’t truly understand this one thing. That’s why they claim to propound the complete understanding of all kinds of grasping, but they don’t really. 

In\marginnote{13.1} such a teaching and training, confidence in the Teacher is said to be far from ideal. Likewise, confidence in the teaching, fulfillment of the precepts, and love and affection for those sharing the same path are said to be far from ideal. Why is that? It’s because that teaching and training is poorly explained and poorly propounded, not emancipating, not leading to peace, proclaimed by someone who is not a fully awakened Buddha. 

The\marginnote{14.1} Realized One, the perfected one, the fully awakened Buddha claims to propound the complete understanding of all kinds of grasping. He describes the complete understanding of grasping at sensual pleasures, views, precepts and observances, and theories of a self. 

In\marginnote{15.1} such a teaching and training, confidence in the Teacher is said to be ideal. Likewise, confidence in the teaching, fulfillment of the precepts, and love and affection for those sharing the same path are said to be ideal. Why is that? It’s because that teaching and training is well explained and well propounded, emancipating, leading to peace, proclaimed by a fully awakened Buddha. 

What\marginnote{16.1} is the source, origin, birthplace, and inception of these four kinds of grasping? Craving. And what is the source, origin, birthplace, and inception of craving? Feeling. And what is the source of feeling? Contact. And what is the source of contact? The six sense fields. And what is the source of the six sense fields? Name and form. And what is the source of name and form? Consciousness. And what is the source of consciousness? Choices. And what is the source of choices? Ignorance. 

When\marginnote{17.1} that mendicant has given up ignorance and given rise to knowledge, they don’t grasp at sensual pleasures, views, precepts and observances, or theories of a self. Not grasping, they’re not anxious. Not being anxious, they personally become extinguished. 

They\marginnote{17.3} understand: ‘Rebirth is ended, the spiritual journey has been completed, what had to be done has been done, there is no return to any state of existence.’” 

That\marginnote{17.4} is what the Buddha said. Satisfied, the mendicants were happy with what the Buddha said. 

%
\section*{{\suttatitleacronym MN 12}{\suttatitletranslation The Longer Discourse on the Lion’s Roar }{\suttatitleroot Mahāsīhanādasutta}}
\addcontentsline{toc}{section}{\tocacronym{MN 12} \toctranslation{The Longer Discourse on the Lion’s Roar } \tocroot{Mahāsīhanādasutta}}
\markboth{The Longer Discourse on the Lion’s Roar }{Mahāsīhanādasutta}
\extramarks{MN 12}{MN 12}

\scevam{So\marginnote{1.1} I have heard. }At one time the Buddha was staying near \textsanskrit{Vesālī} in a woodland grove behind the town. 

Now\marginnote{2.1} at that time Sunakkhatta the Licchavi had recently left this teaching and training. He was telling a crowd in \textsanskrit{Vesālī}: 

“The\marginnote{2.3} ascetic Gotama has no superhuman distinction in knowledge and vision worthy of the noble ones. He teaches what he’s worked out by logic, following a line of inquiry, expressing his own perspective. And his teaching leads those who practice it to the complete ending of suffering, the goal for which it’s taught.” 

Then\marginnote{3.1} Venerable \textsanskrit{Sāriputta} robed up in the morning and, taking his bowl and robe, entered \textsanskrit{Vesālī} for alms. He heard what Sunakkhatta was saying. 

Then\marginnote{3.6} he wandered for alms in \textsanskrit{Vesālī}. After the meal, on his return from almsround, he went to the Buddha, bowed, sat down to one side, and told him what had happened. 

“\textsanskrit{Sāriputta},\marginnote{4.1} Sunakkhatta, that silly man, is angry. His words are spoken out of anger. Thinking he criticizes the Realized One, in fact he just praises him. For it is praise of the Realized One to say: ‘His teaching leads those who practice it to the complete ending of suffering, the goal for which it’s taught.’ 

But\marginnote{5.1} there’s no way Sunakkhatta will infer about me from the teaching: ‘That Blessed One is perfected, a fully awakened Buddha, accomplished in knowledge and conduct, holy, knower of the world, supreme guide for those who wish to train, teacher of gods and humans, awakened, blessed.’ 

And\marginnote{6.1} there’s no way Sunakkhatta will infer about me from the teaching: ‘That Blessed One wields the many kinds of psychic power: multiplying himself and becoming one again; appearing and disappearing; going unimpeded through a wall, a rampart, or a mountain as if through space; diving in and out of the earth as if it were water; walking on water as if it were earth; flying cross-legged through the sky like a bird; touching and stroking with the hand the sun and moon, so mighty and powerful; controlling the body as far as the \textsanskrit{Brahmā} realm.’ 

And\marginnote{7.1} there’s no way Sunakkhatta will infer about me from the teaching: ‘That Blessed One, with clairaudience that is purified and superhuman, hears both kinds of sounds, human and divine, whether near or far.’ 

And\marginnote{8.1} there’s no way Sunakkhatta will infer about me from the teaching: ‘That Blessed One understands the minds of other beings and individuals, having comprehended them with his own mind. He understands mind with greed as “mind with greed,” and mind without greed as “mind without greed.” He understands mind with hate … mind without hate … mind with delusion … mind without delusion … constricted mind … scattered mind … expansive mind … unexpansive mind … mind that is supreme … mind that is not supreme … mind immersed in \textsanskrit{samādhi} … mind not immersed in \textsanskrit{samādhi} … freed mind as “freed mind,” and unfreed mind as “unfreed mind.”’ 

The\marginnote{9.1} Realized One possesses ten powers of a Realized One. With these he claims the bull’s place, roars his lion’s roar in the assemblies, and turns the holy wheel. What ten? 

Firstly,\marginnote{10.1} the Realized One truly understands the possible as possible, and the impossible as impossible. Since he truly understands this, this is a power of the Realized One. Relying on this he claims the bull’s place, roars his lion’s roar in the assemblies, and turns the holy wheel. 

Furthermore,\marginnote{11.1} the Realized One truly understands the result of deeds undertaken in the past, future, and present in terms of causes and reasons. Since he truly understands this, this is a power of the Realized One. … 

Furthermore,\marginnote{12.1} the Realized One truly understands where all paths of practice lead. Since he truly understands this, this is a power of the Realized One. … 

Furthermore,\marginnote{13.1} the Realized One truly understands the world with its many and diverse elements. Since he truly understands this, this is a power of the Realized One. … 

Furthermore,\marginnote{14.1} the Realized One truly understands the diverse convictions of sentient beings. Since he truly understands this, this is a power of the Realized One. … 

Furthermore,\marginnote{15.1} the Realized One truly understands the faculties of other sentient beings and other individuals after comprehending them with his mind. Since he truly understands this, this is a power of the Realized One. … 

Furthermore,\marginnote{16.1} the Realized One truly understands corruption, cleansing, and emergence regarding the absorptions, liberations, immersions, and attainments. Since he truly understands this, this is a power of the Realized One. … 

Furthermore,\marginnote{17.1} the Realized One recollects many kinds of past lives. That is: one, two, three, four, five, ten, twenty, thirty, forty, fifty, a hundred, a thousand, a hundred thousand rebirths; many eons of the world contracting, many eons of the world expanding, many eons of the world contracting and expanding. He remembers: ‘There, I was named this, my clan was that, I looked like this, and that was my food. This was how I felt pleasure and pain, and that was how my life ended. When I passed away from that place I was reborn somewhere else. There, too, I was named this, my clan was that, I looked like this, and that was my food. This was how I felt pleasure and pain, and that was how my life ended. When I passed away from that place I was reborn here.’ And so he recollects his many kinds of past lives, with features and details. Since he truly understands this, this is a power of the Realized One. … 

Furthermore,\marginnote{18.1} with clairvoyance that is purified and superhuman, the Realized One sees sentient beings passing away and being reborn—inferior and superior, beautiful and ugly, in a good place or a bad place. He understands how sentient beings are reborn according to their deeds. ‘These dear beings did bad things by way of body, speech, and mind. They spoke ill of the noble ones; they had wrong view; and they chose to act out of that wrong view. When their body breaks up, after death, they’re reborn in a place of loss, a bad place, the underworld, hell. These dear beings, however, did good things by way of body, speech, and mind. They never spoke ill of the noble ones; they had right view; and they chose to act out of that right view. When their body breaks up, after death, they’re reborn in a good place, a heavenly realm.’ And so, with clairvoyance that is purified and superhuman, he sees sentient beings passing away and being reborn—inferior and superior, beautiful and ugly, in a good place or a bad place. He understands how sentient beings are reborn according to their deeds. Since he truly understands this, this is a power of the Realized One. … 

Furthermore,\marginnote{19.1} the Realized One has realized the undefiled freedom of heart and freedom by wisdom in this very life, and lives having realized it with his own insight due to the ending of defilements. Since he truly understands this, this is a power of the Realized One. Relying on this he claims the bull’s place, roars his lion’s roar in the assemblies, and turns the holy wheel. 

A\marginnote{20.1} Realized One possesses these ten powers of a Realized One. With these he claims the bull’s place, roars his lion’s roar in the assemblies, and turns the holy wheel. 

When\marginnote{21.1} I know and see in this way, suppose someone were to say this: ‘The ascetic Gotama has no superhuman distinction in knowledge and vision worthy of the noble ones. He teaches what he’s worked out by logic, following a line of inquiry, expressing his own perspective.’ Unless they give up that speech and that thought, and let go of that view, they will be cast down to hell. Just as a mendicant accomplished in ethics, immersion, and wisdom would reach enlightenment in this very life, such is the consequence, I say. Unless they give up that speech and thought, and let go of that view, they will be cast down to hell. 

\textsanskrit{Sāriputta},\marginnote{22.1} a Realized One has four kinds of self-assurance. With these he claims the bull’s place, roars his lion’s roar in the assemblies, and turns the holy wheel. What four? 

I\marginnote{23.1} see no reason for anyone—whether ascetic, brahmin, god, \textsanskrit{Māra}, or \textsanskrit{Brahmā}, or anyone else in the world—to legitimately scold me, saying: ‘You claim to be fully awakened, but you don’t understand these things.’ Since I see no such reason, I live secure, fearless, and assured. 

I\marginnote{24.1} see no reason for anyone—whether ascetic, brahmin, god, \textsanskrit{Māra}, or \textsanskrit{Brahmā}, or anyone else in the world—to legitimately scold me, saying: ‘You claim to have ended all defilements, but these defilements have not ended.’ Since I see no such reason, I live secure, fearless, and assured. 

I\marginnote{25.1} see no reason for anyone—whether ascetic, brahmin, god, \textsanskrit{Māra}, or \textsanskrit{Brahmā}, or anyone else in the world—to legitimately scold me, saying: ‘The acts that you say are obstructions are not really obstructions for the one who performs them.’ Since I see no such reason, I live secure, fearless, and assured. 

I\marginnote{26.1} see no reason for anyone—whether ascetic, brahmin, god, \textsanskrit{Māra}, or \textsanskrit{Brahmā}, or anyone else in the world—to legitimately scold me, saying: ‘The teaching doesn’t lead those who practice it to the complete ending of suffering, the goal for which you taught it.’ Since I see no such reason, I live secure, fearless, and assured. 

A\marginnote{27.1} Realized One has these four kinds of self-assurance. With these he claims the bull’s place, roars his lion’s roar in the assemblies, and turns the holy wheel. 

When\marginnote{28.1} I know and see in this way, suppose someone were to say this: ‘The ascetic Gotama has no superhuman distinction in knowledge and vision worthy of the noble ones …’ Unless they give up that speech and that thought, and let go of that view, they will be cast down to hell. 

\textsanskrit{Sāriputta},\marginnote{29.1} there are these eight assemblies. What eight? The assemblies of aristocrats, brahmins, householders, and ascetics. An assembly of the gods under the Four Great Kings. An assembly of the gods under the Thirty-Three. An assembly of \textsanskrit{Māras}. An assembly of \textsanskrit{Brahmās}. These are the eight assemblies. Possessing these four kinds of self-assurance, the Realized One approaches and enters right into these eight assemblies. I recall having approached an assembly of hundreds of aristocrats. There I used to sit with them, converse, and engage in discussion. But I don’t see any reason to feel afraid or insecure. Since I see no such reason, I live secure, fearless, and assured. 

I\marginnote{30.1} recall having approached an assembly of hundreds of brahmins … householders … ascetics … the gods under the Four Great Kings … the gods under the Thirty-Three … \textsanskrit{Māras} … \textsanskrit{Brahmās}. There too I used to sit with them, converse, and engage in discussion. But I don’t see any reason to feel afraid or insecure. Since I see no such reason, I live secure, fearless, and assured. 

When\marginnote{31.1} I know and see in this way, suppose someone were to say this: ‘The ascetic Gotama has no superhuman distinction in knowledge and vision worthy of the noble ones …’ Unless they give up that speech and that thought, and let go of that view, they will be cast down to hell. 

\textsanskrit{Sāriputta},\marginnote{32.1} there are these four kinds of reproduction. What four? Reproduction for creatures born from an egg, from a womb, from moisture, or spontaneously. 

And\marginnote{33.1} what is reproduction from an egg? There are beings who are born by breaking out of an eggshell. This is called reproduction from an egg. And what is reproduction from a womb? There are beings who are born by breaking out of the amniotic sac. This is called reproduction from a womb. And what is reproduction from moisture? There are beings who are born in a rotten fish, in a rotten corpse, in rotten dough, in a cesspool or a sump. This is called reproduction from moisture. And what is spontaneous reproduction? Gods, hell-beings, certain humans, and certain beings in the lower realms. This is called spontaneous reproduction. These are the four kinds of reproduction. 

When\marginnote{34.1} I know and see in this way, suppose someone were to say this: ‘The ascetic Gotama has no superhuman distinction in knowledge and vision worthy of the noble ones …’ Unless they give up that speech and that thought, and let go of that view, they will be cast down to hell. 

There\marginnote{35.1} are these five destinations. What five? Hell, the animal realm, the ghost realm, humanity, and the gods. 

I\marginnote{36.1} understand hell, and the path and practice that leads to hell. And I understand how someone practicing that way, when their body breaks up, after death, is reborn in a place of loss, a bad place, the underworld, hell. I understand the animal realm … the ghost realm … humanity … gods, and the path and practice that leads to the world of the gods. And I understand how someone practicing that way, when their body breaks up, after death, is reborn in a good place, a heavenly realm. And I understand extinguishment, and the path and practice that leads to extinguishment. And I understand how someone practicing that way realizes the undefiled freedom of heart and freedom by wisdom in this very life, and lives having realized it with their own insight due to the ending of defilements. 

When\marginnote{37.1} I’ve comprehended the mind of a certain person, I understand: ‘This person is practicing in such a way and has entered such a path that when their body breaks up, after death, they will be reborn in a place of loss, a bad place, the underworld, hell.’ Then some time later I see that they have indeed been reborn in hell, where they experience exclusively painful feelings, sharp and severe. Suppose there was a pit of glowing coals deeper than a man’s height, full of glowing coals that neither flamed nor smoked. Then along comes a person struggling in the oppressive heat, weary, thirsty, and parched. And they have set out on a path that meets with that same pit of coals. If a person with good eyesight saw them, they’d say: ‘This person is proceeding in such a way and has entered such a path that they will arrive at that very pit of coals.’ Then some time later they see that they have indeed fallen into that pit of coals, where they experience exclusively painful feelings, sharp and severe. … 

When\marginnote{38.1} I’ve comprehended the mind of a certain person, I understand: ‘This person … will be reborn in the animal realm.’ Then some time later I see that they have indeed been reborn in the animal realm, where they suffer painful feelings, sharp and severe. Suppose there was a sewer deeper than a man’s height, full to the brim with feces. Then along comes a person struggling in the oppressive heat, weary, thirsty, and parched. And they have set out on a path that meets with that same sewer. If a person with good eyesight saw them, they’d say: ‘This person is proceeding in such a way and has entered such a path that they will arrive at that very sewer.’ Then some time later they see that they have indeed fallen into that sewer, where they suffer painful feelings, sharp and severe. … 

When\marginnote{39.1} I’ve comprehended the mind of a certain person, I understand: ‘This person … will be reborn in the ghost realm.’ Then some time later I see that they have indeed been reborn in the ghost realm, where they experience many painful feelings. Suppose there was a tree growing on rugged ground, with thin foliage casting dappled shade. Then along comes a person struggling in the oppressive heat, weary, thirsty, and parched. And they have set out on a path that meets with that same tree. If a person with good eyesight saw them, they’d say: ‘This person is proceeding in such a way and has entered such a path that they will arrive at that very tree.’ Then some time later they see them sitting or lying under that tree, where they experience many painful feelings. … 

When\marginnote{40.1} I’ve comprehended the mind of a certain person, I understand: ‘This person … will be reborn among human beings.’ Then some time later I see that they have indeed been reborn among human beings, where they experience many pleasant feelings. Suppose there was a tree growing on smooth ground, with abundant foliage casting dense shade. Then along comes a person struggling in the oppressive heat, weary, thirsty, and parched. And they have set out on a path that meets with that same tree. If a person with good eyesight saw them, they’d say: ‘This person is proceeding in such a way and has entered such a path that they will arrive at that very tree.’ Then some time later they see them sitting or lying under that tree, where they experience many pleasant feelings. … 

When\marginnote{41.1} I’ve comprehended the mind of a certain person, I understand: ‘This person … will be reborn in a good place, a heavenly realm.’ Then some time later I see that they have indeed been reborn in a heavenly realm, where they experience exclusively pleasant feelings. Suppose there was a stilt longhouse with a peaked roof, plastered inside and out, draft-free, with latches fastened and windows shuttered. And it had a couch spread with woolen covers—shag-piled, pure white, or embroidered with flowers—and spread with a fine deer hide, with a canopy above and red pillows at both ends. Then along comes a person struggling in the oppressive heat, weary, thirsty, and parched. And they have set out on a path that meets with that same stilt longhouse. If a person with good eyesight saw them, they’d say: ‘This person is proceeding in such a way and has entered such a path that they will arrive at that very stilt longhouse.’ Then some time later they see them sitting or lying in that stilt longhouse, where they experience exclusively pleasant feelings. … 

When\marginnote{42.1} I’ve comprehended the mind of a certain person, I understand: ‘This person is practicing in such a way and has entered such a path that they will realize the undefiled freedom of heart and freedom by wisdom in this very life, and live having realized it with their own insight due to the ending of defilements.’ Then some time later I see that they have indeed realized the undefiled freedom of heart and freedom by wisdom in this very life, and live having realized it with their own insight due to the ending of defilements, experiencing exclusively pleasant feelings. Suppose there was a lotus pond with clear, sweet, cool water, clean, with smooth banks, delightful. And nearby was a dark forest grove. Then along comes a person struggling in the oppressive heat, weary, thirsty, and parched. And they have set out on a path that meets with that same lotus pond. If a person with good eyesight saw them, they’d say: ‘This person is proceeding in such a way and has entered such a path that they will arrive at that very lotus pond.’ Then some time later they would see that person after they had plunged into that lotus pond, bathed and drunk. When all their stress, weariness, and heat exhaustion had faded away, they emerged and sat or lay down in that woodland thicket, where they experienced exclusively pleasant feelings. 

In\marginnote{42.10} the same way, when I’ve comprehended the mind of a person, I understand: ‘This person is practicing in such a way and has entered such a path that they will realize the undefiled freedom of heart and freedom by wisdom in this very life, and live having realized it with their own insight due to the ending of defilements.’ Then some time later I see that they have indeed realized the undefiled freedom of heart and freedom by wisdom in this very life, and live having realized it with their own insight due to the ending of defilements, experiencing exclusively pleasant feelings. These are the five destinations. 

When\marginnote{43.1} I know and see in this way, suppose someone were to say this: ‘The ascetic Gotama has no superhuman distinction in knowledge and vision worthy of the noble ones. He teaches what he’s worked out by logic, following a line of inquiry, expressing his own perspective.’ Unless they give up that speech and that thought, and let go of that view, they will be cast down to hell. Just as a mendicant accomplished in ethics, immersion, and wisdom would reach enlightenment in this very life, such is the consequence, I say. Unless they give up that speech and thought, and let go of that view, they will be cast down to hell. 

\textsanskrit{Sāriputta},\marginnote{44.1} I recall having practiced a spiritual path consisting of four factors. I used to be a self-mortifier, the ultimate self-mortifier. I used to live rough, the ultimate rough-liver. I used to live in disgust at sin, the ultimate one living in disgust at sin. I used to be secluded, in ultimate seclusion. 

And\marginnote{45.1} this is what my self-mortification was like. I went naked, ignoring conventions. I licked my hands, and didn’t come or stop when asked. I didn’t consent to food brought to me, or food prepared specially for me, or an invitation for a meal. I didn’t receive anything from a pot or bowl; or from someone who keeps sheep, or who has a weapon or a shovel in their home; or where a couple is eating; or where there is a woman who is pregnant, breastfeeding, or who has a man in her home; or where food for distribution is advertised; or where there’s a dog waiting or flies buzzing. I accepted no fish or meat or liquor or wine, and drank no beer. I went to just one house for alms, taking just one mouthful, or two houses and two mouthfuls, up to seven houses and seven mouthfuls. I fed on one saucer a day, two saucers a day, up to seven saucers a day. I ate once a day, once every second day, up to once a week, and so on, even up to once a fortnight. I lived committed to the practice of eating food at set intervals. 

I\marginnote{45.6} ate herbs, millet, wild rice, poor rice, water lettuce, rice bran, scum from boiling rice, sesame flour, grass, or cow dung. I survived on forest roots and fruits, or eating fallen fruit. 

I\marginnote{45.7} wore robes of sunn hemp, mixed hemp, corpse-wrapping cloth, rags, lodh tree bark, antelope hide (whole or in strips), kusa grass, bark, wood-chips, human hair, horse-tail hair, or owls’ wings. I tore out hair and beard, committed to this practice. I constantly stood, refusing seats. I squatted, committed to the endeavor of squatting. I lay on a mat of thorns, making a mat of thorns my bed. I was committed to the practice of immersion in water three times a day, including the evening. And so I lived committed to practicing these various ways of mortifying and tormenting the body. Such was my practice of self-mortification. 

And\marginnote{46.1} this is what my rough living was like. The dust and dirt built up on my body over many years until it started flaking off. It’s like the trunk of a pale-moon ebony tree, which builds up bark over many years until it starts flaking off. But it didn’t occur to me: ‘Oh, this dust and dirt must be rubbed off by my hand or another’s.’ That didn’t occur to me. Such was my rough living. 

And\marginnote{47.1} this is what my living in disgust of sin was like. I’d step forward or back ever so mindfully. I was full of pity even regarding a drop of water, thinking: ‘May I not accidentally injure any little creatures that happen to be in the wrong place.’ Such was my living in disgust of sin. 

And\marginnote{48.1} this is what my seclusion was like. I would plunge deep into a wilderness region and stay there. When I saw a cowherd or a shepherd, or someone gathering grass or sticks, or a lumberjack, I’d flee from forest to forest, from thicket to thicket, from valley to valley, from uplands to uplands. Why is that? So that I wouldn’t see them, nor they me. I fled like a wild deer seeing a human being. Such was my practice of seclusion. 

I\marginnote{49.1} would go on all fours into the cow-pens after the cattle had left and eat the dung of the young suckling calves. As long as my own urine and excrement lasted, I would even eat that. Such was my eating of most unnatural things. 

I\marginnote{50.1} would plunge deep into an awe-inspiring forest grove and stay there. It was so awe-inspiring that normally it would make your hair stand on end if you weren’t free of greed. And on days such as the cold spell when the snow falls in the dead of winter, I stayed in the open by night and in the forest by day. But in the last month of summer I’d stay in the open by day and in the forest by night. And then these verses, which were neither supernaturally inspired, nor learned before in the past, occurred to me: 

\begin{verse}%
‘Scorched\marginnote{50.7} and frozen, \\
alone in the awe-inspiring forest. \\
Naked, no fire to sit beside, \\
the sage still pursues his quest.’ 

%
\end{verse}

I\marginnote{51.1} would make my bed in a charnel ground, with the bones of the dead for a pillow. Then the cowboys would come up to me. They’d spit and piss on me, throw mud on me, even poke sticks in my ears. But I don’t recall ever having a bad thought about them. Such was my abiding in equanimity. 

There\marginnote{52.1} are some ascetics and brahmins who have this doctrine and view: ‘Purity comes from food.’ They say: ‘Let’s live on jujubes.’ So they eat jujubes and jujube powder, and drink jujube juice. And they enjoy many jujube concoctions. I recall eating just a single jujube. You might think that at that time the jujubes must have been very big. But you should not see it like this. The jujubes then were at most the same size as today. Eating so very little, my body became extremely emaciated. Due to eating so little, my limbs became like the joints of an eighty-year-old or a corpse, my bottom became like a camel’s hoof, my vertebrae stuck out like beads on a string, and my ribs were as gaunt as the broken-down rafters on an old barn. Due to eating so little, the gleam of my eyes sank deep in their sockets, like the gleam of water sunk deep down a well. Due to eating so little, my scalp shriveled and withered like a green bitter-gourd in the wind and sun. Due to eating so little, the skin of my belly stuck to my backbone, so that when I tried to rub the skin of my belly I grabbed my backbone, and when I tried to rub my backbone I rubbed the skin of my belly. Due to eating so little, when I tried to urinate or defecate I fell face down right there. Due to eating so little, when I tried to relieve my body by rubbing my limbs with my hands, the hair, rotted at its roots, fell out. 

There\marginnote{53{-}55.1} are some ascetics and brahmins who have this doctrine and view: ‘Purity comes from food.’ They say: ‘Let’s live on mung beans.’ … ‘Let’s live on sesame.’ … ‘Let’s live on ordinary rice.’ … Due to eating so little, when I tried to relieve my body by rubbing my limbs with my hands, the hair, rotted at its roots, fell out. 

But\marginnote{56.1} \textsanskrit{Sāriputta}, I did not achieve any superhuman distinction in knowledge and vision worthy of the noble ones by that conduct, that practice, that grueling work. Why is that? Because I didn’t achieve that noble wisdom that’s noble and emancipating, and which leads someone who practices it to the complete ending of suffering. 

There\marginnote{57.1} are some ascetics and brahmins who have this doctrine and view: ‘Purity comes from transmigration.’ But it’s not easy to find a realm that I haven’t previously transmigrated to in all this long time, except for the gods of the pure abodes. For if I had transmigrated to the gods of the pure abodes I would not have returned to this realm again. 

There\marginnote{58.1} are some ascetics and brahmins who have this doctrine and view: ‘Purity comes from rebirth.’ But it’s not easy to find any rebirth that I haven’t previously been reborn in … 

There\marginnote{59.1} are some ascetics and brahmins who have this doctrine and view: ‘Purity comes from abode of rebirth.’ But it’s not easy to find an abode where I haven’t previously abided … 

There\marginnote{60.1} are some ascetics and brahmins who have this doctrine and view: ‘Purity comes from sacrifice.’ But it’s not easy to find a sacrifice that I haven’t previously offered in all this long time, when I was an anointed aristocratic king or a well-to-do brahmin. 

There\marginnote{61.1} are some ascetics and brahmins who have this doctrine and view: ‘Purity comes from serving the sacred flame.’ But it’s not easy to find a fire that I haven’t previously served in all this long time, when I was an anointed aristocratic king or a well-to-do brahmin. 

There\marginnote{62.1} are some ascetics and brahmins who have this doctrine and view: ‘So long as this gentleman is youthful, young, black-haired, blessed with youth, in the prime of life he will be endowed with perfect lucidity of wisdom. But when he’s old, elderly, and senior, advanced in years, and has reached the final stage of life—eighty, ninety, or a hundred years old—he will lose his lucidity of wisdom.’ But you should not see it like this. For now I am old, elderly, and senior, I’m advanced in years, and have reached the final stage of life. I am eighty years old. Suppose I had four disciples with a lifespan of a hundred years. And they each were perfect in memory, range, retention, and perfect lucidity of wisdom. Imagine how easily a well-trained expert archer with a strong bow would shoot a light arrow across the shadow of a palm tree. That’s how extraordinary they were in memory, range, retention, and perfect lucidity of wisdom. They’d bring up questions about the four kinds of mindfulness meditation again and again, and I would answer each question. They’d remember the answers and not ask the same question twice. And they’d pause only to eat and drink, go to the toilet, and sleep to dispel weariness. But the Realized One would not run out of Dhamma teachings, words and phrases of the teachings, or spontaneous answers. And at the end of a hundred years my four disciples would pass away. Even if you have to carry me around on a stretcher, there will never be any deterioration in the Realized One’s lucidity of wisdom. 

And\marginnote{63.1} if there’s anyone of whom it may be rightly said that a being not liable to delusion has arisen in the world for the welfare and happiness of the people, out of compassion for the world, for the benefit, welfare, and happiness of gods and humans, it’s of me that this should be said.” 

Now\marginnote{64.1} at that time Venerable \textsanskrit{Nāgasamāla} was standing behind the Buddha fanning him. Then he said to the Buddha: 

“It’s\marginnote{64.3} incredible, sir, it’s amazing! While I was listening to this exposition of the teaching my hair stood up! What is the name of this exposition of the teaching?” 

“Well,\marginnote{64.6} \textsanskrit{Nāgasamāla}, you may remember this exposition of the teaching as ‘The Hair-raising Discourse’.” 

That\marginnote{64.7} is what the Buddha said. Satisfied, Venerable \textsanskrit{Nāgasamāla} was happy with what the Buddha said. 

%
\section*{{\suttatitleacronym MN 13}{\suttatitletranslation The Longer Discourse on the Mass of Suffering }{\suttatitleroot Mahādukkhakkhandhasutta}}
\addcontentsline{toc}{section}{\tocacronym{MN 13} \toctranslation{The Longer Discourse on the Mass of Suffering } \tocroot{Mahādukkhakkhandhasutta}}
\markboth{The Longer Discourse on the Mass of Suffering }{Mahādukkhakkhandhasutta}
\extramarks{MN 13}{MN 13}

\scevam{So\marginnote{1.1} I have heard. }At one time the Buddha was staying near \textsanskrit{Sāvatthī} in Jeta’s Grove, \textsanskrit{Anāthapiṇḍika}’s monastery. 

Then\marginnote{2.1} several mendicants robed up in the morning and, taking their bowls and robes, entered \textsanskrit{Sāvatthī} for alms. Then it occurred to them, “It’s too early to wander for alms in \textsanskrit{Sāvatthī}. Why don’t we go to the monastery of the wanderers who follow other paths?” Then they went to the monastery of the wanderers who follow other paths, and exchanged greetings with the wanderers there. When the greetings and polite conversation were over, they sat down to one side. The wanderers said to them: 

“Reverends,\marginnote{3.1} the ascetic Gotama advocates the complete understanding of sensual pleasures, and so do we. The ascetic Gotama advocates the complete understanding of sights, and so do we. The ascetic Gotama advocates the complete understanding of feelings, and so do we. What, then, is the difference between the ascetic Gotama’s teaching and instruction and ours?” 

Those\marginnote{4.1} mendicants neither approved nor dismissed that statement of the wanderers who follow other paths. They got up from their seat, thinking, “We will learn the meaning of this statement from the Buddha himself.” 

Then,\marginnote{5.1} after the meal, when they returned from almsround, they went up to the Buddha, bowed, sat down to one side, and told him what had happened. The Buddha said: 

“Mendicants,\marginnote{6.1} when wanderers who follow other paths say this, you should say to them: ‘But reverends, what’s the gratification, the drawback, and the escape when it comes to sensual pleasures? What’s the gratification, the drawback, and the escape when it comes to sights? What’s the gratification, the drawback, and the escape when it comes to feelings?’ Questioned like this, the wanderers who follow other paths would be stumped, and, in addition, would get frustrated. Why is that? Because they’re out of their element. I don’t see anyone in this world—with its gods, \textsanskrit{Māras}, and \textsanskrit{Brahmās}, this population with its ascetics and brahmins, its gods and humans—who could provide a satisfying answer to these questions except for the Realized One or his disciple or someone who has heard it from them. 

And\marginnote{7.1} what is the gratification of sensual pleasures? There are these five kinds of sensual stimulation. What five? Sights known by the eye that are likable, desirable, agreeable, pleasant, sensual, and arousing. Sounds known by the ear … Smells known by the nose … Tastes known by the tongue … Touches known by the body that are likable, desirable, agreeable, pleasant, sensual, and arousing. These are the five kinds of sensual stimulation. The pleasure and happiness that arise from these five kinds of sensual stimulation: this is the gratification of sensual pleasures. 

And\marginnote{8.1} what is the drawback of sensual pleasures? It’s when a gentleman earns a living by means such as computing, accounting, calculating, farming, trade, raising cattle, archery, government service, or one of the professions. But they must face cold and heat, being hurt by the touch of flies, mosquitoes, wind, sun, and reptiles, and risking death from hunger and thirst. This is a drawback of sensual pleasures apparent in this very life, a mass of suffering caused by sensual pleasures. 

That\marginnote{9.1} gentleman might try hard, strive, and make an effort, but fail to earn any money. If this happens, they sorrow and wail and lament, beating their breast and falling into confusion, saying: ‘Oh, my hard work is wasted. My efforts are fruitless!’ This too is a drawback of sensual pleasures apparent in this very life, a mass of suffering caused by sensual pleasures. 

That\marginnote{10.1} gentleman might try hard, strive, and make an effort, and succeed in earning money. But they experience pain and sadness when they try to protect it, thinking: ‘How can I prevent my wealth from being taken by rulers or bandits, consumed by fire, swept away by flood, or taken by unloved heirs?’ And even though they protect it and ward it, rulers or bandits take it, or fire consumes it, or flood sweeps it away, or unloved heirs take it. They sorrow and wail and lament, beating their breast and falling into confusion: ‘What used to be mine is gone.’ This too is a drawback of sensual pleasures apparent in this very life, a mass of suffering caused by sensual pleasures. 

Furthermore,\marginnote{11.1} for the sake of sensual pleasures kings fight with kings, aristocrats fight with aristocrats, brahmins fight with brahmins, and householders fight with householders. A mother fights with her child, child with mother, father with child, and child with father. Brother fights with brother, brother with sister, sister with brother, and friend fights with friend. Once they’ve started quarreling, arguing, and disputing, they attack each other with fists, stones, rods, and swords, resulting in death and deadly pain. This too is a drawback of sensual pleasures apparent in this very life, a mass of suffering caused by sensual pleasures. 

Furthermore,\marginnote{12.1} for the sake of sensual pleasures they don their sword and shield, fasten their bow and arrows, and plunge into a battle massed on both sides, with arrows and spears flying and swords flashing. There they are struck with arrows and spears, and their heads are chopped off, resulting in death and deadly pain. This too is a drawback of sensual pleasures apparent in this very life, a mass of suffering caused by sensual pleasures. 

Furthermore,\marginnote{13.1} for the sake of sensual pleasures they don their sword and shield, fasten their bow and arrows, and charge wetly plastered bastions, with arrows and spears flying and swords flashing. There they are struck with arrows and spears, splashed with dung, crushed with spiked blocks, and their heads are chopped off, resulting in death and deadly pain. This too is a drawback of sensual pleasures apparent in this very life, a mass of suffering caused by sensual pleasures. 

Furthermore,\marginnote{14.1} for the sake of sensual pleasures they break into houses, plunder wealth, steal from isolated buildings, commit highway robbery, and commit adultery. The rulers would arrest them and subject them to various punishments—whipping, caning, and clubbing; cutting off hands or feet, or both; cutting off ears or nose, or both; the ‘porridge pot’, the ‘shell-shave’, the ‘demon’s mouth’, the ‘garland of fire’, the ‘burning hand’, the ‘grass blades’, the ‘bark dress’, the ‘antelope’, the ‘meat hook’, the ‘coins’, the ‘caustic pickle’, the ‘twisting bar’, the ‘straw mat’; being splashed with hot oil, being fed to the dogs, being impaled alive, and being beheaded. These result in death and deadly pain. This too is a drawback of sensual pleasures apparent in this very life, a mass of suffering caused by sensual pleasures. 

Furthermore,\marginnote{15.1} for the sake of sensual pleasures, they conduct themselves badly by way of body, speech, and mind. When their body breaks up, after death, they’re reborn in a place of loss, a bad place, the underworld, hell. This is a drawback of sensual pleasures to do with lives to come, a mass of suffering caused by sensual pleasures. 

And\marginnote{16.1} what is the escape from sensual pleasures? Removing and giving up desire and greed for sensual pleasures: this is the escape from sensual pleasures. 

There\marginnote{17.1} are ascetics and brahmins who don’t truly understand sensual pleasures’ gratification, drawback, and escape in this way for what they are. It’s impossible for them to completely understand sensual pleasures themselves, or to instruct another so that, practicing accordingly, they will completely understand sensual pleasures. There are ascetics and brahmins who do truly understand sensual pleasures’ gratification, drawback, and escape in this way for what they are. It is possible for them to completely understand sensual pleasures themselves, or to instruct another so that, practicing accordingly, they will completely understand sensual pleasures. 

And\marginnote{18.1} what is the gratification of sights? Suppose there was a girl of the brahmins, aristocrats, or householders in her fifteenth or sixteenth year, neither too tall nor too short, neither too thin nor too fat, neither too dark nor too fair. Is she not at the height of her beauty and prettiness?” 

“Yes,\marginnote{18.3} sir.” 

“The\marginnote{18.4} pleasure and happiness that arise from this beauty and prettiness is the gratification of sights. 

And\marginnote{19.1} what is the drawback of sights? Suppose that some time later you were to see that same sister—eighty, ninety, or a hundred years old—bent double, crooked, leaning on a staff, trembling as they walk, ailing, past their prime, with teeth broken, hair grey and scanty or bald, skin wrinkled, and limbs blotchy. 

What\marginnote{19.3} do you think, mendicants? Has not that former beauty vanished and the drawback become clear?” 

“Yes,\marginnote{19.5} sir.” 

“This\marginnote{19.6} is the drawback of sights. 

Furthermore,\marginnote{20.1} suppose that you were to see that same sister sick, suffering, gravely ill, collapsed in her own urine and feces, being picked up by some and put down by others. 

What\marginnote{20.2} do you think, mendicants? Has not that former beauty vanished and the drawback become clear?” 

“Yes,\marginnote{20.4} sir.” 

“This\marginnote{20.5} too is the drawback of sights. 

Furthermore,\marginnote{21.1} suppose that you were to see that same sister as a corpse discarded in a charnel ground. And she had been dead for one, two, or three days, bloated, livid, and festering. 

What\marginnote{21.3} do you think, mendicants? Has not that former beauty vanished and the drawback become clear?” 

“Yes,\marginnote{21.5} sir.” 

“This\marginnote{21.6} too is the drawback of sights. 

Furthermore,\marginnote{22.1} suppose that you were to see that same sister as a corpse discarded in a charnel ground. And she was being devoured by crows, hawks, vultures, herons, dogs, tigers, leopards, jackals, and many kinds of little creatures … 

Furthermore,\marginnote{23{-}28.1} suppose that you were to see that same sister as a corpse discarded in a charnel ground. And she had been reduced to a skeleton with flesh and blood, held together by sinews … a skeleton rid of flesh but smeared with blood, and held together by sinews … a skeleton rid of flesh and blood, held together by sinews … bones rid of sinews scattered in every direction. Here a hand-bone, there a foot-bone, here a shin-bone, there a thigh-bone, here a hip-bone, there a rib-bone, here a back-bone, there an arm-bone, here a neck-bone, there a jaw-bone, here a tooth, there the skull. … 

Furthermore,\marginnote{29.1} suppose that you were to see that same sister as a corpse discarded in a charnel ground. And she had been reduced to white bones, the color of shells … decrepit bones, heaped in a pile … bones rotted and crumbled to powder. 

What\marginnote{29.3} do you think, mendicants? Has not that former beauty vanished and the drawback become clear?” 

“Yes,\marginnote{29.5} sir.” 

“This\marginnote{29.6} too is the drawback of sights. 

And\marginnote{30.1} what is the escape from sights? Removing and giving up desire and greed for sights: this is the escape from sights. 

There\marginnote{31.1} are ascetics and brahmins who don’t truly understand sights’ gratification, drawback, and escape in this way for what they are. It’s impossible for them to completely understand sights themselves, or to instruct another so that, practicing accordingly, they will completely understand sights. There are ascetics and brahmins who do truly understand sights’ gratification, drawback, and escape in this way for what they are. It is possible for them to completely understand sights themselves, or to instruct another so that, practicing accordingly, they will completely understand sights. 

And\marginnote{32.1} what is the gratification of feelings? It’s when a mendicant, quite secluded from sensual pleasures, secluded from unskillful qualities, enters and remains in the first absorption, which has the rapture and bliss born of seclusion, while placing the mind and keeping it connected. At that time a mendicant doesn’t intend to hurt themselves, hurt others, or hurt both; they feel only feelings that are not hurtful. Freedom from being hurt is the ultimate gratification of feelings, I say. 

Furthermore,\marginnote{33{-}35.1} a mendicant enters and remains in the second absorption … third absorption … fourth absorption. At that time a mendicant doesn’t intend to hurt themselves, hurt others, or hurt both; they feel only feelings that are not hurtful. Freedom from being hurt is the ultimate gratification of feelings, I say. 

And\marginnote{36.1} what is the drawback of feelings? That feelings are impermanent, suffering, and perishable: this is their drawback. 

And\marginnote{37.1} what is the escape from feelings? Removing and giving up desire and greed for feelings: this is the escape from feelings. 

There\marginnote{38.1} are ascetics and brahmins who don’t truly understand feelings’ gratification, drawback, and escape in this way for what they are. It’s impossible for them to completely understand feelings themselves, or to instruct another so that, practicing accordingly, they will completely understand feelings. There are ascetics and brahmins who do truly understand feelings’ gratification, drawback, and escape in this way for what they are. It is possible for them to completely understand feelings themselves, or to instruct another so that, practicing accordingly, they will completely understand feelings.” 

That\marginnote{38.3} is what the Buddha said. Satisfied, the mendicants were happy with what the Buddha said. 

%
\section*{{\suttatitleacronym MN 14}{\suttatitletranslation The Shorter Discourse on the Mass of Suffering }{\suttatitleroot Cūḷadukkhakkhandhasutta}}
\addcontentsline{toc}{section}{\tocacronym{MN 14} \toctranslation{The Shorter Discourse on the Mass of Suffering } \tocroot{Cūḷadukkhakkhandhasutta}}
\markboth{The Shorter Discourse on the Mass of Suffering }{Cūḷadukkhakkhandhasutta}
\extramarks{MN 14}{MN 14}

\scevam{So\marginnote{1.1} I have heard. }At one time the Buddha was staying in the land of the Sakyans, near Kapilavatthu in the Banyan Tree Monastery. 

Then\marginnote{2.1} \textsanskrit{Mahānāma} the Sakyan went up to the Buddha, bowed, sat down to one side, and said to him, “For a long time, sir, I have understood your teaching like this: ‘Greed, hate, and delusion are corruptions of the mind.’ Despite understanding this, sometimes my mind is occupied by thoughts of greed, hate, and delusion. I wonder what qualities remain in me that I have such thoughts?” 

“\textsanskrit{Mahānāma},\marginnote{3.1} there is a quality that remains in you that makes you have such thoughts. For if you had given up that quality you would not still be living at home and enjoying sensual pleasures. But because you haven’t given up that quality you are still living at home and enjoying sensual pleasures. 

Sensual\marginnote{4.1} pleasures give little gratification and much suffering and distress, and they are all the more full of drawbacks. Even though a noble disciple has clearly seen this with right wisdom, so long as they don’t achieve the rapture and bliss that are apart from sensual pleasures and unskillful qualities, or something even more peaceful than that, they might still return to sensual pleasures. But when they do achieve that rapture and bliss, or something more peaceful than that, they will not return to sensual pleasures. 

Before\marginnote{5.1} my awakening—when I was still unawakened but intent on awakening—I too clearly saw with right wisdom that: ‘Sensual pleasures give little gratification and much suffering and distress, and they are all the more full of drawbacks.’ But so long as I didn’t achieve the rapture and bliss that are apart from sensual pleasures and unskillful qualities, or something even more peaceful than that, I didn’t announce that I would not return to sensual pleasures. But when I did achieve that rapture and bliss, or something more peaceful than that, I announced that I would not return to sensual pleasures. 

And\marginnote{6.1} what is the gratification of sensual pleasures? There are these five kinds of sensual stimulation. What five? Sights known by the eye that are likable, desirable, agreeable, pleasant, sensual, and arousing. Sounds known by the ear … Smells known by the nose … Tastes known by the tongue … Touches known by the body that are likable, desirable, agreeable, pleasant, sensual, and arousing. These are the five kinds of sensual stimulation. The pleasure and happiness that arise from these five kinds of sensual stimulation: this is the gratification of sensual pleasures. 

And\marginnote{7.1} what is the drawback of sensual pleasures? It’s when a gentleman earns a living by means such as computing, accounting, calculating, farming, trade, raising cattle, archery, government service, or one of the professions. But they must face cold and heat, being hurt by the touch of flies, mosquitoes, wind, sun, and reptiles, and risking death from hunger and thirst. This is a drawback of sensual pleasures apparent in this very life, a mass of suffering caused by sensual pleasures. 

That\marginnote{8.1} gentleman might try hard, strive, and make an effort, but fail to earn any money. If this happens, they sorrow and wail and lament, beating their breast and falling into confusion, saying: ‘Oh, my hard work is wasted. My efforts are fruitless!’ This too is a drawback of sensual pleasures apparent in this very life, a mass of suffering caused by sensual pleasures. 

That\marginnote{9.1} gentleman might try hard, strive, and make an effort, and succeed in earning money. But they experience pain and sadness when they try to protect it, thinking: ‘How can I prevent my wealth from being taken by rulers or bandits, consumed by fire, swept away by flood, or taken by unloved heirs?’ And even though they protect it and ward it, rulers or bandits take it, or fire consumes it, or flood sweeps it away, or unloved heirs take it. They sorrow and wail and lament, beating their breast and falling into confusion: ‘What used to be mine is gone.’ This too is a drawback of sensual pleasures apparent in this very life, a mass of suffering caused by sensual pleasures. 

Furthermore,\marginnote{10.1} for the sake of sensual pleasures kings fight with kings, aristocrats fight with aristocrats, brahmins fight with brahmins, and householders fight with householders. A mother fights with her child, child with mother, father with child, and child with father. Brother fights with brother, brother with sister, sister with brother, and friend fights with friend. Once they’ve started quarreling, arguing, and disputing, they attack each other with fists, stones, rods, and swords, resulting in death and deadly pain. This too is a drawback of sensual pleasures apparent in this very life, a mass of suffering caused by sensual pleasures. 

Furthermore,\marginnote{11.1} for the sake of sensual pleasures they don their sword and shield, fasten their bow and arrows, and plunge into a battle massed on both sides, with arrows and spears flying and swords flashing. There they are struck with arrows and spears, and their heads are chopped off, resulting in death and deadly pain. This too is a drawback of sensual pleasures apparent in this very life, a mass of suffering caused by sensual pleasures. 

Furthermore,\marginnote{12.1} for the sake of sensual pleasures they don their sword and shield, fasten their bow and arrows, and charge wetly plastered bastions, with arrows and spears flying and swords flashing. There they are struck with arrows and spears, splashed with dung, crushed with spiked blocks, and their heads are chopped off, resulting in death and deadly pain. This too is a drawback of sensual pleasures apparent in this very life, a mass of suffering caused by sensual pleasures. 

Furthermore,\marginnote{13.1} for the sake of sensual pleasures they break into houses, plunder wealth, steal from isolated buildings, commit highway robbery, and commit adultery. The rulers would arrest them and subject them to various punishments—whipping, caning, and clubbing; cutting off hands or feet, or both; cutting off ears or nose, or both; the ‘porridge pot’, the ‘shell-shave’, the ‘demon’s mouth’, the ‘garland of fire’, the ‘burning hand’, the ‘grass blades’, the ‘bark dress’, the ‘antelope’, the ‘meat hook’, the ‘coins’, the ‘caustic pickle’, the ‘twisting bar’, the ‘straw mat’; being splashed with hot oil, being fed to the dogs, being impaled alive, and being beheaded. These result in death and deadly pain. This too is a drawback of sensual pleasures apparent in this very life, a mass of suffering caused by sensual pleasures. 

Furthermore,\marginnote{14.1} for the sake of sensual pleasures, they conduct themselves badly by way of body, speech, and mind. When their body breaks up, after death, they’re reborn in a place of loss, a bad place, the underworld, hell. This is a drawback of sensual pleasures to do with lives to come, a mass of suffering caused by sensual pleasures. 

\textsanskrit{Mahānāma},\marginnote{15.1} this one time I was staying near \textsanskrit{Rājagaha}, on the Vulture’s Peak Mountain. Now at that time several Jain ascetics on the slopes of Isigili at the Black Rock were constantly standing, refusing seats. And they felt painful, sharp, severe, acute feelings due to overexertion. 

Then\marginnote{16.1} in the late afternoon, I came out of retreat and went to the Black Rock to visit those Jain ascetics. I said to them, ‘Reverends, why are you constantly standing, refusing seats, so that you suffer painful, sharp, severe, acute feelings due to overexertion?’ 

When\marginnote{17.1} I said this, those Jain ascetics said to me, ‘Reverend, the Jain leader \textsanskrit{Nātaputta} claims to be all-knowing and all-seeing, to know and see everything without exception, thus: “Knowledge and vision are constantly and continually present to me, while walking, standing, sleeping, and waking.” 

He\marginnote{17.4} says, “O Jain ascetics, you have done bad deeds in a past life. Wear them away with these severe and grueling austerities. And when you refrain from such deeds in the present by way of body, speech, and mind, you’re not doing any bad deeds for the future. So, due to eliminating past deeds by mortification, and not doing any new deeds, there’s nothing to come up in the future. With nothing to come up in the future, deeds end. With the ending of deeds, suffering ends. With the ending of suffering, feeling ends. And with the ending of feeling, all suffering will have been worn away.” We like and accept this, and we are satisfied with it.’ 

When\marginnote{18.1} they said this, I said to them, ‘But reverends, do you know for sure that you existed in the past, and it is not the case that you didn’t exist?’ 

‘No\marginnote{18.4} we don’t, reverend.’ 

‘But\marginnote{18.5} reverends, do you know for sure that you did bad deeds in the past?’ 

‘No\marginnote{18.7} we don’t, reverend.’ 

‘But\marginnote{18.8} reverends, do you know that you did such and such bad deeds?’ 

‘No\marginnote{18.10} we don’t, reverend.’ 

‘But\marginnote{18.11} reverends, do you know that so much suffering has already been worn away? Or that so much suffering still remains to be worn away? Or that when so much suffering is worn away all suffering will have been worn away?’ 

‘No\marginnote{18.13} we don’t, reverend.’ 

‘But\marginnote{18.14} reverends, do you know about giving up unskillful qualities in the present life and embracing skillful qualities?’ 

‘No\marginnote{18.16} we don’t, reverend.’ 

‘So\marginnote{19.1} it seems that you don’t know any of these things. That being so, when those in the world who are violent and bloody-handed and make their living by cruelty are reborn among humans they go forth as Jain ascetics.’ 

‘Reverend\marginnote{20.1} Gotama, pleasure is not gained through pleasure; pleasure is gained through pain. For if pleasure were to be gained through pleasure, King Seniya \textsanskrit{Bimbisāra} of \textsanskrit{Māgadha} would gain pleasure, since he lives in greater pleasure than Venerable Gotama.’ 

‘Clearly\marginnote{20.3} the venerables have spoken rashly, without reflection. Rather, I’m the one who should be asked about who lives in greater pleasure, King \textsanskrit{Bimbisāra} or Venerable Gotama?’ 

‘Clearly\marginnote{20.8} we spoke rashly and without reflection. But forget about that. Now we ask Venerable Gotama: “Who lives in greater pleasure, King \textsanskrit{Bimbisāra} or Venerable Gotama?”' 

‘Well\marginnote{21.1} then, reverends, I’ll ask you about this in return, and you can answer as you like. What do you think, reverends? Is King \textsanskrit{Bimbisāra} capable of experiencing perfect happiness for seven days and nights without moving his body or speaking?’ 

‘No\marginnote{21.4} he is not, reverend.’ 

‘What\marginnote{21.5} do you think, reverends? Is King \textsanskrit{Bimbisāra} capable of experiencing perfect happiness for six days … five days … four days … three days … two days … one day?’ 

‘No\marginnote{21.12} he is not, reverend.’ 

‘But\marginnote{21.13} I am capable of experiencing perfect happiness for one day and night without moving my body or speaking. I am capable of experiencing perfect happiness for two days … three days … four days … five days … six days … seven days. What do you think, reverends? This being so, who lives in greater pleasure, King \textsanskrit{Bimbisāra} or I?’ 

‘This\marginnote{21.21} being so, Venerable Gotama lives in greater pleasure than King \textsanskrit{Bimbisāra}.’” 

That\marginnote{21.22} is what the Buddha said. Satisfied, \textsanskrit{Mahānāma} the Sakyan was happy with what the Buddha said. 

%
\section*{{\suttatitleacronym MN 15}{\suttatitletranslation Measuring Up }{\suttatitleroot Anumānasutta}}
\addcontentsline{toc}{section}{\tocacronym{MN 15} \toctranslation{Measuring Up } \tocroot{Anumānasutta}}
\markboth{Measuring Up }{Anumānasutta}
\extramarks{MN 15}{MN 15}

\scevam{So\marginnote{1.1} I have heard. }At one time Venerable \textsanskrit{Mahāmoggallāna} was staying in the land of the Bhaggas on Crocodile Hill, in the deer park at \textsanskrit{Bhesakaḷā}’s Wood. There Venerable \textsanskrit{Mahāmoggallāna} addressed the mendicants: “Reverends, mendicants!” 

“Reverend,”\marginnote{1.5} they replied. Venerable \textsanskrit{Mahāmoggallāna} said this: 

“Suppose\marginnote{2.1} a mendicant invites other mendicants to admonish them. But they’re hard to admonish, having qualities that make them hard to admonish. They're impatient, and don't take instruction respectfully. So their spiritual companions don’t think it’s worth advising and instructing them, and that person doesn’t gain their trust. 

And\marginnote{3.1} what are the qualities that make them hard to admonish? Firstly, a mendicant has wicked desires, having fallen under the sway of wicked desires. This is a quality that makes them difficult to admonish. 

Furthermore,\marginnote{3.5} a mendicant glorifies themselves and puts others down. … 

They’re\marginnote{3.8} irritable, overcome by anger … 

They’re\marginnote{3.11} irritable, and hostile due to anger … 

They’re\marginnote{3.14} irritable, and stubborn due to anger … 

They’re\marginnote{3.17} irritable, and blurt out words bordering on anger … 

When\marginnote{3.20} accused, they object to the accuser … 

When\marginnote{3.23} accused, they rebuke the accuser … 

When\marginnote{3.26} accused, they retort to the accuser … 

When\marginnote{3.29} accused, they dodge the issue, distract the discussion with irrelevant points, and display annoyance, hate, and bitterness … 

When\marginnote{3.32} accused, they are unable to account for their behavior … 

They\marginnote{3.35} are offensive and contemptuous … 

They’re\marginnote{3.38} jealous and stingy … 

They’re\marginnote{3.41} devious and deceitful … 

They’re\marginnote{3.44} obstinate and vain … 

Furthermore,\marginnote{3.47} a mendicant is attached to their own views, holding them tight, and refusing to let go. This too is a quality that makes them difficult to admonish. 

These\marginnote{3.50} are the qualities that make them hard to admonish. 

Suppose\marginnote{4.1} a mendicant doesn’t invite other mendicants to admonish them. But they’re easy to admonish, having qualities that make them easy to admonish. They're accepting, and take instruction respectfully. So their spiritual companions think it’s worth advising and instructing them, and that person gains their trust. 

And\marginnote{5.1} what are the qualities that make them easy to admonish? Firstly, a mendicant doesn’t have wicked desires … 

Furthermore,\marginnote{5.47} a mendicant isn’t attached to their own views, not holding them tight, but letting them go easily. 

These\marginnote{5.50} are the qualities that make them easy to admonish. 

In\marginnote{6.1} such a case, a mendicant should measure themselves against another like this. ‘This person has wicked desires, having fallen under the sway of wicked desires. And I don’t like or approve of this person. And if I were to fall under the sway of wicked desires, others wouldn’t like or approve of me.’ A mendicant who knows this should give rise to the thought: ‘I will not fall under the sway of wicked desires.’ … 

‘This\marginnote{6.47} person is attached to their own views, holding them tight and refusing to let go. And I don’t like or approve of this person. And if I were to be attached to my own views, holding them tight and refusing to let go, others wouldn’t like or approve of me.’ A mendicant who knows this should give rise to the thought: ‘I will not be attached to my own views, holding them tight, but will let them go easily.’ 

In\marginnote{7.1} such a case, a mendicant should check themselves like this: ‘Do I have wicked desires? Have I fallen under the sway of wicked desires?’ Suppose that, upon checking, a mendicant knows that they have fallen under the sway of wicked desires. Then they should make an effort to give up those bad, unskillful qualities. But suppose that, upon checking, a mendicant knows that they haven’t fallen under the sway of wicked desires. Then they should meditate with rapture and joy, training day and night in skillful qualities. … 

Suppose\marginnote{7.87} that, upon checking, a mendicant knows that they are attached to their own views, holding them tight, and refusing to let go. Then they should make an effort to give up those bad, unskillful qualities. Suppose that, upon checking, a mendicant knows that they’re not attached to their own views, holding them tight, but let them go easily. Then they should meditate with rapture and joy, training day and night in skillful qualities. 

Suppose\marginnote{8.1} that, upon checking, a mendicant sees that they haven’t given up all these bad, unskillful qualities. Then they should make an effort to give them all up. But suppose that, upon checking, a mendicant sees that they have given up all these bad, unskillful qualities. Then they should meditate with rapture and joy, training day and night in skillful qualities. 

Suppose\marginnote{8.3} there was a woman or man who was young, youthful, and fond of adornments, and they check their own reflection in a clean bright mirror or a clear bowl of water. If they see any dirt or blemish there, they’d try to remove it. But if they don’t see any dirt or blemish there, they’re happy, thinking: ‘How fortunate that I’m clean!’ 

In\marginnote{8.6} the same way, suppose that, upon checking, a mendicant sees that they haven’t given up all these bad, unskillful qualities. Then they should make an effort to give them all up. But suppose that, upon checking, a mendicant sees that they have given up all these bad, unskillful qualities. Then they should meditate with rapture and joy, training day and night in skillful qualities.” 

This\marginnote{8.8} is what Venerable \textsanskrit{Mahāmoggallāna} said. Satisfied, the mendicants were happy with what Venerable \textsanskrit{Mahāmoggallāna} said. 

%
\section*{{\suttatitleacronym MN 16}{\suttatitletranslation Emotional Barrenness }{\suttatitleroot Cetokhilasutta}}
\addcontentsline{toc}{section}{\tocacronym{MN 16} \toctranslation{Emotional Barrenness } \tocroot{Cetokhilasutta}}
\markboth{Emotional Barrenness }{Cetokhilasutta}
\extramarks{MN 16}{MN 16}

\scevam{So\marginnote{1.1} I have heard. }At one time the Buddha was staying near \textsanskrit{Sāvatthī} in Jeta’s Grove, \textsanskrit{Anāthapiṇḍika}’s monastery. There the Buddha addressed the mendicants, “Mendicants!” 

“Venerable\marginnote{1.5} sir,” they replied. The Buddha said this: 

“Mendicants,\marginnote{2.1} when a mendicant has not given up five kinds of emotional barrenness and cut off five emotional shackles, it’s not possible for them to achieve growth, improvement, or maturity in this teaching and training. 

What\marginnote{3.1} are the five kinds of emotional barrenness they haven’t given up? Firstly, a mendicant has doubts about the Teacher. They’re uncertain, undecided, and lacking confidence. This being so, their mind doesn’t incline toward keenness, commitment, persistence, and striving. This is the first kind of emotional barrenness they haven’t given up. 

Furthermore,\marginnote{4.1} a mendicant has doubts about the teaching … This is the second kind of emotional barrenness. 

They\marginnote{5.1} have doubts about the \textsanskrit{Saṅgha} … This is the third kind of emotional barrenness. 

They\marginnote{6.1} have doubts about the training … This is the fourth kind of emotional barrenness. 

Furthermore,\marginnote{7.1} a mendicant is angry and upset with their spiritual companions, resentful and closed off. This being so, their mind doesn’t incline toward keenness, commitment, persistence, and striving. This is the fifth kind of emotional barrenness they haven’t given up. These are the five kinds of emotional barrenness they haven’t given up. 

What\marginnote{8.1} are the five emotional shackles they haven’t cut off? Firstly, a mendicant isn’t free of greed, desire, fondness, thirst, passion, and craving for sensual pleasures. This being so, their mind doesn’t incline toward keenness, commitment, persistence, and striving. This is the first emotional shackle they haven’t cut off. 

Furthermore,\marginnote{9.1} a mendicant isn’t free of greed for the body … This is the second emotional shackle. 

Furthermore,\marginnote{10.1} a mendicant isn’t free of greed for form … This is the third emotional shackle. 

They\marginnote{11.1} eat as much as they like until their belly is full, then indulge in the pleasures of sleeping, lying down, and drowsing … This is the fourth emotional shackle. 

They\marginnote{12.1} lead the spiritual life hoping to be reborn in one of the orders of gods, thinking: ‘By this precept or observance or mortification or spiritual life, may I become one of the gods!’ This being so, their mind doesn’t incline toward keenness, commitment, persistence, and striving. This is the fifth emotional shackle they haven’t cut off. These are the five emotional shackles they haven’t cut off. 

When\marginnote{13.1} a mendicant has not given up these five kinds of emotional barrenness and cut off these five emotional shackles, it’s not possible for them to achieve growth, improvement, or maturity in this teaching and training. 

When\marginnote{14.1} a mendicant has given up these five kinds of emotional barrenness and cut off these five emotional shackles, it is possible for them to achieve growth, improvement, and maturity in this teaching and training. 

What\marginnote{15.1} are the five kinds of emotional barrenness they’ve given up? Firstly, a mendicant has no doubts about the Teacher. They’re not uncertain, undecided, or lacking confidence. This being so, their mind inclines toward keenness, commitment, persistence, and striving. This is the first kind of emotional barrenness they’ve given up. 

Furthermore,\marginnote{16.1} a mendicant has no doubts about the teaching … 

They\marginnote{17.1} have no doubts about the \textsanskrit{Saṅgha} … 

They\marginnote{18.1} have no doubts about the training … 

They’re\marginnote{19.1} not angry and upset with their spiritual companions, not resentful or closed off. This being so, their mind inclines toward keenness, commitment, persistence, and striving. This is the fifth kind of emotional barrenness they’ve given up. These are the five kinds of emotional barrenness they’ve given up. 

What\marginnote{20.1} are the five emotional shackles they’ve cut off? Firstly, a mendicant is rid of greed, desire, fondness, thirst, passion, and craving for sensual pleasures. This being so, their mind inclines toward keenness, commitment, persistence, and striving. This is the first emotional shackle they’ve cut off. 

Furthermore,\marginnote{21.1} a mendicant is rid of greed for the body … 

They’re\marginnote{22.1} rid of greed for form … 

They\marginnote{23.1} don’t eat as much as they like until their belly is full, then indulge in the pleasures of sleeping, lying down, and drowsing … 

They\marginnote{24.1} don’t lead the spiritual life hoping to be reborn in one of the orders of gods, thinking: ‘By this precept or observance or mortification or spiritual life, may I become one of the gods!’ This being so, their mind inclines toward keenness, commitment, persistence, and striving. This is the fifth emotional shackle they’ve cut off. These are the five emotional shackles they’ve cut off. 

When\marginnote{25.1} a mendicant has given up these five kinds of emotional barrenness and cut off these five emotional shackles, it is possible for them to achieve growth, improvement, or maturity in this teaching and training. 

They\marginnote{26.1} develop the basis of psychic power that has immersion due to enthusiasm, and active effort … the basis of psychic power that has immersion due to energy, and active effort … the basis of psychic power that has immersion due to mental development, and active effort … the basis of psychic power that has immersion due to inquiry, and active effort. And the fifth is sheer vigor. A mendicant who possesses these fifteen factors, including vigor, is capable of breaking out, becoming awakened, and reaching the supreme sanctuary. Suppose there was a chicken with eight or ten or twelve eggs. And she properly sat on them to keep them warm and incubated. Even if that chicken doesn’t wish: ‘If only my chicks could break out of the eggshell with their claws and beak and hatch safely!’ Still they can break out and hatch safely. 

In\marginnote{27.7} the same way, a mendicant who possesses these fifteen factors, including vigor, is capable of breaking out, becoming awakened, and reaching the supreme sanctuary.” 

That\marginnote{27.8} is what the Buddha said. Satisfied, the mendicants were happy with what the Buddha said. 

%
\section*{{\suttatitleacronym MN 17}{\suttatitletranslation Jungle Thickets }{\suttatitleroot Vanapatthasutta}}
\addcontentsline{toc}{section}{\tocacronym{MN 17} \toctranslation{Jungle Thickets } \tocroot{Vanapatthasutta}}
\markboth{Jungle Thickets }{Vanapatthasutta}
\extramarks{MN 17}{MN 17}

\scevam{So\marginnote{1.1} I have heard. }At one time the Buddha was staying near \textsanskrit{Sāvatthī} in Jeta’s Grove, \textsanskrit{Anāthapiṇḍika}’s monastery. There the Buddha addressed the mendicants, “Mendicants!” 

“Venerable\marginnote{1.5} sir,” they replied. The Buddha said this: 

“Mendicants,\marginnote{2.1} I will teach you an exposition about jungle thickets. Listen and pay close attention, I will speak.” 

“Yes,\marginnote{2.3} sir,” they replied. The Buddha said this: 

“Mendicants,\marginnote{3.1} take the case of a mendicant who lives close by a jungle thicket. As they do so, their mindfulness does not become established, their mind does not become immersed in \textsanskrit{samādhi}, their defilements do not come to an end, and they do not arrive at the supreme sanctuary. And the necessities of life that a renunciate requires—robes, almsfood, lodgings, and medicines and supplies for the sick—are hard to come by. That mendicant should reflect: ‘While living close by this jungle thicket, my mindfulness does not become established, my mind does not become immersed in \textsanskrit{samādhi}, my defilements do not come to an end, and I do not arrive at the supreme sanctuary. And the necessities of life that a renunciate requires—robes, almsfood, lodgings, and medicines and supplies for the sick—are hard to come by.’ That mendicant should leave that jungle thicket that very time of night or day; they shouldn’t stay there. 

Take\marginnote{4.1} another case of a mendicant who lives close by a jungle thicket. Their mindfulness does not become established … But the necessities of life are easy to come by. That mendicant should reflect: ‘While living close by this jungle thicket, my mindfulness does not become established … But the necessities of life are easy to come by. But I didn’t go forth from the lay life to homelessness for the sake of a robe, almsfood, lodgings, or medicines and supplies for the sick. Moreover, while living close by this jungle thicket, my mindfulness does not become established …’ After appraisal, that mendicant should leave that jungle thicket; they shouldn’t stay there. 

Take\marginnote{5.1} another case of a mendicant who lives close by a jungle thicket. As they do so, their mindfulness becomes established, their mind becomes immersed in \textsanskrit{samādhi}, their defilements come to an end, and they arrive at the supreme sanctuary. But the necessities of life that a renunciate requires—robes, almsfood, lodgings, and medicines and supplies for the sick—are hard to come by. That mendicant should reflect: ‘While living close by this jungle thicket, my mindfulness becomes established … But the necessities of life are hard to come by. But I didn’t go forth from the lay life to homelessness for the sake of a robe, almsfood, lodgings, or medicines and supplies for the sick. Moreover, while living close by this jungle thicket, my mindfulness becomes established …’ After appraisal, that mendicant should stay in that jungle thicket; they shouldn’t leave. 

Take\marginnote{6.1} another case of a mendicant who lives close by a jungle thicket. Their mindfulness becomes established … And the necessities of life are easy to come by. That mendicant should reflect: ‘While living close by this jungle thicket, my mindfulness becomes established … And the necessities of life are easy to come by.’ That mendicant should stay in that jungle thicket for the rest of their life; they shouldn’t leave. 

Take\marginnote{7{-}22.1} the case of a mendicant who lives supported by a village … town … city … country … an individual. As they do so, their mindfulness does not become established, their mind does not become immersed in \textsanskrit{samādhi}, their defilements do not come to an end, and they do not arrive at the supreme sanctuary. And the necessities of life that a renunciate requires—robes, almsfood, lodgings, and medicines and supplies for the sick—are hard to come by…. That mendicant should leave that person at any time of the day or night, without asking. They shouldn’t follow them. … 

Take\marginnote{26.1} another case of a mendicant who lives supported by an individual. As they do so, their mindfulness becomes established, their mind becomes immersed in \textsanskrit{samādhi}, their defilements come to an end, and they arrive at the supreme sanctuary. And the necessities of life that a renunciate requires—robes, almsfood, lodgings, and medicines and supplies for the sick—are easy to come by. That mendicant should reflect: ‘While living supported by this person, my mindfulness becomes established … And the necessities of life are easy to come by.’ That mendicant should follow that person for the rest of their life. They shouldn’t leave them, even if sent away.” 

That\marginnote{26.8} is what the Buddha said. Satisfied, the mendicants were happy with what the Buddha said. 

%
\section*{{\suttatitleacronym MN 18}{\suttatitletranslation The Honey-Cake }{\suttatitleroot Madhupiṇḍikasutta}}
\addcontentsline{toc}{section}{\tocacronym{MN 18} \toctranslation{The Honey-Cake } \tocroot{Madhupiṇḍikasutta}}
\markboth{The Honey-Cake }{Madhupiṇḍikasutta}
\extramarks{MN 18}{MN 18}

\scevam{So\marginnote{1.1} I have heard. }At one time the Buddha was staying in the land of the Sakyans, near Kapilavatthu in the Banyan Tree Monastery. 

Then\marginnote{2.1} the Buddha robed up in the morning and, taking his bowl and robe, entered Kapilavatthu for alms. He wandered for alms in Kapilavatthu. After the meal, on his return from almsround, he went to the Great Wood, plunged deep into it, and sat at the root of a young wood apple tree for the day’s meditation. 

\textsanskrit{Daṇḍapāṇi}\marginnote{3.1} the Sakyan, while going for a walk, plunged deep into the Great Wood. He approached the Buddha and exchanged greetings with him. When the greetings and polite conversation were over, he stood to one side leaning on his staff, and said to the Buddha, “What does the ascetic teach? What does he explain?” 

“Sir,\marginnote{4.1} my teaching is such that one does not conflict with anyone in this world with its gods, \textsanskrit{Māras}, and \textsanskrit{Brahmās}, this population with its ascetics and brahmins, its gods and humans. And it is such that perceptions do not underlie the brahmin who lives detached from sensual pleasures, without doubting, stripped of worry, and rid of craving for rebirth in this or that state. That’s what I teach, and that’s what I explain.” 

When\marginnote{5.1} he had spoken, \textsanskrit{Daṇḍapāṇi} shook his head, waggled his tongue, raised his eyebrows until his brow puckered in three furrows, and he departed leaning on his staff. 

Then\marginnote{6.1} in the late afternoon, the Buddha came out of retreat and went to the Banyan Tree Monastery, sat down on the seat spread out, and told the mendicants what had happened. 

When\marginnote{6.14} he had spoken, one of the mendicants said to him, “But sir, what is the teaching such that the Buddha does not conflict with anyone in this world with its gods, \textsanskrit{Māras}, and \textsanskrit{Brahmās}, this population with its ascetics and brahmins, its gods and humans? And how is it that perceptions do not underlie the Buddha, the brahmin who lives detached from sensual pleasures, without indecision, stripped of worry, and rid of craving for rebirth in this or that state?” 

“Mendicant,\marginnote{8.1} a person is beset by concepts of identity that emerge from the proliferation of perceptions. If they don’t find anything worth approving, welcoming, or getting attached to in the source from which these arise, just this is the end of the underlying tendencies to desire, repulsion, views, doubt, conceit, the desire to be reborn, and ignorance. This is the end of taking up the rod and the sword, the end of quarrels, arguments, and disputes, of accusations, divisive speech, and lies. This is where these bad, unskillful qualities cease without anything left over.” 

That\marginnote{9.1} is what the Buddha said. When he had spoken, the Holy One got up from his seat and entered his dwelling. 

Soon\marginnote{10.1} after the Buddha left, those mendicants considered, “The Buddha gave this brief passage for recitation, then entered his dwelling without explaining the meaning in detail. Who can explain in detail the meaning of this brief passage for recitation given by the Buddha?” 

Then\marginnote{10.8} those mendicants thought, “This Venerable \textsanskrit{Mahākaccāna} is praised by the Buddha and esteemed by his sensible spiritual companions. He is capable of explaining in detail the meaning of this brief passage for recitation given by the Buddha. Let’s go to him, and ask him about this matter.” 

Then\marginnote{11.1} those mendicants went to \textsanskrit{Mahākaccāna}, and exchanged greetings with him. When the greetings and polite conversation were over, they sat down to one side. They told him what had happened, and said: “May Venerable \textsanskrit{Mahākaccāna} please explain this.” 

“Reverends,\marginnote{12.1} suppose there was a person in need of heartwood. And while wandering in search of heartwood he’d come across a large tree standing with heartwood. But he’d pass over the roots and trunk, imagining that the heartwood should be sought in the branches and leaves. Such is the consequence for the venerables. Though you were face to face with the Buddha, you overlooked him, imagining that you should ask me about this matter. For he is the Buddha, who knows and sees. He is vision, he is knowledge, he is the truth, he is holiness. He is the teacher, the proclaimer, the elucidator of meaning, the bestower of the deathless, the lord of truth, the Realized One. That was the time to approach the Buddha and ask about this matter. You should have remembered it in line with the Buddha’s answer.” 

“Certainly\marginnote{13.1} he is the Buddha, who knows and sees. He is vision, he is knowledge, he is the truth, he is holiness. He is the teacher, the proclaimer, the elucidator of meaning, the bestower of the deathless, the lord of truth, the Realized One. That was the time to approach the Buddha and ask about this matter. We should have remembered it in line with the Buddha’s answer. Still, \textsanskrit{Mahākaccāna} is praised by the Buddha and esteemed by his sensible spiritual companions. You are capable of explaining in detail the meaning of this brief passage for recitation given by the Buddha. Please explain this, if it’s no trouble.” 

“Well\marginnote{14.1} then, reverends, listen and pay close attention, I will speak.” 

“Yes,\marginnote{14.2} reverend,” they replied. Venerable \textsanskrit{Mahākaccāna} said this: 

“Reverends,\marginnote{15.1} the Buddha gave this brief passage for recitation, then entered his dwelling without explaining the meaning in detail: ‘A person is beset by concepts of identity that emerge from the proliferation of perceptions. If they don’t find anything worth approving, welcoming, or getting attached to in the source from which these arise … This is where these bad, unskillful qualities cease without anything left over.’ This is how I understand the detailed meaning of this passage for recitation. 

Eye\marginnote{16.1} consciousness arises dependent on the eye and sights. The meeting of the three is contact. Contact is a condition for feeling. What you feel, you perceive. What you perceive, you think about. What you think about, you proliferate. What you proliferate about is the source from which a person is beset by concepts of identity that emerge from the proliferation of perceptions. This occurs with respect to sights known by the eye in the past, future, and present. 

Ear\marginnote{16.2} consciousness arises dependent on the ear and sounds. … 

Nose\marginnote{16.3} consciousness arises dependent on the nose and smells. … 

Tongue\marginnote{16.4} consciousness arises dependent on the tongue and tastes. … 

Body\marginnote{16.5} consciousness arises dependent on the body and touches. … 

Mind\marginnote{16.6} consciousness arises dependent on the mind and thoughts. The meeting of the three is contact. Contact is a condition for feeling. What you feel, you perceive. What you perceive, you think about. What you think about, you proliferate. What you proliferate about is the source from which a person is beset by concepts of identity that emerge from the proliferation of perceptions. This occurs with respect to thoughts known by the mind in the past, future, and present. 

When\marginnote{17.1} there is the eye, sights, and eye consciousness, it’s possible to point out what’s known as ‘contact’. When there is what’s known as contact, it’s possible to point out what’s known as ‘feeling’. When there is what’s known as feeling, it’s possible to point out what’s known as ‘perception’. When there is what’s known as perception, it’s possible to point out what’s known as ‘thought’. When there is what’s known as thought, it’s possible to point out what’s known as ‘being beset by concepts of identity that emerge from the proliferation of perceptions’. 

When\marginnote{17.6} there is the ear … nose … tongue … body … mind, thoughts, and mind consciousness, it’s possible to point out what’s known as ‘contact’. … When there is what’s known as thought, it’s possible to point out what’s known as ‘being beset by concepts of identity that emerge from the proliferation of perceptions’. 

When\marginnote{18.1} there is no eye, no sights, and no eye consciousness, it’s not possible to point out what’s known as ‘contact’. When there isn’t what’s known as contact, it’s not possible to point out what’s known as ‘feeling’. When there isn’t what’s known as feeling, it’s not possible to point out what’s known as ‘perception’. When there isn’t what’s known as perception, it’s not possible to point out what’s known as ‘thought’. When there isn’t what’s known as thought, it’s not possible to point out what’s known as ‘being beset by concepts of identity that emerge from the proliferation of perceptions’. 

When\marginnote{18.6} there is no ear … nose … tongue … body … mind, no thoughts, and no mind consciousness, it’s not possible to point out what’s known as ‘contact’. … When there isn’t what’s known as thought, it’s not possible to point out what’s known as ‘being beset by concepts of identity that emerge from the proliferation of perceptions’. 

This\marginnote{19.1} is how I understand the detailed meaning of that brief passage for recitation given by the Buddha. If you wish, you may go to the Buddha and ask him about this. You should remember it in line with the Buddha’s answer.” 

“Yes,\marginnote{20.1} reverend,” said those mendicants, approving and agreeing with what \textsanskrit{Mahākaccāna} said. Then they rose from their seats and went to the Buddha, bowed, sat down to one side, and told him what had happened. Then they said: “\textsanskrit{Mahākaccāna} clearly explained the meaning to us in this manner, with these words and phrases.” 

“\textsanskrit{Mahākaccāna}\marginnote{21.1} is astute, mendicants, he has great wisdom. If you came to me and asked this question, I would answer it in exactly the same way as \textsanskrit{Mahākaccāna}. That is what it means, and that’s how you should remember it.” 

When\marginnote{22.1} he said this, Venerable Ānanda said to the Buddha, “Sir, suppose a person who was weak with hunger was to obtain a honey-cake. Wherever they taste it, they would enjoy a sweet, delicious flavor. 

In\marginnote{22.3} the same way, wherever a sincere, capable mendicant might examine with wisdom the meaning of this exposition of the teaching they would only gain joy and clarity. Sir, what is the name of this exposition of the teaching?” 

“Well,\marginnote{22.5} Ānanda, you may remember this exposition of the teaching as ‘The Honey-Cake Discourse’.” 

That\marginnote{22.6} is what the Buddha said. Satisfied, Venerable Ānanda was happy with what the Buddha said. 

%
\section*{{\suttatitleacronym MN 19}{\suttatitletranslation Two Kinds of Thought }{\suttatitleroot Dvedhāvitakkasutta}}
\addcontentsline{toc}{section}{\tocacronym{MN 19} \toctranslation{Two Kinds of Thought } \tocroot{Dvedhāvitakkasutta}}
\markboth{Two Kinds of Thought }{Dvedhāvitakkasutta}
\extramarks{MN 19}{MN 19}

\scevam{So\marginnote{1.1} I have heard. }At one time the Buddha was staying near \textsanskrit{Sāvatthī} in Jeta’s Grove, \textsanskrit{Anāthapiṇḍika}’s monastery. There the Buddha addressed the mendicants, “Mendicants!” 

“Venerable\marginnote{1.5} sir,” they replied. The Buddha said this: 

“Mendicants,\marginnote{2.1} before my awakening—when I was still unawakened but intent on awakening—I thought: ‘Why don’t I meditate by continually dividing my thoughts into two classes?’ So I assigned sensual, malicious, and cruel thoughts to one class. And I assigned thoughts of renunciation, good will, and harmlessness to the second class. 

Then,\marginnote{3.1} as I meditated—diligent, keen, and resolute—a sensual thought arose. I understood: ‘This sensual thought has arisen in me. It leads to hurting myself, hurting others, and hurting both. It blocks wisdom, it’s on the side of anguish, and it doesn’t lead to extinguishment.’ When I reflected that it leads to hurting myself, it went away. When I reflected that it leads to hurting others, it went away. When I reflected that it leads to hurting both, it went away. When I reflected that it blocks wisdom, it’s on the side of anguish, and it doesn’t lead to extinguishment, it went away. So I gave up, got rid of, and eliminated any sensual thoughts that arose. 

Then,\marginnote{4{-}5.1} as I meditated—diligent, keen, and resolute—a malicious thought arose … a cruel thought arose. I understood: ‘This cruel thought has arisen in me. It leads to hurting myself, hurting others, and hurting both. It blocks wisdom, it’s on the side of anguish, and it doesn’t lead to extinguishment.’ When I reflected that it leads to hurting myself … hurting others … hurting both, it went away. When I reflected that it blocks wisdom, it’s on the side of anguish, and it doesn’t lead to extinguishment, it went away. So I gave up, got rid of, and eliminated any cruel thoughts that arose. 

Whatever\marginnote{6.1} a mendicant frequently thinks about and considers becomes their heart’s inclination. If they often think about and consider sensual thoughts, they’ve given up the thought of renunciation to cultivate sensual thought. Their mind inclines to sensual thoughts. If they often think about and consider malicious thoughts … their mind inclines to malicious thoughts. If they often think about and consider cruel thoughts … their mind inclines to cruel thoughts. 

Suppose\marginnote{7.1} it’s the last month of the rainy season, when the crops grow closely together, and a cowherd must take care of the cattle. He’d tap and poke them with his staff on this side and that to keep them in check. Why is that? For he sees that if they wander into the crops he could be executed, imprisoned, fined, or condemned. 

In\marginnote{7.5} the same way, I saw that unskillful qualities have the drawbacks of sordidness and corruption, and that skillful qualities have the benefit and cleansing power of renunciation. 

Then,\marginnote{8.1} as I meditated—diligent, keen, and resolute—a thought of renunciation arose. I understood: ‘This thought of renunciation has arisen in me. It doesn’t lead to hurting myself, hurting others, or hurting both. It nourishes wisdom, it’s on the side of freedom from anguish, and it leads to extinguishment.’ If I were to keep on thinking and considering this all night … all day … all night and day, I see no danger that would come from that. Still, thinking and considering for too long would tire my body. And when the body is tired, the mind is stressed. And when the mind is stressed, it’s far from immersion. So I stilled, settled, unified, and immersed my mind internally. Why is that? So that my mind would not be stressed. 

Then,\marginnote{9{-}10.1} as I meditated—diligent, keen, and resolute—a thought of good will arose … a thought of harmlessness arose. I understood: ‘This thought of harmlessness has arisen in me. It doesn’t lead to hurting myself, hurting others, or hurting both. It nourishes wisdom, it’s on the side of freedom from anguish, and it leads to extinguishment.’ If I were to keep on thinking and considering this all night … all day … all night and day, I see no danger that would come from that. Still, thinking and considering for too long would tire my body. And when the body is tired, the mind is stressed. And when the mind is stressed, it’s far from immersion. So I stilled, settled, unified, and immersed my mind internally. Why is that? So that my mind would not be stressed. 

Whatever\marginnote{11.1} a mendicant frequently thinks about and considers becomes their heart’s inclination. If they often think about and consider thoughts of renunciation, they’ve given up sensual thought to cultivate the thought of renunciation. Their mind inclines to thoughts of renunciation. If they often think about and consider thoughts of good will … their mind inclines to thoughts of good will. If they often think about and consider thoughts of harmlessness … their mind inclines to thoughts of harmlessness. 

Suppose\marginnote{12.1} it’s the last month of summer, when all the crops have been gathered within a village, and a cowherd must take care of the cattle. While at the root of a tree or in the open he need only be mindful that the cattle are there. In the same way I needed only to be mindful that those things were there. 

My\marginnote{13.1} energy was roused up and unflagging, my mindfulness was established and lucid, my body was tranquil and undisturbed, and my mind was immersed in \textsanskrit{samādhi}. 

Quite\marginnote{14.1} secluded from sensual pleasures, secluded from unskillful qualities, I entered and remained in the first absorption, which has the rapture and bliss born of seclusion, while placing the mind and keeping it connected. 

As\marginnote{15.1} the placing of the mind and keeping it connected were stilled, I entered and remained in the second absorption, which has the rapture and bliss born of immersion, with internal clarity and confidence, and unified mind, without placing the mind and keeping it connected. 

And\marginnote{16.1} with the fading away of rapture, I entered and remained in the third absorption, where I meditated with equanimity, mindful and aware, personally experiencing the bliss of which the noble ones declare, ‘Equanimous and mindful, one meditates in bliss.’ 

With\marginnote{17.1} the giving up of pleasure and pain, and the ending of former happiness and sadness, I entered and remained in the fourth absorption, without pleasure or pain, with pure equanimity and mindfulness. 

When\marginnote{18.1} my mind had immersed in \textsanskrit{samādhi} like this—purified, bright, flawless, rid of corruptions, pliable, workable, steady, and imperturbable—I extended it toward recollection of past lives. I recollected many kinds of past lives, with features and details. 

This\marginnote{19.1} was the first knowledge, which I achieved in the first watch of the night. Ignorance was destroyed and knowledge arose; darkness was destroyed and light arose, as happens for a meditator who is diligent, keen, and resolute. 

When\marginnote{20.1} my mind had become immersed in \textsanskrit{samādhi} like this, I extended it toward knowledge of the death and rebirth of sentient beings. With clairvoyance that is purified and superhuman, I saw sentient beings passing away and being reborn—inferior and superior, beautiful and ugly, in a good place or a bad place. I understood how sentient beings are reborn according to their deeds. 

This\marginnote{21.1} was the second knowledge, which I achieved in the middle watch of the night. Ignorance was destroyed and knowledge arose; darkness was destroyed and light arose, as happens for a meditator who is diligent, keen, and resolute. 

When\marginnote{22.1} my mind had become immersed in \textsanskrit{samādhi} like this, I extended it toward knowledge of the ending of defilements. I truly understood: ‘This is suffering’ … ‘This is the origin of suffering’ … ‘This is the cessation of suffering’ … ‘This is the practice that leads to the cessation of suffering.' 

I\marginnote{23.1} truly understood: ‘These are defilements’ … ‘This is the origin of defilements’ … ‘This is the cessation of defilements’ … ‘This is the practice that leads to the cessation of defilements.' Knowing and seeing like this, my mind was freed from the defilements of sensuality, desire to be reborn, and ignorance. I understood: ‘Rebirth is ended; the spiritual journey has been completed; what had to be done has been done; there is no return to any state of existence.’ 

This\marginnote{24.1} was the third knowledge, which I achieved in the last watch of the night. Ignorance was destroyed and knowledge arose; darkness was destroyed and light arose, as happens for a meditator who is diligent, keen, and resolute. 

Suppose\marginnote{25.1} that in a forested wilderness there was an expanse of low-lying marshes, and a large herd of deer lived nearby. Then along comes a person who wants to harm, injure, and threaten them. They close off the safe, secure path that leads to happiness, and open the wrong path. There they plant domesticated male and female deer as decoys so that, in due course, that herd of deer would fall to ruin and disaster. Then along comes a person who wants to help keep the herd of deer safe. They open up the safe, secure path that leads to happiness, and close off the wrong path. They get rid of the decoys so that, in due course, that herd of deer would grow, increase, and mature. 

I’ve\marginnote{26.1} made up this simile to make a point. And this is what it means. ‘An expanse of low-lying marshes’ is a term for sensual pleasures. ‘A large herd of deer’ is a term for sentient beings. ‘A person who wants to harm, injure, and threaten them’ is a term for \textsanskrit{Māra} the Wicked. ‘The wrong path’ is a term for the wrong eightfold path, that is, wrong view, wrong thought, wrong speech, wrong action, wrong livelihood, wrong effort, wrong mindfulness, and wrong immersion. ‘A domesticated male deer’ is a term for greed and relishing. ‘A domesticated female deer’ is a term for ignorance. ‘A person who wants to help keep the herd of deer safe’ is a term for the Realized One, the perfected one, the fully awakened Buddha. ‘The safe, secure path that leads to happiness’ is a term for the noble eightfold path, that is: right view, right thought, right speech, right action, right livelihood, right effort, right mindfulness, and right immersion. 

So,\marginnote{26.13} mendicants, I have opened up the safe, secure path to happiness and closed off the wrong path. And I have got rid of the male and female decoys. 

Out\marginnote{27.1} of compassion, I’ve done what a teacher should do who wants what’s best for their disciples. Here are these roots of trees, and here are these empty huts. Practice absorption, mendicants! Don’t be negligent! Don’t regret it later! This is my instruction to you.” 

That\marginnote{27.3} is what the Buddha said. Satisfied, the mendicants were happy with what the Buddha said. 

%
\section*{{\suttatitleacronym MN 20}{\suttatitletranslation How to Stop Thinking }{\suttatitleroot Vitakkasaṇṭhānasutta}}
\addcontentsline{toc}{section}{\tocacronym{MN 20} \toctranslation{How to Stop Thinking } \tocroot{Vitakkasaṇṭhānasutta}}
\markboth{How to Stop Thinking }{Vitakkasaṇṭhānasutta}
\extramarks{MN 20}{MN 20}

\scevam{So\marginnote{1.1} I have heard. }At one time the Buddha was staying near \textsanskrit{Sāvatthī} in Jeta’s Grove, \textsanskrit{Anāthapiṇḍika}’s monastery. There the Buddha addressed the mendicants, “Mendicants!” 

“Venerable\marginnote{1.5} sir,” they replied. The Buddha said this: 

“Mendicants,\marginnote{2.1} a mendicant committed to the higher mind should focus on five foundations of meditation from time to time. What five? 

Take\marginnote{3.1} a mendicant who is focusing on some foundation of meditation that gives rise to bad, unskillful thoughts connected with desire, hate, and delusion. That mendicant should focus on some other foundation of meditation connected with the skillful. As they do so, those bad thoughts are given up and come to an end. Their mind becomes stilled internally; it settles, unifies, and becomes immersed in \textsanskrit{samādhi}. It’s like a deft carpenter or their apprentice who’d knock out or extract a large peg with a finer peg. In the same way, a mendicant … should focus on some other foundation of meditation connected with the skillful … 

Now,\marginnote{4.1} suppose that mendicant is focusing on some other foundation of meditation connected with the skillful, but bad, unskillful thoughts connected with desire, hate, and delusion keep coming up. They should examine the drawbacks of those thoughts: ‘So these thoughts are unskillful, they’re blameworthy, and they result in suffering.’ As they do so, those bad thoughts are given up and come to an end. Their mind becomes stilled internally; it settles, unifies, and becomes immersed in \textsanskrit{samādhi}. Suppose there was a woman or man who was young, youthful, and fond of adornments. If the corpse of a snake or a dog or a human were hung around their neck, they’d be horrified, repelled, and disgusted. In the same way, a mendicant … should examine the drawbacks of those thoughts … 

Now,\marginnote{5.1} suppose that mendicant is examining the drawbacks of those thoughts, but bad, unskillful thoughts connected with desire, hate, and delusion keep coming up. They should try to ignore and forget about them. As they do so, those bad thoughts are given up and come to an end. Their mind becomes stilled internally; it settles, unifies, and becomes immersed in \textsanskrit{samādhi}. Suppose there was a person with good eyesight, and some undesirable sights came into their range of vision. They’d just close their eyes or look away. In the same way, a mendicant … those bad thoughts are given up and come to an end … 

Now,\marginnote{6.1} suppose that mendicant is ignoring and forgetting about those thoughts, but bad, unskillful thoughts connected with desire, hate, and delusion keep coming up. They should focus on stopping the formation of thoughts. As they do so, those bad thoughts are given up and come to an end. Their mind becomes stilled internally; it settles, unifies, and becomes immersed in \textsanskrit{samādhi}. Suppose there was a person walking quickly. They’d think: ‘Why am I walking so quickly? Why don’t I slow down?’ So they’d slow down. They’d think: ‘Why am I walking slowly? Why don’t I stand still?’ So they’d stand still. They’d think: ‘Why am I standing still? Why don’t I sit down?’ So they’d sit down. They’d think: ‘Why am I sitting? Why don’t I lie down?’ So they’d lie down. And so that person would reject successively coarser postures and adopt more subtle ones. 

In\marginnote{6.22} the same way, a mendicant … those thoughts are given up and come to an end … 

Now,\marginnote{7.1} suppose that mendicant is focusing on stopping the formation of thoughts, but bad, unskillful thoughts connected with desire, hate, and delusion keep coming up. With teeth clenched and tongue pressed against the roof of the mouth, they should squeeze, squash, and torture mind with mind. As they do so, those bad thoughts are given up and come to an end. Their mind becomes stilled internally; it settles, unifies, and becomes immersed in \textsanskrit{samādhi}. It’s like a strong man who grabs a weaker man by the head or throat or shoulder and squeezes, squashes, and tortures them. In the same way, a mendicant … with teeth clenched and tongue pressed against the roof of the mouth, should squeeze, squash, and torture mind with mind. As they do so, those bad thoughts are given up and come to an end. Their mind becomes stilled internally; it settles, unifies, and becomes immersed in \textsanskrit{samādhi}. 

Now,\marginnote{8.1} take the mendicant who is focusing on some foundation of meditation that gives rise to bad, unskillful thoughts connected with desire, hate, and delusion. They focus on some other foundation of meditation connected with the skillful … They examine the drawbacks of those thoughts … They try to ignore and forget about those thoughts … They focus on stopping the formation of thoughts … With teeth clenched and tongue pressed against the roof of the mouth, they squeeze, squash, and torture mind with mind. When they succeed in each of these things, those bad thoughts are given up and come to an end. Their mind becomes stilled internally; it settles, unifies, and becomes immersed in \textsanskrit{samādhi}. This is called a mendicant who is a master of the ways of thought. They’ll think what they want to think, and they won’t think what they don’t want to think. They’ve cut off craving, untied the fetters, and by rightly comprehending conceit have made an end of suffering.” 

That\marginnote{8.14} is what the Buddha said. Satisfied, the mendicants were happy with what the Buddha said. 

%
\addtocontents{toc}{\let\protect\contentsline\protect\nopagecontentsline}
\chapter*{The Chapter of Similes }
\addcontentsline{toc}{chapter}{\tocchapterline{The Chapter of Similes }}
\addtocontents{toc}{\let\protect\contentsline\protect\oldcontentsline}

%
\section*{{\suttatitleacronym MN 21}{\suttatitletranslation The Simile of the Saw }{\suttatitleroot Kakacūpamasutta}}
\addcontentsline{toc}{section}{\tocacronym{MN 21} \toctranslation{The Simile of the Saw } \tocroot{Kakacūpamasutta}}
\markboth{The Simile of the Saw }{Kakacūpamasutta}
\extramarks{MN 21}{MN 21}

\scevam{So\marginnote{1.1} I have heard. }At one time the Buddha was staying near \textsanskrit{Sāvatthī} in Jeta’s Grove, \textsanskrit{Anāthapiṇḍika}’s monastery. 

Now\marginnote{2.1} at that time, Venerable Phagguna of the Top-Knot was mixing too closely together with the nuns. So much so that if any mendicant criticized those nuns in his presence, Phagguna of the Top-Knot got angry and upset, and even instigated disciplinary proceedings. And if any mendicant criticized Phagguna of the Top-Knot in their presence, those nuns got angry and upset, and even instigated disciplinary proceedings. That’s how much Phagguna of the Top-Knot was mixing too closely together with the nuns. 

Then\marginnote{3.1} a mendicant went up to the Buddha, bowed, sat down to one side, and told him what was going on. 

So\marginnote{4.1} the Buddha addressed a certain monk, “Please, monk, in my name tell the mendicant Phagguna of the Top-Knot that the teacher summons him.” 

“Yes,\marginnote{4.4} sir,” that monk replied. He went to Phagguna of the Top-Knot and said to him, “Reverend Phagguna, the teacher summons you.” 

“Yes,\marginnote{4.6} reverend,” Phagguna replied. He went to the Buddha, bowed, and sat down to one side. The Buddha said to him: 

“Is\marginnote{5.1} it really true, Phagguna, that you’ve been mixing overly closely together with the nuns? So much so that if any mendicant criticizes those nuns in your presence, you get angry and upset, and even instigate disciplinary proceedings? And if any mendicant criticizes you in those nuns’ presence, they get angry and upset, and even instigate disciplinary proceedings? Is that how much you’re mixing overly closely together with the nuns?” 

“Yes,\marginnote{5.6} sir.” 

“Phagguna,\marginnote{5.7} are you not a gentleman who has gone forth from the lay life to homelessness?” 

“Yes,\marginnote{5.8} sir.” 

“As\marginnote{6.1} such, it’s not appropriate for you to mix so closely with the nuns. So if anyone criticizes those nuns in your presence, you should give up any desires or thoughts of the lay life. If that happens, you should train like this: ‘My mind will be unaffected. I will blurt out no bad words. I will remain full of compassion, with a heart of love and no secret hate.’ That’s how you should train. 

So\marginnote{6.6} even if someone strikes those nuns with fists, stones, rods, and swords in your presence, you should give up any desires or thoughts of the lay life. If that happens, you should train like this: ‘My mind will be unaffected. I will blurt out no bad words. I will remain full of compassion, with a heart of love and no secret hate.’ That’s how you should train. 

So\marginnote{6.10} if anyone criticizes you in your presence, you should give up any desires or thoughts of the lay life. If that happens, you should train like this: ‘My mind will be unaffected. I will blurt out no bad words. I will remain full of compassion, with a heart of love and no secret hate.’ That’s how you should train. 

So\marginnote{6.13} Phagguna, even if someone strikes you with fists, stones, rods, and swords, you should give up any desires or thoughts of the lay life. If that happens, you should train like this: ‘My mind will be unaffected. I will blurt out no bad words. I will remain full of compassion, with a heart of love and no secret hate.’ That’s how you should train.” 

Then\marginnote{7.1} the Buddha said to the mendicants: 

“Mendicants,\marginnote{7.2} I used to be satisfied with the mendicants. Once, I addressed them: ‘I eat my food in one sitting per day. Doing so, I find that I’m healthy and well, nimble, strong, and living comfortably. You too should eat your food in one sitting per day. Doing so, you’ll find that you’re healthy and well, nimble, strong, and living comfortably.’ I didn’t have to keep on instructing those mendicants; I just had to prompt their mindfulness. 

Suppose\marginnote{7.10} a chariot stood harnessed to thoroughbreds at a level crossroads, with a goad ready. Then a deft horse trainer, a master charioteer, might mount the chariot, taking the reins in his right hand and goad in the left. He’d drive out and back wherever he wishes, whenever he wishes. 

In\marginnote{7.12} the same way, I didn’t have to keep on instructing those mendicants; I just had to prompt their mindfulness. So, mendicants, you too should give up what’s unskillful and devote yourselves to skillful qualities. In this way you’ll achieve growth, improvement, and maturity in this teaching and training. 

Suppose\marginnote{8.1} that not far from a town or village there was a large grove of sal trees that was choked with castor-oil weeds. Then along comes a person who wants to help protect and nurture that grove. They’d cut down the crooked sal saplings that were robbing the sap, and throw them out. They’d clean up the interior of the grove, and properly care for the straight, well-formed sal saplings. In this way, in due course, that sal grove would grow, increase, and mature. 

In\marginnote{8.7} the same way, mendicants, you too should give up what’s unskillful and devote yourselves to skillful qualities. In this way you’ll achieve growth, improvement, and maturity in this teaching and training. 

Once\marginnote{9.1} upon a time, mendicants, right here in \textsanskrit{Sāvatthī} there was a housewife named \textsanskrit{Vedehikā}. She had this good reputation: ‘The housewife \textsanskrit{Vedehikā} is sweet, even-tempered, and calm.’ Now, \textsanskrit{Vedehikā} had a bonded maid named \textsanskrit{Kāḷī} who was skilled, tireless, and well-organized in her work. 

Then\marginnote{9.5} \textsanskrit{Kāḷī} thought, ‘My mistress has a good reputation as being sweet, even-tempered, and calm. But does she actually have anger in her and just not show it? Or does she have no anger? Or is it just because my work is well-organized that she doesn’t show anger, even though she still has it inside? Why don’t I test my mistress?’ 

So\marginnote{9.11} \textsanskrit{Kāḷī} got up during the day. \textsanskrit{Vedehikā} said to her, ‘What the hell, \textsanskrit{Kāḷī}!’ 

‘What\marginnote{9.14} is it, madam?’ 

‘You’re\marginnote{9.15} getting up in the day—what’s up with you, girl?’ 

‘Nothing,\marginnote{9.16} madam.’ 

‘Nothing’s\marginnote{9.17} up, you bad girl, but you get up in the day!’ Angry and upset, she scowled. 

Then\marginnote{9.18} \textsanskrit{Kāḷī} thought, ‘My mistress actually has anger in her and just doesn’t show it; it’s not that she has no anger. It’s just because my work is well-organized that she doesn’t show anger, even though she still has it inside. Why don’t I test my mistress further?’ 

So\marginnote{9.22} \textsanskrit{Kāḷī} got up later in the day. \textsanskrit{Vedehikā} said to her, ‘What the hell, \textsanskrit{Kāḷī}!’ 

‘What\marginnote{9.25} is it, madam?’ 

‘You’re\marginnote{9.26} getting up later in the day—what’s up with you, girl?’ 

‘Nothing,\marginnote{9.27} madam.’ 

‘Nothing’s\marginnote{9.28} up, you bad girl, but you get up later in the day!’ Angry and upset, she blurted out angry words. 

Then\marginnote{9.29} \textsanskrit{Kāḷī} thought, ‘My mistress actually has anger in her and just doesn’t show it; it’s not that she has no anger. It’s just because my work is well-organized that she doesn’t show anger, even though she still has it inside. Why don’t I test my mistress further?’ 

So\marginnote{9.33} \textsanskrit{Kāḷī} got up even later in the day. \textsanskrit{Vedehikā} said to her, ‘What the hell, \textsanskrit{Kāḷī}!’ 

‘What\marginnote{9.36} is it, madam?’ 

‘You’re\marginnote{9.37} getting up even later in the day—what’s up with you, girl?’ 

‘Nothing,\marginnote{9.38} madam.’ 

‘Nothing’s\marginnote{9.39} up, you bad girl, but you get up even later in the day!’ Angry and upset, she grabbed a rolling-pin and hit \textsanskrit{Kāḷī} on the head, cracking it open. 

Then\marginnote{9.40} \textsanskrit{Kāḷī}, with blood pouring from her cracked skull, denounced her mistress to the neighbors, ‘See, ladies, what the sweet one did! See what the even-tempered one did! See what the calm one did! How on earth can she grab a rolling-pin and hit her only maid on the head, cracking it open, just for getting up late?’ 

Then\marginnote{9.44} after some time the housewife \textsanskrit{Vedehikā} got this bad reputation: ‘The housewife \textsanskrit{Vedehikā} is fierce, ill-tempered, and not calm at all.’ 

In\marginnote{10.1} the same way, a mendicant may be the sweetest of the sweet, the most even-tempered of the even-tempered, the calmest of the calm, so long as they don’t encounter any disagreeable criticism. But it’s when they encounter disagreeable criticism that you’ll know whether they’re really sweet, even-tempered, and calm. I don’t say that a mendicant is easy to admonish if they make themselves easy to admonish only for the sake of robes, almsfood, lodgings, and medicines and supplies for the sick. Why is that? Because when they don’t get robes, almsfood, lodgings, and medicines and supplies for the sick, they’re no longer easy to admonish. But when a mendicant is easy to admonish purely because they honor, respect, revere, worship, and venerate the teaching, then I say that they’re easy to admonish. So, mendicants, you should train yourselves: ‘We will be easy to admonish purely because we honor, respect, revere, worship, and venerate the teaching.’ That’s how you should train. 

Mendicants,\marginnote{11.1} there are these five ways in which others might criticize you. Their speech may be timely or untimely, true or false, gentle or harsh, beneficial or harmful, from a heart of love or from secret hate. When others criticize you, they may do so in any of these ways. If that happens, you should train like this: ‘Our minds will remain unaffected. We will blurt out no bad words. We will remain full of compassion, with a heart of love and no secret hate. We will meditate spreading a heart of love to that person. And with them as a basis, we will meditate spreading a heart full of love to everyone in the world—abundant, expansive, limitless, free of enmity and ill will.’ That’s how you should train. 

Suppose\marginnote{12.1} a person was to come along carrying a spade and basket and say, ‘I shall make this great earth be without earth!’ And they’d dig all over, scatter all over, spit all over, and urinate all over, saying, ‘Be without earth! Be without earth!’ 

What\marginnote{12.6} do you think, mendicants? Could that person make this great earth be without earth?” 

“No,\marginnote{12.8} sir. Why is that? Because this great earth is deep and limitless. It’s not easy to make it be without earth. That person will eventually get weary and frustrated.” 

“In\marginnote{13.1} the same way, there are these five ways in which others might criticize you. Their speech may be timely or untimely, true or false, gentle or harsh, beneficial or harmful, from a heart of love or from secret hate. When others criticize you, they may do so in any of these ways. If that happens, you should train like this: ‘Our minds will remain unaffected. We will blurt out no bad words. We will remain full of compassion, with a heart of love and no secret hate. We will meditate spreading a heart of love to that person. And with them as a basis, we will meditate spreading a heart like the earth to everyone in the world—abundant, expansive, limitless, free of enmity and ill will.’ That’s how you should train. 

Suppose\marginnote{14.1} a person was to come along with dye such as red lac, turmeric, indigo, or rose madder, and say, ‘I shall draw pictures on the sky, making pictures appear there.’ 

What\marginnote{14.4} do you think, mendicants? Could that person draw pictures on the sky?” 

“No,\marginnote{14.6} sir. Why is that? Because the sky is formless and invisible. It’s not easy to draw pictures there. That person will eventually get weary and frustrated.” 

“In\marginnote{15.1} the same way, there are these five ways in which others might criticize you … 

Suppose\marginnote{16.1} a person was to come along carrying a blazing grass torch, and say, ‘I shall burn and scorch the river Ganges with this blazing grass torch.’ 

What\marginnote{16.4} do you think, mendicants? Could that person burn and scorch the river Ganges with a blazing grass torch?” 

“No,\marginnote{16.6} sir. Why is that? Because the river Ganges is deep and limitless. It’s not easy to burn and scorch it with a blazing grass torch. That person will eventually get weary and frustrated.” 

“In\marginnote{17.1} the same way, there are these five ways in which others might criticize you … 

Suppose\marginnote{18.1} there was a catskin bag that was rubbed, well-rubbed, very well-rubbed, soft, silky, rid of rustling and crackling. Then a person comes along carrying a stick or a stone, and says, ‘I shall make this soft catskin bag rustle and crackle with this stick or stone.’ 

What\marginnote{18.5} do you think, mendicants? Could that person make that soft catskin bag rustle and crackle with that stick or stone?” 

“No,\marginnote{18.7} sir. Why is that? Because that catskin bag is rubbed, well-rubbed, very well-rubbed, soft, silky, rid of rustling and crackling. It’s not easy to make it rustle or crackle with a stick or stone. That person will eventually get weary and frustrated.” 

“In\marginnote{19.1} the same way, there are these five ways in which others might criticize you. Their speech may be timely or untimely, true or false, gentle or harsh, beneficial or harmful, from a heart of love or from secret hate. When others criticize you, they may do so in any of these ways. If that happens, you should train like this: ‘Our minds will remain unaffected. We will blurt out no bad words. We will remain full of compassion, with a heart of love and no secret hate. We will meditate spreading a heart of love to that person. And with them as a basis, we will meditate spreading a heart like a catskin bag to everyone in the world—abundant, expansive, limitless, free of enmity and ill will.’ That’s how you should train. 

Even\marginnote{20.1} if low-down bandits were to sever you limb from limb, anyone who had a malevolent thought on that account would not be following my instructions. If that happens, you should train like this: ‘Our minds will remain unaffected. We will blurt out no bad words. We will remain full of compassion, with a heart of love and no secret hate. We will meditate spreading a heart of love to that person. And with them as a basis, we will meditate spreading a heart full of love to everyone in the world—abundant, expansive, limitless, free of enmity and ill will.’ That’s how you should train. 

If\marginnote{21.1} you frequently reflect on this advice—the simile of the saw—do you see any criticism, large or small, that you could not endure?” 

“No,\marginnote{21.3} sir.” 

“So,\marginnote{21.4} mendicants, you should frequently reflect on this advice, the simile of the saw. This will be for your lasting welfare and happiness.” 

That\marginnote{21.6} is what the Buddha said. Satisfied, the mendicants were happy with what the Buddha said. 

%
\section*{{\suttatitleacronym MN 22}{\suttatitletranslation The Simile of the Snake }{\suttatitleroot Alagaddūpamasutta}}
\addcontentsline{toc}{section}{\tocacronym{MN 22} \toctranslation{The Simile of the Snake } \tocroot{Alagaddūpamasutta}}
\markboth{The Simile of the Snake }{Alagaddūpamasutta}
\extramarks{MN 22}{MN 22}

\scevam{So\marginnote{1.1} I have heard. }At one time the Buddha was staying near \textsanskrit{Sāvatthī} in Jeta’s Grove, \textsanskrit{Anāthapiṇḍika}’s monastery. 

Now\marginnote{2.1} at that time a mendicant called \textsanskrit{Ariṭtha}, who had previously been a vulture trapper, had the following harmful misconception: “As I understand the Buddha’s teachings, the acts that he says are obstructions are not really obstructions for the one who performs them.” 

Several\marginnote{2.3} mendicants heard about this. They went up to \textsanskrit{Ariṭṭha} and said to him, “Is it really true, Reverend \textsanskrit{Ariṭṭha}, that you have such a harmful misconception: ‘As I understand the Buddha’s teachings, the acts that he says are obstructions are not really obstructions for the one who performs them’?” 

“Absolutely,\marginnote{3.4} reverends. As I understand the Buddha’s teachings, the acts that he says are obstructions are not really obstructions for the one who performs them.” 

Then,\marginnote{3.5} wishing to dissuade \textsanskrit{Ariṭṭha} from his view, the mendicants pursued, pressed, and grilled him, “Don’t say that, \textsanskrit{Ariṭṭha}! Don’t misrepresent the Buddha, for misrepresentation of the Buddha is not good. And the Buddha would not say that. In many ways the Buddha has said that obstructive acts are obstructive, and that they really do obstruct the one who performs them. The Buddha says that sensual pleasures give little gratification and much suffering and distress, and they are all the more full of drawbacks. With the similes of a skeleton … a lump of meat … a grass torch … a pit of glowing coals … a dream … borrowed goods … fruit on a tree … a butcher’s knife and chopping block … a staking sword … a snake’s head, the Buddha says that sensual pleasures give little gratification and much suffering and distress, and they are all the more full of drawbacks.” 

But\marginnote{3.19} even though the mendicants pursued, pressed, and grilled him in this way, \textsanskrit{Ariṭṭha} obstinately stuck to his misconception and insisted on stating it. 

When\marginnote{3.21} they weren’t able to dissuade \textsanskrit{Ariṭṭha} from his view, the mendicants went to the Buddha, bowed, sat down to one side, and told him what had happened. 

So\marginnote{5.1} the Buddha addressed a certain monk, “Please, monk, in my name tell the mendicant \textsanskrit{Ariṭṭha}, formerly a vulture trapper, that the teacher summons him.” 

“Yes,\marginnote{5.4} sir,” that monk replied. He went to \textsanskrit{Ariṭṭha} and said to him, “Reverend \textsanskrit{Ariṭṭha}, the teacher summons you.” 

“Yes,\marginnote{5.6} reverend,” \textsanskrit{Ariṭṭha} replied. He went to the Buddha, bowed, and sat down to one side. The Buddha said to him, 

“Is\marginnote{5.7} it really true, \textsanskrit{Ariṭṭha}, that you have such a harmful misconception: ‘As I understand the Buddha’s teachings, the acts that he says are obstructions are not really obstructions for the one who performs them’?” 

“Absolutely,\marginnote{5.9} sir. As I understand the Buddha’s teachings, the acts that he says are obstructions are not really obstructions for the one who performs them.” 

“Silly\marginnote{6.1} man, who on earth have you ever known me to teach in that way? Haven’t I said in many ways that obstructive acts are obstructive, and that they really do obstruct the one who performs them? I’ve said that sensual pleasures give little gratification and much suffering and distress, and they are all the more full of drawbacks. With the similes of a skeleton … a lump of meat … a grass torch … a pit of glowing coals … a dream … borrowed goods … fruit on a tree … a butcher’s knife and chopping block … a staking sword … a snake’s head, I’ve said that sensual pleasures give little gratification and much suffering and distress, and they are all the more full of drawbacks. But still you misrepresent me by your wrong grasp, harm yourself, and make much bad karma. This will be for your lasting harm and suffering.” 

Then\marginnote{7.1} the Buddha said to the mendicants, “What do you think, mendicants? Has this mendicant \textsanskrit{Ariṭṭha} kindled even a spark of wisdom in this teaching and training?” 

“How\marginnote{7.4} could that be, sir? No, sir.” When this was said, \textsanskrit{Ariṭṭha} sat silent, embarrassed, shoulders drooping, downcast, depressed, with nothing to say. 

Knowing\marginnote{7.7} this, the Buddha said, “Silly man, you will be known by your own harmful misconception. I’ll question the mendicants about this.” 

Then\marginnote{8.1} the Buddha said to the mendicants, “Mendicants, do you understand my teachings as \textsanskrit{Ariṭṭha} does, when he misrepresents me by his wrong grasp, harms himself, and makes much bad karma?” 

“No,\marginnote{8.3} sir. For in many ways the Buddha has said that obstructive acts are obstructive, and that they really do obstruct the one who performs them. The Buddha has said that sensual pleasures give little gratification and much suffering and distress, and they are all the more full of drawbacks. With the similes of a skeleton … a snake’s head, the Buddha has said that sensual pleasures give little gratification and much suffering and distress, and they are all the more full of drawbacks.” 

“Good,\marginnote{8.9} good, mendicants! It’s good that you understand my teaching like this. For in many ways I have said that obstructive acts are obstructive … 

I’ve\marginnote{8.11} said that sensual pleasures give little gratification and much suffering and distress, and they are all the more full of drawbacks. But still this \textsanskrit{Ariṭṭha} misrepresents me by his wrong grasp, harms himself, and makes much bad karma. This will be for his lasting harm and suffering. Truly, mendicants, it’s not possible to perform sensual acts without sensual pleasures, sensual perceptions, and sensual thoughts. 

Take\marginnote{10.1} a foolish person who memorizes the teaching—statements, songs, discussions, verses, inspired exclamations, legends, stories of past lives, amazing stories, and classifications. But they don’t examine the meaning of those teachings with wisdom, and so don’t come to a considered acceptance of them. They just memorize the teaching for the sake of finding fault and winning debates. They don’t realize the goal for which they memorized them. Because they’re wrongly grasped, those teachings lead to their lasting harm and suffering. Why is that? Because of their wrong grasp of the teachings. 

Suppose\marginnote{10.10} there was a person in need of a snake. And while wandering in search of a snake they’d see a big snake, and grasp it by the coil or the tail. But that snake would twist back and bite them on the hand or the arm or limb, resulting in death or deadly pain. Why is that? Because of their wrong grasp of the snake. 

In\marginnote{10.17} the same way, a foolish person memorizes the teaching … and those teachings lead to their lasting harm and suffering. Why is that? Because of their wrong grasp of the teachings. 

Now,\marginnote{11.1} take a gentleman who memorizes the teaching—statements, songs, discussions, verses, inspired exclamations, legends, stories of past lives, amazing stories, and classifications. And once he’s memorized them, he examines their meaning with wisdom, and comes to a considered acceptance of them. He doesn’t memorize the teaching for the sake of finding fault and winning debates. He realizes the goal for which he memorized them. Because they’re correctly grasped, those teachings lead to his lasting welfare and happiness. Why is that? Because of his correct grasp of the teachings. 

Suppose\marginnote{11.10} there was a person in need of a snake. And while wandering in search of a snake they’d see a big snake, and hold it down carefully with a cleft stick. Only then would they correctly grasp it by the neck. And even though that snake might wrap its coils around that person’s hand or arm or some other limb, that wouldn’t result in death or deadly pain. Why is that? Because of their correct grasp of the snake. 

In\marginnote{11.17} the same way, a gentleman memorizes the teaching … and those teachings lead to his lasting welfare and happiness. Why is that? Because of his correct grasp of the teachings. 

So,\marginnote{12.1} mendicants, when you understand what I’ve said, you should remember it accordingly. But if I’ve said anything that you don’t understand, you should ask me about it, or some competent mendicants. 

Mendicants,\marginnote{13.1} I will teach you how the Dhamma is similar to a raft: it’s for crossing over, not for holding on. Listen and pay close attention, I will speak.” 

“Yes,\marginnote{13.3} sir,” they replied. The Buddha said this: 

“Suppose\marginnote{13.5} there was a person traveling along the road. They’d see a large deluge, whose near shore was dubious and perilous, while the far shore was a sanctuary free of peril. But there was no ferryboat or bridge for crossing over. They’d think, ‘Why don’t I gather grass, sticks, branches, and leaves and make a raft? Riding on the raft, and paddling with my hands and feet, I can safely reach the far shore.’ And so they’d do exactly that. And when they’d crossed over to the far shore, they’d think, ‘This raft has been very helpful to me. Riding on the raft, and paddling with my hands and feet, I have safely crossed over to the far shore. Why don’t I hoist it on my head or pick it up on my shoulder and go wherever I want?’ 

What\marginnote{13.17} do you think, mendicants? Would that person be doing what should be done with that raft?” 

“No,\marginnote{13.19} sir.” 

“And\marginnote{13.20} what, mendicants, should that person do with the raft? When they’d crossed over they should think, ‘This raft has been very helpful to me. … Why don’t I beach it on dry land or set it adrift on the water and go wherever I want?’ That’s what that person should do with the raft. 

In\marginnote{13.26} the same way, I have taught how the teaching is similar to a raft: it’s for crossing over, not for holding on. By understanding the simile of the raft, you will even give up the teachings, let alone what is against the teachings. 

Mendicants,\marginnote{15.1} there are these six grounds for views. What six? Take an unlearned ordinary person who has not seen the noble ones, and is neither skilled nor trained in the teaching of the noble ones. They’ve not seen good persons, and are neither skilled nor trained in the teaching of the good persons. They regard form like this: ‘This is mine, I am this, this is my self.’ They also regard feeling … perception … choices … whatever is seen, heard, thought, known, attained, sought, and explored by the mind like this: ‘This is mine, I am this, this is my self.’ And the same for this ground for views: ‘The self and the cosmos are one and the same. After death I will be permanent, everlasting, eternal, imperishable, and will last forever and ever.’ They also regard this: ‘This is mine, I am this, this is my self.’ 

But\marginnote{16.1} a learned noble disciple has seen the noble ones, and is skilled and trained in the teaching of the noble ones. They’ve seen good persons, and are skilled and trained in the teaching of the good persons. They regard form like this: ‘This is not mine, I am not this, this is not my self.’ They also regard feeling … perception … choices … whatever is seen, heard, thought, known, attained, sought, and explored by the mind like this: ‘This is not mine, I am not this, this is not my self.’ And the same for this ground for views: ‘The self and the cosmos are one and the same. After death I will be permanent, everlasting, eternal, imperishable, and will last forever and ever.’ They also regard this: ‘This is not mine, I am not this, this is not my self.’ 

Seeing\marginnote{17.1} in this way they’re not anxious about what doesn’t exist.” 

When\marginnote{18.1} he said this, one of the mendicants asked the Buddha, “Sir, can there be anxiety about what doesn’t exist externally?” 

“There\marginnote{18.3} can, mendicant,” said the Buddha. “It’s when someone thinks, ‘Oh, but it used to be mine, and it is mine no more. Oh, but it could be mine, and I will get it no more.’ They sorrow and wail and lament, beating their breast and falling into confusion. That’s how there is anxiety about what doesn’t exist externally.” 

“But\marginnote{19.1} can there be no anxiety about what doesn’t exist externally?” 

“There\marginnote{19.2} can, mendicant,” said the Buddha. “It’s when someone doesn’t think, ‘Oh, but it used to be mine, and it is mine no more. Oh, but it could be mine, and I will get it no more.’ They don’t sorrow and wail and lament, beating their breast and falling into confusion. That’s how there is no anxiety about what doesn’t exist externally.” 

“But\marginnote{20.1} can there be anxiety about what doesn’t exist internally?” 

“There\marginnote{20.2} can, mendicant,” said the Buddha. “It’s when someone has such a view: ‘The self and the cosmos are one and the same. After death I will be permanent, everlasting, eternal, imperishable, and will last forever and ever.’ They hear the Realized One or their disciple teaching Dhamma for the uprooting of all grounds, fixations, obsessions, insistences, and underlying tendencies regarding views; for the stilling of all activities, the letting go of all attachments, the ending of craving, fading away, cessation, extinguishment. They think, ‘Whoa, I’m going to be annihilated and destroyed! I won’t exist any more!’ They sorrow and wail and lament, beating their breast and falling into confusion. That’s how there is anxiety about what doesn’t exist internally.” 

“But\marginnote{21.1} can there be no anxiety about what doesn’t exist internally?” 

“There\marginnote{21.2} can,” said the Buddha. “It’s when someone doesn’t have such a view: ‘The self and the cosmos are one and the same. After death I will be permanent, everlasting, eternal, imperishable, and will last forever and ever.’ They hear the Realized One or their disciple teaching Dhamma for the uprooting of all grounds, fixations, obsessions, insistences, and underlying tendencies regarding views; for the stilling of all activities, the letting go of all attachments, the ending of craving, fading away, cessation, extinguishment. It never occurs to them, ‘Whoa, I’m going to be annihilated and destroyed! I won’t exist any more!’ They don’t sorrow and wail and lament, beating their breast and falling into confusion. That’s how there is no anxiety about what doesn’t exist internally. 

Mendicants,\marginnote{22.1} it would make sense to be possessive about something that’s permanent, everlasting, eternal, imperishable, and will last forever and ever. But do you see any such possession?” 

“No,\marginnote{22.3} sir.” 

“Good,\marginnote{22.4} mendicants! I also can’t see any such possession. 

It\marginnote{23.1} would make sense to grasp at a doctrine of self that didn’t give rise to sorrow, lamentation, pain, sadness, and distress. But do you see any such doctrine of self?” 

“No,\marginnote{23.3} sir.” 

“Good,\marginnote{23.4} mendicants! I also can’t see any such doctrine of self. 

It\marginnote{24.1} would make sense to rely on a view that didn’t give rise to sorrow, lamentation, pain, sadness, and distress. But do you see any such view to rely on?” 

“No,\marginnote{24.3} sir.” 

“Good,\marginnote{24.4} mendicants! I also can’t see any such view to rely on. 

Mendicants,\marginnote{25.1} were a self to exist, would there be the thought, ‘Belonging to my self’?” 

“Yes,\marginnote{25.2} sir.” 

“Were\marginnote{25.3} what belongs to a self to exist, would there be the thought, ‘My self’?” 

“Yes,\marginnote{25.4} sir.” 

“But\marginnote{25.5} self and what belongs to a self are not acknowledged as a genuine fact. This being so, is not the following a totally foolish teaching: ‘The self and the cosmos are one and the same. After death I will be permanent, everlasting, eternal, imperishable, and will last forever and ever’?” 

“What\marginnote{25.8} else could it be, sir? It’s a totally foolish teaching.” 

“What\marginnote{26.1} do you think, mendicants? Is form permanent or impermanent?” 

“Impermanent,\marginnote{26.3} sir.” 

“But\marginnote{26.4} if it’s impermanent, is it suffering or happiness?” 

“Suffering,\marginnote{26.5} sir.” 

“But\marginnote{26.6} if it’s impermanent, suffering, and liable to wear out, is it fit to be regarded thus: ‘This is mine, I am this, this is my self’?” 

“No,\marginnote{26.8} sir.” 

“What\marginnote{26.9} do you think, mendicants? Is feeling … perception … choices … consciousness permanent or impermanent?” 

“Impermanent,\marginnote{26.14} sir.” 

“But\marginnote{26.15} if it’s impermanent, is it suffering or happiness?” 

“Suffering,\marginnote{26.16} sir.” 

“But\marginnote{26.17} if it’s impermanent, suffering, and liable to wear out, is it fit to be regarded thus: ‘This is mine, I am this, this is my self’?” 

“No,\marginnote{26.19} sir.” 

“So,\marginnote{27.1} mendicants, you should truly see any kind of form at all—past, future, or present; internal or external; coarse or fine; inferior or superior; far or near: \emph{all} form—with right understanding: ‘This is not mine, I am not this, this is not my self.’ You should truly see any kind of feeling … perception … choices … consciousness at all—past, future, or present; internal or external; coarse or fine; inferior or superior; far or near: \emph{all} consciousness—with right understanding: ‘This is not mine, I am not this, this is not my self.’ 

Seeing\marginnote{28.1} this, a learned noble disciple grows disillusioned with form, feeling, perception, choices, and consciousness. Being disillusioned, desire fades away. When desire fades away they’re freed. When they’re freed, they know they’re freed. 

They\marginnote{29.2} understand: ‘Rebirth is ended, the spiritual journey has been completed, what had to be done has been done, there is no return to any state of existence.’ 

This\marginnote{30.1} is called a mendicant who has lifted up the cross-bar, filled in the trench, and pulled up the pillar; who is unbarred, a noble one with banner and burden put down, detached. 

And\marginnote{31.1} how has a mendicant lifted the cross-bar? It’s when a mendicant has given up ignorance, cut it off at the root, made it like a palm stump, obliterated it, so it’s unable to arise in the future. That’s how a mendicant has lifted the cross-bar. 

And\marginnote{32.1} how has a mendicant filled in the trench? It’s when a mendicant has given up transmigrating through births in future lives, cut it off at the root, made it like a palm stump, obliterated it, so it’s unable to arise in the future. That’s how a mendicant has filled in the trench. 

And\marginnote{33.1} how has a mendicant pulled up the pillar? It’s when a mendicant has given up craving, cut it off at the root, made it like a palm stump, obliterated it, so it’s unable to arise in the future. That’s how a mendicant has pulled up the pillar. 

And\marginnote{34.1} how is a mendicant unbarred? It’s when a mendicant has given up the five lower fetters, cut them off at the root, made them like a palm stump, obliterated them, so they’re unable to arise in the future. That’s how a mendicant is unbarred. 

And\marginnote{35.1} how is a mendicant a noble one with banner and burden put down, detached? It’s when a mendicant has given up the conceit ‘I am’, cut it off at the root, made it like a palm stump, obliterated it, so it’s unable to arise in the future. That’s how a mendicant is a noble one with banner and burden put down, detached. 

When\marginnote{36.1} a mendicant’s mind was freed like this, the gods together with Indra, \textsanskrit{Brahmā}, and \textsanskrit{Pajāpati}, search as they may, will not discover: ‘This is what the Realized One’s consciousness depends on.’ Why is that? Because even in the present life the Realized One is not found, I say. 

Though\marginnote{37.1} I speak and explain like this, certain ascetics and brahmins misrepresent me with the false, hollow, lying, untruthful claim: ‘The ascetic Gotama is an exterminator. He advocates the annihilation, eradication, and obliteration of an existing being.’ I have been falsely misrepresented as being what I am not, and saying what I do not say. In the past, as today, what I describe is suffering and the cessation of suffering. This being so, if others abuse, attack, harass, and trouble the Realized One, he doesn’t get resentful, bitter, and emotionally exasperated. 

Or\marginnote{38.1} if others honor, respect, revere, or venerate him, he doesn’t get thrilled, elated, and emotionally excited. He just thinks, ‘They do such things for what has already been completely understood.’ 

So,\marginnote{39.1} mendicants, if others abuse, attack, harass, and trouble you, don’t make yourselves resentful, bitter, and emotionally exasperated. Or if others honor, respect, revere, or venerate you, don’t make yourselves thrilled, elated, and emotionally excited. Just think, ‘They do such things for what has already been completely understood.’ 

So,\marginnote{40.1} mendicants, give up what isn't yours. Giving it up will be for your lasting welfare and happiness. 

And\marginnote{41.1} what isn’t yours? Form isn’t yours: give it up. Giving it up will be for your lasting welfare and happiness. 

Feeling\marginnote{41.4} … perception … choices … consciousness isn’t yours: give it up. Giving it up will be for your lasting welfare and happiness. 

What\marginnote{41.12} do you think, mendicants? Suppose a person was to carry off the grass, sticks, branches, and leaves in this Jeta’s Grove, or burn them, or do what they want with them. Would you think, ‘This person is carrying us off, burning us, or doing what they want with us’?” 

“No,\marginnote{41.16} sir. Why is that? Because that’s neither self nor belonging to self.” 

“In\marginnote{41.19} the same way, mendicants, give up what isn't yours. Giving it up will be for your lasting welfare and happiness. And what isn’t yours? Form … feeling … perception … choices … consciousness isn’t yours: give it up. Giving it up will be for your lasting welfare and happiness. 

Thus\marginnote{42.1} the teaching has been well explained by me, made clear, opened, illuminated, and stripped of patchwork. In this teaching there are mendicants who are perfected, who have ended the defilements, completed the spiritual journey, done what had to be done, laid down the burden, achieved their own goal, utterly ended the fetters of rebirth, and are rightly freed through enlightenment. For them, there is no cycle of rebirths to be found. … 

In\marginnote{43.1} this teaching there are mendicants who have given up the five lower fetters. All of them are reborn spontaneously. They are extinguished there, and are not liable to return from that world. … 

In\marginnote{44.1} this teaching there are mendicants who, having given up three fetters, and weakened greed, hate, and delusion, are once-returners. All of them come back to this world once only, then make an end of suffering. … 

In\marginnote{45.1} this teaching there are mendicants who have ended three fetters. All of them are stream-enterers, not liable to be reborn in the underworld, bound for awakening. … 

In\marginnote{46.1} this teaching there are mendicants who are followers of principles, or followers by faith. All of them are bound for awakening. 

Thus\marginnote{47.1} the teaching has been well explained by me, made clear, opened, illuminated, and stripped of patchwork. In this teaching there are those who have a degree of faith and love for me. All of them are bound for heaven.” 

That\marginnote{47.3} is what the Buddha said. Satisfied, the mendicants were happy with what the Buddha said. 

%
\section*{{\suttatitleacronym MN 23}{\suttatitletranslation The Ant-Hill }{\suttatitleroot Vammikasutta}}
\addcontentsline{toc}{section}{\tocacronym{MN 23} \toctranslation{The Ant-Hill } \tocroot{Vammikasutta}}
\markboth{The Ant-Hill }{Vammikasutta}
\extramarks{MN 23}{MN 23}

\scevam{So\marginnote{1.1} I have heard. }At one time the Buddha was staying near \textsanskrit{Sāvatthī} in Jeta’s Grove, \textsanskrit{Anāthapiṇḍika}’s monastery. Now at that time Venerable Kassapa the Prince was staying in the Dark Forest. 

Then,\marginnote{1.4} late at night, a glorious deity, lighting up the entire Dark Forest, went up to Kassapa the Prince, stood to one side, and said: 

“Monk,\marginnote{2.1} monk! This ant-hill fumes by night and flames by day. The brahmin said, ‘Take up the sword and dig, O sage!’ 

Taking\marginnote{2.4} up the sword and digging, the sage saw a bar: ‘A bar, sir!’ The brahmin said, ‘Throw out the bar! Take up the sword and dig, O sage!’ 

Taking\marginnote{2.9} up the sword and digging, the sage saw a bullfrog: ‘A bullfrog, sir!’ The brahmin said, ‘Throw out the bullfrog! Take up the sword and dig, O sage!’ 

Taking\marginnote{2.14} up the sword and digging, the sage saw a forked path: ‘A forked path, sir!’ The brahmin said, ‘Throw out the forked path! Take up the sword and dig, O sage!’ 

Taking\marginnote{2.19} up the sword and digging, the sage saw a box: ‘A box, sir!’ The brahmin said, ‘Throw out the box! Take up the sword and dig, O sage!’ 

Taking\marginnote{2.24} up the sword and digging, the sage saw a tortoise: ‘A tortoise, sir!’ The brahmin said, ‘Throw out the tortoise! Take up the sword and dig, O sage!’ 

Taking\marginnote{2.29} up the sword and digging, the sage saw an axe and block: ‘An axe and block, sir!’ The brahmin said, ‘Throw out the axe and block! Take up the sword and dig, O sage!’ 

Taking\marginnote{2.34} up the sword and digging, the sage saw a lump of meat: ‘A lump of meat, sir!’ The brahmin said, ‘Throw out the lump of meat! Take up the sword and dig, O sage!’ 

Taking\marginnote{2.39} up the sword and digging, the sage saw a dragon: ‘A dragon, sir!’ The brahmin said, ‘Leave the dragon! Do not disturb the dragon! Worship the dragon!’ 

Mendicant,\marginnote{2.43} go to the Buddha and ask him about this riddle. You should remember it in line with his answer. I don’t see anyone in this world—with its gods, \textsanskrit{Māras}, and \textsanskrit{Brahmās}, this population with its ascetics and brahmins, its gods and humans—who could provide a satisfying answer to this riddle except for the Realized One or his disciple or someone who has heard it from them.” 

That\marginnote{2.45} is what that deity said before vanishing right there. 

Then,\marginnote{3.1} when the night had passed, Kassapa the Prince went to the Buddha, bowed, sat down to one side, and told him what had happened. Then he asked: 

“Sir,\marginnote{3.8} what is the ant-hill? What is the fuming by night and flaming by day? Who is the brahmin, and who the sage? What are the sword, the digging, the bar, the bullfrog, the forked path, the box, the tortoise, the axe and block, and the lump of meat? And what is the dragon?” 

“Mendicant,\marginnote{4.1} ‘ant-hill’ is a term for this body made up of the four primary elements, produced by mother and father, built up from rice and porridge, liable to impermanence, to wearing away and erosion, to breaking up and destruction. 

Thinking\marginnote{4.2} and considering all night about what you did during the day—this is the fuming at night. The work you apply yourself to during the day by body, speech, and mind after thinking about it all night—this is the flaming by day. 

‘Brahmin’\marginnote{4.6} is a term for the Realized One, the perfected one, the fully awakened Buddha. ‘Sage’ is a term for the trainee mendicant. 

‘Sword’\marginnote{4.8} is a term for noble wisdom. ‘Digging’ is a term for being energetic. 

‘Bar’\marginnote{4.10} is a term for ignorance. ‘Throw out the bar’ means ‘give up ignorance, take up the sword, sage, and dig.’ 

‘Bullfrog’\marginnote{4.13} is a term for anger and distress. ‘Throw out the bullfrog’ means ‘give up anger and distress’ … 

‘A\marginnote{4.16} forked path’ is a term for doubt. ‘Throw out the forked path’ means ‘give up doubt’ … 

‘Box’\marginnote{4.19} is a term for the five hindrances, that is: the hindrances of sensual desire, ill will, dullness and drowsiness, restlessness and remorse, and doubt. ‘Throw out the box’ means ‘give up the five hindrances’ … 

‘Tortoise’\marginnote{4.23} is a term for the five grasping aggregates, that is: form, feeling, perception, choices, and consciousness. ‘Throw out the tortoise’ means ‘give up the five grasping aggregates’ … 

‘Axe\marginnote{4.27} and block’ is a term for the five kinds of sensual stimulation. Sights known by the eye that are likable, desirable, agreeable, pleasant, sensual, and arousing. Sounds known by the ear … Smells known by the nose … Tastes known by the tongue … Touches known by the body that are likable, desirable, agreeable, pleasant, sensual, and arousing. ‘Throw out the axe and block’ means ‘give up the five kinds of sensual stimulation’ … 

‘Lump\marginnote{4.35} of meat’ is a term for greed and relishing. ‘Throw out the lump of meat’ means ‘give up greed and relishing’ … 

‘Dragon’\marginnote{4.38} is a term for a mendicant who has ended the defilements. This is the meaning of: ‘Leave the dragon! Do not disturb the dragon! Worship the dragon.’” 

That\marginnote{4.40} is what the Buddha said. Satisfied, Venerable Kassapa the Prince was happy with what the Buddha said. 

%
\section*{{\suttatitleacronym MN 24}{\suttatitletranslation Prepared Chariots }{\suttatitleroot Rathavinītasutta}}
\addcontentsline{toc}{section}{\tocacronym{MN 24} \toctranslation{Prepared Chariots } \tocroot{Rathavinītasutta}}
\markboth{Prepared Chariots }{Rathavinītasutta}
\extramarks{MN 24}{MN 24}

\scevam{So\marginnote{1.1} I have heard. }At one time the Buddha was staying near \textsanskrit{Rājagaha}, in the Bamboo Grove, the squirrels’ feeding ground. 

Then\marginnote{2.1} several mendicants who had completed the rainy season residence in their native land went to the Buddha, bowed, and sat down to one side. The Buddha said to them: 

“In\marginnote{2.2} your native land, mendicants, which of the native mendicants is esteemed in this way: ‘Personally having few wishes, they speak to the mendicants on having few wishes. Personally having contentment, seclusion, aloofness, energy, ethics, immersion, wisdom, freedom, and the knowledge and vision of freedom, they speak to the mendicants on all these things. They’re an adviser and instructor, one who educates, encourages, fires up, and inspires their spiritual companions.’” 

“\textsanskrit{Puṇṇa}\marginnote{2.4} son of \textsanskrit{Mantāṇī}, sir, is esteemed in this way in our native land.” 

Now\marginnote{3.1} at that time Venerable \textsanskrit{Sāriputta} was meditating not far from the Buddha. Then he thought: 

“\textsanskrit{Puṇṇa}\marginnote{3.3} son of \textsanskrit{Mantāṇī} is fortunate, so very fortunate, in that his sensible spiritual companions praise him point by point in the presence of the Teacher, and that the Teacher seconds that appreciation. Hopefully, some time or other I’ll get to meet Venerable \textsanskrit{Puṇṇa}, and we can have a discussion.” 

When\marginnote{4.1} the Buddha had stayed in \textsanskrit{Rājagaha} as long as he wished, he set out for \textsanskrit{Sāvatthī}. Traveling stage by stage, he arrived at \textsanskrit{Sāvatthī}, where he stayed in Jeta’s Grove, \textsanskrit{Anāthapiṇḍika}’s monastery. \textsanskrit{Puṇṇa} heard that the Buddha had arrived at \textsanskrit{Sāvatthī}. 

Then\marginnote{5.1} he set his lodgings in order and, taking his bowl and robe, set out for \textsanskrit{Sāvatthī}. Eventually he came to \textsanskrit{Sāvatthī} and Jeta’s Grove. He went up to the Buddha, bowed, and sat down to one side. The Buddha educated, encouraged, fired up, and inspired him with a Dhamma talk. Then, having approved and agreed with what the Buddha said, \textsanskrit{Puṇṇa} got up from his seat, bowed, and respectfully circled the Buddha, keeping him on his right. Then he went to the Dark Forest for the day’s meditation. 

Then\marginnote{6.1} a certain mendicant went up to Venerable \textsanskrit{Sāriputta}, and said to him, “Reverend \textsanskrit{Sāriputta}, the mendicant named \textsanskrit{Puṇṇa}, of whom you have often spoken so highly, after being inspired by a talk of the Buddha’s, left for the Dark Forest for the day’s meditation.” 

\textsanskrit{Sāriputta}\marginnote{7.1} quickly grabbed his sitting cloth and followed behind \textsanskrit{Puṇṇa}, keeping sight of his head. \textsanskrit{Puṇṇa} plunged deep into the Dark Forest and sat at the root of a tree for the day’s meditation. And \textsanskrit{Sāriputta} did likewise. 

Then\marginnote{8.1} in the late afternoon, \textsanskrit{Sāriputta} came out of retreat, went to \textsanskrit{Puṇṇa}, and exchanged greetings with him. When the greetings and polite conversation were over, he sat down to one side and said to \textsanskrit{Puṇṇa}: 

“Reverend,\marginnote{9.1} is our spiritual life lived under the Buddha?” 

“Yes,\marginnote{9.2} reverend.” 

“Is\marginnote{9.3} the spiritual life lived under the Buddha for the sake of purification of ethics?” 

“Certainly\marginnote{9.4} not.” 

“Then\marginnote{9.5} is the spiritual life lived under the Buddha for the sake of purification of mind?” 

“Certainly\marginnote{9.6} not.” 

“Is\marginnote{9.7} the spiritual life lived under the Buddha for the sake of purification of view?” 

“Certainly\marginnote{9.8} not.” 

“Then\marginnote{9.9} is the spiritual life lived under the Buddha for the sake of purification through overcoming doubt?” 

“Certainly\marginnote{9.10} not.” 

“Is\marginnote{9.11} the spiritual life lived under the Buddha for the sake of purification of knowledge and vision of the variety of paths?” 

“Certainly\marginnote{9.12} not.” 

“Then\marginnote{9.13} is the spiritual life lived under the Buddha for the sake of purification of knowledge and vision of the practice?” 

“Certainly\marginnote{9.14} not.” 

“Is\marginnote{9.15} the spiritual life lived under the Buddha for the sake of purification of knowledge and vision?” 

“Certainly\marginnote{9.16} not.” 

“When\marginnote{10.1} asked each of these questions, you answered, ‘Certainly not.’ Then what exactly is the purpose of leading the spiritual life under the Buddha?” 

“The\marginnote{10.9} purpose of leading the spiritual life under the Buddha is extinguishment by not grasping.” 

“Reverend,\marginnote{11.1} is purification of ethics extinguishment by not grasping?” 

“Certainly\marginnote{11.2} not.” 

“Is\marginnote{11.13} purification of knowledge and vision extinguishment by not grasping?” 

“Certainly\marginnote{11.14} not.” 

“Then\marginnote{11.15} is extinguishment by not grasping something apart from these things?” 

“Certainly\marginnote{11.16} not.” 

“When\marginnote{12.1} asked each of these questions, you answered, ‘Certainly not.’ How then should we see the meaning of this statement?” 

“If\marginnote{13.1} the Buddha had declared purification of ethics to be extinguishment by not grasping, he would have declared that which has grasping to be extinguishment by not grasping. … If the Buddha had declared purification of knowledge and vision to be extinguishment by not grasping, he would have declared that which has grasping to be extinguishment by not grasping. But if extinguishment by not grasping was something apart from these things, an ordinary person would become extinguished. For an ordinary person lacks these things. 

Well\marginnote{14.1} then, reverend, I shall give you a simile. For by means of a simile some sensible people understand the meaning of what is said. 

Suppose\marginnote{14.3} that, while staying in \textsanskrit{Sāvatthī}, King Pasenadi of Kosala had some urgent business come up in \textsanskrit{Sāketa}. Now, between \textsanskrit{Sāvatthī} and \textsanskrit{Sāketa} seven prepared chariots were stationed ready for him. Then Pasenadi, having departed \textsanskrit{Sāvatthī}, mounted the first prepared chariot by the gate of the royal compound. The first prepared chariot would bring him to the second, where he’d dismount and mount the second chariot. The second prepared chariot would bring him to the third … The third prepared chariot would bring him to the fourth … The fourth prepared chariot would bring him to the fifth … The fifth prepared chariot would bring him to the sixth … The sixth prepared chariot would bring him to the seventh, where he’d dismount and mount the seventh chariot. The seventh prepared chariot would bring him to the gate of the royal compound of \textsanskrit{Sāketa}. And when he was at the gate, friends and colleagues, relatives and kin would ask him: ‘Great king, did you come to \textsanskrit{Sāketa} from \textsanskrit{Sāvatthī} by this prepared chariot?’ If asked this, how should King Pasenadi rightly reply?” 

“The\marginnote{14.15} king should reply: ‘Well, while staying in \textsanskrit{Sāvatthī}, I had some urgent business come up in \textsanskrit{Sāketa}. Now, between \textsanskrit{Sāvatthī} and \textsanskrit{Sāketa} seven prepared chariots were stationed ready for me. Then, having departed \textsanskrit{Sāvatthī}, I mounted the first prepared chariot by the gate of the royal compound. The first prepared chariot brought me to the second, where I dismounted and mounted the second chariot. … The sixth prepared chariot brought me to the seventh, where I dismounted and mounted the seventh chariot. The seventh prepared chariot brought me to the gate of the royal compound of \textsanskrit{Sāketa}.’ That’s how King Pasenadi should rightly reply.” 

“In\marginnote{15.1} the same way, reverend, purification of ethics is only for the sake of purification of mind. Purification of mind is only for the sake of purification of view. Purification of view is only for the sake of purification through overcoming doubt. Purification through overcoming doubt is only for the sake of purification of knowledge and vision of the variety of paths. Purification of knowledge and vision of the variety of paths is only for the sake of purification of knowledge and vision of the practice. Purification of knowledge and vision of the practice is only for the sake of purification of knowledge and vision. Purification of knowledge and vision is only for the sake of extinguishment by not grasping. The spiritual life is lived under the Buddha for the sake of extinguishment by not grasping.” 

When\marginnote{16.1} he said this, \textsanskrit{Sāriputta} said to \textsanskrit{Puṇṇa}, “What is the venerable’s name? And how are you known among your spiritual companions?” 

“Reverend,\marginnote{16.3} my name is \textsanskrit{Puṇṇa}. And I am known as \textsanskrit{Mantāṇiputta} among my spiritual companions.” 

“It’s\marginnote{16.5} incredible, reverend, it’s amazing! Venerable \textsanskrit{Puṇṇa} son of \textsanskrit{Mantāṇī} has answered each deep question point by point, as a learned disciple who rightly understands the teacher’s instructions. It is fortunate for his spiritual companions, so very fortunate, that they get to see Venerable \textsanskrit{Puṇṇa} son of \textsanskrit{Mantāṇī} and pay homage to him. Even if they only got to see him and pay respects to him by carrying him around on their heads on a roll of cloth, it would still be very fortunate for them! And it’s fortunate for me, so very fortunate, that I get to see the venerable and pay homage to him.” 

When\marginnote{17.1} he said this, \textsanskrit{Puṇṇa} said to \textsanskrit{Sāriputta}, “What is the venerable’s name? And how are you known among your spiritual companions?” 

“Reverend,\marginnote{17.3} my name is Upatissa. And I am known as \textsanskrit{Sāriputta} among my spiritual companions.” 

“Goodness!\marginnote{17.5} I had no idea I was consulting with \emph{the} Venerable \textsanskrit{Sāriputta}, the disciple who is fit to be compared with the Teacher himself! If I’d known, I wouldn’t have said so much. It’s incredible, reverend, it’s amazing! Venerable \textsanskrit{Sāriputta} has asked each deep question point by point, as a learned disciple who rightly understands the teacher’s instructions. It is fortunate for his spiritual companions, so very fortunate, that they get to see Venerable \textsanskrit{Sāriputta} and pay homage to him. Even if they only got to see him and pay respects to him by carrying him around on their heads on a roll of cloth, it would still be very fortunate for them! And it’s fortunate for me, so very fortunate, that I get to see the venerable and pay homage to him.” 

And\marginnote{17.13} so these two spiritual giants agreed with each others’ fine words. 

%
\section*{{\suttatitleacronym MN 25}{\suttatitletranslation Fodder }{\suttatitleroot Nivāpasutta}}
\addcontentsline{toc}{section}{\tocacronym{MN 25} \toctranslation{Fodder } \tocroot{Nivāpasutta}}
\markboth{Fodder }{Nivāpasutta}
\extramarks{MN 25}{MN 25}

\scevam{So\marginnote{1.1} I have heard. }At one time the Buddha was staying near \textsanskrit{Sāvatthī} in Jeta’s Grove, \textsanskrit{Anāthapiṇḍika}’s monastery. There the Buddha addressed the mendicants, “Mendicants!” 

“Venerable\marginnote{1.5} sir,” they replied. The Buddha said this: 

“Mendicants,\marginnote{2.1} a trapper doesn’t cast bait for deer thinking, ‘May the deer, enjoying this bait, be healthy and in good condition. May they live long and prosper!’ A trapper casts bait for deer thinking, ‘When these deer intrude on where I cast the bait, they’ll recklessly enjoy eating it. They’ll become indulgent, then they’ll become negligent, and then I’ll be able to do what I want with them on account of this bait.’ 

And\marginnote{3.1} indeed, the first herd of deer intruded on where the trapper cast the bait and recklessly enjoyed eating it. They became indulgent, then they became negligent, and then the trapper was able to do what he wanted with them on account of that bait. And that’s how the first herd of deer failed to get free from the trapper’s power. 

So\marginnote{4.1} then a second herd of deer thought up a plan, ‘The first herd of deer became indulgent … and failed to get free of the trapper’s power. Why don’t we avoid eating the bait altogether? Avoiding dangerous food, we can venture deep into a wilderness region and live there.’ And that’s just what they did. But when it came to the last month of summer, the grass and water ran out. Their bodies became much too thin, and they lost their strength and energy. So they went back to that same place where the trapper had cast bait. Intruding on that place, they recklessly enjoyed eating it … And that’s how the second herd failed to get free from the trapper’s power. 

So\marginnote{5.1} then a third herd of deer thought up a plan, ‘The first … and second herds of deer … failed to get free of the trapper’s power. Why don’t we set up our lair close by the place where the trapper has cast the bait? Then we can intrude on it and enjoy eating without being reckless. We won’t become indulgent, then we won’t become negligent, and then  the trapper won’t be able to do what he wants with us on account of that bait.’ And that’s just what they did. 

So\marginnote{5.19} the trapper and his companions thought, ‘Wow, this third herd of deer is so sneaky and devious, they must be some kind of unnatural spirits with psychic power! For they eat the bait we’ve cast without us knowing how they come and go. Why don’t we surround the bait on all sides by staking out high nets? Hopefully we might get to see their lair, where they go to hide out.’ And that’s just what they did. And they saw where the third herd of deer had their lair, where they went to hide out. And that’s how the third herd failed to get free from the trapper’s power. 

So\marginnote{6.1} then a fourth herd of deer thought up a plan, ‘The first … second … and third herds of deer … failed to get free of the trapper’s power. Why don’t we set up our lair somewhere the trapper and his companions can’t go? Then we can intrude on where the trapper has cast the bait and enjoy eating it without being reckless. We won’t become indulgent, then we won’t become negligent, and then the trapper won’t be able to do with them what he wants on account of that bait.’ And that’s just what they did. 

So\marginnote{6.31} the trapper and his companions thought, ‘Wow, this fourth herd of deer is so sneaky and devious, they must be some kind of unnatural spirits with psychic power! For they eat the bait we’ve cast without us knowing how they come and go. Why don’t we surround the bait on all sides by staking out high nets? Hopefully we might get to see their lair, where they go to hide out.’ And that’s just what they did. But they couldn’t see where the fourth herd of deer had their lair, where they went to hide out. So the trapper and his companions thought, ‘If we disturb this fourth herd of deer, they’ll disturb others, who in turn will disturb even more. Then all of the deer will be free from this bait we’ve cast. Why don’t we just keep an eye on that fourth herd?’ And that’s just what they did. And that’s how the fourth herd of deer got free from the trapper’s power. 

I’ve\marginnote{7.1} made up this simile to make a point. And this is what it means. 

‘Bait’\marginnote{7.3} is a term for the five kinds of sensual stimulation. 

‘Trapper’\marginnote{7.4} is a term for \textsanskrit{Māra} the Wicked. 

‘Trapper’s\marginnote{7.5} companions’ is a term for \textsanskrit{Māra}’s assembly. 

‘Deer’\marginnote{7.6} is a term for ascetics and brahmins. 

Now,\marginnote{8.1} the first group of ascetics and brahmins intruded on where the bait and the material delights of the world were cast by \textsanskrit{Māra} and recklessly enjoyed eating it. They became indulgent, then they became negligent, and then \textsanskrit{Māra} was able to do what he wanted with them on account of that bait and the material delights of the world. And that’s how the first group of ascetics and brahmins failed to get free from \textsanskrit{Māra}’s power. This first group of ascetics and brahmins is just like the first herd of deer, I say. 

So\marginnote{9.1} then a second group of ascetics and brahmins thought up a plan, ‘The first group of ascetics and brahmins became indulgent … and failed to get free of \textsanskrit{Māra}’s power. Why don’t we avoid eating the bait and the world’s material delights altogether? Avoiding dangerous food, we can venture deep into a wilderness region and live there.’ And that’s just what they did. They ate herbs, millet, wild rice, poor rice, water lettuce, rice bran, scum from boiling rice, sesame flour, grass, or cow dung. They survived on forest roots and fruits, or eating fallen fruit. 

But\marginnote{9.9} when it came to the last month of summer, the grass and water ran out. Their bodies became much too thin, and they lost their strength and energy. Because of this, they lost their heart’s release, so they went back to that same place where \textsanskrit{Māra} had cast the bait and the material delights of the world. Intruding on that place, they recklessly enjoyed eating them … And that’s how the second group of ascetics and brahmins failed to get free from \textsanskrit{Māra}’s power. This second group of ascetics and brahmins is just like the second herd of deer, I say. 

So\marginnote{10.1} then a third group of ascetics and brahmins thought up a plan, ‘The first … and second groups of ascetics and brahmins … failed to get free of \textsanskrit{Māra}’s power. Why don’t we set up our lair close by the place where \textsanskrit{Māra} has cast the bait and those material delights of the world? Then we can intrude on it and enjoy eating without being reckless. We won’t become indulgent, then we won’t become negligent, and then \textsanskrit{Māra} won’t be able to do what he wants with us on account of that bait and those material delights of the world.’ 

And\marginnote{10.17} that’s just what they did. Still, they had such views as these: ‘The cosmos is eternal’ or ‘The cosmos is not eternal’; ‘The world is finite’ or ‘The world is infinite’; ‘The soul and the body are the same thing’ or ‘The soul and the body are different things’; or that after death, a Realized One exists, or doesn’t exist, or both exists and doesn’t exist, or neither exists nor doesn’t exist. And that’s how the third group of ascetics and brahmins failed to get free from \textsanskrit{Māra}’s power. This third group of ascetics and brahmins is just like the third herd of deer, I say. 

So\marginnote{11.1} then a fourth group of ascetics and brahmins thought up a plan, ‘The first … second … and third groups of ascetics and brahmins … failed to get free of \textsanskrit{Māra}’s power. Why don’t we set up our lair somewhere \textsanskrit{Māra} and his assembly can’t go? Then we can intrude on where \textsanskrit{Māra} has cast the bait and those material delights of the world, and enjoy eating without being reckless. We won’t become indulgent, then we won’t become negligent, and then \textsanskrit{Māra} won’t be able to do what he wants with us on account of that bait and those material delights of the world.’ 

And\marginnote{11.29} that’s just what they did. And that’s how the fourth group of ascetics and brahmins got free from \textsanskrit{Māra}’s power. This fourth group of ascetics and brahmins is just like the fourth herd of deer, I say. 

And\marginnote{12.1} where is it that \textsanskrit{Māra} and his assembly can’t go? It’s when a mendicant, quite secluded from sensual pleasures, secluded from unskillful qualities, enters and remains in the first absorption, which has the rapture and bliss born of seclusion, while placing the mind and keeping it connected. This is called a mendicant who has blinded \textsanskrit{Māra}, put out his eyes without a trace, and gone where the Wicked One cannot see. 

Furthermore,\marginnote{13.1} as the placing of the mind and keeping it connected are stilled, a mendicant enters and remains in the second absorption, which has the rapture and bliss born of immersion, with internal clarity and confidence, and unified mind, without placing the mind and keeping it connected. This is called a mendicant who has blinded \textsanskrit{Māra} … 

Furthermore,\marginnote{14.1} with the fading away of rapture, a mendicant enters and remains in the third absorption, where they meditate with equanimity, mindful and aware, personally experiencing the bliss of which the noble ones declare, ‘Equanimous and mindful, one meditates in bliss.’ This is called a mendicant who has blinded \textsanskrit{Māra} … 

Furthermore,\marginnote{15.1} giving up pleasure and pain, and ending former happiness and sadness, a mendicant enters and remains in the fourth absorption, without pleasure or pain, with pure equanimity and mindfulness. This is called a mendicant who has blinded \textsanskrit{Māra} … 

Furthermore,\marginnote{16.1} a mendicant, going totally beyond perceptions of form, with the ending of perceptions of impingement, not focusing on perceptions of diversity, aware that ‘space is infinite’, enters and remains in the dimension of infinite space. This is called a mendicant who has blinded \textsanskrit{Māra} … 

Furthermore,\marginnote{17.1} a mendicant, going totally beyond the dimension of infinite space, aware that ‘consciousness is infinite’, enters and remains in the dimension of infinite consciousness. This is called a mendicant who has blinded \textsanskrit{Māra} … 

Furthermore,\marginnote{18.1} a mendicant, going totally beyond the dimension of infinite consciousness, aware that ‘there is nothing at all’, enters and remains in the dimension of nothingness. This is called a mendicant who has blinded \textsanskrit{Māra} … 

Furthermore,\marginnote{19.1} a mendicant, going totally beyond the dimension of nothingness, enters and remains in the dimension of neither perception nor non-perception. This is called a mendicant who has blinded \textsanskrit{Māra} … 

Furthermore,\marginnote{20.1} a mendicant, going totally beyond the dimension of neither perception nor non-perception, enters and remains in the cessation of perception and feeling. And, having seen with wisdom, their defilements come to an end. This is called a mendicant who has blinded \textsanskrit{Māra}, put out his eyes without a trace, and gone where the Wicked One cannot see. And they’ve crossed over clinging to the world.” 

That\marginnote{20.3} is what the Buddha said. Satisfied, the mendicants were happy with what the Buddha said. 

%
\section*{{\suttatitleacronym MN 26}{\suttatitletranslation The Noble Search }{\suttatitleroot Pāsarāsisutta}}
\addcontentsline{toc}{section}{\tocacronym{MN 26} \toctranslation{The Noble Search } \tocroot{Pāsarāsisutta}}
\markboth{The Noble Search }{Pāsarāsisutta}
\extramarks{MN 26}{MN 26}

\scevam{So\marginnote{1.1} I have heard. }At one time the Buddha was staying near \textsanskrit{Sāvatthī} in Jeta’s Grove, \textsanskrit{Anāthapiṇḍika}’s monastery. 

Then\marginnote{2.1} the Buddha robed up in the morning and, taking his bowl and robe, entered \textsanskrit{Sāvatthī} for alms. Then several mendicants went up to Venerable Ānanda and said to him, “Reverend, it’s been a long time since we’ve heard a Dhamma talk from the Buddha. It would be good if we got to hear a Dhamma talk from the Buddha.” 

“Well\marginnote{2.5} then, reverends, go to the brahmin Rammaka’s hermitage. Hopefully you’ll get to hear a Dhamma talk from the Buddha.” 

“Yes,\marginnote{2.7} reverend,” they replied. 

Then,\marginnote{3.1} after the meal, on his return from almsround, the Buddha addressed Ānanda, “Come, Ānanda, let’s go to the Eastern Monastery, the stilt longhouse of \textsanskrit{Migāra}’s mother for the day’s meditation.” 

“Yes,\marginnote{3.3} sir,” Ānanda replied. So the Buddha went with Ānanda to the Eastern Monastery. In the late afternoon the Buddha came out of retreat and addressed Ānanda, “Come, Ānanda, let’s go to the eastern gate to bathe.” 

“Yes,\marginnote{3.7} sir,” Ānanda replied. 

So\marginnote{3.8} the Buddha went with Ānanda to the eastern gate to bathe. When he had bathed and emerged from the water he stood in one robe drying himself. Then Ānanda said to the Buddha, “Sir, the hermitage of the brahmin Rammaka is nearby. It’s so delightful, so lovely. Please visit it out of compassion.” The Buddha consented in silence. 

He\marginnote{4.1} went to the brahmin Rammaka’s hermitage. Now at that time several mendicants were sitting together in the hermitage talking about the teaching. The Buddha stood outside the door waiting for the talk to end. When he knew the talk had ended he cleared his throat and knocked with the latch. The mendicants opened the door for the Buddha, and he entered the hermitage, where he sat on the seat spread out and addressed the mendicants, “Mendicants, what were you sitting talking about just now? What conversation was left unfinished?” 

“Sir,\marginnote{4.9} our unfinished discussion on the teaching was about the Buddha himself when the Buddha arrived.” 

“Good,\marginnote{4.10} mendicants! It’s appropriate for gentlemen like you, who have gone forth in faith from the lay life to homelessness, to sit together and talk about the teaching. When you’re sitting together you should do one of two things: discuss the teachings or keep noble silence. 

Mendicants,\marginnote{5.1} there are these two searches: the noble search and the ignoble search. 

And\marginnote{5.3} what is the ignoble search? It’s when someone who is themselves liable to be reborn seeks what is also liable to be reborn. Themselves liable to grow old, fall sick, die, sorrow, and become corrupted, they seek what is also liable to these things. 

And\marginnote{6.1} what should be described as liable to be reborn? Partners and children, male and female bondservants, goats and sheep, chickens and pigs, and elephants and cattle are liable to be reborn. These attachments are liable to be reborn. Someone who is tied, infatuated, and attached to such things, themselves liable to being reborn, seeks what is also liable to be reborn. 

And\marginnote{7.1} what should be described as liable to grow old? Partners and children, male and female bondservants, goats and sheep, chickens and pigs, and elephants and cattle are liable to grow old. These attachments are liable to grow old. Someone who is tied, infatuated, and attached to such things, themselves liable to grow old, seeks what is also liable to grow old. 

And\marginnote{8.1} what should be described as liable to fall sick? Partners and children, male and female bondservants, goats and sheep, chickens and pigs, and elephants and cattle are liable to fall sick. These attachments are liable to fall sick. Someone who is tied, infatuated, and attached to such things, themselves liable to falling sick, seeks what is also liable to fall sick. 

And\marginnote{9.1} what should be described as liable to die? Partners and children, male and female bondservants, goats and sheep, chickens and pigs, and elephants and cattle are liable to die. These attachments are liable to die. Someone who is tied, infatuated, and attached to such things, themselves liable to die, seeks what is also liable to die. 

And\marginnote{10.1} what should be described as liable to sorrow? Partners and children, male and female bondservants, goats and sheep, chickens and pigs, and elephants and cattle are liable to sorrow. These attachments are liable to sorrow. Someone who is tied, infatuated, and attached to such things, themselves liable to sorrow, seeks what is also liable to sorrow. 

And\marginnote{11.1} what should be described as liable to corruption? Partners and children, male and female bondservants, goats and sheep, chickens and pigs, elephants and cattle, and gold and money are liable to corruption. These attachments are liable to corruption. Someone who is tied, infatuated, and attached to such things, themselves liable to corruption, seeks what is also liable to corruption. This is the ignoble search. 

And\marginnote{12.1} what is the noble search? It’s when someone who is themselves liable to be reborn, understanding the drawbacks in being liable to be reborn, seeks the unborn supreme sanctuary, extinguishment. Themselves liable to grow old, fall sick, die, sorrow, and become corrupted, understanding the drawbacks in these things, they seek the unaging, unailing, undying, sorrowless, uncorrupted supreme sanctuary, extinguishment. This is the noble search. 

Mendicants,\marginnote{13.1} before my awakening—when I was still unawakened but intent on awakening—I too, being liable to be reborn, sought what is also liable to be reborn. Myself liable to grow old, fall sick, die, sorrow, and become corrupted, I sought what is also liable to these things. Then it occurred to me: ‘Why do I, being liable to be reborn, grow old, fall sick, sorrow, die, and become corrupted, seek things that have the same nature? Why don’t I seek the unborn, unaging, unailing, undying, sorrowless, uncorrupted supreme sanctuary, extinguishment?’ 

Some\marginnote{14.1} time later, while still black-haired, blessed with youth, in the prime of life—though my mother and father wished otherwise, weeping with tearful faces—I shaved off my hair and beard, dressed in ocher robes, and went forth from the lay life to homelessness. 

Once\marginnote{15.1} I had gone forth I set out to discover what is skillful, seeking the supreme state of sublime peace. I approached \textsanskrit{Āḷāra} \textsanskrit{Kālāma} and said to him, ‘Reverend \textsanskrit{Kālāma}, I wish to lead the spiritual life in this teaching and training.’ 

\textsanskrit{Āḷāra}\marginnote{15.3} \textsanskrit{Kālāma} replied, ‘Stay, venerable. This teaching is such that a sensible person can soon realize their own tradition with their own insight and live having achieved it.’ 

I\marginnote{15.6} quickly memorized that teaching. So far as lip-recital and oral recitation were concerned, I spoke with knowledge and the authority of the elders. I claimed to know and see, and so did others. 

Then\marginnote{15.8} it occurred to me, ‘It is not solely by mere faith that \textsanskrit{Āḷāra} \textsanskrit{Kālāma} declares: “I realize this teaching with my own insight, and live having achieved it.” Surely he meditates knowing and seeing this teaching.’ 

So\marginnote{15.11} I approached \textsanskrit{Āḷāra} \textsanskrit{Kālāma} and said to him, ‘Reverend \textsanskrit{Kālāma}, to what extent do you say you’ve realized this teaching with your own insight?’ When I said this, he declared the dimension of nothingness. 

Then\marginnote{15.14} it occurred to me, ‘It’s not just \textsanskrit{Āḷāra} \textsanskrit{Kālāma} who has faith, energy, mindfulness, immersion, and wisdom; I too have these things. Why don’t I make an effort to realize the same teaching that \textsanskrit{Āḷāra} \textsanskrit{Kālāma} says he has realized with his own insight?’ I quickly realized that teaching with my own insight, and lived having achieved it. 

So\marginnote{15.22} I approached \textsanskrit{Āḷāra} \textsanskrit{Kālāma} and said to him, ‘Reverend \textsanskrit{Kālāma}, have you realized this teaching with your own insight up to this point, and declare having achieved it?’ 

‘I\marginnote{15.24} have, reverend.’ 

‘I\marginnote{15.25} too, reverend, have realized this teaching with my own insight up to this point, and live having achieved it.’ 

‘We\marginnote{15.26} are fortunate, reverend, so very fortunate to see a venerable such as yourself as one of our spiritual companions! So the teaching that I’ve realized with my own insight, and declare having achieved it, you’ve realized with your own insight, and live having achieved it. The teaching that you’ve realized with your own insight, and live having achieved it, I’ve realized with my own insight, and declare having achieved it. So the teaching that I know, you know, and the teaching that you know, I know. I am like you and you are like me. Come now, reverend! We should both lead this community together.’ 

And\marginnote{15.33} that is how my teacher \textsanskrit{Āḷāra} \textsanskrit{Kālāma} placed me, his student, on the same position as him, and honored me with lofty praise. 

Then\marginnote{15.34} it occurred to me, ‘This teaching doesn’t lead to disillusionment, dispassion, cessation, peace, insight, awakening, and extinguishment. It only leads as far as rebirth in the dimension of nothingness.’ Realizing that this teaching was inadequate, I left disappointed. 

I\marginnote{16.1} set out to discover what is skillful, seeking the supreme state of sublime peace. I approached Uddaka, son of \textsanskrit{Rāma}, and said to him, ‘Reverend, I wish to lead the spiritual life in this teaching and training.’ 

Uddaka\marginnote{16.3} replied, ‘Stay, venerable. This teaching is such that a sensible person can soon realize their own tradition with their own insight and live having achieved it.’ 

I\marginnote{16.6} quickly memorized that teaching. So far as lip-recital and oral recitation were concerned, I spoke with knowledge and the authority of the elders. I claimed to know and see, and so did others. 

Then\marginnote{16.8} it occurred to me, ‘It is not solely by mere faith that \textsanskrit{Rāma} declared: “I realize this teaching with my own insight, and live having achieved it.” Surely he meditated knowing and seeing this teaching.’ 

So\marginnote{16.11} I approached Uddaka, son of \textsanskrit{Rāma}, and said to him, ‘Reverend, to what extent did \textsanskrit{Rāma} say he’d realized this teaching with his own insight?’ 

When\marginnote{16.13} I said this, Uddaka, son of \textsanskrit{Rāma}, declared the dimension of neither perception nor non-perception. 

Then\marginnote{16.14} it occurred to me, ‘It’s not just \textsanskrit{Rāma} who had faith, energy, mindfulness, immersion, and wisdom; I too have these things. Why don’t I make an effort to realize the same teaching that \textsanskrit{Rāma} said he had realized with his own insight?’ I quickly realized that teaching with my own insight, and lived having achieved it. 

So\marginnote{16.22} I approached Uddaka, son of \textsanskrit{Rāma}, and said to him, ‘Reverend, had \textsanskrit{Rāma} realized this teaching with his own insight up to this point, and declared having achieved it?’ 

‘He\marginnote{16.24} had, reverend.’ 

‘I\marginnote{16.25} too have realized this teaching with my own insight up to this point, and live having achieved it.’ 

‘We\marginnote{16.26} are fortunate, reverend, so very fortunate to see a venerable such as yourself as one of our spiritual companions! So the teaching that \textsanskrit{Rāma} had realized with his own insight, and declared having achieved it, you’ve realized with your own insight, and live having achieved it. The teaching that you’ve realized with your own insight, and live having achieved it, \textsanskrit{Rāma} had realized with his own insight, and declared having achieved it. So the teaching that \textsanskrit{Rāma} directly knew, you know, and the teaching you know, \textsanskrit{Rāma} directly knew. \textsanskrit{Rāma} was like you and you are like \textsanskrit{Rāma}. Come now, reverend! You should lead this community.’ 

And\marginnote{16.33} that is how my spiritual companion Uddaka, son of \textsanskrit{Rāma}, placed me in the position of a teacher, and honored me with lofty praise. 

Then\marginnote{16.34} it occurred to me, ‘This teaching doesn’t lead to disillusionment, dispassion, cessation, peace, insight, awakening, and extinguishment. It only leads as far as rebirth in the dimension of neither perception nor non-perception.’ Realizing that this teaching was inadequate, I left disappointed. 

I\marginnote{17.1} set out to discover what is skillful, seeking the supreme state of sublime peace. Traveling stage by stage in the Magadhan lands, I arrived at Senanigama near \textsanskrit{Uruvelā}. There I saw a delightful park, a lovely grove with a flowing river that was clean and charming, with smooth banks. And nearby was a village for alms. 

Then\marginnote{17.3} it occurred to me, ‘This park is truly delightful, a lovely grove with a flowing river that’s clean and charming, with smooth banks. And nearby there’s a village to go for alms. This is good enough for a gentleman who wishes to put forth effort in meditation.’ So I sat down right there, thinking, ‘This is good enough for meditation.’ 

And\marginnote{18.1} so, being myself liable to be reborn, understanding the drawbacks in being liable to be reborn, I sought the unborn supreme sanctuary, extinguishment—and I found it. Being myself liable to grow old, fall sick, die, sorrow, and become corrupted, understanding the drawbacks in these things, I sought the unaging, unailing, undying, sorrowless, uncorrupted supreme sanctuary, extinguishment—and I found it. 

Knowledge\marginnote{18.2} and vision arose in me: ‘My freedom is unshakable; this is my last rebirth; now there are no more future lives.’ 

Then\marginnote{19.1} it occurred to me, ‘This principle I have discovered is deep, hard to see, hard to understand, peaceful, sublime, beyond the scope of logic, subtle, comprehensible to the astute. But people like attachment, they love it and enjoy it. It’s hard for them to see this thing; that is, specific conditionality, dependent origination. It’s also hard for them to see this thing; that is, the stilling of all activities, the letting go of all attachments, the ending of craving, fading away, cessation, extinguishment. And if I were to teach the Dhamma, others might not understand me, which would be wearying and troublesome for me.’ 

And\marginnote{19.7} then these verses, which were neither supernaturally inspired, nor learned before in the past, occurred to me: 

\begin{verse}%
‘I’ve\marginnote{19.8} struggled hard to realize this, \\
enough with trying to explain it! \\
This teaching is not easily understood \\
by those mired in greed and hate. 

Those\marginnote{19.12} besotted by greed can’t see \\
what’s subtle, going against the stream, \\
deep, hard to see, and very fine, \\
for they’re shrouded in a mass of darkness.’ 

%
\end{verse}

So,\marginnote{19.16} as I reflected like this, my mind inclined to remaining passive, not to teaching the Dhamma. 

Then\marginnote{20.1} \textsanskrit{Brahmā} Sahampati, knowing what I was thinking, thought, ‘Oh my goodness! The world will be lost, the world will perish! For the mind of the Realized One, the perfected one, the fully awakened Buddha, inclines to remaining passive, not to teaching the Dhamma.’ 

Then,\marginnote{20.3} as easily as a strong person would extend or contract their arm, he vanished from the \textsanskrit{Brahmā} realm and reappeared in front of the Buddha. He arranged his robe over one shoulder, raised his joined palms toward the Buddha, and said, ‘Sir, let the Blessed One teach the Dhamma! Let the Holy One teach the Dhamma! There are beings with little dust in their eyes. They’re in decline because they haven’t heard the teaching. There will be those who understand the teaching!’ 

That’s\marginnote{20.8} what \textsanskrit{Brahmā} Sahampati said. Then he went on to say: 

\begin{verse}%
‘Among\marginnote{20.10} the Magadhans there appeared in the past \\
an impure teaching thought up by those still stained. \\
Fling open the door to the deathless! \\
Let them hear the teaching the immaculate one discovered. 

Standing\marginnote{20.14} high on a rocky mountain, \\
you can see the people all around. \\
In just the same way, all-seer, wise one, \\
having ascended the Temple of Truth, \\
rid of sorrow, look upon the people \\
swamped with sorrow, oppressed by rebirth and old age. 

Rise,\marginnote{20.20} hero! Victor in battle, leader of the caravan, \\
wander the world without obligation. \\
Let the Blessed One teach the Dhamma! \\
There will be those who understand!’ 

%
\end{verse}

Then,\marginnote{21.1} understanding \textsanskrit{Brahmā}’s invitation, I surveyed the world with the eye of a Buddha, because of my compassion for sentient beings. And I saw sentient beings with little dust in their eyes, and some with much dust in their eyes; with keen faculties and with weak faculties, with good qualities and with bad qualities, easy to teach and hard to teach. And some of them lived seeing the danger in the fault to do with the next world, while others did not. It’s like a pool with blue water lilies, or pink or white lotuses. Some of them sprout and grow in the water without rising above it, thriving underwater. Some of them sprout and grow in the water reaching the water’s surface. And some of them sprout and grow in the water but rise up above the water and stand with no water clinging to them. In the same way, I saw sentient beings with little dust in their eyes, and some with much dust in their eyes. 

Then\marginnote{21.5} I replied in verse to \textsanskrit{Brahmā} Sahampati: 

\begin{verse}%
‘Flung\marginnote{21.6} open are the doors to the deathless! \\
Let those with ears to hear commit to faith. \\
Thinking it would be troublesome, \textsanskrit{Brahmā}, I did not teach \\
the sophisticated, sublime Dhamma among humans.’ 

%
\end{verse}

Then\marginnote{21.10} \textsanskrit{Brahmā} Sahampati, knowing that his request for me to teach the Dhamma had been granted, bowed and respectfully circled me, keeping me on his right, before vanishing right there. 

Then\marginnote{22.1} I thought, ‘Who should I teach first of all? Who will quickly understand this teaching?’ 

Then\marginnote{22.4} it occurred to me, ‘That \textsanskrit{Āḷāra} \textsanskrit{Kālāma} is astute, competent, clever, and has long had little dust in his eyes. Why don’t I teach him first of all? He’ll quickly understand the teaching.’ 

But\marginnote{22.8} a deity came to me and said, ‘Sir, \textsanskrit{Āḷāra} \textsanskrit{Kālāma} passed away seven days ago.’ 

And\marginnote{22.10} knowledge and vision arose in me, ‘\textsanskrit{Āḷāra} \textsanskrit{Kālāma} passed away seven days ago.’ 

I\marginnote{22.12} thought, ‘This is a great loss for \textsanskrit{Āḷāra} \textsanskrit{Kālāma}. If he had heard the teaching, he would have understood it quickly.’ 

Then\marginnote{23.1} I thought, ‘Who should I teach first of all? Who will quickly understand this teaching?’ 

Then\marginnote{23.4} it occurred to me, ‘That Uddaka, son of \textsanskrit{Rāma}, is astute, competent, clever, and has long had little dust in his eyes. Why don’t I teach him first of all? He’ll quickly understand the teaching.’ 

But\marginnote{23.8} a deity came to me and said, ‘Sir, Uddaka, son of \textsanskrit{Rāma}, passed away just last night.’ 

And\marginnote{23.10} knowledge and vision arose in me, ‘Uddaka, son of \textsanskrit{Rāma}, passed away just last night.’ 

I\marginnote{23.12} thought, ‘This is a great loss for Uddaka. If he had heard the teaching, he would have understood it quickly.’ 

Then\marginnote{24.1} I thought, ‘Who should I teach first of all? Who will quickly understand this teaching?’ 

Then\marginnote{24.4} it occurred to me, ‘The group of five mendicants were very helpful to me. They looked after me during my time of resolute striving. Why don’t I teach them first of all?’ 

Then\marginnote{24.7} I thought, ‘Where are the group of five mendicants staying these days?’ With clairvoyance that is purified and superhuman I saw that the group of five mendicants were staying near Benares, in the deer park at Isipatana. So, when I had stayed in \textsanskrit{Uruvelā} as long as I wished, I set out for Benares. 

While\marginnote{25.1} I was traveling along the road between \textsanskrit{Gayā} and Bodhgaya, the \textsanskrit{Ājīvaka} ascetic Upaka saw me and said, ‘Reverend, your faculties are so very clear, and your complexion is pure and bright. In whose name have you gone forth, reverend? Who is your Teacher? Whose teaching do you believe in?’ 

I\marginnote{25.5} replied to Upaka in verse: 

\begin{verse}%
‘I\marginnote{25.6} am the champion, the knower of all, \\
unsullied in the midst of all things. \\
I’ve given up all, freed through the ending of craving. \\
Since I know for myself, whose follower should I be? 

I\marginnote{25.10} have no teacher. \\
There is no-one like me. \\
In the world with its gods, \\
I have no counterpart. 

For\marginnote{25.14} in this world, I am the perfected one; \\
I am the supreme Teacher. \\
I alone am fully awakened, \\
cooled, extinguished. 

I\marginnote{25.18} am going to the city of \textsanskrit{Kāsi} \\
to roll forth the Wheel of Dhamma. \\
In this world that is so blind, \\
I’ll beat the deathless drum!’ 

%
\end{verse}

‘According\marginnote{25.22} to what you claim, reverend, you ought to be the Infinite Victor.’ 

\begin{verse}%
‘The\marginnote{25.23} victors are those who, like me, \\
have reached the ending of defilements. \\
I have conquered bad qualities, Upaka—\\
that’s why I’m a victor.’ 

%
\end{verse}

When\marginnote{25.27} I had spoken, Upaka said: ‘If you say so, reverend.’ Shaking his head, he took a wrong turn and left. 

Traveling\marginnote{26.1} stage by stage, I arrived at Benares, and went to see the group of five mendicants in the deer park at Isipatana. The group of five mendicants saw me coming off in the distance and stopped each other, saying, ‘Here comes the ascetic Gotama. He’s so indulgent; he strayed from the struggle and returned to indulgence. We shouldn’t bow to him or rise for him or receive his bowl and robe. But we can set out a seat; he can sit if he likes.’ Yet as I drew closer, the group of five mendicants were unable to stop themselves as they had agreed. Some came out to greet me and receive my bowl and robe, some spread out a seat, while others set out water for washing my feet. But they still addressed me by name and as ‘reverend’. 

So\marginnote{27.1} I said to them, ‘Mendicants, don’t address me by name and as ‘reverend’. The Realized One is perfected, a fully awakened Buddha. Listen up, mendicants: I have achieved the Deathless! I shall instruct you, I will teach you the Dhamma. By practicing as instructed you will soon realize the supreme end of the spiritual path in this very life. You will live having achieved with your own insight the goal for which gentlemen rightly go forth from the lay life to homelessness.’ 

But\marginnote{27.6} they said to me, ‘Reverend Gotama, even by that conduct, that practice, that grueling work you did not achieve any superhuman distinction in knowledge and vision worthy of the noble ones. How could you have achieved such a state now that you’ve become indulgent, strayed from the struggle and returned to indulgence?’ 

So\marginnote{27.8} I said to them, ‘The Realized One has not become indulgent, strayed from the struggle and returned to indulgence. The Realized One is perfected, a fully awakened Buddha. Listen up, mendicants: I have achieved the Deathless! I shall instruct you, I will teach you the Dhamma. By practicing as instructed you will soon realize the supreme end of the spiritual path in this very life.’ 

But\marginnote{27.13} for a second time they said to me, ‘Reverend Gotama … you’ve returned to indulgence.’ 

So\marginnote{27.15} for a second time I said to them, ‘The Realized One has not become indulgent …’ 

But\marginnote{27.18} for a third time they said to me, ‘Reverend Gotama, even by that conduct, that practice, that grueling work you did not achieve any superhuman distinction in knowledge and vision worthy of the noble ones. How could you have achieved such a state now that you’ve become indulgent, strayed from the struggle and returned to indulgence?’ 

So\marginnote{28.1} I said to them, ‘Mendicants, have you ever known me to speak like this before?’ 

‘No\marginnote{28.3} sir, we have not.’ 

‘The\marginnote{28.4} Realized One is perfected, a fully awakened Buddha. Listen up, mendicants: I have achieved the Deathless! I shall instruct you, I will teach you the Dhamma. By practicing as instructed you will soon realize the supreme end of the spiritual path in this very life. You will live having achieved with your own insight the goal for which gentlemen rightly go forth from the lay life to homelessness.’ 

I\marginnote{29.1} was able to persuade the group of five mendicants. Then sometimes I advised two mendicants, while the other three went for alms. Then those three would feed all six of us with what they brought back. Sometimes I advised three mendicants, while the other two went for alms. Then those two would feed all six of us with what they brought back. 

As\marginnote{30.1} the group of five mendicants were being advised and instructed by me like this, being themselves liable to be reborn, understanding the drawbacks in being liable to be reborn, they sought the unborn supreme sanctuary, extinguishment—and they found it. Being themselves liable to grow old, fall sick, die, sorrow, and become corrupted, understanding the drawbacks in these things, they sought the unaging, unailing, undying, sorrowless, uncorrupted supreme sanctuary, extinguishment—and they found it. Knowledge and vision arose in them: ‘Our freedom is unshakable; this is our last rebirth; now there are no more future lives.’ 

Mendicants,\marginnote{31.1} there are these five kinds of sensual stimulation. What five? Sights known by the eye that are likable, desirable, agreeable, pleasant, sensual, and arousing. Sounds known by the ear … Smells known by the nose … Tastes known by the tongue … Touches known by the body that are likable, desirable, agreeable, pleasant, sensual, and arousing. These are the five kinds of sensual stimulation. 

There\marginnote{32.1} are ascetics and brahmins who enjoy these five kinds of sensual stimulation tied, infatuated, attached, blind to the drawbacks, and not understanding the escape. You should understand that they have met with calamity and disaster, and the Wicked One can do with them what he wants. 

Suppose\marginnote{32.3} a deer in the wilderness was lying caught on a pile of snares. You’d know that it has met with calamity and disaster, and the hunter can do with them what he wants. And when the hunter comes, it cannot flee where it wants. 

In\marginnote{32.7} the same way, there are ascetics and brahmins who enjoy these five kinds of sensual stimulation tied, infatuated, attached, blind to the drawbacks, and not understanding the escape. You should understand that they have met with calamity and disaster, and the Wicked One can do with them what he wants. 

There\marginnote{33.1} are ascetics and brahmins who enjoy these five kinds of sensual stimulation without being tied, infatuated, or attached, seeing the drawbacks, and understanding the escape. You should understand that they haven’t met with calamity and disaster, and the Wicked One cannot do what he wants with them. 

Suppose\marginnote{33.3} a deer in the wilderness was lying on a pile of snares without being caught. You’d know that it hasn’t met with calamity and disaster, and the hunter cannot do what he wants with them. And when the hunter comes, it can flee where it wants. 

In\marginnote{33.7} the same way, there are ascetics and brahmins who enjoy these five kinds of sensual stimulation without being tied, infatuated, or attached, seeing the drawbacks, and understanding the escape. You should understand that they haven’t met with calamity and disaster, and the Wicked One cannot do what he wants with them. 

Suppose\marginnote{34.1} there was a wild deer wandering in the forest that walked, stood, sat, and laid down in confidence. Why is that? Because it’s out of the hunter’s range. 

In\marginnote{34.4} the same way, a mendicant, quite secluded from sensual pleasures, secluded from unskillful qualities, enters and remains in the first absorption, which has the rapture and bliss born of seclusion, while placing the mind and keeping it connected. This is called a mendicant who has blinded \textsanskrit{Māra}, put out his eyes without a trace, and gone where the Wicked One cannot see. 

Furthermore,\marginnote{35.1} as the placing of the mind and keeping it connected are stilled, a mendicant enters and remains in the second absorption, which has the rapture and bliss born of immersion, with internal clarity and confidence, and unified mind, without placing the mind and keeping it connected. This is called a mendicant who has blinded \textsanskrit{Māra} … 

Furthermore,\marginnote{36.1} with the fading away of rapture, a mendicant enters and remains in the third absorption, where they meditate with equanimity, mindful and aware, personally experiencing the bliss of which the noble ones declare, ‘Equanimous and mindful, one meditates in bliss.’ This is called a mendicant who has blinded \textsanskrit{Māra} … 

Furthermore,\marginnote{37.1} giving up pleasure and pain, and ending former happiness and sadness, a mendicant enters and remains in the fourth absorption, without pleasure or pain, with pure equanimity and mindfulness. This is called a mendicant who has blinded \textsanskrit{Māra} … 

Furthermore,\marginnote{38.1} a mendicant, going totally beyond perceptions of form, with the ending of perceptions of impingement, not focusing on perceptions of diversity, aware that ‘space is infinite’, enters and remains in the dimension of infinite space. This is called a mendicant who has blinded \textsanskrit{Māra} … 

Furthermore,\marginnote{39.1} a mendicant, going totally beyond the dimension of infinite space, aware that ‘consciousness is infinite’, enters and remains in the dimension of infinite consciousness. This is called a mendicant who has blinded \textsanskrit{Māra} … 

Furthermore,\marginnote{40.1} a mendicant, going totally beyond the dimension of infinite consciousness, aware that ‘there is nothing at all’, enters and remains in the dimension of nothingness. This is called a mendicant who has blinded \textsanskrit{Māra} … 

Furthermore,\marginnote{41.1} a mendicant, going totally beyond the dimension of nothingness, enters and remains in the dimension of neither perception nor non-perception. This is called a mendicant who has blinded \textsanskrit{Māra} … 

Furthermore,\marginnote{42.1} a mendicant, going totally beyond the dimension of neither perception nor non-perception, enters and remains in the cessation of perception and feeling. And, having seen with wisdom, their defilements come to an end. This is called a mendicant who has blinded \textsanskrit{Māra}, put out his eyes without a trace, and gone where the Wicked One cannot see. They’ve crossed over clinging to the world. And they walk, stand, sit, and lie down in confidence. Why is that? Because they’re out of the Wicked One’s range.” 

That\marginnote{42.6} is what the Buddha said. Satisfied, the mendicants were happy with what the Buddha said. 

%
\section*{{\suttatitleacronym MN 27}{\suttatitletranslation The Shorter Simile of the Elephant’s Footprint }{\suttatitleroot Cūḷahatthipadopamasutta}}
\addcontentsline{toc}{section}{\tocacronym{MN 27} \toctranslation{The Shorter Simile of the Elephant’s Footprint } \tocroot{Cūḷahatthipadopamasutta}}
\markboth{The Shorter Simile of the Elephant’s Footprint }{Cūḷahatthipadopamasutta}
\extramarks{MN 27}{MN 27}

\scevam{So\marginnote{1.1} I have heard. }At one time the Buddha was staying near \textsanskrit{Sāvatthī} in Jeta’s Grove, \textsanskrit{Anāthapiṇḍika}’s monastery. 

Now\marginnote{2.1} at that time the brahmin \textsanskrit{Jāṇussoṇi} drove out from \textsanskrit{Sāvatthī} in the middle of the day in an all-white chariot drawn by mares. He saw the wanderer Pilotika coming off in the distance, and said to him, “So, Master \textsanskrit{Vacchāyana}, where are you coming from in the middle of the day?” 

“Just\marginnote{2.5} now, good sir, I’ve come from the presence of the ascetic Gotama.” 

“What\marginnote{2.6} do you think of the ascetic Gotama’s lucidity of wisdom? Do you think he’s astute?” 

“My\marginnote{2.7} good man, who am I to judge the ascetic Gotama’s lucidity of wisdom? You’d really have to be on the same level to judge his lucidity of wisdom.” 

“Master\marginnote{2.9} \textsanskrit{Vacchāyana} praises the ascetic Gotama with lofty praise indeed.” 

“Who\marginnote{2.10} am I to praise the ascetic Gotama? He is praised by the praised as the first among gods and humans.” 

“But\marginnote{2.12} for what reason are you so devoted to the ascetic Gotama?” 

“Suppose\marginnote{3.1} that a skilled elephant tracker were to enter an elephant wood. There he’d see a large elephant’s footprint, long and broad. He’d draw the conclusion, ‘This must be a big bull elephant.’ 

In\marginnote{3.5} the same way, because I saw four footprints of the ascetic Gotama I drew the conclusion, ‘The Blessed One is a fully awakened Buddha. The teaching is well explained. The \textsanskrit{Saṅgha} is practicing well.’ 

What\marginnote{4.1} four? Firstly, I see some clever aristocrats who are subtle, accomplished in the doctrines of others, hair-splitters. You’d think they live to demolish convictions with their intellect. They hear, ‘So, gentlemen, that ascetic Gotama will come down to such and such village or town.’ They formulate a question, thinking, ‘We’ll approach the ascetic Gotama and ask him this question. If he answers like this, we’ll refute him like that; and if he answers like that, we’ll refute him like this.’ 

When\marginnote{4.9} they hear that he has come down they approach him. The ascetic Gotama educates, encourages, fires up, and inspires them with a Dhamma talk. They don’t even get around to asking their question to the ascetic Gotama, so how could they refute his answer? Invariably, they become his disciples. When I saw this first footprint of the ascetic Gotama, I drew the conclusion, ‘The Blessed One is a fully awakened Buddha. The teaching is well explained. The \textsanskrit{Saṅgha} is practicing well.’ 

Furthermore,\marginnote{5.1} I see some clever brahmins … some clever householders … they become his disciples. 

Furthermore,\marginnote{7.1} I see some clever ascetics who are subtle, accomplished in the doctrines of others, hair-splitters. … They don’t even get around to asking their question to the ascetic Gotama, so how could they refute his answer? Invariably, they ask the ascetic Gotama for the chance to go forth. And he gives them the going-forth. Soon after going forth, living withdrawn, diligent, keen, and resolute, they realize the supreme end of the spiritual path in this very life. They live having achieved with their own insight the goal for which gentlemen rightly go forth from the lay life to homelessness. 

They\marginnote{7.14} say, ‘We were almost lost! We almost perished! For we used to claim that we were ascetics, brahmins, and perfected ones, but we were none of these things. But now we really are ascetics, brahmins, and perfected ones!’ When I saw this fourth footprint of the ascetic Gotama, I drew the conclusion, ‘The Blessed One is a fully awakened Buddha. The teaching is well explained. The \textsanskrit{Saṅgha} is practicing well.’ 

It’s\marginnote{7.20} because I saw these four footprints of the ascetic Gotama that I drew the conclusion, ‘The Blessed One is a fully awakened Buddha. The teaching is well explained. The \textsanskrit{Saṅgha} is practicing well.’” 

When\marginnote{8.1} he had spoken, \textsanskrit{Jāṇussoṇi} got down from his chariot, arranged his robe over one shoulder, raised his joined palms toward the Buddha, and expressed this heartfelt sentiment three times: 

“Homage\marginnote{8.2} to that Blessed One, the perfected one, the fully awakened Buddha! 

Homage\marginnote{8.3} to that Blessed One, the perfected one, the fully awakened Buddha! 

Homage\marginnote{8.4} to that Blessed One, the perfected one, the fully awakened Buddha! 

Hopefully,\marginnote{8.5} some time or other I’ll get to meet Master Gotama, and we can have a discussion.” 

Then\marginnote{9.1} the brahmin \textsanskrit{Jāṇussoṇi} went up to the Buddha, and exchanged greetings with him. When the greetings and polite conversation were over, he sat down to one side, and informed the Buddha of all they had discussed. 

When\marginnote{9.4} he had spoken, the Buddha said to him, “Brahmin, the simile of the elephant’s footprint is not yet completed in detail. As to how it is completed in detail, listen and pay close attention, I will speak.” 

“Yes\marginnote{9.8} sir,” \textsanskrit{Jāṇussoṇi} replied. The Buddha said this: 

“Suppose\marginnote{10.1} that an elephant tracker were to enter an elephant wood. There they’d see a large elephant’s footprint, long and broad. A skilled elephant tracker wouldn’t yet come to the conclusion, ‘This must be a big bull elephant.’ Why not? Because in an elephant wood there are dwarf she-elephants with big footprints, and this footprint might be one of theirs. 

They\marginnote{10.7} keep following the track until they see a big footprint, long and broad, and, high up, signs of usage. A skilled elephant tracker wouldn’t yet come to the conclusion, ‘This must be a big bull elephant.’ Why not? Because in an elephant wood there are tall she-elephants with long trunks and big footprints, and this footprint might be one of theirs. 

They\marginnote{10.13} keep following the track until they see a big footprint, long and broad, and, high up, signs of usage and tusk-marks. A skilled elephant tracker wouldn’t yet come to the conclusion, ‘This must be a big bull elephant.’ Why not? Because in an elephant wood there are tall and fully-grown she-elephants with big footprints, and this footprint might be one of theirs. 

They\marginnote{10.19} keep following the track until they see a big footprint, long and broad, and, high up, signs of usage, tusk-marks, and broken branches. And they see that bull elephant walking, standing, sitting, or lying down at the root of a tree or in the open. Then they’d come to the conclusion, ‘This is that big bull elephant.’ 

In\marginnote{11.1} the same way, brahmin, a Realized One arises in the world, perfected, a fully awakened Buddha, accomplished in knowledge and conduct, holy, knower of the world, supreme guide for those who wish to train, teacher of gods and humans, awakened, blessed. He realizes with his own insight this world—with its gods, \textsanskrit{Māras} and \textsanskrit{Brahmās}, this population with its ascetics and brahmins, gods and humans—and he makes it known to others. He teaches Dhamma that’s good in the beginning, good in the middle, and good in the end, meaningful and well-phrased. And he reveals a spiritual practice that’s entirely complete and pure. 

A\marginnote{12.1} householder hears that teaching, or a householder’s child, or someone reborn in some good family. They gain faith in the Realized One, and reflect, ‘Living in a house is cramped and dirty, but the life of one gone forth is wide open. It’s not easy for someone living at home to lead the spiritual life utterly full and pure, like a polished shell. Why don’t I shave off my hair and beard, dress in ocher robes, and go forth from the lay life to homelessness?’ After some time they give up a large or small fortune, and a large or small family circle. They shave off hair and beard, dress in ocher robes, and go forth from the lay life to homelessness. 

Once\marginnote{13.1} they’ve gone forth, they take up the training and livelihood of the mendicants. They give up killing living creatures, renouncing the rod and the sword. They’re scrupulous and kind, living full of compassion for all living beings. 

They\marginnote{13.2} give up stealing. They take only what’s given, and expect only what’s given. They keep themselves clean by not thieving. 

They\marginnote{13.3} give up unchastity. They are celibate, set apart, avoiding the common practice of sex. 

They\marginnote{13.4} give up lying. They speak the truth and stick to the truth. They’re honest and trustworthy, and don’t trick the world with their words. 

They\marginnote{13.5} give up divisive speech. They don’t repeat in one place what they heard in another so as to divide people against each other. Instead, they reconcile those who are divided, supporting unity, delighting in harmony, loving harmony, speaking words that promote harmony. 

They\marginnote{13.6} give up harsh speech. They speak in a way that’s mellow, pleasing to the ear, lovely, going to the heart, polite, likable, and agreeable to the people. 

They\marginnote{13.7} give up talking nonsense. Their words are timely, true, and meaningful, in line with the teaching and training. They say things at the right time which are valuable, reasonable, succinct, and beneficial. 

They\marginnote{13.8} avoid injuring plants and seeds. They eat in one part of the day, abstaining from eating at night and at the wrong time. They avoid dancing, singing, music, and seeing shows. They avoid beautifying and adorning themselves with garlands, perfumes, and makeup. They avoid high and luxurious beds. They avoid receiving gold and money, raw grains, raw meat, women and girls, male and female bondservants, goats and sheep, chickens and pigs, elephants, cows, horses, and mares, and fields and land. They avoid running errands and messages; buying and selling; falsifying weights, metals, or measures; bribery, fraud, cheating, and duplicity; mutilation, murder, abduction, banditry, plunder, and violence. 

They’re\marginnote{14.1} content with robes to look after the body and almsfood to look after the belly. Wherever they go, they set out taking only these things. They’re like a bird: wherever it flies, wings are its only burden. In the same way, a mendicant is content with robes to look after the body and almsfood to look after the belly. Wherever they go, they set out taking only these things. When they have this entire spectrum of noble ethics, they experience a blameless happiness inside themselves. 

When\marginnote{15.1} they see a sight with their eyes, they don’t get caught up in the features and details. If the faculty of sight were left unrestrained, bad unskillful qualities of desire and aversion would become overwhelming. For this reason, they practice restraint, protecting the faculty of sight, and achieving its restraint. When they hear a sound with their ears … When they smell an odor with their nose … When they taste a flavor with their tongue … When they feel a touch with their body … When they know a thought with their mind, they don’t get caught up in the features and details. If the faculty of mind were left unrestrained, bad unskillful qualities of desire and aversion would become overwhelming. For this reason, they practice restraint, protecting the faculty of mind, and achieving its restraint. When they have this noble sense restraint, they experience an unsullied bliss inside themselves. 

They\marginnote{16.1} act with situational awareness when going out and coming back; when looking ahead and aside; when bending and extending the limbs; when bearing the outer robe, bowl and robes; when eating, drinking, chewing, and tasting; when urinating and defecating; when walking, standing, sitting, sleeping, waking, speaking, and keeping silent. 

When\marginnote{17.1} they have this noble spectrum of ethics, this noble contentment, this noble sense restraint, and this noble mindfulness and situational awareness, they frequent a secluded lodging—a wilderness, the root of a tree, a hill, a ravine, a mountain cave, a charnel ground, a forest, the open air, a heap of straw. 

After\marginnote{18.1} the meal, they return from almsround, sit down cross-legged with their body straight, and establish mindfulness right there. Giving up desire for the world, they meditate with a heart rid of desire, cleansing the mind of desire. Giving up ill will and malevolence, they meditate with a mind rid of ill will, full of compassion for all living beings, cleansing the mind of ill will. Giving up dullness and drowsiness, they meditate with a mind rid of dullness and drowsiness, perceiving light, mindful and aware, cleansing the mind of dullness and drowsiness. Giving up restlessness and remorse, they meditate without restlessness, their mind peaceful inside, cleansing the mind of restlessness and remorse. Giving up doubt, they meditate having gone beyond doubt, not undecided about skillful qualities, cleansing the mind of doubt. 

They\marginnote{19.1} give up these five hindrances, corruptions of the heart that weaken wisdom. Then, quite secluded from sensual pleasures, secluded from unskillful qualities, they enter and remain in the first absorption, which has the rapture and bliss born of seclusion, while placing the mind and keeping it connected. This, brahmin, is called ‘a footprint of the Realized One’ and also ‘used by the Realized One’ and also ‘marked by the Realized One’. But a noble disciple wouldn’t yet come to the conclusion, ‘The Blessed One is a fully awakened Buddha. The teaching is well explained. The \textsanskrit{Saṅgha} is practicing well.’ 

Furthermore,\marginnote{20.1} as the placing of the mind and keeping it connected are stilled, a mendicant enters and remains in the second absorption, which has the rapture and bliss born of immersion, with internal clarity and confidence, and unified mind, without placing the mind and keeping it connected. This too is called ‘a footprint of the Realized One’ … 

Furthermore,\marginnote{21.1} with the fading away of rapture, a mendicant enters and remains in the third absorption, where they meditate with equanimity, mindful and aware, personally experiencing the bliss of which the noble ones declare, ‘Equanimous and mindful, one meditates in bliss.’ This too is called ‘a footprint of the Realized One’ … 

Furthermore,\marginnote{22.1} giving up pleasure and pain, and ending former happiness and sadness, a mendicant enters and remains in the fourth absorption, without pleasure or pain, with pure equanimity and mindfulness. This too is called ‘a footprint of the Realized One’ … 

When\marginnote{23.1} their mind has become immersed in \textsanskrit{samādhi} like this—purified, bright, flawless, rid of corruptions, pliable, workable, steady, and imperturbable—they extend it toward recollection of past lives. They recollect many kinds of past lives, that is, one, two, three, four, five, ten, twenty, thirty, forty, fifty, a hundred, a thousand, a hundred thousand rebirths; many eons of the world contracting, many eons of the world expanding, many eons of the world contracting and expanding. … They recollect their many kinds of past lives, with features and details. This too is called ‘a footprint of the Realized One’ … 

When\marginnote{24.1} their mind has become immersed in \textsanskrit{samādhi} like this—purified, bright, flawless, rid of corruptions, pliable, workable, steady, and imperturbable—they extend it toward knowledge of the death and rebirth of sentient beings. With clairvoyance that is purified and surpasses the human, they understand how sentient beings are reborn according to their deeds. This too is called ‘a footprint of the Realized One’ … 

When\marginnote{25.1} their mind has become immersed in \textsanskrit{samādhi} like this—purified, bright, flawless, rid of corruptions, pliable, workable, steady, and imperturbable—they extend it toward knowledge of the ending of defilements. They truly understand: ‘This is suffering’ … ‘This is the origin of suffering’ … ‘This is the cessation of suffering’ … ‘This is the practice that leads to the cessation of suffering.’ They truly understand: ‘These are defilements’ … ‘This is the origin of defilements’ … ‘This is the cessation of defilements’ … ‘This is the practice that leads to the cessation of defilements.’ This, brahmin, is called ‘a footprint of the Realized One’ and also ‘used by the Realized One’ and also ‘marked by the Realized One’. At this point a noble disciple has not yet come to a conclusion, but they are coming to the conclusion, ‘The Blessed One is a fully awakened Buddha. The teaching is well explained. The \textsanskrit{Saṅgha} is practicing well.’ 

Knowing\marginnote{26.1} and seeing like this, their mind is freed from the defilements of sensuality, desire to be reborn, and ignorance. When they’re freed, they know they’re freed. 

They\marginnote{26.3} understand: ‘Rebirth is ended, the spiritual journey has been completed, what had to be done has been done, there is no return to any state of existence.’ This, brahmin, is called ‘a footprint of the Realized One’ and also ‘used by the Realized One’ and also ‘marked by the Realized One’. At this point a noble disciple has come to the conclusion, ‘The Blessed One is a fully awakened Buddha. The teaching is well explained. The \textsanskrit{Saṅgha} is practicing well.’ And it is at this point that the simile of the elephant’s footprint has been completed in detail.” 

When\marginnote{27.1} he had spoken, the brahmin \textsanskrit{Jāṇussoṇi} said to the Buddha, “Excellent, Master Gotama! Excellent! As if he were righting the overturned, or revealing the hidden, or pointing out the path to the lost, or lighting a lamp in the dark so people with good eyes can see what’s there, Master Gotama has made the teaching clear in many ways. I go for refuge to Master Gotama, to the teaching, and to the mendicant \textsanskrit{Saṅgha}. From this day forth, may Master Gotama remember me as a lay follower who has gone for refuge for life.” 

%
\section*{{\suttatitleacronym MN 28}{\suttatitletranslation The Longer Simile of the Elephant’s Footprint }{\suttatitleroot Mahāhatthipadopamasutta}}
\addcontentsline{toc}{section}{\tocacronym{MN 28} \toctranslation{The Longer Simile of the Elephant’s Footprint } \tocroot{Mahāhatthipadopamasutta}}
\markboth{The Longer Simile of the Elephant’s Footprint }{Mahāhatthipadopamasutta}
\extramarks{MN 28}{MN 28}

\scevam{So\marginnote{1.1} I have heard. }At one time the Buddha was staying near \textsanskrit{Sāvatthī} in Jeta’s Grove, \textsanskrit{Anāthapiṇḍika}’s monastery. There \textsanskrit{Sāriputta} addressed the mendicants, “Reverends, mendicants!” 

“Reverend,”\marginnote{1.5} they replied. \textsanskrit{Sāriputta} said this: 

“The\marginnote{2.1} footprints of all creatures that walk can fit inside an elephant’s footprint, so an elephant’s footprint is said to be the biggest of them all. In the same way, all skillful qualities can be included in the four noble truths. What four? The noble truths of suffering, the origin of suffering, the cessation of suffering, and the practice that leads to the cessation of suffering. 

And\marginnote{3.1} what is the noble truth of suffering? Rebirth is suffering; old age is suffering; death is suffering; sorrow, lamentation, pain, sadness, and distress are suffering; not getting what you wish for is suffering. In brief, the five grasping aggregates are suffering. And what are the five grasping aggregates? They are as follows: the grasping aggregates of form, feeling, perception, choices, and consciousness. 

And\marginnote{4.1} what is the grasping aggregate of form? The four primary elements, and form derived from the four primary elements. 

And\marginnote{5.1} what are the four primary elements? The elements of earth, water, fire, and air. 

And\marginnote{6.1} what is the earth element? The earth element may be interior or exterior. And what is the interior earth element? Anything hard, solid, and appropriated that’s internal, pertaining to an individual. This includes: head hair, body hair, nails, teeth, skin, flesh, sinews, bones, bone marrow, kidneys, heart, liver, diaphragm, spleen, lungs, intestines, mesentery, undigested food, feces, or anything else hard, solid, and appropriated that’s internal, pertaining to an individual. This is called the interior earth element. The interior earth element and the exterior earth element are just the earth element. This should be truly seen with right understanding like this: ‘This is not mine, I am not this, this is not my self.’ When you truly see with right understanding, you grow disillusioned with the earth element, detaching the mind from the earth element. 

There\marginnote{7.1} comes a time when the exterior water element flares up. At that time the exterior earth element vanishes. So for all its great age, the earth element will be revealed as impermanent, liable to end, vanish, and perish. What then of this ephemeral body appropriated by craving? Rather than take it to be ‘I’ or ‘mine’ or ‘I am’, they still just consider it to be none of these things. 

If\marginnote{8.1} others abuse, attack, harass, and trouble that mendicant, they understand: ‘This painful feeling born of ear contact has arisen in me. That’s dependent, not independent. Dependent on what? Dependent on contact.’ They see that contact, feeling, perception, choices, and consciousness are impermanent. Based on that element alone, their mind becomes eager, confident, settled, and decided. 

Others\marginnote{9.1} might treat that mendicant with disliking, loathing, and detestation, striking them with fists, stones, sticks, and swords. They understand: ‘This body is such that fists, stones, sticks, and swords strike it. But the Buddha has said in the Simile of the Saw: 

“Even\marginnote{9.6} if low-down bandits were to sever you limb from limb, anyone who had a malevolent thought on that account would not be following my instructions.” My energy shall be roused up and unflagging, my mindfulness established and lucid, my body tranquil and undisturbed, and my mind immersed in \textsanskrit{samādhi}. Gladly now, let fists, stones, sticks, and swords strike this body! For this is how the instructions of the Buddhas are followed.’ 

While\marginnote{10.1} recollecting the Buddha, the teaching, and the \textsanskrit{Saṅgha} in this way, equanimity based on the skillful may not become stabilized in them. In that case they stir up a sense of urgency: ‘It’s my loss, my misfortune, that while recollecting the Buddha, the teaching, and the \textsanskrit{Saṅgha} in this way, equanimity based on the skillful does not become stabilized in me.’ They’re like a daughter-in-law who stirs up a sense of urgency when they see their father-in-law. But if, while recollecting the Buddha, the teaching, and the \textsanskrit{Saṅgha} in this way, equanimity based on the skillful does become stabilized in them, they’re happy with that. At this point, much has been done by that mendicant. 

And\marginnote{11.1} what is the water element? The water element may be interior or exterior. And what is the interior water element? Anything that’s water, watery, and appropriated that’s internal, pertaining to an individual. This includes: bile, phlegm, pus, blood, sweat, fat, tears, grease, saliva, snot, synovial fluid, urine, or anything else that’s water, watery, and appropriated that’s internal, pertaining to an individual. This is called the interior water element. The interior water element and the exterior water element are just the water element. This should be truly seen with right understanding like this: ‘This is not mine, I am not this, this is not my self.’ When you truly see with right understanding, you grow disillusioned with the water element, detaching the mind from the water element. 

There\marginnote{12.1} comes a time when the exterior water element flares up. It sweeps away villages, towns, cities, countries, and regions. There comes a time when the water in the ocean sinks down a hundred leagues, or two, three, four, five, six, up to seven hundred leagues. There comes a time when the water in the ocean stands just seven palm trees deep, or six, five, four, three, two, or even just one palm tree deep. There comes a time when the water in the ocean stands just seven fathoms deep, or six, five, four, three, two, or even just one fathom deep. There comes a time when the water in the ocean stands just half a fathom deep, or waist deep, or knee deep, or even just ankle deep. There comes a time when there isn’t enough water in the ocean even to wet the tip of your finger. So for all its great age, the water element will be revealed as impermanent, liable to end, vanish, and perish. What then of this ephemeral body appropriated by craving? Rather than take it to be ‘I’ or ‘mine’ or ‘I am’, they still just consider it to be none of these things. … If, while recollecting the Buddha, the teaching, and the \textsanskrit{Saṅgha} in this way, equanimity based on the skillful does become stabilized in them, they’re happy with that. At this point, much has been done by that mendicant. 

And\marginnote{16.1} what is the fire element? The fire element may be interior or exterior. And what is the interior fire element? Anything that’s fire, fiery, and appropriated that’s internal, pertaining to an individual. This includes: that which warms, that which ages, that which heats you up when feverish, that which properly digests food and drink, or anything else that’s fire, fiery, and appropriated that’s internal, pertaining to an individual. This is called the interior fire element. The interior fire element and the exterior fire element are just the fire element. This should be truly seen with right understanding like this: ‘This is not mine, I am not this, this is not my self.’ When you truly see with right understanding, you grow disillusioned with the fire element, detaching the mind from the fire element. 

There\marginnote{17.1} comes a time when the exterior fire element flares up. It burns up villages, towns, cities, countries, and regions until it reaches a green field, a roadside, a cliff’s edge, a body of water, or cleared parkland, where it’s extinguished for lack of fuel. There comes a time when they go looking for a fire, taking just a chicken feather or a scrap of sinew as kindling. So for all its great age, the fire element will be revealed as impermanent, liable to end, vanish, and perish. What then of this ephemeral body appropriated by craving? Rather than take it to be ‘I’ or ‘mine’ or ‘I am’, they still just consider it to be none of these things. … 

If,\marginnote{18{-}20.1} while recollecting the Buddha, the teaching, and the \textsanskrit{Saṅgha} in this way, equanimity based on the skillful does become stabilized in them, they’re happy with that. At this point, much has been done by that mendicant. 

And\marginnote{21.1} what is the air element? The air element may be interior or exterior. And what is the interior air element? Anything that’s wind, windy, and appropriated that’s internal, pertaining to an individual. This includes: winds that go up or down, winds in the belly or the bowels, winds that flow through the limbs, in-breaths and out-breaths, or anything else that’s wind, windy, and appropriated that’s internal, pertaining to an individual. This is called the interior air element. The interior air element and the exterior air element are just the air element. This should be truly seen with right understanding like this: ‘This is not mine, I am not this, this is not my self.’ When you truly see with right understanding, you reject the air element, detaching the mind from the air element. 

There\marginnote{22.1} comes a time when the exterior air element flares up. It sweeps away villages, towns, cities, countries, and regions. There comes a time, in the last month of summer, when they look for wind by using a palm-leaf or fan, and even the grasses in the drip-fringe of a thatch roof don’t stir. So for all its great age, the air element will be revealed as impermanent, liable to end, vanish, and perish. What then of this ephemeral body appropriated by craving? Rather than take it to be ‘I’ or ‘mine’ or ‘I am’, they still just consider it to be none of these things. … 

If\marginnote{23.1} others abuse, attack, harass, and trouble that mendicant, they understand: ‘This painful feeling born of ear contact has arisen in me. That’s dependent, not independent. Dependent on what? Dependent on contact. They see that contact, feeling, perception, choices, and consciousness are impermanent. Based on that element alone, their mind becomes eager, confident, settled, and decided. 

Others\marginnote{24.1} might treat that mendicant with disliking, loathing, and detestation, striking them with fists, stones, sticks, and swords. They understand: ‘This body is such that fists, stones, sticks, and swords strike it. But the Buddha has said in the Simile of the Saw: “Even if low-down bandits were to sever you limb from limb, anyone who had a thought of hate on that account would not be following my instructions.” My energy shall be roused up and unflagging, my mindfulness established and lucid, my body tranquil and undisturbed, and my mind immersed in \textsanskrit{samādhi}. Gladly now, let fists, stones, sticks, and swords strike this body! For this is how the instructions of the Buddhas are followed.’ 

While\marginnote{25.1} recollecting the Buddha, the teaching, and the \textsanskrit{Saṅgha} in this way, equanimity based on the skillful may not become stabilized in them. In that case they stir up a sense of urgency: ‘It’s my loss, my misfortune, that while recollecting the Buddha, the teaching, and the \textsanskrit{Saṅgha} in this way, equanimity based on the skillful does not become stabilized in me.’ They’re like a daughter-in-law who stirs up a sense of urgency when they see their father-in-law. But if, while recollecting the Buddha, the teaching, and the \textsanskrit{Saṅgha} in this way, equanimity based on the skillful does become stabilized in them, they’re happy with that. At this point, much has been done by that mendicant. 

When\marginnote{26.1} a space is enclosed by sticks, creepers, grass, and mud it becomes known as a ‘building’. In the same way, when a space is enclosed by bones, sinews, flesh, and skin it becomes known as a ‘form’. 

Reverends,\marginnote{27.1} though the eye is intact internally, so long as exterior sights don’t come into range and there’s no corresponding engagement, there’s no manifestation of the corresponding type of consciousness. Though the eye is intact internally and exterior sights come into range, so long as there’s no corresponding engagement, there’s no manifestation of the corresponding type of consciousness. But when the eye is intact internally and exterior sights come into range and there is corresponding engagement, there is the manifestation of the corresponding type of consciousness. 

The\marginnote{28.1} form produced in this way is included in the grasping aggregate of form. The feeling, perception, choices, and consciousness produced in this way are each included in the corresponding grasping aggregate. 

They\marginnote{28.2} understand: ‘So this is how there comes to be inclusion, gathering together, and joining together into these five grasping aggregates. But the Buddha has said: “One who sees dependent origination sees the teaching. One who sees the teaching sees dependent origination.” And these five grasping aggregates are indeed dependently originated. The desire, adherence, attraction, and attachment for these five grasping aggregates is the origin of suffering. Giving up and getting rid of desire and greed for these five grasping aggregates is the cessation of suffering.’ At this point, much has been done by that mendicant. 

Though\marginnote{29{-}30.1} the ear … nose … tongue … body … mind is intact internally, so long as exterior thoughts don’t come into range and there’s no corresponding engagement, there’s no manifestation of the corresponding type of consciousness. 

Though\marginnote{38.1} the mind is intact internally and exterior thoughts come into range, so long as there’s no corresponding engagement, there’s no manifestation of the corresponding type of consciousness. But when the mind is intact internally and exterior thoughts come into range and there is corresponding engagement, there is the manifestation of the corresponding type of consciousness. 

The\marginnote{38.3} form produced in this way is included in the grasping aggregate of form. The feeling, perception, choices, and consciousness produced in this way are each included in the corresponding grasping aggregate. They understand: ‘So this is how there comes to be inclusion, gathering together, and joining together into these five grasping aggregates. 

But\marginnote{38.6} the Buddha has also said: “One who sees dependent origination sees the teaching. One who sees the teaching sees dependent origination.” And these five grasping aggregates are indeed dependently originated. The desire, adherence, attraction, and attachment for these five grasping aggregates is the origin of suffering. Giving up and getting rid of desire and greed for these five grasping aggregates is the cessation of suffering.’ At this point, much has been done by that mendicant.” 

That’s\marginnote{38.13} what Venerable \textsanskrit{Sāriputta} said. Satisfied, the mendicants were happy with what \textsanskrit{Sāriputta} said. 

%
\section*{{\suttatitleacronym MN 29}{\suttatitletranslation The Longer Simile of the Heartwood }{\suttatitleroot Mahāsāropamasutta}}
\addcontentsline{toc}{section}{\tocacronym{MN 29} \toctranslation{The Longer Simile of the Heartwood } \tocroot{Mahāsāropamasutta}}
\markboth{The Longer Simile of the Heartwood }{Mahāsāropamasutta}
\extramarks{MN 29}{MN 29}

\scevam{So\marginnote{1.1} I have heard. }At one time the Buddha was staying near \textsanskrit{Rājagaha}, on the Vulture’s Peak Mountain, not long after Devadatta had left. There the Buddha spoke to the mendicants about Devadatta: 

“Mendicants,\marginnote{2.1} take the case of a gentleman who has gone forth from the lay life to homelessness, thinking, ‘I’m swamped by rebirth, old age, and death; by sorrow, lamentation, pain, sadness, and distress. I’m swamped by suffering, mired in suffering. Hopefully I can find an end to this entire mass of suffering.’ When they’ve gone forth they generate possessions, honor, and popularity. They’re happy with that, and they’ve got all they wished for. And they glorify themselves and put others down because of that: ‘I’m the one with possessions, honor, and popularity. These other mendicants are obscure and insignificant.’ And so they become indulgent and fall into negligence regarding those possessions, honor, and popularity. And being negligent they live in suffering. 

Suppose\marginnote{2.9} there was a person in need of heartwood. And while wandering in search of heartwood he’d come across a large tree standing with heartwood. But, passing over the heartwood, softwood, bark, and shoots, he’d cut off the branches and leaves and depart imagining they were heartwood. If a person with good eyesight saw him they’d say, ‘This gentleman doesn’t know what heartwood, softwood, bark, shoots, or branches and leaves are. That’s why he passed them over, cut off the branches and leaves, and departed imagining they were heartwood. Whatever he needs to make from heartwood, he won’t succeed.’ … 

This\marginnote{2.14} is called a mendicant who has grabbed the branches and leaves of the spiritual life and stopped short with that. 

Next,\marginnote{3.1} take a gentleman who has gone forth from the lay life to homelessness … When they’ve gone forth they generate possessions, honor, and popularity. They’re not happy with that, and haven’t got all they wished for. They don’t glorify themselves and put others down on account of that. Nor do they become indulgent and fall into negligence regarding those possessions, honor, and popularity. Being diligent, they become accomplished in ethics. They’re happy with that, and they’ve got all they wished for. And they glorify themselves and put others down on account of that: ‘I’m the one who is ethical, of good character. These other mendicants are unethical, of bad character.’ And so they become indulgent and fall into negligence regarding their accomplishment in ethics. And being negligent they live in suffering. 

Suppose\marginnote{3.13} there was a person in need of heartwood. And while wandering in search of heartwood he’d come across a large tree standing with heartwood. But, passing over the heartwood, softwood, and bark, he’d cut off the shoots and depart imagining they were heartwood. If a person with good eyesight saw him they’d say, ‘This gentleman doesn’t know what heartwood, softwood, bark, shoots, or branches and leaves are. That’s why he passed them over, cut off the shoots, and departed imagining they were heartwood. Whatever he needs to make from heartwood, he won’t succeed.’ … 

This\marginnote{4.1} is called a mendicant who has grabbed the shoots of the spiritual life and stopped short with that. 

Next,\marginnote{4.15} take a gentleman who has gone forth from the lay life to homelessness … When they’ve gone forth they generate possessions, honor, and popularity. … Being diligent, they achieve immersion. They’re happy with that, and they’ve got all they wished for. And they glorify themselves and put others down on account of that: ‘I’m the one with immersion and unified mind. These other mendicants lack immersion, they have straying minds.’ And so they become indulgent and fall into negligence regarding that accomplishment in immersion. And being negligent they live in suffering. 

Suppose\marginnote{4.30} there was a person in need of heartwood. And while wandering in search of heartwood he’d come across a large tree standing with heartwood. But, passing over the heartwood and softwood, he’d cut off the bark and depart imagining it was heartwood. If a person with good eyesight saw him they’d say: ‘This gentleman doesn’t know what heartwood, softwood, bark, shoots, or branches and leaves are. That’s why he passed them over, cut off the bark, and departed imagining it was heartwood. Whatever he needs to make from heartwood, he won’t succeed.’ … 

This\marginnote{4.34} is called a mendicant who has grabbed the bark of the spiritual life and stopped short with that. 

Next,\marginnote{5.1} take a gentleman who has gone forth from the lay life to homelessness … When they’ve gone forth they generate possessions, honor, and popularity. … Being diligent, they achieve knowledge and vision. They’re happy with that, and they’ve got all they wished for. And they glorify themselves and put others down on account of that, ‘I’m the one who meditates knowing and seeing. These other mendicants meditate without knowing and seeing.’ And so they become indulgent and fall into negligence regarding that knowledge and vision. And being negligent they live in suffering. 

Suppose\marginnote{5.20} there was a person in need of heartwood. And while wandering in search of heartwood he’d come across a large tree standing with heartwood. But, passing over the heartwood, he’d cut out the softwood and depart imagining it was heartwood. If a person with good eyesight saw him they’d say, ‘This gentleman doesn’t know what heartwood, softwood, bark, shoots, or branches and leaves are. That’s why he passed them over, cut out the softwood, and departed imagining it was heartwood. Whatever he needs to make from heartwood, he won’t succeed.’ … 

This\marginnote{5.25} is called a mendicant who has grabbed the softwood of the spiritual life and stopped short with that. 

Next,\marginnote{6.1} take a gentleman who has gone forth from the lay life to homelessness, thinking, ‘I’m swamped by rebirth, old age, and death; by sorrow, lamentation, pain, sadness, and distress. I’m swamped by suffering, mired in suffering. Hopefully I can find an end to this entire mass of suffering.’ When they’ve gone forth they generate possessions, honor, and popularity. They’re not happy with that, and haven’t got all they wished for. They don’t glorify themselves and put others down on account of that. Nor do they become indulgent and fall into negligence regarding those possessions, honor, and popularity. Being diligent, they become accomplished in ethics. They’re happy with that, but they haven’t got all they wished for. They don’t glorify themselves and put others down on account of that. Nor do they become indulgent and fall into negligence regarding that accomplishment in ethics. Being diligent, they achieve immersion. They’re happy with that, but they haven’t got all they wished for. They don’t glorify themselves and put others down on account of that. Nor do they become indulgent and fall into negligence regarding that accomplishment in immersion. Being diligent, they achieve knowledge and vision. They’re happy with that, but they haven’t got all they wished for. They don’t glorify themselves and put others down on account of that. Nor do they become indulgent and fall into negligence regarding that knowledge and vision. Being diligent, they achieve permanent liberation. And it’s impossible for that mendicant to fall away from that irreversible liberation. 

Suppose\marginnote{6.18} there was a person in need of heartwood. And while wandering in search of heartwood he’d come across a large tree standing with heartwood. He’d cut out just the heartwood and depart knowing it was heartwood. If a person with good eyesight saw him they’d say, ‘This gentleman knows what heartwood, softwood, bark, shoots, and branches and leaves are. That’s why he cut out just the heartwood and departed knowing it was heartwood. Whatever he needs to make from heartwood, he will succeed.’ … 

It’s\marginnote{6.23} impossible for that mendicant to fall away from that irreversible liberation. 

And\marginnote{7.1} so, mendicants, this spiritual life is not lived for the sake of possessions, honor, and popularity, or for accomplishment in ethics, or for accomplishment in immersion, or for knowledge and vision. Rather, the goal, heartwood, and final end of the spiritual life is the unshakable freedom of heart.” 

That\marginnote{7.4} is what the Buddha said. Satisfied, the mendicants were happy with what the Buddha said. 

%
\section*{{\suttatitleacronym MN 30}{\suttatitletranslation The Shorter Simile of the Heartwood }{\suttatitleroot Cūḷasāropamasutta}}
\addcontentsline{toc}{section}{\tocacronym{MN 30} \toctranslation{The Shorter Simile of the Heartwood } \tocroot{Cūḷasāropamasutta}}
\markboth{The Shorter Simile of the Heartwood }{Cūḷasāropamasutta}
\extramarks{MN 30}{MN 30}

\scevam{So\marginnote{1.1} I have heard. }At one time the Buddha was staying near \textsanskrit{Sāvatthī} in Jeta’s Grove, \textsanskrit{Anāthapiṇḍika}’s monastery. 

Then\marginnote{2.1} the brahmin \textsanskrit{Piṅgalakoccha} went up to the Buddha, and exchanged greetings with him. When the greetings and polite conversation were over, he sat down to one side and said to the Buddha: 

“Master\marginnote{2.3} Gotama, there are those ascetics and brahmins who lead an order and a community, and teach a community. They’re well-known and famous religious founders, regarded as holy by many people. Namely: \textsanskrit{Pūraṇa} Kassapa, Makkhali \textsanskrit{Gosāla}, \textsanskrit{Nigaṇṭha} \textsanskrit{Nāṭaputta}, \textsanskrit{Sañjaya} \textsanskrit{Belaṭṭhiputta}, Pakudha \textsanskrit{Kaccāyana}, and Ajita Kesakambala. According to their own claims, did all of them have direct knowledge, or none of them, or only some?” 

“Enough,\marginnote{2.6} brahmin, let this be: ‘According to their own claims, did all of them have direct knowledge, or none of them, or only some?’ I will teach you the Dhamma. Listen and pay close attention, I will speak.” 

“Yes\marginnote{2.10} sir,” \textsanskrit{Piṅgalakoccha} replied. The Buddha said this: 

“Suppose\marginnote{3.1} there was a person in need of heartwood. And while wandering in search of heartwood he’d come across a large tree standing with heartwood. But, passing over the heartwood, softwood, bark, and shoots, he’d cut off the branches and leaves and depart imagining they were heartwood. If a person with good eyesight saw him they’d say: ‘This gentleman doesn’t know what heartwood, softwood, bark, shoots, or branches and leaves are. That’s why he passed them over, cut off the branches and leaves, and departed imagining they were heartwood. Whatever he needs to make from heartwood, he won’t succeed.’ 

Suppose\marginnote{4.1} there was another person in need of heartwood … he’d cut off the shoots and depart imagining they were heartwood … 

Suppose\marginnote{5.1} there was another person in need of heartwood … he’d cut off the bark and depart imagining it was heartwood … 

Suppose\marginnote{6.1} there was another person in need of heartwood … he’d cut out the softwood and depart imagining it was heartwood … 

Suppose\marginnote{7.1} there was another person in need of heartwood. And while wandering in search of heartwood he’d come across a large tree standing with heartwood. He’d cut out just the heartwood and depart knowing it was heartwood. If a person with good eyesight saw him they’d say: ‘This gentleman knows what heartwood, softwood, bark, shoots, or branches and leaves are. That’s why he cut out just the heartwood and departed knowing it was heartwood. Whatever he needs to make from heartwood, he will succeed.’ 

In\marginnote{8.1} the same way, take a certain person who goes forth from the lay life to homelessness, thinking: ‘I’m swamped by rebirth, old age, and death; by sorrow, lamentation, pain, sadness, and distress. I’m swamped by suffering, mired in suffering. Hopefully I can find an end to this entire mass of suffering.’ When they’ve gone forth they generate possessions, honor, and popularity. They’re happy with that, and they’ve got all they wished for. And they glorify themselves and put others down on account of that: ‘I’m the one with possessions, honor, and popularity. These other mendicants are obscure and insignificant.’ They become lazy and slack regarding their possessions, honor, and popularity, not generating enthusiasm or trying to realize those things that are better and finer. … They’re like the person who mistakes branches and leaves for heartwood, I say. 

Next,\marginnote{9.1} take a gentleman who has gone forth from the lay life to homelessness … They become lazy and slack regarding their accomplishment in ethics, not generating enthusiasm or trying to realize those things that are better and finer. … They’re like the person who mistakes shoots for heartwood, I say. 

Next,\marginnote{10.1} take a gentleman who has gone forth from the lay life to homelessness … They become lazy and slack regarding their accomplishment in immersion, not generating enthusiasm or trying to realize those things that are better and finer. … They’re like the person who mistakes bark for heartwood, I say. 

Next,\marginnote{11.1} take a gentleman who has gone forth from the lay life to homelessness … They become lazy and slack regarding their knowledge and vision, not generating enthusiasm or trying to realize those things that are better and finer. … They’re like the person who mistakes softwood for heartwood, I say. 

Next,\marginnote{12.1} take a gentleman who has gone forth from the lay life to homelessness, thinking: ‘I’m swamped by rebirth, old age, and death; by sorrow, lamentation, pain, sadness, and distress. I’m swamped by suffering, mired in suffering. Hopefully I can find an end to this entire mass of suffering.’ When they’ve gone forth they generate possessions, honor, and popularity. They’re not happy with that, and haven’t got all they wished for. They don’t glorify themselves and put others down on account of that. They don’t become lazy and slack regarding their possessions, honor, and popularity, but generate enthusiasm and try to realize those things that are better and finer. They become accomplished in ethics. They’re happy with that, but they haven’t got all they wished for. They don’t glorify themselves and put others down on account of that. They don’t become lazy and slack regarding their accomplishment in ethics, but generate enthusiasm and try to realize those things that are better and finer. They become accomplished in immersion. They’re happy with that, but they haven’t got all they wished for. They don’t glorify themselves and put others down on account of that. They don’t become lazy and slack regarding their accomplishment in immersion, but generate enthusiasm and try to realize those things that are better and finer. They achieve knowledge and vision. They’re happy with that, but they haven’t got all they wished for. They don’t glorify themselves and put others down on account of that. They don’t become lazy and slack regarding their knowledge and vision, but generate enthusiasm and try to realize those things that are better and finer. 

And\marginnote{12.20} what are those things that are better and finer than knowledge and vision? 

Take\marginnote{13.1} a mendicant who, quite secluded from sensual pleasures, secluded from unskillful qualities, enters and remains in the first absorption, which has the rapture and bliss born of seclusion, while placing the mind and keeping it connected. This is something better and finer than knowledge and vision. 

Furthermore,\marginnote{14.1} as the placing of the mind and keeping it connected are stilled, a mendicant enters and remains in the second absorption, which has the rapture and bliss born of immersion, with internal clarity and confidence, and unified mind, without placing the mind and keeping it connected. This too is something better and finer than knowledge and vision. 

Furthermore,\marginnote{15.1} with the fading away of rapture, a mendicant enters and remains in the third absorption, where they meditate with equanimity, mindful and aware, personally experiencing the bliss of which the noble ones declare, ‘Equanimous and mindful, one meditates in bliss.’ This too is something better and finer than knowledge and vision. 

Furthermore,\marginnote{16.1} giving up pleasure and pain, and ending former happiness and sadness, a mendicant enters and remains in the fourth absorption, without pleasure or pain, with pure equanimity and mindfulness. This too is something better and finer than knowledge and vision. 

Furthermore,\marginnote{17.1} a mendicant, going totally beyond perceptions of form, with the ending of perceptions of impingement, not focusing on perceptions of diversity, aware that ‘space is infinite’, enters and remains in the dimension of infinite space. This too is something better and finer than knowledge and vision. 

Furthermore,\marginnote{18.1} a mendicant, going totally beyond the dimension of infinite space, aware that ‘consciousness is infinite’, enters and remains in the dimension of infinite consciousness. This too is something better and finer than knowledge and vision. 

Furthermore,\marginnote{19.1} a mendicant, going totally beyond the dimension of infinite consciousness, aware that ‘there is nothing at all’, enters and remains in the dimension of nothingness. This too is something better and finer than knowledge and vision. 

Furthermore,\marginnote{20.1} take a mendicant who, going totally beyond the dimension of nothingness, enters and remains in the dimension of neither perception nor non-perception. This too is something better and finer than knowledge and vision. 

Furthermore,\marginnote{21.1} take a mendicant who, going totally beyond the dimension of neither perception nor non-perception, enters and remains in the cessation of perception and feeling. And, having seen with wisdom, their defilements come to an end. This too is something better and finer than knowledge and vision. These are the things that are better and finer than knowledge and vision. 

Suppose\marginnote{22.1} there was a person in need of heartwood. And while wandering in search of heartwood he’d come across a large tree standing with heartwood. He’d cut out just the heartwood and depart knowing it was heartwood. Whatever he needs to make from heartwood, he will succeed. That’s what this person is like, I say. 

And\marginnote{23.1} so, brahmin, this spiritual life is not lived for the sake of possessions, honor, and popularity, or for accomplishment in ethics, or for accomplishment in immersion, or for knowledge and vision. Rather, the goal, heartwood, and final end of the spiritual life is the unshakable freedom of heart.” 

When\marginnote{24.1} he had spoken, the brahmin \textsanskrit{Piṅgalakoccha} said to the Buddha, “Excellent, Master Gotama! Excellent! … From this day forth, may Master Gotama remember me as a lay follower who has gone for refuge for life.” 

%
\addtocontents{toc}{\let\protect\contentsline\protect\nopagecontentsline}
\chapter*{The Greater Chapter on Pairs }
\addcontentsline{toc}{chapter}{\tocchapterline{The Greater Chapter on Pairs }}
\addtocontents{toc}{\let\protect\contentsline\protect\oldcontentsline}

%
\section*{{\suttatitleacronym MN 31}{\suttatitletranslation The Shorter Discourse at Gosiṅga }{\suttatitleroot Cūḷagosiṅgasutta}}
\addcontentsline{toc}{section}{\tocacronym{MN 31} \toctranslation{The Shorter Discourse at Gosiṅga } \tocroot{Cūḷagosiṅgasutta}}
\markboth{The Shorter Discourse at Gosiṅga }{Cūḷagosiṅgasutta}
\extramarks{MN 31}{MN 31}

\scevam{So\marginnote{1.1} I have heard. }At one time the Buddha was staying at \textsanskrit{Nādika} in the brick house. 

Now\marginnote{2.1} at that time the venerables Anuruddha, Nandiya, and Kimbila were staying in the sal forest park at \textsanskrit{Gosiṅga}. 

Then\marginnote{3.1} in the late afternoon, the Buddha came out of retreat and went to that park. The park keeper saw the Buddha coming off in the distance and said to him, “Don’t come into this park, ascetic. There are three gentlemen who love themselves staying here. Don’t disturb them.” 

Anuruddha\marginnote{4.1} heard the park keeper conversing with the Buddha, and said to him, “Don’t keep the Buddha out, good park keeper! Our Teacher, the Blessed One, has arrived.” Then Anuruddha went to Nandiya and Kimbila, and said to them, “Come forth, venerables, come forth! Our Teacher, the Blessed One, has arrived!” 

Then\marginnote{5.1} Anuruddha, Nandiya, and Kimbila came out to greet the Buddha. One received his bowl and robe, one spread out a seat, and one set out water for washing his feet. He sat on the seat spread out and washed his feet. Those venerables bowed and sat down to one side. 

The\marginnote{5.6} Buddha said to Anuruddha, “I hope you’re keeping well, Anuruddha and friends; I hope you’re alright. And I hope you’re having no trouble getting almsfood.” 

“We’re\marginnote{5.8} alright, Blessed One, we’re getting by. And we have no trouble getting almsfood.” 

“I\marginnote{6.1} hope you’re living in harmony, appreciating each other, without quarreling, blending like milk and water, and regarding each other with kindly eyes?” 

“Indeed,\marginnote{6.2} sir, we live in harmony like this.” 

“But\marginnote{6.3} how do you live this way?” 

“In\marginnote{7.1} this case, sir, I think, ‘I’m fortunate, so very fortunate, to live together with spiritual companions such as these.’ I consistently treat these venerables with kindness by way of body, speech, and mind, both in public and in private. I think, ‘Why don’t I set aside my own ideas and just go along with these venerables’ ideas?’ And that’s what I do. Though we’re different in body, sir, we’re one in mind, it seems to me.” 

And\marginnote{7.11} the venerables Nandiya and Kimbila spoke likewise, and they added: “That’s how we live in harmony, appreciating each other, without quarreling, blending like milk and water, and regarding each other with kindly eyes.” 

“Good,\marginnote{8.1} good, Anuruddha and friends! But I hope you’re living diligently, keen, and resolute?” 

“Indeed,\marginnote{9.1} sir, we live diligently.” 

“But\marginnote{9.2} how do you live this way?” 

“In\marginnote{9.3} this case, sir, whoever returns first from almsround prepares the seats, and puts out the drinking water and the rubbish bin. If there’s anything left over, whoever returns last eats it if they like. Otherwise they throw it out where there is little that grows, or drop it into water that has no living creatures. Then they put away the seats, drinking water, and rubbish bin, and sweep the refectory. If someone sees that the pot of water for washing, drinking, or the toilet is empty they set it up. If he can’t do it, he summons another with a wave of the hand, and they set it up by lifting it with their hands. But we don’t break into speech for that reason. And every five days we sit together for the whole night and discuss the teachings. That’s how we live diligently, keen, and resolute.” 

“Good,\marginnote{10.1} good, Anuruddha and friends! But as you live diligently like this, have you achieved any superhuman distinction in knowledge and vision worthy of the noble ones, a meditation at ease?” 

“How\marginnote{10.3} could we not, sir? Whenever we want, quite secluded from sensual pleasures, secluded from unskillful qualities, we enter and remain in the first absorption, which has the rapture and bliss born of seclusion, while placing the mind and keeping it connected. This is a superhuman distinction in knowledge and vision worthy of the noble ones, a meditation at ease, that we have achieved while living diligent, keen, and resolute.” 

“Good,\marginnote{11{-}13.1} good! But have you achieved any other superhuman distinction for going beyond and stilling that meditation?” 

“How\marginnote{11{-}13.3} could we not, sir? Whenever we want, as the placing of the mind and keeping it connected are stilled, we enter and remain in the second absorption, which has the rapture and bliss born of immersion, with internal clarity and confidence, and unified mind, without placing the mind and keeping it connected. This is another superhuman distinction that we have achieved for going beyond and stilling that meditation.” 

“Good,\marginnote{14.1} good! But have you achieved any other superhuman distinction for going beyond and stilling that meditation?” 

“How\marginnote{14.3} could we not, sir? Whenever we want, with the fading away of rapture, we enter and remain in the third absorption, where we meditate with equanimity, mindful and aware, personally experiencing the bliss of which the noble ones declare, ‘Equanimous and mindful, one meditates in bliss.’ This is another superhuman distinction that we have achieved for going beyond and stilling that meditation.” 

“Good,\marginnote{15.1} good! But have you achieved any other superhuman distinction for going beyond and stilling that meditation?” 

“How\marginnote{15.3} could we not, sir? Whenever we want, with the giving up of pleasure and pain, and the ending of former happiness and sadness, we enter and remain in the fourth absorption, without pleasure or pain, with pure equanimity and mindfulness. This is another superhuman distinction that we have achieved for going beyond and stilling that meditation.” 

“Good,\marginnote{16.1} good! But have you achieved any other superhuman distinction for going beyond and stilling that meditation?” 

“How\marginnote{16.3} could we not, sir? Whenever we want, going totally beyond perceptions of form, with the ending of perceptions of impingement, not focusing on perceptions of diversity, aware that ‘space is infinite’, we enter and remain in the dimension of infinite space. This is another superhuman distinction that we have achieved for going beyond and stilling that meditation.” 

“Good,\marginnote{17.1} good! But have you achieved any other superhuman distinction for going beyond and stilling that meditation?” 

“How\marginnote{17.3} could we not, sir? Whenever we want, going totally beyond the dimension of infinite space, aware that ‘consciousness is infinite’, we enter and remain in the dimension of infinite consciousness. … going totally beyond the dimension of infinite consciousness, aware that ‘there is nothing at all’, we enter and remain in the dimension of nothingness. … going totally beyond the dimension of nothingness, we enter and remain in the dimension of neither perception nor non-perception. This is another superhuman distinction that we have achieved for going beyond and stilling that meditation.” 

“Good,\marginnote{18.1} good! But have you achieved any other superhuman distinction for going beyond and stilling that meditation?” 

“How\marginnote{18.3} could we not, sir? Whenever we want, going totally beyond the dimension of neither perception nor non-perception, we enter and remain in the cessation of perception and feeling. And, having seen with wisdom, our defilements have come to an end. This is another superhuman distinction in knowledge and vision worthy of the noble ones, a meditation at ease, that we have achieved for going beyond and stilling that meditation. And we don’t see any better or finer way of meditating at ease than this.” 

“Good,\marginnote{18.7} good! There is no better or finer way of meditating at ease than this.” 

Then\marginnote{19.1} the Buddha educated, encouraged, fired up, and inspired the venerables Anuruddha, Nandiya, and Kimbila with a Dhamma talk, after which he got up from his seat and left. 

The\marginnote{20.1} venerables then accompanied the Buddha for a little way before turning back. Nandiya and Kimbila said to Anuruddha, “Did we ever tell you that we had gained such and such meditations and attainments, up to the ending of defilements, as you revealed to the Buddha?” 

“The\marginnote{20.4} venerables did not tell me that they had gained such meditations and attainments. But I discovered it by comprehending your minds, and deities also told me. I answered when the Buddha directly asked about it.” 

Then\marginnote{21.1} the native spirit \textsanskrit{Dīgha} Parajana went up to the Buddha, bowed, stood to one side, and said to him, “The Vajjis are lucky! The Vajjian people are so very lucky that the Realized One, the perfected one, the fully awakened Buddha stays there, as well as these three gentlemen, the venerables Anuruddha, Nandiya, and Kimbila.” 

Hearing\marginnote{21.5} the cry of \textsanskrit{Dīgha} Parajana, the Earth Gods raised the cry … 

Hearing\marginnote{21.9} the cry of the Earth Gods, the Gods of the Four Great Kings … the Gods of the Thirty-Three … the Gods of Yama … the Joyful Gods … the Gods Who Love to Create … the Gods Who Control the Creations of Others … the Gods of \textsanskrit{Brahmā}’s Host raised the cry, “The Vajjis are lucky! The Vajjian people are so very lucky that the Realized One, the perfected one, the fully awakened Buddha stays there, as well as these three gentlemen, the venerables Anuruddha, Nandiya, and Kimbila.” 

And\marginnote{21.19} so at that moment, in that instant, those venerables were known as far as the \textsanskrit{Brahmā} realm. 

“That’s\marginnote{22.1} so true, \textsanskrit{Dīgha}! That’s so true! If the family from which those three gentlemen went forth from the lay life to homelessness were to recollect those venerables with confident heart, that would be for that family’s lasting welfare and happiness. If the family circle … village … town … city … country … all the aristocrats … all the brahmins … all the merchants … all the workers were to recollect those venerables with confident heart, that would be for all those workers’ lasting welfare and happiness. 

If\marginnote{22.12} the whole world—with its gods, \textsanskrit{Māras} and \textsanskrit{Brahmās}, this population with its ascetics and brahmins, gods and humans—were to recollect those venerables with confident heart, that would be for the whole world’s lasting welfare and happiness. 

See,\marginnote{22.13} \textsanskrit{Dīgha}, how those three gentlemen are practicing for the welfare and happiness of the people, out of compassion for the world, for the benefit, welfare, and happiness of gods and humans!” 

That\marginnote{22.14} is what the Buddha said. Satisfied, the native spirit \textsanskrit{Dīgha} Parajana was happy with what the Buddha said. 

%
\section*{{\suttatitleacronym MN 32}{\suttatitletranslation The Longer Discourse at Gosiṅga }{\suttatitleroot Mahāgosiṅgasutta}}
\addcontentsline{toc}{section}{\tocacronym{MN 32} \toctranslation{The Longer Discourse at Gosiṅga } \tocroot{Mahāgosiṅgasutta}}
\markboth{The Longer Discourse at Gosiṅga }{Mahāgosiṅgasutta}
\extramarks{MN 32}{MN 32}

\scevam{So\marginnote{1.1} I have heard. }At one time the Buddha was staying in the sal forest park at \textsanskrit{Gosiṅga}, together with several well-known senior disciples, such as the venerables \textsanskrit{Sāriputta}, \textsanskrit{Mahāmoggallāna}, \textsanskrit{Mahākassapa}, Anuruddha, Revata, Ānanda, and others. 

Then\marginnote{2.1} in the late afternoon, Venerable \textsanskrit{Mahāmoggallāna} came out of retreat, went to Venerable \textsanskrit{Mahākassapa}, and said, “Come, Reverend Kassapa, let’s go to Venerable \textsanskrit{Sāriputta} to hear the teaching.” 

“Yes,\marginnote{2.3} reverend,” \textsanskrit{Mahākassapa} replied. Then, together with Venerable Anuruddha, they went to \textsanskrit{Sāriputta} to hear the teaching. 

Seeing\marginnote{3.1} them, Venerable Ānanda went to Venerable Revata, told him what was happening, and invited him also. 

\textsanskrit{Sāriputta}\marginnote{4.1} saw them coming off in the distance and said to Ānanda, “Come, Venerable Ānanda. Welcome to Ānanda, the Buddha’s attendant, who is so close to the Buddha. Ānanda, the sal forest park at \textsanskrit{Gosiṅga} is lovely, the night is bright, the sal trees are in full blossom, and divine scents seem to float on the air. What kind of mendicant would beautify this park?” 

“Reverend\marginnote{4.7} \textsanskrit{Sāriputta}, it’s a mendicant who is very learned, remembering and keeping what they’ve learned. These teachings are good in the beginning, good in the middle, and good in the end, meaningful and well-phrased, describing a spiritual practice that’s entirely full and pure. They are very learned in such teachings, remembering them, reinforcing them by recitation, mentally scrutinizing them, and comprehending them theoretically. And they teach the four assemblies in order to uproot the underlying tendencies with well-rounded and systematic words and phrases. That’s the kind of mendicant who would beautify this park.” 

When\marginnote{5.1} he had spoken, \textsanskrit{Sāriputta} said to Revata, “Reverend Revata, Ānanda has answered by speaking from his heart. And now we ask you the same question.” 

“Reverend\marginnote{5.6} \textsanskrit{Sāriputta}, it’s a mendicant who enjoys retreat and loves retreat. They’re committed to inner serenity of the heart, they don’t neglect absorption, they’re endowed with discernment, and they frequent empty huts. That’s the kind of mendicant who would beautify this park.” 

When\marginnote{6.1} he had spoken, \textsanskrit{Sāriputta} said to Anuruddha, “Reverend Anuruddha, Revata has answered by speaking from his heart. And now we ask you the same question.” 

“Reverend\marginnote{6.6} \textsanskrit{Sāriputta}, it’s a mendicant who surveys the entire galaxy with clairvoyance that is purified and surpasses the human, just as a person with good sight could survey a thousand wheel rims from the upper floor of a stilt longhouse. That’s the kind of mendicant who would beautify this park.” 

When\marginnote{7.1} he had spoken, \textsanskrit{Sāriputta} said to \textsanskrit{Mahākassapa}, “Reverend Kassapa, Anuruddha has answered by speaking from his heart. And now we ask you the same question.” 

“Reverend\marginnote{7.6} \textsanskrit{Sāriputta}, it’s a mendicant who lives in the wilderness, eats only almsfood, wears rag robes, and owns just three robes; and they praise these things. They are of few wishes, content, secluded, aloof, and energetic; and they praise these things. They are accomplished in ethics, immersion, wisdom, freedom, and the knowledge and vision of freedom; and they praise these things. That’s the kind of mendicant who would beautify this park.” 

When\marginnote{8.1} he had spoken, \textsanskrit{Sāriputta} said to \textsanskrit{Mahāmoggallāna}, “Reverend \textsanskrit{Moggallāna}, \textsanskrit{Mahākassapa} has answered by speaking from his heart. And now we ask you the same question.” 

“Reverend\marginnote{8.6} \textsanskrit{Sāriputta}, it’s when two mendicants engage in discussion about the teaching. They question each other and answer each other’s questions without faltering, and their discussion on the teaching flows on. That’s the kind of mendicant who would beautify this park.” 

Then\marginnote{9.1} \textsanskrit{Mahāmoggallāna} said to \textsanskrit{Sāriputta}, “Each of us has spoken from our heart. And now we ask you: \textsanskrit{Sāriputta}, the sal forest park at \textsanskrit{Gosiṅga} is lovely, the night is bright, the sal trees are in full blossom, and divine scents seem to float on the air. What kind of mendicant would beautify this park?” 

“Reverend\marginnote{9.6} \textsanskrit{Moggallāna}, it’s when a mendicant masters their mind and is not mastered by it. In the morning, they abide in whatever meditation or attainment they want. At midday, and in the evening, they abide in whatever meditation or attainment they want. Suppose that a ruler or their minister had a chest full of garments of different colors. In the morning, they’d don whatever pair of garments they wanted. At midday, and in the evening, they’d don whatever pair of garments they wanted. 

In\marginnote{9.14} the same way, a mendicant masters their mind and is not mastered by it. In the morning, they abide in whatever meditation or attainment they want. At midday, and in the evening, they abide in whatever meditation or attainment they want. That’s the kind of mendicant who would beautify this park.” 

Then\marginnote{10.1} \textsanskrit{Sāriputta} said to those venerables, “Each of us has spoken from the heart. Come, reverends, let’s go to the Buddha, and inform him about this. As he answers, so we’ll remember it.” 

“Yes,\marginnote{10.5} reverend,” they replied. Then those venerables went to the Buddha, bowed, and sat down to one side. Venerable \textsanskrit{Sāriputta} told the Buddha of how the mendicants had come to see him, and how he had asked Ānanda: “‘Ānanda, the sal forest park at \textsanskrit{Gosiṅga} is lovely, the night is bright, the sal trees are in full blossom, and divine scents seem to float on the air. What kind of mendicant would beautify this park?’ When I had spoken, Ānanda said to me: ‘Reverend \textsanskrit{Sāriputta}, it’s a mendicant who is very learned … That’s the kind of mendicant who would beautify this park.’” 

“Good,\marginnote{11.12} good, \textsanskrit{Sāriputta}! Ānanda answered in the right way for him. For Ānanda is very learned …” 

“Next\marginnote{12.1} I asked Revata the same question. He said: ‘It’s a mendicant who enjoys retreat … That’s the kind of mendicant who would beautify this park.’” 

“Good,\marginnote{12.9} good, \textsanskrit{Sāriputta}! Revata answered in the right way for him. For Revata enjoys retreat …” 

“Next\marginnote{13.1} I asked Anuruddha the same question. He said: ‘It’s a mendicant who surveys the entire galaxy with clairvoyance that is purified and surpasses the human … That’s the kind of mendicant who would beautify this park.’” 

“Good,\marginnote{13.8} good, \textsanskrit{Sāriputta}! Anuruddha answered in the right way for him. For Anuruddha surveys the entire galaxy with clairvoyance that is purified and surpasses the human.” 

“Next\marginnote{14.1} I asked \textsanskrit{Mahākassapa} the same question. He said: ‘It’s a mendicant who lives in the wilderness … and is accomplished in the knowledge and vision of freedom; and they praise these things. That’s the kind of mendicant who would beautify this park.’” 

“Good,\marginnote{14.8} good, \textsanskrit{Sāriputta}! Kassapa answered in the right way for him. For Kassapa lives in the wilderness … and is accomplished in the knowledge and vision of freedom; and he praises these things.” 

“Next\marginnote{15.1} I asked \textsanskrit{Mahāmoggallāna} the same question. He said: ‘It’s when two mendicants engage in discussion about the teaching … That’s the kind of mendicant who would beautify this park.’” 

“Good,\marginnote{15.8} good, \textsanskrit{Sāriputta}! \textsanskrit{Moggallāna} answered in the right way for him. For \textsanskrit{Moggallāna} is a Dhamma speaker.” 

When\marginnote{16.1} he had spoken, \textsanskrit{Moggallāna} said to the Buddha, “Next, I asked \textsanskrit{Sāriputta}: ‘Each of us has spoken from our heart. And now we ask you: \textsanskrit{Sāriputta}, the sal forest park at \textsanskrit{Gosiṅga} is lovely, the night is bright, the sal trees are in full blossom, and divine scents seem to float on the air. What kind of mendicant would beautify this park?’ When I had spoken, \textsanskrit{Sāriputta} said to me: ‘Reverend \textsanskrit{Moggallāna}, it’s when a mendicant masters their mind and is not mastered by it … That’s the kind of mendicant who would beautify this park.’” 

“Good,\marginnote{16.21} good, \textsanskrit{Moggallāna}! \textsanskrit{Sāriputta} answered in the right way for him. For \textsanskrit{Sāriputta} masters his mind and is not mastered by it …” 

When\marginnote{17.1} he had spoken, \textsanskrit{Sāriputta} asked the Buddha, “Sir, who has spoken well?” 

“You’ve\marginnote{17.3} all spoken well in your own way. However, listen to me also as to what kind of mendicant would beautify this sal forest park at \textsanskrit{Gosiṅga}. It’s a mendicant who, after the meal, returns from almsround, sits down cross-legged with their body straight, and establishes mindfulness right there, thinking: ‘I will not break this sitting posture until my mind is freed from the defilements by not grasping!’ That’s the kind of mendicant who would beautify this park.” 

That\marginnote{17.8} is what the Buddha said. Satisfied, those venerables were happy with what the Buddha said. 

%
\section*{{\suttatitleacronym MN 33}{\suttatitletranslation The Longer Discourse on the Cowherd }{\suttatitleroot Mahāgopālakasutta}}
\addcontentsline{toc}{section}{\tocacronym{MN 33} \toctranslation{The Longer Discourse on the Cowherd } \tocroot{Mahāgopālakasutta}}
\markboth{The Longer Discourse on the Cowherd }{Mahāgopālakasutta}
\extramarks{MN 33}{MN 33}

\scevam{So\marginnote{1.1} I have heard. }At one time the Buddha was staying near \textsanskrit{Sāvatthī} in Jeta’s Grove, \textsanskrit{Anāthapiṇḍika}’s monastery. There the Buddha addressed the mendicants, “Mendicants!” 

“Venerable\marginnote{1.5} sir,” they replied. The Buddha said this: 

“Mendicants,\marginnote{2.1} a cowherd with eleven factors can’t maintain and expand a herd of cattle. What eleven? It’s when a cowherd doesn’t know form, is unskilled in characteristics, doesn’t pick out flies’ eggs, doesn’t dress wounds, doesn’t smoke out pests, doesn’t know the ford, doesn’t know satisfaction, doesn’t know the trail, is not skilled in pastures, milks dry, and doesn’t show extra respect to the bulls who are fathers and leaders of the herd. A cowherd with these eleven factors can’t maintain and expand a herd of cattle. 

In\marginnote{3.1} the same way, a mendicant with eleven qualities can’t achieve growth, improvement, or maturity in this teaching and training. What eleven? It’s when a mendicant doesn’t know form, is unskilled in characteristics, doesn’t pick out flies’ eggs, doesn’t dress wounds, doesn’t smoke out pests, doesn’t know the ford, doesn’t know satisfaction, doesn’t know the trail, is not skilled in pastures, milks dry, and doesn’t show extra respect to senior mendicants of long standing, long gone forth, fathers and leaders of the \textsanskrit{Saṅgha}. 

And\marginnote{4.1} how does a mendicant not know form? It’s when a mendicant doesn’t truly understand that all form is the four primary elements, or form derived from the four primary elements. That’s how a mendicant doesn’t know form. 

And\marginnote{5.1} how is a mendicant not skilled in characteristics? It’s when a mendicant doesn’t understand that a fool is characterized by their deeds, and an astute person is characterized by their deeds. That’s how a mendicant isn’t skilled in characteristics. 

And\marginnote{6.1} how does a mendicant not pick out flies’ eggs? It’s when a mendicant tolerates a sensual, malicious, or cruel thought that has arisen. They tolerate any bad, unskillful qualities that have arisen. They don’t give them up, get rid of them, eliminate them, and obliterate them. That’s how a mendicant doesn’t pick out flies’ eggs. 

And\marginnote{7.1} how does a mendicant not dress wounds? When a mendicant sees a sight with their eyes, they get caught up in the features and details. Since the faculty of sight is left unrestrained, bad unskillful qualities of desire and aversion become overwhelming. They don’t practice restraint, they don’t protect the faculty of sight, and they don’t achieve its restraint. When they hear a sound with their ears … smell an odor with their nose … taste a flavor with their tongue … feel a touch with their body … know a thought with their mind, they get caught up in the features and details. Since the faculty of the mind is left unrestrained, bad unskillful qualities of desire and aversion become overwhelming. They don’t practice restraint, they don’t protect the faculty of the mind, and they don’t achieve its restraint. That’s how a mendicant doesn’t dress wounds. 

And\marginnote{8.1} how does a mendicant not smoke out pests? It’s when a mendicant doesn’t teach others the Dhamma in detail as they learned and memorized it. That’s how a mendicant doesn’t smoke out pests. 

And\marginnote{9.1} how does a mendicant not know the ford? It’s when a mendicant doesn’t from time to time go up to those mendicants who are very learned—knowledgeable in the scriptures, who have memorized the teachings, the monastic law, and the outlines—and ask them questions: ‘Why, sir, does it say this? What does that mean?’ Those venerables don’t clarify what is unclear, reveal what is obscure, and dispel doubt regarding the many doubtful matters. That’s how a mendicant doesn’t know the ford. 

And\marginnote{10.1} how does a mendicant not know satisfaction? It’s when a mendicant, when the teaching and training proclaimed by the Realized One are being taught, finds no inspiration in the meaning and the teaching, and finds no joy connected with the teaching. That’s how a mendicant doesn’t know satisfaction. 

And\marginnote{11.1} how does a mendicant not know the trail? It’s when a mendicant doesn’t truly understand the noble eightfold path. That’s how a mendicant doesn’t know the trail. 

And\marginnote{12.1} how is a mendicant not skilled in pastures? It’s when a mendicant doesn’t truly understand the four kinds of mindfulness meditation. That’s how a mendicant is not skilled in pastures. 

And\marginnote{13.1} how does a mendicant milk dry? It’s when a mendicant is invited by a householder to accept robes, almsfood, lodgings, and medicines and supplies for the sick, and that mendicant doesn’t know moderation in accepting. That’s how a mendicant milks dry. 

And\marginnote{14.1} how does a mendicant not show extra respect to senior mendicants of long standing, long gone forth, fathers and leaders of the \textsanskrit{Saṅgha}? It’s when a mendicant doesn’t consistently treat senior mendicants of long standing, long gone forth, fathers and leaders of the \textsanskrit{Saṅgha} with kindness by way of body, speech, and mind, both in public and in private. That’s how a mendicant doesn’t show extra respect to senior mendicants of long standing, long gone forth, fathers and leaders of the \textsanskrit{Saṅgha}. 

A\marginnote{14.6} mendicant with these eleven qualities can’t achieve growth, improvement, or maturity in this teaching and training. 

A\marginnote{15.1} cowherd with eleven factors can maintain and expand a herd of cattle. What eleven? It’s when a cowherd knows form, is skilled in characteristics, picks out flies’ eggs, dresses wounds, smokes out pests, knows the ford, knows satisfaction, knows the trail, is skilled in pastures, doesn’t milk dry, and shows extra respect to the bulls who are fathers and leaders of the herd. A cowherd with these eleven factors can maintain and expand a herd of cattle. 

In\marginnote{16.1} the same way, a mendicant with eleven qualities can achieve growth, improvement, and maturity in this teaching and training. What eleven? It’s when a mendicant knows form, is skilled in characteristics, picks out flies’ eggs, dresses wounds, smokes out pests, knows the ford, knows satisfaction, knows the trail, is skilled in pastures, doesn’t milk dry, and shows extra respect to senior mendicants of long standing, long gone forth, fathers and leaders of the \textsanskrit{Saṅgha}. 

And\marginnote{17.1} how does a mendicant know form? It’s when a mendicant truly understands that all form is the four primary elements, or form derived from the four primary elements. That’s how a mendicant knows form. 

And\marginnote{18.1} how is a mendicant skilled in characteristics? It’s when a mendicant understands that a fool is characterized by their deeds, and an astute person is characterized by their deeds. That’s how a mendicant is skilled in characteristics. 

And\marginnote{19.1} how does a mendicant pick out flies’ eggs? It’s when a mendicant doesn’t tolerate a sensual, malicious, or cruel thought that has arisen. They don’t tolerate any bad, unskillful qualities that have arisen, but give them up, get rid of them, eliminate them, and obliterate them. That’s how a mendicant picks out flies’ eggs. 

And\marginnote{20.1} how does a mendicant dress wounds? When a mendicant sees a sight with their eyes, they don’t get caught up in the features and details. If the faculty of sight were left unrestrained, bad unskillful qualities of desire and aversion would become overwhelming. For this reason, they practice restraint, protecting the faculty of sight, and achieving its restraint. When they hear a sound with their ears … smell an odor with their nose … taste a flavor with their tongue … feel a touch with their body … know a thought with their mind, they don’t get caught up in the features and details. If the faculty of mind were left unrestrained, bad unskillful qualities of desire and aversion would become overwhelming. For this reason, they practice restraint, protecting the faculty of mind, and achieving its restraint. That’s how a mendicant dresses wounds. 

And\marginnote{21.1} how does a mendicant smoke out pests? It’s when a mendicant teaches others the Dhamma in detail as they learned and memorized it. That’s how a mendicant smokes out pests. 

And\marginnote{22.1} how does a mendicant know the ford? It’s when from time to time a mendicant goes up to those mendicants who are very learned—knowledgeable in the scriptures, who have memorized the teachings, the monastic law, and the outlines—and asks them questions: ‘Why, sir, does it say this? What does that mean?’ Those venerables clarify what is unclear, reveal what is obscure, and dispel doubt regarding the many doubtful matters. That’s how a mendicant knows the ford. 

And\marginnote{23.1} how does a mendicant know satisfaction? It’s when a mendicant, when the teaching and training proclaimed by the Realized One are being taught, finds inspiration in the meaning and the teaching, and finds joy connected with the teaching. That’s how a mendicant knows satisfaction. 

And\marginnote{24.1} how does a mendicant know the trail? It’s when a mendicant truly understands the noble eightfold path. That’s how a mendicant knows the trail. 

And\marginnote{25.1} how is a mendicant skilled in pastures? It’s when a mendicant truly understands the four kinds of mindfulness meditation. That’s how a mendicant is skilled in pastures. 

And\marginnote{26.1} how does a mendicant not milk dry? It’s when a mendicant is invited by a householder to accept robes, almsfood, lodgings, and medicines and supplies for the sick, and that mendicant knows moderation in accepting. That’s how a mendicant doesn’t milk dry. 

And\marginnote{27.1} how does a mendicant show extra respect to senior mendicants of long standing, long gone forth, fathers and leaders of the \textsanskrit{Saṅgha}? It’s when a mendicant consistently treats senior mendicants of long standing, long gone forth, fathers and leaders of the \textsanskrit{Saṅgha} with kindness by way of body, speech, and mind, both in public and in private. That’s how a mendicant shows extra respect to senior mendicants of long standing, long gone forth, fathers and leaders of the \textsanskrit{Saṅgha}. 

A\marginnote{27.6} mendicant with these eleven qualities can achieve growth, improvement, and maturity in this teaching and training.” 

That\marginnote{27.7} is what the Buddha said. Satisfied, the mendicants were happy with what the Buddha said. 

%
\section*{{\suttatitleacronym MN 34}{\suttatitletranslation The Shorter Discourse on the Cowherd }{\suttatitleroot Cūḷagopālakasutta}}
\addcontentsline{toc}{section}{\tocacronym{MN 34} \toctranslation{The Shorter Discourse on the Cowherd } \tocroot{Cūḷagopālakasutta}}
\markboth{The Shorter Discourse on the Cowherd }{Cūḷagopālakasutta}
\extramarks{MN 34}{MN 34}

\scevam{So\marginnote{1.1} I have heard. }At one time the Buddha was staying in the land of the Vajjis near \textsanskrit{Ukkacelā} on the bank of the Ganges river. There the Buddha addressed the mendicants, “Mendicants!” 

“Venerable\marginnote{1.5} sir,” they replied. The Buddha said this: 

“Once\marginnote{2.1} upon a time, mendicants, there was an unintelligent Magadhan cowherd. In the last month of the rainy season, without inspecting the near shore or the far shore, he drove his cattle across a place with no ford on the Ganges river to the land of the Suvidehans on the northern shore. 

But\marginnote{3.1} the cattle bunched up in mid-stream and came to ruin right there. Why is that? Because the unintelligent cowherd failed to inspect the shores before driving the cattle across at a place with no ford. In the same way, there are ascetics and brahmins who are unskilled in this world and the other world, unskilled in \textsanskrit{Māra}’s domain and its opposite, and unskilled in Death’s domain and its opposite. If anyone thinks they are worth listening to and trusting, it will be for their lasting harm and suffering. 

Once\marginnote{4.1} upon a time, mendicants, there was an intelligent Magadhan cowherd. In the last month of the rainy season, after inspecting the near shore and the far shore, he drove his cattle across a ford on the Ganges river to the land of the Suvidehans on the northern shore. 

First\marginnote{5.1} he drove across the bulls, the fathers and leaders of the herd. They breasted the stream of the Ganges and safely reached the far shore. Then he drove across the strong and tractable cattle. They too breasted the stream of the Ganges and safely reached the far shore. Then he drove across the bullocks and heifers. They too breasted the stream of the Ganges and safely reached the far shore. Then he drove across the calves and weak cattle. They too breasted the stream of the Ganges and safely reached the far shore. Once it happened that a baby calf had just been born. Urged on by its mother’s lowing, even it managed to breast the stream of the Ganges and safely reach the far shore. Why is that? Because the intelligent cowherd inspected both shores before driving the cattle across at a ford. 

In\marginnote{5.12} the same way, there are ascetics and brahmins who are skilled in this world and the other world, skilled in \textsanskrit{Māra}’s domain and its opposite, and skilled in Death’s domain and its opposite. If anyone thinks they are worth listening to and trusting, it will be for their lasting welfare and happiness. 

Just\marginnote{6.1} like the bulls, fathers and leaders of the herd, who crossed the Ganges to safety are the mendicants who are perfected, who have ended the defilements, completed the spiritual journey, done what had to be done, laid down the burden, achieved their own goal, utterly ended the fetters of rebirth, and are rightly freed through enlightenment. Having breasted \textsanskrit{Māra}’s stream, they have safely crossed over to the far shore. 

Just\marginnote{7.1} like the strong and tractable cattle who crossed the Ganges to safety are the mendicants who, with the ending of the five lower fetters, are reborn spontaneously. They’re extinguished there, and are not liable to return from that world. They too, having breasted \textsanskrit{Māra}’s stream, will safely cross over to the far shore. 

Just\marginnote{8.1} like the bullocks and heifers who crossed the Ganges to safety are the mendicants who, with the ending of three fetters, and the weakening of greed, hate, and delusion, are once-returners. They come back to this world once only, then make an end of suffering. They too, having breasted \textsanskrit{Māra}’s stream, will safely cross over to the far shore. 

Just\marginnote{9.1} like the calves and weak cattle who crossed the Ganges to safety are the mendicants who, with the ending of three fetters are stream-enterers, not liable to be reborn in the underworld, bound for awakening. They too, having breasted \textsanskrit{Māra}’s stream, will safely cross over to the far shore. 

Just\marginnote{10.1} like the baby calf who had just been born, but, urged on by its mother’s lowing, still managed to cross the Ganges to safety are the mendicants who are followers of principles, followers by faith. They too, having breasted \textsanskrit{Māra}’s stream, will safely cross over to the far shore. 

Mendicants,\marginnote{11.1} I am skilled in this world and the other world, skilled in \textsanskrit{Māra}’s domain and its opposite, and skilled in Death’s domain and its opposite. If anyone thinks I am worth listening to and trusting, it will be for their lasting welfare and happiness.” 

That\marginnote{12.1} is what the Buddha said. Then the Holy One, the Teacher, went on to say: 

\begin{verse}%
“This\marginnote{12.3} world and the other world \\
have been clearly explained by one who knows; \\
as well as \textsanskrit{Māra}’s reach, \\
and what’s out of Death’s reach. 

Directly\marginnote{12.7} knowing the whole world, \\
the Buddha who understands \\
has flung open the door of the deathless, \\
for realizing the sanctuary, extinguishment. 

The\marginnote{12.11} Wicked One’s stream has been cut, \\
it’s blown away and mown down. \\
Be full of joy, mendicants, \\
set your heart on the sanctuary!” 

%
\end{verse}

%
\section*{{\suttatitleacronym MN 35}{\suttatitletranslation The Shorter Discourse With Saccaka }{\suttatitleroot Cūḷasaccakasutta}}
\addcontentsline{toc}{section}{\tocacronym{MN 35} \toctranslation{The Shorter Discourse With Saccaka } \tocroot{Cūḷasaccakasutta}}
\markboth{The Shorter Discourse With Saccaka }{Cūḷasaccakasutta}
\extramarks{MN 35}{MN 35}

\scevam{So\marginnote{1.1} I have heard. }At one time the Buddha was staying near \textsanskrit{Vesālī}, at the Great Wood, in the hall with the peaked roof. 

Now\marginnote{2.1} at that time Saccaka, the son of Jain parents, was staying in \textsanskrit{Vesālī}. He was a debater and clever speaker regarded as holy by many people. He was telling a crowd in \textsanskrit{Vesālī}, “If I was to take them on in debate, I don’t see any ascetic or brahmin—leader of an order or a community, or the teacher of a community, even one who claims to be a perfected one, a fully awakened Buddha—who would not shake and rock and tremble, sweating from the armpits. Even if I took on an insentient post in debate, it would shake and rock and tremble. How much more then a human being!” 

Then\marginnote{3.1} Venerable Assaji robed up in the morning and, taking his bowl and robe, entered \textsanskrit{Vesālī} for alms. As Saccaka was going for a walk he saw Assaji coming off in the distance. He approached him and exchanged greetings with him. 

When\marginnote{3.4} the greetings and polite conversation were over, Saccaka stood to one side and said to Assaji, “Master Assaji, how does the ascetic Gotama guide his disciples? And how does instruction to his disciples generally proceed?” 

“Aggivessana,\marginnote{4.2} this is how the ascetic Gotama guides his disciples, and how instruction to his disciples generally proceeds: ‘Form, feeling, perception, choices, and consciousness are impermanent. Form, feeling, perception, choices, and consciousness are not-self. All conditions are impermanent. All things are not-self.’ This is how the ascetic Gotama guides his disciples, and how instruction to his disciples generally proceeds.” 

“It’s\marginnote{4.7} sad to hear, Master Assaji, that the ascetic Gotama has such a doctrine. Hopefully, some time or other I’ll get to meet Master Gotama, and we can have a discussion. And hopefully I can dissuade him from this harmful misconception.” 

Now\marginnote{5.1} at that time around five hundred Licchavis were sitting together at the town hall on some business. Then Saccaka went up to them and said, “Come forth, good \textsanskrit{Licchavīs}, come forth! Today I am going to have a discussion with the ascetic Gotama. If he stands by the position stated to me by one of his well-known disciples—a mendicant named Assaji—I’ll take him on in debate and drag him to and fro and round about, like a strong man would drag a fleecy sheep to and fro and round about! Taking him on in debate, I’ll drag him to and fro and round about, like a strong brewer’s worker would toss a large brewer’s sieve into a deep lake, grab it by the corners, and drag it to and fro and round about! Taking him on in debate, I’ll shake him down and about, and give him a beating, like a strong brewer’s mixer would grab a strainer by the corners and shake it down and about, and give it a beating! I’ll play a game of ear-washing with the ascetic Gotama, like a sixty-year-old elephant would plunge into a deep lotus pond and play a game of ear-washing! Come forth, good \textsanskrit{Licchavīs}, come forth! Today I am going to have a discussion with the ascetic Gotama.” 

At\marginnote{6.1} that, some of the Licchavis said, “How can the ascetic Gotama refute Saccaka’s doctrine, when it is Saccaka who will refute Gotama’s doctrine?” 

But\marginnote{6.3} some of the Licchavis said, “Who is Saccaka to refute the Buddha’s doctrine, when it is the Buddha who will refute Saccaka’s doctrine?” 

Then\marginnote{6.5} Saccaka, escorted by the five hundred Licchavis, went to the hall with the peaked roof in the Great Wood. 

At\marginnote{7.1} that time several mendicants were walking mindfully in the open air. Then Saccaka went up to them and said, “Gentlemen, where is Master Gotama at present? For we want to see him.” 

“Aggivessana,\marginnote{7.5} the Buddha has plunged deep into the Great Wood and is sitting at the root of a tree for the day’s meditation.” 

Then\marginnote{8.1} Saccaka, together with a large group of Licchavis, went to see the Buddha in the Great Wood, and exchanged greetings with him. When the greetings and polite conversation were over, he sat down to one side. Before sitting down to one side, some of the \textsanskrit{Licchavīs} bowed, some exchanged greetings and polite conversation, some held up their joined palms toward the Buddha, some announced their name and clan, while some kept silent. 

Then\marginnote{9.1} Saccaka said to the Buddha, “I’d like to ask Master Gotama about a certain point, if you’d take the time to answer.” 

“Ask\marginnote{9.3} what you wish, Aggivessana.” 

“How\marginnote{9.4} does the ascetic Gotama guide his disciples? And how does instruction to his disciples generally proceed?” 

“This\marginnote{9.5} is how I guide my disciples, and how instruction to my disciples generally proceeds: ‘Form, feeling, perception, choices, and consciousness are impermanent. Form, feeling, perception, choices, and consciousness are not-self. All conditions are impermanent. All things are not-self.’ This is how I guide my disciples, and how instruction to my disciples generally proceeds.” 

“A\marginnote{10.1} simile strikes me, Master Gotama.” 

“Then\marginnote{10.2} speak as you feel inspired,” said the Buddha. 

“All\marginnote{10.3} the plants and seeds that achieve growth, increase, and maturity do so depending on the earth and grounded on the earth. All the hard work that gets done depends on the earth and is grounded on the earth. 

In\marginnote{10.7} the same way, an individual’s self is form. Grounded on form they make good and bad choices. An individual’s self is feeling … perception … choices … consciousness. Grounded on consciousness they make good and bad choices.” 

“Aggivessana,\marginnote{11.1} are you not saying this: ‘Form is my self, feeling is my self, perception is my self, choices are my self, consciousness is my self’?” 

“Indeed,\marginnote{11.3} Master Gotama, that is what I am saying. And this big crowd agrees with me!” 

“What\marginnote{11.5} has this big crowd to do with you? Please just explain your own statement.” 

“Then,\marginnote{11.7} Master Gotama, what I am saying is this: ‘Form is my self, feeling is my self, perception is my self, choices are my self, consciousness is my self’.” 

“Well\marginnote{12.1} then, Aggivessana, I’ll ask you about this in return, and you can answer as you like. What do you think, Aggivessana? Consider an anointed aristocratic king such as Pasenadi of Kosala or \textsanskrit{Ajātasattu} Vedehiputta of Magadha. Would they have the power in their own realm to execute, fine, or banish those who are guilty?” 

“An\marginnote{12.5} anointed king would have such power, Master Gotama. Even federations such as the Vajjis and Mallas have such power in their own realm. So of course an anointed king such as Pasenadi or \textsanskrit{Ajātasattu} would wield such power, as is their right.” 

“What\marginnote{13.1} do you think, Aggivessana? When you say, ‘Form is my self,’ do you have power over that form to say: ‘May my form be like this! May it not be like that’?” When he said this, Saccaka kept silent. The Buddha asked the question a second time, but Saccaka still kept silent. So the Buddha said to Saccaka, “Answer now, Aggivessana. Now is not the time for silence. If someone fails to answer a legitimate question when asked three times by the Buddha, their head explodes into seven pieces there and then.” 

Now\marginnote{14.1} at that time the spirit \textsanskrit{Vajirapāṇi}, taking up a burning iron thunderbolt, blazing and glowing, stood in the sky above Saccaka, thinking, “If this Saccaka doesn’t answer when asked a third time, I’ll blow his head into seven pieces there and then!” And both the Buddha and Saccaka could see \textsanskrit{Vajirapāṇi}. 

Saccaka\marginnote{14.4} was terrified, shocked, and awestruck. Looking to the Buddha for shelter, protection, and refuge, he said, “Ask me, Master Gotama. I will answer.” 

“What\marginnote{15.1} do you think, Aggivessana? When you say, ‘Form is my self,’ do you have power over that form to say: ‘May my form be like this! May it not be like that’?” 

“No,\marginnote{15.5} Master Gotama.” 

“Think\marginnote{16.1} about it, Aggivessana! You should think before answering. What you said before and what you said after don’t match up. What do you think, Aggivessana? When you say, ‘Feeling is my self,’ do you have power over that feeling to say: ‘May my feeling be like this! May it not be like that’?” 

“No,\marginnote{16.8} Master Gotama.” 

“Think\marginnote{17.1} about it, Aggivessana! You should think before answering. What you said before and what you said after don’t match up. What do you think, Aggivessana? When you say, ‘Perception is my self,’ do you have power over that perception to say: ‘May my perception be like this! May it not be like that’?” 

“No,\marginnote{17.8} Master Gotama.” 

“Think\marginnote{18.1} about it, Aggivessana! You should think before answering. What you said before and what you said after don’t match up. What do you think, Aggivessana? When you say, ‘Choices are my self,’ do you have power over those choices to say: ‘May my choices be like this! May they not be like that’?” 

“No,\marginnote{18.8} Master Gotama.” 

“Think\marginnote{19.1} about it, Aggivessana! You should think before answering. What you said before and what you said after don’t match up. What do you think, Aggivessana? When you say, ‘Consciousness is my self,’ do you have power over that consciousness to say: ‘May my consciousness be like this! May it not be like that’?” 

“No,\marginnote{19.8} Master Gotama.” 

“Think\marginnote{20.1} about it, Aggivessana! You should think before answering. What you said before and what you said after don’t match up. What do you think, Aggivessana? Is form permanent or impermanent?” 

“Impermanent.”\marginnote{20.6} 

“But\marginnote{20.7} if it’s impermanent, is it suffering or happiness?” 

“Suffering.”\marginnote{20.8} 

“But\marginnote{20.9} if it’s impermanent, suffering, and liable to wear out, is it fit to be regarded thus: ‘This is mine, I am this, this is my self’?” 

“No,\marginnote{20.11} Master Gotama.” 

“What\marginnote{20.12} do you think, Aggivessana? Is feeling … perception … choices … consciousness permanent or impermanent?” 

“Impermanent.”\marginnote{20.17} 

“But\marginnote{20.18} if it’s impermanent, is it suffering or happiness?” 

“Suffering.”\marginnote{20.19} 

“But\marginnote{20.20} if it’s impermanent, suffering, and liable to wear out, is it fit to be regarded thus: ‘This is mine, I am this, this is my self’?” 

“No,\marginnote{20.22} Master Gotama.” 

“What\marginnote{21.1} do you think, Aggivessana? Consider someone who clings, holds, and attaches to suffering, regarding it thus: ‘This is mine, I am this, this is my self.’ Would such a person be able to completely understand suffering themselves, or live having wiped out suffering?” 

“How\marginnote{21.3} could they? No, Master Gotama.” 

“What\marginnote{21.5} do you think, Aggivessana? This being so, aren’t you someone who clings, holds, and attaches to suffering, regarding it thus: ‘This is mine, I am this, this is my self’?” 

“How\marginnote{21.8} could I not? Yes, Master Gotama.” 

“Suppose,\marginnote{22.1} Aggivessana, there was a person in need of heartwood. Wandering in search of heartwood, they’d take a sharp axe and enter a forest. There they’d see a big banana tree, straight and young and grown free of defects. They’d cut it down at the base, cut off the top, and unroll the coiled sheaths. But they wouldn’t even find sapwood, much less heartwood. 

In\marginnote{22.5} the same way, when pursued, pressed, and grilled by me on your own doctrine, you turn out to be void, hollow, and mistaken. But it was you who stated before the assembly of \textsanskrit{Vesālī}: ‘If I was to take them on in debate, I don’t see any ascetic or brahmin—leader of an order or a community, or the teacher of a community, even one who claims to be a perfected one, a fully awakened Buddha—who would not shake and rock and tremble, sweating from the armpits. Even if I took on an insentient post in debate, it would shake and rock and tremble. How much more then a human being!’ But sweat is pouring from your forehead; it’s soaked through your robe and drips on the ground. While I now have no sweat on my body.” So the Buddha revealed his golden body to the assembly. When this was said, Saccaka sat silent, embarrassed, shoulders drooping, downcast, depressed, with nothing to say. 

Knowing\marginnote{23.1} this, the Licchavi Dummukha said to the Buddha, “A simile strikes me, Blessed One.” 

“Then\marginnote{23.3} speak as you feel inspired,” said the Buddha. 

“Sir,\marginnote{23.4} suppose there was a lotus pond not far from a town or village, and a crab lived there. Then several boys or girls would leave the town or village and go to the pond, where they’d pull out the crab and put it on dry land. Whenever that crab extended a claw, those boys or girls would snap, crack, and break it off with a stick or a stone. And when that crab’s claws had all been snapped, cracked, and broken off it wouldn’t be able to return down into that lotus pond. In the same way, sir, the Buddha has snapped, cracked, and broken off all Saccaka’s tricks, dodges, and evasions. Now he can’t get near the Buddha again looking for a debate.” 

But\marginnote{24.1} Saccaka said to him, “Hold on, Dummukha, hold on! I wasn’t talking with you, I was talking with Master Gotama. 

Master\marginnote{24.3} Gotama, leave aside that statement I made—as did various other ascetics and brahmins—it was, like, just a bit of nonsense. How do you define a disciple of Master Gotama who follows instructions and responds to advice; who has gone beyond doubt, got rid of indecision, gained assurance, and is independent of others in the Teacher’s instructions?” 

“It’s\marginnote{24.6} when one of my disciples truly sees any kind of form at all—past, future, or present; internal or external; coarse or fine; inferior or superior; far or near: \emph{all} form—with right understanding: ‘This is not mine, I am not this, this is not my self.’ They truly see any kind of feeling … perception … choices … consciousness at all—past, future, or present; internal or external; coarse or fine; inferior or superior; far or near: \emph{all} consciousness—with right understanding: ‘This is not mine, I am not this, this is not my self.’ That’s how to define one of my disciples who follows instructions and responds to advice; who has gone beyond doubt, got rid of indecision, gained assurance, and is independent of others in the Teacher’s instructions.” 

“But\marginnote{25.1} how do you define a mendicant who is a perfected one, with defilements ended, who has completed the spiritual journey, done what had to be done, laid down the burden, achieved their own true goal, utterly ended the fetters of rebirth, and is rightly freed through enlightenment?” 

“It’s\marginnote{25.2} when one of my disciples truly sees any kind of form at all—past, future, or present; internal or external; coarse or fine; inferior or superior; far or near: \emph{all} form—with right understanding: ‘This is not mine, I am not this, this is not my self.’ And having seen this with right understanding they’re freed by not grasping. They truly see any kind of feeling … perception … choices … consciousness at all—past, future, or present; internal or external; coarse or fine; inferior or superior; far or near: \emph{all} consciousness—with right understanding: ‘This is not mine, I am not this, this is not my self.’ And having seen this with right understanding they’re freed by not grasping. That’s how to define a mendicant who is a perfected one, with defilements ended, who has completed the spiritual journey, done what had to be done, laid down the burden, achieved their own true goal, utterly ended the fetters of rebirth, and is rightly freed through enlightenment. 

A\marginnote{26.1} mendicant whose mind is freed like this has three unsurpassable qualities: unsurpassable vision, practice, and freedom. They honor, respect, esteem, and venerate only the Realized One: ‘The Blessed One is awakened, tamed, serene, crossed over, and extinguished. And he teaches Dhamma for awakening, self-control, serenity, crossing over, and extinguishment.’” 

When\marginnote{27.1} he had spoken, Saccaka said to him, “Master Gotama, it was rude and impudent of me to imagine I could attack you in debate. For a person might find safety after attacking a rutting elephant, but not after attacking Master Gotama. A person might find safety after attacking a blazing mass of fire, but not after attacking Master Gotama. They might find safety after attacking a poisonous viper, but not after attacking Master Gotama. It was rude and impudent of me to imagine I could attack you in debate. Would Master Gotama together with the mendicant \textsanskrit{Saṅgha} please accept tomorrow’s meal from me?” The Buddha consented in silence. 

Then,\marginnote{28.1} knowing that the Buddha had consented, Saccaka addressed those Licchavis, “Listen, gentlemen. I have invited the ascetic Gotama together with the \textsanskrit{Saṅgha} of mendicants for tomorrow’s meal. You may all bring me what you think is suitable.” 

Then,\marginnote{29.1} when the night had passed, those Licchavis presented Saccaka with an offering of five hundred servings of food. And Saccaka had a variety of delicious foods prepared in his own home. Then he had the Buddha informed of the time, saying, “It’s time, Master Gotama, the meal is ready.” 

Then\marginnote{30.1} the Buddha robed up in the morning and, taking his bowl and robe, went to Saccaka’s park, where he sat on the seat spread out, together with the \textsanskrit{Saṅgha} of mendicants. Then Saccaka served and satisfied the mendicant \textsanskrit{Saṅgha} headed by the Buddha with his own hands with a variety of delicious foods. When the Buddha had eaten and washed his hand and bowl, Saccaka took a low seat and sat to one side. 

Then\marginnote{30.4} Saccaka said to the Buddha, “Master Gotama, may the merit and the growth of merit in this gift be for the happiness of the donors.” 

“Aggivessana,\marginnote{30.6} whatever comes from giving to a recipient of a religious donation such as yourself—who is not free of greed, hate, and delusion—will accrue to the donors. Whatever comes from giving to a recipient of a religious donation such as myself—who is free of greed, hate, and delusion—will accrue to you.” 

%
\section*{{\suttatitleacronym MN 36}{\suttatitletranslation The Longer Discourse With Saccaka }{\suttatitleroot Mahāsaccakasutta}}
\addcontentsline{toc}{section}{\tocacronym{MN 36} \toctranslation{The Longer Discourse With Saccaka } \tocroot{Mahāsaccakasutta}}
\markboth{The Longer Discourse With Saccaka }{Mahāsaccakasutta}
\extramarks{MN 36}{MN 36}

\scevam{So\marginnote{1.1} I have heard. }At one time the Buddha was staying near \textsanskrit{Vesālī}, at the Great Wood, in the hall with the peaked roof. 

Now\marginnote{2.1} at that time in the morning the Buddha, being properly dressed, took his bowl and robe, wishing to enter \textsanskrit{Vesālī} for alms. 

Then\marginnote{3.1} as Saccaka, the son of Jain parents, was going for a walk he approached the hall with the peaked roof in the Great Wood. Venerable Ānanda saw him coming off in the distance, and said to the Buddha, “Sir, Saccaka, the son of Jain parents, is coming. He’s a debater and clever speaker regarded as holy by many people. He wants to discredit the Buddha, the teaching, and the \textsanskrit{Saṅgha}. Please, sir, sit for a moment out of compassion.” The Buddha sat on the seat spread out. 

Then\marginnote{3.8} Saccaka went up to the Buddha, and exchanged greetings with him. When the greetings and polite conversation were over, he sat down to one side and said to the Buddha, 

“Master\marginnote{4.1} Gotama, there are some ascetics and brahmins who live committed to the practice of developing physical endurance, without developing the mind. They suffer painful physical feelings. This happened to someone once. Their thighs became paralyzed, their heart burst, hot blood gushed from their mouth, and they went mad and lost their mind. Their mind was subject to the body, and the body had power over it. Why is that? Because their mind was not developed. There are some ascetics and brahmins who live committed to the practice of developing the mind, without developing physical endurance. They suffer painful mental feelings. This happened to someone once. Their thighs became paralyzed, their heart burst, hot blood gushed from their mouth, and they went mad and lost their mind. Their body was subject to the mind, and the mind had power over it. Why is that? Because their physical endurance was not developed. It occurs to me that Master Gotama’s disciples must live committed to the practice of developing the mind, without developing physical endurance.” 

“But\marginnote{5.1} Aggivessana, what have you heard about the development of physical endurance?” 

“Take,\marginnote{5.2} for example, Nanda Vaccha, Kisa \textsanskrit{Saṅkicca}, and Makkhali \textsanskrit{Gosāla}. They go naked, ignoring conventions. They lick their hands, and don’t come or wait when called. They don’t consent to food brought to them, or food prepared on purpose for them, or an invitation for a meal. They don’t receive anything from a pot or bowl; or from someone who keeps sheep, or who has a weapon or a shovel in their home; or where a couple is eating; or where there is a woman who is pregnant, breastfeeding, or who has a man in her home; or where there’s a dog waiting or flies buzzing. They accept no fish or meat or liquor or wine, and drink no beer. They go to just one house for alms, taking just one mouthful, or two houses and two mouthfuls, up to seven houses and seven mouthfuls. They feed on one saucer a day, two saucers a day, up to seven saucers a day. They eat once a day, once every second day, up to once a week, and so on, even up to once a fortnight. They live committed to the practice of eating food at set intervals.” 

“But\marginnote{6.1} Aggivessana, do they get by on so little?” 

“No,\marginnote{6.2} Master Gotama. Sometimes they eat a variety of luxury foods and drink a variety of luxury beverages. They gather their body’s strength, build it up, and get fat.” 

“What\marginnote{6.5} they earlier gave up, they later got back. That is how there is the increase and decrease of this body. But Aggivessana, what have you heard about development of the mind?” When Saccaka was questioned by the Buddha about development of the mind, he was stumped. 

So\marginnote{7.1} the Buddha said to Saccaka, “The development of physical endurance that you have described is not the legitimate development of physical endurance in the noble one’s training. And since you don’t even understand the development of physical endurance, how can you possibly understand the development of the mind? Still, as to how someone is undeveloped in physical endurance and mind, and how someone is developed in physical endurance and mind, listen and pay close attention, I will speak.” 

“Yes,\marginnote{7.6} sir,” replied Saccaka. The Buddha said this: 

“And\marginnote{8.1} how is someone undeveloped in physical endurance and mind? Take an unlearned ordinary person who has a pleasant feeling. When they experience pleasant feeling they become full of lust for it. Then that pleasant feeling ceases. And when it ceases, a painful feeling arises. When they suffer painful feeling, they sorrow and wail and lament, beating their breast and falling into confusion. Because their physical endurance is undeveloped, pleasant feelings occupy the mind. And because their mind is undeveloped, painful feelings occupy the mind. Someone whose mind is occupied by both pleasant and painful feelings like this is undeveloped in physical endurance and in mind. 

And\marginnote{9.1} how is someone developed in physical endurance and mind? Take a learned noble disciple who has a pleasant feeling. When they experience pleasant feeling they don’t become full of lust for it. Then that pleasant feeling ceases. And when it ceases, painful feeling arises. When they suffer painful feelings they don’t sorrow or wail or lament, beating their breast and falling into confusion. Because their physical endurance is developed, pleasant feelings don’t occupy the mind. And because their mind is developed, painful feelings don’t occupy the mind. Someone whose mind is not occupied by both pleasant and painful feelings like this is developed in physical endurance and in mind.” 

“I\marginnote{10.1} am quite confident that Master Gotama is developed in physical endurance and in mind.” 

“Your\marginnote{10.3} words are clearly invasive and intrusive, Aggivessana. Nevertheless, I will answer you. Ever since I shaved off my hair and beard, dressed in ocher robes, and went forth from the lay life to homelessness, it has not been possible for any pleasant or painful feeling to occupy my mind.” 

“Surely\marginnote{11.1} you must have had feelings so pleasant or so painful that they could occupy your mind?” 

“How\marginnote{12.1} could I not, Aggivessana? Before my awakening—when I was still unawakened but intent on awakening—I thought: ‘Living in a house is cramped and dirty, but the life of one gone forth is wide open. It’s not easy for someone living at home to lead the spiritual life utterly full and pure, like a polished shell. Why don’t I shave off my hair and beard, dress in ocher robes, and go forth from the lay life to homelessness?’ 

Some\marginnote{13.1} time later, while still black-haired, blessed with youth, in the prime of life—though my mother and father wished otherwise, weeping with tearful faces—I shaved off my hair and beard, dressed in ocher robes, and went forth from the lay life to homelessness. 

Once\marginnote{13.2} I had gone forth I set out to discover what is skillful, seeking the supreme state of sublime peace. I approached \textsanskrit{Āḷāra} \textsanskrit{Kālāma} and said to him, ‘Reverend \textsanskrit{Kālāma}, I wish to lead the spiritual life in this teaching and training.’ 

\textsanskrit{Āḷāra}\marginnote{13.4} \textsanskrit{Kālāma} replied, ‘Stay, venerable. This teaching is such that a sensible person can soon realize their own tradition with their own insight and live having achieved it.’ 

I\marginnote{13.7} quickly memorized that teaching. So far as lip-recital and oral recitation were concerned, I spoke with knowledge and the authority of the elders. I claimed to know and see, and so did others. 

Then\marginnote{13.9} it occurred to me, ‘It is not solely by mere faith that \textsanskrit{Āḷāra} \textsanskrit{Kālāma} declares: “I realize this teaching with my own insight, and live having achieved it.” Surely he meditates knowing and seeing this teaching.’ 

So\marginnote{13.12} I approached \textsanskrit{Āḷāra} \textsanskrit{Kālāma} and said to him, ‘Reverend \textsanskrit{Kālāma}, to what extent do you say you’ve realized this teaching with your own insight?’ When I said this, he declared the dimension of nothingness. 

Then\marginnote{13.15} it occurred to me, ‘It’s not just \textsanskrit{Āḷāra} \textsanskrit{Kālāma} who has faith, energy, mindfulness, immersion, and wisdom; I too have these things. Why don’t I make an effort to realize the same teaching that \textsanskrit{Āḷāra} \textsanskrit{Kālāma} says he has realized with his own insight?’ I quickly realized that teaching with my own insight, and lived having achieved it. 

So\marginnote{14.1} I approached \textsanskrit{Āḷāra} \textsanskrit{Kālāma} and said to him, ‘Reverend \textsanskrit{Kālāma}, have you realized this teaching with your own insight up to this point, and declare having achieved it?’ 

‘I\marginnote{14.3} have, reverend.’ 

‘I\marginnote{14.4} too have realized this teaching with my own insight up to this point, and live having achieved it.’ 

‘We\marginnote{14.5} are fortunate, reverend, so very fortunate to see a venerable such as yourself as one of our spiritual companions! So the teaching that I’ve realized with my own insight, and declare having achieved it, you’ve realized with your own insight, and live having achieved it. The teaching that you’ve realized with your own insight, and live having achieved it, I’ve realized with my own insight, and declare having achieved it. So the teaching that I know, you know, and the teaching you know, I know. I am like you and you are like me. Come now, reverend! We should both lead this community together.’ And that is how my teacher \textsanskrit{Āḷāra} \textsanskrit{Kālāma} placed me, his student, on the same position as him, and honored me with lofty praise. 

Then\marginnote{14.13} it occurred to me, ‘This teaching doesn’t lead to disillusionment, dispassion, cessation, peace, insight, awakening, and extinguishment. It only leads as far as rebirth in the dimension of nothingness.’ Realizing that this teaching was inadequate, I left disappointed. 

I\marginnote{15.1} set out to discover what is skillful, seeking the supreme state of sublime peace. I approached Uddaka, son of \textsanskrit{Rāma}, and said to him, ‘Reverend, I wish to lead the spiritual life in this teaching and training.’ 

Uddaka\marginnote{15.3} replied, ‘Stay, venerable. This teaching is such that a sensible person can soon realize their own tradition with their own insight and live having achieved it.’ 

I\marginnote{15.6} quickly memorized that teaching. So far as lip-recital and oral recitation were concerned, I spoke with knowledge and the authority of the elders. I claimed to know and see, and so did others. 

Then\marginnote{15.8} it occurred to me, ‘It is not solely by mere faith that \textsanskrit{Rāma} declared: “I realize this teaching with my own insight, and live having achieved it.” Surely he meditated knowing and seeing this teaching.’ 

So\marginnote{15.11} I approached Uddaka, son of \textsanskrit{Rāma}, and said to him, ‘Reverend, to what extent did \textsanskrit{Rāma} say he’d realized this teaching with his own insight?’ When I said this, Uddaka, son of \textsanskrit{Rāma}, declared the dimension of neither perception nor non-perception. 

Then\marginnote{15.14} it occurred to me, ‘It’s not just \textsanskrit{Rāma} who had faith, energy, mindfulness, immersion, and wisdom; I too have these things. Why don’t I make an effort to realize the same teaching that \textsanskrit{Rāma} said he had realized with his own insight?’ I quickly realized that teaching with my own insight, and lived having achieved it. 

So\marginnote{15.22} I approached Uddaka, son of \textsanskrit{Rāma}, and said to him, ‘Reverend, had \textsanskrit{Rāma} realized this teaching with his own insight up to this point, and declared having achieved it?’ 

‘He\marginnote{15.24} had, reverend.’ 

‘I\marginnote{15.25} too have realized this teaching with my own insight up to this point, and live having achieved it.’ 

‘We\marginnote{15.26} are fortunate, reverend, so very fortunate to see a venerable such as yourself as one of our spiritual companions! The teaching that \textsanskrit{Rāma} had realized with his own insight, and declared having achieved it, you have realized with your own insight, and live having achieved it. The teaching that you’ve realized with your own insight, and live having achieved it, \textsanskrit{Rāma} had realized with his own insight, and declared having achieved it. So the teaching that \textsanskrit{Rāma} directly knew, you know, and the teaching you know, \textsanskrit{Rāma} directly knew. \textsanskrit{Rāma} was like you and you are like \textsanskrit{Rāma}. Come now, reverend! You should lead this community.’ And that is how my spiritual companion Uddaka, son of \textsanskrit{Rāma}, placed me in the position of a teacher, and honored me with lofty praise. 

Then\marginnote{15.33} it occurred to me, ‘This teaching doesn’t lead to disillusionment, dispassion, cessation, peace, insight, awakening, and extinguishment. It only leads as far as rebirth in the dimension of neither perception nor non-perception.’ Realizing that this teaching was inadequate, I left disappointed. 

I\marginnote{16.1} set out to discover what is skillful, seeking the supreme state of sublime peace. Traveling stage by stage in the Magadhan lands, I arrived at Senanigama near \textsanskrit{Uruvelā}. There I saw a delightful park, a lovely grove with a flowing river that was clean and charming, with smooth banks. And nearby was a village to go for alms. Then it occurred to me, ‘This park is truly delightful, a lovely grove with a flowing river that’s clean and charming, with smooth banks. And nearby there’s a village to go for alms. This is good enough for a gentleman who wishes to put forth effort in meditation.’ So I sat down right there, thinking: ‘This is good enough for meditation.’ 

And\marginnote{17.1} then these three examples, which were neither supernaturally inspired, nor learned before in the past, occurred to me. Suppose there was a green, sappy log, and it was lying in water. Then a person comes along with a drill-stick, thinking to light a fire and produce heat. What do you think, Aggivessana? By drilling the stick against that green, sappy log lying in the water, could they light a fire and produce heat?” 

“No,\marginnote{17.7} Master Gotama. Why not? Because it’s a green, sappy log, and it’s lying in the water. That person will eventually get weary and frustrated.” 

“In\marginnote{17.11} the same way, there are ascetics and brahmins who don’t live withdrawn in body and mind from sensual pleasures. They haven’t internally given up or stilled desire, affection, infatuation, thirst, and passion for sensual pleasures. Regardless of whether or not they feel painful, sharp, severe, acute feelings due to overexertion, they are incapable of knowledge and vision, of supreme awakening. This was the first example that occurred to me. 

Then\marginnote{18.1} a second example occurred to me. Suppose there was a green, sappy log, and it was lying on dry land far from the water. Then a person comes along with a drill-stick, thinking to light a fire and produce heat. What do you think, Aggivessana? By drilling the stick against that green, sappy log on dry land far from water, could they light a fire and produce heat?” 

“No,\marginnote{18.7} Master Gotama. Why not? Because it’s still a green, sappy log, despite the fact that it’s lying on dry land far from water. That person will eventually get weary and frustrated.” 

“In\marginnote{18.11} the same way, there are ascetics and brahmins who live withdrawn in body and mind from sensual pleasures. But they haven’t internally given up or stilled desire, affection, infatuation, thirst, and passion for sensual pleasures. Regardless of whether or not they suffer painful, sharp, severe, acute feelings due to overexertion, they are incapable of knowledge and vision, of supreme awakening. This was the second example that occurred to me. 

Then\marginnote{19.1} a third example occurred to me. Suppose there was a dried up, withered log, and it was lying on dry land far from the water. Then a person comes along with a drill-stick, thinking to light a fire and produce heat. What do you think, Aggivessana? By drilling the stick against that dried up, withered log on dry land far from water, could they light a fire and produce heat?” 

“Yes,\marginnote{19.7} Master Gotama. Why is that? Because it’s a dried up, withered log, and it’s lying on dry land far from water.” 

“In\marginnote{19.10} the same way, there are ascetics and brahmins who live withdrawn in body and mind from sensual pleasures. And they have internally given up and stilled desire, affection, infatuation, thirst, and passion for sensual pleasures. Regardless of whether or not they suffer painful, sharp, severe, acute feelings due to overexertion, they are capable of knowledge and vision, of supreme awakening. This was the third example that occurred to me. These are the three examples, which were neither supernaturally inspired, nor learned before in the past, that occurred to me. 

Then\marginnote{20.1} it occurred to me, ‘Why don’t I, with teeth clenched and tongue pressed against the roof of my mouth, squeeze, squash, and torture mind with mind.’ So that’s what I did, until sweat ran from my armpits. It was like when a strong man grabs a weaker man by the head or throat or shoulder and squeezes, squashes, and tortures them. In the same way, with teeth clenched and tongue pressed against the roof of my mouth, I squeezed, squashed, and tortured mind with mind until sweat ran from my armpits. My energy was roused up and unflagging, and my mindfulness was established and lucid, but my body was disturbed, not tranquil, because I’d pushed too hard with that painful striving. But even such painful feeling did not occupy my mind. 

Then\marginnote{21.1} it occurred to me, ‘Why don’t I practice the breathless absorption?’ So I cut off my breathing through my mouth and nose. But then winds came out my ears making a loud noise, like the puffing of a blacksmith’s bellows. My energy was roused up and unflagging, and my mindfulness was established and lucid, but my body was disturbed, not tranquil, because I’d pushed too hard with that painful striving. But even such painful feeling did not occupy my mind. 

Then\marginnote{22.1} it occurred to me, ‘Why don’t I keep practicing the breathless absorption?’ So I cut off my breathing through my mouth and nose and ears. But then strong winds ground my head, like a strong man was drilling into my head with a sharp point. My energy was roused up and unflagging, and my mindfulness was established and lucid, but my body was disturbed, not tranquil, because I’d pushed too hard with that painful striving. But even such painful feeling did not occupy my mind. 

Then\marginnote{23.1} it occurred to me, ‘Why don’t I keep practicing the breathless absorption?’ So I cut off my breathing through my mouth and nose and ears. But then I got a severe headache, like a strong man was tightening a tough leather strap around my head. My energy was roused up and unflagging, and my mindfulness was established and lucid, but my body was disturbed, not tranquil, because I’d pushed too hard with that painful striving. But even such painful feeling did not occupy my mind. 

Then\marginnote{24.1} it occurred to me, ‘Why don’t I keep practicing the breathless absorption?’ So I cut off my breathing through my mouth and nose and ears. But then strong winds carved up my belly, like a deft butcher or their apprentice was slicing my belly open with a meat cleaver. My energy was roused up and unflagging, and my mindfulness was established and lucid, but my body was disturbed, not tranquil, because I’d pushed too hard with that painful striving. But even such painful feeling did not occupy my mind. 

Then\marginnote{25.1} it occurred to me, ‘Why don’t I keep practicing the breathless absorption?’ So I cut off my breathing through my mouth and nose and ears. But then there was an intense burning in my body, like two strong men grabbing a weaker man by the arms to burn and scorch him on a pit of glowing coals. My energy was roused up and unflagging, and my mindfulness was established and lucid, but my body was disturbed, not tranquil, because I’d pushed too hard with that painful striving. But even such painful feeling did not occupy my mind. 

Then\marginnote{26.1} some deities saw me and said, ‘The ascetic Gotama is dead.’ Others said, ‘He’s not dead, but he’s dying.’ Others said, ‘He’s not dead or dying. The ascetic Gotama is a perfected one, for that is how the perfected ones live.’ 

Then\marginnote{27.1} it occurred to me, ‘Why don’t I practice completely cutting off food?’ But deities came to me and said, ‘Good sir, don’t practice totally cutting off food. If you do, we’ll infuse divine nectar into your pores and you will live on that.’ Then I thought, ‘If I claim to be completely fasting while these deities are infusing divine nectar in my pores, that would be a lie on my part.’ So I dismissed those deities, saying, ‘There’s no need.’ 

Then\marginnote{28.1} it occurred to me, ‘Why don’t I just take a little bit of food each time, a cup of broth made from mung beans, lentils, chickpeas, or green gram.’ So that’s what I did, until my body became extremely emaciated. Due to eating so little, my limbs became like the joints of an eighty-year-old or a corpse, my bottom became like a camel’s hoof, my vertebrae stuck out like beads on a string, and my ribs were as gaunt as the broken-down rafters on an old barn. Due to eating so little, the gleam of my eyes sank deep in their sockets, like the gleam of water sunk deep down a well. Due to eating so little, my scalp shriveled and withered like a green bitter-gourd in the wind and sun. 

Due\marginnote{28.11} to eating so little, the skin of my belly stuck to my backbone, so that when I tried to rub the skin of my belly I grabbed my backbone, and when I tried to rub my backbone I rubbed the skin of my belly. Due to eating so little, when I tried to urinate or defecate I fell face down right there. Due to eating so little, when I tried to relieve my body by rubbing my limbs with my hands, the hair, rotted at its roots, fell out. 

Then\marginnote{29.1} some people saw me and said: ‘The ascetic Gotama is black.’ Some said: ‘He’s not black, he’s brown.’ Some said: ‘He’s neither black nor brown. The ascetic Gotama has tawny skin.’ That’s how far the pure, bright complexion of my skin had been ruined by taking so little food. 

Then\marginnote{30.1} I thought, ‘Whatever ascetics and brahmins have experienced painful, sharp, severe, acute feelings due to overexertion—whether in the past, future, or present—this is as far as it goes, no-one has done more than this. But I have not achieved any superhuman distinction in knowledge and vision worthy of the noble ones by this severe, grueling work. Could there be another path to awakening?’ 

Then\marginnote{31.1} it occurred to me, ‘I recall sitting in the cool shade of the rose-apple tree while my father the Sakyan was off working. Quite secluded from sensual pleasures, secluded from unskillful qualities, I entered and remained in the first absorption, which has the rapture and bliss born of seclusion, while placing the mind and keeping it connected. Could that be the path to awakening?’ 

Stemming\marginnote{31.4} from that memory came the realization: ‘\emph{That} is the path to awakening!’ 

Then\marginnote{32.1} it occurred to me, ‘Why am I afraid of that pleasure, for it has nothing to do with sensual pleasures or unskillful qualities?’ Then I thought, ‘I’m not afraid of that pleasure, for it has nothing to do with sensual pleasures or unskillful qualities.’ 

Then\marginnote{33.1} I thought, ‘I can’t achieve that pleasure with a body so excessively emaciated. Why don’t I eat some solid food, some rice and porridge?’ So I ate some solid food. 

Now\marginnote{33.4} at that time the five mendicants were attending on me, thinking, ‘The ascetic Gotama will tell us of any truth that he realizes.’ But when I ate some solid food, they left disappointed in me, saying, ‘The ascetic Gotama has become indulgent; he has strayed from the struggle and returned to indulgence.’ 

After\marginnote{34.1} eating solid food and gathering my strength, quite secluded from sensual pleasures, secluded from unskillful qualities, I entered and remained in the first absorption, which has the rapture and bliss born of seclusion, while placing the mind and keeping it connected. But even such pleasant feeling did not occupy my mind. 

As\marginnote{35{-}37.1} the placing of the mind and keeping it connected were stilled, I entered and remained in the second absorption, which has the rapture and bliss born of immersion, with internal clarity and confidence, and unified mind, without placing the mind and keeping it connected. But even such pleasant feeling did not occupy my mind. And with the fading away of rapture, I entered and remained in the third absorption, where I meditated with equanimity, mindful and aware, personally experiencing the bliss of which the noble ones declare, ‘Equanimous and mindful, one meditates in bliss.’ But even such pleasant feeling did not occupy my mind. With the giving up of pleasure and pain, and the ending of former happiness and sadness, I entered and remained in the fourth absorption, without pleasure or pain, with pure equanimity and mindfulness. But even such pleasant feeling did not occupy my mind. 

When\marginnote{38.1} my mind had immersed in \textsanskrit{samādhi} like this—purified, bright, flawless, rid of corruptions, pliable, workable, steady, and imperturbable—I extended it toward recollection of past lives. I recollected my many kinds of past lives, with features and details. 

This\marginnote{39.1} was the first knowledge, which I achieved in the first watch of the night. Ignorance was destroyed and knowledge arose; darkness was destroyed and light arose, as happens for a meditator who is diligent, keen, and resolute. But even such pleasant feeling did not occupy my mind. 

When\marginnote{40.1} my mind had immersed in \textsanskrit{samādhi} like this—purified, bright, flawless, rid of corruptions, pliable, workable, steady, and imperturbable—I extended it toward knowledge of the death and rebirth of sentient beings. With clairvoyance that is purified and superhuman, I saw sentient beings passing away and being reborn—inferior and superior, beautiful and ugly, in a good place or a bad place. I understood how sentient beings are reborn according to their deeds. 

This\marginnote{41.1} was the second knowledge, which I achieved in the middle watch of the night. Ignorance was destroyed and knowledge arose; darkness was destroyed and light arose, as happens for a meditator who is diligent, keen, and resolute. But even such pleasant feeling did not occupy my mind. 

When\marginnote{42.1} my mind had immersed in \textsanskrit{samādhi} like this—purified, bright, flawless, rid of corruptions, pliable, workable, steady, and imperturbable—I extended it toward knowledge of the ending of defilements. I truly understood: ‘This is suffering’ … ‘This is the origin of suffering’ … ‘This is the cessation of suffering’ … ‘This is the practice that leads to the cessation of suffering.’ I truly understood: ‘These are defilements’ … ‘This is the origin of defilements’ … ‘This is the cessation of defilements’ … ‘This is the practice that leads to the cessation of defilements.’ 

Knowing\marginnote{43.1} and seeing like this, my mind was freed from the defilements of sensuality, desire to be reborn, and ignorance. When it was freed, I knew it was freed. 

I\marginnote{43.3} understood: ‘Rebirth is ended; the spiritual journey has been completed; what had to be done has been done; there is no return to any state of existence.’ 

This\marginnote{44.1} was the third knowledge, which I achieved in the last watch of the night. Ignorance was destroyed and knowledge arose; darkness was destroyed and light arose, as happens for a meditator who is diligent, keen, and resolute. But even such pleasant feeling did not occupy my mind. 

Aggivessana,\marginnote{45.1} I recall teaching the Dhamma to an assembly of many hundreds, and each person thinks that I am teaching the Dhamma especially for them. But it should not be seen like this. The Realized One teaches others only so that they can understand. When that talk is finished, I still, settle, unify, and immerse my mind in \textsanskrit{samādhi} internally, using the same meditation subject as a foundation of immersion that I used before, which is my usual meditation.” 

“I’d\marginnote{45.7} believe that of Master Gotama, just like a perfected one, a fully awakened Buddha. But do you ever recall sleeping during the day?” 

“I\marginnote{46.1} do recall that in the last month of the summer, I have spread out my outer robe folded in four and lain down in the lion’s posture—on the right side, placing one foot on top of the other—mindful and aware.” 

“Some\marginnote{46.2} ascetics and brahmins call that a deluded abiding.” 

“That’s\marginnote{46.3} not how to define whether someone is deluded or not. But as to how to define whether someone is deluded or not, listen and pay close attention, I will speak.” 

“Yes,\marginnote{46.6} sir,” replied Saccaka. 

The\marginnote{46.7} Buddha said this: 

“Whoever\marginnote{47.1} has not given up the defilements that are corrupting, leading to future lives, hurtful, resulting in suffering and future rebirth, old age, and death is deluded, I say. For it’s not giving up the defilements that makes you deluded. Whoever has given up the defilements that are corrupting, leading to future lives, hurtful, resulting in suffering and future rebirth, old age, and death—is not deluded, I say. For it’s giving up the defilements that makes you not deluded. 

The\marginnote{47.5} Realized One has given up the defilements that are corrupting, leading to future lives, hurtful, resulting in suffering and future rebirth, old age, and death. He has cut them off at the root, made them like a palm stump, obliterated them so they are unable to arise in the future. Just as a palm tree with its crown cut off is incapable of further growth, in the same way, the Realized One has given up the defilements so they are unable to arise in the future.” 

When\marginnote{48.1} he had spoken, Saccaka said to him, “It’s incredible, Master Gotama, it’s amazing! When Master Gotama is repeatedly attacked with inappropriate and intrusive criticism, the complexion of his skin brightens and the color of his face becomes clear, just like a perfected one, a fully awakened Buddha. 

I\marginnote{48.4} recall taking on \textsanskrit{Pūraṇa} Kassapa in debate. He dodged the issue, distracting the discussion with irrelevant points, and displaying annoyance, hate, and bitterness. But when Master Gotama is repeatedly attacked with inappropriate and intrusive criticism, the complexion of his skin brightens and the color of his face becomes clear, just like a perfected one, a fully awakened Buddha. 

I\marginnote{48.7} recall taking on Makkhali \textsanskrit{Gosāla}, Ajita Kesakambala, Pakudha \textsanskrit{Kaccāyana}, \textsanskrit{Sañjaya} \textsanskrit{Belaṭṭhiputta}, and \textsanskrit{Nigaṇṭha} \textsanskrit{Nātaputta} in debate. They all dodged the issue, distracting the discussion with irrelevant points, and displaying annoyance, hate, and bitterness. But when Master Gotama is repeatedly attacked with inappropriate and intrusive criticism, the complexion of his skin brightens and the color of his face becomes clear, just like a perfected one, a fully awakened Buddha. 

Well,\marginnote{48.14} now, Master Gotama, I must go. I have many duties, and much to do.” 

“Please,\marginnote{48.16} Aggivessana, go at your convenience.” 

Then\marginnote{48.17} Saccaka, the son of Jain parents, having approved and agreed with what the Buddha said, got up from his seat and left. 

%
\section*{{\suttatitleacronym MN 37}{\suttatitletranslation The Shorter Discourse on the Ending of Craving }{\suttatitleroot Cūḷataṇhāsaṅkhayasutta}}
\addcontentsline{toc}{section}{\tocacronym{MN 37} \toctranslation{The Shorter Discourse on the Ending of Craving } \tocroot{Cūḷataṇhāsaṅkhayasutta}}
\markboth{The Shorter Discourse on the Ending of Craving }{Cūḷataṇhāsaṅkhayasutta}
\extramarks{MN 37}{MN 37}

\scevam{So\marginnote{1.1} I have heard. }At one time the Buddha was staying near \textsanskrit{Sāvatthī} in the Eastern Monastery, the stilt longhouse of \textsanskrit{Migāra}’s mother. 

And\marginnote{2.1} then Sakka, lord of gods, went up to the Buddha, bowed, stood to one side, and said to him: 

“Sir,\marginnote{2.2} how do you briefly define a mendicant who is freed through the ending of craving, who has reached the ultimate end, the ultimate sanctuary, the ultimate spiritual life, the ultimate goal, and is best among gods and humans?” 

“Lord\marginnote{3.1} of Gods, it’s when a mendicant has heard: ‘Nothing is worth insisting on.’ When a mendicant has heard that nothing is worth insisting on, they directly know all things. Directly knowing all things, they completely understand all things. Having completely understood all things, when they experience any kind of feeling—pleasant, unpleasant, or neutral—they meditate observing impermanence, dispassion, cessation, and letting go in those feelings. Meditating in this way, they don’t grasp at anything in the world. Not grasping, they’re not anxious. Not being anxious, they personally become extinguished. They understand: ‘Rebirth is ended, the spiritual journey has been completed, what had to be done has been done, there is no return to any state of existence.’ That’s how I briefly define a mendicant who is freed through the ending of craving, who has reached the ultimate end, the ultimate sanctuary, the ultimate spiritual life, the ultimate goal, and is best among gods and humans.” 

Then\marginnote{4.1} Sakka, lord of gods, having approved and agreed with what the Buddha said, bowed and respectfully circled the Buddha, keeping him on his right, before vanishing right there. 

Now\marginnote{5.1} at that time Venerable \textsanskrit{Mahāmoggallāna} was sitting not far from the Buddha. He thought, “Did that spirit comprehend what the Buddha said when he agreed with him, or not? Why don’t I find out?” 

And\marginnote{6.1} then Venerable \textsanskrit{Mahāmoggallāna}, as easily as a strong person would extend or contract their arm, vanished from the Eastern Monastery and reappeared among the gods of the Thirty-Three. Now at that time Sakka was amusing himself in the Single Lotus Park, supplied and provided with a divine orchestra. 

Seeing\marginnote{7.2} \textsanskrit{Mahāmoggallāna} coming off in the distance, he dismissed the orchestra, approached \textsanskrit{Mahāmoggallāna}, and said, “Come, my good \textsanskrit{Moggallāna}! Welcome, good sir! It’s been a long time since you took the opportunity to come here. Sit, my good \textsanskrit{Moggallāna}, this seat is for you.” \textsanskrit{Mahāmoggallāna} sat down on the seat spread out, while Sakka took a low seat and sat to one side. 

\textsanskrit{Mahāmoggallāna}\marginnote{7.9} said to him, “Kosiya, how did the Buddha briefly explain freedom through the ending of craving? Please share this talk with me so that I can also get to hear it.” 

“My\marginnote{8.3} good \textsanskrit{Moggallāna}, I have many duties, and much to do, not only for myself, but also for the Gods of the Thirty-Three. Besides, I quickly forget even things I’ve properly heard, learned, attended, and memorized. Once upon a time, a battle was fought between the gods and the demons. In that battle the gods won and the demons lost. When I returned from that battle as a conqueror, I created the Palace of Victory. The Palace of Victory has a hundred towers. Each tower has seven hundred chambers. Each chamber has seven nymphs. Each nymph has seven maids. Would you like to see the lovely Palace of Victory?” \textsanskrit{Mahāmoggallāna} consented in silence. 

Then,\marginnote{9.1} putting Venerable \textsanskrit{Mahāmoggallāna} in front, Sakka, lord of gods, and \textsanskrit{Vessavaṇa}, the Great King, went to the Palace of Victory. When they saw \textsanskrit{Moggallāna} coming off in the distance, Sakka’s maids, being prudent and discreet, each went to her own bedroom. They were just like a daughter-in-law who is prudent and discreet when they see their father-in-law. 

Then\marginnote{10.3} Sakka and \textsanskrit{Vessavaṇa} encouraged \textsanskrit{Moggallāna} to wander and explore the palace, saying, “See, in the palace, my good \textsanskrit{Moggallāna}, this lovely thing! And that lovely thing!” 

“That\marginnote{10.6} looks nice for Venerable Kosiya, just like for someone who has made merit in the past. Humans, when they see something lovely, also say: ‘It looks nice enough for the Gods of the Thirty-Three!’ That looks nice for Venerable Kosiya, just like for someone who has made merit in the past.” 

Then\marginnote{11.1} \textsanskrit{Moggallāna} thought, “This spirit lives much too negligently. Why don’t I stir up a sense of urgency in him?” 

Then\marginnote{11.4} \textsanskrit{Moggallāna} used his psychic power to make the Palace of Victory shake and rock and tremble with his big toe. Then Sakka, \textsanskrit{Vessavaṇa}, and the Gods of the Thirty-Three, their minds full of wonder and amazement, thought, “It’s incredible, it’s amazing! The ascetic has such power and might that he makes the god’s home shake and rock and tremble with his big toe!” 

Knowing\marginnote{12.1} that Sakka was shocked and awestruck, \textsanskrit{Moggallāna} said to him, “Kosiya, how did the Buddha briefly explain freedom through the ending of craving? Please share this talk with me so that I can also get to hear it.” 

“My\marginnote{12.4} dear \textsanskrit{Moggallāna}, I approached the Buddha, bowed, stood to one side, and said to him, ‘Sir, how do you briefly define a mendicant who is freed with the ending of craving, who has reached the ultimate end, the ultimate sanctuary, the ultimate spiritual life, the ultimate goal, and is best among gods and humans?’ 

When\marginnote{13.1} I had spoken the Buddha said to me: ‘Lord of Gods, it’s when a mendicant has heard: “Nothing is worth insisting on” When a mendicant has heard that nothing is worth insisting on, they directly know all things. Directly knowing all things, they completely understand all things. Having completely understood all things, when they experience any kind of feeling—pleasant, unpleasant, or neutral—they meditate observing impermanence, dispassion, cessation, and letting go in those feelings. Meditating in this way, they don’t grasp at anything in the world. Not grasping, they’re not anxious. Not being anxious, they personally become extinguished. They understand: “Rebirth is ended, the spiritual journey has been completed, what had to be done has been done, there is no return to any state of existence.” That’s how I briefly define a mendicant who is freed through the ending of craving, who has reached the ultimate end, the ultimate sanctuary, the ultimate spiritual life, the ultimate goal, and is best among gods and humans.’ That’s how the Buddha briefly explained freedom through the ending of craving to me.” 

\textsanskrit{Moggallāna}\marginnote{14.1} approved and agreed with what Sakka said. As easily as a strong person would extend or contract their arm, he vanished from among the Gods of the Thirty-Three and reappeared in the Eastern Monastery. 

Soon\marginnote{14.2} after \textsanskrit{Moggallāna} left, Sakka’s maids said to him, “Good sir, was that the Blessed One, your Teacher?” 

“No,\marginnote{14.4} it was not. That was my spiritual companion Venerable \textsanskrit{Mahāmoggallāna}.” 

“You’re\marginnote{14.6} fortunate, good sir, so very fortunate, to have a spiritual companion of such power and might! We can’t believe that’s not the Blessed One, your Teacher!” 

Then\marginnote{15.1} \textsanskrit{Mahāmoggallāna} went up to the Buddha, bowed, sat down to one side, and said to him, “Sir, do you recall briefly explaining freedom through the ending of craving to a certain well-known and illustrious spirit?” 

“I\marginnote{15.3} do, \textsanskrit{Moggallāna}.” And the Buddha retold all that happened when Sakka came to visit him, adding: 

“That’s\marginnote{15.5} how I recall briefly explaining freedom through the ending of craving to Sakka, lord of gods.” 

That\marginnote{15.18} is what the Buddha said. Satisfied, Venerable \textsanskrit{Mahāmoggallāna} was happy with what the Buddha said. 

%
\section*{{\suttatitleacronym MN 38}{\suttatitletranslation The Longer Discourse on the Ending of Craving }{\suttatitleroot Mahātaṇhāsaṅkhayasutta}}
\addcontentsline{toc}{section}{\tocacronym{MN 38} \toctranslation{The Longer Discourse on the Ending of Craving } \tocroot{Mahātaṇhāsaṅkhayasutta}}
\markboth{The Longer Discourse on the Ending of Craving }{Mahātaṇhāsaṅkhayasutta}
\extramarks{MN 38}{MN 38}

\scevam{So\marginnote{1.1} I have heard. }At one time the Buddha was staying near \textsanskrit{Sāvatthī} in Jeta’s Grove, \textsanskrit{Anāthapiṇḍika}’s monastery. 

Now\marginnote{2.1} at that time a mendicant called \textsanskrit{Sāti}, the fisherman’s son, had the following harmful misconception: “As I understand the Buddha’s teachings, it is this very same consciousness that roams and transmigrates, not another.” 

Several\marginnote{3.1} mendicants heard about this. They went up to \textsanskrit{Sāti} and said to him, “Is it really true, Reverend \textsanskrit{Sāti}, that you have such a harmful misconception: ‘As I understand the Buddha’s teachings, it is this very same consciousness that roams and transmigrates, not another’?” 

“Absolutely,\marginnote{3.7} reverends. As I understand the Buddha’s teachings, it is this very same consciousness that roams and transmigrates, not another.” 

Then,\marginnote{3.8} wishing to dissuade \textsanskrit{Sāti} from his view, the mendicants pursued, pressed, and grilled him, “Don’t say that, \textsanskrit{Sāti}! Don’t misrepresent the Buddha, for misrepresentation of the Buddha is not good. And the Buddha would not say that. In many ways the Buddha has said that consciousness is dependently originated, since without a cause, consciousness does not come to be.” 

But\marginnote{3.11} even though the mendicants pressed him in this way, \textsanskrit{Sāti} obstinately stuck to his misconception and insisted on stating it. 

When\marginnote{4.1} they weren’t able to dissuade \textsanskrit{Sāti} from his view, the mendicants went to the Buddha, bowed, sat down to one side, and told him what had happened. 

So\marginnote{5.1} the Buddha addressed a certain monk, “Please, monk, in my name tell the mendicant \textsanskrit{Sāti} that the teacher summons him.” 

“Yes,\marginnote{5.4} sir,” that monk replied. He went to \textsanskrit{Sāti} and said to him, “Reverend \textsanskrit{Sāti}, the teacher summons you.” 

“Yes,\marginnote{5.6} reverend,” \textsanskrit{Sāti} replied. He went to the Buddha, bowed, and sat down to one side. The Buddha said to him, “Is it really true, \textsanskrit{Sāti}, that you have such a harmful misconception: ‘As I understand the Buddha’s teachings, it is this very same consciousness that roams and transmigrates, not another’?” 

“Absolutely,\marginnote{5.9} sir. As I understand the Buddha’s teachings, it is this very same consciousness that roams and transmigrates, not another.” 

“\textsanskrit{Sāti},\marginnote{5.10} what is that consciousness?” 

“Sir,\marginnote{5.11} it is he who speaks and feels and experiences the results of good and bad deeds in all the different realms.” 

“Silly\marginnote{5.12} man, who on earth have you ever known me to teach in that way? Haven’t I said in many ways that consciousness is dependently originated, since consciousness does not arise without a cause? But still you misrepresent me by your wrong grasp, harm yourself, and make much bad karma. This will be for your lasting harm and suffering.” 

Then\marginnote{6.1} the Buddha said to the mendicants, “What do you think, mendicants? Has this mendicant \textsanskrit{Sāti} kindled even a spark of wisdom in this teaching and training?” 

“How\marginnote{6.4} could that be, sir? No, sir.” When this was said, \textsanskrit{Sāti} sat silent, embarrassed, shoulders drooping, downcast, depressed, with nothing to say. 

Knowing\marginnote{6.7} this, the Buddha said, “Silly man, you will be known by your own harmful misconception. I’ll question the mendicants about this.” 

Then\marginnote{7.1} the Buddha said to the mendicants, “Mendicants, do you understand my teachings as \textsanskrit{Sāti} does, when he misrepresents me by his wrong grasp, harms himself, and makes much bad karma?” 

“No,\marginnote{7.3} sir. For in many ways the Buddha has told us that consciousness is dependently originated, since without a cause, consciousness does not come to be.” 

“Good,\marginnote{7.5} good, mendicants! It’s good that you understand my teaching like this. For in many ways I have told you that consciousness is dependently originated, since without a cause, consciousness does not come to be. But still this \textsanskrit{Sāti} misrepresents me by his wrong grasp, harms himself, and makes much bad karma. This will be for his lasting harm and suffering. 

Consciousness\marginnote{8.1} is reckoned according to the specific conditions dependent upon which it arises. Consciousness that arises dependent on the eye and sights is reckoned as eye consciousness. Consciousness that arises dependent on the ear and sounds is reckoned as ear consciousness. Consciousness that arises dependent on the nose and smells is reckoned as nose consciousness. Consciousness that arises dependent on the tongue and tastes is reckoned as tongue consciousness. Consciousness that arises dependent on the body and touches is reckoned as body consciousness. Consciousness that arises dependent on the mind and thoughts is reckoned as mind consciousness. 

It’s\marginnote{8.8} like fire, which is reckoned according to the specific conditions dependent upon which it burns. A fire that burns dependent on logs is reckoned as a log fire. A fire that burns dependent on twigs is reckoned as a twig fire. A fire that burns dependent on grass is reckoned as a grass fire. A fire that burns dependent on cow-dung is reckoned as a cow-dung fire. A fire that burns dependent on husks is reckoned as a husk fire. A fire that burns dependent on rubbish is reckoned as a rubbish fire. 

In\marginnote{8.15} the same way, consciousness is reckoned according to the specific conditions dependent upon which it arises. … 

Mendicants,\marginnote{9.1} do you see that this has come to be?” 

“Yes,\marginnote{9.2} sir.” 

“Do\marginnote{9.3} you see that it originated with that as fuel?” 

“Yes,\marginnote{9.4} sir.” 

“Do\marginnote{9.5} you see that when that fuel ceases, what has come to be is liable to cease?” 

“Yes,\marginnote{9.6} sir.” 

“Does\marginnote{10.1} doubt arise when you’re uncertain whether or not this has come to be?” 

“Yes,\marginnote{10.2} sir.” 

“Does\marginnote{10.3} doubt arise when you’re uncertain whether or not this has originated with that as fuel?” 

“Yes,\marginnote{10.4} sir.” 

“Does\marginnote{10.5} doubt arise when you’re uncertain whether or not when that fuel ceases, what has come to be is liable to cease?” 

“Yes,\marginnote{10.6} sir.” 

“Is\marginnote{11.1} doubt given up in someone who truly sees with right understanding that this has come to be?” 

“Yes,\marginnote{11.2} sir.” 

“Is\marginnote{11.3} doubt given up in someone who truly sees with right understanding that this has originated with that as fuel?” 

“Yes,\marginnote{11.4} sir.” 

“Is\marginnote{11.5} doubt given up in someone who truly sees with right understanding that when that fuel ceases, what has come to be is liable to cease?” 

“Yes,\marginnote{11.6} sir.” 

“Are\marginnote{12.1} you free of doubt as to whether this has come to be?” 

“Yes,\marginnote{12.2} sir.” 

“Are\marginnote{12.3} you free of doubt as to whether this has originated with that as fuel?” 

“Yes,\marginnote{12.4} sir.” 

“Are\marginnote{12.5} you free of doubt as to whether when that fuel ceases, what has come to be is liable to cease?” 

“Yes,\marginnote{12.6} sir.” 

“Have\marginnote{13.1} you truly seen clearly with right understanding that this has come to be?” 

“Yes,\marginnote{13.2} sir.” 

“Have\marginnote{13.3} you truly seen clearly with right understanding that this has originated with that as fuel?” 

“Yes,\marginnote{13.4} sir.” 

“Have\marginnote{13.5} you truly seen clearly with right understanding that when that fuel ceases, what has come to be is liable to cease?” 

“Yes,\marginnote{13.6} sir.” 

“Pure\marginnote{14.1} and bright as this view is, mendicants, if you cherish it, fancy it, treasure it, and treat it as your own, would you be understanding how the Dhamma is similar to a raft: for crossing over, not for holding on?” 

“No,\marginnote{14.2} sir.” 

“Pure\marginnote{14.3} and bright as this view is, mendicants, if you don’t cherish it, fancy it, treasure it, and treat it as your own, would you be understanding how the Dhamma is similar to a raft: for crossing over, not for holding on?” 

“Yes,\marginnote{14.4} sir.” 

“Mendicants,\marginnote{15.1} there are these four fuels. They maintain sentient beings that have been born and help those that are about to be born. What four? Solid food, whether coarse or fine; contact is the second, mental intention the third, and consciousness the fourth. 

What\marginnote{16.1} is the source, origin, birthplace, and inception of these four fuels? Craving. 

And\marginnote{16.3} what is the source of craving? Feeling. 

And\marginnote{16.5} what is the source of feeling? Contact. 

And\marginnote{16.7} what is the source of contact? The six sense fields. 

And\marginnote{16.9} what is the source of the six sense fields? Name and form. 

And\marginnote{16.11} what is the source of name and form? Consciousness. 

And\marginnote{16.13} what is the source of consciousness? Choices. 

And\marginnote{16.15} what is the source of choices? Ignorance. 

So,\marginnote{17.1} ignorance is a condition for choices. Choices are a condition for consciousness. Consciousness is a condition for name and form. Name and form are conditions for the six sense fields. The six sense fields are conditions for contact. Contact is a condition for feeling. Feeling is a condition for craving. Craving is a condition for grasping. Grasping is a condition for continued existence. Continued existence is a condition for rebirth. Rebirth is a condition for old age and death, sorrow, lamentation, pain, sadness, and distress to come to be. That is how this entire mass of suffering originates. 

‘Rebirth\marginnote{18.1} is a condition for old age and death.’ That’s what I said. Is that how you see this or not?” 

“That’s\marginnote{18.3} how we see it.” 

“‘Continued\marginnote{18.6} existence is a condition for rebirth.’ … 

‘Ignorance\marginnote{18.50} is a condition for choices.’ That’s what I said. Is that how you see this or not?” 

“That’s\marginnote{18.52} how we see it.” 

“Good,\marginnote{19.1} mendicants! So both you and I say this. When this exists, that is; due to the arising of this, that arises. That is: Ignorance is a condition for choices. Choices are a condition for consciousness. Consciousness is a condition for name and form. Name and form are conditions for the six sense fields. The six sense fields are conditions for contact. Contact is a condition for feeling. Feeling is a condition for craving. Craving is a condition for grasping. Grasping is a condition for continued existence. Continued existence is a condition for rebirth. Rebirth is a condition for old age and death, sorrow, lamentation, pain, sadness, and distress to come to be. That is how this entire mass of suffering originates. 

When\marginnote{20.1} ignorance fades away and ceases with nothing left over, choices cease. When choices cease, consciousness ceases. When consciousness ceases, name and form cease. When name and form cease, the six sense fields cease. When the six sense fields cease, contact ceases. When contact ceases, feeling ceases. When feeling ceases, craving ceases. When craving ceases, grasping ceases. When grasping ceases, continued existence ceases. When continued existence ceases, rebirth ceases. When rebirth ceases, old age and death, sorrow, lamentation, pain, sadness, and distress cease. That is how this entire mass of suffering ceases. 

‘When\marginnote{21.1} rebirth ceases, old age and death cease.’ That’s what I said. Is that how you see this or not?” 

“That’s\marginnote{21.3} how we see it.” 

‘When\marginnote{21.6} continued existence ceases, rebirth ceases.’ … 

‘When\marginnote{21.51} ignorance ceases, choices cease.’ That’s what I said. Is that how you see this or not?” 

“That’s\marginnote{21.53} how we see it.” 

“Good,\marginnote{22.1} mendicants! So both you and I say this. When this doesn’t exist, that is not; due to the cessation of this, that ceases. That is: When ignorance ceases, choices cease. When choices cease, consciousness ceases. When consciousness ceases, name and form cease. When name and form cease, the six sense fields cease. When the six sense fields cease, contact ceases. When contact ceases, feeling ceases. When feeling ceases, craving ceases. When craving ceases, grasping ceases. When grasping ceases, continued existence ceases. When continued existence ceases, rebirth ceases. When rebirth ceases, old age and death, sorrow, lamentation, pain, sadness, and distress cease. That is how this entire mass of suffering ceases. 

Knowing\marginnote{23.1} and seeing in this way, mendicants, would you turn back to the past, thinking, ‘Did we exist in the past? Did we not exist in the past? What were we in the past? How were we in the past? After being what, what did we become in the past?’?” 

“No,\marginnote{23.3} sir.” 

“Knowing\marginnote{23.4} and seeing in this way, mendicants, would you turn forward to the future, thinking, ‘Will we exist in the future? Will we not exist in the future? What will we be in the future? How will we be in the future? After being what, what will we become in the future?’?” 

“No,\marginnote{23.6} sir.” 

“Knowing\marginnote{24.1} and seeing in this way, mendicants, would you be undecided about the present, thinking, ‘Am I? Am I not? What am I? How am I? This sentient being—where did it come from? And where will it go?’?” 

“No,\marginnote{24.3} sir.” 

“Knowing\marginnote{24.4} and seeing in this way, would you say, ‘Our teacher is respected. We speak like this out of respect for our teacher.’?” 

“No,\marginnote{24.6} sir.” 

“Knowing\marginnote{24.7} and seeing in this way, would you say, ‘Our ascetic says this. It’s only because of him that we say this’?” 

“No,\marginnote{24.9} sir.” 

“Knowing\marginnote{24.10} and seeing in this way, would you acknowledge another teacher?” 

“No,\marginnote{24.11} sir.” 

“Knowing\marginnote{24.12} and seeing in this way, would you believe that the observances and noisy, superstitious rites of the various ascetics and brahmins are the most important things?” 

“No,\marginnote{24.13} sir.” 

“Aren’t\marginnote{24.14} you speaking only of what you have known and seen and realized for yourselves?” 

“Yes,\marginnote{24.15} sir.” 

“Good,\marginnote{25.1} mendicants! You have been guided by me with this teaching that’s visible in this very life, immediately effective, inviting inspection, relevant, so that sensible people can know it for themselves. For when I said that this teaching is visible in this very life, immediately effective, inviting inspection, relevant, so that sensible people can know it for themselves, this is what I was referring to. 

Mendicants,\marginnote{26.1} when three things come together an embryo is conceived. In a case where the mother and father come together, but the mother is not in the fertile part of her menstrual cycle, and the virile spirit is not potent, the embryo is not conceived. In a case where the mother and father come together, the mother is in the fertile part of her menstrual cycle, but the virile spirit is not potent, the embryo is not conceived. But when these three things come together—the mother and father come together, the mother is in the fertile part of her menstrual cycle, and the virile spirit is potent—an embryo is conceived. 

The\marginnote{27.1} mother nurtures the embryo in her womb for nine or ten months at great risk to her heavy burden. When nine or ten months have passed, the mother gives birth at great risk to her heavy burden. When the infant is born she nourishes it with her own blood. For mother’s milk is regarded as blood in the training of the Noble One. 

That\marginnote{28.1} boy grows up and his faculties mature. He accordingly plays childish games such as toy plows, tipcat, somersaults, pinwheels, toy measures, toy carts, and toy bows. 

That\marginnote{29.1} boy grows up and his faculties mature further. He accordingly amuses himself, supplied and provided with the five kinds of sensual stimulation. Sights known by the eye that are likable, desirable, agreeable, pleasant, sensual, and arousing. 

Sounds\marginnote{29.4} known by the ear … 

Smells\marginnote{29.5} known by the nose … 

Tastes\marginnote{29.6} known by the tongue … 

Touches\marginnote{29.7} known by the body that are likable, desirable, agreeable, pleasant, sensual, and arousing. 

When\marginnote{30.1} they see a sight with their eyes, if it’s pleasant they desire it, but if it’s unpleasant they dislike it. They live with mindfulness of the body unestablished and their heart restricted. And they don’t truly understand the freedom of heart and freedom by wisdom where those arisen bad, unskillful qualities cease without anything left over. 

Being\marginnote{30.3} so full of favoring and opposing, when they experience any kind of feeling—pleasant, unpleasant, or neutral—they approve, welcome, and keep clinging to it. This gives rise to relishing. Relishing feelings is grasping. Their grasping is a condition for continued existence. Continued existence is a condition for rebirth. Rebirth is a condition for old age and death, sorrow, lamentation, pain, sadness, and distress to come to be. That is how this entire mass of suffering originates. 

When\marginnote{30.7} they hear a sound with their ears … 

When\marginnote{30.8} they smell an odor with their nose … 

When\marginnote{30.9} they taste a flavor with their tongue … 

When\marginnote{30.10} they feel a touch with their body … 

When\marginnote{30.11} they know a thought with their mind, if it’s pleasant they desire it, but if it’s unpleasant they dislike it. They live with mindfulness of the body unestablished and their heart restricted. And they don’t truly understand the freedom of heart and freedom by wisdom where those arisen bad, unskillful qualities cease without anything left over. 

Being\marginnote{30.13} so full of favoring and opposing, when they experience any kind of feeling—pleasant, unpleasant, or neutral—they approve, welcome, and keep clinging to it. This gives rise to relishing. Relishing feelings is grasping. Their grasping is a condition for continued existence. Continued existence is a condition for rebirth. Rebirth is a condition for old age and death, sorrow, lamentation, pain, sadness, and distress to come to be. That is how this entire mass of suffering originates. 

But\marginnote{31.1} consider when a Realized One arises in the world, perfected, a fully awakened Buddha, accomplished in knowledge and conduct, holy, knower of the world, supreme guide for those who wish to train, teacher of gods and humans, awakened, blessed. He has realized with his own insight this world—with its gods, \textsanskrit{Māras} and \textsanskrit{Brahmās}, this population with its ascetics and brahmins, gods and humans—and he makes it known to others. He proclaims a teaching that is good in the beginning, good in the middle, and good in the end, with the right meaning and phrasing. He reveals an entirely full and pure spiritual life. 

A\marginnote{32.1} householder hears that teaching, or a householder’s child, or someone reborn in some good family. They gain faith in the Realized One, and reflect, ‘Living in a house is cramped and dirty, but the life of one gone forth is wide open. It’s not easy for someone living at home to lead the spiritual life utterly full and pure, like a polished shell. Why don’t I shave off my hair and beard, dress in ocher robes, and go forth from lay life to homelessness?’ After some time they give up a large or small fortune, and a large or small family circle. They shave off hair and beard, dress in ocher robes, and go forth from the lay life to homelessness. 

Once\marginnote{33.1} they’ve gone forth, they take up the training and livelihood of the mendicants. They give up killing living creatures, renouncing the rod and the sword. They’re scrupulous and kind, living full of compassion for all living beings. 

They\marginnote{33.2} give up stealing. They take only what’s given, and expect only what’s given. They keep themselves clean by not thieving. 

They\marginnote{33.3} give up unchastity. They are celibate, set apart, avoiding the common practice of sex. 

They\marginnote{33.4} give up lying. They speak the truth and stick to the truth. They’re honest and trustworthy, and don’t trick the world with their words. 

They\marginnote{33.5} give up divisive speech. They don’t repeat in one place what they heard in another so as to divide people against each other. Instead, they reconcile those who are divided, supporting unity, delighting in harmony, loving harmony, speaking words that promote harmony. 

They\marginnote{33.6} give up harsh speech. They speak in a way that’s mellow, pleasing to the ear, lovely, going to the heart, polite, likable and agreeable to the people. 

They\marginnote{33.7} give up talking nonsense. Their words are timely, true, and meaningful, in line with the teaching and training. They say things at the right time which are valuable, reasonable, succinct, and beneficial. 

They\marginnote{33.8} avoid injuring plants and seeds. They eat in one part of the day, abstaining from eating at night and food at the wrong time. They avoid dancing, singing, music, and seeing shows. They avoid beautifying and adorning themselves with garlands, perfumes, and makeup. They avoid high and luxurious beds. They avoid receiving gold and money, raw grains, raw meat, women and girls, male and female bondservants, goats and sheep, chickens and pigs, elephants, cows, horses, and mares, and fields and land. They avoid running errands and messages; buying and selling; falsifying weights, metals, or measures; bribery, fraud, cheating, and duplicity; mutilation, murder, abduction, banditry, plunder, and violence. 

They’re\marginnote{34.1} content with robes to look after the body and almsfood to look after the belly. Wherever they go, they set out taking only these things. They’re like a bird: wherever it flies, wings are its only burden. In the same way, a mendicant is content with robes to look after the body and almsfood to look after the belly. Wherever they go, they set out taking only these things. When they have this entire spectrum of noble ethics, they experience a blameless happiness inside themselves. 

When\marginnote{35.1} they see a sight with their eyes, they don’t get caught up in the features and details. If the faculty of sight were left unrestrained, bad unskillful qualities of desire and aversion would become overwhelming. For this reason, they practice restraint, protecting the faculty of sight, and achieving its restraint. 

When\marginnote{35.3} they hear a sound with their ears … 

When\marginnote{35.4} they smell an odor with their nose … 

When\marginnote{35.5} they taste a flavor with their tongue … 

When\marginnote{35.6} they feel a touch with their body … 

When\marginnote{35.7} they know a thought with their mind, they don’t get caught up in the features and details. If the faculty of mind were left unrestrained, bad unskillful qualities of desire and aversion would become overwhelming. For this reason, they practice restraint, protecting the faculty of mind, and achieving its restraint. When they have this noble sense restraint, they experience an unsullied bliss inside themselves. 

They\marginnote{36.1} act with situational awareness when going out and coming back; when looking ahead and aside; when bending and extending the limbs; when bearing the outer robe, bowl and robes; when eating, drinking, chewing, and tasting; when urinating and defecating; when walking, standing, sitting, sleeping, waking, speaking, and keeping silent. 

When\marginnote{37.1} they have this noble spectrum of ethics, this noble sense restraint, and this noble mindfulness and situational awareness, they frequent a secluded lodging—a wilderness, the root of a tree, a hill, a ravine, a mountain cave, a charnel ground, a forest, the open air, a heap of straw. 

After\marginnote{38.1} the meal, they return from almsround, sit down cross-legged with their body straight, and establish mindfulness right there. Giving up desire for the world, they meditate with a heart rid of desire, cleansing the mind of desire. Giving up ill will and malevolence, they meditate with a mind rid of ill will, full of compassion for all living beings, cleansing the mind of ill will. Giving up dullness and drowsiness, they meditate with a mind rid of dullness and drowsiness, perceiving light, mindful and aware, cleansing the mind of dullness and drowsiness. Giving up restlessness and remorse, they meditate without restlessness, their mind peaceful inside, cleansing the mind of restlessness and remorse. Giving up doubt, they meditate having gone beyond doubt, not undecided about skillful qualities, cleansing the mind of doubt. 

They\marginnote{39.1} give up these five hindrances, corruptions of the heart that weaken wisdom. Then, quite secluded from sensual pleasures, secluded from unskillful qualities, they enter and remain in the first absorption, which has the rapture and bliss born of seclusion, while placing the mind and keeping it connected. Furthermore, as the placing of the mind and keeping it connected are stilled, a mendicant enters and remains in the second absorption … third absorption … fourth absorption. 

When\marginnote{40.1} they see a sight with their eyes, if it’s pleasant they don’t desire it, and if it’s unpleasant they don’t dislike it. They live with mindfulness of the body established and a limitless heart. And they truly understand the freedom of heart and freedom by wisdom where those arisen bad, unskillful qualities cease without anything left over. 

Having\marginnote{40.3} given up favoring and opposing, when they experience any kind of feeling—pleasant, unpleasant, or neutral—they don’t approve, welcome, or keep clinging to it. As a result, relishing of feelings ceases. When their relishing ceases, grasping ceases. When grasping ceases, continued existence ceases. When continued existence ceases, rebirth ceases. When rebirth ceases, old age and death, sorrow, lamentation, pain, sadness, and distress cease. That is how this entire mass of suffering ceases. 

When\marginnote{41.1} they hear a sound with their ears … 

When\marginnote{41.2} they smell an odor with their nose … 

When\marginnote{41.3} they taste a flavor with their tongue … 

When\marginnote{41.4} they feel a touch with their body … 

When\marginnote{41.5} they know a thought with their mind, if it’s pleasant they don’t desire it, and if it’s unpleasant they don’t dislike it. They live with mindfulness of the body established and a limitless heart. And they truly understand the freedom of heart and freedom by wisdom where those arisen bad, unskillful qualities cease without anything left over. 

Having\marginnote{41.7} given up favoring and opposing, when they experience any kind of feeling—pleasant, unpleasant, or neutral—they don’t approve, welcome, or keep clinging to it. As a result, relishing of feelings ceases. When their relishing ceases, grasping ceases. When grasping ceases, continued existence ceases. When continued existence ceases, rebirth ceases. When rebirth ceases, old age and death, sorrow, lamentation, pain, sadness, and distress cease. That is how this entire mass of suffering ceases. 

Mendicants,\marginnote{41.11} you should memorize that brief statement on freedom through the ending of craving. But the mendicant \textsanskrit{Sāti}, the fisherman’s son, is caught in a vast net of craving, a tangle of craving.” 

That\marginnote{41.12} is what the Buddha said. Satisfied, the mendicants were happy with what the Buddha said. 

%
\section*{{\suttatitleacronym MN 39}{\suttatitletranslation The Longer Discourse at Assapura }{\suttatitleroot Mahāassapurasutta}}
\addcontentsline{toc}{section}{\tocacronym{MN 39} \toctranslation{The Longer Discourse at Assapura } \tocroot{Mahāassapurasutta}}
\markboth{The Longer Discourse at Assapura }{Mahāassapurasutta}
\extramarks{MN 39}{MN 39}

\scevam{So\marginnote{1.1} I have heard. }At one time the Buddha was staying in the land of the \textsanskrit{Aṅgas}, near the \textsanskrit{Aṅgan} town named Assapura. There the Buddha addressed the mendicants, “Mendicants!” 

“Venerable\marginnote{1.5} sir,” they replied. The Buddha said this: 

“Mendicants,\marginnote{2.1} people label you as ascetics. And when they ask you what you are, you claim to be ascetics. 

Given\marginnote{2.3} this label and this claim, you should train like this: ‘We will undertake and follow the things that make one an ascetic and a brahmin. That way our label will be accurate and our claim correct. Any robes, almsfood, lodgings, and medicines and supplies for the sick that we use will be very fruitful and beneficial for the donor. And our going forth will not be wasted, but will be fruitful and fertile.’ 

And\marginnote{3.1} what are the things that make one an ascetic and a brahmin? You should train like this: ‘We will have conscience and prudence.’ Now, mendicants, you might think, ‘We have conscience and prudence. Just this much is enough. We have achieved the goal of life as an ascetic. There is nothing more to do.’ And you might rest content with just that much. I declare this to you, mendicants, I announce this to you: ‘You who seek to be true ascetics, do not lose sight of the goal of the ascetic life while there is still more to do.’ 

What\marginnote{4.1} more is there to do? You should train like this: ‘Our bodily behavior will be pure, clear, open, neither inconsistent nor secretive. And we won’t glorify ourselves or put others down on account of our pure bodily behavior.’ Now, mendicants, you might think, ‘We have conscience and prudence, and our bodily behavior is pure. Just this much is enough …’ I declare this to you, mendicants, I announce this to you: ‘You who seek to be true ascetics, do not lose sight of the goal of the ascetic life while there is still more to do.’ 

What\marginnote{5.1} more is there to do? You should train like this: ‘Our verbal behavior … mental behavior … livelihood will be pure, clear, open, neither inconsistent nor secretive. And we won’t glorify ourselves or put others down on account of our pure livelihood.’ Now, mendicants, you might think, ‘We have conscience and prudence, our bodily, verbal, and mental behavior is pure, and our livelihood is pure. Just this much is enough. We have achieved the goal of life as an ascetic. There is nothing more to do.’ And you might rest content with just that much. I declare this to you, mendicants, I announce this to you: ‘You who seek to be true ascetics, do not lose sight of the goal of the ascetic life while there is still more to do.’ 

What\marginnote{8.1} more is there to do? You should train yourselves like this: ‘We will restrain our sense doors. When we see a sight with our eyes, we won’t get caught up in the features and details. If the faculty of sight were left unrestrained, bad unskillful qualities of desire and aversion would become overwhelming. For this reason, we will practice restraint, we will protect the faculty of sight, and we will achieve its restraint. When we hear a sound with our ears … When we smell an odor with our nose … When we taste a flavor with our tongue … When we feel a touch with our body … When we know a thought with our mind, we won’t get caught up in the features and details. If the faculty of mind were left unrestrained, bad unskillful qualities of desire and aversion would become overwhelming. For this reason, we will practice restraint, we will protect the faculty of mind, and we will achieve its restraint.’ Now, mendicants, you might think, ‘We have conscience and prudence, our bodily, verbal, and mental behavior is pure, our livelihood is pure, and our sense doors are restrained. Just this much is enough …’ 

What\marginnote{9.1} more is there to do? You should train yourselves like this: ‘We will not eat too much. We will only eat after reflecting properly on our food. We will eat not for fun, indulgence, adornment, or decoration, but only to sustain this body, to avoid harm, and to support spiritual practice. In this way, we shall put an end to old discomfort and not give rise to new discomfort, and we will live blamelessly and at ease.’ Now, mendicants, you might think, ‘We have conscience and prudence, our bodily, verbal, and mental behavior is pure, our livelihood is pure, our sense doors are restrained, and we don’t eat too much. Just this much is enough …’ 

What\marginnote{10.1} more is there to do? You should train yourselves like this: ‘We will be dedicated to wakefulness. When practicing walking and sitting meditation by day, we will purify our mind from obstacles. In the evening, we will continue to practice walking and sitting meditation. In the middle of the night, we will lie down in the lion’s posture—on the right side, placing one foot on top of the other—mindful and aware, and focused on the time of getting up. In the last part of the night, we will get up and continue to practice walking and sitting meditation, purifying our mind from obstacles.’ Now, mendicants, you might think, ‘We have conscience and prudence, our bodily, verbal, and mental behavior is pure, our livelihood is pure, our sense doors are restrained, we don’t eat too much, and we are dedicated to wakefulness. Just this much is enough …’ 

What\marginnote{11.1} more is there to do? You should train yourselves like this: ‘We will have situational awareness and mindfulness. We will act with situational awareness when going out and coming back; when looking ahead and aside; when bending and extending the limbs; when bearing the outer robe, bowl and robes; when eating, drinking, chewing, and tasting; when urinating and defecating; when walking, standing, sitting, sleeping, waking, speaking, and keeping silent.’ Now, mendicants, you might think, ‘We have conscience and prudence, our bodily, verbal, and mental behavior is pure, our livelihood is pure, our sense doors are restrained, we don’t eat too much, we are dedicated to wakefulness, and we have mindfulness and situational awareness. Just this much is enough …’ 

What\marginnote{12.1} more is there to do? Take a mendicant who frequents a secluded lodging—a wilderness, the root of a tree, a hill, a ravine, a mountain cave, a charnel ground, a forest, the open air, a heap of straw. 

After\marginnote{13.1} the meal, they return from almsround, sit down cross-legged with their body straight, and establish mindfulness right there. Giving up desire for the world, they meditate with a heart rid of desire, cleansing the mind of desire. Giving up ill will and malevolence, they meditate with a mind rid of ill will, full of compassion for all living beings, cleansing the mind of ill will. Giving up dullness and drowsiness, they meditate with a mind rid of dullness and drowsiness, perceiving light, mindful and aware, cleansing the mind of dullness and drowsiness. Giving up restlessness and remorse, they meditate without restlessness, their mind peaceful inside, cleansing the mind of restlessness and remorse. Giving up doubt, they meditate having gone beyond doubt, not undecided about skillful qualities, cleansing the mind of doubt. 

Suppose\marginnote{14.1} a man who has gotten into debt were to apply himself to work, and his efforts proved successful. He would pay off the original loan and have enough left over to support his partner. Thinking about this, he’d be filled with joy and happiness. 

Suppose\marginnote{14.8} a person was sick, suffering, and gravely ill. They’d lose their appetite and get physically weak. But after some time they’d recover from that illness, and regain their appetite and their strength. Thinking about this, they’d be filled with joy and happiness. 

Suppose\marginnote{14.13} a person was imprisoned in a jail. But after some time they were released from jail, safe and sound, with no loss of wealth. Thinking about this, they’d be filled with joy and happiness. 

Suppose\marginnote{14.18} a person was a bondservant. They belonged to someone else and were unable to go where they wished. But after some time they’d be freed from servitude and become their own master, an emancipated individual able to go where they wished. Thinking about this, they’d be filled with joy and happiness. 

Suppose\marginnote{14.23} there was a person with wealth and property who was traveling along a desert road. But after some time they crossed over the desert, safe and sound, with no loss of wealth. Thinking about this, they’d be filled with joy and happiness. 

In\marginnote{14.29} the same way, as long as these five hindrances are not given up inside themselves, a mendicant regards them as a debt, a disease, a prison, slavery, and a desert crossing. But when these five hindrances are given up inside themselves, a mendicant regards this as freedom from debt, good health, release from prison, emancipation, and sanctuary. 

They\marginnote{15.1} give up these five hindrances, corruptions of the heart that weaken wisdom. Then, quite secluded from sensual pleasures, secluded from unskillful qualities, they enter and remain in the first absorption, which has the rapture and bliss born of seclusion, while placing the mind and keeping it connected. They drench, steep, fill, and spread their body with rapture and bliss born of seclusion. There’s no part of the body that’s not spread with rapture and bliss born of seclusion. It’s like when a deft bathroom attendant or their apprentice pours bath powder into a bronze dish, sprinkling it little by little with water. They knead it until the ball of bath powder is soaked and saturated with moisture, spread through inside and out; yet no moisture oozes out. 

In\marginnote{15.5} the same way, a mendicant drenches, steeps, fills, and spreads their body with rapture and bliss born of seclusion. There’s no part of the body that’s not spread with rapture and bliss born of seclusion. 

Furthermore,\marginnote{16.1} as the placing of the mind and keeping it connected are stilled, a mendicant enters and remains in the second absorption, which has the rapture and bliss born of immersion, with internal clarity and confidence, and unified mind, without placing the mind and keeping it connected. They drench, steep, fill, and spread their body with rapture and bliss born of immersion. There’s no part of the body that’s not spread with rapture and bliss born of immersion. It’s like a deep lake fed by spring water. There’s no inlet to the east, west, north, or south, and no rainfall to replenish it from time to time. But the stream of cool water welling up in the lake drenches, steeps, fills, and spreads throughout the lake. There’s no part of the lake that’s not spread through with cool water. 

In\marginnote{16.4} the same way, a mendicant drenches, steeps, fills, and spreads their body with rapture and bliss born of immersion. There’s no part of the body that’s not spread with rapture and bliss born of immersion. 

Furthermore,\marginnote{17.1} with the fading away of rapture, a mendicant enters and remains in the third absorption, where they meditate with equanimity, mindful and aware, personally experiencing the bliss of which the noble ones declare, ‘Equanimous and mindful, one meditates in bliss.’ They drench, steep, fill, and spread their body with bliss free of rapture. There’s no part of the body that’s not spread with bliss free of rapture. It’s like a pool with blue water lilies, or pink or white lotuses. Some of them sprout and grow in the water without rising above it, thriving underwater. From the tip to the root they’re drenched, steeped, filled, and soaked with cool water. There’s no part of them that’s not soaked with cool water. 

In\marginnote{17.4} the same way, a mendicant drenches, steeps, fills, and spreads their body with bliss free of rapture. There’s no part of the body that’s not spread with bliss free of rapture. 

Furthermore,\marginnote{18.1} giving up pleasure and pain, and ending former happiness and sadness, a mendicant enters and remains in the fourth absorption, without pleasure or pain, with pure equanimity and mindfulness. They sit spreading their body through with pure bright mind. There’s no part of the body that’s not spread with pure bright mind. It’s like someone sitting wrapped from head to foot with white cloth. There’s no part of the body that’s not spread over with white cloth. 

In\marginnote{18.4} the same way, they sit spreading their body through with pure bright mind. There’s no part of the body that’s not spread with pure bright mind. 

When\marginnote{19.1} their mind has become immersed in \textsanskrit{samādhi} like this—purified, bright, flawless, rid of corruptions, pliable, workable, steady, and imperturbable—they extend it toward recollection of past lives. They recollect many kinds of past lives, with features and details. Suppose a person was to leave their home village and go to another village. From that village they’d go to yet another village. And from that village they’d return to their home village. They’d think: ‘I went from my home village to another village. There I stood like this, sat like that, spoke like this, or kept silent like that. From that village I went to yet another village. There too I stood like this, sat like that, spoke like this, or kept silent like that. And from that village I returned to my home village.’ 

In\marginnote{19.4} the same way, a mendicant recollects their many kinds of past lives, with features and details. 

When\marginnote{20.1} their mind has become immersed in \textsanskrit{samādhi} like this—purified, bright, flawless, rid of corruptions, pliable, workable, steady, and imperturbable—they extend it toward knowledge of the death and rebirth of sentient beings. With clairvoyance that is purified and superhuman, they see sentient beings passing away and being reborn—inferior and superior, beautiful and ugly, in a good place or a bad place. They understand how sentient beings are reborn according to their deeds. Suppose there were two houses with doors. A person with good eyesight standing in between them would see people entering and leaving a house and wandering to and fro. 

In\marginnote{20.4} the same way, with clairvoyance that is purified and superhuman, they see sentient beings passing away and being reborn—inferior and superior, beautiful and ugly, in a good place or a bad place. They understand how sentient beings are reborn according to their deeds. 

When\marginnote{21.1} their mind has become immersed in \textsanskrit{samādhi} like this—purified, bright, flawless, rid of corruptions, pliable, workable, steady, and imperturbable—they extend it toward knowledge of the ending of defilements. They truly understand: ‘This is suffering’ … ‘This is the origin of suffering’ … ‘This is the cessation of suffering’ … ‘This is the practice that leads to the cessation of suffering.’ They truly understand: ‘These are defilements’ … ‘This is the origin of defilements’ … ‘This is the cessation of defilements’ … ‘This is the practice that leads to the cessation of defilements.’ Knowing and seeing like this, their mind is freed from the defilements of sensuality, desire to be reborn, and ignorance. When they’re freed, they know they’re freed. They understand: ‘Rebirth is ended, the spiritual journey has been completed, what had to be done has been done, there is no return to any state of existence.’ 

Suppose\marginnote{21.7} that in a mountain glen there was a lake that was transparent, clear, and unclouded. A person with good eyesight standing on the bank would see the clams and mussels, and pebbles and gravel, and schools of fish swimming about or staying still. They’d think: ‘This lake is transparent, clear, and unclouded. And here are the clams and mussels, and pebbles and gravel, and schools of fish swimming about or staying still.’ 

In\marginnote{21.11} the same way, a mendicant truly understands: ‘This is suffering’ … ‘This is the origin of suffering’ … ‘This is the cessation of suffering’ … ‘This is the practice that leads to the cessation of suffering.’ They understand: ‘… there is no return to any state of existence.’ 

This\marginnote{22.1} mendicant is called an ‘ascetic’, a ‘brahmin’, a ‘bathed initiate’, a ‘knowledge master’, a ‘scholar’, a ‘noble one’, and a ‘perfected one’. 

And\marginnote{23.1} how is a mendicant an ascetic? They have quelled the bad, unskillful qualities that are corrupting, leading to future lives, hurtful, resulting in suffering and future rebirth, old age, and death. That’s how a mendicant is an ascetic. 

And\marginnote{24.1} how is a mendicant a brahmin? They have banished the bad, unskillful qualities. That’s how a mendicant is a brahmin. 

And\marginnote{25.1} how is a mendicant a bathed initiate? They have bathed off the bad, unskillful qualities. That’s how a mendicant is a bathed initiate. 

And\marginnote{26.1} how is a mendicant a knowledge master? They have known the bad, unskillful qualities. That’s how a mendicant is a knowledge master. 

And\marginnote{27.1} how is a mendicant a scholar? They have scoured off the bad, unskillful qualities. That’s how a mendicant is a scholar. 

And\marginnote{28.1} how is a mendicant a noble one? They are far away from the bad, unskillful qualities. That’s how a mendicant is a noble one. 

And\marginnote{29.1} how is a mendicant a perfected one? They are far away from the bad, unskillful qualities that are corrupting, leading to future lives, hurtful, resulting in suffering and future rebirth, old age, and death. That’s how a mendicant is a perfected one.” 

That\marginnote{29.4} is what the Buddha said. Satisfied, the mendicants were happy with what the Buddha said. 

%
\section*{{\suttatitleacronym MN 40}{\suttatitletranslation The Shorter Discourse at Assapura }{\suttatitleroot Cūḷaassapurasutta}}
\addcontentsline{toc}{section}{\tocacronym{MN 40} \toctranslation{The Shorter Discourse at Assapura } \tocroot{Cūḷaassapurasutta}}
\markboth{The Shorter Discourse at Assapura }{Cūḷaassapurasutta}
\extramarks{MN 40}{MN 40}

\scevam{So\marginnote{1.1} I have heard. }At one time the Buddha was staying in the land of the \textsanskrit{Aṅgas}, near the \textsanskrit{Aṅgan} town named Assapura. There the Buddha addressed the mendicants, “Mendicants!” 

“Venerable\marginnote{1.5} sir,” they replied. The Buddha said this: 

“Mendicants,\marginnote{2.1} people label you as ascetics. And when they ask you what you are, you claim to be ascetics. 

Given\marginnote{2.3} this label and this claim, you should train like this: ‘We will practice in the way that is proper for an ascetic. That way our label will be accurate and our claim correct. Any robes, almsfood, lodgings, and medicines and supplies for the sick that we use will be very fruitful and beneficial for the donor. And our going forth will not be wasted, but will be fruitful and fertile.’ 

And\marginnote{3.1} how does a mendicant not practice in the way that is proper for an ascetic? 

There\marginnote{3.2} are some mendicants who have not given up covetousness, ill will, irritability, hostility, disdain, contempt, jealousy, stinginess, deviousness, deceit, bad desires, and wrong view. These stains, defects, and dregs of an ascetic are grounds for rebirth in places of loss, and are experienced in bad places. As long as they have not given these up, they do not practice in the way that is proper for an ascetic, I say. I say that such a mendicant’s going forth may be compared to the kind of weapon called ‘death-dealer’—double-edged, hardened, and keen—covered and wrapped in the outer robe. 

I\marginnote{5.1} say that you don’t deserve the label ‘outer robe wearer’ just because you wear an outer robe. You don’t deserve the label ‘naked ascetic’ just because you go naked. You don’t deserve the label ‘dust and dirt wearer’ just because you’re caked in dust and dirt. You don’t deserve the label ‘water immerser’ just because you immerse yourself in water. You don’t deserve the label ‘tree root dweller’ just because you stay at the root of a tree. You don’t deserve the label ‘open air dweller’ just because you stay in the open air. You don’t deserve the label ‘stander’ just because you continually stand. You don’t deserve the label ‘interval eater’ just because you eat food at set intervals. You don’t deserve the label ‘reciter’ just because you recite scriptures. You don’t deserve the label ‘matted-hair ascetic’ just because you have matted hair. 

Imagine\marginnote{6.1} that just by wearing an outer robe someone with covetousness, ill will, irritability, hostility, disdain, contempt, jealousy, stinginess, deviousness, deceit, bad desires, and wrong view could give up these things. If that were the case, your friends and colleagues, relatives and kin would make you an outer robe wearer as soon as you were born. They’d encourage you: ‘Please, my dear, wear an outer robe! By doing so you will give up covetousness, ill will, irritability, hostility, disdain, contempt, jealousy, stinginess, deviousness, deceit, bad desires, and wrong view.’ But sometimes I see someone with these bad qualities who is an outer robe wearer. That’s why I say that you don’t deserve the label ‘outer robe wearer’ just because you wear an outer robe. 

Imagine\marginnote{6.4} that just by going naked … wearing dust and dirt … immersing in water … staying at the root of a tree … staying in the open air … standing continually … eating at set intervals … reciting scriptures … having matted hair someone with covetousness, ill will, irritability, hostility, disdain, contempt, jealousy, stinginess, deviousness, deceit, bad desires, and wrong view could give up these things. If that were the case, your friends and colleagues, relatives and kin would make you a matted-hair ascetic as soon as you were born. They’d encourage you: ‘Please, my dear, become a matted-hair ascetic! By doing so you will give up covetousness, ill will, irritability, hostility, disdain, contempt, jealousy, stinginess, deviousness, deceit, bad desires, and wrong view.’ But sometimes I see someone with these bad qualities who is a matted-hair ascetic. That’s why I say that you don’t deserve the label ‘matted-hair ascetic’ just because you have matted hair. 

And\marginnote{7.1} how does a mendicant practice in the way that is proper for an ascetic? 

There\marginnote{7.2} are some mendicants who have given up covetousness, ill will, irritability, hostility, disdain, contempt, jealousy, stinginess, deviousness, deceit, bad desires, and wrong view. These stains, defects, and dregs of an ascetic are grounds for rebirth in places of loss, and are experienced in bad places. When they have given these up, they are practicing in the way that is proper for an ascetic, I say. 

They\marginnote{8.1} see themselves purified from all these bad, unskillful qualities. Seeing this, joy springs up. Being joyful, rapture springs up. When the mind is full of rapture, the body becomes tranquil. When the body is tranquil, they feel bliss. And when blissful, the mind becomes immersed in \textsanskrit{samādhi}. 

They\marginnote{9.1} meditate spreading a heart full of love to one direction, and to the second, and to the third, and to the fourth. In the same way above, below, across, everywhere, all around, they spread a heart full of love to the whole world—abundant, expansive, limitless, free of enmity and ill will. 

They\marginnote{10{-}12.1} meditate spreading a heart full of compassion … 

They\marginnote{10{-}12.2} meditate spreading a heart full of rejoicing … 

They\marginnote{10{-}12.3} meditate spreading a heart full of equanimity to one direction, and to the second, and to the third, and to the fourth. In the same way above, below, across, everywhere, all around, they spread a heart full of equanimity to the whole world—abundant, expansive, limitless, free of enmity and ill will. 

Suppose\marginnote{13.1} there was a lotus pond with clear, sweet, cool water, clean, with smooth banks, delightful. Then along comes a person—whether from the east, west, north, or south—struggling in the oppressive heat, weary, thirsty, and parched. No matter what direction they come from, when they arrive at that lotus pond they would alleviate their thirst and heat exhaustion. 

In\marginnote{13.4} the same way, suppose someone has gone forth from the lay life to homelessness—whether from a family of aristocrats, brahmins, merchants, or workers—and has arrived at the teaching and training proclaimed by a Realized One. Having developed love, compassion, rejoicing, and equanimity in this way they gain inner peace. Because of that inner peace they are practicing the way proper for an ascetic, I say. 

And\marginnote{14.1} suppose someone has gone forth from the lay life to homelessness—whether from a family of aristocrats, brahmins, merchants, or workers—and they realize the undefiled freedom of heart and freedom by wisdom in this very life. And they live having realized it with their own insight due to the ending of defilements. They’re an ascetic because of the ending of defilements.” 

That\marginnote{14.4} is what the Buddha said. Satisfied, the mendicants were happy with what the Buddha said. 

%
\addtocontents{toc}{\let\protect\contentsline\protect\nopagecontentsline}
\chapter*{The Lesser Chapter on Pairs }
\addcontentsline{toc}{chapter}{\tocchapterline{The Lesser Chapter on Pairs }}
\addtocontents{toc}{\let\protect\contentsline\protect\oldcontentsline}

%
\section*{{\suttatitleacronym MN 41}{\suttatitletranslation The People of Sālā }{\suttatitleroot Sāleyyakasutta}}
\addcontentsline{toc}{section}{\tocacronym{MN 41} \toctranslation{The People of Sālā } \tocroot{Sāleyyakasutta}}
\markboth{The People of Sālā }{Sāleyyakasutta}
\extramarks{MN 41}{MN 41}

\scevam{So\marginnote{1.1} I have heard. }At one time the Buddha was wandering in the land of the Kosalans together with a large \textsanskrit{Saṅgha} of mendicants when he arrived at a village of the Kosalan brahmins named \textsanskrit{Sālā}. 

The\marginnote{2.1} brahmins and householders of \textsanskrit{Sālā} heard, “It seems the ascetic Gotama—a Sakyan, gone forth from a Sakyan family—while wandering in the land of the Kosalans has arrived at \textsanskrit{Sālā}, together with a large \textsanskrit{Saṅgha} of mendicants. He has this good reputation: ‘That Blessed One is perfected, a fully awakened Buddha, accomplished in knowledge and conduct, holy, knower of the world, supreme guide for those who wish to train, teacher of gods and humans, awakened, blessed.’ He has realized with his own insight this world—with its gods, \textsanskrit{Māras} and \textsanskrit{Brahmās}, this population with its ascetics and brahmins, gods and humans—and he makes it known to others. He proclaims a teaching that is good in the beginning, good in the middle, and good in the end, with the right meaning and phrasing. He reveals an entirely full and pure spiritual life. It’s good to see such perfected ones.” 

Then\marginnote{3.1} the brahmins and householders of \textsanskrit{Sālā} went up to the Buddha. Before sitting down to one side, some bowed, some exchanged greetings and polite conversation, some held up their joined palms toward the Buddha, some announced their name and clan, while some kept silent. Seated to one side they said to the Buddha: 

“What\marginnote{4.1} is the cause, Master Gotama, what is the reason why some sentient beings, when their body breaks up, after death, are reborn in a place of loss, a bad place, the underworld, hell? And what is the cause, Master Gotama, what is the reason why some sentient beings, when their body breaks up, after death, are reborn in a good place, a heavenly realm?” 

“Unprincipled\marginnote{5.1} and immoral conduct is the reason why some sentient beings, when their body breaks up, after death, are reborn in a place of loss, a bad place, the underworld, hell. Principled and moral conduct is the reason why some sentient beings, when their body breaks up, after death, are reborn in a good place, a heavenly realm.” 

“We\marginnote{6.1} don’t understand the detailed meaning of Master Gotama’s brief statement. Master Gotama, please teach us this matter in detail so we can understand the meaning.” 

“Well\marginnote{6.3} then, householders, listen and pay close attention, I will speak.” 

“Yes,\marginnote{6.4} sir,” they replied. The Buddha said this: 

“Householders,\marginnote{7.1} unprincipled and immoral conduct is threefold by way of body, fourfold by way of speech, and threefold by way of mind. 

And\marginnote{8.1} how is unprincipled and immoral conduct threefold by way of body? It’s when a certain person kills living creatures. They’re violent, bloody-handed, a hardened killer, merciless to living beings. 

They\marginnote{8.3} steal. With the intention to commit theft, they take the wealth or belongings of others from village or wilderness. 

They\marginnote{8.4} commit sexual misconduct. They have sexual relations with women who have their mother, father, both mother and father, brother, sister, relatives, or clan as guardian. They have sexual relations with a woman who is protected on principle, or who has a husband, or whose violation is punishable by law, or even one who has been garlanded as a token of betrothal. This is how unprincipled and immoral conduct is threefold by way of body. 

And\marginnote{9.1} how is unprincipled and immoral conduct fourfold by way of speech? It’s when a certain person lies. They’re summoned to a council, an assembly, a family meeting, a guild, or to the royal court, and asked to bear witness: ‘Please, mister, say what you know.’ Not knowing, they say ‘I know.’ Knowing, they say ‘I don’t know.’ Not seeing, they say ‘I see.’ And seeing, they say ‘I don’t see.’ So they deliberately lie for the sake of themselves or another, or for some trivial worldly reason. 

They\marginnote{9.3} speak divisively. They repeat in one place what they heard in another so as to divide people against each other. And so they divide those who are harmonious, supporting division, delighting in division, loving division, speaking words that promote division. 

They\marginnote{9.4} speak harshly. They use the kinds of words that are cruel, nasty, hurtful, offensive, bordering on anger, not leading to immersion. 

They\marginnote{9.5} talk nonsense. Their speech is untimely, and is neither factual nor beneficial. It has nothing to do with the teaching or the training. Their words have no value, and are untimely, unreasonable, rambling, and pointless. This is how unprincipled and immoral conduct is fourfold by way of speech. 

And\marginnote{10.1} how is unprincipled and immoral conduct threefold by way of mind? It's when a certain person is covetous. They covet the wealth and belongings of others: ‘Oh, if only their belongings were mine!’ 

They\marginnote{10.3} have ill will and malicious intentions: ‘May these sentient beings be killed, slaughtered, slain, destroyed, or annihilated!’ 

They\marginnote{10.4} have wrong view. Their perspective is distorted: ‘There’s no meaning in giving, sacrifice, or offerings. There’s no fruit or result of good and bad deeds. There’s no afterlife. There’s no such thing as mother and father, or beings that are reborn spontaneously. And there’s no ascetic or brahmin who is well attained and practiced, and who describes the afterlife after realizing it with their own insight.’ This is how unprincipled and immoral conduct is threefold by way of mind. 

That’s\marginnote{10.7} how unprincipled and immoral conduct is the reason why some sentient beings, when their body breaks up, after death, are reborn in a place of loss, a bad place, the underworld, hell. 

Householders,\marginnote{11.1} principled and moral conduct is threefold by way of body, fourfold by way of speech, and threefold by way of mind. 

And\marginnote{12.1} how is principled and moral conduct threefold by way of body? It’s when a certain person gives up killing living creatures. They renounce the rod and the sword. They’re scrupulous and kind, living full of compassion for all living beings. 

They\marginnote{12.3} give up stealing. They don’t, with the intention to commit theft, take the wealth or belongings of others from village or wilderness. 

They\marginnote{12.4} give up sexual misconduct. They don’t have sexual relations with women who have their mother, father, both mother and father, brother, sister, relatives, or clan as guardian. They don’t have sexual relations with a woman who is protected on principle, or who has a husband, or whose violation is punishable by law, or even one who has been garlanded as a token of betrothal. This is how principled and moral conduct is threefold by way of body. 

And\marginnote{13.1} how is principled and moral conduct fourfold by way of speech? It’s when a certain person gives up lying. They’re summoned to a council, an assembly, a family meeting, a guild, or to the royal court, and asked to bear witness: ‘Please, mister, say what you know.’ Not knowing, they say ‘I don’t know.’ Knowing, they say ‘I know.’ Not seeing, they say ‘I don’t see.’ And seeing, they say ‘I see.’ So they don’t deliberately lie for the sake of themselves or another, or for some trivial worldly reason. 

They\marginnote{13.3} give up divisive speech. They don’t repeat in one place what they heard in another so as to divide people against each other. Instead, they reconcile those who are divided, supporting unity, delighting in harmony, loving harmony, speaking words that promote harmony. 

They\marginnote{13.4} give up harsh speech. They speak in a way that’s mellow, pleasing to the ear, lovely, going to the heart, polite, likable, and agreeable to the people. 

They\marginnote{13.5} give up talking nonsense. Their words are timely, true, and meaningful, in line with the teaching and training. They say things at the right time which are valuable, reasonable, succinct, and beneficial. This is how principled and moral conduct is fourfold by way of speech. 

And\marginnote{14.1} how is principled and moral conduct threefold by way of mind? It's when a certain person is not covetous. They don’t covet the wealth and belongings of others: ‘Oh, if only their belongings were mine!’ 

They\marginnote{14.3} have a kind heart and loving intentions: ‘May these sentient beings live free of enmity and ill will, untroubled and happy!’ 

They\marginnote{14.4} have right view, an undistorted perspective: ‘There is meaning in giving, sacrifice, and offerings. There are fruits and results of good and bad deeds. There is an afterlife. There are such things as mother and father, and beings that are reborn spontaneously. And there are ascetics and brahmins who are well attained and practiced, and who describe the afterlife after realizing it with their own insight.’ This is how principled and moral conduct is threefold by way of mind. 

This\marginnote{14.7} is how principled and moral conduct is the reason why some sentient beings, when their body breaks up, after death, are reborn in a good place, a heavenly realm. 

A\marginnote{15.1} person of principled and moral conduct might wish: ‘If only, when my body breaks up, after death, I would be reborn in the company of well-to-do aristocrats!’ It’s possible that this might happen. Why is that? Because they have principled and moral conduct. 

A\marginnote{16{-}17.1} person of principled and moral conduct might wish: ‘If only, when my body breaks up, after death, I would be reborn in the company of well-to-do brahmins … well-to-do householders … the Gods of the Four Great Kings … the Gods of the Thirty-Three … the Gods of Yama … the Joyful Gods … the Gods Who Love to Create … the Gods Who Control the Creations of Others … the Gods of \textsanskrit{Brahmā}’s Host … the Radiant Gods … the Gods of Limited Radiance … the Gods of Limitless Radiance … the Gods of Streaming Radiance … the Gods of Limited Glory … the Gods of Limitless Glory … the Gods Replete with Glory … the Gods of Abundant Fruit … the Gods of Aviha … the Gods of Atappa … the Gods Fair to See … the Fair Seeing Gods … the Gods of \textsanskrit{Akaniṭṭha} … the gods of the dimension of infinite space … the gods of the dimension of infinite consciousness … the gods of the dimension of nothingness … the gods of the dimension of neither perception nor non-perception.’ It’s possible that this might happen. Why is that? Because they have principled and moral conduct. 

A\marginnote{43.1} person of principled and moral conduct might wish: ‘If only I might realize the undefiled freedom of heart and freedom by wisdom in this very life, and live having realized it with my own insight due to the ending of defilements.’ It’s possible that this might happen. Why is that? Because they have principled and moral conduct.” 

When\marginnote{44.1} he had spoken, the brahmins and householders of \textsanskrit{Sālā} said to the Buddha, “Excellent, Master Gotama! Excellent! As if he were righting the overturned, or revealing the hidden, or pointing out the path to the lost, or lighting a lamp in the dark so people with good eyes can see what’s there, Master Gotama has made the teaching clear in many ways. We go for refuge to Master Gotama, to the teaching, and to the mendicant \textsanskrit{Saṅgha}. From this day forth, may Master Gotama remember us as lay followers who have gone for refuge for life.” 

%
\section*{{\suttatitleacronym MN 42}{\suttatitletranslation The People of Verañja }{\suttatitleroot Verañjakasutta}}
\addcontentsline{toc}{section}{\tocacronym{MN 42} \toctranslation{The People of Verañja } \tocroot{Verañjakasutta}}
\markboth{The People of Verañja }{Verañjakasutta}
\extramarks{MN 42}{MN 42}

\scevam{So\marginnote{1.1} I have heard. }At one time the Buddha was staying near \textsanskrit{Sāvatthī} in Jeta’s Grove, \textsanskrit{Anāthapiṇḍika}’s monastery. 

Now\marginnote{2.1} at that time the brahmins and householders of \textsanskrit{Verañja} were residing in \textsanskrit{Sāvatthī} on some business. The brahmins and householders of \textsanskrit{Verañja} heard: 

“It\marginnote{2.3} seems the ascetic Gotama—a Sakyan, gone forth from a Sakyan family—is staying near \textsanskrit{Sāvatthī} in Jeta’s Grove, \textsanskrit{Anāthapiṇḍika}’s monastery. He has this good reputation …” … 

“Householders,\marginnote{7.1} a person of unprincipled and immoral conduct is threefold by way of body, fourfold by way of speech, and threefold by way of mind. …” … 

(The\marginnote{8.1} remainder of this discourse is identical with MN 41.) 

%
\section*{{\suttatitleacronym MN 43}{\suttatitletranslation The Great Classification }{\suttatitleroot Mahāvedallasutta}}
\addcontentsline{toc}{section}{\tocacronym{MN 43} \toctranslation{The Great Classification } \tocroot{Mahāvedallasutta}}
\markboth{The Great Classification }{Mahāvedallasutta}
\extramarks{MN 43}{MN 43}

\scevam{So\marginnote{1.1} I have heard. }At one time the Buddha was staying near \textsanskrit{Sāvatthī} in Jeta’s Grove, \textsanskrit{Anāthapiṇḍika}’s monastery. 

Then\marginnote{1.3} in the late afternoon, Venerable \textsanskrit{Mahākoṭṭhita} came out of retreat, went to Venerable \textsanskrit{Sāriputta}, and exchanged greetings with him. When the greetings and polite conversation were over, he sat down to one side and said to \textsanskrit{Sāriputta}: 

“Reverend,\marginnote{2.1} they speak of ‘a witless person’. How is a witless person defined?” 

“Reverend,\marginnote{2.3} they’re called witless because they don’t understand. And what don’t they understand? They don’t understand: ‘This is suffering’ … ‘This is the origin of suffering’ … ‘This is the cessation of suffering’ … ‘This is the practice that leads to the cessation of suffering.’ They’re called witless because they don’t understand.” 

Saying\marginnote{2.7} “Good, reverend,” \textsanskrit{Mahākoṭṭhita} approved and agreed with what \textsanskrit{Sāriputta} said. Then he asked another question: 

“They\marginnote{3.1} speak of ‘a wise person’. How is a wise person defined?” 

“They’re\marginnote{3.3} called wise because they understand. And what do they understand? They understand: ‘This is suffering’ … ‘This is the origin of suffering’ … ‘This is the cessation of suffering’ … ‘This is the practice that leads to the cessation of suffering.’ They’re called wise because they understand.” 

“They\marginnote{4.1} speak of ‘consciousness’. How is consciousness defined?” 

“It’s\marginnote{4.3} called consciousness because it cognizes. And what does it cognize? It cognizes ‘pleasure’ and ‘pain’ and ‘neutral’. It’s called consciousness because it cognizes.” 

“Wisdom\marginnote{5.1} and consciousness—are these things mixed or separate? And can we completely dissect them so as to describe the difference between them?” 

“Wisdom\marginnote{5.4} and consciousness—these things are mixed, not separate. And you can never completely dissect them so as to describe the difference between them. For you understand what you cognize, and you cognize what you understand. That’s why these things are mixed, not separate. And you can never completely dissect them so as to describe the difference between them.” 

“Wisdom\marginnote{6.1} and consciousness—what is the difference between these things that are mixed, not separate?” 

“The\marginnote{6.3} difference between these things is that wisdom should be developed, while consciousness should be completely understood.” 

“They\marginnote{7.1} speak of this thing called ‘feeling’. How is feeling defined?” 

“It’s\marginnote{7.3} called feeling because it feels. And what does it feel? It feels pleasure, pain, and neutral. It’s called feeling because it feels.” 

“They\marginnote{8.1} speak of this thing called ‘perception’. How is perception defined?” 

“It’s\marginnote{8.3} called perception because it perceives. And what does it perceive? It perceives blue, yellow, red, and white. It’s called perception because it perceives.” 

“Feeling,\marginnote{9.1} perception, and consciousness—are these things mixed or separate? And can we completely dissect them so as to describe the difference between them?” 

“Feeling,\marginnote{9.4} perception, and consciousness—these things are mixed, not separate. And you can never completely dissect them so as to describe the difference between them. For you perceive what you feel, and you cognize what you perceive. That’s why these things are mixed, not separate. And you can never completely dissect them so as to describe the difference between them.” 

“What\marginnote{10.1} can be known by purified mind consciousness released from the five senses?” 

“Aware\marginnote{10.2} that ‘space is infinite’ it can know the dimension of infinite space. Aware that ‘consciousness is infinite’ it can know the dimension of infinite consciousness. Aware that ‘there is nothing at all’ it can know the dimension of nothingness.” 

“How\marginnote{11.1} do you understand something that can be known?” 

“You\marginnote{11.2} understand something that can be known with the eye of wisdom.” 

“What\marginnote{12.1} is the purpose of wisdom?” 

“The\marginnote{12.2} purpose of wisdom is direct knowledge, complete understanding, and giving up.” 

“How\marginnote{13.1} many conditions are there for the arising of right view?” 

“There\marginnote{13.2} are two conditions for the arising of right view: the words of another and proper attention. These are the two conditions for the arising of right view.” 

“When\marginnote{14.1} right view is supported by how many factors does it have freedom of heart and freedom by wisdom as its fruit and benefit?” 

“When\marginnote{14.2} right view is supported by five factors it has freedom of heart and freedom by wisdom as its fruit and benefit. It’s when right view is supported by ethics, learning, discussion, serenity, and discernment. When right view is supported by these five factors it has freedom of heart and freedom by wisdom as its fruit and benefit.” 

“How\marginnote{15.1} many states of existence are there?” 

“Reverend,\marginnote{15.2} there are these three states of existence. Existence in the sensual realm, the realm of luminous form, and the formless realm.” 

“But\marginnote{16.1} how is there rebirth into a new state of existence in the future?” 

“It’s\marginnote{16.2} because of sentient beings—shrouded by ignorance and fettered by craving—chasing pleasure in various realms. That’s how there is rebirth into a new state of existence in the future.” 

“But\marginnote{17.1} how is there no rebirth into a new state of existence in the future?” 

“It’s\marginnote{17.2} when ignorance fades away, knowledge arises, and craving ceases. That’s how there is no rebirth into a new state of existence in the future.” 

“But\marginnote{18.1} what, reverend, is the first absorption?” 

“Reverend,\marginnote{18.2} it’s when a mendicant, quite secluded from sensual pleasures, secluded from unskillful qualities, enters and remains in the first absorption, which has the rapture and bliss born of seclusion, while placing the mind and keeping it connected. This is called the first absorption.” 

“But\marginnote{19.1} how many factors does the first absorption have?” 

“The\marginnote{19.2} first absorption has five factors. When a mendicant has entered the first absorption, placing the mind, keeping it connected, rapture, bliss, and unification of mind are present. That’s how the first absorption has five factors.” 

“But\marginnote{20.1} how many factors has the first absorption given up and how many does it possess?” 

“The\marginnote{20.2} first absorption has given up five factors and possesses five factors. When a mendicant has entered the first absorption, sensual desire, ill will, dullness and drowsiness, restlessness and remorse, and doubt are given up. Placing the mind, keeping it connected, rapture, bliss, and unification of mind are present. That’s how the first absorption has given up five factors and possesses five factors.” 

“Reverend,\marginnote{21.1} these five faculties have different scopes and different ranges, and don’t experience each others’ scope and range. That is, the faculties of the eye, ear, nose, tongue, and body. What do these five faculties, with their different scopes and ranges, have recourse to? What experiences their scopes and ranges?” 

“These\marginnote{21.4} five faculties, with their different scopes and ranges, have recourse to the mind. And the mind experiences their scopes and ranges.” 

“These\marginnote{22.1} five faculties depend on what to continue?” 

“These\marginnote{22.4} five faculties depend on life to continue.” 

“But\marginnote{22.7} what does life depend on to continue?” 

“Life\marginnote{22.8} depends on warmth to continue.” 

“But\marginnote{22.9} what does warmth depend on to continue?” 

“Warmth\marginnote{22.10} depends on life to continue.” 

“Just\marginnote{22.11} now I understood you to say: ‘Life depends on warmth to continue.’ But I also understood you to say: ‘Warmth depends on life to continue.’ How then should we see the meaning of this statement?” 

“Well\marginnote{22.16} then, reverend, I shall give you a simile. For by means of a simile some sensible people understand the meaning of what is said. Suppose there was an oil lamp burning. The light appears dependent on the flame, and the flame appears dependent on the light. In the same way, life depends on warmth to continue, and warmth depends on life to continue.” 

“Are\marginnote{23.1} the life forces the same things as the phenomena that are felt? Or are they different things?” 

“The\marginnote{23.2} life forces are not the same things as the phenomena that are felt. For if the life forces and the phenomena that are felt were the same things, a mendicant who had attained the cessation of perception and feeling would not emerge from it. But because the life forces and the phenomena that are felt are different things, a mendicant who has attained the cessation of perception and feeling can emerge from it.” 

“How\marginnote{24.1} many things must this body lose before it lies forsaken, tossed aside like an insentient log?” 

“This\marginnote{24.2} body must lose three things before it lies forsaken, tossed aside like an insentient log: vitality, warmth, and consciousness.” 

“What’s\marginnote{25.1} the difference between someone who has passed away and a mendicant who has attained the cessation of perception and feeling?” 

“When\marginnote{25.2} someone dies, their physical, verbal, and mental processes have ceased and stilled; their vitality is spent; their warmth is dissipated; and their faculties have disintegrated. When a mendicant has attained the cessation of perception and feeling, their physical, verbal, and mental processes have ceased and stilled. But their vitality is not spent; their warmth is not dissipated; and their faculties are very clear. That’s the difference between someone who has passed away and a mendicant who has attained the cessation of perception and feeling.” 

“How\marginnote{26.1} many conditions are necessary to attain the neutral release of the heart?” 

“Four\marginnote{26.2} conditions are necessary to attain the neutral release of the heart. Giving up pleasure and pain, and ending former happiness and sadness, a mendicant enters and remains in the fourth absorption, without pleasure or pain, with pure equanimity and mindfulness. These four conditions are necessary to attain the neutral release of the heart.” 

“How\marginnote{27.1} many conditions are necessary to attain the signless release of the heart?” 

“Two\marginnote{27.2} conditions are necessary to attain the signless release of the heart: not focusing on any signs, and focusing on the signless. These two conditions are necessary to attain the signless release of the heart.” 

“How\marginnote{27.5} many conditions are necessary to remain in the signless release of the heart?” 

“Three\marginnote{28.1} conditions are necessary to remain in the signless release of the heart: not focusing on any signs, focusing on the signless, and a previous determination. These three conditions are necessary to remain in the signless release of the heart.” 

“How\marginnote{29.1} many conditions are necessary to emerge from the signless release of the heart?” 

“Two\marginnote{29.2} conditions are necessary to emerge from the signless release of the heart: focusing on all signs, and not focusing on the signless. These two conditions are necessary to emerge from the signless release of the heart.” 

“The\marginnote{29.5} limitless release of the heart, and the release of the heart through nothingness, and the release of the heart through emptiness, and the signless release of the heart: do these things differ in both meaning and phrasing? Or do they mean the same thing, and differ only in the phrasing?” 

“There\marginnote{30.1} is a way in which these things differ in both meaning and phrasing. But there’s also a way in which they mean the same thing, and differ only in the phrasing. 

And\marginnote{31.1} what’s the way in which these things differ in both meaning and phrasing? 

Firstly,\marginnote{31.2} a mendicant meditates spreading a heart full of love to one direction, and to the second, and to the third, and to the fourth. In the same way above, below, across, everywhere, all around, they spread a heart full of love to the whole world—abundant, expansive, limitless, free of enmity and ill will. They meditate spreading a heart full of compassion … They meditate spreading a heart full of rejoicing … They meditate spreading a heart full of equanimity to one direction, and to the second, and to the third, and to the fourth. In the same way above, below, across, everywhere, all around, they spread a heart full of equanimity to the whole world—abundant, expansive, limitless, free of enmity and ill will. This is called the limitless release of the heart. 

And\marginnote{32.1} what is the release of the heart through nothingness? It’s when a mendicant, going totally beyond the dimension of infinite consciousness, aware that ‘there is nothing at all’, enters and remains in the dimension of nothingness. This is called the heart’s release through nothingness. 

And\marginnote{33.1} what is the release of the heart through emptiness? It’s when a mendicant has gone to a wilderness, or to the root of a tree, or to an empty hut, and reflects like this: ‘This is empty of a self or what belongs to a self.’ This is called the release of the heart through emptiness. 

And\marginnote{34.1} what is the signless release of the heart? It’s when a mendicant, not focusing on any signs, enters and remains in the signless immersion of the heart. This is called the signless release of the heart. 

This\marginnote{34.4} is the way in which these things differ in both meaning and phrasing. 

And\marginnote{35.1} what’s the way in which they mean the same thing, and differ only in the phrasing? 

Greed,\marginnote{35.2} hate, and delusion are makers of limits. A mendicant who has ended the defilements has given these up, cut them off at the root, made them like a palm stump, and obliterated them, so they are unable to arise in the future. The unshakable release of the heart is said to be the best kind of limitless release of the heart. That unshakable release of the heart is empty of greed, hate, and delusion. 

Greed\marginnote{36.1} is something, hate is something, and delusion is something. A mendicant who has ended the defilements has given these up, cut them off at the root, made them like a palm stump, and obliterated them, so they are unable to arise in the future. The unshakable release of the heart is said to be the best kind of release of the heart through nothingness. That unshakable release of the heart is empty of greed, hate, and delusion. 

Greed,\marginnote{37.1} hate, and delusion are makers of signs. A mendicant who has ended the defilements has given these up, cut them off at the root, made them like a palm stump, and obliterated them, so they are unable to arise in the future. The unshakable release of the heart is said to be the best kind of signless release of the heart. That unshakable release of the heart is empty of greed, hate, and delusion. 

This\marginnote{37.5} is the way in which they mean the same thing, and differ only in the phrasing.” 

This\marginnote{37.6} is what Venerable \textsanskrit{Sāriputta} said. Satisfied, Venerable \textsanskrit{Mahākoṭṭhita} was happy with what \textsanskrit{Sāriputta} said. 

%
\section*{{\suttatitleacronym MN 44}{\suttatitletranslation The Shorter Classification }{\suttatitleroot Cūḷavedallasutta}}
\addcontentsline{toc}{section}{\tocacronym{MN 44} \toctranslation{The Shorter Classification } \tocroot{Cūḷavedallasutta}}
\markboth{The Shorter Classification }{Cūḷavedallasutta}
\extramarks{MN 44}{MN 44}

\scevam{So\marginnote{1.1} I have heard. }At one time the Buddha was staying near \textsanskrit{Rājagaha}, in the Bamboo Grove, the squirrels’ feeding ground. 

Then\marginnote{1.3} the layman \textsanskrit{Visākha} went to see the nun \textsanskrit{Dhammadinnā}, bowed, sat down to one side, and said to her: 

“Ma’am,\marginnote{2.1} they speak of this thing called ‘identity’. What is this identity that the Buddha spoke of?” 

“\textsanskrit{Visākha},\marginnote{2.3} the Buddha said that these five grasping aggregates are identity. That is: form, feeling, perception, choices, and consciousness. The Buddha said that these five grasping aggregates are identity.” 

Saying\marginnote{2.6} “Good, ma’am,” \textsanskrit{Visākha} approved and agreed with what \textsanskrit{Dhammadinnā} said. Then he asked another question: 

“Ma’am,\marginnote{3.1} they speak of this thing called ‘the origin of identity’. What is the origin of identity that the Buddha spoke of?” 

“It’s\marginnote{3.3} the craving that leads to future lives, mixed up with relishing and greed, chasing pleasure in various realms. That is, craving for sensual pleasures, craving to continue existence, and craving to end existence. The Buddha said that this is the origin of identity.” 

“Ma’am,\marginnote{4.1} they speak of this thing called ‘the cessation of identity’. What is the cessation of identity that the Buddha spoke of?” 

“It’s\marginnote{4.3} the fading away and cessation of that very same craving with nothing left over; giving it away, letting it go, releasing it, and not adhering to it. The Buddha said that this is the cessation of identity.” 

“Ma’am,\marginnote{5.1} they speak of the practice that leads to the cessation of identity. What is the practice that leads to the cessation of identity that the Buddha spoke of?” 

“The\marginnote{5.3} practice that leads to the cessation of identity that the Buddha spoke of is simply this noble eightfold path, that is: right view, right thought, right speech, right action, right livelihood, right effort, right mindfulness, and right immersion.” 

“But\marginnote{6.1} ma’am, is that grasping the exact same thing as the five grasping aggregates? Or is grasping one thing and the five grasping aggregates another?” 

“That\marginnote{6.2} grasping is not the exact same thing as the five grasping aggregates. Nor is grasping one thing and the five grasping aggregates another. The desire and greed for the five grasping aggregates is the grasping there.” 

“But\marginnote{7.1} ma’am, how does identity view come about?” 

“It’s\marginnote{7.2} when an unlearned ordinary person has not seen the noble ones, and is neither skilled nor trained in the teaching of the noble ones. They’ve not seen good persons, and are neither skilled nor trained in the teaching of the good persons. They regard form as self, self as having form, form in self, or self in form. They regard feeling … perception … choices … consciousness as self, self as having consciousness, consciousness in self, or self in consciousness. That’s how identity view comes about.” 

“But\marginnote{8.1} ma’am, how does identity view not come about?” 

“It’s\marginnote{8.2} when a learned noble disciple has seen the noble ones, and is skilled and trained in the teaching of the noble ones. They’ve seen good persons, and are skilled and trained in the teaching of the good persons. They don’t regard form as self, self as having form, form in self, or self in form. They don’t regard feeling … perception … choices … consciousness as self, self as having consciousness, consciousness in self, or self in consciousness. That’s how identity view does not come about.” 

“But\marginnote{9.1} ma’am, what is the noble eightfold path?” 

“It\marginnote{9.2} is simply this noble eightfold path, that is: right view, right thought, right speech, right action, right livelihood, right effort, right mindfulness, and right immersion.” 

“But\marginnote{10.1} ma’am, is the noble eightfold path conditioned or unconditioned?” 

“The\marginnote{10.2} noble eightfold path is conditioned.” 

“Are\marginnote{11.1} the three practice categories included in the noble eightfold path? Or is the noble eightfold path included in the three practice categories?” 

“The\marginnote{11.2} three practice categories are not included in the noble eightfold path. Rather, the noble eightfold path is included in the three practice categories. Right speech, right action, and right livelihood: these things are included in the category of ethics. Right effort, right mindfulness, and right immersion: these things are included in the category of immersion. Right view and right thought: these things are included in the category of wisdom.” 

“But\marginnote{12.1} ma’am, what is immersion? What things are the foundations of immersion? What things are the prerequisites for immersion? What is the development of immersion?” 

“Unification\marginnote{12.2} of the mind is immersion. The four kinds of mindfulness meditation are the foundations of immersion. The four right efforts are the prerequisites for immersion. The cultivation, development, and making much of these very same things is the development of immersion.” 

“How\marginnote{13.1} many processes are there?” 

“There\marginnote{13.2} are these three processes. Physical, verbal, and mental processes.” 

“But\marginnote{14.1} ma’am, what is the physical process? What’s the verbal process? What’s the mental process?” 

“Breathing\marginnote{14.2} is a physical process. Placing the mind and keeping it connected are verbal processes. Perception and feeling are mental processes.” 

“But\marginnote{15.1} ma’am, why is breathing a physical process? Why are placing the mind and keeping it connected verbal processes? Why are perception and feeling mental processes?” 

“Breathing\marginnote{15.2} is physical. It’s tied up with the body, that’s why breathing is a physical process. First you place the mind and keep it connected, then you break into speech. That’s why placing the mind and keeping it connected are verbal processes. Perception and feeling are mental. They’re tied up with the mind, that’s why perception and feeling are mental processes.” 

“But\marginnote{16.1} ma’am, how does someone attain the cessation of perception and feeling?” 

“A\marginnote{16.2} mendicant who is entering such an attainment does not think: ‘I will enter the cessation of perception and feeling’ or ‘I am entering the cessation of perception and feeling’ or ‘I have entered the cessation of perception and feeling.’ Rather, their mind has been previously developed so as to lead to such a state.” 

“But\marginnote{17.1} ma’am, which cease first for a mendicant who is entering the cessation of perception and feeling: physical, verbal, or mental processes?” 

“Verbal\marginnote{17.2} processes cease first, then physical, then mental.” 

“But\marginnote{18.1} ma’am, how does someone emerge from the cessation of perception and feeling?” 

“A\marginnote{18.2} mendicant who is emerging from such an attainment does not think: ‘I will emerge from the cessation of perception and feeling’ or ‘I am emerging from the cessation of perception and feeling’ or ‘I have emerged from the cessation of perception and feeling.’ Rather, their mind has been previously developed so as to lead to such a state.” 

“But\marginnote{19.1} ma’am, which arise first for a mendicant who is emerging from the cessation of perception and feeling: physical, verbal, or mental processes?” 

“Mental\marginnote{19.2} processes arise first, then physical, then verbal.” 

“But\marginnote{20.1} ma’am, when a mendicant has emerged from the attainment of the cessation of perception and feeling, how many kinds of contact do they experience?” 

“They\marginnote{20.2} experience three kinds of contact: emptiness, signless, and undirected contacts.” 

“But\marginnote{21.1} ma’am, when a mendicant has emerged from the attainment of the cessation of perception and feeling, what does their mind slant, slope, and incline to?” 

“Their\marginnote{21.2} mind slants, slopes, and inclines to seclusion.” 

“But\marginnote{22.1} ma’am, how many feelings are there?” 

“There\marginnote{22.2} are three feelings: pleasant, painful, and neutral feeling.” 

“What\marginnote{23.1} are these three feelings?” 

“Anything\marginnote{23.2} felt physically or mentally as pleasant or enjoyable. This is pleasant feeling. Anything felt physically or mentally as painful or unpleasant. This is painful feeling. Anything felt physically or mentally as neither pleasurable nor painful. This is neutral feeling.” 

“What\marginnote{24.1} is pleasant and what is painful in each of the three feelings?” 

“Pleasant\marginnote{24.2} feeling is pleasant when it remains and painful when it perishes. Painful feeling is painful when it remains and pleasant when it perishes. Neutral feeling is pleasant when there is knowledge, and painful when there is ignorance.” 

“What\marginnote{25.1} underlying tendencies underlie each of the three feelings?” 

“The\marginnote{25.2} underlying tendency for greed underlies pleasant feeling. The underlying tendency for repulsion underlies painful feeling. The underlying tendency for ignorance underlies neutral feeling.” 

“Do\marginnote{26.1} these underlying tendencies always underlie these feelings?” 

“No,\marginnote{26.2} they do not.” 

“What\marginnote{27.1} should be given up in regard to each of these three feelings?” 

“The\marginnote{27.2} underlying tendency to greed should be given up when it comes to pleasant feeling. The underlying tendency to repulsion should be given up when it comes to painful feeling. The underlying tendency to ignorance should be given up when it comes to neutral feeling.” 

“Should\marginnote{28.1} these underlying tendencies be given up regarding all instances of these feelings?” 

“No,\marginnote{28.2} not in all instances. Take a mendicant who, quite secluded from sensual pleasures, secluded from unskillful qualities, enters and remains in the first absorption, which has the rapture and bliss born of seclusion, while placing the mind and keeping it connected. With this they give up greed, and the underlying tendency to greed does not lie within that. And take a mendicant who reflects: ‘Oh, when will I enter and remain in the same dimension that the noble ones enter and remain in today?’ Nursing such a longing for the supreme liberations gives rise to sadness due to longing. With this they give up repulsion, and the underlying tendency to repulsion does not lie within that. Take a mendicant who, giving up pleasure and pain, and ending former happiness and sadness, enters and remains in the fourth absorption, without pleasure or pain, with pure equanimity and mindfulness. With this they give up ignorance, and the underlying tendency to ignorance does not lie within that.” 

“But\marginnote{29.1} ma’am, what is the counterpart of pleasant feeling?” 

“Painful\marginnote{29.2} feeling.” 

“What\marginnote{29.3} is the counterpart of painful feeling?” 

“Pleasant\marginnote{29.4} feeling.” 

“What\marginnote{29.5} is the counterpart of neutral feeling?” 

“Ignorance.”\marginnote{29.6} 

“What\marginnote{29.7} is the counterpart of ignorance?” 

“Knowledge.”\marginnote{29.8} 

“What\marginnote{29.9} is the counterpart of knowledge?” 

“Freedom.”\marginnote{29.10} 

“What\marginnote{29.11} is the counterpart of freedom?” 

“Extinguishment.”\marginnote{29.12} 

“What\marginnote{29.13} is the counterpart of extinguishment?” 

“Your\marginnote{29.14} question goes too far, \textsanskrit{Visākha}. You couldn’t figure out the limit of questions. For extinguishment is the culmination, destination, and end of the spiritual life. If you wish, go to the Buddha and ask him this question. You should remember it in line with his answer.” 

And\marginnote{30.1} then the layman \textsanskrit{Visākha} approved and agreed with what the nun \textsanskrit{Dhammadinnā} said. He got up from his seat, bowed, and respectfully circled her, keeping her on his right. Then he went up to the Buddha, bowed, sat down to one side, and informed the Buddha of all they had discussed. 

When\marginnote{30.3} he had spoken, the Buddha said to him, “The nun \textsanskrit{Dhammadinnā} is astute, \textsanskrit{Visākha}, she has great wisdom. If you came to me and asked this question, I would answer it in exactly the same way as the nun \textsanskrit{Dhammadinnā}. That is what it means, and that’s how you should remember it.” 

That\marginnote{30.7} is what the Buddha said. Satisfied, the layman \textsanskrit{Visākha} was happy with what the Buddha said. 

%
\section*{{\suttatitleacronym MN 45}{\suttatitletranslation The Shorter Discourse on Taking Up Practices }{\suttatitleroot Cūḷadhammasamādānasutta}}
\addcontentsline{toc}{section}{\tocacronym{MN 45} \toctranslation{The Shorter Discourse on Taking Up Practices } \tocroot{Cūḷadhammasamādānasutta}}
\markboth{The Shorter Discourse on Taking Up Practices }{Cūḷadhammasamādānasutta}
\extramarks{MN 45}{MN 45}

\scevam{So\marginnote{1.1} I have heard. }At one time the Buddha was staying near \textsanskrit{Sāvatthī} in Jeta’s Grove, \textsanskrit{Anāthapiṇḍika}’s monastery. There the Buddha addressed the mendicants, “Mendicants!” 

“Venerable\marginnote{1.5} sir,” they replied. The Buddha said this: 

“Mendicants,\marginnote{2.1} there are these four ways of taking up practices. What four? There is a way of taking up practices that is pleasant now but results in future pain. There is a way of taking up practices that is painful now and results in future pain. There is a way of taking up practices that is painful now but results in future pleasure. There is a way of taking up practices that is pleasant now and results in future pleasure. 

And\marginnote{3.1} what is the way of taking up practices that is pleasant now but results in future pain? There are some ascetics and brahmins who have this doctrine and view: ‘There’s nothing wrong with sensual pleasures.’ They throw themselves into sensual pleasures, cavorting with female wanderers with fancy hair-dos. They say, ‘What future danger do those ascetics and brahmins see in sensual pleasures that they speak of giving up sensual pleasures, and advocate the complete understanding of sensual pleasures? Pleasant is the touch of this female wanderer’s arm, tender, soft, and downy!’ And they throw themselves into sensual pleasures. When their body breaks up, after death, they’re reborn in a place of loss, a bad place, the underworld, hell. And there they feel painful, sharp, severe, acute feelings. They say, ‘This is that future danger that those ascetics and brahmins saw. For it is because of sensual pleasures that I’m feeling painful, sharp, severe, acute feelings.’ 

Suppose\marginnote{4.1} that in the last month of summer a camel’s foot creeper pod were to burst open and a seed were to fall at the root of a sal tree. Then the deity haunting that sal tree would become apprehensive and nervous. But their friends and colleagues, relatives and kin—deities of the parks, forests, trees, and those who haunt the herbs, grass, and big trees—would come together to reassure them, ‘Do not fear, sir, do not fear! Hopefully that seed will be swallowed by a peacock, or eaten by a deer, or burnt by a forest fire, or picked up by a lumberjack, or eaten by termites, or it may not even be fertile.’ But none of these things happened. And the seed was fertile, so that when the clouds soaked it with rain, it sprouted. And the creeper wound its tender, soft, and downy tendrils around that sal tree. Then the deity thought, ‘What future danger did my friends see when they said: ‘Do not fear, sir, do not fear! Hopefully that seed will be swallowed by a peacock, or eaten by a deer, or burnt by a forest fire, or picked up by a lumberjack, or eaten by termites, or it may not even be fertile.’ Pleasant is the touch of this creeper’s tender, soft, and downy tendrils.’ Then the creeper enfolded the sal tree, made a canopy over it, draped a curtain around it, and split apart all the main branches. Then the deity thought, ‘This is the future danger that my friends saw! It’s because of that camel’s foot creeper seed that I’m feeling painful, sharp, severe, acute feelings.’ 

In\marginnote{4.22} the same way, there are some ascetics and brahmins who have this doctrine and view: ‘There’s nothing wrong with sensual pleasures’ … This is called the way of taking up practices that is pleasant now but results in future pain. 

And\marginnote{5.1} what is the way of taking up practices that is painful now and results in future pain? It’s when someone goes naked, ignoring conventions. They lick their hands, and don’t come or wait when called. They don’t consent to food brought to them, or food prepared on purpose for them, or an invitation for a meal. They don’t receive anything from a pot or bowl; or from someone who keeps sheep, or who has a weapon or a shovel in their home; or where a couple is eating; or where there is a woman who is pregnant, breastfeeding, or who has a man in her home; or where there’s a dog waiting or flies buzzing. They accept no fish or meat or liquor or wine, and drink no beer. They go to just one house for alms, taking just one mouthful, or two houses and two mouthfuls, up to seven houses and seven mouthfuls. They feed on one saucer a day, two saucers a day, up to seven saucers a day. They eat once a day, once every second day, up to once a week, and so on, even up to once a fortnight. They live committed to the practice of eating food at set intervals. 

They\marginnote{5.7} eat herbs, millet, wild rice, poor rice, water lettuce, rice bran, scum from boiling rice, sesame flour, grass, or cow dung. They survive on forest roots and fruits, or eating fallen fruit. 

They\marginnote{5.8} wear robes of sunn hemp, mixed hemp, corpse-wrapping cloth, rags, lodh tree bark, antelope hide (whole or in strips), kusa grass, bark, wood-chips, human hair, horse-tail hair, or owls’ wings. They tear out their hair and beard, committed to this practice. They stand forever, refusing seats. They squat, committed to persisting in the squatting position. They lie on a mat of thorns, making a mat of thorns their bed. They’re committed to the practice of immersion in water three times a day, including the evening. And so they live committed to practicing these various ways of mortifying and tormenting the body. When their body breaks up, after death, they’re reborn in a place of loss, a bad place, the underworld, hell. This is called the way of taking up practices that is painful now and results in future pain. 

And\marginnote{6.1} what is the way of taking up practices that is painful now but results in future pleasure? It’s when someone is ordinarily full of acute greed, hate, and delusion. They often feel the pain and sadness that greed, hate, and delusion bring. They lead the full and pure spiritual life in pain and sadness, weeping, with tearful faces. When their body breaks up, after death, they’re reborn in a good place, a heavenly realm. This is called the way of taking up practices that is painful now but results in future pleasure. 

And\marginnote{7.1} what is the way of taking up practices that is pleasant now and results in future pleasure? It’s when someone is not ordinarily full of acute greed, hate, and delusion. They rarely feel the pain and sadness that greed, hate, and delusion bring. Quite secluded from sensual pleasures, secluded from unskillful qualities, they enter and remain in the first absorption … second absorption … third absorption … fourth absorption. When their body breaks up, after death, they’re reborn in a good place, a heavenly realm. This is called the way of taking up practices that is pleasant now and results in future pleasure. These are the four ways of taking up practices.” 

That\marginnote{7.12} is what the Buddha said. Satisfied, the mendicants were happy with what the Buddha said. 

%
\section*{{\suttatitleacronym MN 46}{\suttatitletranslation The Great Discourse on Taking Up Practices }{\suttatitleroot Mahādhammasamādānasutta}}
\addcontentsline{toc}{section}{\tocacronym{MN 46} \toctranslation{The Great Discourse on Taking Up Practices } \tocroot{Mahādhammasamādānasutta}}
\markboth{The Great Discourse on Taking Up Practices }{Mahādhammasamādānasutta}
\extramarks{MN 46}{MN 46}

\scevam{So\marginnote{1.1} I have heard. }At one time the Buddha was staying near \textsanskrit{Sāvatthī} in Jeta’s Grove, \textsanskrit{Anāthapiṇḍika}’s monastery. There the Buddha addressed the mendicants, “Mendicants!” 

“Venerable\marginnote{1.5} sir,” they replied. The Buddha said this: 

“Mendicants,\marginnote{2.1} sentient beings typically have the wish, desire, and hope: ‘Oh, if only unlikable, undesirable, and disagreeable things would decrease, and likable, desirable, and agreeable things would increase!’ But exactly the opposite happens to them. What do you take to be the reason for this?” 

“Our\marginnote{2.5} teachings are rooted in the Buddha. He is our guide and our refuge. Sir, may the Buddha himself please clarify the meaning of this. The mendicants will listen and remember it.” 

“Well\marginnote{2.6} then, mendicants, listen and pay close attention, I will speak.” 

“Yes,\marginnote{2.7} sir,” they replied. The Buddha said this: 

“Take\marginnote{3.1} an unlearned ordinary person who has not seen the noble ones, and is neither skilled nor trained in the teaching of the noble ones. They’ve not seen good persons, and are neither skilled nor trained in the teaching of the good persons. They don’t know what practices they should cultivate and foster, and what practices they shouldn’t cultivate and foster. So they cultivate and foster practices they shouldn’t, and don’t cultivate and foster practices they should. When they do so, unlikable, undesirable, and disagreeable things increase, and likable, desirable, and agreeable things decrease. Why is that? Because that’s what it’s like for someone who doesn’t know. 

But\marginnote{4.1} a learned noble disciple has seen the noble ones, and is skilled and trained in the teaching of the noble ones. They’ve seen good persons, and are skilled and trained in the teaching of the good persons. They know what practices they should cultivate and foster, and what practices they shouldn’t cultivate and foster. So they cultivate and foster practices they should, and don’t cultivate and foster practices they shouldn’t. When they do so, unlikable, undesirable, and disagreeable things decrease, and likable, desirable, and agreeable things increase. Why is that? Because that’s what it’s like for someone who knows. 

Mendicants,\marginnote{5.1} there are these four ways of taking up practices. What four? There is a way of taking up practices that is painful now and results in future pain. There is a way of taking up practices that is pleasant now but results in future pain. There is a way of taking up practices that is painful now but results in future pleasure. There is a way of taking up practices that is pleasant now and results in future pleasure. 

When\marginnote{6.1} it comes to the way of taking up practices that is painful now and results in future pain, an ignoramus, without knowing this, doesn’t truly understand: ‘This is the way of taking up practices that is painful now and results in future pain.’ So instead of avoiding that practice, they cultivate it. When they do so, unlikable, undesirable, and disagreeable things increase, and likable, desirable, and agreeable things decrease. Why is that? Because that’s what it’s like for someone who doesn’t know. 

When\marginnote{7.1} it comes to the way of taking up practices that is pleasant now and results in future pain, an ignoramus … cultivates it … and disagreeable things increase … 

When\marginnote{8.1} it comes to the way of taking up practices that is painful now and results in future pleasure, an ignoramus … doesn’t cultivate it … and disagreeable things increase … 

When\marginnote{9.1} it comes to the way of taking up practices that is pleasant now and results in future pleasure, an ignoramus … doesn’t cultivate it … and disagreeable things increase … Why is that? Because that’s what it’s like for someone who doesn’t know. 

When\marginnote{10.1} it comes to the way of taking up practices that is painful now and results in future pain, a wise person, knowing this, truly understands: ‘This is the way of taking up practices that is painful now and results in future pain.’ So instead of cultivating that practice, they avoid it. When they do so, unlikable, undesirable, and disagreeable things decrease, and likable, desirable, and agreeable things increase. Why is that? Because that’s what it’s like for someone who knows. 

When\marginnote{11.1} it comes to the way of taking up practices that is pleasant now and results in future pain, a wise person … doesn’t cultivate it … and agreeable things increase … 

When\marginnote{12.1} it comes to the way of taking up practices that is painful now and results in future pleasure, a wise person … cultivates it … and agreeable things increase … 

When\marginnote{13.1} it comes to the way of taking up practices that is pleasant now and results in future pleasure, a wise person, knowing this, truly understands: ‘This is the way of taking up practices that is pleasant now and results in future pleasure.’ So instead of avoiding that practice, they cultivate it. When they do so, unlikable, undesirable, and disagreeable things decrease, and likable, desirable, and agreeable things increase. Why is that? Because that’s what it’s like for someone who knows. 

And\marginnote{14.1} what is the way of taking up practices that is painful now and results in future pain? It’s when someone in pain and sadness kills living creatures, steals, and commits sexual misconduct. They use speech that’s false, divisive, harsh, or nonsensical. And they’re covetous, malicious, with wrong view. Because of these things they experience pain and sadness. And when their body breaks up, after death, they’re reborn in a place of loss, a bad place, the underworld, hell. This is called the way of taking up practices that is painful now and results in future pain. 

And\marginnote{15.1} what is the way of taking up practices that is pleasant now but results in future pain? It’s when someone with pleasure and happiness kills living creatures, steals, and commits sexual misconduct. They use speech that’s false, divisive, harsh, or nonsensical. And they’re covetous, malicious, with wrong view. Because of these things they experience pleasure and happiness. But when their body breaks up, after death, they’re reborn in a place of loss, a bad place, the underworld, hell. This is called the way of taking up practices that is pleasant now but results in future pain. 

And\marginnote{16.1} what is the way of taking up practices that is painful now but results in future pleasure? It’s when someone in pain and sadness doesn’t kill living creatures, steal, or commit sexual misconduct. They don’t use speech that’s false, divisive, harsh, or nonsensical. And they’re contented, kind-hearted, with right view. Because of these things they experience pain and sadness. But when their body breaks up, after death, they’re reborn in a good place, a heavenly realm. This is called the way of taking up practices that is painful now but results in future pleasure. 

And\marginnote{17.1} what is the way of taking up practices that is pleasant now and results in future pleasure? It’s when someone with pleasure and happiness doesn’t kill living creatures, steal, or commit sexual misconduct. They don’t use speech that’s false, divisive, harsh, or nonsensical. And they’re contented, kind-hearted, with right view. Because of these things they experience pleasure and happiness. And when their body breaks up, after death, they’re reborn in a good place, a heavenly realm. This is called the way of taking up practices that is pleasant now and results in future pleasure. These are the four ways of taking up practices. 

Suppose\marginnote{18.1} there was some bitter gourd mixed with poison. Then a man would come along who wants to live and doesn’t want to die, who wants to be happy and recoils from pain. They’d say to him: ‘Here, mister, this is bitter gourd mixed with poison. Drink it if you like. If you drink it, the color, aroma, and flavor will be unappetizing, and it will result in death or deadly pain.’ He wouldn’t reject it. Without reflection, he’d drink it. The color, aroma, and flavor would be unappetizing, and it would result in death or deadly pain. This is comparable to the way of taking up practices that is painful now and results in future pain, I say. 

Suppose\marginnote{19.1} there was a bronze cup of beverage that had a nice color, aroma, and flavor. But it was mixed with poison. Then a man would come along who wants to live and doesn’t want to die, who wants to be happy and recoils from pain. They’d say to him: ‘Here, mister, this bronze cup of beverage has a nice color, aroma, and flavor. But it’s mixed with poison. Drink it if you like. If you drink it, the color, aroma, and flavor will be appetizing, but it will result in death or deadly pain.’ He wouldn’t reject it. Without reflection, he’d drink it. The color, aroma, and flavor would be appetizing, but it would result in death or deadly pain. This is comparable to the way of taking up practices that is pleasant now and results in future pain, I say. 

Suppose\marginnote{20.1} there was some fermented urine mixed with different medicines. Then a man with jaundice would come along. They’d say to him: ‘Here, mister, this is fermented urine mixed with different medicines. Drink it if you like. If you drink it, the color, aroma, and flavor will be unappetizing, but after drinking it you will be happy.’ He wouldn’t reject it. After appraisal, he’d drink it. The color, aroma, and flavor would be unappetizing, but after drinking it he would be happy. This is comparable to the way of taking up practices that is painful now and results in future pleasure, I say. 

Suppose\marginnote{21.1} there was some curds, honey, ghee, and molasses all mixed together. Then a man with bloody dysentery would come along. They’d say to him: ‘Here, mister, this is curds, honey, ghee, and molasses all mixed together. Drink it if you like. If you drink it, the color, aroma, and flavor will be appetizing, and after drinking it you will be happy.’ He wouldn’t reject it. After appraisal, he’d drink it. The color, aroma, and flavor would be appetizing, and after drinking it he would be happy. This is comparable to the way of taking up practices that is pleasant now and results in future pleasure, I say. 

It’s\marginnote{22.1} like the time after the rainy season when the sky is clear and cloudless. And when the sun rises, it dispels all the darkness from the sky as it shines and glows and radiates. In the same way, this way of taking up practices that is pleasant now and results in future pleasure dispels the doctrines of the various other ascetics and brahmins as it shines and glows and radiates.” 

That\marginnote{22.3} is what the Buddha said. Satisfied, the mendicants were happy with what the Buddha said. 

%
\section*{{\suttatitleacronym MN 47}{\suttatitletranslation The Inquirer }{\suttatitleroot Vīmaṁsakasutta}}
\addcontentsline{toc}{section}{\tocacronym{MN 47} \toctranslation{The Inquirer } \tocroot{Vīmaṁsakasutta}}
\markboth{The Inquirer }{Vīmaṁsakasutta}
\extramarks{MN 47}{MN 47}

\scevam{So\marginnote{1.1} I have heard. }At one time the Buddha was staying near \textsanskrit{Sāvatthī} in Jeta’s Grove, \textsanskrit{Anāthapiṇḍika}’s monastery. There the Buddha addressed the mendicants, “Mendicants!” 

“Venerable\marginnote{1.5} sir,” they replied. The Buddha said this: 

“Mendicants,\marginnote{2.1} a mendicant who is an inquirer, unable to comprehend another’s mind, should scrutinize the Realized One to see whether he is a fully awakened Buddha or not.” 

“Our\marginnote{3.1} teachings are rooted in the Buddha. He is our guide and our refuge. Sir, may the Buddha himself please clarify the meaning of this. The mendicants will listen and remember it.” 

“Well\marginnote{3.2} then, mendicants, listen and pay close attention, I will speak.” 

“Yes,\marginnote{3.3} sir,” they replied. The Buddha said this: 

“Mendicants,\marginnote{4.1} a mendicant who is an inquirer, unable to comprehend another’s mind, should scrutinize the Realized One for two things—things that can be seen and heard: ‘Can anything corrupt be seen or heard in the Realized One or not?’ Scrutinizing him they find that nothing corrupt can be seen or heard in the Realized One. 

They\marginnote{5.1} scrutinize further: ‘Can anything mixed be seen or heard in the Realized One or not?’ Scrutinizing him they find that nothing mixed can be seen or heard in the Realized One. 

They\marginnote{6.1} scrutinize further: ‘Can anything clean be seen or heard in the Realized One or not?’ Scrutinizing him they find that clean things can be seen and heard in the Realized One. 

They\marginnote{7.1} scrutinize further: ‘Did the venerable attain this skillful state a long time ago, or just recently?’ Scrutinizing him they find that the venerable attained this skillful state a long time ago, not just recently. 

They\marginnote{8.1} scrutinize further: ‘Are certain dangers found in that venerable mendicant who has achieved fame and renown?’ For, mendicants, so long as a mendicant has not achieved fame and renown, certain dangers are not found in them. But when they achieve fame and renown, those dangers appear. Scrutinizing him they find that those dangers are not found in that venerable mendicant who has achieved fame and renown. 

They\marginnote{9.1} scrutinize further: ‘Is this venerable securely stopped or insecurely stopped? Is the reason they don’t indulge in sensual pleasures that they’re free of greed because greed has ended?’ Scrutinizing him they find that that venerable is securely stopped, not insecurely stopped. The reason they don’t indulge in sensual pleasures is that they’re free of greed because greed has ended. 

If\marginnote{10.1} others should ask that mendicant, ‘But what reason and evidence does the venerable have for saying this?’ Answering rightly, the mendicant should say, ‘Because, whether that venerable is staying in a community or alone, some people there are in a good state or a sorry state, some instruct a group, and some indulge in material pleasures, while others remain unsullied. Yet that venerable doesn’t look down on them for that. Also, I have heard and learned this in the presence of the Buddha: “I am securely stopped, not insecurely stopped. The reason I don’t indulge in sensual pleasures is that I’m free of greed because greed has ended.”’ 

Next,\marginnote{11.1} they should ask the Realized One himself about this, ‘Can anything corrupt be seen or heard in the Realized One or not?’ The Realized One would answer, ‘Nothing corrupt can be seen or heard in the Realized One.’ 

‘Can\marginnote{12.1} anything mixed be seen or heard in the Realized One or not?’ The Realized One would answer, ‘Nothing mixed can be seen or heard in the Realized One.’ 

‘Can\marginnote{13.1} anything clean be seen or heard in the Realized One or not?’ The Realized One would answer, ‘Clean things can be seen and heard in the Realized One. I am that range and that territory, but I do not identify with that.’ 

A\marginnote{14.1} disciple ought to approach a teacher who has such a doctrine in order to listen to the teaching. The teacher explains Dhamma with its higher and higher stages, with its better and better stages, with its dark and bright sides. When they directly know a certain principle of those teachings, in accordance with how they were taught, the mendicant comes to a conclusion about the teachings. They have confidence in the teacher: ‘The Blessed One is a fully awakened Buddha! The teaching is well explained! The \textsanskrit{Saṅgha} is practicing well!’ 

If\marginnote{15.1} others should ask that mendicant, ‘But what reason and evidence does the venerable have for saying this?’ Answering rightly, the mendicant should say, ‘Reverends, I approached the Buddha to listen to the teaching. He explained Dhamma with its higher and higher stages, with its better and better stages, with its dark and bright sides. When I directly knew a certain principle of those teachings, in accordance with how I was taught, I came to a conclusion about the teachings. I had confidence in the Teacher: “The Blessed One is a fully awakened Buddha! The teaching is well explained! The \textsanskrit{Saṅgha} is practicing well!”’ 

When\marginnote{16.1} someone’s faith is settled, rooted, and planted in the Realized One in this manner, with these words and phrases, it’s said to be grounded faith that’s based on evidence. It is firm, and cannot be shifted by any ascetic or brahmin or god or \textsanskrit{Māra} or \textsanskrit{Brahmā} or by anyone in the world. This is how to scrutinize the Realized One’s qualities. But the Realized One has already been properly searched in this way by nature.” 

That\marginnote{16.5} is what the Buddha said. Satisfied, the mendicants were happy with what the Buddha said. 

%
\section*{{\suttatitleacronym MN 48}{\suttatitletranslation The Mendicants of Kosambi }{\suttatitleroot Kosambiyasutta}}
\addcontentsline{toc}{section}{\tocacronym{MN 48} \toctranslation{The Mendicants of Kosambi } \tocroot{Kosambiyasutta}}
\markboth{The Mendicants of Kosambi }{Kosambiyasutta}
\extramarks{MN 48}{MN 48}

\scevam{So\marginnote{1.1} I have heard. }At one time the Buddha was staying near Kosambi, in Ghosita’s Monastery. 

Now\marginnote{2.1} at that time the mendicants of Kosambi were arguing, quarreling, and disputing, continually wounding each other with barbed words. They couldn’t persuade each other or be persuaded, nor could they convince each other or be convinced. 

Then\marginnote{3.1} a mendicant went up to the Buddha, bowed, sat down to one side, and told him what was happening. 

So\marginnote{4.1} the Buddha addressed a certain monk, “Please, monk, in my name tell those mendicants that the teacher summons them. 

“Yes,\marginnote{4.4} sir,” that monk replied. He went to those monks and said, “Venerables, the teacher summons you.” 

“Yes,\marginnote{4.6} reverend,” those monks replied. They went to the Buddha, bowed, and sat down to one side. The Buddha said to them, 

“Is\marginnote{4.7} it really true, mendicants, that you have been arguing, quarreling, and disputing, continually wounding each other with barbed words? And that you can’t persuade each other or be persuaded, nor can you convince each other or be convinced?” 

“Yes,\marginnote{4.9} sir,” they said. 

“What\marginnote{5.1} do you think, mendicants? When you’re arguing, quarreling, and disputing, continually wounding each other with barbed words, are you treating your spiritual companions with kindness by way of body, speech, and mind, both in public and in private?” 

“No,\marginnote{5.3} sir.” 

“So\marginnote{5.4} it seems that when you’re arguing you are not treating each other with kindness. So what exactly do you know and see, you foolish men, that you behave in such a way? This will be for your lasting harm and suffering.” 

Then\marginnote{6.1} the Buddha said to the mendicants: 

“Mendicants,\marginnote{6.2} these six warm-hearted qualities make for fondness and respect, conducing to inclusion, harmony, and unity, without quarreling. What six? Firstly, a mendicant consistently treats their spiritual companions with bodily kindness, both in public and in private. This warm-hearted quality makes for fondness and respect, conducing to inclusion, harmony, and unity, without quarreling. 

Furthermore,\marginnote{6.6} a mendicant consistently treats their spiritual companions with verbal kindness … 

Furthermore,\marginnote{6.8} a mendicant consistently treats their spiritual companions with mental kindness … 

Furthermore,\marginnote{6.10} a mendicant shares without reservation any material possessions they have gained by legitimate means, even the food placed in the alms-bowl, using them in common with their ethical spiritual companions … 

Furthermore,\marginnote{6.12} a mendicant lives according to the precepts shared with their spiritual companions, both in public and in private. Those precepts are unbroken, impeccable, spotless, and unmarred, liberating, praised by sensible people, not mistaken, and leading to immersion. … 

Furthermore,\marginnote{6.14} a mendicant lives according to the view shared with their spiritual companions, both in public and in private. That view is noble and emancipating, and leads one who practices it to the complete ending of suffering. This warm-hearted quality makes for fondness and respect, conducing to inclusion, harmony, and unity, without quarreling. 

These\marginnote{6.16} six warm-hearted qualities make for fondness and respect, conducing to inclusion, harmony, and unity, without quarreling. 

Of\marginnote{7.1} these six warm-hearted qualities, the chief is the view that is noble and emancipating, and leads one who practices it to the complete ending of suffering. It holds and binds everything together. It’s like a bungalow. The roof-peak is the chief point, which holds and binds everything together. In the same way, of these six warm-hearted qualities, the chief is the view that is noble and emancipating, and leads one who practices it to the complete ending of suffering. It holds and binds everything together. 

And\marginnote{8.1} how does the view that is noble and emancipating lead one who practices it to the complete ending of suffering? It’s when a mendicant has gone to a wilderness, or to the root of a tree, or to an empty hut, and reflects like this, ‘Is there anything that I’m overcome with internally and haven’t given up, because of which I might not accurately know and see?’ If a mendicant is overcome with sensual desire, it’s their mind that’s overcome. If a mendicant is overcome with ill will, dullness and drowsiness, restlessness and remorse, doubt, pursuing speculation about this world, pursuing speculation about the next world, or arguing, quarreling, and disputing, continually wounding others with barbed words, it’s their mind that’s overcome. They understand, ‘There is nothing that I’m overcome with internally and haven’t given up, because of which I might not accurately know and see. My mind is properly disposed for awakening to the truths.’ This is the first knowledge they have achieved that is noble and transcendent, and is not shared with ordinary people. 

Furthermore,\marginnote{9.1} a noble disciple reflects, ‘When I develop, cultivate, and make much of this view, do I personally gain serenity and quenching?’ They understand, ‘When I develop, cultivate, and make much of this view, I personally gain serenity and quenching.’ This is their second knowledge … 

Furthermore,\marginnote{10.1} a noble disciple reflects, ‘Are there any ascetics or brahmins outside of the Buddhist community who have the same kind of view that I have?’ They understand, ‘There are no ascetics or brahmins outside of the Buddhist community who have the same kind of view that I have.’ This is their third knowledge … 

Furthermore,\marginnote{11.1} a noble disciple reflects, ‘Do I have the same nature as a person accomplished in view?’ And what, mendicants, is the nature of a person accomplished in view? This is the nature of a person accomplished in view. Though they may fall into a kind of offense for which rehabilitation has been laid down, they quickly disclose, clarify, and reveal it to the Teacher or a sensible spiritual companion. And having revealed it they restrain themselves in the future. Suppose there was a little baby boy. If he puts his hand or foot on a burning coal, he quickly pulls it back. In the same way, this is the nature of a person accomplished in view. Though they may still fall into a kind of offense for which rehabilitation has been laid down, they quickly reveal it to the Teacher or a sensible spiritual companion. And having revealed it they restrain themselves in the future. They understand, ‘I have the same nature as a person accomplished in view.’ This is their fourth knowledge … 

Furthermore,\marginnote{12.1} a noble disciple reflects, ‘Do I have the same nature as a person accomplished in view?’ And what, mendicants, is the nature of a person accomplished in view? This is the nature of a person accomplished in view. Though they might manage a diverse spectrum of duties for their spiritual companions, they still feel a keen regard for the training in higher ethics, higher mind, and higher wisdom. Suppose there was a cow with a baby calf. She keeps the calf close as she grazes. In the same way, this is the nature of a person accomplished in view. Though they might manage a diverse spectrum of duties for their spiritual companions, they still feel a keen regard for the training in higher ethics, higher mind, and higher wisdom. They understand, ‘I have the same nature as a person accomplished in view.’ This is their fifth knowledge … 

Furthermore,\marginnote{13.1} a noble disciple reflects, ‘Do I have the same strength as a person accomplished in view?’ And what, mendicants, is the strength of a person accomplished in view? The strength of a person accomplished in view is that, when the teaching and training proclaimed by the Realized One are being taught, they pay heed, pay attention, engage wholeheartedly, and lend an ear. They understand, ‘I have the same strength as a person accomplished in view.’ This is their sixth knowledge … 

Furthermore,\marginnote{14.1} a noble disciple reflects, ‘Do I have the same strength as a person accomplished in view?’ And what, mendicants, is the strength of a person accomplished in view? The strength of a person accomplished in view is that, when the teaching and training proclaimed by the Realized One are being taught, they find inspiration in the meaning and the teaching, and find joy connected with the teaching. They understand, ‘I have the same strength as a person accomplished in view.’ This is the seventh knowledge they have achieved that is noble and transcendent, and is not shared with ordinary people. 

When\marginnote{15.1} a noble disciple has these seven factors, they have properly investigated their own nature with respect to the realization of the fruit of stream-entry. A noble disciple with these seven factors has the fruit of stream-entry.” 

That\marginnote{15.3} is what the Buddha said. Satisfied, the mendicants were happy with what the Buddha said. 

%
\section*{{\suttatitleacronym MN 49}{\suttatitletranslation On the Invitation of Brahmā }{\suttatitleroot Brahmanimantanikasutta}}
\addcontentsline{toc}{section}{\tocacronym{MN 49} \toctranslation{On the Invitation of Brahmā } \tocroot{Brahmanimantanikasutta}}
\markboth{On the Invitation of Brahmā }{Brahmanimantanikasutta}
\extramarks{MN 49}{MN 49}

\scevam{So\marginnote{1.1} I have heard. }At one time the Buddha was staying near \textsanskrit{Sāvatthī} in Jeta’s Grove, \textsanskrit{Anāthapiṇḍika}’s monastery. There the Buddha addressed the mendicants, “Mendicants!” 

“Venerable\marginnote{1.5} sir,” they replied. The Buddha said this: 

“At\marginnote{2.1} one time, mendicants, I was staying near \textsanskrit{Ukkaṭṭhā}, in the Subhaga Forest at the root of a magnificent sal tree. Now at that time Baka the \textsanskrit{Brahmā} had the following harmful misconception: ‘This is permanent, this is everlasting, this is eternal, this is whole, this is imperishable. For this is where there’s no being born, growing old, dying, passing away, or being reborn. And there’s no other escape beyond this.’ 

Then\marginnote{3.1} I knew what Baka the \textsanskrit{Brahmā} was thinking. As easily as a strong person would extend or contract their arm, I vanished from the Subhaga Forest and reappeared in that \textsanskrit{Brahmā} realm. 

Baka\marginnote{3.3} saw me coming off in the distance and said, ‘Come, good sir! Welcome, good sir! It’s been a long time since you took the opportunity to come here. For this is permanent, this is everlasting, this is eternal, this is complete, this is imperishable. For this is where there’s no being born, growing old, dying, passing away, or being reborn. And there’s no other escape beyond this.’ 

When\marginnote{4.1} he had spoken, I said to him, ‘Alas, Baka the \textsanskrit{Brahmā} is lost in ignorance! Alas, Baka the \textsanskrit{Brahmā} is lost in ignorance! Because what is actually impermanent, not lasting, transient, incomplete, and perishable, he says is permanent, everlasting, eternal, complete, and imperishable. And where there is being born, growing old, dying, passing away, and being reborn, he says that there’s no being born, growing old, dying, passing away, or being reborn. And although there is another escape beyond this, he says that there’s no other escape beyond this.’ 

Then\marginnote{5.1} \textsanskrit{Māra} the Wicked took possession of a member of \textsanskrit{Brahmā}’s retinue and said this to me, ‘Mendicant, mendicant! Don’t attack this one! Don’t attack this one! For this is \textsanskrit{Brahmā}, the Great \textsanskrit{Brahmā}, the Undefeated, the Champion, the Universal Seer, the Wielder of Power, the Lord God, the Maker, the Author, the First, the Begetter, the Controller, the Father of those who have been born and those yet to be born. 

There\marginnote{5.3} have been ascetics and brahmins before you, mendicant, who criticized and loathed earth, water, air, fire, creatures, gods, the Creator, and \textsanskrit{Brahmā}. When their bodies broke up and their breath was cut off they were reborn in a lower realm. 

There\marginnote{5.5} have been ascetics and brahmins before you, mendicant, who praised and approved earth, water, air, fire, creatures, gods, the Creator, and \textsanskrit{Brahmā}. When their bodies broke up and their breath was cut off they were reborn in a higher realm. 

So,\marginnote{5.7} mendicant, I tell you this: please, good sir, do exactly what \textsanskrit{Brahmā} says. Don’t go beyond the word of \textsanskrit{Brahmā}. If you do, then the consequence for you will be like that of a person who, when Lady Luck approaches, wards her off with a staff, or someone who shoves away the ground as they fall down the abyss into hell. Please, dear sir, do exactly what \textsanskrit{Brahmā} says. Don’t go beyond the word of \textsanskrit{Brahmā}. Do you not see the assembly of \textsanskrit{Brahmā} gathered here?’ 

And\marginnote{5.12} that is how \textsanskrit{Māra} the Wicked presented the assembly of \textsanskrit{Brahmā} to me as an example. 

When\marginnote{6.1} he had spoken, I said to \textsanskrit{Māra}, ‘I know you, Wicked One. Do not think, “He does not know me.” You are \textsanskrit{Māra} the Wicked. And \textsanskrit{Brahmā}, \textsanskrit{Brahmā}’s assembly, and \textsanskrit{Brahmā}’s retinue have all fallen into your hands; they’re under your sway. And you think, “Maybe this one, too, has fallen into my hands; maybe he’s under my sway!” But I haven’t fallen into your hands; I’m not under your sway.’ 

When\marginnote{7.1} I had spoken, Baka the \textsanskrit{Brahmā} said to me, ‘But, good sir, what I say is permanent, everlasting, eternal, complete, and imperishable is in fact permanent, everlasting, eternal, complete, and imperishable. And where I say there’s no being born, growing old, dying, passing away, or being reborn there is in fact no being born, growing old, dying, passing away, or being reborn. And when I say there’s no other escape beyond this there is in fact no other escape beyond this. There have been ascetics and brahmins in the world before you, mendicant, whose self-mortification lasted as long as your entire life. When there was another escape beyond this they knew it, and when there was no other escape beyond this, they knew it. So, mendicant, I tell you this: you will never find another escape beyond this, and you will eventually get weary and frustrated. If you attach to earth, you will lie close to me, in my domain, subject to my will, and expendable. If you attach to water … fire … air … creatures … gods … the Creator … \textsanskrit{Brahmā}, you will lie close to me, in my domain, subject to my will, and expendable.’ 

‘\textsanskrit{Brahmā},\marginnote{8.1} I too know that if I attach to earth, I will lie close to you, in your domain, subject to your will, and expendable. If I attach to water … fire … air … creatures … gods … the Creator … \textsanskrit{Brahmā}, I will lie close to you, in your domain, subject to your will, and expendable. And in addition, \textsanskrit{Brahmā}, I understand your range and your light: “That’s how powerful is Baka the \textsanskrit{Brahmā}, how illustrious and mighty.”’ 

‘But\marginnote{8.11} in what way do you understand my range and my light?’ 

\begin{verse}%
‘A\marginnote{9.1} galaxy extends a thousand times as far \\
as the moon and sun revolve \\
and the shining ones light up the quarters. \\
And there you wield your power. 

You\marginnote{9.5} know the high and low, \\
the passionate and dispassionate, \\
and the coming and going of sentient beings \\
from this realm to another. 

%
\end{verse}

That’s\marginnote{9.9} how I understand your range and your light. 

But\marginnote{10.1} there is another realm that you don’t know or see. But I know it and see it. There is the realm named after the gods of streaming radiance. You passed away from there and were reborn here. You’ve dwelt here so long that you’ve forgotten about that, so you don’t know it or see it. But I know it and see it. So \textsanskrit{Brahmā}, I am not your equal in knowledge, still less your inferior. Rather, I know more than you. 

There\marginnote{10.8} is the realm named after the gods replete with glory … the realm named after the gods of abundant fruit … the realm named after the Overlord, which you don’t know or see. But I know it and see it. So \textsanskrit{Brahmā}, I am not your equal in knowledge, still less your inferior. Rather, I know more than you. 

Having\marginnote{11.1} directly known earth as earth, and having directly known that which does not fall within the scope of experience based on earth, I did not identify with earth, I did not identify regarding earth, I did not identify as earth, I did not identify ‘earth is mine’, I did not enjoy earth. So \textsanskrit{Brahmā}, I am not your equal in knowledge, still less your inferior. Rather, I know more than you. 

Having\marginnote{12.1} directly known water … fire … air … creatures … gods … the Creator … \textsanskrit{Brahmā} … the gods of streaming radiance … the gods replete with glory … the gods of abundant fruit … the Overlord … Having directly known all as all, and having directly known that which does not fall within the scope of experience based on all, I did not identify with all, I did not identify regarding all, I did not identify as all, I did not identify ‘all is mine’, I did not enjoy all. So \textsanskrit{Brahmā}, I am not your equal in knowledge, still less your inferior. Rather, I know more than you.’ 

‘Well,\marginnote{24.1} good sir, if you have directly known that which is not within the scope of experience based on all, may your words not turn out to be void and hollow! 

Consciousness\marginnote{25.1} that is invisible, infinite, entirely given up—\emph{that’s} what is not within the scope of experience based on earth, water, fire, air, creatures, gods, the Creator, \textsanskrit{Brahmā}, the gods of streaming radiance, the gods replete with glory, the gods of abundant fruit, the Overlord, and the all. 

Well\marginnote{26.1} look now, good sir, I will vanish from you!’ 

‘All\marginnote{26.2} right, then, \textsanskrit{Brahmā}, vanish from me—if you can.’ 

Then\marginnote{26.3} Baka the \textsanskrit{Brahmā} said, ‘I will vanish from the ascetic Gotama! I will vanish from the ascetic Gotama!’ But he was unable to vanish from me. 

So\marginnote{26.5} I said to him, ‘Well look now, \textsanskrit{Brahmā}, I will vanish from you!’ 

‘All\marginnote{26.7} right, then, good sir, vanish from me—if you can.’ 

Then\marginnote{26.8} I used my psychic power to will that my voice would extend so that \textsanskrit{Brahmā}, his assembly, and his retinue would hear me, but they would not see me. And while invisible I recited this verse: 

\begin{verse}%
‘Seeing\marginnote{27.1} the danger in continued existence—\\
that life in any existence will cease to be—\\
I didn’t welcome any kind of existence, \\
and didn’t grasp at relishing.’ 

%
\end{verse}

Then\marginnote{28.1} \textsanskrit{Brahmā}, his assembly, and his retinue, their minds full of wonder and amazement, thought, ‘It’s incredible, it’s amazing! The ascetic Gotama has such psychic power and might! We’ve never before seen or heard of any other ascetic or brahmin with psychic power and might like the ascetic Gotama, who has gone forth from the Sakyan clan. Though people enjoy continued existence, loving it so much, he has extracted it, root and all.’ 

Then\marginnote{29.1} \textsanskrit{Māra} the Wicked took possession of a member of \textsanskrit{Brahmā}’s retinue and said this to me, ‘If such is your understanding, good sir, do not present it to your disciples or those gone forth! Do not teach this Dhamma to your disciples or those gone forth! Do not wish this for your disciples or those gone forth! 

There\marginnote{29.5} have been ascetics and brahmins before you, mendicant, who claimed to be perfected ones, fully awakened Buddhas. They presented, taught, and wished this for their disciples and those gone forth. When their bodies broke up and their breath was cut off they were reborn in a lower realm. 

But\marginnote{29.8} there have also been other ascetics and brahmins before you, mendicant, who claimed to be perfected ones, fully awakened Buddhas. They did not present, teach, or wish this for their disciples and those gone forth. When their bodies broke up and their breath was cut off they were reborn in a higher realm. 

So,\marginnote{29.11} mendicant, I tell you this: please, good sir, remain passive, dwelling in blissful meditation in the present life, for this is better left unsaid. Good sir, do not instruct others.’ 

When\marginnote{30.1} he had spoken, I said to \textsanskrit{Māra}, ‘I know you, Wicked One. Do not think, “He doesn’t know me.” You are \textsanskrit{Māra} the Wicked. You don’t speak to me like this out of compassion, but with no compassion. For you think, “Those who the ascetic Gotama teaches will go beyond my reach.” 

Those\marginnote{30.9} who formerly claimed to be fully awakened Buddhas were not in fact fully awakened Buddhas. But I am. The Realized One remains as such whether or not he teaches disciples. The Realized One remains as such whether or not he presents the teaching to disciples. Why is that? Because the Realized One has given up the defilements that are corrupting, leading to future lives, hurtful, resulting in suffering and future rebirth, old age, and death. He has cut them off at the root, made them like a palm stump, obliterated them so they are unable to arise in the future. Just as a palm tree with its crown cut off is incapable of further growth, the Realized One has given up the defilements that are corrupting, leading to future lives, hurtful, resulting in suffering and future rebirth, old age, and death. He has cut them off at the root, made them like a palm stump, obliterated them so they are unable to arise in the future.’” 

And\marginnote{31.1} so, because of the silencing of \textsanskrit{Māra}, and because of the invitation of \textsanskrit{Brahmā}, the name of this discussion is “On the Invitation of \textsanskrit{Brahmā}”. 

%
\section*{{\suttatitleacronym MN 50}{\suttatitletranslation The Rebuke of Māra }{\suttatitleroot Māratajjanīyasutta}}
\addcontentsline{toc}{section}{\tocacronym{MN 50} \toctranslation{The Rebuke of Māra } \tocroot{Māratajjanīyasutta}}
\markboth{The Rebuke of Māra }{Māratajjanīyasutta}
\extramarks{MN 50}{MN 50}

\scevam{So\marginnote{1.1} I have heard. }At one time Venerable \textsanskrit{Mahāmoggallāna} was staying in the land of the Bhaggas on Crocodile Hill, in the deer park at \textsanskrit{Bhesakaḷā}’s Wood. 

At\marginnote{2.1} that time \textsanskrit{Moggallāna} was walking mindfully in the open air. 

Now\marginnote{2.2} at that time \textsanskrit{Māra} the Wicked had got inside \textsanskrit{Moggallāna}’s belly. \textsanskrit{Moggallāna} thought, “Why now is my belly so very heavy, like I’ve just eaten a load of beans?” Then he stepped down from the walking path, entered his dwelling, sat down on the seat spread out, and investigated inside himself. 

He\marginnote{3.2} saw that \textsanskrit{Māra} the Wicked had got inside his belly. So he said to \textsanskrit{Māra}, “Come out, Wicked One, come out! Do not harass the Realized One or his disciple. Don’t create lasting harm and suffering for yourself!” 

Then\marginnote{4.1} \textsanskrit{Māra} thought, “This ascetic doesn’t really know me or see me when he tells me to come out. Not even the Teacher could recognize me so quickly, so how could a disciple?” 

Then\marginnote{5.1} \textsanskrit{Moggallāna} said to \textsanskrit{Māra}, “I know you even when you’re like this, Wicked One. Do not think, ‘He doesn’t know me.’ You are \textsanskrit{Māra} the Wicked. And you think, ‘This ascetic doesn’t really know me or see me when he tells me to come out. Not even the Teacher could recognize me so quickly, so how could a disciple?’” 

Then\marginnote{6.1} \textsanskrit{Māra} thought, “This ascetic really does know me and see me when he tells me to come out.” 

Then\marginnote{6.7} \textsanskrit{Māra} came up out of \textsanskrit{Moggallāna}’s mouth and stood against the door bar. \textsanskrit{Moggallāna} saw him there and said, “I see you even there, Wicked One. Do not think, ‘He doesn’t see me.’ That’s you, Wicked One, standing against the door bar. 

Once\marginnote{8.1} upon a time, Wicked One, I was a \textsanskrit{Māra} named \textsanskrit{Dūsī}, and I had a sister named \textsanskrit{Kāḷī}. You were her son, which made you my nephew. 

At\marginnote{9.1} that time Kakusandha, the Blessed One, the perfected one, the fully awakened Buddha arose in the world. Kakusandha had a fine pair of chief disciples named Vidhura and \textsanskrit{Sañjīva}. Of all the disciples of the Buddha Kakusandha, none were the equal of Venerable Vidhura in teaching Dhamma. And that’s how he came to be known as Vidhura. 

But\marginnote{10.1} when Venerable \textsanskrit{Sañjīva} had gone to a wilderness, or to the root of a tree, or to an empty hut, he easily attained the cessation of perception and feeling. Once upon a time, \textsanskrit{Sañjīva} was sitting at the root of a certain tree having attained the cessation of perception and feeling. Some cowherds, shepherds, farmers, and passers-by saw him sitting there and said, ‘It’s incredible, it’s amazing! This ascetic passed away while sitting. We should cremate him.’ They collected grass, wood, and cow-dung, heaped it all on \textsanskrit{Sañjīva}’s body, set it on fire, and left. 

Then,\marginnote{11.1} when the night had passed, \textsanskrit{Sañjīva} emerged from that attainment, shook out his robes, and, since it was morning, he robed up and entered the village for alms. Those cowherds, shepherds, farmers, and passers-by saw him wandering for alms and said, ‘It’s incredible, it’s amazing! This ascetic passed away while sitting, and now he has come back to life!’ And that’s how he came to be known as \textsanskrit{Sañjīva}. 

Then\marginnote{12.1} it occurred to \textsanskrit{Māra} \textsanskrit{Dūsī}, ‘I don’t know the course of rebirth of these ethical mendicants of good character. Why don’t I take possession of these brahmins and householders and say, “Come, all of you, abuse, attack, harass, and trouble the ethical mendicants of good character. Hopefully by doing this we can upset their minds so that \textsanskrit{Māra} \textsanskrit{Dūsī} can find a vulnerability.”’ And that’s exactly what he did. 

Then\marginnote{13.1} those brahmins and householders abused, attacked, harassed, and troubled the ethical mendicants of good character: ‘These shavelings, fake ascetics, riffraff, black spawn from the feet of our kinsman, say, “We practice absorption meditation! We practice absorption meditation!” Slouching, downcast, and dopey, they meditate and concentrate and contemplate and ruminate. They’re just like an owl on a branch, which meditates and concentrates and contemplates and ruminates as it hunts a mouse. They’re just like a jackal on a river-bank, which meditates and concentrates and contemplates and ruminates as it hunts a fish. They’re just like a cat by an alley or a drain or a dustbin, which meditates and concentrates and contemplates and ruminates as it hunts a mouse. They’re just like an unladen donkey by an alley or a drain or a dustbin, which meditates and concentrates and contemplates and ruminates. In the same way, these shavelings, fake ascetics, riffraff, black spawn from the feet of our kinsman, say, “We practice absorption meditation! We practice absorption meditation!” Slouching, downcast, and dopey, they meditate and concentrate and contemplate and ruminate.’ 

Most\marginnote{13.11} of the people who died at that time—when their body broke up, after death—were reborn in a place of loss, a bad place, the underworld, hell. 

Then\marginnote{14.1} Kakusandha the Blessed One, the perfected one, the fully awakened Buddha, addressed the mendicants: ‘Mendicants, the brahmins and householders have been possessed by \textsanskrit{Māra} \textsanskrit{Dūsī}. He told them to abuse you in the hope of upsetting your minds so that he can find a vulnerability. Come, all of you mendicants, meditate spreading a heart full of love to one direction, and to the second, and to the third, and to the fourth. In the same way above, below, across, everywhere, all around, spread a heart full of love to the whole world—abundant, expansive, limitless, free of enmity and ill will. Meditate spreading a heart full of compassion … Meditate spreading a heart full of rejoicing … Meditate spreading a heart full of equanimity to one direction, and to the second, and to the third, and to the fourth. In the same way above, below, across, everywhere, all around, spread a heart full of equanimity to the whole world—abundant, expansive, limitless, free of enmity and ill will.’ 

When\marginnote{15.1} those mendicants were instructed and advised by the Buddha Kakusandha in this way, they went to a wilderness, or to the root of a tree, or to an empty hut, where they meditated spreading a heart full of love … compassion … rejoicing … equanimity. 

Then\marginnote{16.1} it occurred to \textsanskrit{Māra} \textsanskrit{Dūsī}, ‘Even when I do this I don’t know the course of rebirth of these ethical mendicants of good character. Why don’t I take possession of these brahmins and householders and say, “Come, all of you, honor, respect, esteem, and venerate the ethical mendicants of good character. Hopefully by doing this we can upset their minds so that \textsanskrit{Māra} \textsanskrit{Dūsī} can find a vulnerability.”’ 

And\marginnote{17.1} that’s exactly what he did. Then those brahmins and householders honored, respected, esteemed, and venerated the ethical mendicants of good character. 

Most\marginnote{17.5} of the people who died at that time—when their body broke up, after death—were reborn in a good place, a heavenly realm. 

Then\marginnote{18.1} Kakusandha the Blessed One, the perfected one, the fully awakened Buddha, addressed the mendicants: ‘Mendicants, the brahmins and householders have been possessed by \textsanskrit{Māra} \textsanskrit{Dūsī}. He told them to venerate you in the hope of upsetting your minds so that he can find a vulnerability. Come, all you mendicants, meditate observing the ugliness of the body, perceiving the repulsiveness of food, perceiving dissatisfaction with the whole world, and observing the impermanence of all conditions.’ 

When\marginnote{19.1} those mendicants were instructed and advised by the Buddha Kakusandha in this way, they went to a wilderness, or to the root of a tree, or to an empty hut, where they meditated observing the ugliness of the body, perceiving the repulsiveness of food, perceiving dissatisfaction with the whole world, and observing the impermanence of all conditions. 

Then\marginnote{20.1} the Buddha Kakusandha robed up in the morning and, taking this bowl and robe, entered the village for alms with Venerable Vidhura as his second monk. 

Then\marginnote{21.1} \textsanskrit{Māra} \textsanskrit{Dūsī} took possession of a certain boy, picked up a rock, and hit Vidhura on the head, cracking it open. Then Vidhura, with blood pouring from his cracked skull, still followed behind the Buddha Kakusandha. Then the Buddha Kakusandha turned his whole body, the way that elephants do, to look back, saying, ‘This \textsanskrit{Māra} \textsanskrit{Dūsī} knows no bounds.’ And with that look \textsanskrit{Māra} \textsanskrit{Dūsī} fell from that place and was reborn in the Great Hell. 

Now\marginnote{22.1} that Great Hell is known by three names: ‘The Six Fields of Contact’ and also ‘The Impaling With Spikes’ and also ‘Individually Painful’. Then the wardens of hell came to me and said, ‘When spike meets spike in your heart, you will know that you’ve been roasting in hell for a thousand years.’ 

I\marginnote{23.1} roasted for many years, many centuries, many millennia in that Great Hell. For ten thousand years I roasted in the annex of that Great Hell, experiencing the pain called ‘this is emergence’. My body was in human form, but I had the head of a fish. 

\begin{verse}%
What\marginnote{24.1} kind of hell was that, \\
where \textsanskrit{Dūsī} was roasted \\
after attacking the disciple Vidhura \\
along with the brahmin Kakusandha? 

There\marginnote{24.5} were 100 iron spikes, \\
each one individually painful. \\
That’s the kind of hell \\
where \textsanskrit{Dūsī} was roasted \\
after attacking the disciple Vidhura \\
along with the brahmin Kakusandha. 

Dark\marginnote{24.11} One, if you attack \\
a mendicant who directly knows this, \\
a disciple of the Buddha, \\
you’ll fall into suffering. 

There\marginnote{25.1} are mansions that last for an eon \\
standing in the middle of a lake. \\
Sapphire-colored, brilliant, \\
they sparkle and shine. \\
Dancing there are nymphs \\
shining in all different colors. 

Dark\marginnote{25.7} One, if you attack \\
a mendicant who directly knows this, \\
a disciple of the Buddha, \\
you’ll fall into suffering. 

I’m\marginnote{26.1} the one who, urged by the Buddha, \\
shook the stilt longhouse of \textsanskrit{Migāra}’s mother \\
with his big toe \\
as the \textsanskrit{Saṅgha} of mendicants watched. 

Dark\marginnote{26.5} One, if you attack \\
a mendicant who directly knows this, \\
a disciple of the Buddha, \\
you’ll fall into suffering. 

I’m\marginnote{27.1} the one who shook the Palace of Victory \\
with his big toe \\
owing to psychic power, \\
inspiring deities to awe. 

Dark\marginnote{27.5} One, if you attack \\
a mendicant who directly knows this, \\
a disciple of the Buddha, \\
you’ll fall into suffering. 

I’m\marginnote{28.1} the one who asked Sakka \\
in the Palace of Victory: \\
‘\textsanskrit{Vāsava}, do you know the freedom \\
that comes with the ending of craving?’ \\
And I’m the one to whom Sakka \\
admitted the truth when asked. 

Dark\marginnote{28.7} One, if you attack \\
a mendicant who directly knows this, \\
a disciple of the Buddha, \\
you’ll fall into suffering. 

I’m\marginnote{29.1} the one who asked \textsanskrit{Brahmā} \\
in the Hall of Justice before the assembly: \\
‘Friend, do you still have the same view \\
that you had in the past? \\
Or do you see the radiance \\
transcending the \textsanskrit{Brahmā} realm?’ 

And\marginnote{29.7} I’m the one to whom \textsanskrit{Brahmā} \\
truthfully admitted his progress: \\
‘Good sir, I don’t have that view \\
that I had in the past. 

I\marginnote{29.11} see the radiance \\
transcending the \textsanskrit{Brahmā} realm. \\
So how could I say today \\
that I am permanent and eternal?’ 

Dark\marginnote{29.15} One, if you attack \\
a mendicant who directly knows this, \\
a disciple of the Buddha, \\
you’ll fall into suffering. 

I’m\marginnote{30.1} the one who has touched the peak of Mount Meru \\
using the power of meditative liberation. \\
I’ve visited the forests of the people \\
who dwell in the Eastern Continent. 

Dark\marginnote{30.5} One, if you attack \\
a mendicant who directly knows this, \\
a disciple of the Buddha, \\
you’ll fall into suffering. 

Though\marginnote{31.1} a fire doesn’t think, \\
‘I’ll burn the fool!’ \\
Still the fool who attacks \\
the fire gets burnt. 

In\marginnote{31.5} the same way, \textsanskrit{Māra}, \\
in attacking the Realized One, \\
you’ll only burn yourself, \\
like a fool touching the flames. 

\textsanskrit{Māra}’s\marginnote{31.9} done a bad thing \\
in attacking the Realized One. \\
Wicked One, do you imagine that \\
your wickedness won’t bear fruit? 

Your\marginnote{31.13} deeds heap up wickedness \\
that will last a long time, terminator! \\
Forget about the Buddha, \textsanskrit{Māra}! \\
And give up your hopes for the mendicants!” 

That\marginnote{31.17} is how, in the \textsanskrit{Bhesekaḷā} grove, \\
the mendicant rebuked \textsanskrit{Māra}. \\
That spirit, downcast, \\
disappeared right there! 

%
\end{verse}

%
\backmatter%
\chapter*{Colophon}
\addcontentsline{toc}{chapter}{Colophon}
\markboth{Colophon}{Colophon}

\section*{The Translator}

Bhikkhu Sujato was born as Anthony Aidan Best on 4/11/1966 in Perth, Western Australia. He grew up in the pleasant suburbs of Mt Lawley and Attadale alongside his sister Nicola, who was the good child. His mother, Margaret Lorraine Huntsman née Pinder, said “he’ll either be a priest or a poet”, while his father, Anthony Thomas Best, advised him to “never do anything for money”. He attended Aquinas College, a Catholic school, where he decided to become an atheist. At the University of WA he studied philosophy, aiming to learn what he wanted to do with his life. Finding that what he wanted to do was play guitar, he dropped out. His main band was named Martha’s Vineyard, which achieved modest success in the indie circuit. 

A seemingly random encounter with a roadside joey took him to Thailand, where he entered his first meditation retreat at Wat Ram Poeng, Chieng Mai in 1992. Feeling the call to the Buddha’s path, he took full ordination in Wat Pa Nanachat in 1994, where his teachers were Ajahn Pasanno and Ajahn Jayasaro. In 1997 he returned to Perth to study with Ajahn Brahm at Bodhinyana Monastery. 

He spent several years practicing in seclusion in Malaysia and Thailand before establishing Santi Forest Monastery in Bundanoon, NSW, in 2003. There he was instrumental in supporting the establishment of the Theravada bhikkhuni order in Australia and advocating for women’s rights. He continues to teach in Australia and globally, with a special concern for the moral implications of climate change and other forms of environmental destruction. He has published a series of books of original and groundbreaking research on early Buddhism. 

In 2005 he founded SuttaCentral together with Rod Bucknell and John Kelly. In 2015, seeing the need for a complete, accurate, plain English translation of the Pali texts, he undertook the task, spending nearly three years in isolation on the isle of Qi Mei off the coast of the nation of Taiwan. He completed the four main \textsanskrit{Nikāyas} in 2018, and the early books of the Khuddaka \textsanskrit{Nikāya} were complete by 2021. All this work is dedicated to the public domain and is entirely free of copyright encumbrance. 

In 2019 he returned to Sydney where he established Lokanta Vihara (The Monastery at the End of the World). 

\section*{Creation Process}

Primary source was the digital \textsanskrit{Mahāsaṅgīti} edition of the Pali \textsanskrit{Tipiṭaka}. Translated from the Pali, with reference to several English translations, especially those of Bhikkhu Bodhi.

\section*{The Translation}

This translation was part of a project to translate the four Pali \textsanskrit{Nikāyas} with the following aims: plain, approachable English; consistent terminology; accurate rendition of the Pali; free of copyright. It was made during 2016–2018 while Bhikkhu Sujato was staying in Qimei, Taiwan.

\section*{About SuttaCentral}

SuttaCentral publishes early Buddhist texts. Since 2005 we have provided root texts in Pali, Chinese, Sanskrit, Tibetan, and other languages, parallels between these texts, and translations in many modern languages. We build on the work of generations of scholars, and offer our contribution freely.

SuttaCentral is driven by volunteer contributions, and in addition we employ professional developers. We offer a sponsorship program for high quality translations from the original languages. Financial support for SuttaCentral is handled by the SuttaCentral Development Trust, a charitable trust registered in Australia.

\section*{About Bilara}

“Bilara” means “cat” in Pali, and it is the name of our Computer Assisted Translation (CAT) software. Bilara is a web app that enables translators to translate early Buddhist texts into their own language. These translations are published on SuttaCentral with the root text and translation side by side.

\section*{About SuttaCentral Editions}

The SuttaCentral Editions project makes high quality books from selected Bilara translations. These are published in formats including HTML, EPUB, PDF, and print.

If you want to print any of our Editions, please let us know and we will help prepare a file to your specifications.

%
\end{document}