\documentclass[12pt,openany]{book}%
\usepackage{lastpage}%
%
\usepackage{ragged2e}
\usepackage{verse}
\usepackage[a-3u]{pdfx}
\usepackage[inner=1in, outer=1in, top=.7in, bottom=1in, papersize={6in,9in}, headheight=13pt]{geometry}
\usepackage{polyglossia}
\usepackage[12pt]{moresize}
\usepackage{soul}%
\usepackage{microtype}
\usepackage{tocbasic}
\usepackage{realscripts}
\usepackage{epigraph}%
\usepackage{setspace}%
\usepackage{sectsty}
\usepackage{fontspec}
\usepackage{marginnote}
\usepackage[bottom]{footmisc}
\usepackage{enumitem}
\usepackage{fancyhdr}
\usepackage{emptypage}
\usepackage{extramarks}
\usepackage{graphicx}
\usepackage{relsize}
\usepackage{etoolbox}

% improve ragged right headings by suppressing hyphenation and orphans. spaceskip plus and minus adjust interword spacing; increase rightskip stretch to make it want to push a word on the first line(s) to the next line; reduce parfillskip stretch to make line length more equal . spacefillskip and xspacefillskip can be deleted to use defaults.
\protected\def\BalancedRagged{
\leftskip     0pt
\rightskip    0pt plus 10em
\spaceskip=1\fontdimen2\font plus .5\fontdimen3\font minus 1.5\fontdimen4\font
\xspaceskip=1\fontdimen2\font plus 1\fontdimen3\font minus 1\fontdimen4\font
\parfillskip  0pt plus 15em
\relax
}

\hypersetup{
colorlinks=true,
urlcolor=black,
linkcolor=black,
citecolor=black,
allcolors=black
}

% use a small amount of tracking on small caps
\SetTracking[ spacing = {25*,166, } ]{ encoding = *, shape = sc }{ 25 }

% add a blank page
\newcommand{\blankpage}{
\newpage
\thispagestyle{empty}
\mbox{}
\newpage
}

% define languages
\setdefaultlanguage[]{english}
\setotherlanguage[script=Latin]{sanskrit}

%\usepackage{pagegrid}
%\pagegridsetup{top-left, step=.25in}

% define fonts
% use if arno sanskrit is unavailable
%\setmainfont{Gentium Plus}
%\newfontfamily\Marginalfont[]{Gentium Plus}
%\newfontfamily\Allsmallcapsfont[RawFeature=+c2sc]{Gentium Plus}
%\newfontfamily\Noligaturefont[Renderer=Basic]{Gentium Plus}
%\newfontfamily\Noligaturecaptionfont[Renderer=Basic]{Gentium Plus}
%\newfontfamily\Fleuronfont[Ornament=1]{Gentium Plus}

% use if arno sanskrit is available. display is applied to \chapter and \part, subhead to \section and \subsection.
\setmainfont[
  FontFace={sb}{n}{Font = {Arno Pro Semibold}},
  FontFace={sb}{it}{Font = {Arno  Pro Semibold Italic}}
]{Arno Pro}

% create commands for using semibold
\DeclareRobustCommand{\sbseries}{\fontseries{sb}\selectfont}
\DeclareTextFontCommand{\textsb}{\sbseries}

\newfontfamily\Marginalfont[RawFeature=+subs]{Arno Pro Regular}
\newfontfamily\Allsmallcapsfont[RawFeature=+c2sc]{Arno Pro}
\newfontfamily\Noligaturefont[Renderer=Basic]{Arno Pro}
\newfontfamily\Noligaturecaptionfont[Renderer=Basic]{Arno Pro Caption}

% chinese fonts
\newfontfamily\cjk{Noto Serif TC}
\newcommand*{\langlzh}[1]{\cjk{#1}\normalfont}%

% logo
\newfontfamily\Logofont{sclogo.ttf}
\newcommand*{\sclogo}[1]{\large\Logofont{#1}}

% use subscript numerals for margin notes
\renewcommand*{\marginfont}{\Marginalfont}

% ensure margin notes have consistent vertical alignment
\renewcommand*{\marginnotevadjust}{-.17em}

% use compact lists
\setitemize{noitemsep,leftmargin=1em}
\setenumerate{noitemsep,leftmargin=1em}
\setdescription{noitemsep, style=unboxed, leftmargin=1em}

% style ToC
\DeclareTOCStyleEntries[
  raggedentrytext,
  linefill=\hfill,
  pagenumberwidth=.5in,
  pagenumberformat=\normalfont,
  entryformat=\normalfont
]{tocline}{chapter,section}


  \setlength\topsep{0pt}%
  \setlength\parskip{0pt}%

% define new \centerpars command for use in ToC. This ensures centering, proper wrapping, and no page break after
\def\startcenter{%
  \par
  \begingroup
  \leftskip=0pt plus 1fil
  \rightskip=\leftskip
  \parindent=0pt
  \parfillskip=0pt
}
\def\stopcenter{%
  \par
  \endgroup
}
\long\def\centerpars#1{\startcenter#1\stopcenter}

% redefine part, so that it adds a toc entry without page number
\let\oldcontentsline\contentsline
\newcommand{\nopagecontentsline}[3]{\oldcontentsline{#1}{#2}{}}

    \makeatletter
\renewcommand*\l@part[2]{%
  \ifnum \c@tocdepth >-2\relax
    \addpenalty{-\@highpenalty}%
    \addvspace{0em \@plus\p@}%
    \setlength\@tempdima{3em}%
    \begingroup
      \parindent \z@ \rightskip \@pnumwidth
      \parfillskip -\@pnumwidth
      {\leavevmode
       \setstretch{.85}\large\scshape\centerpars{#1}\vspace*{-1em}\llap{#2}}\par
       \nobreak
         \global\@nobreaktrue
         \everypar{\global\@nobreakfalse\everypar{}}%
    \endgroup
  \fi}
\makeatother

\makeatletter
\def\@pnumwidth{2em}
\makeatother

% define new sectioning command, which is only used in volumes where the pannasa is found in some parts but not others, especially in an and sn

\newcommand*{\pannasa}[1]{\clearpage\thispagestyle{empty}\begin{center}\vspace*{14em}\setstretch{.85}\huge\itshape\scshape\MakeLowercase{#1}\end{center}}

    \makeatletter
\newcommand*\l@pannasa[2]{%
  \ifnum \c@tocdepth >-2\relax
    \addpenalty{-\@highpenalty}%
    \addvspace{.5em \@plus\p@}%
    \setlength\@tempdima{3em}%
    \begingroup
      \parindent \z@ \rightskip \@pnumwidth
      \parfillskip -\@pnumwidth
      {\leavevmode
       \setstretch{.85}\large\itshape\scshape\lowercase{\centerpars{#1}}\vspace*{-1em}\llap{#2}}\par
       \nobreak
         \global\@nobreaktrue
         \everypar{\global\@nobreakfalse\everypar{}}%
    \endgroup
  \fi}
\makeatother

% don't put page number on first page of toc (relies on etoolbox)
\patchcmd{\chapter}{plain}{empty}{}{}

% global line height
\setstretch{1.05}

% allow linebreak after em-dash
\catcode`\—=13
\protected\def—{\unskip\textemdash\allowbreak}

% style headings with secsty. chapter and section are defined per-edition
\partfont{\setstretch{.85}\normalfont\centering\textsc}
\subsectionfont{\setstretch{.95}\normalfont\BalancedRagged}%
\subsubsectionfont{\setstretch{1}\normalfont\itshape\BalancedRagged}

% style elements of suttatitle
\newcommand*{\suttatitleacronym}[1]{\smaller[2]{#1}\vspace*{.3em}}
\newcommand*{\suttatitletranslation}[1]{\linebreak{#1}}
\newcommand*{\suttatitleroot}[1]{\linebreak\smaller[2]\itshape{#1}}

\DeclareTOCStyleEntries[
  indent=3.3em,
  dynindent,
  beforeskip=.2em plus -2pt minus -1pt,
]{tocline}{section}

\DeclareTOCStyleEntries[
  indent=0em,
  dynindent,
  beforeskip=.4em plus -2pt minus -1pt,
]{tocline}{chapter}

\newcommand*{\tocacronym}[1]{\hspace*{-3.3em}{#1}\quad}
\newcommand*{\toctranslation}[1]{#1}
\newcommand*{\tocroot}[1]{(\textit{#1})}
\newcommand*{\tocchapterline}[1]{\bfseries\itshape{#1}}


% redefine paragraph and subparagraph headings to not be inline
\makeatletter
% Change the style of paragraph headings %
\renewcommand\paragraph{\@startsection{paragraph}{4}{\z@}%
            {-2.5ex\@plus -1ex \@minus -.25ex}%
            {1.25ex \@plus .25ex}%
            {\noindent\normalfont\itshape\small}}

% Change the style of subparagraph headings %
\renewcommand\subparagraph{\@startsection{subparagraph}{5}{\z@}%
            {-2.5ex\@plus -1ex \@minus -.25ex}%
            {1.25ex \@plus .25ex}%
            {\noindent\normalfont\itshape\footnotesize}}
\makeatother

% use etoolbox to suppress page numbers on \part
\patchcmd{\part}{\thispagestyle{plain}}{\thispagestyle{empty}}
  {}{\errmessage{Cannot patch \string\part}}

% and to reduce margins on quotation
\patchcmd{\quotation}{\rightmargin}{\leftmargin 1.2em \rightmargin}{}{}
\AtBeginEnvironment{quotation}{\small}

% titlepage
\newcommand*{\titlepageTranslationTitle}[1]{{\begin{center}\begin{large}{#1}\end{large}\end{center}}}
\newcommand*{\titlepageCreatorName}[1]{{\begin{center}\begin{normalsize}{#1}\end{normalsize}\end{center}}}

% halftitlepage
\newcommand*{\halftitlepageTranslationTitle}[1]{\setstretch{2.5}{\begin{Huge}\uppercase{\so{#1}}\end{Huge}}}
\newcommand*{\halftitlepageTranslationSubtitle}[1]{\setstretch{1.2}{\begin{large}{#1}\end{large}}}
\newcommand*{\halftitlepageFleuron}[1]{{\begin{large}\Fleuronfont{{#1}}\end{large}}}
\newcommand*{\halftitlepageByline}[1]{{\begin{normalsize}\textit{{#1}}\end{normalsize}}}
\newcommand*{\halftitlepageCreatorName}[1]{{\begin{LARGE}{\textsc{#1}}\end{LARGE}}}
\newcommand*{\halftitlepageVolumeNumber}[1]{{\begin{normalsize}{\Allsmallcapsfont{\textsc{#1}}}\end{normalsize}}}
\newcommand*{\halftitlepageVolumeAcronym}[1]{{\begin{normalsize}{#1}\end{normalsize}}}
\newcommand*{\halftitlepageVolumeTranslationTitle}[1]{{\begin{Large}{\textsc{#1}}\end{Large}}}
\newcommand*{\halftitlepageVolumeRootTitle}[1]{{\begin{normalsize}{\Allsmallcapsfont{\textsc{\itshape #1}}}\end{normalsize}}}
\newcommand*{\halftitlepagePublisher}[1]{{\begin{large}{\Noligaturecaptionfont\textsc{#1}}\end{large}}}

% epigraph
\renewcommand{\epigraphflush}{center}
\renewcommand*{\epigraphwidth}{.85\textwidth}
\newcommand*{\epigraphTranslatedTitle}[1]{\vspace*{.5em}\footnotesize\textsc{#1}\\}%
\newcommand*{\epigraphRootTitle}[1]{\footnotesize\textit{#1}\\}%
\newcommand*{\epigraphReference}[1]{\footnotesize{#1}}%

% map
\newsavebox\IBox

% custom commands for html styling classes
\newcommand*{\scnamo}[1]{\begin{Center}\textit{#1}\end{Center}\bigskip}
\newcommand*{\scendsection}[1]{\begin{Center}\begin{small}\textit{#1}\end{small}\end{Center}\addvspace{1em}}
\newcommand*{\scendsutta}[1]{\begin{Center}\textit{#1}\end{Center}\addvspace{1em}}
\newcommand*{\scendbook}[1]{\bigskip\begin{Center}\uppercase{#1}\end{Center}\addvspace{1em}}
\newcommand*{\scendkanda}[1]{\begin{Center}\textbf{#1}\end{Center}\addvspace{1em}} % use for ending vinaya rule sections and also samyuttas %
\newcommand*{\scend}[1]{\begin{Center}\begin{small}\textit{#1}\end{small}\end{Center}\addvspace{1em}}
\newcommand*{\scendvagga}[1]{\begin{Center}\textbf{#1}\end{Center}\addvspace{1em}}
\newcommand*{\scrule}[1]{\textsb{#1}}
\newcommand*{\scadd}[1]{\textit{#1}}
\newcommand*{\scevam}[1]{\textsc{#1}}
\newcommand*{\scspeaker}[1]{\hspace{2em}\textit{#1}}
\newcommand*{\scbyline}[1]{\begin{flushright}\textit{#1}\end{flushright}\bigskip}
\newcommand*{\scexpansioninstructions}[1]{\begin{small}\textit{#1}\end{small}}
\newcommand*{\scuddanaintro}[1]{\medskip\noindent\begin{footnotesize}\textit{#1}\end{footnotesize}\smallskip}

\newenvironment{scuddana}{%
\setlength{\stanzaskip}{.5\baselineskip}%
  \vspace{-1em}\begin{verse}\begin{footnotesize}%
}{%
\end{footnotesize}\end{verse}
}%

% custom command for thematic break = hr
\newcommand*{\thematicbreak}{\begin{center}\rule[.5ex]{6em}{.4pt}\begin{normalsize}\quad\Fleuronfont{•}\quad\end{normalsize}\rule[.5ex]{6em}{.4pt}\end{center}}

% manage and style page header and footer. "fancy" has header and footer, "plain" has footer only

\pagestyle{fancy}
\fancyhf{}
\fancyfoot[RE,LO]{\thepage}
\fancyfoot[LE,RO]{\footnotesize\lastleftxmark}
\fancyhead[CE]{\setstretch{.85}\Noligaturefont\MakeLowercase{\textsc{\firstrightmark}}}
\fancyhead[CO]{\setstretch{.85}\Noligaturefont\MakeLowercase{\textsc{\firstleftmark}}}
\renewcommand{\headrulewidth}{0pt}
\fancypagestyle{plain}{ %
\fancyhf{} % remove everything
\fancyfoot[RE,LO]{\thepage}
\fancyfoot[LE,RO]{\footnotesize\lastleftxmark}
\renewcommand{\headrulewidth}{0pt}
\renewcommand{\footrulewidth}{0pt}}
\fancypagestyle{plainer}{ %
\fancyhf{} % remove everything
\fancyfoot[RE,LO]{\thepage}
\renewcommand{\headrulewidth}{0pt}
\renewcommand{\footrulewidth}{0pt}}

% style footnotes
\setlength{\skip\footins}{1em}

\makeatletter
\newcommand{\@makefntextcustom}[1]{%
    \parindent 0em%
    \thefootnote.\enskip #1%
}
\renewcommand{\@makefntext}[1]{\@makefntextcustom{#1}}
\makeatother

% hang quotes (requires microtype)
\microtypesetup{
  protrusion = true,
  expansion  = true,
  tracking   = true,
  factor     = 1000,
  patch      = all,
  final
}

% Custom protrusion rules to allow hanging punctuation
\SetProtrusion
{ encoding = *}
{
% char   right left
  {-} = {    , 500 },
  % Double Quotes
  \textquotedblleft
      = {1000,     },
  \textquotedblright
      = {    , 1000},
  \quotedblbase
      = {1000,     },
  % Single Quotes
  \textquoteleft
      = {1000,     },
  \textquoteright
      = {    , 1000},
  \quotesinglbase
      = {1000,     }
}

% make latex use actual font em for parindent, not Computer Modern Roman
\AtBeginDocument{\setlength{\parindent}{1em}}%
%

% Default values; a bit sloppier than normal
\tolerance 1414
\hbadness 1414
\emergencystretch 1.5em
\hfuzz 0.3pt
\clubpenalty = 10000
\widowpenalty = 10000
\displaywidowpenalty = 10000
\hfuzz \vfuzz
 \raggedbottom%

\title{Middle Discourses}
\author{Bhikkhu Sujato}
\date{}%
% define a different fleuron for each edition
\newfontfamily\Fleuronfont[Ornament=4]{Arno Pro}

% Define heading styles per edition for chapter, section, and subsection. Suttatitle can be any one of these, depending on the volume. 

\let\oldfrontmatter\frontmatter
\renewcommand{\frontmatter}{%
\chapterfont{\setstretch{.85}\normalfont\centering}%
\sectionfont{\setstretch{.85}\normalfont\BalancedRagged}%
\oldfrontmatter}

\let\oldmainmatter\mainmatter
\renewcommand{\mainmatter}{%
\chapterfont{\thispagestyle{empty}\vspace*{4em}\setstretch{.85}\LARGE\normalfont\itshape\scshape\centering\MakeLowercase}
\sectionfont{\clearpage\thispagestyle{plain}\vspace*{2em}\setstretch{.85}\normalfont\centering}%
\oldmainmatter}

\let\oldbackmatter\backmatter
\renewcommand{\backmatter}{%
\chapterfont{\setstretch{.85}\normalfont\centering}%
\sectionfont{\setstretch{.85}\normalfont\BalancedRagged}%
\pagestyle{plainer}%
\oldbackmatter}
%
%
\begin{document}%
\normalsize%
\frontmatter%
\setlength{\parindent}{0cm}

\pagestyle{empty}

\maketitle

\blankpage%
\begin{center}

\vspace*{2.2em}

\halftitlepageTranslationTitle{Middle Discourses}

\vspace*{1em}

\halftitlepageTranslationSubtitle{A lucid translation of the Majjhima Nikāya}

\vspace*{2em}

\halftitlepageFleuron{•}

\vspace*{2em}

\halftitlepageByline{translated and introduced by}

\vspace*{.5em}

\halftitlepageCreatorName{Bhikkhu Sujato}

\vspace*{4em}

\halftitlepageVolumeNumber{Volume 1}

\smallskip

\halftitlepageVolumeAcronym{MN 1–50}

\smallskip

\halftitlepageVolumeTranslationTitle{The First Fifty}

\smallskip

\halftitlepageVolumeRootTitle{Mūlapaṇṇāsa}

\vspace*{\fill}

\sclogo{0}
 \halftitlepagePublisher{SuttaCentral}

\end{center}

\newpage
%
\setstretch{1.05}

\begin{footnotesize}

\textit{Middle Discourses} is a translation of the Majjhimanikāya by Bhikkhu Sujato.

\medskip

Creative Commons Zero (CC0)

To the extent possible under law, Bhikkhu Sujato has waived all copyright and related or neighboring rights to \textit{Middle Discourses}.

\medskip

This work is published from Australia.

\begin{center}
\textit{This translation is an expression of an ancient spiritual text that has been passed down by the Buddhist tradition for the benefit of all sentient beings. It is dedicated to the public domain via Creative Commons Zero (CC0). You are encouraged to copy, reproduce, adapt, alter, or otherwise make use of this translation. The translator respectfully requests that any use be in accordance with the values and principles of the Buddhist community.}
\end{center}

\medskip

\begin{description}
    \item[Web publication date] 2018
    \item[This edition] 2025-01-13 01:01:43
    \item[Publication type] hardcover
    \item[Edition] ed3
    \item[Number of volumes] 3
    \item[Publication ISBN] 978-1-76132-058-3
    \item[Volume ISBN] 978-1-76132-059-0
    \item[Publication URL] \href{https://suttacentral.net/editions/mn/en/sujato}{https://suttacentral.net/editions/mn/en/sujato}
    \item[Source URL] \href{https://github.com/suttacentral/bilara-data/tree/published/translation/en/sujato/sutta/mn}{https://github.com/suttacentral/bilara-data/tree/published/translation/en/sujato/sutta/mn}
    \item[Publication number] scpub3
\end{description}

\medskip

Map of Jambudīpa is by Jonas David Mitja Lang, and is released by him under Creative Commons Zero (CC0).

\medskip

Published by SuttaCentral

\medskip

\textit{SuttaCentral,\\
c/o Alwis \& Alwis Pty Ltd\\
Kaurna Country,\\
Suite 12,\\
198 Greenhill Road,\\
Eastwood,\\
SA 5063,\\
Australia}

\end{footnotesize}

\newpage

\setlength{\parindent}{1em}%%
\newpage

\vspace*{\fill}

\begin{center}
\epigraph{The sage at peace is not reborn, does not grow old, and does not die. They are not shaken, and do not yearn. For they have nothing which would cause them to be reborn. Not being reborn, how could they grow old? Not growing old, how could they die? Not dying, how could they be shaken? Not shaking, for what could they yearn?}
{
\epigraphTranslatedTitle{“The Analysis of the Elements”}
\epigraphRootTitle{\textsanskrit{Dhātuvibhaṅgasutta}}
\epigraphReference{Majjhima \textsanskrit{Nikāya} 140}
}
\end{center}

\vspace*{2in}

\vspace*{\fill}

\newgeometry{inner=0mm, outer=.5in, top=.6in, bottom=0mm}
\setlength{\parindent}{0em}
\sbox\IBox{\includegraphics{/app/sutta_publisher/images/jambudipa_map.png}}%
\includegraphics[trim=0 0 \dimexpr\wd\IBox-\textwidth{} 0,clip]{/app/sutta_publisher/images/jambudipa_map.png}
\newpage
\includegraphics[trim=\textwidth{} 0 0 0,clip]{/app/sutta_publisher/images/jambudipa_map.png}
\newpage
\restoregeometry

\blankpage%

\setlength{\parindent}{1em}
%
\tableofcontents
\newpage
\pagestyle{fancy}
%
\chapter*{The SuttaCentral Editions Series}
\addcontentsline{toc}{chapter}{The SuttaCentral Editions Series}
\markboth{The SuttaCentral Editions Series}{The SuttaCentral Editions Series}

Since 2005 SuttaCentral has provided access to the texts, translations, and parallels of early Buddhist texts. In 2018 we started creating and publishing our translations of these seminal spiritual classics. The “Editions” series now makes selected translations available as books in various forms, including print, PDF, and EPUB.

Editions are selected from our most complete, well-crafted, and reliable translations. They aim to bring these texts to a wider audience in forms that reward mindful reading. Care is taken with every detail of the production, and we aim to meet or exceed professional best standards in every way. These are the core scriptures underlying the entire Buddhist tradition, and we believe that they deserve to be preserved and made available in the highest quality without compromise.

SuttaCentral is a charitable organization. Our work is accomplished by volunteers and through the generosity of our donors. Everything we create is offered to all of humanity free of any copyright or licensing restrictions. 

%
\chapter*{Preface to \emph{Middle Discourses}}
\addcontentsline{toc}{chapter}{Preface to \emph{Middle Discourses}}
\markboth{Preface to \emph{Middle Discourses}}{Preface to \emph{Middle Discourses}}

It was in 1992 that I first encountered the Buddha’s words. I was in Chieng Mai, having just completed my very first meditation retreat, a month-long \emph{\textsanskrit{vipassanā}} intensive in the Mahasi style. During the retreat, the guides at Wat Ram Poeng told us not to read any books, but to learn only from our own experience. This was an entirely novel concept for me at the time. The retreat, quite frankly, blew my mind.

When the retreat was over reading was allowed, and as a confirmed bookworm from childhood, I couldn’t wait to start making sense of all that had happened on my retreat. They gave me some of the then-available books, collections of Dhamma talks and the like. They were fine, but nothing that really sparked my enthusiasm. 

I went back and said, “They talk about these things called ‘the Suttas’. What are they?”

This was the days before the internet and before the excellent editions by Bhikkhu Bodhi. What they had was the selected discourses from the Majjhima \textsanskrit{Nikāya} translated by Bhikkhu \textsanskrit{Ñāṇamoḷi}, edited and compiled by Bhikkhu Khantipalo as \textit{A Treasury of the Buddha’s Words}. Very excited, I took the first volume back to my room and began to read.

Right away I knew. There was something about it. It was not at all what I was expecting. I don’t know what I was expecting, but this wasn’t it. It was so different than my usual reading. At once transparent and obscure; forbidding and intimate. The Buddha’s voice was refreshingly free of the apologetics and mysticism I associated with “spiritual” literature. He was direct, blunt even. He spoke with the quiet authority of someone who knew what he was talking about.

I took it slow, reading a Sutta per day. There was much that I didn’t understand, even with the notes and introductions. But I knew that there were great depths there. This was the Buddha! This would be just the start of a long journey. So I read, slowly, going back over it again and again, digesting as much as I could. When I had the time, I’d read for an hour a day and meditate for a few hours. The teachers at Wat Ram Poeng told us to sit mindfully when learning Dhamma, so I did. I brought as much care and mindful attention to reading as I did to meditation.

Gradually I made my way through the Majjhima \textsanskrit{Nikāya}, learning volumes along the way. I remember some months after that retreat I was staying in Mae Hong Son, at a farm for orphaned children called Buddhakaset. I was there as a volunteer for a while. When the kids had gone to bed, I’d sit in the evening under the shelter of a simple wooden verandah. I’d carefully open my book of Suttas and read the next one, never quite knowing what was to come. There are many things that most Buddhists will encounter and know through Buddhist culture, then rediscover when reading the Suttas. “Ahh, that’s where that came from!” For me it was the opposite: I discovered things in the Suttas, then noticed them in Buddhist culture.

I continued these practices through my early years as a monk. Gradually, I read my way through the Suttas, often in the archaic translations that were then available. But I kept the simple approach I had from the start. Read a bit, meditate a lot. Things that I read then have stayed with me until the present.

I reread the Majjhima \textsanskrit{Nikāya} in other translations. First that by I.B. Horner for the Pali Text Society, which was roughly contemporary to that of \textsanskrit{Ñāṇamoḷi}, but which in publication and language seemed of an earlier generation. Then Bhikkhu Bodhi’s edition came out, a completely revised and seemingly perfected translation based on that of \textsanskrit{Ñāṇamoḷi}. Exciting times! I was beginning my Pali studies, but I never expected to one day be making my own translation.

I hope that you might experience something of the joy and inspiration that I found in my discovery of the Suttas. At this point, having been a monk for twenty-five years, it is redundant to say that these teachings changed my life. They might change yours, too, if you let them.

%
\chapter*{The Middle Discourses: conversations on matters of deep truth}
\addcontentsline{toc}{chapter}{The Middle Discourses: conversations on matters of deep truth}
\markboth{The Middle Discourses: conversations on matters of deep truth}{The Middle Discourses: conversations on matters of deep truth}

\scbyline{Bhikkhu Sujato, 2019}

The Majjhima \textsanskrit{Nikāya} is the second of the four main divisions in the Sutta \textsanskrit{Piṭaka} of the Pali Canon (\textit{\textsanskrit{tipiṭaka}}). It is translated here as \textit{Middle Discourses}, and is sometimes known as the \textit{Middle-Length Discourses}. As the title suggests, its discourses are somewhat shorter than those of the \textsanskrit{Dīgha} \textsanskrit{Nikāya} (\textit{Long Discourses}), but longer than the many short discourses collected in the \textsanskrit{Saṁyutta} \textsanskrit{Nikāya} (\textit{Linked Discourses}) and \textsanskrit{Aṅguttara} \textsanskrit{Nikāya} (\textit{Numbered Discourses}).

In the classic introduction to his translation of the Majjhima \textsanskrit{Nikāya}, Bhikkhu Bodhi described the Majjhima as “the collection that combines the richest variety of contextual settings with the deepest and most comprehensive assortment of teachings”. He went on to situate it among the other \textit{\textsanskrit{nikāyas}}:

\begin{quotation}%
Like the \textsanskrit{Dīgha} \textsanskrit{Nikāya}, the Majjhima is replete with drama and narrative, while lacking much of its predecessor’s tendency towards imaginative embellishment and profusion of legend. Like the \textsanskrit{Saṁyutta}, it contains some of the profoundest discourses in the Canon, disclosing the Buddha’s radical insights into the nature of existence; and like the \textsanskrit{Aṅguttara}, it covers a wide range of topics of practical applicability. In contrast to those two \textsanskrit{Nikāyas}, however, the Majjhima sets forth this material not in the form of short, self-contained utterances, but in the context of a fascinating procession of scenarios that exhibit the Buddha’s resplendence of wisdom, his skill in adapting his teachings to the needs and proclivities of his interlocutors, his wit and gentle humour, his majestic sublimity, and his compassionate humanity.

%
\end{quotation}

In this guide, I describe some of the special features of the Majjhima. After noting some formal and doctrinal features, I focus this essay on the Buddha and the people he encountered, whether his monastic Sangha, his lay followers, or the many non-Buddhists he spoke with. The Majjhima includes many of the most important autobiographical discourses, so it is a natural place to discuss the Buddha’s life and person. And the dialogue format makes the Majjhima a specially rich context to see how the Dhamma emerged through interaction and conversation with people of all kinds.

\section*{How the Majjhima is Organized}

There are 152 discourses in the Majjhima. These are collected into groups of 50 discourses (\textit{\textsanskrit{paṇṇāsa}}), although the final \textit{\textsanskrit{paṇṇāsa}} contains 52.

Within each \textit{\textsanskrit{paṇṇāsa}} is a set of five \textit{vaggas}. As usual, most of the \textit{vaggas} are simply named after their first sutta, but a few exhibit some thematic unity:

\begin{description}%
\item[\textbf{Opamma Vagga (Chapter With Similes)}] All these discourses prominently feature similes.%
\item[\textbf{\textsanskrit{Mahāyamaka} Vagga (Large Chapter With Pairs)}] Paired “long” and “short” discourses. (The following Short Chapter With Pairs only has two sets of pairs.)%
\item[\textbf{Vaggas 6–10}] These collect discourses featuring certain kinds of people: householders, mendicants, wanderers, kings, and brahmins.%
\item[\textbf{\textsanskrit{Vibhaṅga} Vagga (Chapter on Analysis)}] These discourses consist of a lengthy “analysis” of a short saying.%
\item[\textbf{\textsanskrit{Saḷāyatana} Vagga (Chapter on the Six Senses)}] These focus on the six senses.%
\end{description}

\section*{Imagery and Narrative}

The Majjhima includes an astonishing range of imagery, with similes found in almost all discourses. Sometimes these are extended to short parables. \href{https://suttacentral.net/mn21}{MN 21} \textit{The Simile of the Saw} (\textit{\textsanskrit{Kakacūpamasutta}}) tells the memorable story of the bold maid \textsanskrit{Kāḷī} who tested her mistress. In \href{https://suttacentral.net/mn56\#27}{MN 56:27} \textit{With \textsanskrit{Upāli}} we meet the brahmin lady who wanted to not only dye her pet monkey, but press him and wring him out.

While the Buddha himself speaks only short tales, in the background narratives we find more developed narratives, including famous stories such as the taming of the vicious serial-killer \textsanskrit{Aṅgulimāla} (\href{https://suttacentral.net/mn86}{MN 86}) or the uncompromising commitment of the wealthy youth \textsanskrit{Raṭṭhapāla} (\href{https://suttacentral.net/mn82}{MN 82}), who defied his parents’ will to take ordination, and whose discourse finishes with an extraordinary set of teachings for his king.

Some stories are less earthbound. \href{https://suttacentral.net/mn37}{MN 37} \textit{The Shorter Discourse on the Ending of Craving} (\textit{\textsanskrit{Cūḷataṇhāsaṅkhayasutta}}) depicts \textsanskrit{Moggallāna} ascending to the heaven of Sakka, the lord of gods, to check his indulgence. \href{https://suttacentral.net/mn50}{MN 50} \textit{The Rebuke of \textsanskrit{Māra}} (\textit{\textsanskrit{Māratajjanīyasutta}}) also features \textsanskrit{Moggallāna}, this time in a dialogue with \textsanskrit{Māra}, the lord of deceit and death; and it contains the startling revelation that \textsanskrit{Moggallāna} himself was a \textsanskrit{Māra} in a past life. Such cosmic drama reaches its apex in \href{https://suttacentral.net/mn49}{MN 49} \textit{On the Invitation of \textsanskrit{Brahmā}} (\textit{Brahmanimantanikasutta}), where the Buddha takes on no less than \textsanskrit{Brahmā} himself in a high-level philosophical debate, forcing \textsanskrit{Māra} to reveal himself on the side of \textsanskrit{Brahmā}. From the perspective of early Buddhism, God and the Devil are not so very different.

\section*{Theory \& Practice}

The Majjhima is perhaps the richest of the early Buddhist collections in matters of doctrine. It contains an extraordinary series of discourses that delve into profound topics with detail and complexity not found elsewhere. It’s worth bearing in mind, though, that such discourses are for advanced students, and fascinating as they are, it is important to get a solid grounding on the fundamental doctrines collected in the \textsanskrit{Saṁyutta}. For this reason, I reserve most doctrinal explanations for my guide to the Linked Discourses and make only a few brief remarks here.

The teachings familiar from the \textsanskrit{Saṁyutta} are all found in the Majjhima, and in several cases, the Majjhima offers more detailed explanations. These discourses are important and deserve close study, but beware of equating length with importance. In Buddhist texts it’s just as likely that length implies, not that it is something the Buddha regarded as important, but that it is a late compilation.

Such is the case with \href{https://suttacentral.net/mn10}{MN 10} \textit{Mindfulness Meditation} (\textit{\textsanskrit{Satipaṭṭhānasutta}}), which, together with its expanded version at \href{https://suttacentral.net/dn22}{DN 22}, is the most detailed explanation of mindfulness meditation in the canon. Yet critical and comparative analysis reveals that the discourse as found in the Pali has been subject to considerable late development. \href{https://suttacentral.net/mn141}{MN 141} \textit{The Analysis of the Truths} (\textit{\textsanskrit{Saccavibhaṅgasutta}}), the most detailed discourse of the four noble truths, is closely related to \href{https://suttacentral.net/mn10}{MN 10}—in fact \href{https://suttacentral.net/dn22}{DN 22} is virtually a combination of this and \href{https://suttacentral.net/mn10}{MN 10}—and it too might be suspected to have late features. No such doubt attends to \href{https://suttacentral.net/mn111}{MN 111} \textit{One by One} (\textit{Anupadasutta}), which is clearly a late sutta. This is not the place for a complex discussion of text-critical method, but it is common and natural to assume that length implies importance, and it is worth bearing in mind that the situation is more complicated than that.

Majjhima suttas that deal with key doctrinal teachings can be understood as offering in-depth analyses of particular factors of the four noble truths. The first truth of suffering is explored in detail in \href{https://suttacentral.net/mn13}{MN 13} and \href{https://suttacentral.net/mn14}{MN 14} on the “Mass of Suffering”. Various topics under this heading are also treated in detail; the six sense fields are taught in several suttas (\href{https://suttacentral.net/mn18}{MN 18}, \href{https://suttacentral.net/mn137}{MN 137}, \href{https://suttacentral.net/mn138}{MN 138}) and even an entire \textit{vagga} (\href{https://suttacentral.net/mn143}{MN 143}–152), while several suttas investigate the teaching on the elements in great detail, exposing depths that one might not suspect for what appears to be such a simple teaching (\href{https://suttacentral.net/mn28}{MN 28}, \href{https://suttacentral.net/mn115}{MN 115}, \href{https://suttacentral.net/mn140}{MN 140}). The aggregates appear, but are treated in less detail.

The second and third noble truths are featured in \href{https://suttacentral.net/mn38}{MN 38}, a complex and rewarding discourse on dependent origination, as well as \href{https://suttacentral.net/mn135}{MN 135} and \href{https://suttacentral.net/mn136}{MN 136}, two of the most detailed and influential discourses on the topic of \textit{kamma}; see too \href{https://suttacentral.net/mn120}{MN 120}.

But it is the fourth noble truth, the path, that dominates the Majjhima. The vast majority of discourses deal with the path as a central topic. The noble eightfold path is treated in complex detail at \href{https://suttacentral.net/mn117}{MN 117} \textit{The Great Forty} (\textit{\textsanskrit{Mahācattārīsakasutta}}), and several discourses deal with specific path factors. In \href{https://suttacentral.net/mn9}{MN 9}, Venerable \textsanskrit{Sāriputta} presents the topic of right view from a diverse range of perspectives. The second path factor, right thought or right intention, is the special subject of \href{https://suttacentral.net/mn19}{MN 19} and \href{https://suttacentral.net/mn20}{MN 20}, which give advice from the Buddha’s own experience on how to first purify thought and then let it go entirely. The path factors on ethics are treated in too many suttas to be summarized here; some of these are covered in the sections on the \textsanskrit{Saṅgha} and the lay communities. Right effort is featured in many suttas but is especially emphasized in texts such as \href{https://suttacentral.net/mn29}{MN 29}, \href{https://suttacentral.net/mn30}{MN 30}, and \href{https://suttacentral.net/mn32}{MN 32}. \href{https://suttacentral.net/mn10}{MN 10}, as noted, deals with right mindfulness, and the topic is treated from more specialized angles in \href{https://suttacentral.net/mn118}{MN 118} on mindfulness of breathing, and \href{https://suttacentral.net/mn119}{MN 119} on mindfulness of the body. In \href{https://suttacentral.net/mn77}{MN 77} we find a long list of different presentations of the path, including the topics found as heads in the \textsanskrit{Saṁyutta}, and quite a few more.

Right immersion or \emph{\textsanskrit{jhāna}}, the final factor of the path, is prominent throughout the Majjhima. It is difficult to overstate how central the \textit{\textsanskrit{jhānas}} are to Buddhist meditation. The formula for the four absorptions appears around 50 times in the Majjhima, which is probably more than the formulas for all the other path factors \emph{combined}. Moreover, right immersion appears also in other guises, such as the four “divine meditations” (\textit{\textsanskrit{brahmavihāra}}), where the pure emotions of love, compassion, rejoicing, and equanimity serve as a basis for immersion. These appear around 13 times in the Majjhima. The even more subtle “formless” meditations are also a frequent topic, also appearing no less than 13 times. Unlike the absorptions, these are not an absolute requirement for the path to awakening, but clearly they were part of the practice for many talented meditators. Discourses such as \href{https://suttacentral.net/mn52}{MN 52} \textit{The Man From the City of \textsanskrit{Aṭṭhaka}} (\textit{\textsanskrit{Aṭṭhakanāgarasutta}}) combine these three sets of meditations, while advanced texts such as \href{https://suttacentral.net/mn43}{MN 43} \textit{The Great Classification} (\textit{\textsanskrit{Mahāvedallasutta}}), \href{https://suttacentral.net/mn44}{MN 44} \textit{The Shorter Classification} (\textit{\textsanskrit{Cūḷavedallasutta}}), \href{https://suttacentral.net/mn106}{MN 106} \textit{\textsanskrit{Āneñjasappāyasutta}}, \href{https://suttacentral.net/mn121}{MN 121} \textit{The Shorter Discourse on Emptiness} (\textit{\textsanskrit{Cūḷasuññatasutta}}), and \href{https://suttacentral.net/mn122}{MN 122} \textit{The Longer Discourse on Emptiness} (\textit{\textsanskrit{Mahāsuññatasutta}}) deal with rarely-discussed subtleties and refinements pertaining to the most advanced forms of meditation.

\section*{The Buddha}

The story of the Buddha’s life has become a primary vehicle for sharing and passing down Buddhist teachings and values in the Buddhist traditions. Yet there is no coherent biography of the Buddha in the early texts. Such information as we do have is scattered and piecemeal, found in the occasional details shared by the Buddha himself or inferred from the various situations in which we find him.

But perhaps this shouldn’t come as such a surprise. There’s no particular reason for the Buddha to be interested in telling his life story—he had lived it. And his disciples knew him personally. Only after he had passed away did the community feel the need to keep their teacher’s memory alive through vivid and detailed stories, growing more elaborate with every telling.

The broad outlines of the later Buddha legends grew out of the kernels in the early texts, many of which are found in the Majjhima. There we find the Buddha’s birth, his early upbringing, renunciation, practice and awakening, challenges involved in setting up a community, and various encounters while teaching. In the \textsanskrit{Dīgha} we find the longest narrative of early Buddhism, an extensive record of the Buddha’s last days found in \href{https://suttacentral.net/dn16}{DN 16} \textit{\textsanskrit{Mahāparinibbāna}}. And scattered throughout the texts we find isolated events and encounters. In one sense, most of the discourses can be considered as episodes of the Buddha’s life, vignettes in a magnificent myth, each one contributing a little to understanding the man and his message. These texts form our primary source of knowledge for the Buddha’s early life and teaching career. While the tendency towards legend-building is apparent in some places, most of these episodes are simple and realistic.

An exception is \href{https://suttacentral.net/mn123}{MN 123} \textit{Incredible and Amazing} (\textit{Acchariyaabbhutasutta}), where we find Ānanda, the founder of Buddhist biography, recounting an extended series of apparently miraculous events that accompanied the birth of the Buddha-to-be. This is evidently derived from \href{https://suttacentral.net/dn14}{DN 14} \textit{\textsanskrit{Mahāpadāna}} with some expansions. With its devotional tone and emphasis on the extraordinary, this text shows a shift in emphasis towards honoring the person of the Buddha rather than practicing his teachings; a shift that the Buddha resists by pointing out that the truly extraordinary thing is to be aware of one’s own mind.

The Buddha’s immediate family is mentioned only rarely in the \textit{\textsanskrit{nikāyas}}. The Buddha’s wife appears only in the Vinaya and \textsanskrit{Therīgāthā}, where she has some verses. His son \textsanskrit{Rāhula} is prominently featured in several discourses (\href{https://suttacentral.net/mn61}{MN 61}, \href{https://suttacentral.net/mn62}{MN 62}, \href{https://suttacentral.net/mn147}{MN 147}), showing the Buddha’s patient teaching and \textsanskrit{Rāhula}’s eventual awakening. The Buddha’s father is briefly mentioned in the passage below on meditation as a child. Both his parents are mentioned in several Majjhima discourses (\href{https://suttacentral.net/mn26}{MN 26}, \href{https://suttacentral.net/mn36}{MN 36}, \href{https://suttacentral.net/mn85}{MN 85}, \href{https://suttacentral.net/mn100}{MN 100}), but they are only named elsewhere: Suddhodana his father in \href{https://suttacentral.net/snp3.11}{Snp 3.11} and \href{https://suttacentral.net/pli-tv-kd1.\#54}{Kd 1:54}; \textsanskrit{Māyā} his mother in \href{https://suttacentral.net/thig6.6}{Thig 6.6}; and both in \href{https://suttacentral.net/dn14}{DN 14} and \href{https://suttacentral.net/thag10.1}{Thag 10.1}. His stepmother \textsanskrit{Mahāpajāpatī} \textsanskrit{Gotamī}, however, appears in a few discourses, notably \href{https://suttacentral.net/mn142}{MN 142} \textit{The Analysis of Religious Donations} (\textit{\textsanskrit{Dakkhiṇāvibhaṅgasutta}}), and \href{https://suttacentral.net/an8.51}{AN 8.51}. More distant relatives include several well-known monastic and lay figures such as Ānanda, Anuruddha, and \textsanskrit{Mahānāma}.

In \href{https://suttacentral.net/mn75\#10}{MN 75:10} \textit{With \textsanskrit{Māgaṇḍiya}} the Buddha recounts the luxuries he enjoyed as a young man (cp. \href{https://suttacentral.net/an3.39}{AN 3.39}), and in \href{https://suttacentral.net/mn26\#13}{MN 26:13} he tells of the painful moment when he left home, though his parents wept in distress. An alternate account of the going forth is found in the distinctive and early \textsanskrit{Attadaṇḍa} Sutta, where going forth is prompted not by the sight of old age, sickness, and death, but by seeing the unceasing conflict and distress of the world (\href{https://suttacentral.net/snp4.15}{Snp 4.15}).

There follows the story of the six years of striving, divided into three periods. It seems that he first practiced under a Brahmanical tradition, probably following the \textsanskrit{Upaniṣadic} philosophy. \href{https://suttacentral.net/mn26}{MN 26} \textit{The Noble Search} (\textit{Ariyapariyesanasutta}, also known as the \textit{\textsanskrit{Pāsarāsisutta}}) tells of his experience under the famed spiritual guides \textsanskrit{Āḷāra} \textsanskrit{Kālāma} and Uddaka \textsanskrit{Rāmaputta}. Under their tutelage, he realized deep immersion (\textit{\textsanskrit{samādhi}}) and formless attainments, but he was still unsatisfied, so he left to embark on a more severe ascetic path.

According to the traditions, the bulk of this period was spent enduring harsh austerities, practices that are similar or identical to those observed by the Jains. These are described in detail in \href{https://suttacentral.net/mn12}{MN 12} \textit{The Longer Discourse on the Lion’s Roar} (\textit{\textsanskrit{Mahāsīhanādasutta}}) and \href{https://suttacentral.net/mn36}{MN 36} \textit{The Longer Discourse With Saccaka}. But after many years of such self-mortification, he was getting nowhere. So he took some solid food and, recalling his childhood experience of absorption, began the third and final phase of his practice, which led to his awakening.

While it might seem as if the night of awakening followed directly from his rejection of austerities, several discourses indicate that this period involved a rather extensive development of meditation. \href{https://suttacentral.net/mn19}{MN 19} \textit{Two Kinds of Thought} (\textit{\textsanskrit{Dvedhāvitakkasutta}}), for example, tells of his analysis and training in wholesome thought; \href{https://suttacentral.net/mn4}{MN 4} \textit{Fear and Terror} (\textit{Bhayabheravasutta}) depicts his struggles to overcome fear; and \href{https://suttacentral.net/mn128}{MN 128} \textit{Corruptions} (\textit{Upakkilesasutta}) speaks of a series of specific meditative hindrances he encountered (see too \href{https://suttacentral.net/an8.64}{AN 8.64} and \href{https://suttacentral.net/an9.40}{AN 9.40}). While his reflections on pursuing a path of practice are always logical, in \href{https://suttacentral.net/an5.196}{AN 5.196} we read about an extraordinary series of dreams that foretold his awakening; the symbolism of the dream imagery repays careful attention.

It is in the Majjhima (\href{https://suttacentral.net/mn36}{MN 36}, \href{https://suttacentral.net/mn85}{MN 85}, \href{https://suttacentral.net/mn100}{MN 100}) that the Buddha speaks of that crucial moment in his childhood where he spontaneously entered the first absorption, a state of profound peace and stillness in meditation. Much later, when he arrived at a crisis in his spiritual progress, he was to recall this event and realize that meditative absorption was the only true path to awakening (\textit{eso’va maggo \textsanskrit{bodhāya}}).

The Buddha’s awakening is told from several different perspectives. In \href{https://suttacentral.net/mn4}{MN 4} \textit{Bhayabherava}, after describing how he overcame his fears, the Buddha tells how he developed the absorptions and gained the three higher knowledges. In \href{https://suttacentral.net/mn14}{MN 14} \textit{The Shorter Discourse on the Mass of Suffering} (\textit{\textsanskrit{Cūḷadukkhakkhandha}}) he speaks of the escape from suffering by letting go of sensual pleasures; see too \href{https://suttacentral.net/sn35.117}{SN 35.117}. Several discourses employ the stock framework of understanding the three aspects of gratification, drawback, and escape, applying this to the five aggregates (\href{https://suttacentral.net/sn22.26}{SN 22.26}), the elements (\href{https://suttacentral.net/sn14.31}{SN 14.31}), feelings (\href{https://suttacentral.net/sn36.24}{SN 36.24}), or the world (\href{https://suttacentral.net/an3.101}{AN 3.101}). Elsewhere awakening is seen as a result of understanding dependent origination (\href{https://suttacentral.net/sn12.10}{SN 12.10}, \href{https://suttacentral.net/dn14}{DN 14}). Other contexts depict awakening as emerging from different practices, such as the four kinds of mindfulness meditation (\href{https://suttacentral.net/sn47.31}{SN 47.31}), the four bases of psychic powers (\href{https://suttacentral.net/sn51.9}{SN 51.9}), or the abandoning of thoughts (\href{https://suttacentral.net/mn19}{MN 19}). This is far from an exhaustive list, as all of the Buddha's teachings depict different aspects of the wisdom that stems from awakening.

The period after the awakening is told in some detail, recounting the Buddha’s journeys, various encounters along the way, his first conversions, and setting up the Sangha. However the detailed account of this is in the Vinaya (\href{https://suttacentral.net/pli-tv-kd1}{Kd 1}), and only portions of these events are found in the \textit{\textsanskrit{nikāyas}}, such as the first three sermons (\href{https://suttacentral.net/sn56.11}{SN 56.11}, \href{https://suttacentral.net/sn22.59}{SN 22.59}, \href{https://suttacentral.net/sn35.28}{SN 35.28}).

The Buddha followed the same simple lifestyle as his monastic disciples. His possessions consisted of a single set of three robes, a bowl, and a few other sundry items. He stayed for the most part in \textsanskrit{Anāthapiṇḍika}’s monastery near \textsanskrit{Sāvatthī}, although he spent time also in other monasteries. While in a monastery he lived in a simple hut. The year was divided into three seasons—hot, rainy, and cold. During the rainy season he always stayed in a monastery, while for the other seasons he might also wander the countryside.

A typical day would begin with early rising for meditation. The approaching dawn signaled the start of the day’s activities, particularly the daily alms round. Now, while in the monastery, monastics would typically wear only a lower sarong-like robe (and an upper cloth for the nuns). So, some time after dawn, they would dress in the full set of three robes before proceeding to a nearby village or town for alms round. During the alms round, as may be seen in Buddhist lands today, lay folk would put some food in the bowl to be eaten that day. Buddhist monastics are not allowed to receive money. The monastics would eat once or twice a day, but always between dawn and noon.

Normally the mendicants would retire after the morning meal to solitude for the day’s meditation, either in their huts or perhaps in a nearby forest or some other quiet spot. In the hot season the Buddha would sometimes have a short siesta in the afternoon (\href{https://suttacentral.net/mn36\#46}{MN 36:46}). Towards evening the Buddha would emerge and would often give teachings or answer questions. But this was spontaneous, and not a fixed routine. Then he would meditate until late at night, needing only a few hours sleep in the middle of the night.

\href{https://suttacentral.net/mn91}{MN 91} \textit{With \textsanskrit{Brahmāyu}} describes his behavior in minute detail, recording the tiny nuances of mindfulness and care that he brought to every activity. His followers saw him as the embodiment of the highest truth. Discerning his realization in every detail of his words and acts, they could be moved to exalted praise in passages of joy and high beauty (\href{https://suttacentral.net/mn92}{MN 92} \textit{With Sela}, \href{https://suttacentral.net/an6.43}{AN 6.43}, \href{https://suttacentral.net/an10.30}{AN 10.30}, \href{https://suttacentral.net/sn8.7}{SN 8.7}). However, far from encouraging mindless devotion, the Buddha encouraged his students to investigate him, prescribing a rigorous and detailed set of tests that a good spiritual teacher should pass (\href{https://suttacentral.net/mn47}{MN 47} \textit{The Inquirer}).

This brief description does not do justice to the impact that the Buddha had on those who encountered him. He is constantly praised as the one who arises in the world out of compassion for sentient beings (\href{https://suttacentral.net/mn4}{MN 4}), who is perfected in good qualities (\href{https://suttacentral.net/mn38}{MN 38}), whose arising is the manifestation of a great light (\href{https://suttacentral.net/an1.175}{AN 1.175}–\href{https://suttacentral.net/an1.177}{AN 1.177}), who fully understands whatever is knowable (\href{https://suttacentral.net/mn1}{MN 1}; \href{https://suttacentral.net/an4.23}{AN 4.23}), and possesses complete confidence and courage no matter what the context (\href{https://suttacentral.net/mn12}{MN 12}). He possessed overwhelming physical beauty and charisma, making an unforgettable impression on many of the people he encountered (\href{https://suttacentral.net/mn26}{MN 26}, \href{https://suttacentral.net/an4.36}{AN 4.36}), but he told overzealous devotees to forget about his “putrid body” (\href{https://suttacentral.net/sn22.87}{SN 22.87}).

While as a person he was unique and unequaled in his time (\href{https://suttacentral.net/mn56}{MN 56}; \href{https://suttacentral.net/an1.174}{AN 1.174}), as the “Realized One” (\textit{\textsanskrit{tathāgatha}}) the Buddha was one of a series of awakened masters of truly cosmic significance. His understanding did not just pertain to the narrow locale of his own time and place, but was equally relevant to all sentient beings in all realms at all times.

Other Buddhas have arisen in the past and will arise again in the future, with the same realization and teachings. A detailed list of past Buddhas is found at \href{https://suttacentral.net/dn14}{DN 14} \textit{\textsanskrit{Mahāpadāna}}. Mentions elsewhere, such as the Buddhas Kakusandha in \href{https://suttacentral.net/mn50}{MN 50} and Kassapa in \href{https://suttacentral.net/mn81}{MN 81}, show that the underlying idea was current throughout the \textit{\textsanskrit{nikāyas}} (See too \href{https://suttacentral.net/sn12.14}{SN 12.4}–10, \href{https://suttacentral.net/an3.80}{AN 3.80}, and \href{https://suttacentral.net/an5.180}{AN 5.180}).

\href{https://suttacentral.net/mn116}{MN 116} \textit{At Isigili} also refers to the so-called “Buddhas awakened for themselves” (\textit{paccekabuddha}), a mysterious kind of sage who has realized the same truths as the “fully awakened Buddha” (\textit{\textsanskrit{sammāsambuddha}}) but does not establish a religious movement.

Despite his exalted and revered status, the Buddha in the \textit{\textsanskrit{nikāyas}} had not yet been elevated to the cosmic divinity who appears in later Buddhism. For all his extraordinary qualities, he remains a very human figure. Nowhere does he say that his practice in past lives led to his awakening in this life; there is no mention of the \textit{\textsanskrit{pāramīs}}. Indeed, in one of the rare occasions when he refers to a past life (\href{https://suttacentral.net/mn81}{MN 81} \textit{With \textsanskrit{Ghaṭikāra}}), he appears decidedly un-enlightened.

In the early texts, the term \textit{bodhisatta} means “one intent on awakening”. It primarily refers to the period after leaving home and before awakening. The Buddha-to-be is not described as following a path that he had started long ago, but as exploring the different options available to him, uncertain as to how awakening may be gained. His crucial insight came, not through vows made in past lives, but when he remembered the time he fell into absorption (\textit{\textsanskrit{jhāna}}) as a child.

The Buddha goes to great lengths to detail the many trials and experiments he undertook before his path was mature. His truly special quality was that he discovered the path through his own efforts, and later taught that same path to his disciples. Even disciples who had realized the same liberation and understanding revered the Buddha as the one of unsurpassed wisdom and compassion who illuminated the path for others.

\section*{The Stages of Awakening}

Spiritual enlightenment or awakening in Buddhism is not seen as a vague or unknown quantity. On the contrary, it is specific, precise, and repeatable. The texts develop a detailed typology of enlightened beings. There are four main stages, subdivided into eight. These are called the “noble disciples” (\textit{\textsanskrit{ariyasāvaka}}) or “good people” (\textit{sappurisa}). These are presented from different perspectives throughout the texts; here is an overview.

The factors of the path are developed until they are all present to a sufficient degree of maturity. At this point one is said to be a “follower of the teachings” (\textit{\textsanskrit{dhammānusāri}}) or “follower by faith” (\textit{\textsanskrit{saddhānusāri}}), depending on whether wisdom or faith is predominant (\href{https://suttacentral.net/mn70\#20}{MN 70:20}). Such a person is also called “one who is on the path to stream-entry” (\href{https://suttacentral.net/mn142\#5}{MN 142:5}, \href{https://suttacentral.net/mn48\#15}{MN 48:15}). They will realize the Dhamma in this life, yet they have still not seen a vision of the Dhamma and their understanding is still to a degree conceptual (\href{https://suttacentral.net/sn25.1}{SN 25.1}). How far and fast they proceed depends on the strength of their faculties and their effort.

The texts are unclear as to whether such a person definitively knows that they have reached this point. No such ambiguity attaches to the moment of actually realizing stream-entry, however. It hits like a flash of lightning in the dead of night (\href{https://suttacentral.net/an3.25}{AN 3.25}), and you will remember precisely when and where it happened (\href{https://suttacentral.net/pj4}{Pj 4}). You have a vision of the four noble truths, at which point the conceptual understanding of the path becomes fully experiential (\href{https://suttacentral.net/sn25.1}{SN 25.1}). Letting go of three fetters (\textit{\textsanskrit{saṁyojana}})—doubt, misapprehension of precepts and vows, and any views that identify a self with the aggregates—one reaches the first stage of awakening, known as “stream entry”.

A stream-enterer is bound for awakening, free from any rebirth in lower realms (\href{https://suttacentral.net/mn6}{MN 6}; \href{https://suttacentral.net/sn55.1}{SN 55.1}), and is reborn a maximum of seven times (\href{https://suttacentral.net/an3.88}{AN 3.88}). They have eliminated a huge mass of suffering; what’s left is like seven small pebbles compared to the Himalayas (\href{https://suttacentral.net/sn56.59}{SN 56.59}). They are generous, devoted, and naturally keep the five precepts at a minimum (\href{https://suttacentral.net/mn53}{MN 53}; \href{https://suttacentral.net/sn12.41}{SN 12.41}; \href{https://suttacentral.net/an7.6}{AN 7.6}). They understand the four noble truths and dependent origination and have no doubts as to the Buddha, his teaching, or his \textsanskrit{Saṅgha} (\href{https://suttacentral.net/mn7}{MN 7}; \href{https://suttacentral.net/sn12.41}{SN 12.41}). Yet there are still attachments; one may still sorrow when relatives pass away (\href{https://suttacentral.net/ud8.8}{Ud 8.8}), or suffer moments of grief or despair (\href{https://suttacentral.net/dn16/en/sujato\#5}{DN 16:5}.13).

When a stream-enterer further develops the path, they are said to be on the path to once-return; with the lessening of greed and hate they reach the state of a once-returner. At this point one will be reborn in this world once only. Again developing the path one completely eliminates greed and hate, at which point one is considered a non-returner. Such a person is reborn usually in one of the special realms of high divinity known as the Pure Abodes, and from there attains full extinguishment. However, in certain cases a non-returner may become fully extinguished before being reborn (\href{https://suttacentral.net/an3.88}{AN 3.88}).

Yet even the exalted state of the non-returner is not entirely free from attachments. They have perfected ethics and immersion, yet their wisdom is still not complete. They still have the five higher fetters: attachment to rebirth in the realms of luminous form and the formless; a restless urgency to reach full awakening; a residual sense of self, or conceit; and ignorance.

Once again they further develop the path until they attain full perfection. A perfected one, or arahant, has fully eliminated all defilements, has made an end of rebirth, and when this life is over will suffer no more. Their liberation is identical to that of the Buddha. Since they have no hindrances or defilements of mind, they can attain deep meditation whenever they want, and their lives are full of joy and peace. They feel no grief, no anxiety, no confusion or doubt. A perfected one lives only in accordance with the Dhamma, and is incapable of reverting to worldly ways, or of indulging in material desires. They live a life of serenity and virtue, preferring seclusion and meditation, but are selfless in service, helping others whenever they can, especially through teaching. They are devoid of fear and have complete equanimity when faced with death (\href{https://suttacentral.net/mn140\#24}{MN 140:24}). They still experience the pain of the body, but have no mental suffering at all (\href{https://suttacentral.net/sn36.6}{SN 36.6}).

\section*{The Monastic Disciples}

Although we think of the suttas as “teachings of the Buddha”, in fact only about a quarter of the discourses in the Majjhima feature the Buddha simply delivering a discourse to the assembly. Over half the discourses consist of dialogues (\textit{\textsanskrit{vyākaraṇa}}). Sometimes the Buddha asks a question and leads the assembly through a process of Socratic inquiry. Sometimes a seeker asks a question or a series of questions. The Buddha may answer in various ways—sometimes directly, sometimes with a counter-question, sometimes with a lengthy analysis—depending on the nature of the question and the questioner. And intriguingly, he sometimes doesn’t answer at all.

And as well as featuring in one way or another in most of the discourses, various disciples serve as primary teachers in about a fifth of the discourses of the Majjhima, continuing the Buddha’s work of exploring and explaining the teachings. In the Majjhima \textsanskrit{Nikāya} we meet a range of skilled and accomplished teachers, many of whom became widely famed in the Buddhist traditions. From the beginning of his dispensation, the Buddha was eager to empower his followers, encouraging them to share the teaching to the best of their ability.

Nine discourses are spoken by the Buddha’s chief disciple and renowned General of the Dhamma, \textsanskrit{Sāriputta}. These include \href{https://suttacentral.net/mn9}{MN 9}, a wide-ranging exploration of the many facets of right view; \href{https://suttacentral.net/mn28}{MN 28}, which shows how all the teachings can be included in the four noble truths; and \href{https://suttacentral.net/mn141}{MN 141}, giving a fully detailed explanation of the four noble truths based on the Buddha’s first sermon. In these discourses \textsanskrit{Sāriputta} shows his interest in developing a systematic overview of the Dhamma. Such analyses are one of the primary inspirations behind the later development of the Abhidhamma texts, and \textsanskrit{Sāriputta} is rightly regarded as one of the fathers of the Abhidhamma.

\textsanskrit{Mahākaccāna} is another monk of renowned wisdom, whose incisive analytic style, with a special focus on the process of sense experience, also influenced the Abhidhamma. He spoke four discourses in the Majjhima. \href{https://suttacentral.net/mn18}{MN 18}, the “Honey Cake” demonstrates his unmatched skill in drawing out subtle implications of brief teachings by the Buddha. This discourse received a detailed treatment in Katukurunde Nyanananda’s \textit{Concept and Reality in Early Buddhist Thought}, one of the most influential monographs in modern Buddhist studies.

\href{https://suttacentral.net/mn24}{MN 24}, a dialogue between \textsanskrit{Sāriputta} and \textsanskrit{Puṇṇa} \textsanskrit{Mantāṇiputta}, speaks of a relay of chariots, introducing the idea of the seven stages of purification that was to greatly influence the concept of stages of insight in Buddhaghosa’s \textit{Visuddhimagga} and, from there, contemporary Theravada meditation.

The nun \textsanskrit{Dhammadinnā} presents the only discourse by a \textit{\textsanskrit{bhikkhunī}} in the Majjhima. It seems that the teachings by women in the suttas, while rare, were included because of their unique and striking wisdom. In \href{https://suttacentral.net/mn44}{MN 44} she responds to a series of questions by her former husband, revealing her depth of meditative accomplishment and wisdom.

Further discourses by disciples include several by \textsanskrit{Moggallāna}, Anuruddha, and others. And we cannot pass a discussion on disciples without mentioning Ānanda, who lived closest to the Buddha, and through whom, according to tradition, all the texts passed. Ānanda is the main teacher in seven discourses, and features in many more. Ānanda has a specially close interest in the personal life of the Buddha, and it is with him that the Buddha’s life story began to take on its familiar form.

\section*{The \textsanskrit{Saṅgha}}

Mighty in wisdom though they were, the enlightened disciples of the Buddha did not exist in isolation. They were part of an organized spiritual community called the \textsanskrit{Saṅgha}, or “Monastic Order”. \textit{\textsanskrit{Saṅgha}} as a religious term is used in two specific senses: the “monastic order” (\textit{\textsanskrit{bhikkhusaṅgha}}) and the “community of noble disciples” (\textit{\textsanskrit{sāvakasaṅgha}}; the expected term \textit{\textsanskrit{ariyasaṅgha}} only appears in one verse at \href{https://suttacentral.net/an6.54}{AN 6.54}). The former term refers to those who have taken ordination and practice as a Buddhist mendicant, while the latter refers to someone who has reached one of the stages of awakening, which may, of course, include lay followers.

The distinction between these two is not as clear-cut as one might imagine. The standard formula describing the \textsanskrit{Saṅgha} in the recollection of the Triple Gem mentions the eight kinds of noble disciple on the path. But it also describes them in terms typical of the monastic order, for example as a “field of merit”. It is difficult to draw a decisive conclusion from this, however, as this formula appears to be a composite one. As a general rule, it is safe to assume that, unless the context specifies the \textsanskrit{Saṅgha} of noble disciples, it is referring to the monastic order.

The monastic community was established shortly after the Buddha’s awakening, as told in the first chapter of the Vinaya Khandhakas, known as the \textsanskrit{Mahākkhandhaka} or the Pabbajjakkhandhaka (\href{https://suttacentral.net/pli-tv-kd1}{Kd 1}). Portions of that narrative appear in the suttas, but the entire story should really be read.

The early \textsanskrit{Saṅgha} consisted of a small community of advanced spiritual practitioners; according to the texts, they were all perfected ones. Famously, the Buddha urged his followers to wander the countryside, teaching the Dhamma for the benefit of all people (\href{https://suttacentral.net/sn4.5}{SN 4.5}). This sets the example for how the Buddha was to relate to his community. He did not operate as a guru figure who insisted on obedience and treated his followers as dependents. On the contrary, he treated his community as adults who could make their own choices, and he trusted them to make their own contributions. To be sure, at this early stage they were all awakened; but this policy continued throughout his life, even up to his deathbed when this had long ceased to be the case.

\href{https://suttacentral.net/pli-tv-kd1}{Kd 1} tells the story of how the ordination procedure evolved. Originally the Buddha himself simply invited his followers with the words “Come, mendicant!” Later, as the number of candidates grew, the Buddha allowed the mendicants to perform ordination themselves using a simple ceremony of going to the three refuges; this is still used for novice ordinations. Eventually, the procedure was formalized in the mature form as a “motion and three announcements”. The mature form of ordination ceremony had various formal and legalistic elements: a set of questions was given to vet the candidates; the questioners were formally appointed by the \textsanskrit{Saṅgha}; the candidates, having been questioned, were assigned a mentor; and the entire \textsanskrit{Saṅgha} gave their assent to the ordination. The entire procedure is straightforward and legalistic, devoid of ritual or embellishment. The core of the procedure is still followed today, although the traditions have adorned the bare bones of the procedure with a colorful range of rituals and celebrations.

This procedure set the template for the proceedings of the Sangha. It is by consensus: the \textsanskrit{Saṅgha} as a whole gives the ordination (\emph{\textsanskrit{saṁgho} \textsanskrit{itthannāmaṁ} \textsanskrit{upasampādeti}}). If even a single mendicant dissents the ordination does not proceed. Contrary to a misunderstanding that is unfortunately common even within the \textsanskrit{Saṅgha}, the mentor—called \textit{\textsanskrit{upajjhāya}} for the monks or \textit{\textsanskrit{pavattinī}} for the nuns—does not perform the ordination; they are appointed by the \textsanskrit{Saṅgha} to support the new monastics.

For legal purposes the \textsanskrit{Saṅgha} is the community within the “boundary” (\textit{\textsanskrit{sīmā}}), which is an arbitrary area formally designated by the local community; typically it would have been the grounds of a monastery, but it could have been much bigger or smaller (see \href{https://suttacentral.net/pli-tv-kd2}{Kd 2}). This is an important pragmatic point: the \textsanskrit{Saṅgha} is decentralized. It is impractical to expect all mendicants from all over the world to come together to agree, so all procedures are based on the local community.

There is no hierarchy in the \textsanskrit{Saṅgha}: all members have the same say. Respect is owed to seniors on account of their experience and wisdom, but this does not translate to a power of command. No \textsanskrit{Saṅgha} member has the right to force anyone to do anything, and if a senior \textsanskrit{Saṅgha} member, even one’s mentor or teacher, tells one to do something that is against the Dhamma or Vinaya, one is obligated to disobey. As an example of how the mendicants were to make decisions, \href{https://suttacentral.net/mn17}{MN 17} \textit{Jungle Thickets} (\textit{Vanapatthasutta}) gives some guidelines for whether a mendicant should stay in a monastery or leave; there is no question of being ordered to go to one place or the other.

The \textsanskrit{Saṅgha} soon set up monasteries, with relatively settled communities. Typically mendicants would stay in monasteries for part of the year, especially in the three months of the rainy season retreat, while much of the rest of the year may have been spent wandering. They would meet together each fortnight for the “sabbath” (\textit{uposatha}), during which time there would be teachings (\href{https://suttacentral.net/mn109}{MN 109} \textit{The Longer Discourse on the Full-Moon Night}, \textit{\textsanskrit{Mahāpuṇṇamasutta}}), and later, the recitation of the monastic rules (\textit{\textsanskrit{pātimokkha}}).

At some point a community of nuns (\textit{\textsanskrit{bhikkhunīs}}) was set up along the lines of the monks’ order. The traditional account says that this was on the instigation of the Buddha’s stepmother, \textsanskrit{Mahapajāpatī} \textsanskrit{Gotamī}. However, the account as preserved today is deeply problematic both textually and ethically, and cannot be accepted without reservation. In any case, we know that a nuns’ community was established and that it ran on mostly independent grounds. The nuns built their own monasteries (\href{https://suttacentral.net/pli-tv-bi-vb-pj5}{Bi Pj 5}), took their own students (\href{https://suttacentral.net/thig5.11}{Thig 5.11}), studied the texts (\href{https://suttacentral.net/pli-tv-bi-vb-pc-33}{Bi Pc 33}), developed meditation (\href{https://suttacentral.net/sn47.10}{SN 47.10}), wandered the countryside (\href{https://suttacentral.net/pli-tv-bi-vb-pc-50}{Bi Pc 50}), and achieved the wisdom of awakening (\href{https://suttacentral.net/mn44}{MN 44} \textit{The Shorter Classification}, \textit{\textsanskrit{Cūḷavedallasutta}}). While it is true that certain of the rules as they exist today discriminate against the nuns, other rules protect them; for example, the monks are forbidden from having a nun wash their robes, thus preventing the monks from treating the nuns like domestic servants. When monks taught the nuns, they did so respectfully, engaging with them as equals (\href{https://suttacentral.net/mn146}{MN 146} \textit{Advice from Nandaka}). The order of nuns gave women of the time a rare opportunity to pursue their own spiritual path, supported by the community. It survived through the years in the East Asian traditions, and in recent years has been revived within the Tibetan and \textsanskrit{Theravādin} schools.

Note that in the suttas, the term \textit{bhikkhu} (masculine gender) is used as a generic term to include both monks and nuns. That nuns were included in the generic masculine is clear from such contexts as \href{https://suttacentral.net/an4.170}{AN 4.170}, where Ānanda begins by referring to “monks and nuns” and continues with just “monks”, or \href{https://suttacentral.net/dn16}{DN 16}, which speaks of “monks and nuns” but uses a masculine pronoun to refer to them both. That the default masculine may refer to women is further confirmed by passages such as \href{https://suttacentral.net/thig16.1}{Thig 16.1}, where the lady \textsanskrit{Sumedhā} is called \textit{putta} by her father. \textit{Putta} as “son” contrasts with \textit{\textsanskrit{dhītā}} as “daughter”, but this passage shows it can be used in the generic sense of “child” as well. In general teaching, it is likely that monks and nuns, as well as lay people, would have been present, yet the texts by convention are addressed to “monks” (\textit{bhikkhave}). Hence I have rendered \textit{bhikkhu} throughout with the gender-neutral “mendicant”, except where it is necessary to distinguish the genders, in which case I use “monk”. In the Vinaya \textsanskrit{Piṭaka}, however, the texts are by default separated by gender, so it is best to use “monk” there. Note too that “mendicant” is what \textit{bhikkhu} actually means: it refers to someone who makes a living by walking for alms.

Originally the \textsanskrit{Saṅgha} followed an informal set of principles considered appropriate for ascetics, which was similar to those followed by other ascetic groups. These are retained in detail in the ethics portion of the Gradual Training (\href{https://suttacentral.net/mn51}{MN 51}; \href{https://suttacentral.net/dn1}{DN 1}, etc.). The Buddha initially refused to set up a formal system of monastic law, but eventually the \textsanskrit{Saṅgha} grew so large that such a system became necessary (\href{https://suttacentral.net/pli-tv-bu-vb-pj1}{Bu Pj 1}). This is detailed in the extensive texts of the Vinaya \textsanskrit{Piṭaka}. Note that when the word \textit{vinaya} is used in the four \textit{\textsanskrit{nikāyas}}, it rarely refers to the Vinaya \textsanskrit{Piṭaka}. Normally it is a general term for the practical application of the teaching: \textit{dhammavinaya} means something like “theory and practice”.

Scattered throughout the four \textit{\textsanskrit{nikāyas}} we find a fair number of teachings intended for the monastic community. Sometimes these refer to technical procedures of the Vinaya \textsanskrit{Piṭaka}, probably to make sure that students of the suttas would be familiar with them (\href{https://suttacentral.net/mn104}{MN 104} \textit{At \textsanskrit{Sāmagāma}}). More commonly, however, they were general principles of ethics, laying down guidelines for a harmonious and flourishing spiritual community.

In \href{https://suttacentral.net/mn3}{MN 3} \textit{\textsanskrit{Dhammadāyāda}} the Buddha teaches his mendicants to be his “heirs in the teaching”, inheriting spiritual, not material things from their Teacher. In \href{https://suttacentral.net/mn5}{MN 5} \textit{Unblemished} (\textit{\textsanskrit{Anaṅgaṇasutta}}), Venerables \textsanskrit{Sāriputta} and \textsanskrit{Moggallāna} speak of the those who have “blemishes” that spoil their spiritual integrity. When guilty of an offense, rather than clearing it by confession, they hide it. Or they hope the Buddha asks them a question about the teaching, or that they get the best food at mealtime. Such things appear small, but over time they corrupt, so they must be polished off diligently, for scrupulous attention to ethics is the foundation for all higher achievements in the spiritual path (\href{https://suttacentral.net/mn6}{MN 6} \textit{One Might Wish}, \textit{\textsanskrit{Ākaṅkheyyasutta}}). It is essential for community members to be open to admonition, for otherwise they cannot identify their flaws and heal them (\href{https://suttacentral.net/mn15}{MN 15} \textit{Measuring Up}, \textit{\textsanskrit{Anumānasutta}}). But a community cannot be based on sniping and criticism, but on love, generosity, and respect (\href{https://suttacentral.net/mn16}{MN 16} \textit{Emotional Barrenness}, \textit{Cetokhilasutta}).

Not all were satisfied with the Buddha’s path. A certain Sunakkhatta—familiar from \href{https://suttacentral.net/dn24}{DN 24}—disrobed, for he wanted to see more miracles, and thought the mere ending of suffering was a poor goal (\href{https://suttacentral.net/mn12}{MN 12}; see too \href{https://suttacentral.net/mn63}{MN 63}). The monk \textsanskrit{Ariṭṭha} went even further, directly contradicting the fundamental principles of the Dhamma by declaring that the things the Buddha said were harmful were not, in fact, harmful (\href{https://suttacentral.net/mn22}{MN 22} \textit{The Simile of the Snake}, \textit{\textsanskrit{Alagaddūpamasutta}}). This event prompted the establishment of several Vinaya rules, showing the interdependence of these bodies of literature (\href{https://suttacentral.net/pli-tv-bu-vb-pc68}{Bu Pc 68}, \href{https://suttacentral.net/pli-tv-bu-vb-pc69}{Bu Pc 69}; \href{https://suttacentral.net/pli-tv-kd11}{Kd 11}).

Sometimes problems went beyond just an individual, and a whole community could split apart. \href{https://suttacentral.net/mn48}{MN 48} \textit{The Mendicants of Kosambi}, which also has parallels in the Vinaya, tells of how a community can be split because of an apparently trivial difference. They became so consumed by anger and conceit that they even ignored the Buddha’s attempts at reconciliation. Eventually, though, they came to their senses. Likewise, the monks at \textsanskrit{Cātuma} (\href{https://suttacentral.net/mn67}{MN 67}) were so unruly the Buddha dismissed them, but was persuaded to relent.

Reconciliation and growth is also the message of \href{https://suttacentral.net/mn65}{MN 65} \textit{With \textsanskrit{Bhaddāli}}, where the Buddha requests that the mendicants eat before noon. \textsanskrit{Bhaddāli} refuses, out of greed and stubbornness, but later accepts the Buddha’s ruling and is forgiven. This rule was a big deal, for it is also discussed in \href{https://suttacentral.net/mn66}{MN 66} and \href{https://suttacentral.net/mn70}{MN 70}.

But in general the suttas paint a glowingly positive view of the renunciate life, leaving the Vinaya \textsanskrit{Piṭaka} to deal with the nitty-gritty of human failings. \href{https://suttacentral.net/mn31}{MN 31} \textit{The Shorter Discourse at \textsanskrit{Gosiṅga}} (\textit{\textsanskrit{Cūḷagosiṅgasutta}} depicts an idyllic fellowship of three monks living in gracious and fulfilling harmony, where the simplicity and purity of their lifestyle form the basis for advanced meditation (see \href{https://suttacentral.net/mn128}{MN 128} \textit{Corruptions}, \textit{Upakkilesasutta}, which shares the same setting). The \textsanskrit{Saṅgha} lives honoring the Buddha not out of fear but from genuine love and respect (\href{https://suttacentral.net/mn77}{MN 77} \textit{The Longer Discourse with \textsanskrit{Sakuludāyī}}). By practicing in line with the Dhamma, they honor their Teacher and become worthy of their alms-food (\href{https://suttacentral.net/mn151}{MN 151} \textit{The Purification of Alms}, \textit{\textsanskrit{Piṇḍapātapārisuddhisutta}}).

\section*{The Wider Community}

The monastics are far from the only people we meet. In his wide wanderings across the Ganges plain, the Buddha interacted with a wide cross-section of the local peoples: learned brahmins and simple villagers; kings and slaves; priests and prostitutes; women and men; children and the elderly; the devout and the skeptical; the sick and the disabled; those seeking to disparage and those sincerely seeking the truth.

It should not be thought that teachings for monastics and lay were completely separated. For example, \href{https://suttacentral.net/mn7}{MN 7} \textit{The Simile of the Cloth} (\textit{Vatthasutta} or \textit{\textsanskrit{Vatthūpamasutta}}) begins with a standard teaching to the mendicants about purifying the mind, leading up to the divine meditations and full awakening. The Buddha calls such a person “bathed with the inner bathing”. At this, the brahmin Sundarika—who happened to be sitting nearby—protested, saying that the brahmins attested to the purifying properties of bathing in the river \textsanskrit{Bāhuka}. This casual example shows that, even when a discourse is addressed to the mendicants, a wide range of people, including non-Buddhists, might be present.

The brahmin from \href{https://suttacentral.net/mn7}{MN 7} ended up taking ordination. However, there are also many cases of long-term practitioners who remained in the lay life. In \href{https://suttacentral.net/mn14}{MN 14} \textit{\textsanskrit{Cūḷadukkhakkhandha}}, the Buddha’s relative \textsanskrit{Mahānāma} laments that despite his long years of practice, he still has greed, hate, and delusion. Lay meditators struggled to find deep peace of mind in those days, just like today.

Such cases show how the Buddha spoke to individuals, addressing their specific needs and concerns. Elsewhere he gave more general discourses on ethical principles. In \href{https://suttacentral.net/mn41}{MN 41} \textit{The People of \textsanskrit{Sālā}} (\textit{\textsanskrit{Sāleyyakasutta}}), he teaches a group of non-Buddhist lay people what are commonly called the “ten ways of doing skillful deeds” (\textit{\textsanskrit{dasakusalakammapathā}}; see \href{https://suttacentral.net/an10.176}{AN 10.176}):

\begin{enumerate}%
\item No killing%
\item No stealing%
\item No sexual misconduct%
\item True speech%
\item Harmonious speech%
\item Gentle speech%
\item Meaningful speech%
\item No covetousness%
\item No ill will%
\item Right view%
\end{enumerate}

The first three pertain to the body; the second four to speech, and the final three to the mind. These extend the well-known five precepts, offering a complete guide to ethical living. They are taught in many places in the suttas, and here they are defined in detail.

While in the Majjhima, and everywhere in the canon, formalistic and artificial settings dominate, several discourses have somewhat messy narrative structures, which seem to preserve the memory of real-life incidents. In \href{https://suttacentral.net/mn51}{MN 51} \textit{With Kandaraka} the Buddha is approached by an elephant driver named Pessa and a wandering ascetic named Kandaraka. Kandaraka is impressed by the silence of the \textsanskrit{Saṅgha}, prompting the Buddha to attribute this to their practice of mindfulness meditation. Pessa intervenes, saying that lay folk also sometimes practice mindfulness, and the Buddha engages him on the topic of people who act for their own harm or benefit, or that of others; a topic familiar from the \textsanskrit{Aṅguttara} (\href{https://suttacentral.net/an4.198}{AN 4.198}). But Pessa has to leave, at which the Buddha says he would have benefited by staying. The mendicants ask the Buddha to finish what he was about to say, at which he gives a lengthy version of the Gradual Training. We don’t hear about what happened to either Pessa or Kandaraka. We do, however, find the Gradual Training taught to lay folk elsewhere, for example by Ānanda on the occasion of opening a new community hall belonging to his clan, the Sakyans (\href{https://suttacentral.net/mn53}{MN 53} \textit{A Trainee}, \textit{Sekhasutta}). Indeed, the gradual progress of a mendicant is compared with a graduated professional education of an accountant (\href{https://suttacentral.net/mn107}{MN 107} \textit{With \textsanskrit{Moggallāna} the Accountant}, \textit{\textsanskrit{Gaṇakamoggallānasutta}}).

In addition to personal problems and general ethical teachings, the Buddha responded to criticisms of his community, not with defensiveness, but by a clear explanation. In \href{https://suttacentral.net/mn55}{MN 55} \textit{With \textsanskrit{Jīvaka}}, the Buddha’s physician reports that people are saying that the Buddha eats meat that has been slaughtered on purpose for him. The Buddha denies this, saying that he and his mendicants only accept food that has been freely offered, and will refuse any meat they suspect has been killed for them.

While the critics are not identified in \href{https://suttacentral.net/mn55}{MN 55}, \href{https://suttacentral.net/mn56}{MN 56} \textit{With \textsanskrit{Upāli}} and \href{https://suttacentral.net/mn58}{MN 58} \textit{With Prince Abhaya} (\textit{\textsanskrit{Abhayarājakumārasutta}}) depicts the Jains as deliberately trying to take down the Buddha in debate, and—since these are Buddhist texts—failing. Brahmins attempt the same trick, sending their most precocious scholars against the Buddha with no more success (\href{https://suttacentral.net/mn93}{MN 93}, \href{https://suttacentral.net/mn95}{MN 95}).

A few discourses focus not on the Buddha’s message as such, but on the impact it had on family life. When the wealthy young man \textsanskrit{Raṭṭhapāla} ordains, his parents are highly distressed and try to entice him to disrobe (\href{https://suttacentral.net/mn82}{MN 82}). \href{https://suttacentral.net/mn87}{MN 87} \textit{Born From the Beloved} (\textit{\textsanskrit{Piyajātikasutta}}) gives us a glimpse into the family life of King Pasenadi and Queen \textsanskrit{Mallikā}. They hear of the Buddha’s teaching that our loved ones bring us suffering, and when \textsanskrit{Mallikā} agrees with this, her husband is not pleased at all. \textsanskrit{Mallikā} is careful to first confirm that the teaching was what the Buddha said, then she gently and kindly explains the teaching to the King, leading him to announce his faith in the Buddha. He became a devoted follower featured in many discourses. \href{https://suttacentral.net/mn100}{MN 100} \textit{With \textsanskrit{Saṅgārava}} is another case where a wife leads her husband to the Dhamma.

It is not only the great and the good who need teachings. The suttas depict Venerable \textsanskrit{Sāriputta} helping his old friend \textsanskrit{Dhanañjāni}, who was redeemed near the end of his life, despite his long career of corruption (\href{https://suttacentral.net/mn97}{MN 97}). Even the serial killer \textsanskrit{Aṅgulimāla} found redemption and forgiveness (\href{https://suttacentral.net/mn86}{MN 86}; cf. \href{https://suttacentral.net/thag16.8}{Thag 16.8}).

\section*{A Brief Textual History}

The Majjhima \textsanskrit{Nikāya} was edited by V. Trenckner (vol. 1) and Robert Chalmers (vols. 2 and 3) based on manuscripts in Sinhalese, Burmese, and Thai scripts, and published in Latin script by the Pali Text Society from 1888 to 1899. The first translation into English followed in 1926–7 by Robert Chalmers under the title \textit{Further Dialogues of the Buddha}.

Rapid improvements in understanding of Pali and Buddhism during the early 20th century soon made it clear that an improved translation was needed. This was undertaken in the 1950s by I.B. Horner and was published by the PTS as \textit{The Book of Middle Length Sayings} in 1954–9.

Her translation, while a significant improvement on Chalmers’, was soon eclipsed by that of Bhikkhu \textsanskrit{Ñāṇamoḷi}. \textsanskrit{Ñāṇamoḷi}’s extraordinary career as a Pali scholar and translator was tragically cut short by his early death, and his Majjhima \textsanskrit{Nikāya} translation remained as an unfinished hand-written manuscript. Nevertheless, its value was so clear that it was published, first as a selection of 90 discourses edited by Bhikkhu Khantipalo and published in Bangkok in 1976 as \textit{A Treasury of the Buddha’s Words}, then as a fully edited and updated version by Bhikkhu Bodhi in 1995 under the title \textit{The Middle Length Discourses of the Buddha: A Translation of the Majjhima Nikaya}. The latter version reached a peak of accuracy, consistency, and readability that has become the standard to which all later translations of the \textit{\textsanskrit{nikāyas}} have aspired.

When the Pali was unclear I frequently referred to the earlier work of \textsanskrit{Ñāṇamoḷi} \& Bodhi (both the published work and \textsanskrit{Ñāṇamoḷi}’s hand-written original), and less often to Horner and various translations of specific texts. I also had access to notes by Bhikkhus \textsanskrit{Ñāṇadīpa} and \textsanskrit{Ñāṇatusita} on Bhikkhu Bodhi’s translation. In addition, I consulted the Chinese and other parallel texts, and the detailed studies on these by Bhikkhu \textsanskrit{Anālayo}. However, I found these to be useful for translation in only a few instances, as I believe it is important to preserve the integrity of the different textual lineages. Comparative studies lose value when the underlying texts have already been reconciled.

%
\chapter*{Acknowledgements}
\addcontentsline{toc}{chapter}{Acknowledgements}
\markboth{Acknowledgements}{Acknowledgements}

I remember with gratitude all those from whom I have learned the Dhamma, especially Ajahn Brahm and Bhikkhu Bodhi, the two monks who more than anyone else showed me the depth, meaning, and practical value of the Suttas.

Special thanks to Dustin and Keiko Cheah and family, who sponsored my stay in Qi Mei while I made this translation.

Thanks also for Blake Walshe, who provided essential software support for my translation work.

Throughout the process of translation, I have frequently sought feedback and suggestions from the SuttaCentral community on our forum, “Discuss and Discover”. I want to thank all those who have made suggestions and contributed to my understanding, as well as to the moderators who have made the forum possible. These translations were significantly improved due to the careful work of my proofreaders: \textsanskrit{Ayyā} \textsanskrit{Pāsādā}, John and Lynn Kelly, and Derek Sola. Special thanks are due to \textsanskrit{Sabbamittā}, a true friend of all, who has tirelessly and precisely checked my work.

Finally my everlasting thanks to all those people, far too many to mention, who have supported SuttaCentral, and those who have supported my life as a monastic. None of this would be possible without you.

%
\chapter*{Summary of Contents}
\addcontentsline{toc}{chapter}{Summary of Contents}
\markboth{Summary of Contents}{Summary of Contents}

\begin{description}%
\item[The Chapter on the Root of All Things (\textit{\textsanskrit{Mūlapariyāyavagga}})] This chapter, though beginning with the abstruse \textsanskrit{Mūlapariyāya} Sutta, mostly contains foundational teachings and can, as a whole, serve as an introduction to the discourses.%
\item[MN 1: The Root of All Things (\textit{\textsanskrit{Mūlapariyāyasutta}})] The Buddha examines how the notion of a permanent self emerges from the process of perception. A wide range of phenomena are considered, embracing both naturalistic and cosmological dimensions. An unawakened person interprets experience in terms of a self, while those more advanced have the same experiences without attachment.%
\item[MN 2: All the Defilements (\textit{\textsanskrit{Sabbāsavasutta}})] The diverse problems of the spiritual journey demand a diverse range of responses. Rather than applying the same solution to every problem, the Buddha outlines seven methods of dealing with defilements, each of which works in certain cases.%
\item[MN 3: Heirs in the Teaching (\textit{\textsanskrit{Dhammadāyādasutta}})] Some of the Buddha’s students inherit from him only material profits and fame. But his true inheritance is the spiritual path, the way of contentment. Venerable \textsanskrit{Sāriputta} explains how by following the Buddha’s example we can experience the fruits of the path.%
\item[MN 4: Fear and Dread (\textit{\textsanskrit{Bhayabheravasutta}})] The Buddha explains the difficulties of living in the wilderness, and how they are overcome by purity of conduct and meditation. He recounts some of the fears and obstacles he faced during his own practice.%
\item[MN 5: Unblemished (\textit{\textsanskrit{Anaṅgaṇasutta}})] The Buddha’s chief disciples, \textsanskrit{Sāriputta} and \textsanskrit{Moggallāna}, use a simile of a tarnished bowl to illustrate the blemishes of the mind and conduct. They emphasize how the crucial thing is not so much whether there are blemishes, but whether we are aware of them.%
\item[MN 6: One Might Wish (\textit{\textsanskrit{Ākaṅkheyyasutta}})] According to the Buddha, careful observance of ethical precepts is the foundation of all higher achievements in the spiritual life.%
\item[MN 7: The Simile of the Cloth (\textit{\textsanskrit{Vatthasutta}})] The many different kinds of impurities that defile the mind are compared to a dirty cloth. When the mind is clean we find joy, which leads to states of higher consciousness. Finally, the Buddha rejects the Brahmanical notion that purity comes from bathing in sacred rivers.%
\item[MN 8: Self-Effacement (\textit{\textsanskrit{Sallekhasutta}})] The Buddha differentiates between peaceful meditation and spiritual practices that encompass the whole of life. He lists forty-four aspects, which he explains as “effacement”, the wearing away of conceit.%
\item[MN 9: Right View (\textit{\textsanskrit{Sammādiṭṭhisutta}})] Venerable \textsanskrit{Sāriputta} gives a detailed explanation of right view, the first factor of the noble eightfold path. At the prompting of the other mendicants, he approaches the topic from a wide range of perspectives.%
\item[MN 10: The Discourse on Mindfulness Meditation (\textit{\textsanskrit{Mahāsatipaṭṭhānasutta}})] Here the Buddha details the seventh factor of the noble eightfold path, mindfulness meditation. This collects many of the meditation teachings found throughout the canon, especially the foundational practices focusing on the body, and is regarded as one of the most important meditation discourses.%
\item[The Chapter on the Lion’s Roar (\textit{\textsanskrit{Sīhanādavagga}})] Beginning with two discourses containing a “lion’s roar”, this chapter deals with suffering, community life, and practical meditation advice.%
\item[MN 11: The Shorter Discourse on the Lion’s Roar (\textit{\textsanskrit{Cūḷasīhanādasutta}})] The Buddha declares that only those following his path can genuinely experience the four stages of awakening. This is because, while much is shared with other systems, none of them go so far as to fully reject all attachment to the idea of a self.%
\item[MN 12: The Longer Discourse on the Lion’s Roar (\textit{\textsanskrit{Mahāsīhanādasutta}})] A disrobed monk, Sunakkhatta, attacks the Buddha’s teaching because it merely leads to the end of suffering. The Buddha counters that this is, in fact, praise, and goes on to enumerate his many profound and powerful achievements.%
\item[MN 13: The Longer Discourse on the Mass of Suffering (\textit{\textsanskrit{Mahādukkhakkhandhasutta}})] Challenged to show the difference between his teaching and that of other ascetics, the Buddha points out that they speak of letting go, but do not really understand why. He then explains in great detail the suffering that arises from attachment to sensual stimulation.%
\item[MN 14: The Shorter Discourse on the Mass of Suffering (\textit{\textsanskrit{Cūḷadukkhakkhandhasutta}})] A lay person is puzzled at how, despite their long practice, they still have greedy or hateful thoughts. The Buddha explains the importance of absorption meditation for letting go such attachments. But he also criticizes self-mortification, and recounts a previous dialog with Jain ascetics.%
\item[MN 15: Measuring Up (\textit{\textsanskrit{Anumānasutta}})] Venerable \textsanskrit{Moggallāna} raises the topic of admonishment, without which healthy community is not possible. He lists a number of qualities that will encourage others to think it worthwhile to admonish you in a constructive way.%
\item[MN 16: Emotional Barrenness (\textit{\textsanskrit{Cetokhilasutta}})] The Buddha explains various ways one can become emotionally cut off from one’s spiritual community.%
\item[MN 17: Jungle Thickets (\textit{\textsanskrit{Vanapatthasutta}})] While living in the wilderness is great, not everyone is ready for it. The Buddha encourages meditators to reflect on whether one’s environment is genuinely supporting their meditation practice, and if not, to leave.%
\item[MN 18: The Honey-Cake (\textit{\textsanskrit{Madhupiṇḍikasutta}})] Challenged by a brahmin, the Buddha gives an enigmatic response on how conflict arises due to proliferation based on perceptions. Venerable \textsanskrit{Kaccāna} draws out the detailed implications of this in one of the most insightful passages in the entire canon.%
\item[MN 19: Two Kinds of Thought (\textit{\textsanskrit{Dvedhāvitakkasutta}})] Recounting his own experiences in developing meditation, the Buddha explains how to understand harmful and harmless thoughts, and how to go beyond thought altogether.%
\item[MN 20: How to Stop Thinking (\textit{\textsanskrit{Vitakkasaṇṭhānasutta}})] In a practical meditation teaching, the Buddha describes five different approaches to stopping thoughts.%
\item[The Chapter of Similes (\textit{\textsanskrit{Opammavagga}})] A diverse chapter including biography, non-violence, not-self, and an influential teaching on the progress of meditation.%
\item[MN 21: The Simile of the Saw (\textit{\textsanskrit{Kakacūpamasutta}})] A discourse full of vibrant and memorable similes, on the importance of patience and love even when faced with abuse and criticism. The Buddha finishes with the simile of the saw, one of the most memorable similes found in the discourses.%
\item[MN 22: The Simile of the Snake (\textit{\textsanskrit{Alagaddūpamasutta}})] One of the monks denies that prohibited conduct is really a problem. The monks and then the Buddha subject him to an impressive dressing down. The Buddha compares someone who understands only the letter of the teachings to someone who grabs a snake by the tail, and also invokes the famous simile of the raft.%
\item[MN 23: The Ant-Hill (\textit{\textsanskrit{Vammikasutta}})] In a curious discourse laden with evocative imagery, a deity presents a riddle to a mendicant, who seeks an answer from the Buddha.%
\item[MN 24: Prepared Chariots (\textit{\textsanskrit{Rathavinītasutta}})] Venerable \textsanskrit{Sāriputta} seeks a dialog with an esteemed monk, Venerable \textsanskrit{Puṇṇa} \textsanskrit{Mantāniputta}, and they discuss the stages of purification.%
\item[MN 25: Fodder (\textit{\textsanskrit{Nivāpasutta}})] The Buddha compares getting trapped by \textsanskrit{Māra} with a deer getting caught in a snare, illustrating the ever more complex strategies employed by hunter and hunted.%
\item[MN 26: The Noble Search (\textit{\textsanskrit{Pāsarāsisutta}})] This is one of the most important biographical discourses, telling the Buddha’s experiences from leaving home to realizing awakening. Throughout, he was driven by the imperative to fully escape from rebirth and suffering.%
\item[MN 27: The Shorter Elephant’s Footprint Simile (\textit{\textsanskrit{Cūḷahatthipadopamasutta}})] The Buddha cautions against swift conclusions about a teacher’s spiritual accomplishments, comparing it to the care a tracker would use when tracking elephants. He presents the full training of a monastic.%
\item[MN 28: The Longer Simile of the Elephant’s Footprint (\textit{\textsanskrit{Mahāhatthipadopamasutta}})] \textsanskrit{Sāriputta} gives an elaborate demonstration of how, just as any footprint can fit inside an elephant’s, all the Buddha’s teaching can fit inside the four noble truths. This offers an overall template for organizing the Buddha’s teachings.%
\item[MN 29: The Longer Simile of the Heartwood (\textit{\textsanskrit{Mahāsāropamasutta}})] Following the incident with Devadatta, the Buddha cautions the mendicants against becoming complacent with superficial benefits of spiritual life and points to liberation as the true heart of the teaching.%
\item[MN 30: The Shorter Simile of the Heartwood (\textit{\textsanskrit{Cūḷasāropamasutta}})] Similar to the previous. After the incident with Devadatta, the Buddha cautions the mendicants against becoming complacent and points to liberation as the true heart of the teaching.%
\item[The Greater Chapter on Pairs (\textit{\textsanskrit{Mahāyamakavagga}})] Discourses arranged as pairs of longer and shorter texts.%
\item[MN 31: The Shorter Discourse at \textsanskrit{Gosiṅga} (\textit{\textsanskrit{Cūḷagosiṅgasutta}})] The Buddha comes across three mendicants practicing diligently and harmoniously, and asks them how they do it. Reluctant to disclose their higher attainments, they explain how they deal with the practical affairs of living together. But when pressed by the Buddha, they reveal their meditation attainments.%
\item[MN 32: The Longer Discourse at \textsanskrit{Gosiṅga} (\textit{\textsanskrit{Mahāgosiṅgasutta}})] Several senior mendicants, reveling in the beauty of the night, discuss what kind of practitioner would adorn the park. They take their answers to the Buddha, who praises their answers, but gives his own twist.%
\item[MN 33: The Longer Discourse on the Cowherd (\textit{\textsanskrit{Mahāgopālakasutta}})] For eleven reasons a cowherd is not able to properly look after a herd. The Buddha compares this to the spiritual growth of a mendicant.%
\item[MN 34: The Shorter Discourse on the Cowherd (\textit{\textsanskrit{Cūḷagopālakasutta}})] Drawing parallels with a cowherd guiding his herd across a dangerous river, the Buddha presents the various kinds of enlightened disciples who cross the stream of transmigration.%
\item[MN 35: The Shorter Discourse With Saccaka (\textit{\textsanskrit{Cūḷasaccakasutta}})] Saccaka was a debater, who challenged the Buddha to a contest. Despite his bragging, the Buddha is not at all perturbed at his attacks.%
\item[MN 36: The Longer Discourse With Saccaka (\textit{\textsanskrit{Mahāsaccakasutta}})] In a less confrontational meeting, the Buddha and Saccaka discuss the difference between physical and mental development. The Buddha gives a long account of the various practices he did before awakening, detailing the astonishing lengths he took to mortify the body.%
\item[MN 37: The Shorter Discourse on the Ending of Craving (\textit{\textsanskrit{Cūḷataṇhāsaṅkhayasutta}})] \textsanskrit{Moggallāna} visits the heaven of Sakka, the lord of gods, to see whether he really understands what the Buddha is teaching.%
\item[MN 38: The Longer Discourse on the Ending of Craving (\textit{\textsanskrit{Mahātaṇhāsaṅkhayasutta}})] To counter the wrong view that a self-identical consciousness transmigrates from one life to the next, the Buddha teaches dependent origination, showing that consciousness invariably arises dependent on conditions.%
\item[MN 39: The Longer Discourse at Assapura (\textit{\textsanskrit{Mahāassapurasutta}})] The Buddha encourages the mendicants to live up to their name, by actually practicing in a way that meets or exceeds the expectations people have for renunciants.%
\item[MN 40: The Shorter Discourse at Assapura (\textit{\textsanskrit{Cūḷaassapurasutta}})] The labels of being a spiritual practitioner don’t just come from external trappings, but from sincere inner change.%
\item[The Lesser Chapter on Pairs (\textit{\textsanskrit{Cūḷayamakavagga}})] A similar arrangement to the previous.%
\item[MN 41: The People of \textsanskrit{Sālā} (\textit{\textsanskrit{Sāleyyakasutta}})] The Buddha explains to a group of brahmins the conduct leading to rebirth in higher or lower states, including detailed explanations of the ten core practices which lay people should undertake, and which also form the basis for liberation.%
\item[MN 42: The People of \textsanskrit{Verañja} (\textit{\textsanskrit{Verañjakasutta}})] Similar to the previous. The Buddha explains the conduct leading to rebirth in higher or lower states, including detailed explanations of the ten core practices.%
\item[MN 43: The Great Classification (\textit{\textsanskrit{Mahāvedallasutta}})] A series of questions and answers between \textsanskrit{Sāriputta} and \textsanskrit{Mahākoṭṭhita}, examining various subtle and abstruse aspects of the teachings.%
\item[MN 44: The Shorter Classification (\textit{\textsanskrit{Cūḷavedallasutta}})] The layman \textsanskrit{Visākha} asks the nun \textsanskrit{Dhammadinnā} about various difficult matters, including some of the highest meditation attainments. The Buddha fully endorses her answers.%
\item[MN 45: The Shorter Discourse on Taking Up Practices (\textit{\textsanskrit{Cūḷadhammasamādānasutta}})] The Buddha explains how taking up different practices may have harmful or beneficial results. The memorable simile of the creeper shows how insidious temptations can be.%
\item[MN 46: The Great Discourse on Taking Up Practices (\textit{\textsanskrit{Mahādhammasamādānasutta}})] While we all want to be happy, we often find the opposite happens. The Buddha explains why.%
\item[MN 47: The Inquirer (\textit{\textsanskrit{Vīmaṁsakasutta}})] While some spiritual teachers prefer to remain in obscurity, the Buddha not only encouraged his followers to closely investigate him, but gave them a detailed and demanding method to do so.%
\item[MN 48: The Mendicants of Kosambi (\textit{\textsanskrit{Kosambiyasutta}})] Despite the Buddha’s presence, the monks of Kosambi fell into a deep and bitter dispute. The Buddha taught the reluctant monks to develop love and harmony, reminding them of the state of peace that they sought.%
\item[MN 49: On the Invitation of \textsanskrit{Brahmā} (\textit{\textsanskrit{Brahmanimantanikasutta}})] The Buddha ascends to a high heavenly realm where he engages in a cosmic contest with a powerful divinity, who had fallen into the delusion that he was eternal and all-powerful.%
\item[MN 50: The Rebuke of \textsanskrit{Māra} (\textit{\textsanskrit{Māratajjanīyasutta}})] \textsanskrit{Māra}, the trickster and god of death, tried to annoy \textsanskrit{Moggallāna}. He not only failed but was subject to a stern sermon warning of the dangers of attacking the Buddha’s disciples.%
\end{description}

%
\mainmatter%
\pagestyle{fancy}%
\addtocontents{toc}{\let\protect\contentsline\protect\nopagecontentsline}
\part*{The First Fifty }
\addcontentsline{toc}{part}{The First Fifty }
\markboth{}{}
\addtocontents{toc}{\let\protect\contentsline\protect\oldcontentsline}

%
\addtocontents{toc}{\let\protect\contentsline\protect\nopagecontentsline}
\chapter*{The Chapter on the Root of All Things }
\addcontentsline{toc}{chapter}{\tocchapterline{The Chapter on the Root of All Things }}
\addtocontents{toc}{\let\protect\contentsline\protect\oldcontentsline}

%
\section*{{\suttatitleacronym MN 1}{\suttatitletranslation The Root of All Things }{\suttatitleroot Mūlapariyāyasutta}}
\addcontentsline{toc}{section}{\tocacronym{MN 1} \toctranslation{The Root of All Things } \tocroot{Mūlapariyāyasutta}}
\markboth{The Root of All Things }{Mūlapariyāyasutta}
\extramarks{MN 1}{MN 1}

\scevam{So\marginnote{1.1} I have heard.\footnote{Just as the \textsanskrit{Dīghanikāya} begins with the complex and demanding \textsanskrit{Brahmajālasutta}, the Majjhima opens with one of the most abstruse discourses in the canon. It examines the ways that the process of perception and identification evolves with progress on the path. It was translated, together with its commentary and extensive analysis, by Bhikkhu Bodhi as \emph{The Discourse on the Root of Existence}. The commentarial background is also found in the \textsanskrit{Mūlapariyāya} \textsanskrit{Jātaka} (\href{https://suttacentral.net/ja245/en/sujato}{Ja 245}). The commentary connects this sutta with the Gotamakacetiyasutta (\href{https://suttacentral.net/an3.125/en/sujato}{AN 3.125}), but there is no internal evidence to support this. } }At one time the Buddha was staying near \textsanskrit{Ukkaṭṭhā}, in the Subhaga Forest at the root of a magnificent sal tree.\footnote{\textsanskrit{Ukkaṭṭhā}, near \textsanskrit{Sāvatthī}, is mentioned only rarely, and always in the context of extraordinary teachings and events that emphasize the cosmic grandeur of the Buddha against the brahmins, likely because it was the home of the prominent Kosalan brahmin \textsanskrit{Pokkharasāti} (\href{https://suttacentral.net/dn3/en/sujato\#1.2.1}{DN 3:1.2.1}, \href{https://suttacentral.net/dn14/en/sujato\#3.29.1}{DN 14:3.29.1}, \href{https://suttacentral.net/mn49/en/sujato\#2.1}{MN 49:2.1}). } There the Buddha addressed the mendicants, “Mendicants!”\footnote{The pattern of this discourse answers to such passages as \textsanskrit{Bṛhadāraṇyaka} \textsanskrit{Upaniṣad} 3.7, where \textsanskrit{Yājñavalkya} expounds a series of principles in relation to which the “immortal self” is conceived. The commentary says that this discourse was delivered to a group of former brahmins who had become conceited when they learned the Buddha’s teaching. While the text certainly responds to ideas and methods of Brahmanical texts, that interpretation is not supported by the text. } 

“Venerable\marginnote{1.5} sir,” they replied. The Buddha said this: 

“Mendicants,\marginnote{2.1} I will teach you the explanation of the root of all things.\footnote{In his third discourse, speaking to Brahmanical ascetics, the Buddha reframed the “all” as the experience of the six senses (\href{https://suttacentral.net/sn35.28/en/sujato}{SN 35.28}). The distinctive “conceiving” pattern of this sutta is therefore also applied to the “all” of the six senses (\href{https://suttacentral.net/sn35.30/en/sujato\#1.19}{SN 35.30:1.19}, \href{https://suttacentral.net/sn35.90/en/sujato\#3.7}{SN 35.90:3.7}). More broadly, the same pattern is also applied to the “aggregates, elements, and sense fields” (\href{https://suttacentral.net/sn35.31/en/sujato\#1.21}{SN 35.31:1.21}, \href{https://suttacentral.net/sn35.91/en/sujato\#4.1}{SN 35.91:4.1}). | The meaning of “root” is clarified later (\href{https://suttacentral.net/mn1/en/sujato\#171.4}{MN 1:171.4}) as “taking pleasure”, i.e. craving, which is the “root” of suffering. } Listen and apply your minds well, I will speak.” 

“Yes,\marginnote{2.3} sir,” they replied. The Buddha said this: 

“Take\marginnote{3.1} an unlearned ordinary person who has not seen the noble ones, and is neither skilled nor trained in the teaching of the noble ones. They’ve not seen true persons, and are neither skilled nor trained in the teaching of the true persons.\footnote{An “unlearned ordinary person”, who has not realized any of the stages of the noble path, is contrasted with one who has entered the path. | “Noble one” (\textit{ariya}) loosely conveys the sense “cultured” or “civilized”; it is a term for the inheritors of the Aryan culture that originated among the proto-Indo-European peoples of the central Asian steppes. | “True person” (\textit{sappurisa}) indicates one who is authentic and genuine in their realization of the truth, and hence is virtuous and good. Both “noble one” and “true person” are technical terms referring to any person who has at least entered the path to stream-entry. } They perceive earth as earth.\footnote{Although their perception (\textit{\textsanskrit{saññā}}) is accurate, to perceive something “as” something is to recognize it filtered through memory and concepts learned in the past, a subtle pre-processing that interprets present experience in light of expectations and desires. | The ending \textit{-to} here and throughout is the “ablative of perspective”, which is used with verbs of cognition to express the idea of seeing something in a certain light; for example, one contemplates the body “as impermanent” (\href{https://suttacentral.net/mn74/en/sujato\#9.1}{MN 74:9.1}). } Having perceived earth as earth, they conceive it to be earth, they conceive it in earth, they conceive it as earth, they conceive that ‘earth is mine’, they approve earth.\footnote{To “conceive” or “imagine” (\textit{\textsanskrit{maññati}}) is, according to the commentary, to think in terms of a “self”, proliferating experience through craving, conceit, or views until it is constructed \emph{for me}. This usage draws upon such passages as \textsanskrit{Bṛhadāraṇyaka} \textsanskrit{Upaniṣad} 4.3.20, where due to ignorance, a person “imagines” in a dream the fearful things they saw when awake, or at the highest level, “imagines I am this all” (\textit{\textsanskrit{ahamevedaṁ} sarvo’\textsanskrit{smīti} manyate}). | Each of the five phrases takes the “perception of earth as earth” and conceives, imagines, or construes that perception in progressively more differentiated and objectified ways, until it becomes something that is owned and enjoyed. } Why is that? Because they haven’t completely understood it, I say.\footnote{“Complete understanding” (\textit{\textsanskrit{pariññā}}) is the understanding of the arahant that permanently cuts through all delusions and conceits. } 

They\marginnote{4.1} perceive water as water.\footnote{The sutta proceeds through the four main physical elements or properties before proceeding to beings and then various deities. The difference between these things is not as clear-cut as one might think. The elements were worshiped as gods, while the gods were often anthropomorphized natural phenomena such as the sky (\textit{deva}) or the sun (“streaming radiance”). To identify with a material element is to share the essence of a powerful force of nature. | A similar list, but with fewer items, is found starting at \href{https://suttacentral.net/mn49/en/sujato\#11.1}{MN 49:11.1}. } Having perceived water as water, they conceive it to be water … Why is that? Because they haven’t completely understood it, I say. 

They\marginnote{5.1} perceive fire as fire. Having perceived fire as fire, they conceive it to be fire … Why is that? Because they haven’t completely understood it, I say. 

They\marginnote{6.1} perceive air as air. Having perceived air as air, they conceive it to be air … Why is that? Because they haven’t completely understood it, I say. 

They\marginnote{7.1} perceive creatures as creatures.\footnote{“Creatures” (\textit{\textsanskrit{bhūta}}) can refer to any living being, including humans and animals, as well as invisible entities such as ghosts. } Having perceived creatures as creatures, they conceive them to be creatures … Why is that? Because they haven’t completely understood it, I say. 

They\marginnote{8.1} perceive gods as gods.\footnote{“Gods” (\textit{deva}) or “deities” (\textit{\textsanskrit{devatā}}) is a generic description of the many divine entities of ancient Indian belief. Some were inherited from the old Vedic theology, while others reflect local customs and beliefs. All are impermanent and subject to suffering. } Having perceived gods as gods, they conceive them to be gods … Why is that? Because they haven’t completely understood it, I say. 

They\marginnote{9.1} perceive the Progenitor as the Progenitor.\footnote{This is \textsanskrit{Pajāpati}, the lonely creator god of Vedic belief. Having set the world in motion he was largely forgotten. } Having perceived the Progenitor as the Progenitor, they conceive him to be the Progenitor … Why is that? Because they haven’t completely understood it, I say. 

They\marginnote{10.1} perceive the Divinity as the Divinity.\footnote{\textsanskrit{Brahmā} is also regarded as a creator, but in the sense of the underlying divine force that sustains the life of the cosmos. In Buddhism, several individual \textsanskrit{Brahmās} appear, depicted as high deities who achieved their status due to the practice of first \textit{\textsanskrit{jhāna}} in a past life. } Having perceived the Divinity as the Divinity, they conceive him to be the Divinity … Why is that? Because they haven’t completely understood it, I say. 

They\marginnote{11.1} perceive those of streaming radiance as those of streaming radiance.\footnote{This and the next two are higher \textsanskrit{Brahmā} realms. Beings in this realm are sometimes called “gods” (\textit{\textsanskrit{devā}}). They achieved their status through the second, third, and fourth \textit{\textsanskrit{jhānas}} respectively. Later Brahmanical texts mention a class of \textsanskrit{Ābhāsvara} deities, but it does not appear to be a Vedic concept. } Having perceived those of streaming radiance as those of streaming radiance, they conceive them to be those of streaming radiance … Why is that? Because they haven’t completely understood it, I say. 

They\marginnote{12.1} perceive those of universal beauty as those of universal beauty.\footnote{“Universal beauty” is \textit{\textsanskrit{subhakiṇha}}. \textit{Subha} is “beauty, radiance”. \textit{\textsanskrit{Kiṇha}} is “universal, entire, total” (= Sanskrit \textit{\textsanskrit{kṛtsna}}); the same word is the basis for the meditation on “universals” (\textit{\textsanskrit{kasiṇa}}). The concept appears to be Buddhist, but we find a precedent when \textsanskrit{Yājñavalkya} says that, just as salt is “entirely” salty, the Self is an “entire mass of consciousness” (\textit{\textsanskrit{kṛtsnaḥ} \textsanskrit{prajñānaghana} eva}, \textsanskrit{Bṛhadāraṇyaka} \textsanskrit{Upaniṣad} 4.5.13). } But then they conceive them to be those of universal beauty … Why is that? Because they haven’t completely understood it, I say. 

They\marginnote{13.1} perceive those of abundant fruit as those of abundant fruit.\footnote{The gods of “abundant fruit” (\textit{vehapphala}; Sanskrit \textit{\textsanskrit{bṛhatphala}}) do not appear in Brahmanical literature, but \textit{\textsanskrit{bṛhat}} is a common descriptor of divinity. See eg. the Vedic god \textsanskrit{Bṛhaspati}, identified with the planet Jupiter, or Rig Veda 9.107.15, which describes Soma as \textit{\textsanskrit{ṛtam} \textsanskrit{bṛhat}}, “vast and true”. } Having perceived those of abundant fruit as those of abundant fruit, they conceive them to be those of abundant fruit … Why is that? Because they haven’t completely understood it, I say. 

They\marginnote{14.1} perceive the Vanquisher as the Vanquisher.\footnote{“Vanquisher” (\textit{\textsanskrit{abhibhū}}) is an epithet of \textsanskrit{Brahmā} (\href{https://suttacentral.net/mn49/en/sujato\#5.2}{MN 49:5.2}) that was appropriated for the Buddha (\href{https://suttacentral.net/an4.23/en/sujato\#5.1}{AN 4.23:5.1}). In Rig Veda 8.97.10 it is an epithet of Indra, but it is not a regular name for a deity in either Buddhism or Brahmanism. Here it appears to be the name of the highest of the \textsanskrit{Brahmā} gods. } Having perceived the Vanquisher as the Vanquisher, they conceive him to be the Vanquisher … Why is that? Because they haven’t completely understood it, I say. 

They\marginnote{15.1} perceive the dimension of infinite space as the dimension of infinite space.\footnote{Here begins the series of realms associated with the practice of formless meditations. These were practiced by the most advanced non-dualist Brahmanical teachers before the Buddha, who identified such experiences with the highest Self that is the cosmic divinity. The Buddha adopted the practices as part of the development of meditation, divesting them of metaphysical significance. } Having perceived the dimension of infinite space as the dimension of infinite space, they conceive it to be the dimension of infinite space …\footnote{“Space” (\textit{\textsanskrit{ākāsa}}) is a principle of deep significance in the \textsanskrit{Upaniṣads}, yet it is ultimately a lesser manifestation of the Absolute. See eg. \textsanskrit{Bṛhadāraṇyaka} \textsanskrit{Upaniṣad} 3.8.7; \textsanskrit{Taittirīya} \textsanskrit{Upaniṣad} 2.1.1; \textsanskrit{Chāndogya} \textsanskrit{Upaniṣad} 7.12. } Why is that? Because they haven’t completely understood it, I say. 

They\marginnote{16.1} perceive the dimension of infinite consciousness as the dimension of infinite consciousness.\footnote{“Infinite consciousness” is identified with the highest Self by \textsanskrit{Yājñavalkya} at \textsanskrit{Bṛhadāraṇyaka} \textsanskrit{Upaniṣad} 2.4.12. } Having perceived the dimension of infinite consciousness as the dimension of infinite consciousness, they conceive it to be the dimension of infinite consciousness … Why is that? Because they haven’t completely understood it, I say. 

They\marginnote{17.1} perceive the dimension of nothingness as the dimension of nothingness.\footnote{Taught by the Brahmanical sage \textsanskrit{Āḷāra} \textsanskrit{Kālāma} at \href{https://suttacentral.net/mn26/en/sujato\#15.13}{MN 26:15.13}. } Having perceived the dimension of nothingness as the dimension of nothingness, they conceive it to be the dimension of nothingness … Why is that? Because they haven’t completely understood it, I say. 

They\marginnote{18.1} perceive the dimension of neither perception nor non-perception as the dimension of neither perception nor non-perception.\footnote{Taught by the Brahmanical sage Uddaka \textsanskrit{Rāmaputta} at \href{https://suttacentral.net/mn26/en/sujato\#16.13}{MN 26:16.13}. } Having perceived the dimension of neither perception nor non-perception as the dimension of neither perception nor non-perception, they conceive it to be the dimension of neither perception nor non-perception … Why is that? Because they haven’t completely understood it, I say. 

They\marginnote{19.1} perceive the seen as the seen.\footnote{The discourse presents four items—the seen, heard, thought, and known—which describe the means of knowing spiritual truths: the sight of a holy person, hearing a teaching, contemplating the truth, and meditative awareness. This framework, which is found commonly in the suttas, was adopted from \textsanskrit{Yājñavalkya}; for example at \textsanskrit{Bṛhadāraṇyaka} \textsanskrit{Upaniṣad} 3.8.11 he describes the Absolute as “the unseen seer, the unheard hearer, the unthought thinker, the unknown knower”. } Having perceived the seen as the seen, they conceive it to be the seen …\footnote{See \href{https://suttacentral.net/snp4.4/en/sujato}{Snp 4.4} for a more detailed critique of “seeing” a holy person as a standard of truth. } Why is that? Because they haven’t completely understood it, I say. 

They\marginnote{20.1} perceive the heard as the heard.\footnote{This refers to teachings that are “heard” or passed down in oral tradition. It includes Vedic scriptures (\textit{\textsanskrit{śruti}}) that were believed to have been “heard” or transmitted by divine inspiration, as well Buddhist scriptures, which begin, “So I have heard”. No scripture is infallible (\href{https://suttacentral.net/mn76/en/sujato\#25.2}{MN 76:25.2}). } Having perceived the heard as the heard, they conceive it to be the heard … Why is that? Because they haven’t completely understood it, I say. 

They\marginnote{21.1} perceive the thought as the thought.\footnote{\textit{Muta} means “(what is) thought”, but is often mistranslated as “sensed”, a meaning that does not apply in the early texts. Philosophical thought, like scripture, is fallible (\href{https://suttacentral.net/mn76/en/sujato\#27.3}{MN 76:27.3}), but people get attached to their theories (\href{https://suttacentral.net/snp4.5/en/sujato}{Snp 4.5}). } Having perceived the thought as the thought, they conceive it to be the thought … Why is that? Because they haven’t completely understood it, I say. 

They\marginnote{22.1} perceive the known as the known.\footnote{The “known” (\textit{\textsanskrit{viññāta}}) is that which is cognized with consciousness (\textit{\textsanskrit{viññāṇa}}), especially states of expanded consciousness in deep meditation. Even such states are not immune to misinterpretation (eg. \href{https://suttacentral.net/mn136/en/sujato\#9.1}{MN 136:9.1}, \href{https://suttacentral.net/dn1/en/sujato\#1.31.1}{DN 1:1.31.1}). } Having perceived the known as the known, they conceive it to be the known … Why is that? Because they haven’t completely understood it, I say. 

They\marginnote{23.1} perceive oneness as oneness.\footnote{Perception of “oneness” (\textit{ekatta}) sees the world as manifold reflections of an underlying unity. Arising from meditative experience or philosophical reflection, it is associated with the non-dual schools of Brahmanism. \textsanskrit{Īśa} \textsanskrit{Upaniṣad} 7, for example, speaks of “contemplating the oneness” (\textit{ekatvam \textsanskrit{anupaśyataḥ}}) of all creatures with the supreme soul. \textsanskrit{Yājñavalkya} said in the state of non-differentiation the Self “becomes clear as water, one, the seer without a second; this is the \textsanskrit{Brahmā} realm” (\textsanskrit{Bṛhadāraṇyaka} \textsanskrit{Upaniṣad} 4.3.32: \textit{salila eko \textsanskrit{draṣṭādvaito} bhavati, \textsanskrit{eṣa} brahmalokaḥ}). } Having perceived oneness as oneness, they conceive it to be oneness … Why is that? Because they haven’t completely understood it, I say. 

They\marginnote{24.1} perceive diversity as diversity.\footnote{“Diversity” (\textit{\textsanskrit{nānatta}}) is the opposite fallacy to “oneness”; based on the diversity of sense experience, it interprets the world as irreducibly manifold. An example would be the Jains, who believed the world was made up of countless separate entities, a view criticized in \textsanskrit{Māṇḍūkya} \textsanskrit{Upaniṣad} 3.13 (\textit{\textsanskrit{nānātvaṁ} nindyate}). Both these fallacies take a particular mode of perception, which is true in certain respects, and apply it as an absolute. } Having perceived diversity as diversity, they conceive it to be diversity … Why is that? Because they haven’t completely understood it, I say. 

They\marginnote{25.1} perceive all as all.\footnote{The “all” is another critical term in the \textsanskrit{Upaniṣads}, representing the totality of creation as an expression of divinity. See eg. \textsanskrit{Chāndogya} \textsanskrit{Upaniṣad} 7.25.2, “the self is all this” (\textit{\textsanskrit{ātmaivedaṁ} sarvamiti}), or \textsanskrit{Bṛhadāraṇyaka} \textsanskrit{Upaniṣad} 2.5.19, “this self that experiences all is divinity” (\textit{ayam \textsanskrit{ātmā} brahma \textsanskrit{sarvānubhūḥ}}). } Having perceived all as all, they conceive it to be all … Why is that? Because they haven’t completely understood it, I say. 

They\marginnote{26.1} perceive extinguishment as extinguishment.\footnote{It is puzzling to see “extinguishment” (\textit{\textsanskrit{nibbāna}}; Sanskrit \textit{\textsanskrit{nirvāṇa}}) here, as it is the end of conceiving. The similar sequence at \href{https://suttacentral.net/mn49/en/sujato\#23.1}{MN 49:23.1} culminates with “all”. Three interpretations: (1) \emph{Simple textual error}. Of the three Chinese parallels, EA 44.6 mentions \textit{\textsanskrit{nibbāna}} here, while MA 106 and T 56 do not. If two separate texts have the same error, it predates the separation between the schools, or arose later through contamination. (2) \emph{The five kinds of “extinguishment in the present life”}. These are false liberations believed by sectarians (\textsanskrit{Brahmajālasutta}, \href{https://suttacentral.net/dn1/en/sujato\#3.19.1}{DN 1:3.19.1}). The commentary says they are meant here, but it seems unlikely, given that below the stream-enterer is enjoined to not identify with \textit{\textsanskrit{nibbāna}}, whereas they have already dispelled such misconceptions of the path. (3) \emph{An unenlightened person’s misconception of the Buddhist goal}. At least some ancient Buddhists read it this way, as this passage is quoted in an Abhidhamma discussion as to whether the deathless as an object of thought can be a fetter (\textsanskrit{Kathāvatthu} 9.2). The \textsanskrit{Theravāda} commentary, rejecting this, says it was the view of the Pubbaseliyas, a branch of the \textsanskrit{Mahāsaṅghikas}. } Having perceived extinguishment as extinguishment, they conceive it to be extinguishment, they conceive it in extinguishment, they conceive it as extinguishment, they conceive that ‘extinguishment is mine’, they approve extinguishment. Why is that? Because they haven’t completely understood it, I say. 

A\marginnote{27.1} mendicant who is a trainee, who hasn’t achieved their heart’s desire, but lives aspiring to the supreme sanctuary from the yoke, directly knows earth as earth.\footnote{A “trainee” (\textit{sekha}), who has realized stream-entry, once-return, or non-return, has had a vision of the path and \textsanskrit{Nibbāna}. Yet since they have not fully relinquished the fetters that bind them to transmigration, they continue to deepen their practice of the noble eightfold path. Their “direct knowing” (\textit{\textsanskrit{abhiññā}}) is purified by the twin powers of \textit{samatha} and \textit{\textsanskrit{vipassanā}} meditation, rather than “perception” (\textit{\textsanskrit{saññā}}), which is filtered through the five hindrances and other cognitive distortions. This distinction between perception and higher awareness (\textit{\textsanskrit{vijñāna}} or \textit{\textsanskrit{prajñāna}}) was first made by \textsanskrit{Yājñavalkya} (\textsanskrit{Bṛhadāraṇyaka} \textsanskrit{Upaniṣad} 2.4.12 and 4.5.13). } Having directly known earth as earth, let them not conceive it to be earth, let them not conceive it in earth, let them not conceive it as earth, let them not conceive that ‘earth is mine’, let them not approve earth.\footnote{The sutta depicts progress through the path in three grammatical modes. The ordinary person conceives; the trainee ought not conceive; the perfected one does not conceive. This echoes the “three rounds” of the first sermon (Dhammacakkappavattanasutta, \href{https://suttacentral.net/sn56.11/en/sujato}{SN 56.11}): “there is” suffering; suffering “should be understood”; suffering “has been understood”. } Why is that? So that they may completely understand it, I say. 

They\marginnote{28.1} directly know water … fire … air … creatures … gods … the Progenitor … the Divinity … those of streaming radiance … those replete with glory … those of abundant fruit … the Vanquisher … the dimension of infinite space … the dimension of infinite consciousness … the dimension of nothingness … the dimension of neither perception nor non-perception …\footnote{The stream-enterer has not necessarily had personal experience of all these meditation states and realms of existence. Yet they “directly know” dependent origination, and hence understand that all such states are conditioned, impermanent, and included within the round of transmigration. } the seen … the heard … the thought … the known … oneness … diversity … all … They directly know extinguishment as extinguishment. Having directly known extinguishment as extinguishment, let them not conceive it to be extinguishment, let them not conceive it in extinguishment, let them not conceive it as extinguishment, let them not conceive that ‘extinguishment is mine’, let them not approve extinguishment. Why is that? So that they may completely understand it, I say. 

A\marginnote{51.1} mendicant who is perfected—with defilements ended, who has completed the spiritual journey, done what had to be done, laid down the burden, achieved their own true goal, utterly ended the fetter of continued existence, and is rightly freed through enlightenment—directly knows earth as earth.\footnote{The “perfected one” is the \textit{arahant}, literally “worthy one”, who is the Buddhist spiritual ideal. Their direct knowing is so powerful that it has cut through all fetters binding them to transmigration. } Having directly known earth as earth, they do not conceive it to be earth, they do not conceive it in earth, they do not conceive it as earth, they do not conceive that ‘earth is mine’, they do not approve earth. Why is that? Because they have completely understood it, I say. 

They\marginnote{52.1} directly know water … fire … air … creatures … gods … the Progenitor … the Divinity … those of streaming radiance … those replete with glory … those of abundant fruit … the Vanquisher … the dimension of infinite space … the dimension of infinite consciousness … the dimension of nothingness … the dimension of neither perception nor non-perception … the seen … the heard … the thought … the known … oneness … diversity … all … They directly know extinguishment as extinguishment. Having directly known extinguishment as extinguishment, they do not conceive it to be extinguishment, they do not conceive it in extinguishment, they do not conceive it as extinguishment, they do not conceive that ‘extinguishment is mine’, they do not approve extinguishment. Why is that? Because they have completely understood it, I say. 

A\marginnote{75.1} mendicant who is perfected—with defilements ended, who has completed the spiritual journey, done what had to be done, laid down the burden, achieved their own true goal, utterly ended the fetter of continued existence, and is rightly freed through enlightenment—directly knows earth as earth. Having directly known earth as earth, they do not conceive it to be earth, they do not conceive it in earth, they do not conceive it as earth, they do not conceive that ‘earth is mine’, they do not approve earth. Why is that? Because they’re free of greed due to the ending of greed.\footnote{The text repeats the passage on the perfected one three times, emphasizing the ending of greed, hate, and delusion respectively. } 

They\marginnote{76.1} directly know water … fire … air … creatures … gods … the Progenitor … the Divinity … those of streaming radiance … those replete with glory … those of abundant fruit … the Vanquisher … the dimension of infinite space … the dimension of infinite consciousness … the dimension of nothingness … the dimension of neither perception nor non-perception … the seen … the heard … the thought … the known … oneness … diversity … all … They directly know extinguishment as extinguishment. Having directly known extinguishment as extinguishment, they do not conceive it to be extinguishment, they do not conceive it in extinguishment, they do not conceive it as extinguishment, they do not conceive that ‘extinguishment is mine’, they do not approve extinguishment. Why is that? Because they’re free of greed due to the ending of greed. 

A\marginnote{99.1} mendicant who is perfected—with defilements ended, who has completed the spiritual journey, done what had to be done, laid down the burden, achieved their own true goal, utterly ended the fetter of continued existence, and is rightly freed through enlightenment—directly knows earth as earth. Having directly known earth as earth, they do not conceive it to be earth, they do not conceive it in earth, they do not conceive it as earth, they do not conceive that ‘earth is mine’, they do not approve earth. Why is that? Because they’re free of hate due to the ending of hate. 

They\marginnote{100.1} directly know water … fire … air … creatures … gods … the Progenitor … the Divinity … those of streaming radiance … those replete with glory … those of abundant fruit … the Vanquisher … the dimension of infinite space … the dimension of infinite consciousness … the dimension of nothingness … the dimension of neither perception nor non-perception … the seen … the heard … the thought … the known … oneness … diversity … all … They directly know extinguishment as extinguishment. Having directly known extinguishment as extinguishment, they do not conceive it to be extinguishment, they do not conceive it in extinguishment, they do not conceive it as extinguishment, they do not conceive that ‘extinguishment is mine’, they do not approve extinguishment. Why is that? Because they’re free of hate due to the ending of hate. 

A\marginnote{123.1} mendicant who is perfected—with defilements ended, who has completed the spiritual journey, done what had to be done, laid down the burden, achieved their own true goal, utterly ended the fetter of continued existence, and is rightly freed through enlightenment—directly knows earth as earth. Having directly known earth as earth, they do not conceive it to be earth, they do not conceive it in earth, they do not conceive it as earth, they do not conceive that ‘earth is mine’, they do not approve earth. Why is that? Because they’re free of delusion due to the ending of delusion. 

They\marginnote{124.1} directly know water … fire … air … creatures … gods … the Progenitor … the Divinity … those of streaming radiance … those replete with glory … those of abundant fruit … the Vanquisher … the dimension of infinite space … the dimension of infinite consciousness … the dimension of nothingness … the dimension of neither perception nor non-perception … the seen … the heard … the thought … the known … oneness … diversity … all … They directly know extinguishment as extinguishment. Having directly known extinguishment as extinguishment, they do not conceive it to be extinguishment, they do not conceive it in extinguishment, they do not conceive it as extinguishment, they do not conceive that ‘extinguishment is mine’, they do not approve extinguishment. Why is that? Because they’re free of delusion due to the ending of delusion. 

The\marginnote{147.1} Realized One, the perfected one, the fully awakened Buddha directly knows earth as earth.\footnote{The Buddha is an \textit{arahant}, and he shares his fundamental understanding with other arahants. Yet the suttas elevate his understanding as the one who discovered the path. } Having directly known earth as earth, he does not conceive it to be earth, he does not conceive it in earth, he does not conceive it as earth, he does not conceive that ‘earth is mine’, he does not approve earth. Why is that? Because the Realized One has completely understood it to the end, I say.\footnote{To “completely understood to the end” is a phrase unique to this sutta. It implies that, while other arahants understand phenomena to the extent necessary for release, the Buddha understands all phenomena without exception. } 

He\marginnote{148.1} directly knows water … fire … air … creatures … gods … the Progenitor … the Divinity … those of streaming radiance … those replete with glory … those of abundant fruit … the Vanquisher … the dimension of infinite space … the dimension of infinite consciousness … the dimension of nothingness … the dimension of neither perception nor non-perception … the seen … the heard … the thought … the known … oneness … diversity … all … He directly knows extinguishment as extinguishment. Having directly known extinguishment as extinguishment, he does not conceive it to be extinguishment, he does not conceive it in extinguishment, he does not conceive it as extinguishment, he does not conceive that ‘extinguishment is mine’, he does not approve extinguishment. Why is that? Because the Realized One has completely understood it to the end, I say. 

The\marginnote{171.1} Realized One, the perfected one, the fully awakened Buddha directly knows earth as earth. Having directly known earth as earth, he does not conceive it to be earth, he does not conceive it in earth, he does not conceive it as earth, he does not conceive that ‘earth is mine’, he does not approve earth. Why is that? Because he has understood that approval is the root of suffering,\footnote{This clarifies an ambiguity in the phrase “approve” (or “delights”, “relishes”, \textit{(abhi)-nandati}). This can have a positive sense, as the audience normally “approves” the Buddha’s teachings (but see the end of this sutta). Here, however, the Buddha clarifies that he is using “approve” in the sense of craving, as it is found in the standard definition of the second noble truth (\href{https://suttacentral.net/sn56.11/en/sujato\#4.4}{SN 56.11:4.4}). } and that rebirth comes from continued existence; whoever has come to be gets old and dies.\footnote{With these lines the Buddha connects the teachings of this sutta with dependent origination. He employs a similar strategy at the end of the \textsanskrit{Brahmajālasutta}. } That’s why the Realized One—with the ending, fading away, cessation, giving up, and letting go of all cravings—has awakened to the supreme perfect awakening, I say. 

He\marginnote{172.1} directly knows water … fire … air … creatures … gods … the Progenitor … the Divinity … those of streaming radiance … those replete with glory … those of abundant fruit … the Vanquisher … the dimension of infinite space … the dimension of infinite consciousness … the dimension of nothingness … the dimension of neither perception nor non-perception … the seen … the heard … the thought … the known … oneness … diversity … all … He directly knows extinguishment as extinguishment. Having directly known extinguishment as extinguishment, he does not conceive it to be extinguishment, he does not conceive it in extinguishment, he does not conceive it as extinguishment, he does not conceive that ‘extinguishment is mine’, he does not approve extinguishment. Why is that? Because he has understood that approval is the root of suffering, and that rebirth comes from continued existence; whoever has come to be gets old and dies. That’s why the Realized One—with the ending, fading away, cessation, giving up, and letting go of all cravings—has awakened to the supreme perfect Awakening, I say.” 

That\marginnote{194.8} is what the Buddha said. But the mendicants did not approve what the Buddha said.\footnote{That they “did not approve” (\textit{na \textsanskrit{abhinanduṁ}}) is confirmed in the commentary and in the parallel at EA 44.6, which explains that the mendicants did not understand the discourse. Alternatively, some modern interpreters (Bodhi, \textsanskrit{Ñāṇananda}), relying on the commentarial background explanation, suggest that the problem was that they understood it all too well and were not happy that their beliefs were challenged. These explanations, however, do not take into account the centrality of “approval” (\textit{nanda})—it is literally the “root” that lends the discourse its title. Given that an arahant does not “approve” even \textsanskrit{Nibbāna}, and that “approval” is the root of suffering, is it any wonder that the audience did not “approve” the teaching? It is not that they did not understand, nor that their understanding threatened their egos, but that they understood so well that they received the teaching with perfect equanimity. Elsewhere the Buddha urged that one should “neither approve (\textit{abhinandati}) nor dismiss” a teaching (\href{https://suttacentral.net/an4.180/en/sujato\#2.5}{AN 4.180:2.5} = \href{https://suttacentral.net/dn16/en/sujato\#4.8.4}{DN 16:4.8.4}, \href{https://suttacentral.net/dn29/en/sujato\#18.4}{DN 29:18.4}, \href{https://suttacentral.net/mn112/en/sujato\#3.1}{MN 112:3.1}). It seems that the normal response, where the audience approves a teaching with pleasure, is meant in a conventional sense, whereas this sutta shows how for an arahant all such responses are transcended. } 

%
\section*{{\suttatitleacronym MN 2}{\suttatitletranslation All the Defilements }{\suttatitleroot Sabbāsavasutta}}
\addcontentsline{toc}{section}{\tocacronym{MN 2} \toctranslation{All the Defilements } \tocroot{Sabbāsavasutta}}
\markboth{All the Defilements }{Sabbāsavasutta}
\extramarks{MN 2}{MN 2}

\scevam{So\marginnote{1.1} I have heard.\footnote{A difficult philosophical text opens both the \textsanskrit{Dīghanikāya} (\textsanskrit{Brahmajālasutta}) and the \textsanskrit{Majjhimanikāya} (\textsanskrit{Mūlapariyāyasutta}) followed by a more practical teaching (\textsanskrit{Sāmaññaphalasutta}, \textsanskrit{Sabbāsavasutta}). } }At one time the Buddha was staying near \textsanskrit{Sāvatthī} in Jeta’s Grove, \textsanskrit{Anāthapiṇḍika}’s monastery. There the Buddha addressed the mendicants, “Mendicants!” 

“Venerable\marginnote{1.5} sir,” they replied. The Buddha said this: 

“Mendicants,\marginnote{2.1} I will teach you the explanation of the restraint of all defilements.\footnote{“Defilements” (\textit{\textsanskrit{āsava}}) are the fundamental “pollutions” or “corruptions” that darken awareness, trapping people in transmigration. In practical application, \textit{\textsanskrit{āsava}} can mean the “discharge” from a sore (\href{https://suttacentral.net/an3.27/en/sujato\#2.4}{AN 3.27:2.4}). The Jain \textsanskrit{Tattvārthasūtra} 6.1–2 defines \textit{\textsanskrit{āsrava}} as the influx of deeds; these pollute the pure soul. | Note that the Pali \textit{\textsanskrit{āsava}} represents two distinct words, which are homonyms in Pali but differentiated in Sanskrit. The word here is Sanskrit \textit{\textsanskrit{āsrava}} (or \textit{\textsanskrit{āśrava}}), whereas Sanskrit \textit{\textsanskrit{āsava}} means “intoxicant”, a sense found elsewhere in Pali (\href{https://suttacentral.net/pli-tv-bu-vb-pc51/en/sujato\#2.1.4}{Bu Pc 51:2.1.4}). } Listen and apply your mind well, I will speak.”\footnote{This opening follows the pattern of \href{https://suttacentral.net/mn1/en/sujato}{MN 1}. } 

“Yes,\marginnote{2.3} sir,” they replied. The Buddha said this: 

“Mendicants,\marginnote{3.1} I say that the ending of defilements is for one who knows and sees, not for one who does not know or see.\footnote{The “ending of defilements” (\textit{\textsanskrit{āsavānaṁ} \textsanskrit{khayaṁ}}) is a common sutta term for arahantship. | “Knowing and seeing” refers to the penetrating insight that arises from meditative immersion (\textit{\textsanskrit{samādhi}}, eg. \href{https://suttacentral.net/an5.168/en/sujato\#2.2}{AN 5.168:2.2}, \href{https://suttacentral.net/an10.2/en/sujato\#2.3}{AN 10.2:2.3}, \href{https://suttacentral.net/sn12.23/en/sujato\#6.18}{SN 12.23:6.18}, \href{https://suttacentral.net/dn2/en/sujato\#83.1}{DN 2:83.1}). } For one who knows and sees what? Rational application of mind and irrational application of mind.\footnote{“Rational application of mind” (\textit{yoniso \textsanskrit{manasikāra}}) is a distinctively Buddhist term. It draws on the frequent Vedic image of the “womb of truth” (\textit{\textsanskrit{yonāv} \textsanskrit{ṛtasya}}, Rig Veda 9.13.9), the source of the laws and patterns that govern the natural order. The idea is that by applying the mind by way of cause or reason the hidden truth can be uncovered. Here the meditator’s insight is framed not as uncovering the objective truth about the world, but as reflexively understanding the means of insight itself. } When you apply the mind irrationally, defilements arise, and once arisen they grow.\footnote{This passage and the next are comparable to the contemplation of the principles of the five hindrances and the seven awakening factors respectively in the fourth section of mindfulness meditation (\textit{\textsanskrit{dhammānupassanā}}; see \href{https://suttacentral.net/mn10/en/sujato\#36.4}{MN 10:36.4}). } When you apply the mind rationally, defilements don’t arise, and those that have already arisen are given up. 

Some\marginnote{4.1} defilements should be given up by seeing, some by restraint, some by using, some by enduring, some by avoiding, some by dispelling, and some by developing.\footnote{Here the Buddha gives the scheme of the sutta: seven methods for getting rid of defilements. Omitting the defilements given up by seeing, the remainder are found at \href{https://suttacentral.net/an6.58/en/sujato}{AN 6.58}. } 

\subsection*{1. Defilements Given Up by Seeing }

And\marginnote{5.1} what are the defilements that should be given up by seeing?\footnote{“Seeing” (\textit{dassana}) the four noble truths with the wisdom of insight. } Take an unlearned ordinary person who has not seen the noble ones, and is neither skilled nor trained in the teaching of the noble ones. They’ve not seen true persons, and are neither skilled nor trained in the teaching of the true persons.\footnote{“Noble Ones” (\textit{\textsanskrit{ariyā}}) and “true persons” (\textit{\textsanskrit{sappurisā}}) both refer to the eight kinds of individual who have entered the eightfold path to awakening. } They don’t understand to which things they should apply the mind and to which things they should not apply the mind. So they apply the mind to things they shouldn’t and don’t apply the mind to things they should. 

And\marginnote{6.1} what are the things to which they apply the mind but should not? They are the things that, when the mind is applied to them, give rise to unarisen defilements and make arisen defilements grow: the defilements of sensual desire,\footnote{This indicates one sense of what “irrational” means: it creates the very things that one is trying to escape. } desire to be reborn, and ignorance. These are the things to which they apply the mind but should not. 

And\marginnote{6.6} what are the things to which they do not apply the mind but should? They are the things that, when the mind is applied to them, do not give rise to unarisen defilements and give up arisen defilements: the defilements of sensual desire, desire to be reborn, and ignorance. These are the things to which they do not apply the mind but should. 

Because\marginnote{7.1} of applying the mind to what they should not and not applying the mind to what they should, unarisen defilements arise and arisen defilements grow. 

This\marginnote{7.2} is how they apply the mind irrationally: ‘Did I exist in the past? Did I not exist in the past? What was I in the past? How was I in the past? After being what, what did I become in the past?\footnote{These are existential questions framed from a metaphysical perspective, i.e. they are based on the underlying assumption of a self. They are “irrational” because they avoid the question of cause: they only ask \emph{what} happens, not \emph{why} it happens. } Will I exist in the future? Will I not exist in the future? What will I be in the future? How will I be in the future? After being what, what will I become in the future?’ Or they are undecided about the present thus: ‘Am I? Am I not? What am I? How am I? This sentient being—where did it come from? And where will it go?’ 

When\marginnote{8.1} they apply the mind irrationally in this way, one of the following six views arises in them and is taken as a genuine fact.\footnote{To wonder is natural, but when we take speculations seriously they become dogmas, insisted on as the truth, though in reality we have no way of knowing. } The view: ‘My self survives.’\footnote{See also the discussions on the “self” at \href{https://suttacentral.net/sn44.10/en/sujato}{SN 44.10} and “gods” at \href{https://suttacentral.net/mn100/en/sujato\#42.4}{MN 100:42.4}. All three passages are phrased the same way. In each case the question, drawing on the doubts about the past and future as depicted above, is whether the self continues to exist (\textit{atthi}) or not (\textit{natthi}) after death. } The view: ‘My self does not survive.’ The view: ‘I perceive the self with the self.’\footnote{Compare \textsanskrit{Bṛhadāraṇyaka} \textsanskrit{Upaniṣad} 4.4.23, which, in a discussion of the hereafter, says that a sage who is tamed and stilled (\textit{\textsanskrit{samāhito}}) “sees the self in the self” (\textit{\textsanskrit{ātmanyevātmānaṁ} \textsanskrit{paśyati}}). Note the different verb here; unlike the \textsanskrit{Upaniṣad}, the sutta is not speaking of a sage who “sees” but of a theorist who “perceives”. } The view: ‘I perceive what is not-self with the self.’\footnote{The method of negation was employed by \textsanskrit{Yājñavalkya} to discard false, shallow views of the Self: “This Self is that which is not that, not that” (\textit{sa \textsanskrit{eṣa} neti \textsanskrit{netyātmā}}, \textsanskrit{Bṛhadāraṇyaka} \textsanskrit{Upaniṣad} 4.5.15, etc.). } The view: ‘I perceive the self with what is not-self.’\footnote{The theorist still relies on perception (\textit{\textsanskrit{saññā}}) and hence does not see the Self, which is pure \textit{\textsanskrit{viññāṇa}} cognized by \textit{\textsanskrit{viññāṇa}}: “Through what should one know the Knower?” (\textit{\textsanskrit{vijñātāramare} kena \textsanskrit{vijānīyāt}}, \textsanskrit{Bṛhadāraṇyaka} \textsanskrit{Upaniṣad} 2.4.14). } Or they have such a view: ‘This self of mine is he, the speaker, the knower who experiences the results of good and bad deeds in all the different realms. This self is permanent, everlasting, eternal, and imperishable, and will last forever and ever.’\footnote{This view, attributed to the mendicant \textsanskrit{Sāti} at \href{https://suttacentral.net/mn36/en/sujato\#5.11}{MN 36:5.11}, is reminiscent of \textsanskrit{Yājñavalkya}’s discussion of the Self as a person’s “light” in the \textsanskrit{Bṛhadāraṇyaka} \textsanskrit{Upaniṣad}. He illustrates the departure of the Self from the body at death by analogy with dreams, where a person sets aside the physical body and takes up a body of light (4.3.9). There (\textit{tatra}) he is the “agent” (\textit{\textsanskrit{kartā}}, 4.3.10) who “moves between the worlds contemplating and playing, as it were” (4.3.7). He creates his own experiences, seeing good and bad (\textit{\textsanskrit{dṛṣṭvaiva} \textsanskrit{puṇyaṁ} ca \textsanskrit{pāpaṁ} ca}, 4.3.15). Being “immortal” (\textit{\textsanskrit{amṛto}}, 4.3.12), he wanders where he likes and returns unaffected (4.3.16). In 4.4.5, he additionally says that “as one acts, so one becomes”, doing good becoming good, and doing bad becoming bad. See too Aitareya \textsanskrit{Upaniṣad} 3.1, which says the Self is that by which one sees, hears, smells, tastes, speaks and knows. | For the phrase \textit{\textsanskrit{sassatisamaṁ}} (“lasting forever and ever”), compare \textsanskrit{Bṛhadāraṇyaka} \textsanskrit{Upaniṣad} 5.10.1: “He reaches that world free of sorrow and snow, where he lives forever and ever” (\textit{sa lokam \textsanskrit{āgacchaty} \textsanskrit{aśokam} ahimam. tasmin vasati \textsanskrit{śāśvatīḥ} \textsanskrit{samāḥ}}). } This is called a misconception, the thicket of views, the desert of views, the twist of views, the dodge of views, the fetter of views.\footnote{“Twist” (\textit{\textsanskrit{visūka}}) is used with “dodge” (\textit{vipphandita}) and sometimes “duck” (\textit{visevita}) for an horse fighting the bit (\href{https://suttacentral.net/mn65/en/sujato\#33.2}{MN 65:33.2}) or a crab escaping the torment of children (\href{https://suttacentral.net/mn35/en/sujato\#23.9}{MN 35:23.9}). \textit{\textsanskrit{Visūka}} is also used for a “show” of dance, etc., where the commentarial gloss \textit{\textsanskrit{paṭāṇi}} (“screw”) reinforces the sense “twist, gyrate”. As descriptions of views, they suggest the active process of denial and distortion through which views shape how we see the world. } An unlearned ordinary person who is fettered by views is not freed from rebirth, old age, and death, from sorrow, lamentation, pain, sadness, and distress. They’re not freed from suffering, I say. 

But\marginnote{9.1} take a learned noble disciple who has seen the noble ones, and is skilled and trained in the teaching of the noble ones. They’ve seen true persons, and are skilled and trained in the teaching of the true persons. They understand to which things they should apply the mind and to which things they should not apply the mind. So they apply the mind to things they should and don’t apply the mind to things they shouldn’t. 

And\marginnote{10.1} what are the things to which they don’t apply the mind and should not? They are the things that, when the mind is applied to them, give rise to unarisen defilements and make arisen defilements grow: the defilements of sensual desire, desire to be reborn, and ignorance. These are the things to which they don’t apply the mind and should not. 

And\marginnote{10.6} what are the things to which they do apply the mind and should? They are the things that, when the mind is applied to them, do not give rise to unarisen defilements and give up arisen defilements: the defilements of sensual desire, desire to be reborn, and ignorance. These are the things to which they do apply the mind and should. 

Because\marginnote{10.11} of not applying the mind to what they should not and applying the mind to what they should, unarisen defilements don’t arise and arisen defilements are given up. 

They\marginnote{11.1} rationally apply the mind: ‘This is suffering’ … ‘This is the origin of suffering’ … ‘This is the cessation of suffering’ … ‘This is the practice that leads to the cessation of suffering’.\footnote{This is the realization of the four noble truths. The suttas distinguish between someone who accepts the truth of the Dhamma either by faith or by logic, and someone who truly sees with direct experience (\href{https://suttacentral.net/sn25.1/en/sujato}{SN 25.1}). This direct vision, here called “rational application of mind”, may be expressed any number of different ways. } And as they do so, they give up three fetters: substantialist view, doubt, and misapprehension of precepts and observances.\footnote{They are a stream-enterer who has entered the first of the four stages of awakening. “Seeing” the four noble truths permanently severs these three fetters. This is the decisive difference between “seeing” and knowing by faith or logic (\href{https://suttacentral.net/sn25.1/en/sujato}{SN 25.1}). } These are called the defilements that should be given up by seeing. 

\subsection*{2. Defilements Given Up by Restraint }

And\marginnote{12.1} what are the defilements that should be given up by restraint?\footnote{In the Gradual Training (eg. \href{https://suttacentral.net/mn38/en/sujato\#35.1}{MN 38:35.1}), sense restraint (\textit{\textsanskrit{saṁvarā}}) comes before seeing the four noble truths. The sequence in this sutta does not follow the order of practice; rather, it starts and ends with the most important items. } Take a mendicant who, reflecting rationally, lives restraining the faculty of the eye. For the distressing and feverish defilements that might arise in someone who lives without restraint of the eye faculty do not arise when there is such restraint.\footnote{Sense restraint is not about denial, but about freeing the mind from addiction. } Reflecting rationally, they live restraining the faculty of the ear … the nose … the tongue … the body … the mind. For the distressing and feverish defilements that might arise in someone who lives without restraint of the mind faculty do not arise when there is such restraint.\footnote{In the first and last items in this sutta, “giving up” defilements means their permanent eradication by means of the noble path. For the interim items, however, “giving up” refers to the more modest goal of practicing so that they do not arise in the mind. This creates the supportive conditions for deeper realizations, while giving the mind time for understanding to mature. } 

For\marginnote{12.10} the distressing and feverish defilements that might arise in someone who lives without restraint do not arise when there is such restraint. These are called the defilements that should be given up by restraint. 

\subsection*{3. Defilements Given Up by Using }

And\marginnote{13.1} what are the defilements that should be given up by using?\footnote{These are the four requisites used (\textit{\textsanskrit{paṭisevana}}) by a mendicant. They are provided to a mendicant at their ordination. Other possessions of a minor nature are also allowed, such as a razor, waistband, sandals, and so on. These passages are used as mindfulness reminders when making use of possessions. } Take a mendicant who, reflecting rationally, makes use of robes: ‘Only for the sake of warding off cold and heat; for warding off the touch of flies, mosquitoes, wind, sun, and reptiles; and for covering up the private parts.’ 

Reflecting\marginnote{14.1} rationally, they make use of almsfood: ‘Not for fun, indulgence, adornment, or decoration, but only to sustain this body, to avoid harm, and to support spiritual practice. In this way, I shall put an end to old discomfort and not give rise to new discomfort, and I will have the means to keep going, blamelessness, and a comfortable abiding.’\footnote{The next sutta (\href{https://suttacentral.net/mn3/en/sujato}{MN 3}) gives an example of this. | In several Chinese translations, “adornment” is applied to robes rather than almsfood, where it seems more fitting. | Compare the phrase \textit{\textsanskrit{yātrāmātraprasiddhyarthaṁ}} (“for accomplishing mere maintenance”) at \textsanskrit{Manusmṛti} 4.3. } 

Reflecting\marginnote{15.1} rationally, they make use of lodgings: ‘Only for the sake of warding off cold and heat; for warding off the touch of flies, mosquitoes, wind, sun, and reptiles; to shelter from harsh weather and to enjoy retreat.’ 

Reflecting\marginnote{16.1} rationally, they make use of medicines and supplies for the sick: ‘Only for the sake of warding off the pains of illness and to promote good health.’\footnote{While it may seem odd to use medicines for purposes other than treating illness, this is a broad category. It includes anything used as a tonic, pick-me-up, or refreshment so long as it is not solid food, such as fruit juice, honey, ghee, ginger, etc. } 

For\marginnote{17.1} the distressing and feverish defilements that might arise in someone who lives without using these things do not arise when they are used. These are called the defilements that should be given up by using. 

\subsection*{4. Defilements Given Up by Enduring }

And\marginnote{18.1} what are the defilements that should be given up by enduring?\footnote{The Buddha depicted the spiritual path as one of freedom and happiness. Nonetheless, practitioners will invariably encounter adversities along the way. Such things should be endured (\textit{\textsanskrit{adhivāsana}}) with patience and strength, without giving up. } Take a mendicant who, reflecting rationally, endures cold, heat, hunger, and thirst. They endure the touch of flies, mosquitoes, wind, sun, and reptiles. They endure rude and unwelcome criticism. And they put up with physical pain—sharp, severe, acute, unpleasant, disagreeable, and life-threatening. 

For\marginnote{18.3} the distressing and feverish defilements that might arise in someone who lives without enduring these things do not arise when they are endured. These are called the defilements that should be given up by enduring. 

\subsection*{5. Defilements Given Up by Avoiding }

And\marginnote{19.1} what are the defilements that should be given up by avoiding?\footnote{This item shows that endurance is not always the right response to adversity. Some things are best avoided (\textit{parivajjana}) where possible. This is different, of course, from a strategy or compulsion of avoiding things that are uncomfortable. } Take a mendicant who, reflecting rationally, avoids a wild elephant, a wild horse, a wild ox, a wild dog, a snake, a stump, thorny ground, a pit, a cliff, a swamp, and a sewer. Reflecting rationally, they avoid sitting on inappropriate seats, walking in inappropriate neighborhoods, and mixing with bad friends—whatever sensible spiritual companions would believe to be a bad setting.\footnote{This introduces the hard-to-translate idea of \textit{gocara}, literally “pasture”. It refers to the places or people to whom a mendicant “resorts”, especially when on alms. What is appropriate depends on context. For example, whereas there is no issue with accepting a meal from a sex worker (\href{https://suttacentral.net/dn16/en/sujato\#2.14.7}{DN 16:2.14.7}), it would provoke suspicion if a monk entered a brothel for the meal. | \textit{Okappeti} means “believe, trust” rather than “suspect”. } 

For\marginnote{19.4} the distressing and feverish defilements that might arise in someone who lives without avoiding these things do not arise when they are avoided. These are called the defilements that should be given up by avoiding. 

\subsection*{6. Defilements Given Up by Dispelling }

And\marginnote{20.1} what are the defilements that should be given up by dispelling?\footnote{This section shows that, rather than being “non-judgmental” about ones’ thoughts, a meditator should recognize and “dispel” (\textit{vinodana}) those that are harmful. The first step, however, in dispelling harmful thoughts is to recognize that they are harmful, which requires a degree of mindfulness and equanimity. Often that is sufficient: once one is mindful of the bad thought, it undercuts the greed, hate, and delusion that fuels it and it fades away. In cases where the mind is too caught up in the harmful thinking, a more deliberate practice can be required (\href{https://suttacentral.net/mn20/en/sujato}{MN 20}). } Take a mendicant who, reflecting rationally, doesn’t tolerate a sensual, malicious, or cruel thought that has arisen, but gives it up, gets rid of it, eliminates it, and obliterates it. They don’t tolerate any bad, unskillful qualities that have arisen, but give them up, get rid of them, eliminate them, and obliterate them. 

For\marginnote{20.3} the distressing and feverish defilements that might arise in someone who lives without dispelling these things do not arise when they are dispelled. These are called the defilements that should be given up by dispelling. 

\subsection*{7. Defilements Given Up by Developing }

And\marginnote{21.1} what are the defilements that should be given up by developing?\footnote{To develop (\textit{\textsanskrit{bhāvanā}}) is literally to “make be more”, to “grow” or “amplify”. The good factors that are already present, especially in the stream-enterer, are cultivated to support the realization of full awakening. } It’s when a mendicant, reflecting rationally, develops the awakening factors of mindfulness,\footnote{The seven awakening factors especially emphasize the emotional and holistic dimension of meditative growth. } investigation of principles, energy, rapture, tranquility, immersion, and equanimity, which rely on seclusion, fading away, and cessation, and ripen as letting go.\footnote{The four terms here—seclusion, fading away, cessation, ripening as letting go—are commonly applied to the different formulations of the path, but especially the seven awakening factors. Each expresses a fundamental quality of the path. They can be understood as a process of deepening that moves towards \textsanskrit{Nibbāna}, the ultimate letting go, while each of the four is also a term for \textsanskrit{Nibbāna} itself. } 

For\marginnote{21.9} the distressing and feverish defilements that might arise in someone who lives without developing these things do not arise when they are developed. These are called the defilements that should be given up by developing. 

Now,\marginnote{22.1} take a mendicant who, by seeing, has given up the defilements that should be given up by seeing. By restraint, they’ve given up the defilements that should be given up by restraint. By using, they’ve given up the defilements that should be given up by using. By enduring, they’ve given up the defilements that should be given up by enduring. By avoiding, they’ve given up the defilements that should be given up by avoiding. By dispelling, they’ve given up the defilements that should be given up by dispelling. By developing, they’ve given up the defilements that should be given up by developing.\footnote{Spiritual teachings sometimes emphasize the critical role of a single practice to overcome different defilements, for example through chanting a mantra or by mindful awareness: many problems, one tool. This sutta, in line with the early texts generally, takes the opposing line, that the diversity of defilements requires a diversity of practices in response: many problems, many tools. } They’re called a mendicant who lives having restrained all defilements, who has cut off craving, untied the fetters, and by rightly comprehending conceit has made an end of suffering.”\footnote{This is the arahant. | The “fetters” are enumerated as ten at eg. \href{https://suttacentral.net/an10.13/en/sujato}{AN 10.13}. | “Conceit”, which is one of the fetters, is the tendency of the mind to judge and assess in terms of oneself through the process of “conceiving” discussed in \href{https://suttacentral.net/mn1/en/sujato}{MN 1}. | They have “made an end of suffering” in the sense that they have cut off the root of transmigration. Nonetheless while an arahant is living they still experience suffering such as physical sickness. } 

That\marginnote{22.3} is what the Buddha said. Satisfied, the mendicants approved what the Buddha said. 

%
\section*{{\suttatitleacronym MN 3}{\suttatitletranslation Heirs in the Teaching }{\suttatitleroot Dhammadāyādasutta}}
\addcontentsline{toc}{section}{\tocacronym{MN 3} \toctranslation{Heirs in the Teaching } \tocroot{Dhammadāyādasutta}}
\markboth{Heirs in the Teaching }{Dhammadāyādasutta}
\extramarks{MN 3}{MN 3}

\scevam{So\marginnote{1.1} I have heard. }At one time the Buddha was staying near \textsanskrit{Sāvatthī} in Jeta’s Grove, \textsanskrit{Anāthapiṇḍika}’s monastery. There the Buddha addressed the mendicants, “Mendicants!” 

“Venerable\marginnote{1.5} sir,” they replied. The Buddha said this: 

“Mendicants,\marginnote{2.1} be my heirs in the teaching, not in things of the flesh.\footnote{“Things of the flesh” renders \textit{\textsanskrit{āmisa}}, literally “meat”. In Vedic sacrifice the flesh of the slaughtered beast was made holy and fit for the gods (Rig Veda 1.162.10). In this way the guilt of killing was assuaged and the flesh became allowable for the brahmin priests. Here the “flesh” is extended by implication to the material pleasures of the world. } Out of sympathy for you, I think, ‘How can my disciples become heirs in the teaching, not in things of the flesh?’ 

If\marginnote{2.4} you become heirs in things of the flesh, not in the teaching, that will make you liable to the accusation:\footnote{Text’s \textit{\textsanskrit{ādiya}} (and variant \textit{\textsanskrit{ādissā}}) are future passive participles from √\textit{dis} (“point”) having the sense “liable to be pointed out or accused”. } ‘The Teacher’s disciples live as heirs in things of the flesh, not in the teaching.’ And it will make me liable to the accusation: ‘The Teacher’s disciples live as heirs in things of the flesh, not in the teaching.’ 

If\marginnote{2.8} you become heirs in the teaching, not in things of the flesh, that will make you not liable to the accusation: ‘The Teacher’s disciples live as heirs in the teaching, not in things of the flesh.’ And it will make me not liable to the accusation: ‘The Teacher’s disciples live as heirs in the teaching, not in things of the flesh.’ 

So,\marginnote{2.12} mendicants, be my heirs in the teaching, not in things of the flesh. Out of sympathy for you, I think, ‘How can my disciples become heirs in the teaching, not in things of the flesh?’ 

Suppose\marginnote{3.1} that I had eaten and refused more food, being full, and having had as much as I needed.\footnote{A mendicant tries to eat just what they need. In some circumstances, especially when eating on invitation in a house, the donors will first place a modest amount of food in the bowl, then offer more during the meal if it is needed. This passage concerns such extra food that has been refused. } And there was some extra almsfood that was going to be thrown away. Then two mendicants were to come who were weak with hunger. I’d say to them, ‘Mendicants, I have eaten and refused more food, being full, and having had as much as I need. And there is this extra almsfood that’s going to be thrown away. Eat it if you like. Otherwise I’ll throw it out where there is little that grows, or drop it into water that has no living creatures.’ 

Then\marginnote{3.8} one of those mendicants thought, ‘The Buddha has eaten and refused more food. And he has some extra almsfood that’s going to be thrown away. If we don’t eat it he’ll throw it away. But the Buddha has also said: “Be my heirs in the teaching, not in things of the flesh.” And almsfood is one of the things of the flesh. Instead of eating this almsfood, why don’t I spend this day and night weak with hunger?’ And that’s what they did. 

Then\marginnote{3.17} the second of those mendicants thought, ‘The Buddha has eaten and refused more food. And he has some extra almsfood that’s going to be thrown away. If we don’t eat it he’ll throw it away. Why don’t I eat this almsfood, then spend the day and night having got rid of my hunger and weakness?’ And that’s what they did. 

Even\marginnote{3.23} though that mendicant, after eating the almsfood, spent the day and night rid of hunger and weakness, it is the former mendicant who is more worthy of respect and praise. Why is that? Because for a long time that will conduce to that mendicant being of few wishes, content, self-effacing, unburdensome, and energetic.\footnote{This is an example of the “defilements given up by using” (\href{https://suttacentral.net/mn2/en/sujato\#14.2}{MN 2:14.2}). It makes a point: short term discomfort is outweighed by the benefits of overcoming greed. Note that the Buddha encouraged mendicants to eat regularly, and did not support fasting or other extreme diets. } 

So,\marginnote{3.26} mendicants, be my heirs in the teaching, not in things of the flesh. Out of sympathy for you, I think, ‘How can my disciples become heirs in the teaching, not in things of the flesh?’” 

That\marginnote{4.1} is what the Buddha said. When he had spoken, the Holy One got up from his seat and entered his dwelling. 

Then\marginnote{4.3} soon after the Buddha left, Venerable \textsanskrit{Sāriputta} said to the mendicants,\footnote{This marks the first time in the \textsanskrit{Majjhimanikāya} that we hear a teaching from a disciple. \textsanskrit{Sāriputta} was the Buddha’s leading disciple, famed for his wisdom. He gave many teachings in the suttas, typically, as here, preferring systematic expositions. } “Reverends, mendicants!” 

“Reverend,”\marginnote{4.5} they replied. \textsanskrit{Sāriputta} said this: 

“Reverends,\marginnote{5.1} how do the disciples of a Teacher who lives in seclusion not train in seclusion? And how do they train in seclusion?”\footnote{\textsanskrit{Sāriputta} continues on a similar theme, phrased in terms of “seclusion” (\textit{viveka}) rather than “things of the flesh” (\textit{\textsanskrit{āmisa}}). Seclusion is explained in the commentaries as threefold: physical seclusion achieved by living a virtuous life in a quiet place; mental seclusion from the hindrances achieved through the \textit{\textsanskrit{jhānas}}; and seclusion from attachments, namely \textsanskrit{Nibbāna}. } 

“Reverend,\marginnote{5.2} we would travel a long way to learn the meaning of this statement in the presence of Venerable \textsanskrit{Sāriputta}. May Venerable \textsanskrit{Sāriputta} himself please clarify the meaning of this. The mendicants will listen and remember it.” 

“Well\marginnote{5.5} then, reverends, listen and apply your mind well, I will speak.” 

“Yes,\marginnote{5.6} reverend,” they replied. \textsanskrit{Sāriputta} said this: 

“Reverends,\marginnote{6.1} how do the disciples of a Teacher who lives in seclusion not train in seclusion? The disciples of a teacher who lives in seclusion do not train in seclusion. They don’t give up what the Teacher tells them to give up. They’re indulgent and slack, leaders in backsliding, neglecting seclusion. In this case, the senior mendicants should be criticized on three grounds.\footnote{“Senior mendicants” (\textit{\textsanskrit{therā} \textsanskrit{bhikkhū}}) have been ordained for more than ten rains retreats, i.e. ten years. } ‘The disciples of a teacher who lives in seclusion do not train in seclusion.’ This is the first ground. ‘They don’t give up what the Teacher tells them to give up.’ This is the second ground. ‘They’re indulgent and slack, leaders in backsliding, neglecting seclusion.’ This is the third ground. The senior mendicants should be criticized on these three grounds. In this case, the middle mendicants\footnote{“Middle mendicants” (\textit{\textsanskrit{majjhimā} \textsanskrit{bhikkhū}}) have between five and ten rains. } and the junior mendicants should be criticized on the same three grounds.\footnote{“Junior mendicants” (\textit{\textsanskrit{navā} \textsanskrit{bhikkhū}}) have less than five rains. } This is how the disciples of a Teacher who lives in seclusion do not train in seclusion. 

And\marginnote{7.1} how do the disciples of a teacher who lives in seclusion train in seclusion? The disciples of a teacher who lives in seclusion train in seclusion. They give up what the Teacher tells them to give up. They’re not indulgent and slack, leaders in backsliding, neglecting seclusion. In this case, the senior mendicants should be praised on three grounds. ‘The disciples of a teacher who lives in seclusion train in seclusion.’ This is the first ground. ‘They give up what the Teacher tells them to give up.’ This is the second ground. ‘They’re not indulgent and slack, leaders in backsliding, neglecting seclusion.’ This is the third ground. The senior mendicants should be praised on these three grounds. In this case, the middle mendicants and the junior mendicants should be praised on the same three grounds. This is how the disciples of a Teacher who lives in seclusion train in seclusion. 

The\marginnote{8.1} bad thing here is greed and hate. There is a middle way of practice for giving up greed and hate. It gives vision and knowledge, and leads to peace, direct knowledge, awakening, and extinguishment.\footnote{\textsanskrit{Sāriputta} is adapting a passage from the Buddha’s first sermon, the Dhammacakkappavattanasutta (\href{https://suttacentral.net/sn56.11/en/sujato\#2.4}{SN 56.11:2.4}). } And what is that middle way of practice? It is simply this noble eightfold path, that is:\footnote{The noble eightfold path is the fourth of the four noble truths. It is defined in detail at \href{https://suttacentral.net/mn141/en/sujato\#23.2}{MN 141:23.2}. } right view, right thought, right speech, right action, right livelihood, right effort, right mindfulness, and right immersion. This is that middle way of practice, which gives vision and knowledge, and leads to peace, direct knowledge, awakening, and extinguishment. 

The\marginnote{9{-}15.1} bad thing here is anger and acrimony. … disdain and contempt … jealousy and stinginess … deceit and deviousness … obstinacy and aggression … conceit and arrogance … vanity and negligence. There is a middle way of practice for giving up vanity and negligence. It gives vision and knowledge, and leads to peace, direct knowledge, awakening, and extinguishment. And what is that middle way of practice? It is simply this noble eightfold path, that is: right view, right thought, right speech, right action, right livelihood, right effort, right mindfulness, and right immersion. This is that middle way of practice, which gives vision and knowledge, and leads to peace, direct knowledge, awakening, and extinguishment.” 

This\marginnote{9{-}15.13} is what Venerable \textsanskrit{Sāriputta} said. Satisfied, the mendicants approved what \textsanskrit{Sāriputta} said. 

%
\section*{{\suttatitleacronym MN 4}{\suttatitletranslation Fear and Dread }{\suttatitleroot Bhayabheravasutta}}
\addcontentsline{toc}{section}{\tocacronym{MN 4} \toctranslation{Fear and Dread } \tocroot{Bhayabheravasutta}}
\markboth{Fear and Dread }{Bhayabheravasutta}
\extramarks{MN 4}{MN 4}

\scevam{So\marginnote{1.1} I have heard. }At one time the Buddha was staying near \textsanskrit{Sāvatthī} in Jeta’s Grove, \textsanskrit{Anāthapiṇḍika}’s monastery. 

Then\marginnote{2.1} the brahmin \textsanskrit{Jānussoṇi} went up to the Buddha, and exchanged greetings with him.\footnote{We meet \textsanskrit{Jānussoṇi} many times in the suttas, where he converses with the Buddha on a number of topics, with a special interest in the afterlife. His name appears in Sanskrit as \textsanskrit{Jānaśruti} or \textsanskrit{Jānaśruteya}, which are patronymics of \textsanskrit{Janaśruti}, “famed among the people”, with the element \textit{\textsanskrit{soṇi}} coming from the Vedic \textit{\textsanskrit{ṣvaṇi}} (“sound”). The Chinese rendering \langlzh{生聞} (“Born Famous”, T 125.2.665b18) evidently assumes the same derivation (\textit{jana} “people” is from √\textit{jan} “born”). The Pali commentary says it is a title awarded the family priest (\textit{purohita}) of Kosala, which would make him one of the most powerful brahmins alive. However, I can find no confirmation of such a title, whereas the use of \textsanskrit{Jānaśruti} as a patronymic by the descendants of \textsanskrit{Janaśruti} is well attested. } When the greetings and polite conversation were over, he sat down to one side and said to the Buddha: 

“Mister\marginnote{2.3} Gotama, those gentlemen who have gone forth out of faith from the lay life to homelessness out of faith in you have Mister Gotama to lead the way, help them out, and give them encouragement.\footnote{This statement is sometimes translated as a question, but it lacks the usual question markers. The parallel at \href{https://suttacentral.net/ea31.1/lzh/taisho}{EA 31.1} reverses the sequence: \textsanskrit{Jānussoṇi} first says that living in the forest is hard, then introduces the idea that the Buddha is their guide, following which the Buddha tells his life story. This makes for a better flow of ideas. } And those people follow Mister Gotama’s example.”\footnote{Picking up from the previous sutta the themes of following the Buddha’s example and of seclusion. | In \textit{\textsanskrit{diṭṭhānugatiṁ}}, read \textit{\textsanskrit{diṭṭha}} (“what has been seen”, i.e. “example”), rather than \textit{\textsanskrit{diṭṭhi}} (“view”) per commentary. } 

“That’s\marginnote{2.5} so true, brahmin! Everything you say is true, brahmin!” 

“But\marginnote{2.8} Mister Gotama, remote lodgings in the wilderness and the forest are challenging. It’s hard to maintain seclusion and hard to find joy in solitude. The forests seem to rob the mind of a mendicant who isn’t immersed in \textsanskrit{samādhi}.” 

“That’s\marginnote{2.10} so true, brahmin! Everything you say is true, brahmin! 

Before\marginnote{3.1} my awakening—when I was still unawakened but intent on awakening—I too thought,\footnote{This is the first of several times in the \textsanskrit{Majjhimanikāya} that the Buddha spoke of his meditation before enlightenment (see eg. \href{https://suttacentral.net/mn19/en/sujato}{MN 19}, \href{https://suttacentral.net/mn128/en/sujato}{MN 128}). In these texts the Buddha takes pains to depict himself as an ordinary person who struggled with the same issues in meditation as anyone else. | “Intent on awakening” is \textit{bodhisatta} (Sanskrit \textit{bodhisattva}). Such passages as this are apparently the original context for this term. The Pali \textit{satta} represents a number of different words in Sanskrit, including \textit{sajjita} (“intent on, devoted to, fixed upon”), which makes better sense here than \textit{sattva} (“being”). } ‘Remote lodgings in the wilderness and the forest are challenging. It’s hard to maintain seclusion and hard to find joy in solitude. The forests seem to rob the mind of a mendicant who isn’t immersed in \textsanskrit{samādhi}.’ 

Then\marginnote{4.1} I thought, ‘There are ascetics and brahmins with unpurified conduct of body, speech, and mind who frequent remote lodgings in the wilderness and the forest. Those ascetics and brahmins summon unskillful fear and dread because of these defects in their conduct.\footnote{That is, their guilt will manifest as fear. Throughout this sutta, the Buddha emphasizes that fear is overcome by the proper development of the path, which leads to confidence and strength of mind. This is in implicit contrast to the notion that one overcomes fear by either brute endurance and force of will or by magical spells and charms. } But I don’t frequent remote lodgings in the wilderness and the forest with unpurified conduct of body, speech, and mind. My conduct is purified. I am one of those noble ones who frequent remote lodgings in the wilderness and the forest with purified conduct of body, speech, and mind.’ Seeing this purity of conduct in myself I felt even more unruffled about staying in the forest.\footnote{“Unruffled” is \textit{\textsanskrit{pallomamāpādiṁ}}. } 

Then\marginnote{5.1} I thought, ‘There are ascetics and brahmins with unpurified livelihood who frequent remote lodgings in the wilderness and the forest. Those ascetics and brahmins summon unskillful fear and dread because of these defects in their livelihood. But I don’t frequent remote lodgings in the wilderness and the forest with unpurified livelihood. My livelihood is purified. I am one of those noble ones who frequent remote lodgings in the wilderness and the forest with purified livelihood.’\footnote{The purification of livelihood for an ascetic is spelled out in detail in \href{https://suttacentral.net/dn2/en/sujato\#56.1}{DN 2:56.1} as part of the Gradual Training of a Buddhist mendicant. The current sutta can, in fact, be read as a variant presentation of the Gradual Training. } Seeing this purity of livelihood in myself I felt even more unruffled about staying in the forest. 

Then\marginnote{8.1} I thought, ‘There are ascetics and brahmins full of desire for sensual pleasures, with acute lust … I am not full of desire …’ 

‘There\marginnote{9.1} are ascetics and brahmins full of ill will, with malicious intentions … I have a heart full of love …’ 

‘There\marginnote{10.1} are ascetics and brahmins overcome with dullness and drowsiness … I am free of dullness and drowsiness …’ 

‘There\marginnote{11.1} are ascetics and brahmins who are restless, with no peace of mind … My mind is peaceful …’ 

‘There\marginnote{12.1} are ascetics and brahmins who are doubting and uncertain … I’ve gone beyond doubt …’ 

‘There\marginnote{13.1} are ascetics and brahmins who glorify themselves and put others down … I don’t glorify myself and put others down …’ 

‘There\marginnote{14.1} are ascetics and brahmins who are cowardly and craven … I don’t get startled …’ 

‘There\marginnote{15.1} are ascetics and brahmins who enjoy possessions, honor, and popularity … I have few wishes …’ 

‘There\marginnote{16.1} are ascetics and brahmins who are lazy and lack energy … I am energetic …’ 

‘There\marginnote{17.1} are ascetics and brahmins who are unmindful and lack situational awareness … I am mindful …’ 

‘There\marginnote{18.1} are ascetics and brahmins who lack immersion, with straying minds …\footnote{“Straying minds” is \textit{\textsanskrit{vibbhantacittā}}. The verbal form \textit{vibhamati} is used of a monastic who disrobes. } I am accomplished in immersion …’ 

‘There\marginnote{19.1} are ascetics and brahmins who are witless and idiotic who frequent remote lodgings in the wilderness and the forest. Those ascetics and brahmins summon unskillful fear and dread because of the defects of witlessness and stupidity. But I don’t frequent remote lodgings in the wilderness and the forest witless and idiotic. I am accomplished in wisdom. I am one of those noble ones who frequent remote lodgings in the wilderness and the forest accomplished in wisdom.’ Seeing this accomplishment of wisdom in myself I felt even more unruffled about staying in the forest. 

Then\marginnote{20.1} I thought, ‘There are certain nights that are recognized as specially portentous: the fourteenth, fifteenth, and eighth of the fortnight.\footnote{These are the days of \textit{uposatha} (“sabbath”) marking the lunar cycle. } On such nights, why don’t I stay in awe-inspiring and hair-raising shrines in parks, forests, and trees? In such lodgings, hopefully I might see that fear and dread.’ Some time later, that’s what I did. As I was staying there a deer came by, or a peacock snapped a twig, or the wind rustled the leaves. Then I thought, ‘Is this that fear and dread coming?’\footnote{The dual feelings of love for wilderness and fear of her (possibly imagined) dangers echo the hymn to \textsanskrit{Araṇyānī}, the wilderness as goddess, at Rig Veda 10.146. } Then I thought, ‘Why do I always meditate expecting that fear to come? Why don’t I get rid of that fear and dread just as it comes, while remaining just as I am?’ Then that fear and dread came upon me as I was walking. I didn’t stand still or sit down or lie down until I had got rid of that fear and dread while walking. Then that fear and dread came upon me as I was standing. I didn’t walk or sit down or lie down until I had got rid of that fear and dread while standing. Then that fear and dread came upon me as I was sitting. I didn’t lie down or stand still or walk until I had got rid of that fear and dread while sitting. Then that fear and dread came upon me as I was lying down. I didn’t sit up or stand still or walk until I had got rid of that fear and dread while lying down. 

There\marginnote{21.1} are some ascetics and brahmins who perceive that it’s day when in fact it’s night, or perceive that it’s night when in fact it’s day.\footnote{The Buddha illustrates his claim that the forests can rob the mind of a meditator without \textit{\textsanskrit{samādhi}}. During the development of meditation, perception can change in strange and unsettling ways, making the meditator question the nature of reality. In some cases, over-exertion in meditation can indeed lead to mental breakdown. } This meditation of theirs is delusional, I say. I perceive that it’s night when in fact it is night, and perceive that it’s day when in fact it is day.\footnote{This alludes to the oft-repeated phrase “as by day, so by night; as by night, so by day”, which is associated with the meditative perception of light (\href{https://suttacentral.net/an4.41/en/sujato\#3.3}{AN 4.41:3.3}), the development of the bases of psychic power (\href{https://suttacentral.net/sn51.11/en/sujato\#1.10}{SN 51.11:1.10}), and the dispelling of drowsiness (\href{https://suttacentral.net/an7.61/en/sujato\#7.2}{AN 7.61:7.2}). The point of that phrase is that the mind has been developed until it is full of light regardless of whether it is light or dark outside. Here, the Buddha forestalls the accusation that this is deluded: he knows perfectly well what is day and what is night. } And if there’s anyone of whom it may be rightly said that a being not liable to delusion has arisen in the world for the welfare and happiness of the people, out of sympathy for the world, for the benefit, welfare, and happiness of gods and humans, it’s of me that this should be said. 

My\marginnote{22{-}26.1} energy was roused up and unflagging, my mindfulness was established and lucid, my body was tranquil and undisturbed, and my mind was immersed in \textsanskrit{samādhi}.\footnote{To this point, the sutta has been focusing on the foundations of meditation from the special perspective of overcoming fear. Now that this has been achieved, it returns to the standard presentation of immersion and then wisdom. Taken together, then, it is the first presentation of the Gradual Training in the Majjhima \textsanskrit{Nikāya}. } Quite secluded from sensual pleasures, secluded from unskillful qualities, I entered and remained in the first absorption, which has the rapture and bliss born of seclusion, while placing the mind and keeping it connected.\footnote{\textit{\textsanskrit{Jhāna}} is a state of “elevated consciousness” (\textit{adhicitta}), so all the terms have an elevated sense. The plural form indicates that “sensual pleasures” includes sense experience, which the meditator can turn away from since they no longer have any desire for it. The “unskillful qualities” are the five hindrances. The “rapture and bliss born of seclusion” is the happiness of abandoning the hindrances and freedom from sense impingement. “Placing the mind and keeping it connected” (\textit{vitakka}, \textit{\textsanskrit{vicāra}}) uses terms that mean “thought” in coarse consciousness, but which in “elevated consciousness” refer to the subtle function of applying the mind to the meditation. } As the placing of the mind and keeping it connected were stilled, I entered and remained in the second absorption, which has the rapture and bliss born of immersion, with internal clarity and mind at one, without placing the mind and keeping it connected.\footnote{Each \textit{\textsanskrit{jhāna}} begins as the least refined aspect of the previous \textit{\textsanskrit{jhāna}} ends. This is not consciously directed, but describes the natural process of settling. The meditator is now fully confident and no longer needs to apply their mind: it is simply still and fully unified. } And with the fading away of rapture, I entered and remained in the third absorption, where I meditated with equanimity, mindful and aware, personally experiencing the bliss of which the noble ones declare, ‘Equanimous and mindful, one meditates in bliss.’\footnote{The emotional response to bliss matures from the subtle thrill of rapture to the poise of equanimity. Mindfulness is present in all states of deep meditation, but with equanimity it becomes prominent. } With the giving up of pleasure and pain, and the ending of former happiness and sadness, I entered and remained in the fourth absorption, without pleasure or pain, with pure equanimity and mindfulness.\footnote{The emotional poise of equanimity leads to the feeling of pleasure settling into the more subtle neutral feeling. Pain and sadness have been abandoned long before, but are emphasized here as they are subtle counterpart of pleasure. } 

When\marginnote{27.1} my mind had become immersed in \textsanskrit{samādhi} like this—purified, bright, flawless, rid of corruptions, pliable, workable, steady, and imperturbable—I extended it toward recollection of past lives. I recollected many kinds of past lives.\footnote{The equanimity of the fourth \textit{\textsanskrit{jhāna}} is not dullness and indifference, but a brilliant and radiant awareness. The fourth \textit{\textsanskrit{jhāna}} is the ideal basis for developing higher knowledges, although elsewhere the canon shows that even the first \textit{\textsanskrit{jhāna}} can be a basis for liberating insight. Without \textit{\textsanskrit{jhāna}}, however, the eightfold path is incomplete and liberating insight is impossible. | The verb \textit{\textsanskrit{abhininnāmeti}} (“extend”) indicates that the meditator comes out of full immersion like a tortoise sticking out its limbs (\href{https://suttacentral.net/sn35.240/en/sujato\#1.7}{SN 35.240:1.7}). } That is: one, two, three, four, five, ten, twenty, thirty, forty, fifty, a hundred, a thousand, a hundred thousand rebirths; many eons of the world contracting, many eons of the world expanding, many eons of the world contracting and expanding. I remembered: ‘There, I was named this, my clan was that, I looked like this, and that was my food. This was how I felt pleasure and pain, and that was how my life ended. When I passed away from that place I was reborn somewhere else. There, too, I was named this, my clan was that, I looked like this, and that was my food. This was how I felt pleasure and pain, and that was how my life ended. When I passed away from that place I was reborn here.’ And so I recollected my many kinds of past lives, with features and details.\footnote{Empowered by the fourth \textit{\textsanskrit{jhāna}}, memory breaks through the veil of birth and death, revealing the vast expanse of time and dispelling the illusion that there is any place of eternal rest or sanctuary in the cycle of transmigration. The knowledge of these events is not hazy or murky, but clear and precise, illuminated by the brilliance of purified consciousness. } 

This\marginnote{28.1} was the first knowledge, which I achieved in the first watch of the night. Ignorance was destroyed and knowledge arose; darkness was destroyed and light arose, as happens for a meditator who is diligent, keen, and resolute. 

When\marginnote{29.1} my mind had become immersed in \textsanskrit{samādhi} like this—purified, bright, flawless, rid of corruptions, pliable, workable, steady, and imperturbable—I extended it toward knowledge of the death and rebirth of sentient beings. With clairvoyance that is purified and superhuman, I saw sentient beings passing away and being reborn—inferior and superior, beautiful and ugly, in a good place or a bad place. I understood how sentient beings are reborn according to their deeds: ‘These dear beings did bad things by way of body, speech, and mind. They denounced the noble ones; they had wrong view; and they chose to act out of that wrong view. When their body breaks up, after death, they’re reborn in a place of loss, a bad place, the underworld, hell. These dear beings, however, did good things by way of body, speech, and mind. They never denounced the noble ones; they had right view; and they chose to act out of that right view. When their body breaks up, after death, they’re reborn in a good place, a heavenly realm.’ And so, with clairvoyance that is purified and superhuman, I saw sentient beings passing away and being reborn—inferior and superior, beautiful and ugly, in a good place or a bad place. I understood how sentient beings are reborn according to their deeds.\footnote{Here knowledge extends to the rebirths of others as well as oneself. Even more significant, it brings in the understanding of cause and effect; \emph{why} rebirth happens the way it does. Such knowledge, however, is not infallible, as the Buddha warns in \href{https://suttacentral.net/dn1/en/sujato\#2.5.3}{DN 1:2.5.3} and \href{https://suttacentral.net/mn136/en/sujato}{MN 136}. The experience is one thing; the inferences drawn from it are another. One should draw conclusions only tentatively, after long experience. | “Clairvoyance” renders \textit{dibbacakkhu} (“celestial eye”), for which see \textsanskrit{Chāndogya} \textsanskrit{Upaniṣad} 8.12.5, which says of the self that “the mind is its celestial eye” (\textit{mano’sya \textsanskrit{daivaṁ} \textsanskrit{cakṣuḥ}}). } 

This\marginnote{30.1} was the second knowledge, which I achieved in the middle watch of the night. Ignorance was destroyed and knowledge arose; darkness was destroyed and light arose, as happens for a meditator who is diligent, keen, and resolute. 

When\marginnote{31.1} my mind had become immersed in \textsanskrit{samādhi} like this—purified, bright, flawless, rid of corruptions, pliable, workable, steady, and imperturbable—I extended it toward knowledge of the ending of defilements.\footnote{This is the experience of awakening that is the true goal of the Buddhist path. The defilements—properties of the mind that create suffering—have been curbed by the practice of ethics and suppressed by the power of \textit{\textsanskrit{jhāna}}. Here they are eliminated forever. } I truly understood: ‘This is suffering’ … ‘This is the origin of suffering’ … ‘This is the cessation of suffering’ … ‘This is the practice that leads to the cessation of suffering’.\footnote{These are the four noble truths, which form the main content of the Buddha’s first sermon. They are the overarching principle into which all other teachings fall. As established in \href{https://suttacentral.net/mn2/en/sujato}{MN 2}, seeing the four noble truths indicates stream-entry, while the ending of all defilements indicates arahantship. } I truly understood: ‘These are defilements’ … ‘This is the origin of defilements’ … ‘This is the cessation of defilements’ … ‘This is the practice that leads to the cessation of defilements’.\footnote{The application of the four noble truths to defilements indicates that this is the final stage of awakening, perfection (or “arahantship”, \textit{arahatta}). | Many translators use “defilement” to render \textit{kilesa}, but since \textit{kilesa} appears only rarely in the early texts, I use “defilement” for \textit{\textsanskrit{āsava}}. Both terms refer to a stain, corruption, or pollution in the mind. } 

Knowing\marginnote{32.1} and seeing like this, my mind was freed from the defilements of sensuality, desire to be reborn, and ignorance.\footnote{\textit{\textsanskrit{Bhavāsava}} is the defilement that craves to continue life in a new birth. } When it was freed, I knew it was freed.\footnote{This is a reflective awareness of the fact of awakening. The meditator reviews their mind and sees that it is free from all forces that lead to suffering. } 

I\marginnote{32.3} understood: ‘Rebirth is ended, the spiritual journey has been completed, what had to be done has been done, there is nothing further for this place.’”\footnote{This is a standard declaration of full awakening in the suttas, said both of the Buddha and of any arahant (“perfected one”). Each of the four phrases illustrates a cardinal principle of awakening. (1) Further transmigration through rebirths has come to an end due to the exhaustion (\textit{\textsanskrit{khīṇa}}) of that which propels rebirth, namely deeds motivated by craving. (2) The eightfold path has been developed fully in all respects. (3) All functions relating to the four noble truths have been completed, namely: understanding suffering, letting go craving, witnessing extinguishment, and developing the path. (4) Extinguishment is final, with no falling back to this or any other state of existence. } 

This\marginnote{33.1} was the third knowledge, which I achieved in the final watch of the night. Ignorance was destroyed and knowledge arose; darkness was destroyed and light arose, as happens for a meditator who is diligent, keen, and resolute. 

Brahmin,\marginnote{34.1} you might think: ‘Perhaps the ascetic Gotama is not free of greed, hate, and delusion even today, and that is why he still frequents remote lodgings in the wilderness and the forest.’ But you should not see it like this. I see two reasons to frequent remote lodgings in the wilderness and the forest. I see happiness for myself in this life, and I have sympathy for future generations.”\footnote{Also at \href{https://suttacentral.net/an2.30/en/sujato\#1.1}{AN 2.30:1.1}. The Buddha confirms \textsanskrit{Jānussoṇi}’s belief that he acts to set a good example for others. } 

“Indeed,\marginnote{35.1} Mister Gotama has sympathy for future generations, since he is a perfected one, a fully awakened Buddha. Excellent, Mister Gotama! Excellent, Mister Gotama! As if he were righting the overturned, or revealing the hidden, or pointing out the path to the lost, or lighting a lamp in the dark so people with clear eyes can see what’s there, Mister Gotama has made the teaching clear in many ways. I go for refuge to Mister Gotama, to the teaching, and to the mendicant \textsanskrit{Saṅgha}.\footnote{\textsanskrit{Jānussoṇi} goes for refuge so often in the suttas that it is almost a genre unto itself (\href{https://suttacentral.net/mn27/en/sujato}{MN 27}, \href{https://suttacentral.net/sn12/en/sujato\#47}{SN 12:47}, \href{https://suttacentral.net/an2.17/en/sujato\#3.1}{AN 2.17:3.1}, \href{https://suttacentral.net/an3/en/sujato\#55}{AN 3:55}, \href{https://suttacentral.net/an3/en/sujato\#59}{AN 3:59}, \href{https://suttacentral.net/an4/en/sujato\#184}{AN 4:184}, \href{https://suttacentral.net/an6/en/sujato\#52}{AN 6:52}, \href{https://suttacentral.net/an7/en/sujato\#47}{AN 7:47}, \href{https://suttacentral.net/an10/en/sujato\#119}{AN 10:119}, \href{https://suttacentral.net/an10/en/sujato\#167}{AN 10:167}, \href{https://suttacentral.net/an10/en/sujato\#177}{AN 10:177}). Such narrative accounts of conversion or enlightenment at the end of discourses are not very reliable, as they are not the Buddha’s words but were added by editors at some point. } From this day forth, may Mister Gotama remember me as a lay follower who has gone for refuge for life.” 

%
\section*{{\suttatitleacronym MN 5}{\suttatitletranslation Unblemished }{\suttatitleroot Anaṅgaṇasutta}}
\addcontentsline{toc}{section}{\tocacronym{MN 5} \toctranslation{Unblemished } \tocroot{Anaṅgaṇasutta}}
\markboth{Unblemished }{Anaṅgaṇasutta}
\extramarks{MN 5}{MN 5}

\scevam{So\marginnote{1.1} I have heard. }At one time the Buddha was staying near \textsanskrit{Sāvatthī} in Jeta’s Grove, \textsanskrit{Anāthapiṇḍika}’s monastery.\footnote{The Buddha does not appear in this sutta, yet he is mentioned by way of respect, indicating that these events took place while he was alive. } There \textsanskrit{Sāriputta} addressed the mendicants: “Reverends, mendicants!” 

“Reverend,”\marginnote{1.5} they replied. \textsanskrit{Sāriputta} said this: 

“Reverends,\marginnote{2.1} these four people are found in the world.\footnote{This sutta emphasizes how even small, apparently trivial defects betray a deeper corruption. Rather than hiding them away in shame, they will only be healed when brought to light. } What four? One person with a blemish doesn’t truly understand: ‘There is a blemish in me.’\footnote{“Blemish” (\textit{\textsanskrit{aṅgaṇa}}) is used literally as a mark or discoloration on the face (\href{https://suttacentral.net/mn15/en/sujato\#8.3}{MN 15:8.3}), and metaphorically in the sense of a psychological defect. It stems from the root √\textit{\textsanskrit{añj}} in the sense of something smeared on the skin such as ointment (\href{https://suttacentral.net/mn75/en/sujato\#23.13}{MN 75:23.13}) or eyeshadow (\href{https://suttacentral.net/mn82/en/sujato\#25.14}{MN 82:25.14}). } But another person with a blemish does truly understand: ‘There is a blemish in me.’ One person without a blemish doesn’t truly understand: ‘There is no blemish in me.’ But another person without a blemish does truly understand: ‘There is no blemish in me.’ In this case, of the two persons with a blemish, the one who doesn’t understand is said to be worse, while the one who does understand is better. And of the two persons without a blemish, the one who doesn’t understand is said to be worse, while the one who does understand is better.” 

When\marginnote{3.1} he said this, Venerable \textsanskrit{Mahāmoggallāna} said to him: 

“What\marginnote{3.2} is the cause, Reverend \textsanskrit{Sāriputta}, what is the reason why, of the two persons with a blemish, one is said to be worse and one better? And what is the cause, what is the reason why, of the two persons without a blemish, one is said to be worse and one better?” 

“Reverends,\marginnote{4.1} take the case of the person who has a blemish and does not understand it. You can expect that they won’t generate enthusiasm, make an effort, or rouse up energy to give up that blemish.\footnote{When \textsanskrit{Sāriputta} and \textsanskrit{Moggallāna} address each other, they use their name together with “reverend” (\textit{\textsanskrit{āvuso}}). Here we just have \textit{\textsanskrit{āvuso}}. I take it that \textsanskrit{Moggallāna} is asking questions on behalf of the group, and \textsanskrit{Sāriputta} addresses the group. } And they will die with greed, hate, and delusion, blemished, with a corrupted mind. Suppose a bronze cup was brought from a shop or smithy covered with dirt or stains. And the owners neither used it or had it cleaned, but kept it in a dirty place. Over time, wouldn’t that bronze cup get even dirtier and more stained?” 

“Yes,\marginnote{4.6} reverend.” 

“In\marginnote{4.7} the same way, take the case of the person who has a blemish and does not understand it. You can expect that … they will die with a corrupted mind. 

Take\marginnote{5.1} the case of the person who has a blemish and does understand it. You can expect that they will generate enthusiasm, make an effort, and rouse up energy to give up that blemish. And they will die without greed, hate, and delusion, unblemished, with an uncorrupted mind. Suppose a bronze cup was brought from a shop or smithy covered with dirt or stains. But the owners used it and had it cleaned, and didn’t keep it in a dirty place. Over time, wouldn’t that bronze cup get cleaner and brighter?” 

“Yes,\marginnote{5.6} reverend.” 

“In\marginnote{5.7} the same way, take the case of the person who has a blemish and does understand it. You can expect that … they will die with an uncorrupted mind. 

Take\marginnote{6.1} the case of the person who doesn’t have a blemish but does not understand it. You can expect that they will focus on the feature of beauty, and because of that, lust will infect their mind.\footnote{“Feature of beauty” is \textit{subhanimitta}. In early Pali, \textit{nimitta} is used for a feature or quality of the mind that, when focused on, promotes the growth of similar or related qualities. Thus focusing on beauty fosters the desire for that beauty. } And they will die with greed, hate, and delusion, blemished, with a corrupted mind. Suppose a bronze cup was brought from a shop or smithy clean and bright. And the owners neither used it or had it cleaned, but kept it in a dirty place. Over time, wouldn’t that bronze cup get dirtier and more stained?” 

“Yes,\marginnote{6.6} reverend.” 

“In\marginnote{6.7} the same way, take the case of the person who has no blemish and does not understand it. You can expect that … they will die with a corrupted mind. 

Take\marginnote{7.1} the case of the person who doesn’t have a blemish and does understand it. You can expect that they won’t focus on the feature of beauty, and because of that, lust won’t infect their mind. And they will die without greed, hate, and delusion, unblemished, with an uncorrupted mind. Suppose a bronze cup was brought from a shop or smithy clean and bright. And the owners used it and had it cleaned, and didn’t keep it in a dirty place. Over time, wouldn’t that bronze cup get cleaner and brighter?” 

“Yes,\marginnote{7.6} reverend.” 

“In\marginnote{7.7} the same way, take the case of the person who doesn’t have a blemish and does understand it. You can expect that … they will die with an uncorrupted mind. 

This\marginnote{8.1} is the cause, this is the reason why, of the two persons with a blemish, one is said to be worse and one better. And this is the cause, this is the reason why, of the two persons without a blemish, one is said to be worse and one better.” 

“Reverend,\marginnote{9.1} the word ‘blemish’ is spoken of.\footnote{Following the commentary, I take this question as an interjection by \textsanskrit{Moggallāna}, even though the expected personal name is lacking. } But what is ‘blemish’ a term for?” 

“Reverend,\marginnote{9.3} ‘blemish’ is a term for the spheres of bad, unskillful wishes. 

It’s\marginnote{10.1} possible that some mendicant might wish: ‘If I commit an offense, I hope the mendicants don’t find out!’\footnote{The Vinaya depends on mutual acknowledgement of offences and mutual rehabilitation. It is a system of social morality, not just of individual responsibility. } But it’s possible that the mendicants do find out that that mendicant has committed an offense. Thinking, ‘The mendicants have found out about my offense,’ they get angry and bitter. And that anger and that bitterness are both blemishes. 

It’s\marginnote{11.1} possible that some mendicant might wish: ‘If I commit an offense, I hope the mendicants accuse me in private, not in the middle of the \textsanskrit{Saṅgha}.’\footnote{Many Vinaya offences can be cleared by a simple confession to a single fellow monastic. In more serious cases, however, the entire resident \textsanskrit{Saṅgha} must be involved in the rehabilitation process. In such cases, a sense of collective shame is essential to motivate the offender to change their ways. } But it’s possible that the mendicants do accuse that mendicant in the middle of the \textsanskrit{Saṅgha} … 

It’s\marginnote{12.1} possible that some mendicant might wish: ‘If I commit an offense, I hope I’m accused by a counterpart, not by someone who is not a counterpart.’\footnote{“Counterpart” (\textit{\textsanskrit{paṭipuggala}}) is elsewhere used of the Buddha, where it is paired with \textit{sama} in a negative sense, “without equal or rival” (eg. \href{https://suttacentral.net/mn56/en/sujato\#29.54}{MN 56:29.54}). Here, the commentary explains it as someone who is a counterpart in that they too have committed an offense (\textit{\textsanskrit{sāpattiko}}, cp. \href{https://suttacentral.net/pli-tv-kd2/en/sujato\#27.3.1}{Kd 2:27.3.1}), or because they are similar in caste, learning, etc. } But it’s possible that someone who is not a counterpart accuses that mendicant … 

It’s\marginnote{13.1} possible that some mendicant might wish: ‘Oh, I hope the Teacher will teach the mendicants by repeatedly questioning me alone, not some other mendicant.’ But it’s possible that the Teacher will teach the mendicants by repeatedly questioning some other mendicant … 

It’s\marginnote{14.1} possible that some mendicant might wish: ‘Oh, I hope the mendicants will enter the village for the meal putting me at the very front, not some other mendicant.’\footnote{Normally, mendicants would wander for alms, which was usually done alone or with an attendant. In some places today, such as north-east Thailand, mendicant go for alms in a line. The phrasing “for the meal”, however, seems to refer to an occasion when a group of mendicants have been invited to take the meal in someone’s house. } But it’s possible that the mendicants will enter the village for the meal putting some other mendicant at the very front … 

It’s\marginnote{15.1} possible that some mendicant might wish: ‘Oh, I hope that I alone get the best seat, the best drink, and the best almsfood in the refectory, not some other mendicant.’ But it’s possible that some other mendicant gets the best seat, the best drink, and the best almsfood in the refectory … 

It’s\marginnote{16.1} possible that some mendicant might wish: ‘I hope that I alone give the verses of appreciation after eating in the refectory, not some other mendicant.’\footnote{This is the \textit{anumodana}, which is still recited in Pali today. It was a short teaching on the benefits of giving, emphasizing the positive effects of the donor’s actions. See examples at \href{https://suttacentral.net/dn16/en/sujato\#1.31.2}{DN 16:1.31.2} = \href{https://suttacentral.net/ud8.6/en/sujato\#22.1}{Ud 8.6:22.1}, \href{https://suttacentral.net/mn92/en/sujato\#26.1}{MN 92:26.1} = \href{https://suttacentral.net/snp3.7/en/sujato\#35.1}{Snp 3.7:35.1}, \href{https://suttacentral.net/sn55.26/en/sujato\#20.1}{SN 55.26:20.1}, \href{https://suttacentral.net/an5.44/en/sujato\#8.1}{AN 5.44:8.1}, \href{https://suttacentral.net/an54/en/sujato\#8.1}{AN 54:8.1}, \href{https://suttacentral.net/pli-tv-kd1/en/sujato\#15.14.4}{Kd 1:15.14.4}, and \href{https://suttacentral.net/pli-tv-kd1/en/sujato\#1.5.1}{Kd 1:1.5.1}. } But it’s possible that some other mendicant gives the verses of appreciation after eating in the refectory … 

It’s\marginnote{17.1} possible that some mendicant might wish: ‘Oh, I hope that I might teach the Dhamma to the monks, nuns, laymen, and laywomen in the monastery, not some other mendicant.’\footnote{It is an offence for a monk to teach nuns with a worldly motive (\href{https://suttacentral.net/pli-tv-bu-vb-pc24/en/sujato}{Bu Pc 24}). } 

But\marginnote{18{-}20.1} it’s possible that some other mendicant teaches the Dhamma … 

It’s\marginnote{21.1} possible that some mendicant might wish: ‘Oh, I hope that the monks, nuns, laymen, and laywomen will honor, respect, revere, and venerate me alone, not some other mendicant.’ 

But\marginnote{22{-}24.1} it’s possible that some other mendicant is honored, respected, revered, and venerated … 

It’s\marginnote{25.1} possible that some mendicant might wish: ‘I hope I get the nicest robes, almsfood, lodgings, and medicines and supplies for the sick, not some other mendicant.’ But it’s possible that some other mendicant gets the nicest robes, almsfood, lodgings, and medicines and supplies for the sick … 

Thinking,\marginnote{26{-}27.1} ‘Some other mendicant has got the nicest robes, almsfood, lodgings, and medicines and supplies for the sick’, they get angry and bitter. And that anger and that bitterness are both blemishes. 

‘Blemish’\marginnote{28.1} is a term for these spheres of bad, unskillful wishes. 

Suppose\marginnote{29.1} these spheres of bad, unskillful wishes are seen and heard to be not given up by a mendicant. Even though they dwell in the wilderness, in remote lodgings, eat only almsfood, wander indiscriminately for almsfood, wear rag robes, and wear shabby robes, their spiritual companions don’t honor, respect, revere, and venerate them.\footnote{These are the outward signs of someone living a dedicated ascetic life. These and other practices are called \textit{\textsanskrit{dhutaṅga}} (“shaking off”) and may be voluntarily undertaken in order to live more freely without attachments. } Why is that? It’s because these spheres of bad, unskillful wishes are seen and heard to be not given up by that venerable. Suppose a bronze cup was brought from a shop or smithy clean and bright. The owners were to prepare it with the carcass of a snake, a dog, or a human, cover it with a bronze lid, and parade it through the market-place. When people saw it they’d say: ‘My good man, what is it that you’re carrying like a precious treasure?’ So they’d open up the lid for people to look inside. But as soon as they saw it they were filled with loathing, revulsion, and disgust. Not even those who were hungry wanted to eat it, let alone those who had eaten. 

In\marginnote{29.11} the same way, when these spheres of bad, unskillful wishes are seen and heard to be not given up by a mendicant … their spiritual companions don’t honor, respect, revere, and venerate them. Why is that? It’s because these spheres of bad, unskillful wishes are seen and heard to be not given up by that venerable. 

Suppose\marginnote{30.1} these spheres of bad, unskillful wishes are seen and heard to be given up by a mendicant. Even though they dwell within a village, accept invitations to a meal, and wear robes offered by householders, their spiritual companions honor, respect, revere, and venerate them. Why is that? It’s because these spheres of bad, unskillful wishes are seen and heard to be given up by that venerable. Suppose a bronze cup was brought from a shop or smithy clean and bright. The owners were to prepare it with boiled fine rice with the dark grains picked out and served with many soups and sauces, cover it with a bronze lid, and parade it through the market-place. When people saw it they’d say: ‘My good man, what is it that you’re carrying like a precious treasure?’ So they’d open up the lid for people to look inside. And as soon as they saw it they were filled with liking, attraction, and relish. Even those who had eaten wanted to eat it, let alone those who were hungry. 

In\marginnote{30.11} the same way, when these spheres of bad, unskillful wishes are seen and heard to be given up by a mendicant … their spiritual companions honor, respect, revere, and venerate them. Why is that? It’s because these spheres of bad, unskillful wishes are seen and heard to be given up by that venerable.” 

When\marginnote{31.1} he said this, Venerable \textsanskrit{Mahāmoggallāna} said to him, “Reverend \textsanskrit{Sāriputta}, a simile strikes me.” 

“Then\marginnote{31.3} speak as you feel inspired,” said \textsanskrit{Sāriputta}. 

“Reverend,\marginnote{31.4} this one time I was staying right here in \textsanskrit{Rājagaha}, the Mountainfold. Then I robed up in the morning and, taking my bowl and robe, entered \textsanskrit{Rājagaha} for alms. Now at that time \textsanskrit{Samīti} of the wainwrights was planing the rim of a chariot wheel.\footnote{\textsanskrit{Samīti} is mentioned only here. | The rare \textit{\textsanskrit{yānakāra}} (“wainwright”) is used here instead of the normal \textit{\textsanskrit{rathakāra}} (also at \href{https://suttacentral.net/tha-ap150/pli/ms\#1.1}{Tha{-}ap 150:1.1} and \textsanskrit{Varāhamihira}’s \textsanskrit{Bṛhatsaṁhitā} 10.17). } The \textsanskrit{Ājīvaka} ascetic \textsanskrit{Paṇḍuputta}, who was formerly of the wainwrights, was standing by,\footnote{The primary Pali account of \textsanskrit{Ājīvaka} beliefs is at \href{https://suttacentral.net/dn2/en/sujato\#20.2}{DN 2:20.2}. | \textsanskrit{Paṇḍuputta} is mentioned only here, although he shares a name with the five “sons of \textsanskrit{Paṇḍu}”, better known as \textsanskrit{Pāṇḍava}, who fought the \textsanskrit{Kurukṣetra} war in the \textsanskrit{Mahābhārata}. } and this thought came to his mind: ‘Oh, I hope \textsanskrit{Samīti} the wainwright planes out the crooks, bends, and flaws in this rim. Then the rim will be rid of crooks, bends, and flaws, pure, and consolidated in the core.’\footnote{“Pure and consolidated in the core” (\textit{\textsanskrit{suddhā} assa \textsanskrit{sāre} \textsanskrit{patiṭṭhitā}}) recurs at \href{https://suttacentral.net/mn72/en/sujato\#21.5}{MN 72:21.5} (of an assembly) and \href{https://suttacentral.net/an3.93/en/sujato\#6.13}{AN 3.93:6.13} (of a crop). } And \textsanskrit{Samīti} planed out the flaws in the rim just as \textsanskrit{Paṇḍuputta} thought. Then \textsanskrit{Paṇḍuputta} expressed his gladness: ‘He planes like he knows my heart with his heart!’ 

In\marginnote{32.1} the same way, there are those faithless people who went forth from the lay life to homelessness not out of faith but to earn a livelihood. They’re devious, deceitful, and sneaky. They’re restless, insolent, fickle, scurrilous, and loose-tongued. They do not guard their sense doors or eat in moderation, and they are not dedicated to wakefulness. They don’t care about the ascetic life, and don’t keenly respect the training. They’re indulgent and slack, leaders in backsliding, neglecting seclusion, lazy, and lacking energy. They’re unmindful, lacking situational awareness and immersion, with straying minds, witless and idiotic. Venerable \textsanskrit{Sāriputta} planes their faults with this exposition of the teaching as if he knows my heart with his heart!\footnote{Also at \href{https://suttacentral.net/mn107/en/sujato\#15.2}{MN 107:15.2}. } 

But\marginnote{32.2} there are those gentlemen who went forth from the lay life to homelessness out of faith. They’re not devious, deceitful, and sneaky. They’re not restless, insolent, fickle, scurrilous, and loose-tongued. They guard their sense doors and eat in moderation, and they are dedicated to wakefulness. They care about the ascetic life, and keenly respect the training. They’re not indulgent or slack, nor are they leaders in backsliding, neglecting seclusion. They’re energetic and determined. They’re mindful, with situational awareness, immersion, and unified minds; wise and clever. Hearing this exposition of the teaching from Venerable \textsanskrit{Sāriputta}, they drink it up and devour it, as it were. And in speech and thought they say: ‘It’s good, sirs, that he draws his spiritual companions away from the unskillful and establishes them in the skillful.’\footnote{The Pali Text Society edition treats this sentence as a simple continuation of \textsanskrit{Moggallāna}’s speech. But it reads better as an exclamation by the good monks, as indicated by \textsanskrit{Mahāsaṅgīti}’s close \textit{-ti}. } 

Suppose\marginnote{33.1} there was a woman or man who was young, youthful, and fond of adornments, and had bathed their head. Presented with a garland of lotuses, jasmine, or liana flowers, they would take them in both hands and place them on the crown of the head.\footnote{This simile is also found at \href{https://suttacentral.net/an8.51/en/sujato\#24.1}{AN 8.51:24.1}. It is the inverse of the simile where a vain young person would try to remove any “blemish” from their face (eg. \href{https://suttacentral.net/mn15/en/sujato\#8.3}{MN 15:8.3}, \href{https://suttacentral.net/an10.51/en/sujato\#3.2}{AN 10.51:3.2}; with \textit{\textsanskrit{kaṇika}} instead of \textit{\textsanskrit{aṅgaṇa}} at \href{https://suttacentral.net/dn2/en/sujato\#92.1}{DN 2:92.1}). } In the same way, those gentlemen who went forth from the lay life to homelessness out of faith … say: ‘It’s good, sirs, that he draws his spiritual companions away from the unskillful and establishes them in the skillful.’” And so these two spiritual giants agreed with each others’ fine words.\footnote{“Spiritual giant” is \textit{\textsanskrit{nāga}}, which also means a dragon, a bull elephant, or a king cobra. | This mutual admiration is also expressed between \textsanskrit{Sāriputta} and \textsanskrit{Moggallāna} at \href{https://suttacentral.net/sn21.3/en/sujato\#7.1}{SN 21.3:7.1} and \href{https://suttacentral.net/ud4.4/en/sujato\#8.1}{Ud 4.4:8.1}, and between \textsanskrit{Sāriputta} and \textsanskrit{Puṇṇa} son of \textsanskrit{Mantāṇī} at \href{https://suttacentral.net/mn24/en/sujato\#17.13}{MN 24:17.13}.  } 

%
\section*{{\suttatitleacronym MN 6}{\suttatitletranslation One Might Wish }{\suttatitleroot Ākaṅkheyyasutta}}
\addcontentsline{toc}{section}{\tocacronym{MN 6} \toctranslation{One Might Wish } \tocroot{Ākaṅkheyyasutta}}
\markboth{One Might Wish }{Ākaṅkheyyasutta}
\extramarks{MN 6}{MN 6}

\scevam{So\marginnote{1.1} I have heard. }At one time the Buddha was staying near \textsanskrit{Sāvatthī} in Jeta’s Grove, \textsanskrit{Anāthapiṇḍika}’s monastery. There the Buddha addressed the mendicants, “Mendicants!” 

“Venerable\marginnote{1.5} sir,” they replied. The Buddha said this: 

“Mendicants,\marginnote{2.1} live by the ethical precepts and the monastic code. Live restrained in the monastic code, conducting yourselves well and resorting for alms in suitable places. Seeing danger in the slightest fault, keep the rules you’ve undertaken.\footnote{The “monastic code” (\textit{\textsanskrit{pātimokkha}}) is the primary list of rules for Buddhist monastics. There are many extant \textit{\textsanskrit{pātimokkhas}} in Chinese, Tibetan, and Sanskrit from early Buddhist schools, all similar but with some variations especially in minor rules of etiquette. Each fortnight the \textit{\textsanskrit{pātimokkha}} is recited to affirm the communal rules of the \textsanskrit{Saṅgha}. Other passages are recited as inspiration, including this exhortation to keep the rules (also at \href{https://suttacentral.net/an4.12/en/sujato\#1.1}{AN 4.12:1.1}, \href{https://suttacentral.net/an10.71/en/sujato\#2.1}{AN 10.71:2.1}, and \href{https://suttacentral.net/iti111/en/sujato\#2.1}{Iti 111:2.1}), as well as the verses known as \textit{\textsanskrit{Ovāda} \textsanskrit{Pātimokkha}} (\href{https://suttacentral.net/dn14/en/sujato\#3.28.1}{DN 14:3.28.1}). | Commentaries offer several explanations for the word \textit{\textsanskrit{pātimokkha}}, favoring the sense “leading to release” (from remorse and suffering). The Buddha says he laid down the \textit{\textsanskrit{pātimokkha}} to tie the community together like flowers bunched with string (\href{https://suttacentral.net/pli-tv-bu-vb-pj1/en/sujato\#3.2.6}{Bu Pj 1:3.2.6}), which suggests a connection with \textit{\textsanskrit{paṭimukka}}, “binding” (\href{https://suttacentral.net/mn38/en/sujato\#41.11}{MN 38:41.11}). } 

A\marginnote{3.1} mendicant might wish: ‘May I be liked and approved by my spiritual companions, respected and admired.’ So let them fulfill their precepts, be committed to inner serenity of the heart, not neglect absorption, be endowed with discernment, and frequent empty huts.\footnote{To “fulfill precepts” repeats the commitment to the monastic code. | “Serenity” and “absorption” are the practice of meditation to develop tranquility (\textit{samatha}). |  “Discernment” is \textit{\textsanskrit{vipassanā}} (“insight”) meditation | “Empty huts” indicates solitude. } 

A\marginnote{4.1} mendicant might wish: ‘May I receive robes, almsfood, lodgings, and medicines and supplies for the sick.’ So let them fulfill their precepts, be committed to inner serenity of the heart, not neglect absorption, be endowed with discernment, and frequent empty huts.\footnote{This series of arguments is conditional: \emph{if} this is what you want, \emph{then} this is how to get it. The Buddha is not encouraging people to want these things; on the contrary, he frequently warned of the dangers of such attachments. He is being pragmatic: given that people have desires, how can they be encouraged to desire something better? } 

A\marginnote{5.1} mendicant might wish: ‘May the services of those whose robes, almsfood, lodgings, and medicines and supplies for the sick I enjoy be very fruitful and beneficial for them.’ So let them fulfill their precepts … 

A\marginnote{6.1} mendicant might wish: ‘When deceased family and relatives who have passed away recollect me with a confident mind, may this be very fruitful and beneficial for them.’ So let them fulfill their precepts … 

A\marginnote{7.1} mendicant might wish: ‘May I prevail over desire and discontent, and may desire and discontent not prevail over me. May I live having mastered desire and discontent whenever they arose.’ So let them fulfill their precepts … 

A\marginnote{8.1} mendicant might wish: ‘May I prevail over fear and dread, and may fear and dread not prevail over me. May I live having mastered fear and dread whenever they arose.’ So let them fulfill their precepts … 

A\marginnote{9.1} mendicant might wish: ‘May I get the four absorptions—blissful meditations in this life that belong to the higher mind—when I want, without trouble or difficulty.’ So let them fulfill their precepts … 

A\marginnote{10.1} mendicant might wish: ‘May I have direct meditative experience of the peaceful liberations that are formless, transcending form.’ So let them fulfill their precepts … 

A\marginnote{11.1} mendicant might wish: ‘May I, with the ending of three fetters, become a stream-enterer, not liable to be reborn in the underworld, bound for awakening.’ So let them fulfill their precepts … 

A\marginnote{12.1} mendicant might wish: ‘May I, with the ending of three fetters, and the weakening of greed, hate, and delusion, become a once-returner, coming back to this world once only, then making an end of suffering.’ So let them fulfill their precepts … 

A\marginnote{13.1} mendicant might wish: ‘May I, with the ending of the five lower fetters, be reborn spontaneously and become extinguished there, not liable to return from that world.’ So let them fulfill their precepts … 

A\marginnote{14.1} mendicant might wish: ‘May I wield the many kinds of psychic power: multiplying myself and becoming one again; appearing and disappearing; going unobstructed through a wall, a rampart, or a mountain as if through space; diving in and out of the earth as if it were water; walking on water as if it were earth; flying cross-legged through the sky like a bird; touching and stroking with my hand the sun and moon, so mighty and powerful; controlling the body as far as the realm of divinity.’ So let them fulfill their precepts …\footnote{“Psychic powers” (\textit{iddhi}) were much cultivated in the Buddha’s day, but the means to acquire them varied: devotion to a god, brutal penances, or magic rituals. The Buddha taught that the mind developed in \textit{\textsanskrit{samādhi}} was capable of things that are normally incomprehensible. | Only a few of these are attested as events in the early texts. The most common is the ability to disappear and reappear, exhibited by the Buddha (\href{https://suttacentral.net/an8.30/en/sujato\#2.1}{AN 8.30:2.1}), some disciples (\href{https://suttacentral.net/mn37/en/sujato\#6.1}{MN 37:6.1}), and deities (\href{https://suttacentral.net/mn67/en/sujato\#8.1}{MN 67:8.1}). } 

A\marginnote{15.1} mendicant might wish: ‘With clairaudience that is purified and superhuman, may I hear both kinds of sounds, human and heavenly, whether near or far.’ So let them fulfill their precepts …\footnote{“Clairaudience” is a literal rendition of \textit{dibbasota}. The root sense of \textit{dibba} is to “shine” like the bright sky or a divine being. The senses of clarity and divinity are both present. } 

A\marginnote{16.1} mendicant might wish: ‘May I understand the minds of other beings and individuals, having comprehended them with my mind.\footnote{Note that the Indic idiom is not the “reading” of minds, which suggests hearing the words spoken in inner dialogue. While this is exhibited by the Buddha (eg. \href{https://suttacentral.net/an8.30/en/sujato\#2.1}{AN 8.30:2.1}), the main emphasis is on the comprehension of the overall state of mind. } May I understand mind with greed as “mind with greed”, and mind without greed as “mind without greed”; mind with hate as “mind with hate”, and mind without hate as “mind without hate”; mind with delusion as “mind with delusion”, and mind without delusion as “mind without delusion”; constricted mind as “constricted mind”, and scattered mind as “scattered mind”; expansive mind as “expansive mind”, and unexpansive mind as “unexpansive mind”; mind that is not supreme as “mind that is not supreme”, and mind that is supreme as “mind that is supreme”; mind immersed in \textsanskrit{samādhi} as “mind immersed in \textsanskrit{samādhi}”, and mind not immersed in \textsanskrit{samādhi} as “mind not immersed in \textsanskrit{samādhi}”; freed mind as “freed mind”, and unfreed mind as “unfreed mind”.’ So let them fulfill their precepts … 

A\marginnote{17.1} mendicant might wish: ‘May I recollect many kinds of past lives. That is: one, two, three, four, five, ten, twenty, thirty, forty, fifty, a hundred, a thousand, a hundred thousand rebirths; many eons of the world contracting, many eons of the world expanding, many eons of the world contracting and expanding. May I remember: “There, I was named this, my clan was that, I looked like this, and that was my food. This was how I felt pleasure and pain, and that was how my life ended. When I passed away from that place I was reborn somewhere else. There, too, I was named this, my clan was that, I looked like this, and that was my food. This was how I felt pleasure and pain, and that was how my life ended. When I passed away from that place I was reborn here.” May I thus recollect my many kinds of past lives, with features and details.’\footnote{Empowered by the fourth \textit{\textsanskrit{jhāna}}, memory breaks through the veil of birth and death, revealing the vast expanse of time and dispelling the illusion that there is any place of eternal rest or sanctuary in the cycle of transmigration. } So let them fulfill their precepts … 

A\marginnote{18.1} mendicant might wish: ‘With clairvoyance that is purified and superhuman, may I see sentient beings passing away and being reborn—inferior and superior, beautiful and ugly, in a good place or a bad place—and understand how sentient beings are reborn according to their deeds: “These dear beings did bad things by way of body, speech, and mind. They denounced the noble ones; they had wrong view; and they chose to act out of that wrong view. When their body breaks up, after death, they’re reborn in a place of loss, a bad place, the underworld, hell. These dear beings, however, did good things by way of body, speech, and mind. They never denounced the noble ones; they had right view; and they chose to act out of that right view. When their body breaks up, after death, they’re reborn in a good place, a heavenly realm.” And so, with clairvoyance that is purified and superhuman, may I see sentient beings passing away and being reborn—inferior and superior, beautiful and ugly, in a good place or a bad place. And may I understand how sentient beings are reborn according to their deeds.’\footnote{Here knowledge extends to the rebirths of others as well as oneself. Even more significant, it brings in the understanding of cause and effect; \emph{why} rebirth happens the way it does. Such knowledge, however, is not infallible, as the Buddha warns in \href{https://suttacentral.net/dn1/en/sujato\#2.5.3}{DN 1:2.5.3} and \href{https://suttacentral.net/mn136/en/sujato}{MN 136}. The experience is one thing; the inferences drawn from it are another. One should draw conclusions only tentatively, after long experience. } So let them fulfill their precepts … 

A\marginnote{19.1} mendicant might wish: ‘May I realize the undefiled freedom of heart and freedom by wisdom in this very life, and live having realized it with my own insight due to the ending of defilements.’\footnote{This is the experience of awakening that is the true goal of the Buddhist path. The defilements—properties of the mind that create suffering—have been curbed by the practice of ethics and suppressed by the power of \textit{\textsanskrit{jhāna}}. Here they are eliminated forever. } So let them fulfill their precepts, be committed to inner serenity of the heart, not neglect absorption, be endowed with discernment, and frequent empty huts. 

‘Mendicants,\marginnote{20.1} live by the ethical precepts and the monastic code. Live restrained in the monastic code, conducting yourselves well and resorting for alms in suitable places. Seeing danger in the slightest fault, keep the rules you’ve undertaken.’ That’s what I said, and this is why I said it.” 

That\marginnote{20.3} is what the Buddha said. Satisfied, the mendicants approved what the Buddha said. 

%
\section*{{\suttatitleacronym MN 7}{\suttatitletranslation The Simile of the Cloth }{\suttatitleroot Vatthasutta}}
\addcontentsline{toc}{section}{\tocacronym{MN 7} \toctranslation{The Simile of the Cloth } \tocroot{Vatthasutta}}
\markboth{The Simile of the Cloth }{Vatthasutta}
\extramarks{MN 7}{MN 7}

\scevam{So\marginnote{1.1} I have heard. }At one time the Buddha was staying near \textsanskrit{Sāvatthī} in Jeta’s Grove, \textsanskrit{Anāthapiṇḍika}’s monastery. There the Buddha addressed the mendicants, “Mendicants!” 

“Venerable\marginnote{1.5} sir,” they replied. The Buddha said this: 

“Suppose,\marginnote{2.1} mendicants, there was a cloth that was dirty and soiled. No matter what dye the dyer applied—whether blue or yellow or red or magenta—it would look poorly dyed and impure in color. Why is that? Because of the impurity of the cloth. 

In\marginnote{2.5} the same way, when the mind is corrupt, a bad destiny is to be expected.\footnote{A “bad destiny” (\textit{duggati}) is any realm below the human, namely the animal, ghost, and hell realms. } Suppose there was a cloth that was pure and clean. No matter what dye the dyer applied—whether blue or yellow or red or magenta—it would look well dyed and pure in color. Why is that? Because of the purity of the cloth. 

In\marginnote{2.10} the same way, when the mind isn’t corrupt, a good destiny is to be expected. 

And\marginnote{3.1} what are the corruptions of the mind?\footnote{Here the text shifts from \textit{\textsanskrit{saṅkiliṭṭha}} (“corrupt”) to \textit{upakkilesa} (“corruption”). These are general terms for unskillful qualities of mind and the change in prefix has no particular significence. } Covetousness and immoral greed, ill will, anger, acrimony, disdain, contempt, jealousy, stinginess, deceit, deviousness, obstinacy, aggression, conceit, arrogance, vanity, and negligence are corruptions of the mind. 

A\marginnote{4.1} mendicant who understands that covetousness and immoral greed are corruptions of the mind gives them up. A mendicant who understands that ill will … negligence is a corruption of the mind gives it up. 

When\marginnote{5.1} they have understood these corruptions of the mind for what they are, and have given them up, they have experiential confidence in the Buddha:\footnote{“Experiential” is \textit{avecca}, literally “having undergone”. “Experiential confidence” is the faith of a stream-enterer, who has seen for themselves. } ‘That Blessed One is perfected, a fully awakened Buddha, accomplished in knowledge and conduct, holy, knower of the world, supreme guide for those who wish to train, teacher of gods and humans, awakened, blessed.’\footnote{They have experiential confidence in the Buddha as a teacher because they have followed his path and realized the results that he speaks of. } 

They\marginnote{6.1} have experiential confidence in the teaching: ‘The teaching is well explained by the Buddha—apparent in the present life, immediately effective, inviting inspection, relevant, so that sensible people can know it for themselves.’\footnote{A stream-enterer has direct experience of the four noble truths, so they have confirmed that the teaching is indeed realizable in this very life. } 

They\marginnote{7.1} have experiential confidence in the \textsanskrit{Saṅgha}: ‘The \textsanskrit{Saṅgha} of the Buddha’s disciples is practicing the way that’s good, sincere, systematic, and proper. It consists of the four pairs, the eight individuals. This is the \textsanskrit{Saṅgha} of the Buddha’s disciples that is worthy of offerings dedicated to the gods, worthy of hospitality, worthy of a religious donation, worthy of greeting with joined palms, and is the supreme field of merit for the world.’\footnote{The suttas distinguish between two senses of \textsanskrit{Saṅgha}. The “mendicant \textsanskrit{Saṅgha}” (\textit{\textsanskrit{bhikkhusaṅgha}}) is the conventional community of monks and nuns. The “\textsanskrit{Saṅgha} of disciples” (\textit{\textsanskrit{sāvakasaṅgha}}) is classified as fourfold according to the stages of awakening: stream-entry, once-return, non-return, and perfection. Each of these stages is further subdivided into those of the path who are practicing for realization and those of the fruit who have realized. These are referred to as “noble disciples”, four of the path and four of the fruit, making eight individuals in total. \textsanskrit{Saṅgha} is not used in the sense of “spiritual community”. } 

When\marginnote{8.1} a mendicant has discarded, eliminated, released, given up, and relinquished to this extent, thinking, ‘I have experiential confidence in the Buddha … the teaching … the \textsanskrit{Saṅgha},’ they find inspiration in the meaning and the teaching, and find joy connected with the teaching. Thinking: ‘I have discarded, eliminated, released, given up, and relinquished to this extent,’ they find inspiration in the meaning and the teaching, and find joy connected with the teaching. When they’re joyful, rapture springs up. When the mind is full of rapture, the body becomes tranquil. When the body is tranquil, they feel bliss. And when they’re blissful, the mind becomes immersed in \textsanskrit{samādhi}.\footnote{Reflection on one’s progress brings joy and spurs further progress. This can be so powerful as to be a basis for \textit{\textsanskrit{samādhi}}. } 

When\marginnote{12.1} a mendicant of such ethics, such qualities, and such wisdom eats boiled fine rice with the dark grains picked out and served with many soups and sauces, that is no obstacle for them.\footnote{An offering of delicious food is normally the most sensual temptation in a mendicant’s day. In the Chinese parallel at EA 13.5, this is mentioned only after arahantship. } Compare with cloth that is dirty and soiled; it can be made pure and clean by pure water. Or native gold, which can be made pure and bright by a forge. In the same way, when a mendicant of such ethics, such qualities, and such wisdom eats boiled fine rice with the dark grains picked out and served with many soups and sauces, that is no obstacle for them. 

They\marginnote{13.1} meditate spreading a heart full of love to one direction, and to the second, and to the third, and to the fourth. In the same way above, below, across, everywhere, all around, they spread a heart full of love to the whole world—abundant, expansive, limitless, free of enmity and ill will.\footnote{Here the Buddha introduces the so-called “divine meditations” (\textit{\textsanskrit{brahmavihāra}}) or “immeasurables” (\textit{\textsanskrit{appamañña}}). These are four wholesome emotional states that can be developed as a basis for \textit{\textsanskrit{samādhi}}. They were evidently pre-Buddhist, although they have not been traced as a group in pre-Buddhist texts. However, they are shared with later non-Buddhist texts such as  \textsanskrit{Yogasūtra} 1.33 and the Jain \textsanskrit{Tattvārthasūtra} 7.11. | “Love” (\textit{\textsanskrit{mettā}}) is a universal positive regard and well-wishing free of personal desires or attachments. } They meditate spreading a heart full of compassion to one direction, and to the second, and to the third, and to the fourth. In the same way above, below, across, everywhere, all around, they spread a heart full of compassion to the whole world—abundant, expansive, limitless, free of enmity and ill will.\footnote{“Compassion” (\textit{\textsanskrit{karuṇā}}) is the quality of empathy with the suffering of another or oneself and the wish to remove it. } They meditate spreading a heart full of rejoicing to one direction, and to the second, and to the third, and to the fourth. In the same way above, below, across, everywhere, all around, they spread a heart full of rejoicing to the whole world—abundant, expansive, limitless, free of enmity and ill will.\footnote{“Rejoicing” (\textit{\textsanskrit{muditā}}) is joyful celebration in the success of others or oneself, free of jealousy or cynicism. } They meditate spreading a heart full of equanimity to one direction, and to the second, and to the third, and to the fourth. In the same way above, below, across, everywhere, all around, they spread a heart full of equanimity to the whole world—abundant, expansive, limitless, free of enmity and ill will.\footnote{Equanimity (\textit{\textsanskrit{upekkhā}}) is literally “close watching”, not interfering but standing ready when needed. It is not indifference, which is why it emerges only at the end, after the positive emotions are developed. } 

They\marginnote{17.1} understand: ‘There is this, there is what is worse than this, there is what is better than this, and there is an escape beyond the scope of perception.’\footnote{This is a description of advanced insight by a stream-enterer. The meditator understands the current state of their experience, namely the mind developed through \textit{\textsanskrit{samādhi}}. They know that this is conditioned and hence liable to decline to a lower state “worse than this”. They know that there are still higher states of mind that can be developed. And they know that, while all states of meditation fall within the scope of perception, there is an ultimate escape, namely \textsanskrit{Nibbāna}. } 

Knowing\marginnote{18.1} and seeing like this, their mind is freed from the defilements of sensuality, desire to be reborn, and ignorance. When they’re freed, they know they’re freed. 

They\marginnote{18.3} understand: ‘Rebirth is ended, the spiritual journey has been completed, what had to be done has been done, there is nothing further for this place.’ This is called a mendicant who is bathed with the inner bathing.”\footnote{“Bathed” is \textit{\textsanskrit{sināta}}, a rare variant spelling of \textit{nhata} (Sanskrit \textit{\textsanskrit{snāta}}. \textit{\textsanskrit{Sināta}} also occurs at \href{https://suttacentral.net/sn7.9/en/sujato\#15.3}{SN 7.9:15.3}). | Later Brahmanical texts regarded external bathing as spiritually effective only when accompanied by inner bathing or purity of soul. The \textsanskrit{Mahābhārata} says that “he who is bathed in the bath of self-disciple is clean inside and out” (13.111.9c, \textit{sa \textsanskrit{snāto} yo \textsanskrit{damasnātaḥ} \textsanskrit{sabāhyābhyantaraḥ} \textsanskrit{śuciḥ}}), while the \textsanskrit{Liṅga}-\textsanskrit{purāṇa} (1.8.33) insists that one who has bathed externally “must also practice the inner bathing” (\textit{\textsanskrit{snānaṁ} … \textsanskrit{ābhyantaraṁ} caret}). } 

Now\marginnote{19.1} at that time the brahmin \textsanskrit{Bhāradvāja} of \textsanskrit{Sundarikā} was sitting not far from the Buddha.\footnote{It seems that the Buddha mentioned the inner bathing to provoke the brahmin. | Four of the six Chinese parallels to this passage situate it on the bank of a river. } He said to the Buddha, “But does Mister Gotama go to the river \textsanskrit{Bāhuka} to bathe?” 

“Brahmin,\marginnote{19.4} why go to the river \textsanskrit{Bāhuka}? What can the river \textsanskrit{Bāhuka} do?” 

“Many\marginnote{19.6} people deem that the river \textsanskrit{Bāhukā} leads to a heavenly world and bestows merit. And many people wash off their bad deeds in the river \textsanskrit{Bāhukā}.”\footnote{\textit{Lokkha}, elsewhere unattested in Pali, is Sanskrit \textit{lokya}, which occurs several times in the Śatapatha \textsanskrit{Brāhmaṇa} in the sense “conducive to a (heavenly) world” (9.5.2.16, 10.5.2.12, 11.3.3.7). The root sense of \textit{loka} is “light” and it was originally the bright sky with its \textit{devas} (eg. Śatapatha \textsanskrit{Brāhmaṇa} \textsanskrit{Mādyandina} 11.2.3.1–6, \textsanskrit{Kāṇva} 3.2.5.1), later being extended to include all worlds. The PTS reading \textit{mokkha} is an unwarranted normalization, although it is supported by the parallel at MA 93, which has \langlzh{度}. Note that the expected form \textit{lokiya}, although common in later Pali, occurs only in one early verse (\href{https://suttacentral.net/thag2.18/en/sujato\#2.4}{Thag 2.18:2.4}). | Bathing for purity from misdeeds as a pre-Buddhist Brahmanical custom is attested in Śukla Yajurveda 3.48. The \textsanskrit{bhikkhunī} \textsanskrit{Puṇṇikā} pointed out that if this were true then the fish, frogs, and turtles would go to heaven (\href{https://suttacentral.net/thig12.1/en/sujato\#6.1}{Thig 12.1:6.1}); her argument was echoed by the Jains (\textsanskrit{Sūyagaḍa} 1.7.14–16) and later Brahmins (\textsanskrit{Liṅga}-\textsanskrit{purāṇa} 1.8.33–4). } 

Then\marginnote{20.1} the Buddha addressed \textsanskrit{Bhāradvāja} of \textsanskrit{Sundarikā} in verse: 

\begin{verse}%
“The\marginnote{20.2} \textsanskrit{Bāhukā} and the \textsanskrit{Adhikakkā},\footnote{The \textsanskrit{Bāhukā} (variant \textsanskrit{Bahukā}) is mentioned only here. It may be the \textsanskrit{Bāhudā} found in various Sanskrit texts (\textsanskrit{Bhāgavata} \textsanskrit{Purāṇa} 5.19.18), both words having the sense “granter of abundance”. | \textsanskrit{Adhikakkā} (variant \textsanskrit{Avikakkā}) is mentioned only here. The commentary says it was a ford (\textit{tittha}). } \\
at \textsanskrit{Gayā} and the \textsanskrit{Sundarikā} too,\footnote{\textsanskrit{Gayā} is a ford on the Phalgu River by the town of \textsanskrit{Gayā}, which is still a popular site for sacred bathing. | \textsanskrit{Sundarikā} was a river in Kosala used for Brahmanical rituals (\href{https://suttacentral.net/sn7.9/en/sujato}{SN 7.9}, \href{https://suttacentral.net/snp3.4/en/sujato}{Snp 3.4}). Bathing in the ford there was said to bring beauty (\textsanskrit{Mahābhārata} 3.82.51). } \\
\textsanskrit{Sarasvatī} and \textsanskrit{Payāga},\footnote{Pali \textit{\textsanskrit{sarassatī}} is better known in the Sanskrit form \textit{\textsanskrit{sarasvatī}}, the most prominent river of the Vedas, who “like a snorting boar, broke the back of the mountains with her mighty waves” (Rig Veda 6.61.2). Change stole her waters and today she is lost. Researchers have identified her with the now-residual Ghaggar-Hakra system in north-west India and Pakistan or the Helmand River in Afghanistan. | \textsanskrit{Payāga} (modern Prayagraj, formerly Allahabad) is the sacred ford at the confluence of the Ganges and the \textsanskrit{Yamunā} beside \textsanskrit{Kosambī} (see \href{https://suttacentral.net/pli-tv-bu-vb-pj1/en/sujato\#4.18}{Bu Pj 1:4.18}). The \textsanskrit{Mahābhārata} calls it the “vulva of the world” (3.83.71), the most meritorious of all fords, where bathing can wash away a hundred crimes (3.83.82). } \\
and the river \textsanskrit{Bāhumatī}:\footnote{The \textsanskrit{Bāhumatī} River is elsewhere only known by a passing mention in Sanskrit reference works. As an epithet of Indra it means “strong of arm”. } \\
a fool can constantly plunge into them \\
but it won’t purify their dark deeds. 

What\marginnote{20.8} can the \textsanskrit{Sundarikā} do? \\
What the \textsanskrit{Payāga} or the \textsanskrit{Bāhukā}? \\
They can’t cleanse a cruel person, a sinner \\
from their bad deeds. 

For\marginnote{20.12} the pure in heart it’s always \\
the spring festival or the sabbath.\footnote{“Spring festival” is \textit{phaggu}, said to be held at \textsanskrit{Gayā} (\href{https://suttacentral.net/thag5.7/en/sujato\#1.4}{Thag 5.7:1.4}, \href{https://suttacentral.net/thag4.6/en/sujato\#1.2}{Thag 4.6:1.2}), although Sanskrit sources take it as the name of a river at \textsanskrit{Gayā}. } \\
For the pure in heart and clean of deed, \\
their vows will always be fulfilled. \\
It’s here alone that you should bathe, brahmin, \\
making yourself a sanctuary for all creatures.\footnote{A \textit{khema} (“sanctuary”) is a place of safety for wild creatures, a meaning featured in several \textsanskrit{Jātaka} stories (eg. \href{https://suttacentral.net/ja482}{Ja 482}). } 

And\marginnote{20.18} if you speak no lies,\footnote{The straightforward ethical teachings here contrast with the more demanding teachings for the mendicants in the previous prose. Such teachings presage the universal values of \textit{dharma} promoted by King Asoka in his edicts. } \\
nor harm any living creature, \\
nor steal anything not given, \\
and you’re faithful and not stingy: \\
what’s the point of going to \textsanskrit{Gayā}? \\
For any well may be your \textsanskrit{Gayā}!” 

%
\end{verse}

When\marginnote{21.1} he had spoken, the brahmin \textsanskrit{Bhāradvāja} of \textsanskrit{Sundarikā} said to the Buddha, “Excellent, Mister Gotama! Excellent! As if he were righting the overturned, or revealing the hidden, or pointing out the path to the lost, or lighting a lamp in the dark so people with clear eyes can see what’s there, Mister Gotama has made the teaching clear in many ways. I go for refuge to Mister Gotama, to the teaching, and to the mendicant \textsanskrit{Saṅgha}. May I receive the going forth, the ordination in Mister Gotama’s presence?” 

And\marginnote{22.1} the brahmin \textsanskrit{Bhāradvāja} of \textsanskrit{Sundarikā} received the going forth, the ordination in the Buddha’s presence. Not long after his ordination, Venerable \textsanskrit{Bhāradvāja}, living alone, withdrawn, diligent, keen, and resolute, soon realized the supreme end of the spiritual path in this very life. He lived having achieved with his own insight the goal for which gentlemen rightly go forth from the lay life to homelessness. 

He\marginnote{22.3} understood: “Rebirth is ended; the spiritual journey has been completed; what had to be done has been done; there is nothing further for this place.” And Venerable \textsanskrit{Bhāradvāja} became one of the perfected. 

%
\section*{{\suttatitleacronym MN 8}{\suttatitletranslation Self-Effacement }{\suttatitleroot Sallekhasutta}}
\addcontentsline{toc}{section}{\tocacronym{MN 8} \toctranslation{Self-Effacement } \tocroot{Sallekhasutta}}
\markboth{Self-Effacement }{Sallekhasutta}
\extramarks{MN 8}{MN 8}

\scevam{So\marginnote{1.1} I have heard. }At one time the Buddha was staying near \textsanskrit{Sāvatthī} in Jeta’s Grove, \textsanskrit{Anāthapiṇḍika}’s monastery.\footnote{Of the two parallels to this sutta, MA 91 takes place at the \textsanskrit{Ghositārāma} in \textsanskrit{Kosambī}, while EA 47.9 is set in the Squirrels’ Feeding Ground at \textsanskrit{Rājagaha}. The specifics of narrative backgrounds often vary; they are not “the Buddha’s word”, but rather were added by editors at some point. } 

Then\marginnote{2.1} in the late afternoon, Venerable \textsanskrit{Mahācunda} came out of retreat and went to the Buddha. He bowed, sat down to one side, and said to the Buddha:\footnote{There are various Cundas whose connection is unclear. \textsanskrit{Mahācunda} was one of the great disciples (\href{https://suttacentral.net/mn118/en/sujato}{MN 118:}, \href{https://suttacentral.net/an6.17/en/sujato}{AN 6.17}), who later brought the Dhamma to the land of the \textsanskrit{Cetīs} (\href{https://suttacentral.net/an6.46/en/sujato}{AN 6.46}, \href{https://suttacentral.net/an10.24/en/sujato}{AN 10.24}). He once stayed with Channa and \textsanskrit{Sāriputta} (\href{https://suttacentral.net/mn144/en/sujato\#4.1}{MN 144:4.1}, \href{https://suttacentral.net/sn35.87/en/sujato\#1.2}{SN 35.87:1.2}); the commentaries say he was in fact \textsanskrit{Sāriputta}’s younger brother. They also identify him with the “novice Cunda” who reported the deaths of \textsanskrit{Sāriputta} (\href{https://suttacentral.net/sn47.13/en/sujato\#1.3}{SN 47.13:1.3}) and \textsanskrit{Mahāvīra} (\href{https://suttacentral.net/dn29/en/sujato\#2.1}{DN 29:2.1}, \href{https://suttacentral.net/mn104/en/sujato\#3.1}{MN 104:3.1}), explaining that the title “novice” was a nickname that persisted from the time he ordained as a young novice. It is unclear whether the commentaries take the Cundaka of \href{https://suttacentral.net/dn16/en/sujato\#4.39.2}{DN 16:4.39.2} = \href{https://suttacentral.net/ud8.5/en/sujato\#14.3}{Ud 8.5:14.3} to be the same person, but he is there performing a similar role as carer adjacent to the Buddha’s death. Later there appears a \textsanskrit{Cūḷacunda} (\href{https://suttacentral.net/tha-ap52/en/sujato\#13.3}{Tha Ap 52:13.3}). Thus there may have been one person known by different names, or several people whose stories have become conflated. } 

“Sir,\marginnote{3.1} there are many different views that arise in the world connected with theories of self or with theories of the cosmos.\footnote{These are discussed many times in the suttas, for example doctrines of the self at \href{https://suttacentral.net/mn2/en/sujato\#8.2}{MN 2:8.2} and \href{https://suttacentral.net/mn44/en/sujato}{MN 44}; of the cosmos at \href{https://suttacentral.net/mn63/en/sujato}{MN 63} and \href{https://suttacentral.net/mn72/en/sujato}{MN 72}; and of both at \href{https://suttacentral.net/dn1/en/sujato\#1.30.1}{DN 1:1.30.1} and \href{https://suttacentral.net/mn102/en/sujato\#14.1}{MN 102:14.1}. } How does a mendicant who is focusing on the starting point give up and let go of these views?”\footnote{The idiom \textit{\textsanskrit{ādimeva}} elsewhere occurs only when the Buddha gives a mendicant instructions for a retreat. Typically (\href{https://suttacentral.net/sn47.3/en/sujato\#1.6}{SN 47.3:1.6}, \href{https://suttacentral.net/sn47.15/en/sujato\#1.4}{SN 47.15:1.4}, \href{https://suttacentral.net/sn47.16/en/sujato\#1.4}{SN 47.16:1.4}) he says to “purify the starting point of skillful qualities”, namely ethics (\textit{\textsanskrit{sīla}}) and views (\textit{\textsanskrit{diṭṭhi}}), on the latter of which \textsanskrit{Mahācunda} is seeking clarification. | In the Pali the question is phrased directly and is sometimes translated “do they give them up?”, but the idiom \textit{\textsanskrit{evametāsaṁ}} shifts the sense to “how do they give them up?” (\href{https://suttacentral.net/mn74/en/sujato\#6.14}{MN 74:6.14}, \href{https://suttacentral.net/mn137/en/sujato\#16.8}{MN 137:16.8}). And that is indeed the answer the Buddha gives. The commentary explains it similarly (\textit{etena \textsanskrit{upāyena} \textsanskrit{etāsaṁ}}), and both Chinese texts have \langlzh{云何} here, which typically stands for \textit{\textsanskrit{kathaṁ}} (“how”). } 

“Cunda,\marginnote{3.4} there are many different views that arise in the world connected with theories of self or with theories of the cosmos. A mendicant gives up and lets go of these views by truly seeing with right wisdom where they arise, where they settle in, and where they operate as: ‘This is not mine, I am not this, this is not my self.’\footnote{This is the insight of stream-entry, where wrong views are fully abandoned by means of the clarified wisdom that sees the truth of conditionality. The remainder of the sutta discusses the practice leading to this point. } 

It’s\marginnote{4.1} possible that a certain mendicant, quite secluded from sensual pleasures, secluded from unskillful qualities, might enter and remain in the first absorption, which has the rapture and bliss born of seclusion, while placing the mind and keeping it connected. They might think they’re practicing self-effacement.\footnote{The term “self-effacement” that lends the sutta its title is Pali \textit{sallekha}, which has the sense of “rubbing out” (an inscription or mark). It is associated with such virtues as contentment, seclusion, and simplicity (eg. \href{https://suttacentral.net/mn3/en/sujato\#3.25}{MN 3:3.25}). Here it refers to the strength of character to persist in “effacing” unwholesome qualities through developing their opposites, no matter what others might do. | Not to be confused with the Jain \textit{\textsanskrit{sallekhanā}}, where practitioners at the end of life refuse all food (\textsanskrit{Tattvārthasūtra} 7.22). In fact, this sutta appears to emphasize the gentle nature of the Buddha’s approach in deliberate contrast with the stern austerities of the Jains. } But in the training of the Noble One these are not called ‘self-effacement’; they’re called ‘blissful meditations in this life’.\footnote{The sutta is not deprecating the \textit{\textsanskrit{jhānas}}, for they are a fundamental part of the early Buddhist path, and as such are included later under the heading of “right immersion” (\href{https://suttacentral.net/mn8/en/sujato\#12.19}{MN 8:12.19}). Indeed, the sutta ends with the Buddha encouraging Cunda to practice \textit{\textsanskrit{jhāna}}. Rather, this text emphasizes the need to actively develop all aspects of the path, humble or exalted, and integrate them in every aspect of life, rather than solely relying on meditation in seclusion. } 

It’s\marginnote{5.1} possible that some mendicant, as the placing of the mind and keeping it connected are stilled, might enter and remain in the second absorption, which has the rapture and bliss born of immersion, with internal clarity and mind at one, without placing the mind and keeping it connected. They might think they’re practicing self-effacement. But in the training of the Noble One these are not called ‘self-effacement’; they’re called ‘blissful meditations in this life’. 

It’s\marginnote{6.1} possible that some mendicant, with the fading away of rapture, might enter and remain in the third absorption, where they meditate with equanimity, mindful and aware, personally experiencing the bliss of which the noble ones declare, ‘Equanimous and mindful, one meditates in bliss.’ They might think they’re practicing self-effacement. But in the training of the Noble One these are not called ‘self-effacement’; they’re called ‘blissful meditations in this life’. 

It’s\marginnote{7.1} possible that some mendicant, with the giving up of pleasure and pain, and the ending of former happiness and sadness, might enter and remain in the fourth absorption, without pleasure or pain, with pure equanimity and mindfulness. They might think they’re practicing self-effacement. But in the training of the Noble One these are not called ‘self-effacement’; they’re called ‘blissful meditations in this life’. 

It’s\marginnote{8.1} possible that some mendicant, going totally beyond perceptions of form, with the ending of perceptions of impingement, not focusing on perceptions of diversity, aware that ‘space is infinite’, might enter and remain in the dimension of infinite space. They might think they’re practicing self-effacement. But in the training of the Noble One these are not called ‘self-effacement’; they’re called ‘peaceful meditations’. 

It’s\marginnote{9.1} possible that some mendicant, going totally beyond the dimension of infinite space, aware that ‘consciousness is infinite’, might enter and remain in the dimension of infinite consciousness. They might think they’re practicing self-effacement. But in the training of the Noble One these are not called ‘self-effacement’; they’re called ‘peaceful meditations’. 

It’s\marginnote{10.1} possible that some mendicant, going totally beyond the dimension of infinite consciousness, aware that ‘there is nothing at all’, might enter and remain in the dimension of nothingness. They might think they’re practicing self-effacement. But in the training of the Noble One these are not called ‘self-effacement’; they’re called ‘peaceful meditations’. 

It’s\marginnote{11.1} possible that some mendicant, going totally beyond the dimension of nothingness, might enter and remain in the dimension of neither perception nor non-perception. They might think they’re practicing self-effacement. But in the training of the Noble One these are not called ‘self-effacement’; they’re called ‘peaceful meditations’. 

\subsection*{1. The Exposition of Self-Effacement }

Now,\marginnote{12.1} Cunda, you should work on self-effacement in each of the following ways.\footnote{The phrase “work on self-effacement” (\textit{sallekho \textsanskrit{karaṇīyo}}) recurs in each clause but is abbreviated in translation. It echoes the phrase \textit{\textsanskrit{sikkhā} \textsanskrit{karaṇīyā}} (“this training should be done”) which ends each of the Vinaya “training rules”. Both cases emphasize the active application of effort. } 

‘Others\marginnote{12.2} will be cruel, but here we will not be cruel.’\footnote{The list of unskillful qualities forms the skeleton of the rest of the sutta, being repeated five times with variations in the presentation. The Pali has 44 items, while MA 91 has 31 and EA 47.9 has 16. I note where the items appear in standard doctrinal lists. | Non-cruelty is mentioned first as it is the foundation of good qualities and the first moral precept, to not kill. } 

‘Others\marginnote{12.3} will kill living creatures, but here we will not kill living creatures.’\footnote{These ten make up the ten ways of doing of skillful deeds. } 

‘Others\marginnote{12.4} will steal, but here we will not steal.’ 

‘Others\marginnote{12.5} will be unchaste, but here we will not be unchaste.’ 

‘Others\marginnote{12.6} will lie, but here we will not lie.’ 

‘Others\marginnote{12.7} will speak divisively, but here we will not speak divisively.’ 

‘Others\marginnote{12.8} will speak harshly, but here we will not speak harshly.’ 

‘Others\marginnote{12.9} will talk nonsense, but here we will not talk nonsense.’ 

‘Others\marginnote{12.10} will be covetous, but here we will not be covetous.’ 

‘Others\marginnote{12.11} will have ill will, but here we will not have ill will.’ 

‘Others\marginnote{12.12} will have wrong view, but here we will have right view.’\footnote{This item performs double-duty as both the last of the ten ways of doing skillful deeds and the first of the noble eight (or ten) fold path. } 

‘Others\marginnote{12.13} will have wrong thought, but here we will have right thought.’ 

‘Others\marginnote{12.14} will have wrong speech, but here we will have right speech.’ 

‘Others\marginnote{12.15} will have wrong action, but here we will have right action.’ 

‘Others\marginnote{12.16} will have wrong livelihood, but here we will have right livelihood.’ 

‘Others\marginnote{12.17} will have wrong effort, but here we will have right effort.’ 

‘Others\marginnote{12.18} will have wrong mindfulness, but here we will have right mindfulness.’ 

‘Others\marginnote{12.19} will have wrong immersion, but here we will have right immersion.’ 

‘Others\marginnote{12.20} will have wrong knowledge, but here we will have right knowledge.’\footnote{These two items are sometimes added to the eightfold path to make ten (eg. \href{https://suttacentral.net/mn117/en/sujato\#35.13}{MN 117:35.13}). They are missing in both parallels. } 

‘Others\marginnote{12.21} will have wrong freedom, but here we will have right freedom.’ 

‘Others\marginnote{12.22} will be overcome with dullness and drowsiness, but here we will be rid of dullness and drowsiness.’\footnote{This is the third of the five hindrances, and the final two follow in sequence. The first two of the hindrances appear above as “covetousness” and “ill will” in the ten ways of doing unskillful deeds. Notably, the same pattern occurs in MA 91, which speaks to a common ancestor on this point. This shows that these items were organized to eliminate repetition, but in other cases (such as the kinds of wrong speech and wrong action) items do repeat. } 

‘Others\marginnote{12.23} will be restless, but here we will not be restless.’ 

‘Others\marginnote{12.24} will have doubts, but here we will have gone beyond doubt.’ 

‘Others\marginnote{12.25} will be irritable, but here we will be without anger.’\footnote{These ten are included in the sixteen blemishes of \href{https://suttacentral.net/mn7/en/sujato}{MN 7}. } 

‘Others\marginnote{12.26} will be acrimonious, but here we will be without acrimony.’ 

‘Others\marginnote{12.27} will be offensive, but here we will be inoffensive.’ 

‘Others\marginnote{12.28} will be contemptuous, but here we will be without contempt.’ 

‘Others\marginnote{12.29} will be jealous, but here we will be without jealousy.’ 

‘Others\marginnote{12.30} will be stingy, but here we will be without stinginess.’ 

‘Others\marginnote{12.31} will be devious, but here we will not be devious.’ 

‘Others\marginnote{12.32} will be deceitful, but here we will not be deceitful.’ 

‘Others\marginnote{12.33} will be pompous, but here we will not be pompous.’ 

‘Others\marginnote{12.34} will be arrogant, but here we will not be arrogant.’ 

‘Others\marginnote{12.35} will be hard to admonish, but here we will not be hard to admonish.’\footnote{These three items are not part of a defined set, and in addition are absent from MA 91. } 

‘Others\marginnote{12.36} will have bad friends, but here we will have good friends.’ 

‘Others\marginnote{12.37} will be negligent, but here we will be diligent.’ 

‘Others\marginnote{12.38} will be faithless, but here we will have faith.’\footnote{These seven are the seven good (and bad) qualities of \href{https://suttacentral.net/mn53/en/sujato\#11.1}{MN 53:11.1}. } 

‘Others\marginnote{12.39} will be conscienceless, but here we will have a sense of conscience.’\footnote{These seven are also in \href{https://suttacentral.net/mn53/en/sujato}{MN 53}. } 

‘Others\marginnote{12.40} will be imprudent, but here we will be prudent.’ 

‘Others\marginnote{12.41} will be unlearned, but here we will be well learned.’ 

‘Others\marginnote{12.42} will be lazy, but here we will be energetic.’ 

‘Others\marginnote{12.43} will be unmindful, but here we will be mindful.’ 

‘Others\marginnote{12.44} will be witless, but here we will be accomplished in wisdom.’ 

‘Others\marginnote{12.45} will be attached to their own views, holding them tight, and refusing to let go, but here we will not be attached to our own views, not holding them tight, but will let them go easily.’\footnote{Typically this follows “corrupt wishes and wrong view” (eg. \href{https://suttacentral.net/an6.36/en/sujato\#2.5}{AN 6.36:2.5}, \href{https://suttacentral.net/dn33/en/sujato\#2.2.61}{DN 33:2.2.61}), thus is not part of a defined set in this sutta, and is also absent from MA 91. } 

\subsection*{2. Giving Rise to the Thought }

Cunda,\marginnote{13.1} I say that even giving rise to the thought of skillful qualities is very helpful, let alone following that path in body and speech.\footnote{This round foregrounds the “starting point”. Before we do anything, we think of doing it, and that in itself is powerful, as it sets the mind in a certain direction. For the Buddha, the mental intentions that underlie bodily behaviors are crucial, whereas for the Jains it is the bodily behaviors themselves that matter. } That’s why you should give rise to the following thoughts. ‘Others will be cruel, but here we will not be cruel.’ ‘Others will kill living creatures, but here we will not kill living creatures.’ … ‘Others will be attached to their own views, holding them tight, and refusing to let go, but here we will not be attached to our own views, not holding them tight, but will let them go easily.’ 

\subsection*{3. Bypassing }

Cunda,\marginnote{14.1} suppose there was a rough path and another smooth path to get around it.\footnote{When the path is rough, the Buddha encourages us to find a smoother path. The undertakings of the Buddhist path should be pleasant and lead to happiness. While the path can be difficult, the difficulty itself is of no value. This is in implicit contrast with the Jains, who gave themselves to “burning off” (\textit{tapas}) kamma through agonizing austerities. } Or suppose there was a rough ford and another smooth ford to get around it. In the same way, a cruel individual bypasses it by not being cruel. An individual who kills bypasses it by not killing. … 

An\marginnote{14.5} individual who is attached to their own views, holding them tight, and refusing to let go, gets around it by not being attached to their own views, not holding them tight, but letting them go easily. 

\subsection*{4. Going Up }

Cunda,\marginnote{15.1} all unskillful qualities lead downwards, while all skillful qualities lead upwards.\footnote{This round emphasizes the unifying quality of good and bad, so that any good deed, no matter how small or insignificant, leads in the same direction. } In the same way, a cruel individual is led upwards by not being cruel. An individual who kills is led upwards by not killing … An individual who is attached to their own views, holding them tight, and refusing to let go, is led upwards by not being attached to their own views, not holding them tight, but letting them go easily. 

\subsection*{5. The Exposition by Extinguishment }

If\marginnote{16.1} you’re sinking in the mud yourself, Cunda, it is quite impossible for you to pull out someone else who is sinking in the mud. But if you’re not sinking in the mud yourself, it is quite possible for you to pull out someone else who is sinking in the mud. If you’re not tamed, trained, and quenched yourself, it is quite impossible for you to help tame, train, and extinguish someone else. But if you are tamed, trained, and quenched yourself, it is quite possible for you to help tame, train, and extinguish someone else.\footnote{The Buddha only ever taught people to do what he himself had achieved, following the same path he had discovered. } 

In\marginnote{16.5} the same way, a cruel individual extinguishes it by not being cruel. An individual who kills extinguishes it by not killing. … 

An\marginnote{16.24} individual who is attached to their own views, holding them tight, and refusing to let go, extinguishes it by not being attached to their own views, not holding them tight, but letting them go easily. 

So,\marginnote{17.1} Cunda, I’ve taught the expositions by way of self-effacement, arising of thought, bypassing, going up, and extinguishing.\footnote{The Buddha lists the five rounds of the forty-four items. } Out of sympathy, I’ve done what a teacher should do who wants what’s best for their disciples. Here are these roots of trees, and here are these empty huts. Practice absorption, Cunda! Don’t be negligent! Don’t regret it later! This is my instruction.”\footnote{While this exhortation is addressed to Cunda, it is phrased in plural, indicating that it intended for the whole audience. } 

That\marginnote{17.4} is what the Buddha said. Satisfied, Venerable \textsanskrit{Mahācunda} approved what the Buddha said. 

\begin{verse}%
Forty-four\marginnote{17.6} items have been stated,\footnote{It is unusual for a summary verse (\textit{\textsanskrit{uddāna}}) to be added to an individual sutta. It is not mentioned in the commentary or subcommentary, nor is it found in the PTS edition, nor is there an equivalent in either Chinese parallel. It has obviously been added by redactors at some point. I translate it here due to its inherent interest; normally I leave the \textit{\textsanskrit{uddānas}} at the end of a chapter or other section untranslated, as they consist of little more than a list of the titles of suttas.  } \\
organized into five sections. \\
“Effacement” is the name of this discourse, \\
which is deep as the ocean. 

%
\end{verse}

%
\section*{{\suttatitleacronym MN 9}{\suttatitletranslation Right View }{\suttatitleroot Sammādiṭṭhisutta}}
\addcontentsline{toc}{section}{\tocacronym{MN 9} \toctranslation{Right View } \tocroot{Sammādiṭṭhisutta}}
\markboth{Right View }{Sammādiṭṭhisutta}
\extramarks{MN 9}{MN 9}

\scevam{So\marginnote{1.1} I have heard. }At one time the Buddha was staying near \textsanskrit{Sāvatthī} in Jeta’s Grove, \textsanskrit{Anāthapiṇḍika}’s monastery. There \textsanskrit{Sāriputta} addressed the mendicants:\footnote{The Chinese and Sanskrit parallels depict this discourse as a conversation between \textsanskrit{Sāriputta} and \textsanskrit{Mahākoṭṭhita}, a context that appears to have been lost in the Pali. The sutta proceeds by starting with a simple analysis of right view, gradually stepping into deeper waters. Much of the discourse is framed in terms of dependent origination, but it focuses on the dependently originated phenomena, rather than the process of causality (see \href{https://suttacentral.net/sn12.20/en/sujato}{SN 12.20} for this distinction). } “Reverends, mendicants!” 

“Reverend,”\marginnote{1.5} they replied. \textsanskrit{Sāriputta} said this: 

“Reverends,\marginnote{2.1} they speak of this thing called ‘right view’. How do you define a noble disciple who has right view, whose view is correct, who has experiential confidence in the teaching, and has come to the true teaching?”\footnote{In other words, a stream-enterer. In \href{https://suttacentral.net/mn141/en/sujato\#5.6}{MN 141:5.6}, \textsanskrit{Sāriputta} is said to teach students as far as as stream-entry, while his friend \textsanskrit{Moggallāna} leads them to arahantship. } 

“Reverend,\marginnote{2.3} we would travel a long way to learn the meaning of this statement in the presence of Venerable \textsanskrit{Sāriputta}. May Venerable \textsanskrit{Sāriputta} himself please clarify the meaning of this. The mendicants will listen and remember it.” 

“Well\marginnote{2.6} then, reverends, listen and apply your mind well, I will speak.” 

“Yes,\marginnote{2.7} reverend,” they replied. \textsanskrit{Sāriputta} said this: 

“A\marginnote{3.1} noble disciple understands the unskillful and its root, and the skillful and its root.\footnote{That is, the deed and the motivating force behind the deed. } When they’ve done this, they’re defined as a noble disciple who has right view, whose view is correct, who has experiential confidence in the teaching, and has come to the true teaching.\footnote{\textsanskrit{Sāriputta} had a deliberate, systematic, and unhurried approach to teaching. First he introduces a fundamental question, in this case stream-entry. Then he gives a simple and practical answer. Then he goes on to draw out implications both broad and deep. } 

But\marginnote{4.1} what is the unskillful and what is its root? And what is the skillful and what is its root? Killing living creatures, stealing, and sexual misconduct; speech that’s false, divisive, harsh, or nonsensical; and covetousness, ill will, and wrong view.\footnote{Note that covetousness, ill will, and wrong view are strong forms of greed, hate, and delusion respectively. } This is called the unskillful. 

And\marginnote{5.1} what is the root of the unskillful? Greed, hate, and delusion. This is called the root of the unskillful. 

And\marginnote{6.1} what is the skillful? Avoiding killing living creatures, stealing, and sexual misconduct; avoiding speech that’s false, divisive, harsh, or nonsensical; contentment, good will, and right view.\footnote{These are the “ten ways of doing skillful deeds” (\textit{\textsanskrit{dasakusalakammapathā}}). There is a detailed explanation at \href{https://suttacentral.net/mn41/en/sujato\#7.1}{MN 41:7.1}. } This is called the skillful. 

And\marginnote{7.1} what is the root of the skillful? Contentment, love, and understanding. This is called the root of the skillful. 

A\marginnote{8.1} noble disciple understands in this way the unskillful and its root, and the skillful and its root. They’ve completely given up the underlying tendency to greed, got rid of the underlying tendency to repulsion, and eradicated the underlying tendency to the view and conceit ‘I am’. They’ve given up ignorance and given rise to knowledge, and make an end of suffering in this very life.\footnote{This passage, which is repeated throughout the sutta, indicates the arahant. The phrasing is problematic, as it suggests that one has right view (stream-entry) only after fully relinquishing all defilements (arahantship). The commentary records a discussion of this problem, but in fact it is probably due to a textual corruption in the Pali text, as the parallels at MA 29 and SA 344 lack this passage. This shows how difficulties in the Pali text can sometimes lead to fruitless discussions in the absence of the broader context offered by parallels. } When they’ve done this, they’re defined as a noble disciple who has right view, whose view is correct, who has experiential confidence in the teaching, and has come to the true teaching.” 

Saying\marginnote{9.1} “Good, sir,” those mendicants approved and agreed with what \textsanskrit{Sāriputta} said. Then they asked another question: “But reverend, might there be another way to describe a noble disciple who has right view, whose view is correct, who has experiential confidence in the teaching, and has come to the true teaching?”\footnote{By giving a brief and simple answer to a profound question, \textsanskrit{Sāriputta} leaves room for the audience to request further explanation. This teaching method ensures audience engagement and conveys information in digestible doses. } 

“There\marginnote{10.1} might, reverends. A noble disciple understands fuel, its origin, its cessation, and the practice that leads to its cessation.\footnote{\textsanskrit{Sāriputta} introduces the framework of the four noble truths, but structured in terms of “fuel” rather than suffering. | “Fuel” (or “food” or “nutriment”, \textit{\textsanskrit{āhāra}}) refers to both the thing that acts as a condition, fuel, or support, as well as the internal grasping and attachment to that thing. Its use as a philosophical term appears to be an innovation by the Buddha, replacing the \textit{anna} so dear to Vedic seers. } When they’ve done this, they’re defined as a noble disciple who has right view, whose view is correct, who has experiential confidence in the teaching, and has come to the true teaching. 

But\marginnote{11.1} what is fuel? What is its origin, its cessation, and the practice that leads to its cessation? There are these four fuels. They maintain sentient beings that have been born and help those that are about to be born.\footnote{Also at \href{https://suttacentral.net/mn38/en/sujato\#15.1}{MN 38:15.1}, \href{https://suttacentral.net/sn12.11/en/sujato}{SN 12.11}, \href{https://suttacentral.net/sn12.12/en/sujato}{SN 12.12}, \href{https://suttacentral.net/sn12.63/en/sujato}{SN 12.63}, \href{https://suttacentral.net/sn12.64/en/sujato}{SN 12.64}; cp \href{https://suttacentral.net/snp1.8/en/sujato\#5.3}{Snp 1.8:5.3}. | “About to be born” is \textit{\textsanskrit{sambhavesī}}, which I follow Norman and Bodhi in reading as a future active participle, although the commentary takes it in the sense “seeking” to be born. Compare such Sanskrit constructions as Rig Veda 1.66.8, \textit{yamo ha \textsanskrit{jāto} yamo \textsanskrit{janitvaṁ}} (“the twin that is born and the twin about to be born”) and Śatapatha \textsanskrit{Brāhmaṇa} 2.3.1.24, \textit{\textsanskrit{bhūtaṁ} caiva \textsanskrit{bhaviṣyacca} \textsanskrit{jātaṁ} ca \textsanskrit{janiṣyamāṇaṁ}} (“has become and will be, born and to be born”). | The phrase is one of several in the suttas that appears to indicate an intermediate state between one life and the next, despite the fact that this view is formally rejected by the Theravada Abhidhamma (\href{https://suttacentral.net/kv8.2}{Kv 8.2}). } What four? Solid food, whether solid or subtle; contact is the second, mental intention the third, and consciousness the fourth.\footnote{Coarse solid food sustains the bodies of beings in the human and animal realms, while fine solid food sustains the gods and ancestors, a belief intertwined with the Vedic notion that the gods partake of the food offered in sacrifice (eg. Rig Veda 1.187). | “Contact” (\textit{phassa}) is the interaction between the inner and outer worlds, allowing us to situate ourselves in a sensory world full of fears and joys, stimulating feeling and hence the craving for more. | “Volition” (\textit{\textsanskrit{manosañcetanā}}) allows us to act in the world revealed by the senses and secure further “fuel”. | “Consciousness” (\textit{\textsanskrit{viññāṇa}}) is aware of all this, experiencing suffering, and giving rise to a new “name and form” in a future life in a fruitless search to find a world free of pain. Thus the four “fuels” can be considered as a distinctive perspective on dependent origination, which is expanded further in the subsequent items. } Fuel originates from craving. Fuel ceases when craving ceases. The practice that leads to the cessation of fuel is simply this noble eightfold path, that is:\footnote{The word \textit{\textsanskrit{āhāra}} (“fuel”, “food”, “nutriment”) means literally “intake”, and is etymologically parallel to \textit{\textsanskrit{upādāna}}, “grasping”, “uptake”. Both terms have dual senses, on the one hand denoting fuel or sustenance, and on the other grasping and attachment. That is why here (as at \href{https://suttacentral.net/mn38/en/sujato\#16.1}{MN 38:16.1}), \textit{\textsanskrit{āhāra}} is created by craving, just like \textit{\textsanskrit{upādāna}} in the standard sequence (\href{https://suttacentral.net/mn38/en/sujato\#17.8}{MN 38:17.8}). } right view, right thought, right speech, right action, right livelihood, right effort, right mindfulness, and right immersion. 

A\marginnote{12.1} noble disciple understands in this way fuel, its origin, its cessation, and the practice that leads to its cessation. They’ve completely given up the underlying tendency to greed, got rid of the underlying tendency to repulsion, and eradicated the underlying tendency to the view and conceit ‘I am’. They’ve given up ignorance and given rise to knowledge, and make an end of suffering in this very life. When they’ve done this, they’re defined as a noble disciple who has right view, whose view is correct, who has experiential confidence in the teaching, and has come to the true teaching.” 

Saying\marginnote{13.1} “Good, sir,” those mendicants … asked another question: “But reverend, might there be another way to describe a noble disciple who … has come to the true teaching?” 

“There\marginnote{14.1} might, reverends. A noble disciple understands suffering, its origin, its cessation, and the practice that leads to its cessation.\footnote{Here \textsanskrit{Sāriputta} gives the classic statement on the four noble truths in terms of suffering (\textit{dukkha}). } When they’ve done this, they’re defined as a noble disciple who … has come to the true teaching. But what is suffering? What is its origin, its cessation, and the practice that leads to its cessation? Rebirth is suffering; old age is suffering; death is suffering; sorrow, lamentation, pain, sadness, and distress are suffering; association with the disliked is suffering; separation from the liked is suffering; not getting what you wish for is suffering. In brief, the five grasping aggregates are suffering.\footnote{The definitions of the four noble truths are direct quotes from the Buddha’s first sermon, the Dhammacakkappavattanasutta (\href{https://suttacentral.net/sn56.11/en/sujato\#4.2}{SN 56.11:4.2}). They are notably emphasized by \textsanskrit{Sāriputta}, who delves even further into them (\href{https://suttacentral.net/mn28/en/sujato\#3.2}{MN 28:3.2}, \href{https://suttacentral.net/mn141/en/sujato\#10.2}{MN 141:10.2}), as he does later in this sutta also. } This is called suffering. And what is the origin of suffering? It’s the craving that leads to future lives, mixed up with relishing and greed, taking pleasure wherever it lands. That is,\footnote{This definition clarifies two common misunderstandings. First, not all desire causes suffering, for some kinds of desire lead out of suffering (\href{https://suttacentral.net/sn51.15/en/sujato\#4.1}{SN 51.15:4.1}). Second, rebirth is not a tangential part of the Buddha’s teachings, uncritically inherited from cultural suppositions; it is baked into the fundamental meaning of the four noble truths. | The idiom \textit{tatratatra} (literally “there, there”) is not selective (“here and there”) but distributive (“everywhere”). By taking pleasure in sense experience, the mind binds itself to the need for continued stimulation in future lives. The distancing sense “there” is important, for craving does not merely satisfy itself with what it currently experiences, but must always seek out renewed gratification. For an \textsanskrit{Upaniṣadic} precursor, see the note to the same idiom at \href{https://suttacentral.net/mn2/en/sujato\#8.8}{MN 2:8.8}. } craving for sensual pleasures, craving for continued existence, and craving to end existence.\footnote{Craving is a primal desire or hunger, often unconscious. | “Craving for sensual pleasures” (\textit{\textsanskrit{kāmataṇhā}}): the desire to experience pleasure through any of the five senses. | “Craving for continued existence” (\textit{\textsanskrit{bhavataṇhā}}): the desire to continue living in some form after death. | “Craving to end existence” (\textit{\textsanskrit{vibhavataṇhā}}): the desire to annihilate the self. } This is called the origin of suffering. And what is the cessation of suffering? It’s the fading away and cessation of that very same craving with nothing left over; giving it away, letting it go, releasing it, and not clinging to it.\footnote{These terms are used widely in different senses, but here they are all synonyms of “extinction” (\textit{\textsanskrit{nibbāna}}). | Note especially the appearance of \textit{\textsanskrit{cāga}} (“giving”) here. Giving is regarded as the most basic foundation of moral practice, yet even in its simplest form it partakes of the same nature as Nibbana. } This is called the cessation of suffering. And what is the practice that leads to the cessation of suffering? It is simply this noble eightfold path, that is: right view … right immersion. This is called the practice that leads to the cessation of suffering. 

A\marginnote{19.1} noble disciple understands in this way suffering, its origin, its cessation, and the practice that leads to its cessation. They’ve completely given up the underlying tendency to greed, got rid of the underlying tendency to repulsion, and eradicated the underlying tendency to the view and conceit ‘I am’. They’ve given up ignorance and given rise to knowledge, and make an end of suffering in this very life. When they’ve done this, they’re defined as a noble disciple who has right view, whose view is correct, who has experiential confidence in the teaching, and has come to the true teaching.” 

Saying\marginnote{20.1} “Good, sir,” those mendicants … asked another question: “But reverend, might there be another way to describe a noble disciple who … has come to the true teaching?” 

“There\marginnote{21.1} might, reverends. A noble disciple understands old age and death, their origin, their cessation, and the practice that leads to their cessation …\footnote{This is a nice example of \textsanskrit{Sāriputta}’s analytical genius. He picks up “old age and death” from the explanation of “suffering” and treats it within the framework of the four noble truths. This creates continuity with the previous sections, as well as situating it within the Buddha’s teachings as a whole, since all footprints fit in an elephant’s footprint (\href{https://suttacentral.net/mn28/en/sujato\#2.1}{MN 28:2.1}). Further, he \emph{deepens} understanding by applying the four noble truths recursively, bringing small details into close focus. Meanwhile, this item, following on from “suffering”, leads into the familiar sequence of items from dependent origination. Thus he simultaneously \emph{broadens} the scope by linking the four noble truths with dependent origination. } But what are old age and death? What is their origin, their cessation, and the practice that leads to their cessation? The old age, decrepitude, broken teeth, gray hair, wrinkly skin, diminished vitality, and failing faculties of the various sentient beings in the various orders of sentient beings.\footnote{\textit{\textsanskrit{Jarā}} is neither the process of getting old (“ageing”), nor a psychological metaphor, but the physical state of being old (“old age”). The same applies to the definitions of “death” and “rebirth”. } This is called old age. And what is death?\footnote{Text lacks the expected question “what is old age?” on the previous section. } The passing away, perishing, disintegration, demise, mortality, death, decease, breaking up of the aggregates, laying to rest of the corpse, and cutting off of the life faculty of the various sentient beings in the various orders of sentient beings. This is called death. Such is old age, and such is death. This is called old age and death. Old age and death originate from rebirth. Old age and death cease when rebirth ceases. The practice that leads to the cessation of old age and death is simply this noble eightfold path …”\footnote{\textit{\textsanskrit{Jāti}} refers to ongoing transmigration into new lives, “rebirth”. The only way to escape old age and death is to not be reborn. } 

“Might\marginnote{24.1} there be another way to describe a noble disciple?” 

“There\marginnote{25.1} might, reverends. A noble disciple understands rebirth, its origin, its cessation, and the practice that leads to its cessation … But what is rebirth? What is its origin, its cessation, and the practice that leads to its cessation? The rebirth, inception, conception, reincarnation, manifestation of the aggregates, and acquisition of the sense fields of the various sentient beings in the various orders of sentient beings. This is called rebirth. Rebirth originates from continued existence. Rebirth ceases when continued existence ceases. The practice that leads to the cessation of rebirth is simply this noble eightfold path …” 

“Might\marginnote{28.1} there be another way to describe a noble disciple?” 

“There\marginnote{29.1} might, reverends. A noble disciple understands continued existence, its origin, its cessation, and the practice that leads to its cessation.\footnote{\textit{Bhava} means “being, existence, life” in the sense of “past and future lives”. The concept of \textit{bhava} includes both active and resultant dimensions of life. Consider it by analogy with, say, visiting a park. First there is the spark of an idea, a vague impression of a “park” in the mind, which solidifies into an intention. Acting on it, one goes to the park, where it becomes ones’ reality, the “world” one inhabits. Likewise, by creating ideas and volitions in the mind, one is aligning or tuning into the corresponding realm of existence, priming the mind to project itself into that state and make it a reality when one is reborn there. } But what is continued existence? What is its origin, its cessation, and the practice that leads to its cessation? There are these three states of continued existence.\footnote{The idea of three “worlds” (\textit{loka}) or “states of existence” (\textit{bhava}), variously defined, is shared between Buddhism, Hinduism, and Jainism. It originally referred to the earth (\textit{\textsanskrit{bhū}}, \textit{\textsanskrit{pṛthivi}}), the midspace (\textit{\textsanskrit{antarikṣa}}), and the heavens or sky (\textit{diva}, \textit{svarga}), the respective abodes of humanity, the ancestors, and the gods. In the mythology underlying the Vedas they were formed by Indra’s heroic deeds: empowered by soma, he first slew the dragon \textsanskrit{Vṛtra} who bound the world in a mass of darkness, then he separated earth from sky, leaving the midspace between (Rig Veda 2.15), thus creating the visible or intelligible world. } Existence in the sensual realm, the realm of luminous form, and the formless realm.\footnote{The “sensual realm” encompasses all realms, including the human, from the lowest hell to the highest of the sensual heavens, the gods who control what is imagined by others. | “Luminous form” (\textit{\textsanskrit{rūpa}}) refers to the \textsanskrit{Brahmā} realms attained through the luminous mind of \textit{\textsanskrit{jhāna}}. | The “formless” realms are attained through the formless meditations. } Continued existence originates from grasping. Continued existence ceases when grasping ceases. The practice that leads to the cessation of continued existence is simply this noble eightfold path …” 

“Might\marginnote{32.1} there be another way to describe a noble disciple?” 

“There\marginnote{33.1} might, reverends. A noble disciple understands grasping, its origin, its cessation, and the practice that leads to its cessation …\footnote{“Grasping” (\textit{\textsanskrit{upādāna}}) has the active sense of “taking up” a new life, not just “clinging” to what one has. As noted above, it has a dual sense, because, like \textit{\textsanskrit{āhāra}}, it also means the “fuel” that sustains the fire of existence. } But what is grasping? What is its origin, its cessation, and the practice that leads to its cessation? There are these four kinds of grasping. Grasping at sensual pleasures, views, precepts and observances, and theories of a self.\footnote{Grasping begins with the primal desire of the senses, but the three other graspings are rather intellectual and sophisticated. Only a grown human being with a developed linguistic ability is able to formulate a view to become attached to, and likewise with attachment to religious observances and vows, and to theories of a self. } Grasping originates from craving. Grasping ceases when craving ceases. The practice that leads to the cessation of grasping is simply this noble eightfold path …” 

“Might\marginnote{36.1} there be another way to describe a noble disciple?” 

“There\marginnote{37.1} might, reverends. A noble disciple understands craving, its origin, its cessation, and the practice that leads to its cessation … But what is craving? What is its origin, its cessation, and the practice that leads to its cessation? There are these six classes of craving. Craving for sights, sounds, smells, tastes, touches, and ideas.\footnote{We have met the three kinds of craving above, while here craving is analyzed in terms of the six senses, which relates it to the items to come. | \textit{\textsanskrit{Dhammataṇhā}} is idiomatically translated as “craving for ideas”, although its scope is broader than just discursive thinking. It refers to any kind of craving for mental phenomena, which includes, for example, craving for the mental pleasure of deep meditation. } Craving originates from feeling. Craving ceases when feeling ceases. The practice that leads to the cessation of craving is simply this noble eightfold path …” 

“Might\marginnote{40.1} there be another way to describe a noble disciple?” 

“There\marginnote{41.1} might, reverends. A noble disciple understands feeling, its origin, its cessation, and the practice that leads to its cessation …\footnote{“Feeling” is \textit{\textsanskrit{vedanā}}, from the root \textit{vid} (“to experience”) and having the applied meaning of “hedonic tone”, whether pleasant, painful, or neutral. } But what is feeling? What is its origin, its cessation, and the practice that leads to its cessation? There are these six classes of feeling. Feeling born of contact through the eye, ear, nose, tongue, body, and mind. Feeling originates from contact. Feeling ceases when contact ceases. The practice that leads to the cessation of feeling is simply this noble eightfold path …” 

“Might\marginnote{44.1} there be another way to describe a noble disciple?” 

“There\marginnote{45.1} might, reverends. A noble disciple understands contact, its origin, its cessation, and the practice that leads to its cessation …\footnote{“Contact” is \textit{phassa}, otherwise  “touch”, or “impingement”. It occurs when sense stimulus meets sense organ in experience. } But what is contact? What is its origin, its cessation, and the practice that leads to its cessation? There are these six classes of contact. Contact through the eye, ear, nose, tongue, body, and mind.\footnote{Contact through the physical senses occurs by way of impingement or physical resistance, the impact of a stimulating energy with the sense organ. Mental contact occurs by way of designation or conceptualization. } Contact originates from the six sense fields. Contact ceases when the six sense fields cease. The practice that leads to the cessation of contact is simply this noble eightfold path …” 

“Might\marginnote{48.1} there be another way to describe a noble disciple?” 

“There\marginnote{49.1} might, reverends. A noble disciple understands the six sense fields, their origin, their cessation, and the practice that leads to their cessation …\footnote{“Field” is \textit{\textsanskrit{āyatana}}, literally a “stretching out”, i.e. a field or dimension. } But what are the six sense fields? What is their origin, their cessation, and the practice that leads to their cessation? There are these six sense fields. The sense fields of the eye, ear, nose, tongue, body, and mind.\footnote{In Buddhism, the mind is the “sixth sense”, which knows mental phenomena just as physical senses know physical phenomena. } The six sense fields originate from name and form. The six sense fields cease when name and form cease. The practice that leads to the cessation of the six sense fields is simply this noble eightfold path …” 

“Might\marginnote{52.1} there be another way to describe a noble disciple?” 

“There\marginnote{53.1} might, reverends. A noble disciple understands name and form, their origin, their cessation, and the practice that leads to their cessation …\footnote{“Name and form” (\textit{\textsanskrit{nāmarūpa}}) is a Vedic concept referring to the multiplicity of material forms (\textit{\textsanskrit{rūpa}}) and associated names (\textit{\textsanskrit{nāma}}), especially the individual “sentient organisms” such as gods and humans (Rig Veda 5.43.10, \textsanskrit{Bṛhadāraṇyaka} \textsanskrit{Upaniṣad} 1.6.1), which are ultimately absorbed into the divine, like rivers in the ocean (\textsanskrit{Muṇḍaka} \textsanskrit{Upaniṣad} 3.2.8, \textsanskrit{Praśna} \textsanskrit{Upaniṣad} 6.5). } But what are name and form? What is their origin, their cessation, and the practice that leads to their cessation? Feeling, perception, intention, contact, and application of mind—\footnote{Name and form are treated analytically (see also \href{https://suttacentral.net/dn15/en/sujato\#20.8}{DN 15:20.8}, \href{https://suttacentral.net/sn12.2/en/sujato\#11.1}{SN 12.2:11.1}). This brings them out of the world of metaphysics and theology and into the realm of mindful experience and rational inquiry. These five factors make it possible for consciousness to function. } this is called name. The four primary elements, and form derived from the four primary elements—\footnote{The primary elements are earth, water, fire, and air, corresponding to the modern concepts of solid, liquid, plasma, and gas. | “Derived” form is not explained in the suttas, but one passage indicates it includes “space” (\href{https://suttacentral.net/mn28/en/sujato\#26.1}{MN 28:26.1}). The Abhidhamma traditions explain it as including the objects of the senses, the subtle matter that receives sense impressions, and various other material phenomena. } this is called form. Such is name and such is form. This is called name and form. Name and form originate from consciousness. Name and form cease when consciousness ceases. The practice that leads to the cessation of name and form is simply this noble eightfold path …” 

“Might\marginnote{56.1} there be another way to describe a noble disciple?” 

“There\marginnote{57.1} might, reverends. A noble disciple understands consciousness, its origin, its cessation, and the practice that leads to its cessation …\footnote{“Consciousness” (\textit{\textsanskrit{viññāṇa}}) is simple subjective awareness, the sense of knowing. It is the subjective awareness that makes the entire multiform world of concepts and appearances possible. } But what is consciousness? What is its origin, its cessation, and the practice that leads to its cessation? There are these six classes of consciousness. Eye, ear, nose, tongue, body, and mind consciousness.\footnote{External sense consciousness is the sheer awareness of a physical property. The eye, for example, is only aware of light. The mind, relying on the five factors of “name”, processes this data into meaningful concepts and ideas. } Consciousness originates from choices. Consciousness ceases when choices cease. The practice that leads to the cessation of consciousness is simply this noble eightfold path …” 

“Might\marginnote{59.2} there be another way to describe a noble disciple?” 

“There\marginnote{61.1} might, reverends. A noble disciple understands choices, their origin, their cessation, and the practice that leads to their cessation …\footnote{This is the final item in the Chinese and Sanskrit parallels. } But what are choices? What is their origin, their cessation, and the practice that leads to their cessation?\footnote{\textit{\textsanskrit{Saṅkhāra}} in early Buddhism has three main doctrinal senses. (1) The broadest sense is “conditioned phenomena”, which essentially means “everything except Nibbana”. (2) Sometimes it is a physical or mental “process” or “activity”, such as vitality or the breath. (3) In the five aggregates and dependent origination it has the sense of “morally potent volitions or choices” and is a synonym for \textit{\textsanskrit{cetanā}} (“intention”) or \textit{kamma} (“deed”). It is defined as good, bad, and imperturbable choices (\href{https://suttacentral.net/dn33/en/sujato\#1.10.77}{DN 33:1.10.77}, \href{https://suttacentral.net/sn12.51/en/sujato\#9.4}{SN 12.51:9.4}), the latter of which refers to the kamma of the fourth \textit{\textsanskrit{jhāna}} and above. In this sense it is the moral “choices” for good or ill that propel consciousness into a new rebirth. } There are these three kinds of choice. Choices by way of body, speech, and mind.\footnote{This set of three refers to deeds carried out through the body, speech, and mind (eg. \href{https://suttacentral.net/mn57/en/sujato\#8.2}{MN 57:8.2}). “Mind” is \textit{citta}, while the Sanskrit \textsanskrit{Dṛṣṭisampannasūtra} has \textit{mano} in the parallel passage (\href{https://suttacentral.net/sf172/san/tripathi\#tri23.17b}{SF 172}). In dependent origination these are synonyms (\href{https://suttacentral.net/sn12.61/en/sujato\#2.1}{SN 12.61:2.1}). Not to be confused with the same terms when used in the context of mindfulness of breathing (\href{https://suttacentral.net/sn41.6/en/sujato\#1.5}{SN 41.6:1.5}). } Choices originate from ignorance. Choices cease when ignorance ceases. The practice that leads to the cessation of choices is simply this noble eightfold path …” 

“Might\marginnote{64.1} there be another way to describe a noble disciple?” 

“There\marginnote{64.3} might, reverends. A noble disciple understands ignorance, its origin, its cessation, and the practice that leads to its cessation …\footnote{“Ignorance” is the only item not found in the Chinese and Sanskrit parallels. Perhaps the original form of the sutta treated ignorance solely under its inverse, which is right view itself. Thus the lack of ignorance as a separate item could have been deliberate, emphasizing how the entirety of right view, encompassing all items in the discourse, implies the opposite of ignorance. } But what is ignorance? What is its origin, its cessation, and the practice that leads to its cessation? Not knowing about suffering, the origin of suffering, the cessation of suffering, and the practice that leads to the cessation of suffering. This is called ignorance. Ignorance originates from defilement. Ignorance ceases when defilement ceases. The practice that leads to the cessation of ignorance is simply this noble eightfold path …”\footnote{The current presentation is unique, as ignorance is itself one of the defilements (\textit{\textsanskrit{āsava}}). Ignorance is normally the final item in dependent origination, the “head” from which all follows (\href{https://suttacentral.net/snp5.1/en/sujato\#51.1}{Snp 5.1:51.1}). Nonetheless, causality is not a linear process, so a starting point is identified only for practical reasons. At \href{https://suttacentral.net/an10.62/en/sujato\#2.4}{AN 10.62:2.4} the five hindrances are said to be the “fuel” for ignorance. } 

Saying\marginnote{68.1} “Good, sir,” those mendicants approved and agreed with what \textsanskrit{Sāriputta} said. Then they asked another question: “But reverend, might there be another way to describe a noble disciple who has right view, whose view is correct, who has experiential confidence in the teaching, and has come to the true teaching?” 

“There\marginnote{69.1} might, reverends. A noble disciple understands defilement, its origin, its cessation, and the practice that leads to its cessation.\footnote{The appearance of “defilement” (\textit{\textsanskrit{āsava}}) at the end is appropriate, as its appearance in the four noble truths formula usually signifies arahantship. Nonetheless, the Chinese and Sanskrit parallels place it much earlier, after the four fuels. The sequence of the remaining items is mostly consistent between parallels. } When they’ve done this, they’re defined as a noble disciple who has right view, whose view is correct, who has experiential confidence in the teaching, and has come to the true teaching. 

But\marginnote{70.1} what is defilement? What is its origin, its cessation, and the practice that leads to its cessation? There are these three defilements. The defilements of sensuality, desire to be reborn, and ignorance. Defilement originates from ignorance. Defilement ceases when ignorance ceases. The practice that leads to the cessation of defilement is simply this noble eightfold path, that is:\footnote{Ignorance and defilements are locked in a cycle of mutual dependent conditioning: ignorance begets more ignorance. Compare the mutual conditioning of consciousness with name and form at \href{https://suttacentral.net/dn15/en/sujato\#20.1}{DN 15:20.1}. One implication of this is that no first point of ignorance and hence no first point of transmigration can be known. } right view, right thought, right speech, right action, right livelihood, right effort, right mindfulness, and right immersion. 

A\marginnote{71.1} noble disciple understands in this way defilement, its origin, its cessation, and the practice that leads to its cessation. They’ve completely given up the underlying tendency to greed, got rid of the underlying tendency to repulsion, and eradicated the underlying tendency to the view and conceit ‘I am’. They’ve given up ignorance and given rise to knowledge, and make an end of suffering in this very life. When they’ve done this, they’re defined as a noble disciple who has right view, whose view is correct, who has experiential confidence in the teaching, and has come to the true teaching.” 

This\marginnote{71.3} is what Venerable \textsanskrit{Sāriputta} said. Satisfied, the mendicants approved what \textsanskrit{Sāriputta} said. 

%
\section*{{\suttatitleacronym MN 10}{\suttatitletranslation Mindfulness Meditation }{\suttatitleroot Mahāsatipaṭṭhānasutta}}
\addcontentsline{toc}{section}{\tocacronym{MN 10} \toctranslation{Mindfulness Meditation } \tocroot{Mahāsatipaṭṭhānasutta}}
\markboth{Mindfulness Meditation }{Mahāsatipaṭṭhānasutta}
\extramarks{MN 10}{MN 10}

\scevam{So\marginnote{1.1} I have heard.\footnote{This discourse is copied at \href{https://suttacentral.net/dn22/en/sujato}{DN 22}, where the section on the four noble truths has been expanded with material mostly drawn from \href{https://suttacentral.net/mn141/en/sujato}{MN 141}. These discourses are the most influential texts for modern Theravada meditation, prompting countless modern commentaries. Comparative study of the several parallel versions reveals that this discourse, while comprised almost entirely of early material, was compiled in this form as one of the latest texts in the Pali suttas. | While mindfulness is always useful (\href{https://suttacentral.net/sn46.53/en/sujato\#15.4}{SN 46.53:15.4}), the “establishment of mindfulness” (\textit{\textsanskrit{satipaṭṭhāna}}) refers especially to a conscious development of contemplative practices based on mindfulness, i.e. “mindfulness meditation” or simply “meditation”. } }At one time the Buddha was staying in the land of the Kurus, near the Kuru town named \textsanskrit{Kammāsadamma}. There the Buddha addressed the mendicants, “Mendicants!” 

“Venerable\marginnote{1.5} sir,” they replied. The Buddha said this: 

“Mendicants,\marginnote{2.1} the four kinds of mindfulness meditation are the path to convergence. They are in order to purify sentient beings, to get past sorrow and crying, to make an end of pain and sadness, to discover the system, and to realize extinguishment.\footnote{The phrase \textit{\textsanskrit{ekāyano} maggo} (“path to convergence”) is given multiple meanings in commentaries and ancient translations. Outside of \textit{\textsanskrit{satipaṭṭhāna}}, it is used in only one context in Pali, where it means to “come together with” (\href{https://suttacentral.net/mn12/en/sujato\#37.5}{MN 12:37.5}). At \href{https://suttacentral.net/sn47.18/en/sujato\#3.4}{SN 47.18:3.4} the phrase is spoken by \textsanskrit{Brahmā}, which suggests it was a Brahmanical term. At \textsanskrit{Bṛhadāraṇyaka} \textsanskrit{Upaniṣad} 2.4.11—a passage full of details shared with the suttas—it means a place where things unite or converge. Thus \textit{\textsanskrit{satipaṭṭhāna}} leads to everything “coming together as one”. In other words, as seventh factor of the noble eightfold path, it leads to \textit{\textsanskrit{samādhi}}, the eighth factor (\href{https://suttacentral.net/sn45.1/en/sujato\#3.9}{SN 45.1:3.9}; see also \href{https://suttacentral.net/mn44/en/sujato\#12.3}{MN 44:12.3}). } 

What\marginnote{3.1} four? It’s when a mendicant meditates by observing an aspect of the body—keen, aware, and mindful, rid of covetousness and displeasure for the world.\footnote{The idiom \textit{\textsanskrit{kāye} \textsanskrit{kāyānupassī}}, literally “one who observes a body in the body” refers to focusing on a specific aspect of embodied experience, such as the breath, the postures, etc. | “Keen” (or “ardent”, \textit{\textsanskrit{ātāpī}}) implies effort, while “aware” (\textit{\textsanskrit{sampajāno}}) is the wisdom of understanding situation and context. | “Covetousness and displeasure” (\textit{\textsanskrit{abhijjhādomanassaṁ}}) are the strong forms of desire and aversion that are overcome by sense restraint in preparation for meditation. } They meditate observing an aspect of feelings—keen, aware, and mindful, rid of covetousness and displeasure for the world.\footnote{“Feelings” (\textit{\textsanskrit{vedanā}}) are the basic tones of pleasant, painful, or neutral, not the complexes we call “emotions”. } They meditate observing an aspect of the mind—keen, aware, and mindful, rid of covetousness and displeasure for the world.\footnote{“Mind” (\textit{citta}) is simple awareness. In meditation contexts, “mind” is often similar in meaning to \textit{\textsanskrit{samādhi}}. } They meditate observing an aspect of principles—keen, aware, and mindful, rid of covetousness and displeasure for the world.\footnote{“Principles” (\textit{\textsanskrit{dhammā}}) are the natural “systems” of cause and effect that underlie the “teachings”. The renderings “mind objects” or “mental qualities” are incorrect, as many of the things spoken of in this section are neither mind objects nor mental qualities. “Phenomena” is a possible translation, but the emphasis is not on the “appearance” of things, but on the “principles” governing their conditional relations. } 

\subsection*{1. Observing the Body }

\subsubsection*{1.1. Mindfulness of Breathing }

And\marginnote{4.1} how does a mendicant meditate observing an aspect of the body? 

It’s\marginnote{4.2} when a mendicant—gone to a wilderness, or to the root of a tree, or to an empty hut—sits down cross-legged, sets their body straight, and establishes mindfulness in their presence.\footnote{The situation here—a mendicant gone to the forest—establishes that this practice takes place in the wider context of the Gradual Training. Indeed, this whole sutta can be understood as an expansion of this phrase, mentioned briefly at \href{https://suttacentral.net/dn2/en/sujato\#67.3}{DN 2:67.3}. } Just mindful, they breathe in. Mindful, they breathe out.\footnote{The most fundamental meditation instruction. Notice how the Buddha phrases it: not “concentrate on the breath” as an object, but rather “breathing” as an activity to which one brings mindfulness. The stages of breath meditation are not meant to be done deliberately, but to be observed and understood as the natural process of deepening meditation. } 

Breathing\marginnote{4.4} in heavily they know: ‘I’m breathing in heavily.’ Breathing out heavily they know: ‘I’m breathing out heavily.’\footnote{In the beginning, the breath is somewhat rough and coarse. The Pali idiom is “long” and “short” breath, but in English we usually say to breathe “heavily” or “lightly”. } 

When\marginnote{4.5} breathing in lightly they know: ‘I’m breathing in lightly.’ Breathing out lightly they know: ‘I’m breathing out lightly.’\footnote{Over time, the breath becomes more subtle and soft. } 

They\marginnote{4.6} practice like this: ‘I’ll breathe in experiencing the whole body.’ They practice like this: ‘I’ll breathe out experiencing the whole body.’\footnote{Contextually the idiom “whole body” (\textit{\textsanskrit{sabbakāya}}) here refers to the breath, marking the fuller and more continuous awareness that arises with tranquility. Some practitioners, however, interpret it as the “whole physical body”, broadening awareness to encompass the movement and settling of energies throughout the body. } 

They\marginnote{4.7} practice like this: ‘I’ll breathe in stilling the physical process.’ They practice like this: ‘I’ll breathe out stilling the physical process.’\footnote{The “physical process” (\textit{\textsanskrit{kāyasaṅkhāraṁ}}) is the breath itself (\href{https://suttacentral.net/sn41.6/en/sujato\#1.8}{SN 41.6:1.8}). This can become so soft as to be imperceptible. } 

It’s\marginnote{4.8} like a deft carpenter or carpenter’s apprentice. When making a deep cut they know: ‘I’m making a deep cut,’ and when making a shallow cut they know: ‘I’m making a shallow cut.’\footnote{Text has “long” and “short”, but “deep” and “shallow” or “heavy” and “light” are more idiomatic for describing the breath in English. } 

And\marginnote{5.1} so they meditate observing an aspect of the body internally, externally, and both internally and externally.\footnote{“Internally” is one’s own body, “externally” the bodies of others, or external physical phenomena. This distinction is applied broadly in Buddhist meditation, but it is more relevant in some contexts than others. In the case of the breath, one is focusing on one’s own breath, but when contemplating, say, a dead body, or the material elements, there is more of an external dimension. Starting with “me” in here and the “world” out there, this practice dissolves this distinction so that we see we are of the same nature as everything else. } They meditate observing the body as liable to originate, as liable to vanish, and as liable to both originate and vanish.\footnote{This is the \textit{\textsanskrit{vipassanā}} (“insight” or “discernment”) dimension of meditation, observing not just the rise and fall of phenomena, but also their conditioned “nature” as being “liable” (\textit{-dhamma}) to impermanence. The meaning of this passage is explained at (\href{https://suttacentral.net/sn47.42/en/sujato}{SN 47.42}). Apart from these passages, \textit{\textsanskrit{vipassanā}} in \textit{\textsanskrit{satipaṭṭhāna}} pertains specially to the observation of principles. } Or mindfulness is established that the body exists, to the extent necessary for knowledge and mindfulness. They meditate independent, not grasping at anything in the world.\footnote{Mindfulness meditation leads to a range of knowledges as detailed by Anuruddha at \href{https://suttacentral.net/sn52.6/en/sujato}{SN 52.6} and \href{https://suttacentral.net/sn52.11/en/sujato}{SN 52.11}–24. An arahant is “independent” of any attachment (eg. \href{https://suttacentral.net/mn143/en/sujato}{MN 143}), but \textit{\textsanskrit{satipaṭṭhāna}} is also taught to give up dependency on views of the past and future (\href{https://suttacentral.net/dn29/en/sujato\#40.1}{DN 29:40.1}). } 

That’s\marginnote{5.4} how a mendicant meditates by observing an aspect of the body. 

\subsubsection*{1.2. The Postures }

Furthermore,\marginnote{6.1} when a mendicant is walking they know: ‘I am walking.’ When standing they know: ‘I am standing.’ When sitting they know: ‘I am sitting.’ And when lying down they know: ‘I am lying down.’\footnote{In early Pali, this practice is found only in the two \textsanskrit{Satipaṭṭhānasuttas} and the closely related \textsanskrit{Kāyagatāsatisutta} (\href{https://suttacentral.net/mn119/en/sujato}{MN 119}). It is practiced by developing a reflexive awareness of one’s posture and activity as it proceeds, often assisted by moving slowly and carefully. } Whatever posture their body is in, they know it. 

And\marginnote{7.1} so they meditate observing an aspect of the body internally, externally, and both internally and externally. They meditate observing the body as liable to originate, as liable to vanish, and as liable to both originate and vanish. Or mindfulness is established that the body exists, to the extent necessary for knowledge and mindfulness. They meditate independent, not grasping at anything in the world. 

That\marginnote{7.4} too is how a mendicant meditates by observing an aspect of the body. 

\subsubsection*{1.3. Situational Awareness }

Furthermore,\marginnote{8.1} a mendicant acts with situational awareness when going out and coming back; when looking ahead and aside; when bending and extending the limbs; when bearing the outer robe, bowl and robes; when eating, drinking, chewing, and tasting; when urinating and defecating; when walking, standing, sitting, sleeping, waking, speaking, and keeping silent.\footnote{“Situational awareness” (\textit{\textsanskrit{sampajañña}}) understands the context and purpose of activities. The main examples here illustrate the activities of daily monastic life: leaving the monastery on almsround, restraint while in the town, care wearing the robes, then mindfully eating and going to the toilet. } 

And\marginnote{9.1} so they meditate observing an aspect of the body internally … 

That\marginnote{9.2} too is how a mendicant meditates by observing an aspect of the body. 

\subsubsection*{1.4. Focusing on the Repulsive }

Furthermore,\marginnote{10.1} a mendicant examines their own body, up from the soles of the feet and down from the tips of the hairs, wrapped in skin and full of many kinds of filth.\footnote{This practice is intended to counter sexual desire and obsession. The primary focus is on one’s own body, rather than another’s body, although that can be brought in also. By focusing on aspects of our body that we normally prefer to ignore, we move towards a healthy sense of acceptance and neutrality towards our body. } ‘In this body there is head hair, body hair, nails, teeth, skin, flesh, sinews, bones, bone marrow, kidneys, heart, liver, diaphragm, spleen, lungs, intestines, mesentery, undigested food, feces, bile, phlegm, pus, blood, sweat, fat, tears, grease, saliva, snot, synovial fluid, urine.’\footnote{Thirty-one parts are mentioned in early texts, later expanded to thirty-two with the addition of the “brain” (\textit{\textsanskrit{matthaluṅga}}). } 

It’s\marginnote{10.3} as if there were a bag with openings at both ends, filled with various kinds of grains, such as fine rice, wheat, mung beans, peas, sesame, and ordinary rice. And someone with clear eyes were to open it and examine the contents: ‘These grains are fine rice, these are wheat, these are mung beans, these are peas, these are sesame, and these are ordinary rice.’\footnote{The “bag with openings at both ends” is the body. Not all the varieties of grains and beans can be positively identified. } 

And\marginnote{11.1} so they meditate observing an aspect of the body internally … 

That\marginnote{11.2} too is how a mendicant meditates by observing an aspect of the body. 

\subsubsection*{1.5. Focusing on the Elements }

Furthermore,\marginnote{12.1} a mendicant examines their own body, whatever its placement or posture, according to the elements: ‘In this body there is the earth element, the water element, the fire element, and the air element.’\footnote{While meditation on the elements is commonly taught in early texts, this phrase is found only in the two \textsanskrit{Satipaṭṭhānasuttas} and the \textsanskrit{Kāyagatāsatisutta}. Detailed instructions are found in such suttas as \href{https://suttacentral.net/mn28/en/sujato}{MN 28} and \href{https://suttacentral.net/mn140/en/sujato}{MN 140}. This meditation works in any posture, whereas breath meditation is best done sitting, to allow the breath to become still. } 

It’s\marginnote{12.3} as if a deft butcher or butcher’s apprentice were to kill a cow and sit down at the crossroads with the meat cut into chops.\footnote{This gruesome image shows that butchery of cows was a normal feature of ancient Indian life. | A wide range of skilled workers have an “apprentice”, including potters, goldsmiths, accountants, carpenters, magicians, etc. The word is \textit{\textsanskrit{antevāsī}}, literally “one who dwells within”, suggesting that apprentices would stay with their teacher. The same word is used for a monastic student. } 

And\marginnote{13.1} so they meditate observing an aspect of the body internally … 

That\marginnote{13.2} too is how a mendicant meditates by observing an aspect of the body. 

\subsubsection*{1.6. The Charnel Ground Contemplations }

Furthermore,\marginnote{14.1} suppose a mendicant were to see a corpse discarded in a charnel ground. And it had been dead for one, two, or three days, bloated, livid, and festering.\footnote{Cremation was expensive and not available to everyone. Bodies might be left in the charnel ground for a variety of reasons, such as local customs, lack of funds, or in cases of inauspicious death such as murder or execution. This is still seen in some places today, and monastics occasionally take the opportunity to practice meditation beside a corpse. However the wording of the Pali sounds like an imaginative exercise. } They’d compare it with their own body:\footnote{The observed corpse is not gendered. The purpose is not to become repulsed by an objectified other, but to understand the mortality of one’s own body. } ‘This body is also of that same nature, that same kind, and cannot go beyond that.’ And so they meditate observing an aspect of the body internally … 

That\marginnote{15.2} too is how a mendicant meditates by observing an aspect of the body. 

Furthermore,\marginnote{16.1} suppose they were to see a corpse discarded in a charnel ground being devoured by crows, hawks, vultures, herons, dogs, tigers, leopards, jackals, and many kinds of little creatures. They’d compare it with their own body: ‘This body is also of that same nature, that same kind, and cannot go beyond that.’ And so they meditate observing an aspect of the body internally … 

That\marginnote{17.2} too is how a mendicant meditates by observing an aspect of the body. 

Furthermore,\marginnote{18{-}23.1} suppose they were to see a corpse discarded in a charnel ground, a skeleton with flesh and blood, held together by sinews … 

A\marginnote{18{-}23.2} skeleton without flesh but smeared with blood, and held together by sinews … 

A\marginnote{18{-}23.3} skeleton rid of flesh and blood, held together by sinews … 

Bones\marginnote{24.1} rid of sinews scattered in every direction. Here a hand-bone, there a foot-bone, here an ankle bone, there a shin-bone, here a thigh-bone, there a hip-bone, here a rib-bone, there a back-bone, here an arm-bone, there a neck-bone, here a jaw-bone, there a tooth, here the skull. … 

White\marginnote{26{-}28.1} bones, the color of shells … 

Decrepit\marginnote{29.1} bones, heaped in a pile … 

Bones\marginnote{30.1} rotted and crumbled to powder.\footnote{It takes decades for bones to rot to powder, again suggesting it is an imaginative contemplation. } They’d compare it with their own body: ‘This body is also of that same nature, that same kind, and cannot go beyond that.’ 

And\marginnote{31.1} so they meditate observing an aspect of the body internally, externally, and both internally and externally. They meditate observing the body as liable to originate, as liable to vanish, and as liable to both originate and vanish. Or mindfulness is established that the body exists, to the extent necessary for knowledge and mindfulness. They meditate independent, not grasping at anything in the world. 

That\marginnote{31.4} too is how a mendicant meditates by observing an aspect of the body. 

\subsection*{2. Observing the Feelings }

And\marginnote{32.1} how does a mendicant meditate observing an aspect of feelings?\footnote{Literally “a feeling among the feelings”; the practice shows that the meditator contemplates specific feelings as they occur. } 

It’s\marginnote{32.2} when a mendicant who feels a pleasant feeling knows: ‘I feel a pleasant feeling.’\footnote{Pali employs direct quotes to indicate reflexive awareness: you feel the feeling and you know that you feel the feeling. It does not mean that you have to literally say “I feel a pleasant feeling”, although some adopt that as a meditation method. } 

When\marginnote{32.3} they feel a painful feeling, they know: ‘I feel a painful feeling.’ 

When\marginnote{32.4} they feel a neutral feeling, they know: ‘I feel a neutral feeling.’ 

When\marginnote{32.5} they feel a pleasant feeling of the flesh, they know: ‘I feel a pleasant feeling of the flesh.’\footnote{Feelings “of the flesh” (\textit{\textsanskrit{sāmisa}})  are associated with the body and sensual desires (\href{https://suttacentral.net/sn36.31/en/sujato\#4.1}{SN 36.31:4.1}). } 

When\marginnote{32.6} they feel a pleasant feeling not of the flesh, they know: ‘I feel a pleasant feeling not of the flesh.’\footnote{Feelings “not of the flesh” (\textit{\textsanskrit{nirāmisa}}) are associated with renunciation and especially with the \textit{\textsanskrit{jhānas}} and liberation (\href{https://suttacentral.net/sn36.31/en/sujato\#5.1}{SN 36.31:5.1}). } 

When\marginnote{32.7} they feel a painful feeling of the flesh, they know: ‘I feel a painful feeling of the flesh.’ 

When\marginnote{32.8} they feel a painful feeling not of the flesh, they know: ‘I feel a painful feeling not of the flesh.’\footnote{This would include the feelings of loss, doubt, and dejection that can occur during the spiritual path (see \href{https://suttacentral.net/mn44/en/sujato\#28.6}{MN 44:28.6}). } 

When\marginnote{32.9} they feel a neutral feeling of the flesh, they know: ‘I feel a neutral feeling of the flesh.’ 

When\marginnote{32.10} they feel a neutral feeling not of the flesh, they know: ‘I feel a neutral feeling not of the flesh.’\footnote{The feeling of the fourth \textit{\textsanskrit{jhāna}} and higher liberations (\href{https://suttacentral.net/sn36.31/en/sujato\#8.2}{SN 36.31:8.2}). } 

And\marginnote{33.1} so they meditate observing an aspect of feelings internally, externally, and both internally and externally. They meditate observing feelings as liable to originate, as liable to vanish, and as liable to both originate and vanish. Or mindfulness is established that feelings exist, to the extent necessary for knowledge and mindfulness. They meditate independent, not grasping at anything in the world. 

That’s\marginnote{33.4} how a mendicant meditates by observing an aspect of feelings. 

\subsection*{3. Observing the Mind }

And\marginnote{34.1} how does a mendicant meditate observing an aspect of the mind?\footnote{In Buddhist theory, awareness of the presence or absence of qualities such as greed is explained on three levels. There is the simple happenstance of whether greed is present at that time or not. Then there is the mind freed of greed through the power of absorption. Finally there is the liberation from greed which comes with full awakening. } 

It’s\marginnote{34.2} when a mendicant understands mind with greed as ‘mind with greed,’ and mind without greed as ‘mind without greed.’ They understand mind with hate as ‘mind with hate,’ and mind without hate as ‘mind without hate.’ They understand mind with delusion as ‘mind with delusion,’ and mind without delusion as ‘mind without delusion.’ They know constricted mind as ‘constricted mind,’\footnote{The mind is “constricted internally” due to dullness and “scattered externally” due to the distractions of desire (\href{https://suttacentral.net/sn51.20/en/sujato\#18.1}{SN 51.20:18.1}). } and scattered mind as ‘scattered mind.’ They know expansive mind as ‘expansive mind,’\footnote{The following terms “expansive” (\textit{mahaggata}), “supreme” (\textit{anuttara}), “immersed” (\textit{\textsanskrit{samāhita}}), and “freed” (\textit{vimutta}) all refer to states of absorption and/or awakening. } and unexpansive mind as ‘unexpansive mind.’ They know mind that is not supreme as ‘mind that is not supreme,’ and mind that is supreme as ‘mind that is supreme.’ They know mind immersed in \textsanskrit{samādhi} as ‘mind immersed in \textsanskrit{samādhi},’ and mind not immersed in \textsanskrit{samādhi} as ‘mind not immersed in \textsanskrit{samādhi}.’ They know freed mind as ‘freed mind,’ and unfreed mind as ‘unfreed mind.’ 

And\marginnote{35.1} so they meditate observing an aspect of the mind internally, externally, and both internally and externally. They meditate observing the mind as liable to originate, as liable to vanish, and as liable to both originate and vanish. Or mindfulness is established that the mind exists, to the extent necessary for knowledge and mindfulness. They meditate independent, not grasping at anything in the world. 

That’s\marginnote{35.4} how a mendicant meditates by observing an aspect of the mind. 

\subsection*{4. Observing Principles }

\subsubsection*{4.1. The Hindrances }

And\marginnote{36.1} how does a mendicant meditate observing an aspect of principles? 

It’s\marginnote{36.2} when a mendicant meditates by observing an aspect of principles with respect to the five hindrances.\footnote{The \textsanskrit{Satipaṭṭhānavibhaṅga} of the Pali Abhidhamma only mentions the hindrances and awakening factors in this section (\href{https://suttacentral.net/vb7}{Vb 7}). This, together with a range of other evidence, suggests that this was the original content of the observation of principles. } And how does a mendicant meditate observing an aspect of principles with respect to the five hindrances? 

It’s\marginnote{36.4} when a mendicant who has sensual desire in them understands: ‘I have sensual desire in me.’ When they don’t have sensual desire in them, they understand: ‘I don’t have sensual desire in me.’ They understand how sensual desire arises; how, when it’s already arisen, it’s given up; and how, once it’s given up, it doesn’t arise again in the future.\footnote{Here causality is introduced. In the contemplation of mind, the meditator was aware of the presence or absence of desire in the mind. Now they look deeper, investigating the cause of desire and understanding how to be free of it forever. This contemplation of the “principles” of cause and effect is the distinctive feature of this section. } 

When\marginnote{36.5} they have ill will in them, they understand: ‘I have ill will in me.’ When they don’t have ill will in them, they understand: ‘I don’t have ill will in me.’ They understand how ill will arises; how, when it’s already arisen, it’s given up; and how, once it’s given up, it doesn’t arise again in the future. 

When\marginnote{36.6} they have dullness and drowsiness in them, they understand: ‘I have dullness and drowsiness in me.’ When they don’t have dullness and drowsiness in them, they understand: ‘I don’t have dullness and drowsiness in me.’ They understand how dullness and drowsiness arise; how, when they’ve already arisen, they’re given up; and how, once they’re given up, they don’t arise again in the future.\footnote{The Buddhist schools debated whether this included physical tiredness or not. The Theravada argued that it was purely a mental laziness, as even the Buddha got sleepy. } 

When\marginnote{36.7} they have restlessness and remorse in them, they understand: ‘I have restlessness and remorse in me.’ When they don’t have restlessness and remorse in them, they understand: ‘I don’t have restlessness and remorse in me.’ They understand how restlessness and remorse arise; how, when they’ve already arisen, they’re given up; and how, once they’re given up, they don’t arise again in the future. 

When\marginnote{36.8} they have doubt in them, they understand: ‘I have doubt in me.’ When they don’t have doubt in them, they understand: ‘I don’t have doubt in me.’ They understand how doubt arises; how, when it’s already arisen, it’s given up; and how, once it’s given up, it doesn’t arise again in the future. 

And\marginnote{37.1} so they meditate observing an aspect of principles internally, externally, and both internally and externally. They meditate observing the principles as liable to originate, as liable to vanish, and as liable to both originate and vanish. Or mindfulness is established that principles exist, to the extent necessary for knowledge and mindfulness. They meditate independent, not grasping at anything in the world. 

That’s\marginnote{37.4} how a mendicant meditates by observing an aspect of principles with respect to the five hindrances. 

\subsubsection*{4.2. The Aggregates }

Furthermore,\marginnote{38.1} a mendicant meditates by observing an aspect of principles with respect to the five grasping aggregates. And how does a mendicant meditate observing an aspect of principles with respect to the five grasping aggregates? It’s when a mendicant contemplates: ‘Such is form, such is the origin of form, such is the ending of form.\footnote{“Form” (\textit{\textsanskrit{rūpa}}) is one’s own body and the external material world experienced through the senses. More subtly, it represents the “appearance” of physical phenomena, even when experienced solely in the mind as color, visions, etc. } Such is feeling, such is the origin of feeling, such is the ending of feeling. Such is perception, such is the origin of perception, such is the ending of perception.\footnote{“Perception” (\textit{\textsanskrit{saññā}}) is the recognition or interpretation of experience in terms of meaningful wholes. We see, for example, “color” yet we perceive a “person”. In the Vinaya we find many examples where a person perceived things in one way, yet they turned out to be something else. } Such are choices, such is the origin of choices, such is the ending of choices.\footnote{In the five aggregates, \textit{\textsanskrit{saṅkhārā}} is a synonym for “volition” (\textit{\textsanskrit{cetanā}}). The traditions later used it as a catch-all category for everything that does not fit in the other aggregates. In the suttas, however, the purpose of the aggregates is not to classify everything that exists, but to contemplate aspects of experience that we tend to identify as a “self”. } Such is consciousness, such is the origin of consciousness, such is the ending of consciousness.’ 

And\marginnote{39.1} so they meditate observing an aspect of principles internally … 

That’s\marginnote{39.4} how a mendicant meditates by observing an aspect of principles with respect to the five grasping aggregates. 

\subsubsection*{4.3. The Sense Fields }

Furthermore,\marginnote{40.1} a mendicant meditates by observing an aspect of principles with respect to the six interior and exterior sense fields. And how does a mendicant meditate observing an aspect of principles with respect to the six interior and exterior sense fields? 

It’s\marginnote{40.3} when a mendicant understands the eye, sights, and the fetter that arises dependent on both of these. They understand how the fetter that has not arisen comes to arise; how the arisen fetter comes to be abandoned; and how the abandoned fetter comes to not rise again in the future.\footnote{At \href{https://suttacentral.net/sn35.232/en/sujato\#3.2}{SN 35.232:3.2} the “fetter that arises dependent on both” is identified as “desire and lust” (\textit{\textsanskrit{chandarāga}}). } 

They\marginnote{40.4} understand the ear, sounds, and the fetter … 

They\marginnote{40.5} understand the nose, smells, and the fetter … 

They\marginnote{40.6} understand the tongue, tastes, and the fetter … 

They\marginnote{40.7} understand the body, touches, and the fetter … 

They\marginnote{40.8} understand the mind, ideas, and the fetter that arises dependent on both of these. They understand how the fetter that has not arisen comes to arise; how the arisen fetter comes to be abandoned; and how the abandoned fetter comes to not rise again in the future. 

And\marginnote{41.1} so they meditate observing an aspect of principles internally … 

That’s\marginnote{41.4} how a mendicant meditates by observing an aspect of principles with respect to the six internal and external sense fields. 

\subsubsection*{4.4. The Awakening Factors }

Furthermore,\marginnote{42.1} a mendicant meditates by observing an aspect of principles with respect to the seven awakening factors.\footnote{These seven factors that lead to awakening (\textit{\textsanskrit{bojjhaṅgā}}, \href{https://suttacentral.net/sn46.5/en/sujato}{SN 46.5}) are commonly presented in opposition to the five hindrances (eg. \href{https://suttacentral.net/sn46.2/en/sujato}{SN 46.2}, \href{https://suttacentral.net/sn46.23/en/sujato}{SN 46.23}, \href{https://suttacentral.net/sn46.55/en/sujato}{SN 46.55}). } And how does a mendicant meditate observing an aspect of principles with respect to the seven awakening factors? 

It’s\marginnote{42.3} when a mendicant who has the awakening factor of mindfulness in them understands: ‘I have the awakening factor of mindfulness in me.’ When they don’t have the awakening factor of mindfulness in them, they understand: ‘I don’t have the awakening factor of mindfulness in me.’ They understand how the awakening factor of mindfulness that has not arisen comes to arise; and how the awakening factor of mindfulness that has arisen becomes fulfilled by development.\footnote{“Mindfulness” includes the recollection of the teachings (\href{https://suttacentral.net/sn46.3/en/sujato\#1.8}{SN 46.3:1.8}) as well as mindfulness meditation. } 

When\marginnote{42.4} they have the awakening factor of investigation of principles …\footnote{Likewise, this includes the inquiry into \textit{dhammas} as “teachings” as well as “phenomena” or “principles”. } energy … rapture … tranquility … immersion … equanimity in them, they understand: ‘I have the awakening factor of equanimity in me.’ When they don’t have the awakening factor of equanimity in them, they understand: ‘I don’t have the awakening factor of equanimity in me.’ They understand how the awakening factor of equanimity that has not arisen comes to arise; and how the awakening factor of equanimity that has arisen becomes fulfilled by development. 

And\marginnote{43.1} so they meditate observing an aspect of principles internally, externally, and both internally and externally. They meditate observing the principles as liable to originate, as liable to vanish, and as liable to both originate and vanish. Or mindfulness is established that principles exist, to the extent necessary for knowledge and mindfulness. They meditate independent, not grasping at anything in the world. 

That’s\marginnote{43.4} how a mendicant meditates by observing an aspect of principles with respect to the seven awakening factors. 

\subsubsection*{4.5. The Truths }

Furthermore,\marginnote{44.1} a mendicant meditates by observing an aspect of principles with respect to the four noble truths.\footnote{Due to their development of the two wings of \textit{samatha} and \textit{\textsanskrit{vipassanā}} meditation as described in this sutta, practised in the context of the teaching and training as a whole, the meditator realizes the four noble truths at the moment of stream-entry. } 

And\marginnote{44.2} how does a mendicant meditate observing an aspect of principles with respect to the four noble truths? It’s when a mendicant truly understands: ‘This is suffering’ … ‘This is the origin of suffering’ … ‘This is the cessation of suffering’ … ‘This is the practice that leads to the cessation of suffering.’\footnote{Following this section, \href{https://suttacentral.net/dn22/en/sujato\#17.4}{DN 22:17.4} announces the end of the first recitation section, and goes on to expand each of the four noble truths in detail. } 

And\marginnote{45.1} so they meditate observing an aspect of principles internally, externally, and both internally and externally. They meditate observing the principles as liable to originate, as liable to vanish, and as liable to both originate and vanish. Or mindfulness is established that principles exist, to the extent necessary for knowledge and mindfulness. They meditate independent, not grasping at anything in the world. 

That’s\marginnote{45.4} how a mendicant meditates by observing an aspect of principles with respect to the four noble truths. 

Anyone\marginnote{46.1} who develops these four kinds of mindfulness meditation in this way for seven years can expect one of two results:\footnote{The emphasis is on “develop in this way” (\textit{\textsanskrit{evaṁ} \textsanskrit{bhāveyya}}), that is, with the full practice including deep absorption as the culmination of the path as a whole. } enlightenment in this very life, or if there’s something left over, non-return. 

Let\marginnote{46.3} alone seven years,\footnote{A similar promise of results in at most seven years is found at \href{https://suttacentral.net/dn22/en/sujato\#22.3}{DN 22:22.3}, \href{https://suttacentral.net/dn25/en/sujato\#22.9}{DN 25:22.9}, and \href{https://suttacentral.net/mn85/en/sujato\#59.3}{MN 85:59.3}; and at most ten years at \href{https://suttacentral.net/an10.46/en/sujato\#7.3}{AN 10.46:7.3}. } anyone who develops these four kinds of mindfulness meditation in this way for six years … five years … four years … three years … two years … one year … seven months … six months … five months … four months … three months … two months … one month … a fortnight … Let alone a fortnight, anyone who develops these four kinds of mindfulness meditation in this way for seven days can expect one of two results: enlightenment in this very life, or if there’s something left over, non-return. 

‘The\marginnote{47.1} four kinds of mindfulness meditation are the path to convergence. They are in order to purify sentient beings, to get past sorrow and crying, to make an end of pain and sadness, to discover the system, and to realize extinguishment.’ That’s what I said, and this is why I said it.” 

That\marginnote{47.3} is what the Buddha said. Satisfied, the mendicants approved what the Buddha said. 

%
\addtocontents{toc}{\let\protect\contentsline\protect\nopagecontentsline}
\chapter*{The Chapter on the Lion’s Roar }
\addcontentsline{toc}{chapter}{\tocchapterline{The Chapter on the Lion’s Roar }}
\addtocontents{toc}{\let\protect\contentsline\protect\oldcontentsline}

%
\section*{{\suttatitleacronym MN 11}{\suttatitletranslation The Shorter Discourse on the Lion’s Roar }{\suttatitleroot Cūḷasīhanādasutta}}
\addcontentsline{toc}{section}{\tocacronym{MN 11} \toctranslation{The Shorter Discourse on the Lion’s Roar } \tocroot{Cūḷasīhanādasutta}}
\markboth{The Shorter Discourse on the Lion’s Roar }{Cūḷasīhanādasutta}
\extramarks{MN 11}{MN 11}

\scevam{So\marginnote{1.1} I have heard.\footnote{Multiple suttas (eg. \href{https://suttacentral.net/dn8/en/sujato}{DN 8}, \href{https://suttacentral.net/dn25/en/sujato}{DN 25}, \href{https://suttacentral.net/mn12/en/sujato}{MN 12}, \href{https://suttacentral.net/an6.64/en/sujato}{AN 6.64}, \href{https://suttacentral.net/an9.11/en/sujato}{AN 9.11}, and \href{https://suttacentral.net/an10.21/en/sujato}{AN 10.21}) invoke the indomitable force of the lion’s roar, the most mighty sound of the jungle (\href{https://suttacentral.net/sn22.78/en/sujato\#1.2}{SN 22.78:1.2}). } }At one time the Buddha was staying near \textsanskrit{Sāvatthī} in Jeta’s Grove, \textsanskrit{Anāthapiṇḍika}’s monastery. There the Buddha addressed the mendicants, “Mendicants!” 

“Venerable\marginnote{1.5} sir,” they replied. The Buddha said this: 

“‘Only\marginnote{2.1} here is there a true ascetic, here a second ascetic, here a third ascetic, and here a fourth ascetic.\footnote{The same teaching is given more briefly at \href{https://suttacentral.net/an4.241/en/sujato\#1.1}{AN 4.241:1.1}, where the four ascetics are defined as the stream-enterer, once-returner, non-returner, and perfected one,  and at \href{https://suttacentral.net/dn16/en/sujato\#5.27.3}{DN 16:5.27.3}, where the presence of the four ascetics is attributed to the practice of the eightfold path. } Other sects are empty of ascetics.’ This, mendicants, is how you should rightly roar your lion’s roar. 

It’s\marginnote{3.1} possible that wanderers of other religions might say: ‘But what is the source of the venerables’ certainty and forcefulness that they say this?’\footnote{\textit{Ko \textsanskrit{assāso} \textsanskrit{kiṁ} \textsanskrit{balaṁ}} is found only here and at \href{https://suttacentral.net/mn93/en/sujato\#6.7}{MN 93:6.7}, where it also deals with an unequivocal claim to superiority. } You should say to them: ‘There are four things explained by the Blessed One, who knows and sees, the perfected one, the fully awakened Buddha. Seeing these things in ourselves we say that:\footnote{Self-confidence is reflective, not dogmatic. } 

“Only\marginnote{3.7} here is there a true ascetic, here a second ascetic, here a third ascetic, and here a fourth ascetic. Other sects are empty of ascetics.” What four? We have confidence in the Teacher, we have confidence in the teaching, and we have fulfilled the precepts. And we have love and affection for those who share our path,\footnote{This appears to be the only occurrence of \textit{sahadhammika} in this sense in early Pali. Normally it has the sense “legitimate” (as eg. \href{https://suttacentral.net/mn90/en/sujato\#5.4}{MN 90:5.4}). The parallel at MA 103 has \langlzh{同道}. } both laypeople and renunciates. These are the four things.’ 

It’s\marginnote{4.1} possible that wanderers of other religions might say: ‘We too have confidence in the Teacher—our Teacher; we have confidence in the teaching—our teaching; and we have fulfilled the precepts—our precepts. And we have love and affection for those who share our path, both laypeople and renunciates. What, then, is the difference between you and us?’ 

You\marginnote{5.1} should say to them: ‘Well, reverends, is the goal one or many?’ Answering rightly, the wanderers would say: ‘The goal is one, reverends, not many.’ 

‘But\marginnote{5.5} is that goal for the greedy or for those free of greed?’ Answering rightly, the wanderers would say: ‘That goal is for those free of greed, not for the greedy.’ 

‘Is\marginnote{5.8} it for the hateful or those free of hate?’ ‘It’s for those free of hate.’ 

‘Is\marginnote{5.11} it for the delusional or those free of delusion?’ ‘It’s for those free of delusion.’ 

‘Is\marginnote{5.14} it for those who crave or those rid of craving?’ ‘It’s for those rid of craving.’ 

‘Is\marginnote{5.17} it for those who have fuel for grasping or those who do not?’ ‘It’s for those who do not have fuel for grasping.’ 

‘Is\marginnote{5.20} it for the knowledgeable or the ignorant?’ ‘It’s for the knowledgeable.’ 

‘Is\marginnote{5.23} it for those who favor and oppose or for those who don’t favor and oppose?’ ‘It’s for those who don’t favor and oppose.’ 

‘But\marginnote{5.26} is that goal for those who enjoy proliferation or for those who enjoy non-proliferation?’\footnote{The topic of “proliferation” (\textit{\textsanskrit{papañca}}) is explored in \href{https://suttacentral.net/mn18/en/sujato}{MN 18}. } Answering rightly, the wanderers would say: ‘It’s for those who enjoy non-proliferation, not for those who enjoy proliferation.’ 

Mendicants,\marginnote{6.1} there are these two views: views favoring continued existence and views favoring ending existence.\footnote{That is, the eternalists and the annihilationists. } Any ascetics or brahmins who resort to, draw near to, and cling to a view favoring continued existence will oppose a view favoring ending existence.\footnote{The sequence of operative terms here is \textit{\textsanskrit{allīna}}, \textit{upagata}, \textit{ajjhosita}. Normally \textit{\textsanskrit{allīna}} in Pali means to “stick to, cling to”.  But read this passage compared to \href{https://suttacentral.net/sn12.15/en/sujato\#2.5}{SN 12.15:2.5} where we find a similar movement of drawing close (\textit{upagata})and getting attached. The \textsanskrit{Mahāvastu} in Buddhist Hybrid Sanskrit features \textit{\textsanskrit{allīyati}} in the sense “approach”, “resort”. This lends a greater coherence to this passage. } Any ascetics or brahmins who resort to, draw near to, and cling to a view favoring ending existence will oppose a view favoring continued existence. 

There\marginnote{7.1} are some ascetics and brahmins who don’t truly understand these two views’ origin, ending, gratification, drawback, and escape. They’re greedy, hateful, delusional, craving, grasping, and ignorant. They favor and oppose, and they enjoy proliferation. They’re not freed from rebirth, old age, and death, from sorrow, lamentation, pain, sadness, and distress. They’re not freed from suffering, I say. 

There\marginnote{8.1} are some ascetics and brahmins who do truly understand these two views’ origin, ending, gratification, drawback, and escape. They’re rid of greed, hate, delusion, craving, grasping, and ignorance. They don’t favor and oppose, and they enjoy non-proliferation. They’re freed from rebirth, old age, and death, from sorrow, lamentation, pain, sadness, and distress. They’re freed from suffering, I say. 

There\marginnote{9.1} are these four kinds of grasping. What four? Grasping at sensual pleasures, views, precepts and observances, and theories of a self. 

There\marginnote{10.1} are some ascetics and brahmins who claim to propound the complete understanding of all kinds of grasping. But they don’t correctly describe the complete understanding of all kinds of grasping. They describe the complete understanding of grasping at sensual pleasures, but not views, precepts and observances, and theories of a self.\footnote{Renunciate orders such as the Jains abhorred sensual pleasures, while the meditations of Brahmanical rishis depended on letting go all pleasures of the flesh. } Why is that? Because those gentlemen don’t truly understand these three things. That’s why they claim to propound the complete understanding of all kinds of grasping, but they don’t really. 

There\marginnote{11.1} are some other ascetics and brahmins who claim to propound the complete understanding of all kinds of grasping, but they don’t really. They describe the complete understanding of grasping at sensual pleasures and views, but not precepts and observances, and theories of a self.\footnote{Perhaps referring to those such as \textsanskrit{Dīghanakha} who claimed to believe in nothing (\href{https://suttacentral.net/mn74/en/sujato\#2.4}{MN 74:2.4}). } Why is that? Because those gentlemen don’t truly understand these two things. That’s why they claim to propound the complete understanding of all kinds of grasping, but they don’t really. 

There\marginnote{12.1} are some other ascetics and brahmins who claim to propound the complete understanding of all kinds of grasping, but they don’t really. They describe the complete understanding of grasping at sensual pleasures, views, and precepts and observances, but not theories of a self.\footnote{The teaching on not-self is what sets Buddhism apart from all other teachings. } Why is that? Because those gentlemen don’t truly understand this one thing. That’s why they claim to propound the complete understanding of all kinds of grasping, but they don’t really. 

In\marginnote{13.1} such a teaching and training, confidence in the Teacher is said to be not rightly placed.\footnote{Elsewhere \textit{sammaggata} (“rightly placed”) is an epithet of the Buddha or other well-practiced sages, where I translate as “rightly comported”. } Likewise, confidence in the teaching, fulfillment of the precepts, and love and affection for those sharing the same path are said to be not rightly placed. Why is that? It’s because that teaching and training is poorly explained and poorly propounded, not emancipating, not leading to peace, proclaimed by someone who is not a fully awakened Buddha. 

The\marginnote{14.1} Realized One, the perfected one, the fully awakened Buddha claims to propound the complete understanding of all kinds of grasping. He describes the complete understanding of grasping at sensual pleasures, views, precepts and observances, and theories of a self. 

In\marginnote{15.1} such a teaching and training, confidence in the Teacher is said to be rightly placed. Likewise, confidence in the teaching, fulfillment of the precepts, and love and affection for those sharing the same path are said to be rightly placed. Why is that? It’s because that teaching and training is well explained and well propounded, emancipating, leading to peace, proclaimed by a fully awakened Buddha. 

What\marginnote{16.1} is the source, origin, birthplace, and inception of these four kinds of grasping?\footnote{Grasping is fully understood only when its source is known. Here the Buddha begins a partial treatment of dependent origination, implicitly connecting this to his special doctrine of not-self. } Craving. And what is the source, origin, birthplace, and inception of craving? Feeling. And what is the source of feeling? Contact. And what is the source of contact? The six sense fields. And what is the source of the six sense fields? Name and form. And what is the source of name and form? Consciousness. And what is the source of consciousness? Choices. And what is the source of choices? Ignorance. 

When\marginnote{17.1} that mendicant has given up ignorance and given rise to knowledge, they don’t grasp at sensual pleasures, views, precepts and observances, or theories of a self. Not grasping, they’re not anxious. Not being anxious, they personally become extinguished. 

They\marginnote{17.3} understand: ‘Rebirth is ended, the spiritual journey has been completed, what had to be done has been done, there is nothing further for this place.’” 

That\marginnote{17.4} is what the Buddha said. Satisfied, the mendicants approved what the Buddha said. 

%
\section*{{\suttatitleacronym MN 12}{\suttatitletranslation The Longer Discourse on the Lion’s Roar }{\suttatitleroot Mahāsīhanādasutta}}
\addcontentsline{toc}{section}{\tocacronym{MN 12} \toctranslation{The Longer Discourse on the Lion’s Roar } \tocroot{Mahāsīhanādasutta}}
\markboth{The Longer Discourse on the Lion’s Roar }{Mahāsīhanādasutta}
\extramarks{MN 12}{MN 12}

\scevam{So\marginnote{1.1} I have heard. }At one time the Buddha was staying outside the city of \textsanskrit{Vesālī} in a woodland grove west of the town.\footnote{This description of the Buddha’s location is unique. } 

Now\marginnote{2.1} at that time Sunakkhatta the Licchavi had recently left this teaching and training.\footnote{Sunakkhatta’s dismal spiritual career began when he met the Buddha in \href{https://suttacentral.net/mn105/en/sujato}{MN 105}. In \href{https://suttacentral.net/dn6/en/sujato\#5.3}{DN 6:5.3} we learn that, after being ordained three years, he spoke of his limited success in meditation. The current sutta and \href{https://suttacentral.net/dn24/en/sujato}{DN 24} deal with Sunakkhatta’s bitter criticisms of the Buddha shortly after his disrobal. } He was telling a crowd in \textsanskrit{Vesālī}: 

“The\marginnote{2.3} ascetic Gotama has no superhuman distinction in knowledge and vision worthy of the noble ones.\footnote{These distinctions are defined in the Vinaya as “absorption, release, immersion, attainment, knowledge and vision, development of the path, realization of the fruits, giving up the defilements, a mind without hindrances, delighting in an empty dwelling” (\href{https://suttacentral.net/pli-tv-bu-vb-pj4/en/sujato\#3.8}{Bu Pj 4:3.8}). They may be summarized as absorption, psychic abilities, and realization of the paths and fruits. } He teaches what he’s worked out by logic, following a line of inquiry, expressing his own perspective.\footnote{The Buddha taught that logic alone is an unreliable guide to the truth (\href{https://suttacentral.net/mn76/en/sujato\#27.2}{MN 76:27.2}, \href{https://suttacentral.net/dn1/en/sujato\#1.34.2}{DN 1:1.34.2}), a fact of which Sunakkhatta was apparently not aware. } And his teaching leads those who practice it to the complete ending of suffering, the goal for which it’s taught.”\footnote{The Chinese parallel at T 757, rather, raises the question, “How could such a teaching lead to freedom from suffering?” } 

Then\marginnote{3.1} Venerable \textsanskrit{Sāriputta} robed up in the morning and, taking his bowl and robe, entered \textsanskrit{Vesālī} for alms. He heard what Sunakkhatta was saying. 

Then\marginnote{3.6} he wandered for alms in \textsanskrit{Vesālī}. After the meal, on his return from almsround, he went to the Buddha, bowed, sat down to one side, and told him what had happened. 

“\textsanskrit{Sāriputta},\marginnote{4.1} Sunakkhatta, that futile man, is angry.\footnote{“Futile man” (\textit{moghapurisa}) refers to someone who has been led astray by delusion. } His words are spoken out of anger. Thinking he criticizes the Realized One, in fact he just praises him. For it is praise of the Realized One to say: ‘His teaching leads those who practice it to the complete ending of suffering, the goal for which it’s taught.’ 

But\marginnote{5.1} there’s no way Sunakkhatta will infer about me from the teaching:\footnote{“Inference from the teaching” (\textit{dhammanvaya}) is a valid form of knowledge, where someone who has seen the Dhamma draws a reasoned conclusion based on fundamental principles. } ‘That Blessed One is perfected, a fully awakened Buddha, accomplished in knowledge and conduct, holy, knower of the world, supreme guide for those who wish to train, teacher of gods and humans, awakened, blessed.’ 

And\marginnote{6.1} there’s no way Sunakkhatta will infer about me from the teaching: ‘That Blessed One wields the many kinds of psychic power: multiplying himself and becoming one again; appearing and disappearing; going unobstructed through a wall, a rampart, or a mountain as if through space; diving in and out of the earth as if it were water; walking on water as if it were earth; flying cross-legged through the sky like a bird; touching and stroking with the hand the sun and moon, so mighty and powerful; controlling the body as far as the realm of divinity.’\footnote{The Buddha introduces these here because, as we know from \href{https://suttacentral.net/dn24/en/sujato\#1.4.1}{DN 24:1.4.1}, Sunakkhatta thought that such displays were a worthy goal of spiritual practice. The Chinese parallel, however, omits the psychic powers and instead has the \textit{\textsanskrit{jhānas}}. | This passage begins the six direct knowledges. } 

And\marginnote{7.1} there’s no way Sunakkhatta will infer about me from the teaching: ‘That Blessed One, with clairaudience that is purified and superhuman, hears both kinds of sounds, human and heavenly, whether near or far.’ 

And\marginnote{8.1} there’s no way Sunakkhatta will infer about me from the teaching: ‘That Blessed One understands the minds of other beings and individuals, having comprehended them with his own mind. He understands mind with greed as “mind with greed,” and mind without greed as “mind without greed.” He understands mind with hate … mind without hate … mind with delusion … mind without delusion … constricted mind … scattered mind … expansive mind … unexpansive mind … mind that is supreme … mind that is not supreme … mind immersed in \textsanskrit{samādhi} … mind not immersed in \textsanskrit{samādhi} … freed mind as “freed mind,” and unfreed mind as “unfreed mind.”’ 

There\marginnote{9.1} are these ten powers of a Realized One that the Realized One possesses. With these he claims the bull’s place, roars his lion’s roar in the assemblies, and turns the divine wheel.\footnote{After the first three of the direct knowledges, the sequence is interrupted with the insertion of these ten powers of a Realized One, also found at \href{https://suttacentral.net/an10.21/en/sujato}{AN 10.21} and \href{https://suttacentral.net/an10.22/en/sujato}{AN 10.22}. The \textsanskrit{Vibhaṅga}, an early Abhidhamma book, gives a detailed explanation (\href{https://suttacentral.net/vb16/pli/ms\#353.1}{Vb 16:353} ff.). | Five powers are listed at \href{https://suttacentral.net/an5.11/en/sujato}{AN 5.11} and six at \href{https://suttacentral.net/an6.64/en/sujato}{AN 6.64}. } What ten? 

Firstly,\marginnote{10.1} the Realized One truly understands the possible as possible, and the impossible as impossible.\footnote{Explained in detail at \href{https://suttacentral.net/mn115/en/sujato\#12.1}{MN 115:12.1}–17. } Since he truly understands this, this is a power of the Realized One. Relying on this he claims the bull’s place, roars his lion’s roar in the assemblies, and turns the divine wheel. 

Furthermore,\marginnote{11.1} the Realized One truly understands the result of deeds undertaken in the past, future, and present in terms of grounds and causes.\footnote{The \textsanskrit{Vibhaṅga} explains this as knowing the circumstances that prevent or support the fruition of deeds, namely: place of rebirth (\textit{gati}), form of reincarnation (\textit{upadhi}, explained by the commentary as \textit{\textsanskrit{attabhāva}}), time (\textit{\textsanskrit{kāla}}), and effort (\textit{payoga}). | See also the analysis of “deeds” (\textit{kamma}) in \href{https://suttacentral.net/mn57/en/sujato}{MN 57}, \href{https://suttacentral.net/mn135/en/sujato}{MN 135}, and \href{https://suttacentral.net/mn136/en/sujato}{MN 136}. } Since he truly understands this, this is a power of the Realized One. … 

Furthermore,\marginnote{12.1} the Realized One truly understands where all paths of practice lead.\footnote{Expanded below in \href{https://suttacentral.net/mn12/en/sujato\#35.1}{MN 12:35.1}–42.13. } Since he truly understands this, this is a power of the Realized One. … 

Furthermore,\marginnote{13.1} the Realized One truly understands the world with its many and diverse elements.\footnote{See \href{https://suttacentral.net/mn115/en/sujato\#4.1}{MN 115:4.1}–9.6. } Since he truly understands this, this is a power of the Realized One. … 

Furthermore,\marginnote{14.1} the Realized One truly understands the diverse convictions of sentient beings.\footnote{This is not explained explicitly, but an example is found at \href{https://suttacentral.net/dn1/en/sujato\#1.3.2}{DN 1:1.3.2}, where the Buddha knows the difference between Brahmadatta, who had faith in the triple gem, and Suppiya who did not. Thus \textit{adhimutti} (“conviction”) means “faith, belief” rather than “disposition”. } Since he truly understands this, this is a power of the Realized One. … 

Furthermore,\marginnote{15.1} the Realized One truly understands the faculties of other sentient beings and other individuals after comprehending them with his mind.\footnote{Exemplified by the Buddha when he surveyed the world after his awakening to see the diverse spiritual potentials of beings (\href{https://suttacentral.net/mn26/en/sujato\#21.2}{MN 26:21.2}, \href{https://suttacentral.net/mn85/en/sujato\#45.4}{MN 85:45.4}, \href{https://suttacentral.net/dn14/en/sujato\#3.6.2}{DN 14:3.6.2}, \href{https://suttacentral.net/sn6.1/en/sujato\#9.2}{SN 6.1:9.2}, \href{https://suttacentral.net/pli-tv-kd1/en/sujato\#5.11.1}{Kd 1:5.11.1}). } Since he truly understands this, this is a power of the Realized One. … 

Furthermore,\marginnote{16.1} the Realized One truly understands corruption, cleansing, and emergence regarding the absorptions, liberations, immersions, and attainments.\footnote{“Corruption” is any mental factor that darkens and diminishes meditation; “cleansing” is anything that brightens and enhances it. “Emergence” refers to understanding the reasons why one comes out of a given state of meditation (eg. \href{https://suttacentral.net/mn43/en/sujato\#29.1}{MN 43:29.1}). | The phrase “absorptions, liberations, immersions, and attainments” indicates that, although they are given various technical definitions, these are overlapping terms that describe in different ways the same field of deep meditation experiences. } Since he truly understands this, this is a power of the Realized One. … 

Furthermore,\marginnote{17.1} the Realized One recollects many kinds of past lives. That is: one, two, three, four, five, ten, twenty, thirty, forty, fifty, a hundred, a thousand, a hundred thousand rebirths; many eons of the world contracting, many eons of the world expanding, many eons of the world contracting and expanding. He remembers: ‘There, I was named this, my clan was that, I looked like this, and that was my food. This was how I felt pleasure and pain, and that was how my life ended. When I passed away from that place I was reborn somewhere else. There, too, I was named this, my clan was that, I looked like this, and that was my food. This was how I felt pleasure and pain, and that was how my life ended. When I passed away from that place I was reborn here.’ And so he recollects his many kinds of past lives, with features and details.\footnote{Here we return to the final three of the six direct knowledges, which also form the final three of the ten powers. } Since he truly understands this, this is a power of the Realized One. … 

Furthermore,\marginnote{18.1} with clairvoyance that is purified and superhuman, the Realized One sees sentient beings passing away and being reborn—inferior and superior, beautiful and ugly, in a good place or a bad place. He understands how sentient beings are reborn according to their deeds. ‘These dear beings did bad things by way of body, speech, and mind. They denounced the noble ones; they had wrong view; and they chose to act out of that wrong view. When their body breaks up, after death, they’re reborn in a place of loss, a bad place, the underworld, hell. These dear beings, however, did good things by way of body, speech, and mind. They never denounced the noble ones; they had right view; and they chose to act out of that right view. When their body breaks up, after death, they’re reborn in a good place, a heavenly realm.’ And so, with clairvoyance that is purified and superhuman, he sees sentient beings passing away and being reborn—inferior and superior, beautiful and ugly, in a good place or a bad place. He understands how sentient beings are reborn according to their deeds. Since he truly understands this, this is a power of the Realized One. … 

Furthermore,\marginnote{19.1} the Realized One has realized the undefiled freedom of heart and freedom by wisdom in this very life, and lives having realized it with his own insight due to the ending of defilements. Since he truly understands this, this is a power of the Realized One. Relying on this he claims the bull’s place, roars his lion’s roar in the assemblies, and turns the divine wheel. 

A\marginnote{20.1} Realized One possesses these ten powers of a Realized One. With these he claims the bull’s place, roars his lion’s roar in the assemblies, and turns the divine wheel. 

When\marginnote{21.1} I know and see in this way, suppose someone were to say this: ‘The ascetic Gotama has no superhuman distinction in knowledge and vision worthy of the noble ones. He teaches what he’s worked out by logic, following a line of inquiry, expressing his own perspective.’ Unless they give up that speech and that thought, and let go of that view, they will be cast down to hell.\footnote{This seems harsh, but the suttas see such denigration of the Buddha as deeply wicked, since it closes off the path to Nibbana, as at \href{https://suttacentral.net/sn42.9/en/sujato\#4.12}{SN 42.9:4.12} where a village chief accuses the Buddha of destroying families by encouraging renunciation. At \href{https://suttacentral.net/dn24/en/sujato\#1.16.13}{DN 24:1.16.13} the Buddha warns Sunakkhatta that the naked ascetic \textsanskrit{Pāṭikaputta}, without likewise giving up his speech, thought, and view, will not even be able to enter the Buddha’s presence, much less debate him. } Just as a mendicant accomplished in ethics, immersion, and wisdom would reach enlightenment in this very life, such is the consequence, I say. Unless they give up that speech and that thought, and let go of that view, they will be cast down to hell. 

\textsanskrit{Sāriputta},\marginnote{22.1} a Realized One has four kinds of self-assurance. With these he claims the bull’s place, roars his lion’s roar in the assemblies, and turns the divine wheel.\footnote{These four “kinds of self-assurance” (\textit{\textsanskrit{vesārajja}}) are also found at \href{https://suttacentral.net/an4.8/en/sujato}{AN 4.8}. The first three are included in the Buddha’s dressing down of Sarabha at \href{https://suttacentral.net/an3.64/en/sujato}{AN 3.64}. } What four? 

I\marginnote{23.1} see no reason for anyone—whether ascetic, brahmin, god, \textsanskrit{Māra}, or the Divinity, or anyone else in the world—to legitimately scold me, saying: ‘You claim to be a fully awakened Buddha, but you don’t understand these things.’\footnote{In his first discourse, the Buddha says that he did not claim to be fully awakened until he had understand the four noble truths in all three rounds and twelve aspects (\href{https://suttacentral.net/sn56.11/en/sujato\#10.1}{SN 56.11:10.1}). | Here \textit{nimitta} means “reason, ground, basis”. } Since I see no such reason, I live secure, fearless, and assured. 

I\marginnote{24.1} see no reason for anyone—whether ascetic, brahmin, god, \textsanskrit{Māra}, or the Divinity, or anyone else in the world—to legitimately scold me, saying: ‘You claim to have ended all defilements, but you still have these defilements.’ Since I see no such reason, I live secure, fearless, and assured. 

I\marginnote{25.1} see no reason for anyone—whether ascetic, brahmin, god, \textsanskrit{Māra}, or the Divinity, or anyone else in the world—to legitimately scold me, saying: ‘The acts that you say are obstructions are not really obstructions for the one who performs them.’\footnote{This was the accusation of \textsanskrit{Ariṭṭha} (\href{https://suttacentral.net/mn22/en/sujato}{MN 22}). The Chinese parallel identifies the “obstructions” as desire and lust. } Since I see no such reason, I live secure, fearless, and assured. 

I\marginnote{26.1} see no reason for anyone—whether ascetic, brahmin, god, \textsanskrit{Māra}, or the Divinity, or anyone else in the world—to legitimately scold me, saying: ‘The teaching doesn’t lead those who practice it to the complete ending of suffering, the goal for which it is taught.’ Since I see no such reason, I live secure, fearless, and assured. 

A\marginnote{27.1} Realized One has these four kinds of self-assurance. With these he claims the bull’s place, roars his lion’s roar in the assemblies, and turns the divine wheel. 

When\marginnote{28.1} I know and see in this way, suppose someone were to say this: ‘The ascetic Gotama has no superhuman distinction in knowledge and vision worthy of the noble ones …’ Unless they give up that speech and that thought, and let go of that view, they will be cast down to hell. 

\textsanskrit{Sāriputta},\marginnote{29.1} there are these eight assemblies.\footnote{Mentioned also at \href{https://suttacentral.net/an8.69/en/sujato}{AN 8.69}, \href{https://suttacentral.net/dn16/en/sujato\#3.21.3}{DN 16:3.21.3}, and \href{https://suttacentral.net/dn33/en/sujato\#3.1.139}{DN 33:3.1.139}. } What eight? The assemblies of aristocrats, brahmins, householders, and ascetics. An assembly of the gods of the four great kings. An assembly of the gods of the thirty-three. An assembly of \textsanskrit{Māras}. An assembly of divinities. These are the eight assemblies. Possessing these four kinds of self-assurance, the Realized One approaches and enters right into these eight assemblies.\footnote{\href{https://suttacentral.net/an8.69/en/sujato}{AN 8.69} and \href{https://suttacentral.net/dn16/en/sujato\#3.21.3}{DN 16:3.21.3} both relate the Buddha entering these assemblies, but neither they, nor the Chinese parallel to this passage, attribute his confidence to the four kinds of self-assurance. } I recall having approached an assembly of hundreds of aristocrats.\footnote{\textit{\textsanskrit{Anekasataṁ} \textsanskrit{khattiyaparisaṁ}} is singular, “an assembly of hundreds of aristocrats”, rather than “hundreds of assemblies of aristocrats”. } There I used to sit with them, converse, and engage in discussion. But I don’t see any reason to feel afraid or insecure. Since I see no such reason, I live secure, fearless, and assured. 

I\marginnote{30.1} recall having approached an assembly of hundreds of brahmins … householders … ascetics … the gods of the four great kings … the gods of the thirty-three … \textsanskrit{Māras} … divinities. There too I used to sit with them, converse, and engage in discussion. But I don’t see any reason to feel afraid or insecure. Since I see no such reason, I live secure, fearless, and assured. 

When\marginnote{31.1} I know and see in this way, suppose someone were to say this: ‘The ascetic Gotama has no superhuman distinction in knowledge and vision worthy of the noble ones …’ Unless they give up that speech and that thought, and let go of that view, they will be cast down to hell. 

\textsanskrit{Sāriputta},\marginnote{32.1} there are these four kinds of reproduction.\footnote{These are absent from the Chinese parallel. } What four? Reproduction for creatures born from an egg, from a womb, from moisture, or spontaneously.\footnote{By this the Buddha shows 
 how his knowledge of birth extends to all life. Certain mystical beings such as dragons (\textit{\textsanskrit{nāga}}, \href{https://suttacentral.net/sn29.1/en/sujato\#1.4}{SN 29.1:1.4}) and phoenixes (\textit{\textsanskrit{supaṇṇa}} or \textit{\textsanskrit{garuḍa}}, \href{https://suttacentral.net/sn30.1/en/sujato\#1.4}{SN 30.1:1.4}) are said to be born in all these ways. } 

And\marginnote{33.1} what is reproduction from an egg? There are beings who are born by breaking out of an eggshell. This is called reproduction from an egg. And what is reproduction from a womb? There are beings who are born by breaking out of the amniotic sac. This is called reproduction from a womb. And what is reproduction from moisture? There are beings who are born in a rotten fish, in a rotten carcass, in rotten dough, in a cesspool or a sump. This is called reproduction from moisture. And what is spontaneous reproduction? Gods, hell-beings, certain humans, and certain beings in the lower realms. This is called spontaneous reproduction. These are the four kinds of reproduction. 

When\marginnote{34.1} I know and see in this way, suppose someone were to say this: ‘The ascetic Gotama has no superhuman distinction in knowledge and vision worthy of the noble ones …’ Unless they give up that speech and that thought, and let go of that view, they will be cast down to hell. 

There\marginnote{35.1} are these five destinations. What five? Hell, the animal realm, the ghost realm, humanity, and the gods.\footnote{The \textit{asuras} (“titans”) are included with the gods in this classification; gods and titans are close enough to allow intermarriage (\href{https://suttacentral.net/sn11.13/en/sujato\#11.1}{SN 11.13:11.1}). The Chinese parallel, in common with later traditions, adds the titans as a separate realm here. } 

I\marginnote{36.1} understand hell, and the path and practice that leads to hell. And I understand how someone practicing that way, when their body breaks up, after death, is reborn in a place of loss, a bad place, the underworld, hell. I understand the animal realm … the ghost realm … humanity … gods, and the path and practice that leads to the world of the gods. And I understand how someone practicing that way, when their body breaks up, after death, is reborn in a good place, a heavenly realm. And I understand extinguishment, and the path and practice that leads to extinguishment. And I understand how someone practicing that way realizes the undefiled freedom of heart and freedom by wisdom in this very life, and lives having realized it with their own insight due to the ending of defilements. 

When\marginnote{37.1} I’ve comprehended the mind of a certain person, I understand:\footnote{Here the Buddha describes the process by which he knows where paths lead. First he comprehends their minds, then he infers their destiny, then he confirms the outcome. It is only in this limited way that he knows the future. If, as popular tradition would have it, he was literally omniscient, there would be no need to undertake such a laborious process involving inference, as he could simply directly see whatever he wanted. } ‘This person is practicing in such a way and is riding upon such a path that when their body breaks up, after death, they will be reborn in a place of loss, a bad place, the underworld, hell.’ Some time later I see that they have indeed been reborn in hell, where they experience exclusively painful feelings, sharp and severe. Suppose there was a pit of glowing coals deeper than a man’s height, full of glowing coals that neither flamed nor smoked.\footnote{Such pits are frequently mentioned in the suttas and must have been a common sight. Since fired bricks were not regularly produced, perhaps they were used for metals or pottery, although kilns are also mentioned (\href{https://suttacentral.net/an7.66/en/sujato\#8.3}{AN 7.66:8.3}, \href{https://suttacentral.net/sn12.51/en/sujato\#12.1}{SN 12.51:12.1}). } Then along comes a person struggling in the oppressive heat, weary, thirsty, and parched. And they have set out on a path that meets with that same pit of coals. If a person with clear eyes saw them, they’d say: ‘This person is proceeding in such a way and is riding upon such a path that they will arrive at that very pit of coals.’ Some time later they see that they have indeed fallen into that pit of coals, where they experience exclusively painful feelings, sharp and severe. … 

When\marginnote{38.1} I’ve comprehended the mind of a certain person, I understand: ‘This person … will be reborn in the animal realm.’ Some time later I see that they have indeed been reborn in the animal realm, where they suffer painful feelings, sharp and severe.\footnote{Text omits \textit{ekanta} (“exclusively”), as animals experience both pleasure and pain. } Suppose there was a sewer deeper than a man’s height, full to the brim with feces. Then along comes a person struggling in the oppressive heat, weary, thirsty, and parched. And they have set out on a path that meets with that same sewer. If a person with clear eyes saw them, they’d say: ‘This person is proceeding in such a way and is riding upon such a path that they will arrive at that very sewer.’ Some time later they see that they have indeed fallen into that sewer, where they suffer painful feelings, sharp and severe. … 

When\marginnote{39.1} I’ve comprehended the mind of a certain person, I understand: ‘This person … will be reborn in the ghost realm.’ Some time later I see that they have indeed been reborn in the ghost realm, where they experience mostly painful feelings.\footnote{The pain in the ghost realm is not sharp, severe, and exclusive like in hell, nor just sharp and severe as in the animal realm, but merely “mostly painful” (\textit{\textsanskrit{dukkhabahulā} \textsanskrit{vedanā}}). The pain level is gradually diminishing, and with the human realm it flips, becoming mostly pleasant, then exclusively pleasant in the heaven realms. } Suppose there was a tree growing on rugged ground, with thin foliage casting dappled shade.\footnote{Perhaps the tree was in Australia, where the ground is stony, the leaf coverage scanty, and the flies ravenous. } Then along comes a person struggling in the oppressive heat, weary, thirsty, and parched. And they have set out on a path that meets with that same tree. If a person with clear eyes saw them, they’d say: ‘This person is proceeding in such a way and is riding upon such a path that they will arrive at that very tree.’ Some time later they see them sitting or lying under that tree, where they experience mostly painful feelings. … 

When\marginnote{40.1} I’ve comprehended the mind of a certain person, I understand: ‘This person … will be reborn among human beings.’ Some time later I see that they have indeed been reborn among human beings, where they experience mostly pleasant feelings. Suppose there was a tree growing on smooth ground, with abundant foliage casting dense shade. Then along comes a person struggling in the oppressive heat, weary, thirsty, and parched. And they have set out on a path that meets with that same tree. If a person with clear eyes saw them, they’d say: ‘This person is proceeding in such a way and is riding upon such a path that they will arrive at that very tree.’ Some time later they see them sitting or lying under that tree, where they experience mostly pleasant feelings. … 

When\marginnote{41.1} I’ve comprehended the mind of a certain person, I understand: ‘This person … will be reborn in a good place, a heavenly realm.’ Some time later I see that they have indeed been reborn in a heavenly realm, where they experience feelings of perfect happiness. Suppose there was a stilt longhouse with a peaked roof, plastered inside and out, draft-free, with door fastened and window shuttered. And it had a couch spread with woolen covers—shag-piled, pure white, or embroidered with flowers—and spread with a fine deer hide, with a canopy above and red pillows at both ends. Then along comes a person struggling in the oppressive heat, weary, thirsty, and parched. And they have set out on a path that meets with that same stilt longhouse. If a person with clear eyes saw them, they’d say: ‘This person is proceeding in such a way and is riding upon such a path that they will arrive at that very stilt longhouse.’ Some time later they see them sitting or lying in that stilt longhouse, where they experience feelings of perfect happiness. … 

When\marginnote{42.1} I’ve comprehended the mind of a certain person, I understand: ‘This person is practicing in such a way and is riding upon such a path that they will realize the undefiled freedom of heart and freedom by wisdom in this very life, and live having realized it with their own insight due to the ending of defilements.’ Some time later I see that they have indeed realized the undefiled freedom of heart and freedom by wisdom in this very life, and live having realized it with their own insight due to the ending of defilements, experiencing feelings of perfect happiness. Suppose there was a lotus pond with clear, sweet, cool water, clean, with smooth banks, delightful. Nearby was a dense forest grove.\footnote{We have met \textit{tibba} above in the sense of “sharp” feelings. At \href{https://suttacentral.net/mnd14/en/sujato\#58.13}{Mnd 14:58.13}, such a forest is said to be one of the four kinds of darkness. } Then along comes a person struggling in the oppressive heat, weary, thirsty, and parched. And they have set out on a path that meets with that same lotus pond. If a person with clear eyes saw them, they’d say: ‘This person is proceeding in such a way and is riding upon such a path that they will arrive at that very lotus pond.’ Some time later they would see that person after they had plunged into that lotus pond, bathed and drunk. When all their stress, weariness, and heat exhaustion had faded away, they emerged and sat or lay down in that woodland thicket, where they experienced feelings of perfect happiness. 

In\marginnote{42.10} the same way, when I’ve comprehended the mind of a person, I understand: ‘This person is practicing in such a way and is riding upon such a path that they will realize the undefiled freedom of heart and freedom by wisdom in this very life, and live having realized it with their own insight due to the ending of defilements.’ Some time later I see that they have indeed realized the undefiled freedom of heart and freedom by wisdom in this very life, and live having realized it with their own insight due to the ending of defilements, experiencing feelings of perfect happiness. These are the five destinations. 

When\marginnote{43.1} I know and see in this way, suppose someone were to say this: ‘The ascetic Gotama has no superhuman distinction in knowledge and vision worthy of the noble ones. He teaches what he’s worked out by logic, following a line of inquiry, expressing his own perspective.’ Unless they give up that speech and that thought, and let go of that view, they will be cast down to hell. Just as a mendicant accomplished in ethics, immersion, and wisdom would reach enlightenment in this very life, such is the consequence, I say. Unless they give up that speech and that thought, and let go of that view, they will be cast down to hell. 

\textsanskrit{Sāriputta},\marginnote{44.1} I recall having practiced a spiritual path consisting of four factors.\footnote{The Buddha evidently introduces this topic because Sunakkhatta was impressed by displays of austerity. It is a unique description of the Bodhisatta’s practices before awakening. The practices, which for the most part sound very Jain-like, were undertaken in various periods during the six years after the Bodhisatta went forth and before his awakening. | The supposedly Jain “fourfold restraint” is recorded at \href{https://suttacentral.net/mn56/en/sujato\#12.2}{MN 56:12.2} and \href{https://suttacentral.net/dn2/en/sujato\#29.2}{DN 2:29.2}, while at \href{https://suttacentral.net/dn25/en/sujato\#16.4}{DN 25:16.4} the Buddha reinterprets it as keeping precepts. } I used to be a fervent mortifier, the ultimate fervent mortifier. I used to live rough, the ultimate rough-liver. I used to live in disgust of sin, the ultimate one living in disgust of sin. I used to be secluded, in ultimate seclusion.\footnote{“Fervent mortifier” (\textit{\textsanskrit{tapassī}}), “rough-liver” (\textit{\textsanskrit{lūkha}}), “one living in disgust of sin” (\textit{\textsanskrit{jegucchī}}), “secluded” (\textit{pavivitta}). } 

And\marginnote{45.1} this is what my fervent mortification was like. I went naked, ignoring conventions. I licked my hands, and didn’t come or stop when asked. I didn’t consent to food brought to me, or food prepared specially for me, or an invitation for a meal.\footnote{\textit{Tapas} (“ardor, fire. fervor, fervent mortification”) is the raging flame of righteous pain that courses through body when it is pushed to its extremes. This “fervor” burns off the corrupting traces of \textit{kamma} and defilements. | Buddhist mendicants may not receive food in their hands, nor lick them while eating. Followers of the practices listed here would have walked steadily and randomly for alms, accepting only what was given at the time. } I didn’t receive anything from a pot or bowl; or from someone who keeps sheep, or who has a weapon or a shovel in their home; or where a couple is eating; or where there is a woman who is pregnant, breastfeeding, or who lives with a man; or where food for distribution is advertised; or where there’s a dog waiting or flies buzzing. I accepted no fish or meat or beer or wine, and drank no fermented gruel.\footnote{Keeping sheep (\textit{\textsanskrit{eḷaka}}, for slaughter) goes against the Jain principle of non-violence, as does keeping weapons (\textit{\textsanskrit{daṇḍa}}). | A \textit{musala} often means “pestle”, but it can also be a “shovel”; at \href{https://suttacentral.net/mn81/en/sujato\#18.12}{MN 81:18.12} it is regarded as a virtue to not use one to dig the soil (which is regarded as being alive in Jainism). | \textit{Thusodaka} is an alcoholic porridge fermented from grain-husks, mentioned alongside \textit{\textsanskrit{sovīraka}} in the Pali commentaries and \textsanskrit{Carakasaṁhitā} 27g.191. } I went to just one house for alms, taking just one mouthful, or two houses and two mouthfuls, up to seven houses and seven mouthfuls. I fed on one saucer a day, two saucers a day, up to seven saucers a day. I ate once a day, once every second day, up to once a week, and so on, even up to once a fortnight. I lived committed to the practice of eating food at set intervals. 

I\marginnote{45.6} ate herbs, millet, wild rice, poor rice, water lettuce, rice bran, scum from boiling rice, sesame flour, grass, or cow dung. I survived on forest roots and fruits, or eating fallen fruit.\footnote{It is not easy to meaningfully distinguish the various kinds of grain. } 

I\marginnote{45.7} wore robes of sunn hemp, mixed hemp, corpse-wrapping cloth, rags, lodh tree bark, antelope hide (whole or in strips), kusa grass, bark, wood-chips, human hair, horse-tail hair, or owls’ wings.\footnote{All are extremely uncomfortable. Christian ascetics wore a “hair shirt” in order to “mortify the flesh” . } I tore out hair and beard, committed to this practice.\footnote{Jain ascetics tear out their hair at ordination, rather than shaving. } I constantly stood, refusing seats.\footnote{Remaining in one posture for months or years at a time is one of the most difficult practices. } I squatted, committed to the endeavor of squatting. I lay on a mat of thorns, making a mat of thorns my bed. I was devoted to ritual bathing three times a day, including the evening.\footnote{This seems out of place here. It was a Brahmanical practice (\href{https://suttacentral.net/sn7.21/en/sujato}{SN 7.21}), as the Jains refused to bathe at all, hence the “rough living” of the following section. Indeed, bathing three times a day in the Indian climate would, for most of the year, be quite pleasant. Sometimes, however, it was deliberately done in the cold (\href{https://suttacentral.net/ud1.9/en/sujato\#1.3}{Ud 1.9:1.3}). } And so I lived committed to practicing these various ways of mortifying and tormenting the body. Such was my practice of fervent mortification. 

And\marginnote{46.1} this is what my rough living was like.\footnote{“Rough” (\textit{\textsanskrit{lūkha}}) is related to the Sanskrit \textit{\textsanskrit{rūkṣa}}, among the meanings of which is “soiled, smeared, dirtied”. } The dust and dirt built up on my body over many years until it started flaking off. It’s like the trunk of a pale-moon ebony tree, which builds up bark over many years until it starts flaking off. But it didn’t occur to me: ‘Oh, this dust and dirt must be rubbed off by my hand or another’s.’ That didn’t occur to me. Such was my rough living. 

And\marginnote{47.1} this is what my living in disgust of sin was like. I’d step forward or back ever so mindfully, so I was full of pity regarding even a drop of water, thinking:\footnote{The Bodhisatta learned mindfulness while doing Jain-like practices and later incorporated it into the eightfold path (see also \href{https://suttacentral.net/mn36/en/sujato\#20.7}{MN 36:20.7}). He also learned mindfulness under Brahmanical teachers (\href{https://suttacentral.net/mn26/en/sujato\#15.17}{MN 26:15.17}). | “Pity” is \textit{\textsanskrit{dayā}}, which is similar to the more commonly found terms \textit{\textsanskrit{karuṇā}} or \textit{anukampa}. } ‘May I not injure any little creatures on unclear ground.’\footnote{To this day, Jain ascetics carry a soft broom to gently sweep away any insects in their path. | “Unclear ground” (\textit{visamagata}) is explained by the commentary as ground that is uneven or covered in grass, branches, etc. A lion is said to reflect in the same way when setting out on a hunt (\href{https://suttacentral.net/an10.21/en/sujato\#1.7}{AN 10.21:1.7}). | Compare \textsanskrit{Bhāgavata} \textsanskrit{Purāṇa} 5.10.2, where a man carrying a palanquin is so scrupulous about avoiding treading on ants that he stops and starts, making the progress of the palanquin unsteady (\textit{\textsanskrit{viṣama}-\textsanskrit{gatāṁ}}). } Such was my living in disgust of sin. 

And\marginnote{48.1} this is what my seclusion was like.\footnote{“Seclusion” (\textit{\textsanskrit{viviktaśayyāsana}}) is classified by the Jains as an “external austerity” (\textit{\textsanskrit{bāhyatapas}}) at \textsanskrit{Tattvārthasūtra} 9.19. } I would plunge deep into a wilderness region and stay there. When I saw a cowherd or a shepherd, or someone gathering grass or sticks, or a lumberjack, I’d flee from forest to forest, from thicket to thicket, from valley to valley, from uplands to uplands. Why is that? So that I wouldn’t see them, nor they me. I fled like a wild deer seeing a human being. Such was my practice of seclusion.\footnote{This concludes the “spiritual path consisting of four factors” of \href{https://suttacentral.net/mn12/en/sujato\#44.1}{MN 12:44.1}. } 

I\marginnote{49.1} would go on all fours into the cow-pens after the cattle had left and eat the dung of the young suckling calves. As long as my own urine and excrement lasted, I would even eat that. Such was my eating of most unnatural things.\footnote{At \href{https://suttacentral.net/pli-tv-kd6/en/sujato\#14.6.3}{Kd 6:14.6.3} the four “great unnaturals” (or “filthy edibles”, \textit{\textsanskrit{mahāvikaṭa}}) are said to be feces, urine, ash, and clay. } 

I\marginnote{50.1} would plunge deep into an awe-inspiring forest grove and stay there. It was so awe-inspiring that normally it would make your hair stand on end if you weren’t free of greed. And on days such as the cold spell when the snow falls in the January winter, I stayed in the open by night and in the forest by day.\footnote{\textit{\textsanskrit{Antaraṭṭhakā}} (Buddhist Hybrid Sanskrit \textit{\textsanskrit{aṣṭaka}-\textsanskrit{rātri}}) is the coldest part of Indian winter between the festival days that fall on the eighth day after the fullmoon in the two winter months of \textsanskrit{Māgha} and Phagguna. This roughly corresponds with January, the coldest month in northern India. } But in the last month of summer I’d stay in the open by day and in the forest by night. And then these verses, which were neither supernaturally inspired, nor learned before in the past, occurred to me:\footnote{“Not supernaturally inspired” (\textit{\textsanskrit{anacchariyā}}) rejects the Vedic “channeling” of scripture from the Divinity, while “not learned before in the past” (\textit{pubbe \textsanskrit{assutapubbā}}), echoing the Dhammacakkappavattanasutta (\href{https://suttacentral.net/sn56.11/en/sujato\#5.1}{SN 56.11:5.1}), rejects the oral tradition. } 

\begin{verse}%
‘Scorched\marginnote{50.7} and frozen, \\
alone in the awe-inspiring forest. \\
Naked, no fire to sit beside, \\
the sage still pursues his quest.’ 

%
\end{verse}

I\marginnote{51.1} would make my bed in a charnel ground, with the bones of the dead for a pillow. Then village louts would come up to me. They’d spit and piss on me, throw mud on me, even poke sticks in my ears.\footnote{For “village louts” (\textit{\textsanskrit{gāmaṇḍalā}}), compare \textit{\textsanskrit{aṇḍaka}} in the sense “nasty, cruel” at \href{https://suttacentral.net/mn41/en/sujato\#9.4}{MN 41:9.4}. } But I don’t recall ever having a bad thought about them. Such was my abiding in equanimity. 

There\marginnote{52.1} are some ascetics and brahmins who have this doctrine and view: ‘Purity comes from food.’\footnote{Compare \textsanskrit{Chāndogya} \textsanskrit{Upaniṣad} 7.26.2: “from purity of food there is purity of being; from purity of being there is stable memory (mindfulness); relying on memory there is release from all ties” (\textit{\textsanskrit{āhāraśuddhau} \textsanskrit{sattvaśuddhiḥ}; \textsanskrit{sattvaśuddhau} \textsanskrit{dhruvā} \textsanskrit{smṛtiḥ}; \textsanskrit{smṛtilambhe} \textsanskrit{sarvagranthīnāṁ} \textsanskrit{vipramokṣaḥ}}). } They say: ‘Let’s live on jujubes.’ So they eat jujubes and jujube powder, and drink jujube juice. And they enjoy many jujube concoctions. I recall eating just a single jujube. You might think that at that time the jujubes must have been very big. But you should not see it like this. The jujubes then were at most the same size as today. Eating so very little, my body became extremely emaciated. Due to eating so little, my major and minor limbs became like the joints of an eighty-year-old or a dying man,\footnote{\textit{\textsanskrit{Āsītika}} is “eighty-year-old”, \textit{\textsanskrit{kāḷa}} is “time (of death)” (read \textit{\textsanskrit{kāla}}), \textit{pabba} is “joint”. } my bottom became like a camel’s hoof, my vertebrae stuck out like beads on a string, and my ribs were as gaunt as the broken-down rafters on an old barn. Due to eating so little, the gleam of my eyes sank deep in their sockets, like the gleam of water sunk deep down a well. Due to eating so little, my scalp shriveled and withered like a green bitter-gourd in the wind and sun. Due to eating so little, the skin of my belly stuck to my backbone, so that when I tried to rub the skin of my belly I grabbed my backbone, and when I tried to rub my backbone I rubbed the skin of my belly. Due to eating so little, when I tried to urinate or defecate I fell face down right there. Due to eating so little, when I tried to relieve my body by rubbing my limbs with my hands, the hair, rotted at its roots, fell out. 

There\marginnote{53{-}55.1} are some ascetics and brahmins who have this doctrine and view: ‘Purity comes from food.’ They say: ‘Let’s live on mung beans.’ … ‘Let’s live on sesame.’ … ‘Let’s live on ordinary rice.’ … Due to eating so little, when I tried to relieve my body by rubbing my limbs with my hands, the hair, rotted at its roots, fell out. 

But\marginnote{56.1} \textsanskrit{Sāriputta}, I did not achieve any superhuman distinction in knowledge and vision worthy of the noble ones by that conduct, that practice, that grueling work. Why is that? Because I didn’t achieve that noble wisdom that’s noble and emancipating, and which delivers one who practices it to the complete ending of suffering. 

There\marginnote{57.1} are some ascetics and brahmins who have this doctrine and view: ‘Purity comes from transmigration.’\footnote{A view attributed to Makkhali \textsanskrit{Gosāla} (\href{https://suttacentral.net/dn2/en/sujato\#21.3}{DN 2:21.3}). The idea is that transmigration is limited, so one will automatically be released after a certain number of lives. } But it’s not easy to find a realm that I haven’t previously transmigrated to in all this long time, except for the gods of the pure abodes. For if I had transmigrated to the gods of the pure abodes I would not have returned to this realm again.\footnote{T'he gods there are all “non-returners” and will attain arahantship in that realm. } 

There\marginnote{58.1} are some ascetics and brahmins who have this doctrine and view: ‘Purity comes from rebirth.’\footnote{This view and the next appear to be different ways of repeating the same view as before. Perhaps there is a subtle difference between them. } But it’s not easy to find any rebirth that I haven’t previously been reborn in … 

There\marginnote{59.1} are some ascetics and brahmins who have this doctrine and view: ‘Purity comes from abode of rebirth.’ But it’s not easy to find an abode where I haven’t previously abided … 

There\marginnote{60.1} are some ascetics and brahmins who have this doctrine and view: ‘Purity comes from sacrifice.’\footnote{This is, of course, the brahmins. } But it’s not easy to find a sacrifice that I haven’t previously offered in all this long time, when I was an anointed aristocratic king or a well-to-do brahmin. 

There\marginnote{61.1} are some ascetics and brahmins who have this doctrine and view: ‘Purity comes from serving the sacred flame.’\footnote{Again, serving the sacred flame is a fundamental, perhaps \emph{the} fundamental, rite of the brahmins. } But it’s not easy to find a fire that I haven’t previously served in all this long time, when I was an anointed aristocratic king or a well-to-do brahmin. 

There\marginnote{62.1} are some ascetics and brahmins who have this doctrine and view: ‘So long as this gentleman is youthful, young, with pristine black hair, blessed with youth, in the prime of life he will be endowed with perfect lucidity of wisdom. But when he’s old, elderly, and senior, advanced in years, and has reached the final stage of life—eighty, ninety, or a hundred years old—he will lose his lucidity of wisdom.’ But you should not see it like this. For now I am old, elderly, and senior, I’m advanced in years, and have reached the final stage of life. I am eighty years old.\footnote{This situates this discourse, and hence the Sunakkhatta cycle as a whole, near the end of the Buddha’s life. } Suppose I had four disciples with a lifespan of a hundred years. And they each were perfect in memory, range, retention, and perfect lucidity of wisdom.\footnote{Memory (\textit{sati}), range (\textit{gati}), retention (\textit{dhiti}), and perfect lucidity of wisdom (\textit{\textsanskrit{paññāveyyattiya}}). } Imagine how easily a well-trained expert archer with a strong bow would shoot a light arrow across the shadow of a palm tree. That’s how extraordinary they were in memory, range, retention, and perfect lucidity of wisdom. They’d bring up questions about the four kinds of mindfulness meditation again and again, and I would answer each question. They’d remember the answers and not ask the same question twice. And they’d pause only to eat and drink, go to the toilet, and sleep to dispel weariness. But the Realized One would not run out of Dhamma teachings, words and phrases of the teachings, or spontaneous answers. And at the end of a hundred years my four disciples would pass away. Even if you have to carry me around on a stretcher, there will never be any deterioration in the Realized One’s lucidity of wisdom. 

And\marginnote{63.1} if there’s anyone of whom it may be rightly said that a being not liable to delusion has arisen in the world for the welfare and happiness of the people, out of sympathy for the world, for the benefit, welfare, and happiness of gods and humans, it’s of me that this should be said.” 

Now\marginnote{64.1} at that time Venerable \textsanskrit{Nāgasamāla} was standing behind the Buddha fanning him.\footnote{We meet \textsanskrit{Nāgasamāla} in less auspicious circumstances at \href{https://suttacentral.net/ud8.7/en/sujato}{Ud 8.7}, where he stubbornly disagrees with the Buddha on which path to take. } Then he said to the Buddha: 

“It’s\marginnote{64.3} incredible, sir, it’s amazing! While I was listening to this exposition of the teaching my hair stood up! What is the name of this exposition of the teaching?” 

“Well\marginnote{64.6} then, \textsanskrit{Nāgasamāla}, you may remember this exposition of the teaching as ‘The Hair-raising Discourse’.”\footnote{This original title is widely attested in the tradition, being found in some \textit{\textsanskrit{uddāna}} summary verses in Burmese and Sinhalese manuscripts, \href{https://suttacentral.net/mil7.5.7/en/sujato\#2.1}{Mil 7.5.7:2.1}, a \textsanskrit{Jātaka} of the same name (\href{https://suttacentral.net/ja94/en/sujato}{Ja 94}), several commentaries, and the Sanskrit and Chinese parallels. The subcommentary says that the name \textsanskrit{Mahāsīhanāda} was given by the redactors. } 

That\marginnote{64.7} is what the Buddha said. Satisfied, Venerable \textsanskrit{Nāgasamāla} approved what the Buddha said. 

%
\section*{{\suttatitleacronym MN 13}{\suttatitletranslation The Longer Discourse on the Mass of Suffering }{\suttatitleroot Mahādukkhakkhandhasutta}}
\addcontentsline{toc}{section}{\tocacronym{MN 13} \toctranslation{The Longer Discourse on the Mass of Suffering } \tocroot{Mahādukkhakkhandhasutta}}
\markboth{The Longer Discourse on the Mass of Suffering }{Mahādukkhakkhandhasutta}
\extramarks{MN 13}{MN 13}

\scevam{So\marginnote{1.1} I have heard. }At one time the Buddha was staying near \textsanskrit{Sāvatthī} in Jeta’s Grove, \textsanskrit{Anāthapiṇḍika}’s monastery. 

Then\marginnote{2.1} several mendicants robed up in the morning and, taking their bowls and robes, entered \textsanskrit{Sāvatthī} for alms. Then it occurred to them, “It’s too early to wander for alms in \textsanskrit{Sāvatthī}. Why don’t we visit the monastery of the wanderers of other religions?” Then they went to the monastery of the wanderers of other religions and exchanged greetings with the wanderers there. When the greetings and polite conversation were over, they sat down to one side. The wanderers said to them: 

“Reverends,\marginnote{3.1} the ascetic Gotama advocates the complete understanding of sensual pleasures, and so do we.\footnote{This group of three—\textit{\textsanskrit{kāma}}, \textit{\textsanskrit{rūpa}}, \textit{\textsanskrit{vedanā}}—is also found at \href{https://suttacentral.net/an3.126/en/sujato}{AN 3.126} and \href{https://suttacentral.net/an10.29/en/sujato}{AN 10.29}, where it also relates to the teachings of outsiders. } The ascetic Gotama advocates the complete understanding of forms, and so do we. The ascetic Gotama advocates the complete understanding of feelings, and so do we.\footnote{The text does not say what the wanderers understood by these three full understandings. The commentary says they spoke in reference to, respectively, the first absorption, the formless attainments, and the non-percipient state. } What, then, is the difference between the ascetic Gotama’s teaching and instruction and ours?”\footnote{In questioning this, the wanderers display better discernment than many today, who leap all too readily from the discovery of something in common between religions to the assertion that they are therefore the same. } 

Those\marginnote{4.1} mendicants neither approved nor dismissed that statement of the wanderers of other religions. They got up from their seat, thinking, “We will learn the meaning of this statement from the Buddha himself.”\footnote{This is the quintessential Buddhist attitude when encountering something unknown: to neither accept nor reject, but inquire. } 

Then,\marginnote{5.1} after the meal, when they returned from almsround, they went up to the Buddha, bowed, sat down to one side, and told him what had happened. The Buddha said: 

“Mendicants,\marginnote{6.1} when wanderers of other religions say this, you should say to them: ‘But reverends, what’s the gratification, the drawback, and the escape when it comes to sensual pleasures?\footnote{This set of three questions is a key analytical approach in the suttas. Without an appreciation of both the good and bad sides of things we cannot properly understand them and be free of craving for them. } What’s the gratification, the drawback, and the escape when it comes to forms? What’s the gratification, the drawback, and the escape when it comes to feelings?’ Questioned like this, the wanderers of other religions would be stumped, and, in addition, would get frustrated. Why is that? Because they’re out of their element. I don’t see anyone in this world—with its gods, \textsanskrit{Māras}, and Divinities, this population with its ascetics and brahmins, its gods and humans—who could provide a satisfying answer to these questions except for the Realized One or his disciple or someone who has heard it from them. 

And\marginnote{7.1} what is the gratification of sensual pleasures? There are these five kinds of sensual stimulation. What five? Sights known by the eye, which are likable, desirable, agreeable, pleasant, sensual, and arousing. Sounds known by the ear … Smells known by the nose … Tastes known by the tongue … Touches known by the body, which are likable, desirable, agreeable, pleasant, sensual, and arousing. These are the five kinds of sensual stimulation. The pleasure and happiness that arise from these five kinds of sensual stimulation: this is the gratification of sensual pleasures. 

And\marginnote{8.1} what is the drawback of sensual pleasures? It’s when a gentleman earns a living by means such as arithmetic, accounting, calculating, farming, trade, raising cattle, archery, government service, or one of the professions. But they must face cold and heat, being hurt by the touch of flies, mosquitoes, wind, sun, and reptiles, and risking death from hunger and thirst. This is a drawback of sensual pleasures apparent in the present life, a mass of suffering caused by sensual pleasures. 

That\marginnote{9.1} gentleman might try hard, strive, and make an effort, but fail to earn any money. If this happens, they sorrow and wail and lament, beating their breast and falling into confusion, saying: ‘Oh, my hard work is wasted. My efforts are fruitless!’ This too is a drawback of sensual pleasures apparent in the present life, a mass of suffering caused by sensual pleasures. 

That\marginnote{10.1} gentleman might try hard, strive, and make an effort, and succeed in earning money. But they experience pain and sadness when they try to protect it, thinking: ‘How can I prevent my wealth from being taken by rulers or bandits, consumed by fire, swept away by flood, or taken by unloved heirs?’ And even though they protect it and ward it, rulers or bandits take it, or fire consumes it, or flood sweeps it away, or unloved heirs take it. They sorrow and wail and lament, beating their breast and falling into confusion: ‘What once was mine is gone.’ This too is a drawback of sensual pleasures apparent in the present life, a mass of suffering caused by sensual pleasures. 

Furthermore,\marginnote{11.1} for the sake of sensual pleasures kings fight with kings, aristocrats fight with aristocrats, brahmins fight with brahmins, and householders fight with householders. A mother fights with her child, child with mother, father with child, and child with father. Brother fights with brother, brother with sister, sister with brother, and friend fights with friend. Once they’ve started quarreling, arguing, and disputing, they attack each other with fists, stones, rods, and swords, resulting in death and deadly pain. This too is a drawback of sensual pleasures apparent in the present life, a mass of suffering caused by sensual pleasures. 

Furthermore,\marginnote{12.1} for the sake of sensual pleasures they don their sword and shield, fasten their bow and arrows, and plunge into a battle massed on both sides, with arrows and spears flying and swords flashing. There they are struck with arrows and spears, and their heads are chopped off, resulting in death and deadly pain. This too is a drawback of sensual pleasures apparent in the present life, a mass of suffering caused by sensual pleasures. 

Furthermore,\marginnote{13.1} for the sake of sensual pleasures they don their sword and shield, fasten their bow and arrows, and charge wetly plastered bastions, with arrows and spears flying and swords flashing.\footnote{Bastions were “wetly plastered” (\textit{\textsanskrit{addāvalepana}}) for resistance from fire (\href{https://suttacentral.net/sn35.243/en/sujato\#7.1}{SN 35.243:7.1}). } There they are struck with arrows and spears, splashed with dung, crushed by a superior force, and their heads are chopped off,\footnote{For “splashed with dung” (\textit{\textsanskrit{chakaṇakāyapi} \textsanskrit{osiñcanti}}), PTS reads \textit{\textsanskrit{pakkaṭṭhī}}, but this word appears to be spurious. The commentary says \textit{\textsanskrit{chakaṇakā}} means “boiling cowdung”, but nothing in the word itself suggests “boiling”. The Chinese parallels at MA 99 and T 53 have “molten copper” \langlzh{融銅}, while EA 21.9 has “molten iron” (\langlzh{消鐵}). | \textit{Abhivagga} is only found here in Pali. The Atharvaveda (3.5.2, 6.54.2, 11.2.4) has \textit{\textsanskrit{abhīvarga}} apparently in the sense of “domain”. The commentary says \textit{abhivagga} was a “hundred-toothed” weapon that crushed invaders of a castle. However, \textit{vagga} has the recognized military sense of “cadre, company, platoon”, so I take \textit{abhivagga} to mean “superior force”. } resulting in death and deadly pain. This too is a drawback of sensual pleasures apparent in the present life, a mass of suffering caused by sensual pleasures. 

Furthermore,\marginnote{14.1} for the sake of sensual pleasures they break into houses, plunder wealth, steal from isolated buildings, commit highway robbery, and commit adultery. The rulers would arrest them and subject them to various punishments—whipping, caning, and clubbing; cutting off hands or feet, or both; cutting off ears or nose, or both; the ‘porridge pot’, the ‘shell-shave’, the ‘\textsanskrit{Rāhu}’s mouth’, the ‘garland of fire’, the ‘burning hand’, the ‘bulrush twist’, the ‘bark dress’, the ‘antelope’, the ‘meat hook’, the ‘coins’, the ‘caustic pickle’, the ‘twisting bar’, the ‘straw mat’; being splashed with hot oil, being fed to the dogs, being impaled alive, and being beheaded.\footnote{The commentary explains these punishments thus. “Porridge pot”: remove the top of the skull and drop in a hot iron ball so that the brains boil over. “Shell-shave”: grind the skull with gravel until it is smooth. “\textsanskrit{Rāhu}”s mouth’: force open the mouth with a skewer, put in oil and wick, and light it so it burns like the sun swallowed by the titan \textsanskrit{Rāhu} (\href{https://suttacentral.net/sn2.8/en/sujato}{SN 2.8}). “Garland of fire”: smear the body with oil and set it alight. “Burning hand”: wrap the hand with oiled rags and set it alight. “Bulrush twist”: flay the skin from the neck down, then twist it into a band by which to hang the victim. “Bark dress”: cut the skin in strips and make it into a garment. “Antelope”: pin the bound victim to the ground and roast them alive. “Meat hook”: flay with double fish-hooks. “Coins”: slice off disks of flesh like coins. “Caustic pickle”: beat the victim, then rub the wounds with caustic solution. “Twisting bar”: pin the victim to the ground by the ears and twirl them by the feet. “Straw mat”: beat them until every bone is broken and the body becomes limp as a mattress. } These result in death and deadly pain. This too is a drawback of sensual pleasures apparent in the present life, a mass of suffering caused by sensual pleasures. 

Furthermore,\marginnote{15.1} for the sake of sensual pleasures, they conduct themselves badly by way of body, speech, and mind. When their body breaks up, after death, they’re reborn in a place of loss, a bad place, the underworld, hell.\footnote{The drawbacks are escalating, the point being that hell is worse than the punishments described above. } This is a drawback of sensual pleasures to do with lives to come, a mass of suffering caused by sensual pleasures.\footnote{Previous drawbacks pertained to what is “apparent in this very life” (\textit{\textsanskrit{sandiṭṭhika}}), whereas this applies to “lives to come” (\textit{\textsanskrit{samparāyika}}). } 

And\marginnote{16.1} what is the escape from sensual pleasures? Removing and giving up desire and greed for sensual pleasures: this is the escape from sensual pleasures.\footnote{This happens with the realization of non-return. } 

There\marginnote{17.1} are ascetics and brahmins who don’t truly understand sensual pleasures’ gratification, drawback, and escape in this way for what they are. It’s impossible for them to completely understand sensual pleasures themselves, or to instruct another so that, practicing accordingly, they will completely understand sensual pleasures. There are ascetics and brahmins who do truly understand sensual pleasures’ gratification, drawback, and escape in this way for what they are. It is possible for them to completely understand sensual pleasures themselves, or to instruct another so that, practicing accordingly, they will completely understand sensual pleasures. 

And\marginnote{18.1} what is the gratification of forms? Suppose there was a girl of the brahmins, aristocrats, or householders in her fifteenth or sixteenth year, neither too tall nor too short, neither too thin nor too fat, neither too dark nor too fair. Is she not at the height of her beauty and prettiness?” 

“Yes,\marginnote{18.3} sir.” 

“The\marginnote{18.4} pleasure and happiness that arise from this beauty and prettiness is the gratification of forms. 

And\marginnote{19.1} what is the drawback of forms? Suppose that some time later you were to see that same sister—eighty, ninety, or a hundred years old—bent double, crooked, leaning on a staff, trembling as they walk, ailing, past their prime, with teeth broken, hair grey and scanty or bald, skin wrinkled, and limbs blotchy. 

What\marginnote{19.3} do you think, mendicants? Has not that former beauty vanished and the drawback become clear?” 

“Yes,\marginnote{19.5} sir.” 

“This\marginnote{19.6} is the drawback of forms. 

Furthermore,\marginnote{20.1} suppose that you were to see that same sister sick, suffering, gravely ill, collapsed in her own urine and feces, being picked up by some and put down by others. 

What\marginnote{20.2} do you think, mendicants? Has not that former beauty vanished and the drawback become clear?” 

“Yes,\marginnote{20.4} sir.” 

“This\marginnote{20.5} too is the drawback of forms. 

Furthermore,\marginnote{21.1} suppose that you were to see that same sister as a corpse discarded in a charnel ground. And it had been dead for one, two, or three days, bloated, livid, and festering.\footnote{As shown in \href{https://suttacentral.net/mn10/en/sujato\#14.3}{MN 10:14.3}, this meditation proceeds not by objectifying the other’s corpse as repulsive, but by identifying “it”—the neuter-gendered \textit{\textsanskrit{sarīra}}—with one’s own body. } 

What\marginnote{21.3} do you think, mendicants? Has not that former beauty vanished and the drawback become clear?” 

“Yes,\marginnote{21.5} sir.” 

“This\marginnote{21.6} too is the drawback of forms. 

Furthermore,\marginnote{22.1} suppose that you were to see that same sister as a corpse discarded in a charnel ground. And it was being devoured by crows, hawks, vultures, herons, dogs, tigers, leopards, jackals, and many kinds of little creatures … 

Furthermore,\marginnote{23{-}28.1} suppose that you were to see that same sister as a corpse discarded in a charnel ground. And it had been reduced to a skeleton with flesh and blood, held together by sinews … a skeleton rid of flesh but smeared with blood, and held together by sinews … a skeleton rid of flesh and blood, held together by sinews … bones rid of sinews scattered in every direction. Here a hand-bone, there a foot-bone, here an ankle bone, there a shin-bone, here a thigh-bone, there a hip-bone, here a rib-bone, there a back-bone, here an arm-bone, there a neck-bone, here a jaw-bone, there a tooth, here the skull. … 

Furthermore,\marginnote{29.1} suppose that you were to see that same sister as a corpse discarded in a charnel ground. And it had been reduced to white bones, the color of shells … decrepit bones, heaped in a pile … bones rotted and crumbled to powder. 

What\marginnote{29.3} do you think, mendicants? Has not that former beauty vanished and the drawback become clear?” 

“Yes,\marginnote{29.5} sir.” 

“This\marginnote{29.6} too is the drawback of forms. 

And\marginnote{30.1} what is the escape from forms? Removing and giving up desire and greed for forms: this is the escape from forms.\footnote{Since “forms” also includes the refined visions of meditation, the full understanding of forms only occurs with arahantship. } 

There\marginnote{31.1} are ascetics and brahmins who don’t truly understand forms’ gratification, drawback, and escape in this way for what they are. It’s impossible for them to completely understand forms themselves, or to instruct another so that, practicing accordingly, they will completely understand forms. There are ascetics and brahmins who do truly understand forms’ gratification, drawback, and escape in this way for what they are. It is possible for them to completely understand forms themselves, or to instruct another so that, practicing accordingly, they will completely understand forms. 

And\marginnote{32.1} what is the gratification of feelings? It’s when a mendicant, quite secluded from sensual pleasures, secluded from unskillful qualities, enters and remains in the first absorption, which has the rapture and bliss born of seclusion, while placing the mind and keeping it connected.\footnote{The Buddha illustrates “feelings” with the highest and most refined possible feelings, those of \textit{\textsanskrit{jhāna}}. } At that time a mendicant doesn’t intend to hurt themselves, hurt others, or hurt both; they feel only feelings that are not hurtful. Freedom from being hurt is the ultimate gratification of feelings, I say. 

Furthermore,\marginnote{33{-}35.1} a mendicant enters and remains in the second absorption … third absorption … fourth absorption. At that time a mendicant doesn’t intend to hurt themselves, hurt others, or hurt both; they feel only feelings that are not hurtful. Freedom from being hurt is the ultimate gratification of feelings, I say. 

And\marginnote{36.1} what is the drawback of feelings? That feelings are impermanent, suffering, and perishable: this is their drawback.\footnote{This describes the stage of insight meditation. After the meditator has attained absorption, they reflect that even those sublime feelings are impermanent. } 

And\marginnote{37.1} what is the escape from feelings? Removing and giving up desire and greed for feelings: this is the escape from feelings.\footnote{This occurs at arahantship. } 

There\marginnote{38.1} are ascetics and brahmins who don’t truly understand feelings’ gratification, drawback, and escape in this way for what they are. It’s impossible for them to completely understand feelings themselves, or to instruct another so that, practicing accordingly, they will completely understand feelings. There are ascetics and brahmins who do truly understand feelings’ gratification, drawback, and escape in this way for what they are. It is possible for them to completely understand feelings themselves, or to instruct another so that, practicing accordingly, they will completely understand feelings.” 

That\marginnote{38.3} is what the Buddha said. Satisfied, the mendicants approved what the Buddha said. 

%
\section*{{\suttatitleacronym MN 14}{\suttatitletranslation The Shorter Discourse on the Mass of Suffering }{\suttatitleroot Cūḷadukkhakkhandhasutta}}
\addcontentsline{toc}{section}{\tocacronym{MN 14} \toctranslation{The Shorter Discourse on the Mass of Suffering } \tocroot{Cūḷadukkhakkhandhasutta}}
\markboth{The Shorter Discourse on the Mass of Suffering }{Cūḷadukkhakkhandhasutta}
\extramarks{MN 14}{MN 14}

\scevam{So\marginnote{1.1} I have heard. }At one time the Buddha was staying in the land of the Sakyans, near Kapilavatthu in the Banyan Tree Monastery. 

Then\marginnote{2.1} \textsanskrit{Mahānāma} the Sakyan went up to the Buddha, bowed, sat down to one side, and said to him,\footnote{\textsanskrit{Mahānāma} was the brother of Anuruddha and Ānanda (\href{https://suttacentral.net/pli-tv-kd17/en/sujato\#1.1.3}{Kd 17:1.1.3}) which, according to the commentary, makes the Buddha his cousin. He remained in the lay life as a devoted and generous follower, keenly interested in developing his practice. The early texts do not record any occasion of his awakening. However, the commentarial claim that he was a once-returner is supported by passages such as \href{https://suttacentral.net/sn55.22/en/sujato\#2.3}{SN 55.22:2.3} and \href{https://suttacentral.net/sn55.21/en/sujato\#3.6}{SN 55.21:3.6} that suggest he was indeed a noble disciple. } “For a long time, sir, I have understood your teaching like this: ‘Greed, hate, and delusion are corruptions of the mind.’ Despite understanding this, sometimes my mind is occupied by thoughts of greed, hate, and delusion. I wonder what qualities remain in me that I have such thoughts?”\footnote{As in the opening of the previous discourse, he exhibits curiosity and openness to inquiry. } 

“\textsanskrit{Mahānāma},\marginnote{3.1} there is a quality that remains in you that makes you have such thoughts. For if you had given up that quality you would not still be living at home and enjoying sensual pleasures. But because you haven’t given up that quality you are still living at home and enjoying sensual pleasures. 

Sensual\marginnote{4.1} pleasures give little gratification and much suffering and distress, and they are all the more full of drawbacks. So, \textsanskrit{Mahānāma}, even though a noble disciple has clearly seen this with right wisdom, as long as they do not achieve the rapture and bliss that are apart from sensual pleasures and unskillful qualities, or something even more peaceful than that,\footnote{The “rapture and bliss that are apart from sensual pleasures and unskillful qualities” includes the first and second \textit{\textsanskrit{jhānas}}. The text is in present tense: they \emph{do not attain} \textit{\textsanskrit{jhāna}}, not \emph{they have never attained} \textit{\textsanskrit{jhāna}}. In order to realize the noble truths, \textsanskrit{Mahānāma} must, of course, have practiced the noble eightfold path, which includes \textit{\textsanskrit{jhāna}}. Indeed, at \href{https://suttacentral.net/an3.73/en/sujato\#1.6}{AN 3.73:1.6} \textsanskrit{Mahānāma} said he understood that in the Buddha’s teaching, “Knowledge is for those with immersion, not those without immersion.” However, a stream-enterer and a once-returner have not fully given up the underlying attachment to sensual pleasures, so unless they are dedicated to regular meditation following that realization, it is possible for their mental clarity to deteriorate and sensual desires to return. It seems that the temptations and business of the lay life had distracted \textsanskrit{Mahānāma} from his meditation. } they can return to sensual pleasures.\footnote{The Pali is expressed in a double negative: “they are not one who does not return”. } But when they do achieve that rapture and bliss, or something more peaceful than that, they do not return to sensual pleasures. 

Before\marginnote{5.1} my awakening—when I was still unawakened but intent on awakening—I too clearly saw with right wisdom that: ‘Sensual pleasures give little gratification and much suffering and distress, and they are all the more full of drawbacks.’ But so long as I didn’t achieve the rapture and bliss that are apart from sensual pleasures and unskillful qualities, or something even more peaceful than that, I didn’t announce that I would not return to sensual pleasures. But when I did achieve that rapture and bliss, or something more peaceful than that, I announced that I would not return to sensual pleasures.\footnote{The phrasing of this passage echoes that of the first sermon (\href{https://suttacentral.net/sn56.11/en/sujato\#9.1}{SN 56.11:9.1}). } 

And\marginnote{6.1} what is the gratification of sensual pleasures? There are these five kinds of sensual stimulation. What five? Sights known by the eye, which are likable, desirable, agreeable, pleasant, sensual, and arousing. Sounds known by the ear … Smells known by the nose … Tastes known by the tongue … Touches known by the body, which are likable, desirable, agreeable, pleasant, sensual, and arousing. These are the five kinds of sensual stimulation. The pleasure and happiness that arise from these five kinds of sensual stimulation: this is the gratification of sensual pleasures. 

And\marginnote{7.1} what is the drawback of sensual pleasures? It’s when a gentleman earns a living by means such as arithmetic, accounting, calculating, farming, trade, raising cattle, archery, government service, or one of the professions. But they must face cold and heat, being hurt by the touch of flies, mosquitoes, wind, sun, and reptiles, and risking death from hunger and thirst. This is a drawback of sensual pleasures apparent in the present life, a mass of suffering caused by sensual pleasures. 

That\marginnote{8.1} gentleman might try hard, strive, and make an effort, but fail to earn any money. If this happens, they sorrow and wail and lament, beating their breast and falling into confusion, saying: ‘Oh, my hard work is wasted. My efforts are fruitless!’ This too is a drawback of sensual pleasures apparent in the present life, a mass of suffering caused by sensual pleasures. 

That\marginnote{9.1} gentleman might try hard, strive, and make an effort, and succeed in earning money. But they experience pain and sadness when they try to protect it, thinking: ‘How can I prevent my wealth from being taken by rulers or bandits, consumed by fire, swept away by flood, or taken by unloved heirs?’ And even though they protect it and ward it, rulers or bandits take it, or fire consumes it, or flood sweeps it away, or unloved heirs take it. They sorrow and wail and lament, beating their breast and falling into confusion: ‘What once was mine is gone.’ This too is a drawback of sensual pleasures apparent in the present life, a mass of suffering caused by sensual pleasures. 

Furthermore,\marginnote{10.1} for the sake of sensual pleasures kings fight with kings, aristocrats fight with aristocrats, brahmins fight with brahmins, and householders fight with householders. A mother fights with her child, child with mother, father with child, and child with father. Brother fights with brother, brother with sister, sister with brother, and friend fights with friend. Once they’ve started quarreling, arguing, and disputing, they attack each other with fists, stones, rods, and swords, resulting in death and deadly pain. This too is a drawback of sensual pleasures apparent in the present life, a mass of suffering caused by sensual pleasures. 

Furthermore,\marginnote{11.1} for the sake of sensual pleasures they don their sword and shield, fasten their bow and arrows, and plunge into a battle massed on both sides, with arrows and spears flying and swords flashing. There they are struck with arrows and spears, and their heads are chopped off, resulting in death and deadly pain. This too is a drawback of sensual pleasures apparent in the present life, a mass of suffering caused by sensual pleasures. 

Furthermore,\marginnote{12.1} for the sake of sensual pleasures they don their sword and shield, fasten their bow and arrows, and charge wetly plastered bastions, with arrows and spears flying and swords flashing. There they are struck with arrows and spears, splashed with dung, crushed by a superior force, and their heads are chopped off, resulting in death and deadly pain. This too is a drawback of sensual pleasures apparent in the present life, a mass of suffering caused by sensual pleasures. 

Furthermore,\marginnote{13.1} for the sake of sensual pleasures they break into houses, plunder wealth, steal from isolated buildings, commit highway robbery, and commit adultery. The rulers would arrest them and subject them to various punishments—whipping, caning, and clubbing; cutting off hands or feet, or both; cutting off ears or nose, or both; the ‘porridge pot’, the ‘shell-shave’, the ‘\textsanskrit{Rāhu}’s mouth’, the ‘garland of fire’, the ‘burning hand’, the ‘bulrush twist’, the ‘bark dress’, the ‘antelope’, the ‘meat hook’, the ‘coins’, the ‘caustic pickle’, the ‘twisting bar’, the ‘straw mat’; being splashed with hot oil, being fed to the dogs, being impaled alive, and being beheaded. These result in death and deadly pain. This too is a drawback of sensual pleasures apparent in the present life, a mass of suffering caused by sensual pleasures. 

Furthermore,\marginnote{14.1} for the sake of sensual pleasures, they conduct themselves badly by way of body, speech, and mind. When their body breaks up, after death, they’re reborn in a place of loss, a bad place, the underworld, hell. This is a drawback of sensual pleasures to do with lives to come, a mass of suffering caused by sensual pleasures. 

\textsanskrit{Mahānāma},\marginnote{15.1} this one time I was staying near \textsanskrit{Rājagaha}, on the Vulture’s Peak Mountain. Now at that time several Jain ascetics on the slopes of Isigili at the Black Rock were constantly standing, refusing seats. And they felt painful, sharp, severe, acute feelings due to overexertion.\footnote{A large open area where the Buddha taught occasionally (\href{https://suttacentral.net/sn8.10/en/sujato}{SN 8.10}), but it is most famous as the place the monks Godhika (\href{https://suttacentral.net/sn4.23/en/sujato}{SN 4.23}) and Vakkali took their lives (\href{https://suttacentral.net/sn22.87/en/sujato}{SN 22.87}). | The practice of constant standing was formerly undertaken by the Bodhisatta (\href{https://suttacentral.net/mn12/en/sujato\#45.9}{MN 12:45.9}). } 

Then\marginnote{16.1} in the late afternoon, I came out of retreat and went to the Black Rock to visit those Jain ascetics. I said to them,\footnote{The Buddha reports a similar, but more detailed, conversation with Jains at \href{https://suttacentral.net/mn101/en/sujato}{MN 101}. } ‘Reverends, why are you constantly standing, refusing seats, so that you suffer painful, sharp, severe, acute feelings due to overexertion?’\footnote{Even with his own extensive experience in such practices, the Buddha still asks. } 

When\marginnote{17.1} I said this, those Jain ascetics said to me, ‘Reverend, the Jain ascetic of the \textsanskrit{Ñātika} clan claims to be all-knowing and all-seeing, to know and see everything without exception, thus:\footnote{The Jain leader \textsanskrit{Mahāvīra} \textsanskrit{Vardhamāna} is known as \textsanskrit{Nigaṇṭha} \textsanskrit{Nātaputta} in Pali texts. He is regarded as the 24th supreme leader of the Jains, although only he and his predecessor \textsanskrit{Pārśvanātha} (not mentioned in the Pali) are historical. \textit{\textsanskrit{Nigaṇṭha}} means “knotless” (i.e. without attachments); it is a term for a Jain ascetic. \textsanskrit{Nātaputta} indicates his clan the \textsanskrit{Ñātikas} (Sanskrit \textit{\textsanskrit{jñātiputra}}; \textsanskrit{Prākrit} \textit{\textsanskrit{nāyaputta}}). The Pali tradition has confused \textit{\textsanskrit{ñāti}} (“family”) with \textit{\textsanskrit{nāṭa}} (“dancer”). Thus \textsanskrit{Nigaṇṭha} \textsanskrit{Nātaputta} means “the Jain monk of the \textsanskrit{Ñātika} clan”. } “Knowledge and vision are constantly and continually present to me, while walking, standing, sleeping, and waking.”\footnote{The Buddha denied possessing such omniscience; rather, he knows the three knowledges (\href{https://suttacentral.net/mn71/en/sujato\#5.5}{MN 71:5.5}). Nonetheless, he can be poetically considered as “all-knowing” (\href{https://suttacentral.net/thag2.6/en/sujato\#1.4}{Thag 2.6:1.4}, \href{https://suttacentral.net/thag16.1/en/sujato\#18.1}{Thag 16.1:18.1}, \href{https://suttacentral.net/thag1.69/en/sujato\#1.2}{Thag 1.69:1.2}), since he understands the “all” that is the six senses (\href{https://suttacentral.net/sn35.28/en/sujato}{SN 35.28}). } 

He\marginnote{17.4} says, “O Jain ascetics, you have done bad deeds in a past life. Wear them away with these severe and grueling austerities.\footnote{The Buddha emphasized understanding \textit{kamma} and relinquishing the causes of new \textit{kamma}, rather than wearing away the unknowable mass of past \textit{kamma}. | \textsanskrit{Mahāvīra}’s emphasis on \emph{bad} karma of the past presages the modern usage where karma is evoked when something inescapably bad happens. } And when in the present you are restrained in body, speech, and mind, you’re not doing any bad deeds for the future. So, due to eliminating past deeds by fervent mortification, and not doing any new deeds, there’s nothing to come up in the future. With no future consequence, deeds end. With the ending of deeds, suffering ends. With the ending of suffering, feeling ends. And with the ending of feeling, all suffering will have been worn away.”\footnote{The past participle with future tense (\textit{\textsanskrit{nijjiṇṇaṁ} bhavissati}) indicates a future perfect sense, “will have been worn away”. } We endorse and accept this, and we are satisfied with it.’ 

When\marginnote{18.1} they said this, I said to them, ‘But reverends, do you know for sure that you existed in the past, and it is not the case that you did not exist?’ 

‘No\marginnote{18.4} we don’t, reverend.’ 

‘But\marginnote{18.5} reverends, do you know for sure that you did bad deeds in the past?’ 

‘No\marginnote{18.7} we don’t, reverend.’ 

‘But\marginnote{18.8} reverends, do you know that you did such and such bad deeds?’ 

‘No\marginnote{18.10} we don’t, reverend.’ 

‘But\marginnote{18.11} reverends, do you know that so much suffering has already been worn away? Or that so much suffering still remains to be worn away? Or that when so much suffering is worn away all suffering will have been worn away?’ 

‘No\marginnote{18.13} we don’t, reverend.’ 

‘But\marginnote{18.14} reverends, do you know about giving up unskillful qualities in this very life and embracing skillful qualities?’ 

‘No\marginnote{18.16} we don’t, reverend.’ 

‘So\marginnote{19.1} it seems that you don’t know any of these things. That being so, when those in the world who are violent and bloody-handed and of cruel livelihood are reborn among humans they go forth as Jain ascetics.’\footnote{This point is argued in more detail at \href{https://suttacentral.net/mn101/en/sujato\#12.2}{MN 101:12.2} ff. | “Cruel livelihood” (\textit{\textsanskrit{kurūrakammantā}}) is defined at \href{https://suttacentral.net/mn51/en/sujato\#9.2}{MN 51:9.2}, etc. } 

‘Reverend\marginnote{20.1} Gotama, pleasure is not gained through pleasure; pleasure is gained through pain.\footnote{At \href{https://suttacentral.net/mn85/en/sujato\#10.2}{MN 85:10.2} the Buddha says that he too held this belief while still an unenlightened Bodhisatta. Thus not only were his austerities identical with the Jains, so were his beliefs. The Jain text \textsanskrit{Sūyagaḍa} (known in Sanskrit as \textsanskrit{Sūtrakṛtāṅga}) 1.3.4.6 rejects the view that pleasure is gained through pleasure, attributing it to those who disdain the noble path, which its commentary identifies with Buddhists. } For if pleasure were to be gained through pleasure, King Seniya \textsanskrit{Bimbisāra} of Magadha would gain pleasure, since he lives in greater pleasure than Venerable Gotama.’\footnote{With their path emphasizing painful austerity, there is little to support the idea that the Jains practiced meditation in the Buddhist sense, and certainly not the deep pleasure of \textit{\textsanskrit{jhāna}}. } 

‘Clearly\marginnote{20.3} the venerables have spoken rashly, without reflection. Rather, I’m the one who should be asked about who lives in greater pleasure, King \textsanskrit{Bimbisāra} or Venerable Gotama?’\footnote{He holds them to the same standard that he himself has already demonstrated: don’t make assumptions, ask. } 

‘Clearly\marginnote{20.8} we spoke rashly and without reflection. But let that be. Now we ask Venerable Gotama: “Who lives in greater pleasure, King \textsanskrit{Bimbisāra} or Venerable Gotama?”' 

‘Well\marginnote{21.1} then, reverends, I’ll ask you about this in return, and you can answer as you like. What do you think, reverends? Is King \textsanskrit{Bimbisāra} capable of experiencing perfect happiness for seven days and nights without moving his body or speaking?’ 

‘No\marginnote{21.4} he is not, reverend.’ 

‘What\marginnote{21.5} do you think, reverends? Is King \textsanskrit{Bimbisāra} capable of experiencing perfect happiness for six days … five days … four days … three days … two days … one day?’ 

‘No\marginnote{21.12} he is not, reverend.’ 

‘But\marginnote{21.13} I am capable of experiencing perfect happiness for one day and night without moving my body or speaking. I am capable of experiencing perfect happiness for two days … three days … four days … five days … six days … seven days.\footnote{The Buddha sat for seven days after his awakening (\href{https://suttacentral.net/ud1.1/en/sujato\#1.3}{Ud 1.1:1.3}), an achievement also attested for disciples such as \textsanskrit{Mahākassapa} (\href{https://suttacentral.net/ud3.7/en/sujato\#1.3}{Ud 3.7:1.3}) and the nun \textsanskrit{Uttamā} (\href{https://suttacentral.net/thig3.2/en/sujato\#3.3}{Thig 3.2:3.3}). } What do you think, reverends? This being so, who lives in greater pleasure, King \textsanskrit{Bimbisāra} or I?’ 

‘This\marginnote{21.21} being so, Venerable Gotama lives in greater pleasure than King \textsanskrit{Bimbisāra}.’” 

That\marginnote{21.22} is what the Buddha said. Satisfied, \textsanskrit{Mahānāma} the Sakyan approved what the Buddha said. 

%
\section*{{\suttatitleacronym MN 15}{\suttatitletranslation Measuring Up }{\suttatitleroot Anumānasutta}}
\addcontentsline{toc}{section}{\tocacronym{MN 15} \toctranslation{Measuring Up } \tocroot{Anumānasutta}}
\markboth{Measuring Up }{Anumānasutta}
\extramarks{MN 15}{MN 15}

\scevam{So\marginnote{1.1} I have heard. }At one time Venerable \textsanskrit{Mahāmoggallāna} was staying in the land of the Bhaggas at Crocodile Hill, in the deer park at \textsanskrit{Bhesakaḷā}’s Wood.\footnote{The Bhaggas were an isolated republican clan, sandwiched between the Vacchas south of the Yamuna and the Kosalans north of the Ganges. By the time of the Buddha they were evidently subject to the Vacchas. | “Crocodile Hill” (\textit{\textsanskrit{susumāragira}}) was evidently the name of the capital city. Its name is something of a mystery, as the land is notably flat with the exception of a hill today called Pabhosa (also Prabhashgiri or Rohitgiri), which is now the site of a Jain temple. Perhaps it was this hill that from the river resembled a crocodile. The commentary, however, says the capital was was so-named because a crocodile made a noise in a nearby lake during the founding of the city. } There Venerable \textsanskrit{Mahāmoggallāna} addressed the mendicants:\footnote{While \textsanskrit{Moggallāna} is known for his mastery of advanced psychic abilities, suttas such as this and \href{https://suttacentral.net/mn5/en/sujato}{MN 5} show his concern for even the smallest defects in ethical conduct. } “Reverends, mendicants!” 

“Reverend,”\marginnote{1.5} they replied. Venerable \textsanskrit{Mahāmoggallāna} said this: 

“Suppose\marginnote{2.1} a mendicant invites\footnote{The term “invites” (\textit{\textsanskrit{pavāreti}}) suggests a connection with the “Invitation” (\textit{\textsanskrit{pavāraṇā}}) ceremony at the end of each rains retreat, where the monastics invite each other for admonition (\href{https://suttacentral.net/pli-tv-kd4/en/sujato}{Kd 4}). The Chinese parallels confirm this connection, saying the discourse was spoken on the \textsanskrit{Pavāraṇā}. } other mendicants to admonish them. But they’re hard to admonish, having qualities that make them hard to admonish. They're impatient, and don't take instruction respectfully. So their spiritual companions don’t think it’s worth advising and instructing them, and that person doesn’t gain their trust.\footnote{This discourse acts as a sutta counterpart to the Vinaya rule \href{https://suttacentral.net/pli-tv-bu-vb-ss12/en/sujato}{Bu Ss 12}, which lays down a procedure to deal with a mendicant who makes themselves incorrigible. The Pali origin story for that rule lays the blame on the monk Channa, the Buddha’s former charioteer, who boasts “mine is the Buddha, mine the Dhamma!” It was evidently unsuccessful, as the Buddha on his deathbed had to further lay down the “divine punishment” for him (\href{https://suttacentral.net/dn16/en/sujato\#6.4.1}{DN 16:6.4.1}). } 

And\marginnote{3.1} what are the qualities that make them hard to admonish? Firstly, a mendicant has corrupt wishes, having fallen under the sway of corrupt wishes. This is a quality that makes them difficult to admonish. 

Furthermore,\marginnote{3.5} a mendicant glorifies themselves and puts others down. …\footnote{“Glorifies themselves” is \textit{\textsanskrit{attukkaṁsaka}}. } 

They’re\marginnote{3.8} irritable, overcome by anger … 

They’re\marginnote{3.11} irritable, and acrimonious due to anger … 

They’re\marginnote{3.14} irritable, and stubborn due to anger … 

They’re\marginnote{3.17} irritable, and blurt out words bordering on anger … 

When\marginnote{3.20} accused, they object to the accuser … 

When\marginnote{3.23} accused, they rebuke the accuser … 

When\marginnote{3.26} accused, they retort to the accuser … 

When\marginnote{3.29} accused, they dodge the issue, distract the discussion with irrelevant points, and display annoyance, hate, and bitterness … 

When\marginnote{3.32} accused, they are unable to account for the evidence …\footnote{\textit{\textsanskrit{Apadāna}} means the traces or marks left behind, in this case the evidence of misdeeds. } 

They\marginnote{3.35} are offensive and contemptuous … 

They’re\marginnote{3.38} jealous and stingy … 

They’re\marginnote{3.41} devious and deceitful … 

They’re\marginnote{3.44} obstinate and arrogant … 

Furthermore,\marginnote{3.47} a mendicant is attached to their own views, holding them tight, and refusing to let go.\footnote{Compare \href{https://suttacentral.net/mn8/en/sujato\#12.45}{MN 8:12.45}. } This too is a quality that makes them difficult to admonish. 

These\marginnote{3.50} are the qualities that make them hard to admonish. 

Suppose\marginnote{4.1} a mendicant doesn’t invite other mendicants to admonish them. But they’re easy to admonish, having qualities that make them easy to admonish. They're accepting, and take instruction respectfully. So their spiritual companions think it’s worth advising and instructing them, and that person gains their trust. 

And\marginnote{5.1} what are the qualities that make them easy to admonish? Firstly, a mendicant doesn’t have corrupt wishes … 

Furthermore,\marginnote{5.47} a mendicant isn’t attached to their own views, not holding them tight, but letting them go easily. 

These\marginnote{5.50} are the qualities that make them easy to admonish. 

In\marginnote{6.1} such a case, a mendicant should measure themselves like this.\footnote{“Should measure against” is \textit{anuminitabba}, which in its noun form \textit{\textsanskrit{anumāna}} lends the sutta its title. The normal sense in later literature is “inference” (eg. \href{https://suttacentral.net/mil6.4.1/en/sujato}{Mil 6.4.1}). \textit{\textsanskrit{Anumāna}} occurs in the early texts only here and in the passive form \textit{\textsanskrit{anumīyati}} at \href{https://suttacentral.net/sn22.36/en/sujato\#1.4}{SN 22.36:1.4}, where the sense must be to “measure against”.  This passage also discusses how to measure oneself in relation to another, an external standard, whereas the next section on reflection applies an inner standard. Hence the commentary glosses with: “should compare, should judge” (\textit{tuletabbo \textsanskrit{tīretabbo}}). } ‘This person has corrupt wishes, having fallen under the sway of corrupt wishes. And I don’t like or approve of this person. And if I were to fall under the sway of corrupt wishes, others wouldn’t like or approve of me.’ A mendicant who knows this should give rise to the thought: ‘I will not fall under the sway of corrupt wishes.’ … 

‘This\marginnote{6.47} person is attached to their own views, holding them tight and refusing to let go. And I don’t like or approve of this person. And if I were to be attached to my own views, holding them tight and refusing to let go, others wouldn’t like or approve of me.’ A mendicant who knows this should give rise to the thought: ‘I will not be attached to my own views, holding them tight, but will let them go easily.’ 

In\marginnote{7.1} such a case, a mendicant should check themselves like this: ‘Do I have corrupt wishes? Have I fallen under the sway of corrupt wishes?’ Suppose that, upon checking, a mendicant knows that they have fallen under the sway of corrupt wishes. Then they should make an effort to give up those bad, unskillful qualities. But suppose that, upon checking, a mendicant knows that they haven’t fallen under the sway of corrupt wishes. Then they should meditate with rapture and joy, training day and night in skillful qualities. … 

Suppose\marginnote{7.87} that, upon checking, a mendicant knows that they are attached to their own views, holding them tight, and refusing to let go. Then they should make an effort to give up those bad, unskillful qualities. Suppose that, upon checking, a mendicant knows that they’re not attached to their own views, holding them tight, but let them go easily. Then they should meditate with rapture and joy, training day and night in skillful qualities. 

Suppose\marginnote{8.1} that, upon checking, a mendicant sees that they haven’t given up all these bad, unskillful qualities. Then they should make an effort to give them all up. But suppose that, upon checking, a mendicant sees that they have given up all these bad, unskillful qualities. Then they should meditate with rapture and joy, training day and night in skillful qualities. 

Suppose\marginnote{8.3} there was a woman or man who was young, youthful, and fond of adornments, and they check their own reflection in a clean bright mirror or a clear bowl of water. If they see any dirt or blemish there, they’d try to remove it.\footnote{\textsanskrit{Moggallāna} also takes part in a similar conversation about blemishes at \href{https://suttacentral.net/mn5/en/sujato\#3.1}{MN 5:3.1}. } But if they don’t see any dirt or blemish there, they’re happy, thinking: ‘How fortunate that I’m clean!’ 

In\marginnote{8.6} the same way, suppose that, upon checking, a mendicant sees that they haven’t given up all these bad, unskillful qualities. Then they should make an effort to give them all up. But suppose that, upon checking, a mendicant sees that they have given up all these bad, unskillful qualities. Then they should meditate with rapture and joy, training day and night in skillful qualities.”\footnote{The Chinese parallels continue at this point, expanding the sequence that leads from joy to immersion to insight and liberation. } 

This\marginnote{8.8} is what Venerable \textsanskrit{Mahāmoggallāna} said. Satisfied, the mendicants approved what Venerable \textsanskrit{Mahāmoggallāna} said.\footnote{The commentary says that the ancient teachers called this the “\textsanskrit{Bhikkhupātimokkha}”, perhaps an echo of the memory that it was recited on the \textit{\textsanskrit{pavāraṇā}}, which replaces the recitation of the \textsanskrit{Pātimokkha} normally held on the full moon day. It goes on to say that a mendicant should use these teachings to review themselves three times a day, or at least once. } 

%
\section*{{\suttatitleacronym MN 16}{\suttatitletranslation Hard-heartedness }{\suttatitleroot Cetokhilasutta}}
\addcontentsline{toc}{section}{\tocacronym{MN 16} \toctranslation{Hard-heartedness } \tocroot{Cetokhilasutta}}
\markboth{Hard-heartedness }{Cetokhilasutta}
\extramarks{MN 16}{MN 16}

\scevam{So\marginnote{1.1} I have heard. }At one time the Buddha was staying near \textsanskrit{Sāvatthī} in Jeta’s Grove, \textsanskrit{Anāthapiṇḍika}’s monastery. There the Buddha addressed the mendicants, “Mendicants!” 

“Venerable\marginnote{1.5} sir,” they replied. The Buddha said this: 

“Mendicants,\marginnote{2.1} when a mendicant has not given up five kinds of hard-heartedness and severed five shackles of the heart, it’s not possible for them to achieve growth, improvement, or maturity in this teaching and training.\footnote{The two sets of five are also found at \href{https://suttacentral.net/an10.14/en/sujato}{AN 10.14}. The five kinds of “hard-heartedness” are also at \href{https://suttacentral.net/an5.205/en/sujato}{AN 5.205}, \href{https://suttacentral.net/an9.71/en/sujato}{AN 9.71}, \href{https://suttacentral.net/dn33/en/sujato\#2.1.76}{DN 33:2.1.76}, and \href{https://suttacentral.net/dn34/en/sujato\#1.6.24}{DN 34:1.6.24}, and the five “shackles of the heart” are at \href{https://suttacentral.net/an5.206/en/sujato}{AN 5.206}, \href{https://suttacentral.net/an9.72/en/sujato}{AN 9.72}, \href{https://suttacentral.net/an9.82/en/sujato}{AN 9.82}, \href{https://suttacentral.net/an9.92/en/sujato}{AN 9.92}, and \href{https://suttacentral.net/dn33/en/sujato\#2.1.84}{DN 33:2.1.84}. | While this sutta is addressed as normal to the \textit{bhikkhus}, the parallels at \href{https://suttacentral.net/an10.14/en/sujato}{AN 10.14}, MA 206, and EA 51.4 are all addressed to both monks and nuns. This is one of the instances showing that the term \textit{bhikkhu} is meant to include both monks and nuns, a common feature of Pali where the default gender is masculine. Hence I translate \textit{bhikkhu} with “mendicant”, which is both literal and gender-neutral, unless it is clear that males are meant, in which case I use “monk”. } 

What\marginnote{3.1} are the five kinds of hard-heartedness they haven’t given up?\footnote{“Hard-heartedness” or “emotional barrenness” (\textit{cetokhila}): \textit{ceto} is one of several words for “mind, heart” and is functionally equivalent to \textit{citta}, yet it tends to be used in contexts that emphasize the emotional dimensions of the mind such as here. \textit{Khila} is said, in the commentary to \href{https://suttacentral.net/snp1.2/en/sujato}{Snp 1.2}, to be land so barren that nothing grows even if it rains for four months. } Firstly, a mendicant has doubts about the Teacher. They’re uncertain, undecided, and lacking confidence. This being so, their mind doesn’t incline toward keenness, commitment, persistence, and striving. This is the first kind of hard-heartedness they haven’t given up. 

Furthermore,\marginnote{4.1} a mendicant has doubts about the teaching … This is the second kind of hard-heartedness. 

They\marginnote{5.1} have doubts about the \textsanskrit{Saṅgha} … This is the third kind of hard-heartedness. 

They\marginnote{6.1} have doubts about the training …\footnote{While “training” includes the threefold training of ethics, meditation, and wisdom, it specially indicates the foundational training in ethics. This is emphasized in the Chinese parallels, which mention the “precepts” here. } This is the fourth kind of hard-heartedness. 

Furthermore,\marginnote{7.1} a mendicant is angry and upset with their spiritual companions, resentful and closed off. This being so, their mind doesn’t incline toward keenness, commitment, persistence, and striving. This is the fifth kind of hard-heartedness they haven’t given up. These are the five kinds of hard-heartedness they haven’t given up. 

What\marginnote{8.1} are the five shackles of the heart they haven’t severed? Firstly, a mendicant isn’t free of greed, desire, fondness, thirst, passion, and craving for sensual pleasures. This being so, their mind doesn’t incline toward keenness, commitment, persistence, and striving. This is the first shackle of the heart they haven’t severed. 

Furthermore,\marginnote{9.1} a mendicant isn’t free of greed for the body … This is the second shackle of the heart. 

Furthermore,\marginnote{10.1} a mendicant isn’t free of greed for form … This is the third shackle of the heart. 

They\marginnote{11.1} eat as much as they like until their belly is full, then indulge in the pleasures of sleeping, lying down, and drowsing … This is the fourth heart shackle. 

They\marginnote{12.1} lead the spiritual life hoping to be reborn in one of the orders of gods, thinking: ‘By this precept or observance or fervent austerity or spiritual life, may I become one of the gods!’ This being so, their mind doesn’t incline toward keenness, commitment, persistence, and striving. This is the fifth shackle of the heart they haven’t severed. These are the five shackles of the heart they haven’t severed. 

When\marginnote{13.1} a mendicant has not given up these five kinds of hard-heartedness and severed these five shackles of the heart, it’s not possible for them to achieve growth, improvement, or maturity in this teaching and training. 

When\marginnote{14.1} a mendicant has given up these five kinds of hard-heartedness and severed these five shackles of the heart, it is possible for them to achieve growth, improvement, and maturity in this teaching and training. 

What\marginnote{15.1} are the five kinds of hard-heartedness they’ve given up? Firstly, a mendicant has no doubts about the Teacher. They’re not uncertain, undecided, or lacking confidence. This being so, their mind inclines toward keenness, commitment, persistence, and striving. This is the first kind of hard-heartedness they’ve given up. 

Furthermore,\marginnote{16.1} a mendicant has no doubts about the teaching … 

They\marginnote{17.1} have no doubts about the \textsanskrit{Saṅgha} … 

They\marginnote{18.1} have no doubts about the training … 

They’re\marginnote{19.1} not angry and upset with their spiritual companions, not resentful or closed off. This being so, their mind inclines toward keenness, commitment, persistence, and striving. This is the fifth kind of hard-heartedness they’ve given up. These are the five kinds of hard-heartedness they’ve given up. 

What\marginnote{20.1} are the five shackles of the heart they’ve severed? Firstly, a mendicant is rid of greed, desire, fondness, thirst, passion, and craving for sensual pleasures. This being so, their mind inclines toward keenness, commitment, persistence, and striving. This is the first shackle of the heart they’ve severed. 

Furthermore,\marginnote{21.1} a mendicant is rid of greed for the body … 

They’re\marginnote{22.1} rid of greed for form … 

They\marginnote{23.1} don’t eat as much as they like until their belly is full, then indulge in the pleasures of sleeping, lying down, and drowsing … 

They\marginnote{24.1} don’t lead the spiritual life hoping to be reborn in one of the orders of gods, thinking: ‘By this precept or observance or fervent austerity or spiritual life, may I become one of the gods!’ This being so, their mind inclines toward keenness, commitment, persistence, and striving. This is the fifth shackle of the heart they’ve severed. These are the five shackles of the heart they’ve severed. 

When\marginnote{25.1} a mendicant has given up these five kinds of hard-heartedness and severed these five shackles of the heart, it is possible for them to achieve growth, improvement, or maturity in this teaching and training.\footnote{Up to this point, the discourse is mostly identical to \href{https://suttacentral.net/an10.14/en/sujato}{AN 10.14}. In the next section, the Buddha goes on to show what “growth” means; with the emotional fundamentals in place, meditation can proceed. } 

They\marginnote{26.1} develop the basis of psychic power that has immersion due to enthusiasm, and active effort …\footnote{The four \textit{iddhipada} are included in the so-called 37 “wings to awakening” (\textit{bodhipakkhiyadhamma}). They are an alternate presentation of the path of meditation, with a special focus on developing \textit{\textsanskrit{samādhi}}. It seems the Buddha designed this presentation of Dhamma to appeal to the many meditators who are fascinated by psychic powers, with the aim to lead them on to the higher goal of liberation. | “Enthusiasm” (\textit{chanda}) is the desire or will to practice the Dhamma for liberation. } the basis of psychic power that has immersion due to energy, and active effort …\footnote{One who has the requisite enthusiasm will put forth “effort” to achieve the goal. } the basis of psychic power that has immersion due to mental development, and active effort …\footnote{When striving, one will purify and develop the “mind”. When \textit{citta} appears in meditation contexts, it is often virtually a synonym of \textit{\textsanskrit{samādhi}} (see eg. \textit{\textsanskrit{cittabhāvanā}}, \textit{\textsanskrit{cittasampadā}}). } the basis of psychic power that has immersion due to inquiry, and active effort. And the fifth is sheer vigor.\footnote{The mind clarified in \textit{\textsanskrit{samādhi}} can investigate the true nature of reality. | \textit{\textsanskrit{Ussoḷhi}} (“vigor”) is similar in meaning to “energy” (\textit{viriya}) and “active effort” (\textit{\textsanskrit{saṅkhāra}}), underscoring the importance of effort here. } A mendicant who possesses these fifteen factors, including vigor, is capable of breaking out, becoming awakened, and reaching the supreme sanctuary from the yoke.\footnote{\textit{Yogakkhema} (“sanctuary from the yoke”) is a metaphor for Nibbana that draws from the literal sense of \textit{khema} as an oasis or sanctuary for men and beasts. After a hard day’s journey, a caravan would stop to rest, “unyoke” the draft animals, lay down the burden, and refresh themselves in the cool waters. } Suppose there was a chicken with eight or ten or twelve eggs.\footnote{This simile recurs at \href{https://suttacentral.net/mn53/en/sujato\#19.2}{MN 53:19.2} and elsewhere. } And she properly sat on them to keep them warm and incubated. Even if that chicken doesn’t wish: ‘If only my chicks could break out of the eggshell with their claws and beak and hatch safely!’ Still they can break out and hatch safely. 

In\marginnote{27.7} the same way, a mendicant who possesses these fifteen factors, including vigor, is capable of breaking out, becoming awakened, and reaching the supreme sanctuary from the yoke.” 

That\marginnote{27.8} is what the Buddha said. Satisfied, the mendicants approved what the Buddha said. 

%
\section*{{\suttatitleacronym MN 17}{\suttatitletranslation Jungle Thickets }{\suttatitleroot Vanapatthasutta}}
\addcontentsline{toc}{section}{\tocacronym{MN 17} \toctranslation{Jungle Thickets } \tocroot{Vanapatthasutta}}
\markboth{Jungle Thickets }{Vanapatthasutta}
\extramarks{MN 17}{MN 17}

\scevam{So\marginnote{1.1} I have heard. }At one time the Buddha was staying near \textsanskrit{Sāvatthī} in Jeta’s Grove, \textsanskrit{Anāthapiṇḍika}’s monastery. There the Buddha addressed the mendicants, “Mendicants!” 

“Venerable\marginnote{1.5} sir,” they replied. The Buddha said this: 

“Mendicants,\marginnote{2.1} I will teach you an exposition about jungle thickets.\footnote{This sutta acts as a counterbalance to the many places where the Buddha encourages living secluded in the forest. The purpose of the life of solitude is to develop meditation and find freedom. But life is complicated and people are complicated, and sometimes what we think will support us becomes a hindrance. So regardless of how secluded and inspiring a place may be, it is crucial to always reflect on one’s actual progress in developing the wholesome. } Listen and apply your mind well, I will speak.” 

“Yes,\marginnote{2.3} sir,” they replied. The Buddha said this: 

“Mendicants,\marginnote{3.1} take the case of a mendicant who lives close by a jungle thicket.\footnote{Here I translate \textit{upanissaya} as “close by”. } As they do so, their mindfulness does not become established, their mind does not become immersed in \textsanskrit{samādhi}, their defilements do not come to an end, and they do not arrive at the supreme sanctuary from the yoke.\footnote{This gives a brief summary of the path of meditation. First one “establishes mindfulness”, i.e. undertakes meditation based on the four \textit{\textsanskrit{satipaṭṭhānas}}. This leads to \textit{\textsanskrit{samādhi}} as the mind becomes immersed in the four \textit{\textsanskrit{jhānas}}. Then, seeing reality with the unclouded clarity of a purified mind, defilements are abandoned and one is freed from transmigration. } And the necessities of life that a renunciate requires—robes, almsfood, lodgings, and medicines and supplies for the sick—are hard to come by. That mendicant should reflect: ‘While living close by this jungle thicket, my mindfulness does not become established, my mind does not become immersed in \textsanskrit{samādhi}, my defilements do not come to an end, and I do not arrive at the supreme sanctuary from the yoke. And the necessities of life that a renunciate requires—robes, almsfood, lodgings, and medicines and supplies for the sick—are hard to come by.’ That mendicant should leave that jungle thicket that very time of night or day; they should not stay there. 

Take\marginnote{4.1} another case of a mendicant who lives close by a jungle thicket. Their mindfulness does not become established … But the necessities of life are easy to come by. That mendicant should reflect: ‘While living close by this jungle thicket, my mindfulness does not become established … But the necessities of life are easy to come by. But I didn’t go forth from the lay life to homelessness for the sake of a robe, almsfood, lodgings, or medicines and supplies for the sick. Moreover, while living close by this jungle thicket, my mindfulness does not become established …’ That mendicant should, after appraisal, leave that jungle thicket; they should not stay there.\footnote{In this case they should leave after reflection, not necessarily “that very day”, since at least they can live comfortably and healthily, which is no small thing. } 

Take\marginnote{5.1} another case of a mendicant who lives close by a jungle thicket. As they do so, their mindfulness becomes established, their mind becomes immersed in \textsanskrit{samādhi}, their defilements come to an end, and they arrive at the supreme sanctuary from the yoke. But the necessities of life that a renunciate requires—robes, almsfood, lodgings, and medicines and supplies for the sick—are hard to come by. That mendicant should reflect: ‘While living close by this jungle thicket, my mindfulness becomes established … But the necessities of life are hard to come by. But I didn’t go forth from the lay life to homelessness for the sake of a robe, almsfood, lodgings, or medicines and supplies for the sick. Moreover, while living close by this jungle thicket, my mindfulness becomes established …’ That mendicant should, after appraisal, stay in that jungle thicket; they should not leave.\footnote{Even if life is hard, meditation progress is more important. } 

Take\marginnote{6.1} another case of a mendicant who lives close by a jungle thicket. Their mindfulness becomes established … And the necessities of life are easy to come by. That mendicant should reflect: ‘While living close by this jungle thicket, my mindfulness becomes established … And the necessities of life are easy to come by.’ That mendicant should stay in that jungle thicket for the rest of their life; they should not leave. 

Take\marginnote{7{-}10.1} the case of a mendicant who lives supported by a village …\footnote{In the remaining cases I translate \textit{upanissaya} as “supported by”. } town … city … country … an individual. As they do so, their mindfulness does not become established, their mind does not become immersed in \textsanskrit{samādhi}, their defilements do not come to an end, and they do not arrive at the supreme sanctuary from the yoke. And the necessities of life that a renunciate requires—robes, almsfood, lodgings, and medicines and supplies for the sick—are hard to come by. That mendicant should reflect: ‘… my mindfulness does not become established … And the necessities of life are hard to come by.’ That mendicant should leave that person at any time of the day or night, without taking leave; they should not follow them. …\footnote{The mendicant's supporters have not fulfilled their duty to provide adequate requisites, so the mendicant is not obliged to take leave from them. } 

Take\marginnote{24.1} another case of a mendicant who lives supported by an individual. Their mindfulness does not become established … But the necessities of life are easy to come by. That mendicant should reflect: ‘… my mindfulness does not become established … But the necessities of life are easy to come by.’ … That mendicant should, after appraisal, leave that person having taken leave; they should not follow them. … 

Take\marginnote{25.1} another case of a mendicant who lives supported by an individual. Their mindfulness becomes established … But the necessities of life are hard to come by. That mendicant should reflect: ‘… my mindfulness becomes established … But the necessities of life are hard to come by.’ … That mendicant should, after appraisal, follow that person; they should not leave. 

Take\marginnote{26.1} another case of a mendicant who lives supported by an individual. As they do so, their mindfulness becomes established, their mind becomes immersed in \textsanskrit{samādhi}, their defilements come to an end, and they arrive at the supreme sanctuary from the yoke. And the necessities of life that a renunciate requires—robes, almsfood, lodgings, and medicines and supplies for the sick—are easy to come by. That mendicant should reflect: ‘While living supported by this person, my mindfulness becomes established … And the necessities of life are easy to come by.’ That mendicant should follow that person for the rest of their life; they should not leave them, even if sent away.”\footnote{The idiom “even if sent away” (\textit{api \textsanskrit{panujjamānena}}) occurs a few times in this sense (\href{https://suttacentral.net/mn122/en/sujato\#19.2}{MN 122:19.2}, \href{https://suttacentral.net/an7.37/en/sujato\#1.1}{AN 7.37:1.1}, \href{https://suttacentral.net/an9.6/en/sujato\#6.4}{AN 9.6:6.4}). } 

That\marginnote{26.8} is what the Buddha said. Satisfied, the mendicants approved what the Buddha said. 

%
\section*{{\suttatitleacronym MN 18}{\suttatitletranslation The Honey-Cake }{\suttatitleroot Madhupiṇḍikasutta}}
\addcontentsline{toc}{section}{\tocacronym{MN 18} \toctranslation{The Honey-Cake } \tocroot{Madhupiṇḍikasutta}}
\markboth{The Honey-Cake }{Madhupiṇḍikasutta}
\extramarks{MN 18}{MN 18}

\scevam{So\marginnote{1.1} I have heard.\footnote{This is one of the most famous discourses in studies of early Buddhism, largely due to the ground-breaking analysis by Venerable \textsanskrit{Kaṭukurunde} \textsanskrit{Ñāṇananda} in his 1971  monograph \emph{Concept and Reality in Early Buddhist Thought}, which established the meaning of \textit{\textsanskrit{papañca}} as “conceptual proliferation”. | It is the first of the “thought trilogy”, a series of discourses that deal with the activity of thinking in meditation (also \href{https://suttacentral.net/mn19/en/sujato}{MN 19}, \href{https://suttacentral.net/mn20/en/sujato}{MN 20}). } }At one time the Buddha was staying in the land of the Sakyans, near Kapilavatthu in the Banyan Tree Monastery.\footnote{The Banyan Tree Monastery (\textit{\textsanskrit{nigrodhārāma}}) was the normal residence for the Buddha and his disciples in the Sakyan republic. It was named, according to northern traditions, after the banyan trees that grew there, while the Pali commentaries say it was named after a Sakyan called Nigrodha who donated it. The two stories are not incompatible, as the owner could have been known by his most famous attribute, his banyan grove. It has been identified by stupas located next to the village Kudan, just north of the India-Nepal border. } 

Then\marginnote{2.1} the Buddha robed up in the morning and, taking his bowl and robe, entered Kapilavatthu for alms.\footnote{It would have been about thirty minutes walk to Kapilavatthu. } He wandered for alms in Kapilavatthu. After the meal, on his return from almsround, he went to the Great Wood for the day’s meditation,\footnote{The Great Wood (\textit{\textsanskrit{mahāvana}}) was a favorite meditation place of the Buddha. The commentaries say it was a stretch of wilderness that reached as far as the Himalayas on one side (200 km) and the ocean on the other (1000 km). Later tradition says that a town should have three woods: a “great wood” for wilderness (\textit{\textsanskrit{mahāvana}}); a “prosperity wood” for resources (\textit{sirivana}); and an “ascetic wood” for spiritual practice (\textit{tapovana}). } plunged deep into it, and sat at the root of a young wood apple tree to meditate. 

\textsanskrit{Daṇḍapāṇi}\marginnote{3.1} the Sakyan, while going for a walk,\footnote{\textsanskrit{Daṇḍapāṇi} was said to be the brother of the Buddha’s birth mother \textsanskrit{Māyā} and foster mother \textsanskrit{Mahāpajāpatī} (\textsanskrit{Mahāvaṁsa} 2.19), or else the father of the Buddha’s former wife \textsanskrit{Yasodharā} (or \textsanskrit{Gopā}, Lalitavistara 12.15). Both could be true, making Siddhattha’s wife his cousin. Reading between the lines, it seems \textsanskrit{Daṇḍapāṇi} nursed a grudge against the Buddha. This would be understandable if Siddhattha’s birth resulted in the death of one of \textsanskrit{Daṇḍapāṇi}’s sisters, while the other sister was left distraught when he went forth; and even more so if he abandoned \textsanskrit{Daṇḍapāṇi}’s daughter with their newborn son. } plunged deep into the Great Wood. He approached the Buddha and exchanged greetings with him. When the greetings and polite conversation were over, he stood to one side leaning on his staff, and said to the Buddha, “What is the ascetic’s doctrine? What does he assert?”\footnote{There is a minor Vinaya training against teaching anyone with “a staff in their hand” (\textit{\textsanskrit{daṇḍapāṇi}}, \href{https://suttacentral.net/pli-tv-bu-vb-sk58/en/sujato\#1.3.1}{Bu Sk 58:1.3.1}). This was evidently laid down because a staff could be used as a weapon, and hence was associated with royal authority or with policing and the exercise of violence. In a vision interpreted at Śatapatha \textsanskrit{Brāhmaṇa} 11.6.1.13, a man with staff in hand is identified with wrath (\textit{krodha}). } 

“Sir,\marginnote{4.1} my doctrine is such that one does not conflict with anyone in this world with its gods, \textsanskrit{Māras}, and Divinities, this population with its ascetics and brahmins, its gods and humans. And it is such that perceptions do not underlie the brahmin who lives detached from sensual pleasures, without indecision, stripped of worry, and rid of craving for rebirth in this or that state.\footnote{In the face of \textsanskrit{Daṇḍapāṇi}’s evidently hostile attitude, the Buddha addresses his uncle with the respectful \textit{\textsanskrit{āvuso}} (“sir”), and emphasizes non-conflict in line with his claim that, “I do not argue with the world; it is the world that argues with me” (\href{https://suttacentral.net/sn22.94/en/sujato}{SN 22.94}). | The expression “perceptions do not underlie” (\textit{\textsanskrit{saññā} \textsanskrit{nānusenti}}) is unique to this context and must pertain to the highly-charged relation between the Buddha and \textsanskrit{Daṇḍapāṇi}. “Perception” is that mode of knowing that interprets the present in terms of the past, and hence it might sometimes be translated as “recognition”. The Buddha, by asserting he is no longer bound by past perceptions, is hinting that this is how \textsanskrit{Daṇḍapāṇi} can get over his grudge. } That is my doctrine, and that is what I assert.” 

When\marginnote{5.1} he had spoken, \textsanskrit{Daṇḍapāṇi} shook his head, waggled his tongue, raised his eyebrows until his brow puckered in three furrows, and departed leaning on his staff.\footnote{While the Buddha succeeding in deescalating possible conflict, clearly the teaching did not have the desired effect, at least right away. | \textsanskrit{Māra} responds in the same way at \href{https://suttacentral.net/sn4.21/en/sujato\#1.12}{SN 4.21:1.12}. } 

Then\marginnote{6.1} in the late afternoon, the Buddha came out of retreat and went to the Banyan Tree Monastery, sat down on the seat spread out, and told the mendicants what had happened. 

When\marginnote{6.14} he had spoken, one of the mendicants said to him, “But sir, asserting what doctrine does the Buddha not conflict with anyone in this world with its gods, \textsanskrit{Māras}, and Divinities, this population with its ascetics and brahmins, its gods and humans?\footnote{The “Buddha” (\textit{\textsanskrit{bhagavā}}) is the subject and object respectively of this sentence and the next, a detail not always captured in translations. } And how is it that perceptions do not underlie the Buddha, the brahmin who lives detached from sensual pleasures, without indecision, stripped of worry, and rid of craving for rebirth in this or that state?”\footnote{The following passage, with its longer explanation below, is one of the most dense and enigmatic statements in the suttas. I shall explain the terms as they occur, and draw out the structure of the argument as it is revealed. } 

“Mendicant,\marginnote{8.1} judgments driven by proliferating perceptions beset a person.\footnote{“Judgment” (\textit{\textsanskrit{saṅkhā}}) is the way we “appraise” or “assess” ourselves, especially in relation to others (cf. \href{https://suttacentral.net/mn1/en/sujato\#3.3}{MN 1:3.3}, \href{https://suttacentral.net/dn1/en/sujato\#1.3.2}{DN 1:1.3.2}). | “Proliferation” (\textit{\textsanskrit{papañca}}) is the compulsion of the mind to spread out in endless inner commentary that hides reality. | “Beset” (\textit{\textsanskrit{samudācaranti}}) conveys the sense that the person is overwhelmed and swamped, no longer the agent of their existence. | A “person” (\textit{purisa}) is the conventional sense of self that arises from desire and identification. } If they don’t find anything worth approving, welcoming, or getting attached to in the source from which these arise,\footnote{\textit{Ettha} (“regarding that”) refers back to \textit{\textsanskrit{yatonidānaṁ}} (“the source from which”) in the previous line. This “source” has not yet been identified. } just this is the end of the underlying tendencies to desire, repulsion, views, doubt, conceit, the desire to be reborn, and ignorance. This is the end of taking up the rod and the sword, the end of quarrels, arguments, and disputes, of accusations, divisive speech, and lies.\footnote{Here the Buddha uses “underlying tendencies” (\textit{anusaya}) for the normal set of seven, implying that these are meant by “perceptions” in the phrase “perceptions do not underlie the brahmin” at \href{https://suttacentral.net/mn18/en/sujato\#4.1}{MN 18:4.1}. | He identifies the doctrine that leads to peace as the ending of these tendencies that create proliferation and judgment about the supposed “person”. } This is where these bad, unskillful qualities cease without anything left over.”\footnote{By this the Buddha indicates arahantship. } 

That\marginnote{9.1} is what the Buddha said. When he had spoken, the Holy One got up from his seat and entered his dwelling.\footnote{While the Buddha usually took pains to make his teaching explicit, he sometimes left his students with puzzling or enigmatic statements as a way of encouraging them to figure them out for themselves. } 

Soon\marginnote{10.1} after the Buddha left, those mendicants considered, “The Buddha gave this brief summary recital, then entered his dwelling without explaining the meaning in detail. Who can explain in detail the meaning of this brief summary recital given by the Buddha?” 

Then\marginnote{10.8} those mendicants thought, “This Venerable \textsanskrit{Mahākaccāna} is praised by the Buddha and esteemed by his sensible spiritual companions.\footnote{\textsanskrit{Mahākaccāna} was one of the great disciples, whose teachings specially emphasized the incisive analysis of consciousness through the lens of the six senses. He was said to be the most skilled at given detailed explanations of brief teachings (\href{https://suttacentral.net/an1.197/en/sujato}{AN 1.197}), a skill he displayed also at \href{https://suttacentral.net/mn133/en/sujato}{MN 133}, \href{https://suttacentral.net/mn138/en/sujato}{MN 138}, and \href{https://suttacentral.net/an10.172/en/sujato}{AN 10.172}. Later he was to settle to the southwest in Avanti, where he established the Dhamma in the region. } He is capable of explaining in detail the meaning of this brief summary recital given by the Buddha. Let’s go to him, and ask him about this matter.” 

Then\marginnote{11.1} those mendicants went to \textsanskrit{Mahākaccāna}, and exchanged greetings with him. When the greetings and polite conversation were over, they sat down to one side. They told him what had happened, and said: “May Venerable \textsanskrit{Mahākaccāna} please explain this.” 

“Reverends,\marginnote{12.1} suppose there was a person in need of heartwood. And while wandering in search of heartwood he’d come across a large tree standing with heartwood. But he’d pass over the roots and trunk, imagining that the heartwood should be sought in the branches and leaves. Such is the consequence for the venerables. Though you were face to face with the Buddha, you overlooked him, imagining that you should ask me about this matter. For he is the Buddha, the one who knows and sees. He is vision, he is knowledge, he is the manifestation of principle, he is the manifestation of divinity. He is the teacher, the proclaimer, the elucidator of meaning, the bestower of freedom from death, the lord of truth, the Realized One. That was the time to approach the Buddha and ask about this matter. You should have remembered it in line with the Buddha’s answer.” 

“Certainly\marginnote{13.1} he is the Buddha, the one who knows and sees. He is vision, he is knowledge, he is the manifestation of principle, he is the manifestation of divinity. He is the teacher, the proclaimer, the elucidator of meaning, the bestower of freedom from death, the lord of truth, the Realized One. That was the time to approach the Buddha and ask about this matter. We should have remembered it in line with the Buddha’s answer. Still, \textsanskrit{Mahākaccāna} is praised by the Buddha and esteemed by his sensible spiritual companions. You are capable of explaining in detail the meaning of this brief summary recital given by the Buddha. Please explain this, if it’s no trouble.” 

“Well\marginnote{14.1} then, reverends, listen and apply your mind well, I will speak.” 

“Yes,\marginnote{14.2} reverend,” they replied. Venerable \textsanskrit{Mahākaccāna} said this: 

“Reverends,\marginnote{15.1} the Buddha gave this brief summary recital, then entered his dwelling without explaining the meaning in detail:\footnote{“Passage for recitation” is \textit{uddesa}, which is used for a short passage to be memorized verbatim, to which is then attached a longer analysis. } ‘Judgments driven by proliferating perceptions beset a person. If they don’t find anything worth approving, welcoming, or getting attached to in the source from which these arise, just this is the end of the underlying tendencies to desire, repulsion, views, doubt, conceit, the desire to be reborn, and ignorance. This is the end of taking up the rod and the sword, the end of quarrels, arguments, and disputes, of accusations, divisive speech, and lies. This is where these bad, unskillful qualities cease without anything left over.’ This is how I understand the detailed meaning of this summary recital. 

Eye\marginnote{16.1} consciousness arises dependent on the eye and sights. The meeting of the three is contact. Contact is a condition for feeling. What you feel, you perceive. What you perceive, you think about. What you think about, you proliferate. What you proliferate is the source from which judgments driven by proliferating perceptions beset a person. This occurs with respect to sights known by the eye in the past, future, and present.\footnote{In this passage, \textsanskrit{Mahākaccāna} deftly unfolds the meaning inside the syntax. For consciousness, contact, and feeling, he repeats the standard analysis of sense experience linked to dependent origination (\href{https://suttacentral.net/sn12.43/en/sujato}{SN 12.43}), where each item, expressed as a noun, leads to the next like falling dominoes. Pivoting on feeling (cp. \href{https://suttacentral.net/sn12.43/en/sujato\#4.5}{SN 12.43:4.5}, \href{https://suttacentral.net/dn15/en/sujato\#18.6}{DN 15:18.6}), he switches to verbs; feeling exerts a force that motivates desire, even though desire itself is left unstated here. In the Pali, the subject of the verbs is implicit, assuming an agent who is feeling, perceiving, thinking, and proliferating. But with proliferating, the syntax changes again. The agent is fully manifest as the “person” who, tragically, is no longer the subject in control of the process, but the hapless object of the swarm of judgments that beset them. It is at this point that time is introduced, as the concept of the “person” binds the mind to suffering in the three periods of time. If we relate this to the origin story, \textsanskrit{Daṇḍapāṇi} has become the “person” he is, full of bitterness and resentment, because of his chronic ruminations on perceived injustices of the past. | \textsanskrit{Mahākaccāna} identifies the “source” left undefined in the Buddha’s statement with proliferation itself. | This passage also clarifies the grammatical relationship between the main terms: perception leads to proliferation and proliferation results in judgments. } 

Ear\marginnote{16.2} consciousness arises dependent on the ear and sounds. … 

Nose\marginnote{16.3} consciousness arises dependent on the nose and smells. … 

Tongue\marginnote{16.4} consciousness arises dependent on the tongue and tastes. … 

Body\marginnote{16.5} consciousness arises dependent on the body and touches. … 

Mind\marginnote{16.6} consciousness arises dependent on the mind and ideas. The meeting of the three is contact. Contact is a condition for feeling. What you feel, you perceive. What you perceive, you think about. What you think about, you proliferate. What you proliferate is the source from which judgments driven by proliferating perceptions beset a person. This occurs with respect to ideas known by the mind in the past, future, and present.\footnote{Here, rather unsatisfactorily, “ideas” renders \textit{\textsanskrit{dhammā}}. \textit{Dhamma} in the sense of “what is known by the mind rather than the senses” doesn’t readily map on to a common concept in English. Attempts include “mind objects”, which introduces the Abhidhammic idea of “object” to the suttas, where it is entirely absent; or “(mental) phenomena”, which doesn’t really fit the common meaning of “phenomena” as being what is perceptible by the senses. Etymologically, the correct word would be “noumena”, but this is used only as a technical term in Kantian philosophy where it has a rather different sense. \textsanskrit{Ñāṇamoḷi}’s “idea” might be the least bad option, in the sense of a thought, concept, sensation, or image present in consciousness. } 

Where\marginnote{17.1} there is the eye, sights, and eye consciousness, it will be possible to discover evidence of contact.\footnote{I take this passage as an encouragement to meditators who may be intimidated by the complex analysis that preceded. \textsanskrit{Mahākaccāna} is assuring his audience that if they can see the fundamentals of sense experience, the rest of the process “will make itself known” (\textit{\textsanskrit{paññāpessati}}). | I render the repetitive phrase \textit{\textsanskrit{phassapaññattiṁ} \textsanskrit{paññāpessati}} idiomatically as “will discover evidence of contact”, but more literally it might be “the making known of contact will make itself known”. } Where there is evidence of contact, it will be possible to discover evidence of feeling. Where there is evidence of feeling, it will be possible to discover evidence of perception. Where there is evidence of perception, it will be possible to discover evidence of thought. Where there is evidence of thought, it will be possible to discover evidence of being beset by judgments driven by proliferating perceptions. 

Where\marginnote{17.6} there is the ear … nose … tongue … body … mind, ideas, and mind consciousness, it will be possible to discover evidence of contact. Where there is evidence of contact, it will be possible to discover evidence of feeling. Where there is evidence of feeling, it will be possible to discover evidence of perception. Where there is evidence of perception, it will be possible to discover evidence of thinking. Where there is evidence of thinking, it will be possible to discover evidence of being beset by judgments driven by proliferating perceptions. 

Where\marginnote{18.1} there is no eye, no sights, and no eye consciousness, it will not be possible to discover evidence of contact. Where there is no evidence of contact, it will not be possible to discover evidence of feeling. Where there is no evidence of feeling, it will not be possible to discover evidence of perception. Where there is no evidence of perception, it will not be possible to discover evidence of thinking. Where there is no evidence of thinking, it will not be possible to discover evidence of being beset by judgments driven by proliferating perceptions. 

Where\marginnote{18.6} there is no ear … no nose … no tongue … no body … no mind, no ideas, and no mind consciousness, it will not be possible to discover evidence of contact. Where there is no evidence of contact, it will not be possible to discover evidence of feeling. Where there is no evidence of feeling, it will not be possible to discover evidence of perception. Where there is no evidence of perception, it will not be possible to discover evidence of thinking. Where there is no evidence of thinking, it will not be possible to discover evidence of being beset by judgments driven by proliferating perceptions. 

This\marginnote{19.1} is how I understand the detailed meaning of that brief summary recital given by the Buddha. If you wish, you may go to the Buddha and ask him about this. You should remember it in line with the Buddha’s answer.” 

Then\marginnote{20.1} those mendicants, approving and agreeing with what \textsanskrit{Mahākaccāna} said, rose from their seats and went to the Buddha, bowed, sat down to one side, and told him what had happened, adding: “\textsanskrit{Mahākaccāna} clearly explained the meaning to us in this manner, with these words and phrases.” 

“\textsanskrit{Mahākaccāna}\marginnote{21.1} is astute, mendicants, he has great wisdom. If you came to me and asked this question, I would answer it in exactly the same way as \textsanskrit{Mahākaccāna}. That is what it means, and that’s how you should remember it.” 

When\marginnote{22.1} he said this, Venerable Ānanda said to the Buddha, “Sir, suppose a person who was weak with hunger was to obtain a honey-cake. Wherever they taste it, they would enjoy a sweet, delicious flavor.\footnote{This simile is also used for the Buddha’s teachings at \href{https://suttacentral.net/an5.194/en/sujato\#4.1}{AN 5.194:4.1}. } 

In\marginnote{22.3} the same way, wherever a sincere, capable mendicant might examine with wisdom the meaning of this exposition of the teaching they would only gain joy and clarity. Sir, what is the name of this exposition of the teaching?” 

“Well\marginnote{22.5} then, Ānanda, you may remember this exposition of the teaching as ‘The Honey-Cake Discourse’.” 

That\marginnote{22.6} is what the Buddha said. Satisfied, Venerable Ānanda approved what the Buddha said. 

%
\section*{{\suttatitleacronym MN 19}{\suttatitletranslation Two Kinds of Thought }{\suttatitleroot Dvedhāvitakkasutta}}
\addcontentsline{toc}{section}{\tocacronym{MN 19} \toctranslation{Two Kinds of Thought } \tocroot{Dvedhāvitakkasutta}}
\markboth{Two Kinds of Thought }{Dvedhāvitakkasutta}
\extramarks{MN 19}{MN 19}

\scevam{So\marginnote{1.1} I have heard.\footnote{This discourse shows that a meditator must abandon unwholesome thought then wholesome thought before entering absorption. It is one of several discourses that give detailed instructions on dealing with thought in meditation (eg. \href{https://suttacentral.net/mn18/en/sujato}{MN 18}, \href{https://suttacentral.net/mn20/en/sujato}{MN 20}, \href{https://suttacentral.net/an3.101/en/sujato}{AN 3.101}). While meditation existed before the Buddha, we do not find this kind of practical advice on how to go about it. } }At one time the Buddha was staying near \textsanskrit{Sāvatthī} in Jeta’s Grove, \textsanskrit{Anāthapiṇḍika}’s monastery. There the Buddha addressed the mendicants, “Mendicants!” 

“Venerable\marginnote{1.5} sir,” they replied. The Buddha said this: 

“Mendicants,\marginnote{2.1} before my awakening—when I was still unawakened but intent on awakening—I thought:\footnote{This period of meditative development must have taken place after giving up self-mortification practices. \href{https://suttacentral.net/mn36/en/sujato\#34.1}{MN 36:34.1} says that at this point, after restoring his strength by eating solid food, he developed the absorptions, giving the impression that this happened immediately. However, the current sutta, supported by passages such as \href{https://suttacentral.net/mn128/en/sujato\#16.2}{MN 128:16.2}, shows that this took some time, although it is not clear how long. } ‘Why don’t I meditate by continually dividing my thoughts into two classes?’ So I assigned sensual, malicious, and cruel thoughts\footnote{By analyzing thoughts (\textit{vitakka}), he is consciously developing the second factor of the noble eightfold path, right thought (\textit{\textsanskrit{sammāsaṅkappa}}). In this context, \textit{vitakka} and \textit{\textsanskrit{saṅkappa}} are synonyms. } to one class. And I assigned thoughts of renunciation, good will, and harmlessness to the second class. 

Then,\marginnote{3.1} as I meditated—diligent, keen, and resolute—a sensual thought arose. I understood: ‘This sensual thought has arisen in me.\footnote{First one knows the thought, understanding it in terms of the framework. } It leads to hurting myself, hurting others, and hurting both. It blocks wisdom, it’s on the side of distress, and it doesn’t lead to extinguishment.’\footnote{Then one reflects on the causal outcome of the thought. } When I reflected that it leads to hurting myself, it went away.\footnote{Thoughts are eradicated not by force or judgment, but by reflective wisdom. } When I reflected that it leads to hurting others, it went away. When I reflected that it leads to hurting both, it went away. When I reflected that it blocks wisdom, it’s on the side of distress, and it doesn’t lead to extinguishment, it went away. So I gave up, got rid of, and eliminated any sensual thoughts that arose. 

Then,\marginnote{4{-}5.1} as I meditated—diligent, keen, and resolute—a malicious thought arose …\footnote{The difference between “malice” (or “ill will”, \textit{\textsanskrit{byāpāda}}) and “cruelty” (\textit{\textsanskrit{vihiṁsā}}) is subtle; they are the respective opposites of “love” (\textit{\textsanskrit{mettā}}) and “compassion” (\textit{\textsanskrit{karuṇā}}). \textit{\textsanskrit{Mettā}} wishes well simply and without qualification, just as “malice” wishes ill. But \textit{\textsanskrit{karuṇa}} takes pleasure in the alleviation of pain, while \textit{\textsanskrit{vihiṁsā}} takes pleasure in inflicting pain. } a cruel thought arose. I understood: ‘This cruel thought has arisen in me. It leads to hurting myself, hurting others, and hurting both. It blocks wisdom, it’s on the side of distress, and it doesn’t lead to extinguishment.’ When I reflected that it leads to hurting myself … hurting others … hurting both, it went away. When I reflected that it blocks wisdom, it’s on the side of distress, and it doesn’t lead to extinguishment, it went away. So I gave up, got rid of, and eliminated any cruel thoughts that arose. 

Whatever\marginnote{6.1} a mendicant frequently thinks about and considers becomes their heart’s inclination.\footnote{This is a key principle underlying Buddhist meditation. } If they often think about and consider sensual thoughts, they’ve given up the thought of renunciation to cultivate sensual thought. Their mind inclines to sensual thoughts. If they often think about and consider malicious thoughts … their mind inclines to malicious thoughts. If they often think about and consider cruel thoughts … their mind inclines to cruel thoughts. 

Suppose\marginnote{7.1} it’s the last month of the rainy season, in autumn, when the crops grow closely together, and a cowherd must take care of the cattle.\footnote{Compare \href{https://suttacentral.net/an3.101/en/sujato}{AN 3.101}, where the Buddha illustrates the same point with a simile of smelting gold. } He’d tap and poke them with his staff on this side and that to keep them in check. Why is that? For he sees that if they wander into the crops he could be executed, imprisoned, fined, or condemned.\footnote{At this point, a meditator guards against unwholesome thoughts, gently and persistently. } 

In\marginnote{7.5} the same way, I saw that unskillful qualities have the drawbacks of sordidness and corruption, and that skillful qualities have the benefit and cleansing power of renunciation. 

Then,\marginnote{8.1} as I meditated—diligent, keen, and resolute—a thought of renunciation arose. I understood: ‘This thought of renunciation has arisen in me. It doesn’t lead to hurting myself, hurting others, or hurting both. It nourishes wisdom, it’s on the side of freedom from distress, and it leads to extinguishment.’ If I were to keep on thinking and considering this all night … all day … all night and day, I see no danger that would come from that.\footnote{Meditators often wish to reach a state free of thought, but thought is a natural and essential function of the mind and wholesome habits of thought are a part of the eightfold path. } Still, thinking and considering for too long would tire my body. And when the body is tired, the mind is stressed. And when the mind is stressed, it’s far from immersion.\footnote{Even good thought has a limit as the mind is still active. | \textsanskrit{Bṛhadāraṇyaka} \textsanskrit{Upaniṣad} 4.4.21 enjoins the contemplative not to think too much, as it is fatiguing to speech. } So I stilled, settled, unified, and immersed my mind internally.\footnote{As the sutta later makes clear, this means that he entered absorption. The process described here extends over a period of time during which these different factors arose, rather than a single meditation sitting. } Why is that? So that my mind would not be stressed. 

Then,\marginnote{9{-}10.1} as I meditated—diligent, keen, and resolute—a thought of good will arose … a thought of harmlessness arose. I understood: ‘This thought of harmlessness has arisen in me. It doesn’t lead to hurting myself, hurting others, or hurting both. It nourishes wisdom, it’s on the side of freedom from distress, and it leads to extinguishment.’ If I were to keep on thinking and considering this all night … all day … all night and day, I see no danger that would come from that. Still, thinking and considering for too long would tire my body. And when the body is tired, the mind is stressed. And when the mind is stressed, it’s far from immersion. So I stilled, settled, unified, and immersed my mind internally. Why is that? So that my mind would not be stressed. 

Whatever\marginnote{11.1} a mendicant frequently thinks about and considers becomes their heart’s inclination. If they often think about and consider thoughts of renunciation, they’ve given up sensual thought to cultivate the thought of renunciation. Their mind inclines to thoughts of renunciation. If they often think about and consider thoughts of good will … their mind inclines to thoughts of good will. If they often think about and consider thoughts of harmlessness … their mind inclines to thoughts of harmlessness. 

Suppose\marginnote{12.1} it’s the last month of summer, when all the crops have been gathered within a village, and a cowherd must take care of the cattle. While at the root of a tree or in the open he need only be mindful that the cattle are there. In the same way I needed only to be mindful that those things were there.\footnote{Compare \href{https://suttacentral.net/mn140/en/sujato\#20.2}{MN 140:20.2} and \href{https://suttacentral.net/an3.102/en/sujato\#2.1}{AN 3.102:2.1}, where, when smelting is going well, the goldsmith merely observes with equanimity. } 

My\marginnote{13.1} energy was roused up and unflagging, my mindfulness was established and lucid, my body was tranquil and undisturbed, and my mind was immersed in \textsanskrit{samādhi}.\footnote{This passage, which describes the process of entering absorption, is here shown to be the opposite of the previous situation where the mind was continuing to think even wholesome thoughts. } 

Quite\marginnote{14.1} secluded from sensual pleasures, secluded from unskillful qualities, I entered and remained in the first absorption, which has the rapture and bliss born of seclusion, while placing the mind and keeping it connected.\footnote{The first absorption still has \textit{vitakka}, which above was rendered “thought” since it was clearly a verbal process. \textit{\textsanskrit{Jhāna}} is, however, a state of “higher mind” where all mental processes are elevated. Pleasure is no longer the coarse stimulus of the senses, seclusion is no longer just being physically isolated, rapture is no longer exciting. And \textit{vitakka} is no longer the activity of verbalizing thought, as the burden of the sutta is to show that even wholesome thought prevents absorption. Rather, it is explained as a subtle process of placing the mind and keeping it in place (as defined at \href{https://suttacentral.net/mn117/en/sujato\#14.2}{MN 117:14.2}). The English word “thought” in the sense “to bring something to mind” might be elastic enough to cover this sense, but it is apt to be misleading. | The late canonical \textsanskrit{Peṭakopadesa} has an interesting analysis that precedes the commentarial Theravadin understanding of this point (\href{https://suttacentral.net/pe7/pli/ms}{Pe 7}). } 

As\marginnote{15.1} the placing of the mind and keeping it connected were stilled, I entered and remained in the second absorption, which has the rapture and bliss born of immersion, with internal clarity and mind at one, without placing the mind and keeping it connected.\footnote{Each \textit{\textsanskrit{jhāna}} begins as the least refined aspect of the previous \textit{\textsanskrit{jhāna}} ends. This is not consciously directed, but describes the natural process of settling. The meditator is now fully confident and no longer needs to apply their mind: it is simply still and fully unified. } 

And\marginnote{16.1} with the fading away of rapture, I entered and remained in the third absorption, where I meditated with equanimity, mindful and aware, personally experiencing the bliss of which the noble ones declare, ‘Equanimous and mindful, one meditates in bliss.’\footnote{The emotional response to bliss matures from the subtle thrill of rapture to the poise of equanimity. Mindfulness is present in all states of deep meditation, but with equanimity it becomes prominent. } 

With\marginnote{17.1} the giving up of pleasure and pain, and the ending of former happiness and sadness, I entered and remained in the fourth absorption, without pleasure or pain, with pure equanimity and mindfulness.\footnote{The emotional poise of equanimity leads to the feeling of pleasure settling into the more subtle neutral feeling. Pain and sadness have been abandoned long before, but are emphasized here as they are subtle counterpart of pleasure. } 

When\marginnote{18.1} my mind had immersed in \textsanskrit{samādhi} like this—purified, bright, flawless, rid of corruptions, pliable, workable, steady, and imperturbable—I extended it toward recollection of past lives. I recollected many kinds of past lives, with features and details.\footnote{The text is elided in the Pali, but clearly is meant to be understood in full per \href{https://suttacentral.net/mn4/en/sujato\#27.2}{MN 4:27.2}. } 

This\marginnote{19.1} was the first knowledge, which I achieved in the first watch of the night. Ignorance was destroyed and knowledge arose; darkness was destroyed and light arose, as happens for a meditator who is diligent, keen, and resolute. 

When\marginnote{20.1} my mind had become immersed in \textsanskrit{samādhi} like this, I extended it toward knowledge of the death and rebirth of sentient beings. With clairvoyance that is purified and superhuman, I saw sentient beings passing away and being reborn—inferior and superior, beautiful and ugly, in a good place or a bad place. I understood how sentient beings are reborn according to their deeds. 

This\marginnote{21.1} was the second knowledge, which I achieved in the middle watch of the night. Ignorance was destroyed and knowledge arose; darkness was destroyed and light arose, as happens for a meditator who is diligent, keen, and resolute. 

When\marginnote{22.1} my mind had become immersed in \textsanskrit{samādhi} like this, I extended it toward knowledge of the ending of defilements. I truly understood: ‘This is suffering’ … ‘This is the origin of suffering’ … ‘This is the cessation of suffering’ … ‘This is the practice that leads to the cessation of suffering.' 

I\marginnote{23.1} truly understood: ‘These are defilements’ … ‘This is the origin of defilements’ … ‘This is the cessation of defilements’ … ‘This is the practice that leads to the cessation of defilements.' Knowing and seeing like this, my mind was freed from the defilements of sensuality, desire to be reborn, and ignorance. I understood: ‘Rebirth is ended; the spiritual journey has been completed; what had to be done has been done; there is nothing further for this place.’ 

This\marginnote{24.1} was the third knowledge, which I achieved in the last watch of the night. Ignorance was destroyed and knowledge arose; darkness was destroyed and light arose, as happens for a meditator who is diligent, keen, and resolute. 

Suppose\marginnote{25.1} that in a forested wilderness there was an expanse of low-lying marshes, and a large herd of deer lived nearby. Then along comes a person who wants to harm, injure, and threaten them. They close off the safe, secure path that leads to happiness, and open the wrong path. There they plant domesticated male and female deer as decoys so that, in due course, that herd of deer would fall to ruin and disaster. Then along comes a person who wants to help keep the herd of deer safe. They open up the safe, secure path that leads to happiness, and close off the wrong path. They get rid of the decoys so that, in due course, that herd of deer would grow, increase, and mature. 

I’ve\marginnote{26.1} made up this simile to make a point. And this is what it means. ‘An expanse of low-lying marshes’ is a term for sensual pleasures.\footnote{As at \href{https://suttacentral.net/sn22.84/en/sujato\#10.11}{SN 22.84:10.11}. } ‘A large herd of deer’ is a term for sentient beings. ‘A person who wants to harm, injure, and threaten them’ is a term for \textsanskrit{Māra} the Wicked. ‘The wrong path’ is a term for the wrong eightfold path, that is, wrong view, wrong thought, wrong speech, wrong action, wrong livelihood, wrong effort, wrong mindfulness, and wrong immersion. ‘A domesticated male deer’ is a term for greed and relishing. ‘A domesticated female deer’ is a term for ignorance. ‘A person who wants to help keep the herd of deer safe’ is a term for the Realized One, the perfected one, the fully awakened Buddha. ‘The safe, secure path that leads to happiness’ is a term for the noble eightfold path, that is: right view, right thought, right speech, right action, right livelihood, right effort, right mindfulness, and right immersion. 

So,\marginnote{26.13} mendicants, I have opened up the safe, secure path to happiness and closed off the wrong path. And I have got rid of the male and female decoys. 

Out\marginnote{27.1} of sympathy, I’ve done what a teacher should do who wants what’s best for their disciples. Here are these roots of trees, and here are these empty huts. Practice absorption, mendicants! Don’t be negligent! Don’t regret it later! This is my instruction to you.” 

That\marginnote{27.3} is what the Buddha said. Satisfied, the mendicants approved what the Buddha said. 

%
\section*{{\suttatitleacronym MN 20}{\suttatitletranslation How to Stop Thinking }{\suttatitleroot Vitakkasaṇṭhānasutta}}
\addcontentsline{toc}{section}{\tocacronym{MN 20} \toctranslation{How to Stop Thinking } \tocroot{Vitakkasaṇṭhānasutta}}
\markboth{How to Stop Thinking }{Vitakkasaṇṭhānasutta}
\extramarks{MN 20}{MN 20}

\scevam{So\marginnote{1.1} I have heard. }At one time the Buddha was staying near \textsanskrit{Sāvatthī} in Jeta’s Grove, \textsanskrit{Anāthapiṇḍika}’s monastery. There the Buddha addressed the mendicants, “Mendicants!” 

“Venerable\marginnote{1.5} sir,” they replied. The Buddha said this: 

“Mendicants,\marginnote{2.1} a mendicant committed to the higher mind should focus on five subjects from time to time.\footnote{The “higher mind” (\textit{adhicitta}) is the four \textit{\textsanskrit{jhānas}} (\href{https://suttacentral.net/an3.90/en/sujato\#3.1}{AN 3.90:3.1}). | “Subject” (\textit{nimitta}) here has its normal meaning in meditation contexts: a property of experience that, when focused on, fosters similar properties. This is primarily the chosen subject of meditation, but it also includes any subject that the mind dwells on. | “From time to time” (\textit{\textsanskrit{kālena} \textsanskrit{kālaṁ}}): these are not fundamentals of meditation like mindfulness, which should always be developed, but rather expedients for dealing with specific problems when they arise. A meditator should be familiar with these methods and apply them when needed. However, one should not be over-eager to reach for these methods; they are meant for those times when the normal process of meditation has gone awry. } What five? 

Take\marginnote{3.1} a mendicant who is focusing on some subject that gives rise to bad, unskillful thoughts connected with desire, hate, and delusion. That mendicant should focus on some other subject connected with the skillful.\footnote{For example, a meditator who finds themselves plagued with thoughts of annoyance, having recognized that this is happening per \href{https://suttacentral.net/mn19/en/sujato}{MN 19}, should switch to a new meditation such as \textit{\textsanskrit{mettā}}. } As they do so, those bad thoughts are given up and come to an end. Their mind becomes stilled internally; it settles, unifies, and becomes immersed in \textsanskrit{samādhi}. It’s like a deft carpenter or their apprentice who’d knock out or extract a large peg with a finer peg. In the same way, a mendicant … should focus on some other basis of meditation connected with the skillful … 

Now,\marginnote{4.1} suppose that mendicant is focusing on some other subject connected with the skillful, but bad, unskillful thoughts connected with desire, hate, and delusion keep coming up. They should examine the drawbacks of those thoughts:\footnote{These methods are progressive; each one assumes that the former method has failed. | Looking at the drawbacks of unskillful thoughts was the cornerstone of the Bodhisatta’s method at per \href{https://suttacentral.net/mn19/en/sujato\#3.4}{MN 19:3.4}. } ‘So these thoughts are unskillful, they’re blameworthy, and they result in suffering.’\footnote{As per \href{https://suttacentral.net/mn19/en/sujato\#3.4}{MN 19:3.4}. } As they do so, those bad thoughts are given up and come to an end. Their mind becomes stilled internally; it settles, unifies, and becomes immersed in \textsanskrit{samādhi}. Suppose there was a woman or man who was young, youthful, and fond of adornments. If the carcass of a snake or a dog or a human were hung around their neck, they’d be horrified, repelled, and disgusted. In the same way, a mendicant … should examine the drawbacks of those thoughts … 

Now,\marginnote{5.1} suppose that mendicant is examining the drawbacks of those thoughts, but bad, unskillful thoughts connected with desire, hate, and delusion keep coming up. They should try to forget and ignore them.\footnote{“Forget” is \textit{asati} and “ignore” is \textit{\textsanskrit{amanasikāra}}. } As they do so, those bad thoughts are given up and come to an end. Their mind becomes stilled internally; it settles, unifies, and becomes immersed in \textsanskrit{samādhi}. Suppose there was a person with clear eyes, and some undesirable sights came into their range of vision. They’d just close their eyes or look away. In the same way, a mendicant … those bad thoughts are given up and come to an end …\footnote{The repetitions here are dubious. The \textsanskrit{Mahāsaṅgīti} edition here and following omits the method that stops the thoughts. It is surely implied, but in the absence of any witnesses, I translate to preserve the roughness of the Pali. } 

Now,\marginnote{6.1} suppose that mendicant is ignoring and forgetting about those thoughts, but bad, unskillful thoughts connected with desire, hate, and delusion keep coming up. They should focus on stopping the formation of thoughts.\footnote{The unique phrase “stopping the formation of thoughts” (\textit{\textsanskrit{vitakkasaṅkhārasaṇṭhānaṁ}}) lends the sutta its title. Here \textit{\textsanskrit{saṅkhāra}} refers to the energy that drives the formation of thoughts. Understanding the cause helps to deprive it of its power. } As they do so, those bad thoughts are given up and come to an end. Their mind becomes stilled internally; it settles, unifies, and becomes immersed in \textsanskrit{samādhi}. Suppose there was a person walking quickly.\footnote{“Walking quickly” like a person swamped by many thoughts (\href{https://suttacentral.net/mn18/en/sujato\#16.1}{MN 18:16.1}). } They’d think: ‘Why am I walking so quickly? Why don’t I slow down?’ So they’d slow down.\footnote{Multiple methods of quelling thought have failed, so this method focuses on gradually slowing down rather than stopping all at once. Asking “why am I thinking so much?” turns attention around, focusing on the previous thought rather than the next thing. } They’d think: ‘Why am I walking slowly? Why don’t I stand still?’ So they’d stand still. They’d think: ‘Why am I standing still? Why don’t I sit down?’ So they’d sit down. They’d think: ‘Why am I sitting? Why don’t I lie down?’ So they’d lie down. And so that person would reject successively coarser postures and adopt more subtle ones. 

In\marginnote{6.22} the same way, a mendicant … those thoughts are given up and come to an end … 

Now,\marginnote{7.1} suppose that mendicant is focusing on stopping the formation of thoughts, but bad, unskillful thoughts connected with desire, hate, and delusion keep coming up. With teeth clenched and tongue pressed against the roof of the mouth, they should squeeze, squash, and crush mind with mind.\footnote{As a last resort, the meditator forcibly crushes unwholesome thoughts and makes themselves think wholesome thoughts. This is one of the Jain-like mortifying meditations that the Bodhisatta undertook before discovering the middle way (\href{https://suttacentral.net/mn36/en/sujato\#20.2}{MN 36:20.2}, \href{https://suttacentral.net/mn85/en/sujato\#20.2}{MN 85:20.2}, \href{https://suttacentral.net/mn100/en/sujato\#17.2}{MN 100:17.2}). There, compared to the other practices such as meditating without breathing, it is the first and gentlest. Thus the most gentle of the mortifying meditations becomes the harshest of the Buddhist methods. | The first two terms (\textit{\textsanskrit{abhiniggaṇhāti}}, \textit{\textsanskrit{abhinippīḷeti}}) also occur in the context of defeating an opponent in debate (\href{https://suttacentral.net/an10.116/en/sujato\#5.1}{AN 10.116:5.1}) and sexual assault (\href{https://suttacentral.net/pli-tv-bu-vb-ss2/en/sujato\#2.2.1}{Bu Ss 2:2.2.1}). \textit{\textsanskrit{Abhisantāpeti}} does not occur elsewhere in early Pali, but at Atharvaveda 2.12.6c it is a divine punishment for heretics. } As they do so, those bad thoughts are given up and come to an end. Their mind becomes stilled internally; it settles, unifies, and becomes immersed in \textsanskrit{samādhi}. It’s like a strong man who grabs a weaker man by the head or throat or shoulder and squeezes, squashes, and crushes them. In the same way, a mendicant … with teeth clenched and tongue pressed against the roof of the mouth, should squeeze, squash, and crush mind with mind. As they do so, those bad thoughts are given up and come to an end. Their mind becomes stilled internally; it settles, unifies, and becomes immersed in \textsanskrit{samādhi}. 

Now,\marginnote{8.1} take the mendicant who is focusing on some subject that gives rise to bad, unskillful thoughts connected with desire, hate, and delusion. They focus on some other subject connected with the skillful … They examine the drawbacks of those thoughts … They try to forget and ignore about those thoughts … They focus on stopping the formation of thoughts … With teeth clenched and tongue pressed against the roof of the mouth, they squeeze, squash, and crush mind with mind. When they succeed in each of these things, those bad thoughts are given up and come to an end. Their mind becomes stilled internally; it settles, unifies, and becomes immersed in \textsanskrit{samādhi}. This is called a mendicant who is a master of the ways of thought. They will think what they want to think, and they won’t think what they don’t want to think. They’ve cut off craving, untied the fetters, and by rightly comprehending conceit have made an end of suffering.”\footnote{This sentence has a number of dubious features. Up until now the Chinese parallel at MA 101 is fairly similar, but it lacks this phrase. In addition, the passage lacks the expected conditional \textit{yato ca} construction (“When a mendicant … then they are called one who has cut off craving …”, eg. \href{https://suttacentral.net/mn22.1/en/sujato}{MN 22.1}). Moreover, being phrased in the past tense it sits uneasily with the future tense of the previous phrase, and would have made better sense reversed (“Having cut off craving … they will think what they want to think”). Taken together, these considerations suggest that this passage may have been inserted by error in the Pali. The original scope of the sutta, then, would have focused solely on the quieting of thoughts for attaining \textit{\textsanskrit{samādhi}}. } 

That\marginnote{8.14} is what the Buddha said. Satisfied, the mendicants approved what the Buddha said. 

%
\addtocontents{toc}{\let\protect\contentsline\protect\nopagecontentsline}
\chapter*{The Chapter of Similes }
\addcontentsline{toc}{chapter}{\tocchapterline{The Chapter of Similes }}
\addtocontents{toc}{\let\protect\contentsline\protect\oldcontentsline}

%
\section*{{\suttatitleacronym MN 21}{\suttatitletranslation The Simile of the Saw }{\suttatitleroot Kakacūpamasutta}}
\addcontentsline{toc}{section}{\tocacronym{MN 21} \toctranslation{The Simile of the Saw } \tocroot{Kakacūpamasutta}}
\markboth{The Simile of the Saw }{Kakacūpamasutta}
\extramarks{MN 21}{MN 21}

\scevam{So\marginnote{1.1} I have heard. }At one time the Buddha was staying near \textsanskrit{Sāvatthī} in Jeta’s Grove, \textsanskrit{Anāthapiṇḍika}’s monastery. 

Now\marginnote{2.1} at that time, Venerable Phagguna of the Top-Knot was spending too long mixing closely with some nuns.\footnote{This monk appears in three discourses, in none of which he distinguishes himself. At \href{https://suttacentral.net/sn12.12/en/sujato}{SN 12.12} he persistently attempts to rephrase questions on dependent origination in terms of a self. And at \href{https://suttacentral.net/sn12.32/en/sujato}{SN 12.32} Venerable \textsanskrit{Sāriputta} gets word that Phagguna had disrobed. | According to the commentary, he was called Phagguna “of the Top-Knot” because of the style he wore his hair as a layperson. The name stuck, presumably to distinguish him from the Phagguna of \href{https://suttacentral.net/an6.56/en/sujato}{AN 6.56} (and probably \href{https://suttacentral.net/sn35.83/en/sujato}{SN 35.83}) who died in robes as a non-returner after hearing a Dhamma talk. } So much so that if any mendicant criticized those nuns in his presence, Phagguna of the Top-Knot got angry and upset, and even instigated disciplinary proceedings.\footnote{A “disciplinary proceeding” (\textit{\textsanskrit{adhikaraṇa}}) is a legal case in Vinaya. A standard set of four is listed at \href{https://suttacentral.net/mn104/en/sujato\#12.3}{MN 104:12.3}. The kind of case is not specified in the text, but the commentary says they accused other mendicants of various offences and got Vinaya experts to litigate them, making it a “disciplinary issue arising from accusations” (\textit{\textsanskrit{anuvādādhikaraṇa}}, \href{https://suttacentral.net/pli-tv-kd14/en/sujato\#14.2.7}{Kd 14:14.2.7}). } And if any mendicant criticized Phagguna of the Top-Knot in their presence, those nuns got angry and upset, and even instigated disciplinary proceedings. That’s how close Phagguna of the Top-Knot was with those nuns. 

Then\marginnote{3.1} a mendicant went up to the Buddha, bowed, sat down to one side, and told him what was going on. 

So\marginnote{4.1} the Buddha addressed one of the monks, “Please, monk, in my name tell the mendicant Phagguna of the Top-Knot that the teacher summons him.” 

“Yes,\marginnote{4.4} sir,” that monk replied. He went to Phagguna of the Top-Knot and said to him, “Reverend Phagguna, the teacher summons you.” 

“Yes,\marginnote{4.6} reverend,” Phagguna replied. He went to the Buddha, bowed, and sat down to one side. The Buddha said to him: 

“Is\marginnote{5.1} it really true, Phagguna, that you’ve been spending too long mixing closely with some nuns? So much so that if any mendicant criticizes those nuns in your presence, you get angry and upset, and even instigate disciplinary proceedings? And if any mendicant criticizes you in those nuns’ presence, they get angry and upset, and even instigate disciplinary proceedings? Is that how close you’ve become with those nuns?” 

“Yes,\marginnote{5.6} sir.” 

“Phagguna,\marginnote{5.7} are you not a gentleman who has gone forth out of faith from the lay life to homelessness?” 

“Yes,\marginnote{5.8} sir.” 

“As\marginnote{6.1} such, it’s not appropriate for you to mix so closely with those nuns. So if anyone criticizes those nuns in your presence, you should give up any desires or thoughts of domestic life. If that happens, you should train like this: ‘My mind will not degenerate. I will blurt out no bad words. I will remain full of sympathy, with a heart of love and no secret hate.’ That’s how you should train. 

So\marginnote{6.6} even if someone strikes those nuns with fists, stones, rods, and swords in your presence, you should give up any desires or thoughts of domestic life.\footnote{This does not mean, of course, that one should not defend people against violence (see eg. \href{https://suttacentral.net/dn16/en/sujato\#1.4.15}{DN 16:1.4.15}), but that even in such circumstances one should not give way to anger. } If that happens, you should train like this: ‘My mind will not degenerate. I will blurt out no bad words. I will remain full of sympathy, with a heart of love and no secret hate.’ That’s how you should train. 

So\marginnote{6.10} if anyone criticizes you in your presence, you should give up any desires or thoughts of domestic life.\footnote{The grammar here is a little tricky. The genitive plays two roles in these passages: specifying the context (\textit{tava \textsanskrit{sammukhā}}, “in your presence”), and as the object of criticism or blows (\textit{\textsanskrit{pahāraṁ} dadeyya} “if they strike” operates above on “those nuns” (\textit{\textsanskrit{tāsaṁ} \textsanskrit{bhikkhunīnaṁ}}) and below on “you” (\textit{tava})). Here we would expect both senses to be used, but the text only has \textit{tava} once. I assume that either one instance of \textit{tava} has been lost or it is meant to be distributed. } If that happens, you should train like this: ‘My mind will not degenerate. I will blurt out no bad words. I will remain full of sympathy, with a heart of love and no secret hate.’ That’s how you should train. 

So\marginnote{6.13} Phagguna, even if someone strikes you with fists, stones, rods, and swords, you should give up any desires or thoughts of domestic life.\footnote{Here \textit{\textsanskrit{sammukhā}} is omitted, as it is unnecessary to specify being hit in your own presence. } If that happens, you should train like this: ‘My mind will not degenerate. I will blurt out no bad words. I will remain full of sympathy, with a heart of love and no secret hate.’ That’s how you should train.” 

Then\marginnote{7.1} the Buddha said to the mendicants: 

“Mendicants,\marginnote{7.2} I used to be satisfied with the mendicants. Once, I addressed them: ‘I eat my food in one sitting per day.\footnote{Eating once a day is also encouraged at \href{https://suttacentral.net/mn65/en/sujato}{MN 65}, while \href{https://suttacentral.net/mn70/en/sujato}{MN 70} encourages not eating at night. In both cases, monks objected when asked to follow this practice. Thus all three instances depict the growing influence of recalcitrant monks in the Sangha, culminating in the Buddha laying down a formal Vinaya rule against eating at the wrong time (\href{https://suttacentral.net/pli-tv-bu-vb-pc37/en/sujato}{Bu Pc 37}). Per the allowance at \href{https://suttacentral.net/mn65/en/sujato\#4.1}{MN 65:4.1}, the mendicants are not required to eat in one sitting, which remained as an optional ascetic practice. } Doing so, I find that I’m healthy and well, nimble, strong, and living comfortably. You too should eat your food in one sitting per day. Doing so, you’ll find that you’re healthy and well, nimble, strong, and living comfortably.’ I didn’t have to keep on instructing those mendicants; I just had to prompt their mindfulness.\footnote{This is referring to the early years of the dispensation, before the formal Vinaya code was laid down. The Buddha was reluctant to create a legal code and did so only when it became necessary (\href{https://suttacentral.net/pli-tv-bu-vb-pj1/en/sujato\#3.4.6}{Bu Pj 1:3.4.6}). He preferred to lead without judgment and punishment, but by reason, positive encouragement, and inspiring with his own example. } 

Suppose\marginnote{7.10} a chariot stood harnessed to thoroughbreds at a level crossroads, with a goad ready.\footnote{This simile is also at \href{https://suttacentral.net/mn119/en/sujato\#31.2}{MN 119:31.2}, \href{https://suttacentral.net/sn35.239/en/sujato\#2.10}{SN 35.239:2.10}, and \href{https://suttacentral.net/an5.28/en/sujato\#10.2}{AN 5.28:10.2}. } A deft horse trainer, a master charioteer, might mount the chariot, taking the reins in his right hand and goad in the left. He’d drive out and back wherever he wishes, whenever he wishes. 

In\marginnote{7.12} the same way, I didn’t have to keep on instructing those mendicants; I just had to prompt their mindfulness. So, mendicants, you too should give up what’s unskillful and devote yourselves to skillful qualities. In this way you’ll achieve growth, improvement, and maturity in this teaching and training. 

Suppose\marginnote{8.1} that not far from a town or village there was a large grove of sal trees\footnote{This is a unique simile. } that was choked with castor-oil weeds. Then along comes a person who wants to help protect and nurture that grove. They’d cut down the crooked sal saplings that were robbing the sap, and throw them out. They’d clean up the interior of the grove, and properly care for the straight, well-formed sal saplings. In this way, in due course, that sal grove would grow, increase, and mature. 

In\marginnote{8.7} the same way, mendicants, you too should give up what’s unskillful and devote yourselves to skillful qualities. In this way you’ll achieve growth, improvement, and maturity in this teaching and training. 

Once\marginnote{9.1} upon a time, mendicants, right here in \textsanskrit{Sāvatthī} there was a housewife named \textsanskrit{Vedehikā}.\footnote{The lady \textsanskrit{Vedehikā} is known only here. According to the commentary, her name means either “daughter of the lady from Videha”, or (implausibly) “wise one”. However, the much earlier \textsanskrit{Arthaśāstra} treats \textit{vaidehika} either as a “merchant” (2.21.7) or as a caste name (3.7.32). This story paints a rare portrait of domestic life. } She had this good reputation: ‘The housewife \textsanskrit{Vedehikā} is sweet, even-tempered, and calm.’\footnote{The same set of three in a similar context occur at \href{https://suttacentral.net/an6.60/en/sujato}{AN 6.60}. } Now, \textsanskrit{Vedehikā} had a bonded maid named \textsanskrit{Kāḷī} who was deft, tireless, and well-organized in her work.\footnote{According to ancient Indian law (\textsanskrit{Arthaśāstra} 3.13), slaves may be born such, be captured in war, or a person in a time of trouble may bind themselves in service for a fee. Non-Aryans (\textit{mleccha}) may indenture their children, but this is forbidden for Aryans. Such bondservants were protected against cruelty, sexual abuse, and unfair work. After earning back the fee of their indenture they were freed, retaining their original inheritance and status. | The maid’s name means “black” (compare the \textsanskrit{Kāḷī} at \href{https://suttacentral.net/thag2.16/en/sujato\#1.1}{Thag 2.16:1.1} who “looks like a crow”). This is a persistent pattern in early Pali. The only other named slaves are \textsanskrit{Disā} (“foe”, \href{https://suttacentral.net/dn3/en/sujato\#1.16.1}{DN 3:1.16.1}), her son \textsanskrit{Kaṇha} (“black”, \href{https://suttacentral.net/dn3/en/sujato\#1.16.1}{DN 3:1.16.1}), and \textsanskrit{Kāka} (“crow”, \href{https://suttacentral.net/pli-tv-kd8/en/sujato\#1.26.6}{Kd 8:1.26.6}). These people were evidently non-Aryans descended from the native peoples of India, perhaps Tamils or other tribal groups. } 

Then\marginnote{9.5} \textsanskrit{Kāḷī} thought, ‘My mistress has a good reputation as being sweet, even-tempered, and calm. But does she actually have anger in her and just not show it? Or does she have no anger? Or is it just because my work is well-organized that she doesn’t show anger, even though she still has it inside? Why don’t I test my mistress?’\footnote{To “test” (or “inquire”) is \textit{\textsanskrit{vīmaṁsā}}, which features as the fourth of the bases of psychic power. } 

So\marginnote{9.11} \textsanskrit{Kāḷī} got up during the day. \textsanskrit{Vedehikā} said to her, ‘Oi wench, \textsanskrit{Kāḷī}!’\footnote{\textit{He je} is an idiomatic and disrespectful form of address, offering a glimpse of colloquial speech in the formal Pali of scripture. \textit{He} is an exclamation for calling attention, like “hey” but cruder. \textit{Je} is a derogatory term for women, used here and by the \textsanskrit{Vajjīs} to \textsanskrit{Ambapālī} at \href{https://suttacentral.net/dn16/en/sujato\#2.16.3}{DN 16:2.16.3}. } 

‘What\marginnote{9.14} is it, ma’am?’\footnote{She uses the respectful address \textit{ayye} (“ma’am”). } 

‘You’re\marginnote{9.15} getting up in the day—what’s up with you, wench?’ 

‘Nothing,\marginnote{9.16} ma’am.’ 

‘Oh,\marginnote{9.17} so nothing’s up, you naughty maid, but you get up in the day!’ Angry and upset, she scowled. 

Then\marginnote{9.18} \textsanskrit{Kāḷī} thought, ‘My mistress actually has anger in her and just doesn’t show it; it’s not that she has no anger. It’s just because my work is well-organized that she doesn’t show anger, even though she still has it inside. Why don’t I test my mistress further?’ 

So\marginnote{9.22} \textsanskrit{Kāḷī} got up later in the day. \textsanskrit{Vedehikā} said to her, ‘Oi wench, \textsanskrit{Kāḷī}!’ 

‘What\marginnote{9.25} is it, ma’am?’ 

‘You’re\marginnote{9.26} getting up later in the day—what’s up with you, wench?’ 

‘Nothing,\marginnote{9.27} ma’am.’ 

‘Oh,\marginnote{9.28} so nothing’s up, you naughty maid, but you get up later in the day!’ Angry and upset, she blurted out angry words. 

Then\marginnote{9.29} \textsanskrit{Kāḷī} thought, ‘My mistress actually has anger in her and just doesn’t show it; it’s not that she has no anger. It’s just because my work is well-organized that she doesn’t show anger, even though she still has it inside. Why don’t I test my mistress further?’ 

So\marginnote{9.33} \textsanskrit{Kāḷī} got up even later in the day.\footnote{Assuming that the three rounds are meant to be \textit{\textsanskrit{divā}}, \textit{\textsanskrit{divātaraṁ}}, \textit{\textsanskrit{divātaraṁyeva}} (“in the day”, “later in the day”, “even later in the day”), although this is not followed consistently in the text. } \textsanskrit{Vedehikā} said to her, ‘Oi wench, \textsanskrit{Kāḷī}!’ 

‘What\marginnote{9.36} is it, ma’am?’ 

‘You’re\marginnote{9.37} getting up even later in the day—what’s up with you, wench?’ 

‘Nothing,\marginnote{9.38} ma’am.’ 

‘Oh,\marginnote{9.39} so nothing’s up, you naughty maid, but you get up even later in the day!’ Angry and upset, she grabbed a door-pin and hit \textsanskrit{Kāḷī} on the head, cracking it open.\footnote{\textit{\textsanskrit{Aggaḷasūci}} is a “pin” (\textit{\textsanskrit{sūci}}) for fastening the “door” (\textit{aggala}). It is sometimes translated as “rolling pin”, but that would be Sanskrit \textit{vellana} (Hindi \textit{belan}). } 

Then\marginnote{9.40} \textsanskrit{Kāḷī}, with blood pouring from her cracked skull, denounced her mistress to the neighbors, ‘See, ladies, what the sweet one did! See what the even-tempered one did! See what the calm one did! How on earth can she grab a door-pin and hit her only maid on the head, cracking it open, just for getting up late?’\footnote{\textsanskrit{Arthaśāstra} 3.13.9 says that inflicting punishment (\textit{\textsanskrit{daṇḍapreṣaṇam}}; cf. Pali \textit{\textsanskrit{daṇḍāpesuṁ}} at \href{https://suttacentral.net/pli-tv-bi-vb-ss1/en/sujato\#1.38}{Bi Ss 1:1.38}) on a slave is a crime for which a master incurs a fine equivalent to the cost of the slave. } 

Then\marginnote{9.44} after some time the housewife \textsanskrit{Vedehikā} got this bad reputation: ‘The housewife \textsanskrit{Vedehikā} is fierce, ill-tempered, and not calm at all.’\footnote{No blame is given to \textsanskrit{Kāḷī} for her deliberate provocation. } 

In\marginnote{10.1} the same way, a mendicant may be the sweetest of the sweet, the most even-tempered of the even-tempered, the calmest of the calm, so long as they don’t encounter any disagreeable criticism. But it’s when they encounter disagreeable criticism that you’ll know whether they’re really sweet, even-tempered, and calm. I don’t say that a mendicant is easy to admonish if they make themselves easy to admonish only for the sake of robes, almsfood, lodgings, and medicines and supplies for the sick. Why is that? Because when they don’t get robes, almsfood, lodgings, and medicines and supplies for the sick, they’re no longer easy to admonish. But when a mendicant is easy to admonish purely because they honor, respect, revere, worship, and venerate the teaching, then I say that they’re easy to admonish. So, mendicants, you should train yourselves: ‘We will be easy to admonish purely because we honor, respect, revere, worship, and venerate the teaching.’ That’s how you should train. 

Mendicants,\marginnote{11.1} there are these five ways in which others might criticize you.\footnote{At \href{https://suttacentral.net/an5.167/en/sujato}{AN 5.167} and \href{https://suttacentral.net/an10.44/en/sujato\#7.1}{AN 10.44:7.1} the five positive ways should be established before admonishing anyone. } Their speech may be timely or untimely, true or false, gentle or harsh, beneficial or harmful, from a heart of love or from secret hate. When others criticize you, they may do so in any of these ways. If that happens, you should train like this: ‘Our minds will not degenerate. We will blurt out no bad words. We will remain full of sympathy, with a heart of love and no secret hate. We will meditate spreading a heart of love to that person. And with them as a basis, we will meditate spreading a heart full of love to everyone in the world—abundant, expansive, limitless, free of enmity and ill will.’\footnote{Up until now, the sutta has concerned itself with good behavior and motivations in everyday life. Now this forms a basis to support meditation. } That’s how you should train. 

Suppose\marginnote{12.1} a person was to come along carrying a spade and basket and say, ‘I shall make this great earth be without earth!’ And they’d dig all over, scatter all over, spit all over, and urinate all over, saying, ‘Be without earth! Be without earth!’ 

What\marginnote{12.6} do you think, mendicants? Could that person make this great earth be without earth?” 

“No,\marginnote{12.8} sir. Why is that? Because this great earth is deep and limitless. It’s not easy to make it be without earth. That person will eventually get weary and frustrated.” 

“In\marginnote{13.1} the same way, there are these five ways in which others might criticize you. Their speech may be timely or untimely, true or false, gentle or harsh, beneficial or harmful, from a heart of love or from secret hate. When others criticize you, they may do so in any of these ways. If that happens, you should train like this: ‘Our minds will not degenerate. We will blurt out no bad words. We will remain full of sympathy, with a heart of love and no secret hate. We will meditate spreading a heart of love to that person. And with them as a basis, we will meditate spreading a heart like the earth to everyone in the world—abundant, expansive, limitless, free of enmity and ill will.’ That’s how you should train. 

Suppose\marginnote{14.1} a person was to come along with dye such as red lac, turmeric, indigo, or rose madder, and say, ‘I shall draw pictures in space, making pictures appear there.’ 

What\marginnote{14.4} do you think, mendicants? Could that person draw pictures in space?” 

“No,\marginnote{14.6} sir. Why is that? Because space has no form or appearance.\footnote{“No form or appearance” is \textit{\textsanskrit{arūpī} anidassano}. } It’s not easy to draw pictures there. That person will eventually get weary and frustrated.” 

“In\marginnote{15.1} the same way, if others criticize you in any of these five ways … you should train like this: ‘… We will meditate spreading a heart of love to that person. And with them as a basis, we will meditate spreading a heart like space to everyone in the world—abundant, expansive, limitless, free of enmity and ill will.’ That’s how you should train. 

Suppose\marginnote{16.1} a person was to come along carrying a blazing grass torch, and say, ‘I shall burn and scorch the river Ganges with this blazing grass torch.’ 

What\marginnote{16.4} do you think, mendicants? Could that person burn and scorch the river Ganges with a blazing grass torch?” 

“No,\marginnote{16.6} sir. Why is that? Because the river Ganges is deep and limitless. It’s not easy to burn and scorch it with a blazing grass torch. That person will eventually get weary and frustrated.” 

“In\marginnote{17.1} the same way, if others criticize you in any of these five ways … you should train like this: ‘… We will meditate spreading a heart of love to that person. And with them as a basis, we will meditate spreading a heart like the Ganges to everyone in the world—abundant, expansive, limitless, free of enmity and ill will.’ That’s how you should train. 

Suppose\marginnote{18.1} there was a catskin bag that was rubbed, well-rubbed, very well-rubbed, soft, silky, rid of rustling and crackling. Then a person comes along carrying a stick or a stone, and says, ‘I shall make this soft catskin bag rustle and crackle with this stick or stone.’ 

What\marginnote{18.5} do you think, mendicants? Could that person make that soft catskin bag rustle and crackle with that stick or stone?” 

“No,\marginnote{18.7} sir. Why is that? Because that catskin bag is rubbed, well-rubbed, very well-rubbed, soft, silky, rid of rustling and crackling. It’s not easy to make it rustle or crackle with a stick or stone. That person will eventually get weary and frustrated.” 

“In\marginnote{19.1} the same way, there are these five ways in which others might criticize you. Their speech may be timely or untimely, true or false, gentle or harsh, beneficial or harmful, from a heart of love or from secret hate. When others criticize you, they may do so in any of these ways. If that happens, you should train like this: ‘Our minds will not degenerate. We will blurt out no bad words. We will remain full of sympathy, with a heart of love and no secret hate. We will meditate spreading a heart of love to that person. And with them as a basis, we will meditate spreading a heart like a catskin bag to everyone in the world—abundant, expansive, limitless, free of enmity and ill will.’ That’s how you should train. 

Even\marginnote{20.1} if low-down bandits were to sever you limb from limb with a two-handed saw, anyone who had a malevolent thought on that account would not be following my instructions.\footnote{This is often depicted as one of the torments of hell. } If that happens, you should train like this: ‘Our minds will not degenerate. We will blurt out no bad words. We will remain full of sympathy, with a heart of love and no secret hate. We will meditate spreading a heart of love to that person. And with them as a basis, we will meditate spreading a heart full of love to everyone in the world—abundant, expansive, limitless, free of enmity and ill will.’ That’s how you should train. 

If\marginnote{21.1} you frequently reflect on this advice on the simile of the saw,\footnote{This dramatic passage is quoted at \href{https://suttacentral.net/mn28/en/sujato\#24.3}{MN 28:24.3}, which adopts the name “The Advice on the Simile of the Saw” (\textit{\textsanskrit{kakacūpamovāda}}). It became one of the most famous similes in Buddhism. In addition to these two Majjhima suttas and their Chinese parallels, it is cited in canonical texts at \href{https://suttacentral.net/thag6.12/en/sujato\#5.2}{Thag 6.12:5.2} and SA 497, and in commentarial texts such as the Sanskrit Abhidharma text \textsanskrit{Mahāvibhāṣā} (T 1545 at T xxvii 190a28), the Pali commentary to \href{https://suttacentral.net/ud3.3/en/sujato}{Ud 3.3}, and the Visuddhimagga chapter on loving-kindness meditation (Vsm 2.9.15). A similar sentiment, moreover, is expressed in the Jain \textsanskrit{Isibhāsiyāiṁ} chapter 34. } do you see any criticism, large or small, that you could not endure?” 

“No,\marginnote{21.3} sir.” 

“So,\marginnote{21.4} mendicants, you should frequently reflect on this advice on the simile of the saw. This will be for your lasting welfare and happiness.” 

That\marginnote{21.6} is what the Buddha said. Satisfied, the mendicants approved what the Buddha said. 

%
\section*{{\suttatitleacronym MN 22}{\suttatitletranslation The Simile of the Cobra }{\suttatitleroot Alagaddūpamasutta}}
\addcontentsline{toc}{section}{\tocacronym{MN 22} \toctranslation{The Simile of the Cobra } \tocroot{Alagaddūpamasutta}}
\markboth{The Simile of the Cobra }{Alagaddūpamasutta}
\extramarks{MN 22}{MN 22}

\scevam{So\marginnote{1.1} I have heard. }At one time the Buddha was staying near \textsanskrit{Sāvatthī} in Jeta’s Grove, \textsanskrit{Anāthapiṇḍika}’s monastery. 

Now\marginnote{2.1} at that time a mendicant called \textsanskrit{Ariṭṭha}, who had previously been a vulture trapper, had the following harmful misconception:\footnote{For “vulture trapper” (\textit{\textsanskrit{gaddhabādhi}}), see \href{https://suttacentral.net/sn47.7/en/sujato\#1.4}{SN 47.7:1.4} where \textit{√\textsanskrit{bādh}} clearly means to “trap” a monkey. The commentary’s “killer” seems unjustified. } “As I understand the Buddha’s teaching, the acts that he says are obstructions are not really obstructions for the one who performs them.”\footnote{By this he denies the third of the four kinds of self-assurance (\href{https://suttacentral.net/mn12/en/sujato\#25.1}{MN 12:25.1}). } 

Several\marginnote{2.3} mendicants heard about this. They went up to \textsanskrit{Ariṭṭha} and said to him, “Is it really true, Reverend \textsanskrit{Ariṭṭha}, that you have such a harmful misconception: ‘As I understand the Buddha’s teaching, the acts that he says are obstructions are not really obstructions for the one who performs them’?” 

“Absolutely,\marginnote{3.4} reverends. As I understand the Buddha’s teaching, the acts that he says are obstructions are not really obstructions for the one who performs them.” 

Then,\marginnote{3.5} wishing to dissuade \textsanskrit{Ariṭṭha} from his view, the mendicants pursued, pressed, and grilled him,\footnote{The stock phrase “pursued, pressed, and grilled” (\textit{\textsanskrit{samanuyuñjanti} \textsanskrit{samanugāhanti} \textsanskrit{samanubhāsanti}}) is sometimes rendered as if it meant to “question”, but here there is no question. While these terms are commonly used in a context of questioning, they do not, in and of themselves, mean to question. Rather they mean to engage with a person and push for an answer or response. } “Don’t say that, \textsanskrit{Ariṭṭha}! Don’t misrepresent the Buddha, for misrepresentation of the Buddha is not good. And the Buddha would not say that.\footnote{\textsanskrit{Ariṭṭha} makes two mistakes: misunderstanding sensual pleasures, and misrepresenting the Buddha. } In many ways the Buddha has said that obstructive acts are obstructive, and that they really do obstruct the one who performs them.\footnote{\textsanskrit{Ariṭṭha} did not specify what “obstructions” he was referring to, but this reply by the mendicants indicates that he meant indulgence in sensual pleasures, a conclusion supported by the commentary and several parallel texts. See too \href{https://suttacentral.net/thig16.1/en/sujato\#45.2}{Thig 16.1:45.2} where sensual pleasures are said to be “obstructive”. Other things said to be obstructive are “possessions, honor, and popularity” (eg. \href{https://suttacentral.net/sn17.2/en/sujato\#1.2}{SN 17.2:1.2}) and “false speech” (\href{https://suttacentral.net/pli-tv-kd2/en/sujato\#3.3.14}{Kd 2:3.3.14}). In the latter context, “obstructive” is explained as preventing the attainment of \textit{\textsanskrit{jhāna}} and higher spiritual realizations. } The Buddha says that sensual pleasures give little gratification and much suffering and distress, and they are all the more full of drawbacks. With the similes of a skeleton …\footnote{The first seven of these ten similes are taught with explanations at \href{https://suttacentral.net/mn54/en/sujato\#15.1}{MN 54:15.1}. The full ten are quoted at \href{https://suttacentral.net/an5.76/en/sujato\#11.2}{AN 5.76:11.2} and by the \textsanskrit{bhikkhunī} \textsanskrit{Sumedhā} at \href{https://suttacentral.net/thig16.1/en/sujato\#41.1}{Thig 16.1:41.1}. With slight variations the list recurs in various parallels and some similes are found individually. } a scrap of meat … a grass torch … a pit of glowing coals … a dream … borrowed goods … fruit on a tree … a butcher’s knife and chopping board … swords and spears … a snake’s head, the Buddha says that sensual pleasures give little gratification and much suffering and distress, and they are all the more full of drawbacks.” 

But\marginnote{3.19} even though the mendicants pursued, pressed, and grilled him in this way, \textsanskrit{Ariṭṭha} obstinately stuck to his misconception and insisted on it. 

When\marginnote{3.21} they weren’t able to dissuade \textsanskrit{Ariṭṭha} from his view, the mendicants went to the Buddha, bowed, sat down to one side, and told him what had happened. 

So\marginnote{5.1} the Buddha addressed one of the monks, “Please, monk, in my name tell the mendicant \textsanskrit{Ariṭṭha}, formerly a vulture trapper, that the teacher summons him.” 

“Yes,\marginnote{5.4} sir,” that monk replied. He went to \textsanskrit{Ariṭṭha} and said to him, “Reverend \textsanskrit{Ariṭṭha}, the teacher summons you.” 

“Yes,\marginnote{5.6} reverend,” \textsanskrit{Ariṭṭha} replied. He went to the Buddha, bowed, and sat down to one side. The Buddha said to him, 

“Is\marginnote{5.7} it really true, \textsanskrit{Ariṭṭha}, that you have such a harmful misconception: ‘As I understand the Buddha’s teaching, the acts that he says are obstructions are not really obstructions for the one who performs them’?” 

“Absolutely,\marginnote{5.9} sir. As I understand the Buddha’s teaching, the acts that he says are obstructions are not really obstructions for the one who performs them.”\footnote{“Absolutely” renders the particle \textit{\textsanskrit{byā}}, which is a rare intensive form of \textit{iva}. It is employed in the same manner by \textsanskrit{Sāti} at \href{https://suttacentral.net/mn38/en/sujato\#3.7}{MN 38:3.7}, who is equally confident and equally wrong. } 

“Futile\marginnote{6.1} man, who on earth have you ever known me to teach in that way? Haven’t I said in many ways that obstructive acts are obstructive, and that they really do obstruct the one who performs them? I’ve said that sensual pleasures give little gratification and much suffering and distress, and they are all the more full of drawbacks. With the similes of a skeleton … a scrap of meat … a grass torch … a pit of glowing coals … a dream … borrowed goods … fruit on a tree … a butcher’s knife and chopping board … swords and spears … a snake’s head, I’ve said that sensual pleasures give little gratification and much suffering and distress, and they are all the more full of drawbacks. But still you misrepresent me by your wrong grasp, harm yourself, and create much wickedness. This will be for your lasting harm and suffering.”\footnote{The discourse up to this point is also found twice in the Vinaya. At \href{https://suttacentral.net/pli-tv-bu-vb-pc68/en/sujato}{Bu Pc 68} the Buddha makes it a confessable offence to persistently insist on a pernicious wrong view of this sort. At \href{https://suttacentral.net/pli-tv-kd11/en/sujato\#32.1.1}{Kd 11:32.1.1} the Buddha asks the Sangha to perform an act of suspension (or “ejection”, \textit{\textsanskrit{ukkhepanīyakamma}}) against \textsanskrit{Ariṭṭha}. This portion is also found in the parallel Vinayas of the Dharmaguptaka,\textsanskrit{Kāśyapīya}, \textsanskrit{Mahāsāṅghika}, \textsanskrit{Mahīśāsaka}, \textsanskrit{Mūlasarvāstivāda}, and \textsanskrit{Sarvāstivāda} schools. } 

Then\marginnote{7.1} the Buddha said to the mendicants, “What do you think, mendicants? Has this mendicant \textsanskrit{Ariṭṭha} kindled even a spark of ardor in this teaching and training?”\footnote{“Kindled even a spark of ardor” renders \textit{\textsanskrit{usmīkatopi}}, found in a similar context at \href{https://suttacentral.net/mn38/en/sujato\#6.3}{MN 38:6.3}. \textit{\textsanskrit{Usmā}} elsewhere appears as bodily warmth (\href{https://suttacentral.net/mn43/en/sujato\#22.9}{MN 43:22.9}, cf. \textsanskrit{Brahmasūtra} 4.2.11), or the initial heating of fire-sticks when rubbed together (\href{https://suttacentral.net/mn140/en/sujato\#19.13}{MN 140:19.13}). But the most pertinent context is \href{https://suttacentral.net/ja526/en/sujato\#55.4}{Ja 526:55.4}, where a young ascetic will swiftly lose their \textit{\textsanskrit{usmāgataṁ}}—explained by the commentary as “the fire of an ascetic” (\textit{\textsanskrit{samaṇatejaṁ}})—should they fall prey to sensual temptation. Thus, drawing on the traditional imagery of \textit{tapas} as heat and fervor, it refers to the kindling of ascetic ardor. It never became a technical term in Pali, but in Sanskrit Abhidharma, \textit{\textsanskrit{uṣmagata}} refers to the conjunction of radiant \textit{\textsanskrit{samādhi}} with wisdom in the initial realization of the truths, which “burns up” the defilements (\textsanskrit{Abhidharmakoṣabhāṣya} 6.17, Abhidharmasamuccaya 2.4). } 

“How\marginnote{7.4} could that be, sir? No, sir.” When this was said, \textsanskrit{Ariṭṭha} sat silent, dismayed, shoulders drooping, downcast, depressed, with nothing to say. 

Knowing\marginnote{7.7} this, the Buddha said, “Futile man, you will be known by your own harmful misconception. I’ll question the mendicants about this.” 

Then\marginnote{8.1} the Buddha said to the mendicants, “Mendicants, do you understand my teaching as \textsanskrit{Ariṭṭha} does, when he misrepresents me by his wrong grasp, harms himself, and creates much wickedness?” 

“No,\marginnote{8.3} sir. For in many ways the Buddha has told us that obstructive acts are obstructive, and that they really do obstruct the one who performs them. The Buddha has said that sensual pleasures give little gratification and much suffering and distress, and they are all the more full of drawbacks. With the similes of a skeleton … a snake’s head, the Buddha has said that sensual pleasures give little gratification and much suffering and distress, and they are all the more full of drawbacks.” 

“Good,\marginnote{8.9} good, mendicants! It’s good that you understand my teaching like this. For in many ways I have said that obstructive acts are obstructive … 

I’ve\marginnote{8.11} said that sensual pleasures give little gratification and much suffering and distress, and they are all the more full of drawbacks. But still this \textsanskrit{Ariṭṭha} misrepresents me by his wrong grasp, harms himself, and creates much wickedness. This will be for his lasting harm and suffering. Truly, mendicants, it is quite impossible to perform sensual acts without sensual desires, sensual perceptions, and sensual thoughts.\footnote{According to the commentary, this refers to sexual intercourse. This sentence is not found in the Chinese parallel at MA 200. } 

Take\marginnote{10.1} a futile person who memorizes the teaching—statements, mixed prose \& verse, discussions, verses, inspired exclamations, legends, stories of past lives, amazing stories, and elaborations.\footnote{These nine categories (\textit{\textsanskrit{aṅga}}) of the teaching were an early organization of the Dhamma before the system of \textit{\textsanskrit{nikāyas}} (or \textit{\textsanskrit{āgamas}}) was introduced at the First Council. While their exact specification is uncertain, in my view they are most likely as follows (with an example of each). \textit{Sutta} is short doctrinal statements (\href{https://suttacentral.net/sn12.1/en/sujato}{SN 12.1}). \textit{Geyya} is mixed prose and verse (“prosimetra”, \href{https://suttacentral.net/sn1.1/en/sujato}{SN 1.1}). \textit{\textsanskrit{Veyyākaraṇa}} is questions and answers (\href{https://suttacentral.net/mn22/en/sujato}{MN 22}). \textit{\textsanskrit{Gāthā}} is pure verse (\href{https://suttacentral.net/thig1.1/en/sujato}{Thig 1.1}). \textit{\textsanskrit{Udāna}} is the inspired statements identified as such in the early texts (\href{https://suttacentral.net/mn75/en/sujato\#19.1}{MN 75:19.1}). \textit{Itivuttaka} perhaps means “legends of the past”  (\href{https://suttacentral.net/dn27/en/sujato}{DN 27}) rather than the book of that name (\href{https://suttacentral.net/iti1/en/sujato}{Iti 1}). \textit{\textsanskrit{Jātaka}} are the past life stories of the Buddha found in the early texts (\href{https://suttacentral.net/mn81/en/sujato}{MN 81}). \textit{Abbhutadhamma} are stories of the amazing qualities of the Buddha or disciples (\href{https://suttacentral.net/mn123/en/sujato}{MN 123}). \textit{Vedalla} are detailed analytical elaborations (\href{https://suttacentral.net/mn43/en/sujato}{MN 43}). Once the system of \textit{\textsanskrit{aṅgas}} fell into disuse, some names were repurposed as specific books (\textsanskrit{Udāna}, Itivuttaka, \textsanskrit{Jātaka}). Northern traditions, including the parallels to this passage, usually extend the list to twelve with the addition of \textit{\textsanskrit{nidāna}} (background stories), \textit{\textsanskrit{apadāna}} (past lives of disciples), and \textit{upadesa} (explanatory treatises). } But they don’t examine the meaning of those teachings with wisdom, and so don’t come to an acceptance of them after deliberation.\footnote{“Considered acceptance” is \textit{\textsanskrit{nijjhānaṁ} khamanti}. } They memorize the teaching for the sake of finding fault and winning debates.\footnote{This theme is expanded in several discourses of the \textsanskrit{Aṭṭhakavagga}, such as \href{https://suttacentral.net/snp4.8/en/sujato}{Snp 4.8}. } They don’t realize the goal for which they memorized them. Because they’re wrongly grasped, those teachings lead to their lasting harm and suffering. Why is that? Because of their wrong grasp of the teachings.\footnote{\textsanskrit{Bṛhadāraṇyaka} \textsanskrit{Upaniṣad} 4.4.10 says that those who worship ignorance enter darkness, but those who love knowledge (\textit{veda}) enter even greater darkness. This is a recurring theme; for example, \textsanskrit{Chāndogya} \textsanskrit{Upaniṣad} 6.1.2 tells how \textsanskrit{Uddālaka} saw the vanity of his son Śvetaketu when he returned from his schooling. } 

Suppose\marginnote{10.10} there was a person in need of a cobra. And while wandering in search of a cobra\footnote{The \textit{alagadda} appears only here in early Pali. Sanskrit sources identify \textit{alagarda} either as a water-snake—in which case, however, it is said to be non-venomous (\textsanskrit{Kṣīrasvāmin}’s \textsanskrit{Amarakoṣodghāṭana} 1.7.5)—or as a kind of cobra (\textit{\textsanskrit{darvīkāra}}, \textsanskrit{Suśrutasaṁhitā} 5.4). | As to why the man was looking for a cobra, the commentary says he was looking to harvest the snake’s venom. But \textsanskrit{Candrakīrti}, drawing on the pan-Indic legend that certain serpents have a “snake-gem” (\textit{\textsanskrit{nāgamaṇi}}) in their heads, says that a serpent captured with the proper herbs and incantations brings great riches, but should these fail it will turn deadly (\textsanskrit{Mūlamadhyamakavṛtti}-\textsanskrit{prasannapadā}, L. de La Vallée Poussin’s translation, page 497). } they’d see a big cobra, and grasp it by the coil or the tail. But that cobra would twist back and bite them on the hand or the arm or other major or minor limb, resulting in death or deadly pain. Why is that? Because of their wrong grasp of the cobra. 

In\marginnote{10.17} the same way, a futile person memorizes the teaching … and those teachings lead to their lasting harm and suffering. Why is that? Because of their wrong grasp of the teachings. 

Now,\marginnote{11.1} take a gentleman who memorizes the teaching—statements, mixed prose \& verse, discussions, verses, inspired exclamations, legends, stories of past lives, amazing stories, and elaborations. And once he’s memorized them, he examines their meaning with wisdom, and comes to an acceptance of them after deliberation.\footnote{This is the “follower of teachings” (\textit{\textsanskrit{dhammānusāri}}), who is mentioned near the end of the sutta. } He doesn’t memorize the teaching for the sake of finding fault and winning debates. He realizes the goal for which he memorized them. Because they’re correctly grasped, those teachings lead to his lasting welfare and happiness. Why is that? Because of his correct grasp of the teachings. 

Suppose\marginnote{11.10} there was a person in need of a cobra. And while wandering in search of a cobra they’d see a big cobra, and hold it down carefully with a cleft stick. Only then would they correctly grasp it by the neck. And even though that cobra might wrap its coils around that person’s hand or arm or some other major or minor limb, that wouldn’t result in death or deadly pain. Why is that? Because of their correct grasp of the cobra. 

In\marginnote{11.17} the same way, a gentleman memorizes the teaching … and those teachings lead to his lasting welfare and happiness. Why is that? Because of his correct grasp of the teachings. 

So,\marginnote{12.1} mendicants, when you understand what I’ve said, you should remember it accordingly. But if I’ve said anything that you don’t understand, you should ask me about it, or some competent mendicants.\footnote{As in, for example, \href{https://suttacentral.net/mn18/en/sujato}{MN 18}. } 

Mendicants,\marginnote{13.1} I will teach you a simile of the teaching as a raft: for crossing over, not for holding on.\footnote{This is the second renowned simile introduced in this discourse. It is referenced by name in \href{https://suttacentral.net/mn38/en/sujato\#14.1}{MN 38:14.1}. Metaphors of flood and crossing rivers abound in the Buddha’s teaching, and the specific idea of using a raft to cross over recurs at \href{https://suttacentral.net/ud8.6/en/sujato\#27.4}{Ud 8.6:27.4} = \href{https://suttacentral.net/dn16/en/sujato\#1.33.2}{DN 16:1.33.2} = \href{https://suttacentral.net/pli-tv-kd6/en/sujato\#28.12.7}{Kd 6:28.12.7}, \href{https://suttacentral.net/snp1.2/en/sujato\#4.1}{Snp 1.2:4.1}, \href{https://suttacentral.net/sn35.238/en/sujato\#5.6}{SN 35.238:5.6}. } Listen and apply your mind well, I will speak.” 

“Yes,\marginnote{13.3} sir,” they replied. The Buddha said this: 

“Suppose\marginnote{13.5} there was a person traveling along the road. They’d see a large deluge, whose near shore was dubious and perilous, while the far shore was a sanctuary free of peril. But there was no ferryboat or bridge for crossing over. They’d think, ‘Why don’t I gather grass, sticks, branches, and leaves and make a raft? Riding on the raft, and paddling with my hands and feet, I can safely reach the far shore.’ And so they’d do exactly that. And when they’d crossed over to the far shore, they’d think, ‘This raft has been very helpful to me. Riding on the raft, and paddling with my hands and feet, I have safely crossed over to the far shore. Why don’t I hoist it on my head or pick it up on my shoulder and go wherever I want?’ 

What\marginnote{13.17} do you think, mendicants? Would that person be doing what should be done with that raft?” 

“No,\marginnote{13.19} sir.” 

“And\marginnote{13.20} what, mendicants, should that person do with the raft? When they’d crossed over they should think, ‘This raft has been very helpful to me. … Why don’t I beach it on dry land or set it adrift on the water and go wherever I want?’ That’s what that person should do with the raft. 

In\marginnote{13.26} the same way, I have taught a simile of the teaching as a raft: for crossing over, not for holding on. By understanding the simile of the raft, you will even give up the teachings, let alone what is not the teachings.\footnote{\textit{\textsanskrit{Dhammā}} in the plural refers back to “those teachings” (\textit{\textsanskrit{tesaṁ} \textsanskrit{dhammānaṁ}}) of the nine categories. Accordingly, when this simile is invoked at \href{https://suttacentral.net/mn38/en/sujato\#14.1}{MN 38:14.1}, it is in reference to views. The pair \textit{dhamma} and \textit{adhamma} usually means “the teaching” and “what is not the teaching” (eg. \href{https://suttacentral.net/an2.104/en/sujato}{AN 2.104}). Just as the positive form, however, means more than just “teaching”, but rather a teaching of natural and moral truth, the negative form implies there is something unnatural, in conflict with the way the world is. } 

Mendicants,\marginnote{15.1} there are these six grounds for views.\footnote{“Grounds for views” (\textit{\textsanskrit{diṭṭhiṭṭhānāni}}) are the experiences or reasoning from which views are derived. The Buddha is explaining how \textsanskrit{Ariṭṭha} fell into his wrong view. One of the purposes of this analysis is to show how, while divergent views might seem insightful, innovative, or courageous, they all fall back on the same basic fallacies. } What six? Take an unlearned ordinary person who has not seen the noble ones, and is neither skilled nor trained in the teaching of the noble ones. They’ve not seen true persons, and are neither skilled nor trained in the teaching of the true persons. They regard form as: ‘This is mine, I am this, this is my self.’\footnote{The misguided person assumes that one or other of the aggregates is their self. In modern times, the teaching of the aggregates is often presented as a reductive argument: “What you take to be your self is in fact just the aggregates”. But the Buddha’s point is, rather: “The aggregates that you take to be your self do not have the properties of a self,” namely permanence, etc. } They also regard feeling … perception … choices … whatever is seen, heard, thought, known, attained, sought, and explored by the mind as: ‘This is mine, I am this, this is my self.’\footnote{Standing in place of the fifth aggregate, “consciousness”, this includes all kinds of knowledge or spiritual wisdom, especially that gained through mysticism or meditation. } And as for this ground for views: ‘The cosmos and the self are one and the same. After death I will be that, permanent, everlasting, eternal, imperishable, and will last forever and ever.’\footnote{This is a both view and a ground for views. Such views, lacking empirical basis, have something dissatisfying about them, so adherent is driven to develop more and more complex metaphysical abstractions. At \href{https://suttacentral.net/dn1/en/sujato\#1.30.1}{DN 1:1.30.1}, this is treated as an eternalist view that arises from the grounds of either recollection of past lives or logic. | The repeated demonstrative pronouns in the Pali \textit{so loko so \textsanskrit{attā}} (literally, “This is the cosmos, this is the self”) affirm an emphatic and absolute identity: “The cosmos and the self are one and the same”. This phrase has a lexical parallel at \textsanskrit{Bṛhadāraṇyaka} \textsanskrit{Upaniṣad} 4.4.22, which argues for the renunciate life: “What can descendants do for those for whom the self and the cosmos are one and the same” (\textit{[a]\textsanskrit{yamātmāyaṁ} loka}). See also the last words of a father to his son at 1.5.17: “You are divinity, you are the sacrifice, you are the cosmos” (\textit{\textsanskrit{tvaṁ} brahma \textsanskrit{tvaṁ} \textsanskrit{yajñas} \textsanskrit{tvaṁ} loka}). | In this passage the terms \textit{\textsanskrit{attā}} and \textit{loko} are reversed, presumably by textual error. It does not affect the meaning. } They regard this also as: ‘This is mine, I am this, this is my self.’ 

But\marginnote{16.1} a learned noble disciple has seen the noble ones, and is skilled and trained in the teaching of the noble ones. They’ve seen true persons, and are skilled and trained in the teaching of the true persons. They regard form like this: ‘This is not mine, I am not this, this is not my self.’ They also regard feeling … perception … choices … whatever is seen, heard, thought, known, attained, sought, and explored by the mind like this: ‘This is not mine, I am not this, this is not my self.’ And the same for this ground for views: ‘The cosmos and the self are one and the same. After death I will be that, permanent, everlasting, eternal, imperishable, and will last forever and ever.’ They also regard like this: ‘This is not mine, I am not this, this is not my self.’ 

Seeing\marginnote{17.1} in this way they’re not anxious about what doesn’t exist.”\footnote{“What does not exist” (\textit{asati}) is the self. | To be “anxious” (\textit{paritassati}) is to be caught between desire and fear. It seems to be a word of specifically Buddhist usage, leaning equally on the roots \textit{√tras} (to tremble in fear) and \textit{√\textsanskrit{tṛṣ}} (to thirst for or crave). } 

When\marginnote{18.1} he said this, one of the mendicants asked the Buddha, “Sir, can there be anxiety about what doesn’t exist externally?”\footnote{By adding “externally” the focus is shifted from the self to “what belongs to the self”, namely possessions. However, the Chinese parallel at MA 200 apparently takes it in reference to the anxiety of an eternalist who hears the Buddha teaching “externally”. } 

“There\marginnote{18.3} can, mendicant,” said the Buddha. “It’s when someone thinks, ‘Oh, it once was mine but is mine no more. Oh, it could be mine but I do not get it.’ They sorrow and wail and lament, beating their breast and falling into confusion. That’s how there is anxiety about what doesn’t exist externally.” 

“But\marginnote{19.1} can there be no anxiety about what doesn’t exist externally?” 

“There\marginnote{19.2} can, mendicant,” said the Buddha. “It’s when someone doesn’t think, ‘Oh, it once was mine but is mine no more. Oh, it could be mine but I do not get it.’ They don’t sorrow and wail and lament, beating their breast and falling into confusion. That’s how there is no anxiety about what doesn’t exist externally.” 

“But\marginnote{20.1} can there be anxiety about what doesn’t exist internally?” 

“There\marginnote{20.2} can, mendicant,” said the Buddha. “It’s when someone has such a view: ‘The cosmos and the self are one and the same. After death I will be that, permanent, everlasting, eternal, imperishable, and will last forever and ever.’ They hear the Realized One or their disciple teaching Dhamma for the uprooting of all grounds, fixations, obsessions, insistences, and underlying tendencies regarding views; for the stilling of all activities, the letting go of all attachments, the ending of craving, fading away, cessation, extinguishment.\footnote{The structure of this long compound is revealed at \href{https://suttacentral.net/an10.96/en/sujato\#14.1}{AN 10.96:14.1}, which shows that “views” is not one item on the list, but applies to each item. } They think, ‘Whoa, I’m going to be annihilated and destroyed! I won’t even exist any more!’\footnote{They can only see the extremes: if not eternalism it must be annihilationism. } They sorrow and wail and lament, beating their breast and falling into confusion. That’s how there is anxiety about what doesn’t exist internally.” 

“But\marginnote{21.1} can there be no anxiety about what doesn’t exist internally?” 

“There\marginnote{21.2} can,” said the Buddha. “It’s when someone doesn’t have such a view: ‘The cosmos and the self are one and the same. After death I will be that, permanent, everlasting, eternal, imperishable, and will last forever and ever.’ They hear the Realized One or their disciple teaching Dhamma for the uprooting of all grounds, fixations, obsessions, insistences, and underlying tendencies regarding views; for the stilling of all activities, the letting go of all attachments, the ending of craving, fading away, cessation, extinguishment. They don’t think, ‘Whoa, I’m going to be annihilated and destroyed! I won’t even exist any more!’ They don’t sorrow and wail and lament, beating their breast and falling into confusion. That’s how there is no anxiety about what doesn’t exist internally. 

Mendicants,\marginnote{22.1} it would make sense to be possessive about something that’s permanent, everlasting, eternal, imperishable, and will last forever and ever. But do you see any such possession?” 

“No,\marginnote{22.3} sir.” 

“Good,\marginnote{22.4} mendicants! I also can’t see any such possession. 

It\marginnote{23.1} would make sense to grasp at a theory of self that didn’t give rise to sorrow, lamentation, pain, sadness, and distress. But do you see any such theory of self?” 

“No,\marginnote{23.3} sir.” 

“Good,\marginnote{23.4} mendicants! I also can’t see any such theory of self. 

It\marginnote{24.1} would make sense to rely on a view that didn’t give rise to sorrow, lamentation, pain, sadness, and distress. But do you see any such view to rely on?” 

“No,\marginnote{24.3} sir.” 

“Good,\marginnote{24.4} mendicants! I also can’t see any such view to rely on. 

Mendicants,\marginnote{25.1} were a self to exist, would there be the thought, ‘Belonging to my self’?” 

“Yes,\marginnote{25.2} sir.” 

“Were\marginnote{25.3} what belongs to a self to exist, would there be the thought, ‘My self’?” 

“Yes,\marginnote{25.4} sir.” 

“But\marginnote{25.5} since a self and what belongs to a self are not actually found, is not the following a totally foolish teaching:\footnote{“Not actually found” renders \textit{saccato thetato \textsanskrit{anupalabbhamāne}}. } ‘The cosmos and the self are one and the same. After death I will be that, permanent, everlasting, eternal, imperishable, and will last forever and ever’?” 

“How\marginnote{25.8} could it not, sir? It’s a totally foolish teaching.” 

“What\marginnote{26.1} do you think, mendicants?\footnote{The following question and answer section is quoted verbatim from the \textsanskrit{Anattalakkhaṇasutta} (\href{https://suttacentral.net/sn22.59/en/sujato\#6.1}{SN 22.59:6.1}). It appears twice elsewhere in the Majjhima (\href{https://suttacentral.net/mn35/en/sujato\#20.4}{MN 35:20.4}, \href{https://suttacentral.net/mn109/en/sujato\#15.1}{MN 109:15.1}), in the Vinaya (\href{https://suttacentral.net/pli-tv-kd1/en/sujato\#6.42.1}{Kd 1:6.42.1}), and about fifty times in the \textsanskrit{Saṁyutta}. } Is form permanent or impermanent?” 

“Impermanent,\marginnote{26.3} sir.” 

“But\marginnote{26.4} if it’s impermanent, is it suffering or happiness?” 

“Suffering,\marginnote{26.5} sir.” 

“But\marginnote{26.6} if it’s impermanent, suffering, and perishable, is it fit to be regarded thus: ‘This is mine, I am this, this is my self’?” 

“No,\marginnote{26.8} sir.” 

“What\marginnote{26.9} do you think, mendicants? Is feeling … perception … choices … consciousness permanent or impermanent?” 

“Impermanent,\marginnote{26.14} sir.” 

“But\marginnote{26.15} if it’s impermanent, is it suffering or happiness?” 

“Suffering,\marginnote{26.16} sir.” 

“But\marginnote{26.17} if it’s impermanent, suffering, and perishable, is it fit to be regarded thus: ‘This is mine, I am this, this is my self’?” 

“No,\marginnote{26.19} sir.” 

“So,\marginnote{27.1} mendicants, you should truly see any kind of form at all—past, future, or present; internal or external; solid or subtle; inferior or superior; far or near: \emph{all} form—with right understanding: ‘This is not mine, I am not this, this is not my self.’ You should truly see any kind of feeling … perception … choices … consciousness at all—past, future, or present; internal or external; solid or subtle; inferior or superior; far or near: \emph{all} consciousness—with right understanding: ‘This is not mine, I am not this, this is not my self.’ 

Seeing\marginnote{28.1} this, a learned noble disciple grows disillusioned with form, feeling, perception, choices, and consciousness. Being disillusioned, desire fades away. When desire fades away they’re freed. When they’re freed, they know they’re freed. 

They\marginnote{29.2} understand: ‘Rebirth is ended, the spiritual journey has been completed, what had to be done has been done, there is nothing further for this place.’ 

Such\marginnote{30.1} a mendicant is one who is called ‘one who has lifted the cross-bar’, ‘one who has filled in the moat’, ‘one who has pulled up the pillar’, ‘one who is unimpeded’, and also ‘a noble one with banner lowered and burden dropped, detached’.\footnote{These similes recur at \href{https://suttacentral.net/an5.71/en/sujato}{AN 5.71} and \href{https://suttacentral.net/an5.72/en/sujato}{AN 5.72}. They reflect ideas of entrapment and freedom. } 

And\marginnote{31.1} how has a mendicant raised the cross-bar?\footnote{Ignorance is a “hindrance” (\textit{\textsanskrit{nīvaraṇa}}), an image that ultimately draws on the Vedic myth of \textsanskrit{Vṛtra}, a giant “constrictor” who trapped the waters (and sometimes cattle and sun) in darkness until it was slain by Indra and the waters released. Removing the bar permits escape from entrapment. At \textsanskrit{Chāndogya} \textsanskrit{Upaniṣad} 2.24.6, it is said that a devoted sacrificer may, at the time of death, pray that the cross-bar blocking entry to the next world be removed. } It’s when a mendicant has given up ignorance, cut it off at the root, made it like a palm stump, obliterated it, so it’s unable to arise in the future. That’s how a mendicant has lifted the cross-bar. 

And\marginnote{32.1} how has a mendicant filled in the moat?\footnote{An elephant may be trapped by a moat \href{https://suttacentral.net/cp11/en/sujato\#3.1}{Cariyapiṭaka 11:3.1}, so when it is filled in it may roam free. } It’s when a mendicant has given up transmigrating through births in future lives, cut it off at the root, made it like a palm stump, obliterated it, so it’s unable to arise in the future. That’s how a mendicant has filled in the moat. 

And\marginnote{33.1} how has a mendicant pulled up the pillar?\footnote{Likewise, an elephant may be tethered to a pillar (\href{https://suttacentral.net/mn125/en/sujato\#12.11}{MN 125:12.11}, \href{https://suttacentral.net/thag19.1/en/sujato\#51.2}{Thag 19.1:51.2}, \href{https://suttacentral.net/cp11/en/sujato\#3.2}{Cariyapiṭaka 11:3.2}). The term for “pillar” here (\textit{\textsanskrit{esikā}}) also serves as a pun for “search” (\textit{\textsanskrit{esanā}}). Two of the three “searches” (sensual pleasures and continued existence, \href{https://suttacentral.net/sn45.161/en/sujato}{SN 45.161}) overlap with two of the three “cravings”. } It’s when a mendicant has given up craving, cut it off at the root, made it like a palm stump, obliterated it, so it’s unable to arise in the future. That’s how a mendicant has pulled up the pillar. 

And\marginnote{34.1} how is a mendicant unimpeded?\footnote{Giving up the five lower “fetters” (thus becoming a non-returner) makes you “unbarred” (\textit{\textsanskrit{niraggaḷa}}). This is also the name of a Vedic sacrifice (\href{https://suttacentral.net/sn3.9/en/sujato\#4.3}{SN 3.9:4.3}), where it is probably a name for the horse sacrifice, during which the horse is set free to roam for a year. } It’s when a mendicant has given up the five lower fetters, cut them off at the root, made them like a palm stump, obliterated them, so they’re unable to arise in the future. That’s how a mendicant is unimpeded. 

And\marginnote{35.1} how is a mendicant a noble one with with banner lowered and burden dropped, detached?\footnote{A “banner” is held aloft as a sign of identity. | At \href{https://suttacentral.net/sn22.22/en/sujato}{SN 22.22}, the “burden” is said to be the five aggregates, and the burden is put down with the end of craving. } It’s when a mendicant has given up the conceit ‘I am’, cut it off at the root, made it like a palm stump, obliterated it, so it’s unable to arise in the future. That’s how a mendicant is a noble one with banner lowered and burden dropped, detached. 

When\marginnote{36.1} a mendicant’s mind was freed like this, the gods together with Indra, the Divinity, and the Progenitor, search as they may, will not discover:\footnote{This expression conveys a deep sense of wonder when faced with a state of meditative consciousness so profound that its “basis” is unknown (\href{https://suttacentral.net/an11.9/en/sujato\#3.15}{AN 11.9:3.15}, \href{https://suttacentral.net/sn22.79/en/sujato\#14.26}{SN 22.79:14.26}). The “basis” is that on which consciousness depends, which fuels it and keeps it going. All such dependencies have been uprooted. } ‘This is the basis of that realized one’s consciousness.’\footnote{“Realized one” (\textit{\textsanskrit{tahāgata}}) usually refers to the Buddha, but sometimes, as here, it applies to any arahant. } Why is that? Because even in this very life that realized one is not found, I say.\footnote{Here “not found” renders \textit{ananuvijja}, whereas at \href{https://suttacentral.net/mn22/en/sujato\#25.5}{MN 22:25.5} above it rendered \textit{\textsanskrit{anupalabbhamāna}}. Both terms refer to the self that is not discovered or discoverable. } 

Though\marginnote{37.1} I state and assert this, certain ascetics and brahmins misrepresent me with the incorrect, hollow, false, untruthful claim: ‘The ascetic Gotama is an exterminator. He advocates the annihilation, eradication, and obliteration of an existing being.’\footnote{This is the annihilationist view, exemplified by Ajita of the hair blanket (\href{https://suttacentral.net/dn2/en/sujato\#22.1}{DN 2:22.1}), and referred to above at \href{https://suttacentral.net/mn22/en/sujato\#20.7}{MN 22:20.7}. } They misrepresent me as what I am not, and saying what I do not say. In the past, as today, what I describe is suffering and the cessation of suffering.\footnote{The particle \textit{c’eva} here has its normal role of slight emphasis in conjunctions, and so this phrase does not mean, “I \emph{only} teach suffering and the end of suffering”. The point, rather, is that when he speaks of cessation, it is the cessation of suffering, not the cessation of an “existing being”. } This being so, if others abuse, attack, harass, and trouble the Realized One, he doesn’t get resentful, bitter, and emotionally exasperated. 

Or\marginnote{38.1} if others honor, respect, revere, or venerate him, he doesn’t get thrilled, elated, and emotionally excited.\footnote{The Buddha teaches a similar equanimity at \href{https://suttacentral.net/dn1/en/sujato\#1.5.1}{DN 1:1.5.1}. } If they praise him, he just thinks, ‘They do such things for me regarding what in the past was completely understood.’\footnote{Read \textit{tattha me} “regarding that for me” rather than commentary’s \textit{tattha (i)me} “regarding that for these [aggregates]”. The latter reading is supported by BJT’s \textit{tatrime}, but I think this is probably a backreading from the commentary. In the parallel phrase for the monks below, we would expect \textit{tattha no}, which is in fact attested in the Thai and PTS editions, so I translate “us” accordingly. } 

So,\marginnote{39.1} mendicants, if others abuse, attack, harass, and trouble you, don’t make yourselves resentful, bitter, and emotionally exasperated. Or if others honor, respect, revere, or venerate you, don’t make yourselves thrilled, elated, and emotionally excited. If they praise you, just think, ‘They do such things for us regarding what in the past was completely understood.’\footnote{This seems out of place here, for “completely understood” refers to arahantship, yet as a general teaching this should include everyone. In the Chinese parallel at MA 200, after being abused, one reflects that it is due to past deeds, and after being praised, one reflects that it is due to present knowledge and elimination of defilements. \textsanskrit{Anālayo} (\emph{Comparative Study}, pp. 56–7) suggests that the Pali phrase \textit{pubbe \textsanskrit{pariññātaṁ}} might be a conflation of these two ideas due to a textual confusion which also left the Pali without a reflection on being abused. While this must remain conjectural, it is a neat explanation for the several doctrinal and textual problems of this passage. } 

So,\marginnote{40.1} mendicants, give up what isn't yours.\footnote{This passage is found in several other suttas such as \href{https://suttacentral.net/sn22.33/en/sujato}{SN 22.33}, and applied to the six senses at \href{https://suttacentral.net/sn35.101/en/sujato}{SN 35.101}. } Giving it up will be for your lasting welfare and happiness. 

And\marginnote{41.1} what isn’t yours? Form isn’t yours: give it up. Giving it up will be for your lasting welfare and happiness. 

Feeling\marginnote{41.4} … perception … choices … consciousness isn’t yours: give it up. Giving it up will be for your lasting welfare and happiness. 

What\marginnote{41.12} do you think, mendicants? Suppose a person was to carry off the grass, sticks, branches, and leaves in this Jeta’s Grove, or burn them, or do what they want with them. Would you think, ‘This person is carrying us off, burning us, or doing what they want with us’?” 

“No,\marginnote{41.16} sir. Why is that? Because  to us that’s neither self nor belonging to self.” 

“In\marginnote{41.19} the same way, mendicants, give up what isn't yours. Giving it up will be for your lasting welfare and happiness. And what isn’t yours? Form … feeling … perception … choices … consciousness isn’t yours: give it up. Giving it up will be for your lasting welfare and happiness. 

Thus\marginnote{42.1} the teaching has been well explained by me, made clear, opened, illuminated, and stripped of patchwork.\footnote{Also at \href{https://suttacentral.net/sn12.22/en/sujato\#2.1}{SN 12.22:2.1}. | “Stripped of patchwork” (\textit{chinnapilotika}) because, while the Dhamma has many and varied aspects, they all form a unified whole, not just scraps sewn together. } In this teaching there are mendicants who are perfected, who have ended the defilements, completed the spiritual journey, done what had to be done, laid down the burden, achieved their own goal, utterly ended the fetter of continued existence, and are rightly freed through enlightenment. For them, there is no cycle of rebirths to be found. …\footnote{For the phrase “no cycle of rebirths to be found” (\textit{\textsanskrit{vaṭṭaṁ} \textsanskrit{tesaṁ} natthi \textsanskrit{paññāpanāya}}), compare \href{https://suttacentral.net/dn15/en/sujato\#22.6}{DN 15:22.6}. } 

In\marginnote{43.1} this teaching there are mendicants who have given up the five lower fetters. All of them are reborn spontaneously. They are extinguished there, and are not liable to return from that world. …\footnote{The qualifier “all of them” (\textit{sabbe te}) here and below is not found in similar passages. } 

In\marginnote{44.1} this teaching there are mendicants who, having given up three fetters, and weakened greed, hate, and delusion, are once-returners. All of them come back to this world once only, then make an end of suffering. … 

In\marginnote{45.1} this teaching there are mendicants who have ended three fetters. All of them are stream-enterers, not liable to be reborn in the underworld, bound for awakening. … 

In\marginnote{46.1} this teaching there are mendicants who are followers of teachings, or followers by faith. All of them are bound for awakening.\footnote{A “follower of teachings” is someone who accepts the teachings after reflecting on them with wisdom, while a “follower by faith” accepts the teachings due to their confidence in the teacher. They are both considered to be “practicing to realize the fruit of stream-entry” (\textit{\textsanskrit{sotāpattiphalasacchikiriyāya} \textsanskrit{paṭipanno}}), and, with the maturing of their faculties (\href{https://suttacentral.net/sn48.18/en/sujato\#1.5}{SN 48.18:1.5}), will assuredly attain stream-entry in this life (\href{https://suttacentral.net/sn25.1/en/sujato}{SN 25.1}). } 

Thus\marginnote{47.1} the teaching has been well explained by me, made clear, opened, illuminated, and stripped of patchwork. In this teaching there are those who have a degree of faith and love for me. All of them are bound for heaven.”\footnote{\textit{\textsanskrit{Saddhāmatta}} (“a degree of faith”) is to be distinguished from \textit{\textsanskrit{saddhāmattaka}} (“mere faith”). The former is sufficient for one to progress in the path (\href{https://suttacentral.net/mn70/en/sujato\#21.2}{MN 70:21.2}), but with the latter one just gets by (\href{https://suttacentral.net/mn65/en/sujato\#28.7}{MN 65:28.7}). } 

That\marginnote{47.3} is what the Buddha said. Satisfied, the mendicants approved what the Buddha said. 

%
\section*{{\suttatitleacronym MN 23}{\suttatitletranslation The Termite Mound }{\suttatitleroot Vammikasutta}}
\addcontentsline{toc}{section}{\tocacronym{MN 23} \toctranslation{The Termite Mound } \tocroot{Vammikasutta}}
\markboth{The Termite Mound }{Vammikasutta}
\extramarks{MN 23}{MN 23}

\scevam{So\marginnote{1.1} I have heard. }At one time the Buddha was staying near \textsanskrit{Sāvatthī} in Jeta’s Grove, \textsanskrit{Anāthapiṇḍika}’s monastery. Now at that time Venerable Kassapa the Prince was staying in the Dark Forest.\footnote{“Kassapa the Prince” (\textsanskrit{Kumārakassapa}) was ordained at twenty (\href{https://suttacentral.net/pli-tv-kd1/en/sujato\#75.1.1}{Kd 1:75.1.1}). His verses are collected in the \textsanskrit{Theragāthā} (\href{https://suttacentral.net/thag2.41/en/sujato}{Thag 2.41}). His love of elaborate similes and playful similes is shown in the \textsanskrit{Pāyāsisutta} (\href{https://suttacentral.net/dn23/en/sujato}{DN 23}), and echoed here in the teachings to him. He was declared the foremost of those with brilliant speech (\href{https://suttacentral.net/an1.217/en/sujato}{AN 1.217}). | The Dark Forest (\textit{andhavana}) was a thick grove south of \textsanskrit{Sāvatthī} often visited by monks and nuns for meditation. However, this is the only discourse where someone is said to be staying there. } 

Then,\marginnote{1.4} late at night, a glorious deity, lighting up the entire Dark Forest, went up to Kassapa the Prince, stood to one side, and said:\footnote{The deity offers a series of obscure riddles full of secret meaning, almost like a dream sequence. The deliberate use of obscurity is a hallmark of Brahmanical literature, for “the gods love hidden things” (\textit{\textsanskrit{parokṣakāmā} hi \textsanskrit{devāḥ}}, Śatapatha \textsanskrit{Brāhmaṇa} 10.6.2.2 and throughout). The sequence is set out as a quest for buried treasure. This mytheme is implicit in the Vedic myth of Indra freeing the cattle and the sun from the Vala cave, an act that liberates the truth. Sometimes it is made explicit, as when Indra is said to bring up the treasures buried deep (Rig Veda 8.66.4). The \textsanskrit{Aśvins} are also associated with unearthing buried gems (Rig Veda 1.117.5) or gold (Rig Veda 1.117.12), imagery ultimately based in mining for the wealth of minerals underground. } 

“Monk,\marginnote{2.1} monk! This termite mound fumes by night and flames by day.\footnote{Worship of termite mounds and the deadly snakes they harbor is still common in India today and is probably pre-Vedic. The Śatapatha \textsanskrit{Brāhmaṇa} acknowledges a ritual significance to the anthill as a creative manifestation of the earth upon which offerings may be laid and whose ants were divine (2.6.2.17, 6.3.3.5, 14.1.2.10). In \textsanskrit{Bṛhadāraṇyaka} \textsanskrit{Upaniṣad} 4.4.7, \textsanskrit{Yājñavalkya} compares a dead body bereft of soul to the lifeless slough of a snake on an anthill. | While anthills don’t “fume” and “flame”, large termite mounds leverage the temperature differential of day and night to create convection flows that regulate temperature and flush carbon dioxide. } The brahmin said, ‘Dig, clever one, having picked up the sword!’\footnote{The sword is taken up, a symbol whose roots lie in the discovery of metal smelting and the power this grants to the one who wields the sword. But the phrase has a double meaning, for \textit{sattha} means both “sword” and “sacred treatise”, while \textit{\textsanskrit{abhikkhaṇa}} means both “dig” (\textit{abhi} + \textit{√\textsanskrit{khaṇ}}) and “see” (\textit{abhi} + \textit{√ikkh}; this sense not in Pali, but cf. Vedic \textit{\textsanskrit{abhikhyā}}). Thus, having entered the Dark Wood, the “clever one” (\textit{sumedha}) is urged to use the wisdom of scripture in order to see. This sense is reinforced by the fact that is the “brahmin” who says this. } 

Picking\marginnote{2.4} up the sword and digging, the clever one saw a sticking point:\footnote{\textit{\textsanskrit{Laṅgi}} occurs only here, and is glossed by the commentary with \textit{paligha} (“obstacle”). Such an obscure term must have been selected for a play on words. Sanskrit \textit{lagna} has the primary sense “to stick, adhere”, while also having a sense of the beginning or coming into contact with something. In modern languages such as Hindi, Marathi, or Kannada it has taken on the sense of “marriage”, and perhaps here we see an anticipation of this. If we are on the right track, the \textit{\textsanskrit{laṅgi}} would be the initial challenge for a mendicant seeker, namely their attachment to family. This would explain why it is the first obstacle. } ‘A sticking point, sir!’ The brahmin said, ‘Throw out the sticking point! Dig, clever one, having picked up the sword!’ 

Picking\marginnote{2.9} up the sword and digging, the clever one saw a bullfrog:\footnote{The male Indian bullfrog possesses a pair of prominent dark-blue vocal sacs that puff up and down as it croaks, hence the Pali name \textit{\textsanskrit{uddhumāyika}} (“puffer”). | Rig Veda 7.10 is addressed to frogs, who lie fallow in the dry but spring to life in the rains, filling the countryside with their amorous croaks, like the brahmin priests reciting their verses and passing them down to students. } ‘A bullfrog, sir!’ The brahmin said, ‘Throw out the bullfrog! Dig, clever one, having picked up the sword!’ 

Picking\marginnote{2.14} up the sword and digging, the clever one saw a forked path:\footnote{Rig Veda 10.88.15 speaks of the two paths of gods and mortals (\textit{\textsanskrit{devānām} uta \textsanskrit{martyānām}}), later formalized as the path of the forefathers (\textit{\textsanskrit{pitṛyāna}}) leading to rebirth and that of the gods (\textit{\textsanskrit{devayāna}}) leading to liberation. This corresponds to the twofold choice of the young Siddhattha: to stay at home or to go forth. } ‘A forked path, sir!’ The brahmin said, ‘Throw out the forked path! Dig, clever one, having picked up the sword!’ 

Picking\marginnote{2.19} up the sword and digging, the clever one saw a filter of ash:\footnote{\textit{\textsanskrit{Caṅgavāra}} is a filter through which, according to \href{https://suttacentral.net/ja525/en/sujato\#29.2}{Ja 525:29.2}, water drips away like the brief days of our lives. The commentaries say that such alkaline (\textit{\textsanskrit{khāra}}) filters were used by laundrymen. This refers to the traditional method of creating lye (a soap precursor) for washing, where a barrel is filled with wood ash through which water is passed. No matter how many pots of water are poured in, the commentary adds, it keeps dripping. } ‘A filter of ash, sir!’ The brahmin said, ‘Throw out the filter of ash! Dig, clever one, having picked up the sword!’ 

Picking\marginnote{2.24} up the sword and digging, the clever one saw a tortoise:\footnote{As described in Śatapatha \textsanskrit{Brāhmaṇa} 7.5.1, a tortoise was placed on the altar to represent the primordial powers of creation: the sap of life, the sun, the breath, and the act of creation itself. Its lower and upper shells and its body in-between correspond to the three worlds: the earth, the heavens, and the midspace. It therefore connotes the ancient, endless movement of life. } ‘A tortoise, sir!’ The brahmin said, ‘Throw out the tortoise! Dig, clever one, having taken up the sword!’ 

Picking\marginnote{2.29} up the sword and digging, the clever one saw a butcher’s knife and chopping board:\footnote{In this brutal simile, \textit{asi} is a knife and \textit{\textsanskrit{sūna}} is a place where animals are slaughtered, either a slaughterhouse or, as here, an implement for the slaughter. \href{https://suttacentral.net/mnd1/en/sujato\#23.18}{Mnd 1:23.18} says this phrase has the meaning of “chopping” (\textit{\textsanskrit{adhikuṭṭanaṭṭhena}}). Compare with \textit{\textsanskrit{asiṁ} \textsanskrit{sūnāṁ}} in the violent, sexually graphic hymn at Rig Veda 10.86.18c. Vedic \textit{\textsanskrit{sūna}} was a (sewn) “basket” for carrying the flesh of a slaughtered beast (see also Rig Veda 1.161.10), a sense that developed to “chopping block” and “slaughterhouse”. Jamison and Brereton read the Vedic passage as an analogy for the horse sacrifice, at the climax of which the queen has vulgar and very public sex with the dead horse. This cements in the most explicit way possible the oneness of sexual potency with the cycle of life and death. } ‘A butcher’s knife and chopping board, sir!’ The brahmin said, ‘Throw out the butcher’s knife and chopping board! Dig, clever one, having picked up the sword!’ 

Picking\marginnote{2.34} up the sword and digging, the clever one saw a scrap of meat:\footnote{Another simile shared with \href{https://suttacentral.net/mn22/en/sujato\#3.10}{MN 22:3.10}. A \textit{\textsanskrit{maṁsapesi}} is small enough to be grabbed by a crow (\href{https://suttacentral.net/mn54/en/sujato\#16.1}{MN 54:16.1}), or to quickly disintegrate on a stove (\href{https://suttacentral.net/an7.74/en/sujato\#7.1}{AN 7.74:7.1}), thus is a “scrap of meat” rather than a substantial piece. It emphasizes the meanness and poverty of mortal life, our desires and attachments bound up with this transient “scrap of meat” we call a body. } ‘A scrap of meat, sir!’ The brahmin said, ‘Throw out the scrap of meat! Dig, clever one, having picked up the sword!’ 

Picking\marginnote{2.39} up the sword and digging, the clever one saw a mighty serpent:\footnote{\textit{\textsanskrit{Nāga}} here means both a cobra—the literal snake that lives underneath an anthill—and the serpent of mysterious power that is analogous to an arahant, the “spiritual giant”. } ‘A mighty serpent, sir!’ The brahmin said, ‘Leave the mighty serpent! Do not disturb the mighty serpent! Worship the mighty serpent!’ 

Mendicant,\marginnote{2.43} go to the Buddha and ask him about this riddle. You should remember it in line with his answer. I don’t see anyone in this world—with its gods, \textsanskrit{Māras}, and Divinities, this population with its ascetics and brahmins, its gods and humans—who could provide a satisfying answer to this riddle except for the Realized One or his disciple or someone who has heard it from them.”\footnote{This is, of course, a direct slight on the brahmins who prided themselves on finding explanations for the most obscure and puzzling passages. } 

That\marginnote{2.45} is what that deity said before vanishing right there. 

Then,\marginnote{3.1} when the night had passed, Kassapa the Prince went to the Buddha, bowed, sat down to one side, and told him what had happened. Then he asked: 

“Sir,\marginnote{3.8} what is the termite mound? What is the fuming by night and flaming by day? Who is the brahmin, and who the clever one? What are the sword, the digging, the sticking point, the bullfrog, the forked path, the filter of ash, the tortoise, the butcher’s knife and chopping board, and the scrap of meat? And what is the mighty serpent?” 

“Mendicant,\marginnote{4.1} ‘termite mound’ is a term for this body made up of the four principal states, produced by mother and father, built up from rice and porridge, liable to impermanence, to wearing away and erosion, to breaking up and destruction.\footnote{Just as a termite mound is created by activity driven by eating and excreting, so this body is created from within by food. Also, the body, like a termite mound, is home to countless small creatures. } 

Thinking\marginnote{4.2} and considering all night about what you did during the day—\footnote{Apparently stressing oneself at night over work is not a modern phenomenon. } this is the fuming at night. The work you apply yourself to during the day by body, speech, and mind after thinking about it all night—this is the flaming by day. 

‘Brahmin’\marginnote{4.6} is a term for the Realized One, the perfected one, the fully awakened Buddha. ‘Clever one’ is a term for the trainee mendicant. 

‘Sword’\marginnote{4.8} is a term for noble wisdom. ‘Digging’ is a term for rousing energy. 

‘Sticking\marginnote{4.10} point’ is a term for ignorance. ‘Throw out the sticking point’ means ‘give up ignorance, dig, clever one, having picked up the sword.’ 

‘Bullfrog’\marginnote{4.13} is a term for anger and distress.\footnote{The bullfrog (\textit{\textsanskrit{uddhumāyika}}) evokes those renunciates who become puffed up (\textit{uddhata}) and argumentative over doctrines (eg. \href{https://suttacentral.net/thag17.1/en/sujato\#11.1}{Thag 17.1:11.1}). } ‘Throw out the bullfrog’ means ‘give up anger and distress’ … 

‘A\marginnote{4.16} forked path’ is a term for doubt. ‘Throw out the forked path’ means ‘give up doubt’ … 

‘A\marginnote{4.19} filter of ash’ is a term for the five hindrances, that is:\footnote{The commentary says that, just as water continually drips through such a filter, the mind of someone beset by the five hindrances cannot stay fixed internally on what is wholesome. This contrasts with the mind in absorption, which is illustrated with similes of water that emphasize stillness and containment (\href{https://suttacentral.net/dn2/en/sujato\#76.1}{DN 2:76.1}, etc.). } the hindrances of sensual desire, ill will, dullness and drowsiness, restlessness and remorse, and doubt. ‘Throw out the filter of ash’ means ‘give up the five hindrances’ … 

‘Tortoise’\marginnote{4.23} is a term for the five grasping aggregates, that is:\footnote{The five aggregates are the tortoise’s five limbs, with consciousness as the head. The aggregates represent the changing world of conditions driven by karma, just as the tortoise is the creative force and life of the three worlds. In \textit{\textsanskrit{samādhi}}, the senses are withdrawn like a tortoise’s limbs (\href{https://suttacentral.net/sn35.240/en/sujato\#1.4}{SN 35.240:1.4}). } form, feeling, perception, choices, and consciousness. ‘Throw out the tortoise’ means ‘give up the five grasping aggregates’ … 

‘Butcher’s\marginnote{4.27} knife and chopping board’ is a term for the five kinds of sensual stimulation.\footnote{The senses work because the sense stimulus smashes into the sense organ, like a knife on a chopping block. We seek pleasurable sensations to mask the inherently turbulent nature of sense experience. } Sights known by the eye, which are likable, desirable, agreeable, pleasant, sensual, and arousing. Sounds known by the ear … Smells known by the nose … Tastes known by the tongue … Touches known by the body, which are likable, desirable, agreeable, pleasant, sensual, and arousing. ‘Throw out the butcher’s knife and chopping board’ means ‘give up the five kinds of sensual stimulation’ … 

‘Scrap\marginnote{4.35} of meat’ is a term for greed and relishing. ‘Throw out the scrap of meat’ means ‘give up greed and relishing’ … 

‘Mighty\marginnote{4.38} serpent’ is a term for a mendicant who has ended the defilements. This is the meaning of: ‘Leave the mighty serpent! Do not disturb the mighty serpent! Worship the mighty serpent.’” 

That\marginnote{4.40} is what the Buddha said. Satisfied, Venerable Kassapa the Prince approved what the Buddha said. 

%
\section*{{\suttatitleacronym MN 24}{\suttatitletranslation Chariots at the Ready }{\suttatitleroot Rathavinītasutta}}
\addcontentsline{toc}{section}{\tocacronym{MN 24} \toctranslation{Chariots at the Ready } \tocroot{Rathavinītasutta}}
\markboth{Chariots at the Ready }{Rathavinītasutta}
\extramarks{MN 24}{MN 24}

\scevam{So\marginnote{1.1} I have heard. }At one time the Buddha was staying near \textsanskrit{Rājagaha}, in the Bamboo Grove, the squirrels’ feeding ground.\footnote{The Bamboo Grove was a major monastery near \textsanskrit{Rājagaha}. According to \href{https://suttacentral.net/pli-tv-kd1/en/sujato\#22.16.1}{Kd 1:22.16.1}, it was offered by King \textsanskrit{Bimbisāra} as the very first monastery dedicated to the \textsanskrit{Saṅgha}. } 

Then\marginnote{2.1} several mendicants who had completed the rainy season residence in their native land went to the Buddha, bowed, and sat down to one side. The Buddha said to them:\footnote{The commentary says \textit{\textsanskrit{jātibhūmi}} means the Buddha’s native land, but the word occurs at \href{https://suttacentral.net/an6.54/en/sujato}{AN 6.54}, \href{https://suttacentral.net/pli-tv-pvr14/en/sujato\#2.1}{Pvr 14:2.1}, \href{https://suttacentral.net/ja537/en/sujato\#26.1}{Ja 537:26.1}, and \href{https://suttacentral.net/ja546/en/sujato\#43.1}{Ja 546:43.1}, where it always means “one’s native land”. Given that \textsanskrit{Puṇṇa}, the most famous teacher of their land, was unknown by \textsanskrit{Sāriputta}, it seems likely that it was somewhat remote. One of the Chinese parallels (MA 9) describes him as of light skin and prominent nose, a description that might befit someone from the north-west of India. } 

“In\marginnote{2.2} your native land, mendicants, which of the native mendicants is esteemed in this way: ‘Personally having few wishes, they speak to the mendicants on having few wishes. Personally having contentment, seclusion, aloofness, energy, ethics, immersion, wisdom, freedom, and the knowledge and vision of freedom, they speak to the mendicants on all these things. They’re an adviser and counselor, one who educates, encourages, fires up, and inspires their spiritual companions.’” 

“\textsanskrit{Puṇṇa}\marginnote{2.4} son of \textsanskrit{Mantāṇī}, sir, is esteemed in this way in our native land.”\footnote{Although said to be the foremost Dhamma speaker (\href{https://suttacentral.net/an1.196/en/sujato\#1.1}{AN 1.196:1.1}, see \href{https://suttacentral.net/sn14.15/en/sujato\#2.14}{SN 14.15:2.14}), few of his teachings are recorded. At \href{https://suttacentral.net/sn22.83/en/sujato}{SN 22.83}, Ānanda attributes his initial breakthrough to \textsanskrit{Puṇṇa}, while a single verse is attributed to him at \href{https://suttacentral.net/thag1.4/en/sujato}{Thag 1.4}. | \textsanskrit{Puṇṇa} was evidently from the same maternal clan as \textsanskrit{Aṅgulimāla} (\href{https://suttacentral.net/mn86/en/sujato\#12.12}{MN 86:12.12}). This name does not seem to occur in Sanskrit texts, but \textsanskrit{Mahāvastu} 3.377 spells it as \textsanskrit{Maitrāyaṇī}, a name shared with the \textsanskrit{Maitrāyaṇī} \textsanskrit{Saṁhitā}, the oldest ritual text of the Black Yajurveda. } 

Now\marginnote{3.1} at that time Venerable \textsanskrit{Sāriputta} was sitting not far from the Buddha. Then he thought: 

“\textsanskrit{Puṇṇa}\marginnote{3.3} son of \textsanskrit{Mantāṇī} is fortunate, so very fortunate, in that his sensible spiritual companions praise him point by point in the presence of the Teacher, and that the Teacher seconds that appreciation. Hopefully, some time or other I’ll get to meet Venerable \textsanskrit{Puṇṇa}, and we can have a discussion.” 

When\marginnote{4.1} the Buddha had stayed in \textsanskrit{Rājagaha} as long as he pleased, he set out for \textsanskrit{Sāvatthī}. Traveling stage by stage, he arrived at \textsanskrit{Sāvatthī}, where he stayed in Jeta’s Grove, \textsanskrit{Anāthapiṇḍika}’s monastery. \textsanskrit{Puṇṇa} heard that the Buddha had arrived at \textsanskrit{Sāvatthī}. 

Then\marginnote{5.1} he set his lodgings in order and, taking his bowl and robe, set out for \textsanskrit{Sāvatthī}. Eventually he came to \textsanskrit{Sāvatthī} and Jeta’s Grove. He went up to the Buddha, bowed, and sat down to one side. The Buddha educated, encouraged, fired up, and inspired him with a Dhamma talk. Then, having approved and agreed with what the Buddha said, \textsanskrit{Puṇṇa} got up from his seat, bowed, and respectfully circled the Buddha, keeping him on his right. Then he went to the Dark Forest for the day’s meditation. 

Then\marginnote{6.1} a certain mendicant went up to Venerable \textsanskrit{Sāriputta}, and said to him, “Reverend \textsanskrit{Sāriputta}, the mendicant named \textsanskrit{Puṇṇa}, of whom you have often spoken so highly, after being inspired by a talk of the Buddha’s, left for the Dark Forest for the day’s meditation.” 

\textsanskrit{Sāriputta}\marginnote{7.1} quickly grabbed his sitting cloth and followed behind \textsanskrit{Puṇṇa}, keeping sight of his head.\footnote{While similar actions are found elsewhere, they are not said to be done “quickly” (\textit{\textsanskrit{taramānarūpo}}). \textsanskrit{Sāriputta}’s urgency here reinforces the impression that \textsanskrit{Puṇṇa} lived somewhere distant and visited the central regions only occasionally. } \textsanskrit{Puṇṇa} plunged deep into the Dark Forest and sat at the root of a tree for the day’s meditation. And \textsanskrit{Sāriputta} did likewise. 

Then\marginnote{8.1} in the late afternoon, \textsanskrit{Sāriputta} came out of retreat, went to \textsanskrit{Puṇṇa}, and exchanged greetings with him. When the greetings and polite conversation were over, he sat down to one side and said to \textsanskrit{Puṇṇa}:\footnote{The following seven “purifications” became one of the primary textual sources for the commentarial system of “insight knowledges” (\textit{\textsanskrit{vipassanañāṇa}}) that became central to later Theravada meditation. It would be a mistake, however, to interpret this sutta in light of the insight knowledges, which were developed over a thousand years. } 

“Reverend,\marginnote{9.1} is our spiritual life lived under the Buddha?”\footnote{\textsanskrit{Puṇṇa} was a stranger, so \textsanskrit{Sāriputta} begins by establishing common ground before asking a series of questions to which he expects a negative answer. } 

“Yes,\marginnote{9.2} reverend.” 

“Is\marginnote{9.3} the spiritual life lived under the Buddha for the sake of purification of ethics?”\footnote{The questions are framed in an unusual paired syntax: \textit{\textsanskrit{kiṁ} nu kho, \textsanskrit{āvuso}} alternates with \textit{\textsanskrit{kiṁ} \textsanskrit{panāvuso}}. } 

“Certainly\marginnote{9.4} not.” 

“Well,\marginnote{9.5} is the spiritual life lived under the Buddha for the sake of purification of mind?”\footnote{“Purification of mind” is the development of absorption to abandon the hindrances (\href{https://suttacentral.net/an4.194/en/sujato\#4.1}{AN 4.194:4.1}). } 

“Certainly\marginnote{9.6} not.” 

“Is\marginnote{9.7} the spiritual life lived under the Buddha for the sake of purification of view?”\footnote{“Purified view” is the right view of the stream-enterer who has seen the four noble truths (\href{https://suttacentral.net/an4.194/en/sujato\#5.1}{AN 4.194:5.1}) and rejected the fetter of views of a self (\textit{\textsanskrit{sakkāyadiṭṭhi}}). | Likewise, the five items from here down to “knowledge and vision” all refer to stream-entry. This sutta departs from other early presentations in presenting these as a sequence, whereas normally they are different facets of the experience of stream-entry. } 

“Certainly\marginnote{9.8} not.” 

“Well,\marginnote{9.9} is the spiritual life lived under the Buddha for the sake of purification by traversing doubt?”\footnote{Doubt is one of the fetters overcome by the stream-enterer (eg. \href{https://suttacentral.net/mn64/en/sujato\#6.4}{MN 64:6.4}), since they have the confidence born of direct experience of the four noble truths. | At \href{https://suttacentral.net/ud5.7/en/sujato}{Ud 5.7} “purification by traversing doubt” is said to be achieved by the practice of \textit{\textsanskrit{jhāna}}. } 

“Certainly\marginnote{9.10} not.” 

“Is\marginnote{9.11} the spiritual life lived under the Buddha for the sake of purification of knowledge and vision of what is the path and what is not the path?”\footnote{From their own experience, a stream-enterer knows what leads to the goal and what does not, hence they give up the fetter of “misapprehension of precepts and observances”. | For \textit{\textsanskrit{maggāmagga}}, see the debate between brahmin students at \href{https://suttacentral.net/dn13/en/sujato\#3.1}{DN 13:3.1}, which shows that this term is not unique to Buddhism. There is, however, no evidence that the sequence of purifications is found outside of Buddhism. } 

“Certainly\marginnote{9.12} not.” 

“Well,\marginnote{9.13} is the spiritual life lived under the Buddha for the sake of purification of knowledge and vision of the practice?”\footnote{A stream-enterer not only understands what the correct path is, they have actually practiced it (\href{https://suttacentral.net/sn55.5/en/sujato\#3.3}{SN 55.5:3.3}). } 

“Certainly\marginnote{9.14} not.” 

“Is\marginnote{9.15} the spiritual life lived under the Buddha for the sake of purification of knowledge and vision?”\footnote{This is the vision of the four noble truths at stream-entry (\href{https://suttacentral.net/sn25.1/en/sujato\#2.4}{SN 25.1:2.4}). } 

“Certainly\marginnote{9.16} not.” 

“When\marginnote{10.1} asked each of these questions, you answered, ‘Certainly not.’ Then what exactly is the purpose of leading the spiritual life under the Buddha?” 

“The\marginnote{10.9} purpose of leading the spiritual life under the Buddha is extinguishment by not grasping.”\footnote{In other words, for arahantship not stream-entry. } 

“Reverend,\marginnote{11.1} is purification of ethics extinguishment by not grasping?” 

“Certainly\marginnote{11.2} not, reverend.” 

“Is\marginnote{11.3} purification of mind … 

purification\marginnote{11.5} of view … 

purification\marginnote{11.7} by traversing doubt … 

purification\marginnote{11.9} of knowledge and vision of what is the path and what is not the path … 

purification\marginnote{11.11} of knowledge and vision of the practice … 

Is\marginnote{11.13} purification of knowledge and vision extinguishment by not grasping?” 

“Certainly\marginnote{11.14} not.” 

“Then\marginnote{11.15} is extinguishment by not grasping something apart from these things?” 

“Certainly\marginnote{11.16} not.” 

“When\marginnote{12.1} asked each of these questions, you answered, ‘Certainly not.’ How then should we see the meaning of this statement?” 

“If\marginnote{13.1} the Buddha had declared purification of ethics to be extinguishment by not grasping, he would have declared that which has fuel for grasping to be extinguishment by not grasping.\footnote{Compare the simile of the raft at \href{https://suttacentral.net/mn22/en/sujato\#13.1}{MN 22:13.1}. } If the Buddha had declared purification of mind … purification of view … purification by traversing doubt … purification of knowledge and vision of what is the path and what is not the path … purification of knowledge and vision of the practice … If the Buddha had declared purification of knowledge and vision to be extinguishment by not grasping, he would have declared that which has fuel for grasping to be extinguishment by not grasping. But if extinguishment by not grasping was something apart from these things, an ordinary person would become extinguished.\footnote{That is to say, the goal is not the same as the path, but it cannot be attained without the path. } For an ordinary person lacks these things. 

Well\marginnote{14.1} then, reverend, I shall give you a simile. For by means of a simile some sensible people understand the meaning of what is said. 

Suppose\marginnote{14.3} that, while staying in \textsanskrit{Sāvatthī}, King Pasenadi of Kosala had some urgent business come up in \textsanskrit{Sāketa}. Now, between \textsanskrit{Sāvatthī} and \textsanskrit{Sāketa} seven chariots were stationed at the ready for him.\footnote{\textit{\textsanskrit{Rathavinīta}}, which lends the sutta its title, is explained by the commentary as “seven chariots with horses at the ready”, where \textit{\textsanskrit{vinīta}} has its usual sense of “trained, readied”. While the series of chariots forms a relay, the word \textit{\textsanskrit{vinīta}} does not mean “relay”. The term and the simile are unique to this sutta. } Then Pasenadi, having departed \textsanskrit{Sāvatthī}, mounted the first chariot at the ready by the gate of the royal compound. The first chariot at the ready would bring him to the second, where he’d dismount and mount the second chariot. The second chariot at the ready would bring him to the third … The third chariot at the ready would bring him to the fourth … The fourth chariot at the ready would bring him to the fifth … The fifth chariot at the ready would bring him to the sixth … The sixth chariot at the ready would bring him to the seventh, where he’d dismount and mount the seventh chariot. The seventh chariot at the ready would bring him to the gate of the royal compound of \textsanskrit{Sāketa}. And when he was at the gate, friends and colleagues, relatives and kin would ask him: ‘Great king, did you come to \textsanskrit{Sāketa} from \textsanskrit{Sāvatthī} by this chariot at the ready?’ If asked this, how should King Pasenadi rightly reply?” 

“The\marginnote{14.15} king should reply: ‘Well, while staying in \textsanskrit{Sāvatthī}, I had some urgent business come up in \textsanskrit{Sāketa}. Now, between \textsanskrit{Sāvatthī} and \textsanskrit{Sāketa} seven chariots were stationed at the ready for me. Then, having departed \textsanskrit{Sāvatthī}, I mounted the first chariot at the ready by the gate of the royal compound. The first chariot at the ready brought me to the second, where I dismounted and mounted the second chariot. … The second chariot brought me to the third … the fourth … the fifth … the sixth … The sixth chariot at the ready brought me to the seventh, where I dismounted and mounted the seventh chariot. The seventh chariot at the ready brought me to the gate of the royal compound of \textsanskrit{Sāketa}.’ That’s how King Pasenadi should rightly reply.” 

“In\marginnote{15.1} the same way, reverend, purification of ethics is only for the sake of purification of mind. Purification of mind is only for the sake of purification of view. Purification of view is only for the sake of purification by traversing doubt. Purification by traversing doubt is only for the sake of purification of knowledge and vision of what is the path and what is not the path. Purification of knowledge and vision of what is the path and what is not the path is only for the sake of purification of knowledge and vision of the practice. Purification of knowledge and vision of the practice is only for the sake of purification of knowledge and vision. Purification of knowledge and vision is only for the sake of extinguishment by not grasping. The spiritual life is lived under the Buddha for the sake of extinguishment by not grasping.” 

When\marginnote{16.1} he said this, \textsanskrit{Sāriputta} said to \textsanskrit{Puṇṇa}, “What is the venerable’s name? And how are you known among your spiritual companions?” 

“Reverend,\marginnote{16.3} my name is \textsanskrit{Puṇṇa}. And I am known as ‘son of \textsanskrit{Mantāṇī}’ among my spiritual companions.” 

“It’s\marginnote{16.5} incredible, reverend, it’s amazing! Venerable \textsanskrit{Puṇṇa} son of \textsanskrit{Mantāṇī} has answered each deep question point by point, as a learned disciple who rightly understands the teacher’s instructions. It is fortunate for his spiritual companions, so very fortunate, that they get to see Venerable \textsanskrit{Puṇṇa} son of \textsanskrit{Mantāṇī} and pay homage to him. Even if they only got to see him and pay respects to him by carrying him around on their heads on a roll of cloth, it would still be very fortunate for them! And it’s fortunate for me, so very fortunate, that I get to see the venerable and pay homage to him.”\footnote{“A roll of cloth” (\textit{\textsanskrit{celaṇḍuka}}) is a unique term, and these are unique words of praise. } 

When\marginnote{17.1} he said this, \textsanskrit{Puṇṇa} said to \textsanskrit{Sāriputta}, “What is the venerable’s name? And how are you known among your spiritual companions?” 

“Reverend,\marginnote{17.3} my name is Upatissa. And I am known as \textsanskrit{Sāriputta} among my spiritual companions.” 

“Goodness!\marginnote{17.5} I had no idea I was consulting with \emph{the} Venerable \textsanskrit{Sāriputta}, the disciple who is fit to be compared with the Teacher himself!\footnote{“Fit to be compared with the Teacher” (\textit{satthukappa}) is another term that is unique in early Pali. } If I’d known, I would not have said so much. It’s incredible, reverend, it’s amazing! Venerable \textsanskrit{Sāriputta} has asked each deep question point by point, as a learned disciple who rightly understands the teacher’s instructions.\footnote{“Point by point” (\textit{anumassa anumassa}), found several times above, is yet another phrasing unique to this sutta. | This sutta has several features that, taken together, indicate that it is later than the bulk of the four \textsanskrit{nikāyas}: many unique phrases; the hints that \textsanskrit{Puṇṇa} is from a distant land; and most of all, the doctrine. The text here takes the set of four purifications at \href{https://suttacentral.net/an4.194/en/sujato}{AN 4.194} and expands “view” into five to make the seven of the current sutta. Now, AN 4.194 lacks any mention of where the Buddha was at the time and hence was likely taught by Ānanda after the Buddha’s death. Then the set is expanded to nine (plus wisdom and freedom) at \href{https://suttacentral.net/dn34/en/sujato\#2.2.6}{DN 34:2.2.6}, one of the latest suttas in the \textsanskrit{nikāyas}. Most importantly, the five items pertaining to stream-entry are presented here as a linear sequence, anticipating the detailed analysis of stages of insight in later Theravada. Thus not only are the associated suttas late, this sutta is developed further than them. } It is fortunate for his spiritual companions, so very fortunate, that they get to see Venerable \textsanskrit{Sāriputta} and pay homage to him. Even if they only got to see him and pay respects to him by carrying him around on their heads on a roll of cloth, it would still be very fortunate for them! And it’s fortunate for me, so very fortunate, that I get to see the venerable and pay homage to him.” 

And\marginnote{17.13} so these two spiritual giants agreed with each others’ fine words.\footnote{See \href{https://suttacentral.net/mn5/en/sujato\#33.4}{MN 5:33.4}. } 

%
\section*{{\suttatitleacronym MN 25}{\suttatitletranslation Sowing }{\suttatitleroot Nivāpasutta}}
\addcontentsline{toc}{section}{\tocacronym{MN 25} \toctranslation{Sowing } \tocroot{Nivāpasutta}}
\markboth{Sowing }{Nivāpasutta}
\extramarks{MN 25}{MN 25}

\scevam{So\marginnote{1.1} I have heard. }At one time the Buddha was staying near \textsanskrit{Sāvatthī} in Jeta’s Grove, \textsanskrit{Anāthapiṇḍika}’s monastery. There the Buddha addressed the mendicants, “Mendicants!” 

“Venerable\marginnote{1.5} sir,” they replied. The Buddha said this: 

“Mendicants,\marginnote{2.1} a sower does not sow seed for deer thinking,\footnote{“Sower sows seed” renders the triplicated expression \textit{\textsanskrit{nevāpiko} \textsanskrit{nivāpaṁ} nivapati}. The root sense of \textit{vap} is to “cast down” or “sow” seed; it is also used of offerings for the ancestors in the \textit{\textsanskrit{śrāddha}} ceremony. It is tempting to translate this phrase as a “trapper” who lays “bait”. But the aim of the “sower” is simply to grow a crop, and to do this they need to keep the deer out. That is the guile of the story: they appear harmless. | For an amicable solution to this problem, see the conclusion to the \textsanskrit{Rurumigarāja} \textsanskrit{Jātaka} (\href{https://suttacentral.net/ja482/en/sujato}{Ja 482}) | \textit{\textsanskrit{Migajāta}} (“deer”) might be better translated as “wild animals” here. } ‘May the deer, enjoying this seed, be long-lived and beautiful. May they live long and prosper!’ A sower sows seed for deer thinking, ‘When these deer encroach on where I sow the seed, they’ll recklessly enjoy eating it. They’ll become indulgent, then they’ll become negligent, and then I’ll be able to do what I want with them on account of this seed.’ 

And\marginnote{3.1} indeed, the first herd of deer encroached on where the sower sowed the seed and recklessly enjoyed eating it. They became indulgent, then they became negligent, and then the sower was able to do what he wanted with them on account of that seed. And that’s how the first herd of deer failed to get free from the sower’s power. 

So\marginnote{4.1} then a second herd of deer thought up a plan, ‘The first herd of deer became indulgent … and failed to get free of the sower’s power. Why don’t we refrain from eating the seed altogether? Avoiding dangerous food, we can venture deep into a wilderness region and live there.’ And that’s just what they did. But when it came to the last month of summer, the grass and water ran out. Their bodies became much too thin, and they lost their strength and energy. So they returned to where the sower had sown the seed. Encroaching, they recklessly enjoyed eating it … And that’s how the second herd of deer failed to get free from the sower’s power. 

So\marginnote{5.1} then a third herd of deer thought up a plan, ‘The first … and second herds of deer … failed to get free of the sower’s power. Why don’t we set up our lair close by where the sower has sown the seed? Then we can encroach and enjoy eating without being reckless. We won’t become indulgent, then we won’t become negligent, and then the sower won’t be able to do what he wants with us on account of that seed.’ And that’s just what they did. 

So\marginnote{5.19} the sower and his helpers thought, ‘Wow, this third herd of deer is so sneaky and devious, they must be some kind of strange spirits with magical abilities!\footnote{The repeated \textit{\textsanskrit{nāma}} is strongly emphatic. | \textit{Parajana} sometimes means “stranger” (\href{https://suttacentral.net/mil6.3.1/en/sujato\#13.2}{Mil 6.3.1:13.2}), but the commentary here glosses it as \textit{yakkha}, and a \textit{yakkha} named Parajana makes an appearance in (\href{https://suttacentral.net/mn31/en/sujato\#21.1}{MN 31:21.1}). Presumably it has a sense not dissimilar to \textit{amanussa}, “non-human”. Compare the \textit{\textsanskrit{itarajanā}} of Atharvaveda \textsanskrit{Saṁhita} 8.10.28a. } For they eat the seed we’ve sown without us knowing how they come and go. Why don’t we surround the seed on all sides by staking out high nets? Hopefully we might get to see the lair where they go to hide out.’\footnote{\textit{\textsanskrit{Vākarā}} (variant \textit{\textsanskrit{vāgurā}}, which is also the Sanskrit form) is a net or snare, which here is staked out on sticks. The deer would have to leap over it, revealing their position. | \textit{\textsanskrit{Gāha}} here is from \textit{√\textsanskrit{gāh}} (“deep place, hidey-hole”) rather than \textit{√gah} (“take”). } And that’s just what they did. And they saw the lair where the third herd of deer went to hide out. And that’s how the third herd failed to get free from the sower’s power. 

So\marginnote{6.1} then a fourth herd of deer thought up a plan, ‘The first … second … and third herds of deer … failed to get free of the sower’s power. Why don’t we set up our lair somewhere the sower and his helpers can’t go? Then we can intrude on where the sower has sown the seed and enjoy eating it without being reckless. We won’t become indulgent, then we won’t become negligent, and then the sower won’t be able to do with us what he wants on account of that seed.’ And that’s just what they did. 

So\marginnote{6.31} the sower and his helpers thought, ‘Wow, this fourth herd of deer is so sneaky and devious, they must be some kind of strange spirits with magical abilities! For they eat the seed we’ve sown without us knowing how they come and go. Why don’t we surround the seed on all sides by staking out high nets? Hopefully we might get to see the lair where they go to hide out.’ And that’s just what they did. But they couldn’t see the lair where the fourth herd of deer went to hide out. So the sower and his helpers thought, ‘If we disturb this fourth herd of deer, they’ll disturb others, who in turn will disturb even more. Then all of the deer will escape this seed we’ve sown. Why don’t we just keep an eye on that fourth herd?’ And that’s just what they did. And that’s how the fourth herd of deer escaped the sower’s power. 

I’ve\marginnote{7.1} made up this simile to make a point. And this is what it means. 

‘Seed’\marginnote{7.3} is a term for the five kinds of sensual stimulation. 

‘Sower’\marginnote{7.4} is a term for \textsanskrit{Māra} the Wicked. 

‘Sower’s\marginnote{7.5} helpers’ is a term for \textsanskrit{Māra}’s assembly. 

‘Deer’\marginnote{7.6} is a term for ascetics and brahmins. 

Now,\marginnote{8.1} the first group of ascetics and brahmins encroached on where the seed and the worldly pleasures of the flesh were sown by \textsanskrit{Māra} and recklessly enjoyed eating it. They became indulgent, then they became negligent, and then \textsanskrit{Māra} was able to do what he wanted with them on account of that seed and the worldly pleasures of the flesh. And that’s how the first group of ascetics and brahmins failed to get free from \textsanskrit{Māra}’s power. This first group of ascetics and brahmins is just like the first herd of deer, I say. 

So\marginnote{9.1} then a second group of ascetics and brahmins thought up a plan, ‘The first group of ascetics and brahmins became indulgent … and failed to get free of \textsanskrit{Māra}’s power. Why don’t we refrain from eating the seed and the worldly pleasures of the flesh altogether? Avoiding dangerous food, we can venture deep into a wilderness region and live there.’\footnote{Following this the \textsanskrit{Mahāsaṅgīti} edition has a ghost sentence, formed by adding the first part of the subsequent sentence with the second part of the previous. It is absent from the PTS text. } And that’s just what they did. They ate herbs, millet, wild rice, poor rice, water lettuce, rice bran, scum from boiling rice, sesame flour, grass, or cow dung. They survived on forest roots and fruits, or eating fallen fruit. 

But\marginnote{9.9} when it came to the last month of summer, the grass and water ran out. Their bodies became much too thin, and they lost their strength and energy. Because of this, they lost their heart’s release,\footnote{“Heart’s release” (\textit{cetovimutti}) is a term for the meditative absorptions, which are listed below. | While these meditators did indeed escape \textsanskrit{Māra} for a while, from the point of view of the eightfold path, they were neglecting right livelihood. A Buddhist mendicant relies on alms, which not only ensures adequate nutrition, but helps build community and spread the Dhamma. | For the dependency of absorption on food, see also the Buddha’s account of his own practice before awakening (\href{https://suttacentral.net/mn36/en/sujato\#33.2}{MN 36:33.2}). } so they went back to where \textsanskrit{Māra} had sown the seed and the worldly pleasures of the flesh. Intruding on that place, they recklessly enjoyed eating them … And that’s how the second group of ascetics and brahmins failed to get free from \textsanskrit{Māra}’s power. This second group of ascetics and brahmins is just like the second herd of deer, I say. 

So\marginnote{10.1} then a third group of ascetics and brahmins thought up a plan, ‘The first … and second groups of ascetics and brahmins … failed to get free of \textsanskrit{Māra}’s power. Why don’t we set up our lair close by where \textsanskrit{Māra} has sown the seed and those worldly pleasures of the flesh? Then we can encroach on it and enjoy eating without being reckless. We won’t become indulgent, then we won’t become negligent, and then \textsanskrit{Māra} won’t be able to do what he wants with us on account of that seed and those worldly pleasures of the flesh.’ 

And\marginnote{10.17} that’s just what they did. Still, they had such views as these:\footnote{This group failed to develop right view. For more on these speculative views, see \href{https://suttacentral.net/mn63/en/sujato}{MN 63} and \href{https://suttacentral.net/mn72/en/sujato}{MN 72}. } ‘The cosmos is eternal’ or ‘The cosmos is not eternal’;\footnote{This is the famous list of ten “undeclared points”, which are found throughout the suttas (eg. \href{https://suttacentral.net/dn9/en/sujato\#25.3}{DN 9:25.3}, \href{https://suttacentral.net/mn63/en/sujato\#2.3}{MN 63:2.3}, \href{https://suttacentral.net/mn72/en/sujato\#3.1}{MN 72:3.1}, and the whole of SN 44). They seem to have functioned as a kind of checklist by which philosophers could be evaluated and classified. In the Jain \textsanskrit{Bhagavatisūtra} 9.33, \textsanskrit{Indrabhūti} Gautama tests the impostor \textsanskrit{Jamāli}’s bona fides by asking about the eternity of the world and the soul. When \textsanskrit{Jamāli} failed to answer, \textsanskrit{Mahāvīra} stepped in, explaining that both the sould and the world are in one sense eternal since they last forever, but in another sense they are transient since they go through different phases. | The word \textit{loka} occurs in a number of senses, but here it refers to the entire “cosmos” of countless worlds. } ‘The cosmos is finite’ or ‘The cosmos is infinite’; ‘The soul and the body are the same thing’ or ‘The soul and the body are different things’; or that after death, a realized one still exists, or no longer exists, or both still exists and no longer exists, or neither still exists nor no longer exists. And that’s how the third group of ascetics and brahmins failed to get free from \textsanskrit{Māra}’s power. This third group of ascetics and brahmins is just like the third herd of deer, I say. 

So\marginnote{11.1} then a fourth group of ascetics and brahmins thought up a plan, ‘The first … second … and third groups of ascetics and brahmins … failed to get free of \textsanskrit{Māra}’s power. Why don’t we set up our lair where \textsanskrit{Māra} and his assembly can’t go? Then we can encroach on where \textsanskrit{Māra} has sown the seed and those worldly pleasures of the flesh, and enjoy eating without being reckless. We won’t become indulgent, then we won’t become negligent, and then \textsanskrit{Māra} won’t be able to do what he wants with us on account of that seed and those worldly pleasures of the flesh.’ 

And\marginnote{11.29} that’s just what they did. And that’s how the fourth group of ascetics and brahmins got free from \textsanskrit{Māra}’s power. This fourth group of ascetics and brahmins is just like the fourth herd of deer, I say. 

And\marginnote{12.1} where is it that \textsanskrit{Māra} and his assembly can’t go? It’s when a mendicant, quite secluded from sensual pleasures, secluded from unskillful qualities, enters and remains in the first absorption, which has the rapture and bliss born of seclusion, while placing the mind and keeping it connected.\footnote{Here the text shifts from “ascetics and brahmins”—that is, any kind of religious practitioner—to “mendicants”, who were Buddhist alms-gatherers. Having learned the importance of right livelihood and right view, they are now practicing absorption as part of the noble eightfold path. } This is called a mendicant who has blinded \textsanskrit{Māra}, put out his eyes without a trace, and gone where the Wicked One cannot see. 

Furthermore,\marginnote{13.1} as the placing of the mind and keeping it connected are stilled, a mendicant enters and remains in the second absorption, which has the rapture and bliss born of immersion, with internal clarity and mind at one, without placing the mind and keeping it connected. This is called a mendicant who has blinded \textsanskrit{Māra} … 

Furthermore,\marginnote{14.1} with the fading away of rapture, a mendicant enters and remains in the third absorption, where they meditate with equanimity, mindful and aware, personally experiencing the bliss of which the noble ones declare, ‘Equanimous and mindful, one meditates in bliss.’ This is called a mendicant who has blinded \textsanskrit{Māra} … 

Furthermore,\marginnote{15.1} giving up pleasure and pain, and ending former happiness and sadness, a mendicant enters and remains in the fourth absorption, without pleasure or pain, with pure equanimity and mindfulness. This is called a mendicant who has blinded \textsanskrit{Māra} … 

Furthermore,\marginnote{16.1} a mendicant, going totally beyond perceptions of form, with the ending of perceptions of impingement, not focusing on perceptions of diversity, aware that ‘space is infinite’, enters and remains in the dimension of infinite space. This is called a mendicant who has blinded \textsanskrit{Māra} … 

Furthermore,\marginnote{17.1} a mendicant, going totally beyond the dimension of infinite space, aware that ‘consciousness is infinite’, enters and remains in the dimension of infinite consciousness. This is called a mendicant who has blinded \textsanskrit{Māra} … 

Furthermore,\marginnote{18.1} a mendicant, going totally beyond the dimension of infinite consciousness, aware that ‘there is nothing at all’, enters and remains in the dimension of nothingness. This is called a mendicant who has blinded \textsanskrit{Māra} … 

Furthermore,\marginnote{19.1} a mendicant, going totally beyond the dimension of nothingness, enters and remains in the dimension of neither perception nor non-perception. This is called a mendicant who has blinded \textsanskrit{Māra} … 

Furthermore,\marginnote{20.1} a mendicant, going totally beyond the dimension of neither perception nor non-perception, enters and remains in the cessation of perception and feeling. And, having seen with wisdom, their defilements come to an end.\footnote{The “cessation of perception and feeling” (\textit{\textsanskrit{saññāvedayitanirodha}}) is a culminating meditation state of supreme subtlety that often leads directly to awakening (but see \href{https://suttacentral.net/an5.166/en/sujato}{AN 5.166}). The state itself, like all meditation states, is temporary, but afterwards the defilements can be eliminated forever. This liberating insight is the consequence of the balanced development of all eight factors of the path. } This is called a mendicant who has blinded \textsanskrit{Māra}, put out his eyes without a trace, and gone where the Wicked One cannot see. And they’ve crossed over clinging to the world.” 

That\marginnote{20.3} is what the Buddha said. Satisfied, the mendicants approved what the Buddha said. 

%
\section*{{\suttatitleacronym MN 26}{\suttatitletranslation The Noble Quest }{\suttatitleroot Pāsarāsisutta}}
\addcontentsline{toc}{section}{\tocacronym{MN 26} \toctranslation{The Noble Quest } \tocroot{Pāsarāsisutta}}
\markboth{The Noble Quest }{Pāsarāsisutta}
\extramarks{MN 26}{MN 26}

\scevam{So\marginnote{1.1} I have heard.\footnote{This discourse is known in Pali manuscripts and commentaries either as “The Noble Quest” or “The Pile of Snares”, whereas the Chinese parallel at MA 204 is titled “The Discourse at \textsanskrit{Rāma}’s Hermitage”. It is one of several discourses in the Majjhima that include a partial account of Siddhattha’s practice before awakening, an account that became one of the key events in the Buddha’s biography. This particular version focuses on his experience with Brahmanical teachers, while \href{https://suttacentral.net/mn36/en/sujato}{MN 36}, \href{https://suttacentral.net/mn85/en/sujato}{MN 85}, and \href{https://suttacentral.net/mn100/en/sujato}{MN 100} include a long passage detailing his Jain-like fervent austerities. } }At one time the Buddha was staying near \textsanskrit{Sāvatthī} in Jeta’s Grove, \textsanskrit{Anāthapiṇḍika}’s monastery. 

Then\marginnote{2.1} the Buddha robed up in the morning and, taking his bowl and robe, entered \textsanskrit{Sāvatthī} for alms. Then several mendicants went up to Venerable Ānanda and said to him, “Reverend, it’s been a long time since we’ve heard a Dhamma talk from the Buddha.\footnote{Why was the Buddha reluctant to teach? This is puzzling because where we find this trope elsewhere there is a clear reason: at \href{https://suttacentral.net/an7.52/en/sujato\#1.3}{AN 7.52:1.3} the Buddha was making one of his rare visits to \textsanskrit{Campā}, while at \href{https://suttacentral.net/sn22.81/en/sujato\#3.5}{SN 22.81:3.5} he had left the squabbling Sangha for the seclusion of the forest. Here, in the central teaching location of \textsanskrit{Sāvatthī}, there is no such reason. Rather, the events are best understood as narrative foreshadowing. His reluctance requires an intervention by others, and when the teaching finally takes place it is not in a Buddhist monastery but on Brahmanical grounds. The story goes on to tell how, after his awakening, he was reluctant to teach until the intervention of \textsanskrit{Brahmā} (\href{https://suttacentral.net/mn26/en/sujato\#19.1}{MN 26:19.1}). This narrative mirroring creates a hidden link (\textit{sandhi}) that frames the awakening in a Brahmanical context. The very obscurity of the connection is the point, for “the gods love hidden things”. The theme of reluctance is extended further to the reluctance of the five mendicants to receive the Buddha. It draws on the precedent of \textsanskrit{Yājñavalkya}’s reluctance to teach Janaka (\textsanskrit{Bṛhadāraṇyaka} \textsanskrit{Upaniṣad} 4.3.1). } It would be good if we got to hear a Dhamma talk from the Buddha.” 

“Well\marginnote{2.5} then, reverends, go to the brahmin Rammaka’s hermitage.\footnote{Rammaka appears only here, and the commentary offers no information. His name presumably signifies that he was a follower of the teacher \textsanskrit{Rāma} who appears below. This complements \href{https://suttacentral.net/an3.126/en/sujato\#3.3}{AN 3.126:3.3}, where the Buddha stays in the hermitage of \textsanskrit{Bharaṇḍu} the \textsanskrit{Kālāma}, who was apparently a student of \textsanskrit{Āḷāra} \textsanskrit{Kālāma}. } Hopefully you’ll get to hear a Dhamma talk from the Buddha.” 

“Yes,\marginnote{2.7} reverend,” they replied. 

Then,\marginnote{3.1} after the meal, on his return from almsround, the Buddha addressed Ānanda,\footnote{The narrative sequence from the Buddha returning from alms round to coming out from his bath is also found at \href{https://suttacentral.net/an6.43/en/sujato\#1.3}{AN 6.43:1.3}. There, there is a clear justification for the story, as the discourse concerns a royal elephant parade seen after bathing. Here, once more, the purpose is implicit: a brahmanical student must bathe before any ritual. } “Come, Ānanda, let’s go to the stilt longhouse of \textsanskrit{Migāra}’s mother in the Eastern Monastery for the day’s meditation.”\footnote{After the Jetavana, this was the best-known monastery in \textsanskrit{Sāvatthī}. It was offered by the lady \textsanskrit{Visākhā}, known as \textsanskrit{Migāra}’s mother. } 

“Yes,\marginnote{3.3} sir,” Ānanda replied. So the Buddha went with Ānanda to the Eastern Monastery for the day’s meditation. In the late afternoon the Buddha came out of retreat and addressed Ānanda, “Come, Ānanda, let’s go to the eastern gate to bathe.” 

“Yes,\marginnote{3.7} sir,” Ānanda replied. 

So\marginnote{3.8} the Buddha went with Ānanda to the eastern gate to bathe. When he had bathed and emerged from the water he stood in one robe drying his limbs. Then Ānanda said to the Buddha, “Sir, the hermitage of the brahmin Rammaka is nearby. It’s so delightful, so lovely. Please visit it out of sympathy.” The Buddha consented with silence. 

He\marginnote{4.1} went to the brahmin Rammaka’s hermitage. Now at that time several mendicants were sitting together in the hermitage talking about the teaching.\footnote{Here (and at \href{https://suttacentral.net/mn92/en/sujato\#4.1}{MN 92:4.1}) a “hermitage” (\textit{assama}, “ashram”) is a sizable building suitable for gatherings, whereas at \href{https://suttacentral.net/sn11.9/en/sujato\#1.4}{SN 11.9:1.4} it is a gated compound with leaf huts. It is normally used for Brahmanical places. } The Buddha stood outside the door waiting for the talk to end. When he knew the talk had ended he cleared his throat and knocked on the door-panel.\footnote{As at \href{https://suttacentral.net/an9.4/en/sujato\#1.4}{AN 9.4:1.4}. } The mendicants opened the door for the Buddha, and he entered the hermitage, where he sat on the seat spread out and addressed the mendicants, “Mendicants, what were you sitting talking about just now? What conversation was left unfinished?” 

“Sir,\marginnote{4.9} our unfinished discussion on the teaching was about the Buddha himself when the Buddha arrived.” 

“Good,\marginnote{4.10} mendicants! It’s appropriate for gentlemen like you, who have gone forth out of faith from the lay life to homelessness, to sit together and talk about the teaching. When you’re sitting together you should do one of two things: discuss the teachings or keep noble silence.\footnote{“Noble silence” is narrowly defined as the second absorption (\href{https://suttacentral.net/sn21.1/en/sujato}{SN 21.1}). } 

Mendicants,\marginnote{5.1} there are these two quests:\footnote{The Buddha picks up from the fact that the mendicants were talking about him, but continues in a depersonalized manner. | The substance of this passage is taught more briefly at \href{https://suttacentral.net/an4.255/en/sujato}{AN 4.255}. } the noble quest and the ignoble quest.\footnote{These opposing quests (or “searches”, \textit{\textsanskrit{pariyesanā}}) respond to \textsanskrit{Bṛhadāraṇyaka} \textsanskrit{Upaniṣad} 4.4.22, which we have already noted as a source for the phrase \textit{so \textsanskrit{attā} so loko} (\href{https://suttacentral.net/mn22/en/sujato\#15.10}{MN 22:15.10}). It says that the wise ones of old, renouncing the search for sons, for wealth, and for the heavenly worlds, lived the mendicant’s life (\textit{\textsanskrit{putraiṣaṇāyāśca} \textsanskrit{vittaiṣaṇāyāśca} \textsanskrit{lokaiṣaṇāyāśca} \textsanskrit{vyutthāyātha} \textsanskrit{bhikśācaryaṁ} caranti}). The inner self sought by the brahmins is described as unborn, made of consciousness, undecaying, unattached, and unaffected by karma; knowing it one becomes a sage. } 

And\marginnote{5.3} what is the ignoble quest? It’s when someone who is themselves liable to be reborn seeks what is also liable to be reborn. Themselves liable to grow old, fall sick, die, sorrow, and become corrupted, they seek what is also liable to these things. 

And\marginnote{6.1} what should be described as liable to be reborn?\footnote{The Pali includes “gold and money” among those things subject to birth, old age, and defilement. The Chinese parallel MA 204 (T i 776a7; cf. T 765 at T xvii 679b23) does not mention “birth” but says money and jewelry are subject to old age, defilement, etc. Elsewhere in the suttas, rebirth, old age, and defilement are qualities of sentient beings, not inanimate objects, so this might be a textual corruption. Nonetheless, the commentary offers explanations for why gold and money are included under these specific categories and not the rest, and this, together with the Chinese text, shows that if it is a corruption it is an old one. I give the commentarial explanations. } Partners and children, male and female bondservants, goats and sheep, chickens and pigs, elephants and cattle, and gold and currency are liable to be reborn.\footnote{Gold and money are subject to “birth” because they are produced by heat (\textit{\textsanskrit{utusamuṭṭhāna}}). | The word for “gold” here, one of several in Pali, is \textit{\textsanskrit{jātarūpa}}, literally “born form”, i.e. that which is naturally beautiful. In \textsanskrit{Bṛhadāraṇyaka} \textsanskrit{Upaniṣad} 6.4.25, a newborn infant  is to be fed mixed curd, honey, and ghee with a piece of \textit{\textsanskrit{jātarūpa}}, thus placing “golden” Vedic speech in him. This rare early instance of \textit{\textsanskrit{jātarūpa}} shows that it had a felt connection with the idea of birth. Perhaps this connection prompted its inclusion here. From that, “money” (\textit{rajata}) was brought in as the two form a stock phrase, and they were then applied elsewhere in the list. } These attachments are liable to be reborn. Someone who is tied, infatuated, and attached to such things, themselves liable to being reborn, seeks what is also liable to be reborn. 

And\marginnote{7.1} what should be described as liable to grow old? Partners and children, male and female bondservants, goats and sheep, chickens and pigs, elephants and cattle, and gold and currency are liable to grow old.\footnote{Gold and money grow old due to rust or grime. } These attachments are liable to grow old. Someone who is tied, infatuated, and attached to such things, themselves liable to grow old, seeks what is also liable to grow old. 

And\marginnote{8.1} what should be described as liable to fall sick? Partners and children, male and female bondservants, goats and sheep, chickens and pigs, and elephants and cattle are liable to fall sick. These attachments are liable to fall sick. Someone who is tied, infatuated, and attached to such things, themselves liable to falling sick, seeks what is also liable to fall sick. 

And\marginnote{9.1} what should be described as liable to die? Partners and children, male and female bondservants, goats and sheep, chickens and pigs, and elephants and cattle are liable to die. These attachments are liable to die. Someone who is tied, infatuated, and attached to such things, themselves liable to die, seeks what is also liable to die. 

And\marginnote{10.1} what should be described as liable to sorrow? Partners and children, male and female bondservants, goats and sheep, chickens and pigs, and elephants and cattle are liable to sorrow. These attachments are liable to sorrow. Someone who is tied, infatuated, and attached to such things, themselves liable to sorrow, seeks what is also liable to sorrow. 

And\marginnote{11.1} what should be described as liable to corruption? Partners and children, male and female bondservants, goats and sheep, chickens and pigs, elephants and cattle, and gold and currency are liable to corruption.\footnote{Gold and money are liable to corruption by iron, etc. } These attachments are liable to corruption. Someone who is tied, infatuated, and attached to such things, themselves liable to corruption, seeks what is also liable to corruption. This is the ignoble quest. 

And\marginnote{12.1} what is the noble quest? It’s when someone who is themselves liable to be reborn, understanding the drawbacks in being liable to be reborn, seeks that which is free of rebirth, the supreme sanctuary from the yoke, extinguishment. Themselves liable to grow old, fall sick, die, sorrow, and become corrupted, understanding the drawbacks in these things, they seek that which is free of old age, sickness, death, sorrow, and corruption, the supreme sanctuary from the yoke, extinguishment.\footnote{As pointed out by K.R. Norman in his \emph{Mistaken Ideas About \textsanskrit{Nibbāna}} (1994), the epithets \textit{\textsanskrit{ajātaṁ}} (etc.) don’t mean that \textsanskrit{Nibbāna} is “unborn”, but rather than in \textsanskrit{Nibbāna} there is no rebirth. Likewise, \textit{amata} does not mean “immortal” or “deathless” but “freedom from death”. } This is the noble quest. 

Mendicants,\marginnote{13.1} before my awakening—when I was still unawakened but intent on awakening—I too, being liable to be reborn, sought what is also liable to be reborn. Myself liable to grow old, fall sick, die, sorrow, and become corrupted, I sought what is also liable to these things. Then it occurred to me: ‘Why do I, being liable to be reborn, grow old, fall sick, sorrow, die, and become corrupted, seek things that have the same nature? Why don’t I seek that which is free of rebirth, old age, sickness, death, sorrow, and corruption, the supreme sanctuary from the yoke, extinguishment?’ 

Some\marginnote{14.1} time later, while still with pristine black hair, blessed with youth, in the prime of life—though my mother and father wished otherwise, weeping with tearful faces—I shaved off my hair and beard, dressed in ocher robes, and went forth from the lay life to homelessness.\footnote{This contrasts with the legend that he slipped away in the dead of night. } 

Once\marginnote{15.1} I had gone forth I set out to discover what is skillful, seeking the supreme state of sublime peace. I approached \textsanskrit{Āḷāra} \textsanskrit{Kālāma} and said to him,\footnote{“To discover what is skillful” (\textit{\textsanskrit{kiṅkusalagavesī}}) might also be translated, “seeking to answer the question, ‘what is good?’.” At this point he did not know the path so he sought an answer from the best teachers of his day. | At \href{https://suttacentral.net/mn102/en/sujato\#25.1}{MN 102:25.1}, the “supreme state of sublime peace” (\textit{santivarapada}) is said to be “liberation without grasping” (\textit{\textsanskrit{anupādāvimokkha}}). | \textsanskrit{Āḷāra} \textsanskrit{Kālāma} was a senior teacher in the contemplative and renunciate tradition of the \textsanskrit{Upaniṣads}. His meditative prowess is praised by his student Pukkusa the Mallian at \href{https://suttacentral.net/dn16/en/sujato\#4.26.1}{DN 16:4.26.1}, but is seen to be inferior to the Buddha’s. At \href{https://suttacentral.net/an3.126/en/sujato\#6.2}{AN 3.126:6.2} an apparent former student of \textsanskrit{Āḷāra} advocates that teachers who teach full understanding of different things are nonetheless leading to the same goal. Thus, in line with \textsanskrit{Upaniṣadic} philosophy, \textsanskrit{Āḷāra} saw the apparent diversity of phenomena as partial manifestations of the immanent cosmic divinity. \textsanskrit{Āḷāra} is usually understood to be named after his people, the \textsanskrit{Kālāmas}. But it is possible we should accept the variant \textsanskrit{Kālāpa}, indicating that he follows the Black Yajurveda. } ‘Reverend \textsanskrit{Kālāma}, I wish to lead the spiritual life in this teaching and training.’\footnote{Śatapatha \textsanskrit{Brāhmaṇa} 11.5.4.1 sets out the initiation (\textit{upanayana}) into “spiritual life” (\textit{brahmacarya}) according to \textsanskrit{Yājñavalkya}. The student says, “I have come for \textit{brahmacarya}” or “Let me be a \textit{\textsanskrit{brahmacāri}}”. The teacher responds by asking, “What is your name” (\textit{ko \textsanskrit{nāmāsīti}}). The same phrase is used in the Buddhist ordination procedure (\textit{\textsanskrit{kiṁnāmosi}}, \href{https://suttacentral.net/pli-tv-kd1/en/sujato\#76.1.15}{Kd 1:76.1.15}). But whereas in Buddhism this is merely a personal identification, for the brahmins this naming signifies a mystical identity with the creator \textsanskrit{Prajāpati}, who is \textit{ka}. } 

\textsanskrit{Āḷāra}\marginnote{15.3} \textsanskrit{Kālāma} replied, ‘Stay, venerable.\footnote{\textsanskrit{Āḷāra} \textsanskrit{Kālāma} calls his student \textit{\textsanskrit{āyasmā}} (Sanskrit \textit{\textsanskrit{āyuṣmant}}), an honorific that in the Śatapatha \textsanskrit{Brāhmaṇa} is reserved for the god Agni (13.8.4.8–9). By this he indicates the student’s divine status. | Compare with the first phrase used by the Buddha to ordain his students, “Come, mendicant” (\textit{ehi bhikkhu}, \href{https://suttacentral.net/pli-tv-kd1/en/sujato\#6.32.3}{Kd 1:6.32.3}). } This teaching is such that a sensible person can soon realize their own tradition with their own insight and live having achieved it.’\footnote{“Their own tradition” (\textit{\textsanskrit{sakaṁ} \textsanskrit{ācariyakaṁ}}), literally “what belongs to their own teacher” (\textit{\textsanskrit{ācariya}}, Sanskrit \textit{\textsanskrit{ācarya}}). In the Brahmanical initiation, having accepted the student, the brahmin takes him by the hand and says, “Agni is your teacher, I am your teacher” (Śatapatha \textsanskrit{Brāhmaṇa} 11.5.4.2). This claims both the universal authority of the divine lineage as well as the personal lineage of that teacher. | “Their own insight” (\textit{\textsanskrit{sayaṁ} \textsanskrit{abhiññā}}); Śatapatha \textsanskrit{Brāhmaṇa} repeatedly emphasizes that the benefits of initiation are for “one who knows this” (\textit{ya \textsanskrit{evaṁ} veda}, 11.5.4.2, etc.). } 

I\marginnote{15.6} quickly memorized that teaching.\footnote{The text does not specify the scripture that he learned, but it must have been Brahmanical, for they were the only scriptures known. Śatapatha \textsanskrit{Brāhmaṇa} 11.5.4.13 instructs the new student to first learn the \textsanskrit{Sāvitrī}, specifying the version in \textsanskrit{Gāyatrī} metre (see also \textsanskrit{Bṛhadāraṇyaka} \textsanskrit{Upaniṣad} 5.14). The Buddha refers to this as the foremost verse (\href{https://suttacentral.net/mn92/en/sujato\#26.2}{MN 92:26.2}), and he even specifies the \textsanskrit{Gāyatrī} version (\href{https://suttacentral.net/snp3.4/en/sujato\#7.3}{Snp 3.4:7.3}). Śatapatha \textsanskrit{Brāhmaṇa} 11.5.7 goes on to encourage the student in daily recitation of the four Vedas and ancillary literature. } As far as lip-recital and verbal repetition went, I spoke the doctrine of knowledge, the elder doctrine. I claimed to know and see, and so did others.\footnote{“I spoke the doctrine of knowledge, the elder doctrine” (\textit{\textsanskrit{ñāṇavādañca} \textsanskrit{vadāmi} \textsanskrit{theravādañca}}). Taken together, the sense is that his personal understanding agreed with tradition. A less literal translation might be, “I spoke with knowledge and authority.” | For \textit{\textsanskrit{ñāṇavāda}}, compare such passages as \href{https://suttacentral.net/snp4.3/en/sujato\#2.4}{Snp 4.3:2.4}, “as you know, so you speak” (\textit{\textsanskrit{yathā} hi \textsanskrit{jāneyya} \textsanskrit{tathā} vadeyya}). In the \textsanskrit{Brāhmaṇas} and \textsanskrit{Upaniṣads}, knowledge refers to the hidden connections between the Vedas and the world which reveal the divinity immanent in all things. | \textit{Thera} (Sanskrit \textit{sthavira}) has the senses “steady, strong” (so commentary’s \textit{\textsanskrit{thirabhāva}}) as well as “elder”, for which see \textsanskrit{Kauṣītaki} \textsanskrit{Brāhmaṇa} 26.2.5, where \textsanskrit{Jātūkarṇya}, a teacher of old, is called \textit{sthavira}. I take it to mean the “lasting doctrine of the ancients”. This is the oldest use of the word \textit{\textsanskrit{theravāda}}, long before it was applied to a Buddhist school. } 

Then\marginnote{15.8} it occurred to me, ‘It is not solely by mere faith that \textsanskrit{Āḷāra} \textsanskrit{Kālāma} declares: “I realize this teaching with my own insight, and live having achieved it.”\footnote{“Faith” (or “trust”, \textit{\textsanskrit{saddhā}}, Sanskrit \textit{\textsanskrit{śraddhā}}) was regarded as a quality of the “heart” through which one gained remuneration in the form of fees for priestly services (Rig Veda 10.151.4, \textsanskrit{Bṛhadāraṇyaka} \textsanskrit{Upaniṣad} 3.9.21). It is like the “trust” that one would place in a bank: an investment is made in expectation of a reward. | For the syntax here, compare with \textit{\textsanskrit{kevalaṁ} \textsanskrit{vassagaṇanamattena}} at \href{https://suttacentral.net/an7.43/en/sujato\#5.6}{AN 7.43:5.6}. } Surely he meditates knowing and seeing this teaching.’ 

So\marginnote{15.11} I approached \textsanskrit{Āḷāra} \textsanskrit{Kālāma} and said to him, ‘Reverend \textsanskrit{Kālāma}, to what extent do you say you’ve realized this teaching with your own insight?’ When I said this, he declared the dimension of nothingness.\footnote{This is said to be the best of the four perceptions at (\href{https://suttacentral.net/an10.29/en/sujato\#18.1}{AN 10.29:18.1}). Buddhist texts portray this as a favorite meditation among brahmin contemplatives (eg. \href{https://suttacentral.net/snp5.15/en/sujato\#2.1}{Snp 5.15:2.1}, \href{https://suttacentral.net/snp5.7/en/sujato\#2.1}{Snp 5.7:2.1}). While one might expect that the brahmins would identify their goal in a positive sense, this is not always the case. A text of the Black Yajurveda, \textsanskrit{Kaṭha} \textsanskrit{Upaniṣad} 1.3.11, says that beyond the great Soul, beyond the unmanifest, beyond the Person, is nothing: and that is the ultimate goal (\textit{\textsanskrit{puruśān} na paraṃ kiṃcit}). See too \textsanskrit{Chāndogya} \textsanskrit{Upaniṣad} 6.12.1, which illustrate how divinity is too subtle for normal perception. The teacher, breaking the kernel of a seed, asked the student, ““What do you see there?’ ‘Nothing’” (\textit{kimatra \textsanskrit{paśyasīti}, na \textsanskrit{kiṁcana}}). Compare the description of this state at \href{https://suttacentral.net/snp5.15/en/sujato\#2.4}{Snp 5.15:2.4}, “one sees nothing at all” (\textit{natthi \textsanskrit{kiñcīti} passato}). This is also similar to the teaching of Uddaka (\href{https://suttacentral.net/mn26/en/sujato\#16.1}{MN 26:16.1}). } 

Then\marginnote{15.14} it occurred to me, ‘It’s not just \textsanskrit{Āḷāra} \textsanskrit{Kālāma} who has faith,\footnote{The Buddha called these the five “faculties” (\href{https://suttacentral.net/sn48.10/en/sujato}{SN 48.10}) or “powers” (\href{https://suttacentral.net/an5.14/en/sujato}{AN 5.14}). All except “immersion” (\textit{\textsanskrit{samādhi}}) are frequently mentioned in pre-Buddhist texts, but not as a set of five. } energy, mindfulness, immersion, and wisdom; I too have these things. Why don’t I make an effort to realize the same teaching that \textsanskrit{Āḷāra} \textsanskrit{Kālāma} says he has realized with his own insight?’ I quickly realized that teaching with my own insight, and lived having achieved it.\footnote{This phrasing echoes part of the description of arahantship (\href{https://suttacentral.net/mn85/en/sujato\#51.12}{MN 85:51.12}). } 

So\marginnote{15.22} I approached \textsanskrit{Āḷāra} \textsanskrit{Kālāma} and said to him, ‘Reverend \textsanskrit{Kālāma}, is it up to this point that you realized this teaching with your own insight, and declare having achieved it?’\footnote{Having achieved that state, \textsanskrit{Āḷāra} \textsanskrit{Kālāma} “declares” (\textit{pavedeti}) it as the teacher, while the Bodhisatta “dwells” in it (\textit{viharati}). } 

‘I\marginnote{15.24} have, reverend.’ 

‘I\marginnote{15.25} too, reverend, have realized this teaching with my own insight up to this point, and live having achieved it.’ 

‘We\marginnote{15.26} are fortunate, reverend, so very fortunate to see a venerable such as yourself as one of our spiritual companions! So the teaching that I’ve realized with my own insight, and declare having achieved it, you’ve realized with your own insight, and dwell having achieved it. The teaching that you’ve realized with your own insight, and dwell having achieved it, I’ve realized with my own insight, and declare having achieved it. So the teaching that I know, you know, and the teaching that you know, I know.\footnote{Almost exactly the same words were spoken by \textsanskrit{Pokkharasāti} to his student \textsanskrit{Ambaṭṭha} (\href{https://suttacentral.net/dn3/en/sujato\#1.3.2}{DN 3:1.3.2}). This connects \textsanskrit{Pokkharasāti} with \textsanskrit{Āḷāra} \textsanskrit{Kālāma} and Uddaka \textsanskrit{Rāmaputta}, suggesting that the uplifting of a talented student in this way was a regular practice of wise brahmins. } I am like you and you are like me. Come now, reverend! We should both lead this community together.’\footnote{This invitation shows his grace and humility. } 

And\marginnote{15.33} that is how my tutor \textsanskrit{Āḷāra} \textsanskrit{Kālāma} placed me, his pupil, on the same position as him, and honored me with lofty praise. 

Then\marginnote{15.34} it occurred to me, ‘This teaching doesn’t lead to disillusionment, dispassion, cessation, peace, insight, awakening, and extinguishment. It only leads as far as rebirth in the dimension of nothingness.’\footnote{It does not fulfill the requirements of the “noble quest”. | Compare the pre-Buddhist meditation practice of Govinda (\href{https://suttacentral.net/dn19/en/sujato\#50.9}{DN 19:50.9}), \textsanskrit{Mātaṅga} (\href{https://suttacentral.net/snp1.7/en/sujato\#27.6}{Snp 1.7:27.6}), and Maghadeva (\href{https://suttacentral.net/mn83/en/sujato\#21.7}{MN 83:21.7}), which in all cases “leads to rebirth in the \textsanskrit{Brahmā} realm” (\textit{\textsanskrit{brahmalokūpapattiyā}}). \textsanskrit{Kaṭha} \textsanskrit{Upaniṣad} 1.3.16, concluding the ancient portion of this text, says that the wise man attains glory in the world of divinity (\textit{\textsanskrit{medhāvī} brahmaloke \textsanskrit{mahīyate}}). } Realizing that this teaching was inadequate, I left disappointed. 

I\marginnote{16.1} set out to discover what is skillful, seeking the supreme state of sublime peace. I approached Uddaka son of \textsanskrit{Rāma} and said to him,\footnote{Uddaka’s teachings are cited at \href{https://suttacentral.net/sn35.103/en/sujato\#1.1}{SN 35.103:1.1} and \href{https://suttacentral.net/dn29/en/sujato\#16.13}{DN 29:16.13}. In the former, his obscure verse assumes the identity of “this” (the impersonal cosmic divinity) with himself, while the latter is a distorted reference to \textsanskrit{Bṛhadāraṇyaka} \textsanskrit{Upaniṣad} 1.4.7. Further, \href{https://suttacentral.net/an10.29/en/sujato\#20.1}{AN 10.29:20.1} implies that he taught the “ultimate purity of the spirit”. While “spirit” (\textit{yakkha}, Sanskrit \textit{\textsanskrit{yakṣa}}) is normally a worldly deity, it is identified with the ultimate Brahman at \textsanskrit{Bṛhadāraṇyaka} \textsanskrit{Upaniṣad} 5.4.1 and Kena \textsanskrit{Upaniṣad} 4.1. Uddaka is apparently the “ascetic \textsanskrit{Rāmaputta}” who was revered by the king, criticized by some brahmins, but defended by Todeyya, a leading brahmin of Kosala often mentioned along with \textsanskrit{Pokkharasāti} (\href{https://suttacentral.net/an4.187/en/sujato\#4.2}{AN 4.187:4.2}). } ‘Reverend, I wish to lead the spiritual life in this teaching and training.’\footnote{In contrast with \textsanskrit{Āḷāra} \textsanskrit{Kālāma}, here the name is omitted. } 

Uddaka\marginnote{16.3} replied, ‘Stay, venerable. This teaching is such that a sensible person can soon realize their own tradition with their own insight and live having achieved it.’ 

I\marginnote{16.6} quickly memorized that teaching. As far as lip-recital and verbal repetition went, I spoke the doctrine of knowledge, the elder doctrine. I claimed to know and see, and so did others. 

Then\marginnote{16.8} it occurred to me, ‘It is not solely by mere faith that \textsanskrit{Rāma} declared: “I realize this teaching with my own insight, and live having achieved it.”\footnote{The text shifts from \textsanskrit{Rāmaputta} to just \textsanskrit{Rāma}, while at the same time shifting to the past tense. Uddaka was thus the (spiritual and/or biological) “son of \textsanskrit{Rāma}”. This detail is preserved reliably in the different versions of the text. It seems that \textsanskrit{Rāmaputta} had not personally attained the meditation he was teaching. This would also explain why, just a little below, he invites the Bodhisatta to lead the community  rather than to share leadership like \textsanskrit{Āḷāra} \textsanskrit{Kālāma} (\href{https://suttacentral.net/mn36/en/sujato\#16.32}{MN 36:16.32}). \href{https://suttacentral.net/sn35.103/en/sujato\#1.6}{SN 35.103:1.6} shows that he did, however, make claims to being spiritually attained. } Surely he meditated knowing and seeing this teaching.’ 

So\marginnote{16.11} I approached Uddaka son of \textsanskrit{Rāma} and said to him, ‘Reverend, to what extent did \textsanskrit{Rāma} say he’d realized this teaching with his own insight?’ 

When\marginnote{16.13} I said this, Uddaka son of \textsanskrit{Rāma} declared the dimension of neither perception nor non-perception.\footnote{In \textsanskrit{Bṛhadāraṇyaka} \textsanskrit{Upaniṣad} 2.4.12 and 4.5.13, the sage \textsanskrit{Yājñavalkya} says that the true Self is a sheer mass of “consciousness” (\textit{\textsanskrit{vijñāna}}, Pali \textit{\textsanskrit{viññāṇa}}), which is “great, endless, infinite reality”. After realizing this, he says, there is no “perception” (\textit{\textsanskrit{saṁjñā}}, Pali \textit{\textsanskrit{saññā}}). This passage seems to have sparked the conversation at \href{https://suttacentral.net/dn9/en/sujato\#6.4}{DN 9:6.4}, where the Buddha goes on to speak of meditative training to refine perception. Notably, there he mentions all the \textit{\textsanskrit{jhānas}} and formless attainments \emph{except} the dimension of neither perception nor non-perception, which is beyond the topic of perception under discussion. This meditative state may be related to the state with “no perception” spoken of by \textsanskrit{Yājñavalkya}. However, it is not possible to literally identify such subtle meditative states on such slender evidence. The point is simply that the contemplative brahmins of the \textsanskrit{Upaniṣadic} tradition of \textsanskrit{Yājñavalkya} did indeed describe their highest state in terms of perception. } 

Then\marginnote{16.14} it occurred to me, ‘It’s not just \textsanskrit{Rāma} who had faith, energy, mindfulness, immersion, and wisdom; I too have these things. Why don’t I make an effort to realize the same teaching that \textsanskrit{Rāma} said he had realized with his own insight?’ I quickly realized that teaching with my own insight, and lived having achieved it. 

So\marginnote{16.22} I approached Uddaka son of \textsanskrit{Rāma} and said to him, ‘Reverend, had \textsanskrit{Rāma} realized this teaching with his own insight up to this point, and declared having achieved it?’ 

‘He\marginnote{16.24} had, reverend.’ 

‘I\marginnote{16.25} too have realized this teaching with my own insight up to this point, and live having achieved it.’ 

‘We\marginnote{16.26} are fortunate, reverend, so very fortunate to see a venerable such as yourself as one of our spiritual companions! So the teaching that \textsanskrit{Rāma} had realized with his own insight, and declared having achieved it, you’ve realized with your own insight, and dwell having achieved it. The teaching that you’ve realized with your own insight, and dwell having achieved it, \textsanskrit{Rāma} had realized with his own insight, and declared having achieved it. So the teaching that \textsanskrit{Rāma} directly knew, you know, and the teaching you know, \textsanskrit{Rāma} directly knew. \textsanskrit{Rāma} was like you and you are like \textsanskrit{Rāma}. Come now, reverend! You should lead this community.’ 

And\marginnote{16.33} that is how my spiritual companion Uddaka son of \textsanskrit{Rāma} placed me in the position of a tutor and honored me with lofty praise.\footnote{An \textit{\textsanskrit{ācariya}} is special teacher who has a close, ongoing relationship with a student, signified by the fact that the student of an \textit{\textsanskrit{ācāriya}} is called an \textit{\textsanskrit{antevāsī}}, literally “one who lives within” the teacher’s home. This passage confirms that this usage carries over from the Brahmanical tradition, where it is a common term in the same sense. Thus \textsanskrit{Bṛhadāraṇyaka} \textsanskrit{Upaniṣad} 4.1.2 speaks of “mother, father, and teacher”, while \textsanskrit{Chāndogya} \textsanskrit{Upaniṣad} 2.23.1 enjoins a student to “live with the teacher until death”, and at 4.9.1 ff. we see the journey of a student to their teacher’s house, where he is looked after by the teacher and his wife. The same sense prevails throughout the suttas and Vinaya, so I translate as “tutor” to identify a close personal teacher. } 

Then\marginnote{16.34} it occurred to me, ‘This teaching doesn’t lead to disillusionment, dispassion, cessation, peace, insight, awakening, and extinguishment. It only leads as far as rebirth in the dimension of neither perception nor non-perception.’ Realizing that this teaching was inadequate, I left disappointed. 

I\marginnote{17.1} set out to discover what is skillful, seeking the supreme state of sublime peace. Traveling stage by stage in the Magadhan lands, I arrived at \textsanskrit{Senānigama} in \textsanskrit{Uruvelā}.\footnote{In modern times this is the pilgrimage site Bodhgaya in the Gaya district of Bihar. It lies by a wide river that is dry in summer but abundant in the rains. Today it is named Lilajan and in Pali \textsanskrit{Nerañjarā}, although that name is not mentioned in this passage. \textsanskrit{Uruvelā}, which aptly means “Broadbanks”, was a locality centered around the market town of \textsanskrit{Senānigama} (“Marshalltown”). There was also a sizable hermitage nearby with matted-hair Brahmanical ascetics (\textit{\textsanskrit{jaṭilā}}) who ritually bathed in the water (\href{https://suttacentral.net/pli-tv-kd1/en/sujato\#20.15.1}{Kd 1:20.15.1}). After teaching the five monks in Varanasi, the Buddha returned here to teach them, but there is no record of him visiting subsequently. } There I saw a delightful park, a lovely grove with a flowing river that was clean and charming, with smooth banks. And nearby was a village to resort to for alms.\footnote{The Buddha’s rejection of sensual pleasures does not preclude his appreciation of nature’s beauty. | The local alms villages in \textsanskrit{Uruvelā} are named in \textsanskrit{Mahāvastu} 2.207 as Praskandaka, \textsanskrit{Balākalpa}, \textsanskrit{Ujjaṁgala}, and \textsanskrit{Jaṁgala}. } 

Then\marginnote{17.3} it occurred to me, ‘This park is truly delightful, a lovely grove with a flowing river that’s clean and charming, with smooth banks. And nearby there’s a village to resort to for alms. This is good enough for striving for a gentleman wanting to strive.’\footnote{\textit{\textsanskrit{Padhāna}} means to “strive”. It is frequently used in the sense of applying oneself to meditation, but here it leads into the painful “striving” of fervent austerity (\href{https://suttacentral.net/mn36/en/sujato\#20.7}{MN 36:20.7}). } So I sat down right there, thinking, ‘This is good enough for striving.’ 

And\marginnote{18.1} so, being myself liable to be reborn, understanding the drawbacks in being liable to be reborn, I sought that which is free of rebirth, the supreme sanctuary from the yoke, extinguishment—and I found it. Being myself liable to grow old, fall sick, die, sorrow, and become corrupted, understanding the drawbacks in these things, I sought that which is free of old age, sickness, death, sorrow, and corruption, the supreme sanctuary from the yoke, extinguishment—and I found it.\footnote{While this skips directly from his arrival in \textsanskrit{Uruvelā} to his awakening, other texts describe his practice of fervent austerities here, which must have taken a considerable time (\href{https://suttacentral.net/mn36/en/sujato\#16.7}{MN 36:16.7}, \href{https://suttacentral.net/mn85/en/sujato\#15.1}{MN 85:15.1}). See note on \href{https://suttacentral.net/mn26/en/sujato\#18.1}{MN 26:18.1}. } 

Knowledge\marginnote{18.2} and vision arose in me: ‘My freedom is unshakable; this is my last rebirth; now there’ll be no more future lives.’\footnote{This is the realization of arahantship. Elsewhere the Buddha goes into more detail as to the process of meditation that led to this point. } 

Then\marginnote{19.1} it occurred to me, ‘This principle I have discovered is deep, hard to see, hard to understand, peaceful, sublime, beyond the scope of logic, subtle, comprehensible to the astute. But people like clinging, they love it and enjoy it. It’s hard for them to see this topic; that is, specific conditionality, dependent origination.\footnote{The Buddha identifies the two most difficult topics in his philosophy: dependent origination and Nibbana. } It’s also hard for them to see this topic; that is, the stilling of all activities, the letting go of all attachments, the ending of craving, fading away, cessation, extinguishment. And if I were to teach the Dhamma, others might not understand me, which would be wearying and troublesome for me.’\footnote{Up to this point, the Bodhisatta has been solely concerned with finding the answer to his quest, and only now does he think of sharing it with others. The idea that he had made an aspiration for Buddhahood in the long ago past out of a desire to help all sentient beings is not supported in early texts. } 

And\marginnote{19.7} then these verses, which were neither supernaturally inspired, nor learned before in the past, occurred to me: 

\begin{verse}%
‘I’ve\marginnote{19.8} struggled hard to realize this, \\
enough with trying to explain it! \\
Those mired in greed and hate \\
can’t really understand this teaching. 

It\marginnote{19.12} goes against the stream, subtle, \\
deep, obscure, and very fine. \\
Those besotted by greed cannot see, \\
for they’re shrouded in a mass of darkness.’ 

%
\end{verse}

So,\marginnote{19.16} as I reflected like this, my mind inclined to remaining passive, not to teaching the Dhamma. 

Then\marginnote{20.1} the divinity Sahampati, knowing my train of thought, thought,\footnote{While Sahampati features prominently in the suttas, no deity of that name is found in early Brahmanical texts. He seems, however, to be the Buddhist version of \textsanskrit{Brahmā} \textsanskrit{Svayambhū}, the “Self-born Divinity”, to whom \textsanskrit{Yājñavalkya} traces the authority of his teaching lineage (\textsanskrit{Bṛhadāraṇyaka} \textsanskrit{Upaniṣad} 2.6.3, 4.5.3, 6.5.4). \textsanskrit{Yājñavalkya} identifies this “self-born” with the sun (Śatapatha \textsanskrit{Brāhmaṇa} 1.9.3.16; see also \textsanskrit{Maitrāyaṇī} \textsanskrit{Saṁhitā} 4.6.6) and with \textsanskrit{Prajāpati} (13.5.3.1). Explaining the origin of the soma rite of the “All-Sacrifice”, he says that \textsanskrit{Brahmā} \textsanskrit{Svayambhū}, while performing fervent austerities, decided to offer his self to all creatures and all creatures to his self, thus establishing lordship (\textit{adhipatya}) over all creatures (Śatapatha \textsanskrit{Brāhmaṇa} 13.7.1.1). The root \textit{pati} (“lord”) here establishes a linguistic link with \textit{sahampati}. Further, in the earliest use of \textit{\textsanskrit{svayambhū}} at Rig Veda 10.83.4, the “self-born” deity Manyu “prevails” (\textit{sahuriḥ \textsanskrit{sahāvān}}) in battle. Thus we can identify Sahampati, the “Lord Who Prevails”, with the highest divinity recognized by \textsanskrit{Yājñavalkya}, who originated as victor in battle, whose physical manifestation is the sun, whose spiritual function is to imbue all creatures with divinity, and who serves as the ultimate source of authority. Meanwhile, the epithet \textit{\textsanskrit{sayambhū}} was taken for the Buddha (\href{https://suttacentral.net/pli-tv-kd11/en/sujato\#35.1.37}{Kd 11:35.1.37}). } ‘Alas! The world will be lost, the world will perish! For the mind of the Realized One, the perfected one, the fully awakened Buddha, inclines to remaining passive, not to teaching the Dhamma.’\footnote{Having surpassed the greatest of the Brahmanical contemplatives, the Buddha’s achievement is recognized by the chief Brahmanical divinity. } 

Then,\marginnote{20.3} as easily as a strong person would extend or contract their arm, he vanished from the realm of divinity and reappeared in front of me. He arranged his robe over one shoulder, raised his joined palms toward me, and said, ‘Sir, let the Blessed One teach the Dhamma! Let the Holy One teach the Dhamma! There are beings with little dust in their eyes. They’re in decline because they haven’t heard the teaching. There will be those who understand the teaching!’\footnote{Sahampati is the one who recognizes the potential in all beings, since it was he who imbued them with divinity. He did this to ensure dominance, but now his motive has transformed into compassion. } 

That’s\marginnote{20.8} what the divinity Sahampati said. Then he went on to say: 

\begin{verse}%
‘Among\marginnote{20.10} the Magadhans there appeared in the past \\
an impure teaching thought up by the stained.\footnote{The “impure teaching” was either that of \textsanskrit{Āḷāra} \textsanskrit{Kālāma} and Uddaka \textsanskrit{Rāmaputta}, or the Jain-like practices of fervent austerity that the Bodhisatta had followed, although not depicted in this sutta. } \\
Fling open the door to freedom from death!\footnote{“Freedom from death” (\textit{amata}, Sanskrit \textit{\textsanskrit{amṛta}}) is yet another Brahmanical term transformed by the Buddha. Its Vedic meaning was “immortality”, and hence the “ambrosia” of soma through which immortality was gained. For the Buddha, as seen in the “noble quest” itself, it was the freedom from the cycles of birth, aging, and death. } \\
Let them hear the teaching \\>the immaculate one discovered. 

Standing\marginnote{20.14} high on a rocky mountain, \\
you can see the people all around. \\
In just the same way, All-seer, so intelligent, \\
having ascended the Temple of Truth, \\
rid of sorrow, look upon the people \\
swamped with sorrow,  \\>oppressed by rebirth and old age. 

Rise,\marginnote{20.20} hero! Victor in battle, leader of the caravan,\footnote{Here we see a verbal echo of Sahampati’s martial origins. } \\
wander the world free of debt. \\
Let the Blessed One teach the Dhamma! \\
There will be those who understand!’ 

%
\end{verse}

Then,\marginnote{21.1} understanding the Divinity’s invitation, I surveyed the world with the eye of a Buddha, because of my compassion for sentient beings. And I saw sentient beings with little dust in their eyes, and some with much dust in their eyes; with keen faculties and with weak faculties, with good qualities and with bad qualities, easy to teach and hard to teach. And some of them lived seeing the danger in the fault to do with the next world, while others did not.\footnote{Here we see the sense of \textit{indriya} as “spiritual potential”. Originally it was the state of excited energy produced by soma that allowed Indra to manifest his full power and overcome his foes. It implicitly links back to the five spiritual qualities that the Buddha attributed to his teachers, which he later formalized as “faculties”. Here as there, the idea creates a connection between beings, a recognition of shared qualities that, when properly nurtured, can mature into awakening. } It’s like a pool with blue water lilies, or pink or white lotuses. Some of them sprout and grow in the water without rising above it, thriving underwater. Some of them sprout and grow in the water reaching the water’s surface. And some of them sprout and grow in the water but rise up above the water and stand with no water clinging to them. In the same way, I saw sentient beings with little dust in their eyes, and some with much dust in their eyes. 

Then\marginnote{21.5} I replied in verse to the divinity Sahampati: 

\begin{verse}%
‘Flung\marginnote{21.6} open are the doors to freedom from death! \\
Let those with ears to hear commit to faith.\footnote{\textit{\textsanskrit{Pamuñcantu} \textsanskrit{saddhaṁ}} has long troubled translators, as the basic sense of \textit{\textsanskrit{pamuñcantu}} is “release”. Sanskrit variants include \textit{pramodantu} (“celebrate”) or \textit{\textsanskrit{praṇudantu} \textsanskrit{kāṅkṣāḥ}} (“dispel doubts”). I think it is a poetic variant of \textit{\textsanskrit{adhimuñcantu}}, to “decide” or “commit” to faith. Pali commonly uses synonymous verbs to reinforce the sense of the noun. In \href{https://suttacentral.net/snp5.19/en/sujato}{Snp 5.19}, \textit{muttasaddho}, \textit{\textsanskrit{pamuñcassu} \textsanskrit{saddhaṁ}}, and \textit{\textsanskrit{adhimuttacittaṁ}} are all used in this sense. } \\
Thinking it would be troublesome, Divinity, \\>I did not teach \\
the sophisticated, sublime Dhamma among humans.’ 

%
\end{verse}

Then\marginnote{21.10} the divinity Sahampati, knowing that his request for me to teach the Dhamma had been granted, bowed and respectfully circled me, keeping me on his right, before vanishing right there. 

Then\marginnote{22.1} I thought, ‘Who should I teach first of all? Who will quickly understand this teaching?’ 

Then\marginnote{22.4} it occurred to me, ‘That \textsanskrit{Āḷāra} \textsanskrit{Kālāma} is astute, competent, clever, and has long had little dust in his eyes. Why don’t I teach him first of all? He’ll quickly understand the teaching.’ 

But\marginnote{22.8} a deity came to me and said, ‘Sir, \textsanskrit{Āḷāra} \textsanskrit{Kālāma} passed away seven days ago.’ 

And\marginnote{22.10} knowledge and vision arose in me, ‘\textsanskrit{Āḷāra} \textsanskrit{Kālāma} passed away seven days ago.’ 

I\marginnote{22.12} thought, ‘This is a great loss for \textsanskrit{Āḷāra} \textsanskrit{Kālāma}. If he had heard the teaching, he would have understood it quickly.’ 

Then\marginnote{23.1} I thought, ‘Who should I teach first of all? Who will quickly understand this teaching?’ 

Then\marginnote{23.4} it occurred to me, ‘That Uddaka son of \textsanskrit{Rāma} is astute, competent, clever, and has long had little dust in his eyes. Why don’t I teach him first of all? He’ll quickly understand the teaching.’ 

But\marginnote{23.8} a deity came to me and said, ‘Sir, Uddaka son of \textsanskrit{Rāma} passed away just last night.’ 

And\marginnote{23.10} knowledge and vision arose in me, ‘Uddaka son of \textsanskrit{Rāma} passed away just last night.’ 

I\marginnote{23.12} thought, ‘This is a great loss for Uddaka. If he had heard the teaching, he would have understood it quickly.’\footnote{While this passage is clearly heightened for dramatic effect, it is not surprising that they have passed away. It has been several years since he studied with these teachers, and they were probably elderly at the time, as perhaps hinted in their eagerness to appoint Siddhattha as heirs. } 

Then\marginnote{24.1} I thought, ‘Who should I teach first of all? Who will quickly understand this teaching?’ 

Then\marginnote{24.4} it occurred to me, ‘The group of five mendicants were very helpful to me. They looked after me during my time of resolute striving.\footnote{The five mendicants appear abruptly here and the reason for their behavior is only clear in light of the fuller accounts (\href{https://suttacentral.net/mn36/en/sujato\#33.4}{MN 36:33.4}, \href{https://suttacentral.net/mn85/en/sujato\#33.4}{MN 85:33.4}, \href{https://suttacentral.net/mn100/en/sujato\#30.5}{MN 100:30.5}). The Buddha’s praise for them is more muted than for the Brahmanical teachers, showing how he valued advanced meditation over austerities. | Note that he calls them “mendicants”, a term typically reserved for Buddhist renunciates. For a similar usage, see \href{https://suttacentral.net/mn140/en/sujato\#3.4}{MN 140:3.4}. } Why don’t I teach them first of all?’ 

Then\marginnote{24.7} I thought, ‘Where are the group of five mendicants staying these days?’ With clairvoyance that is purified and superhuman I saw that the group of five mendicants were staying near Varanasi, in the deer park at Isipatana. So, when I had stayed in \textsanskrit{Uruvelā} as long as I pleased, I set out for Varanasi.\footnote{Several suttas relate further details in this period (eg. \href{https://suttacentral.net/sn47.43/en/sujato}{SN 47.43}, \href{https://suttacentral.net/ud1.3/en/sujato}{Ud 1.3}–4, etc.) } 

While\marginnote{25.1} I was traveling along the road between \textsanskrit{Gayā} and Bodhgaya, the \textsanskrit{Ājīvaka} ascetic Upaka saw me\footnote{For the \textsanskrit{Ājīvakas}, see \href{https://suttacentral.net/mn5/en/sujato\#31.7}{MN 5:31.7} and \href{https://suttacentral.net/dn2/en/sujato\#20.2}{DN 2:20.2}. Upaka is met only here; his name means “nearly there”. | The place of awakening is referred to as \textit{bodhi}. These places are about twenty kilometers apart. } and said, ‘Reverend, your faculties are so very clear, and your complexion is pure and bright.\footnote{Here “faculties” refers to visible features such as bright eyes. Compare \textsanskrit{Chāndogya} \textsanskrit{Upaniṣad} 4.14.2, where a students face is seen as bright and glowing due to their recent insights. } In whose name have you gone forth, reverend? Who is your Teacher? Whose teaching do you believe in?’ 

I\marginnote{25.5} replied to Upaka in verse: 

\begin{verse}%
‘I\marginnote{25.6} am the champion, the knower of all, \\
unsullied in the midst of all things. \\
I’ve given up all, freed through the ending of craving. \\
Since I know for myself, whose follower should I be? 

I\marginnote{25.10} have no tutor. \\
There is no-one like me. \\
In the world with its gods, \\
I have no rival. 

For\marginnote{25.14} in this world, I am the perfected one; \\
I am the supreme Teacher. \\
I alone am fully awakened, \\
cooled, quenched. 

I\marginnote{25.18} am going to the city of \textsanskrit{Kāsi} \\
to roll forth the Wheel of Dhamma. \\
In this world that is so blind, \\
I’ll beat the drum of freedom from death!’\footnote{\textit{\textsanskrit{Āhañchaṁ}} is first person future singular of \textit{\textsanskrit{āhanati}}, “one beats”. } 

%
\end{verse}

‘According\marginnote{25.22} to what you claim, reverend, you ought to be the Infinite Victor.’\footnote{“Infinite Victor” (\textit{anantajina}) is unknown elsewhere and may be an \textsanskrit{Ājīvaka} term. It relates to the epithet \textit{jina} of the teacher \textsanskrit{Mahāvīra}, after which his followers the Jains were named. \textsanskrit{Mahāvīra} and the \textsanskrit{Ājīvaka} founder \textsanskrit{Gosāla} practiced together for six years, so it comes as no surprise that they shared terminology. } 

\begin{verse}%
‘The\marginnote{25.23} victors are those who, like me, \\
have reached the ending of defilements. \\
I have conquered bad qualities, Upaka—\\
that’s why I’m a victor.’ 

%
\end{verse}

When\marginnote{25.27} I had spoken, Upaka said: ‘If you say so, reverend.’ Shaking his head, he took a wrong turn and left. 

Traveling\marginnote{26.1} stage by stage, I arrived at Varanasi, and went to see the group of five mendicants in the deer park at Isipatana. The group of five mendicants saw me coming off in the distance and stopped each other, saying,\footnote{For “they stopped each other” (\textit{\textsanskrit{aññamaññaṁ} \textsanskrit{saṇṭhapesuṁ}}), see \href{https://suttacentral.net/an10.93/en/sujato\#2.4}{AN 10.93:2.4}. } ‘Here comes the ascetic Gotama. He’s so indulgent; he strayed from the struggle and returned to indulgence. We shouldn’t bow to him or rise for him or receive his bowl and robe. But we can set out a seat; he can sit if he likes.’ Yet as I drew closer, the group of five mendicants were unable to stop themselves as they had agreed. Some came out to greet me and receive my bowl and robe, some spread out a seat, while others set out water for washing my feet. But they still addressed me by name and as ‘reverend’.\footnote{These ways of address are suitable for a senior speaking to a junior (\href{https://suttacentral.net/dn16/en/sujato\#6.2.2}{DN 16:6.2.2}). } 

So\marginnote{27.1} I said to them, ‘Mendicants, don’t address me by name and as “reverend”. The Realized One is perfected, a fully awakened Buddha. Listen up, mendicants: I have achieved freedom from death! I shall instruct you, I will teach you the Dhamma. By practicing as instructed you will soon realize the supreme end of the spiritual path in this very life. You will live having achieved with your own insight the goal for which gentlemen rightly go forth from the lay life to homelessness.’ 

But\marginnote{27.6} they said to me, ‘Reverend Gotama, even by that conduct, that practice, that grueling work you did not achieve any superhuman distinction in knowledge and vision worthy of the noble ones. How could you have achieved such a state now that you’ve become indulgent, strayed from the struggle and returned to indulgence?’\footnote{Where “that practice” appears in \href{https://suttacentral.net/mn12/en/sujato\#56.1}{MN 12:56.1} and \href{https://suttacentral.net/mn85/en/sujato\#51.7}{MN 85:51.7}, it refers to the austere practices that have just been described. Taken together with the abrupt skip at \href{https://suttacentral.net/mn26/en/sujato\#18.1}{MN 26:18.1} above, and the equally abrupt mention of the five mendicants at \href{https://suttacentral.net/mn26/en/sujato\#24.5}{MN 26:24.5}, it seems likely the passage on the austerities has been removed in order to focus on the Brahmanical context. } 

So\marginnote{27.8} I said to them, ‘The Realized One has not become indulgent, strayed from the struggle and returned to indulgence. The Realized One is perfected, a fully awakened Buddha. Listen up, mendicants: I have achieved freedom from death! I shall instruct you, I will teach you the Dhamma. By practicing as instructed you will soon realize the supreme end of the spiritual path in this very life.’ 

But\marginnote{27.13} for a second time they said to me, ‘Reverend Gotama … you’ve returned to indulgence.’ 

So\marginnote{27.15} for a second time I said to them, ‘The Realized One has not become indulgent …’ 

But\marginnote{27.18} for a third time they said to me, ‘Reverend Gotama, even by that conduct, that practice, that grueling work you did not achieve any superhuman distinction in knowledge and vision worthy of the noble ones. How could you have achieved such a state now that you’ve become indulgent, strayed from the struggle and returned to indulgence?’ 

So\marginnote{28.1} I said to them, ‘Mendicants, have you ever known me to speak like this before?’ 

‘No\marginnote{28.3} sir, we have not.’\footnote{Here they adopt the more respectful term of address \textit{bhante} (“sir”). } 

‘The\marginnote{28.4} Realized One is perfected, a fully awakened Buddha. Listen up, mendicants: I have achieved freedom from death! I shall instruct you, I will teach you the Dhamma. By practicing as instructed you will soon realize the supreme end of the spiritual path in this very life. You will live having achieved with your own insight the goal for which gentlemen rightly go forth from the lay life to homelessness.’ 

I\marginnote{29.1} was able to persuade the group of five mendicants.\footnote{Again the narrative skips. Here the Buddha taught the “Rolling Forth of the Wheel of the Dhamma” (\href{https://suttacentral.net/sn56.11/en/sujato}{SN 56.11}). The full sequence of events is related in \href{https://suttacentral.net/pli-tv-kd1/en/sujato\#6.16.9}{Kd 1:6.16.9}. } Then sometimes I advised two mendicants, while the other three went for alms. Then those three would feed all six of us with what they brought back. Sometimes I advised three mendicants, while the other two went for alms. Then those two would feed all six of us with what they brought back. 

As\marginnote{30.1} the group of five mendicants were being advised and instructed by me like this, being themselves liable to be reborn, understanding the drawbacks in being liable to be reborn, they sought that which is free of rebirth, the supreme sanctuary from the yoke, extinguishment—and they found it. Being themselves liable to grow old, fall sick, die, sorrow, and become corrupted, understanding the drawbacks in these things, they sought that which is free of old age, sickness, death, sorrow, and corruption, the supreme sanctuary from the yoke, extinguishment—and they found it. Knowledge and vision arose in them: ‘Our freedom is unshakable; this is our last rebirth; now there’ll be no more future lives.’\footnote{This occurred with the teaching of the “Discourse on the Characteristic of Not Self” (\href{https://suttacentral.net/sn22.59/en/sujato}{SN 22.59}). } 

Mendicants,\marginnote{31.1} there are these five kinds of sensual stimulation.\footnote{The Buddha rather abruptly returns to a direct teaching for the mendicants in Rammaka’s hermitage. The topic harks back to the simile of the sappy log, absent from here, but which in \href{https://suttacentral.net/mn36/en/sujato\#17.1}{MN 36:17.1} is introduced as the Bodhisatta undertook striving at \textsanskrit{Uruvelā}. } What five? Sights known by the eye, which are likable, desirable, agreeable, pleasant, sensual, and arousing. Sounds known by the ear … Smells known by the nose … Tastes known by the tongue … Touches known by the body, which are likable, desirable, agreeable, pleasant, sensual, and arousing. These are the five kinds of sensual stimulation. 

There\marginnote{32.1} are ascetics and brahmins who enjoy these five kinds of sensual stimulation tied, infatuated, attached, blind to the drawbacks, and not understanding the escape. You should understand that they\footnote{The Pali word \textit{\textsanskrit{paribhuñjati}} means “to have something satisfying” as well as “to take pleasure in”, much like the English word “enjoys”. } have met with calamity and disaster, and the Wicked One can do with them what he wants. 

Suppose\marginnote{32.3} a deer in the wilderness was lying caught on a pile of snares.\footnote{This image and the teaching that follows links this sutta with the previous. } You’d know that it has met with calamity and disaster, and the hunter can do with them what he wants. And when the hunter comes, it cannot flee where it wants. 

In\marginnote{32.7} the same way, there are ascetics and brahmins who enjoy these five kinds of sensual stimulation tied, infatuated, attached, blind to the drawbacks, and not understanding the escape. You should understand that they have met with calamity and disaster, and the Wicked One can do with them what he wants. 

There\marginnote{33.1} are ascetics and brahmins who enjoy these five kinds of sensual stimulation without being tied, infatuated, or attached, seeing the drawbacks, and understanding the escape. You should understand that they haven’t met with calamity and disaster, and the Wicked One cannot do what he wants with them. 

Suppose\marginnote{33.3} a deer in the wilderness was lying on a pile of snares without being caught. You’d know that it hasn’t met with calamity and disaster, and the hunter cannot do what he wants with them. And when the hunter comes, it can flee where it wants. 

In\marginnote{33.7} the same way, there are ascetics and brahmins who enjoy these five kinds of sensual stimulation without being tied, infatuated, or attached, seeing the drawbacks, and understanding the escape. You should understand that they haven’t met with calamity and disaster, and the Wicked One cannot do what he wants with them. 

Suppose\marginnote{34.1} there was a wild deer wandering in the forest that walked, stood, sat, and laid down in confidence. Why is that? Because it’s out of the hunter’s range. 

In\marginnote{34.4} the same way, a mendicant, quite secluded from sensual pleasures, secluded from unskillful qualities, enters and remains in the first absorption, which has the rapture and bliss born of seclusion, while placing the mind and keeping it connected. This is called a mendicant who has blinded \textsanskrit{Māra}, put out his eyes without a trace, and gone where the Wicked One cannot see. 

Furthermore,\marginnote{35.1} as the placing of the mind and keeping it connected are stilled, a mendicant enters and remains in the second absorption, which has the rapture and bliss born of immersion, with internal clarity and mind at one, without placing the mind and keeping it connected. This is called a mendicant who has blinded \textsanskrit{Māra} … 

Furthermore,\marginnote{36.1} with the fading away of rapture, a mendicant enters and remains in the third absorption, where they meditate with equanimity, mindful and aware, personally experiencing the bliss of which the noble ones declare, ‘Equanimous and mindful, one meditates in bliss.’ This is called a mendicant who has blinded \textsanskrit{Māra} … 

Furthermore,\marginnote{37.1} giving up pleasure and pain, and ending former happiness and sadness, a mendicant enters and remains in the fourth absorption, without pleasure or pain, with pure equanimity and mindfulness. This is called a mendicant who has blinded \textsanskrit{Māra} … 

Furthermore,\marginnote{38.1} a mendicant, going totally beyond perceptions of form, with the ending of perceptions of impingement, not focusing on perceptions of diversity, aware that ‘space is infinite’, enters and remains in the dimension of infinite space. This is called a mendicant who has blinded \textsanskrit{Māra} … 

Furthermore,\marginnote{39.1} a mendicant, going totally beyond the dimension of infinite space, aware that ‘consciousness is infinite’, enters and remains in the dimension of infinite consciousness. This is called a mendicant who has blinded \textsanskrit{Māra} … 

Furthermore,\marginnote{40.1} a mendicant, going totally beyond the dimension of infinite consciousness, aware that ‘there is nothing at all’, enters and remains in the dimension of nothingness.\footnote{This passage and the next affirm that these meditations, learned under \textsanskrit{Āḷāra} \textsanskrit{Kālāma} and Uddaka \textsanskrit{Rāmaputta}, were adopted by the Buddha as part of his practice. } This is called a mendicant who has blinded \textsanskrit{Māra} … 

Furthermore,\marginnote{41.1} a mendicant, going totally beyond the dimension of nothingness, enters and remains in the dimension of neither perception nor non-perception. This is called a mendicant who has blinded \textsanskrit{Māra} … 

Furthermore,\marginnote{42.1} a mendicant, going totally beyond the dimension of neither perception nor non-perception, enters and remains in the cessation of perception and feeling. And, having seen with wisdom, their defilements come to an end. This is called a mendicant who has blinded \textsanskrit{Māra}, put out his eyes without a trace, and gone where the Wicked One cannot see. They’ve crossed over clinging to the world. And they walk, stand, sit, and lie down in confidence. Why is that? Because they’re out of the Wicked One’s range.” 

That\marginnote{42.6} is what the Buddha said. Satisfied, the mendicants approved what the Buddha said. 

%
\section*{{\suttatitleacronym MN 27}{\suttatitletranslation The Shorter Simile of the Elephant’s Footprint }{\suttatitleroot Cūḷahatthipadopamasutta}}
\addcontentsline{toc}{section}{\tocacronym{MN 27} \toctranslation{The Shorter Simile of the Elephant’s Footprint } \tocroot{Cūḷahatthipadopamasutta}}
\markboth{The Shorter Simile of the Elephant’s Footprint }{Cūḷahatthipadopamasutta}
\extramarks{MN 27}{MN 27}

\scevam{So\marginnote{1.1} I have heard.\footnote{This sutta urges against drawing rash conclusions, insisting on a cautious and pragmatic approach to the truth. The theme of epistemological caution is further developed in such suttas as \href{https://suttacentral.net/mn47/en/sujato}{MN 47}, \href{https://suttacentral.net/mn60/en/sujato}{MN 60}, and \href{https://suttacentral.net/mn99/en/sujato}{MN 99}. | According to the Sinhalese chronicle \textsanskrit{Mahāvaṁsa} 14.22, this was the first discourse taught by the arahant Mahinda to King Devanampiyatissa, leading to his conversion and the spread of Buddhism in Sri Lanka. This would have been around 250 BCE. } }At one time the Buddha was staying near \textsanskrit{Sāvatthī} in Jeta’s Grove, \textsanskrit{Anāthapiṇḍika}’s monastery. 

Now\marginnote{2.1} at that time the brahmin \textsanskrit{Jānussoṇi} drove out from \textsanskrit{Sāvatthī} in the middle of the day in an all-white chariot drawn by mares.\footnote{A chariot drawn by mares was the preferred vehicle of \textsanskrit{Jānussoṇi} (\href{https://suttacentral.net/mn99/en/sujato\#30.1}{MN 99:30.1}, \href{https://suttacentral.net/sn45.4/en/sujato\#1.3}{SN 45.4:1.3}), as well as the brahmin student \textsanskrit{Ambaṭṭha} \href{https://suttacentral.net/dn3/en/sujato\#1.6.1}{DN 3:1.6.1}. } He saw the wanderer Pilotika coming off in the distance,\footnote{Pilotika \textsanskrit{Vacchāyana} is not mentioned elsewhere. His first name means “patch” and may be a reference to his patchwork robe (cp. “Ajita of the hair-blanket”). \textsanskrit{Vacchāyana} is a patronymic, possibly indicating he was of the lineage of \textsanskrit{Vātsya}, a student of \textsanskrit{Yājñavalkya} (Śatapatha \textsanskrit{Brāhmaṇa} 9.5.1.62, \textsanskrit{Brahmāṇḍa} \textsanskrit{Purāṇa} 2.35.29). } and said to him, “So, Mister \textsanskrit{Vacchāyana}, where are you coming from in the middle of the day?” 

“Just\marginnote{2.5} now, good sir, I’ve come from the presence of the ascetic Gotama.” 

“What\marginnote{2.6} do you think of the ascetic Gotama’s lucidity of wisdom? Do you think he’s astute?” 

“My\marginnote{2.7} good man, who am I to judge the ascetic Gotama’s lucidity of wisdom?\footnote{As at \href{https://suttacentral.net/mn99/en/sujato\#30.7}{MN 99:30.7} and \href{https://suttacentral.net/an5.194/en/sujato\#2.4}{AN 5.194:2.4}. The Buddha speaks of his own lucidity of wisdom at \href{https://suttacentral.net/mn12/en/sujato\#62.6}{MN 12:62.6}. } You’d really have to be on the same level to judge his lucidity of wisdom.” 

“Mister\marginnote{2.9} \textsanskrit{Vacchāyana} praises the ascetic Gotama with lofty praise indeed.” 

“Who\marginnote{2.10} am I to praise the ascetic Gotama? He is praised by the praised as the first among gods and humans.” 

“But\marginnote{2.12} for what reason are you so devoted to the ascetic Gotama?” 

“Suppose\marginnote{3.1} that a skilled bull elephant tracker were to enter a bull elephant wood.\footnote{The text differentiates “bull elephant” (\textit{\textsanskrit{nāga}}), “elephant” (\textit{hatthi}), and “cow elephant” (\textit{\textsanskrit{hatthinī}}). This is essential to the parable, as the tracker is specifically seeking a bull. } There he’d see a large elephant’s footprint, long and broad. He would come to the conclusion,\footnote{The Pali phrase \textit{\textsanskrit{niṭṭhaṁ} gaccheyya} is an exact match for the English idiom “come to a conclusion”. } ‘This must be a big bull elephant.’ 

In\marginnote{3.5} the same way, because I saw four footprints of the ascetic Gotama I came to the conclusion, ‘The Blessed One is a fully awakened Buddha. The teaching is well explained. The \textsanskrit{Saṅgha} is practicing well.’ 

What\marginnote{4.1} four? Firstly, I see some clever aristocrats who are subtle, accomplished in the doctrines of others, hair-splitters. You’d think they live to demolish convictions with their intellect. They hear, ‘So, gentlemen, that ascetic Gotama will come down to such and such village or town.’ They formulate a question, thinking, ‘We’ll approach the ascetic Gotama and ask him this question. If he answers like this, we’ll refute him like that; and if he answers like that, we’ll refute him like this.’ 

When\marginnote{4.9} they hear that he has come down they approach him. The ascetic Gotama educates, encourages, fires up, and inspires them with a Dhamma talk. They don’t even get around to asking their question to the ascetic Gotama, so how could they refute his answer? Invariably, they become his disciples. When I saw this first footprint of the ascetic Gotama, I came to the conclusion, ‘The Blessed One is a fully awakened Buddha. The teaching is well explained. The \textsanskrit{Saṅgha} is practicing well.’ 

Furthermore,\marginnote{5.1} I see some clever brahmins … some clever householders … they become his disciples. 

Furthermore,\marginnote{7.1} I see some clever ascetics who are subtle, accomplished in the doctrines of others, hair-splitters. … They don’t even get around to asking their question to the ascetic Gotama, so how could they refute his answer? Invariably, they ask the ascetic Gotama for the chance to go forth. And he gives them the going-forth. Soon after going forth, living withdrawn, diligent, keen, and resolute, they realize the supreme end of the spiritual path in this very life. They live having achieved with their own insight the goal for which gentlemen rightly go forth from the lay life to homelessness. 

They\marginnote{7.14} say, ‘We were almost lost! We almost perished! For we used to claim that we were ascetics, brahmins, and perfected ones, but we were none of these things. But now we really are ascetics, brahmins, and perfected ones!’ When I saw this fourth footprint of the ascetic Gotama, I came to the conclusion, ‘The Blessed One is a fully awakened Buddha. The teaching is well explained. The \textsanskrit{Saṅgha} is practicing well.’ 

It’s\marginnote{7.20} because I saw these four footprints of the ascetic Gotama that I came to the conclusion, ‘The Blessed One is a fully awakened Buddha. The teaching is well explained. The \textsanskrit{Saṅgha} is practicing well.’” 

When\marginnote{8.1} he had spoken, \textsanskrit{Jānussoṇi} got down from his chariot, arranged his robe over one shoulder, raised his joined palms toward the Buddha, and expressed this heartfelt sentiment three times: 

“Homage\marginnote{8.2} to that Blessed One, the perfected one, the fully awakened Buddha! 

Homage\marginnote{8.3} to that Blessed One, the perfected one, the fully awakened Buddha! 

Homage\marginnote{8.4} to that Blessed One, the perfected one, the fully awakened Buddha! 

Hopefully,\marginnote{8.5} some time or other I’ll get to meet Mister Gotama, and we can have a discussion.” 

Then\marginnote{9.1} the brahmin \textsanskrit{Jānussoṇi} went up to the Buddha, and exchanged greetings with him. When the greetings and polite conversation were over, he sat down to one side, and informed the Buddha of all he had discussed with the wanderer Pilotika. 

When\marginnote{9.4} he had spoken, the Buddha said to him, “Brahmin, the simile of the elephant’s footprint is not yet completed in detail. As to how it is completed in detail, listen and apply your mind well, I will speak.” 

“Yes\marginnote{9.8} sir,” \textsanskrit{Jānussoṇi} replied. The Buddha said this: 

“Suppose\marginnote{10.1} a bull elephant tracker were to enter a bull elephant wood. There they’d see a large elephant’s footprint, long and broad. A skilled bull elephant tracker does not yet come to the conclusion, ‘This must be a big bull elephant.’ Why not? Because in an elephant wood there are dwarf cow elephants with big footprints, and this footprint might be one of theirs.\footnote{This passage names three kinds of cow elephants unknown elsewhere. From the context they must be of ascending height. The first is the “dwarf” (\textit{\textsanskrit{vāmanikā}}). } 

They\marginnote{10.7} keep following the track until they see a big footprint, long and broad, and traces high up. A skilled bull elephant tracker does not yet come to the conclusion, ‘This must be a big bull elephant.’ Why not? Because in an elephant wood there are tall lofty cow elephants with big footprints, and this footprint might be one of theirs.\footnote{\textit{\textsanskrit{Kāḷārikā}} is related to Sanskrit \textit{\textsanskrit{karāla}}, which may mean “lofty” or “with gaping teeth”. The commentary applies the latter meaning, but surely the former fits the context. } 

They\marginnote{10.13} keep following the track until they see a big footprint, long and broad, and traces and tusk-marks high up. A skilled bull elephant tracker does not yet come to the conclusion, ‘This must be a big bull elephant.’ Why not? Because in an elephant wood there are tall matriarch cow elephants with big footprints, and this footprint might be one of theirs.\footnote{One of the many Indic names for elephants, especially cows, is \textit{\textsanskrit{kareṇu}} (Pali \textit{\textsanskrit{kaṇeru}} by metathesis), which, being from the root \textit{kara} (“doer” = “hand”), has the same meaning as the more common \textit{hatthi}, namely “handy”. I think the \textit{\textsanskrit{kaṇerukā}} is the “leader of the cows”, i.e. the matriarch. } 

They\marginnote{10.19} keep following the track until they see a big footprint, long and broad, and traces, tusk-marks, and broken branches high up. And they see that bull elephant walking, standing, sitting, or lying down at the root of a tree or in the open. Then they’d come to the conclusion, ‘This is that big bull elephant.’ 

In\marginnote{11.1} the same way, brahmin, a Realized One arises in the world, perfected, a fully awakened Buddha, accomplished in knowledge and conduct, holy, knower of the world, supreme guide for those who wish to train, teacher of gods and humans, awakened, blessed.\footnote{This is the start of the teaching on the Gradual Training, encompassing ethics (\textit{\textsanskrit{sīla}}), meditation (\textit{\textsanskrit{samādhi}}), and wisdom (\textit{\textsanskrit{paññā}}). } He realizes with his own insight this world—with its gods, \textsanskrit{Māras}, and divinities, this population with its ascetics and brahmins, gods and humans—and he makes it known to others. He proclaims a teaching that is good in the beginning, good in the middle, and good in the end, meaningful and well-phrased. And he reveals a spiritual practice that’s entirely complete and pure. 

A\marginnote{12.1} householder hears that teaching, or a householder’s child, or someone reborn in a good family. They gain faith in the Realized One, and reflect, ‘Life at home is cramped and dirty, life gone forth is wide open. It’s not easy for someone living at home to lead the spiritual life utterly full and pure, like a polished shell. Why don’t I shave off my hair and beard, dress in ocher robes, and go forth from the lay life to homelessness?’ After some time they give up a large or small fortune, and a large or small family circle. They shave off hair and beard, dress in ocher robes, and go forth from the lay life to homelessness. 

Once\marginnote{13.1} they’ve gone forth, they take up the training and livelihood of the mendicants. They give up killing living creatures, renouncing the rod and the sword. They’re scrupulous and kind, living full of sympathy for all living beings.\footnote{The first and most important precept. It is not just the negative injunction to avoid killing, but also the positive injunction to have compassion for all creatures. If a monastic murders a human being they are immediately and permanently expelled. } 

They\marginnote{13.2} give up stealing. They take only what’s given, and expect only what’s given. They keep themselves clean by not thieving.\footnote{To steal anything of substantial value is an expulsion offence. } 

They\marginnote{13.3} give up unchastity. They are celibate, set apart, avoiding the vulgar act of sex.\footnote{“Chastity” is \textit{brahmacariya}, literally “divine conduct”. Here it is used in the narrow sense of refraining from sex, but more commonly it has a broader sense of “spiritual life”.  Buddhist monastics are forbidden from any form of sexual activity. To engage in penetrative intercourse is an expulsion offence. } 

They\marginnote{13.4} give up lying. They speak the truth and stick to the truth. They’re honest and dependable, and don’t trick the world with their words.\footnote{This is the first of the four kinds of right speech. Just as the precept of not killing implies the positive injunction to live with compassion, the precepts on speech enjoin a positive and constructive use of speech. If a monastic falsely claims states of enlightenment or deep meditation they are expelled. } 

They\marginnote{13.5} give up divisive speech. They don’t repeat in one place what they heard in another so as to divide people against each other. Instead, they reconcile those who are divided, supporting unity, delighting in harmony, loving harmony, speaking words that promote harmony.\footnote{“Harmony” (or “unanimity”, \textit{samagga}) does not excuse untrue, bigoted, or otherwise harmful speech. True harmony is only achieved in the presence of the Dhamma. } 

They\marginnote{13.6} give up harsh speech. They speak in a way that’s mellow, pleasing to the ear, lovely, going to the heart, polite, likable, and agreeable to the people. 

They\marginnote{13.7} give up talking nonsense. Their words are timely, true, and meaningful, in line with the teaching and training. They say things at the right time which are valuable, reasonable, succinct, and beneficial.\footnote{\textit{Attha} is a polyvalent term, here taking the senses  “meaningful” and “beneficial”. Elsewhere it means “goal”, “need”, “purpose”, “lawsuit”, or “ending”, and the senses are not always easy to untangle. } 

They\marginnote{13.8} refrain from injuring plants and seeds.\footnote{Buddhists generally do not regard plants as sentient, but value them as part of the ecosystem that supports all life. } They eat in one part of the day, abstaining from eating at night and at the wrong time.\footnote{From \href{https://suttacentral.net/mn66/en/sujato\#6.4}{MN 66:6.4} and \href{https://suttacentral.net/mn70/en/sujato\#4.8}{MN 70:4.8} we can see that “at night” means after dark, while “at the wrong time” means in the afternoon. More explicitly, these are the “wrong time at night” and the “wrong time in the day”, in which case they are both the “wrong time”. } They refrain from seeing shows of dancing, singing, and music.\footnote{Such sensual entertainments distract and excite the mind. This and the next three precepts encourage peace of mind for meditation. } They refrain from beautifying and adorning themselves with garlands, fragrance, and makeup.\footnote{This was ignored by the Buddha’s cousin, Nanda (\href{https://suttacentral.net/sn21.8/en/sujato\#1.2}{SN 21.8:1.2}). } They refrain from high and luxurious beds.\footnote{To avoid sleeping too much. } They refrain from receiving gold and currency,\footnote{Literally “gold and silver” (\textit{\textsanskrit{jātarūparajata}}), but \textit{rajata} is explained in \href{https://suttacentral.net/pli-tv-bu-vb-np18/en/sujato\#2.8}{Bu NP 18:2.8} as currency of any kind. } raw grains,\footnote{Mendicants receive only the day’s meal and do not store or cook food. } raw meat, women and girls, male and female bondservants,\footnote{According to ancient Indian law (\textsanskrit{Arthaśāstra} 3.13), a person in a time of trouble may bind themselves in service for a fee. Such bondservants were protected against cruelty, sexual abuse, and unfair work. After earning back the fee of their indenture they were freed, retaining their original inheritance and status. } goats and sheep,\footnote{These are animals raised for food. } chickens and pigs, elephants, cows, horses, and mares, and fields and land.\footnote{Land for a monastery may be accepted by the \textsanskrit{Saṅgha} as a community, but not by individual mendicants. } They refrain from running errands and messages;\footnote{These items are discussed in detail below. | Acting as a go-between for lay business was tempting due to the mendicants’ wandering lifestyle. However, it exposes them to risk if the message is not delivered or if it is bad news. } buying and selling;\footnote{For example, trading in monastery property. } falsifying weights, metals, or measures; bribery, fraud, cheating, and duplicity; mutilation, murder, abduction, banditry, plunder, and violence. 

They’re\marginnote{14.1} content with robes to look after the body and almsfood to look after the belly. Wherever they go, they set out taking only these things.\footnote{A Buddhist monk has three robes: a lower robe (sabong or sarong), an upper robe, and an outer cloak. | Here begins the series of practices that build on moral fundamentals to lay the groundwork for meditation. The sequence of contentment, mindfulness, and sense restraint sometimes varies (eg. \href{https://suttacentral.net/dn2/en/sujato\#64.1}{DN 2:64.1}). } They’re like a bird: wherever it flies, wings are its only burden. In the same way, a mendicant is content with robes to look after the body and almsfood to look after the belly. Wherever they go, they set out taking only these things. When they have this entire spectrum of noble ethics, they experience a blameless happiness inside themselves.\footnote{As at \href{https://suttacentral.net/dn2/en/sujato\#63.4}{DN 2:63.4}. This phrase perhaps belongs before contentment. } 

When\marginnote{15.1} they see a sight with their eyes, they don’t get caught up in the features and details. If the faculty of sight were left unrestrained, bad unskillful qualities of covetousness and displeasure would become overwhelming. For this reason, they practice restraint, protecting the faculty of sight, and achieving its restraint.\footnote{It is not that one cannot see things, but that, mindful of its effect, one avoids unnecessary stimulation. | “Covetousness and bitterness” (\textit{\textsanskrit{abhijjhā} \textsanskrit{domanassā}}) are the strong forms of desire and aversion caused by lack of restraint. } When they hear a sound with their ears … When they smell an odor with their nose … When they taste a flavor with their tongue … When they feel a touch with their body … When they know an idea with their mind, they don’t get caught up in the features and details. If the faculty of mind were left unrestrained, bad unskillful qualities of covetousness and displeasure would become overwhelming. For this reason, they practice restraint, protecting the faculty of mind, and achieving its restraint. When they have this noble sense restraint, they experience an unsullied bliss inside themselves.\footnote{As at \href{https://suttacentral.net/dn2/en/sujato\#64.10}{DN 2:64.10}, their happiness deepens, as they see that not only their actions but also their mind is becoming free of anything unwholesome. } 

They\marginnote{16.1} act with situational awareness when going out and coming back; when looking ahead and aside; when bending and extending the limbs; when bearing the outer robe, bowl and robes; when eating, drinking, chewing, and tasting; when urinating and defecating; when walking, standing, sitting, sleeping, waking, speaking, and keeping silent.\footnote{Situational awareness is a psychological term popularized in the 1990s. It has to do with the perception of environmental phenomena and the comprehension of their meaning, which is very close to the sense of the Pali term \textit{\textsanskrit{sampajañña}}. | These acts describe the daily life of  a mendicant: going into the village for alms, at which time there are many distracting sights. Then they return, eat their meal, and spend their day in meditation. } 

When\marginnote{17.1} they have this entire spectrum of noble ethics, this noble contentment, this noble sense restraint, and this noble mindfulness and situational awareness, they frequent a secluded lodging—a wilderness, the root of a tree, a hill, a ravine, a mountain cave, a charnel ground, a forest, the open air, a heap of straw.\footnote{The Jain \textsanskrit{Ācārāṅgasūtra} 1.8.2 notes many places used by \textsanskrit{Mahāvīra} for meditation, including a charnel ground, an empty hut, and the root of a tree. } 

After\marginnote{18.1} the meal, they return from almsround, sit down cross-legged, set their body straight, and establish mindfulness in their presence.\footnote{For \textit{parimukha} (“in their presence”) we find \textit{pratimukha} in Sanskrit, which can mean “presence” or the reflection of the face. Late canonical Pali explains \textit{parimukha} as “the tip of the nose or the reflection of the face (\textit{mukhanimitta})”. \textit{Parimukha} in Sanskrit is rare, but it appears in \textsanskrit{Pāṇini} 4.4.29, which the commentary illustrates with the example of a servant “in the presence” of their master (cp. \href{https://suttacentral.net/sn47.8/en/sujato}{SN 47.8}). So it seems the sense is “before the face” or more generally “in the presence”. | To “establish mindfulness” (\textit{\textsanskrit{satiṁ} \textsanskrit{upaṭṭhapetvā}}) is literally to “do \textsanskrit{satipaṭṭhāna}”. } Giving up covetousness for the world, they meditate with a heart rid of covetousness, cleansing the mind of covetousness.\footnote{Covetousness (\textit{abhijjha}) has been curbed by sense restraint, and now is fully abandoned. } Giving up ill will and malevolence, they meditate with a mind rid of ill will, full of sympathy for all living beings, cleansing the mind of ill will.\footnote{Likewise ill will (\textit{\textsanskrit{byāpādapadosa}}), which was called \textit{domanassa} in the formula for sense restraint. } Giving up dullness and drowsiness, they meditate with a mind rid of dullness and drowsiness, perceiving light, mindful and aware, cleansing the mind of dullness and drowsiness.\footnote{“Mindfulness and situational awareness” has a prominent role in abandoning dullness. } Giving up restlessness and remorse, they meditate without restlessness, their mind peaceful inside, cleansing the mind of restlessness and remorse.\footnote{Restlessness hankers for the future and is countered by contentment. Remorse digs up the past and is countered by ethical purity. } Giving up doubt, they meditate having gone beyond doubt, not undecided about skillful qualities, cleansing the mind of doubt.\footnote{The meditator set out on their path after gaining faith in the Buddha. } 

They\marginnote{19.1} give up these five hindrances, corruptions of the heart that weaken wisdom.\footnote{The five hindrances remain a pillar of meditation teaching. The root sense means to “obstruct” but also to “obscure, darken, veil”. } Then, quite secluded from sensual pleasures, secluded from unskillful qualities, they enter and remain in the first absorption, which has the rapture and bliss born of seclusion, while placing the mind and keeping it connected. This, brahmin, is that which is called ‘a footprint of the Realized One’ and also ‘a trace of the Realized One’ and also ‘a mark of the Realized One’.\footnote{The first true footprint of the Buddha is \textsanskrit{jhāna}, not skill in debate or conversion, nor even ethical conduct. Compare \href{https://suttacentral.net/dn2/en/sujato\#76.3}{DN 2:76.3}, where \textsanskrit{jhāna} was the first superior fruit of the spiritual life. } But a noble disciple does not yet come to the conclusion,\footnote{Just as the bull elephant tracker does not come to a conclusion until they have found a bull elephant, here the meditator has not confirmed that the Buddha’s teaching leads to the goal that was promised. } ‘The Blessed One is a fully awakened Buddha. The teaching is well explained. The \textsanskrit{Saṅgha} is practicing well.’ 

Furthermore,\marginnote{20.1} as the placing of the mind and keeping it connected are stilled, a mendicant enters and remains in the second absorption, which has the rapture and bliss born of immersion, with internal clarity and mind at one, without placing the mind and keeping it connected. This too is what is called ‘a footprint of the Realized One’ … 

Furthermore,\marginnote{21.1} with the fading away of rapture, a mendicant enters and remains in the third absorption, where they meditate with equanimity, mindful and aware, personally experiencing the bliss of which the noble ones declare, ‘Equanimous and mindful, one meditates in bliss.’ This too is what is called ‘a footprint of the Realized One’ … 

Furthermore,\marginnote{22.1} giving up pleasure and pain, and ending former happiness and sadness, a mendicant enters and remains in the fourth absorption, without pleasure or pain, with pure equanimity and mindfulness. This too is what is called ‘a footprint of the Realized One’ … 

When\marginnote{23.1} their mind has become immersed in \textsanskrit{samādhi} like this—purified, bright, flawless, rid of corruptions, pliable, workable, steady, and imperturbable—they extend it toward recollection of past lives. They recollect many kinds of past lives, that is, one, two, three, four, five, ten, twenty, thirty, forty, fifty, a hundred, a thousand, a hundred thousand rebirths; many eons of the world contracting, many eons of the world expanding, many eons of the world contracting and expanding. … They recollect their many kinds of past lives, with features and details. This too is what is called ‘a footprint of the Realized One’ … 

When\marginnote{24.1} their mind has become immersed in \textsanskrit{samādhi} like this—purified, bright, flawless, rid of corruptions, pliable, workable, steady, and imperturbable—they extend it toward knowledge of the death and rebirth of sentient beings. With clairvoyance that is purified and surpasses the human, they understand how sentient beings are reborn according to their deeds. This too is what is called ‘a footprint of the Realized One’ … 

When\marginnote{25.1} their mind has become immersed in \textsanskrit{samādhi} like this—purified, bright, flawless, rid of corruptions, pliable, workable, steady, and imperturbable—they extend it toward knowledge of the ending of defilements. They truly understand: ‘This is suffering’ … ‘This is the origin of suffering’ … ‘This is the cessation of suffering’ … ‘This is the practice that leads to the cessation of suffering.’ They truly understand: ‘These are defilements’ … ‘This is the origin of defilements’ … ‘This is the cessation of defilements’ … ‘This is the practice that leads to the cessation of defilements.’ This, brahmin, is what is called ‘a footprint of the Realized One’ and also ‘a trace of the Realized One’ and also ‘a mark of the Realized One’. At this point a noble disciple has not yet come to a conclusion, but they are coming to the conclusion,\footnote{They are in the culminating stages of the path. } ‘The Blessed One is a fully awakened Buddha. The teaching is well explained. The \textsanskrit{Saṅgha} is practicing well.’ 

Knowing\marginnote{26.1} and seeing like this, their mind is freed from the defilements of sensuality, desire to be reborn, and ignorance. When they’re freed, they know they’re freed. 

They\marginnote{26.3} understand: ‘Rebirth is ended, the spiritual journey has been completed, what had to be done has been done, there is nothing further for this place.’ This, brahmin, is what is called ‘a footprint of the Realized One’ and also ‘a trace of the Realized One’ and also ‘a mark of the Realized One’. At this point a noble disciple has come to the conclusion,\footnote{Faith, reason, evidence, and experience have all played a role in the discovery of truth. But there can be no certainty until the truth has accomplished its ultimate function, freedom from suffering. } ‘The Blessed One is a fully awakened Buddha. The teaching is well explained. The \textsanskrit{Saṅgha} is practicing well.’ And it is at this point that the simile of the elephant’s footprint has been completed in detail.” 

When\marginnote{27.1} he had spoken, the brahmin \textsanskrit{Jānussoṇi} said to the Buddha, “Excellent, Mister Gotama! Excellent! As if he were righting the overturned, or revealing the hidden, or pointing out the path to the lost, or lighting a lamp in the dark so people with clear eyes can see what’s there, Mister Gotama has made the teaching clear in many ways. I go for refuge to Mister Gotama, to the teaching, and to the mendicant \textsanskrit{Saṅgha}. From this day forth, may Mister Gotama remember me as a lay follower who has gone for refuge for life.” 

%
\section*{{\suttatitleacronym MN 28}{\suttatitletranslation The Longer Simile of the Elephant’s Footprint }{\suttatitleroot Mahāhatthipadopamasutta}}
\addcontentsline{toc}{section}{\tocacronym{MN 28} \toctranslation{The Longer Simile of the Elephant’s Footprint } \tocroot{Mahāhatthipadopamasutta}}
\markboth{The Longer Simile of the Elephant’s Footprint }{Mahāhatthipadopamasutta}
\extramarks{MN 28}{MN 28}

\scevam{So\marginnote{1.1} I have heard. }At one time the Buddha was staying near \textsanskrit{Sāvatthī} in Jeta’s Grove, \textsanskrit{Anāthapiṇḍika}’s monastery. There \textsanskrit{Sāriputta} addressed the mendicants,\footnote{This sutta is a masterclass on the methods employed by the Buddha’s greatest student, \textsanskrit{Sāriputta}. He begins with the four noble truths, then proceeds to unpack them systematically, leading to a lengthy analysis of the four elements. But the unpacking takes surprising directions as \textsanskrit{Sāriputta} draws on unexpected layers of the Dhamma to illustrate a familiar teaching in new ways. All the while, he conveys warmth and compassion, illustrating in his manner of teaching the connection that is also the topic of the teaching. } “Reverends, mendicants!” 

“Reverend,”\marginnote{1.5} they replied. \textsanskrit{Sāriputta} said this: 

“The\marginnote{2.1} footprints of all creatures that walk can fit inside an elephant’s footprint, so an elephant’s footprint is said to be the biggest of them all. In the same way, all skillful qualities are included in the four noble truths.\footnote{\textsanskrit{Sāriputta}’s emphasis on the four noble truths is also shown in \href{https://suttacentral.net/mn141/en/sujato}{MN 141}, where he is said to focus on teaching new students as far as stream-entry. The current sutta illustrates how he did this, carefully explaining fundamental concepts and showing their real world impacts, while offering pragmatic and reassuring advice. Along the way, he introduces all the major wisdom teachings of Buddhism, demonstrating exactly how they fit into the four noble truths. | The idea of “inclusion” (\textit{\textsanskrit{saṅgaha}}) became a fundamental method of the Abhidhamma \textsanskrit{Piṭaka}, especially the \textsanskrit{Dhātukathā}, which systematized the mapping of inclusion or exclusion regarding all phenomena (see too \href{https://suttacentral.net/mn44/en/sujato\#11.1}{MN 44:11.1}). } What four? The noble truths of suffering, the origin of suffering, the cessation of suffering, and the practice that leads to the cessation of suffering. 

And\marginnote{3.1} what is the noble truth of suffering? Rebirth is suffering; old age is suffering; death is suffering; sorrow, lamentation, pain, sadness, and distress are suffering; not getting what you wish for is suffering. In brief, the five grasping aggregates are suffering. And what are the five grasping aggregates? They are as follows: the grasping aggregates of form, feeling, perception, choices, and consciousness. 

And\marginnote{4.1} what is the grasping aggregate of form? The four principal states, and form derived from the four principal states.\footnote{The Buddha’s use of \textit{\textsanskrit{mahābhūtā}} (“principal states”) responds to \textsanskrit{Yājñavalkya}’s core teaching at \textsanskrit{Bṛhadāraṇyaka} \textsanskrit{Upaniṣad} 2.4.12, where the several “states” or “real entities” (\textit{\textsanskrit{bhūtā}})—namely the diverse manifestations of creation—arise from and dissolve into the “principal state” (\textit{\textsanskrit{mahābhūta}}) of the Self, singular and infinite. For the Buddha, the “principal states” are themselves plural, as there is no underlying singular reality. Later Sanskrit literature lists the “five states” (\textit{\textsanskrit{pañcabhūta}}) as earth, water, fire, air, and space. } 

And\marginnote{5.1} what are the four principal states?\footnote{Delving into the four noble truths leads us to the actual topic, the four elements. But in this topic we will find not just matter, but connections that lead us back to the four noble truths. } The elements of earth, water, fire, and air.\footnote{Each of these “elements” was a major deity of the Vedas, called by many names in addition to the obvious \textsanskrit{Pṛthivī}, Agni, Āpas, and \textsanskrit{Vāyu}. Nature was a realm of divine powers invoked and hopefully tamed by hymn and ritual. The Buddha treated elements as natural phenomena rather than divinities, but this creates a potential problem. If prayer and sacrifice is no longer effective, are we just victims of the arbitrary threats of an uncaring world? Our modern solution is to subjugate the world of nature, to set ourselves above it and beat it into submission with our technology. For the Buddha, the solution was, rather, to dissolve the difference between self and other: if we are no more than elements, elements are no less than us. While the impermanence of the elements was inescapable, we can master our own responses. Thus we do not live in a world stripped of meaning, but one full of the potential for freedom. } 

And\marginnote{6.1} what is the earth element? The earth element may be interior or exterior.\footnote{That is to say, inside oneself or in the world outside oneself. } And what is the interior earth element? Anything internal, pertaining to an individual, that’s hard, solid, and appropriated. This includes:\footnote{“Appropriated” (\textit{\textsanskrit{upādinna}}) is a technical term referring to matter that has been “taken up” or “grasped” at birth, namely the organic body. } head hair, body hair, nails, teeth, skin, flesh, sinews, bones, bone marrow, kidneys, heart, liver, diaphragm, spleen, lungs, intestines, mesentery, undigested food, feces; or anything else internal, pertaining to an individual, that’s hard, solid, and appropriated.\footnote{The phrase “or anything else” (\textit{\textsanskrit{yaṁ} \textsanskrit{vā} \textsanskrit{panaññampi} \textsanskrit{kiñci}}) indicates that the analysis is meant to be illustrative rather than exhaustive. Early Buddhism makes no claim to list all the different manifestations of the elements. } This is called the interior earth element. The interior earth element and the exterior earth element are just the earth element.\footnote{The exterior element is not defined here. Rather, the sutta moves by collapsing the distinction between interior and exterior. } This should be truly seen with right understanding like this: ‘This is not mine, I am not this, this is not my self.’\footnote{Since what is “in here” and what is “out there” are the same element, how can it be “my” body? | In this standard reflection on not-self, “this is not mine” counters craving due to a sense of ownership; “I am not this” counters the conceit of identification; and “this is not my self” counters views based on speculation. } When you truly see with right understanding, you grow disillusioned with the earth element, detaching the mind from the earth element. 

There\marginnote{7.1} comes a time when the exterior water element flares up.\footnote{This is in reference to the belief that the earth rests upon water, and if the water is disturbed it shakes the earth (\href{https://suttacentral.net/an8.70/en/sujato\#14.3}{AN 8.70:14.3}, \href{https://suttacentral.net/dn16/en/sujato\#3.13.3}{DN 16:3.13.3}). Thus this is a reference to calamity by earthquake rather than flood. Apparently it was believed that an extreme earthquake could shake apart the whole earth. } At that time the exterior earth element vanishes. So for all its great age, the earth element will be revealed as impermanent, liable to end, vanish, and perish.\footnote{Read \textit{\textsanskrit{mahallikā}} as “old” rather than “large”. | Note that “impermanence” in the suttas is not reduced to “momentariness”. It applies equally to the largest and smallest scales, as well as the living scale of the human life. } What then of this ephemeral body appropriated by craving? Rather than ‘I’ or ‘mine’ or ‘I am’, they consider it to be none of these things.\footnote{The key to this difficult sentence is recognizing that \textit{atha kho} here is adversative (compare \href{https://suttacentral.net/dn23/en/sujato\#8.1}{DN 23:8.1}: “Even though Master Kassapa says this, nonetheless I think that”, \textit{\textsanskrit{kiñcāpi} \textsanskrit{bhavaṁ} kassapo \textsanskrit{evamāha}, atha kho \textsanskrit{evaṁ} me ettha hoti}). | \textit{Notevettha hoti} resolves as follows: \textit{not(i)} is the negative, which is distributed across the three terms, I, mine, I am; \textit{ev(a)} conveys exclusivity; \textit{ettha} with \textit{hoti} and the genitive means “think about that” (eg. \href{https://suttacentral.net/mn38/en/sujato\#18.4}{MN 38:18.4}). | “Ephemeral” is \textit{\textsanskrit{mattaṭṭhaka}}, “standing” (\textit{(\textsanskrit{ṭ})\textsanskrit{ṭha}}) for “a while” (\textit{matta}). } 

If\marginnote{8.1} others abuse, attack, harass, and trouble that mendicant, they understand:\footnote{The motif of mendicants being attacked is found occasionally in the suttas, but more commonly in texts of the Jains, who went naked and did not bathe. } ‘This painful feeling born of ear contact has arisen in me.\footnote{\textsanskrit{Sāriputta} brings in a reflection on conditionality, not as a philosophy, but as a skillful method of defusing reactive emotions. } That’s dependent, not independent. Dependent on what? Dependent on contact.’ They see that contact, feeling, perception, choices, and consciousness are impermanent.\footnote{With the previously-mentioned form, now all five aggregates are included. } Based on that element alone, their mind leaps forth, gains confidence, settles down, and becomes decided.\footnote{Here the elements serve for calming the mind (\textit{samatha}) as earlier they served for insight (\textit{\textsanskrit{vipassanā}}) | \textit{\textsanskrit{Ārammaṇa}} does not mean “object”, an Abhidhamma concept foreign to the suttas: existence is relational, not objective. Rather, it is a “support” upon which the mind relies. The “support” itself is impermanent and conditioned, but it serves to achieve the purpose. | The phrase \textit{pakkhandati \textsanskrit{pasīdati} \textsanskrit{santiṭṭhati} adhimuccati} is stock, so \textit{pakkhandati} does not mean to “enter into” but rather to be “secure”. } 

Others\marginnote{9.1} might treat that mendicant with disliking, loathing, and detestation, striking them with fists, stones, sticks, and swords. They understand: ‘This body is such that fists, stones, sticks, and swords strike it. But the Buddha has said in the Advice on the Simile of the Saw:\footnote{\textsanskrit{Sāriputta} is quoting from \href{https://suttacentral.net/mn21/en/sujato\#21.1}{MN 21:21.1}. } 

“Even\marginnote{9.6} if low-down bandits were to sever you limb from limb with a two-handed saw, anyone who had a malevolent thought on that account would not be following my instructions.” My energy shall be roused up and unflagging, my mindfulness established and lucid, my body tranquil and undisturbed, and my mind immersed in \textsanskrit{samādhi}. Gladly now, let fists, stones, sticks, and swords strike this body! For this is how the instructions of the Buddhas are followed.’ 

While\marginnote{10.1} recollecting the Buddha, the teaching, and the \textsanskrit{Saṅgha} in this way, equanimity based on the skillful may not become stabilized in them.\footnote{Again the teaching is pragmatic rather than theoretical, as \textsanskrit{Sāriputta} is reassuring young students. It often happens in the course of practice that even when doing what is said to be the right thing, the results are not what you hoped. } In that case they stir up a sense of urgency: ‘It’s my loss, my misfortune, that while recollecting the Buddha, the teaching, and the \textsanskrit{Saṅgha} in this way, equanimity based on the skillful does not become stabilized in me.’ They’re like a daughter-in-law who stirs up a sense of urgency when they see their father-in-law.\footnote{This alludes to the idea that a newlywed bride moving in to her husband’s home would be on her best behavior. } But if, while recollecting the Buddha, the teaching, and the \textsanskrit{Saṅgha} in this way, equanimity based on the skillful does become stabilized in them, they’re happy with that. At this point, much has been done by that mendicant.\footnote{\textsanskrit{Sāriputta} offers positive reinforcement. Even an apparently small instance of overcoming anger is a significant step in mental training. } 

And\marginnote{11.1} what is the water element? The water element may be interior or exterior. And what is the interior water element? Anything internal, pertaining to an individual, that’s water, watery, and appropriated. This includes: bile, phlegm, pus, blood, sweat, fat, tears, grease, saliva, snot, synovial fluid, urine; or anything else internal, pertaining to an individual, that’s water, watery, and appropriated. This is called the interior water element. The interior water element and the exterior water element are just the water element. This should be truly seen with right understanding like this: ‘This is not mine, I am not this, this is not my self.’ When you truly see with right understanding, you grow disillusioned with the water element, detaching the mind from the water element. 

There\marginnote{12.1} comes a time when the exterior water element flares up. It sweeps away villages, towns, cities, countries, and regions.\footnote{Flood was and remains an ever-present threat in the Indo-Gangetic Plain. | I am writing this note in eastern Australia in 2023, where, due to the impermanence of the elements driven by human greed, we have just endured multiple devastating floods and now face summers of fire. } There comes a time when the water in the ocean sinks down a hundred leagues, or two, three, four, five, six, up to seven hundred leagues. There comes a time when the water in the ocean stands just seven palm trees deep, or six, five, four, three, two, or even just one palm tree deep. There comes a time when the water in the ocean stands just seven fathoms deep, or six, five, four, three, two, or even just one fathom deep. There comes a time when the water in the ocean stands just half a fathom deep, or waist deep, or knee deep, or even just ankle deep. There comes a time when there’s not even enough water left in the great ocean to wet the tip of the toe. So for all its great age, the water element will be revealed as impermanent, liable to end, vanish, and perish. What then of this ephemeral body appropriated by craving? Rather than ‘I’ or ‘mine’ or ‘I am’, they consider it to be none of these things. … If, while recollecting the Buddha, the teaching, and the \textsanskrit{Saṅgha} in this way, equanimity based on the skillful does become stabilized in them, they’re happy with that. At this point, much has been done by that mendicant. 

And\marginnote{16.1} what is the fire element? The fire element may be interior or exterior. And what is the interior fire element? Anything internal, pertaining to an individual that’s fire, fiery, and appropriated. This includes: that which warms, that which ages, that which heats you up when feverish, that which properly digests food and drink; or anything else internal, pertaining to an individual, that’s fire, fiery, and appropriated. This is called the interior fire element. The interior fire element and the exterior fire element are just the fire element. This should be truly seen with right understanding like this: ‘This is not mine, I am not this, this is not my self.’ When you truly see with right understanding, you grow disillusioned with the fire element, detaching the mind from the fire element. 

There\marginnote{17.1} comes a time when the exterior fire element flares up. It burns up villages, towns, cities, countries, and regions until it reaches a green field, a roadside, a cliff’s edge, a body of water, or cleared parkland, where it’s extinguished due to not being fed. There comes a time when they go looking for a fire, taking just a chicken feather or a scrap of sinew as kindling. So for all its great age, the fire element will be revealed as impermanent, liable to end, vanish, and perish. What then of this ephemeral body appropriated by craving? Rather than ‘I’ or ‘mine’ or ‘I am’, they consider it to be none of these things. … 

If,\marginnote{18{-}20.1} while recollecting the Buddha, the teaching, and the \textsanskrit{Saṅgha} in this way, equanimity based on the skillful does become stabilized in them, they’re happy with that. At this point, much has been done by that mendicant. 

And\marginnote{21.1} what is the air element? The air element may be interior or exterior. And what is the interior air element? Anything internal, pertaining to an individual, that’s air, airy, and appropriated. This includes: winds that go up or down, winds in the belly or the bowels, winds that flow through the limbs, in-breaths and out-breaths; or anything else internal, pertaining to an individual, that’s air, airy, and appropriated.\footnote{The “winds” (\textit{\textsanskrit{vātā}}) include both literal gas and movements of energy. | The study of such winds was a major theme of the early \textsanskrit{Upaniṣads}. The very first sentence of the \textsanskrit{Bṛhadāraṇyaka} \textsanskrit{Upaniṣad} says that, for the sacrificial horse, the wind (\textit{\textsanskrit{vātā}}) is the breath (or vital force, \textit{\textsanskrit{prāṇa}}), a concept it goes on to mention no fewer than 160 times. } This is called the interior air element. The interior air element and the exterior air element are just the air element. This should be truly seen with right understanding like this: ‘This is not mine, I am not this, this is not my self.’ When you truly see with right understanding, you grow disillusioned with the air element, detaching the mind from the air element. 

There\marginnote{22.1} comes a time when the exterior air element flares up. It sweeps away villages, towns, cities, countries, and regions. There comes a time, in the last month of summer, when they look for wind by using a palm-leaf or fan, and even the grasses in the drip-fringe of a thatch roof don’t stir. So for all its great age, the air element will be revealed as impermanent, liable to end, vanish, and perish. What then of this ephemeral body appropriated by craving? Rather than ‘I’ or ‘mine’ or ‘I am’, they consider it to be none of these things. 

If\marginnote{23.1} others abuse, attack, harass, and trouble that mendicant, they understand: ‘This painful feeling born of ear contact has arisen in me. That’s dependent, not independent. Dependent on what? Dependent on contact.’ They see that contact, feeling, perception, choices, and consciousness are impermanent. Based on that element alone, their mind leaps forth, gains confidence, settles down, and becomes decided. 

Others\marginnote{24.1} might treat that mendicant with disliking, loathing, and detestation, striking them with fists, stones, sticks, and swords. They understand: ‘This body is such that fists, stones, sticks, and swords strike it. But the Buddha has said in the Advice on the Simile of the Saw: “Even if low-down bandits were to sever you limb from limb with a two-handed saw, anyone who had a thought of hate on that account would not be following my instructions.” My energy shall be roused up and unflagging, my mindfulness established and lucid, my body tranquil and undisturbed, and my mind immersed in \textsanskrit{samādhi}. Gladly now, let fists, stones, sticks, and swords strike this body! For this is how the instructions of the Buddhas are followed.’ 

While\marginnote{25.1} recollecting the Buddha, the teaching, and the \textsanskrit{Saṅgha} in this way, equanimity based on the skillful may not become stabilized in them. In that case they stir up a sense of urgency: ‘It’s my loss, my misfortune, that while recollecting the Buddha, the teaching, and the \textsanskrit{Saṅgha} in this way, equanimity based on the skillful does not become stabilized in me.’ They’re like a daughter-in-law who stirs up a sense of urgency when they see their father-in-law. But if, while recollecting the Buddha, the teaching, and the \textsanskrit{Saṅgha} in this way, equanimity based on the skillful does become stabilized in them, they’re happy with that. At this point, much has been done by that mendicant. 

When\marginnote{26.1} a space is enclosed by sticks, creepers, grass, and mud it becomes known as a ‘building’.\footnote{\textsanskrit{Sāriputta} extends the teaching in a new way by including the space element. But this is not arbitrarily added to the previous discussion. Rather, he begins by defining space as that which is physically delimited by the other elements. Thus space is considered as a kind of “derived form” (\textit{\textsanskrit{upādāyarūpa}}). This point was contested by some later schools of Buddhism, who believed that space was unconditioned. } In the same way, when a space is enclosed by bones, sinews, flesh, and skin it becomes known as a ‘form’. 

Reverends,\marginnote{27.1} though the eye is intact internally, so long as exterior sights don’t come into range and there’s no corresponding engagement, there’s no manifestation of the corresponding type of consciousness.\footnote{This expands the normal explanation of sense experience, found for example at \href{https://suttacentral.net/mn148/en/sujato\#7.3}{MN 148:7.3}. } Though the eye is intact internally and exterior sights come into range, so long as there’s no corresponding engagement, there’s no manifestation of the corresponding type of consciousness. But when the eye is intact internally and exterior sights come into range and there is corresponding engagement, there is the manifestation of the corresponding type of consciousness. 

The\marginnote{28.1} form produced in this way is included in the grasping aggregate of form. The feeling, perception, choices, and consciousness produced in this way are each included in the corresponding grasping aggregate.\footnote{\textsanskrit{Sāriputta} shows the connection between the six senses and the five aggregates. } 

They\marginnote{28.2} understand: ‘So this is how there comes to be inclusion, gathering together, and joining together into these five grasping aggregates. But the Buddha has said:\footnote{This statement is not found in the Pali canon. } “One who sees dependent origination sees the teaching.\footnote{Bringing in dependent origination, \textsanskrit{Sāriputta} integrates in a single discourse all the major wisdom frameworks: the truths, aggregates, senses, elements, and dependent origination. This is especially useful for students, as they would have heard these individual teachings many times, but might be unclear how they are connected. } One who sees the teaching sees dependent origination.” And these five grasping aggregates are indeed dependently originated. The desire, clinging, attraction, and attachment for these five grasping aggregates is the origin of suffering.\footnote{He brings the teaching back around to the four noble truths. } Giving up and getting rid of desire and greed for these five grasping aggregates is the cessation of suffering.’ At this point, much has been done by that mendicant. 

Though\marginnote{29{-}30.1} the ear … nose … tongue … body … mind is intact internally, so long as exterior ideas don’t come into range and there’s no corresponding engagement, there’s no manifestation of the corresponding type of consciousness.\footnote{The “mind” (\textit{mano}, Sanskrit \textit{manas}) here is presumably what is elsewhere called the “mind element” (\textit{\textsanskrit{manodhātu}}, eg. \href{https://suttacentral.net/mn155/en/sujato\#4.8}{MN 155:4.8}), since both are connected with “ideas” (\textit{\textsanskrit{dhammā}}) and “mind consciousness” (\textit{\textsanskrit{manoviññāṇa}}). The exact nature of \textit{mano} here is not specified in early texts. Judging from this sutta, it parallels the physical basis for the other five kinds of consciousness (already in \textsanskrit{Bṛhadāraṇyaka} \textsanskrit{Upaniṣad} 6.1.8: “seeing through the eye, hearing through the ear, knowing through the mind”), in which case it might be the physical basis for consciousness. The suttas do not specify what this is; tradition calls it the \textit{hadayavatthu}, the “heart basis”, while today we would recognize it as the brain. Compare Aitareya \textsanskrit{Upaniṣad}, which says that the mind (\textit{manas}) creates the heart, which in turn creates the moon (1.1.4); when man was formed, the moon, having become \textit{manas}, entered the heart (1.2.4); mind and heart are then said to be the same (3.2). As with the other senses, the mind remains inert until actuated by the Self. Leaving aside the metaphysics, the conception of \textit{manas} as the underlying potential for consciousness is similar, regardless of whether this was a physical organ, or a process of ongoing consciousness awaiting stimulation per the commentary (\textit{\textsanskrit{bhavaṅgacitta}}). } 

Though\marginnote{38.1} the mind is intact internally and exterior ideas come into range, so long as there’s no corresponding engagement, there’s no manifestation of the corresponding type of consciousness. But when the mind is intact internally and exterior ideas come into range and there is corresponding engagement, there is the manifestation of the corresponding type of consciousness. 

The\marginnote{38.3} form produced in this way is included in the grasping aggregate of form. The feeling, perception, choices, and consciousness produced in this way are each included in the corresponding grasping aggregate. They understand: ‘So this is how there comes to be inclusion, gathering together, and joining together into these five grasping aggregates. 

But\marginnote{38.6} the Buddha has also said: “One who sees dependent origination sees the teaching. One who sees the teaching sees dependent origination.” And these five grasping aggregates are indeed dependently originated. The desire, clinging, attraction, and attachment for these five grasping aggregates is the origin of suffering. Giving up and getting rid of desire and greed for these five grasping aggregates is the cessation of suffering.’ At this point, much has been done by that mendicant.” 

That’s\marginnote{38.13} what Venerable \textsanskrit{Sāriputta} said. Satisfied, the mendicants approved what \textsanskrit{Sāriputta} said. 

%
\section*{{\suttatitleacronym MN 29}{\suttatitletranslation The Longer Simile of the Heartwood }{\suttatitleroot Mahāsāropamasutta}}
\addcontentsline{toc}{section}{\tocacronym{MN 29} \toctranslation{The Longer Simile of the Heartwood } \tocroot{Mahāsāropamasutta}}
\markboth{The Longer Simile of the Heartwood }{Mahāsāropamasutta}
\extramarks{MN 29}{MN 29}

\scevam{So\marginnote{1.1} I have heard. }At one time the Buddha was staying near \textsanskrit{Rājagaha}, on the Vulture’s Peak Mountain, not long after Devadatta had left.\footnote{Devadatta was the Buddha’s cousin and nemesis. After going forth together with several other relatives of the Buddha (\href{https://suttacentral.net/pli-tv-kd17/en/sujato\#1.4.1}{Kd 17:1.4.1}), his initial success in meditation was corrupted by his desire for gains and fame (\href{https://suttacentral.net/sn17.35/en/sujato}{SN 17.35}). He tried to take over the \textsanskrit{Saṅgha} but was denounced by the Buddha. His attempt to create a schismatic faction failed and he died in disgrace. } There the Buddha spoke to the mendicants about Devadatta: 

“Mendicants,\marginnote{2.1} take the case of a gentleman who has gone forth out of faith from the lay life to homelessness, thinking, ‘I’m swamped by rebirth, old age, and death; by sorrow, lamentation, pain, sadness, and distress. I’m swamped by suffering, mired in suffering. Hopefully I can find an end to this entire mass of suffering.’ When they’ve gone forth they generate possessions, honor, and popularity. They’re happy with that, and they’ve got all they wished for. And they glorify themselves and put others down because of that: ‘I’m the one with possessions, honor, and popularity. These other mendicants are obscure and insignificant.’ And so they become indulgent and fall into negligence on account of those possessions, honor, and popularity. And being negligent they live in suffering.\footnote{The Chinese parallel at EA 43.4 says that this discourse was given during the time he was being lavishly supported by Prince \textsanskrit{Ajātasattu} (\href{https://suttacentral.net/sn17.36/en/sujato}{SN 17.36}, \href{https://suttacentral.net/pli-tv-kd17/en/sujato\#2.1.1}{Kd 17:2.1.1}). } 

Suppose\marginnote{2.9} there was a person in need of heartwood. And while wandering in search of heartwood he’d come across a large tree standing with heartwood. But, passing over the heartwood, softwood, bark, and shoots, he’d cut off the branches and leaves and depart imagining they were heartwood.\footnote{Compare the “noble quest” on \href{https://suttacentral.net/mn26/en/sujato}{MN 26} and the search for a “bull elephant” in \href{https://suttacentral.net/mn27/en/sujato}{MN 27}. In each case the metaphor emphasizes the specific and uncompromising nature of the goal. The moral is to not stop until the journey is truly complete. } If a person with clear eyes saw him they’d say, ‘This gentleman doesn’t know what heartwood, softwood, bark, shoots, or branches and leaves are. That’s why he passed them over, cut off the branches and leaves, and departed imagining they were heartwood. Whatever he needs to make from heartwood, he won’t succeed.’ … 

This\marginnote{2.14} is called a mendicant who has grabbed the branches and leaves of the spiritual life and stopped short with that. 

Next,\marginnote{3.1} take a gentleman who has gone forth out of faith from the lay life to homelessness … When they’ve gone forth they generate possessions, honor, and popularity. They’re not happy with that, and haven’t got all they wished for. They don’t glorify themselves and put others down on account of that. Nor do they become indulgent and fall into negligence on account of those possessions, honor, and popularity. Being diligent, they achieve accomplishment in ethics. They’re happy with that, and they’ve got all they wished for. And they glorify themselves and put others down on account of that: ‘I’m the one who is ethical, of good character. These other mendicants are unethical, of bad character.’ And so they become indulgent and fall into negligence regarding their accomplishment in ethics. And being negligent they live in suffering. 

Suppose\marginnote{3.13} there was a person in need of heartwood. And while wandering in search of heartwood he’d come across a large tree standing with heartwood. But, passing over the heartwood, softwood, and bark, he’d cut off the shoots and depart imagining they were heartwood. If a person with clear eyes saw him they’d say, ‘This gentleman doesn’t know what heartwood, softwood, bark, shoots, or branches and leaves are. That’s why he passed them over, cut off the shoots, and departed imagining they were heartwood. Whatever he needs to make from heartwood, he won’t succeed.’ … 

This\marginnote{4.1} is called a mendicant who has grabbed the shoots of the spiritual life and stopped short with that. 

Next,\marginnote{4.15} take a gentleman who has gone forth out of faith from the lay life to homelessness … When they’ve gone forth they generate possessions, honor, and popularity. … Being diligent, they achieve accomplishment in immersion. They’re happy with that, and they’ve got all they wished for. And they glorify themselves and put others down on account of that: ‘I’m the one with immersion and unified mind. These other mendicants lack immersion, they have straying minds.’ And so they become indulgent and fall into negligence regarding that accomplishment in immersion. And being negligent they live in suffering. 

Suppose\marginnote{4.30} there was a person in need of heartwood. And while wandering in search of heartwood he’d come across a large tree standing with heartwood. But, passing over the heartwood and softwood, he’d cut off the bark and depart imagining it was heartwood. If a person with clear eyes saw him they’d say: ‘This gentleman doesn’t know what heartwood, softwood, bark, shoots, or branches and leaves are. That’s why he passed them over, cut off the bark, and departed imagining it was heartwood. Whatever he needs to make from heartwood, he won’t succeed.’ … 

This\marginnote{4.34} is called a mendicant who has grabbed the bark of the spiritual life and stopped short with that. 

Next,\marginnote{5.1} take a gentleman who has gone forth out of faith from the lay life to homelessness … When they’ve gone forth they generate possessions, honor, and popularity. … Being diligent, they achieve knowledge and vision.\footnote{While “knowledge and vision” (\textit{\textsanskrit{ñāṇadassana}}) that is “in accordance with reality” (\textit{\textsanskrit{yathābhūta}}) refers to the vision of the four noble truths at stream-entry, “knowledge and vision” by itself does not have a well established technical meaning. At \href{https://suttacentral.net/dn2/en/sujato\#83.1}{DN 2:83.1} it is said to be a vision of the body with its consciousness, although this passage is rare and in \href{https://suttacentral.net/mn77/en/sujato\#29.11}{MN 77:29.11} is not described as “knowledge and vision”. Elsewhere it is used in a variety of senses for a kind of clear and enhanced spiritual understanding. } They’re happy with that, and they’ve got all they wished for. And they glorify themselves and put others down on account of that, ‘I’m the one who meditates knowing and seeing. These other mendicants meditate without knowing and seeing.’ And so they become indulgent and fall into negligence regarding that knowledge and vision. And being negligent they live in suffering. 

Suppose\marginnote{5.20} there was a person in need of heartwood. And while wandering in search of heartwood he’d come across a large tree standing with heartwood. But, passing over the heartwood, he’d cut out the softwood and depart imagining it was heartwood. If a person with clear eyes saw him they’d say, ‘This gentleman doesn’t know what heartwood, softwood, bark, shoots, or branches and leaves are. That’s why he passed them over, cut out the softwood, and departed imagining it was heartwood. Whatever he needs to make from heartwood, he won’t succeed.’ … 

This\marginnote{5.25} is called a mendicant who has grabbed the softwood of the spiritual life and stopped short with that. 

Next,\marginnote{6.1} take a gentleman who has gone forth out of faith from the lay life to homelessness, thinking, ‘I’m swamped by rebirth, old age, and death; by sorrow, lamentation, pain, sadness, and distress. I’m swamped by suffering, mired in suffering. Hopefully I can find an end to this entire mass of suffering.’ When they’ve gone forth they generate possessions, honor, and popularity. They’re not happy with that, and haven’t got all they wished for. They don’t glorify themselves and put others down on account of that. Nor do they become indulgent and fall into negligence on account of those possessions, honor, and popularity. Being diligent, they achieve accomplishment in ethics. They’re happy with that, but they haven’t got all they wished for. They don’t glorify themselves and put others down on account of that. Nor do they become indulgent and fall into negligence regarding that accomplishment in ethics. Being diligent, they achieve accomplishment in immersion. They’re happy with that, but they haven’t got all they wished for. They don’t glorify themselves and put others down on account of that. Nor do they become indulgent and fall into negligence regarding that accomplishment in immersion. Being diligent, they achieve knowledge and vision. They’re happy with that, but they haven’t got all they wished for. They don’t glorify themselves and put others down on account of that. Nor do they become indulgent and fall into negligence regarding that knowledge and vision. Being diligent, they achieve irreversible freedom.\footnote{The “irreversible freedom” (\textit{asamaya}, literally “not temporary”) is that of arahantship, as opposed to the “temporary freedom” of \textit{\textsanskrit{samādhi}}, from which one can fall away due to bad conduct (\href{https://suttacentral.net/an5.149/en/sujato}{AN 5.149}), like Devadatta. | Text inconsistently has \textit{vimokkha} (“liberation”) here and \textit{vimutti} (“freedom”) in the next line, although elsewhere the suttas always read \textit{vimutti}. It seems that when these attainments came to be included in the \textit{Vimokkha} Chapter of the \textsanskrit{Paṭisambhidāmagga} (a late canonical Abhidhamma-style text), they were renamed \textit{vimokkha} to fit the context (\href{https://suttacentral.net/ps1.5/en/sujato\#2.10}{Ps 1.5:2.10}). The commentary then cited that passage here using \textit{vimokkha}, but the commentarial gloss must have contaminated the text. I translate under the assumption that the correct reading is \textit{vimutti}, although it makes no practical difference. } And it’s impossible for that mendicant to fall away from that irreversible freedom. 

Suppose\marginnote{6.18} there was a person in need of heartwood. And while wandering in search of heartwood he’d come across a large tree standing with heartwood. He’d cut out just the heartwood and depart knowing it was heartwood. If a person with clear eyes saw him they’d say, ‘This gentleman knows what heartwood, softwood, bark, shoots, and branches and leaves are. That’s why he cut out just the heartwood and departed knowing it was heartwood. Whatever he needs to make from heartwood, he will succeed.’ … 

It’s\marginnote{6.23} impossible for that mendicant to fall away from that irreversible freedom. 

And\marginnote{7.1} so, mendicants, this spiritual life is not lived for the sake of possessions, honor, and popularity, or for accomplishment in ethics, or for accomplishment in immersion, or for knowledge and vision. Rather, the goal, heartwood, and final end of the spiritual life is the unshakable freedom of heart.”\footnote{This links the “irreversible” freedom with the “unshakable” freedom of arahantship; the two are equated at \href{https://suttacentral.net/mn122/en/sujato\#4.1}{MN 122:4.1}. } 

That\marginnote{7.4} is what the Buddha said. Satisfied, the mendicants approved what the Buddha said. 

%
\section*{{\suttatitleacronym MN 30}{\suttatitletranslation The Shorter Simile of the Heartwood }{\suttatitleroot Cūḷasāropamasutta}}
\addcontentsline{toc}{section}{\tocacronym{MN 30} \toctranslation{The Shorter Simile of the Heartwood } \tocroot{Cūḷasāropamasutta}}
\markboth{The Shorter Simile of the Heartwood }{Cūḷasāropamasutta}
\extramarks{MN 30}{MN 30}

\scevam{So\marginnote{1.1} I have heard. }At one time the Buddha was staying near \textsanskrit{Sāvatthī} in Jeta’s Grove, \textsanskrit{Anāthapiṇḍika}’s monastery. 

Then\marginnote{2.1} the brahmin \textsanskrit{Piṅgalakoccha} went up to the Buddha, and exchanged greetings with him.\footnote{Not mentioned elsewhere in the Pali, this brahmin’s name means “tawny” (or “blotchy”) Koccha. His name suggests he was of the lineage of Kutsa \textsanskrit{Āṅgirasa}, an oft-mentioned sage and ally (and sometime foe) of Indra in the Rig Veda. The name Kautsa appears in the same lineage lists as \textsanskrit{Vātsya} (eg. \textsanskrit{Bṛhadāraṇyaka} \textsanskrit{Upaniṣad} 6.5.4; see \href{https://suttacentral.net/mn27/en/sujato\#2.2}{MN 27:2.2}). } When the greetings and polite conversation were over, he sat down to one side and said to the Buddha: 

“Mister\marginnote{2.3} Gotama, there are those ascetics and brahmins who lead an order and a community, and tutor a community. They’re well-known and famous religious founders, deemed holy by many people. Namely: \textsanskrit{Pūraṇa} Kassapa, the bamboo-staffed ascetic \textsanskrit{Gosāla}, Ajita of the hair blanket, Pakudha \textsanskrit{Kaccāyana}, \textsanskrit{Sañjaya} \textsanskrit{Belaṭṭhiputta}, and the Jain ascetic of the \textsanskrit{Ñātika} clan. According to their own claims, did all of them have direct knowledge, or none of them, or only some?”\footnote{While some such as \textsanskrit{Mahāvīra} the \textsanskrit{Ñātika} (\href{https://suttacentral.net/mn14/en/sujato\#17.2}{MN 14:17.2}) and \textsanskrit{Pūraṇa} Kassapa (\href{https://suttacentral.net/an9.38/en/sujato\#2.1}{AN 9.38:2.1}) claimed to have direct knowledge, others such as Ajita Kesakambala denied that such knowledge was possible (\href{https://suttacentral.net/dn2/en/sujato\#23.2}{DN 2:23.2}). } 

“Enough,\marginnote{2.6} brahmin, let this be:\footnote{The Buddha responded the same way when asked this question by Subhadda (\href{https://suttacentral.net/dn16/en/sujato\#5.26.5}{DN 16:5.26.5}). } ‘According to their own claims, did all of them have direct knowledge, or none of them, or only some?’ I will teach you the Dhamma. Listen and apply your mind well, I will speak.” 

“Yes\marginnote{2.10} sir,” \textsanskrit{Piṅgalakoccha} replied. The Buddha said this: 

“Suppose\marginnote{3.1} there was a person in need of heartwood. And while wandering in search of heartwood he’d come across a large tree standing with heartwood. But, passing over the heartwood, softwood, bark, and shoots, he’d cut off the branches and leaves and depart imagining they were heartwood. If a person with clear eyes saw him they’d say: ‘This gentleman doesn’t know what heartwood, softwood, bark, shoots, or branches and leaves are. That’s why he passed them over, cut off the branches and leaves, and departed imagining they were heartwood. Whatever he needs to make from heartwood, he won’t succeed.’ 

Suppose\marginnote{4.1} there was another person in need of heartwood … he’d cut off the shoots and depart imagining they were heartwood … 

Suppose\marginnote{5.1} there was another person in need of heartwood … he’d cut off the bark and depart imagining it was heartwood … 

Suppose\marginnote{6.1} there was another person in need of heartwood … he’d cut out the softwood and depart imagining it was heartwood … 

Suppose\marginnote{7.1} there was another person in need of heartwood. And while wandering in search of heartwood he’d come across a large tree standing with heartwood. He’d cut out just the heartwood and depart knowing it was heartwood. If a person with clear eyes saw him they’d say: ‘This gentleman knows what heartwood, softwood, bark, shoots, or branches and leaves are. That’s why he cut out just the heartwood and departed knowing it was heartwood. Whatever he needs to make from heartwood, he will succeed.’ 

In\marginnote{8.1} the same way, take a certain person who goes forth from the lay life to homelessness, thinking: ‘I’m swamped by rebirth, old age, and death; by sorrow, lamentation, pain, sadness, and distress. I’m swamped by suffering, mired in suffering. Hopefully I can find an end to this entire mass of suffering.’ When they’ve gone forth they generate possessions, honor, and popularity. They’re happy with that, and they’ve got all they wished for. And they glorify themselves and put others down on account of that: ‘I’m the one with possessions, honor, and popularity. These other mendicants are obscure and insignificant.’ They become lazy and slack on account of their possessions, honor, and popularity, not generating enthusiasm or trying to realize those things that are better and finer. …\footnote{This sentence differentiates this from the corresponding passage in the preceding sutta \href{https://suttacentral.net/mn29/en/sujato\#2.8}{MN 29:2.8}. } They’re like the person who mistakes branches and leaves for heartwood, I say. 

Next,\marginnote{9.1} take a person who has gone forth out of faith from the lay life to homelessness … When they’ve gone forth they generate possessions, honor, and popularity. They’re not happy with that, and haven’t got all they wished for. They don’t glorify themselves and put others down on account of that. They don’t become lazy and slack regarding their possessions, honor, and popularity, but generate enthusiasm and try to realize those things that are better and finer. They achieve accomplishment in ethics. They’re happy with that, and they’ve got all they wished for. And they glorify themselves and put others down on account of that: ‘I’m the one who is ethical, of good character. These other mendicants are unethical, of bad character.’ They become lazy and slack regarding their accomplishment in ethics, not generating enthusiasm or trying to realize those things that are better and finer. … They’re like the person who mistakes shoots for heartwood, I say. 

Next,\marginnote{10.1} take a person who has gone forth out of faith from the lay life to homelessness … They achieve accomplishment in ethics … and immersion. … They become lazy and slack regarding their accomplishment in immersion, not generating enthusiasm or trying to realize those things that are better and finer. … They’re like the person who mistakes bark for heartwood, I say. 

Next,\marginnote{11.1} take a person who has gone forth out of faith from the lay life to homelessness … They achieve accomplishment in ethics … immersion … and knowledge and vision. They become lazy and slack regarding their knowledge and vision, not generating enthusiasm or trying to realize those things that are better and finer. … They’re like the person who mistakes softwood for heartwood, I say. 

Next,\marginnote{12.1} take a person who has gone forth out of faith from the lay life to homelessness, thinking: ‘I’m swamped by rebirth, old age, and death; by sorrow, lamentation, pain, sadness, and distress. I’m swamped by suffering, mired in suffering. Hopefully I can find an end to this entire mass of suffering.’ When they’ve gone forth they generate possessions, honor, and popularity. They’re not happy with that, and haven’t got all they wished for. They don’t glorify themselves and put others down on account of that. They don’t become lazy and slack on account of their possessions, honor, and popularity, but generate enthusiasm and try to realize those things that are better and finer. They become accomplished in ethics. They’re happy with that, but they haven’t got all they wished for. They don’t glorify themselves and put others down on account of that. They don’t become lazy and slack regarding their accomplishment in ethics, but generate enthusiasm and try to realize those things that are better and finer. They become accomplished in immersion. They’re happy with that, but they haven’t got all they wished for. They don’t glorify themselves and put others down on account of that. They don’t become lazy and slack regarding their accomplishment in immersion, but generate enthusiasm and try to realize those things that are better and finer. They achieve knowledge and vision. They’re happy with that, but they haven’t got all they wished for. They don’t glorify themselves and put others down on account of that. They don’t become lazy and slack regarding their knowledge and vision, but generate enthusiasm and try to realize those things that are better and finer. 

And\marginnote{12.20} what are those things that are better and finer than knowledge and vision? 

Take\marginnote{13.1} a mendicant who, quite secluded from sensual pleasures, secluded from unskillful qualities, enters and remains in the first absorption, which has the rapture and bliss born of seclusion, while placing the mind and keeping it connected.\footnote{Where the previous sutta went directly from “knowledge and vision” to final liberation (\href{https://suttacentral.net/mn29/en/sujato\#6.16}{MN 29:6.16}), the current sutta adds an extensive passage on the nine attainments.   Several details indicate that this passage is a later insertion. First, the \textit{\textsanskrit{jhānas}} are already included under immersion above (\href{https://suttacentral.net/mn30/en/sujato\#10.12}{MN 30:10.12}). There, immersion leads to knowledge and vision following the normal sequence (eg. \href{https://suttacentral.net/dn2/en/sujato\#83.1}{DN 2:83.1}), rather than being superior to it as here. Further, this discourse is distinguished as “short” compared to the previous “large” discourse, but it is in fact longer due to this extra material, which was presumably added after the title was established. EA 43.4, which is the only Chinese parallel to both these discourses, also lacks this extra passage. } This is something better and finer than knowledge and vision. 

Furthermore,\marginnote{14.1} as the placing of the mind and keeping it connected are stilled, a mendicant enters and remains in the second absorption, which has the rapture and bliss born of immersion, with internal clarity and mind at one, without placing the mind and keeping it connected. This too is something better and finer than knowledge and vision. 

Furthermore,\marginnote{15.1} with the fading away of rapture, a mendicant enters and remains in the third absorption, where they meditate with equanimity, mindful and aware, personally experiencing the bliss of which the noble ones declare, ‘Equanimous and mindful, one meditates in bliss.’ This too is something better and finer than knowledge and vision. 

Furthermore,\marginnote{16.1} giving up pleasure and pain, and ending former happiness and sadness, a mendicant enters and remains in the fourth absorption, without pleasure or pain, with pure equanimity and mindfulness. This too is something better and finer than knowledge and vision. 

Furthermore,\marginnote{17.1} a mendicant, going totally beyond perceptions of form, with the ending of perceptions of impingement, not focusing on perceptions of diversity, aware that ‘space is infinite’, enters and remains in the dimension of infinite space. This too is something better and finer than knowledge and vision. 

Furthermore,\marginnote{18.1} a mendicant, going totally beyond the dimension of infinite space, aware that ‘consciousness is infinite’, enters and remains in the dimension of infinite consciousness. This too is something better and finer than knowledge and vision. 

Furthermore,\marginnote{19.1} a mendicant, going totally beyond the dimension of infinite consciousness, aware that ‘there is nothing at all’, enters and remains in the dimension of nothingness. This too is something better and finer than knowledge and vision. 

Furthermore,\marginnote{20.1} take a mendicant who, going totally beyond the dimension of nothingness, enters and remains in the dimension of neither perception nor non-perception. This too is something better and finer than knowledge and vision. 

Furthermore,\marginnote{21.1} take a mendicant who, going totally beyond the dimension of neither perception nor non-perception, enters and remains in the cessation of perception and feeling. And, having seen with wisdom, their defilements come to an end. This too is something better and finer than knowledge and vision. These are the things that are better and finer than knowledge and vision. 

Suppose\marginnote{22.1} there was a person in need of heartwood. And while wandering in search of heartwood he’d come across a large tree standing with heartwood. He’d cut out just the heartwood and depart knowing it was heartwood. Whatever he needs to make from heartwood, he will succeed. That’s what this person is like, I say. 

And\marginnote{23.1} so, brahmin, this spiritual life is not lived for the sake of possessions, honor, and popularity, or for accomplishment in ethics, or for accomplishment in immersion, or for knowledge and vision. Rather, the goal, heartwood, and final end of the spiritual life is the unshakable freedom of heart.” 

When\marginnote{24.1} he had spoken, the brahmin \textsanskrit{Piṅgalakoccha} said to the Buddha, “Excellent, Mister Gotama! Excellent! … From this day forth, may Mister Gotama remember me as a lay follower who has gone for refuge for life.” 

%
\addtocontents{toc}{\let\protect\contentsline\protect\nopagecontentsline}
\chapter*{The Greater Chapter on Pairs }
\addcontentsline{toc}{chapter}{\tocchapterline{The Greater Chapter on Pairs }}
\addtocontents{toc}{\let\protect\contentsline\protect\oldcontentsline}

%
\section*{{\suttatitleacronym MN 31}{\suttatitletranslation The Shorter Discourse at Gosiṅga }{\suttatitleroot Cūḷagosiṅgasutta}}
\addcontentsline{toc}{section}{\tocacronym{MN 31} \toctranslation{The Shorter Discourse at Gosiṅga } \tocroot{Cūḷagosiṅgasutta}}
\markboth{The Shorter Discourse at Gosiṅga }{Cūḷagosiṅgasutta}
\extramarks{MN 31}{MN 31}

\scevam{So\marginnote{1.1} I have heard.\footnote{This discourse is a favorite among monastics, depicting an ideal life of simplicity and friendship. The introduction recurs at \href{https://suttacentral.net/pli-tv-kd10/en/sujato\#4.2.1}{Kd 10:4.2.1} and, with a different teaching portion, at \href{https://suttacentral.net/mn128/en/sujato\#81}{MN 128:81}. In both those cases it is set after the Buddha left the quarreling monks of \textsanskrit{Kosambī}. | This discourse and \href{https://suttacentral.net/mn128/en/sujato}{MN 128} appear to be influenced by \textsanskrit{Bṛhadāraṇyaka} \textsanskrit{Upaniṣad} 4.3, a famous dialogue where an unusually reluctant \textsanskrit{Yājñavalkya} is repeatedly pressed by King Janaka to reveal the true nature of a person’s light. } }At one time the Buddha was staying at \textsanskrit{Ñātika} in the brick house.\footnote{\textsanskrit{Ñātika} (also spelled \textit{\textsanskrit{nātika}}, or \textit{\textsanskrit{nādika}}; Sanskrit \textit{\textsanskrit{jñātṛka}}; \textsanskrit{Prākrit} \textit{\textsanskrit{nāyika}}) was the clan to which the Jain leader \textsanskrit{Mahāvīra} (called \textsanskrit{Nāṭaputta}, i.e. \textsanskrit{Ñātiputta}) belonged. This is the chief town of the clan. It was part of the Vajjian Federation (See \href{https://suttacentral.net/dn16/en/sujato\#2.5.1}{DN 16:2.5.1}). | Over a millennium previously, the Indus Valley Civilization had built cities of fired brick with standardized size and construction methods, but in the Buddha’s day most buildings were wood. This is the only brick building mentioned in the suttas, although brick construction is also discussed in the Vinaya. } 

Now\marginnote{2.1} at that time the venerables Anuruddha, Nandiya, and Kimbila were staying in the sal forest park at \textsanskrit{Gosiṅga}.\footnote{Anuruddha was the brother of Ānanda and \textsanskrit{Mahānāma} (\href{https://suttacentral.net/pli-tv-kd17/en/sujato\#1.1.3}{Kd 17:1.1.3}) which, according to the commentary, makes the Buddha his cousin. He was an exponent of mindfulness meditation (\href{https://suttacentral.net/sn52.1/en/sujato}{SN 52.1}, etc.) and master of clairvoyance (\href{https://suttacentral.net/an1.192/en/sujato\#1.1}{AN 1.192:1.1}. | Kimbila was one of the seven leading Sakyans, including Anuruddha and Ānanda, who went forth together (\href{https://suttacentral.net/pli-tv-kd17/en/sujato\#1.4.1}{Kd 17:1.4.1}). | Several Nandiyas are known in the suttas, one of whom was a Sakyan (\href{https://suttacentral.net/sn55.40/en/sujato}{SN 55.40}, \href{https://suttacentral.net/an11.13/en/sujato}{AN 11.13}; the root \textit{nand} (“joy”) was a popular choice for Sakyan names: Ānanda, Nanda, Upananda). While Nandiya is not mentioned as having gone forth with the seven, Kimbila refers to their group as the “Sakyan friends” in his verses at \href{https://suttacentral.net/thag2.18/en/sujato}{Thag 2.18}, so it seems likely he was indeed the Sakyan Nandiya; perhaps he went forth later than his friends. } 

Then\marginnote{3.1} in the late afternoon, the Buddha came out of retreat and went to that park. The park keeper saw the Buddha coming off in the distance\footnote{This is one of several instances showing that the Buddha looked like an ordinary ascetic (see also \href{https://suttacentral.net/dn2/en/sujato\#11.2}{DN 2:11.2}, \href{https://suttacentral.net/mn140/en/sujato\#5.11}{MN 140:5.11}). } and said to him, “Don’t come into this park, ascetic. There are three gentlemen staying here whose nature is to desire only the self.\footnote{\textit{\textsanskrit{Attakāmarūpa}} is glossed by the commentary as “those whose nature is to desire their own welfare”. Elsewhere for \textit{\textsanskrit{attakāma}} I have “who cares for their own welfare”, while in the Vinaya it means to serve “one’s own desires” (\href{https://suttacentral.net/pli-tv-bu-vb-ss4/en/sujato}{Bu Ss 4}). But here it draws on \textsanskrit{Yājnavalkya}’s description of one whose form is such that, having attained their only desire, the Self, they have become without desire (\textit{\textsanskrit{ātmakāmam} \textsanskrit{āptakāmam} \textsanskrit{akāmaṁ} \textsanskrit{rūpaṁ}}, Śatapatha \textsanskrit{Brāhmaṇa} 14.7.1.21, \textsanskrit{Bṛhadāraṇyaka} \textsanskrit{Upaniṣad} 4.3.21. Cf. \textit{\textsanskrit{niṣkāma} \textsanskrit{āptakāma} \textsanskrit{ātmakāmo}}, Śatapatha \textsanskrit{Brāhmaṇa} 14.7.2.8, \textsanskrit{Bṛhadāraṇyaka} \textsanskrit{Upaniṣad} 4.4.6. In both cases the \textsanskrit{Bṛhadāraṇyaka} reverses the sequence of \textit{\textsanskrit{āpta}} … \textit{\textsanskrit{ātma}}.) This describes a sage who is fully immersed in the realization of Brahman in death or dreamless sleep. Here the suffix \textit{-\textsanskrit{rūpa}} does not mean “seemingly”, nor is it to be discarded as a mere idiom, but has the sense given in the commentary, a person’s true form or true nature (\textit{\textsanskrit{sabhāva}}). This is, of course, not a Buddhist concept. But it is spoken by the keeper of a park (not a monastery) who fails to even recognize the Buddha. There is no reason he should be Buddhist. To this day, Buddhist monastics are received with kindness and unhesitatingly supported by Hindus, who readily interpret and describe the monastics’ practice in their own terms. } Do not disturb them.” 

Anuruddha\marginnote{4.1} heard the park keeper conversing with the Buddha, and said to him, “Don’t keep the Buddha out, good park keeper! Our Teacher, the Blessed One, has arrived.” Then Anuruddha went to Nandiya and Kimbila, and said to them, “Come forth, venerables, come forth! Our Teacher, the Blessed One, has arrived!” 

Then\marginnote{5.1} Anuruddha, Nandiya, and Kimbila came out to greet the Buddha. One received his bowl and robe, one spread out a seat, and one set out water for washing his feet. The Buddha sat on the seat spread out, and washed his feet. Those venerables bowed and sat down to one side. 

The\marginnote{5.6} Buddha said to Anuruddha, “I hope you’re keeping well, Anuruddha and friends; I hope you’re all right. And I hope you’re having no trouble getting almsfood.”\footnote{The text refers to “Anuruddhas” in plural. It was apparently the custom to refer to the group by their most senior member. } 

“We’re\marginnote{5.8} keeping well, Blessed One, we’re getting by. And we have no trouble getting almsfood.” 

“I\marginnote{6.1} hope you’re living in harmony, appreciating each other, without quarreling, blending like milk and water, and regarding each other with kindly eyes?”\footnote{The emphasis on harmonious living fits the accounts in \href{https://suttacentral.net/pli-tv-kd10/en/sujato\#4.2.1}{Kd 10:4.2.1} and \href{https://suttacentral.net/mn128/en/sujato}{MN 128}, where the Buddha had just left the quarreling community of \textsanskrit{Kosambī}. This suggests that the introduction belongs there, when the three friends were practicing together in the Eastern Bamboo Park near \textsanskrit{Kosambī}. The current sutta is set around 400 kms to the east near \textsanskrit{Vesālī}, and it is hardly likely that the Buddha went straight there after leaving \textsanskrit{Kosambī}. In addition, in \href{https://suttacentral.net/mn128/en/sujato}{MN 128} the three friends are still developing their meditation, whereas in the current sutta they have already achieved arahantship. It seems that the popular introduction of \href{https://suttacentral.net/mn128/en/sujato}{MN 128} was reused to frame a later conversation with the three friends, despite the implausibility of the same events happening in the same way twice. } 

“Indeed,\marginnote{6.2} sir, we live in harmony like this.” 

“But\marginnote{6.3} how do you live this way?” 

“In\marginnote{7.1} this case, sir, I think, ‘I’m fortunate, so very fortunate, to live together with spiritual companions such as these.’ I consistently treat these venerables with kindness by way of body, speech, and mind, both in public and in private.\footnote{These are three of the six “warm-hearted qualities” of \href{https://suttacentral.net/mn48/en/sujato\#6.2}{MN 48:6.2} and \href{https://suttacentral.net/an6.11/en/sujato\#1.1}{AN 6.11:1.1}. } I think, ‘Why don’t I set aside my own ideas and just go along with these venerables’ ideas?’ And that’s what I do. Though we’re different in body, sir, we’re one in mind, it seems to me.” 

And\marginnote{7.11} the venerables Nandiya and Kimbila spoke likewise, and they added: “That’s how we live in harmony, appreciating each other, without quarreling, blending like milk and water, and regarding each other with kindly eyes.” 

“Good,\marginnote{8.1} good, Anuruddha and friends! But I hope you’re living diligently, keen, and resolute?”\footnote{Normally I render \textit{viharati} in such instances as “meditate”. But Anuruddha’s response avoids talking about meditation. I think he is deliberately avoiding the topic, skirting the obvious intended meaning by seizing on an ambiguity. His reticence echoes \textsanskrit{Yājñavalkya}’s reticence to answer Janaka on the topic of a person’s light (\textsanskrit{Bṛhadāraṇyaka} \textsanskrit{Upaniṣad} 4.3.1). } 

“Indeed,\marginnote{9.1} sir, we live diligently.” 

“But\marginnote{9.2} how do you live this way?” 

“In\marginnote{9.3} this case, sir, whoever returns first from almsround prepares the seats, and puts out the drinking water and the rubbish bin. If there’s anything left over, whoever returns last eats it if they like. Otherwise they throw it out where there is little that grows, or drop it into water that has no living creatures. Then they put away the seats, drinking water, and rubbish bin, and sweep the refectory. If someone sees that the pot of water for washing, drinking, or the toilet is empty they set it up. If he can’t do it, he summons another with a wave of the hand, and they set it up by lending each other a hand to lift. But we don’t break into speech for that reason.\footnote{For \textit{\textsanskrit{hatthavilaṅghaka}} (“lending a hand up”) see \href{https://suttacentral.net/mn125/en/sujato\#9.2}{MN 125:9.2}. | The Chinese parallels (MA 128 at T i 729c9 and EA 24.8 at T ii 629a23) both add that they then retire for meditation. } And every five days we sit together for the whole night and discuss the teachings.\footnote{That is, three times in each \textit{uposatha} period lasting a fortnight. | The Chinese parallels add that they might also sit in noble silence (in accord with \href{https://suttacentral.net/mn26/en/sujato\#4.13}{MN 26:4.13}). } That’s how we live diligently, keen, and resolute.”\footnote{Here the Vinaya account ends, as it is concerned with the settlement of the quarrel among the monks of \textsanskrit{Kosambī}. } 

“Good,\marginnote{10.1} good, Anuruddha and friends! But as you live diligently like this, have you achieved any superhuman distinction in knowledge and vision worthy of the noble ones, a comfortable meditation?” 

“How\marginnote{10.3} could we not, sir? Whenever we want, quite secluded from sensual pleasures, secluded from unskillful qualities, we enter and remain in the first absorption, which has the rapture and bliss born of seclusion, while placing the mind and keeping it connected.\footnote{The first absorption is defined at \href{https://suttacentral.net/pli-tv-bu-vb-pj4/en/sujato\#4.1.3.1}{Bu Pj 4:4.1.3.1} as the most basic “superhuman distinction”. Thus Anuruddha is answering the question directly but humbly, revealing as little as possible. } This is a superhuman distinction in knowledge and vision worthy of the noble ones, a comfortable meditation, that we have achieved while living diligent, keen, and resolute.” 

“Good,\marginnote{11{-}13.1} good! But have you achieved any other superhuman distinction for going beyond and stilling that meditation?”\footnote{\textit{\textsanskrit{Samatikkamāya}} (“for going beyond”) is used in a similar sense at \href{https://suttacentral.net/mn10/en/sujato\#47.1}{MN 10:47.1}. } 

“How\marginnote{11{-}13.3} could we not, sir? Whenever we want, as the placing of the mind and keeping it connected are stilled, we enter and remain in the second absorption, which has the rapture and bliss born of immersion, with internal clarity and mind at one, without placing the mind and keeping it connected. This is another superhuman distinction that we have achieved for going beyond and stilling that meditation.” 

“Good,\marginnote{14.1} good! But have you achieved any other superhuman distinction for going beyond and stilling that meditation?” 

“How\marginnote{14.3} could we not, sir? Whenever we want, with the fading away of rapture, we enter and remain in the third absorption, where we meditate with equanimity, mindful and aware, personally experiencing the bliss of which the noble ones declare, ‘Equanimous and mindful, one meditates in bliss.’ This is another superhuman distinction that we have achieved for going beyond and stilling that meditation.” 

“Good,\marginnote{15.1} good! But have you achieved any other superhuman distinction for going beyond and stilling that meditation?” 

“How\marginnote{15.3} could we not, sir? Whenever we want, with the giving up of pleasure and pain, and the ending of former happiness and sadness, we enter and remain in the fourth absorption, without pleasure or pain, with pure equanimity and mindfulness. This is another superhuman distinction that we have achieved for going beyond and stilling that meditation.” 

“Good,\marginnote{16.1} good! But have you achieved any other superhuman distinction for going beyond and stilling that meditation?” 

“How\marginnote{16.3} could we not, sir? Whenever we want, going totally beyond perceptions of form, with the ending of perceptions of impingement, not focusing on perceptions of diversity, aware that ‘space is infinite’, we enter and remain in the dimension of infinite space. This is another superhuman distinction that we have achieved for going beyond and stilling that meditation.” 

“Good,\marginnote{17.1} good! But have you achieved any other superhuman distinction for going beyond and stilling that meditation?” 

“How\marginnote{17.3} could we not, sir? Whenever we want, going totally beyond the dimension of infinite space, aware that ‘consciousness is infinite’, we enter and remain in the dimension of infinite consciousness. … going totally beyond the dimension of infinite consciousness, aware that ‘there is nothing at all’, we enter and remain in the dimension of nothingness. … going totally beyond the dimension of nothingness, we enter and remain in the dimension of neither perception nor non-perception. This is another superhuman distinction that we have achieved for going beyond and stilling that meditation.” 

“Good,\marginnote{18.1} good! But have you achieved any other superhuman distinction for going beyond and stilling that meditation?” 

“How\marginnote{18.3} could we not, sir? Whenever we want, going totally beyond the dimension of neither perception nor non-perception, we enter and remain in the cessation of perception and feeling. And, having seen with wisdom, our defilements have come to an end. This is another superhuman distinction in knowledge and vision worthy of the noble ones, a comfortable meditation, that we have achieved for going beyond and stilling that meditation. And we don’t see any better or finer way of meditating comfortably than this.” 

“Good,\marginnote{18.7} good! There is no better or finer way of meditating comfortably than this.” 

Then\marginnote{19.1} the Buddha educated, encouraged, fired up, and inspired the venerables Anuruddha, Nandiya, and Kimbila with a Dhamma talk, after which he got up from his seat and left. 

The\marginnote{20.1} venerables then accompanied the Buddha for a little way before turning back. Nandiya and Kimbila said to Anuruddha, “Did we ever tell you that we had gained such and such meditations and attainments, up to the ending of defilements, as you revealed to the Buddha?”\footnote{Even such close monks did not reveal their attainments to each other, a testament to their character. This is a reminder of the virtue of discretion when speaking of such subtle matters. They are not things to be bandied about in common conversation. } 

“The\marginnote{20.4} venerables did not tell me that they had gained such meditations and attainments. But I discovered it by comprehending your minds, and deities also told me. I answered when the Buddha directly asked about it.” 

Then\marginnote{21.1} the native spirit \textsanskrit{Dīgha} Parajana went up to the Buddha, bowed, stood to one side, and said to him,\footnote{\textit{Parajana} means “stranger” (\href{https://suttacentral.net/mil6.3.1/en/sujato\#13.2}{Mil 6.3.1:13.2}) or “strange spirit” (\href{https://suttacentral.net/mn25/en/sujato\#5.20}{MN 25:5.20}). The commentary identifies him with the \textsanskrit{Dīgha} of \href{https://suttacentral.net/dn32/en/sujato\#10.18}{DN 32:10.18}. He must have been the deity of a local shrine. This passage shows how the intercession of even such a humble spirit can resound among all the gods. } “The Vajjis are lucky! The Vajjian people are so very lucky\footnote{The praise that follows is unusually effusive and can be compared with the equally unusual recounting of the enlightened devotees of \textsanskrit{Ñātika} (\href{https://suttacentral.net/dn16/en/sujato\#2.5.1}{DN 16:2.5.1}). Both passages serve to extol Buddhism in the home of \textsanskrit{Mahāvīra}, the Buddha’s elder rival. } that the Realized One, the perfected one, the fully awakened Buddha stays there, as well as these three gentlemen, the venerables Anuruddha, Nandiya, and Kimbila.” 

Hearing\marginnote{21.5} the cry of \textsanskrit{Dīgha} Parajana, the earth gods raised the cry … 

Hearing\marginnote{21.9} the cry of the earth gods, the gods of the four great kings … the gods of the thirty-three … the gods of Yama … the joyful gods … the gods who love to imagine … the gods who control what is imagined by others … the gods of the Divinity’s host raised the cry, “The Vajjis are lucky! The Vajjian people are so very lucky that the Realized One, the perfected one, the fully awakened Buddha stays there, as well as these three gentlemen, the venerables Anuruddha, Nandiya, and Kimbila.” 

And\marginnote{21.19} so at that moment, that second, that hour, those venerables were known as far as the realm of divinity. 

“That’s\marginnote{22.1} so true, \textsanskrit{Dīgha}! That’s so true! If the family from which those three gentlemen went forth from the lay life to homelessness were to recollect those venerables with confident heart, that would be for that family’s lasting welfare and happiness. If the family circle … village … town … city … country … all the aristocrats … all the brahmins … all the peasants … all the workers were to recollect those venerables with confident heart, that would be for all those menials’ lasting welfare and happiness. 

If\marginnote{22.12} the whole world—with its gods, \textsanskrit{Māras}, and divinities, this population with its ascetics and brahmins, gods and humans—were to recollect those venerables with confident heart, that would be for the whole world’s lasting welfare and happiness. 

See,\marginnote{22.13} \textsanskrit{Dīgha}, how those three gentlemen are practicing for the welfare and happiness of the people, out of sympathy for the world, for the benefit, welfare, and happiness of gods and humans!” 

That\marginnote{22.14} is what the Buddha said. Satisfied, the native spirit \textsanskrit{Dīgha} Parajana approved what the Buddha said. 

%
\section*{{\suttatitleacronym MN 32}{\suttatitletranslation The Longer Discourse at Gosiṅga }{\suttatitleroot Mahāgosiṅgasutta}}
\addcontentsline{toc}{section}{\tocacronym{MN 32} \toctranslation{The Longer Discourse at Gosiṅga } \tocroot{Mahāgosiṅgasutta}}
\markboth{The Longer Discourse at Gosiṅga }{Mahāgosiṅgasutta}
\extramarks{MN 32}{MN 32}

\scevam{So\marginnote{1.1} I have heard. }At one time the Buddha was staying in the sal forest park at \textsanskrit{Gosiṅga}, together with several well-known senior disciples, such as\footnote{From \href{https://suttacentral.net/mn31/en/sujato\#3.1}{MN 31:3.1} we know that this park was near \textsanskrit{Ñātika} in the Vajjian country. } the venerables \textsanskrit{Sāriputta}, \textsanskrit{Mahāmoggallāna}, \textsanskrit{Mahākassapa}, Anuruddha, Revata, Ānanda, and others. 

Then\marginnote{2.1} in the late afternoon, Venerable \textsanskrit{Mahāmoggallāna} came out of retreat, went to Venerable \textsanskrit{Mahākassapa}, and said, “Come, Reverend Kassapa, let’s go to Venerable \textsanskrit{Sāriputta} to hear the teaching.” 

“Yes,\marginnote{2.3} reverend,” \textsanskrit{Mahākassapa} replied. Then, together with Venerable Anuruddha, they went to \textsanskrit{Sāriputta} to hear the teaching. 

Seeing\marginnote{3.1} them, Venerable Ānanda went to Venerable Revata, told him what was happening, and invited him also. 

\textsanskrit{Sāriputta}\marginnote{4.1} saw them coming off in the distance and said to Ānanda, “Come, Venerable Ānanda. Welcome to Ānanda, the Buddha’s attendant, who is so close to the Buddha. Ānanda, the sal forest park at \textsanskrit{Gosiṅga} is lovely, the night is bright, the sal trees are in full blossom, and heavenly scents seem to float on the air. What kind of mendicant would beautify this park?” 

“Reverend\marginnote{4.7} \textsanskrit{Sāriputta}, it’s a mendicant who is very learned, remembering and keeping what they’ve learned. These teachings are good in the beginning, good in the middle, and good in the end, meaningful and well-phrased, describing a spiritual practice that’s entirely full and pure. They are very learned in such teachings, remembering them, rehearsing them, mentally scrutinizing them, and comprehending them theoretically.\footnote{Ānanda was renowned as the most learned monk in the Sangha (\href{https://suttacentral.net/an1.219/en/sujato}{AN 1.219}). } And they teach the four assemblies in order to uproot the underlying tendencies with well-rounded and coherent words and phrases.\footnote{Typically inclusive, Ānanda mentions the four assemblies of monks, nuns, laymen, and laywomen. } That’s the kind of mendicant who would beautify this park.” 

When\marginnote{5.1} he had spoken, \textsanskrit{Sāriputta} said to Revata, “Reverend Revata, Ānanda has answered by speaking from his heart. And now we ask you the same question.” 

“Reverend\marginnote{5.6} \textsanskrit{Sāriputta}, it’s a mendicant who enjoys retreat and loves retreat. They’re committed to inner serenity of the heart, they don’t neglect absorption, they’re endowed with discernment, and they frequent empty huts.\footnote{There were two prominent Revatas in the early Sangha. There was Revata of the Acacia Wood, who was said to be the foremost of forest dwellers (\href{https://suttacentral.net/an1.203/en/sujato}{AN 1.203}, \href{https://suttacentral.net/thag14.1/en/sujato}{Thag 14.1}, \href{https://suttacentral.net/thag1.42/en/sujato}{Thag 1.42}). But this is probably Revata the Doubter, who was the most skilled in \textit{\textsanskrit{jhāna}} (\href{https://suttacentral.net/an1.204/en/sujato}{AN 1.204}, \href{https://suttacentral.net/thag1.3/en/sujato}{Thag 1.3}). Early in his monastic life he struggled with doubt over small maters (\href{https://suttacentral.net/pli-tv-kd6/en/sujato\#16.1.1}{Kd 6:16.1.1}). He overcame this (\href{https://suttacentral.net/ud5.7/en/sujato}{Ud 5.7}), but the name stuck. } That’s the kind of mendicant who would beautify this park.” 

When\marginnote{6.1} he had spoken, \textsanskrit{Sāriputta} said to Anuruddha, “Reverend Anuruddha, Revata has answered by speaking from his heart. And now we ask you the same question.” 

“Reverend\marginnote{6.6} \textsanskrit{Sāriputta}, it’s a mendicant who surveys a thousandfold galaxy with clairvoyance that is purified and surpasses the human,\footnote{Anuruddha is identified as foremost in clairvoyance at \href{https://suttacentral.net/an1.192/en/sujato}{AN 1.192}. | The verb \textit{voloketi} (“surveys”) is an elevated usage, only used of Anuruddha’s ability (see \href{https://suttacentral.net/an3.130/en/sujato\#2.2}{AN 3.130:2.2}) and when describing a Buddha “surveying” the world (\href{https://suttacentral.net/mn26/en/sujato\#21.1}{MN 26:21.1}). } just as a person with clear eyes could survey a thousand orbits from the upper floor of a royal longhouse.\footnote{This power is compared (in the Majjhima) with a man standing between two doors watching people come to and fro (\href{https://suttacentral.net/mn39/en/sujato\#20.3}{MN 39:20.3}, \href{https://suttacentral.net/mn77/en/sujato\#35.2}{MN 77:35.2}, \href{https://suttacentral.net/mn130/en/sujato\#2.1}{MN 130:2.1}), and (in the \textsanskrit{Dīgha}) with someone in a longhouse watching people move about in the town square (\href{https://suttacentral.net/dn2/en/sujato\#96.1}{DN 2:96.1}, \href{https://suttacentral.net/dn10/en/sujato\#2.33.1}{DN 10:2.33.1}). The unique phrasing here evokes the Wheel that manifests to a Wheel-Turning Monarch as he stands upstairs in the royal longhouse, which is thousand-spoked and complete with rim (\textit{\textsanskrit{sahassāraṁ} \textsanskrit{sanemikaṁ}}, \href{https://suttacentral.net/mn129/en/sujato\#34.2}{MN 129:34.2}, \href{https://suttacentral.net/dn26/en/sujato\#4.9}{DN 26:4.9}). The simile was evidently unclear to the ancients, as one Chinese parallel has instead “a thousand bricks” (presumably laid out to dry in the sun, MA 184 at T i 727b14). In the Pali tradition, \textit{\textsanskrit{nemimaṇḍala}} is the outer cladding of a wheel rim, often iron (see commentary to \href{https://suttacentral.net/ja475/en/sujato}{Ja 475}). However, normally sutta similes draw directly from experience, and it is hard to imagine how one would survey a thousand wheel rims (unless perhaps a king overseeing chariot construction in his forecourt?) On the other hand, the Wheel is a solar symbol and \textit{\textsanskrit{nemimaṇḍala}} could easily refer to the orbits of the stars, which would explain “looking at the sky” of EA 37.3 ( at T ii 711a3). On balance, I think the simile refers to a star-gazing king. } That’s the kind of mendicant who would beautify this park.” 

When\marginnote{7.1} he had spoken, \textsanskrit{Sāriputta} said to \textsanskrit{Mahākassapa}, “Reverend Kassapa, Anuruddha has answered by speaking from his heart. And now we ask you the same question.” 

“Reverend\marginnote{7.6} \textsanskrit{Sāriputta}, it’s a mendicant who lives in the wilderness, eats only almsfood, wears rag robes, and owns just three robes; and they praise these things. They are of few wishes, content, secluded, aloof, and energetic; and they praise these things. They are accomplished in ethics, immersion, wisdom, freedom, and the knowledge and vision of freedom; and they praise these things.\footnote{\textsanskrit{Mahākassapa}’s austere lifestyle was renowned (\href{https://suttacentral.net/an1.191/en/sujato}{AN 1.191}). } That’s the kind of mendicant who would beautify this park.” 

When\marginnote{8.1} he had spoken, \textsanskrit{Sāriputta} said to \textsanskrit{Mahāmoggallāna}, “Reverend \textsanskrit{Moggallāna}, \textsanskrit{Mahākassapa} has answered by speaking from his heart. And now we ask you the same question.” 

“Reverend\marginnote{8.6} \textsanskrit{Sāriputta}, it’s when two mendicants engage in discussion about the teaching. They question each other and answer each other’s questions without faltering, and their discussion on the teaching flows on.\footnote{In the suttas, \textit{abhidhamma} means “about the teaching”. Here the prefix \textit{abhi-} conveys the same sense as the English “meta-”. | This is a surprising topic for \textsanskrit{Moggallāna}, and all three Chinese parallels as well as Sanskrit fragments say \textsanskrit{Moggallāna} praised psychic powers, in which he was the recognized expert (\href{https://suttacentral.net/an1.190/en/sujato}{AN 1.190}). It seems certain the Pali is corrupt here, and MA 185 more plausibly attributes discussion on the teaching to \textsanskrit{Mahākaccāna}. } That’s the kind of mendicant who would beautify this park.” 

Then\marginnote{9.1} \textsanskrit{Mahāmoggallāna} said to \textsanskrit{Sāriputta}, “Each of us has spoken from our heart. And now we ask you: \textsanskrit{Sāriputta}, the sal forest park at \textsanskrit{Gosiṅga} is lovely, the night is bright, the sal trees are in full blossom, and heavenly scents seem to float on the air. What kind of mendicant would beautify this park?” 

“Reverend\marginnote{9.6} \textsanskrit{Moggallāna}, it’s when a mendicant masters their mind and is not mastered by it.\footnote{While one might expect \textsanskrit{Sāriputta} to praise wisdom, all the parallels agree on this point. Indeed, \textsanskrit{Sāriputta} also speaks in a similar way on mental mastery at \href{https://suttacentral.net/sn46.4/en/sujato}{SN 46.4}. } In the morning, they abide in whatever meditation or attainment they want. At midday, and in the evening, they abide in whatever meditation or attainment they want. Suppose that a ruler or their chief minister had a chest full of garments of different colors. In the morning, they’d don whatever pair of garments they wanted. At midday, and in the evening, they’d don whatever pair of garments they wanted. 

In\marginnote{9.14} the same way, a mendicant masters their mind and is not mastered by it. In the morning, they abide in whatever meditation or attainment they want. At midday, and in the evening, they abide in whatever meditation or attainment they want. That’s the kind of mendicant who would beautify this park.” 

Then\marginnote{10.1} \textsanskrit{Sāriputta} said to those venerables, “Each of us has spoken from the heart. Come, reverends, let’s go to the Buddha, and inform him about this. As he answers, so we’ll remember it.” 

“Yes,\marginnote{10.5} reverend,” they replied. Then those venerables went to the Buddha, bowed, and sat down to one side. Venerable \textsanskrit{Sāriputta} told the Buddha of how the mendicants had come to see him, and how he had asked Ānanda: “‘Ānanda, the sal forest park at \textsanskrit{Gosiṅga} is lovely, the night is bright, the sal trees are in full blossom, and heavenly scents seem to float on the air. What kind of mendicant would beautify this park?’ When I had spoken, Ānanda said to me: ‘Reverend \textsanskrit{Sāriputta}, it’s a mendicant who is very learned … That’s the kind of mendicant who would beautify this park.’” 

“Good,\marginnote{11.12} good, \textsanskrit{Sāriputta}! Ānanda answered in the right way for him. For Ānanda is very learned …” 

“Next\marginnote{12.1} I asked Revata the same question. He said: ‘It’s a mendicant who enjoys retreat … That’s the kind of mendicant who would beautify this park.’” 

“Good,\marginnote{12.9} good, \textsanskrit{Sāriputta}! Revata answered in the right way for him. For Revata enjoys retreat …” 

“Next\marginnote{13.1} I asked Anuruddha the same question. He said: ‘It’s a mendicant who surveys the thousandfold galaxy with clairvoyance that is purified and surpasses the human … That’s the kind of mendicant who would beautify this park.’” 

“Good,\marginnote{13.8} good, \textsanskrit{Sāriputta}! Anuruddha answered in the right way for him. For Anuruddha surveys the thousandfold galaxy with clairvoyance that is purified and surpasses the human.” 

“Next\marginnote{14.1} I asked \textsanskrit{Mahākassapa} the same question. He said: ‘It’s a mendicant who lives in the wilderness … and is accomplished in the knowledge and vision of freedom; and they praise these things. That’s the kind of mendicant who would beautify this park.’” 

“Good,\marginnote{14.8} good, \textsanskrit{Sāriputta}! Kassapa answered in the right way for him. For Kassapa lives in the wilderness … and is accomplished in the knowledge and vision of freedom; and he praises these things.” 

“Next\marginnote{15.1} I asked \textsanskrit{Mahāmoggallāna} the same question. He said: ‘It’s when two mendicants engage in discussion about the teaching … That’s the kind of mendicant who would beautify this park.’” 

“Good,\marginnote{15.8} good, \textsanskrit{Sāriputta}! \textsanskrit{Moggallāna} answered in the right way for him. For \textsanskrit{Moggallāna} is a Dhamma speaker.” 

When\marginnote{16.1} he had spoken, \textsanskrit{Moggallāna} said to the Buddha, “Next, I asked \textsanskrit{Sāriputta}: ‘Each of us has spoken from our heart. And now we ask you: \textsanskrit{Sāriputta}, the sal forest park at \textsanskrit{Gosiṅga} is lovely, the night is bright, the sal trees are in full blossom, and heavenly scents seem to float on the air. What kind of mendicant would beautify this park?’ When I had spoken, \textsanskrit{Sāriputta} said to me: ‘Reverend \textsanskrit{Moggallāna}, it’s when a mendicant masters their mind and is not mastered by it … That’s the kind of mendicant who would beautify this park.’” 

“Good,\marginnote{16.21} good, \textsanskrit{Moggallāna}! \textsanskrit{Sāriputta} answered in the right way for him. For \textsanskrit{Sāriputta} masters his mind and is not mastered by it …” 

When\marginnote{17.1} he had spoken, \textsanskrit{Sāriputta} asked the Buddha, “Sir, who has spoken well?” 

“You’ve\marginnote{17.3} all spoken well in your own way.\footnote{The Buddha gives positive reinforcement, not only to the wholesome qualities of others, but to the fact that they appreciate their own good qualities. This is not conceit, but a form of rejoicing (\textit{\textsanskrit{muditā}}). } However, listen to me also as to what kind of mendicant would beautify this sal forest park at \textsanskrit{Gosiṅga}. It’s a mendicant who, after the meal, returns from almsround, sits down cross-legged, sets their body straight, and establishes mindfulness in their presence, thinking:\footnote{While the others spoke of results, the Buddha speaks of causes. Compare \href{https://suttacentral.net/mn123/en/sujato\#22.1}{MN 123:22.1}, where the Buddha similarly brings the conversation back to meditation practice. } ‘I will not break this sitting posture until my mind is freed from the defilements by not grasping!’ That’s the kind of mendicant who would beautify this park.” 

That\marginnote{17.8} is what the Buddha said. Satisfied, those venerables approved what the Buddha said. 

%
\section*{{\suttatitleacronym MN 33}{\suttatitletranslation The Longer Discourse on the Cowherd }{\suttatitleroot Mahāgopālakasutta}}
\addcontentsline{toc}{section}{\tocacronym{MN 33} \toctranslation{The Longer Discourse on the Cowherd } \tocroot{Mahāgopālakasutta}}
\markboth{The Longer Discourse on the Cowherd }{Mahāgopālakasutta}
\extramarks{MN 33}{MN 33}

\scevam{So\marginnote{1.1} I have heard. }At one time the Buddha was staying near \textsanskrit{Sāvatthī} in Jeta’s Grove, \textsanskrit{Anāthapiṇḍika}’s monastery. There the Buddha addressed the mendicants, “Mendicants!” 

“Venerable\marginnote{1.5} sir,” they replied. The Buddha said this: 

“Mendicants,\marginnote{2.1} a cowherd with eleven factors can’t maintain and expand a herd of cattle.\footnote{Since organizing teachings by number is the hallmark of the \textsanskrit{Aṅguttara} \textsanskrit{Nikāya}, it comes as no surprise to see this pattern repeated in the Book of the Elevens at \href{https://suttacentral.net/an11.17/en/sujato}{AN 11.17}, \href{https://suttacentral.net/an11.22-29/en/sujato}{AN 11.22–29}, and \href{https://suttacentral.net/an11.502-981/en/sujato}{AN 11.502–981}. } What eleven? It’s when a cowherd doesn’t know form, is unskilled in characteristics, doesn’t pick out flies’ eggs, doesn’t dress wounds, doesn’t spread smoke, doesn’t know the ford, doesn’t know satisfaction, doesn’t know the trail, is not skilled in ranges, milks dry, and doesn’t show extra respect to the bulls who are fathers and leaders of the herd.\footnote{“Knowing form” (\textit{\textsanskrit{rūpaññū} }) means being able to keep track of the herd by reckoning or recognizing them. A better translation would be “appearance”, but I use “form” for consistency. | “Ford” (\textit{tittha}) is a way across the river. | “Spreads smoke” (\textit{\textsanskrit{dhūmaṁ} \textsanskrit{kattā}}) is the traditional practice of warding off flies and other  pests by burning cow dung, etc. | “Satisfy” (\textit{\textsanskrit{pīta}}) is less literal than “drink”, but clarifies the simile later on. } A cowherd with these eleven factors can’t maintain and expand a herd of cattle. 

In\marginnote{3.1} the same way, a mendicant with eleven qualities can’t achieve growth, improvement, or maturity in this teaching and training. What eleven? It’s when a mendicant doesn’t know form, is unskilled in characteristics, doesn’t pick out flies’ eggs, doesn’t dress wounds, doesn’t spread smoke, doesn’t know the ford, doesn’t know satisfaction, doesn’t know the trail, is not skilled in ranges, milks dry, and doesn’t show extra respect to senior mendicants of long standing, long gone forth, fathers and leaders of the \textsanskrit{Saṅgha}. 

And\marginnote{4.1} how does a mendicant not know form? It’s when a mendicant doesn’t truly understand that all form is the four principal states, or form derived from the four principal states. That’s how a mendicant doesn’t know form. 

And\marginnote{5.1} how is a mendicant not skilled in characteristics? It’s when a mendicant doesn’t understand that a fool is characterized by their deeds, and an astute person is characterized by their deeds. That’s how a mendicant isn’t skilled in characteristics. 

And\marginnote{6.1} how does a mendicant not pick out flies’ eggs? It’s when a mendicant tolerates a sensual, malicious, or cruel thought that has arisen. They tolerate any bad, unskillful qualities that have arisen. They don’t give them up, get rid of them, eliminate them, and obliterate them. That’s how a mendicant doesn’t pick out flies’ eggs. 

And\marginnote{7.1} how does a mendicant not dress wounds? When a mendicant sees a sight with their eyes, they get caught up in the features and details. Since the faculty of sight is left unrestrained, bad unskillful qualities of covetousness and displeasure become overwhelming. They don’t practice restraint, they don’t protect the faculty of sight, and they don’t achieve its restraint. When they hear a sound with their ears … smell an odor with their nose … taste a flavor with their tongue … feel a touch with their body … know an idea with their mind, they get caught up in the features and details. Since the faculty of the mind is left unrestrained, bad unskillful qualities of covetousness and displeasure become overwhelming. They don’t practice restraint, they don’t protect the faculty of the mind, and they don’t achieve its restraint. That’s how a mendicant doesn’t dress wounds. 

And\marginnote{8.1} how does a mendicant not spread smoke? It’s when a mendicant doesn’t teach others the Dhamma in detail as they learned and memorized it. That’s how a mendicant doesn’t spread smoke. 

And\marginnote{9.1} how does a mendicant not know the ford?\footnote{A \textit{tittha} (“ford”) is a path to salvation, usually used of non-Buddhist religions, whose founders are “ford-makers” (\textit{\textsanskrit{titthakāra}}). } It’s when a mendicant doesn’t from time to time go up to those mendicants who are very learned—inheritors of the heritage, who have memorized the teachings, the monastic law, and the outlines—and ask them questions:\footnote{“Inheritors of the heritage” is \textit{\textsanskrit{āgatāgamā}}, where \textit{\textsanskrit{āgama}} means “what has come down”, namely the scriptural heritage. \textit{Āgama} is a synonym for \textit{\textsanskrit{nikāya}} in the sense of “collection of scripture”. | The “outlines” (\textit{\textsanskrit{mātikā}}, literally “matrix”) are the summary outlines of topics that served as seeds for the development of Abhidhamma. \href{https://suttacentral.net/dn16/en/sujato\#3.50.5}{DN 16:3.50.5} features one of the earliest of such lists, the 37 path factors that the Buddha “taught from his own direct knowledge”. These serve as outline for the section on the path in the \textsanskrit{Saṁyutta}, from where they were adopted in various Abhidhamma texts such as the \textsanskrit{Vibhaṅga}. } ‘Why, sir, does it say this? What does that mean?’ Those venerables don’t clarify what is unclear, reveal what is obscure, and dispel doubt regarding the many doubtful matters. That’s how a mendicant doesn’t know the ford. 

And\marginnote{10.1} how does a mendicant not know satisfaction? It’s when a mendicant, when the teaching and training proclaimed by the Realized One are being taught, finds no inspiration in the meaning and the teaching, and finds no joy connected with the teaching.\footnote{This plays on the similarity of \textit{\textsanskrit{pīta}} (“to have drunk, to have found satisfaction”) and \textit{\textsanskrit{pīti}} (“rapture, joy”); see also \href{https://suttacentral.net/snp2.3/en/sujato\#5.1}{Snp 2.3:5.1}. } That’s how a mendicant doesn’t know satisfaction. 

And\marginnote{11.1} how does a mendicant not know the trail? It’s when a mendicant doesn’t truly understand the noble eightfold path. That’s how a mendicant doesn’t know the trail. 

And\marginnote{12.1} how is a mendicant not skilled in ranges? It’s when a mendicant doesn’t truly understand the four kinds of mindfulness meditation.\footnote{See \href{https://suttacentral.net/sn47.6 /en/sujato}{SN 47.6 }. } That’s how a mendicant is not skilled in ranges. 

And\marginnote{13.1} how does a mendicant milk dry? It’s when a mendicant is invited by a householder to accept robes, almsfood, lodgings, and medicines and supplies for the sick, and that mendicant doesn’t know moderation in accepting. That’s how a mendicant milks dry. 

And\marginnote{14.1} how does a mendicant not show extra respect to senior mendicants of long standing, long gone forth, fathers and leaders of the \textsanskrit{Saṅgha}? It’s when a mendicant doesn’t consistently treat senior mendicants of long standing, long gone forth, fathers and leaders of the \textsanskrit{Saṅgha} with kindness by way of body, speech, and mind, both in public and in private. That’s how a mendicant doesn’t show extra respect to senior mendicants of long standing, long gone forth, fathers and leaders of the \textsanskrit{Saṅgha}. 

A\marginnote{14.6} mendicant with these eleven qualities can’t achieve growth, improvement, or maturity in this teaching and training. 

A\marginnote{15.1} cowherd with eleven factors can maintain and expand a herd of cattle. What eleven? It’s when a cowherd knows form, is skilled in characteristics, picks out flies’ eggs, dresses wounds, spreads smoke, knows the ford, knows satisfaction, knows the trail, is skilled in ranges, doesn’t milk dry, and shows extra respect to the bulls who are fathers and leaders of the herd. A cowherd with these eleven factors can maintain and expand a herd of cattle. 

In\marginnote{16.1} the same way, a mendicant with eleven qualities can achieve growth, improvement, and maturity in this teaching and training. What eleven? It’s when a mendicant knows form, is skilled in characteristics, picks out flies’ eggs, dresses wounds, spreads smoke, knows the ford, knows satisfaction, knows the trail, is skilled in ranges, doesn’t milk dry, and shows extra respect to senior mendicants of long standing, long gone forth, fathers and leaders of the \textsanskrit{Saṅgha}. 

And\marginnote{17.1} how does a mendicant know form? It’s when a mendicant truly understands that all form is the four principal states, or form derived from the four principal states. That’s how a mendicant knows form. 

And\marginnote{18.1} how is a mendicant skilled in characteristics? It’s when a mendicant understands that a fool is characterized by their deeds, and an astute person is characterized by their deeds. That’s how a mendicant is skilled in characteristics. 

And\marginnote{19.1} how does a mendicant pick out flies’ eggs? It’s when a mendicant doesn’t tolerate a sensual, malicious, or cruel thought that has arisen. They don’t tolerate any bad, unskillful qualities that have arisen, but give them up, get rid of them, eliminate them, and obliterate them. That’s how a mendicant picks out flies’ eggs. 

And\marginnote{20.1} how does a mendicant dress wounds? When a mendicant sees a sight with their eyes, they don’t get caught up in the features and details. If the faculty of sight were left unrestrained, bad unskillful qualities of covetousness and displeasure would become overwhelming. For this reason, they practice restraint, protecting the faculty of sight, and achieving its restraint. When they hear a sound with their ears … smell an odor with their nose … taste a flavor with their tongue … feel a touch with their body … know an idea with their mind, they don’t get caught up in the features and details. If the faculty of mind were left unrestrained, bad unskillful qualities of covetousness and displeasure would become overwhelming. For this reason, they practice restraint, protecting the faculty of mind, and achieving its restraint. That’s how a mendicant dresses wounds. 

And\marginnote{21.1} how does a mendicant spread smoke? It’s when a mendicant teaches others the Dhamma in detail as they learned and memorized it. That’s how a mendicant spreads smoke. 

And\marginnote{22.1} how does a mendicant know the ford? It’s when from time to time a mendicant goes up to those mendicants who are very learned—inheritors of the heritage, who have memorized the teachings, the monastic law, and the outlines—and asks them questions: ‘Why, sir, does it say this? What does that mean?’ Those venerables clarify what is unclear, reveal what is obscure, and dispel doubt regarding the many doubtful matters. That’s how a mendicant knows the ford. 

And\marginnote{23.1} how does a mendicant know satisfaction? It’s when a mendicant, when the teaching and training proclaimed by the Realized One are being taught, finds inspiration in the meaning and the teaching, and finds joy connected with the teaching. That’s how a mendicant knows satisfaction. 

And\marginnote{24.1} how does a mendicant know the trail? It’s when a mendicant truly understands the noble eightfold path. That’s how a mendicant knows the trail. 

And\marginnote{25.1} how is a mendicant skilled in ranges? It’s when a mendicant truly understands the four kinds of mindfulness meditation. That’s how a mendicant is skilled in ranges. 

And\marginnote{26.1} how does a mendicant not milk dry? It’s when a mendicant is invited by a householder to accept robes, almsfood, lodgings, and medicines and supplies for the sick, and that mendicant knows moderation in accepting. That’s how a mendicant doesn’t milk dry. 

And\marginnote{27.1} how does a mendicant show extra respect to senior mendicants of long standing, long gone forth, fathers and leaders of the \textsanskrit{Saṅgha}? It’s when a mendicant consistently treats senior mendicants of long standing, long gone forth, fathers and leaders of the \textsanskrit{Saṅgha} with kindness by way of body, speech, and mind, both in public and in private. That’s how a mendicant shows extra respect to senior mendicants of long standing, long gone forth, fathers and leaders of the \textsanskrit{Saṅgha}. 

A\marginnote{27.6} mendicant with these eleven qualities can achieve growth, improvement, and maturity in this teaching and training.” 

That\marginnote{27.7} is what the Buddha said. Satisfied, the mendicants approved what the Buddha said. 

%
\section*{{\suttatitleacronym MN 34}{\suttatitletranslation The Shorter Discourse on the Cowherd }{\suttatitleroot Cūḷagopālakasutta}}
\addcontentsline{toc}{section}{\tocacronym{MN 34} \toctranslation{The Shorter Discourse on the Cowherd } \tocroot{Cūḷagopālakasutta}}
\markboth{The Shorter Discourse on the Cowherd }{Cūḷagopālakasutta}
\extramarks{MN 34}{MN 34}

\scevam{So\marginnote{1.1} I have heard. }At one time the Buddha was staying in the land of the Vajjis near \textsanskrit{Ukkacelā} on the bank of the Ganges river.\footnote{The Buddha also gave a discourse here shortly after the deaths of \textsanskrit{Sāriputta} and \textsanskrit{Moggallāna} (\href{https://suttacentral.net/sn47.14/en/sujato\#1.1}{SN 47.14:1.1}). } There the Buddha addressed the mendicants, “Mendicants!” 

“Venerable\marginnote{1.5} sir,” they replied. The Buddha said this: 

“Once\marginnote{2.1} upon a time, mendicants, there was an unintelligent Magadhan cowherd. In the last month of the rainy season, in autumn, without inspecting the near shore or the far shore, he drove his cattle across a place with no ford on the Ganges river to the northern shore among the Suvidehans.\footnote{Thus illustrating the need to “know the ford” in the previous sutta (\href{https://suttacentral.net/mn33/en/sujato\#2.3}{MN 33:2.3}). | The Suvidehans are not mentioned anywhere else, and the commentary simply treats them as Videhans. But Videha lay to the north-east of \textsanskrit{Vajjī}, and since the expansion of \textsanskrit{Vajjī} and contraction of Videha before the Buddha’s time, it is unlikely that it bordered the Ganges opposite Magadha. Perhaps this story stems from a time when Videha did expand to the river, or perhaps there were folks who called themselves “good Videhans” after the eclipse of the kingdom. } 

But\marginnote{3.1} the cattle bunched up in mid-stream and came to ruin right there. Why is that? Because the unintelligent cowherd failed to inspect the shores before driving the cattle across at a place with no ford. In the same way, there are ascetics and brahmins who are unskilled in this world and the other world, unskilled in \textsanskrit{Māra}’s dominion and its opposite, and unskilled in Death’s dominion and its opposite. If anyone thinks they are worth listening to and trusting, it will be for their lasting harm and suffering. 

Once\marginnote{4.1} upon a time, mendicants, there was an intelligent Magadhan cowherd. In the last month of the rainy season, in autumn, after inspecting the near shore and the far shore, he drove his cattle across a ford on the Ganges river to the northern shore among the Suvidehans. 

First\marginnote{5.1} he drove across the bulls, the fathers and leaders of the herd. They breasted the stream of the Ganges and safely reached the far shore. Then he drove across the strong and tractable cattle. They too breasted the stream of the Ganges and safely reached the far shore. Then he drove across the bullocks and heifers. They too breasted the stream of the Ganges and safely reached the far shore. Then he drove across the calves and weak cattle. They too breasted the stream of the Ganges and safely reached the far shore. Once it happened that a baby calf had just been born. Urged on by its mother’s lowing, even it managed to breast the stream of the Ganges and safely reach the far shore.\footnote{Typically \textit{\textsanskrit{bhūtapubbaṁ}} is used for fables (“once upon a time”), but see also \href{https://suttacentral.net/mn36/en/sujato\#4.3}{MN 36:4.3}. } Why is that? Because the intelligent cowherd inspected both shores before driving the cattle across at a ford. 

In\marginnote{5.12} the same way, there are ascetics and brahmins who are skilled in this world and the other world, skilled in \textsanskrit{Māra}’s dominion and its opposite, and skilled in Death’s dominion and its opposite. If anyone thinks they are worth listening to and trusting, it will be for their lasting welfare and happiness. 

Just\marginnote{6.1} like the bulls, fathers and leaders of the herd, who crossed the Ganges to safety are the mendicants who are perfected, who have ended the defilements, completed the spiritual journey, done what had to be done, laid down the burden, achieved their own goal, utterly ended the fetter of continued existence, and are rightly freed through enlightenment. Having breasted \textsanskrit{Māra}’s stream, they have safely crossed over to the far shore. 

Just\marginnote{7.1} like the strong and tractable cattle who crossed the Ganges to safety are the mendicants who, with the ending of the five lower fetters, are reborn spontaneously. They’re extinguished there, and are not liable to return from that world. They too, having breasted \textsanskrit{Māra}’s stream, will safely cross over to the far shore. 

Just\marginnote{8.1} like the bullocks and heifers who crossed the Ganges to safety are the mendicants who, with the ending of three fetters, and the weakening of greed, hate, and delusion, are once-returners. They come back to this world once only, then make an end of suffering. They too, having breasted \textsanskrit{Māra}’s stream, will safely cross over to the far shore. 

Just\marginnote{9.1} like the calves and weak cattle who crossed the Ganges to safety are the mendicants who, with the ending of three fetters are stream-enterers, not liable to be reborn in the underworld, bound for awakening. They too, having breasted \textsanskrit{Māra}’s stream, will safely cross over to the far shore. 

Just\marginnote{10.1} like the baby calf who had just been born, but, urged on by its mother’s lowing, still managed to cross the Ganges to safety are the mendicants who are followers of teachings, followers by faith. They too, having breasted \textsanskrit{Māra}’s stream, will safely cross over to the far shore. 

Mendicants,\marginnote{11.1} I am skilled in this world and the other world, skilled in \textsanskrit{Māra}’s dominion and its opposite, and skilled in Death’s dominion and its opposite. If anyone thinks I am worth listening to and trusting, it will be for their lasting welfare and happiness.” 

That\marginnote{12.1} is what the Buddha said. Then the Holy One, the Teacher, went on to say:\footnote{While commonly found elsewhere, this tag line is only used three times in the Majjhima \textsanskrit{Nikāya} (\href{https://suttacentral.net/mn130/en/sujato\#30.2}{MN 130:30.2}, \href{https://suttacentral.net/mn142/en/sujato\#14.2}{MN 142:14.2}). } 

\begin{verse}%
“This\marginnote{12.3} world and the other world\footnote{This echoes \textsanskrit{Bṛhadāraṇyaka} \textsanskrit{Upaniṣad} 3.7.1 (\textit{\textsanskrit{ayaṁ} ca lokaḥ \textsanskrit{paraśca} lokaḥ}). } \\
have been clearly explained by one who knows; \\
as well as \textsanskrit{Māra}’s reach, \\
and what’s out of Death’s reach. 

Directly\marginnote{12.7} knowing the whole world, \\
the Buddha who understands \\
has opened the door to freedom from death,\footnote{Compare \href{https://suttacentral.net/mn26/en/sujato\#20.12}{MN 26:20.12}. } \\
for finding the sanctuary, extinguishment. 

The\marginnote{12.11} Wicked One’s stream has been cut, \\
it’s blown away and mown down. \\
Be full of joy, mendicants, \\
set your heart on sanctuary!”\footnote{Read \textit{patthetha} (“set your heart”). } 

%
\end{verse}

%
\section*{{\suttatitleacronym MN 35}{\suttatitletranslation The Shorter Discourse With Saccaka }{\suttatitleroot Cūḷasaccakasutta}}
\addcontentsline{toc}{section}{\tocacronym{MN 35} \toctranslation{The Shorter Discourse With Saccaka } \tocroot{Cūḷasaccakasutta}}
\markboth{The Shorter Discourse With Saccaka }{Cūḷasaccakasutta}
\extramarks{MN 35}{MN 35}

\scevam{So\marginnote{1.1} I have heard. }At one time the Buddha was staying near \textsanskrit{Vesālī}, at the Great Wood, in the hall with the peaked roof. 

Now\marginnote{2.1} at that time Saccaka, the son of Jain parents, was staying in \textsanskrit{Vesālī}. He was a debater and clever speaker deemed holy by many people.\footnote{Saccaka is called \textit{\textsanskrit{nigaṇṭhaputta}}, being the only person so described. The \textit{\textsanskrit{nigaṇṭhas}} (“knotless ones”) were Jain ascetics and their followers were \textit{\textsanskrit{nigaṇṭhasāvaka}} (\href{https://suttacentral.net/sn42.9/en/sujato\#2.3}{SN 42.9:2.3}). The commentary says he was the son of Jain parents. From the ending of this sutta, where he makes an offering of food, we can see he was not an ascetic, and from \href{https://suttacentral.net/mn36/en/sujato\#48.11}{MN 36:48.11} it is clear he was not a Jain. | This sutta includes a number of images, phrases, and allusions that appear reminiscent of the Brahmanical teacher \textsanskrit{Uddālaka} \textsanskrit{Āruṇi}, although none of the references are definitive. The bulk of \textsanskrit{Uddālaka}’s teachings were to his son Śvetaketu, who became arrogant due to his learning (\textsanskrit{Chāndogya} \textsanskrit{Upaniṣad} 6.1.2), paralleling Saccaka’s arrogance. } He was telling a crowd in \textsanskrit{Vesālī}, “I don’t see any ascetic or brahmin who would not shake and rock and tremble, sweating from the armpits, were I to take them on in debate—not a leader of an order or a community, or the tutor of a community, and not even one who claims to be a perfected one, a fully awakened Buddha. Even an insentient post would shake and rock and tremble were I to take it on in debate. How much more then a human being!” 

Then\marginnote{3.1} Venerable Assaji robed up in the morning and, taking his bowl and robe, entered \textsanskrit{Vesālī} for alms.\footnote{Assaji was one the group of five mendicants who received the Buddha’s first discourse (\href{https://suttacentral.net/pli-tv-kd1/en/sujato\#6.36.1}{Kd 1:6.36.1}). Shortly afterwards he inspired \textsanskrit{Sāriputta} with a classic epitome of the teaching (\href{https://suttacentral.net/pli-tv-kd1/en/sujato\#23.2.1}{Kd 1:23.2.1}). Note that \textsanskrit{Sāriputta} waited politely until almsround was complete before approaching, whereas Saccaka just goes right ahead. We hear of a grave illness Assaji suffered later (\href{https://suttacentral.net/sn22.88/en/sujato}{SN 22.88}). | Another Assaji, who is always paired with his friend Punabbasuka, was a lazy and indulgent monk (\href{https://suttacentral.net/mn70/en/sujato\#4.1}{MN 70:4.1}). } As Saccaka was going for a walk he saw Assaji coming off in the distance. He approached him and exchanged greetings with him. 

When\marginnote{3.4} the greetings and polite conversation were over, Saccaka stood to one side and said to Assaji, “Mister Assaji, how does the ascetic Gotama guide his disciples? And on what topics does instruction to his disciples generally proceed?”\footnote{\textit{\textsanskrit{Sāvake} vineti} (“guide his disciples”) is typically used by non-Buddhists. | The phrase \textit{\textsanskrit{kathaṁbhāgā}} (and its counterpart \textit{\textsanskrit{evaṁbhāgā}} in the next line) are unique. The literal sense is “with what portion”, and given the answer I think Saccaka is asking for a list of topics. } 

“Aggivessana,\marginnote{4.2} this is how the ascetic Gotama guides his disciples, and his instructions to disciples generally proceed on these topics:\footnote{Saccaka is addressed with the Bramanical clan name Aggivessana, probably after the lineage of his clan’s \textit{purohita}, just as the Sakyans are called Gotama and the Mallas \textsanskrit{Vāseṭṭha}. } ‘Form, feeling, perception, choices, and consciousness are impermanent.\footnote{The five aggregates are mentioned as if the listener is expected to know them. } Form, feeling, perception, choices, and consciousness are not-self. All conditions are impermanent. All things are not-self.’ This is how the ascetic Gotama guides his disciples, and how instruction to his disciples generally proceeds.”\footnote{Despite this, this teaching that omits “suffering” (\textit{dukkha}) is not taught by the Buddha at all. It only appears in one other sutta, where it is spoken by the mendicants after the Buddha’s passing (\href{https://suttacentral.net/sn22.90/en/sujato\#2.2}{SN 22.90:2.2}; but see below at \href{https://suttacentral.net/mn35/en/sujato\#9.5}{MN 35:9.5}). Strikingly, both the current sutta and SN 22.90 share a common narrative pattern. In both cases, an interlocuter (Saccaka or Channa, both known for their conceit) asks for teaching from a lesser teacher (Assaji or unnamed mendicants). Their reply omits suffering, but this fails. So they seek a better teacher (the Buddha, Ānanda) who not only includes suffering, but specially emphasizes it (\href{https://suttacentral.net/mn35/en/sujato\#21.2}{MN 35:21.2}, \href{https://suttacentral.net/sn22.90/en/sujato\#9.7}{SN 22.90:9.7}). This teaching is successful. This narrative journey mirrors that of Śvetaketu, who studied with others for twelve years before being shown true wisdom by his father \textsanskrit{Uddālaka}. If we are on the right track, the point of the narrative is that it is essential to include suffering. This suggests that the Chinese parallels (SA 110 at T ii 35b4 and EA 37.10 at T ii 715b4), which do include the characteristic of \textit{dukkha} here, would be a product of later expansion and normalization. } 

“It’s\marginnote{4.7} sad to hear, Mister Assaji, that the ascetic Gotama has such a doctrine. Hopefully, some time or other I’ll get to meet Mister Gotama, and we can have a discussion. And hopefully I can dissuade him from this harmful misconception.” 

Now\marginnote{5.1} at that time around five hundred Licchavis were sitting together at the town hall on some business.\footnote{The Licchavis, whose name is derived from “bear”, dominated the Vajji Federation. } Then Saccaka went up to them and said, “Come forth, good \textsanskrit{Licchavīs}, come forth! Today I am going to have a discussion with the ascetic Gotama. If he stands by the position stated to me by one of his well-known disciples—a mendicant named Assaji—I’ll take him on in debate and drag him to and fro and round about, like a strong man would grab a fleecy sheep and drag it to and fro and round about! Taking him on in debate, I’ll drag him to and fro and round about, like a strong brewer’s worker would toss a large brewer’s sieve into a deep lake, grab it by the corners, and drag it to and fro and round about! Taking him on in debate, I’ll shake him down and about, and give him a beating, like a strong brewer’s mixer would grab a strainer by the corners and shake it down and about, and give it a beating!\footnote{See \href{https://suttacentral.net/an6.53/en/sujato}{AN 6.53} and \href{https://suttacentral.net/sn22.102/en/sujato}{SN 22.102}. } I’ll play a game of ear-washing with the ascetic Gotama, like a sixty-year-old elephant would plunge into a deep lotus pond and play a game of ear-washing!\footnote{Emend \textit{\textsanskrit{sāṇadhovikaṁ}} (“hemp-washing”) to \textit{\textsanskrit{kaṇṇadhovikaṁ}} (“ear-washing”) after \href{https://suttacentral.net/an10.99/en/sujato\#3.4}{AN 10.99:3.4}. } Come forth, good \textsanskrit{Licchavīs}, come forth! Today I am going to have a discussion with the ascetic Gotama.” 

At\marginnote{6.1} that, some of the Licchavis said, “How can the ascetic Gotama refute Saccaka’s doctrine, when it is Saccaka who will refute Gotama’s doctrine?” 

But\marginnote{6.3} some of the Licchavis said, “Who is Saccaka to refute the Buddha’s doctrine, when it is the Buddha who will refute Saccaka’s doctrine?” 

Then\marginnote{6.5} Saccaka, escorted by the five hundred Licchavis, went to the hall with the peaked roof in the Great Wood. 

At\marginnote{7.1} that time several mendicants were walking mindfully in the open air. Then Saccaka went up to them and said, “Good sirs, where is Mister Gotama at present? For we want to see him.” 

“Aggivessana,\marginnote{7.5} the Buddha has plunged deep into the Great Wood and is sitting at the root of a tree for the day’s meditation.” 

Then\marginnote{8.1} Saccaka, together with a large group of Licchavis, went to see the Buddha in the Great Wood, and exchanged greetings with him. When the greetings and polite conversation were over, he sat down to one side. Before sitting down to one side, some of the \textsanskrit{Licchavīs} bowed, some exchanged greetings and polite conversation, some held up their joined palms toward the Buddha, some announced their name and clan, while some kept silent. 

Then\marginnote{9.1} Saccaka said to the Buddha, “I’d like to ask Mister Gotama about a certain point, if you’d take the time to answer.” 

“Ask\marginnote{9.3} what you wish, Aggivessana.” 

“How\marginnote{9.4} does Mister Gotama guide his disciples? And on what topics does instruction to his disciples generally proceed?” 

“This\marginnote{9.5} is how I guide my disciples, and my instructions to disciples generally proceed on these topics:\footnote{If, as I have argued, Assaji’s presentation is meant to be inadequate, why does the Buddha endorse it here? The Buddha commonly endorses answers given by disciples, and this could have been a product of editorial standardization. After all, it remains the fact that it simply is not true that the Buddha normally teaches in this way. } ‘Form, feeling, perception, choices, and consciousness are impermanent. Form, feeling, perception, choices, and consciousness are not-self. All conditions are impermanent. All things are not-self.’ This is how I guide my disciples, and how instruction to my disciples generally proceeds.” 

“A\marginnote{10.1} simile strikes me, Mister Gotama.” 

“Then\marginnote{10.2} speak as you feel inspired,” said the Buddha. 

“All\marginnote{10.3} the plants and seeds that achieve growth, increase, and maturity do so depending on the earth and grounded on the earth.\footnote{Saccaka argues by naturalistic analogy from singularity to plurality. Likewise \textsanskrit{Uddālaka}, observing the natural world, pointed to the way that multiplicity emerges from singleness. When asked who he worships as self, \textsanskrit{Uddālaka} answers the “earth”, which is said to be the “ground” (\textit{\textsanskrit{pratiṣṭhā}}, \textsanskrit{Chāndogya} \textsanskrit{Upaniṣad} 5.17); due to this worship he enjoys food and offspring. However, he points to water rather than earth as the origin of \textit{anna}, which is (plant) food but also stands for solidity in general (6.2.4; see too 6.11.1–2). } All the hard work that gets done depends on the earth and is grounded on the earth. 

In\marginnote{10.7} the same way, an individual’s self is form. Grounded on form they create merit and wickedness. An individual’s self is feeling … perception … choices … consciousness. Grounded on consciousness they create merit and wickedness.”\footnote{Saccaka treats the five aggregates as categories for classifying the self. And indeed elsewhere, such as \href{https://suttacentral.net/dn1/en/sujato}{DN 1}, we find various theorists describe the self as “form”, experiencing “feelings”, being “percipient”, etc. In identifying each of the five aggregates as self, Saccaka follows the method of “affirmation” whereby multiple phenomena are affirmed to be the self at deeper and deeper levels, in contrast with the method of “negation” where potential candidates for the self are denied until the final self is revealed. \textsanskrit{Uddālaka} also preferred the method of affirmation, arguing that existence comes first and that the divinity pours itself in all the manifold forms of the world (\textsanskrit{Chāndogya} \textsanskrit{Upaniṣad} 6.2.2 ff.). } 

“Aggivessana,\marginnote{11.1} are you not saying this: ‘Form is my self, feeling is my self, perception is my self, choices are my self, consciousness is my self’?” 

“Indeed,\marginnote{11.3} Mister Gotama, that is what I am saying. And this big crowd agrees with me!” 

“What\marginnote{11.5} has this big crowd to do with you? Please just unpack your own statement.” 

“Then,\marginnote{11.7} Mister Gotama, what I am saying is this: ‘Form is my self, feeling is my self, perception is my self, choices are my self, consciousness is my self’.” 

“Well\marginnote{12.1} then, Aggivessana, I’ll ask you about this in return, and you can answer as you like. What do you think, Aggivessana? Consider an anointed aristocratic king such as Pasenadi of Kosala or \textsanskrit{Ajātasattu} of Magadha, son of the princess of Videha. Would they have the power in their own realm to execute those who have incurred execution, fine those who have incurred fines, or banish those who have incurred banishment?”\footnote{The Pali uses the future passive participle, sometimes interpreted as “deserving of execution”. But this would imply that the Buddha endorsed capital punishment, which is unsupported elsewhere and contradicts his ethical principles. Rather, read in its weaker literal sense, “execute those to be executed”, i.e. those who have been found guilty of a crime incurring capital punishment. It is an acknowledgment of the king’s powers, not an endorsement of them. } 

“An\marginnote{12.5} anointed king would have such power, Mister Gotama. Even federations such as the Vajjis and Mallas have such power in their own realm.\footnote{Saccaka cannily appeals to the Licchavis by asserting their right to rule. } So of course an anointed king such as Pasenadi or \textsanskrit{Ajātasattu} would wield such power, as is their right.”\footnote{Overstepping the Buddha’s position, Saccaka’s endorsement of capital punishment shows how far he has drifted from the non-violent doctrine of his Jain parents. } 

“What\marginnote{13.1} do you think, Aggivessana? When you say, ‘Form is my self,’ do you have power over that form to say: ‘May my form be like this! May it not be like that’?”\footnote{Also at \href{https://suttacentral.net/sn22.59/en/sujato\#2.3}{SN 22.59:2.3}. This responds to the “inner controller” (\textit{\textsanskrit{antaryāmin}}) of \textsanskrit{Bṛhadāraṇyaka} \textsanskrit{Upaniṣad} 3.7. \textsanskrit{Uddālaka} learned this from the centaur Kabandha in Madra (near modern Islamabad). He challenges \textsanskrit{Yājñavalkya}, who explains it as the self who lives in the midst of all things, unknown but knowing, controlling all from within. } When he said this, Saccaka kept silent. The Buddha asked the question a second time, but Saccaka still kept silent. So the Buddha said to Saccaka, “Answer now, Aggivessana. Now is not the time for silence. If someone fails to answer a legitimate question when asked three times by the Buddha, their head explodes into seven pieces there and then.”\footnote{The threat of losing one’s head is found at eg. \textsanskrit{Chāndogya} \textsanskrit{Upaniṣad} 1.8.6 and \textsanskrit{Bṛhadāraṇyaka} \textsanskrit{Upaniṣad} 1.3.24, or at 3.9.26 when it actually did fall off. I cannot trace the detail of heads being split in seven to any early Sanskrit texts, but it is found in later texts such as \textsanskrit{Rāmāyaṇa} 7.26.44c and \textsanskrit{Mahābhārata} 14.7.2c. } 

Now\marginnote{14.1} at that time the spirit \textsanskrit{Vajirapāṇī}, taking up a burning iron thunderbolt, blazing and glowing, stood in the air above Saccaka, thinking,\footnote{\textsanskrit{Vajirapāṇī} (“lightning-bolt in hand”) appears here and in the parallel passage at \href{https://suttacentral.net/dn3/en/sujato\#1.21.1}{DN 3:1.21.1}. The synonymous Vajrahasta (Pali \textit{vajirahattha}, \href{https://suttacentral.net/dn20/en/sujato\#12.1}{DN 20:12.1}) is a frequent epithet of Indra in the Vedas (eg. \textit{indro vajrahastaḥ}, Rig Veda 1.173.10a), confirming the commentary’s identification with Sakka. Much later, Mahayana texts adopted the name for a fierce Bodhisattva who was protector of the Dhamma. } “If this Saccaka doesn’t answer when asked a third time, I’ll blow his head into seven pieces there and then!” And both the Buddha and Saccaka could see \textsanskrit{Vajirapāṇī}. 

Saccaka\marginnote{14.4} was terrified, shocked, and awestruck. Looking to the Buddha for shelter, protection, and refuge, he said, “Ask me, Mister Gotama. I will answer.” 

“What\marginnote{15.1} do you think, Aggivessana? When you say, ‘Form is my self,’ do you have power over that form to say: ‘May my form be like this! May it not be like that’?” 

“No,\marginnote{15.5} Mister Gotama.” 

“Think\marginnote{16.1} about it, Aggivessana!\footnote{Here neither “attention” nor “focus” quite fits \textit{\textsanskrit{manasikāra}}. The Buddha is telling Saccaka to recollect an earlier statement when making a later one, i.e. to practice rationality. } You should think before answering. What you said before and what you said after don’t match up. What do you think, Aggivessana? When you say, ‘Feeling is my self,’ do you have power over that feeling to say: ‘May my feeling be like this! May it not be like that’?” 

“No,\marginnote{16.8} Mister Gotama.” 

“Think\marginnote{17.1} about it, Aggivessana! You should think before answering. What you said before and what you said after don’t match up. What do you think, Aggivessana? When you say, ‘Perception is my self,’ do you have power over that perception to say: ‘May my perception be like this! May it not be like that’?” 

“No,\marginnote{17.8} Mister Gotama.” 

“Think\marginnote{18.1} about it, Aggivessana! You should think before answering. What you said before and what you said after don’t match up. What do you think, Aggivessana? When you say, ‘Choices are my self,’ do you have power over those choices to say: ‘May my choices be like this! May they not be like that’?” 

“No,\marginnote{18.8} Mister Gotama.” 

“Think\marginnote{19.1} about it, Aggivessana! You should think before answering. What you said before and what you said after don’t match up. What do you think, Aggivessana? When you say, ‘Consciousness is my self,’ do you have power over that consciousness to say: ‘May my consciousness be like this! May it not be like that’?” 

“No,\marginnote{19.8} Mister Gotama.” 

“Think\marginnote{20.1} about it, Aggivessana! You should think before answering. What you said before and what you said after don’t match up. What do you think, Aggivessana? Is form permanent or impermanent?” 

“Impermanent.”\marginnote{20.6} 

“But\marginnote{20.7} if it’s impermanent, is it suffering or happiness?”\footnote{Here the Buddha introduces “suffering”, despite its omission earlier. } 

“Suffering.”\marginnote{20.8} 

“But\marginnote{20.9} if it’s impermanent, suffering, and perishable, is it fit to be regarded thus: ‘This is mine, I am this, this is my self’?”\footnote{This phrasing echoes \textsanskrit{Uddālaka}’s famous description of the “subtlest’ self: “all that is the truth, that is the self, you are that” (\textit{\textsanskrit{sarvaṁ} \textsanskrit{tatsatyaṁ} sa \textsanskrit{ātmā} tattvamasi}, \textsanskrit{Chāndogya} \textsanskrit{Upaniṣad} 6.8.7, etc.). In particular the phrase \textit{tattvamasi} is an exact parallel (in second person) to the Pali \textit{esohamasmi}.  See too \textsanskrit{Uddālaka}’s, “I am that, I indeed am that” (\textit{iyam aham \textsanskrit{asmīyam} aham \textsanskrit{asmīti}}, \textsanskrit{Chāndogya} \textsanskrit{Upaniṣad} 6.10.1; cf. 4.11.1, 4.12.1, 8.11.1–2). Likewise it is said, “I myself am that person” (\textit{yo’\textsanskrit{sāvasau} \textsanskrit{puruṣaḥ} so’hamasmi}, \textsanskrit{Bṛhadāraṇyaka} \textsanskrit{Upaniṣad} 5.15 = \textsanskrit{Īśā} \textsanskrit{Upaniṣad} 16). Depending on context, similar phrases might also depict the one who is attached to a false, limited self, thinking “I am he”, “this is mine” (\textit{\textsanskrit{ahaṁ} so, \textsanskrit{mamedaṁ}}, \textsanskrit{Maitrī} \textsanskrit{Upaniṣad} 3.2). } 

“No,\marginnote{20.11} Mister Gotama.” 

“What\marginnote{20.12} do you think, Aggivessana? Is feeling … perception … choices … consciousness permanent or impermanent?” 

“Impermanent.”\marginnote{20.17} 

“But\marginnote{20.18} if it’s impermanent, is it suffering or happiness?” 

“Suffering.”\marginnote{20.19} 

“But\marginnote{20.20} if it’s impermanent, suffering, and perishable, is it fit to be regarded thus: ‘This is mine, I am this, this is my self’?” 

“No,\marginnote{20.22} Mister Gotama.” 

“What\marginnote{21.1} do you think, Aggivessana? Consider someone who resorts, draws near, and clings to suffering, regarding it thus: ‘This is mine, I am this, this is my self.’ Would such a person be able to completely understand suffering themselves, or live having wiped out suffering?” 

“How\marginnote{21.3} could they? No, Mister Gotama.” 

“What\marginnote{21.5} do you think, Aggivessana? This being so, aren’t you someone who resorts, draws near, and clings to suffering, regarding it thus: ‘This is mine, I am this, this is my self’?” 

“How\marginnote{21.8} could I not? Yes, Mister Gotama.” 

“Suppose,\marginnote{22.1} Aggivessana, there was a person in need of heartwood. Wandering in search of heartwood, they’d take a sharp axe and enter a forest. There they’d see a big banana tree, straight and young and grown free of defects. They’d cut it down at the base, cut off the top, and unroll the coiled sheaths. But they wouldn’t even find sapwood, much less heartwood. 

In\marginnote{22.5} the same way, when pursued, pressed, and grilled by me on your own doctrine, you turn out to be vacuous, hollow, and mistaken. But it was you who stated before the assembly of \textsanskrit{Vesālī}: ‘I don’t see any ascetic or brahmin who would not shake and rock and tremble, sweating from the armpits, were I to take them on in debate—not a leader of an order or a community, or the tutor of a community, and not even one who claims to be a perfected one, a fully awakened Buddha. Even an insentient post would shake and rock and tremble were I to take it on in debate. How much more then a human being!’ But sweat is pouring from your forehead; it’s soaked through your robe and drips on the ground.\footnote{The oddly prominent role of sweat in the discourse finds its parallel in \textsanskrit{Chāndogya} \textsanskrit{Upaniṣad} 6.2.3, where \textsanskrit{Uddālaka} argues that existence (\textit{sat}) produces fire (\textit{tejas}) and fire produces water, which is why you sweat when hot. } While I now have no sweat on my body.” So the Buddha revealed his golden body to the assembly. When this was said, Saccaka sat silent, dismayed, shoulders drooping, downcast, depressed, with nothing to say. 

Knowing\marginnote{23.1} this, the Licchavi Dummukha said to the Buddha,\footnote{Dummukha the Licchavi appears only here. The name, meaning “Ugly Face” seems odd, but it is also a term for a horse. A number of individuals of this name appear in Sanskritic literature, including one from \textsanskrit{Uddālaka}’s home country of \textsanskrit{Pañcāla} (Aitareya \textsanskrit{Brāhmaṇa} 8.23). } “A simile strikes me, Blessed One.” 

“Then\marginnote{23.3} speak as you feel inspired,” said the Buddha. 

“Sir,\marginnote{23.4} suppose there was a lotus pond not far from a town or village, and a crab lived there. Then several boys or girls would leave the town or village and go to the pond, where they’d pull out the crab and put it on dry land. Whenever that crab extended a claw, those boys or girls would snap, crack, and break it off with a stick or a stone. And when that crab’s claws had all been snapped, cracked, and broken off it wouldn’t be able to return down into that lotus pond. In the same way, sir, the Buddha has snapped, cracked, and broken off all Saccaka’s twists, ducks, and dodges. Now he can’t get near the Buddha again looking for a debate.” 

But\marginnote{24.1} Saccaka said to him, “Hold on, Dummukha, hold on! I wasn’t talking with you, I was talking with Mister Gotama. 

Mister\marginnote{24.3} Gotama, leave aside that statement I made—as did various other ascetics and brahmins—it was, like, just a bit of nonsense.\footnote{Saccaka tries to wriggle out of his initial statement, just as modern trolls claim to speak in irony or humor. } How do you define a disciple of Mister Gotama who follows instructions and responds to advice; who has gone beyond doubt, got rid of indecision, gained assurance, and is independent of others in the Teacher’s instructions?” 

“It’s\marginnote{24.6} when one of my disciples truly sees any kind of form at all—past, future, or present; internal or external; solid or subtle; inferior or superior; far or near: \emph{all} form—with right understanding: ‘This is not mine, I am not this, this is not my self.’\footnote{This is the trainee (\textit{sekha}). Knowing Saccaka’s attachment to his self theories, the Buddha frames his response with the standard passage on not-self. } They truly see any kind of feeling … perception … choices … consciousness at all—past, future, or present; internal or external; solid or subtle; inferior or superior; far or near: \emph{all} consciousness—with right understanding: ‘This is not mine, I am not this, this is not my self.’ That’s how to define one of my disciples who follows instructions and responds to advice; who has gone beyond doubt, got rid of indecision, gained assurance, and is independent of others in the Teacher’s instructions.” 

“But\marginnote{25.1} how do you define a mendicant who is a perfected one, with defilements ended, who has completed the spiritual journey, done what had to be done, laid down the burden, achieved their own true goal, utterly ended the fetter of continued existence, and is rightly freed through enlightenment?” 

“It’s\marginnote{25.2} when a mendicant truly sees any kind of form at all—past, future, or present; internal or external; coarse or fine; inferior or superior; far or near: \emph{all} form—with right understanding: ‘This is not mine, I am not this, this is not my self.’ And having seen this with right understanding they’re freed by not grasping. They truly see any kind of feeling … perception … choices … consciousness at all—past, future, or present; internal or external; solid or subtle; inferior or superior; far or near: \emph{all} consciousness—with right understanding: ‘This is not mine, I am not this, this is not my self.’ And having seen this with right understanding they’re freed by not grasping. That’s how to define a mendicant who is a perfected one, with defilements ended, who has completed the spiritual journey, done what had to be done, laid down the burden, achieved their own true goal, utterly ended the fetter of continued existence, and is rightly freed through enlightenment. 

A\marginnote{26.1} mendicant whose mind is freed like this has three unsurpassable qualities: unsurpassable seeing, practice, and freedom.\footnote{Also at \href{https://suttacentral.net/dn33/en/sujato\#1.10.119}{DN 33:1.10.119}, which is doubtless lifted from the current passage. A list of six unsurpassables is found at \href{https://suttacentral.net/an6.30/en/sujato}{AN 6.30}, but the only common item is “seeing” (\textit{dassana}), which is defined as seeing the Buddha and his disciples. Here “seeing” is either seeing the Buddha which gives rise to faith, or else “seeing” the four noble truths; practice is the noble eightfold path; and freedom is the result of that path, arahantship. | As to why this unique set is chosen here, it is perhaps notable that three of the final similes of \textsanskrit{Uddālaka}’s “subtlest” self (as above, \href{https://suttacentral.net/mn35/en/sujato\#20.10}{MN 35:20.10}) in the \textsanskrit{Chāndogya} \textsanskrit{Upaniṣad} speak of \emph{seeing} (the kernel of a seed, 6.12.1–2), \emph{walking} (from one village to the next, 6.14.2), and \emph{freedom} (from punishment, 6.16.2). } They honor, respect, esteem, and venerate only the Realized One: ‘The Blessed One is awakened, tamed, serene, crossed over, and quenched. And he teaches Dhamma for awakening, self-control, serenity, crossing over, and extinguishment.’” 

When\marginnote{27.1} he had spoken, Saccaka said to him, “Mister Gotama, it was rude and impudent of me to imagine I could attack you in debate. For a person might find safety after attacking a rutting elephant, but not after attacking Mister Gotama. A person might find safety after attacking a blazing mass of fire, but not after attacking Mister Gotama. They might find safety after attacking a poisonous viper, but not after attacking Mister Gotama. It was rude and impudent of me to imagine I could attack you in debate. Would Mister Gotama together with the mendicant \textsanskrit{Saṅgha} please accept tomorrow’s meal from me?” The Buddha consented with silence. 

Then,\marginnote{28.1} knowing that the Buddha had consented, Saccaka addressed those Licchavis, “Listen, gentlemen. I have invited the ascetic Gotama together with the \textsanskrit{Saṅgha} of mendicants for tomorrow’s meal. You may all bring me what you think is suitable.” 

Then,\marginnote{29.1} when the night had passed, those Licchavis presented Saccaka with an offering of five hundred servings of food.\footnote{The idiom “presented with servings of food” (\textit{\textsanskrit{thālipākaā} \textsanskrit{bhattābhihāraṁ}}) is elevated, occurring only in the context of aristocracy and sacrifice. } And Saccaka had delicious fresh and cooked foods prepared in his own park. Then he had the Buddha informed of the time, saying, “It’s time, Mister Gotama, the meal is ready.” 

Then\marginnote{30.1} the Buddha robed up in the morning and, taking his bowl and robe, went to Saccaka’s park, where he sat on the seat spread out, together with the \textsanskrit{Saṅgha} of mendicants. Then Saccaka served and satisfied the mendicant \textsanskrit{Saṅgha} headed by the Buddha with his own hands with delicious fresh and cooked foods. When the Buddha had eaten and washed his hand and bowl, Saccaka took a low seat and sat to one side. 

Then\marginnote{30.4} Saccaka said to the Buddha, “Mister Gotama, may the merit and the growth of merit in this gift be for the happiness of the donors.”\footnote{\textit{\textsanskrit{Puññamahī}} (“flourishing of merit”), for which there are several variants in text and commentary, is found only here and at \href{https://suttacentral.net/an7.53/en/sujato\#6.10}{AN 7.53:6.10}, where it also concerns the dedication of merit to another. It might refer to the glorious results of merit, for which compare \textsanskrit{Chāndogya} \textsanskrit{Upaniṣad} 8.2, “endowed with that (divine fruit for which he wished) he flourishes (\textit{tena sampanno \textsanskrit{mahīyate}}). Or else it might invoke the idea that merit flourishes depending on the recipient (\href{https://suttacentral.net/mn142/en/sujato}{MN 142}), like a seed on the earth (\textit{\textsanskrit{mahī}}; cp. \textit{\textsanskrit{puññakkhetta}}), which fits nicely with the Buddha’s response here. | Bhikkhu Bodhi’s  remark that Saccaka “must have still considered himself a saint” is unwarranted, as the same phrase at AN 7.53 is used by a devoted lay follower. Clearly he was arrogant, but it was others who called him \textit{\textsanskrit{sādhu}}. } 

“Aggivessana,\marginnote{30.6} whatever comes from giving to a recipient of a religious donation such as yourself—who is not free of greed, hate, and delusion—will accrue to the donors.\footnote{According to one of the Chinese parallels, this statement was made by the Buddha in answer to a question by the mendicants after returning to the monastery (SA 110 at T ii 37b22). } Whatever comes from giving to a recipient of a religious donation such as myself—who is free of greed, hate, and delusion—will accrue to you.” 

%
\section*{{\suttatitleacronym MN 36}{\suttatitletranslation The Longer Discourse With Saccaka }{\suttatitleroot Mahāsaccakasutta}}
\addcontentsline{toc}{section}{\tocacronym{MN 36} \toctranslation{The Longer Discourse With Saccaka } \tocroot{Mahāsaccakasutta}}
\markboth{The Longer Discourse With Saccaka }{Mahāsaccakasutta}
\extramarks{MN 36}{MN 36}

\scevam{So\marginnote{1.1} I have heard. }At one time the Buddha was staying near \textsanskrit{Vesālī}, at the Great Wood, in the hall with the peaked roof. 

Now\marginnote{2.1} at that time in the morning the Buddha, being properly dressed, took his bowl and robe, wishing to enter \textsanskrit{Vesālī} for alms. 

Then\marginnote{3.1} as Saccaka, the son of Jain parents, was going for a walk he approached the hall with the peaked roof in the Great Wood.\footnote{This discourse appears to be set after the previous. In both suttas, while Saccaka is impressed by the Buddha, he does not go for refuge or indicate that he has been persuaded by his teachings. } Venerable Ānanda saw him coming off in the distance, and said to the Buddha, “Sir, Saccaka, the son of Jain parents, is coming. He’s a debater and clever speaker deemed holy by many people. He wants to discredit the Buddha, the teaching, and the \textsanskrit{Saṅgha}. Please, sir, sit for an hour out of sympathy.” The Buddha sat on the seat spread out. 

Then\marginnote{3.8} Saccaka went up to the Buddha, and exchanged greetings with him. When the greetings and polite conversation were over, he sat down to one side and said to the Buddha, 

“Mister\marginnote{4.1} Gotama, there are some ascetics and brahmins who live committed to the practice of developing physical endurance, without developing the mind.\footnote{“Development of physical endurance” is \textit{\textsanskrit{kāyabhāvanā}}, which is not a normal Buddhist term. It is introduced by Saccaka here and collected in \href{https://suttacentral.net/dn33/en/sujato\#1.10.117}{DN 33:1.10.117}. The literal meaning is “development of the body”, but I translate in accord with the sense. } They suffer painful physical feelings. This happened to someone once. Their thighs became paralyzed, their heart burst, hot blood gushed from their mouth, and they went mad and lost their mind. Their mind was subject to the body, and the body had power over it. Why is that? Because their mind was not developed. There are some ascetics and brahmins who live committed to the practice of developing the mind, without developing physical endurance. They suffer painful mental feelings. This happened to someone once. Their thighs became paralyzed, their heart burst, hot blood gushed from their mouth, and they went mad and lost their mind. Their body was subject to the mind, and the mind had power over it. Why is that? Because their physical endurance was not developed.\footnote{Saccaka’s question shows he was aware of the danger of pursuing of extreme practices and the need for balance. It remains the case today that imbalanced and excessive practices lead to physical and mental breakdowns. } It occurs to me that Mister Gotama’s disciples must live committed to the practice of developing the mind, without developing physical endurance.” 

“But\marginnote{5.1} Aggivessana, what have you heard about the development of physical endurance?” 

“Take,\marginnote{5.2} for example, Nanda Vaccha, Kisa \textsanskrit{Saṅkicca}, and the bamboo-staffed ascetic \textsanskrit{Gosāla}.\footnote{Described as prominent sages of the \textsanskrit{Ājīvakas} at \href{https://suttacentral.net/mn76/en/sujato\#53.6}{MN 76:53.6} and \href{https://suttacentral.net/an6.57/en/sujato\#7.1}{AN 6.57:7.1}. \textsanskrit{Gosāla} was one of the oft-mentioned six ascetic teachers (see below \href{https://suttacentral.net/mn36/en/sujato\#48.7}{MN 36:48.7}), while the other two are unknown outside of these passing mentions. But the phrasing of this passage suggests that they too, like \textsanskrit{Gosāla}, were contemporaries of the Buddha. } They go naked, ignoring conventions. They lick their hands, and don’t come or wait when called. They don’t consent to food brought to them, or food prepared on their behalf, or an invitation for a meal. They don’t receive anything from a pot or bowl; or from someone who keeps sheep, or who has a weapon or a shovel in their home; or where a couple is eating; or where there is a woman who is pregnant, breastfeeding, or who lives with a man; or where there’s a dog waiting or flies buzzing. They accept no fish or meat or beer or wine, and drink no fermented gruel. They go to just one house for alms, taking just one mouthful, or two houses and two mouthfuls, up to seven houses and seven mouthfuls. They feed on one saucer a day, two saucers a day, up to seven saucers a day. They eat once a day, once every second day, up to once a week, and so on, even up to once a fortnight. They live committed to the practice of eating food at set intervals.”\footnote{The earliest Jain text speaks of \textsanskrit{Mahāvīra} eating “only every sixth meal, or eighth, or tenth, or twelfth” (\textsanskrit{Ācāraṅgasūtra} 1.8.4.7). } 

“But\marginnote{6.1} Aggivessana, do they get by on so little?” 

“No,\marginnote{6.2} Mister Gotama. Sometimes they eat luxury fresh and cooked foods and drink a variety of luxury beverages. They gather their body’s strength, build it up, and get fat.” 

“What\marginnote{6.5} they earlier gave up, they later got back. That is how there is the increase and decrease of this body. But Aggivessana, what have you heard about development of the mind?” When Saccaka was questioned by the Buddha about development of the mind, he was stumped.\footnote{While pre-Buddhist texts describe a contemplative and reflective culture, they do not contain clear instructions for meditation. It seems that the practice of explicitly teaching meditation methods was an innovation of the Buddha. } 

So\marginnote{7.1} the Buddha said to Saccaka, “The development of physical endurance that you have described is not the legitimate development of physical endurance in the noble one’s training. And since you don’t even understand the development of physical endurance, how can you possibly understand the development of the mind? Still, as to how someone is undeveloped in physical endurance and mind, and how someone is developed in physical endurance and mind, listen and apply your mind well, I will speak.” 

“Yes,\marginnote{7.6} sir,” replied Saccaka. The Buddha said this: 

“And\marginnote{8.1} how is someone undeveloped in physical endurance and mind? Take an unlearned ordinary person who has a pleasant feeling. When they experience a pleasant feeling they become full of lust for it. Then that pleasant feeling ceases. And when it ceases, a painful feeling arises. When they suffer painful feeling, they sorrow and wail and lament, beating their breast and falling into confusion. Because their physical endurance is undeveloped, pleasant feelings occupy the mind. And because their mind is undeveloped, painful feelings occupy the mind. Anyone whose mind is occupied by both pleasant and painful feelings like this is undeveloped both in physical endurance and in mind. 

And\marginnote{9.1} how is someone developed in physical endurance and mind? Take a learned noble disciple who has a pleasant feeling. When they experience a pleasant feeling they don’t become full of lust for it. Then that pleasant feeling ceases. And when it ceases, painful feeling arises. When they suffer painful feelings they don’t sorrow or wail or lament, beating their breast and falling into confusion. Because their physical endurance is developed, pleasant feelings don’t occupy the mind. And because their mind is developed, painful feelings don’t occupy the mind. Anyone whose mind is not occupied by both pleasant and painful feelings like this is developed both in physical endurance and in mind.” 

“I\marginnote{10.1} am quite confident that Mister Gotama is developed in physical endurance and in mind.” 

“Your\marginnote{10.3} words are clearly invasive and intrusive, Aggivessana.\footnote{This is a standard response to someone making unwarranted assumptions about personal spiritual development (\href{https://suttacentral.net/an3.60/en/sujato\#18.1}{AN 3.60:18.1}, \href{https://suttacentral.net/an4.35/en/sujato\#5.1}{AN 4.35:5.1}, \href{https://suttacentral.net/mn127/en/sujato\#17.8}{MN 127:17.8}). } Nevertheless, I will answer you. Ever since I shaved off my hair and beard, dressed in ocher robes, and went forth from the lay life to homelessness, it has not been possible for any pleasant or painful feeling to occupy my mind.”\footnote{The Buddha is not speaking about enlightenment, but about his determination to persevere. } 

“Mister\marginnote{11.1} Gotama mustn’t have experienced the kind of pleasant or painful feelings that would occupy the mind.”\footnote{For the idiom \textit{na hi \textsanskrit{nūna}}, compare \href{https://suttacentral.net/an8.86/en/sujato\#5.3}{AN 8.86:5.3} and \href{https://suttacentral.net/an4.67/en/sujato\#2.1}{AN 4.67:2.1}. } 

“How\marginnote{12.1} could I not, Aggivessana? Before my awakening—when I was still unawakened but intent on awakening—I thought: ‘Life at home is cramped and dirty, life gone forth is wide open. It’s not easy for someone living at home to lead the spiritual life utterly full and pure, like a polished shell. Why don’t I shave off my hair and beard, dress in ocher robes, and go forth from the lay life to homelessness?’ 

Some\marginnote{13.1} time later, while still with pristine black hair, blessed with youth, in the prime of life—though my mother and father wished otherwise, weeping with tearful faces—I shaved off my hair and beard, dressed in ocher robes, and went forth from the lay life to homelessness. 

Once\marginnote{13.2} I had gone forth I set out to discover what is skillful, seeking the supreme state of sublime peace. I approached \textsanskrit{Āḷāra} \textsanskrit{Kālāma} and said to him,\footnote{See \href{https://suttacentral.net/mn26/en/sujato\#14.1}{MN 26:14.1} ff. for notes. } ‘Reverend \textsanskrit{Kālāma}, I wish to lead the spiritual life in this teaching and training.’ 

\textsanskrit{Āḷāra}\marginnote{13.4} \textsanskrit{Kālāma} replied, ‘Stay, venerable. This teaching is such that a sensible person can soon realize their own tradition with their own insight and live having achieved it.’ 

I\marginnote{13.7} quickly memorized that teaching. As far as lip-recital and verbal repetition went, I spoke the doctrine of knowledge, the elder doctrine. I claimed to know and see, and so did others. 

Then\marginnote{13.9} it occurred to me, ‘It is not solely by mere faith that \textsanskrit{Āḷāra} \textsanskrit{Kālāma} declares: “I realize this teaching with my own insight, and live having achieved it.” Surely he meditates knowing and seeing this teaching.’ 

So\marginnote{13.12} I approached \textsanskrit{Āḷāra} \textsanskrit{Kālāma} and said to him, ‘Reverend \textsanskrit{Kālāma}, to what extent do you say you’ve realized this teaching with your own insight?’ When I said this, he declared the dimension of nothingness. 

Then\marginnote{13.15} it occurred to me, ‘It’s not just \textsanskrit{Āḷāra} \textsanskrit{Kālāma} who has faith, energy, mindfulness, immersion, and wisdom; I too have these things. Why don’t I make an effort to realize the same teaching that \textsanskrit{Āḷāra} \textsanskrit{Kālāma} says he has realized with his own insight?’ I quickly realized that teaching with my own insight, and lived having achieved it. 

So\marginnote{14.1} I approached \textsanskrit{Āḷāra} \textsanskrit{Kālāma} and said to him, ‘Reverend \textsanskrit{Kālāma}, is it up to this point that you realized this teaching with your own insight, and declare having achieved it?’ 

‘I\marginnote{14.3} have, reverend.’ 

‘I\marginnote{14.4} too have realized this teaching with my own insight up to this point, and live having achieved it.’ 

‘We\marginnote{14.5} are fortunate, reverend, so very fortunate to see a venerable such as yourself as one of our spiritual companions! So the teaching that I’ve realized with my own insight, and declare having achieved it, you’ve realized with your own insight, and dwell having achieved it. The teaching that you’ve realized with your own insight, and dwell having achieved it, I’ve realized with my own insight, and declare having achieved it. So the teaching that I know, you know, and the teaching you know, I know. I am like you and you are like me. Come now, reverend! We should both lead this community together.’ And that is how my tutor \textsanskrit{Āḷāra} \textsanskrit{Kālāma} placed me, his pupil, on the same position as him, and honored me with lofty praise. 

Then\marginnote{14.13} it occurred to me, ‘This teaching doesn’t lead to disillusionment, dispassion, cessation, peace, insight, awakening, and extinguishment. It only leads as far as rebirth in the dimension of nothingness.’ Realizing that this teaching was inadequate, I left disappointed. 

I\marginnote{15.1} set out to discover what is skillful, seeking the supreme state of sublime peace. I approached Uddaka son of \textsanskrit{Rāma} and said to him, ‘Reverend, I wish to lead the spiritual life in this teaching and training.’ 

Uddaka\marginnote{15.3} replied, ‘Stay, venerable. This teaching is such that a sensible person can soon realize their own tradition with their own insight and live having achieved it.’ 

I\marginnote{15.6} quickly memorized that teaching. As far as lip-recital and verbal repetition went, I spoke the doctrine of knowledge, the elder doctrine. I claimed to know and see, and so did others. 

Then\marginnote{15.8} it occurred to me, ‘It is not solely by mere faith that \textsanskrit{Rāma} declared: “I realize this teaching with my own insight, and live having achieved it.” Surely he meditated knowing and seeing this teaching.’ 

So\marginnote{15.11} I approached Uddaka son of \textsanskrit{Rāma} and said to him, ‘Reverend, to what extent did \textsanskrit{Rāma} say he’d realized this teaching with his own insight?’ When I said this, Uddaka son of \textsanskrit{Rāma} declared the dimension of neither perception nor non-perception. 

Then\marginnote{15.14} it occurred to me, ‘It’s not just \textsanskrit{Rāma} who had faith, energy, mindfulness, immersion, and wisdom; I too have these things. Why don’t I make an effort to realize the same teaching that \textsanskrit{Rāma} said he had realized with his own insight?’ I quickly realized that teaching with my own insight, and lived having achieved it. 

So\marginnote{15.22} I approached Uddaka son of \textsanskrit{Rāma} and said to him, ‘Reverend, had \textsanskrit{Rāma} realized this teaching with his own insight up to this point, and declared having achieved it?’ 

‘He\marginnote{15.24} had, reverend.’ 

‘I\marginnote{15.25} too have realized this teaching with my own insight up to this point, and live having achieved it.’ 

‘We\marginnote{15.26} are fortunate, reverend, so very fortunate to see a venerable such as yourself as one of our spiritual companions! The teaching that \textsanskrit{Rāma} had realized with his own insight, and declared having achieved it, you have realized with your own insight, and dwell having achieved it. The teaching that you’ve realized with your own insight, and dwell having achieved it, \textsanskrit{Rāma} had realized with his own insight, and declared having achieved it. So the teaching that \textsanskrit{Rāma} directly knew, you know, and the teaching you know, \textsanskrit{Rāma} directly knew. \textsanskrit{Rāma} was like you and you are like \textsanskrit{Rāma}. Come now, reverend! You should lead this community.’ And that is how my spiritual companion Uddaka son of \textsanskrit{Rāma} placed me in the position of a tutor and honored me with lofty praise. 

Then\marginnote{15.33} it occurred to me, ‘This teaching doesn’t lead to disillusionment, dispassion, cessation, peace, insight, awakening, and extinguishment. It only leads as far as rebirth in the dimension of neither perception nor non-perception.’ Realizing that this teaching was inadequate, I left disappointed. 

I\marginnote{16.1} set out to discover what is skillful, seeking the supreme state of sublime peace. Traveling stage by stage in the Magadhan lands, I arrived at \textsanskrit{Senānigama} in \textsanskrit{Uruvelā}. There I saw a delightful park, a lovely grove with a flowing river that was clean and charming, with smooth banks. And nearby was a village to resort to for alms. Then it occurred to me, ‘This park is truly delightful, a lovely grove with a flowing river that’s clean and charming, with smooth banks. And nearby there’s a village to resort to for alms. This is good enough for striving for a gentleman wanting to strive.’ So I sat down right there, thinking: ‘This is good enough for striving.’ 

And\marginnote{17.1} then these three similes, which were neither supernaturally inspired, nor learned before in the past, occurred to me.\footnote{Also at \href{https://suttacentral.net/mn85/en/sujato\#15.1}{MN 85:15.1} and \href{https://suttacentral.net/mn100/en/sujato\#14.1}{MN 100:14.1}. These similes, which illustrate the importance of detachment from sensual pleasures and the uselessness of fervent austerities, would fit better below, after the austerities and before \textit{\textsanskrit{jhāna}} (\href{https://suttacentral.net/mn36/en/sujato\#30.1}{MN 36:30.1}). A parallel to this passage preserved in a fragmentary Sanskrit manuscript of the \textsanskrit{Dīrgha} Āgama does indeed place these similes there, while other parallels vary the content or position of the similes to a degree. } Suppose there was a green, sappy log, and it was lying in water. Then a person comes along with a drill-stick, thinking to light a fire and produce heat. What do you think, Aggivessana? By drilling the stick against that green, sappy log lying in the water, could they light a fire and produce heat?”\footnote{\textit{Abhimanthento} must be “drilling” rather than “rubbing”, which does not produce fire. See too the simile of the man “drilling” into the head below (\href{https://suttacentral.net/mn36/en/sujato\#22.5}{MN 36:22.5}). } 

“No,\marginnote{17.7} Mister Gotama. Why not? Because it’s a green, sappy log, and it’s lying in the water. That person will eventually get weary and frustrated.” 

“In\marginnote{17.11} the same way, there are ascetics and brahmins who don’t live withdrawn in body and mind from sensual pleasures. They haven’t internally given up or stilled desire, affection, infatuation, thirst, and passion for sensual pleasures. Regardless of whether or not they feel painful, sharp, severe, acute feelings due to overexertion, they are incapable of knowledge and vision, of supreme awakening. This was the first example that occurred to me. 

Then\marginnote{18.1} a second example occurred to me. Suppose there was a green, sappy log, and it was lying on dry land far from the water. Then a person comes along with a drill-stick, thinking to light a fire and produce heat. What do you think, Aggivessana? By drilling the stick against that green, sappy log on dry land far from water, could they light a fire and produce heat?” 

“No,\marginnote{18.7} Mister Gotama. Why not? Because it’s still a green, sappy log, despite the fact that it’s lying on dry land far from water. That person will eventually get weary and frustrated.” 

“In\marginnote{18.11} the same way, there are ascetics and brahmins who live withdrawn in body and mind from sensual pleasures. But they haven’t internally given up or stilled desire, affection, infatuation, thirst, and passion for sensual pleasures. Regardless of whether or not they suffer painful, sharp, severe, acute feelings due to overexertion, they are incapable of knowledge and vision, of supreme awakening. This was the second example that occurred to me. 

Then\marginnote{19.1} a third example occurred to me. Suppose there was a dried up, withered log, and it was lying on dry land far from the water. Then a person comes along with a drill-stick, thinking to light a fire and produce heat. What do you think, Aggivessana? By drilling the stick against that dried up, withered log on dry land far from water, could they light a fire and produce heat?” 

“Yes,\marginnote{19.7} Mister Gotama. Why is that? Because it’s a dried up, withered log, and it’s lying on dry land far from water.” 

“In\marginnote{19.10} the same way, there are ascetics and brahmins who live withdrawn in body and mind from sensual pleasures. And they have internally given up and stilled desire, affection, infatuation, thirst, and passion for sensual pleasures. Regardless of whether or not they suffer painful, sharp, severe, acute feelings due to overexertion, they are capable of knowledge and vision, of supreme awakening. This was the third example that occurred to me. These are the three similes, which were neither supernaturally inspired, nor learned before in the past, that occurred to me. 

Then\marginnote{20.1} it occurred to me, ‘Why don’t I, with teeth clenched and tongue pressed against the roof of my mouth, squeeze, squash, and scorch mind with mind?’\footnote{This, the most gentle of the austere practices, was occasionally recommended by the Buddha as a last resort (\href{https://suttacentral.net/mn20/en/sujato\#7.2}{MN 20:7.2}). } So that’s what I did, until sweat ran from my armpits. It was like when a strong man grabs a weaker man by the head or throat or shoulder and squeezes, squashes, and crushes them. In the same way, with teeth clenched and tongue pressed against the roof of my mouth, I squeezed, squashed, and crushed mind with mind until sweat ran from my armpits. My energy was roused up and unflagging, and my mindfulness was established and lucid, but my body was disturbed, not tranquil, because I’d pushed too hard with that painful striving. But even such painful feeling did not occupy my mind. 

Then\marginnote{21.1} it occurred to me, ‘Why don’t I practice the breathless absorption?’\footnote{This practice seems like an extreme form of “breath control” (\textit{\textsanskrit{prāṇayāma}}) which was a prominent feature of later Yoga. While the breath played an extremely important role in pre-Buddhist texts, no corresponding practice is detailed. However, precursors may be seen in passages such as \textsanskrit{Bṛhadāraṇyaka} \textsanskrit{Upaniṣad} 3.7.16, where in response to \textsanskrit{Uddālaka}’s question, \textsanskrit{Yājnavalkya} says, “He who controls the breath within is the self, the inner controller, the immortal” (\textit{yaḥ \textsanskrit{prāṇamantaro} yamayati, \textsanskrit{eṣa} ta \textsanskrit{ātmāntaryāmyamṛtaḥ}}). \textsanskrit{Jaiminīya} \textsanskrit{Brāhmaņa} 1.42 describes a practice of deliberate asphyxiation to induce a near-death experience. \textsanskrit{Varuṇa} deprived his son \textsanskrit{Bhṛgu} of breath in order to send him on a journey to the “other world”. There \textsanskrit{Bhṛgu} saw men cutting other men to pieces and eating them, and other sights both horrifying and beautiful, all the while wondering if what he saw was real, before his breath was returned to him. } So I cut off my breathing through my mouth and nose. But then winds came out my ears making a loud noise, like the puffing of a blacksmith’s bellows. My energy was roused up and unflagging, and my mindfulness was established and lucid, but my body was disturbed, not tranquil, because I’d pushed too hard with that painful striving. But even such painful feeling did not occupy my mind. 

Then\marginnote{22.1} it occurred to me, ‘Why don’t I keep practicing the breathless absorption?’ So I cut off my breathing through my mouth and nose and ears. But then strong winds ground my head, like a strong man was drilling into my head with a sharp point. My energy was roused up and unflagging, and my mindfulness was established and lucid, but my body was disturbed, not tranquil, because I’d pushed too hard with that painful striving. But even such painful feeling did not occupy my mind. 

Then\marginnote{23.1} it occurred to me, ‘Why don’t I keep practicing the breathless absorption?’ So I cut off my breathing through my mouth and nose and ears. But then I got a severe headache, like a strong man was tightening a tough leather strap around my head. My energy was roused up and unflagging, and my mindfulness was established and lucid, but my body was disturbed, not tranquil, because I’d pushed too hard with that painful striving. But even such painful feeling did not occupy my mind. 

Then\marginnote{24.1} it occurred to me, ‘Why don’t I keep practicing the breathless absorption?’ So I cut off my breathing through my mouth and nose and ears. But then strong winds carved up my belly, like a deft butcher or their apprentice was slicing my belly open with a sharp meat cleaver. My energy was roused up and unflagging, and my mindfulness was established and lucid, but my body was disturbed, not tranquil, because I’d pushed too hard with that painful striving. But even such painful feeling did not occupy my mind. 

Then\marginnote{25.1} it occurred to me, ‘Why don’t I keep practicing the breathless absorption?’ So I cut off my breathing through my mouth and nose and ears. But then there was an intense burning in my body, like two strong men grabbing a weaker man by the arms to burn and scorch him on a pit of glowing coals. My energy was roused up and unflagging, and my mindfulness was established and lucid, but my body was disturbed, not tranquil, because I’d pushed too hard with that painful striving. But even such painful feeling did not occupy my mind. 

Then\marginnote{26.1} some deities saw me and said, ‘The ascetic Gotama is dead.’ Others said, ‘He’s not dead, but he’s dying.’ Others said, ‘He’s not dead or dying. The ascetic Gotama is a perfected one, for that is how the perfected ones live.’ 

Then\marginnote{27.1} it occurred to me, ‘Why don’t I practice completely cutting off food?’ But deities came to me and said, ‘Good sir, don’t practice totally cutting off food. If you do, we’ll infuse heavenly nectar into your pores and you will live on that.’ Then I thought, ‘If I claim to be completely fasting while these deities are infusing heavenly nectar in my pores, that would be a lie on my part.’ So I dismissed those deities, saying, ‘There’s no need.’ 

Then\marginnote{28.1} it occurred to me, ‘Why don’t I just take a little bit of food each time, a handful of broth made from mung beans, horse gram, chickpeas, or green gram?’ So that’s what I did, until my body became extremely emaciated. Due to eating so little, my major and minor limbs became like the joints of an eighty-year-old or a dying man, my bottom became like a camel’s hoof, my vertebrae stuck out like beads on a string, and my ribs were as gaunt as the broken-down rafters on an old barn. Due to eating so little, the gleam of my eyes sank deep in their sockets, like the gleam of water sunk deep down a well. Due to eating so little, my scalp shriveled and withered like a green bitter-gourd in the wind and sun. 

Due\marginnote{28.11} to eating so little, the skin of my belly stuck to my backbone, so that when I tried to rub the skin of my belly I grabbed my backbone, and when I tried to rub my backbone I rubbed the skin of my belly. Due to eating so little, when I tried to urinate or defecate I fell face down right there. Due to eating so little, when I tried to relieve my body by rubbing my limbs with my hands, the hair, rotted at its roots, fell out. 

Then\marginnote{29.1} some people saw me and said: ‘The ascetic Gotama is black.’ Some said: ‘He’s not black, he’s brown.’ Some said: ‘He’s neither black nor brown. The ascetic Gotama has tawny skin.’ That’s how far the pure, bright complexion of my skin had been ruined by taking so little food. 

Then\marginnote{30.1} I thought, ‘Whatever ascetics and brahmins have experienced painful, sharp, severe, acute feelings due to overexertion—whether in the past, future, or present—this is as far as it goes, no-one has done more than this. But I have not achieved any superhuman distinction in knowledge and vision worthy of the noble ones by this severe, grueling work. Could there be another path to awakening?’ 

Then\marginnote{31.1} it occurred to me, ‘I recall sitting in the cool shade of a black plum tree while my father the Sakyan was off working. Quite secluded from sensual pleasures, secluded from unskillful qualities, I entered and remained in the first absorption, which has the rapture and bliss born of seclusion, while placing the mind and keeping it connected.\footnote{The gentle and pleasant practice of \textit{\textsanskrit{jhāna}} is contrasted with the fiery austerities. The Bodhisatta had previously developed meditation under his former teachers, which included the practice of \textit{\textsanskrit{jhāna}}. But there the practice was embedded in theory, since he first memorized scripture, which would have taught him that such states were a revelation of the true self as the cosmic divinity of sheer consciousness. Thus what he rejected was “that teaching”, i.e. the system of philosophy and practice as a whole, which leads only to rebirth in a realm that, for all its sublimity, is still conditioned. Now he recalls entering deep meditation with the innocence of youth, following no system or philosophy. This showed him that peace of mind is a natural process of letting go, not dependent on (\textsanskrit{Upaniṣadic}) metaphysics. } Could that be the path to awakening?’ 

Stemming\marginnote{31.4} from that memory came the realization: ‘\emph{That} is the path to awakening!’\footnote{Compare \href{https://suttacentral.net/dhp274/en/sujato\#1}{Dhp 274:1}. } 

Then\marginnote{32.1} it occurred to me, ‘Why am I afraid of that pleasure, for it has nothing to do with sensual pleasures or unskillful qualities?’ Then I thought, ‘I’m not afraid of that pleasure, for it has nothing to do with sensual pleasures or unskillful qualities.’ 

Then\marginnote{33.1} I thought, ‘I can’t achieve that pleasure with a body so excessively emaciated. Why don’t I eat some solid food, some rice and porridge?’\footnote{The dependence of the mind on food was proven by \textsanskrit{Uddālaka} when he asked Śvetaketu to go without food for fifteen days, after which he could not recall any of the hymns (\textsanskrit{Chāndogya} \textsanskrit{Upaniṣad} 6.7). | Note how this contrasts with the unhealthy cycle of starving and fattening described above (\href{https://suttacentral.net/mn36/en/sujato\#6.5}{MN 36:6.5}). } So I ate some solid food. 

Now\marginnote{33.4} at that time the five mendicants were attending on me, thinking, ‘The ascetic Gotama will tell us of any truth that he realizes.’ But when I ate some solid food, they left disappointed in me, saying, ‘The ascetic Gotama has become indulgent; he has strayed from the struggle and returned to indulgence.’ 

After\marginnote{34.1} eating solid food and gathering my strength, quite secluded from sensual pleasures, secluded from unskillful qualities, I entered and remained in the first absorption, which has the rapture and bliss born of seclusion, while placing the mind and keeping it connected. But even such pleasant feeling did not occupy my mind. 

As\marginnote{35{-}37.1} the placing of the mind and keeping it connected were stilled, I entered and remained in the second absorption, which has the rapture and bliss born of immersion, with internal clarity and mind at one, without placing the mind and keeping it connected. But even such pleasant feeling did not occupy my mind. And with the fading away of rapture, I entered and remained in the third absorption, where I meditated with equanimity, mindful and aware, personally experiencing the bliss of which the noble ones declare, ‘Equanimous and mindful, one meditates in bliss.’ But even such pleasant feeling did not occupy my mind. With the giving up of pleasure and pain, and the ending of former happiness and sadness, I entered and remained in the fourth absorption, without pleasure or pain, with pure equanimity and mindfulness. But even such pleasant feeling did not occupy my mind. 

When\marginnote{38.1} my mind had immersed in \textsanskrit{samādhi} like this—purified, bright, flawless, rid of corruptions, pliable, workable, steady, and imperturbable—I extended it toward recollection of past lives. I recollected my many kinds of past lives, with features and details. 

This\marginnote{39.1} was the first knowledge, which I achieved in the first watch of the night. Ignorance was destroyed and knowledge arose; darkness was destroyed and light arose, as happens for a meditator who is diligent, keen, and resolute. But even such pleasant feeling did not occupy my mind. 

When\marginnote{40.1} my mind had immersed in \textsanskrit{samādhi} like this—purified, bright, flawless, rid of corruptions, pliable, workable, steady, and imperturbable—I extended it toward knowledge of the death and rebirth of sentient beings. With clairvoyance that is purified and superhuman, I saw sentient beings passing away and being reborn—inferior and superior, beautiful and ugly, in a good place or a bad place. I understood how sentient beings are reborn according to their deeds. 

This\marginnote{41.1} was the second knowledge, which I achieved in the middle watch of the night. Ignorance was destroyed and knowledge arose; darkness was destroyed and light arose, as happens for a meditator who is diligent, keen, and resolute. But even such pleasant feeling did not occupy my mind. 

When\marginnote{42.1} my mind had immersed in \textsanskrit{samādhi} like this—purified, bright, flawless, rid of corruptions, pliable, workable, steady, and imperturbable—I extended it toward knowledge of the ending of defilements. I truly understood: ‘This is suffering’ … ‘This is the origin of suffering’ … ‘This is the cessation of suffering’ … ‘This is the practice that leads to the cessation of suffering.’ I truly understood: ‘These are defilements’ … ‘This is the origin of defilements’ … ‘This is the cessation of defilements’ … ‘This is the practice that leads to the cessation of defilements.’ 

Knowing\marginnote{43.1} and seeing like this, my mind was freed from the defilements of sensuality, desire to be reborn, and ignorance. When it was freed, I knew it was freed. 

I\marginnote{43.3} understood: ‘Rebirth is ended; the spiritual journey has been completed; what had to be done has been done; there is nothing further for this place.’ 

This\marginnote{44.1} was the third knowledge, which I achieved in the last watch of the night. Ignorance was destroyed and knowledge arose; darkness was destroyed and light arose, as happens for a meditator who is diligent, keen, and resolute. But even such pleasant feeling did not occupy my mind. 

Aggivessana,\marginnote{45.1} I recall teaching the Dhamma to an assembly of many hundreds,\footnote{This passage is unique and its purpose here unclear, } and each person thought that I was teaching the Dhamma especially for them. But it should not be seen like this. The Realized One teaches others only so that they can understand. When that talk was finished, I stilled, settled, unified, and immersed my mind in \textsanskrit{samādhi} internally, using the same meditation subject as a basis of immersion that I used before, which I regularly use to meditate.”\footnote{Here \textit{\textsanskrit{samādhinimitta}} as usual means the subject of meditation, probably the breath. } 

“I’d\marginnote{45.7} believe that of Mister Gotama, just like a perfected one, a fully awakened Buddha.\footnote{Saccaka is not saying that the Buddha \emph{is} enlightened, merely he is \emph{like} one who is enlightened (\textit{\textsanskrit{yathā} \textsanskrit{taṁ} arahato}). } But do you ever recall sleeping during the day?”\footnote{Saccaka reveals his bad faith; he is more interested in trapping the Buddha than in learning. } 

“I\marginnote{46.1} do recall that in the last month of the summer, I have spread out my outer robe folded in four and lain down in the lion’s posture—on the right side, placing one foot on top of the other—mindful and aware.” 

“Some\marginnote{46.2} ascetics and brahmins call that a deluded abiding.”\footnote{He deflects by pointing to others, as he also did at \href{https://suttacentral.net/mn35/en/sujato\#11.4}{MN 35:11.4}. } 

“That’s\marginnote{46.3} not how to define whether someone is deluded or not. But as to how to define whether someone is deluded or not, listen and apply your mind well, I will speak.” 

“Yes,\marginnote{46.6} sir,” replied Saccaka. 

The\marginnote{46.7} Buddha said this: 

“Anyone\marginnote{47.1} who has not given up the defilements that are corrupting, leading to future lives, hurtful, resulting in suffering and future rebirth, old age, and death is deluded, I say. For it’s not giving up the defilements that makes you deluded. Anyone who has given up the defilements that are corrupting, leading to future lives, hurtful, resulting in suffering and future rebirth, old age, and death—is not deluded, I say. For it’s giving up the defilements that makes you not deluded. 

The\marginnote{47.5} Realized One has given up the defilements that are corrupting, leading to future lives, hurtful, resulting in suffering and future rebirth, old age, and death. He has cut them off at the root, made them like a palm stump, obliterated them so they are unable to arise in the future. Just as a palm tree with its crown cut off is incapable of further growth, in the same way, the Realized One has given up the defilements so they are unable to arise in the future.” 

When\marginnote{48.1} he had spoken, Saccaka said to him, “It’s incredible, Mister Gotama, it’s amazing! How when Mister Gotama is repeatedly attacked with inappropriate and intrusive criticism, the complexion of his skin brightens and the color of his face becomes clear, just like a perfected one, a fully awakened Buddha! 

I\marginnote{48.4} recall taking on \textsanskrit{Pūraṇa} Kassapa in debate. He dodged the issue, distracting the discussion with irrelevant points, and displaying annoyance, hate, and bitterness. But when Mister Gotama is repeatedly attacked with inappropriate and intrusive criticism, the complexion of his skin brightens and the color of his face becomes clear, just like a perfected one, a fully awakened Buddha. 

I\marginnote{48.7} recall taking on the bamboo-staffed ascetic \textsanskrit{Gosāla}, Ajita of the hair blanket, Pakudha \textsanskrit{Kaccāyana}, \textsanskrit{Sañjaya} \textsanskrit{Belaṭṭhiputta}, and the Jain ascetic of the \textsanskrit{Ñātika} clan in debate.\footnote{This shows that despite his Jain parentage, and despite his citation of \textsanskrit{Ājīvakas} above, Saccaka was no adherent of these philosophies. It seems that he was a contrarian at heart. } They all dodged the issue, distracting the discussion with irrelevant points, and displaying annoyance, hate, and bitterness. But when Mister Gotama is repeatedly attacked with inappropriate and intrusive criticism, the complexion of his skin brightens and the color of his face becomes clear, just like a perfected one, a fully awakened Buddha. 

Well,\marginnote{48.14} now, Mister Gotama, I must go. I have many duties, and much to do.” 

“Please,\marginnote{48.16} Aggivessana, go at your convenience.” 

Then\marginnote{48.17} Saccaka, the son of Jain parents, having approved and agreed with what the Buddha said, got up from his seat and left. 

%
\section*{{\suttatitleacronym MN 37}{\suttatitletranslation The Shorter Discourse on the Ending of Craving }{\suttatitleroot Cūḷataṇhāsaṅkhayasutta}}
\addcontentsline{toc}{section}{\tocacronym{MN 37} \toctranslation{The Shorter Discourse on the Ending of Craving } \tocroot{Cūḷataṇhāsaṅkhayasutta}}
\markboth{The Shorter Discourse on the Ending of Craving }{Cūḷataṇhāsaṅkhayasutta}
\extramarks{MN 37}{MN 37}

\scevam{So\marginnote{1.1} I have heard. }At one time the Buddha was staying near \textsanskrit{Sāvatthī} in the stilt longhouse of \textsanskrit{Migāra}’s mother in the Eastern Monastery. 

And\marginnote{2.1} then Sakka, lord of gods, went up to the Buddha, bowed, stood to one side, and said to him:\footnote{A long account of Sakka’s first meeting with the Buddha and conversion is found at \href{https://suttacentral.net/dn21/en/sujato}{DN 21}. There he is said to have attained stream-entry, and so is destined for full awakening. Yet even a stream-enterer may slow their progress by negligence (\href{https://suttacentral.net/snp2.1/en/sujato\#9.3}{Snp 2.1:9.3}) | Sakka’s epithet “lord of gods” (\textit{\textsanskrit{devānamindo}}) evokes his more familiar Sanskrit name, Indra. } 

“Sir,\marginnote{2.2} how do you briefly define a mendicant who is freed through the ending of craving, who has reached the ultimate end, the ultimate sanctuary from the yoke, the ultimate spiritual life, the ultimate goal, and is best among gods and humans?”\footnote{This question directly follows on from the final question at \href{https://suttacentral.net/dn21/en/sujato\#2.6.9}{DN 21:2.6.9}, where Sakka asks \emph{whether} all ascetics and brahmins have reached the ultimate goal. Here he delves deeper into what that means. | Now that he is a confirmed Buddhist, Sakka switches from “ascetics and brahmins” to “mendicants”. | This passage has an exact parallel at \href{https://suttacentral.net/an7.61/en/sujato\#12.2}{AN 7.61:12.2}, and a variation at \href{https://suttacentral.net/sn35/en/sujato\#80.1}{SN 35:80.1}. } 

“lord\marginnote{3.1} of gods, take a mendicant who has heard: ‘Nothing is worth insisting on.’\footnote{The Buddha is calling back to the previous conversation, where immediately before asking about the ultimate goal, Sakka had asked why all ascetics did not share a common doctrine (\href{https://suttacentral.net/dn21/en/sujato\#2.6.6}{DN 21:2.6.6}). The Buddha answered that, though the world has many and diverse elements, ascetics tend to fixate on one aspect and “insist” it is the only truth. } When a mendicant has heard that nothing is worth insisting on, they directly know all things. Directly knowing all things, they completely understand all things. Completely understanding all things, when they experience any kind of feeling—pleasant, unpleasant, or neutral—\footnote{One who has learned the theory of non-attachment is still liable to become attached, so they apply their knowledge to experiential insight into “all things”. Normally “complete understanding” (\textit{\textsanskrit{pariññā}}) indicates arahantship, but here it reinforces “direct knowledge” (\textit{\textsanskrit{abhiññā}}) as the wisdom of stream-entry. However, the force of attachment is so deeply ingrained that even a stream-enterer, such as Sakka, with both theoretical and experiential wisdom is still not free. So they must then continue to meditate on the feelings in order to let go fully. } they meditate observing impermanence, dispassion, cessation, and letting go in those feelings. Meditating in this way, they don’t grasp at anything in the world. Not grasping, they’re not anxious. Not being anxious, they personally become extinguished. They understand: ‘Rebirth is ended, the spiritual journey has been completed, what had to be done has been done, there is nothing further for this place.’ That’s how I briefly define a mendicant who is freed through the ending of craving, who has reached the ultimate end, the ultimate sanctuary from the yoke, the ultimate spiritual life, the ultimate goal, and is best among gods and humans.” 

Then\marginnote{4.1} Sakka, lord of gods, having approved and agreed with what the Buddha said, bowed and respectfully circled the Buddha, keeping him on his right, before vanishing right there. 

Now\marginnote{5.1} at that time Venerable \textsanskrit{Mahāmoggallāna} was sitting not far from the Buddha. He thought, “Did that spirit comprehend what the Buddha said when he agreed with him, or not?\footnote{\textsanskrit{Moggallāna} refers to Sakka as a “spirit” (\textit{yakkha}). } Why don’t I find out?” 

And\marginnote{6.1} then Venerable \textsanskrit{Mahāmoggallāna}, as easily as a strong person would extend or contract their arm, vanished from the Eastern Monastery and reappeared among the gods of the thirty-three.\footnote{\textsanskrit{Moggallāna} was the foremost in psychic powers. } Now at that time Sakka was amusing himself in the Single Lotus Park, supplied and provided with a heavenly orchestra.\footnote{A \textit{\textsanskrit{pañcaṅgikatūriya}} is a band of five musicians, a “quintet”. Here it is amplified a hundredfold, so I translate “orchestra”. | “Single lotus” (\textit{\textsanskrit{ekapuṇḍarīka}}) appears as an auspicious epithet in a prayer for wealth at \textsanskrit{Bṛhadāraṇyaka} \textsanskrit{Upaniṣad} 6.3.6. } 

Seeing\marginnote{7.2} \textsanskrit{Mahāmoggallāna} coming off in the distance, he dismissed the orchestra, approached \textsanskrit{Mahāmoggallāna}, and said, “Come, my good \textsanskrit{Moggallāna}! Welcome, good sir! It’s been a long time since you took the opportunity to come here. Sit, my good \textsanskrit{Moggallāna}, this seat is for you.” \textsanskrit{Mahāmoggallāna} sat down on the seat spread out, while Sakka took a low seat and sat to one side. 

\textsanskrit{Mahāmoggallāna}\marginnote{7.9} said to him, “Kosiya, how did the Buddha briefly explain freedom through the ending of craving?\footnote{\textsanskrit{Moggallāna} refers to Sakka as “Kosiya” rather than the Buddha’s “lord of gods”. Old gods like Sakka tend to accrue many names. The word \textit{kosiya} is explained by the commentaries as “owl”, which, if correct, would have been the totem for a clan of that name. It is, however, a patronymic: Rig Veda 1.10.11 has \textit{indra \textsanskrit{kauśika}} which means “Indra, son of \textsanskrit{Kuśika} (or \textsanskrit{Kuśa})”. \textit{\textsanskrit{Kuśa}} grass is critical to the performance of Vedic rites, and the label probably initially implied “Brahmanized”, i.e. a king whose reign was authorized according to Vedic ritual. Kosiya is said to be a low class family name (\href{https://suttacentral.net/pli-tv-bu-vb-pc2/en/sujato\#2.1.18}{Bu Pc 2:2.1.18}). } Please share this talk with me so that I can also get to hear it.” 

“My\marginnote{8.3} good \textsanskrit{Moggallāna}, I have many duties, and much to do, not only for myself, but also for the gods of the thirty-three. Still, what is properly heard, learned, attended, and memorized does not vanish all of a sudden.\footnote{The commentary reads \textit{no} as a personal pronoun (\textit{\textsanskrit{amhākaṁ}}) which results in the sense, “Even what I have well remembered quickly vanishes.” This is logically dubious—how is it “properly learned” if it quickly vanishes?—and contradicted below where Sakka shows that he does in fact remember perfectly well (\href{https://suttacentral.net/mn37/en/sujato\#12.5}{MN 37:12.5}). Instead, read \textit{no} as negative particle. Sakka is brushing \textsanskrit{Moggallāna} off by insisting he has learned his lesson, without taking the trouble to actually remember it. } Once upon a time, a battle was fought between the gods and the titans.\footnote{He tries to distract \textsanskrit{Moggallāna} with talk of his glory days, rehashing vendettas that were old even in the Vedas. } In that battle the gods won and the titans lost. When I returned from that battle as a conqueror, I created the Palace of Victory. The Palace of Victory has a hundred towers. Each tower has seven hundred chambers. Each chamber has seven nymphs. Each nymph has seven maids. Would you like to see the lovely Palace of Victory?” \textsanskrit{Mahāmoggallāna} consented with silence. 

Then,\marginnote{9.1} putting Venerable \textsanskrit{Mahāmoggallāna} in front, Sakka, lord of gods, and \textsanskrit{Vessavaṇa}, the Great King, went to the Palace of Victory.\footnote{Also known as Kuvera (\href{https://suttacentral.net/dn20/en/sujato\#9.31}{DN 20:9.31}, \href{https://suttacentral.net/snp2.14/en/sujato\#6.1}{Snp 2.14:6.1}), the name \textsanskrit{Vessavaṇa} means “Son of the Renowned” from his father \textsanskrit{Viśrava}, although \href{https://suttacentral.net/dn32/en/sujato\#7.40}{DN 32:7.40} explains it as from the name of his city. He is one of the four great kings and guards the northern quarter. } When they saw \textsanskrit{Moggallāna} coming off in the distance, Sakka’s maids, being discreet and prudent, each went to her own bedroom.\footnote{The maids are described as possessing the cardinal virtues of \textit{hiri} (conscience, discretion) and \textit{otappa} (prudence), the “guardians of the world” and the foundations of moral integrity. Former translators have rendered this as if the maids were “embarrassed and ashamed”, but this is unwarranted. The maids were born in this realm as the result of their good kamma in past lives and are simply residing there in their own home. Unlike Sakka, who neglected his lesson and then tried to distract attention when reminded of it, they have nothing to be ashamed of. } They were just like a daughter-in-law who is discreet and prudent when they see their father-in-law. 

Then\marginnote{10.3} Sakka and \textsanskrit{Vessavaṇa} encouraged \textsanskrit{Moggallāna} to wander and explore the palace, saying,\footnote{This is a good example of \textit{(anu)-\textsanskrit{vicāra}} in the sense of “explore”. } “See, in the palace, my good \textsanskrit{Moggallāna}, this lovely thing! And that lovely thing!”\footnote{The gods are pointing out various pretty features in the palace. } 

“That\marginnote{10.6} looks beautiful for the venerable Kosiya, as befits one who has made merit in the past.\footnote{After Sakka’s disappointing response, \textsanskrit{Moggallāna} is trying to say something nice. At the same time, he subtly implies the difference between his past merit and present negligence. } Humans, when they see something lovely, also say: ‘It looks beautiful enough for the gods of the thirty-three!’ That looks beautiful for the venerable Kosiya, as befits one who has made merit in the past.” 

Then\marginnote{11.1} \textsanskrit{Moggallāna} thought, “This spirit lives much too negligently. Why don’t I stir up a sense of urgency in him?” 

Then\marginnote{11.4} \textsanskrit{Moggallāna} used his psychic power to make the Palace of Victory shake and rock and tremble with his big toe.\footnote{Referred to at \href{https://suttacentral.net/thag20.1/en/sujato\#55.1}{Thag 20.1:55.1}. \textsanskrit{Moggallāna} displayed a similar wonder at \href{https://suttacentral.net/sn51.14/en/sujato\#3.1}{SN 51.14:3.1}, also a means of last resort in the face of negligence. } Then Sakka, \textsanskrit{Vessavaṇa}, and the gods of the thirty-three, their minds full of wonder and amazement, thought, “Oh, how incredible, how amazing! The ascetic has such power and might that he makes the god’s home shake and rock and tremble with his big toe!” 

Knowing\marginnote{12.1} that Sakka was shocked and awestruck, \textsanskrit{Moggallāna} said to him, “Kosiya, how did the Buddha briefly explain freedom through the ending of craving? Please share this talk with me so that I can also get to hear it.” 

“My\marginnote{12.4} dear \textsanskrit{Moggallāna}, I approached the Buddha, bowed, stood to one side, and said to him, ‘Sir, how do you briefly define a mendicant who is freed with the ending of craving, who has reached the ultimate end, the ultimate sanctuary from the yoke, the ultimate spiritual life, the ultimate goal, and is best among gods and humans?’ 

When\marginnote{13.1} I had spoken the Buddha said to me: ‘lord of gods, it’s when a mendicant has heard: “Nothing is worth insisting on” When a mendicant has heard that nothing is worth insisting on, they directly know all things. Directly knowing all things, they completely understand all things. Having completely understood all things, when they experience any kind of feeling—pleasant, unpleasant, or neutral—they meditate observing impermanence, dispassion, cessation, and letting go in those feelings. Meditating in this way, they don’t grasp at anything in the world. Not grasping, they’re not anxious. Not being anxious, they personally become extinguished. They understand: “Rebirth is ended, the spiritual journey has been completed, what had to be done has been done, there is nothing further for this place.” That’s how I briefly define a mendicant who is freed through the ending of craving, who has reached the ultimate end, the ultimate sanctuary from the yoke, the ultimate spiritual life, the ultimate goal, and is best among gods and humans.’ That’s how the Buddha briefly explained freedom through the ending of craving to me.” 

\textsanskrit{Moggallāna}\marginnote{14.1} approved and agreed with what Sakka said. As easily as a strong person would extend or contract their arm, he vanished from among the gods of the thirty-three and reappeared in the Eastern Monastery. 

Soon\marginnote{14.2} after \textsanskrit{Moggallāna} left, Sakka’s maids said to him, “Good sir, was that the Blessed One, your Teacher?” 

“No,\marginnote{14.4} it was not. That was my spiritual companion Venerable \textsanskrit{Mahāmoggallāna}.”\footnote{“Spiritual companion” (\textit{\textsanskrit{sabrahmacārī}}) normally refers to a fellow monastic, but here it alludes to the fact that both of them have seen the Dhamma. } 

“You’re\marginnote{14.6} fortunate, good sir, so very fortunate, to have a spiritual companion of such power and might! We can’t believe that’s not the Blessed One, your Teacher!” 

Then\marginnote{15.1} \textsanskrit{Mahāmoggallāna} went up to the Buddha, bowed, sat down to one side, and said to him, “Sir, do you recall briefly explaining freedom through the ending of craving to a certain well-known and illustrious spirit?”\footnote{It is unclear why \textsanskrit{Moggallāna} asked for a repetition of the teaching at which he was present. One might have expected that he would report to the Buddha the events at Sakka’s palace. } 

“I\marginnote{15.3} do, \textsanskrit{Moggallāna}.” And the Buddha retold all that happened when Sakka came to visit him, adding: 

“That’s\marginnote{15.5} how I recall briefly explaining freedom through the ending of craving to Sakka, lord of gods.” 

That\marginnote{15.18} is what the Buddha said. Satisfied, Venerable \textsanskrit{Mahāmoggallāna} approved what the Buddha said. 

%
\section*{{\suttatitleacronym MN 38}{\suttatitletranslation The Longer Discourse on the Ending of Craving }{\suttatitleroot Mahātaṇhāsaṅkhayasutta}}
\addcontentsline{toc}{section}{\tocacronym{MN 38} \toctranslation{The Longer Discourse on the Ending of Craving } \tocroot{Mahātaṇhāsaṅkhayasutta}}
\markboth{The Longer Discourse on the Ending of Craving }{Mahātaṇhāsaṅkhayasutta}
\extramarks{MN 38}{MN 38}

\scevam{So\marginnote{1.1} I have heard. }At one time the Buddha was staying near \textsanskrit{Sāvatthī} in Jeta’s Grove, \textsanskrit{Anāthapiṇḍika}’s monastery. 

Now\marginnote{2.1} at that time a mendicant called \textsanskrit{Sāti}, the fisherman’s son, had the following harmful misconception:\footnote{The opening of this sutta is similar to \href{https://suttacentral.net/mn22/en/sujato}{MN 22}. } “As I understand the Buddha’s teaching, it is this very same consciousness that roams and transmigrates, not another.”\footnote{\textsanskrit{Sāti} attributes three teachings to the Buddha. First, that there is a “transmigration” (\textit{\textsanskrit{saṁsāra}}) from one life to another. Second, that the primary locus of transmigration is “consciousness” (\textit{\textsanskrit{viññāṇa}}). And thirdly, that the consciousness that transmigrates remains “this very same” (\textit{\textsanskrit{tadevidaṁ}}), not another (\textit{\textsanskrit{anaññaṁ}}); in other words, it retains its self-same identity through the process of rebirth. The Buddha did in fact teach the first two of these ideas, but not the third, as he will explain below. | The \textsanskrit{Bṛhadāraṇyaka} \textsanskrit{Upaniṣad} says that as death approaches, the senses and vital energies withdraw into the heart (\textit{\textsanskrit{hṛdaya}}), from the top of which the self departs. That same consciousness proceeds to a new body (4.4.2: \textit{\textsanskrit{savijñāno} bhavati, \textsanskrit{savijñānamevānvavakrāmati}}). This core \textsanskrit{Upaniṣadic} chapter on rebirth reflects \textsanskrit{Sāti}’s wording as well as his meaning. \textsanskrit{Sāti} asserts emphatic identity using doubled demonstrative pronouns conjoined with \textit{(e)va} (\textit{\textsanskrit{tadevidaṁ}}), and identical constructions are found throughout the \textsanskrit{Bṛhadāraṇyaka} chapter: \textit{sa \textsanskrit{vā} ayam} (4.4.5), \textit{sa \textsanskrit{vā} \textsanskrit{eṣa}} (4.4.22, 4.4.24, 4.4.25); see also \textit{tameva} (4.4.17). For \textit{\textsanskrit{anaññaṁ}} we find the inverse \textit{anya} for the “other” body (4.4.3, 4.4.4). For the Pali verbs \textit{\textsanskrit{sandhāvati} \textsanskrit{saṁsarati}} we have instead \textit{\textsanskrit{avakrāmati}} (4.4.1, 4.4.2). But the connection with \textit{\textsanskrit{saṁsarati}} is made in the Brahmanical tradition itself, for it says below, “That self is indeed divinity, made of consciousness” (\textit{sa \textsanskrit{vā} \textsanskrit{ayamātmā} brahma \textsanskrit{vijñānamayo}}; 4.4.5, see too 4.4.22), which the commentator \textsanskrit{Śaṅkara} explains as “the transmigrating self” (\textit{\textsanskrit{saṁsaratyātmā}}). } 

Several\marginnote{3.1} mendicants heard about this. They went up to \textsanskrit{Sāti} and said to him, “Is it really true, Reverend \textsanskrit{Sāti}, that you have such a harmful misconception: ‘As I understand the Buddha’s teaching, it is this very same consciousness that roams and transmigrates, not another’?” 

“Absolutely,\marginnote{3.7} reverends. As I understand the Buddha’s teaching, it is this very same consciousness that roams and transmigrates, not another.” 

Then,\marginnote{3.8} wishing to dissuade \textsanskrit{Sāti} from his view, the mendicants pursued, pressed, and grilled him, “Don’t say that, \textsanskrit{Sāti}! Don’t misrepresent the Buddha, for misrepresentation of the Buddha is not good. And the Buddha would not say that. In many ways the Buddha has said that consciousness is dependently originated, since without a cause, consciousness does not come to be.”\footnote{If consciousness is dependent it is changeable and cannot be “that very same”. The Buddha spoke of consciousness as a process of phenomena evolving and flowing, ever changing like a stream. } 

But\marginnote{3.11} even though the mendicants pressed him in this way, \textsanskrit{Sāti} obstinately stuck to his misconception and insisted on it. 

When\marginnote{4.1} they weren’t able to dissuade \textsanskrit{Sāti} from his view, the mendicants went to the Buddha, bowed, sat down to one side, and told him what had happened. 

So\marginnote{5.1} the Buddha addressed one of the monks, “Please, monk, in my name tell the mendicant \textsanskrit{Sāti} that the teacher summons him.” 

“Yes,\marginnote{5.4} sir,” that monk replied. He went to \textsanskrit{Sāti} and said to him, “Reverend \textsanskrit{Sāti}, the teacher summons you.” 

“Yes,\marginnote{5.6} reverend,” \textsanskrit{Sāti} replied. He went to the Buddha, bowed, and sat down to one side. The Buddha said to him, “Is it really true, \textsanskrit{Sāti}, that you have such a harmful misconception: ‘As I understand the Buddha’s teaching, it is this very same consciousness that roams and transmigrates, not another’?” 

“Absolutely,\marginnote{5.9} sir. As I understand the Buddha’s teaching, it is this very same consciousness that roams and transmigrates, not another.” 

“\textsanskrit{Sāti},\marginnote{5.10} what is that consciousness?” 

“Sir,\marginnote{5.11} he is the speaker, the knower who experiences the results of good and bad deeds in all the different realms.”\footnote{See \href{https://suttacentral.net/mn2/en/sujato\#8.8}{MN 2:8.8}. } 

“Futile\marginnote{5.12} man, who on earth have you ever known me to teach in that way? Haven’t I said in many ways that consciousness is dependently originated, since consciousness does not arise without a cause? But still you misrepresent me by your wrong grasp, harm yourself, and create much wickedness. This will be for your lasting harm and suffering.” 

Then\marginnote{6.1} the Buddha said to the mendicants, “What do you think, mendicants? Has this mendicant \textsanskrit{Sāti} kindled even a spark of ardor in this teaching and training?”\footnote{See \href{https://suttacentral.net/mn22/en/sujato\#7.3}{MN 22:7.3}. } 

“How\marginnote{6.4} could that be, sir? No, sir.” When this was said, \textsanskrit{Sāti} sat silent, dismayed, shoulders drooping, downcast, depressed, with nothing to say. 

Knowing\marginnote{6.7} this, the Buddha said, “Futile man, you will be known by your own harmful misconception. I’ll question the mendicants about this.” 

Then\marginnote{7.1} the Buddha said to the mendicants, “Mendicants, do you understand my teachings as \textsanskrit{Sāti} does, when he misrepresents me by his wrong grasp, harms himself, and creates much wickedness?” 

“No,\marginnote{7.3} sir. For in many ways the Buddha has told us that consciousness is dependently originated, since without a cause, consciousness does not come to be.” 

“Good,\marginnote{7.5} good, mendicants! It’s good that you understand my teaching like this. For in many ways I have told you that consciousness is dependently originated, since without a cause, consciousness does not come to be. But still this \textsanskrit{Sāti} misrepresents me by his wrong grasp, harms himself, and creates much wickedness. This will be for his lasting harm and suffering. 

Consciousness\marginnote{8.1} is reckoned according to the very same condition dependent upon which it arises.\footnote{The Buddha’s use of duplicated pronouns with \textit{eva} here echoes \textsanskrit{Sāti}’s language, but to the opposite effect. Rather than emphasizing the self-sameness of transmigrating consciousness, the Buddha states with equal emphasis the dependence of consciousness on specific conditions, whatever they may be. } Consciousness that arises dependent on the eye and sights is reckoned as eye consciousness. Consciousness that arises dependent on the ear and sounds is reckoned as ear consciousness. Consciousness that arises dependent on the nose and smells is reckoned as nose consciousness. Consciousness that arises dependent on the tongue and tastes is reckoned as tongue consciousness. Consciousness that arises dependent on the body and touches is reckoned as body consciousness. Consciousness that arises dependent on the mind and ideas is reckoned as mind consciousness. 

It’s\marginnote{8.8} like fire, which is reckoned according to the very same condition dependent upon which it burns.\footnote{A similar argument is made in the context of caste at \href{https://suttacentral.net/mn93/en/sujato\#11.5}{MN 93:11.5}. } A fire that burns dependent on logs is reckoned as a log fire. A fire that burns dependent on twigs is reckoned as a twig fire. A fire that burns dependent on grass is reckoned as a grass fire. A fire that burns dependent on cow-dung is reckoned as a cow-dung fire. A fire that burns dependent on husks is reckoned as a husk fire. A fire that burns dependent on rubbish is reckoned as a rubbish fire. 

In\marginnote{8.15} the same way, consciousness is reckoned according to the very same condition dependent upon which it arises. … 

Mendicants,\marginnote{9.1} do you see that this has come to be?”\footnote{“This has come to be” (\textit{\textsanskrit{bhūtamidaṁ}}) refers to dependently originated consciousness (implied by the neuter pronoun \textit{\textsanskrit{idaṁ}}). See \href{https://suttacentral.net/sn12.31/en/sujato\#7.1}{SN 12.31:7.1}. } 

“Yes,\marginnote{9.2} sir.” 

“Do\marginnote{9.3} you see that it originated with that as fuel?” 

“Yes,\marginnote{9.4} sir.” 

“Do\marginnote{9.5} you see that when that fuel ceases, what has come to be is liable to cease?” 

“Yes,\marginnote{9.6} sir.” 

“Does\marginnote{10.1} doubt arise when you’re uncertain whether or not this has come to be?” 

“Yes,\marginnote{10.2} sir.” 

“Does\marginnote{10.3} doubt arise when you’re uncertain whether or not this has originated with that as fuel?” 

“Yes,\marginnote{10.4} sir.” 

“Does\marginnote{10.5} doubt arise when you’re uncertain whether or not when that fuel ceases, what has come to be is liable to cease?” 

“Yes,\marginnote{10.6} sir.” 

“Is\marginnote{11.1} doubt given up in someone who truly sees with right understanding that this has come to be?”\footnote{This is the stream-enterer, who has seen dependent origination and given up doubt. } 

“Yes,\marginnote{11.2} sir.” 

“Is\marginnote{11.3} doubt given up in someone who truly sees with right understanding that this has originated with that as fuel?” 

“Yes,\marginnote{11.4} sir.” 

“Is\marginnote{11.5} doubt given up in someone who truly sees with right understanding that when that fuel ceases, what has come to be is liable to cease?” 

“Yes,\marginnote{11.6} sir.” 

“Are\marginnote{12.1} you free of doubt as to whether this has come to be?” 

“Yes,\marginnote{12.2} sir.” 

“Are\marginnote{12.3} you free of doubt as to whether this has originated with that as fuel?” 

“Yes,\marginnote{12.4} sir.” 

“Are\marginnote{12.5} you free of doubt as to whether when that fuel ceases, what has come to be is liable to cease?” 

“Yes,\marginnote{12.6} sir.” 

“Have\marginnote{13.1} you truly seen clearly with right understanding that this has come to be?” 

“Yes,\marginnote{13.2} sir.” 

“Have\marginnote{13.3} you truly seen clearly with right understanding that this has originated with that as fuel?” 

“Yes,\marginnote{13.4} sir.” 

“Have\marginnote{13.5} you truly seen clearly with right understanding that when that fuel ceases, what has come to be is liable to cease?” 

“Yes,\marginnote{13.6} sir.” 

“Pure\marginnote{14.1} and bright as this view is, mendicants, if you cherish it, fancy it, treasure it, and treat it as your own, would you be understanding my simile of the teaching as a raft: for crossing over, not for holding on?”\footnote{An allusion to \href{https://suttacentral.net/mn22/en/sujato\#13.1}{MN 22:13.1}. The verbs here are used of children playing with sandcastles at \href{https://suttacentral.net/sn23.2/en/sujato\#2.2}{SN 23.2:2.2}. } 

“No,\marginnote{14.2} sir.” 

“Pure\marginnote{14.3} and bright as this view is, mendicants, if you don’t cherish it, fancy it, treasure it, and treat it as your own, would you be understanding my simile of the teaching as a raft: for crossing over, not for holding on?” 

“Yes,\marginnote{14.4} sir.” 

“Mendicants,\marginnote{15.1} there are these four fuels. They maintain sentient beings that have been born and help those that are about to be born. What four? Solid food, whether solid or subtle; contact is the second, mental intention the third, and consciousness the fourth.\footnote{As at \href{https://suttacentral.net/mn9/en/sujato\#11.4}{MN 9:11.4}. } 

What\marginnote{16.1} is the source, origin, birthplace, and inception of these four fuels?\footnote{The word \textit{\textsanskrit{āhāra}} (“fuel”, “food”, “nutriment”) means literally “intake”, and is etymologically parallel to \textit{\textsanskrit{upādāna}}, “grasping”, “uptake”. Both terms have dual senses, on the one hand denoting fuel or sustenance, and on the other grasping and attachment. That is why here (as at \href{https://suttacentral.net/mn9/en/sujato\#11.5}{MN 9:11.5}), \textit{\textsanskrit{āhāra}} is created by craving, just like \textit{\textsanskrit{upādāna}} in the standard sequence (\href{https://suttacentral.net/mn38/en/sujato\#17.8}{MN 38:17.8}). } Craving. 

And\marginnote{16.3} what is the source of craving? Feeling. 

And\marginnote{16.5} what is the source of feeling? Contact. 

And\marginnote{16.7} what is the source of contact? The six sense fields. 

And\marginnote{16.9} what is the source of the six sense fields? Name and form. 

And\marginnote{16.11} what is the source of name and form? Consciousness. 

And\marginnote{16.13} what is the source of consciousness? Choices. 

And\marginnote{16.15} what is the source of choices? Ignorance. 

So,\marginnote{17.1} ignorance is a condition for choices.\footnote{Here begins the full presentation of the standard sequence of dependent origination in forward order. Formal definitions are found at \href{https://suttacentral.net/sn12.2/en/sujato}{SN 12.2}. Here I briefly indicate the nature of the conditioned links. | Because we are ignorant of the four noble truths, we make morally potent choices by body, speech, and mind. } Choices are a condition for consciousness.\footnote{These choices are creative forces or energies in the mind that sustain the ongoing stream of sense consciousness from one life to the next. } Consciousness is a condition for name and form.\footnote{Consciousness functions in relation to a cluster of phenomena both mental—feeling, perception, intention, contact, and application of mind—and physical—the four elements. These form an organism that grows and evolves. } Name and form are conditions for the six sense fields.\footnote{The sentient organism of the body requires senses to feed it stimuli. } The six sense fields are conditions for contact.\footnote{Through these the sentient organism encounters the world outside and learns to make sense of it. } Contact is a condition for feeling.\footnote{It distinguishes experiences that are pleasant, unpleasant, and neutral. } Feeling is a condition for craving.\footnote{It reacts by wanting to have more pleasure and to escape pain. } Craving is a condition for grasping.\footnote{Grasping at pleasures, view, observances, and theories of self, one makes sense of the world so as to optimize the capacity of oneself to experience pleasure. } Grasping is a condition for continued existence.\footnote{This grasping binds one to time, to a continuity of existence in the realms of the senses or those of refined consciousness. } Continued existence is a condition for rebirth.\footnote{Shedding the body one takes up a new one in one of the realms of existence, perpetuating the cycle. } Rebirth is a condition for old age and death, sorrow, lamentation, pain, sadness, and distress to come to be.\footnote{Being born, it is inevitable that one will experience the pains of broken teeth, wrinkled skin, crooked back, and ultimately the failure of the body that we call death. } That is how this entire mass of suffering originates. 

‘Rebirth\marginnote{18.1} is a condition for old age and death.’ That’s what I said.\footnote{The Buddha grills his students, reinforcing learning by making sure they understand each point. } Is that how you see this or not?” 

“That’s\marginnote{18.3} how we see it.” 

“‘Continued\marginnote{18.6} existence is a condition for rebirth.’ … 

‘Ignorance\marginnote{18.50} is a condition for choices.’ That’s what I said. Is that how you see this or not?” 

“That’s\marginnote{18.52} how we see it.” 

“Good,\marginnote{19.1} mendicants! So both you and I say this. When this exists, that is; due to the arising of this, that arises. That is:\footnote{This is the abstract principle of dependent origination. It establishes that dependent origination is concerned, not with universal truisms such as “everything is connected” or “everything must have a cause”, but with establishing specific links between one thing and another. This is a form of necessary condition—without one thing, the other cannot be. But it is stronger than mere necessity, as each condition is a close and vital support for its descendant. This abstract principle is often called “specific conditionality” (\textit{\textsanskrit{idappaccayatā}}), but note that in the suttas \textit{\textsanskrit{idappaccayatā}} is a synonym of dependent origination as a whole. } Ignorance is a condition for choices. Choices are a condition for consciousness. Consciousness is a condition for name and form. Name and form are conditions for the six sense fields. The six sense fields are conditions for contact. Contact is a condition for feeling. Feeling is a condition for craving. Craving is a condition for grasping. Grasping is a condition for continued existence. Continued existence is a condition for rebirth. Rebirth is a condition for old age and death, sorrow, lamentation, pain, sadness, and distress to come to be. That is how this entire mass of suffering originates. 

When\marginnote{20.1} ignorance fades away and ceases with nothing left over, choices cease. When choices cease, consciousness ceases. When consciousness ceases, name and form cease. When name and form cease, the six sense fields cease. When the six sense fields cease, contact ceases. When contact ceases, feeling ceases. When feeling ceases, craving ceases. When craving ceases, grasping ceases. When grasping ceases, continued existence ceases. When continued existence ceases, rebirth ceases. When rebirth ceases, old age and death, sorrow, lamentation, pain, sadness, and distress cease. That is how this entire mass of suffering ceases. 

‘When\marginnote{21.1} rebirth ceases, old age and death cease.’ That’s what I said. Is that how you see this or not?” 

“That’s\marginnote{21.3} how we see it.” 

‘When\marginnote{21.6} continued existence ceases, rebirth ceases.’ … 

‘When\marginnote{21.51} ignorance ceases, choices cease.’ That’s what I said. Is that how you see this or not?” 

“That’s\marginnote{21.53} how we see it.” 

“Good,\marginnote{22.1} mendicants! So both you and I say this. When this doesn’t exist, that is not; due to the cessation of this, that ceases. That is: When ignorance ceases, choices cease. When choices cease, consciousness ceases. When consciousness ceases, name and form cease. When name and form cease, the six sense fields cease. When the six sense fields cease, contact ceases. When contact ceases, feeling ceases. When feeling ceases, craving ceases. When craving ceases, grasping ceases. When grasping ceases, continued existence ceases. When continued existence ceases, rebirth ceases. When rebirth ceases, old age and death, sorrow, lamentation, pain, sadness, and distress cease. That is how this entire mass of suffering ceases. 

Knowing\marginnote{23.1} and seeing in this way, mendicants, would you turn back to the past, thinking,\footnote{This passage unpacks certain aspects of ignorance. | Compare \href{https://suttacentral.net/sn12.20/en/sujato\#5.1}{SN 12.20:5.1}. } ‘Did we exist in the past? Did we not exist in the past? What were we in the past? How were we in the past? After being what, what did we become in the past?’?”\footnote{These are called “irrational thoughts” at \href{https://suttacentral.net/mn2/en/sujato\#7.3}{MN 2:7.3}. } 

“No,\marginnote{23.3} sir.” 

“Knowing\marginnote{23.4} and seeing in this way, mendicants, would you turn forward to the future, thinking,\footnote{\textsanskrit{Mahāsaṅgīti} edition has the same verb \textit{\textsanskrit{paṭidhāv}-} here as above (“turn back to”). PTS and BJT have here \textit{\textsanskrit{ādhav}-} with \textit{\textsanskrit{paṭidhāv}-} as variant. At \href{https://suttacentral.net/sn12.20/en/sujato\#5.3}{SN 12.20:5.3} all three editions have \textit{\textsanskrit{upadhāv}-}, with \textit{\textsanskrit{apadhāv}-} as variant in PTS. Whatever the correct reading might be, it is clear the intent is convey the opposite direction. } ‘Will we exist in the future? Will we not exist in the future? What will we be in the future? How will we be in the future? After being what, what will we become in the future?’?” 

“No,\marginnote{23.6} sir.” 

“Knowing\marginnote{24.1} and seeing in this way, mendicants, would you be undecided about the present, thinking, ‘Am I? Am I not? What am I? How am I? This sentient being—where did it come from? And where will it go?’?”\footnote{Although the question is still in plural, the answer shifts to singular, perhaps by mistake because elsewhere this passage is always singular. } 

“No,\marginnote{24.3} sir.” 

“Knowing\marginnote{24.4} and seeing in this way, would you say, ‘Our teacher is respected. We speak like this out of respect for our teacher’?”\footnote{“Respect for our teacher” is \textit{\textsanskrit{satthā} no garu}; compare \textit{\textsanskrit{samaṇo} no garu} at \href{https://suttacentral.net/an3.65/en/sujato\#4.1}{AN 3.65:4.1}. } 

“No,\marginnote{24.6} sir.” 

“Knowing\marginnote{24.7} and seeing in this way, would you say, ‘Our ascetic says this. We speak like this because it is what he says’?”\footnote{Readings here are problematic and not cleared up by the commentary. I follow BJT and MS, which have a similar sense. However, both PTS and BJT plausibly have the pronoun \textit{no} (“our”), which I add though absent from MS. } 

“No,\marginnote{24.9} sir.” 

“Knowing\marginnote{24.10} and seeing in this way, would you dedicate yourself to another teacher?” 

“No,\marginnote{24.11} sir.” 

“Knowing\marginnote{24.12} and seeing in this way, would you believe that the observances and boisterous, superstitious rites of the various ascetics and brahmins are essential?”\footnote{In Buddhism, performance of rituals is not in itself forbidden; the main point is that they are not considered “essential” (\textit{\textsanskrit{sārato}}). Note that rituals were regarded as efficacious acts, and hence correspond to “choices” (\textit{\textsanskrit{saṅkhārā}}), a word that can also mean “rite”. | “Boisterous” (\textit{\textsanskrit{kotūhala}}) is literally “whence the hubbub?” This basic sense comes across clearly in the \textsanskrit{Arthaśāstra}, which describes a spy’s spell for putting to sleep the men or dogs that guard a village, who are always listening out for sounds (14.3.21cd, 14.3.37ab). Vedic rituals, with their multiple reciters and arcane rites, took on a noisy and festive air. } 

“No,\marginnote{24.13} sir.” 

“Aren’t\marginnote{24.14} you speaking only of what you have known and seen and realized for yourselves?” 

“Yes,\marginnote{24.15} sir.” 

“Good,\marginnote{25.1} mendicants! You have been guided by me with this teaching that’s apparent in the present life, immediately effective, inviting inspection, relevant, so that sensible people can know it for themselves. For when I said that this teaching is apparent in the present life, immediately effective, inviting inspection, relevant, so that sensible people can know it for themselves, this is what I was referring to. 

Mendicants,\marginnote{26.1} when three things come together an embryo is conceived.\footnote{This section illustrates dependent origination by way of the birth and physical and psychological development of a person from conception to adulthood. From passages such as \href{https://suttacentral.net/dn15/en/sujato\#21.2}{DN 15:21.2}, we know that conception occurs at the nexus of “consciousness” and “name and form” in dependent origination. Since it starts with this life only, the first two factors, ignorance and choices, are omitted here, but are implicitly covered in the preceding passage. | For the “conception” or more literally “descent” of the embryo, the Buddha uses the same term \textit{avakkanti} that, as we have noted (\href{https://suttacentral.net/mn38/en/sujato\#2.2}{MN 38:2.2}), was preferred by \textsanskrit{Yajñavālkya} in the same context. } In a case where the mother and father come together, but the mother is not in the fertile phase of her menstrual cycle, and the virile spirit is not ready, the embryo is not conceived.\footnote{According to \href{https://suttacentral.net/mn93/en/sujato\#18.61}{MN 93:18.61} this was a doctrine of the brahmins, and it was evidently adopted in this sutta as a popular theory of conception. I discuss the role of the \textit{gandhabba} in my notes there. | \textit{Utu} (“the fertile phase of her menstrual cycle”) literally means “season”. As the earth needs rain, a womb is dry and infertile until it is moistened by blood, for the fortnight following which it is fertile and “in season”. Thus \textit{utu} can be both menstruation, during which sex was taboo for the brahmins, as well as the fertile fortnight that follows, outside of which sex was also taboo (\href{https://suttacentral.net/snp2.7/en/sujato\#9.2}{Snp 2.7:9.2}). Atharvaveda 14.2.37a speaks of parents coming together “in season”. \textsanskrit{Bṛhadāraṇyaka} \textsanskrit{Upaniṣad} 6.4.6 expresses the same idea by saying the woman should be approached for sex when she has removed her soiled garments (since she may not change clothes while menstruating, 6.4.13). } In a case where the mother and father come together, the mother is in the fertile phase of her menstrual cycle, but the virile spirit is not ready, the embryo is not conceived. But when these three things come together—the mother and father come together, the mother is in the fertile phase of her menstrual cycle, and the virile spirit is ready—an embryo is conceived. 

The\marginnote{27.1} mother nurtures the embryo in her womb for nine or ten months at great risk to her heavy burden.\footnote{“At great risk” is \textit{\textsanskrit{mahatā} \textsanskrit{saṁsayena}}. | A term of pregnancy of “nine or ten months” is also found at \textsanskrit{Chāndogya} \textsanskrit{Upaniṣad} 5.9.1. | For “heavy burden” (\textit{\textsanskrit{garubhāra}}) see \href{https://suttacentral.net/pli-tv-bi-vb-pc61/en/sujato\#1.5}{Bi Pc 61:1.5}. } When nine or ten months have passed, the mother gives birth at great risk to her heavy burden. When the infant is born she nourishes it with her own blood. For mother’s milk is regarded as blood in the training of the Noble One.\footnote{The Buddha’s claim that this idea is distinct to him seems to be borne out, as I cannot locate it in non-Buddhist texts. } 

That\marginnote{28.1} boy grows up and his faculties mature.\footnote{This shows that dependent origination does not happen all at once; it is a process of growth and maturation. A child, whose faculties are not developed, does not perpetuate the cycle because they have no formed moral intentions. } He accordingly plays childish games such as toy plows, tipcat, somersaults, pinwheels, toy measures, toy carts, and toy bows.\footnote{A more extensive list of games is found at \href{https://suttacentral.net/dn1/en/sujato\#1.14.2}{DN 1:1.14.2}. } 

That\marginnote{29.1} boy grows up and his faculties mature further. He accordingly amuses himself, supplied and provided with the five kinds of sensual stimulation. Sights known by the eye, which are likable, desirable, agreeable, pleasant, sensual, and arousing.\footnote{In dependent origination, this parallels contact through the senses giving rise to feelings. } 

Sounds\marginnote{29.4} known by the ear … 

Smells\marginnote{29.5} known by the nose … 

Tastes\marginnote{29.6} known by the tongue … 

Touches\marginnote{29.7} known by the body, which are likable, desirable, agreeable, pleasant, sensual, and arousing. 

When\marginnote{30.1} they see a sight with their eyes, if it’s pleasant they desire it, but if it’s unpleasant they dislike it. They live with mindfulness of the body unestablished and their heart restricted.\footnote{In dependent origination, feeling gives rise to craving. | Parallel passages in the \textsanskrit{Saṁyutta} (eg. \href{https://suttacentral.net/sn35.132/en/sujato\#12.3}{SN 35.132:12.3}) in parallels for this passage have \textit{adhimuccati} (‘commits to, holds on to”) rather than \textit{\textsanskrit{sārajjati}} (“desires”). } And they don’t truly understand the freedom of heart and freedom by wisdom where those arisen bad, unskillful qualities cease without anything left over. 

Being\marginnote{30.3} so full of favoring and opposing, when they experience any kind of feeling—pleasant, unpleasant, or neutral—they approve, welcome, and keep clinging to it. This gives rise to relishing. Relishing feelings is grasping. Their grasping is a condition for continued existence. Continued existence is a condition for rebirth. Rebirth is a condition for old age and death, sorrow, lamentation, pain, sadness, and distress to come to be.\footnote{Now we rejoin the standard sequence of dependent origination. } That is how this entire mass of suffering originates. 

When\marginnote{30.7} they hear a sound with their ears … 

When\marginnote{30.8} they smell an odor with their nose … 

When\marginnote{30.9} they taste a flavor with their tongue … 

When\marginnote{30.10} they feel a touch with their body … 

When\marginnote{30.11} they know an idea with their mind, if it’s pleasant they desire it, but if it’s unpleasant they dislike it. They live with mindfulness of the body unestablished and their heart restricted. And they don’t truly understand the freedom of heart and freedom by wisdom where those arisen bad, unskillful qualities cease without anything left over. 

Being\marginnote{30.13} so full of favoring and opposing, when they experience any kind of feeling—pleasant, unpleasant, or neutral—they approve, welcome, and keep clinging to it. This gives rise to relishing. Relishing feelings is grasping. Their grasping is a condition for continued existence. Continued existence is a condition for rebirth. Rebirth is a condition for old age and death, sorrow, lamentation, pain, sadness, and distress to come to be. That is how this entire mass of suffering originates. 

But\marginnote{31.1} consider when a Realized One arises in the world, perfected, a fully awakened Buddha, accomplished in knowledge and conduct, holy, knower of the world, supreme guide for those who wish to train, teacher of gods and humans, awakened, blessed.\footnote{Just as the sutta illustrated the abstract arising of suffering with the concrete example of a child growing up, it now illustrates the unraveling of dependent origination with the Gradual Training (see \href{https://suttacentral.net/mn27/en/sujato\#11.1}{MN 27:11.1}). } He has realized with his own insight this world—with its gods, \textsanskrit{Māras}, and divinities, this population with its ascetics and brahmins, gods and humans—and he makes it known to others. He proclaims a teaching that is good in the beginning, good in the middle, and good in the end, meaningful and well-phrased. He reveals an entirely full and pure spiritual life. 

A\marginnote{32.1} householder hears that teaching, or a householder’s child, or someone reborn in a good family. They gain faith in the Realized One and reflect, ‘Life at home is cramped and dirty, life gone forth is wide open. It’s not easy for someone living at home to lead the spiritual life utterly full and pure, like a polished shell. Why don’t I shave off my hair and beard, dress in ocher robes, and go forth from lay life to homelessness?’ After some time they give up a large or small fortune, and a large or small family circle. They shave off hair and beard, dress in ocher robes, and go forth from the lay life to homelessness. 

Once\marginnote{33.1} they’ve gone forth, they take up the training and livelihood of the mendicants. They give up killing living creatures, renouncing the rod and the sword. They’re scrupulous and kind, living full of sympathy for all living beings. 

They\marginnote{33.2} give up stealing. They take only what’s given, and expect only what’s given. They keep themselves clean by not thieving. 

They\marginnote{33.3} give up unchastity. They are celibate, set apart, avoiding the vulgar act of sex. 

They\marginnote{33.4} give up lying. They speak the truth and stick to the truth. They’re honest and dependable, and don’t trick the world with their words. 

They\marginnote{33.5} give up divisive speech. They don’t repeat in one place what they heard in another so as to divide people against each other. Instead, they reconcile those who are divided, supporting unity, delighting in harmony, loving harmony, speaking words that promote harmony. 

They\marginnote{33.6} give up harsh speech. They speak in a way that’s mellow, pleasing to the ear, lovely, going to the heart, polite, likable and agreeable to the people. 

They\marginnote{33.7} give up talking nonsense. Their words are timely, true, and meaningful, in line with the teaching and training. They say things at the right time which are valuable, reasonable, succinct, and beneficial. 

They\marginnote{33.8} refrain from injuring plants and seeds. They eat in one part of the day, abstaining from eating at night and food at the wrong time. They refrain from seeing shows of dancing, singing, and music . They refrain from beautifying and adorning themselves with garlands, fragrance, and makeup. They refrain from high and luxurious beds. They refrain from receiving gold and currency, raw grains, raw meat, women and girls, male and female bondservants, goats and sheep, chickens and pigs, elephants, cows, horses, and mares, and fields and land. They refrain from running errands and messages; buying and selling; falsifying weights, metals, or measures; bribery, fraud, cheating, and duplicity; mutilation, murder, abduction, banditry, plunder, and violence. 

They’re\marginnote{34.1} content with robes to look after the body and almsfood to look after the belly. Wherever they go, they set out taking only these things. They’re like a bird: wherever it flies, wings are its only burden. In the same way, a mendicant is content with robes to look after the body and almsfood to look after the belly. Wherever they go, they set out taking only these things. When they have this entire spectrum of noble ethics, they experience a blameless happiness inside themselves. 

When\marginnote{35.1} they see a sight with their eyes, they don’t get caught up in the features and details. If the faculty of sight were left unrestrained, bad unskillful qualities of covetousness and displeasure would become overwhelming. For this reason, they practice restraint, protecting the faculty of sight, and achieving its restraint. 

When\marginnote{35.3} they hear a sound with their ears … 

When\marginnote{35.4} they smell an odor with their nose … 

When\marginnote{35.5} they taste a flavor with their tongue … 

When\marginnote{35.6} they feel a touch with their body … 

When\marginnote{35.7} they know an idea with their mind, they don’t get caught up in the features and details. If the faculty of mind were left unrestrained, bad unskillful qualities of covetousness and displeasure would become overwhelming. For this reason, they practice restraint, protecting the faculty of mind, and achieving its restraint. When they have this noble sense restraint, they experience an unsullied bliss inside themselves. 

They\marginnote{36.1} act with situational awareness when going out and coming back; when looking ahead and aside; when bending and extending the limbs; when bearing the outer robe, bowl and robes; when eating, drinking, chewing, and tasting; when urinating and defecating; when walking, standing, sitting, sleeping, waking, speaking, and keeping silent. 

When\marginnote{37.1} they have this entire spectrum of noble ethics, this noble contentment, this noble sense restraint, and this noble mindfulness and situational awareness, they frequent a secluded lodging—a wilderness, the root of a tree, a hill, a ravine, a mountain cave, a charnel ground, a forest, the open air, a heap of straw. 

After\marginnote{38.1} the meal, they return from almsround, sit down cross-legged, set their body straight, and establish mindfulness in their presence. Giving up covetousness for the world, they meditate with a heart rid of covetousness, cleansing the mind of covetousness. Giving up ill will and malevolence, they meditate with a mind rid of ill will, full of sympathy for all living beings, cleansing the mind of ill will. Giving up dullness and drowsiness, they meditate with a mind rid of dullness and drowsiness, perceiving light, mindful and aware, cleansing the mind of dullness and drowsiness. Giving up restlessness and remorse, they meditate without restlessness, their mind peaceful inside, cleansing the mind of restlessness and remorse. Giving up doubt, they meditate having gone beyond doubt, not undecided about skillful qualities, cleansing the mind of doubt. 

They\marginnote{39.1} give up these five hindrances, corruptions of the heart that weaken wisdom. Then, quite secluded from sensual pleasures, secluded from unskillful qualities, they enter and remain in the first absorption, which has the rapture and bliss born of seclusion, while placing the mind and keeping it connected. Furthermore, as the placing of the mind and keeping it connected are stilled, a mendicant enters and remains in the second absorption … third absorption … fourth absorption. 

When\marginnote{40.1} they see a sight with their eyes, if it’s pleasant they don’t desire it, and if it’s unpleasant they don’t dislike it. They live with mindfulness of the body established and a limitless heart.\footnote{This resumes the teaching on attachment to the senses (from \href{https://suttacentral.net/mn38/en/sujato\#30.1}{MN 38:30.1}), having shown what is required to let go such attachment. Here, one experiences the feelings through the senses, but without any attachment. } And they truly understand the freedom of heart and freedom by wisdom where those arisen bad, unskillful qualities cease without anything left over. 

Having\marginnote{40.3} given up favoring and opposing, when they experience any kind of feeling—pleasant, unpleasant, or neutral—they don’t approve, welcome, or keep clinging to it. As a result, relishing of feelings ceases. When their relishing ceases, grasping ceases. When grasping ceases, continued existence ceases. When continued existence ceases, rebirth ceases. When rebirth ceases, old age and death, sorrow, lamentation, pain, sadness, and distress cease. That is how this entire mass of suffering ceases. 

When\marginnote{41.1} they hear a sound with their ears … 

When\marginnote{41.2} they smell an odor with their nose … 

When\marginnote{41.3} they taste a flavor with their tongue … 

When\marginnote{41.4} they feel a touch with their body … 

When\marginnote{41.5} they know an idea with their mind, if it’s pleasant they don’t desire it, and if it’s unpleasant they don’t dislike it. They live with mindfulness of the body established and a limitless heart. And they truly understand the freedom of heart and freedom by wisdom where those arisen bad, unskillful qualities cease without anything left over. 

Having\marginnote{41.7} given up favoring and opposing, when they experience any kind of feeling—pleasant, unpleasant, or neutral—they don’t approve, welcome, or keep clinging to it. As a result, relishing of feelings ceases. When their relishing ceases, grasping ceases. When grasping ceases, continued existence ceases. When continued existence ceases, rebirth ceases. When rebirth ceases, old age and death, sorrow, lamentation, pain, sadness, and distress cease. That is how this entire mass of suffering ceases. 

Mendicants,\marginnote{41.11} you should memorize this brief statement on freedom through the ending of craving. But the mendicant \textsanskrit{Sāti}, the fisherman’s son, is caught in a vast net of craving, a tangle of craving.”\footnote{The mention of the “brief statement” at the end of a long discourse is puzzling. A similar exhortation to “memorize” a “brief” passage is found in only one other passage, where it is in reference to the short summary passage around which the sutta is based (\href{https://suttacentral.net/mn140/en/sujato\#32.3}{MN 140:32.3}). Compare the preceding sutta, \href{https://suttacentral.net/mn37/en/sujato}{MN 37}, which revolves around a short passage for memorization that is fittingly described as “brief” throughout. That “brief” passage opens by saying “nothing is worth insisting on”, advice that is disregarded by \textsanskrit{Sāti} who “insists” on his own view (\href{https://suttacentral.net/mn38/en/sujato\#3.11}{MN 38:3.11}). No “brief statement” is mentioned in the Chinese parallel (MA 201 at T i 769c28), which speaks instead of the shaking of the three-thousand-fold world system. } 

That\marginnote{41.12} is what the Buddha said. Satisfied, the mendicants approved what the Buddha said. 

%
\section*{{\suttatitleacronym MN 39}{\suttatitletranslation The Longer Discourse at Assapura }{\suttatitleroot Mahāassapurasutta}}
\addcontentsline{toc}{section}{\tocacronym{MN 39} \toctranslation{The Longer Discourse at Assapura } \tocroot{Mahāassapurasutta}}
\markboth{The Longer Discourse at Assapura }{Mahāassapurasutta}
\extramarks{MN 39}{MN 39}

\scevam{So\marginnote{1.1} I have heard. }At one time the Buddha was staying in the land of the \textsanskrit{Aṅgas}, near the \textsanskrit{Aṅgan} town named Assapura.\footnote{Assapura, the “horse fort”, appears to be unknown apart from its mention here and the next sutta. Contra \emph{Dictionary of Pali Proper Names}, this is not the same Assapura referred to in \href{https://suttacentral.net/ja422/en/sujato}{Ja 422}, as that lay south of Ceti, far to the west. An \textsanskrit{Aśvapura} is mentioned in Jain legends, but there is nothing to link this with the \textsanskrit{Aṅgan} town. | \textsanskrit{Aṅga} lay to the east of Magadha on the Ganges, towards modern Bengal. } There the Buddha addressed the mendicants, “Mendicants!” 

“Venerable\marginnote{1.5} sir,” they replied. The Buddha said this: 

“Mendicants,\marginnote{2.1} people label you as ascetics. And when they ask you what you are, you claim to be ascetics.\footnote{This sutta deals with the difference between outer perception and inner reality, addressing what we call “imposter syndrome”. The Buddha inspires students to aspire to the highest ideal through the practice of the Gradual Training so that they live up to their labels and claims. } 

Given\marginnote{2.3} this label and this claim, you should train like this: ‘We will undertake and follow the things that make one an ascetic and a brahmin. That way our label will be accurate and our claim correct. Any robes, almsfood, lodgings, and medicines and supplies for the sick that we use will be very fruitful and beneficial for the donor. And our going forth will not be wasted, but will be fruitful and fertile.’ 

And\marginnote{3.1} what are the things that make one an ascetic and a brahmin? You should train like this: ‘We will have conscience and prudence.’ Now, mendicants, you might think, ‘We have conscience and prudence. This is sufficient; enough has been done. We have achieved the goal of life as an ascetic. There is nothing more to do.’ And you might rest content with just that much. I declare this to you, mendicants, I announce this to you: ‘You who seek to be true ascetics, do not lose sight of the goal of the ascetic life while there is still more to do.’ 

What\marginnote{4.1} more is there to do? You should train like this: ‘Our bodily behavior will be pure, clear, open, neither inconsistent nor secretive.\footnote{\textit{\textsanskrit{Chiddavā}} (“inconsistent”) is more literally “full of holes”. Its opposite (\textit{acchidda}) is one of the standard descriptors of “impeccable” precepts (eg. \href{https://suttacentral.net/mn48/en/sujato\#6.12}{MN 48:6.12}). | \textit{\textsanskrit{Saṁvuta}} (past participle of \textit{\textsanskrit{saṁvara}}) can mean either “restrained” or “hidden, secretive”. The sense depends on the syntax; the two instances of \textit{ca} split the listed items into three plus two, with the negative distributed over the last two items. This agrees with the commentary, which explains “secretive” as “concealing one’s faults”. \textit{\textsanskrit{Saṁvara}} appears in the sense of sense restraint below (\href{https://suttacentral.net/mn39/en/sujato\#8.4}{MN 39:8.4}). } And we won’t glorify ourselves or put others down on account of our pure bodily behavior.’ Now, mendicants, you might think, ‘We have conscience and prudence, and our bodily behavior is pure. This is sufficient …’ I declare this to you, mendicants, I announce this to you: ‘You who seek to be true ascetics, do not lose sight of the goal of the ascetic life while there is still more to do.’ 

What\marginnote{5.1} more is there to do? You should train like this: ‘Our verbal behavior … mental behavior … livelihood will be pure, clear, open, neither inconsistent nor secretive. And we won’t glorify ourselves or put others down on account of our pure livelihood.’ Now, mendicants, you might think, ‘We have conscience and prudence, our bodily, verbal, and mental behavior is pure, and our livelihood is pure. This is sufficient; enough has been done. We have achieved the goal of life as an ascetic. There is nothing more to do.’ And you might rest content with just that much. I declare this to you, mendicants, I announce this to you: ‘You who seek to be true ascetics, do not lose sight of the goal of the ascetic life while there is still more to do.’ 

What\marginnote{8.1} more is there to do? You should train yourselves like this: ‘We will restrain our sense doors. When we see a sight with our eyes, we won’t get caught up in the features and details. If the faculty of sight were left unrestrained, bad unskillful qualities of covetousness and displeasure would become overwhelming. For this reason, we will practice restraint, we will protect the faculty of sight, and we will achieve its restraint. When we hear a sound with our ears … When we smell an odor with our nose … When we taste a flavor with our tongue … When we feel a touch with our body … When we know an idea with our mind, we won’t get caught up in the features and details. If the faculty of mind were left unrestrained, bad unskillful qualities of covetousness and displeasure would become overwhelming. For this reason, we will practice restraint, we will protect the faculty of mind, and we will achieve its restraint.’ Now, mendicants, you might think, ‘We have conscience and prudence, our bodily, verbal, and mental behavior is pure, our livelihood is pure, and our sense doors are restrained. This is sufficient …’ 

What\marginnote{9.1} more is there to do? You should train yourselves like this: ‘We will not eat too much. We will only eat after reflecting rationally on our food. We will eat not for fun, indulgence, adornment, or decoration, but only to sustain this body, to avoid harm, and to support spiritual practice. In this way, we shall put an end to old discomfort and not give rise to new discomfort, and we will have the means to keep going, blamelessness, and a comfortable abiding.’ Now, mendicants, you might think, ‘We have conscience and prudence, our bodily, verbal, and mental behavior is pure, our livelihood is pure, our sense doors are restrained, and we don’t eat too much. This is sufficient …’ 

What\marginnote{10.1} more is there to do? You should train yourselves like this: ‘We will be dedicated to wakefulness. When practicing walking and sitting meditation by day, we will purify our mind from obstacles. In the first watch of the night, we will continue to practice walking and sitting meditation. In the middle watch, we will lie down in the lion’s posture—on the right side, placing one foot on top of the other—mindful and aware, and focused on the time of getting up. In the last watch, we will get up and continue to practice walking and sitting meditation, purifying our mind from obstacles.’ Now, mendicants, you might think, ‘We have conscience and prudence, our bodily, verbal, and mental behavior is pure, our livelihood is pure, our sense doors are restrained, we don’t eat too much, and we are dedicated to wakefulness. This is sufficient …’ 

What\marginnote{11.1} more is there to do? You should train yourselves like this: ‘We will have situational awareness and mindfulness. We will act with situational awareness when going out and coming back; when looking ahead and aside; when bending and extending the limbs; when bearing the outer robe, bowl and robes; when eating, drinking, chewing, and tasting; when urinating and defecating; when walking, standing, sitting, sleeping, waking, speaking, and keeping silent.’ Now, mendicants, you might think, ‘We have conscience and prudence, our bodily, verbal, and mental behavior is pure, our livelihood is pure, our sense doors are restrained, we don’t eat too much, we are dedicated to wakefulness, and we have mindfulness and situational awareness. This is sufficient …’ 

What\marginnote{12.1} more is there to do? Take a mendicant who frequents a secluded lodging—a wilderness, the root of a tree, a hill, a ravine, a mountain cave, a charnel ground, a forest, the open air, a heap of straw. 

After\marginnote{13.1} the meal, they return from almsround, sit down cross-legged, set their body straight, and establish mindfulness in their presence. Giving up covetousness for the world, they meditate with a heart rid of covetousness, cleansing the mind of covetousness. Giving up ill will and malevolence, they meditate with a mind rid of ill will, full of sympathy for all living beings, cleansing the mind of ill will. Giving up dullness and drowsiness, they meditate with a mind rid of dullness and drowsiness, perceiving light, mindful and aware, cleansing the mind of dullness and drowsiness. Giving up restlessness and remorse, they meditate without restlessness, their mind peaceful inside, cleansing the mind of restlessness and remorse. Giving up doubt, they meditate having gone beyond doubt, not undecided about skillful qualities, cleansing the mind of doubt. 

Suppose\marginnote{14.1} a man who has gotten into debt were to apply himself to work,\footnote{The similes for the hindrances and \textsanskrit{jhānas} feature also in \href{https://suttacentral.net/dn2/en/sujato\#69.1}{DN 2:69.1} and \href{https://suttacentral.net/dn10/en/sujato\#2.7.1}{DN 10:2.7.1}. } and his efforts proved successful. He would pay off the original loan and have enough left over to support his partner. Thinking about this, he’d be filled with joy and happiness. 

Suppose\marginnote{14.8} a person was sick, suffering, and gravely ill. They’d lose their appetite and get physically weak. But after some time they’d recover from that illness, and regain their appetite and their strength. Thinking about this, they’d be filled with joy and happiness. 

Suppose\marginnote{14.13} a person was imprisoned in a jail. But after some time they were released from jail, safe and sound, with no loss of wealth. Thinking about this, they’d be filled with joy and happiness. 

Suppose\marginnote{14.18} a person was a bondservant. They would not be their own master, but indentured to another, unable to go where they wish. But after some time they’d be freed from servitude. They would be their own master, not indentured to another, an emancipated individual able to go where they wish. Thinking about this, they’d be filled with joy and happiness. 

Suppose\marginnote{14.23} there was a person with wealth and property who was traveling along a desert road. But after some time they crossed over the desert, safe and sound, with no loss of wealth. Thinking about this, they’d be filled with joy and happiness. 

In\marginnote{14.29} the same way, as long as these five hindrances are not given up inside themselves, a mendicant regards them as a debt, a disease, a prison, slavery, and a desert crossing. But when these five hindrances are given up inside themselves, a mendicant regards this as freedom from debt, good health, release from prison, emancipation, and a place of sanctuary at last. 

They\marginnote{15.1} give up these five hindrances, corruptions of the heart that weaken wisdom. Then, quite secluded from sensual pleasures, secluded from unskillful qualities, they enter and remain in the first absorption, which has the rapture and bliss born of seclusion, while placing the mind and keeping it connected. They drench, steep, fill, and spread their body with rapture and bliss born of seclusion. There’s no part of the body that’s not spread with rapture and bliss born of seclusion.\footnote{As a meditator proceeds, their subjective experience of the “body” evolves from tactile sense impressions (\textit{\textsanskrit{phoṭṭhabba}}), to the interior mental experience of bliss and light (\textit{\textsanskrit{manomayakāya}}), to the direct personal realization of highest truth (\href{https://suttacentral.net/mn70/en/sujato\#23.2}{MN 70:23.2}: \textit{\textsanskrit{kāyena} ceva \textsanskrit{paramasaccaṁ} sacchikaroti}). } It’s like when a deft bathroom attendant or their apprentice pours bath powder into a bronze dish, sprinkling it little by little with water. They knead it until the ball of bath powder is soaked and saturated with moisture, spread through inside and out; yet no moisture oozes out.\footnote{The kneading is the “placing the mind and keeping it connected”, the water is bliss, while the lack of leaking speaks to the contained interiority of the experience. } 

In\marginnote{15.5} the same way, a mendicant drenches, steeps, fills, and spreads their body with rapture and bliss born of seclusion. There’s no part of the body that’s not spread with rapture and bliss born of seclusion. 

Furthermore,\marginnote{16.1} as the placing of the mind and keeping it connected are stilled, a mendicant enters and remains in the second absorption, which has the rapture and bliss born of immersion, with internal clarity and mind at one, without placing the mind and keeping it connected. They drench, steep, fill, and spread their body with rapture and bliss born of immersion. There’s no part of the body that’s not spread with rapture and bliss born of immersion. It’s like a deep lake fed by spring water. There’s no inlet to the east, west, north, or south, and the heavens would not properly bestow showers from time to time. But the stream of cool water welling up in the lake drenches, steeps, fills, and spreads throughout the lake. There’s no part of the lake that’s not spread through with cool water.\footnote{The simile emphasizes the water as bliss, while the lack of inflow expresses containment and unification. The water welling up is the rapture, which is the uplifting emotional response to the experience of bliss. } 

In\marginnote{16.4} the same way, a mendicant drenches, steeps, fills, and spreads their body with rapture and bliss born of immersion. There’s no part of the body that’s not spread with rapture and bliss born of immersion. 

Furthermore,\marginnote{17.1} with the fading away of rapture, a mendicant enters and remains in the third absorption, where they meditate with equanimity, mindful and aware, personally experiencing the bliss of which the noble ones declare, ‘Equanimous and mindful, one meditates in bliss.’ They drench, steep, fill, and spread their body with bliss free of rapture. There’s no part of the body that’s not spread with bliss free of rapture. It’s like a pool with blue water lilies, or pink or white lotuses. Some of them sprout and grow in the water without rising above it, thriving underwater. From the tip to the root they’re drenched, steeped, filled, and soaked with cool water. There’s no part of them that’s not soaked with cool water.\footnote{The meditator is utterly immersed in stillness and bliss. } 

In\marginnote{17.4} the same way, a mendicant drenches, steeps, fills, and spreads their body with bliss free of rapture. There’s no part of the body that’s not spread with bliss free of rapture. 

Furthermore,\marginnote{18.1} giving up pleasure and pain, and ending former happiness and sadness, a mendicant enters and remains in the fourth absorption, without pleasure or pain, with pure equanimity and mindfulness. They sit spreading their body through with pure bright mind. There’s no part of the body that’s not spread with pure bright mind. It’s like someone sitting wrapped from head to foot with white cloth. There’s no part of the body that’s not spread over with white cloth.\footnote{The white cloth is the purity and brightness of equanimity. The commentary explains this as a person who has just got out of a bath and sits perfectly dry and content. } 

In\marginnote{18.4} the same way, they sit spreading their body through with pure bright mind. There’s no part of the body that’s not spread with pure bright mind. 

When\marginnote{19.1} their mind has become immersed in \textsanskrit{samādhi} like this—purified, bright, flawless, rid of corruptions, pliable, workable, steady, and imperturbable—they extend it toward recollection of past lives. They recollect many kinds of past lives, with features and details. Suppose a person was to leave their home village and go to another village. From that village they’d go to yet another village. And from that village they’d return to their home village. They’d think: ‘I went from my home village to another village. There I stood like this, sat like that, spoke like this, or kept silent like that. From that village I went to yet another village. There too I stood like this, sat like that, spoke like this, or kept silent like that. And from that village I returned to my home village.’ 

In\marginnote{19.4} the same way, a mendicant recollects their many kinds of past lives, with features and details. 

When\marginnote{20.1} their mind has become immersed in \textsanskrit{samādhi} like this—purified, bright, flawless, rid of corruptions, pliable, workable, steady, and imperturbable—they extend it toward knowledge of the death and rebirth of sentient beings. With clairvoyance that is purified and superhuman, they see sentient beings passing away and being reborn—inferior and superior, beautiful and ugly, in a good place or a bad place. They understand how sentient beings are reborn according to their deeds. Suppose there were two houses with doors. A person with clear eyes standing in between them would see people entering and leaving a house and wandering to and fro. 

In\marginnote{20.4} the same way, with clairvoyance that is purified and superhuman, they see sentient beings passing away and being reborn—inferior and superior, beautiful and ugly, in a good place or a bad place. They understand how sentient beings are reborn according to their deeds. 

When\marginnote{21.1} their mind has become immersed in \textsanskrit{samādhi} like this—purified, bright, flawless, rid of corruptions, pliable, workable, steady, and imperturbable—they extend it toward knowledge of the ending of defilements. They truly understand: ‘This is suffering’ … ‘This is the origin of suffering’ … ‘This is the cessation of suffering’ … ‘This is the practice that leads to the cessation of suffering.’ They truly understand: ‘These are defilements’ … ‘This is the origin of defilements’ … ‘This is the cessation of defilements’ … ‘This is the practice that leads to the cessation of defilements.’ Knowing and seeing like this, their mind is freed from the defilements of sensuality, desire to be reborn, and ignorance. When they’re freed, they know they’re freed. They understand: ‘Rebirth is ended, the spiritual journey has been completed, what had to be done has been done, there is nothing further for this place.’ 

Suppose\marginnote{21.7} that in a mountain glen there was a lake that was transparent, clear, and unclouded. A person with clear eyes standing on the bank would see the clams and mussels, and pebbles and gravel, and schools of fish swimming about or staying still. They’d think: ‘This lake is transparent, clear, and unclouded. And here are the clams and mussels, and pebbles and gravel, and schools of fish swimming about or staying still.’ 

In\marginnote{21.11} the same way, a mendicant truly understands: ‘This is suffering’ … ‘This is the origin of suffering’ … ‘This is the cessation of suffering’ … ‘This is the practice that leads to the cessation of suffering.’ They understand: ‘… there is nothing further for this place.’ 

This\marginnote{22.1} mendicant is one who is called an ‘ascetic’, a ‘brahmin’, a ‘bathed initiate’, a ‘knowledge master’, a ‘scholar’, a ‘noble one’, and also a ‘perfected one’.\footnote{This series of epithets is also found at \href{https://suttacentral.net/an7.85/en/sujato}{AN 7.85}–92, split over separate suttas, and with other terms and explanations in \href{https://suttacentral.net/snp3.6/en/sujato}{Snp 3.6}. The explanations here rely on puns, the feel of which I have tried to echo in English. } 

And\marginnote{23.1} how is a mendicant an ascetic?\footnote{\textit{\textsanskrit{Samaṇa}} (“ascetic”) is from a root meaning to exert or weary oneself, but here is explained as if related to \textit{samita} (“assuaged”, “calmed”). } They have assuaged the bad, unskillful qualities that are corrupting, leading to future lives, hurtful, resulting in suffering and future rebirth, old age, and death. That’s how a mendicant is an ascetic. 

And\marginnote{24.1} how is a mendicant a brahmin?\footnote{The root sense of \textit{brah-} is to “amplify” or “empower”, but is explained as \textit{\textsanskrit{bāhita}} (“banished”, “expelled”). This is a common play on words, eg. \href{https://suttacentral.net/ud1.4/en/sujato\#4.1}{Ud 1.4:4.1}, \href{https://suttacentral.net/dhp388/en/sujato\#1}{Dhp 388:1}. } They have banished the bad, unskillful qualities. That’s how a mendicant is a brahmin. 

And\marginnote{25.1} how is a mendicant a bathed initiate?\footnote{A \textit{\textsanskrit{nhātaka}} (Sanskrit \textit{\textsanskrit{snātaka}}) is one who has performed the ritual ablutions on the completion of his studies. } They have bathed off the bad, unskillful qualities. That’s how a mendicant is a bathed initiate. 

And\marginnote{26.1} how is a mendicant a knowledge master?\footnote{The “knowledge” in \textit{\textsanskrit{vedagū}} is the Vedas in Brahmanism and the three knowledges in Buddhism. Curiously, it is not an attested term in Brahmanical texts. Perhaps it arose by contraction from \textit{(\textsanskrit{tiṇṇaṁ}) \textsanskrit{vedānaṁ} \textsanskrit{pāragū}} (cf. Sanskrit \textit{\textsanskrit{vedapāraga}}). } They have known the bad, unskillful qualities. That’s how a mendicant is a knowledge master. 

And\marginnote{27.1} how is a mendicant a scholar?\footnote{\textit{Sottiya} (“scholar”) is from the root “to hear”, as is the more familiar Pali term \textit{\textsanskrit{sāvaka}} (“disciple”, “student”, literally “hearkener”). In Brahmanism \textit{\textsanskrit{śrotrīya}} refers to one who is well versed in the Vedas. Here, however, it is connected with the root \textit{su} (“to flow”), as in one who has “rinsed off” or “scoured off” bad things. } They have scoured off the bad, unskillful qualities. That’s how a mendicant is a scholar. 

And\marginnote{28.1} how is a mendicant a noble one?\footnote{Both \textit{ariya} and \textit{arahant} are explained with the same word \textit{\textsanskrit{āraka}}, meaning “far from (defilements)”. The commentary gives \textit{\textsanskrit{āraka}} in both cases, but explains \textit{ariya} as “slayer” and \textit{arahant} as “far from”. At \href{https://suttacentral.net/an7.91/en/sujato}{AN 7.91}, PTS has \textit{\textsanskrit{arīhatattā}}, “foe-slayer” for \textit{ariya} (BJT omits, while at least one more variant is attested). This is a slightly better pun for \textit{ariya} than \textit{\textsanskrit{āraka}}, although at \href{https://suttacentral.net/snp3.6/en/sujato\#38.1}{Snp 3.6:38.1} \textit{ariya} is punned with \textit{\textsanskrit{ālaya}} (“clinging”), which is better than both. Where both MA 182 and the parallel to \href{https://suttacentral.net/mn40/en/sujato}{MN 40} at MA 183 at T i 726c13 mention the \textit{ariya}, the Indic pun underlying EA 49.8 seems to be that a \textit{\textsanskrit{kṣatriya}} is so called because they “end” (Sanskit \textit{\textsanskrit{kṣaya}}) the defilements. Neither of the Chinese parallels (MA 182 at T i 725c4, EA 49.8 at T ii 802b4) mentions the \textit{arahant}, suggesting it was added later, which would explain the duplication of \textit{\textsanskrit{āraka}}. However, given that “far from” is consistently attested for the \textit{arahant} in all editions and commentary; that “foe-slayer” is attested for \textit{ariya} in some texts and commentaries; and that \textit{\textsanskrit{kṣatriya}} in the sense of “warrior” is loosely in the same semantic realm as “foe-slayer”, I translate accordingly. } They have nobbled their foes, the bad, unskillful qualities. That’s how a mendicant is a noble one. 

And\marginnote{29.1} how is a mendicant a perfected one?\footnote{\textit{Arahant} (“perfected one”) has the root sense “worthy” and in Buddhism has the sense of one who has perfectly or “impeccably” removed all defilements. } They are impeccably remote from the bad, unskillful qualities that are corrupting, leading to future lives, hurtful, resulting in suffering and future rebirth, old age, and death. That’s how a mendicant is a perfected one.” 

That\marginnote{29.4} is what the Buddha said. Satisfied, the mendicants approved what the Buddha said. 

%
\section*{{\suttatitleacronym MN 40}{\suttatitletranslation The Shorter Discourse at Assapura }{\suttatitleroot Cūḷaassapurasutta}}
\addcontentsline{toc}{section}{\tocacronym{MN 40} \toctranslation{The Shorter Discourse at Assapura } \tocroot{Cūḷaassapurasutta}}
\markboth{The Shorter Discourse at Assapura }{Cūḷaassapurasutta}
\extramarks{MN 40}{MN 40}

\scevam{So\marginnote{1.1} I have heard.\footnote{This sutta deals with a similar topic to the previous, with differences in phrasing and content. } }At one time the Buddha was staying in the land of the \textsanskrit{Aṅgas}, near the \textsanskrit{Aṅgan} town named Assapura. There the Buddha addressed the mendicants, “Mendicants!” 

“Venerable\marginnote{1.5} sir,” they replied. The Buddha said this: 

“Mendicants,\marginnote{2.1} people label you as ascetics. And when they ask you what you are, you claim to be ascetics. 

Given\marginnote{2.3} this label and this claim, you should train like this: ‘We will practice in the way that is proper for an ascetic. That way our label will be accurate and our claim correct. Any robes, almsfood, lodgings, and medicines and supplies for the sick that we use will be very fruitful and beneficial for the donor. And our going forth will not be wasted, but will be fruitful and fertile.’ 

And\marginnote{3.1} how does a mendicant not practice in the way that is proper for an ascetic?\footnote{These are forms of “misapprehension of precepts and observances” which do not in and of themselves signify spiritual purity. } 

Any\marginnote{3.2} mendicant who has not given up covetousness, ill will, irritability, acrimony, disdain, contempt, jealousy, stinginess, deviousness, deceit, corrupt wishes, and wrong view\footnote{Compare the “corruptions of the mind” at \href{https://suttacentral.net/mn7/en/sujato\#3.2}{MN 7:3.2}. } is not practicing in the way that is proper for an ascetic, I say. And that is due to not giving up these stains, defects, and dregs of an ascetic, these grounds for rebirth in places of loss, to be experienced in bad places. I say that such a mendicant’s going forth may be compared to the kind of weapon called ‘deadborn’—double-edged, whetted with yellow arsenic—that has been covered and wrapped in an outer robe.\footnote{“Deadborn” (\textit{mataja}) is explained by the commentary as meaning forged from iron filings recovered from the corpse of a dead heron, which was force-fed them for that purpose. It sounds bizarre, and I have not been able to trace this practice elsewhere, but people do a lot of bizarre things. | \textit{\textsanskrit{Pītanisita}} may mean “sharpened with a wet (\textit{\textsanskrit{pīta}}) stone” per the commentary, or else “sharpened with yellow arsenic” (\textit{\textsanskrit{pīta}}). \textsanskrit{Arthaśāstra} 2.14.48 describes how metals may be rubbed with yellow arsenic (here = \textit{\textsanskrit{haritāla}}) to disguise them. This fits the context, where a supposed monk wrapped in the ocher robe is like the metal, outwardly shiny yellow, but inside is deadly. } 

I\marginnote{5.1} say that you don’t deserve the label ‘outer robe wearer’ just because you wear an outer robe. You don’t deserve the label ‘naked ascetic’ just because you go naked. You don’t deserve the label ‘dust and dirt wearer’ just because you’re caked in dust and dirt.\footnote{Wearers of dust and dirt are mentioned at \href{https://suttacentral.net/dn8/en/sujato\#14.18}{DN 8:14.18} and \href{https://suttacentral.net/dn25/en/sujato\#8.16}{DN 25:8.16}, along with several other items on this list; see too \href{https://suttacentral.net/dhp141/en/sujato}{Dhp 141} and \href{https://suttacentral.net/thag4.5/en/sujato\#1.2}{Thag 4.5:1.2}, and the description of the Bodhisatta at \href{https://suttacentral.net/mn12/en/sujato\#46.5}{MN 12:46.5}. Some ascetics, such as Jains, did not bathe. } You don’t deserve the label ‘water immerser’ just because you immerse yourself in water. You don’t deserve the label ‘tree root dweller’ just because you stay at the root of a tree. You don’t deserve the label ‘open air dweller’ just because you stay in the open air. You don’t deserve the label ‘stander’ just because you continually stand. You don’t deserve the label ‘interval eater’ just because you eat food at set intervals. You don’t deserve the label ‘reciter’ just because you recite hymns. You don’t deserve the label ‘matted-hair ascetic’ just because you have matted hair. 

Imagine\marginnote{6.1} that just by wearing an outer robe someone with covetousness, ill will, irritability, acrimony, disdain, contempt, jealousy, stinginess, deviousness, deceit, corrupt wishes, and wrong view could give up these things. If that were the case, your friends and colleagues, relatives and kin would make you an outer robe wearer as soon as you were born. They’d encourage you: ‘Please, dearest, wear an outer robe! By doing so you will give up covetousness, ill will, irritability, acrimony, disdain, contempt, jealousy, stinginess, deviousness, deceit, corrupt wishes, and wrong view.’\footnote{\textit{Bhadramukha} is used here by family for a child; by lay people addressing monks at \href{https://suttacentral.net/mn81/en/sujato\#21.1}{MN 81:21.1}; by a laywoman addressing a deity at \href{https://suttacentral.net/an7.53/en/sujato\#3.4}{AN 7.53:3.4}; by a king of his grown son at \href{https://suttacentral.net/sn3.7/en/sujato\#1.5}{SN 3.7:1.5}; and, with the addition of the familiar \textit{\textsanskrit{tāta}}, by a brahmin wife to her husband at \href{https://suttacentral.net/mn100/en/sujato\#3.5}{MN 100:3.5}. The term is of Sanskritic form and is found commonly in the \textsanskrit{Avadāna} literature. Turning to later Sanskrit sources, \textsanskrit{Nāṭyaśāstra} 19.12 says it should be used when addressing inferiors, while \textsanskrit{Sāhityadarpaṇa} 6.154 says it is used when addressing a prince (see SN 3.7 above). The early Pali usage, however, suggests it was an affectionate and respectful term of somewhat elevated usage, but not restricted by status. Since \textit{bhadda} by itself is a common form of address in the sense “my dear”, I think the suffix \textit{-mukha} has its intensive sense here, “dearest”, rather than “dear face”. } But sometimes I see someone with these bad qualities who is an outer robe wearer. That’s why I say that you don’t deserve the label ‘outer robe wearer’ just because you wear an outer robe. 

Imagine\marginnote{6.4} that just by going naked … wearing dust and dirt … immersing in water … staying at the root of a tree … staying in the open air … standing continually … eating at set intervals … reciting hymns … having matted hair someone with covetousness, ill will, irritability, acrimony, disdain, contempt, jealousy, stinginess, deviousness, deceit, corrupt wishes, and wrong view could give up these things. If that were the case, your friends and colleagues, relatives and kin would make you a matted-hair ascetic as soon as you were born. They’d encourage you: ‘Please, dearest, become a matted-hair ascetic! By doing so you will give up covetousness, ill will, irritability, hostility, disdain, contempt, jealousy, stinginess, deviousness, deceit, corrupt wishes, and wrong view.’ But sometimes I see someone with these bad qualities who is a matted-hair ascetic. That’s why I say that you don’t deserve the label ‘matted-hair ascetic’ just because you have matted hair. 

And\marginnote{7.1} how does a mendicant practice in the way that is proper for an ascetic? 

Any\marginnote{7.2} mendicant who has given up covetousness, ill will, irritability, acrimony, disdain, contempt, jealousy, stinginess, deviousness, deceit, corrupt wishes, and wrong view is practicing in the way that is proper for an ascetic, I say. And that is due to giving up these stains, defects, and dregs of an ascetic, these grounds for rebirth in places of loss, to be experienced in bad places. 

They\marginnote{8.1} see themselves purified from all these bad, unskillful qualities. Seeing this, joy springs up. Being joyful, rapture springs up. When the mind is full of rapture, the body becomes tranquil. When the body is tranquil, they feel bliss. And when blissful, the mind becomes immersed in \textsanskrit{samādhi}. 

They\marginnote{9.1} meditate spreading a heart full of love to one direction, and to the second, and to the third, and to the fourth. In the same way above, below, across, everywhere, all around, they spread a heart full of love to the whole world—abundant, expansive, limitless, free of enmity and ill will. 

They\marginnote{10{-}12.1} meditate spreading a heart full of compassion … 

They\marginnote{10{-}12.2} meditate spreading a heart full of rejoicing … 

They\marginnote{10{-}12.3} meditate spreading a heart full of equanimity to one direction, and to the second, and to the third, and to the fourth. In the same way above, below, across, everywhere, all around, they spread a heart full of equanimity to the whole world—abundant, expansive, limitless, free of enmity and ill will. 

Suppose\marginnote{13.1} there was a lotus pond with clear, sweet, cool water, clean, with smooth banks, delightful. Then along comes a person—whether from the east, west, north, or south—struggling in the oppressive heat, weary, thirsty, and parched. No matter what direction they come from, when they arrive at that lotus pond they would alleviate their thirst and heat exhaustion.\footnote{In such cases \textit{\textsanskrit{pariḷāha}} refers to “heat exhaustion” rather than “fever”. } 

In\marginnote{13.4} the same way, suppose someone has gone forth from the lay life to homelessness—whether from a family of aristocrats, brahmins, peasants, or menials—and has arrived at the teaching and training proclaimed by a Realized One. Having developed love, compassion, rejoicing, and equanimity in this way they gain inner peace. Because of that inner peace they are practicing the way proper for an ascetic, I say. 

And\marginnote{14.1} suppose someone has gone forth from the lay life to homelessness—whether from a family of aristocrats, brahmins, peasants, or workers—and they realize the undefiled freedom of heart and freedom by wisdom in this very life. And they live having realized it with their own insight due to the ending of defilements. They’re an ascetic because of the ending of defilements.” 

That\marginnote{14.4} is what the Buddha said. Satisfied, the mendicants approved what the Buddha said. 

%
\addtocontents{toc}{\let\protect\contentsline\protect\nopagecontentsline}
\chapter*{The Lesser Chapter on Pairs }
\addcontentsline{toc}{chapter}{\tocchapterline{The Lesser Chapter on Pairs }}
\addtocontents{toc}{\let\protect\contentsline\protect\oldcontentsline}

%
\section*{{\suttatitleacronym MN 41}{\suttatitletranslation The People of Sālā }{\suttatitleroot Sāleyyakasutta}}
\addcontentsline{toc}{section}{\tocacronym{MN 41} \toctranslation{The People of Sālā } \tocroot{Sāleyyakasutta}}
\markboth{The People of Sālā }{Sāleyyakasutta}
\extramarks{MN 41}{MN 41}

\scevam{So\marginnote{1.1} I have heard. }At one time the Buddha was wandering in the land of the Kosalans together with a large \textsanskrit{Saṅgha} of mendicants when he arrived at a village of the Kosalan brahmins named \textsanskrit{Sālā}.\footnote{This village is possibly connected to the \textsanskrit{Śālagrāma} famed in later years as the origin of the ammonite stones found on the \textsanskrit{Gaṇḍakī} river and worshiped as aniconic forms of Vishnu. Against this theory is that \textsanskrit{Śālagrāma} is too far to the north-east, although in truth we do not know the exact extent of Kosala. The feminine ending of the Pali form suggests that the name means “hall”, but Sanskrit sources trace \textsanskrit{Śālagrāma} to the sal trees that grow abundantly in the Himalayan foothills. | The same framing narrative recurs at \href{https://suttacentral.net/mn60/en/sujato}{MN 60}, where the topic of rebirth is also discussed. } 

The\marginnote{2.1} brahmins and householders of \textsanskrit{Sālā} heard, “It seems the ascetic Gotama—a Sakyan, gone forth from a Sakyan family—while wandering in the land of the Kosalans has arrived at \textsanskrit{Sālā}, together with a large \textsanskrit{Saṅgha} of mendicants. He has this good reputation: ‘That Blessed One is perfected, a fully awakened Buddha, accomplished in knowledge and conduct, holy, knower of the world, supreme guide for those who wish to train, teacher of gods and humans, awakened, blessed.’ He has realized with his own insight this world—with its gods, \textsanskrit{Māras}, and divinities, this population with its ascetics and brahmins, gods and humans—and he makes it known to others. He proclaims a teaching that is good in the beginning, good in the middle, and good in the end, meaningful and well-phrased. He reveals an entirely full and pure spiritual life. It’s good to see such perfected ones.” 

Then\marginnote{3.1} the brahmins and householders of \textsanskrit{Sālā} went up to the Buddha. Before sitting down to one side, some bowed, some exchanged greetings and polite conversation, some held up their joined palms toward the Buddha, some announced their name and clan, while some kept silent. Seated to one side they said to the Buddha: 

“What\marginnote{4.1} is the cause, Mister Gotama, what is the reason why some sentient beings, when their body breaks up, after death, are reborn in a place of loss, a bad place, the underworld, hell?\footnote{Multiple theories of rebirth are found in Brahmanical texts. For example, \textsanskrit{Bṛhadāraṇyaka} \textsanskrit{Upaniṣad} 4.4.5 says that rebirth depends on good or bad deeds, but notes the differing opinion that desire is all that matters. \textsanskrit{Chāndogya} \textsanskrit{Upaniṣad} 5.10.1–5 describes two paths: forest ascetics go the \textsanskrit{Brahmā} realm, while ritualists go to the moon, from whence those who did good deeds are reborn as human in a good caste, but those who did bad deeds are reborn as outcastes, pigs, or dogs. Those who follow neither path are reborn as small animals or insects. } And what is the cause, Mister Gotama, what is the reason why some sentient beings, when their body breaks up, after death, are reborn in a good place, a heavenly realm?” 

“Unprincipled\marginnote{5.1} and immoral conduct is the reason why some sentient beings, when their body breaks up, after death, are reborn in a place of loss, a bad place, the underworld, hell. Principled and moral conduct is the reason why some sentient beings, when their body breaks up, after death, are reborn in a good place, a heavenly realm.” 

“We\marginnote{6.1} don’t understand the detailed meaning of Mister Gotama’s brief statement. Mister Gotama, please teach us this matter in detail so we can understand the meaning.” 

“Well\marginnote{6.3} then, householders, listen and apply your mind well, I will speak.” 

“Yes,\marginnote{6.4} sir,” they replied. The Buddha said this: 

“Householders,\marginnote{7.1} unprincipled and immoral conduct is threefold by way of body, fourfold by way of speech, and threefold by way of mind.\footnote{This analysis, found frequently in the suttas, became known as the “ten pathways of skilful deeds” (\textit{dasakusalakammapatha}). It was adopted in the treatment of \textit{karma} in \textsanskrit{Manusmṛti} 12, where we find a very similar division of “ten characteristics” (\textit{\textsanskrit{daśalakṣaṇa}}). } 

And\marginnote{8.1} how is unprincipled and immoral conduct threefold by way of body? It’s when a certain person kills living creatures. They’re violent, bloody-handed, a hardened killer, merciless to living beings.\footnote{The same description is applied to the notorious serial killer \textsanskrit{Aṅgulimāla} (\href{https://suttacentral.net/mn86/en/sujato\#2.1}{MN 86:2.1}). These definitions dramatically present clear cases of violation and are not intended to be complete or to cover all grey areas. The Vinaya rules analyze these transgressions in detail as they apply to monastic conduct. } 

They\marginnote{8.3} steal. With the intention to commit theft, they take the wealth or belongings of others from village or wilderness. 

They\marginnote{8.4} commit sexual misconduct. They have sexual relations with women who have their mother, father, both mother and father, brother, sister, relatives, or clan as guardian. They have sexual relations with a woman who is protected on principle, or who has a husband, or whose violation is punishable by law, or even one who has been garlanded as a token of betrothal.\footnote{This definition covers a number of sexual transgressions from a male hetero-normative perspective that, with due allowance for social change, would also be considered inappropriate today: when a woman is married or engaged, or illegal sex, or if she is under guardianship, such as for a child or teen. Elsewhere, the texts identify a number of other illicit sexual acts, such as rape (\href{https://suttacentral.net/an8.84/en/sujato\#1.3}{AN 8.84:1.3}), underage (\href{https://suttacentral.net/snp1.6/en/sujato\#21.2}{Snp 1.6:21.2}), incest (\href{https://suttacentral.net/an5.55/en/sujato}{AN 5.55}), bestiality (\href{https://suttacentral.net/pli-tv-bu-vb-pj1/en/sujato\#6.30}{Bu Pj 1:6.30}), necrophilia (\href{https://suttacentral.net/pli-tv-bu-vb-pj1/en/sujato\#10.13.1}{Bu Pj 1:10.13.1}), and womanizing or promiscuity in general (\href{https://suttacentral.net/an8.54/en/sujato\#6.2}{AN 8.54:6.2}). Notably absent is any criticism of homosexuality, sex outside of marriage, relations with more than two partners, sex work, masturbation, or specific sexual acts. } This is how unprincipled and immoral conduct is threefold by way of body. 

And\marginnote{9.1} how is unprincipled and immoral conduct fourfold by way of speech? It’s when a certain person lies. They’re summoned to a council, an assembly, a family meeting, a guild, or to the royal court, and asked to bear witness: ‘Please, mister, say what you know.’ Not knowing, they say ‘I know.’ Knowing, they say ‘I don’t know.’ Not seeing, they say ‘I see.’ And seeing, they say ‘I don’t see.’ So they deliberately lie for the sake of themselves or another, or for some trivial worldly reason.\footnote{Again this is a severe example of lying, essentially perjury. } 

They\marginnote{9.3} speak divisively. They repeat in one place what they heard in another so as to divide people against each other. And so they divide those who are harmonious, supporting division, delighting in division, loving division, speaking words that promote division. 

They\marginnote{9.4} speak harshly. They use the kinds of words that are cruel, nasty, hurtful, offensive, bordering on anger, not leading to immersion. 

They\marginnote{9.5} talk nonsense. Their speech is untimely, and is neither factual nor beneficial. It has nothing to do with the teaching or the training. Their words have no value, and are untimely, unreasonable, rambling, and pointless. This is how unprincipled and immoral conduct is fourfold by way of speech. 

And\marginnote{10.1} how is unprincipled and immoral conduct threefold by way of mind?\footnote{These three correspond with the respective defilements greed, hate, and delusion. } It's when a certain person is covetous. They covet the wealth and belongings of others: ‘Oh, if only their belongings were mine!’ 

They\marginnote{10.3} have ill will and malicious intentions: ‘May these sentient beings be killed, slaughtered, slain, destroyed, or annihilated!’ 

They\marginnote{10.4} have wrong view. Their perspective is distorted: ‘There’s no meaning in giving, sacrifice, or offerings. There’s no fruit or result of good and bad deeds. There’s no afterlife. There’s no such thing as mother and father, or beings that are reborn spontaneously. And there’s no ascetic or brahmin who is rightly comported and rightly practiced, and who describes the afterlife after realizing it with their own insight.’\footnote{This includes the wrong view that there is no afterlife, but goes beyond that to assert moral nihilism. } This is how unprincipled and immoral conduct is threefold by way of mind. 

That’s\marginnote{10.7} how unprincipled and immoral conduct is the reason why some sentient beings, when their body breaks up, after death, are reborn in a place of loss, a bad place, the underworld, hell. 

Householders,\marginnote{11.1} principled and moral conduct is threefold by way of body, fourfold by way of speech, and threefold by way of mind. 

And\marginnote{12.1} how is principled and moral conduct threefold by way of body? It’s when a certain person gives up killing living creatures. They renounce the rod and the sword. They’re scrupulous and kind, living full of sympathy for all living beings. 

They\marginnote{12.3} give up stealing. They don’t, with the intention to commit theft, take the wealth or belongings of others from village or wilderness. 

They\marginnote{12.4} give up sexual misconduct. They don’t have sexual relations with women who have their mother, father, both mother and father, brother, sister, relatives, or clan as guardian. They don’t have sexual relations with a woman who is protected on principle, or who has a husband, or whose violation is punishable by law, or even one who has been garlanded as a token of betrothal. This is how principled and moral conduct is threefold by way of body. 

And\marginnote{13.1} how is principled and moral conduct fourfold by way of speech? It’s when a certain person gives up lying. They’re summoned to a council, an assembly, a family meeting, a guild, or to the royal court, and asked to bear witness: ‘Please, mister, say what you know.’ Not knowing, they say ‘I don’t know.’ Knowing, they say ‘I know.’ Not seeing, they say ‘I don’t see.’ And seeing, they say ‘I see.’ So they don’t deliberately lie for the sake of themselves or another, or for some trivial worldly reason. 

They\marginnote{13.3} give up divisive speech. They don’t repeat in one place what they heard in another so as to divide people against each other. Instead, they reconcile those who are divided, supporting unity, delighting in harmony, loving harmony, speaking words that promote harmony. 

They\marginnote{13.4} give up harsh speech. They speak in a way that’s mellow, pleasing to the ear, lovely, going to the heart, polite, likable, and agreeable to the people. 

They\marginnote{13.5} give up talking nonsense. Their words are timely, true, and meaningful, in line with the teaching and training. They say things at the right time which are valuable, reasonable, succinct, and beneficial. This is how principled and moral conduct is fourfold by way of speech. 

And\marginnote{14.1} how is principled and moral conduct threefold by way of mind? It's when a certain person is not covetous. They don’t covet the wealth and belongings of others: ‘Oh, if only their belongings were mine!’ 

They\marginnote{14.3} have a kind heart and loving intentions: ‘May these sentient beings live free of enmity and ill will, untroubled and happy!’ 

They\marginnote{14.4} have right view, an undistorted perspective: ‘There is meaning in giving, sacrifice, and offerings. There are fruits and results of good and bad deeds. There is an afterlife. There are such things as mother and father, and beings that are reborn spontaneously. And there are ascetics and brahmins who are rightly comported and rightly practiced, and who describe the afterlife after realizing it with their own insight.’ This is how principled and moral conduct is threefold by way of mind. 

This\marginnote{14.7} is how principled and moral conduct is the reason why some sentient beings, when their body breaks up, after death, are reborn in a good place, a heavenly realm. 

A\marginnote{15.1} person of principled and moral conduct might wish: ‘If only, when my body breaks up, after death, I would be reborn in the company of well-to-do aristocrats!’\footnote{The Buddha is, of course, not recommending this, merely accommodating it. At this point, the ten ways of skillful deeds are both necessary and sufficient for the desired outcome. This changes with the higher realms, where absorption is also required, and the realization of arahantship, which requires the whole noble eightfold path. See note at \href{https://suttacentral.net/mn120/en/sujato\#12.12}{MN 120:12.12}. } It’s possible that this might happen. Why is that? Because they have principled and moral conduct. 

A\marginnote{16{-}17.1} person of principled and moral conduct might wish: ‘If only, when my body breaks up, after death, I would be reborn in the company of well-to-do brahmins … well-to-do householders … the gods of the four great kings … the gods of the thirty-three … the gods of Yama … the joyful gods … the gods who love to imagine … the gods who control what is imagined by others … the gods of the Divinity’s host …\footnote{Rebirth in this realm and higher is said to depend, not just on the ten ways of doing skillful deeds, but on the development of absorption meditation (\href{https://suttacentral.net/dn13/en/sujato\#76.1}{DN 13:76.1}). Sometimes this is expressed by saying that they are “free from desire” (\href{https://suttacentral.net/an8.35/en/sujato\#4.11}{AN 8.35:4.11}, \href{https://suttacentral.net/dn33/en/sujato\#3.1.136}{DN 33:3.1.136}), a stipulation that is in fact stated in the Chinese parallel to this sutta (SA 1042 at T ii 273a13). } the radiant gods …\footnote{The radiant gods are reborn dependent on the second absorption. The following three classes of gods with “radiance” in their names are subsets of the radiant gods. } the gods of limited radiance … the gods of limitless radiance … the gods of streaming radiance … the gods of limited beauty …\footnote{The three classes of devas with “beauty” (\textit{subha}) in their names are reborn due to the third absorption. Some editions have a separate class of “beautiful gods”, of which these three are the subsets, making these follow the same pattern as the “radiant” gods. } the gods of limitless beauty … the gods of universal beauty … the gods of abundant fruit …\footnote{This corresponds to the fourth absorption. } the gods of Aviha …\footnote{These five realms are the Pure Abodes, which are the resort of non-returners only. } the gods of Atappa … the gods fair to see … the fair seeing gods … the gods of \textsanskrit{Akaniṭṭha} … the gods of the dimension of infinite space …\footnote{These four correespond to the formless attainments. } the gods of the dimension of infinite consciousness … the gods of the dimension of nothingness … the gods of the dimension of neither perception nor non-perception.’ It’s possible that this might happen. Why is that? Because they have principled and moral conduct. 

A\marginnote{43.1} person of principled and moral conduct might wish: ‘If only I might realize the undefiled freedom of heart and freedom by wisdom in this very life, and live having realized it with my own insight due to the ending of defilements.’ It’s possible that this might happen. Why is that? Because they have principled and moral conduct.” 

When\marginnote{44.1} he had spoken, the brahmins and householders of \textsanskrit{Sālā} said to the Buddha, “Excellent, Mister Gotama! Excellent! As if he were righting the overturned, or revealing the hidden, or pointing out the path to the lost, or lighting a lamp in the dark so people with clear eyes can see what’s there, Mister Gotama has made the teaching clear in many ways. We go for refuge to Mister Gotama, to the teaching, and to the mendicant \textsanskrit{Saṅgha}. From this day forth, may Mister Gotama remember us as lay followers who have gone for refuge for life.” 

%
\section*{{\suttatitleacronym MN 42}{\suttatitletranslation The People of Verañjā }{\suttatitleroot Verañjakasutta}}
\addcontentsline{toc}{section}{\tocacronym{MN 42} \toctranslation{The People of Verañjā } \tocroot{Verañjakasutta}}
\markboth{The People of Verañjā }{Verañjakasutta}
\extramarks{MN 42}{MN 42}

\scevam{So\marginnote{1.1} I have heard.\footnote{This discourse is identical with the previous except for the location, and a very slight difference in phrasing at \href{https://suttacentral.net/mn42/en/sujato\#7.1}{MN 42:7.1} compared to \href{https://suttacentral.net/mn41/en/sujato\#7.1}{MN 41:7.1}. The Chinese parallels (SA 1042 and SA 1043 at T ii 272c-273b) are also nearly identical to each other, sharing the same location (\textsanskrit{Sāvatthī}), and varying only in the names of the brahmins and their manner of approaching the Buddha. } }At one time the Buddha was staying near \textsanskrit{Sāvatthī} in Jeta’s Grove, \textsanskrit{Anāthapiṇḍika}’s monastery. 

Now\marginnote{2.1} at that time the brahmins and householders of \textsanskrit{Verañjā} were residing in \textsanskrit{Sāvatthī} on some business.\footnote{\textsanskrit{Verañjā} has not been identified, but it lay between \textsanskrit{Madhurā} (\href{https://suttacentral.net/an4.53/en/sujato\#1.1}{AN 4.53:1.1}) and Soreyya (\href{https://suttacentral.net/pli-tv-bu-vb-pj1/en/sujato\#4.18}{Bu Pj 1:4.18}), placing it roughly in the region of modern Agra, about 450 km west of \textsanskrit{Sāvatthī}. } The brahmins and householders of \textsanskrit{Verañjā} heard: 

“It\marginnote{2.3} seems the ascetic Gotama—a Sakyan, gone forth from a Sakyan family—is staying near \textsanskrit{Sāvatthī} in Jeta’s Grove, \textsanskrit{Anāthapiṇḍika}’s monastery. He has this good reputation … It’s good to see such perfected ones.” 

…\marginnote{3.1} They said to the Buddha: “What is the cause, Mister Gotama, what is the reason why some sentient beings, when their body breaks up, after death, are reborn in a place of loss, a bad place, the underworld, hell? And what is the cause, Mister Gotama, what is the reason why some sentient beings, when their body breaks up, after death, are reborn in a good place, a heavenly realm?” 

“Unprincipled\marginnote{5.1} and immoral conduct is the reason why some sentient beings, when their body breaks up, after death, are reborn in a place of loss, a bad place, the underworld, hell. Principled and moral conduct is the reason why some sentient beings, when their body breaks up, after death, are reborn in a good place, a heavenly realm.” 

“We\marginnote{6.1} don’t understand the detailed meaning of Mister Gotama’s brief statement. …” 

“Householders,\marginnote{7.1} a person of unprincipled and immoral conduct is threefold by way of body, fourfold by way of speech, and threefold by way of mind. …” … 

%
\section*{{\suttatitleacronym MN 43}{\suttatitletranslation The Great Elaboration }{\suttatitleroot Mahāvedallasutta}}
\addcontentsline{toc}{section}{\tocacronym{MN 43} \toctranslation{The Great Elaboration } \tocroot{Mahāvedallasutta}}
\markboth{The Great Elaboration }{Mahāvedallasutta}
\extramarks{MN 43}{MN 43}

\scevam{So\marginnote{1.1} I have heard.\footnote{This sutta and the next are named for the class of scripture called \textit{vedalla}, being the only suttas explicitly labelled as such. The root sense is to “split open” (\textit{vi}-√\textit{dal}) like a blossoming flower. These texts go beyond simple analysis or classification to “elaborate” the basic teachings (cf. the Sanskrit, which here has \textit{vaipulya}, “expansion”). They share a similar method, where a series of questions and answers are carefully laid out, without revealing their purpose, but relentlessly driving towards the unraveling of the deepest meditation states emerging in Nibbana. While the Chinese parallels are broadly similar, there are many differences in detail, so they should be carefully studied before drawing any conclusions based on such “elaborations”. } }At one time the Buddha was staying near \textsanskrit{Sāvatthī} in Jeta’s Grove, \textsanskrit{Anāthapiṇḍika}’s monastery. 

Then\marginnote{1.3} in the late afternoon, Venerable \textsanskrit{Mahākoṭṭhita} came out of retreat, went to Venerable \textsanskrit{Sāriputta}, and exchanged greetings with him. When the greetings and polite conversation were over, he sat down to one side and said to \textsanskrit{Sāriputta}: 

“Reverend,\marginnote{2.1} they speak of ‘a witless person’. How is a witless person defined?” 

“Reverend,\marginnote{2.3} they’re called witless because they don’t understand. And what don’t they understand? They don’t understand: ‘This is suffering’ … ‘This is the origin of suffering’ … ‘This is the cessation of suffering’ … ‘This is the practice that leads to the cessation of suffering.’ They’re called witless because they don’t understand.” 

Saying\marginnote{2.7} “Good, reverend,” \textsanskrit{Mahākoṭṭhita} approved and agreed with what \textsanskrit{Sāriputta} said. Then he asked another question: 

“They\marginnote{3.1} speak of ‘a wise person’. How is a wise person defined?” 

“They’re\marginnote{3.3} called wise because they understand. And what do they understand? They understand: ‘This is suffering’ … ‘This is the origin of suffering’ … ‘This is the cessation of suffering’ … ‘This is the practice that leads to the cessation of suffering.’ They’re called wise because they understand.” 

“They\marginnote{4.1} speak of ‘consciousness’. How is consciousness defined?” 

“It’s\marginnote{4.3} called consciousness because it cognizes.\footnote{In defining a noun by its verb, \textsanskrit{Sāriputta} clarifies that consciousness is not an entity but a function. Consciousness is simply the act of being conscious. Similar definitions are proposed for other fundamental Dhamma terms such as “feeling” and “perception”. } And what does it cognize? It cognizes ‘pleasure’ and ‘pain’ and ‘neutral’.\footnote{This definition is subtly different from that of “feeling” below (\href{https://suttacentral.net/mn43/en/sujato\#7.5}{MN 43:7.5}). The addition of the quotative particle \textit{-ti} distances the verb for cognizing from the noun that is cognized. The Chinese and Tibetan parallels to this passage define consciousness here in the standard way as awareness of sense phenomena (MA 211 at T i 790c7, D 4094 \emph{mngon pa, nyu} 81a7), which is more straightforward, but could be a result of normalization. } It’s called consciousness because it cognizes.” 

“Wisdom\marginnote{5.1} and consciousness—are these things mixed or separate? And can we completely disentangle them so as to describe the difference between them?”\footnote{\textit{Vinibbhujati} (“disentangle”) is used for the dissection of a corpse (\href{https://suttacentral.net/thig16.1/en/sujato\#24.1}{Thig 16.1:24.1}) or the unravelling of coiled banana sheaths (\href{https://suttacentral.net/mn35/en/sujato\#22.3}{MN 35:22.3}). } 

“Wisdom\marginnote{5.4} and consciousness—\footnote{“Wisdom” (\textit{\textsanskrit{paññā}})  and “consciousness” (\textit{\textsanskrit{viññāṇa}}) are two of the very many terms derived from the root \textit{\textsanskrit{ñā}}, “to know”. The prefixes act as intensifiers, but do not decisively distinguish the meaning, so while they are used consistently in doctrinal contexts, more loosely they can be interchangeable. } these things are mixed, not separate. And you can never completely disentangle them so as to describe the difference between them. For you understand what you cognize, and you cognize what you understand. That’s why these things are mixed, not separate. And you can never completely disentangle them so as to describe the difference between them.”\footnote{This dialogue cautions against pushing analysis too far. } 

“Wisdom\marginnote{6.1} and consciousness—what is the difference between these things that are mixed, not separate?” 

“The\marginnote{6.3} difference between these things is that wisdom should be developed, while consciousness should be completely understood.”\footnote{This refers to the fundamental distinction made in the first discourse (\href{https://suttacentral.net/sn56.11/en/sujato}{SN 56.11}): that which is to be developed (“wisdom”) pertains to the fourth noble truth, while that which is to be completely understood (“consciousness”) pertains to the first noble truth. Again, the distinction between them is functional rather than ontological. } 

“They\marginnote{7.1} speak of this thing called ‘feeling’. How is feeling defined?” 

“It’s\marginnote{7.3} called feeling because it feels. And what does it feel? It feels pleasure, pain, and neutral.\footnote{This definition is similar to that for consciousness, except without the distancing \textit{-ti}. The feeling is the experience, whereas consciousnesses is that which is aware of the feeling. | The progress of the text is more subtle than it appears, as the question on feeling and consciousness starts to lay the groundwork for understanding the deep states of meditation that are discussed later. } It’s called feeling because it feels.” 

“They\marginnote{8.1} speak of this thing called ‘perception’. How is perception defined?” 

“It’s\marginnote{8.3} called perception because it perceives. And what does it perceive? It perceives blue, yellow, red, and white.\footnote{“Color” is a more sophisticated and less universal aspect of awareness than “feeling”, reflecting the fact that “perception” relates to higher-order functions like recognition and interpretation, which are involved in concept formation. The “perception” of lights and colors is often connected with the development of the four “form” \textit{\textsanskrit{jhānas}} (eg. \href{https://suttacentral.net/mn77/en/sujato\#23.14}{MN 77:23.14}), so we are paving the way for the apparently abrupt transition to the formless dimensions at \href{https://suttacentral.net/mn43/en/sujato\#10.2}{MN 43:10.2} below. } It’s called perception because it perceives.” 

“Feeling,\marginnote{9.1} perception, and consciousness—are these things mixed or separate? And can we completely disentangle them so as to describe the difference between them?” 

“Feeling,\marginnote{9.4} perception, and consciousness—these things are mixed, not separate. And you can never completely disentangle them so as to describe the difference between them. For you perceive what you feel, and you cognize what you perceive. That’s why these things are mixed, not separate. And you can never completely disentangle them so as to describe the difference between them.” 

“What\marginnote{10.1} can be known by purified mind consciousness released from the five senses?”\footnote{The key term “purified” (\textit{parisuddha}) indicates the fourth absorption. } 

“Aware\marginnote{10.2} that ‘space is infinite’ it can know the dimension of infinite space. Aware that ‘consciousness is infinite’ it can know the dimension of infinite consciousness. Aware that ‘there is nothing at all’ it can know the dimension of nothingness.” 

“How\marginnote{11.1} do you understand something that can be known?” 

“You\marginnote{11.2} understand something that can be known with the eye of wisdom.”\footnote{In other words, through personal experience rather than scripture, tradition, etc. } 

“What\marginnote{12.1} is the purpose of wisdom?” 

“The\marginnote{12.2} purpose of wisdom is direct knowledge, complete understanding, and giving up.”\footnote{Wisdom is explained in functional terms through the effect that it produces. } 

“How\marginnote{13.1} many conditions are there for the arising of right view?”\footnote{That is, for stream-entry. } 

“There\marginnote{13.2} are two conditions for the arising of right view: the voice of another and rational application of mind.\footnote{The “voice of another” is the teachings of the Dhamma, but these have to be actually investigated and applied internally. } These are the two conditions for the arising of right view.” 

“When\marginnote{14.1} right view is supported by how many factors does it have freedom of heart and freedom by wisdom as its fruit and benefit?”\footnote{That is, how does a stream-enterer practice further for arahantship? } 

“When\marginnote{14.2} right view is supported by five factors it has freedom of heart and freedom by wisdom as its fruit and benefit. It’s when right view is supported by ethics, learning, discussion, serenity, and discernment. When right view is supported by these five factors it has freedom of heart and freedom by wisdom as its fruit and benefit.” 

“How\marginnote{15.1} many states of existence are there?” 

“Reverend,\marginnote{15.2} there are these three states of existence.\footnote{These three questions summarize dependent origination. } Existence in the sensual realm, the realm of luminous form, and the formless realm.” 

“But\marginnote{16.1} how is there rebirth into a new state of existence in the future?” 

“It’s\marginnote{16.2} because of sentient beings—shrouded by ignorance and fettered by craving—taking pleasure wherever they land.\footnote{The term “shrouded by ignorance” (\textit{\textsanskrit{avijjānīvaraṇa}}) draws on the connection between \textit{\textsanskrit{nīvaraṇa}}, normally translated “hindrance”, and the cosmic serpent \textsanskrit{Vṛtra}, the “constrictor” who wraps the world in darkness. } That’s how there is rebirth into a new state of existence in the future.” 

“But\marginnote{17.1} how is there no rebirth into a new state of existence in the future?” 

“It’s\marginnote{17.2} when ignorance fades away, knowledge arises, and craving ceases. That’s how there is no rebirth into a new state of existence in the future.” 

“But\marginnote{18.1} what, reverend, is the first absorption?” 

“Reverend,\marginnote{18.2} it’s when a mendicant, quite secluded from sensual pleasures, secluded from unskillful qualities, enters and remains in the first absorption, which has the rapture and bliss born of seclusion, while placing the mind and keeping it connected. This is called the first absorption.” 

“But\marginnote{19.1} how many factors does the first absorption have?” 

“The\marginnote{19.2} first absorption has five factors. When a mendicant has entered the first absorption, placing the mind, keeping it connected, rapture, bliss, and unification of mind are present.\footnote{These five absorption factors are found in the suttas here and, with other factors appended, at \href{https://suttacentral.net/mn111/en/sujato\#4.1}{MN 111:4.1}. The list summarizes the normal depiction of the first absorption, as the first four factors are all part of the standard first absorption formula, and all absorption or \textit{\textsanskrit{samādhi}} is characterized by unification of mind (eg. \href{https://suttacentral.net/mn20/en/sujato\#3.3}{MN 20:3.3}). They became a standard analysis of the first absorption in late canonical texts (eg. \href{https://suttacentral.net/ps1.5/en/sujato\#10.2}{Ps 1.5:10.2}, \href{https://suttacentral.net/ds2.1.1/en/sujato\#3.1}{Ds 2.1.1:3.1}, \href{https://suttacentral.net/vb12/en/sujato\#99.1}{Vb 12:99.1}, \href{https://suttacentral.net/pe6/en/sujato\#66.2}{Pe 6:66.2}). } That’s how the first absorption has five factors.” 

“But\marginnote{20.1} how many factors has the first absorption given up and how many does it possess?” 

“The\marginnote{20.2} first absorption has given up five factors and possesses five factors. When a mendicant has entered the first absorption, sensual desire, ill will, dullness and drowsiness, restlessness and remorse, and doubt are given up. Placing the mind, keeping it connected, rapture, bliss, and unification of mind are present.\footnote{All of these, except unification, are present to some degree before absorption, but now they manifest fully. } That’s how the first absorption has given up five factors and possesses five factors.” 

“Reverend,\marginnote{21.1} these five faculties have different domains and different ranges, and don’t experience each others’ domain and range. That is,\footnote{Again, apparently general questions are in fact laying the groundwork for a discussion on subtle sates of meditation. } the faculties of the eye, ear, nose, tongue, and body. What do these five faculties, with their different domains and ranges, have recourse to? What experiences their domains and ranges?” 

“These\marginnote{21.4} five faculties, with their different domains and ranges, have recourse to the mind. And the mind experiences their domains and ranges.”\footnote{Each kind of sense experience is quite separate and distinct. The mind brings them all together and creates a sensible world in which a “self” can operate. } 

“These\marginnote{22.1} five faculties depend on what to continue?” 

“These\marginnote{22.4} five faculties depend on vitality to continue.” 

“But\marginnote{22.7} what does vitality depend on to continue?” 

“Vitality\marginnote{22.8} depends on warmth to continue.” 

“But\marginnote{22.9} what does warmth depend on to continue?” 

“Warmth\marginnote{22.10} depends on vitality to continue.” 

“Just\marginnote{22.11} now I understood you to say: ‘Vitality depends on warmth to continue.’ But I also understood you to say: ‘Warmth depends on vitality to continue.’ How then should we see the meaning of this statement?” 

“Well\marginnote{22.16} then, reverend, I shall give you a simile. For by means of a simile some sensible people understand the meaning of what is said. Suppose there was an oil lamp burning. The light appears dependent on the flame, and the flame appears dependent on the light. In the same way, vitality depends on warmth to continue, and warmth depends on vitality to continue.” 

“Are\marginnote{23.1} the vital forces the same things as the phenomena that are felt? Or are they different things?”\footnote{“Vital force” is \textit{\textsanskrit{āyusaṅkhāra}}. The suttas also use \textit{\textsanskrit{bhavasaṅkhāra}} (\href{https://suttacentral.net/dn16/en/sujato\#3.10.5}{DN 16:3.10.5}) and \textit{\textsanskrit{jīvitasaṅkhāra}} (\href{https://suttacentral.net/dn16/en/sujato\#2.23.5}{DN 16:2.23.5}) synonymously. } 

“The\marginnote{23.2} vital forces are not the same things as the phenomena that are felt. For if the vital forces and the phenomena that are felt were the same things, a mendicant who had attained the cessation of perception and feeling would not emerge from it.\footnote{This introduces the most subtle of all meditation states, accessible only to non-returners and arahants who are fully accomplished in all the absorptions. } But because the vital forces and the phenomena that are felt are different things, a mendicant who has attained the cessation of perception and feeling can emerge from it.” 

“How\marginnote{24.1} many things must this body lose before it lies abandoned, tossed aside like an insentient log?” 

“This\marginnote{24.2} body must lose three things before it lies abandoned, tossed aside like an insentient log: vitality, warmth, and consciousness.”\footnote{This passage assumes the existence of a distinct vital force that is one of three factors required for life, which is why I have translated \textit{\textsanskrit{āyu}} here as “vitality” rather than “life”. } 

“What’s\marginnote{25.1} the difference between someone who has passed away and a mendicant who has attained the cessation of perception and feeling?”\footnote{This distinction is critical, as it sometimes happens that a person in deep meditation seems as if dead. } 

“When\marginnote{25.2} someone dies, their physical, verbal, and mental processes have ceased and stilled; their vitality is spent; their warmth is dissipated; and their faculties have disintegrated.\footnote{These processes are defined in the next sutta (\href{https://suttacentral.net/mn44/en/sujato\#14.1}{MN 44:14.1}). } When a mendicant has attained the cessation of perception and feeling, their physical, verbal, and mental processes have ceased and stilled. But their vitality is not spent; their warmth is not dissipated; and their faculties are very clear. That’s the difference between someone who has passed away and a mendicant who has attained the cessation of perception and feeling.” 

“How\marginnote{26.1} many conditions are necessary to attain the neutral release of the heart?” 

“Four\marginnote{26.2} conditions are necessary to attain the neutral release of the heart. Giving up pleasure and pain, and ending former happiness and sadness, a mendicant enters and remains in the fourth absorption, without pleasure or pain, with pure equanimity and mindfulness. These four conditions are necessary to attain the neutral release of the heart.” 

“How\marginnote{27.1} many conditions are necessary to attain the signless release of the heart?” 

“Two\marginnote{27.2} conditions are necessary to attain the signless release of the heart: not focusing on any signs, and focusing on the signless. These two conditions are necessary to attain the signless release of the heart.” 

“How\marginnote{27.5} many conditions are necessary to remain in the signless release of the heart?” 

“Three\marginnote{28.1} conditions are necessary to remain in the signless release of the heart: not focusing on any signs, focusing on the signless, and a previous determination.\footnote{Before entering the meditation, one makes an inner resolve to remain there for a certain period of time. } These three conditions are necessary to remain in the signless release of the heart.” 

“How\marginnote{29.1} many conditions are necessary to emerge from the signless release of the heart?” 

“Two\marginnote{29.2} conditions are necessary to emerge from the signless release of the heart: focusing on all signs, and not focusing on the signless. These two conditions are necessary to emerge from the signless release of the heart.” 

“The\marginnote{29.5} limitless release of the heart, and the release of the heart through nothingness, and the release of the heart through emptiness, and the signless release of the heart: do these things differ in both meaning and phrasing? Or do they mean the same thing, and differ only in the phrasing?” 

“There\marginnote{30.1} is a way in which these things differ in both meaning and phrasing. But there’s also a way in which they mean the same thing, and differ only in the phrasing.\footnote{These terms can be used for distinct states of meditation as described below, but they can all be used of arahantship as well. } 

And\marginnote{31.1} what’s the way in which these things differ in both meaning and phrasing? 

Firstly,\marginnote{31.2} a mendicant meditates spreading a heart full of love to one direction, and to the second, and to the third, and to the fourth. In the same way above, below, across, everywhere, all around, they spread a heart full of love to the whole world—abundant, expansive, limitless, free of enmity and ill will. They meditate spreading a heart full of compassion … They meditate spreading a heart full of rejoicing … They meditate spreading a heart full of equanimity to one direction, and to the second, and to the third, and to the fourth. In the same way above, below, across, everywhere, all around, they spread a heart full of equanimity to the whole world—abundant, expansive, limitless, free of enmity and ill will. This is called the limitless release of the heart. 

And\marginnote{32.1} what is the release of the heart through nothingness? It’s when a mendicant, going totally beyond the dimension of infinite consciousness, aware that ‘there is nothing at all’, enters and remains in the dimension of nothingness. This is called the heart’s release through nothingness. 

And\marginnote{33.1} what is the release of the heart through emptiness? It’s when a mendicant has gone to a wilderness, or to the root of a tree, or to an empty hut, and reflects like this: ‘This is empty of a self or what belongs to a self.’ This is called the release of the heart through emptiness. 

And\marginnote{34.1} what is the signless release of the heart? It’s when a mendicant, not focusing on any signs, enters and remains in the signless immersion of the heart. This is called the signless release of the heart. 

This\marginnote{34.4} is the way in which these things differ in both meaning and phrasing. 

And\marginnote{35.1} what’s the way in which they mean the same thing, and differ only in the phrasing? 

Greed,\marginnote{35.2} hate, and delusion are makers of limits. A mendicant who has ended the defilements has given these up, cut them off at the root, made them like a palm stump, and obliterated them, so they are unable to arise in the future. The unshakable release of the heart is said to be the best kind of limitless release of the heart. That unshakable release of the heart is empty of greed, hate, and delusion. 

Greed\marginnote{36.1} is something, hate is something, and delusion is something. A mendicant who has ended the defilements has given these up, cut them off at the root, made them like a palm stump, and obliterated them, so they are unable to arise in the future. The unshakable release of the heart is said to be the best kind of release of the heart through nothingness. That unshakable release of the heart is empty of greed, hate, and delusion. 

Greed,\marginnote{37.1} hate, and delusion are makers of signs. A mendicant who has ended the defilements has given these up, cut them off at the root, made them like a palm stump, and obliterated them, so they are unable to arise in the future. The unshakable release of the heart is said to be the best kind of signless release of the heart. That unshakable release of the heart is empty of greed, hate, and delusion. 

This\marginnote{37.5} is the way in which they mean the same thing, and differ only in the phrasing.” 

This\marginnote{37.6} is what Venerable \textsanskrit{Sāriputta} said. Satisfied, Venerable \textsanskrit{Mahākoṭṭhita} approved what \textsanskrit{Sāriputta} said. 

%
\section*{{\suttatitleacronym MN 44}{\suttatitletranslation The Shorter Elaboration }{\suttatitleroot Cūḷavedallasutta}}
\addcontentsline{toc}{section}{\tocacronym{MN 44} \toctranslation{The Shorter Elaboration } \tocroot{Cūḷavedallasutta}}
\markboth{The Shorter Elaboration }{Cūḷavedallasutta}
\extramarks{MN 44}{MN 44}

\scevam{So\marginnote{1.1} I have heard. }At one time the Buddha was staying near \textsanskrit{Rājagaha}, in the Bamboo Grove, the squirrels’ feeding ground. 

Then\marginnote{1.3} the layman \textsanskrit{Visākha} went to see the nun \textsanskrit{Dhammadinnā}, bowed, sat down to one side, and said to her:\footnote{These two are little-known, in contrast with the previous \textit{vedalla} featuring a dialogue between two mendicants renowned for their wisdom. | \textsanskrit{Dhammadinnā} was recognized as the foremost \textsanskrit{bhikkhunī} in teaching (\href{https://suttacentral.net/an1.239/en/sujato\#1.1}{AN 1.239:1.1}), presumably on the basis of this sutta, and has a single verse to her name (\href{https://suttacentral.net/thig1.12/en/sujato}{Thig 1.12}). | \textsanskrit{Visākha} is otherwise unmentioned in the canon. The commentary says this is not the same person as the mendicant known as \textsanskrit{Visākha}, \textsanskrit{Pañcāli}’s son (\href{https://suttacentral.net/thag2.45/en/sujato}{Thag 2.45}, \href{https://suttacentral.net/an4.48/en/sujato}{AN 4.48}, \href{https://suttacentral.net/sn21.7/en/sujato}{SN 21.7}). } 

“Ma’am,\marginnote{2.1} they speak of this thing called ‘substantial reality’.\footnote{The normal form of address for nuns is \textit{\textsanskrit{ayyā}} (“ma’am”), from the root \textit{ariya} (“noble”); the masculine form \textit{ayya} is occasionally used for monks as well. | “Substantial reality” is \textit{\textsanskrit{sakkāya}}, from \textit{sat} (“real”) and \textit{\textsanskrit{kāya}} (“substance”). The Jain form is \textit{\textsanskrit{astikāya}}, which refers to the five fundamental substances (or ontological categories) comprised of the medium of motion, the medium of rest, space, soul, and matter (\textsanskrit{Bhagavatīsūtra} 1.10). Note that the suttas use \textit{\textsanskrit{kāya}}, in the sense of “substance” or “mass”, as a key doctrinal term for ascetic movements associated with Jainism (eg. \href{https://suttacentral.net/dn2/en/sujato\#23.3}{DN 2:23.3}, \href{https://suttacentral.net/dn2/en/sujato\#26.2}{DN 2:26.2}), whereas \textit{\textsanskrit{kāya}} appears to have had no philosophical importance in pre-Buddhist Vedic texts. The commentaries define \textit{\textsanskrit{sakkāya}} as “the three planes of cyclic existence”, i.e. all that exists. Thus \textit{\textsanskrit{sakkāya}} is the “substantial reality” that is mistakenly assumed to be a “self”. } What is this substantial reality that the Buddha spoke of?” 

“\textsanskrit{Visākha},\marginnote{2.3} the Buddha said that these five grasping aggregates are substantial reality.\footnote{\textsanskrit{Sāriputta} also quotes the Buddha as saying this at \href{https://suttacentral.net/sn38.15/en/sujato\#1.3}{SN 38.15:1.3}, both sources apparently drawing from \href{https://suttacentral.net/sn22.105/en/sujato\#1.5}{SN 22.105:1.5} and \href{https://suttacentral.net/sn22.103/en/sujato\#1.6}{SN 22.103:1.6}. } That is, the grasping aggregates of form, feeling, perception, choices, and consciousness. The Buddha said that these five grasping aggregates are substantial reality.” 

Saying\marginnote{2.6} “Good, ma’am,” \textsanskrit{Visākha} approved and agreed with what \textsanskrit{Dhammadinnā} said. Then he asked another question: 

“Ma’am,\marginnote{3.1} they speak of this thing called ‘the origin of substantial reality’. What is the origin of substantial reality that the Buddha spoke of?” 

“It’s\marginnote{3.3} the craving that leads to future lives, mixed up with relishing and greed, taking pleasure wherever it lands. That is,\footnote{This treats “substantial reality” in terms of the four noble truths, a method also followed in multiple places such as \href{https://suttacentral.net/an4.33/en/sujato\#2.2}{AN 4.33:2.2}, \href{https://suttacentral.net/sn22.103/en/sujato\#1.4}{SN 22.103:1.4}, \href{https://suttacentral.net/sn22.105/en/sujato\#1.2}{SN 22.105:1.2}, and \href{https://suttacentral.net/sn22.78/en/sujato\#4.1}{SN 22.78:4.1}. } craving for sensual pleasures, craving to continue existence, and craving to end existence. The Buddha said that this is the origin of substantial reality.” 

“Ma’am,\marginnote{4.1} they speak of this thing called ‘the cessation of substantial reality’. What is the cessation of substantial reality that the Buddha spoke of?” 

“It’s\marginnote{4.3} the fading away and cessation of that very same craving with nothing left over; giving it away, letting it go, releasing it, and not clinging to it. The Buddha said that this is the cessation of substantial reality.” 

“Ma’am,\marginnote{5.1} they speak of the practice that leads to the cessation of substantial reality. What is the practice that leads to the cessation of substantial reality that the Buddha spoke of?” 

“The\marginnote{5.3} practice that leads to the cessation of substantial reality that the Buddha spoke of is simply this noble eightfold path, that is: right view, right thought, right speech, right action, right livelihood, right effort, right mindfulness, and right immersion.” 

“But\marginnote{6.1} ma’am, is that grasping the exact same thing as the five grasping aggregates? Or is grasping one thing and the five grasping aggregates another?” 

“That\marginnote{6.2} grasping is not the exact same thing as the five grasping aggregates. Nor is grasping one thing and the five grasping aggregates another. The desire and greed for the five grasping aggregates is the grasping there.”\footnote{In the suttas, the five aggregates are not presented as a catch-all category that encompasses all of reality, but rather five types of phenomena that provoke attachment, forming the basis of what we take to be “self”. Desire is what drives the formation of attachment, but it requires all the aggregates to function. Thus the “grasping” aggregates are numbered “five” after the model of the hand that grasps. } 

“But\marginnote{7.1} ma’am, how does substantialist view come about?” 

“It’s\marginnote{7.2} when an unlearned ordinary person has not seen the noble ones, and is neither skilled nor trained in the teaching of the noble ones. They’ve not seen true persons, and are neither skilled nor trained in the teaching of the true persons. They regard form as self, self as having form, form in self, or self in form.\footnote{These are the twenty kinds of substantialist view. } They regard feeling … perception … choices … consciousness as self, self as having consciousness, consciousness in self, or self in consciousness. That’s how substantialist view comes about.” 

“But\marginnote{8.1} ma’am, how does substantialist view not come about?” 

“It’s\marginnote{8.2} when a learned noble disciple has seen the noble ones, and is skilled and trained in the teaching of the noble ones. They’ve seen true persons, and are skilled and trained in the teaching of the true persons. They don’t regard form as self, self as having form, form in self, or self in form. They don’t regard feeling … perception … choices … consciousness as self, self as having consciousness, consciousness in self, or self in consciousness. That’s how substantialist view does not come about.” 

“But\marginnote{9.1} ma’am, what is the noble eightfold path?” 

“It\marginnote{9.2} is simply this noble eightfold path, that is: right view, right thought, right speech, right action, right livelihood, right effort, right mindfulness, and right immersion.” 

“But\marginnote{10.1} ma’am, is the noble eightfold path conditioned or unconditioned?” 

“The\marginnote{10.2} noble eightfold path is conditioned.” 

“Are\marginnote{11.1} the three spectrums of practice included in the noble eightfold path? Or is the noble eightfold path included in the three practice categories?”\footnote{This refers to the “three spectrums” of ethics, immersion, and wisdom (eg. \href{https://suttacentral.net/an3.143/en/sujato\#2.3}{AN 3.143:2.3}). The shorthand reference to them as a group of three is characteristic of later strata of Pali, but it is also found at \href{https://suttacentral.net/dn10/en/sujato\#1.6.1}{DN 10:1.6.1}, and with the addition of \textit{vimutti} at \href{https://suttacentral.net/dn33/en/sujato\#1.11.145}{DN 33:1.11.145}. | “Spectrum” here is \textit{khandha} (elsewhere “aggregate”), which again has the same basic sense of “mass”, “category” as does \textit{\textsanskrit{kāya}} in \textit{\textsanskrit{sakkāya}}. } 

“The\marginnote{11.2} three spectrums of practice are not included in the noble eightfold path. Rather, the noble eightfold path is included in the three practice categories.\footnote{The two teachings cover the same ground, but the more specific factors of the eightfold path are included in the more general categories of the three spectra. } Right speech, right action, and right livelihood: these things are included in the spectrum of ethics. Right effort, right mindfulness, and right immersion: these things are included in the spectrum of immersion. Right view and right thought: these things are included in the spectrum of wisdom.” 

“But\marginnote{12.1} ma’am, what is immersion? What things are the bases of immersion? What things are the prerequisites for immersion? What is the development of immersion?” 

“Unification\marginnote{12.2} of the mind is immersion.\footnote{“Unification” (\textit{\textsanskrit{ekaggatā}}), “immersion” (\textit{\textsanskrit{samādhi}}), and “absorption” (\textit{\textsanskrit{jhāna}}) are synonyms, although they may have specialized nuances. This understanding was common to Indic traditions, possibly under Buddhist influence. As just a few examples, \textsanskrit{Patañjali}’s \textsanskrit{Yogasūtra} 3.11 defines \textit{\textsanskrit{samādhi}} as unification, as does \textsanskrit{Śaṅkāra} on \textsanskrit{Bṛhadāraṇyaka} \textsanskrit{Upaniṣad} 4.4.3 (\textit{\textsanskrit{aikāgryarūpeṇa} \textsanskrit{samāhito}}), and the commentary to Bhagavad \textsanskrit{Gīta} 2.42–44 (\textit{\textsanskrit{samādhiś} \textsanskrit{cittaikāgryam}}) and 17.11 (\textit{\textsanskrit{samādhāyaikāgraṁ}}). The Jain \textsanskrit{Tattvārthasūtra} 9.27 similarly defines \textit{\textsanskrit{dhyāna}}, as does the commentary to Bhagavad \textsanskrit{Gīta} 4.27 (\textit{\textsanskrit{dhyānaikāgryam}}) and 13.24 (\textit{\textsanskrit{ekāgratayā} yac \textsanskrit{cintanaṁ} tad \textsanskrit{dhyānam}}). } The four kinds of mindfulness meditation are the bases for immersion.\footnote{The “foundations of immersion” (\textit{\textsanskrit{samādhinimittā}}) are the practices that lead to \textit{\textsanskrit{jhāna}}. Here \textit{nimitta} means “cause” (Commentary: \textit{paccaya}). } The four right efforts are the prerequisites for immersion.\footnote{Success in mindfulness meditation is the immediate cause of immersion, but to achieve both mindfulness and immersion it is a prerequisite (\textit{\textsanskrit{parikkhāra}}) to make an effort. } The cultivation, development, and making much of these very same things is the development of immersion.”\footnote{“Development” (\textit{\textsanskrit{bhāvanā}}) means “growing”, “amplifying”. } 

“How\marginnote{13.1} many processes are there?” 

“There\marginnote{13.2} are these three processes. Physical, verbal, and mental processes.”\footnote{These “processes” (\textit{\textsanskrit{saṅkhārā}}) apply, as we shall see, in the development of meditation (\href{https://suttacentral.net/sn41.6/en/sujato\#1.5}{SN 41.6:1.5}, \href{https://suttacentral.net/dn18/en/sujato\#24.1}{DN 18:24.1}). They must be distinguished from the similarly named three kinds of “choices” (\textit{\textsanskrit{saṅkhārā}}) that define volitional activity determining rebirth (i.e. karma, \href{https://suttacentral.net/mn57/en/sujato\#8.2}{MN 57:8.2}). In the three “processes”, the final item is always \textit{\textsanskrit{cittasaṅkhāra}} (“mental process”), whereas in the three “choices” the final item is usually \textit{\textsanskrit{manosaṅkhāra}}. While \textit{mano} and \textit{citta} both mean “mind”, \textit{mano} leans to the sense of “volition”, while \textit{citta} leans to the sense of “awareness”. It is a quirk of the Pali texts that they usually have \textit{\textsanskrit{cittasaṅkhāra}} in the context of dependent origination (but see \href{https://suttacentral.net/sn12.25/en/sujato\#13.1}{SN 12.25:13.1}), where the Sanskrit has the expected  \textit{\textsanskrit{manaḥsaṁskāraḥ}} (SF 165 \textsanskrit{Ādisūtra}, Arv 5.2 \textsanskrit{Arthaviniścayasūtra}, SF 238 \textsanskrit{Pratītyasamutpādādivibhaṅganirdeśa}). } 

“But\marginnote{14.1} ma’am, what is the physical process? What’s the verbal process? What’s the mental process?” 

“Breathing\marginnote{14.2} is a physical process. Placing the mind and keeping it connected are verbal processes. Perception and feeling are mental processes.” 

“But\marginnote{15.1} ma’am, why is breathing a physical process? Why are placing the mind and keeping it connected verbal processes? Why are perception and feeling mental processes?” 

“Breathing\marginnote{15.2} is physical. It’s tied up with the body, that’s why breathing is a physical process.\footnote{Breathing is referred to as a “physical process” in the context of breath meditation to emphasize the natural stilling that occurs as the meditation settles. | As in \href{https://suttacentral.net/mn43/en/sujato}{MN 43}, the questions are laying the groundwork for a discussion of deep meditation. Hence in these definitions what is being discussed is not “processes” in general, but the very last residue of activity or disturbance whose cessation marks the transition to a deeper level of consciousness. } First you place the mind and keep it connected, then you break into speech. That’s why placing the mind and keeping it connected are verbal processes.\footnote{\textit{Vitakka} and \textit{\textsanskrit{vicāra}} have a more basic sense in ordinary states of mind (“thought” and “exploring”) and a more refined sense in the elevated consciousness of \textit{\textsanskrit{jhāna}} (“placing the mind” and “keeping it connected”). They act as a condition for breaking into speech, so any hint of such movement, no matter how subtle, must be abandoned. } Perception and feeling are mental. They’re tied up with the mind, that’s why perception and feeling are mental processes.” 

“But\marginnote{16.1} ma’am, how does someone attain the cessation of perception and feeling?” 

“A\marginnote{16.2} mendicant who is entering such an attainment does not think: ‘I will enter the cessation of perception and feeling’ or ‘I am entering the cessation of perception and feeling’ or ‘I have entered the cessation of perception and feeling.’ Rather, their mind has been previously developed so as to lead to such a state.”\footnote{Compare \href{https://suttacentral.net/mn43/en/sujato\#28.2}{MN 43:28.2}. } 

“But\marginnote{17.1} ma’am, which process ceases first for a mendicant who is entering the cessation of perception and feeling: physical, verbal, or mental?” 

“The\marginnote{17.2} verbal process ceases first, then physical, then mental.”\footnote{The verbal process ceases in the first absorption, the breath in the fourth absorption (\href{https://suttacentral.net/sn36.11/en/sujato\#2.19}{SN 36.11:2.19}), and feeling and perception in the attainment of cessation. } 

“But\marginnote{18.1} ma’am, how does someone emerge from the cessation of perception and feeling?” 

“A\marginnote{18.2} mendicant who is emerging from such an attainment does not think: ‘I will emerge from the cessation of perception and feeling’ or ‘I am emerging from the cessation of perception and feeling’ or ‘I have emerged from the cessation of perception and feeling.’ Rather, their mind has been previously developed so as to lead to such a state.” 

“But\marginnote{19.1} ma’am, which process arises first for a mendicant who is emerging from the cessation of perception and feeling: physical, verbal, or mental?” 

“The\marginnote{19.2} mental process arises first, then physical, then verbal.” 

“But\marginnote{20.1} ma’am, when a mendicant has emerged from the attainment of the cessation of perception and feeling, how many kinds of contact do they experience?” 

“They\marginnote{20.2} experience three kinds of contact: emptiness, signless, and undirected contacts.”\footnote{See the discussion on various “releases of the heart” at \href{https://suttacentral.net/mn43/en/sujato\#26.1}{MN 43:26.1} ff. Here, when emerging from cessation, the higher form of release is meant, as the meditator’s mind naturally inclines towards liberating insight. } 

“But\marginnote{21.1} ma’am, when a mendicant has emerged from the attainment of the cessation of perception and feeling, what does their mind slant, slope, and incline to?” 

“Their\marginnote{21.2} mind slants, slopes, and inclines to seclusion.”\footnote{Again, here the higher sense of “seclusion” is meant, namely Nibbana. } 

“But\marginnote{22.1} ma’am, how many feelings are there?” 

“There\marginnote{22.2} are three feelings: pleasant, painful, and neutral feeling.” 

“What\marginnote{23.1} are these three feelings?” 

“Anything\marginnote{23.2} felt physically or mentally as pleasant or enjoyable. This is pleasant feeling. Anything felt physically or mentally as painful or unpleasant. This is painful feeling. Anything felt physically or mentally as neither pleasurable nor painful. This is neutral feeling.” 

“What\marginnote{24.1} is pleasant and what is painful regarding each of the three feelings?” 

“Pleasant\marginnote{24.2} feeling is pleasant when it remains and painful when it perishes. Painful feeling is painful when it remains and pleasant when it perishes. Neutral feeling is pleasant in the presence of knowledge, and painful in the presence of ignorance.”\footnote{Knowledge is what distinguishes equanimity from indifference. } 

“What\marginnote{25.1} underlying tendencies underlie each of the three feelings?” 

“The\marginnote{25.2} underlying tendency for greed underlies pleasant feeling. The underlying tendency for repulsion underlies painful feeling. The underlying tendency for ignorance underlies neutral feeling.” 

“Do\marginnote{26.1} these underlying tendencies always underlie these feelings?” 

“No,\marginnote{26.2} they do not.”\footnote{The arahant still experiences feelings, but without any underlying tendencies. } 

“What\marginnote{27.1} should be given up in regard to each of these three feelings?” 

“The\marginnote{27.2} underlying tendency to greed should be given up when it comes to pleasant feeling. The underlying tendency to repulsion should be given up when it comes to painful feeling. The underlying tendency to ignorance should be given up when it comes to neutral feeling.” 

“Should\marginnote{28.1} these underlying tendencies be given up regarding all instances of these feelings?” 

“No,\marginnote{28.2} not in all instances. Take a mendicant who, quite secluded from sensual pleasures, secluded from unskillful qualities, enters and remains in the first absorption, which has the rapture and bliss born of seclusion, while placing the mind and keeping it connected. With this they give up greed, and the underlying tendency to greed does not lie within that.\footnote{The pleasure of \textit{\textsanskrit{jhāna}} arises from letting go and is not itself a source of attachment. } And take a mendicant who reflects: ‘Oh, when will I enter and remain in the same dimension that the noble ones enter and remain in today?’\footnote{Also at \href{https://suttacentral.net/mn137/en/sujato\#13.3}{MN 137:13.3}. That “dimension” (\textit{\textsanskrit{āyatana}}) is Nibbana (\href{https://suttacentral.net/ud8.1/en/sujato\#3.1}{Ud 8.1:3.1}, \href{https://suttacentral.net/sn35.117/en/sujato\#8.2}{SN 35.117:8.2}). } Nursing such a longing for the supreme liberations gives rise to sadness due to longing.\footnote{This is a kind of “suffering not of the flesh” (\href{https://suttacentral.net/mn10/en/sujato\#32.8}{MN 10:32.8}). } With this they give up repulsion, and the underlying tendency to repulsion does not lie within that. Take a mendicant who, giving up pleasure and pain, and ending former happiness and sadness, enters and remains in the fourth absorption, without pleasure or pain, with pure equanimity and mindfulness. With this they give up ignorance, and the underlying tendency to ignorance does not lie within that.” 

“But\marginnote{29.1} ma’am, what is the counterpart of pleasant feeling?” 

“Painful\marginnote{29.2} feeling.” 

“What\marginnote{29.3} is the counterpart of painful feeling?” 

“Pleasant\marginnote{29.4} feeling.” 

“What\marginnote{29.5} is the counterpart of neutral feeling?” 

“Ignorance.”\marginnote{29.6} 

“What\marginnote{29.7} is the counterpart of ignorance?” 

“Knowledge.”\marginnote{29.8} 

“What\marginnote{29.9} is the counterpart of knowledge?” 

“Freedom.”\marginnote{29.10} 

“What\marginnote{29.11} is the counterpart of freedom?” 

“Extinguishment.”\marginnote{29.12} 

“What\marginnote{29.13} is the counterpart of extinguishment?” 

“Your\marginnote{29.14} question goes too far, \textsanskrit{Visākha}. You couldn’t figure out the limit of questions. For extinguishment is the culmination, destination, and end of the spiritual life. If you wish, go to the Buddha and ask him this question. You should remember it in line with his answer.” 

And\marginnote{30.1} then the layman \textsanskrit{Visākha} approved and agreed with what the nun \textsanskrit{Dhammadinnā} said. He got up from his seat, bowed, and respectfully circled her, keeping her on his right. Then he went up to the Buddha, bowed, sat down to one side, and informed the Buddha of all they had discussed. 

When\marginnote{30.3} he had spoken, the Buddha said to him, “The nun \textsanskrit{Dhammadinnā} is astute, \textsanskrit{Visākha}, she has great wisdom. If you came to me and asked this question, I would answer it in exactly the same way as the nun \textsanskrit{Dhammadinnā}. That is what it means, and that’s how you should remember it.” 

That\marginnote{30.7} is what the Buddha said. Satisfied, the layman \textsanskrit{Visākha} approved what the Buddha said. 

%
\section*{{\suttatitleacronym MN 45}{\suttatitletranslation The Shorter Discourse on Taking Up Practices }{\suttatitleroot Cūḷadhammasamādānasutta}}
\addcontentsline{toc}{section}{\tocacronym{MN 45} \toctranslation{The Shorter Discourse on Taking Up Practices } \tocroot{Cūḷadhammasamādānasutta}}
\markboth{The Shorter Discourse on Taking Up Practices }{Cūḷadhammasamādānasutta}
\extramarks{MN 45}{MN 45}

\scevam{So\marginnote{1.1} I have heard. }At one time the Buddha was staying near \textsanskrit{Sāvatthī} in Jeta’s Grove, \textsanskrit{Anāthapiṇḍika}’s monastery. There the Buddha addressed the mendicants, “Mendicants!” 

“Venerable\marginnote{1.5} sir,” they replied. The Buddha said this: 

“Mendicants,\marginnote{2.1} there are these four ways of taking up practices. What four? There is a way of taking up practices that is pleasant now but results in future pain. There is a way of taking up practices that is painful now and results in future pain. There is a way of taking up practices that is painful now but results in future pleasure. There is a way of taking up practices that is pleasant now and results in future pleasure. 

And\marginnote{3.1} what is the way of taking up practices that is pleasant now but results in future pain? There are some ascetics and brahmins who have this doctrine and view: ‘There’s nothing wrong with sensual pleasures.’\footnote{This view is said to be a cause for lower rebirth at \href{https://suttacentral.net/an3.113/en/sujato\#1.5}{AN 3.113:1.5}, an addiction at \href{https://suttacentral.net/an3.156/en/sujato\#1.6}{AN 3.156:1.6}, and an extreme at \href{https://suttacentral.net/ud6.8/en/sujato\#5.4}{Ud 6.8:5.4}. } They throw themselves into sensual pleasures, cavorting with female wanderers with jeweled bands in their hair.\footnote{\textit{\textsanskrit{Moḷibaddha}} (“head-band”) appears as a bejeweled decoration for lay folk at \href{https://suttacentral.net/mil6.1.3/en/sujato\#1.3}{Mil 6.1.3:1.3} and \href{https://suttacentral.net/mil6.4.2/en/sujato\#4.2}{Mil 6.4.2:4.2}, and with a list of garbs for ascetics at \href{https://suttacentral.net/mil5.1.6/en/sujato\#12.5}{Mil 5.1.6:12.5}. The Sanskrit \textit{maulibandha} appears in a few later texts in the same sense. None of these sources clarify who these fancy wanderer ladies might be. The Chinese parallel at MA 174, however, does not say they are wanderers. It does seem odd to find such a decorative detail being worn by wanderers, so perhaps the Pali text is confused, and the idea of wanderers wearing headbands was later adopted by the Milinda. On the other hand, later texts such as the comedy \textsanskrit{Mattavilāsaprahasana} show that dissolute “renunciates” were not unknown. } They say, ‘What future danger do those ascetics and brahmins see in sensual pleasures that they speak of giving up sensual pleasures, and advocate the complete understanding of sensual pleasures? Pleasant is the touch of this female wanderer’s arm, tender, soft, and downy!’ And they throw themselves into sensual pleasures. When their body breaks up, after death, they’re reborn in a place of loss, a bad place, the underworld, hell. And there they feel painful, sharp, severe, acute feelings. They say, ‘This is that future danger that those ascetics and brahmins saw. For it is because of sensual pleasures that I’m feeling painful, sharp, severe, acute feelings.’ 

Suppose\marginnote{4.1} that in the last month of summer a camel’s foot creeper pod were to burst open\footnote{The “camel’s foot creeper” (\textit{\textsanskrit{māluvā}}) features in a similar simile in the Jain \textsanskrit{Sūyagaḍa} 1.3.2.10. } and a seed were to fall at the root of a sal tree. Then the deity haunting that sal tree would become apprehensive and nervous. But their friends and colleagues, relatives and kin—deities of the parks, forests, trees, and those who haunt the herbs, grass, and big trees—would come together to reassure them, ‘Do not fear, sir, do not fear! Hopefully that seed will be swallowed by a peacock, or eaten by a deer, or burnt by a forest fire, or picked up by a lumberjack, or eaten by termites, or it may not even be fertile.’\footnote{On “eaten by termites”, Bhikkhu \textsanskrit{Ñāṇatusita}, whose expertise in Pali is rivaled only by his knowledge of botany, says in his notes to Bhikkhu Bodhi’s translation, “Perhaps even the whole seed was eaten since the Maloo Creeper, a member of the legume family, has a hard woody seed. It is eaten by forest people in India, supposedly after removing the skin. Termites do not carry away seeds like ants would do, they eat things at the spot. Besides this, the seed is quite large. Thus \textit{\textsanskrit{udrabheyyuṁ}} is probably the right reading. It is supported by the commentary, which glosses it as \textit{\textsanskrit{khādeyyuṁ}}.” } But none of these things happened. And the seed was fertile, so that when the monsoon clouds soaked it with rain, it sprouted. And the creeper wound its tender, soft, and downy tendrils around that sal tree. Then the deity thought, ‘What future danger did my friends see when they said: “Do not fear, sir, do not fear! Hopefully that seed will be swallowed by a peacock, or eaten by a deer, or burnt by a forest fire, or picked up by a lumberjack, or eaten by termites, or it may not even be fertile.” Pleasant is the touch of this creeper’s tender, soft, and downy tendrils.’ Then the creeper enfolded the sal tree, made a canopy over it, draped a curtain around it, and split apart all the main branches. Then the deity thought, ‘This is the future danger that my friends saw! It’s because of that camel’s foot creeper seed that I’m feeling painful, sharp, severe, acute feelings.’ 

In\marginnote{4.22} the same way, there are some ascetics and brahmins who have this doctrine and view: ‘There’s nothing wrong with sensual pleasures’ … This is called the way of taking up practices that is pleasant now but results in future pain. 

And\marginnote{5.1} what is the way of taking up practices that is painful now and results in future pain? It’s when someone goes naked, ignoring conventions. They lick their hands, and don’t come or wait when called. They don’t consent to food brought to them, or food prepared on their behalf, or an invitation for a meal. They don’t receive anything from a pot or bowl; or from someone who keeps sheep, or who has a weapon or a shovel in their home; or where a couple is eating; or where there is a woman who is pregnant, breastfeeding, or who lives with a man; or where there’s a dog waiting or flies buzzing. They accept no fish or meat or beer or wine, and drink no fermented gruel. They go to just one house for alms, taking just one mouthful, or two houses and two mouthfuls, up to seven houses and seven mouthfuls. They feed on one saucer a day, two saucers a day, up to seven saucers a day. They eat once a day, once every second day, up to once a week, and so on, even up to once a fortnight. They live committed to the practice of eating food at set intervals. 

They\marginnote{5.7} eat herbs, millet, wild rice, poor rice, water lettuce, rice bran, scum from boiling rice, sesame flour, grass, or cow dung. They survive on forest roots and fruits, or eating fallen fruit. 

They\marginnote{5.8} wear robes of sunn hemp, mixed hemp, corpse-wrapping cloth, rags, lodh tree bark, antelope hide (whole or in strips), kusa grass, bark, wood-chips, human hair, horse-tail hair, or owls’ wings. They tear out their hair and beard, committed to this practice. They constantly stand, refusing seats. They squat, committed to persisting in the squatting position. They lie on a mat of thorns, making a mat of thorns their bed. They’re devoted to ritual bathing three times a day, including the evening. And so they live committed to practicing these various ways of mortifying and tormenting the body. When their body breaks up, after death, they’re reborn in a place of loss, a bad place, the underworld, hell. This is called the way of taking up practices that is painful now and results in future pain. 

And\marginnote{6.1} what is the way of taking up practices that is painful now but results in future pleasure? It’s when someone is ordinarily full of acute greed, hate, and delusion. They often feel the pain and sadness that greed, hate, and delusion bring. They lead the full and pure spiritual life in pain and sadness, weeping, with tearful faces.\footnote{Described as someone who “goes against the stream” at \href{https://suttacentral.net/an4.5/en/sujato\#2.2}{AN 4.5:2.2}, and as worthy of praise at \href{https://suttacentral.net/an5.5/en/sujato\#2.4}{AN 5.5:2.4}. } When their body breaks up, after death, they’re reborn in a good place, a heavenly realm. This is called the way of taking up practices that is painful now but results in future pleasure. 

And\marginnote{7.1} what is the way of taking up practices that is pleasant now and results in future pleasure? It’s when someone is not ordinarily full of acute greed, hate, and delusion. They rarely feel the pain and sadness that greed, hate, and delusion bring. Quite secluded from sensual pleasures, secluded from unskillful qualities, they enter and remain in the first absorption … second absorption … third absorption … fourth absorption. When their body breaks up, after death, they’re reborn in a good place, a heavenly realm. This is called the way of taking up practices that is pleasant now and results in future pleasure. These are the four ways of taking up practices.” 

That\marginnote{7.12} is what the Buddha said. Satisfied, the mendicants approved what the Buddha said. 

%
\section*{{\suttatitleacronym MN 46}{\suttatitletranslation The Great Discourse on Taking Up Practices }{\suttatitleroot Mahādhammasamādānasutta}}
\addcontentsline{toc}{section}{\tocacronym{MN 46} \toctranslation{The Great Discourse on Taking Up Practices } \tocroot{Mahādhammasamādānasutta}}
\markboth{The Great Discourse on Taking Up Practices }{Mahādhammasamādānasutta}
\extramarks{MN 46}{MN 46}

\scevam{So\marginnote{1.1} I have heard.\footnote{The Buddha delves into this topic in more detail in \href{https://suttacentral.net/mn114/en/sujato}{MN 114}. } }At one time the Buddha was staying near \textsanskrit{Sāvatthī} in Jeta’s Grove, \textsanskrit{Anāthapiṇḍika}’s monastery. There the Buddha addressed the mendicants, “Mendicants!” 

“Venerable\marginnote{1.5} sir,” they replied. The Buddha said this: 

“Mendicants,\marginnote{2.1} sentient beings typically have the wish, desire, and hope: ‘Oh, if only unlikable, undesirable, and disagreeable things would decrease, and likable, desirable, and agreeable things would increase!’ But exactly the opposite happens to them. What do you take to be the reason for this?” 

“Our\marginnote{2.5} teachings are rooted in the Buddha. He is our guide and our refuge. Sir, may the Buddha himself please clarify the meaning of this. The mendicants will listen and remember it.” 

“Well\marginnote{2.6} then, mendicants, listen and apply your mind well, I will speak.” 

“Yes,\marginnote{2.7} sir,” they replied. The Buddha said this: 

“Take\marginnote{3.1} an unlearned ordinary person who has not seen the noble ones, and is neither skilled nor trained in the teaching of the noble ones. They’ve not seen true persons, and are neither skilled nor trained in the teaching of the true persons. They don’t know what practices they should cultivate and foster, and what practices they shouldn’t cultivate and foster. So they cultivate and foster practices they shouldn’t, and don’t cultivate and foster practices they should. When they do so, unlikable, undesirable, and disagreeable things increase, and likable, desirable, and agreeable things decrease. Why is that? Because that’s what it’s like for someone who doesn’t know. 

But\marginnote{4.1} a learned noble disciple has seen the noble ones, and is skilled and trained in the teaching of the noble ones. They’ve seen true persons, and are skilled and trained in the teaching of the true persons. They know what practices they should cultivate and foster, and what practices they shouldn’t cultivate and foster. So they cultivate and foster practices they should, and don’t cultivate and foster practices they shouldn’t. When they do so, unlikable, undesirable, and disagreeable things decrease, and likable, desirable, and agreeable things increase. Why is that? Because that’s what it’s like for someone who knows. 

Mendicants,\marginnote{5.1} there are these four ways of taking up practices. What four? There is a way of taking up practices that is painful now and results in future pain. There is a way of taking up practices that is pleasant now but results in future pain. There is a way of taking up practices that is painful now but results in future pleasure. There is a way of taking up practices that is pleasant now and results in future pleasure. 

When\marginnote{6.1} it comes to the way of taking up practices that is painful now and results in future pain, an ignoramus, without knowing this, doesn’t truly understand: ‘This is the way of taking up practices that is painful now and results in future pain.’ So instead of avoiding that practice, they cultivate it. When they do so, unlikable, undesirable, and disagreeable things increase, and likable, desirable, and agreeable things decrease. Why is that? Because that’s what it’s like for someone who doesn’t know. 

When\marginnote{7.1} it comes to the way of taking up practices that is pleasant now and results in future pain, an ignoramus … cultivates it … and disagreeable things increase … 

When\marginnote{8.1} it comes to the way of taking up practices that is painful now and results in future pleasure, an ignoramus … doesn’t cultivate it … and disagreeable things increase … 

When\marginnote{9.1} it comes to the way of taking up practices that is pleasant now and results in future pleasure, an ignoramus … doesn’t cultivate it … and disagreeable things increase … Why is that? Because that’s what it’s like for someone who doesn’t know. 

When\marginnote{10.1} it comes to the way of taking up practices that is painful now and results in future pain, a wise person, knowing this, truly understands: ‘This is the way of taking up practices that is painful now and results in future pain.’ So instead of cultivating that practice, they avoid it. When they do so, unlikable, undesirable, and disagreeable things decrease, and likable, desirable, and agreeable things increase. Why is that? Because that’s what it’s like for someone who knows. 

When\marginnote{11.1} it comes to the way of taking up practices that is pleasant now and results in future pain, a wise person … doesn’t cultivate it … and agreeable things increase … 

When\marginnote{12.1} it comes to the way of taking up practices that is painful now and results in future pleasure, a wise person … cultivates it … and agreeable things increase … 

When\marginnote{13.1} it comes to the way of taking up practices that is pleasant now and results in future pleasure, a wise person, knowing this, truly understands: ‘This is the way of taking up practices that is pleasant now and results in future pleasure.’ So instead of avoiding that practice, they cultivate it. When they do so, unlikable, undesirable, and disagreeable things decrease, and likable, desirable, and agreeable things increase. Why is that? Because that’s what it’s like for someone who knows. 

And\marginnote{14.1} what is the way of taking up practices that is painful now and results in future pain? It’s when someone in pain and sadness kills living creatures, steals, and commits sexual misconduct. They use speech that’s false, divisive, harsh, or nonsensical. And they’re covetous, malicious, with wrong view. Because of these things they experience pain and sadness. And when their body breaks up, after death, they’re reborn in a place of loss, a bad place, the underworld, hell. This is called the way of taking up practices that is painful now and results in future pain. 

And\marginnote{15.1} what is the way of taking up practices that is pleasant now but results in future pain? It’s when someone with pleasure and happiness kills living creatures, steals, and commits sexual misconduct. They use speech that’s false, divisive, harsh, or nonsensical. And they’re covetous, malicious, with wrong view. Because of these things they experience pleasure and happiness. But when their body breaks up, after death, they’re reborn in a place of loss, a bad place, the underworld, hell. This is called the way of taking up practices that is pleasant now but results in future pain. 

And\marginnote{16.1} what is the way of taking up practices that is painful now but results in future pleasure? It’s when someone in pain and sadness doesn’t kill living creatures, steal, or commit sexual misconduct. They don’t use speech that’s false, divisive, harsh, or nonsensical. And they’re contented, kind-hearted, with right view. Because of these things they experience pain and sadness. But when their body breaks up, after death, they’re reborn in a good place, a heavenly realm. This is called the way of taking up practices that is painful now but results in future pleasure. 

And\marginnote{17.1} what is the way of taking up practices that is pleasant now and results in future pleasure? It’s when someone with pleasure and happiness doesn’t kill living creatures, steal, or commit sexual misconduct. They don’t use speech that’s false, divisive, harsh, or nonsensical. And they’re contented, kind-hearted, with right view. Because of these things they experience pleasure and happiness. And when their body breaks up, after death, they’re reborn in a good place, a heavenly realm. This is called the way of taking up practices that is pleasant now and results in future pleasure. These are the four ways of taking up practices. 

Suppose\marginnote{18.1} there was some bitter gourd mixed with poison. Then a man would come along who wants to live and doesn’t want to die, who wants to be happy and recoils from pain. They’d say to him: ‘Here, mister, this is bitter gourd mixed with poison. Drink it if you like. If you drink it, the color, aroma, and flavor will be unappetizing, and it will result in death or deadly pain.’ He wouldn’t reject it. Without reflection, he’d drink it. The color, aroma, and flavor would be unappetizing, and it would result in death or deadly pain. This is comparable to the way of taking up practices that is painful now and results in future pain, I say. 

Suppose\marginnote{19.1} there was a bronze goblet of beverage that had a nice color, aroma, and flavor. But it was mixed with poison. Then a man would come along who wants to live and doesn’t want to die, who wants to be happy and recoils from pain. They’d say to him: ‘Here, mister, this bronze goblet of beverage has a nice color, aroma, and flavor. But it’s mixed with poison. Drink it if you like. If you drink it, the color, aroma, and flavor will be appetizing, but it will result in death or deadly pain.’ He wouldn’t reject it. Without reflection, he’d drink it. The color, aroma, and flavor would be appetizing, but it would result in death or deadly pain. This is comparable to the way of taking up practices that is pleasant now and results in future pain, I say. 

Suppose\marginnote{20.1} there was some rancid urine mixed with different medicines.\footnote{Rancid urine was the worst of medicines, used when nothing better was available. Specifically, the Vinaya allows it as a purgative in the case of snakebite (\href{https://suttacentral.net/pli-tv-kd6/en/sujato\#14.6.4}{Kd 6:14.6.4}), and gives chebulic myrobalan soaked in cattle urine as a remedy for jaundice (\href{https://suttacentral.net/pli-tv-kd6/en/sujato\#14.7.9}{Kd 6:14.7.9}). Modern research has shown that urine, far from being a sterile cure-all, contains \href{https://suttacentral.nethttps://www.ncbi.nlm.nih.gov/pmc/articles/PMC3032614/}{dangerous} levels of \href{https://suttacentral.nethttps://www.ncbi.nlm.nih.gov/pmc/articles/PMC3957746/}{bacterial pathogens}, and has \href{https://suttacentral.nethttps://www.usatoday.com/story/news/factcheck/2023/03/29/fact-check-drinking-urine-carries-health-risks-no-sight-benefits/11561019002/}{no  medicinal value}. } Then a man with jaundice would come along. They’d say to him: ‘Here, mister, this is rancid urine mixed with different medicines. Drink it if you like. If you drink it, the color, aroma, and flavor will be unappetizing, but after drinking it you will be happy.’ He wouldn’t reject it. After appraisal, he’d drink it. The color, aroma, and flavor would be unappetizing, but after drinking it he would be happy. This is comparable to the way of taking up practices that is painful now and results in future pleasure, I say. 

Suppose\marginnote{21.1} there was some curds, honey, ghee, and molasses all mixed together. Then a man with bloody dysentery would come along. They’d say to him: ‘Here, mister, this is curds, honey, ghee, and molasses all mixed together. Drink it if you like. If you drink it, the color, aroma, and flavor will be appetizing, and after drinking it you will be happy.’ He wouldn’t reject it. After appraisal, he’d drink it. The color, aroma, and flavor would be appetizing, and after drinking it he would be happy. This is comparable to the way of taking up practices that is pleasant now and results in future pleasure, I say. 

It’s\marginnote{22.1} like the last month of the rainy season, in autumn,  when the heavens are clear and cloudless. And as the sun is rising to the firmament, having dispelled all the darkness of space, it shines and glows and radiates. In the same way, this way of taking up practices that is pleasant now and results in future pleasure dispels the doctrines of the various other ascetics and brahmins as it shines and glows and radiates.”\footnote{This text as well as \href{https://suttacentral.net/mn45/en/sujato\#7.9}{MN 45:7.9} answer the question of what practice results in future happiness, and so do not consider awakening. It is surprising to see such bold praise for such a modest teaching. The Chinese parallels to these suttas have the inverse situation, as they describe a practice leading to non-return (MA 174 at T i 712b27) or Nibbana (MA 175 at T i 712c25, T 83 at T i 902c10), yet do not have this passage of praise. } 

That\marginnote{22.3} is what the Buddha said. Satisfied, the mendicants approved what the Buddha said. 

%
\section*{{\suttatitleacronym MN 47}{\suttatitletranslation The Inquirer }{\suttatitleroot Vīmaṁsakasutta}}
\addcontentsline{toc}{section}{\tocacronym{MN 47} \toctranslation{The Inquirer } \tocroot{Vīmaṁsakasutta}}
\markboth{The Inquirer }{Vīmaṁsakasutta}
\extramarks{MN 47}{MN 47}

\scevam{So\marginnote{1.1} I have heard.\footnote{In this sutta, the Buddha shows his commitment to radical accountability and integrity. He does not merely pay lip service to the notion of open inquiry, but lays out a detailed and exacting procedure by which his students should test him. By extension, this approach may be applied to any spiritual teacher. } }At one time the Buddha was staying near \textsanskrit{Sāvatthī} in Jeta’s Grove, \textsanskrit{Anāthapiṇḍika}’s monastery. There the Buddha addressed the mendicants, “Mendicants!” 

“Venerable\marginnote{1.5} sir,” they replied. The Buddha said this: 

“Mendicants,\marginnote{2.1} a mendicant who is an inquirer, unable to comprehend another’s mind, should scrutinize the Realized One to see whether he is a fully awakened Buddha or not.”\footnote{The ability to comprehend the mind of another was considered an advanced meditative skill, not accessible even to all arahants. It allowed one to read the mind of another to the extent of one’s own realization. } 

“Our\marginnote{3.1} teachings are rooted in the Buddha. He is our guide and our refuge. Sir, may the Buddha himself please clarify the meaning of this. The mendicants will listen and remember it.” 

“Well\marginnote{3.2} then, mendicants, listen and apply your mind well, I will speak.” 

“Yes,\marginnote{3.3} sir,” they replied. The Buddha said this: 

“Mendicants,\marginnote{4.1} a mendicant who is an inquirer, unable to comprehend another’s mind, should scrutinize the Realized One for two things—things that can be seen and heard: ‘Can anything corrupt be seen or heard in the Realized One or not?’\footnote{This refers to bodily behavior that might be seen or speech that might be heard. } Scrutinizing him they find that nothing corrupt can be seen or heard in the Realized One. 

They\marginnote{5.1} scrutinize further: ‘Can anything mixed be seen or heard in the Realized One or not?’ Scrutinizing him they find that nothing mixed can be seen or heard in the Realized One. 

They\marginnote{6.1} scrutinize further: ‘Can anything clean be seen or heard in the Realized One or not?’ Scrutinizing him they find that clean things can be seen and heard in the Realized One. 

They\marginnote{7.1} scrutinize further: ‘Did the venerable attain this skillful state a long time ago, or just recently?’\footnote{It is easy to maintain a front for while, but defilements tend to reveal themselves over time. } Scrutinizing him they find that the venerable attained this skillful state a long time ago, not just recently. 

They\marginnote{8.1} scrutinize further: ‘Are certain dangers found in that venerable mendicant who has achieved fame and renown?’ For, mendicants, so long as a mendicant has not achieved fame and renown, certain dangers are not found in them. But when they achieve fame and renown, those dangers appear. Scrutinizing him they find that those dangers are not found in that venerable mendicant who has achieved fame and renown. 

They\marginnote{9.1} scrutinize further: ‘Is this venerable securely stilled or insecurely stilled?\footnote{Here \textit{uparata} means “stilled, ceased” rather than “restrained”. The commentary explains that one has become stilled (\textit{uparata}) due to the absence of perils (\textit{abhaya}). The test aims to distinguish the arahant from those for whom the perils of defilements and rebirth, though possibly suppressed for a time through the power of \textit{\textsanskrit{jhāna}}, are still present. } Is the reason they don’t indulge in sensual pleasures that they’re free of greed because greed has ended?’ Scrutinizing him they find that that venerable is securely stilled, not insecurely stilled. The reason they don’t indulge in sensual pleasures is that they’re free of greed because greed has ended. 

If\marginnote{10.1} others should ask that mendicant, ‘But what reason and evidence does the venerable have for saying this?’ Answering rightly, the mendicant should say, ‘Because, whether that venerable is staying in a community or alone, some people there are in a good state or a sorry state, some instruct a group, and some are seen among pleasures of the flesh while others remain unsullied. Yet that venerable doesn’t look down on them for that.\footnote{The commentary explains \textit{sugata} and \textit{duggata} here as “well practiced” and “poorly practiced”. But the normal sense is one who is in a “good state” or a “sorry state”, typically associated with rebirth, but also with, for example, the “sorry state” of poverty or reduced circumstances (\href{https://suttacentral.net/pli-tv-kd10/en/sujato\#2.4.4}{Kd 10:2.4.4}, \href{https://suttacentral.net/pli-tv-kd6/en/sujato\#15.5.10}{Kd 6:15.5.10}). Notice that there are people nearby even when he is living alone, so this must include lay folk as well as monastics. The point is that the Buddha does not despise the poor or show favors to the rich. } Also, I have heard and learned this in the presence of the Buddha: “I am securely stilled, not insecurely stilled. The reason I don’t indulge in sensual pleasures is that I’m free of greed because greed has ended.”’\footnote{These exact words are not spoken by the Buddha in suttas. However the early Abhidhamma text \textsanskrit{Puggalapaññatti} explains that “insecurely stilled” includes the seven trainees and ethical ordinary people, while the arahant is “securely stilled” (\href{https://suttacentral.net/pp2.1/en/sujato\#11.1}{Pp 2.1:11.1}). Since the \textsanskrit{Puggalapaññatti} consists mainly of lightly adapted quotes from the suttas, it is possible that this passage has been lost from the suttas as they stand. This would explain why these terms are invoked here as if they were known. } 

Next,\marginnote{11.1} they should ask the Realized One himself about this, ‘Can anything corrupt be seen or heard in the Realized One or not?’ The Realized One would answer, ‘Nothing corrupt can be seen or heard in the Realized One.’ 

‘Can\marginnote{12.1} anything mixed be seen or heard in the Realized One or not?’ The Realized One would answer, ‘Nothing mixed can be seen or heard in the Realized One.’ 

‘Can\marginnote{13.1} anything clean be seen or heard in the Realized One or not?’ The Realized One would answer, ‘Clean things can be seen and heard in the Realized One. I am the scope and the range of that, but I am not determined by that.’\footnote{\textit{\textsanskrit{Etaṁ}} is the accusative of relation, familiar from such phrases as \textit{\textsanskrit{taṁ} \textsanskrit{gotamaṁ} \textsanskrit{evaṁ} \textsanskrit{kalyāṇo} kittisaddo abbhuggato} (“a good word \emph{of that} Gotama has spread”). | \textit{Ahamasmi} (“I am”) asserts identity (not possession per the commentary). It is an \textsanskrit{Upaniṣadic} turn of phrase. | \textit{Tammaya} (“determined by that”, literally “made with that”) is found occasionally in the suttas (also \href{https://suttacentral.net/mn137/en/sujato\#20.3}{MN 137:20.3}, \href{https://suttacentral.net/mn113/en/sujato\#21.7}{MN 113:21.7}, \href{https://suttacentral.net/an3.40/en/sujato\#7.4}{AN 3.40:7.4}, \href{https://suttacentral.net/an6.104/en/sujato}{AN 6.104}, \href{https://suttacentral.net/snp4.9/en/sujato\#12.2}{Snp 4.9:12.2}). The Buddha’s usage echoes \textsanskrit{Bṛhadāraṇyaka} \textsanskrit{Upaniṣad} 4.4.5, which begins, “This self is indeed divinity, made with consciousness …” (\textit{sa \textsanskrit{vā} \textsanskrit{ayamātmā} brahma \textsanskrit{vijñānamayo}}). It goes on to list many other items of which the self is comprised (\textit{\textsanskrit{idaṁmayo}}), the point being that the self is ultimately created by its actions (\textit{\textsanskrit{yathākārī} \textsanskrit{yathācārī} \textsanskrit{tathā} bhavati}), a detail that emphasizes the causal sense of \textit{-maya}. Thus the term does not simply point to that which is the self, but rather that by which the self is shaped or determined. \textsanskrit{Yajñavalkya} also uses \textit{etanmaya} in the same sense in such passages as Śatapatha \textsanskrit{Brāhmaṇa} 10.4.2.30: “that (self) is made with hymns, sacrifices, breaths, and divinities” (\textit{candomaya stomamayaḥ \textsanskrit{prāṇamayo} \textsanskrit{devatāmayaḥ} sa etanmaya}) and \textsanskrit{Bṛhadāraṇyaka} \textsanskrit{Upaniṣad} 1.5.3: “that self is made with this: speech, mind, and breath” (\textit{etanmayo \textsanskrit{vā} ayam \textsanskrit{ātmā}: \textsanskrit{vāṅmayo} manomayaḥ \textsanskrit{prāṇamayaḥ}}). Thus the term emphasizes the Buddha’s freedom from conditions. He exemplifies good qualities but is not “determined” by their kammic force. } 

A\marginnote{14.1} disciple ought to approach a teacher who has such a doctrine in order to listen to the teaching. The teacher explains Dhamma with its higher and higher stages, with its better and better stages, with its dark and bright sides.\footnote{Cp. \href{https://suttacentral.net/an5.180/en/sujato}{AN 5.180}. } When they directly know a certain principle of those teachings, in accordance with how they were taught, the mendicant comes to a conclusion about the teachings. They have confidence in the teacher: ‘The Blessed One is a fully awakened Buddha! The teaching is well explained! The \textsanskrit{Saṅgha} is practicing well!’ 

If\marginnote{15.1} others should ask that mendicant, ‘But what reason and evidence does the venerable have for saying this?’ Answering rightly, the mendicant should say, ‘Reverends, I approached the Buddha to listen to the teaching. He explained Dhamma with its higher and higher stages, with its better and better stages, with its dark and bright sides. When I directly knew a certain principle of those teachings, in accordance with how I was taught, I came to a conclusion about the teachings. I had confidence in the Teacher: “The Blessed One is a fully awakened Buddha! The teaching is well explained! The \textsanskrit{Saṅgha} is practicing well!”’ 

When\marginnote{16.1} someone’s faith is settled, rooted, and planted in the Realized One in this manner, with these words and phrases, it’s said to be grounded faith that’s based on evidence.\footnote{In this sentence the word \textit{\textsanskrit{ākāra}} is used in two distinct senses. In the stock phrase “in this manner, with these words and phrases” (\textit{imehi \textsanskrit{ākārehi} imehi padehi imehi \textsanskrit{byañjanehi}}) it means “manner” of exposition (\href{https://suttacentral.net/mn18/en/sujato\#20.17}{MN 18:20.17}, \href{https://suttacentral.net/sn35.116/en/sujato\#10.14}{SN 35.116:10.14}, \href{https://suttacentral.net/an10.115/en/sujato\#21.2}{AN 10.115:21.2}). In the phrase “grounded faith” (\textit{\textsanskrit{ākāravatī} \textsanskrit{saddhā}}) it means a “ground” or “reason” on which faith is based. } It is strong, and cannot be shifted by any ascetic or brahmin or god or \textsanskrit{Māra} or divinity or by anyone in the world.\footnote{Various editions (MS, PTS, BJT) followed by translators (Bodhi, Horner) connect \textit{\textsanskrit{daḷhā}} with the foregoing \textit{\textsanskrit{dassanamūlikā}}. But that foregoing text is missing in the parallels for this passage (\href{https://suttacentral.net/dn27/en/sujato\#9.3}{DN 27:9.3}, \href{https://suttacentral.net/sn48.42/en/sujato\#6.4}{SN 48.42:6.4}, \href{https://suttacentral.net/iti83/en/sujato\#10.1}{Iti 83:10.1}). \textit{\textsanskrit{Daḷhā}} is connected rather with \textit{\textsanskrit{asaṁhāriyā}}, which is in any case semantically closer than \textit{\textsanskrit{dassanamūlikā}}. } That is how there is legitimate scrutiny of the Realized One, and that is how the Realized One is legitimately well-scrutinized.” 

That\marginnote{16.5} is what the Buddha said. Satisfied, the mendicants approved what the Buddha said. 

%
\section*{{\suttatitleacronym MN 48}{\suttatitletranslation The Mendicants of Kosambī }{\suttatitleroot Kosambiyasutta}}
\addcontentsline{toc}{section}{\tocacronym{MN 48} \toctranslation{The Mendicants of Kosambī } \tocroot{Kosambiyasutta}}
\markboth{The Mendicants of Kosambī }{Kosambiyasutta}
\extramarks{MN 48}{MN 48}

\scevam{So\marginnote{1.1} I have heard. }At one time the Buddha was staying near \textsanskrit{Kosambī}, in Ghosita’s Monastery. 

Now\marginnote{2.1} at that time the mendicants of \textsanskrit{Kosambī} were arguing, quarreling, and disputing, continually wounding each other with barbed words.\footnote{This is in reference to the notorious quarrel at \textsanskrit{Kosambī}, also referred to at \href{https://suttacentral.net/mn128/en/sujato\#2.1}{MN 128:2.1}. The events are detailed in the Vinaya at \href{https://suttacentral.net/pli-tv-kd10/en/sujato}{Kd 10}. } They couldn’t persuade each other or be persuaded, nor could they convince each other or be convinced. 

Then\marginnote{3.1} a mendicant went up to the Buddha, bowed, sat down to one side, and told him what was happening. 

So\marginnote{4.1} the Buddha addressed one of the monks, “Please, monk, in my name tell those mendicants that the teacher summons them.” 

“Yes,\marginnote{4.4} sir,” that monk replied. He went to those monks and said, “Venerables, the teacher summons you.” 

“Yes,\marginnote{4.6} reverend,” those monks replied. They went to the Buddha, bowed, and sat down to one side. The Buddha said to them, 

“Is\marginnote{4.7} it really true, mendicants, that you have been arguing, quarreling, and disputing, continually wounding each other with barbed words? And that you can’t persuade each other or be persuaded, nor can you convince each other or be convinced?” 

“Yes,\marginnote{4.9} sir,” they said. 

“What\marginnote{5.1} do you think, mendicants? When you’re arguing, quarreling, and disputing, continually wounding each other with barbed words, are you treating your spiritual companions with kindness by way of body, speech, and mind, both in public and in private?” 

“No,\marginnote{5.3} sir.” 

“So\marginnote{5.4} it seems that when you’re arguing you are not treating each other with kindness. So what exactly do you know and see, you futile men, that you behave in such a way? This will be for your lasting harm and suffering.” 

Then\marginnote{6.1} the Buddha said to the mendicants: 

“Mendicants,\marginnote{6.2} these six warm-hearted qualities make for fondness and respect, conducing to inclusion, harmony, and unity, without quarreling.\footnote{Also taught at \href{https://suttacentral.net/mn104/en/sujato\#21.1}{MN 104:21.1}, \href{https://suttacentral.net/an6.11/en/sujato\#1.1}{AN 6.11:1.1}, and \href{https://suttacentral.net/an6.12/en/sujato\#1.1}{AN 6.12:1.1}, and collected at \href{https://suttacentral.net/dn33/en/sujato\#2.2.37}{DN 33:2.2.37} and \href{https://suttacentral.net/dn34/en/sujato\#1.7.3}{DN 34:1.7.3}. } What six? Firstly, a mendicant consistently treats their spiritual companions with bodily kindness, both in public and in private. This warm-hearted quality makes for fondness and respect, conducing to inclusion, harmony, and unity, without quarreling. 

Furthermore,\marginnote{6.6} a mendicant consistently treats their spiritual companions with verbal kindness … 

Furthermore,\marginnote{6.8} a mendicant consistently treats their spiritual companions with mental kindness … 

Furthermore,\marginnote{6.10} a mendicant shares without reservation any material things they have gained by legitimate means, even the food placed in the alms-bowl, using them in common with their ethical spiritual companions … 

Furthermore,\marginnote{6.12} a mendicant lives according to the precepts shared with their spiritual companions, both in public and in private. Those precepts are intact, impeccable, spotless, and unmarred, liberating, praised by sensible people, not mistaken, and leading to immersion. … 

Furthermore,\marginnote{6.14} a mendicant lives according to the view shared with their spiritual companions, both in public and in private. That view is noble and emancipating, and delivers one who practices it to the complete ending of suffering. This warm-hearted quality makes for fondness and respect, conducing to inclusion, harmony, and unity, without quarreling. 

These\marginnote{6.16} six warm-hearted qualities make for fondness and respect, conducing to inclusion, harmony, and unity, without quarreling. 

Of\marginnote{7.1} these six warm-hearted qualities, the chief is the view that is noble and emancipating, and delivers one who practices it to the complete ending of suffering. It holds and binds everything together. It’s like a bungalow. The roof-peak is the chief point, which holds and binds everything together. In the same way, of these six warm-hearted qualities, the chief is the view that is noble and emancipating, and delivers one who practices it to the complete ending of suffering. It holds and binds everything together. 

And\marginnote{8.1} how does the view that is noble and emancipating lead one who practices it to the complete ending of suffering? It’s when a mendicant has gone to a wilderness, or to the root of a tree, or to an empty hut, and reflects like this, ‘Is there anything that I’m overcome with internally and haven’t given up, because of which I might not accurately know and see?’ If a mendicant is overcome with sensual desire, it’s their mind that’s overcome. If a mendicant is overcome with ill will, dullness and drowsiness, restlessness and remorse, doubt, pursuing speculation about this world, pursuing speculation about the next world, or arguing, quarreling, and disputing, continually wounding others with barbed words, it’s their mind that’s overcome. They understand, ‘There is nothing that I’m overcome with internally and haven’t given up, because of which I might not accurately know and see. My mind is properly disposed for awakening to the truths.’ This is the first knowledge they have achieved that is noble and transcendent, and is not shared with ordinary people. 

Furthermore,\marginnote{9.1} a noble disciple reflects, ‘When I develop, cultivate, and make much of this view, do I personally gain serenity and quenching?’ They understand, ‘When I develop, cultivate, and make much of this view, I personally gain serenity and quenching.’ This is their second knowledge … 

Furthermore,\marginnote{10.1} a noble disciple reflects, ‘Are there any ascetics or brahmins outside of the Buddhist community who have the same kind of view that I have?’ They understand, ‘There are no ascetics or brahmins outside of the Buddhist community who have the same kind of view that I have.’ This is their third knowledge … 

Furthermore,\marginnote{11.1} a noble disciple reflects, ‘Do I have the same nature as a person accomplished in view?’ And what, mendicants, is the nature of a person accomplished in view? This is the nature of a person accomplished in view. Though they may fall into a kind of offense for which resolution is possible, they quickly disclose, clarify, and reveal it to the Teacher or a sensible spiritual companion.\footnote{Vinaya offences fall into a number of classes, all of which must be confessed by a guilty monastic to their fellow monastics. Some are resolved upon confession, others by undergoing a procedure of relinquishment or temporary suspension. However, the most serious offences—sexual intercourse, murder, stealing, and lying about spiritual attainments—entail immediate and permanent expulsion. Offences are further distinguished by intention, as some offences may be transgressed without ill intent or even unknowingly (for example, eating at the wrong time). A stream-enterer cannot commit an expulsion offence, but they may commit one of the other offences without ill intent. } And having revealed it they restrain themselves in the future. Suppose there was a little baby boy. If he puts his hand or foot on a burning coal, he quickly pulls it back. In the same way, this is the nature of a person accomplished in view. Though they may still fall into a kind of offense for which resolution is possible, they quickly reveal it to the Teacher or a sensible spiritual companion. And having revealed it they restrain themselves in the future. They understand, ‘I have the same nature as a person accomplished in view.’ This is their fourth knowledge … 

Furthermore,\marginnote{12.1} a noble disciple reflects, ‘Do I have the same nature as a person accomplished in view?’ And what, mendicants, is the nature of a person accomplished in view? This is the nature of a person accomplished in view. Though they might manage a diverse spectrum of duties for their spiritual companions, they still feel a keen regard for the training in higher ethics, higher mind, and higher wisdom. Suppose there was a cow with a baby calf. She watches her calf devotedly as she grazes.\footnote{The verb \textit{apacinati} is an irregular form of \textit{\textsanskrit{apacāyati}}, to “watch” but also to “revere”. } In the same way, this is the nature of a person accomplished in view. Though they might manage a diverse spectrum of duties for their spiritual companions, they still feel a keen regard for the training in higher ethics, higher mind, and higher wisdom. They understand, ‘I have the same nature as a person accomplished in view.’ This is their fifth knowledge … 

Furthermore,\marginnote{13.1} a noble disciple reflects, ‘Do I have the same strength as a person accomplished in view?’ And what, mendicants, is the strength of a person accomplished in view? The strength of a person accomplished in view is that, when the teaching and training proclaimed by the Realized One are being taught, they pay attention, apply the mind, concentrate wholeheartedly, and actively listen to the teaching. They understand, ‘I have the same strength as a person accomplished in view.’ This is their sixth knowledge … 

Furthermore,\marginnote{14.1} a noble disciple reflects, ‘Do I have the same strength as a person accomplished in view?’ And what, mendicants, is the strength of a person accomplished in view? The strength of a person accomplished in view is that, when the teaching and training proclaimed by the Realized One are being taught, they find inspiration in the meaning and the teaching, and find joy connected with the teaching. They understand, ‘I have the same strength as a person accomplished in view.’ This is the seventh knowledge they have achieved that is noble and transcendent, and is not shared with ordinary people. 

When\marginnote{15.1} a noble disciple has these seven factors, they have properly investigated their nature through the realization of the fruit of stream-entry.\footnote{Normally \textit{\textsanskrit{sotāpattiphalasacchikiriyāya}} is dative, “\emph{for} the realization of the fruit of stream-entry”, but that cannot be the case here as they are already a stream-enterer. The commentary reads it as instrumental. } A noble disciple with these seven factors has the fruit of stream-entry.” 

That\marginnote{15.3} is what the Buddha said. Satisfied, the mendicants approved what the Buddha said. 

%
\section*{{\suttatitleacronym MN 49}{\suttatitletranslation On the Invitation of Divinity }{\suttatitleroot Brahmanimantanikasutta}}
\addcontentsline{toc}{section}{\tocacronym{MN 49} \toctranslation{On the Invitation of Divinity } \tocroot{Brahmanimantanikasutta}}
\markboth{On the Invitation of Divinity }{Brahmanimantanikasutta}
\extramarks{MN 49}{MN 49}

\scevam{So\marginnote{1.1} I have heard.\footnote{This sutta is a unique fusion of high philosophy and cosmic drama, positing an unthinkable alliance between what in Christian terms would be called God and the Devil. Nonetheless, due to a plethora of textual difficulties and uncertain readings, caution in interpretation would be wise. } }At one time the Buddha was staying near \textsanskrit{Sāvatthī} in Jeta’s Grove, \textsanskrit{Anāthapiṇḍika}’s monastery. There the Buddha addressed the mendicants, “Mendicants!” 

“Venerable\marginnote{1.5} sir,” they replied. The Buddha said this: 

“This\marginnote{2.1} one time, mendicants, I was staying near \textsanskrit{Ukkaṭṭhā}, in the Subhaga Forest at the root of a magnificent sal tree.\footnote{The same setting as \href{https://suttacentral.net/mn1/en/sujato}{MN 1}, with which this sutta shares some themes. Both critique the sophisticated philosophy of the Kosalan brahmins, and hence are set in the town of their leader \textsanskrit{Pokkharasāti} (\href{https://suttacentral.net/dn3/en/sujato\#1.2.1}{DN 3:1.2.1}). } Now at that time Baka the Divinity had the following harmful misconception:\footnote{Baka maintains the same wrong view at \href{https://suttacentral.net/sn6.4/en/sujato}{SN 6.4}, where his past lives are revealed. In neither sutta is he said to go for refuge, although SN 6.4 does end with effusive praise of the Buddha. The word \textit{baka} means “stork” or “crane”. To search for a high divinity of this name in Brahmanism is to be disappointed, for instead we find a man-eating demon (\textit{\textsanskrit{rakṣasa}}) in bird form whose fate is to be slain by the hero \textsanskrit{Bhīma} (or \textsanskrit{Kṛṣṇa}). Pali stories (\href{https://suttacentral.net/ja38/en/sujato}{Ja 38}, \href{https://suttacentral.net/ja236/en/sujato}{Ja 236}) tell of how the stork dozes peacefully as if meditating by the water, while in reality he is trying to fool fish into approaching so he can snatch them up. A cunning, large, white, high-flying, predatory bird who fakes meditation is a fitting image for the antagonist of this sutta. } ‘This is permanent, this is everlasting, this is eternal, this is whole, this is not liable to pass away. For this is where there’s no being born, growing old, dying, passing away, or being reborn. And there’s no other escape beyond this.’\footnote{The use of the impersonal pronoun (\textit{\textsanskrit{idaṁ}}) to refer to the self as divinity is a characteristic \textsanskrit{Upaniṣadic} idiom: “you are that” (\textit{tat tvam asi}, \textsanskrit{Chāndogya} \textsanskrit{Upaniṣad} 6.8.7); “that self is divinity” (\textit{ayam \textsanskrit{ātmā} brahma}, \textsanskrit{Māṇḍūkya} \textsanskrit{Upaniṣad} 1.2); “I am that” (\textit{so’ham asmi}, \textsanskrit{Īśa} \textsanskrit{Upaniṣad} 16); “this is that self hidden in all” (\textit{\textsanskrit{eṣa} ta \textsanskrit{ātmā} \textsanskrit{sarvāntaraḥ}}, \textsanskrit{Bṛhadāraṇyaka} \textsanskrit{Upaniṣad} 3.4.1, etc.); “this, verily, is that” (\textit{etad vai tat}, \textsanskrit{Kaṭha} \textsanskrit{Upaniṣad} 2.1.5). | For \textit{\textsanskrit{idaṁ} \textsanskrit{niccaṁ}} see \textit{\textsanskrit{eṣa} nityo} (\textsanskrit{Bṛhadāraṇyaka} \textsanskrit{Upaniṣad} 4.4.23); for \textit{\textsanskrit{idaṁ} \textsanskrit{dhuvaṁ}} see \textit{etad \textsanskrit{apramayaṁ} dhruvam} (4.4.20). } 

Then\marginnote{3.1} I knew what Baka the Divinity was thinking. As easily as a strong person would extend or contract their arm, I vanished from the Subhaga Forest and reappeared in that realm of divinity. 

Baka\marginnote{3.3} saw me coming off in the distance and said, ‘Come, good sir! Welcome, good sir!\footnote{The vocative \textit{\textsanskrit{mārisa}} (possibly equivalent to \textit{\textsanskrit{mādisa}}, “one like me”), is used by the gods in Buddhist texts. } It’s been a long time since you took the opportunity to come here. For this is permanent, this is everlasting, this is eternal, this is complete, this is not liable to pass away. For this is where there’s no being born, growing old, dying, passing away, or being reborn. And there’s no other escape beyond this.’ 

When\marginnote{4.1} he had spoken, I said to him, ‘Alas, Baka the Divinity is lost in ignorance! Alas, Baka the Divinity is lost in ignorance! Because what is actually impermanent, not lasting, transient, incomplete, and liable to pass away, he says is permanent, everlasting, eternal, complete, and not liable to pass away. And where there is being born, growing old, dying, passing away, and being reborn, he says that there’s no being born, growing old, dying, passing away, or being reborn. And although there is another escape beyond this, he says that there’s no other escape beyond this.’ 

Then\marginnote{5.1} \textsanskrit{Māra} the Wicked took possession of a member of the retinue of Divinity and said this to me, ‘Mendicant, mendicant! Don’t attack this one! Don’t attack this one! For this is the Divinity, the Great Divinity, the Vanquisher, the Unvanquished, the Universal Seer, the Wielder of Power, God Almighty, the Maker, the Creator, the First, the Begetter, the Controller, the Father of those who have been born and those yet to be born.\footnote{\textsanskrit{Māra} adopts the boast of \textsanskrit{Brahmā} (\href{https://suttacentral.net/dn1/en/sujato\#2.5.2}{DN 1:2.5.2}). | It is rare for the Buddha to be addressed as “mendicant” (\textit{bhikkhu}) and it is probably meant in a slighting sense. } 

There\marginnote{5.3} have been ascetics and brahmins before you, mendicant, who criticized and loathed earth, water, fire, air, creatures, gods, the Progenitor, and the Divinity.\footnote{This is a summary of the items in \href{https://suttacentral.net/mn1/en/sujato}{MN 1}. | The “ascetics and brahmins” referred to here would include the ancient practitioners of movements such as Jainism, whose rejection of Vedism predates the Buddha. } When their bodies broke up and their breath was cut off they were reborn in a lower realm.\footnote{“With breath cut off” (\textit{\textsanskrit{pāṇupacchedā}}) is a unique idiom for death in Pali, but compare \textsanskrit{Bṛhadāraṇyaka} \textsanskrit{Upaniṣad} 1.5.14: \textit{\textsanskrit{prāṇaṁ} na \textsanskrit{vicchindyāt}}; Atharvaveda 19.58.1c: \textit{\textsanskrit{prāṇo}’chinno}. } 

There\marginnote{5.5} have been ascetics and brahmins before you, mendicant, who praised and approved earth, water, fire, air, creatures, gods, the Progenitor, and the Divinity.\footnote{This would be the ancient sages revered in the Vedic tradition. } When their bodies broke up and their breath was cut off they were reborn in a higher realm. 

So,\marginnote{5.7} mendicant, I tell you this: please, good sir, do exactly what the Divinity says. Don’t go beyond the word of the Divinity. If you do, then you’ll end up like a person who, when approached by Lady Luck, would ward her off with a staff; or who, as they are falling over a cliff, would lose grip of the ground with their hands and feet.\footnote{“Lady Luck” is Siri, later famed as goddess of fortune and prosperity. The same idiom recurs at \href{https://suttacentral.net/thag8.3/en/sujato\#2.1}{Thag 8.3:2.1}. As to why someone would ward her off, Siri (Sanskrit \textit{\textsanskrit{śrī}}) is identified with \textsanskrit{Lakṣmī}, and it is under that name that she makes what may well be her first appearance. Atharva Veda 7.115 opens with, “Fly away wicked \textsanskrit{Lakṣmī}, vanish, fly hence”. Apparently 101 \textsanskrit{Lakṣmīs} attach to a man when born, and the spell chases away the wicked (\textit{\textsanskrit{pāpi}}) while keeping the good (\textit{\textsanskrit{puṇya}}). In this case the Pali text does not draw directly from the Veda, as there is no close verbal parallel, and the Pali presents as irrational an action that makes good sense in the context of the Vedic passage. Rather, it would seem, the Pali draws from a more general cultural awareness. | For “falling over a cliff” see \href{https://suttacentral.net/dn12/en/sujato\#78.2}{DN 12:78.2}. } Please, dear sir, do exactly what the Divinity says. Don’t go beyond the word of the Divinity. Do you not see the assembly of the Divinity gathered here?’ 

And\marginnote{5.12} so \textsanskrit{Māra} the Wicked presented the assembly of the Divinity to me. 

When\marginnote{6.1} he had spoken, I said to \textsanskrit{Māra}, ‘I know you, Wicked One. Do not think, “He does not know me.” You are \textsanskrit{Māra} the Wicked. And the Divinity, the Divinity’s assembly, and the retinue of Divinity have all fallen into your hands; they’re under your sway.\footnote{Normally \textsanskrit{Māra} is believed to hold sway over all the sensual realms, while the \textit{\textsanskrit{jhānas}} (and their corresponding planes of rebirth) are beyond him (\href{https://suttacentral.net/mn25/en/sujato\#12.1}{MN 25:12.1}). Here his reach goes even further, as even the \textsanskrit{Brahmā} realms, while freed from sensuality, are not freed from the attachment to continued existence. } And you think, “Maybe this one, too, has fallen into my hands; maybe he’s under my sway!” But I haven’t fallen into your hands; I’m not under your sway.’ 

When\marginnote{7.1} I had spoken, Baka the Divinity said to me, ‘But, good sir, what I say is permanent, everlasting, eternal, complete, and not liable to pass away is in fact permanent, everlasting, eternal, complete, and not liable to pass away. And where I say there’s no being born, growing old, dying, passing away, or being reborn there is in fact\footnote{One of \textsanskrit{Māra}’s talents is making people believe that his malign ideas are in fact their own. } no being born, growing old, dying, passing away, or being reborn. And when I say there’s no other escape beyond this there is in fact no other escape beyond this. There have been ascetics and brahmins in the world before you, mendicant, whose deeds of fervent mortification lasted as long as your entire life.\footnote{Note the use of \textit{\textsanskrit{kasiṇa}} in the sense “entire”. } When there was another escape beyond this they knew it, and when there was no other escape beyond this, they knew it. So, mendicant, I tell you this: you will never find another escape beyond this, and you will eventually get weary and frustrated. If you attach to earth, you will lie close to me, in my domain, subject to my will, and expendable. If you attach to water … fire … air … creatures … gods … the Progenitor … the Divinity, you will lie close to me, in my domain, subject to my will, and expendable.’ 

‘Divinity,\marginnote{8.1} I too know that if I attach to earth, I will lie close to you, in your domain, subject to your will, and expendable. If I attach to water … fire … air … creatures … gods … the Progenitor … the Divinity, I will lie close to you, in your domain, subject to your will, and expendable. And in addition, Divinity, I understand your range and your light:\footnote{The measuring of a \textsanskrit{Brahmā} by their “light” (\textit{juti}) shows the close connection between divinity and the stars. } “That’s how powerful is Baka the Divinity, how illustrious and mighty.”’ 

‘But\marginnote{8.11} in what way do you understand my range and my light?’ 

\begin{verse}%
‘A\marginnote{9.1} galaxy extends a thousand times as far \\
as the moon and sun revolve \\
and the shining ones light up the quarters. \\
And there you wield your power. 

You\marginnote{9.5} know the high and low, \\
the passionate and dispassionate, \\
and the coming and going of sentient beings \\
from this realm to another. 

%
\end{verse}

That’s\marginnote{9.9} how I understand your range and your light. 

But\marginnote{10.1} there are three other realms that you don’t know or see,\footnote{MS reads \textit{\textsanskrit{añño} \textsanskrit{kāyo}} (“another realm”), but the PTS and BJT reading \textit{\textsanskrit{aññe} tayo \textsanskrit{kāyā}} is supported by the Chinese parallel (MA 78 at T i 548a28). Possibly this change was necessitated by the addition of a fourth realm below. } but which I know and see.\footnote{PTS and BJT have \textit{\textsanskrit{tyāhaṁ}} (= \textit{te \textsanskrit{ahaṁ}}), consistent with their plural reading, as opposed to MS’s singular \textit{\textsanskrit{tamahaṁ}}. This shows that this variation, which is not decided by the commentary, was a deliberate editorial choice. } There is the realm named after the gods of streaming radiance. You passed away from there and were reborn here.\footnote{Here \textit{\textsanskrit{ābhassarā}} is masculine plural, hence not an adjective of \textit{\textsanskrit{kāyo}} but a reference to the “gods” of that realm. MS is inconsistent in this point, for below we find \textit{\textsanskrit{subhakiṇho}} and\textit{vehapphalo} as singular adjectives, where PTS and BJT consistently use the plural form. } You’ve dwelt here so long that you’ve forgotten about that, so you don’t know it or see it.\footnote{\textsanskrit{Brahmā} also forgets his origins at \href{https://suttacentral.net/dn1/en/sujato\#2.2.1}{DN 1:2.2.1}. } But I know it and see it. So Divinity, I am not your equal in knowledge, let alone your inferior. Rather, I know more than you. 

There\marginnote{10.8} is the realm named after the gods of universal beauty … There is the realm named after the gods of abundant fruit, which you don’t know or see.\footnote{MS adds a fourth class here, \textit{\textsanskrit{abhibhū}} (“the Vanquisher”), which is not supported by PTS or BJT, or the Chinese parallel. } But I know it and see it. So Divinity, I am not your equal in knowledge, let alone your inferior. Rather, I know more than you. 

Since\marginnote{11.1} directly knowing earth as earth, and since directly knowing that which does not fall within the scope of experience characterized by earth, I have not become earth, I have not become in earth, I have not become as earth, I have not become one who thinks ‘earth is mine’, I have not affirmed earth.\footnote{This difficult passage directly echoes \href{https://suttacentral.net/mn1/en/sujato\#3.3}{MN 1:3.3}. | \textit{\textsanskrit{Anubhūta}} has its normal sense of “experiences”, “undergoes” (\href{https://suttacentral.net/thig10.1/en/sujato\#8.2}{Thig 10.1:8.2}, more commonly \textit{\textsanskrit{paccanubhūta}}, eg. \href{https://suttacentral.net/mn79/en/sujato\#8.1}{MN 79:8.1}, \href{https://suttacentral.net/sn15.1/en/sujato\#1.14}{SN 15.1:1.14}). Commentary has \textit{\textsanskrit{appattaṁ}}, “not attained” (i.e. not within the scope of absorption). | That which “falls within the scope of experience characterized by earth” (\textit{\textsanskrit{yāvatā} \textsanskrit{pathaviyā} pathavattena \textsanskrit{ananubhūtaṁ}}) is sense experience and the four absorptions; that which does not fall in such a scope are the formless attainments and especially Nibbana. | For \textit{\textsanskrit{nāhosiṁ}} (prefer over MS \textit{\textsanskrit{nāpahosiṁ}}) the commentary glosses “grasps”. The sense is that he does not identify. | For \textit{abhivadati}, compare \textsanskrit{Bṛhadāraṇyaka} \textsanskrit{Upaniṣad} 2.4.14, which says that so long as there is the appearance of duality, one sees, hears, smells, “speaks about” (\textit{abhivadati}), “conceives” (\textit{manute}), and “cognizes” (\textit{\textsanskrit{vijānāti}}) particulars. Notice that this employs \textit{abhivadati} alongside \textit{manute}, just as \textit{abhivadati} in the Pali appears alongside where \textit{\textsanskrit{maññati}} appears in \href{https://suttacentral.net/mn1/en/sujato\#3.3}{MN 1:3.3}. } So Divinity, I am not your equal in knowledge, let alone your inferior. Rather, I know more than you. 

Since\marginnote{12.1} directly knowing water … fire … air … creatures … gods … the Progenitor … the Divinity … the gods of streaming radiance … the gods of universal beauty … the gods of abundant fruit … the Vanquisher … Since directly knowing all as all, and since directly knowing that which does not fall within the scope of experience characterized by all, I have not become all, I have not become in all, I have not become as all, I have not become one who thinks ‘all is mine’, I have not affirmed all.\footnote{This whole passage seems designed to culminate with “experience characterized by all” (\textit{sabbassa sabbattena \textsanskrit{ananubhūtaṁ}}) in answer to \textsanskrit{Bṛhadāraṇyaka} \textsanskrit{Upaniṣad} 2.5.19, which says that “this self that experiences all is divinity” (\textit{ayam \textsanskrit{ātmā} brahma \textsanskrit{sarvānubhūḥ}}). In surpassing even the “experience of all” the Buddha unequivocally asserts his superiority to the \textsanskrit{Upaniṣadic} teaching. } So Divinity, I am not your equal in knowledge, let alone your inferior. Rather, I know more than you.’ 

‘Well,\marginnote{24.1} good sir, if you have directly known that which does not fall within the scope of experience characterized by all, may that not be vacuous and hollow for you!\footnote{The close \textit{-ti} in MS edition indicates the end of Baka’s speech, but this is not found in PTS or BJT editions. If the close \textit{-ti} is accepted here, then the next paragraph is spoken by the Buddha and it should also end with close \textit{-ti}, but no edition has this. We would also expect the Buddha to use a vocative, but there is none. The only reading that is both coherent and attested is that Baka’s speech begins here and ends when he says he will vanish. This is supported by the Chinese parallel, which while lacking an exact equivalent, attributes a similar claim to \textsanskrit{Brahmā} (MA 78 at T i 548b11). } 

\begin{verse}%
Consciousness\marginnote{25.1} where nothing appears,\footnote{These two lines are a fragment of verse found in full at \href{https://suttacentral.net/dn11/en/sujato\#85.18}{DN 11:85.18}. Baka is asserting that what the Buddha has described is none other than the domain of “infinite consciousness”, which is one of the highest attainments attributed to Brahmanical sages. | “Where nothing appears” (\textit{\textsanskrit{anidassanaṁ}}) here is a synonym for “formless” (see eg. \href{https://suttacentral.net/mn21/en/sujato\#14.8}{MN 21:14.8}, “space is formless and invisible”, \textit{\textsanskrit{ākāso} \textsanskrit{arūpī} anidassano}). Normally the colors and images seen in the “form” absorptions are described as “visible” (eg. \href{https://suttacentral.net/dn16/en/sujato\#3.29.1}{DN 16:3.29.1}), so this indicates the formless attainments. } \\
infinite, luminous all-round.

%
\end{verse}

That\marginnote{25.3} is what does not fall within the scope of experience characterized by earth, water, fire, air, creatures, gods, the Progenitor, the Divinity, the gods of streaming radiance, the gods of universal beauty, the gods of abundant fruit, the Vanquisher, and the all. 

Well\marginnote{26.1} look now, good sir, I will vanish from you!’ 

‘All\marginnote{26.2} right, then, Divinity, vanish from me—if you can.’ 

Then\marginnote{26.3} Baka the Divinity said, ‘I will vanish from the ascetic Gotama! I will vanish from the ascetic Gotama!’ But he was unable to vanish from me.\footnote{Since he believes the dimension of infinite consciousness is that which the Buddha speaks of, he mistakenly assumes he has the power. } 

So\marginnote{26.5} I said to him, ‘Well now, Divinity, I will vanish from you!’ 

‘All\marginnote{26.7} right, then, good sir, vanish from me—if you can.’ 

Then\marginnote{26.8} I used my psychic power to will that my voice would extend so that Divinity, his assembly, and his retinue would hear me, but they would not see me.\footnote{This is a dramatic expression of the philosophical gulf between the Buddha and Baka. Baka’s power manifests as “light”, whereas the Buddha demonstrates invisibility. } And while vanished I recited this verse: 

\begin{verse}%
‘Seeing\marginnote{27.1} the danger in continued existence—\footnote{The Buddha shifts the focus from the attainment of a state of exalted consciousness to the cessation of existence. } \\
that life in any existence will cease to be—\footnote{Following Bodhi and Critical Pali Dictionary in reading \textit{vibhavesi}. | The \textit{ca} here answers to the last line of the verse. } \\
I didn’t affirm any kind of existence, \\
and didn’t grasp at relishing.’ 

%
\end{verse}

Then\marginnote{28.1} the Divinity, his assembly, and his retinue, their minds full of wonder and amazement, thought, ‘Oh, how incredible, how amazing! The ascetic Gotama has such psychic power and might! We’ve never before seen or heard of any other ascetic or brahmin with psychic power and might like the ascetic Gotama, who has gone forth from the Sakyan clan. Though people enjoy continued existence, loving it so much, he has extracted it, root and all.’ 

Then\marginnote{29.1} \textsanskrit{Māra} the Wicked took possession of a member of the retinue of Divinity and said this to me, ‘If such is your understanding, good sir, do not present it to your disciples or those gone forth!\footnote{The verb \textit{upanesi} here is used above at \href{https://suttacentral.net/mn49/en/sujato\#5.12}{MN 49:5.12} in the opposite case, where \textsanskrit{Māra} “presents” \textsanskrit{Brahmā}’s assembly to the Buddha. } Do not teach this Dhamma to your disciples or those gone forth! Do not wish this for your disciples or those gone forth! 

There\marginnote{29.5} have been ascetics and brahmins before you, mendicant, who claimed to be perfected ones, fully awakened Buddhas. They presented, taught, and wished this for their disciples and those gone forth. When their bodies broke up and their breath was cut off they were reborn in a lower realm. 

But\marginnote{29.8} there have also been other ascetics and brahmins before you, mendicant, who claimed to be perfected ones, fully awakened Buddhas. They did not present, teach, or wish this for their disciples and those gone forth. When their bodies broke up and their breath was cut off they were reborn in a higher realm. 

So,\marginnote{29.11} mendicant, I tell you this: please, good sir, remain passive, dwelling in blissful meditation in this life, for this is better left unsaid. Good sir, do not instruct others.’ 

When\marginnote{30.1} he had spoken, I said to \textsanskrit{Māra}, ‘I know you, Wicked One. Do not think, “He doesn’t know me.” You are \textsanskrit{Māra} the Wicked. You don’t speak to me like this out of sympathy, but with no sympathy. For you think, “Those who the ascetic Gotama teaches will go beyond my reach.” 

Those\marginnote{30.9} who formerly claimed to be fully awakened Buddhas were not in fact fully awakened Buddhas. But I am. The Realized One remains as such whether or not he teaches disciples. The Realized One remains as such whether or not he presents the teaching to disciples. Why is that? Because the Realized One has given up the defilements that are corrupting, leading to future lives, hurtful, resulting in suffering and future rebirth, old age, and death. He has cut them off at the root, made them like a palm stump, obliterated them so they are unable to arise in the future. Just as a palm tree with its crown cut off is incapable of further growth, the Realized One has given up the defilements that are corrupting, leading to future lives, hurtful, resulting in suffering and future rebirth, old age, and death. He has cut them off at the root, made them like a palm stump, obliterated them so they are unable to arise in the future.’” 

And\marginnote{31.1} so, because of the silencing of \textsanskrit{Māra}, and because of the invitation of the Divinity, the name of this discussion is “On the Invitation of Divinity”. 

%
\section*{{\suttatitleacronym MN 50}{\suttatitletranslation The Condemnation of Māra }{\suttatitleroot Māratajjanīyasutta}}
\addcontentsline{toc}{section}{\tocacronym{MN 50} \toctranslation{The Condemnation of Māra } \tocroot{Māratajjanīyasutta}}
\markboth{The Condemnation of Māra }{Māratajjanīyasutta}
\extramarks{MN 50}{MN 50}

\scevam{So\marginnote{1.1} I have heard. }At one time Venerable \textsanskrit{Mahāmoggallāna} was staying in the land of the Bhaggas at Crocodile Hill, in the deer park at \textsanskrit{Bhesakaḷā}’s Wood.\footnote{This sutta serves as a showcase for \textsanskrit{Moggallāna}, who is able to swiftly best \textsanskrit{Māra} and recount a detailed story of past lives. } 

At\marginnote{2.1} that time \textsanskrit{Moggallāna} was walking mindfully in the open air. 

Now\marginnote{2.2} at that time \textsanskrit{Māra} the Wicked had got inside \textsanskrit{Moggallāna}’s belly.\footnote{\textsanskrit{Māra} is sometimes called Namuci, and in fact seems to descend from the Vedic adversary of that name. Vedic Namuci “the wicked” (\textit{\textsanskrit{pāpmā} vai namuciḥ}, Śatapatha \textsanskrit{Brāhmaṇa} 12.7.3, \textsanskrit{Maitrāyaṇī} \textsanskrit{Saṁhitā} 4.4.4) was a wily trickster (\textit{\textsanskrit{māyin}}, Rig Veda 1.53.7c) who was nonetheless outsmarted by Indra (Rig Veda 5.30.6c), and possessed ineffectual armies reliant on women (Rig Veda 5.30.9a). Buddhist \textsanskrit{Māra} “the wicked” (\textit{\textsanskrit{māro} \textsanskrit{pāpimā}}) was a wily trickster (\emph{passim}) who was nonetheless outsmarted by the Buddha and his followers (\emph{passim}); he possessed ineffectual armies (\href{https://suttacentral.net/dn20/en/sujato\#21.3}{DN 20:21.3}, \href{https://suttacentral.net/an4.13/en/sujato\#2.3}{AN 4.13:2.3}, \href{https://suttacentral.net/snp3.2/en/sujato}{Snp 3.2}), and sent his daughters to do his dirty work (\href{https://suttacentral.net/sn4.25/en/sujato}{SN 4.25}). The decapitated head of Namuci introduced death (\textit{\textsanskrit{mṛtyu}}) in the form of blood to the deathless (\textit{\textsanskrit{amṛta}}) soma (Śatapatha \textsanskrit{Brāhmaṇa} 12.7.3.4, \textsanskrit{Maitrāyaṇī} \textsanskrit{Saṁhitā} 4.4.4), establishing his Pali names \textsanskrit{Maccurāja}, “King of Death” and \textsanskrit{Māra}, “Murderer”. He appears in a similar guise in Jainism too (\textsanskrit{Ācaraṅga} 1.3.1.3). } \textsanskrit{Moggallāna} thought, “Why now is my belly so very heavy, like I’ve just eaten a load of beans?”\footnote{T 66 has \langlzh{猶若食豆}, MA 131 \langlzh{猶如食豆}, both of which confirm the sense “full of beans”. T 67 on the other hand has \langlzh{飢人而負重擔}, “like a hungry man bearing a heavy load.” } Then he stepped down from the walking path, entered his dwelling, sat down on the seat spread out, and investigated inside himself. 

He\marginnote{3.2} saw that \textsanskrit{Māra} the Wicked had got inside his belly. So he said to \textsanskrit{Māra}, “Come out, Wicked One, come out! Do not harass the Realized One or his disciple. Don’t create lasting harm and suffering for yourself!” 

Then\marginnote{4.1} \textsanskrit{Māra} thought, “This ascetic doesn’t really know me or see me when he tells me to come out. Not even the Teacher could recognize me so quickly, so how could a disciple?” 

Then\marginnote{5.1} \textsanskrit{Moggallāna} said to \textsanskrit{Māra}, “I know you even when you’re like this, Wicked One. Do not think, ‘He doesn’t know me.’ You are \textsanskrit{Māra} the Wicked. And you think, ‘This ascetic doesn’t really know me or see me when he tells me to come out. Not even the Teacher could recognize me so quickly, so how could a disciple?’” 

Then\marginnote{6.1} \textsanskrit{Māra} thought, “This ascetic really does know me and see me when he tells me to come out.” 

Then\marginnote{6.7} \textsanskrit{Māra} came up out of \textsanskrit{Moggallāna}’s mouth and stood against the door. \textsanskrit{Moggallāna} saw him there and said, “I see you even there, Wicked One. Do not think, ‘He doesn’t see me.’ That’s you, Wicked One, standing against the door. 

Once\marginnote{8.1} upon a time, Wicked One, I was a \textsanskrit{Māra} named \textsanskrit{Dūsī}, and I had a sister named \textsanskrit{Kāḷī}.\footnote{\textsanskrit{Dūsī} the “corrupter” and \textsanskrit{Kāḷī} the “dark lady”, an early mention of a goddess of that name. | This story is also told at \href{https://suttacentral.net/thag20.1/en/sujato\#48.1}{Thag 20.1:48.1}. } You were her son, which made you my nephew. 

At\marginnote{9.1} that time Kakusandha, the Blessed One, the perfected one, the fully awakened Buddha arose in the world.\footnote{Kakusandha is mentioned also at \href{https://suttacentral.net/sn12.7/en/sujato\#1.1}{SN 12.7:1.1}, \href{https://suttacentral.net/sn15.20/en/sujato\#2.8}{SN 15.20:2.8}, \href{https://suttacentral.net/dn14/en/sujato\#1.4.4}{DN 14:1.4.4}, \href{https://suttacentral.net/dn32/en/sujato\#3.8}{DN 32:3.8}, and \href{https://suttacentral.net/thag7.5/en/sujato\#4.4}{Thag 7.5:4.4}. } Kakusandha had a fine pair of chief disciples named Vidhura and \textsanskrit{Sañjīva}. Of all the disciples of the Buddha Kakusandha, none were the equal of Venerable Vidhura in teaching Dhamma. And that’s how he came to be known as Vidhura.\footnote{At \href{https://suttacentral.net/an3.20/en/sujato\#2.1}{AN 3.20:2.1}, \textit{vidhura} is the cleverness of a shopkeeper in trade, illustrating the “indefatigable” mendicant in striving. } 

But\marginnote{10.1} when Venerable \textsanskrit{Sañjīva} had gone to a wilderness, or to the root of a tree, or to an empty hut, he easily attained the cessation of perception and feeling. Once upon a time, \textsanskrit{Sañjīva} was sitting at the root of a certain tree having attained the cessation of perception and feeling. Some cowherds, shepherds, farmers, and passers-by saw him sitting there and said, ‘Oh, how incredible, how amazing! This ascetic passed away while seated. We should cremate him.’ They collected grass, wood, and cow-dung, heaped it all on \textsanskrit{Sañjīva}’s body, set it on fire, and left. 

Then,\marginnote{11.1} when the night had passed, \textsanskrit{Sañjīva} emerged from that attainment, shook out his robes, and, since it was morning, he robed up and entered the village for alms. Those cowherds, shepherds, farmers, and passers-by saw him wandering for alms and said, ‘Oh, how incredible, how amazing! This ascetic passed away while seated, and now he has come back to life!’ And that’s how he came to be known as \textsanskrit{Sañjīva}.\footnote{\textit{\textsanskrit{Sañjīva}} means “survivor”. } 

Then\marginnote{12.1} it occurred to \textsanskrit{Māra} \textsanskrit{Dūsī}, ‘I don’t know the course of rebirth of these ethical mendicants of good character.\footnote{The idiom \textit{\textsanskrit{āgatiṁ} \textsanskrit{vā} \textsanskrit{gatiṁ} \textsanskrit{vā}}, literally “comings and goings”, refers to the “course of rebirth”. } Why don’t I take possession of these brahmins and householders and say, “Come, all of you, abuse, attack, harass, and trouble the ethical mendicants of good character.\footnote{\textsanskrit{Māra} urges and encourages, but does not control their behavior, so they are still responsible for their deeds. } Hopefully by doing this we can upset their minds so that \textsanskrit{Māra} \textsanskrit{Dūsī} can find a vulnerability.”’ And that’s exactly what he did. 

Then\marginnote{13.1} those brahmins and householders abused, attacked, harassed, and troubled the ethical mendicants of good character: ‘These shavelings, fake ascetics, primitives, black spawn from the feet of our kinsman, say, “We practice absorption meditation! We practice absorption meditation!” Shoulders drooping, downcast, and dopey, they meditate and concentrate and contemplate and ruminate.\footnote{The phrase “meditate and concentrate and contemplate and ruminate” (\textit{\textsanskrit{jhāyanti} \textsanskrit{pajjhāyanti} \textsanskrit{nijjhāyanti} \textsanskrit{apajjhāyanti}}) uses a series of verbs from the same root as \textit{\textsanskrit{jhāna}} with different prefixes to satirical effect. See also \href{https://suttacentral.net/an6.46/en/sujato\#2.2}{AN 6.46:2.2}, \href{https://suttacentral.net/an11.9/en/sujato\#2.10}{AN 11.9:2.10}, and \href{https://suttacentral.net/mn108/en/sujato\#26.4}{MN 108:26.4}. } They’re just like an owl on a branch, which meditates and concentrates and contemplates and ruminates as it hunts a mouse. They’re just like a jackal on a river-bank, which meditates and concentrates and contemplates and ruminates as it hunts a fish. They’re just like a cat by an alley or a drain or a dustbin, which meditates and concentrates and contemplates and ruminates as it hunts a mouse. They’re just like an unladen donkey by an alley or a drain or a dustbin, which meditates and concentrates and contemplates and ruminates. In the same way, these shavelings, fake ascetics, primitives, black spawn from the feet of our kinsman, say, “We practice absorption meditation! We practice absorption meditation!” Shoulders drooping, downcast, and dopey, they meditate and concentrate and contemplate and ruminate.’ 

Most\marginnote{13.11} of the people who died at that time—when their body broke up, after death—were reborn in a place of loss, a bad place, the underworld, hell. 

Then\marginnote{14.1} Kakusandha the Blessed One, the perfected one, the fully awakened Buddha, addressed the mendicants: ‘Mendicants, the brahmins and householders have been possessed by \textsanskrit{Māra} \textsanskrit{Dūsī}. He told them to abuse you in the hope of upsetting your minds so that he can find a vulnerability. Come, all of you mendicants, meditate spreading a heart full of love to one direction, and to the second, and to the third, and to the fourth. In the same way above, below, across, everywhere, all around, spread a heart full of love to the whole world—abundant, expansive, limitless, free of enmity and ill will.\footnote{These meditations counteract the tendency to anger. } Meditate spreading a heart full of compassion … Meditate spreading a heart full of rejoicing … Meditate spreading a heart full of equanimity to one direction, and to the second, and to the third, and to the fourth. In the same way above, below, across, everywhere, all around, spread a heart full of equanimity to the whole world—abundant, expansive, limitless, free of enmity and ill will.’ 

When\marginnote{15.1} those mendicants were instructed and advised by the Buddha Kakusandha in this way, they went to a wilderness, or to the root of a tree, or to an empty hut, where they meditated spreading a heart full of love … compassion … rejoicing … equanimity. 

Then\marginnote{16.1} it occurred to \textsanskrit{Māra} \textsanskrit{Dūsī}, ‘Even when I do this I don’t know the course of rebirth of these ethical mendicants of good character. Why don’t I take possession of these brahmins and householders and say, “Come, all of you, honor, respect, esteem, and venerate the ethical mendicants of good character. Hopefully by doing this we can upset their minds so that \textsanskrit{Māra} \textsanskrit{Dūsī} can find a vulnerability.”’\footnote{\textsanskrit{Māra} did not want to genuinely respect the mendicants, but to corrupt them with adulation. } 

And\marginnote{17.1} that’s exactly what he did. Then those brahmins and householders honored, respected, esteemed, and venerated the ethical mendicants of good character. 

Most\marginnote{17.5} of the people who died at that time—when their body broke up, after death—were reborn in a good place, a heavenly realm. 

Then\marginnote{18.1} Kakusandha the Blessed One, the perfected one, the fully awakened Buddha, addressed the mendicants: ‘Mendicants, the brahmins and householders have been possessed by \textsanskrit{Māra} \textsanskrit{Dūsī}. He told them to venerate you in the hope of upsetting your minds so that he can find a vulnerability. Come, all you mendicants, meditate observing the ugliness of the body, perceiving the repulsiveness of food, perceiving dissatisfaction with the whole world, and observing the impermanence of all conditions.’\footnote{These meditations counteract attachment. } 

When\marginnote{19.1} those mendicants were instructed and advised by the Buddha Kakusandha in this way, they went to a wilderness, or to the root of a tree, or to an empty hut, where they meditated observing the ugliness of the body, perceiving the repulsiveness of food, perceiving dissatisfaction with the whole world, and observing the impermanence of all conditions. 

Then\marginnote{20.1} the Buddha Kakusandha robed up in the morning and, taking this bowl and robe, entered the village for alms with Venerable Vidhura as his second monk. 

Then\marginnote{21.1} \textsanskrit{Māra} \textsanskrit{Dūsī} took possession of a certain boy, picked up a rock, and hit Vidhura on the head, cracking it open. Then Vidhura, with blood pouring from his cracked skull, still followed behind the Buddha Kakusandha.\footnote{\textit{Yeva} (“still”) has an adversative sense here (see \href{https://suttacentral.net/an3.82/en/sujato\#1.4}{AN 3.82:1.4}). } Then the Buddha Kakusandha turned to gaze back, the way that elephants do, saying,\footnote{\textit{\textsanskrit{Nāgāpalokitaṁ}}, the “elephant look”, was famously employed by the Buddha as he made his final departure from \textsanskrit{Vesālī} (\href{https://suttacentral.net/dn16/en/sujato\#4.1.2}{DN 16:4.1.2}). There is a similar Sanskrit term \textit{\textsanskrit{siṁhāvalokana}}, the “lion look”, said to be the slow glance back that a lion makes as he leaves his kill. } ‘This \textsanskrit{Māra} \textsanskrit{Dūsī} knows no bounds.’ And as he was gazing, \textsanskrit{Māra} \textsanskrit{Dūsī} fell from that place and was reborn in the Great Hell.\footnote{The Pali \textit{\textsanskrit{sahāpalokanāya}} makes it clear that \textsanskrit{Māra} fell to hell while the Buddha was looking, not because he looked. | More details on hell are provided in \href{https://suttacentral.net/mn129/en/sujato}{MN 129} and \href{https://suttacentral.net/mn130/en/sujato}{MN 130}. } 

Now\marginnote{22.1} that Great Hell is known by three names: ‘Related to the Six Fields of Contact’ and also ‘The Impaling With Spikes’ and also ‘Individually Painful’.\footnote{The sense of \textit{paccattavedaniya} (“Individually Painful”) is clarified in the verses (\href{https://suttacentral.net/mn50/en/sujato\#24.6}{MN 50:24.6}). } Then the wardens of hell came to me and said, ‘When spike meets spike in your heart, you will know that you’ve been roasting in hell for a thousand years.’ 

I\marginnote{23.1} roasted for many years, many centuries, many millennia in that Great Hell. For ten thousand years I roasted in the annex of that Great Hell, experiencing the pain called ‘this is emergence’.\footnote{I think \textit{\textsanskrit{vuṭṭhānimaṁ}} (“this is emergence”) simply means the feeling that is experienced on “emergence” from the Great Hell to the annex. The commentary, however, says it means the “feeling that emerges as a result of kamma”. } My body was in human form, but I had the head of a fish. 

\begin{verse}%
What\marginnote{24.1} kind of hell was that, \\
where \textsanskrit{Dūsī} was roasted \\
after attacking the disciple Vidhura \\
along with the brahmin Kakusandha? 

There\marginnote{24.5} were 100 iron spikes, \\
each one individually painful. \\
That’s the kind of hell \\
where \textsanskrit{Dūsī} was roasted \\
after attacking the disciple Vidhura \\
along with the brahmin Kakusandha. 

Dark\marginnote{24.11} One, if you attack \\
a mendicant who directly knows this, \\
a disciple of the Buddha, \\
you’ll fall into suffering. 

There\marginnote{25.1} are mansions that last an eon \\
standing in the middle of a lake. \\
Sapphire-colored, brilliant, \\
they sparkle and shine. \\
Dancing there are nymphs \\
shining in all different colors. 

Dark\marginnote{25.7} One, if you attack \\
a mendicant who directly knows this, \\
a disciple of the Buddha, \\
you’ll fall into suffering. 

I’m\marginnote{26.1} the one who, urged by the Buddha, \\
shook the stilt longhouse of \textsanskrit{Migāra}’s mother \\
with his big toe \\
as the \textsanskrit{Saṅgha} of mendicants watched.\footnote{\href{https://suttacentral.net/sn51.14/en/sujato}{SN 51.14}. } 

Dark\marginnote{26.5} One, if you attack \\
a mendicant who directly knows this, \\
a disciple of the Buddha, \\
you’ll fall into suffering. 

I’m\marginnote{27.1} the one who shook the Palace of Victory \\
with his big toe \\
owing to psychic power, \\
inspiring deities to awe.\footnote{\href{https://suttacentral.net/mn37/en/sujato\#11.4}{MN 37:11.4}. } 

Dark\marginnote{27.5} One, if you attack \\
a mendicant who directly knows this, \\
a disciple of the Buddha, \\
you’ll fall into suffering. 

I’m\marginnote{28.1} the one who asked Sakka \\
in the Palace of Victory: \\
‘\textsanskrit{Vāsava}, I hope you recall\footnote{In MN 37, \textsanskrit{Moggallāna} calls him Kosiya, but \textsanskrit{Vāsava} is also frequently used as vocative for Sakka. } \\
the one who is freed through the ending of craving?’\footnote{\textsanskrit{Moggallāna} asked Sakka if he remembered the teaching on this topic that he had received from the Buddha (\href{https://suttacentral.net/mn37/en/sujato\#8.1}{MN 37:8.1}). | The compound \textit{\textsanskrit{taṇhākkhayavimuttiyo}} is translated as a feminine plural by Norman in \emph{Elders’ Verses}, but \href{https://suttacentral.net/mn37/en/sujato\#2.2}{MN 37:2.2} refers to “the mendicant who is freed” in singular. Resolve to \textit{\textsanskrit{taṇhākkhayavimutti} yo}; \textit{vimutti} agrees with \textit{yo} as the nominative singular of the masculine agent noun in \textit{-in}, which occurs in the same phrase at \href{https://suttacentral.net/an4.38/en/sujato\#6.2}{AN 4.38:6.2} and \href{https://suttacentral.net/iti55/en/sujato\#4.2}{Iti 55:4.2}. } \\
And I’m the one to whom Sakka \\
admitted the truth when asked. 

Dark\marginnote{28.7} One, if you attack \\
a mendicant who directly knows this, \\
a disciple of the Buddha, \\
you’ll fall into suffering. 

I’m\marginnote{29.1} the one who asked the Divinity \\
in the Hall of Justice before the assembly: \\
‘Reverend, do you still have the same view \\
that you had in the past? \\
Or do you see the radiance \\
transcending the realm of divinity?’ 

And\marginnote{29.7} I’m the one to whom the Divinity \\
truthfully admitted his progress: \\
‘Good sir, I don’t have that view \\
that I had in the past. 

I\marginnote{29.11} see the radiance \\
transcending the realm of divinity. \\
So how could I say today \\
that I am permanent and eternal?’\footnote{\href{https://suttacentral.net/sn6.5/en/sujato}{SN 6.5}. } 

Dark\marginnote{29.15} One, if you attack \\
a mendicant who directly knows this, \\
a disciple of the Buddha, \\
you’ll fall into suffering. 

I’m\marginnote{30.1} the one who touched the peak of Mount Meru \\
using the power of meditative liberation. \\
I’ve visited the forests of the people \\
who dwell in the Eastern Continent. 

Dark\marginnote{30.5} One, if you attack \\
a mendicant who directly knows this, \\
a disciple of the Buddha, \\
you’ll fall into suffering. 

Though\marginnote{31.1} a fire doesn’t think, \\
‘I’ll burn the fool!’ \\
Still the fool who attacks \\
the fire gets burnt. 

In\marginnote{31.5} the same way, \textsanskrit{Māra}, \\
in attacking the Realized One, \\
you’ll only burn yourself, \\
like a fool touching the flames. 

\textsanskrit{Māra}’s\marginnote{31.9} done a bad thing \\
in attacking the Realized One. \\
Wicked One, do you imagine that \\
your wickedness won’t bear fruit? 

Your\marginnote{31.13} deeds heap up wickedness \\
that will last a long time, terminator! \\
Give up on the Buddha, \textsanskrit{Māra}! \\
And hold no hope for the mendicants!” 

That\marginnote{31.17} is how, in the \textsanskrit{Bhesekaḷā} grove,\footnote{The commentary to \href{https://suttacentral.net/thag20.1/en/sujato\#69.1}{Thag 20.1:69.1} says that this verse was added at the Council. } \\
the mendicant condemned \textsanskrit{Māra}. \\
That spirit, downcast, \\
disappeared right there. 

%
\end{verse}

%
\backmatter%
\chapter*{Colophon}
\addcontentsline{toc}{chapter}{Colophon}
\markboth{Colophon}{Colophon}

\section*{The Translator}

Bhikkhu Sujato was born as Anthony Aidan Best on 4/11/1966 in Perth, Western Australia. He grew up in the pleasant suburbs of Mt Lawley and Attadale alongside his sister Nicola, who was the good child. His mother, Margaret Lorraine Huntsman née Pinder, said “he’ll either be a priest or a poet”, while his father, Anthony Thomas Best, advised him to “never do anything for money”. He attended Aquinas College, a Catholic school, where he decided to become an atheist. At the University of WA he studied philosophy, aiming to learn what he wanted to do with his life. Finding that what he wanted to do was play guitar, he dropped out. His main band was named Martha’s Vineyard, which achieved modest success in the indie circuit. 

A seemingly random encounter with a roadside joey took him to Thailand, where he entered his first meditation retreat at Wat Ram Poeng, Chieng Mai in 1992. Feeling the call to the Buddha’s path, he took full ordination in Wat Pa Nanachat in 1994, where his teachers were Ajahn Pasanno and Ajahn Jayasaro. In 1997 he returned to Perth to study with Ajahn Brahm at Bodhinyana Monastery. 

He spent several years practicing in seclusion in Malaysia and Thailand before establishing Santi Forest Monastery in Bundanoon, NSW, in 2003. There he was instrumental in supporting the establishment of the Theravada bhikkhuni order in Australia and advocating for women’s rights. He continues to teach in Australia and globally, with a special concern for the moral implications of climate change and other forms of environmental destruction. He has published a series of books of original and groundbreaking research on early Buddhism. 

In 2005 he founded SuttaCentral together with Rod Bucknell and John Kelly. In 2015, seeing the need for a complete, accurate, plain English translation of the Pali texts, he undertook the task, spending nearly three years in isolation on the isle of Qi Mei off the coast of the nation of Taiwan. He completed the four main \textsanskrit{Nikāyas} in 2018, and the early books of the Khuddaka \textsanskrit{Nikāya} were complete by 2021. All this work is dedicated to the public domain and is entirely free of copyright encumbrance. 

In 2019 he returned to Sydney where he established Lokanta Vihara (The Monastery at the End of the World). 

\section*{Creation Process}

Primary source was the digital \textsanskrit{Mahāsaṅgīti} edition of the Pali \textsanskrit{Tipiṭaka}. Translated from the Pali, with reference to several English translations, especially those of Bhikkhu Bodhi.

\section*{The Translation}

This translation was part of a project to translate the four Pali \textsanskrit{Nikāyas} with the following aims: plain, approachable English; consistent terminology; accurate rendition of the Pali; free of copyright. It was made during 2016–2018 while Bhikkhu Sujato was staying in Qimei, Taiwan.

\section*{About SuttaCentral}

SuttaCentral publishes early Buddhist texts. Since 2005 we have provided root texts in Pali, Chinese, Sanskrit, Tibetan, and other languages, parallels between these texts, and translations in many modern languages. Building on the work of generations of scholars, we offer our contribution freely.

SuttaCentral is driven by volunteer contributions, and in addition we employ professional developers. We offer a sponsorship program for high quality translations from the original languages. Financial support for SuttaCentral is handled by the SuttaCentral Development Trust, a charitable trust registered in Australia.

\section*{About Bilara}

“Bilara” means “cat” in Pali, and it is the name of our Computer Assisted Translation (CAT) software. Bilara is a web app that enables translators to translate early Buddhist texts into their own language. These translations are published on SuttaCentral with the root text and translation side by side.

\section*{About SuttaCentral Editions}

The SuttaCentral Editions project makes high quality books from selected Bilara translations. These are published in formats including HTML, EPUB, PDF, and print.

You are welcome to print any of our Editions.

%
\end{document}