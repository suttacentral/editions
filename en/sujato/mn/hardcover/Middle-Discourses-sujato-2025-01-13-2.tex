\documentclass[12pt,openany]{book}%
\usepackage{lastpage}%
%
\usepackage{ragged2e}
\usepackage{verse}
\usepackage[a-3u]{pdfx}
\usepackage[inner=1in, outer=1in, top=.7in, bottom=1in, papersize={6in,9in}, headheight=13pt]{geometry}
\usepackage{polyglossia}
\usepackage[12pt]{moresize}
\usepackage{soul}%
\usepackage{microtype}
\usepackage{tocbasic}
\usepackage{realscripts}
\usepackage{epigraph}%
\usepackage{setspace}%
\usepackage{sectsty}
\usepackage{fontspec}
\usepackage{marginnote}
\usepackage[bottom]{footmisc}
\usepackage{enumitem}
\usepackage{fancyhdr}
\usepackage{emptypage}
\usepackage{extramarks}
\usepackage{graphicx}
\usepackage{relsize}
\usepackage{etoolbox}

% improve ragged right headings by suppressing hyphenation and orphans. spaceskip plus and minus adjust interword spacing; increase rightskip stretch to make it want to push a word on the first line(s) to the next line; reduce parfillskip stretch to make line length more equal . spacefillskip and xspacefillskip can be deleted to use defaults.
\protected\def\BalancedRagged{
\leftskip     0pt
\rightskip    0pt plus 10em
\spaceskip=1\fontdimen2\font plus .5\fontdimen3\font minus 1.5\fontdimen4\font
\xspaceskip=1\fontdimen2\font plus 1\fontdimen3\font minus 1\fontdimen4\font
\parfillskip  0pt plus 15em
\relax
}

\hypersetup{
colorlinks=true,
urlcolor=black,
linkcolor=black,
citecolor=black,
allcolors=black
}

% use a small amount of tracking on small caps
\SetTracking[ spacing = {25*,166, } ]{ encoding = *, shape = sc }{ 25 }

% add a blank page
\newcommand{\blankpage}{
\newpage
\thispagestyle{empty}
\mbox{}
\newpage
}

% define languages
\setdefaultlanguage[]{english}
\setotherlanguage[script=Latin]{sanskrit}

%\usepackage{pagegrid}
%\pagegridsetup{top-left, step=.25in}

% define fonts
% use if arno sanskrit is unavailable
%\setmainfont{Gentium Plus}
%\newfontfamily\Marginalfont[]{Gentium Plus}
%\newfontfamily\Allsmallcapsfont[RawFeature=+c2sc]{Gentium Plus}
%\newfontfamily\Noligaturefont[Renderer=Basic]{Gentium Plus}
%\newfontfamily\Noligaturecaptionfont[Renderer=Basic]{Gentium Plus}
%\newfontfamily\Fleuronfont[Ornament=1]{Gentium Plus}

% use if arno sanskrit is available. display is applied to \chapter and \part, subhead to \section and \subsection.
\setmainfont[
  FontFace={sb}{n}{Font = {Arno Pro Semibold}},
  FontFace={sb}{it}{Font = {Arno  Pro Semibold Italic}}
]{Arno Pro}

% create commands for using semibold
\DeclareRobustCommand{\sbseries}{\fontseries{sb}\selectfont}
\DeclareTextFontCommand{\textsb}{\sbseries}

\newfontfamily\Marginalfont[RawFeature=+subs]{Arno Pro Regular}
\newfontfamily\Allsmallcapsfont[RawFeature=+c2sc]{Arno Pro}
\newfontfamily\Noligaturefont[Renderer=Basic]{Arno Pro}
\newfontfamily\Noligaturecaptionfont[Renderer=Basic]{Arno Pro Caption}

% chinese fonts
\newfontfamily\cjk{Noto Serif TC}
\newcommand*{\langlzh}[1]{\cjk{#1}\normalfont}%

% logo
\newfontfamily\Logofont{sclogo.ttf}
\newcommand*{\sclogo}[1]{\large\Logofont{#1}}

% use subscript numerals for margin notes
\renewcommand*{\marginfont}{\Marginalfont}

% ensure margin notes have consistent vertical alignment
\renewcommand*{\marginnotevadjust}{-.17em}

% use compact lists
\setitemize{noitemsep,leftmargin=1em}
\setenumerate{noitemsep,leftmargin=1em}
\setdescription{noitemsep, style=unboxed, leftmargin=1em}

% style ToC
\DeclareTOCStyleEntries[
  raggedentrytext,
  linefill=\hfill,
  pagenumberwidth=.5in,
  pagenumberformat=\normalfont,
  entryformat=\normalfont
]{tocline}{chapter,section}


  \setlength\topsep{0pt}%
  \setlength\parskip{0pt}%

% define new \centerpars command for use in ToC. This ensures centering, proper wrapping, and no page break after
\def\startcenter{%
  \par
  \begingroup
  \leftskip=0pt plus 1fil
  \rightskip=\leftskip
  \parindent=0pt
  \parfillskip=0pt
}
\def\stopcenter{%
  \par
  \endgroup
}
\long\def\centerpars#1{\startcenter#1\stopcenter}

% redefine part, so that it adds a toc entry without page number
\let\oldcontentsline\contentsline
\newcommand{\nopagecontentsline}[3]{\oldcontentsline{#1}{#2}{}}

    \makeatletter
\renewcommand*\l@part[2]{%
  \ifnum \c@tocdepth >-2\relax
    \addpenalty{-\@highpenalty}%
    \addvspace{0em \@plus\p@}%
    \setlength\@tempdima{3em}%
    \begingroup
      \parindent \z@ \rightskip \@pnumwidth
      \parfillskip -\@pnumwidth
      {\leavevmode
       \setstretch{.85}\large\scshape\centerpars{#1}\vspace*{-1em}\llap{#2}}\par
       \nobreak
         \global\@nobreaktrue
         \everypar{\global\@nobreakfalse\everypar{}}%
    \endgroup
  \fi}
\makeatother

\makeatletter
\def\@pnumwidth{2em}
\makeatother

% define new sectioning command, which is only used in volumes where the pannasa is found in some parts but not others, especially in an and sn

\newcommand*{\pannasa}[1]{\clearpage\thispagestyle{empty}\begin{center}\vspace*{14em}\setstretch{.85}\huge\itshape\scshape\MakeLowercase{#1}\end{center}}

    \makeatletter
\newcommand*\l@pannasa[2]{%
  \ifnum \c@tocdepth >-2\relax
    \addpenalty{-\@highpenalty}%
    \addvspace{.5em \@plus\p@}%
    \setlength\@tempdima{3em}%
    \begingroup
      \parindent \z@ \rightskip \@pnumwidth
      \parfillskip -\@pnumwidth
      {\leavevmode
       \setstretch{.85}\large\itshape\scshape\lowercase{\centerpars{#1}}\vspace*{-1em}\llap{#2}}\par
       \nobreak
         \global\@nobreaktrue
         \everypar{\global\@nobreakfalse\everypar{}}%
    \endgroup
  \fi}
\makeatother

% don't put page number on first page of toc (relies on etoolbox)
\patchcmd{\chapter}{plain}{empty}{}{}

% global line height
\setstretch{1.05}

% allow linebreak after em-dash
\catcode`\—=13
\protected\def—{\unskip\textemdash\allowbreak}

% style headings with secsty. chapter and section are defined per-edition
\partfont{\setstretch{.85}\normalfont\centering\textsc}
\subsectionfont{\setstretch{.95}\normalfont\BalancedRagged}%
\subsubsectionfont{\setstretch{1}\normalfont\itshape\BalancedRagged}

% style elements of suttatitle
\newcommand*{\suttatitleacronym}[1]{\smaller[2]{#1}\vspace*{.3em}}
\newcommand*{\suttatitletranslation}[1]{\linebreak{#1}}
\newcommand*{\suttatitleroot}[1]{\linebreak\smaller[2]\itshape{#1}}

\DeclareTOCStyleEntries[
  indent=3.3em,
  dynindent,
  beforeskip=.2em plus -2pt minus -1pt,
]{tocline}{section}

\DeclareTOCStyleEntries[
  indent=0em,
  dynindent,
  beforeskip=.4em plus -2pt minus -1pt,
]{tocline}{chapter}

\newcommand*{\tocacronym}[1]{\hspace*{-3.3em}{#1}\quad}
\newcommand*{\toctranslation}[1]{#1}
\newcommand*{\tocroot}[1]{(\textit{#1})}
\newcommand*{\tocchapterline}[1]{\bfseries\itshape{#1}}


% redefine paragraph and subparagraph headings to not be inline
\makeatletter
% Change the style of paragraph headings %
\renewcommand\paragraph{\@startsection{paragraph}{4}{\z@}%
            {-2.5ex\@plus -1ex \@minus -.25ex}%
            {1.25ex \@plus .25ex}%
            {\noindent\normalfont\itshape\small}}

% Change the style of subparagraph headings %
\renewcommand\subparagraph{\@startsection{subparagraph}{5}{\z@}%
            {-2.5ex\@plus -1ex \@minus -.25ex}%
            {1.25ex \@plus .25ex}%
            {\noindent\normalfont\itshape\footnotesize}}
\makeatother

% use etoolbox to suppress page numbers on \part
\patchcmd{\part}{\thispagestyle{plain}}{\thispagestyle{empty}}
  {}{\errmessage{Cannot patch \string\part}}

% and to reduce margins on quotation
\patchcmd{\quotation}{\rightmargin}{\leftmargin 1.2em \rightmargin}{}{}
\AtBeginEnvironment{quotation}{\small}

% titlepage
\newcommand*{\titlepageTranslationTitle}[1]{{\begin{center}\begin{large}{#1}\end{large}\end{center}}}
\newcommand*{\titlepageCreatorName}[1]{{\begin{center}\begin{normalsize}{#1}\end{normalsize}\end{center}}}

% halftitlepage
\newcommand*{\halftitlepageTranslationTitle}[1]{\setstretch{2.5}{\begin{Huge}\uppercase{\so{#1}}\end{Huge}}}
\newcommand*{\halftitlepageTranslationSubtitle}[1]{\setstretch{1.2}{\begin{large}{#1}\end{large}}}
\newcommand*{\halftitlepageFleuron}[1]{{\begin{large}\Fleuronfont{{#1}}\end{large}}}
\newcommand*{\halftitlepageByline}[1]{{\begin{normalsize}\textit{{#1}}\end{normalsize}}}
\newcommand*{\halftitlepageCreatorName}[1]{{\begin{LARGE}{\textsc{#1}}\end{LARGE}}}
\newcommand*{\halftitlepageVolumeNumber}[1]{{\begin{normalsize}{\Allsmallcapsfont{\textsc{#1}}}\end{normalsize}}}
\newcommand*{\halftitlepageVolumeAcronym}[1]{{\begin{normalsize}{#1}\end{normalsize}}}
\newcommand*{\halftitlepageVolumeTranslationTitle}[1]{{\begin{Large}{\textsc{#1}}\end{Large}}}
\newcommand*{\halftitlepageVolumeRootTitle}[1]{{\begin{normalsize}{\Allsmallcapsfont{\textsc{\itshape #1}}}\end{normalsize}}}
\newcommand*{\halftitlepagePublisher}[1]{{\begin{large}{\Noligaturecaptionfont\textsc{#1}}\end{large}}}

% epigraph
\renewcommand{\epigraphflush}{center}
\renewcommand*{\epigraphwidth}{.85\textwidth}
\newcommand*{\epigraphTranslatedTitle}[1]{\vspace*{.5em}\footnotesize\textsc{#1}\\}%
\newcommand*{\epigraphRootTitle}[1]{\footnotesize\textit{#1}\\}%
\newcommand*{\epigraphReference}[1]{\footnotesize{#1}}%

% map
\newsavebox\IBox

% custom commands for html styling classes
\newcommand*{\scnamo}[1]{\begin{Center}\textit{#1}\end{Center}\bigskip}
\newcommand*{\scendsection}[1]{\begin{Center}\begin{small}\textit{#1}\end{small}\end{Center}\addvspace{1em}}
\newcommand*{\scendsutta}[1]{\begin{Center}\textit{#1}\end{Center}\addvspace{1em}}
\newcommand*{\scendbook}[1]{\bigskip\begin{Center}\uppercase{#1}\end{Center}\addvspace{1em}}
\newcommand*{\scendkanda}[1]{\begin{Center}\textbf{#1}\end{Center}\addvspace{1em}} % use for ending vinaya rule sections and also samyuttas %
\newcommand*{\scend}[1]{\begin{Center}\begin{small}\textit{#1}\end{small}\end{Center}\addvspace{1em}}
\newcommand*{\scendvagga}[1]{\begin{Center}\textbf{#1}\end{Center}\addvspace{1em}}
\newcommand*{\scrule}[1]{\textsb{#1}}
\newcommand*{\scadd}[1]{\textit{#1}}
\newcommand*{\scevam}[1]{\textsc{#1}}
\newcommand*{\scspeaker}[1]{\hspace{2em}\textit{#1}}
\newcommand*{\scbyline}[1]{\begin{flushright}\textit{#1}\end{flushright}\bigskip}
\newcommand*{\scexpansioninstructions}[1]{\begin{small}\textit{#1}\end{small}}
\newcommand*{\scuddanaintro}[1]{\medskip\noindent\begin{footnotesize}\textit{#1}\end{footnotesize}\smallskip}

\newenvironment{scuddana}{%
\setlength{\stanzaskip}{.5\baselineskip}%
  \vspace{-1em}\begin{verse}\begin{footnotesize}%
}{%
\end{footnotesize}\end{verse}
}%

% custom command for thematic break = hr
\newcommand*{\thematicbreak}{\begin{center}\rule[.5ex]{6em}{.4pt}\begin{normalsize}\quad\Fleuronfont{•}\quad\end{normalsize}\rule[.5ex]{6em}{.4pt}\end{center}}

% manage and style page header and footer. "fancy" has header and footer, "plain" has footer only

\pagestyle{fancy}
\fancyhf{}
\fancyfoot[RE,LO]{\thepage}
\fancyfoot[LE,RO]{\footnotesize\lastleftxmark}
\fancyhead[CE]{\setstretch{.85}\Noligaturefont\MakeLowercase{\textsc{\firstrightmark}}}
\fancyhead[CO]{\setstretch{.85}\Noligaturefont\MakeLowercase{\textsc{\firstleftmark}}}
\renewcommand{\headrulewidth}{0pt}
\fancypagestyle{plain}{ %
\fancyhf{} % remove everything
\fancyfoot[RE,LO]{\thepage}
\fancyfoot[LE,RO]{\footnotesize\lastleftxmark}
\renewcommand{\headrulewidth}{0pt}
\renewcommand{\footrulewidth}{0pt}}
\fancypagestyle{plainer}{ %
\fancyhf{} % remove everything
\fancyfoot[RE,LO]{\thepage}
\renewcommand{\headrulewidth}{0pt}
\renewcommand{\footrulewidth}{0pt}}

% style footnotes
\setlength{\skip\footins}{1em}

\makeatletter
\newcommand{\@makefntextcustom}[1]{%
    \parindent 0em%
    \thefootnote.\enskip #1%
}
\renewcommand{\@makefntext}[1]{\@makefntextcustom{#1}}
\makeatother

% hang quotes (requires microtype)
\microtypesetup{
  protrusion = true,
  expansion  = true,
  tracking   = true,
  factor     = 1000,
  patch      = all,
  final
}

% Custom protrusion rules to allow hanging punctuation
\SetProtrusion
{ encoding = *}
{
% char   right left
  {-} = {    , 500 },
  % Double Quotes
  \textquotedblleft
      = {1000,     },
  \textquotedblright
      = {    , 1000},
  \quotedblbase
      = {1000,     },
  % Single Quotes
  \textquoteleft
      = {1000,     },
  \textquoteright
      = {    , 1000},
  \quotesinglbase
      = {1000,     }
}

% make latex use actual font em for parindent, not Computer Modern Roman
\AtBeginDocument{\setlength{\parindent}{1em}}%
%

% Default values; a bit sloppier than normal
\tolerance 1414
\hbadness 1414
\emergencystretch 1.5em
\hfuzz 0.3pt
\clubpenalty = 10000
\widowpenalty = 10000
\displaywidowpenalty = 10000
\hfuzz \vfuzz
 \raggedbottom%

\title{Middle Discourses}
\author{Bhikkhu Sujato}
\date{}%
% define a different fleuron for each edition
\newfontfamily\Fleuronfont[Ornament=4]{Arno Pro}

% Define heading styles per edition for chapter, section, and subsection. Suttatitle can be any one of these, depending on the volume. 

\let\oldfrontmatter\frontmatter
\renewcommand{\frontmatter}{%
\chapterfont{\setstretch{.85}\normalfont\centering}%
\sectionfont{\setstretch{.85}\normalfont\BalancedRagged}%
\oldfrontmatter}

\let\oldmainmatter\mainmatter
\renewcommand{\mainmatter}{%
\chapterfont{\thispagestyle{empty}\vspace*{4em}\setstretch{.85}\LARGE\normalfont\itshape\scshape\centering\MakeLowercase}
\sectionfont{\clearpage\thispagestyle{plain}\vspace*{2em}\setstretch{.85}\normalfont\centering}%
\oldmainmatter}

\let\oldbackmatter\backmatter
\renewcommand{\backmatter}{%
\chapterfont{\setstretch{.85}\normalfont\centering}%
\sectionfont{\setstretch{.85}\normalfont\BalancedRagged}%
\pagestyle{plainer}%
\oldbackmatter}
%
%
\begin{document}%
\normalsize%
\frontmatter%
\setlength{\parindent}{0cm}

\pagestyle{empty}

\maketitle

\blankpage%
\begin{center}

\vspace*{2.2em}

\halftitlepageTranslationTitle{Middle Discourses}

\vspace*{1em}

\halftitlepageTranslationSubtitle{A lucid translation of the Majjhima Nikāya}

\vspace*{2em}

\halftitlepageFleuron{•}

\vspace*{2em}

\halftitlepageByline{translated and introduced by}

\vspace*{.5em}

\halftitlepageCreatorName{Bhikkhu Sujato}

\vspace*{4em}

\halftitlepageVolumeNumber{Volume 2}

\smallskip

\halftitlepageVolumeAcronym{MN 51–100}

\smallskip

\halftitlepageVolumeTranslationTitle{The Middle Fifty}

\smallskip

\halftitlepageVolumeRootTitle{Majjhimapaṇṇāsa}

\vspace*{\fill}

\sclogo{0}
 \halftitlepagePublisher{SuttaCentral}

\end{center}

\newpage
%
\setstretch{1.05}

\begin{footnotesize}

\textit{Middle Discourses} is a translation of the Majjhimanikāya by Bhikkhu Sujato.

\medskip

Creative Commons Zero (CC0)

To the extent possible under law, Bhikkhu Sujato has waived all copyright and related or neighboring rights to \textit{Middle Discourses}.

\medskip

This work is published from Australia.

\begin{center}
\textit{This translation is an expression of an ancient spiritual text that has been passed down by the Buddhist tradition for the benefit of all sentient beings. It is dedicated to the public domain via Creative Commons Zero (CC0). You are encouraged to copy, reproduce, adapt, alter, or otherwise make use of this translation. The translator respectfully requests that any use be in accordance with the values and principles of the Buddhist community.}
\end{center}

\medskip

\begin{description}
    \item[Web publication date] 2018
    \item[This edition] 2025-01-13 01:01:43
    \item[Publication type] hardcover
    \item[Edition] ed3
    \item[Number of volumes] 3
    \item[Publication ISBN] 978-1-76132-058-3
    \item[Volume ISBN] 978-1-76132-060-6
    \item[Publication URL] \href{https://suttacentral.net/editions/mn/en/sujato}{https://suttacentral.net/editions/mn/en/sujato}
    \item[Source URL] \href{https://github.com/suttacentral/bilara-data/tree/published/translation/en/sujato/sutta/mn}{https://github.com/suttacentral/bilara-data/tree/published/translation/en/sujato/sutta/mn}
    \item[Publication number] scpub3
\end{description}

\medskip

Map of Jambudīpa is by Jonas David Mitja Lang, and is released by him under Creative Commons Zero (CC0).

\medskip

Published by SuttaCentral

\medskip

\textit{SuttaCentral,\\
c/o Alwis \& Alwis Pty Ltd\\
Kaurna Country,\\
Suite 12,\\
198 Greenhill Road,\\
Eastwood,\\
SA 5063,\\
Australia}

\end{footnotesize}

\newpage

\setlength{\parindent}{1em}%%
\tableofcontents
\newpage
\pagestyle{fancy}
%
\chapter*{Summary of Contents}
\addcontentsline{toc}{chapter}{Summary of Contents}
\markboth{Summary of Contents}{Summary of Contents}

\begin{description}%
\item[The Chapter on Householders (\textit{\textsanskrit{Gahapativagga}})] This chapter is addressed to a diverse range of lay people.%
\item[MN 51: With Kandaraka (\textit{\textsanskrit{Kandarakasutta}})] The Buddha discusses mindfulness meditation with lay practitioners. Contrasting the openness of animals with the duplicity of humans, he explains how to practice in a way that causes no harm to oneself or others.%
\item[MN 52: The Man From the City of \textsanskrit{Aṭṭhaka} (\textit{\textsanskrit{Aṭṭhakanāgarasutta}})] Asked by a householder to teach a path to freedom, Venerable Ānanda explains no less than eleven meditative states that may serve as doors to the deathless.%
\item[MN 53: A Trainee (\textit{\textsanskrit{Sekhasutta}})] The Buddha is invited by his family, the Sakyans of Kapilavatthu, to inaugurate a new community hall. He invites Venerable Ānanda to explain in detail the stages of spiritual practice for a lay trainee.%
\item[MN 54: With Potaliya the Householder (\textit{\textsanskrit{Potaliyasutta}})] When Potaliya got upset at being referred to as “householder”, the Buddha quizzed him as to the true nature of attachment and renunciation.%
\item[MN 55: With \textsanskrit{Jīvaka} (\textit{\textsanskrit{Jīvakasutta}})] The Buddha’s personal doctor, \textsanskrit{Jīvaka}, hears criticisms of the Buddha’s policy regarding eating meat, and asks him about it.%
\item[MN 56: With \textsanskrit{Upāli} (\textit{\textsanskrit{Upālisutta}})] The Buddha disagrees with a Jain ascetic on the question of whether physical or mental deeds are more important. When he hears of this, the Jain disciple \textsanskrit{Upāli} decides to visit the Buddha and refute him, and proceeds despite all warnings.%
\item[MN 57: The Ascetic Who Behaved Like a Dog (\textit{\textsanskrit{Kukkuravatikasutta}})] Some ascetics in ancient India undertook extreme practices, such as a vow to behave like an ox or a dog. The Buddha meets two such individuals, and is reluctantly pressed to reveal the kammic outcomes of such practice.%
\item[MN 58: With Prince Abhaya (\textit{\textsanskrit{Abhayarājakumārasutta}})] The leader of the Jains, \textsanskrit{Nigaṇṭha} \textsanskrit{Nātaputta}, gives his disciple Prince Abhaya a dilemma to pose to the Buddha, supposing that this will show his weakness. Things don’t go quite as planned.%
\item[MN 59: The Many Kinds of Feeling (\textit{\textsanskrit{Bahuvedanīyasutta}})] The Buddha resolves a disagreement on the number of kinds of feelings that he taught, pointing out that different ways of teaching are appropriate in different contexts, and should not be a cause of disputes. He goes on to show the importance of pleasure in developing higher meditation.%
\item[MN 60: Guaranteed (\textit{\textsanskrit{Apaṇṇakasutta}})] The Buddha teaches a group of uncommitted householders how to use a rational reflection to arrive at practices and principles that are guaranteed to have a good outcome, even if we don’t know all the variables.%
\item[The Chapter on Mendicants (\textit{\textsanskrit{Bhikkhuvagga}})] Ten discourses to monks, many of them focusing on matters of discipline.%
\item[MN 61: Advice to \textsanskrit{Rāhula} at \textsanskrit{Ambalaṭṭhika} (\textit{\textsanskrit{Ambalaṭṭhikarāhulovādasutta}})] Using the “object lesson” of a cup of water, the Buddha explains to his son, \textsanskrit{Rāhula}, the importance of telling the truth and reflecting on one’s motives.%
\item[MN 62: The Longer Advice to \textsanskrit{Rāhula} (\textit{\textsanskrit{Mahārāhulovādasutta}})] The Buddha tells \textsanskrit{Rāhula} to meditate on not-self, which he immediately puts into practice. Seeing him, Venerable \textsanskrit{Sāriputta} advises him to develop breath meditation, but the Buddha suggests a wide range of different practices first.%
\item[MN 63: The Shorter Discourse With \textsanskrit{Māluṅkya} (\textit{\textsanskrit{Cūḷamālukyasutta}})] A monk demands that the Buddha answer his metaphysical questions, or else he will disrobe. The Buddha compares him to a man struck by an arrow, who refuses treatment until he can have all his questions about the arrow and the archer answered.%
\item[MN 64: The Longer Discourse With \textsanskrit{Māluṅkya} (\textit{\textsanskrit{Mahāmālukyasutta}})] A little baby has no wrong views or intentions, but the underlying tendency for these things is still there. Without practicing, they will inevitably recur.%
\item[MN 65: With \textsanskrit{Bhaddāli} (\textit{\textsanskrit{Bhaddālisutta}})] A monk refuses to follow the rule forbidding eating after noon, but is filled with remorse and forgiven.%
\item[MN 66: The Simile of the Quail (\textit{\textsanskrit{Laṭukikopamasutta}})] Again raising the rule regarding eating, but this time as a reflection of gratitude for the Buddha in eliminating things that cause complexity and stress. The Buddha emphasizes how attachment even to little things can be dangerous.%
\item[MN 67: At \textsanskrit{Cātumā} (\textit{\textsanskrit{Cātumasutta}})] After dismissing some unruly monks, the Buddha is persuaded to relent, and teaches them four dangers for those gone forth.%
\item[MN 68: At \textsanskrit{Naḷakapāna} (\textit{\textsanskrit{Naḷakapānasutta}})] Those who practice do so not because they are failures, but because they aspire to higher freedom. When he speaks of the attainments of disciples, the Buddha does so in order to inspire.%
\item[MN 69: With \textsanskrit{Gulissāni} (\textit{\textsanskrit{Goliyānisutta}})] A monk comes down to the community from the wilderness, but doesn’t behave properly. Venerable \textsanskrit{Sāriputta} explains how a mendicant should behave, whether in forest or town.%
\item[MN 70: At \textsanskrit{Kīṭāgiri} (\textit{\textsanskrit{Kīṭāgirisutta}})] A third discourse that presents the health benefits of eating in one part of the day, and the reluctance of some mendicants to follow this.%
\item[The Chapter on Wanderers (\textit{\textsanskrit{Paribbājakavagga}})] The Buddha in dialog with ascetics and wanderers.%
\item[MN 71: To Vacchagotta on the Three Knowledges (\textit{\textsanskrit{Tevijjavacchasutta}})] The Buddha denies being omniscient, and sets forth the three higher knowledges that form the core of his awakened insight.%
\item[MN 72: With Vacchagotta on Fire (\textit{\textsanskrit{Aggivacchasutta}})] Refusing to take a stance regarding useless metaphysical speculations, the Buddha illustrates the spiritual goal with the simile of a flame going out.%
\item[MN 73: The Longer Discourse With Vacchagotta (\textit{\textsanskrit{Mahāvacchasutta}})] In the final installment of the “Vacchagotta trilogy”, Vacchagotta lets go his obsession with meaningless speculation, and asks about practice.%
\item[MN 74: With \textsanskrit{Dīghanakha} (\textit{\textsanskrit{Dīghanakhasutta}})] Deftly outmaneuvering an extreme skeptic, the Buddha discusses the outcomes of belief and disbelief. Rather than getting stuck in abstractions, he encourages staying close to the feelings one experiences.%
\item[MN 75: With \textsanskrit{Māgaṇḍiya} (\textit{\textsanskrit{Māgaṇḍiyasutta}})] Accused by a hedonist of being too negative, the Buddha recounts the luxury of his upbringing, and his realization of how little value there was in such things. Through renunciation he found a far greater pleasure.%
\item[MN 76: With Sandaka (\textit{\textsanskrit{Sandakasutta}})] Venerable Ānanda teaches a group of wanderers how there are many different approaches to the spiritual life, many of which lead nowhere.%
\item[MN 77: The Longer Discourse with \textsanskrit{Sakuludāyī} (\textit{\textsanskrit{Mahāsakuludāyisutta}})] Unlike many teachers, the Buddha’s followers treat him with genuine love and respect, since they see the sincerity of his teaching and practice.%
\item[MN 78: With \textsanskrit{Uggāhamāna} \textsanskrit{Samaṇamaṇḍikāputta} (\textit{\textsanskrit{Samaṇamuṇḍikasutta}})] A wanderer teaches that a person has reached the highest attainment when they keep four basic ethical precepts. The Buddha’s standards are considerably higher.%
\item[MN 79: The Shorter Discourse With \textsanskrit{Sakuludāyī} (\textit{\textsanskrit{Cūḷasakuludāyisutta}})] A wanderer teaches his doctrine of the “highest splendor” but is unable to give a satisfactory account of what that means. The Buddha memorably compares him to someone who is in love with an idealized women who he has never met.%
\item[MN 80: With Vekhanasa (\textit{\textsanskrit{Vekhanasasutta}})] Starting off similar to the previous, the Buddha goes on to explain that one is not converted to his teaching just because of clever arguments, but because you see in yourself the results of the practice.%
\item[The Chapter on Kings (\textit{\textsanskrit{Rājavagga}})] Various dialogs with kings and princes, many of whom followed the Buddha.%
\item[MN 81: With \textsanskrit{Ghaṭīkāra} (\textit{\textsanskrit{Ghaṭikārasutta}})] The Buddha relates an unusual account of a past life in the time of the previous Buddha, Kassapa. At that time he was not interested in Dhamma, and had to be forced to go see the Buddha. This discourse is important in understanding the development of the Bodhisattva doctrine.%
\item[MN 82: With \textsanskrit{Raṭṭhapāla} (\textit{\textsanskrit{Raṭṭhapālasutta}})] A wealthy young man, \textsanskrit{Raṭṭhapāla}, has a strong aspiration to go forth, but has to prevail against the reluctance of his parents. Even after he became a monk, his parents tried to persuade him to disrobe. The discourse ends with a moving series of teachings on the fragility of the world.%
\item[MN 83: About King Maghadeva (\textit{\textsanskrit{Maghadevasutta}})] A rare extended mythic narrative, telling of an ancient kingly lineage and their eventual downfall.%
\item[MN 84: At \textsanskrit{Madhurā} (\textit{\textsanskrit{Madhurasutta}})] In \textsanskrit{Madhurā}, towards the north-eastern limit of the Buddha’s reach during his life, King Avantiputta asks Venerable \textsanskrit{Mahākaccāna} regarding the brahmanical claim to be the highest caste.%
\item[MN 85: With Prince Bodhi (\textit{\textsanskrit{Bodhirājakumārasutta}})] Admitting that he used to believe that pleasure was to be gained through pain, the Buddha explains how his practice showed him the fallacy of that idea.%
\item[MN 86: With \textsanskrit{Aṅgulimāla} (\textit{\textsanskrit{Aṅgulimālasutta}})] Ignoring warnings, the Buddha ventures into the domain of the notorious killer \textsanskrit{Aṅgulimāla} and succeeds in converting him to the path of non-violence. After becoming a monk \textsanskrit{Aṅgulimāla} still suffered for his past deeds, but only to a small extent. He uses his new commitment to non-violence to help a woman in labor.%
\item[MN 87: Born From the Beloved (\textit{\textsanskrit{Piyajātikasutta}})] A rare glimpse into the marital life of King Pasenadi, and how he is led to the Dhamma by his Queen, the incomparable \textsanskrit{Mallikā}. She confirms the Buddha’s teaching that our loved ones bring us sorrow; but that’s not something a husband, father, and king wants to hear.%
\item[MN 88: The Imported Cloth (\textit{\textsanskrit{Bāhitikasutta}})] King Pasenadi takes a chance to visit Venerable Ānanda, where he asks about skillful and unskillful behavior, and what is praised by the Buddha. He offers Ānanda a valuable cloth in gratitude.%
\item[MN 89: Shrines to the Teaching (\textit{\textsanskrit{Dhammacetiyasutta}})] King Pasenadi, near the end of his life, visits the Buddha, and shows moving devotion and love for his teacher.%
\item[MN 90: At \textsanskrit{Kaṇṇakatthala} (\textit{\textsanskrit{Kaṇṇakatthalasutta}})] King Pasenadi questions the Buddha on miscellaneous matters: caste, omniscience, and the gods among them.%
\item[The Chapter on Brahmins (\textit{\textsanskrit{Brāhmaṇavagga}})] The Buddha engages with the powerful caste of brahmins, contesting their claims to spiritual authority.%
\item[MN 91: With \textsanskrit{Brahmāyu} (\textit{\textsanskrit{Brahmāyusutta}})] The oldest and most respected brahmin of the age sends a student to examine the Buddha, and he spends several months following his every move before reporting back. Convinced that the Buddha fulfills an ancient prophecy of the Great Man, the brahmin becomes his disciple.%
\item[MN 92: With Sela (\textit{\textsanskrit{Selasutta}})] A brahmanical ascetic named \textsanskrit{Keṇiya} invites the entire \textsanskrit{Saṅgha} for a meal. When the brahmin Sela sees what is happening, he visits the Buddha and expresses his delight in a moving series of devotional verses.%
\item[MN 93: With \textsanskrit{Assalāyana} (\textit{\textsanskrit{Assalāyanasutta}})] A precocious brahmin student is encouraged against his wishes to challenge the Buddha on the question of caste. His reluctance turns out to be justified.%
\item[MN 94: With \textsanskrit{Ghoṭamukha} (\textit{\textsanskrit{Ghoṭamukhasutta}})] A brahmin denies that there is such a thing as a principled renunciate life, but Venerable Udena persuades him otherwise.%
\item[MN 95: With \textsanskrit{Caṅkī} (\textit{\textsanskrit{Caṅkīsutta}})] The reputed brahmin \textsanskrit{Caṅkī} goes with a large group to visit the Buddha, despite the reservations of other brahmins. A precocious student challenges the Buddha, affirming the validity of the Vedic scriptures. The Buddha gives a detailed explanation of how true understanding gradually emerges through spiritual education.%
\item[MN 96: With \textsanskrit{Esukārī} (\textit{\textsanskrit{Esukārīsutta}})] A brahmin claims that one deserves service and privilege depending on caste, but the Buddha counters that it is conduct, not caste, that show a person’s worth.%
\item[MN 97: With \textsanskrit{Dhanañjāni} (\textit{\textsanskrit{Dhanañjānisutta}})] A corrupt tax-collector is redeemed by his encounter with Venerable \textsanskrit{Sāriputta}.%
\item[MN 98: With \textsanskrit{Vāseṭṭha} (\textit{\textsanskrit{Vāseṭṭhasutta}})] Two brahmin students ask the Buddha about what makes a brahmin: birth or deeds? the Buddha points out that, while the species of animals are determined by birth, for humans what matters is how you chose to live. This discourse anticipates the modern view that there are no such things as clearly defined racial differences among humans.%
\item[MN 99: With Subha (\textit{\textsanskrit{Subhasutta}})] Working hard is not valuable in and of itself; what matters is the outcome. And just as in lay life, spiritual practice may or may not lead to fruitful results.%
\item[MN 100: With \textsanskrit{Saṅgārava} (\textit{\textsanskrit{Saṅgāravasutta}})] Angered by the devotion of a brahmin lady, a brahmin visits the Buddha. He positions himself against traditionalists and rationalists, as someone whose teaching is based on direct experience.%
\end{description}

%
\mainmatter%
\pagestyle{fancy}%
\addtocontents{toc}{\let\protect\contentsline\protect\nopagecontentsline}
\part*{The Middle Fifty }
\addcontentsline{toc}{part}{The Middle Fifty }
\markboth{}{}
\addtocontents{toc}{\let\protect\contentsline\protect\oldcontentsline}

%
\addtocontents{toc}{\let\protect\contentsline\protect\nopagecontentsline}
\chapter*{The Chapter on Householders }
\addcontentsline{toc}{chapter}{\tocchapterline{The Chapter on Householders }}
\addtocontents{toc}{\let\protect\contentsline\protect\oldcontentsline}

%
\section*{{\suttatitleacronym MN 51}{\suttatitletranslation With Kandaraka }{\suttatitleroot Kandarakasutta}}
\addcontentsline{toc}{section}{\tocacronym{MN 51} \toctranslation{With Kandaraka } \tocroot{Kandarakasutta}}
\markboth{With Kandaraka }{Kandarakasutta}
\extramarks{MN 51}{MN 51}

\scevam{So\marginnote{1.1} I have heard. }At one time the Buddha was staying near \textsanskrit{Campā} on the banks of the \textsanskrit{Gaggarā} Lotus Pond together with a large \textsanskrit{Saṅgha} of mendicants.\footnote{\textsanskrit{Campā} is modern Champapur near Bhagalpur in Bihar state, not far from West Bengal. It is near the eastern-most reach of the Buddha’s journeys. \textsanskrit{Campā} was the capital of \textsanskrit{Aṅga}, one of the sixteen “great nations” (\textit{\textsanskrit{mahājanapadā}}). It was a flourishing trade center, and became a sacred city for the Jains. | \textit{\textsanskrit{Gaggarā}}, an onomatopoeic reduplication (“gargle”), is the name of a number of rivers and whirlpools in Sanskrit (cp. the modern Ghaggar River in north-west India). } 

Then\marginnote{1.3} Pessa the elephant driver’s son and Kandaraka the wanderer went to see the Buddha. When they had approached, Pessa bowed and sat down to one side.\footnote{Neither of these appear elsewhere. } But the wanderer Kandaraka exchanged greetings with the Buddha and stood to one side. He looked around the mendicant \textsanskrit{Saṅgha}, who were so very silent, and said to the Buddha: 

“It’s\marginnote{2.1} incredible, Mister Gotama, it’s amazing! How the mendicant \textsanskrit{Saṅgha} has been led to practice properly by Mister Gotama! All the perfected ones, the fully awakened Buddhas in the past or the future who lead the mendicant \textsanskrit{Saṅgha} to practice properly will at best do so like Mister Gotama does in the present.”\footnote{Kandaraka’s status as a (non-Buddhist) wanderer does not preclude his faith in the Buddha. } 

“That’s\marginnote{3.1} so true, Kandaraka! That’s so true!\footnote{We do not hear anything about Kandaraka after this exchange. } All the perfected ones, the fully awakened Buddhas in the past or the future who lead the mendicant \textsanskrit{Saṅgha} to practice properly will at best do so like I do in the present. 

For\marginnote{3.6} in this mendicant \textsanskrit{Saṅgha} there are perfected mendicants, who have ended the defilements, completed the spiritual journey, done what had to be done, laid down the burden, achieved their own goal, utterly ended the fetter of continued existence, and are rightly freed through enlightenment. And in this mendicant \textsanskrit{Saṅgha} there are trainee mendicants who are consistently ethical, living consistently, alert, living alertly. They meditate with their minds firmly established in the four kinds of mindfulness meditation. What four? 

It’s\marginnote{3.10} when a mendicant meditates by observing an aspect of the body—keen, aware, and mindful, rid of covetousness and displeasure for the world. They meditate observing an aspect of feelings—keen, aware, and mindful, rid of covetousness and displeasure for the world. They meditate observing an aspect of the mind—keen, aware, and mindful, rid of covetousness and displeasure for the world. They meditate observing an aspect of principles—keen, aware, and mindful, rid of covetousness and displeasure for the world.” 

When\marginnote{4.1} he had spoken, Pessa said to the Buddha: 

“It’s\marginnote{4.2} incredible, sir, it’s amazing! How well described by the Buddha are the four kinds of mindfulness meditation! They are in order to purify sentient beings, to get past sorrow and crying, to make an end of pain and sadness, to discover the system, and to realize extinguishment.\footnote{\textit{\textsanskrit{Yāva} \textsanskrit{supaññatta}} is also at \href{https://suttacentral.net/dn18/en/sujato\#22.3}{DN 18:22.3}; it is a variation of the standard \textit{\textsanskrit{yāva} \textsanskrit{subhāsita}}. } For we white-clothed laypeople also from time to time meditate with our minds well established in the four kinds of mindfulness meditation.\footnote{The Buddha sometimes taught devotional meditations such as the six recollections for lay people, but passages such as this show there was no fundamental difference between lay and monastic meditations. } We meditate observing an aspect of the body … feelings … mind … principles—keen, aware, and mindful, rid of covetousness and displeasure for the world. 

It’s\marginnote{4.9} incredible, sir, it’s amazing! How the Buddha knows what’s best for sentient beings, even though people continue to be so shady, rotten, and tricky. For human beings are shady, sir, while the animal is obvious. For I can drive an elephant in training, and while going back and forth in \textsanskrit{Campā} it’ll try all the tricks, bluffs, ruses, and feints that it can. But my bondservants, servants, and workers behave one way by body, another by speech, and their minds another. It’s incredible, sir, it’s amazing! How the Buddha knows what’s best for sentient beings, even though people continue to be so shady, rotten, and tricky. For human beings are shady, sir, while the animal is obvious.” 

“That’s\marginnote{5.1} so true, Pessa! That’s so true! For human beings are shady, while the animal is obvious. Pessa, these four people are found in the world. What four? 

\begin{enumerate}%
\item One person mortifies themselves, committed to the practice of mortifying themselves.\footnote{“Mortifies themselves” is \textit{attantapo}. } %
\item One person mortifies others, committed to the practice of mortifying others. %
\item One person mortifies themselves and others, committed to the practice of mortifying themselves and others. %
\item One person doesn’t mortify either themselves or others, committed to the practice of not mortifying themselves or others. They live without wishes in this very life, quenched, cooled, experiencing bliss, with self become divine.\footnote{“With self become divine” (\textit{\textsanskrit{brahmabhūtena} \textsanskrit{attanā}}) deliberately echoes \textsanskrit{Upaniṣadic} language. Pali is sometimes said to lack reference to the cosmic Brahman (in neuter), having only the personal \textsanskrit{Brahmā} (in masculine). The grammatical case of \textit{brahma-} in the compound here is undetermined, yet no scholar of Sanskrit would hesitate to interpret the common phrase \textit{\textsanskrit{brahmabhūtātmā}} in the sense “self become one with the cosmic divinity Brahman”. Surely the Pali draws from the same sense, using it to describe \textsanskrit{Nibbāna}. } %
\end{enumerate}

Which\marginnote{5.11} one of these four people do you like the sound of?” 

“Sir,\marginnote{5.12} I don’t like the sound of the first three people. I only like the sound of the last person, who doesn’t mortify either themselves or others.” 

“But\marginnote{6.1} why don’t you like the sound of those three people?” 

“Sir,\marginnote{6.2} the person who mortifies themselves does so even though they want to be happy and recoil from pain. That’s why I don’t like the sound of that person. The person who mortifies others does so even though others want to be happy and recoil from pain. That’s why I don’t like the sound of that person. The person who mortifies themselves and others does so even though both themselves and others want to be happy and recoil from pain. That’s why I don’t like the sound of that person. The person who doesn’t mortify either themselves or others—living without wishes, quenched, cooled, experiencing bliss, with self become divine—does not torment themselves or others, both of whom want to be happy and recoil from pain. That’s why I like the sound of that person. Well, now, sir, I must go. I have many duties, and much to do.” 

“Please,\marginnote{6.13} Pessa, go at your convenience.” And then Pessa the elephant driver’s son approved and agreed with what the Buddha said. He got up from his seat, bowed, and respectfully circled the Buddha, keeping him on his right, before leaving.\footnote{Presumably Kandaraka left too. This sutta has an odd structure, as the bulk of the teaching occurs after the protagonists leave. This has the ring of a genuinely random encounter. } 

Then,\marginnote{7.1} not long after he had left, the Buddha addressed the mendicants: “Mendicants, Pessa the elephant driver’s son is astute. He has great wisdom. If he had sat here an hour so that I could have analyzed these four people in detail, he would have greatly benefited. Still, even with this much he has already greatly benefited.” 

“Now\marginnote{7.6} is the time, Blessed One! Now is the time, Holy One! May the Buddha analyze these four people in detail. The mendicants will listen and remember it.” 

“Well\marginnote{7.8} then, mendicants, listen and apply your mind well, I will speak.” 

“Yes,\marginnote{7.9} sir,” they replied. The Buddha said this: 

“And\marginnote{8.1} what person mortifies themselves, committed to the practice of mortifying themselves? It’s when a person goes naked, ignoring conventions. They lick their hands, and don’t come or wait when called. They don’t consent to food brought to them, or food prepared for them, or an invitation for a meal.\footnote{For these practices, which were pursued by the Buddha before his awakening, see notes at \href{https://suttacentral.net/mn12/en/sujato\#45.1}{MN 12:45.1}. } They don’t receive anything from a pot or bowl; or from someone who keeps sheep, or who has a weapon or a shovel in their home; or where a couple is eating; or where there is a woman who is pregnant, breastfeeding, or who lives with a man; or where there’s a dog waiting or flies buzzing. They accept no fish or meat or beer or wine, and drink no fermented gruel. They go to just one house for alms, taking just one mouthful, or two houses and two mouthfuls, up to seven houses and seven mouthfuls. They feed on one saucer a day, two saucers a day, up to seven saucers a day. They eat once a day, once every second day, up to once a week, and so on, even up to once a fortnight. They live committed to the practice of eating food at set intervals. 

They\marginnote{8.6} eat herbs, millet, wild rice, poor rice, water lettuce, rice bran, scum from boiling rice, sesame flour, grass, or cow dung. They survive on forest roots and fruits, or eating fallen fruit. 

They\marginnote{8.7} wear robes of sunn hemp, mixed hemp, corpse-wrapping cloth, rags, lodh tree bark, antelope hide (whole or in strips), kusa grass, bark, wood-chips, human hair, horse-tail hair, or owls’ wings. They tear out their hair and beard, committed to this practice. They constantly stand, refusing seats. They squat, committed to the endeavor of squatting. They lie on a mat of thorns, making a mat of thorns their bed. They’re devoted to ritual bathing three times a day, including the evening. And so they live committed to practicing these various ways of mortifying and tormenting the body. This is called a person who mortifies themselves, being committed to the practice of mortifying themselves. 

And\marginnote{9.1} what person mortifies others, committed to the practice of mortifying others? It’s when a person is a slaughterer of sheep, pigs, or poultry, a hunter or trapper, a fisher, a bandit, an executioner, a butcher, a jailer, or someone with some other kind of cruel livelihood. This is called a person who mortifies others, being committed to the practice of mortifying others. 

And\marginnote{10.1} what person mortifies themselves and others, being committed to the practice of mortifying themselves and others? It’s when a person is an anointed aristocratic king or a well-to-do brahmin.\footnote{The same vivid description of ritual is found at \href{https://suttacentral.net/mn94/en/sujato\#12.2}{MN 94:12.2} and \href{https://suttacentral.net/an4.198/en/sujato\#6.2}{AN 4.198:6.2}. | The passage features many details close to the sacrifice as described in the Śatapatha \textsanskrit{Brāhmaṇa}, especially chapter 3, to which I refer below. Of course the attitude is the exact opposite. } He has a new ceremonial hall built to the east of the citadel. He shaves off his hair and beard, dresses in a rough antelope hide, and smears his body with ghee and oil. Scratching his back with antlers, he enters the hall with his chief queen and the brahmin high priest.\footnote{\textit{\textsanskrit{Santhāgāra}} is normally a hall to gather for business, a “town hall”. Here, however, the term draws on the idiom \textit{\textsanskrit{yajñasya} \textsanskrit{saṁsthā}}, (“completion/establishment of the sacrifice”), which is so common the term “sacrifice” may be left out as implied (eg. 13.2.4.2). Thus it is “a building for completing the sacrifice” (commentary: \textit{\textsanskrit{yaññasāla}}). | The hall for sacrifice, referred to simply as \textit{\textsanskrit{śālā}} (3.1.1), is oriented to the east, for “the east is the quarter of the gods” (3.1.1.6). | Shaving hair and beard before the sacrifice is described in detail (3.1.2). | The anointing is with \textit{\textsanskrit{navanīta}} (“fresh butter”, 3.1.3.7 ff.). | The sacrificer wears clean linen (3.1.2.13), but sits on antelope hide (3.2.1.1), scratching himself with the horn (3.2.1.31). | The wife’s fertility cleanses and heals (3.5.3.13, 3.8.2). } There he lies on the bare ground strewn with grass.\footnote{He lies down to sleep in the evening (3.2.2.22). | The spreading of grass is a regular feature of Vedic ritual (eg. 3.6.3.14). } The king feeds on the milk from one teat of a cow that has a calf of the same color. The chief queen feeds on the milk from the second teat. The brahmin high priest feeds on the milk from the third teat. The milk from the fourth teat is served to the sacred flame. The calf feeds on the remainder.\footnote{After shaving, the sacrificer consumes milk only (3.1.2.1), breaking his fast with milk cooked with rice or barley (3.2.2.10, 14). | Milk is drawn from specific teats (3.4.4.26, 9.5.1; \textsanskrit{Bṛhadāraṇyaka} \textsanskrit{Upaniṣad} 5.8.1). | Milk is cooked from a white cow with white calf (5.3.2.1), or in another rite from a black cow with white calf (9.2.3.30). } He says: ‘Slaughter this many bulls, bullocks, heifers, goats, rams, and horses for the sacrifice! Fell this many trees and reap this much grass for the sacrificial equipment!’\footnote{The slaughter of animal victims is described in detail (3.7.3). | The sacrificer ought not use the worldly term “slaughter” (Pali \textit{\textsanskrit{haññantu}} = Sanskrit \textit{jahi}), rather “make them agree”, “persuade” (\textit{\textsanskrit{saṁjñapaya}}, 3.8.1.15) so they look forward to being sacrificed (3.7.3.4). | Trees are felled to make the stake (\textit{\textsanskrit{yūpa}}) to which the animals are tied, while grass (\textit{darbha}) is strewn on the altar (3.6.4). } His bondservants, servants, and workers do their jobs under threat of punishment and danger, weeping with tearful faces. This is called a person who mortifies themselves and others, being committed to the practice of mortifying themselves and others. 

And\marginnote{11.1} what person doesn’t mortify either themselves or others, but lives without wishes, quenched, cooled, experiencing bliss, with self become divine? 

It’s\marginnote{12.1} when a Realized One arises in the world, perfected, a fully awakened Buddha, accomplished in knowledge and conduct, holy, knower of the world, supreme guide for those who wish to train, teacher of gods and humans, awakened, blessed. He has realized with his own insight this world—with its gods, \textsanskrit{Māras}, and divinities, this population with its ascetics and brahmins, gods and humans—and he makes it known to others. He proclaims a teaching that is good in the beginning, good in the middle, and good in the end, meaningful and well-phrased. And he reveals a spiritual practice that’s entirely full and pure. 

A\marginnote{13.1} householder hears that teaching, or a householder’s child, or someone reborn in a good family. They gain faith in the Realized One and reflect: ‘Life at home is cramped and dirty, life gone forth is wide open. It’s not easy for someone living at home to lead the spiritual life utterly full and pure, like a polished shell. Why don’t I shave off my hair and beard, dress in ocher robes, and go forth from the lay life to homelessness?’ After some time they give up a large or small fortune, and a large or small family circle. They shave off hair and beard, dress in ocher robes, and go forth from the lay life to homelessness. 

Once\marginnote{14.1} they’ve gone forth, they take up the training and livelihood of the mendicants. They give up killing living creatures, renouncing the rod and the sword. They’re scrupulous and kind, living full of sympathy for all living beings. They give up stealing. They take only what’s given, and expect only what’s given. They keep themselves clean by not thieving. They give up unchastity. They are celibate, set apart, avoiding the vulgar act of sex. They give up lying. They speak the truth and stick to the truth. They’re honest and dependable, and don’t trick the world with their words. They give up divisive speech. They don’t repeat in one place what they heard in another so as to divide people against each other. Instead, they reconcile those who are divided, supporting unity, delighting in harmony, loving harmony, speaking words that promote harmony. They give up harsh speech. They speak in a way that’s mellow, pleasing to the ear, lovely, going to the heart, polite, likable and agreeable to the people. They give up talking nonsense. Their words are timely, true, and meaningful, in line with the teaching and training. They say things at the right time which are valuable, reasonable, succinct, and beneficial. They refrain from injuring plants and seeds. They eat in one part of the day, abstaining from eating at night and food at the wrong time. They refrain from seeing shows of dancing, singing, and music . They refrain from beautifying and adorning themselves with garlands, fragrance, and makeup. They refrain from high and luxurious beds. They refrain from receiving gold and currency, raw grains, raw meat, women and girls, male and female bondservants, goats and sheep, chickens and pigs, elephants, cows, horses, and mares, and fields and land. They refrain from running errands and messages; buying and selling; falsifying weights, metals, or measures; bribery, fraud, cheating, and duplicity; mutilation, murder, abduction, banditry, plunder, and violence. 

They’re\marginnote{15.1} content with robes to look after the body and almsfood to look after the belly. Wherever they go, they set out taking only these things. They’re like a bird: wherever it flies, wings are its only burden. In the same way, a mendicant is content with robes to look after the body and almsfood to look after the belly. Wherever they go, they set out taking only these things. When they have this entire spectrum of noble ethics, they experience a blameless happiness inside themselves. 

When\marginnote{16.1} they see a sight with their eyes, they don’t get caught up in the features and details. If the faculty of sight were left unrestrained, bad unskillful qualities of covetousness and displeasure would become overwhelming. For this reason, they practice restraint, protecting the faculty of sight, and achieving its restraint. When they hear a sound with their ears … When they smell an odor with their nose … When they taste a flavor with their tongue … When they feel a touch with their body … When they know an idea with their mind, they don’t get caught up in the features and details. If the faculty of mind were left unrestrained, bad unskillful qualities of covetousness and displeasure would become overwhelming. For this reason, they practice restraint, protecting the faculty of mind, and achieving its restraint. When they have this noble sense restraint, they experience an unsullied bliss inside themselves. 

They\marginnote{17.1} act with situational awareness when going out and coming back; when looking ahead and aside; when bending and extending the limbs; when bearing the outer robe, bowl and robes; when eating, drinking, chewing, and tasting; when urinating and defecating; when walking, standing, sitting, sleeping, waking, speaking, and keeping silent. 

When\marginnote{18.1} they have this entire spectrum of noble ethics, this noble contentment, this noble sense restraint, and this noble mindfulness and situational awareness, they frequent a secluded lodging—a wilderness, the root of a tree, a hill, a ravine, a mountain cave, a charnel ground, a forest, the open air, a heap of straw. 

After\marginnote{19.1} the meal, they return from almsround, sit down cross-legged, set their body straight, and establish mindfulness in their presence. Giving up covetousness for the world, they meditate with a heart rid of covetousness, cleansing the mind of covetousness. Giving up ill will, they meditate with a mind rid of ill will, full of sympathy for all living beings, cleansing the mind of ill will and malevolence. Giving up dullness and drowsiness, they meditate with a mind rid of dullness and drowsiness, perceiving light, mindful and aware, cleansing the mind of dullness and drowsiness. Giving up restlessness and remorse, they meditate without restlessness, their mind peaceful inside, cleansing the mind of restlessness and remorse. Giving up doubt, they meditate having gone beyond doubt, not undecided about skillful qualities, cleansing the mind of doubt. 

They\marginnote{20.1} give up these five hindrances, corruptions of the heart that weaken wisdom. Then, quite secluded from sensual pleasures, secluded from unskillful qualities, they enter and remain in the first absorption, which has the rapture and bliss born of seclusion, while placing the mind and keeping it connected. 

As\marginnote{21.1} the placing of the mind and keeping it connected are stilled, they enter and remain in the second absorption, which has the rapture and bliss born of immersion, with internal clarity and mind at one, without placing the mind and keeping it connected. 

And\marginnote{22.1} with the fading away of rapture, they enter and remain in the third absorption, where they meditate with equanimity, mindful and aware, personally experiencing the bliss of which the noble ones declare, ‘Equanimous and mindful, one meditates in bliss.’ 

Giving\marginnote{23.1} up pleasure and pain, and ending former happiness and sadness, they enter and remain in the fourth absorption, without pleasure or pain, with pure equanimity and mindfulness. 

When\marginnote{24.1} their mind has become immersed in \textsanskrit{samādhi} like this—purified, bright, flawless, rid of corruptions, pliable, workable, steady, and imperturbable—they extend it toward recollection of past lives. They recollect many kinds of past lives, that is, one, two, three, four, five, ten, twenty, thirty, forty, fifty, a hundred, a thousand, a hundred thousand rebirths; many eons of the world contracting, many eons of the world expanding, many eons of the world contracting and expanding. They remember: ‘There, I was named this, my clan was that, I looked like this, and that was my food. This was how I felt pleasure and pain, and that was how my life ended. When I passed away from that place I was reborn somewhere else. There, too, I was named this, my clan was that, I looked like this, and that was my food. This was how I felt pleasure and pain, and that was how my life ended. When I passed away from that place I was reborn here.’ And so they recollect their many kinds of past lives, with features and details. 

When\marginnote{25.1} their mind has become immersed in \textsanskrit{samādhi} like this—purified, bright, flawless, rid of corruptions, pliable, workable, steady, and imperturbable—they extend it toward knowledge of the death and rebirth of sentient beings. With clairvoyance that is purified and superhuman, they see sentient beings passing away and being reborn—inferior and superior, beautiful and ugly, in a good place or a bad place. They understand how sentient beings are reborn according to their deeds: ‘These dear beings did bad things by way of body, speech, and mind. They denounced the noble ones; they had wrong view; and they chose to act out of that wrong view. When their body breaks up, after death, they’re reborn in a place of loss, a bad place, the underworld, hell. These dear beings, however, did good things by way of body, speech, and mind. They never denounced the noble ones; they had right view; and they chose to act out of that right view. When their body breaks up, after death, they’re reborn in a good place, a heavenly realm.’ And so, with clairvoyance that is purified and superhuman, they see sentient beings passing away and being reborn—inferior and superior, beautiful and ugly, in a good place or a bad place. They understand how sentient beings are reborn according to their deeds. 

When\marginnote{26.1} their mind has become immersed in \textsanskrit{samādhi} like this—purified, bright, flawless, rid of corruptions, pliable, workable, steady, and imperturbable—they extend it toward knowledge of the ending of defilements. They truly understand: ‘This is suffering’ … ‘This is the origin of suffering’ … ‘This is the cessation of suffering’ … ‘This is the practice that leads to the cessation of suffering’. 

They\marginnote{27.1} truly understand: ‘These are defilements’ … ‘This is the origin of defilements’ … ‘This is the cessation of defilements’ … ‘This is the practice that leads to the cessation of defilements’. Knowing and seeing like this, their mind is freed from the defilements of sensuality, desire to be reborn, and ignorance. When they’re freed, they know they’re freed. 

They\marginnote{27.4} understand: ‘Rebirth is ended, the spiritual journey has been completed, what had to be done has been done, there is nothing further for this place.’ 

This\marginnote{28.1} is called a person who neither mortifies themselves or others, being committed to the practice of not mortifying themselves or others. They live without wishes in this very life, quenched, cooled, experiencing bliss, with self become divine.” 

That\marginnote{28.3} is what the Buddha said. Satisfied, the mendicants approved what the Buddha said. 

%
\section*{{\suttatitleacronym MN 52}{\suttatitletranslation The Wealthy Citizen }{\suttatitleroot Aṭṭhakanāgarasutta}}
\addcontentsline{toc}{section}{\tocacronym{MN 52} \toctranslation{The Wealthy Citizen } \tocroot{Aṭṭhakanāgarasutta}}
\markboth{The Wealthy Citizen }{Aṭṭhakanāgarasutta}
\extramarks{MN 52}{MN 52}

\scevam{So\marginnote{1.1} I have heard.\footnote{This discourse is repeated at \href{https://suttacentral.net/an11.16/en/sujato}{AN 11.16}. } }At one time Venerable Ānanda was staying near \textsanskrit{Vesālī} in the little village of Beluva.\footnote{The Buddha spent his last rains retreat here (\href{https://suttacentral.net/sn47.9/en/sujato\#1.5}{SN 47.9:1.5}, \href{https://suttacentral.net/dn16/en/sujato\#2.22.3}{DN 16:2.22.3}). The fact that the Buddha is not mentioned and the setting at \textsanskrit{Pāṭaliputta} suggest that he had already passed away. } 

Now\marginnote{1.3} at that time the householder Dasama, a wealthy citizen, had arrived at \textsanskrit{Pāṭaliputta} on some business.\footnote{\textit{Dasama} means “tenth”, explained by the commentary as meaning he was tenth in line in his family. | \textit{\textsanskrit{Aṭṭhakanāgara}} is usually taken to be the name of a city, but neither the city nor the possessive form \textit{\textsanskrit{nāgara}} appear elsewhere. Since all we know of him is that he was very rich; since the Pali is in the habit of naming people after their prominent quality; and since we find elsewhere a man named after his wealth (Dhaniya at \href{https://suttacentral.net/snp1.2/en/sujato}{Snp 1.2}), perhaps we should take \textit{\textsanskrit{aṭṭhaka}} from \textit{artha} in the sense of “wealthy”, which agrees with the later Sanskrit sense of \textit{\textsanskrit{nāgara}} as “prominent citizen”. } He went to the Chicken Monastery, approached a certain mendicant, bowed, sat down to one side, and said to him,\footnote{With the rise of \textsanskrit{Pāṭaliputta}, this became an important monastery after the Buddha’s passing. Malalasekera (\emph{Dictionary of Pali Proper Names}) suggests it was the same monastery later renowned as \textsanskrit{Aśokārāma}. Ānanda stayed there often (\href{https://suttacentral.net/sn45.18/en/sujato}{SN 45.18}–20, \href{https://suttacentral.net/sn47.21/en/sujato}{SN 47.21}–3), which explains why Dasama expected to find him there. } “Sir, where is Venerable Ānanda now staying? For I want to see him.” 

“Householder,\marginnote{2.4} Venerable Ānanda is staying near \textsanskrit{Vesālī} in the little village of Beluva.” 

Then\marginnote{3.1} the householder Dasama, having concluded his business there, went to the little village of Beluva in \textsanskrit{Vesālī} to see Ānanda. He bowed, sat down to one side, and said to Ānanda:\footnote{Beluva was immediately north of \textsanskrit{Vesālī}, so about a day’s walk from \textsanskrit{Pāṭaliputta}. But Dasama could doubtless afford a carriage. } 

“Honorable\marginnote{3.2} Ānanda, is there one thing that has been rightly explained by the Blessed One—who knows and sees, the perfected one, the fully awakened Buddha—practicing which a diligent, keen, and resolute mendicant’s mind is freed, their defilements are ended, and they arrive at the supreme sanctuary from the yoke?”\footnote{A number of “one things” are taught throughout the suttas. It seems Dasama wanted a simple summary, but Ānanda wished to challenge him. } 

“There\marginnote{3.3} is, householder.” 

“And\marginnote{3.4} what is that one thing?” 

“Householder,\marginnote{4.1} it’s when a mendicant, quite secluded from sensual pleasures, secluded from unskillful qualities, enters and remains in the first absorption, which has the rapture and bliss born of seclusion, while placing the mind and keeping it connected. Then they reflect: ‘Even this first absorption is produced by choices and intentions.’\footnote{This illustrates the development of insight based on absorption. After emerging from the attainment, the meditator reflects on the nature of that experience, realizing that even the most blissful and peaceful of states is still “chosen and intended” (\textit{\textsanskrit{abhisaṅkhataṁ} \textsanskrit{abhisañcetayitaṁ}}), i.e. constructed by volition. } They understand: ‘But whatever is produced by choices and intentions is impermanent and liable to cessation.’ Abiding in that they attain the ending of defilements.\footnote{Insight must persist to bear fruit. } If they don’t attain the ending of defilements, with the ending of the five lower fetters they’re reborn spontaneously, because of their passion and love for that meditation. They are extinguished there, and are not liable to return from that world.\footnote{“Passion and love for that meditation” (\textit{teneva \textsanskrit{dhammarāgena} \textsanskrit{tāya} \textsanskrit{dhammanandiyā}}); here \textit{dhamma} means the state of meditation, i.e. the \textsanskrit{jhāna}. This is the “higher fetter” of \textit{\textsanskrit{rūparāga}}, the desire for birth in the \textsanskrit{Brahmā} realms of luminous form. } This is one thing that has been rightly explained by the Blessed One—who knows and sees, the perfected one, the fully awakened Buddha—practicing which a diligent, keen, and resolute mendicant’s mind is freed, their defilements are ended, and they arrive at the supreme sanctuary from the yoke. 

Furthermore,\marginnote{5.1} as the placing of the mind and keeping it connected are stilled, they enter and remain in the second absorption … third absorption … fourth absorption … 

Furthermore,\marginnote{8.1} a mendicant meditates spreading a heart full of love to one direction, and to the second, and to the third, and to the fourth. In the same way above, below, across, everywhere, all around, they spread a heart full of love to the whole world—abundant, expansive, limitless, free of enmity and ill will. Then they reflect: ‘Even this heart’s release by love is produced by choices and intentions.’ They understand: ‘But whatever is produced by choices and intentions is impermanent and liable to cessation.’ … 

Furthermore,\marginnote{9.1} a mendicant meditates spreading a heart full of compassion … rejoicing … equanimity … 

Furthermore,\marginnote{12.1} householder, a mendicant, going totally beyond perceptions of form, with the ending of perceptions of impingement, not focusing on perceptions of diversity, aware that ‘space is infinite’, enters and remains in the dimension of infinite space. Then they reflect: ‘Even this attainment of the dimension of infinite space is produced by choices and intentions.’ They understand: ‘But whatever is produced by choices and intentions is impermanent and liable to cessation.’ … 

Furthermore,\marginnote{13.1} a mendicant, going totally beyond the dimension of infinite space, aware that ‘consciousness is infinite’, enters and remains in the dimension of infinite consciousness. … 

Furthermore,\marginnote{14.1} a mendicant, going totally beyond the dimension of infinite consciousness, aware that ‘there is nothing at all’, enters and remains in the dimension of nothingness. Then they reflect: ‘Even this attainment of the dimension of nothingness is produced by choices and intentions.’ They understand: ‘But whatever is produced by choices and intentions is impermanent and liable to cessation.’ Abiding in that they attain the ending of defilements. If they don’t attain the ending of defilements, with the ending of the five lower fetters they’re reborn spontaneously because of their passion and love for that meditation. They are extinguished there, and are not liable to return from that world. This too is one thing that has been rightly explained by the Blessed One—who knows and sees, the perfected one, the fully awakened Buddha—practicing which a diligent, keen, and resolute mendicant’s mind is freed, their defilements are ended, and they arrive at the supreme sanctuary from the yoke.”\footnote{Ānanda omits the final formless attainment, neither perception nor non-perception, which is not regarded as a basis for insight due to its extreme subtlety. See too \href{https://suttacentral.net/mn64/en/sujato\#15.5}{MN 64:15.5}. } 

When\marginnote{15.1} he said this, the householder Dasama said to Venerable Ānanda, “Honorable Ānanda, suppose a person was looking for an entrance to a hidden treasure. And all at once they’d come across eleven entrances! In the same way, I was searching for the door to freedom from death. And all at once I found eleven doors to freedom from death for cultivation.\footnote{“For cultivation”: variants include \textit{\textsanskrit{bhāvanāya}}, \textit{\textsanskrit{savanāya}}, and \textit{\textsanskrit{sevanāya}}. The center point of these is \textit{\textsanskrit{sevanāya}}, which is a synonym of \textit{\textsanskrit{bhāvanāya}} and near-homonym of \textit{\textsanskrit{savanāya}}. Also, the sense of \textit{\textsanskrit{savanāya}} should be “for hearing”, but he has already heard them, so they are now “for cultivation”. } Suppose a person had a house with eleven doors. If the house caught fire they’d be able to flee to safety through any one of those doors. In the same way, I’m able to flee to safety through any one of these eleven doors to freedom from death.\footnote{\textsanskrit{Kaṭha} \textsanskrit{Upaniṣad} 2.2.2 speaks of “a city with eleven doors of the unborn”. } Sir, those of other religions seek a fee for the tutor. Why shouldn’t I make an offering to Venerable Ānanda?”\footnote{“Fee for the teacher”: \textit{\textsanskrit{ācariyadhana}}. | “Other religions” is in reference to the Brahmanical \textit{\textsanskrit{dakṣiṇā}}. } 

Then\marginnote{16.1} the householder Dasama, having assembled the \textsanskrit{Saṅgha} from \textsanskrit{Vesālī} and \textsanskrit{Pāṭaliputta}, served and satisfied them with his own hands with delicious fresh and cooked foods. He clothed each and every mendicant in a pair of garments, with a set of three robes for Ānanda. And he had a dwelling worth five hundred built for Ānanda.\footnote{This lavish demonstration of Dasama’s wealth is an early indication of the largess enjoyed by the Sangha in \textsanskrit{Pāṭaliputta}. Over a century later, under Ashoka, so many corrupt individuals entered the Sangha seeking profits that a Council had to be held to expel them. } 

%
\section*{{\suttatitleacronym MN 53}{\suttatitletranslation A Trainee }{\suttatitleroot Sekhasutta}}
\addcontentsline{toc}{section}{\tocacronym{MN 53} \toctranslation{A Trainee } \tocroot{Sekhasutta}}
\markboth{A Trainee }{Sekhasutta}
\extramarks{MN 53}{MN 53}

\scevam{So\marginnote{1.1} I have heard. }At one time the Buddha was staying in the land of the Sakyans, near Kapilavatthu in the Banyan Tree Monastery. 

Now\marginnote{2.1} at that time a new town hall had recently been constructed for the Sakyans of Kapilavatthu. It had not yet been occupied by an ascetic or brahmin or any person at all.\footnote{The completion of a town hall was celebrated by a talk for the Sakyans at \href{https://suttacentral.net/sn35.243/en/sujato\#1.2}{SN 35.243:1.2} and the Mallas at \href{https://suttacentral.net/dn33/en/sujato\#1.2.1}{DN 33:1.2.1}. Such halls were community meeting places that played a central role in civic society and communal decision-making in democratic republics such as the Mallas and the Sakyans. The Buddha’s participation is a sign of his support for their civic and democratic process. } Then the Sakyans of Kapilavatthu went up to the Buddha, bowed, sat down to one side, and said to him: 

“Sir,\marginnote{2.3} a new town hall has recently been constructed for the Sakyans of Kapilavatthu. It has not yet been occupied by an ascetic or brahmin or any person at all. May the Buddha be the first to use it, and only then will the Sakyans of Kapilavatthu use it. That would be for the lasting welfare and happiness of the Sakyans of Kapilavatthu.” The Buddha consented with silence. 

Then,\marginnote{3.2} knowing that the Buddha had consented, the Sakyans got up from their seat, bowed, and respectfully circled the Buddha, keeping him on their right. Then they went to the new town hall, where they spread carpets all over, prepared seats, set up a water jar, and placed an oil lamp. Then they went back to the Buddha, bowed, stood to one side, and told him of their preparations, saying, “Please, sir, come at your convenience.” 

Then\marginnote{4.1} the Buddha robed up and, taking his bowl and robe, went to the new town hall together with the \textsanskrit{Saṅgha} of mendicants. Having washed his feet he entered the town hall and sat against the central column facing east. The \textsanskrit{Saṅgha} of mendicants also washed their feet, entered the town hall, and sat against the west wall facing east, with the Buddha right in front of them. The Sakyans of Kapilavatthu also washed their feet, entered the town hall, and sat against the east wall facing west, with the Buddha right in front of them. 

The\marginnote{5.1} Buddha spent much of the night educating, encouraging, firing up, and inspiring the Sakyans with a Dhamma talk. Then he addressed Venerable Ānanda, “Ānanda, speak about the practicing trainee to the Sakyans of Kapilavatthu as you feel inspired. My back is sore, I’ll stretch it.” 

“Yes,\marginnote{5.5} sir,” Ānanda replied. And then the Buddha spread out his outer robe folded in four and laid down in the lion’s posture—on the right side, placing one foot on top of the other—mindful and aware, and focused on the time of getting up. 

Then\marginnote{6.1} Ānanda addressed \textsanskrit{Mahānāma} the Sakyan:\footnote{By convention, speech is addressed to the most senior member of the group. } 

“\textsanskrit{Mahānāma},\marginnote{6.2} a noble disciple is accomplished in ethics, guards the sense doors, eats in moderation, and is dedicated to wakefulness. They have seven good qualities, and they get the four absorptions—blissful meditations in this life that belong to the higher mind—when they want, without trouble or difficulty.\footnote{This is a version of the Gradual Training. It is generally oriented towards monastics, as the Gradual Training usually is, but the inclusion of the “seven good qualities” appears to be a nod towards adapting it for a lay audience. } 

And\marginnote{7.1} how is a noble disciple accomplished in ethics? It’s when a noble disciple is ethical, restrained in the monastic code, conducting themselves well and resorting for alms in suitable places. Seeing danger in the slightest fault, they keep the rules they’ve undertaken. That’s how a noble disciple is ethical. 

And\marginnote{8.1} how does a noble disciple guard the sense doors? When a noble disciple sees a sight with their eyes, they don’t get caught up in the features and details. If the faculty of sight were left unrestrained, bad unskillful qualities of covetousness and displeasure would become overwhelming. For this reason, they practice restraint, protecting the faculty of sight, and achieving its restraint. When they hear a sound with their ears … When they smell an odor with their nose … When they taste a flavor with their tongue … When they feel a touch with their body … When they know an idea with their mind, they don’t get caught up in the features and details. If the faculty of mind were left unrestrained, bad unskillful qualities of covetousness and displeasure would become overwhelming. For this reason, they practice restraint, protecting the faculty of mind, and achieving its restraint. That’s how a noble disciple guards the sense doors. 

And\marginnote{9.1} how does a noble disciple eat in moderation? It’s when a noble disciple reflects rationally on the food that they eat: ‘Not for fun, indulgence, adornment, or decoration, but only to sustain this body, to avoid harm, and to support spiritual practice. In this way, I shall put an end to old discomfort and not give rise to new discomfort, and I will have the means to keep going, blamelessness, and a comfortable abiding.’ That’s how a noble disciple eats in moderation. 

And\marginnote{10.1} how is a noble disciple dedicated to wakefulness? It’s when a noble disciple practices walking and sitting meditation by day, purifying their mind from obstacles. In the first watch of the night, they continue to practice walking and sitting meditation. In the middle watch, they lie down in the lion’s posture—on the right side, placing one foot on top of the other—mindful and aware, and focused on the time of getting up. In the last watch, they get up and continue to practice walking and sitting meditation, purifying their mind from obstacles. That’s how a noble disciple is dedicated to wakefulness. 

And\marginnote{11.1} how does a noble disciple have seven good qualities?\footnote{These are illustrated with a simile of a citadel at \href{https://suttacentral.net/an7.67/en/sujato}{AN 7.67}. They partly overlap with other factors of the Gradual Training; but then, these things are not meant to be exclusive. } It’s when a noble disciple has faith in the Realized One’s awakening: ‘That Blessed One is perfected, a fully awakened Buddha, accomplished in knowledge and conduct, holy, knower of the world, supreme guide for those who wish to train, teacher of gods and humans, awakened, blessed.’ 

They\marginnote{12.1} have a conscience. They’re conscientious about bad conduct by way of body, speech, and mind, and conscientious about having any bad, unskillful qualities. 

They\marginnote{13.1} exercise prudence. They’re prudent when it comes to bad conduct by way of body, speech, and mind, and prudent when it comes to acquiring any bad, unskillful qualities. 

They’re\marginnote{14.1} very learned, remembering and keeping what they’ve learned. These teachings are good in the beginning, good in the middle, and good in the end, meaningful and well-phrased, describing a spiritual practice that’s entirely full and pure. They are very learned in such teachings, remembering them, rehearsing them, mentally scrutinizing them, and comprehending them theoretically. 

They\marginnote{15.1} live with energy roused up for giving up unskillful qualities and embracing skillful qualities. They’re strong, staunchly vigorous, not slacking off when it comes to developing skillful qualities. 

They’re\marginnote{16.1} mindful. They have utmost mindfulness and alertness, and can remember and recall what was said and done long ago.\footnote{The pre-Buddhist sense of \textit{sati} is “memory”, while “mindfulness” evolved from the practice of “remembering” scripture, creating an uninterrupted flow state in the present. In this sense, mindfulness can be understood as the element of continuity that knits consciousness together in a coherent stream. Thus when practicing “mindfulness of breathing” one pays continuous attention to the breaths, not “forgetting” what one is doing. } 

They’re\marginnote{17.1} wise. They have the wisdom of arising and passing away which is noble, penetrative, and leads to the complete ending of suffering. That’s how a noble disciple has seven good qualities. 

And\marginnote{18.1} how does a noble disciple get the four absorptions—blissful meditations in this life that belong to the higher mind—when they want, without trouble or difficulty? It’s when a noble disciple, quite secluded from sensual pleasures, secluded from unskillful qualities, enters and remains in the first absorption … second absorption … third absorption … fourth absorption. That’s how a noble disciple gets the four absorptions—blissful meditations in this life that belong to the higher mind—when they want, without trouble or difficulty. 

When\marginnote{19.1} a noble disciple is accomplished in ethics, guards the sense doors, eats in moderation, and is dedicated to wakefulness; and they have seven good qualities, and they get the four absorptions—blissful meditations in this life that belong to the higher mind—when they want, without trouble or difficulty, they are called a noble disciple who is a practicing trainee. Their eggs are unspoiled, and they are capable of breaking out of their shell, becoming awakened, and achieving the supreme sanctuary from the yoke. Suppose there was a chicken with eight or ten or twelve eggs. And she properly sat on them to keep them warm and incubated. Even if that chicken doesn’t wish, ‘If only my chicks could break out of the eggshell with their claws and beak and hatch safely!’ Still they can break out and hatch safely. 

In\marginnote{19.5} the same way, when a noble disciple is practicing all these things they are called a noble disciple who is a practicing trainee. Their eggs are unspoiled, and they are capable of breaking out of their shell, becoming awakened, and achieving the supreme sanctuary from the yoke. 

Relying\marginnote{20.1} on this supreme purity of mindfulness and equanimity, that noble disciple recollects their many kinds of past lives.\footnote{The fourth \textsanskrit{jhāna}. } That is: one, two, three, four, five, ten, twenty, thirty, forty, fifty, a hundred, a thousand, a hundred thousand rebirths; many eons of the world contracting, many eons of the world expanding, many eons of the world contracting and expanding. … And so they recollect their many kinds of past lives, with features and details. This is their first breaking out, like a chick from an eggshell. 

Relying\marginnote{21.1} on this supreme purity of mindfulness and equanimity, that noble disciple, with clairvoyance that is purified and superhuman, sees sentient beings passing away and being reborn—inferior and superior, beautiful and ugly, in a good place or a bad place. … They understand how sentient beings are reborn according to their deeds. This is their second breaking out, like a chick from an eggshell. 

Relying\marginnote{22.1} on this supreme purity of mindfulness and equanimity, that noble disciple realizes the undefiled freedom of heart and freedom by wisdom in this very life. And they live having realized it with their own insight due to the ending of defilements. This is their third breaking out, like a chick from an eggshell. 

A\marginnote{23.1} noble disciple’s conduct includes the following: being accomplished in ethics, guarding the sense doors, moderation in eating, being dedicated to wakefulness, having seven good qualities, and getting the four absorptions when they want, without trouble or difficulty. 

A\marginnote{24.1} noble disciple’s knowledge includes the following: recollecting their past lives, clairvoyance that is purified and superhuman, and realizing the undefiled freedom of heart and freedom by wisdom in this very life due to the ending of defilements. 

This\marginnote{25.1} noble disciple is said to be ‘accomplished in knowledge’, and also ‘accomplished in conduct’, and also ‘accomplished in knowledge and conduct’. 

And\marginnote{25.2} the divinity \textsanskrit{Sanaṅkumāra} also spoke this verse:\footnote{\textsanskrit{Sanaṅkumāra} (“Everyoung”) became a Hindu deity closely associated with the worship of Krishna. He first appears in the seventh chapter of the \textsanskrit{Chāndogya} \textsanskrit{Upaniṣad}. The occasion he spoke this verse is recorded at \href{https://suttacentral.net/sn6.11/en/sujato}{SN 6.11}, and it is repeated several times in the suttas, for example at \href{https://suttacentral.net/dn3/en/sujato\#1.28.2}{DN 3:1.28.2}, where it has a better contextual justification. } 

\begin{verse}%
‘The\marginnote{25.3} aristocrat is best among people \\
who take clan as the standard. \\
But one accomplished in knowledge and conduct \\
is best among gods and humans.’ 

%
\end{verse}

And\marginnote{25.7} that verse was well sung by the Divinity \textsanskrit{Sanaṅkumāra}, not poorly sung; well spoken, not poorly spoken, beneficial, not harmful, and it was approved by the Buddha.” 

Then\marginnote{26.1} the Buddha got up and said to Venerable Ānanda, “Good, good, Ānanda! It’s good that you spoke to the Sakyans of Kapilavatthu about the practicing trainee.” 

This\marginnote{26.4} is what Venerable Ānanda said, and the teacher approved. Satisfied, the Sakyans of Kapilavatthu approved what Venerable Ānanda said. 

%
\section*{{\suttatitleacronym MN 54}{\suttatitletranslation With Potaliya the Householder }{\suttatitleroot Potaliyasutta}}
\addcontentsline{toc}{section}{\tocacronym{MN 54} \toctranslation{With Potaliya the Householder } \tocroot{Potaliyasutta}}
\markboth{With Potaliya the Householder }{Potaliyasutta}
\extramarks{MN 54}{MN 54}

\scevam{So\marginnote{1.1} I have heard. }At one time the Buddha was staying in the land of the \textsanskrit{Aṅguttarāpans}, near the town of theirs named \textsanskrit{Āpaṇa}.\footnote{\textsanskrit{Aṅga} was the country to that controlled the Ganges to the east of both Magadha and \textsanskrit{Vesālī}. The region of \textsanskrit{Aṅga} north of both the Ganges and the \textsanskrit{Mahī} was known as \textsanskrit{Aṅguttarāpa}, “\textsanskrit{Aṅga} North of the Water”. The market town of \textsanskrit{Āpaṇa} was evidently the chief settlement of the region. } 

Then\marginnote{2.1} the Buddha robed up in the morning and, taking his bowl and robe, entered \textsanskrit{Āpaṇa} for alms. He wandered for alms in \textsanskrit{Āpaṇa}. After the meal, on his return from almsround, he went to a certain forest grove for the day’s meditation. Having plunged deep into it, he sat at the root of a certain tree to meditate. 

Potaliya\marginnote{3.1} the householder also approached that forest grove while going for a walk. He was well dressed in sarong and cloak, with parasol and sandals. Having plunged deep into it, he went up to the Buddha, and exchanged greetings with him.\footnote{A wanderer called Potaliya appears at \href{https://suttacentral.net/an4.100/en/sujato}{AN 4.100}. | “Sarong” (\textit{\textsanskrit{nivāsana}}) is lay attire at \href{https://suttacentral.net/an3.39/en/sujato\#1.4}{AN 3.39:1.4} and monastic at \href{https://suttacentral.net/pli-tv-kd1/en/sujato\#25.9.1}{Kd 1:25.9.1}, while “cloak” (\textit{\textsanskrit{pāvuraṇa}}) is lay attire at \href{https://suttacentral.net/sn7.14/en/sujato\#1.2}{SN 7.14:1.2} and monastic at \href{https://suttacentral.net/pli-tv-bi-vb-np11/en/sujato}{Bi NP 11} and \href{https://suttacentral.net/pli-tv-bi-vb-np12/en/sujato}{Bi NP 12}. These are evidently similar to the monastic \textit{\textsanskrit{antaravāsaka}} and \textit{\textsanskrit{saṅghāṭi}}. | Parasols and sandals were considered inappropriate luxuries for monastics (\href{https://suttacentral.net/pli-tv-bi-vb-pc84/en/sujato}{Bi Pc 84}, \href{https://suttacentral.net/pli-tv-kd15/en/sujato\#23.2.19}{Kd 15:23.2.19}, \href{https://suttacentral.net/pli-tv-kd5/en/sujato\#12.1.5}{Kd 5:12.1.5}), and it was considered disrespectful to teach someone thus attired (\href{https://suttacentral.net/pli-tv-bu-vb-sk57/en/sujato}{Bu Sk 57}, \href{https://suttacentral.net/pli-tv-bu-vb-sk62/en/sujato}{Bu Sk 62}). There was, however, an exemption in all cases for health reasons, and monastics were allowed sandals outside of a village. } When the greetings and polite conversation were over, he stood to one side, and the Buddha said to him, “There are seats, householder. Please sit if you wish.” 

When\marginnote{3.4} he said this, Potaliya was angry and upset. Thinking, “The ascetic Gotama addresses me as ‘householder’!” he stayed silent. 

For\marginnote{3.5} a second time … and a third time the Buddha said to him, “There are seats, householder. Please sit if you wish.” 

When\marginnote{3.8} he said this, Potaliya was angry and upset. Thinking, “The ascetic Gotama addresses me as ‘householder’!” he said to the Buddha, “Mister Gotama, it is neither proper nor appropriate for you to address me as ‘householder’.” 

“Well,\marginnote{3.10} householder, you have the features, attributes, and signs of a householder.”\footnote{Monastics were criticized as being “just like householders” for wearing sandals (\href{https://suttacentral.net/pli-tv-kd5/en/sujato\#12.1.3}{Kd 5:12.1.3}) and sunshades (\href{https://suttacentral.net/pli-tv-kd15/en/sujato\#23.2.8}{Kd 15:23.2.8}, \href{https://suttacentral.net/pli-tv-bi-vb-pc8/en/sujato}{Bi Pc 8}). } 

“Mister\marginnote{3.11} Gotama, it’s because I have refused all work and cut off all business.” 

“Householder,\marginnote{3.12} in what way have you refused all work and cut off all business?” 

“Mister\marginnote{3.13} Gotama, all the money, grain, gold, and silver I used to have has been handed over to my children as their inheritance. And in this matter I do not advise or reprimand them, but live with nothing more than food and clothes.\footnote{This explains the cutting off of “business” (\textit{\textsanskrit{vohāra}}) as not getting involved in the family money. | The word “reprimand” (\textit{\textsanskrit{anupavādī}}) is adapted from its worldly sense here to an ethical one below (\href{https://suttacentral.net/mn54/en/sujato\#6.4}{MN 54:6.4}). } That’s how I have refused all work and cut off all business.” 

“The\marginnote{3.15} cutting off of business as you describe it is one thing, householder, but the cutting off of business in the noble one’s training is quite different.” 

“But\marginnote{3.16} what, sir, is cutting off of business in the noble one’s training? Sir, please teach me this.” 

“Well\marginnote{3.18} then, householder, listen and apply your mind well, I will speak.”\footnote{The Buddha appears to be breaking his own rule against teaching someone with sandals and parasol. The Chinese parallel avoids this awkward situation by saying that Potaliya had set aside his parasol and sandals (MA 203 at T i 773a29). } 

“Yes,\marginnote{3.19} sir,” said Potaliya. 

The\marginnote{4.1} Buddha said this: 

“Householder,\marginnote{4.2} these eight things lead to the cutting off of business in the noble one’s training.\footnote{While all these items are standard, this list of eight does not appear elsewhere. It must have been a personal teaching for Potaliya. } What eight? Killing living creatures should be given up, relying on not killing living creatures. Stealing should be given up, relying on not stealing. Lying should be given up, relying on speaking the truth. Divisive speech should be given up, relying on speech that isn’t divisive. Greed and lust should be given up, relying on not being greedy and lustful. Blaming and insulting should be given up, relying on not blaming and not insulting. Anger and distress should be given up, relying on not being angry and distressed. Arrogance should be given up, relying on not being arrogant.\footnote{The Buddha leaves Potaliya’s cardinal sin for last. } These are the eight things—stated in brief without being analyzed in detail—that lead to the cutting off of business in the noble one’s training.” 

“Sir,\marginnote{5.1} please teach me these eight things in detail out of sympathy.” 

“Well\marginnote{5.2} then, householder, listen and apply your mind well, I will speak.” 

“Yes,\marginnote{5.3} sir,” said Potaliya. The Buddha said this: 

“‘Killing\marginnote{6.1} living creatures should be given up, relying on not killing living creatures.’ That’s what I said, but why did I say it? It’s when a noble disciple reflects:\footnote{The practice of ethics arises from reasoned contemplation rather than from commandments or culture. } ‘I am practicing to give up and cut off the fetters that might cause me to kill living creatures. But if I were to kill living creatures, because of that I would reprimand myself; sensible people, after examination, would criticize me; and when my body breaks up, after death, I could expect to be reborn in a bad place. And killing living creatures is itself a fetter and a hindrance. The distressing and feverish defilements that might arise because of killing living creatures do not occur in someone who does not kill living creatures.’ ‘Killing living creatures should be given up, relying on not killing living creatures.’ That’s what I said, and this is why I said it. 

‘Stealing\marginnote{7.1} … lying … divisive speech … greed and lust … blaming and insulting … anger and distress … 

Arrogance\marginnote{13.1} should be given up, relying on not being arrogant.’ That’s what I said, but why did I say it? It’s when a noble disciple reflects: ‘I am practicing to give up and cut off the fetters that might cause me to be arrogant. But if I were to be arrogant, because of that I would reprimand myself; sensible people, after examination, would criticize me; and when my body breaks up, after death, I could expect to be reborn in a bad place. And arrogance is itself a fetter and a hindrance. The distressing and feverish defilements that might arise because of arrogance do not occur in someone who is not arrogant.’ ‘Arrogance should be given up by not being arrogant.’ That’s what I said, and this is why I said it. 

These\marginnote{14.1} are the eight things—stated in brief and analyzed in detail—that lead to the cutting off of business in the noble one’s training. But just this much does not constitute the cutting off of business in each and every respect in the noble one’s training.” 

“But,\marginnote{14.3} sir, how is there the cutting off of business in each and every respect in the noble one’s training? Sir, please teach me this.” 

“Well\marginnote{14.5} then, householder, listen and apply your mind well, I will speak.” 

“Yes,\marginnote{14.6} sir,” said Potaliya. The Buddha said this: 

\subsection*{1. The Dangers of Sensual Pleasures }

“Householder,\marginnote{15.1} suppose a dog weak with hunger was hanging around a butcher’s shop.\footnote{This list of similes is also found at \href{https://suttacentral.net/mn22/en/sujato\#3.8}{MN 22:3.8}, \href{https://suttacentral.net/an5.76/en/sujato\#11.2}{AN 5.76:11.2}, and \href{https://suttacentral.net/thig16.1/en/sujato\#41.1}{Thig 16.1:41.1}, as well as in various parallels to these texts. There, the list is expanded from seven to ten (with minor variations in parallels), and it is quoted as a familiar teaching without any need for explanation. Several of the similes require explanation to make sense; for example “fruit on a tree”, by itself, hardly evokes suffering. Note too that the explanation below at \href{https://suttacentral.net/mn54/en/sujato\#20.2}{MN 54:20.2} mentions the town of \textsanskrit{Āpaṇa} where the discourse is set. These considerations suggest that the list of similes, together with their explanations, first appeared here. } A deft butcher or their apprentice would toss them a skeleton scraped clean of flesh and smeared in blood. What do you think, householder? Gnawing on such a fleshless skeleton, would that dog still get rid of its hunger?” 

“No,\marginnote{15.5} sir. Why not? Because that skeleton is scraped clean of flesh and smeared in blood. That dog will eventually get weary and frustrated.” 

“In\marginnote{15.9} the same way, a noble disciple reflects: ‘With the simile of a skeleton the Buddha said that sensual pleasures give little gratification and much suffering and distress, and they are all the more full of drawbacks.’ Having truly seen this with right understanding, they reject equanimity based on diversity and develop only the equanimity based on unity, where all kinds of grasping to the worldly pleasures of the flesh cease without anything left over. 

Suppose\marginnote{16.1} a vulture or a crow or a hawk was to grab a scrap of meat and fly away.\footnote{In similar passages (\href{https://suttacentral.net/sn19.1/en/sujato\#3.2}{SN 19.1:3.2}, \href{https://suttacentral.net/sn55.21/en/sujato\#2.4}{SN 55.21:2.4}, \href{https://suttacentral.net/pli-tv-bu-vb-pj4/en/sujato\#9.3.2}{Bu Pj 4:9.3.2}) we find “crow” (\textit{\textsanskrit{kāka}}) rather than text’s “heron” (\textit{\textsanskrit{kaṅka}}). The Chinese parallel at MA 203 does indeed have \langlzh{烏} (“crow”). Given the behavior described here, I think it’s likely the original was, in fact, crow. } Other vultures, crows, and hawks would keep chasing it, pecking and clawing.\footnote{The verbs here are uncertain and readings vary considerably. Compare Rig Veda 6.46.14 where horse-racers compete “like birds over raw flesh”. } What do you think, householder? If that vulture, crow, or hawk doesn’t quickly let go of that scrap of meat, wouldn’t that result in death or deadly suffering for them?” 

“Yes,\marginnote{16.5} sir.” … 

“Suppose\marginnote{17.1} a person carrying a blazing grass torch was to walk against the wind. What do you think, householder? If that person doesn’t quickly let go of that blazing grass torch, wouldn’t they burn their hands or arm or other major or minor limb, resulting in death or deadly suffering for them?” 

“Yes,\marginnote{17.4} sir.” … 

“Suppose\marginnote{18.1} there was a pit of glowing coals deeper than a man’s height, full of glowing coals that neither flamed nor smoked. Then a person would come along who wants to live and doesn’t want to die, who wants to be happy and recoils from pain. Two strong men would grab them by the arms and drag them towards the pit of glowing coals. What do you think, householder? Wouldn’t that person writhe and struggle to and fro?” 

“Yes,\marginnote{18.6} sir. Why is that? For that person knows: ‘If I fall in that pit of glowing coals, that’d result in my death or deadly pain.’” … 

“Suppose\marginnote{19.1} a person was to see delightful parks, woods, meadows, and lotus ponds in a dream. But when they woke they couldn’t see them at all. … 

Suppose\marginnote{20.1} a man had borrowed some goods—a gentleman’s carriage and fine jeweled earrings—and preceded and surrounded by these he proceeded through the middle of \textsanskrit{Āpaṇa}. When people saw him they’d say: ‘This must be a wealthy man! For that’s how the wealthy enjoy their wealth.’ But when the owners saw him, they’d take back what was theirs. What do you think? Would that be enough for that man to get upset?” 

“Yes,\marginnote{20.7} sir. Why is that? Because the owners took back what was theirs.” … 

“Suppose\marginnote{21.1} there was a dense forest grove not far from a town or village. And there was a tree laden with fruit, yet none of the fruit had fallen to the ground. And along came a person in need of fruit, wandering in search of fruit. Having plunged deep into that forest grove, they’d see that tree laden with fruit. They’d think: ‘That tree is laden with fruit, yet none of the fruit has fallen to the ground. But I know how to climb a tree. Why don’t I climb the tree, eat as much as I like, then fill my pouch?’ And that’s what they’d do. And along would come a second person in need of fruit, wandering in search of fruit, carrying a sharp axe. Having plunged deep into that forest grove, they’d see that tree laden with fruit. They’d think: ‘That tree is laden with fruit, yet none of the fruit has fallen to the ground. But I don’t know how to climb a tree. Why don’t I chop this tree down at the root, eat as much as I like, then fill my pouch?’ And so they’d chop the tree down at the root. What do you think, householder? If the first person, who climbed the tree, doesn’t quickly come down, when that tree fell wouldn’t they break their hand or arm or other major or minor limb, resulting in death or deadly suffering for them?” 

“Yes,\marginnote{21.19} sir.” 

“In\marginnote{21.20} the same way, a noble disciple reflects: ‘With the simile of the fruit tree the Buddha said that sensual pleasures give little gratification and much suffering and distress, and they are all the more full of drawbacks.’ Having truly seen this with right understanding, they reject equanimity based on diversity and develop only the equanimity based on unity, where all kinds of grasping to the worldly pleasures of the flesh cease without anything left over.\footnote{“Equanimity based on diversity” (\textit{\textsanskrit{upekkhā} \textsanskrit{nānattā}}) is indifference regarding the senses (\href{https://suttacentral.net/mn137/en/sujato\#18.1}{MN 137:18.1}). Here it also relates to Potaliya’s refusal to be involved in family business. | “Equanimity based on unity” refers to the advanced meditations based on the fourth \textit{\textsanskrit{jhāna}} (\href{https://suttacentral.net/mn137/en/sujato\#19.1}{MN 137:19.1}). } 

Relying\marginnote{22.1} on this supreme purity of mindfulness and equanimity, that noble disciple recollects their many kinds of past lives. That is: one, two, three, four, five, ten, twenty, thirty, forty, fifty, a hundred, a thousand, a hundred thousand rebirths; many eons of the world contracting, many eons of the world expanding, many eons of the world contracting and expanding. … They recollect their many kinds of past lives, with features and details. 

Relying\marginnote{23.1} on this supreme purity of mindfulness and equanimity, that noble disciple, with clairvoyance that is purified and superhuman, sees sentient beings passing away and being reborn—inferior and superior, beautiful and ugly, in a good place or a bad place. … They understand how sentient beings are reborn according to their deeds. 

Relying\marginnote{24.1} on this supreme purity of mindfulness and equanimity, that noble disciple realizes the undefiled freedom of heart and freedom by wisdom in this very life. And they live having realized it with their own insight due to the ending of defilements. 

That’s\marginnote{25.1} how there is the cutting off of business in each and every respect in the noble one’s training. 

What\marginnote{25.2} do you think, householder? Do you regard yourself as having cut off business in a way comparable to the cutting off of business in each and every respect in the noble one’s training?” 

“Who\marginnote{25.4} am I compared to one who has cut off business in each and every respect in the noble one’s training? I am far from that. Sir, I used to think that the wanderers following other religions were thoroughbreds, and I fed them and treated them accordingly, but they were not actually thoroughbreds. I thought that the mendicants were not thoroughbreds, and I fed them and treated them accordingly, but they actually were thoroughbreds.\footnote{Notice how \textit{\textsanskrit{bhikkhū}} (“mendicants”) specifically indicates Buddhist monastics. } But now I shall understand that the wanderers following other religions are not actually thoroughbreds, and I will feed them and treat them accordingly. And I shall understand that the mendicants actually are thoroughbreds, and I will feed them and treat them accordingly. The Buddha has inspired me to have love, confidence, and respect for ascetics! 

Excellent,\marginnote{26.1} sir! Excellent! As if he were righting the overturned, or revealing the hidden, or pointing out the path to the lost, or lighting a lamp in the dark so people with clear eyes can see what’s there, the Buddha has made the teaching clear in many ways. I go for refuge to the Buddha, to the teaching, and to the mendicant \textsanskrit{Saṅgha}. From this day forth, may the Buddha remember me as a lay follower who has gone for refuge for life.” 

%
\section*{{\suttatitleacronym MN 55}{\suttatitletranslation With Jīvaka }{\suttatitleroot Jīvakasutta}}
\addcontentsline{toc}{section}{\tocacronym{MN 55} \toctranslation{With Jīvaka } \tocroot{Jīvakasutta}}
\markboth{With Jīvaka }{Jīvakasutta}
\extramarks{MN 55}{MN 55}

\scevam{So\marginnote{1.1} I have heard. }At one time the Buddha was staying near \textsanskrit{Rājagaha} in the Mango Grove of \textsanskrit{Jīvaka} \textsanskrit{Komārabhacca}.\footnote{\textsanskrit{Jīvaka} was the Buddha’s doctor. His story is told in \href{https://suttacentral.net/pli-tv-kd8/en/sujato}{Kd 8}, where we learn that he was abandoned on a trash heap by his mother \textsanskrit{Sālavatī}, only to be discovered and raised by Prince Abhaya of Magadha, a Jain \href{https://suttacentral.net/mn58/en/sujato}{MN 58}. He was called “Survivior” (\textit{\textsanskrit{jīvaka}}) and “Raised by a Prince” (\textit{\textsanskrit{komārabhacca}}). In later Ayurveda, \textit{\textsanskrit{kaumārabhṛtya}} means “paediatrics”, and a lost treatise on that subject is attributed to \textsanskrit{Jīvaka} (\textsanskrit{Jīvakatantra}). } 

Then\marginnote{2.1} \textsanskrit{Jīvaka} went up to the Buddha, bowed, sat down to one side, and said to the Buddha: 

“Sir,\marginnote{3.1} I have heard this: ‘They slaughter living creatures specially for the ascetic Gotama. The ascetic Gotama knowingly eats meat prepared on his behalf: this is a deed he caused.’\footnote{This was an accusation made against the Buddha by Jains (\href{https://suttacentral.net/an8.12/en/sujato\#31.2}{AN 8.12:31.2}, \href{https://suttacentral.net/pli-tv-kd6/en/sujato\#31.13.1}{Kd 6:31.13.1}). They objected only to eating meat “killed on purpose for him”, and so by implication not all eating of meat. This is consistent with several Jain scriptural passages indicating that ascetics may, under certain strictures, eat meat (\textsanskrit{Ācaraṅgasūtra} 2.1.10.5–6, \textsanskrit{Bhagavatīsūtra} 557 (p. 1261a), \textsanskrit{Kalpasūtra} (\textsanskrit{Sāmācārī} 17), \textsanskrit{Dasaveyāliyasutta} 5–1.73). } I trust that those who say this repeat what the Buddha has said, and do not misrepresent him with an untruth? Is their explanation in line with the teaching? Are there any legitimate grounds for rebuttal and criticism?” 

“\textsanskrit{Jīvaka},\marginnote{4.1} those who say this do not repeat what I have said. They misrepresent me with what is false and untrue. 

In\marginnote{5.1} three cases I say that meat may not be eaten: it’s seen, heard, or suspected.\footnote{The relevant Vinaya rule makes the meaning clear: “When you know it was killed on purpose for you” (\href{https://suttacentral.net/pli-tv-kd6/en/sujato\#31.14.4}{Kd 6:31.14.4}). To allay suspicion of improper meat, mendicants are also required to check the provenance of meat before eating (\href{https://suttacentral.net/pli-tv-kd6/en/sujato\#23.9.9}{Kd 6:23.9.9}). } These are three cases in which meat may not be eaten. 

In\marginnote{5.4} three cases I say that meat may be eaten: it’s not seen, heard, or suspected. These are three cases in which meat may be eaten. 

Take\marginnote{6.1} the case of a mendicant living supported by a town or village. They meditate spreading a heart full of love to one direction, and to the second, and to the third, and to the fourth. In the same way above, below, across, everywhere, all around, they spread a heart full of love to the whole world—abundant, expansive, limitless, free of enmity and ill will. A householder or their child approaches and invites them for the next day’s meal. The mendicant accepts if they want. 

When\marginnote{6.5} the night has passed, they robe up in the morning, take their bowl and robe, and approach that householder’s home, where they sit on the seat spread out. That householder or their child serves them with delicious almsfood. It never occurs to them, ‘It’s so good that this householder serves me with delicious almsfood! I hope they serve me with such delicious almsfood in the future!’ They don’t think that. They eat that almsfood untied, uninfatuated, unattached, seeing the drawback, and understanding the escape. 

What\marginnote{6.12} do you think, \textsanskrit{Jīvaka}? At that time is that mendicant intending to hurt themselves, hurt others, or hurt both?” 

“No,\marginnote{6.14} sir.” 

“Aren’t\marginnote{6.15} they eating blameless food at that time?” 

“Yes,\marginnote{7.1} sir. Sir, I have heard that The Divinity abides in love. Now, I’ve seen the Buddha with my own eyes,\footnote{Reading \textit{\textsanskrit{taṁ} … \textsanskrit{idaṁ}} as the accusative of relation, and \textit{me} as instrumental, for which compare \textit{yena \textsanskrit{brahmā} \textsanskrit{sakkhidiṭṭho}} at \href{https://suttacentral.net/dn13/en/sujato\#12.1}{DN 13:12.1}. The point of the passage is that he has merely \emph{heard} that \textsanskrit{Brahmā} abides in love, but he \emph{witnesses} that the Buddha does so. } and it is the Buddha who truly abides in love.” 

“Any\marginnote{7.6} greed, hate, or delusion that might give rise to ill will has been given up by the Realized One, cut off at the root, made like a palm stump, obliterated, and is unable to arise in the future.\footnote{Greed, hate, and delusion are the fundamental underlying defilements that give rise to “ill will”, the opposite of love. } If that’s what you were referring to, I acknowledge it.” 

“That’s\marginnote{7.8} exactly what I was referring to.” 

“Take\marginnote{8{-}10.1} the case, \textsanskrit{Jīvaka}, of a mendicant living supported by a town or village. They meditate spreading a heart full of compassion … 

They\marginnote{8{-}10.3} meditate spreading a heart full of rejoicing … 

They\marginnote{8{-}10.4} meditate spreading a heart full of equanimity to one direction, and to the second, and to the third, and to the fourth. In the same way above, below, across, everywhere, all around, they spread a heart full of equanimity to the whole world—abundant, expansive, limitless, free of enmity and ill will. A householder or their child approaches and invites them for the next day’s meal. The mendicant accepts if they want. 

When\marginnote{8{-}10.8} the night has passed, they robe up in the morning, take their bowl and robe, and approach that householder’s home, where they sit on the seat spread out. That householder or their child serves them with delicious almsfood. It never occurs to them, ‘It’s so good that this householder serves me with delicious almsfood! I hope they serve me with such delicious almsfood in the future!’ They don’t think that. They eat that almsfood untied, uninfatuated, unattached, seeing the drawback, and understanding the escape. 

What\marginnote{8{-}10.15} do you think, \textsanskrit{Jīvaka}? At that time is that mendicant intending to hurt themselves, hurt others, or hurt both?” 

“No,\marginnote{8{-}10.17} sir.” 

“Aren’t\marginnote{8{-}10.18} they eating blameless food at that time?” 

“Yes,\marginnote{11.1} sir. Sir, I have heard that The Divinity abides in equanimity. Now, I’ve seen the Buddha with my own eyes, and it is the Buddha who truly abides in equanimity.” 

“Any\marginnote{11.6} greed, hate, or delusion that might give rise to cruelty, discontent, or repulsion has been given up by the Realized One, cut off at the root, made like a palm stump, obliterated, and is unable to arise in the future.\footnote{Just as ill will is the opposite of love, “cruelty” (\textit{\textsanskrit{vihesā}}), “discontent” (\textit{arati}), and “repulsion” (\textit{\textsanskrit{paṭigha}}) are the respective opposites of compassion, rejoicing, and equanimity (\href{https://suttacentral.net/mn62/en/sujato\#18.2}{MN 62:18.2} ff). } If that’s what you were referring to, I acknowledge it.” 

“That’s\marginnote{11.8} exactly what I was referring to.” 

“\textsanskrit{Jīvaka},\marginnote{12.1} anyone who slaughters a living creature specially for the Realized One or the Realized One’s disciple creates much wickedness for five reasons. 

When\marginnote{12.2} they say: ‘Go, fetch that living creature,’ this is the first reason. 

When\marginnote{12.4} that living creature experiences pain and sadness as it’s led along by a collar, this is the second reason. 

When\marginnote{12.5} they say: ‘Go, slaughter that living creature,’ this is the third reason. 

When\marginnote{12.7} that living creature experiences pain and sadness as it’s being slaughtered, this is the fourth reason. 

When\marginnote{12.8} they serve the Realized One or the Realized One’s disciple with unallowable food, this is the fifth reason. 

Anyone\marginnote{12.9} who slaughters a living creature specially for the Realized One or the Realized One’s disciple creates much wickedness for five reasons.” 

When\marginnote{13.1} he had spoken, \textsanskrit{Jīvaka} said to the Buddha: “It’s incredible, sir, it’s amazing! The mendicants indeed eat allowable food. The mendicants indeed eat blameless food. Excellent, sir! Excellent! … From this day forth, may the Buddha remember me as a lay follower who has gone for refuge for life.”\footnote{This sutta marks the occasion of \textsanskrit{Jīvaka}’s conversion. It would seem that he had previously harbored doubts due to his Jain upbringing. Elsewhere he appears as a devotee of the Buddha. \href{https://suttacentral.net/an1.256/en/sujato\#1.1}{AN 1.256:1.1} records that he was the foremost disciple with “confidence in a person”, establishing that he was a stream-enterer. The occasion of his stream-entry, however, is not recorded in Pali. } 

%
\section*{{\suttatitleacronym MN 56}{\suttatitletranslation With Upāli }{\suttatitleroot Upālisutta}}
\addcontentsline{toc}{section}{\tocacronym{MN 56} \toctranslation{With Upāli } \tocroot{Upālisutta}}
\markboth{With Upāli }{Upālisutta}
\extramarks{MN 56}{MN 56}

\scevam{So\marginnote{1.1} I have heard. }At one time the Buddha was staying near \textsanskrit{Nāḷandā} in \textsanskrit{Pāvārika}’s mango grove.\footnote{\textsanskrit{Pāvārika} means “blanket-weaver”. Places were often named for their primary economic activity. Probably there was a community of blanket weavers and the park was owned by their leading merchant. } 

At\marginnote{2.1} that time the Jain ascetic of the \textsanskrit{Ñātika} clan was residing at \textsanskrit{Nāḷandā} together with a large assembly of Jain ascetics.\footnote{Jain tradition holds that \textsanskrit{Mahāvīra} passed at a \textsanskrit{Pāvā} near \textsanskrit{Nāḷandā}. Buddhist texts know only the \textsanskrit{Pāvā} of the Mallians to the north. Evidently there has been some confusion, and perhaps “\textsanskrit{Pāvārika}’s mango grove”, the site of discussions with Jains here and at \href{https://suttacentral.net/sn42.8/en/sujato}{SN 42.8}, is the \textsanskrit{Pāvā} known to the Jains. Note, however, that the earliest Jain source for \textsanskrit{Mahāvīra}’s passing, the Kalpasutra, does not say where \textsanskrit{Pāvā} is. But it does say the events were commemorated by the rulers of \textsanskrit{Kāsī} and Kosala, and the Mallians and the \textsanskrit{Licchavīs}. The absence of Magadha and the presence of Malla sit better with the location of \textsanskrit{Pāvā} in Malla rather than Magadha. } Then the Jain ascetic \textsanskrit{Dīgha} \textsanskrit{Tapassī} wandered for alms in \textsanskrit{Nāḷandā}. After the meal, on his return from almsround, he went to \textsanskrit{Pāvārika}’s mango grove. There he approached the Buddha, and exchanged greetings with him.\footnote{In Sanskrit, \textsanskrit{Dīrghatapasvin} was a common name, or rather epithet, for ascetics, meaning “long fervor” in honor of lengthy dedication to practices such as starvation. It is presumably the \textsanskrit{Dīgha} \textsanskrit{Tapassī} of this sutta who is inscribed in a Buddhist stupa at Bharhut teaching a group of students. At least one of his students appears to be holding a small stick, which may be a pictorial representation of the subject of this discourse. } 

When\marginnote{2.3} the greetings and polite conversation were over, he stood to one side. The Buddha said to him, “There are seats, \textsanskrit{Tapassī}. Please sit if you wish.” 

When\marginnote{3.1} he said this, \textsanskrit{Dīgha} \textsanskrit{Tapassī} took a low seat and sat to one side. The Buddha said to him, “\textsanskrit{Tapassī}, how many kinds of deed does the Jain ascetic of the \textsanskrit{Ñātika} clan describe for performing bad deeds?” 

“Reverend\marginnote{3.4} Gotama, the Jain \textsanskrit{Ñātika} doesn’t usually speak in terms of ‘deeds’. He usually speaks in terms of ‘rods’.”\footnote{The term “rod” implies violence and punishment unlike the more neutral “deed”. This usage is found in such Jain texts as \textsanskrit{Uttarādhyayanasūtra} 8.10. It was evidently common enough that it could be referred to obliquely as the “three rods” (20.60; see too \textsanskrit{Isibhāsiyāiṁ} 25.2, 35.0.6). It is also found in \textsanskrit{Mahābhārata} 12.228.034c. } 

“Then\marginnote{3.6} how many kinds of rod does the Jain \textsanskrit{Ñātika} describe for performing bad deeds?” 

“The\marginnote{3.7} Jain \textsanskrit{Ñātika} describes three kinds of rod for performing bad deeds: the physical rod, the verbal rod, and the mental rod.” 

“But\marginnote{3.9} are these kinds of rod all distinct from each other?” 

“Yes,\marginnote{3.10} each is quite distinct.” 

“Of\marginnote{3.11} the three rods thus analyzed and differentiated, which rod does the Jain \textsanskrit{Ñātika} describe as being the most blameworthy for performing bad deeds: the physical rod, the verbal rod, or the mental rod?” 

“The\marginnote{3.12} Jain \textsanskrit{Ñātika} describes the physical rod as being the most blameworthy for performing bad deeds, not so much the verbal rod or the mental rod.”\footnote{In Jainism, all three “rods” create \textit{karma}, which in Jainism is a kind of subtle matter (\textit{pudgala}) that adheres to and obscures the soul ( \textsanskrit{Tattvārthasūtra} 6.1–2). Thus all \textit{karma} is material. Nonetheless, according to Dr. Hiralal Jain, \emph{Jainism in Buddhist Literature}, the Jain position is not that physical actions are more important, since all three kinds of action must be intentional. Rather, it is that intentional actions performed through the body have a greater weight than those performed by speech or by thought alone. \textit{Karma} in the Jain system is the result of such intentional action, which clings to the soul as a form of matter. } 

“Do\marginnote{3.13} you say the physical rod, \textsanskrit{Tapassī}?” 

“I\marginnote{3.14} say the physical rod, Reverend Gotama.” 

“Do\marginnote{3.15} you say the physical rod, \textsanskrit{Tapassī}?” 

“I\marginnote{3.16} say the physical rod, Reverend Gotama.” 

“Do\marginnote{3.17} you say the physical rod, \textsanskrit{Tapassī}?” 

“I\marginnote{3.18} say the physical rod, Reverend Gotama.” 

Thus\marginnote{3.19} the Buddha made \textsanskrit{Dīgha} \textsanskrit{Tapassī} stand by this point up to the third time. 

When\marginnote{4.1} this was said, \textsanskrit{Dīgha} \textsanskrit{Tapassī} said to the Buddha, “But Reverend Gotama, how many kinds of rod do you describe for performing bad deeds?” 

“\textsanskrit{Tapassī},\marginnote{4.3} the Realized One doesn’t usually speak in terms of ‘rods’. He usually speaks in terms of ‘deeds’.”\footnote{The terms \textit{\textsanskrit{daṇḍa}} (“rod” or “punishment”) and \textit{kamma} (“deed”) are synonymous in Jainism. The three \textit{\textsanskrit{daṇḍas}} of body, speech, and mind are also found in \textsanskrit{Manusmṛti} 12.8. } 

“Then\marginnote{4.5} how many kinds of deed do you describe for performing bad deeds?” 

“I\marginnote{4.6} describe three kinds of deed for performing bad deeds: physical deeds, verbal deeds, and mental deeds.” 

“But\marginnote{4.8} are these kinds of deed all distinct from each other?” 

“Yes,\marginnote{4.9} each is quite distinct.” 

“Of\marginnote{4.10} the three deeds thus analyzed and differentiated, which deed do you describe as being the most blameworthy for performing bad deeds: physical deeds, verbal deeds, or mental deeds?” 

“I\marginnote{4.11} describe mental deeds as being the most blameworthy for performing bad deeds, not so much physical deeds or verbal deeds.” 

“Do\marginnote{4.12} you say mental deeds, Reverend Gotama?” 

“I\marginnote{4.13} say mental deeds, \textsanskrit{Tapassī}.” 

“Do\marginnote{4.14} you say mental deeds, Reverend Gotama?” 

“I\marginnote{4.15} say mental deeds, \textsanskrit{Tapassī}.” 

“Do\marginnote{4.16} you say mental deeds, Reverend Gotama?” 

“I\marginnote{4.17} say mental deeds, \textsanskrit{Tapassī}.” 

Thus\marginnote{4.18} the Jain ascetic \textsanskrit{Dīgha} \textsanskrit{Tapassī} made the Buddha stand by this point up to the third time, after which he got up from his seat and went to see the Jain \textsanskrit{Ñātika}. 

Now\marginnote{5.1} at that time the Jain \textsanskrit{Ñātika} was sitting together with a large assembly of \textsanskrit{Bālaki} laypeople headed by \textsanskrit{Upāli}.\footnote{\textsanskrit{Bālaki} is obscure. The commentary says it could mean either “foolish” (which seems harsh) or “from the village of the child salt miners” (which is obscure and distant, \href{https://suttacentral.net/mn128/en/sujato\#7.1}{MN 128:7.1}). A \textsanskrit{Dṛpta} \textsanskrit{Bālāki} of the \textsanskrit{Gārgya} clan features in \textsanskrit{Bṛhadāraṇyaka} \textsanskrit{Upaniṣad} 2.1.1, so perhaps it was a proper name (from \textit{\textsanskrit{balāka}}, “crane”). } The Jain \textsanskrit{Ñātika} saw \textsanskrit{Dīgha} \textsanskrit{Tapassī} coming off in the distance and said to him, “So, \textsanskrit{Tapassī}, where are you coming from in the middle of the day?” 

“Just\marginnote{5.5} now, sir, I’ve come from the presence of the ascetic Gotama.” 

“But\marginnote{5.6} did you have some discussion with him?” 

“I\marginnote{5.7} did.” 

“And\marginnote{5.8} what kind of discussion did you have with him?” Then \textsanskrit{Dīgha} \textsanskrit{Tapassī} informed the Jain \textsanskrit{Ñātika} of all they had discussed. When he had spoken, the Jain \textsanskrit{Ñātika} said to him, “Good, good, \textsanskrit{Tapassī}! \textsanskrit{Dīgha} \textsanskrit{Tapassī} has answered the ascetic Gotama like a learned disciple who rightly understands their teacher’s instructions. For how impressive is the measly mental rod when compared with the solid physical rod? Rather, the physical rod is the most blameworthy for performing bad deeds, not so much the verbal rod or the mental rod.” 

When\marginnote{7.1} he said this, the householder \textsanskrit{Upāli} said to him, “Good, sir! Well done, \textsanskrit{Dīgha} \textsanskrit{Tapassī}! The Honorable \textsanskrit{Tapassī} has answered the ascetic Gotama like a learned disciple who rightly understands their teacher’s instructions. For how impressive is the measly mental rod when compared with the solid physical rod? Rather, the physical rod is the most blameworthy for performing bad deeds, not so much the verbal rod or the mental rod. 

I’d\marginnote{7.6} better go and refute the ascetic Gotama’s doctrine regarding this point. If he stands by the position that he stated to \textsanskrit{Dīgha} \textsanskrit{Tapassī}, I’ll take him on in debate and drag him to and fro and round about, like a strong man would grab a fleecy sheep and drag it to and fro and round about! Taking him on in debate, I’ll drag him to and fro and round about, like a strong brewer’s worker would toss a large brewer’s sieve into a deep lake, grab it by the corners, and drag it to and fro and round about! Taking him on in debate, I’ll shake him down and about and give him a beating, like a strong brewer’s mixer would grab a strainer by the corners and shake it down and about, and give it a beating! I’ll play a game of ear-washing with the ascetic Gotama, like a sixty-year-old elephant would plunge into a deep lotus pond and play a game of ear-washing! Sir, I’d better go and refute the ascetic Gotama’s doctrine on this point.” 

“Go,\marginnote{7.12} householder, refute the ascetic Gotama’s doctrine on this point. For either I should do so, or \textsanskrit{Dīgha} \textsanskrit{Tapassī}, or you.”\footnote{It is rather a shame that neither in Buddhist nor Jain tradition is there a record of the two actually meeting. } 

When\marginnote{8.1} he said this, \textsanskrit{Dīgha} \textsanskrit{Tapassī} said to the Jain \textsanskrit{Ñātika}, “Sir, I don’t believe it’s a good idea for the householder \textsanskrit{Upāli} to rebut the ascetic Gotama’s doctrine. For the ascetic Gotama is a magician. He knows a conversion magic, and uses it to convert the disciples of other religions.” 

“It\marginnote{8.4} is impossible, \textsanskrit{Tapassī}, it cannot happen that \textsanskrit{Upāli} could become Gotama’s disciple.\footnote{This is a setup intended to satirize \textsanskrit{Mahāvīra}’s claims to omniscience. } But it is possible that Gotama could become \textsanskrit{Upāli}’s disciple. Go, householder, refute the ascetic Gotama’s doctrine on this point. For either I should do so, or \textsanskrit{Dīgha} \textsanskrit{Tapassī}, or you.” 

For\marginnote{9.1} a second time … and a third time, \textsanskrit{Dīgha} \textsanskrit{Tapassī} said to the Jain \textsanskrit{Ñātika}, “Sir, I don’t believe it’s a good idea for the householder \textsanskrit{Upāli} to rebut the ascetic Gotama’s doctrine. For the ascetic Gotama is a magician. He knows a conversion magic, and uses it to convert the disciples of other religions.” 

“It\marginnote{9.5} is impossible, \textsanskrit{Tapassī}, it cannot happen that \textsanskrit{Upāli} could become Gotama’s disciple. But it is possible that Gotama could become \textsanskrit{Upāli}’s disciple. Go, householder, refute the ascetic Gotama’s doctrine on this point. For either I should do so, or \textsanskrit{Dīgha} \textsanskrit{Tapassī}, or you.” 

“Yes,\marginnote{9.9} sir,” replied the householder \textsanskrit{Upāli} to the Jain \textsanskrit{Ñātika}. He got up from his seat, bowed, and respectfully circled him, keeping him on his right. Then he went to the Buddha, bowed, sat down to one side, and said to him, “Sir, did the Jain ascetic \textsanskrit{Dīgha} \textsanskrit{Tapassī} come here?” 

“He\marginnote{9.11} did, householder.” 

“But\marginnote{9.12} did you have some discussion with him?” 

“I\marginnote{9.13} did.” 

“And\marginnote{9.14} what kind of discussion did you have with him?”\footnote{\textsanskrit{Upāli} makes sure to hear the Buddha’s side before proceeding. } 

Then\marginnote{9.15} the Buddha informed \textsanskrit{Upāli} of all they had discussed. 

When\marginnote{10.1} he said this, the householder \textsanskrit{Upāli} said to him, “Good, sir, well done by \textsanskrit{Tapassī}! The Jain ascetic \textsanskrit{Tapassī} has answered the ascetic Gotama like a learned disciple who rightly understands their teacher’s instructions. For how impressive is the measly mental rod when compared with the solid physical rod? Rather, the physical rod is the most blameworthy for performing bad deeds, not so much the verbal rod or the mental rod.” 

“Householder,\marginnote{10.6} so long as you debate on the basis of truth, we can have some discussion about this.” 

“I\marginnote{10.7} will debate on the basis of truth, sir. Let us have some discussion about this.” 

“What\marginnote{11.1} do you think, householder? Take a Jain ascetic who is sick, suffering, gravely ill. They reject cold water and use only hot water.\footnote{Jain ascetics could not use cold water on account of the living creatures therein. \textsanskrit{Ācāraṅgasūtra} 2.2.1.8 says a \textit{\textsanskrit{nigaṇṭha}} must not stay at a lay person’s house because they may fall ill and be treated with, among other things, cold water (see also 2.6.2.2 and 2.1.7.7).  \textsanskrit{Uttarādhyayanasūtra} 1.1.3.7 dismisses the Buddhist claim that mere filtering is sufficient to render water allowable (see also 2.4). } Not getting cold water, they might die. Now, where does the Jain \textsanskrit{Ñātika} say they would be reborn?” 

“Sir,\marginnote{11.5} there are gods called ‘mind-bound’. They would be reborn there. Why is that? Because they died with mental attachment.”\footnote{It is not clear from the text why they must have attachment. } 

“Think\marginnote{11.8} about it, householder! You should think before answering. What you said before and what you said after don’t match up. But you said that you would debate on the basis of truth.” 

“Even\marginnote{11.13} though the Buddha says this, still the physical rod is the most blameworthy for performing bad deeds, not so much the verbal rod or the mental rod.” 

“What\marginnote{12.1} do you think, householder? Take a Jain ascetic who is restrained in all that is to be restrained, is bridled in all that is to be restrained, has shaken off evil in all that is to be restrained, and is curbed in all that is to be restrained.\footnote{MS has \textit{\textsanskrit{nigaṇṭho} \textsanskrit{nāṭaputto}}, but I follow PTS \textit{\textsanskrit{nigaṇṭho}}, in line with the pattern in this sutta and with the parallel at \href{https://suttacentral.net/dn2/en/sujato\#29.4}{DN 2:29.4}. | At \textsanskrit{Isibhāsiyāiṁ} 29.19, \textsanskrit{Vardhamāna} ( \textsanskrit{Mahāvīra}) teaches that a sage is \textit{savva-\textsanskrit{vārīhiṁ} \textsanskrit{vārie}}, “restrained in all restraints”, which clearly parallels our current passage. In that passage, “restraint” refers to stopping the influx of defilements through the five senses, neither delighting in the pleasant nor loathing the unpleasant. | Read \textit{\textsanskrit{vāri}} as future passive participle (cf. Sanskrit \textit{\textsanskrit{vārya}}). | \textit{Dhuta} in the sense “shaken off (evil by means of ascetic practices)” is a characteristic Jain term. | For \textit{\textsanskrit{sabbavāriphuṭo}} compare \textit{\textsanskrit{ophuṭo}} at \href{https://suttacentral.net/mn99/en/sujato\#15.5}{MN 99:15.5}. In both cases \textit{\textsanskrit{phuṭ}} appears in a string of terms from the root \textit{var}, and is possibly a corrupted form, or at least has the same meaning. } When going out and coming back they accidentally injure many little creatures.\footnote{Jain ascetics use a soft broom to sweep the path before them to avoid this. } Now, what result does the Jain \textsanskrit{Ñātika} say they would incur?” 

“Sir,\marginnote{12.5} the Jain \textsanskrit{Ñātika} says that unintentional acts are not very blameworthy.” 

“But\marginnote{12.6} if they are intentional?” 

“Then\marginnote{12.7} they are very blameworthy.” 

“But\marginnote{12.8} where does the Jain \textsanskrit{Ñātika} say that intention is classified?” 

“In\marginnote{12.9} the mental rod, sir.” 

“Think\marginnote{12.10} about it, householder! You should think before answering. What you said before and what you said after don’t match up. But you said that you would debate on the basis of truth.” 

“Even\marginnote{12.15} though the Buddha says this, still the physical rod is the most blameworthy for performing bad deeds, not so much the verbal rod or the mental rod.” 

“What\marginnote{13.1} do you think, householder? Is this \textsanskrit{Nāḷandā} successful and prosperous, populous, full of people?” 

“Indeed\marginnote{13.3} it is, sir.” 

“What\marginnote{13.4} do you think, householder? Suppose a man were to come along with a drawn sword and say: ‘In one hour I will reduce all the living creatures within the bounds of \textsanskrit{Nāḷandā} to one heap and mass of flesh!’ What do you think, householder? Could he do that?” 

“Sir,\marginnote{13.10} even ten, twenty, thirty, forty, or fifty men couldn’t do that. How impressive is one measly man?” 

“What\marginnote{13.12} do you think, householder? Suppose an ascetic or brahmin with psychic power, who has achieved mastery of the mind, were to come along and say: ‘I will reduce \textsanskrit{Nāḷandā} to ashes with a single malevolent act of will!’ What do you think, householder? Could he do that?” 

“Sir,\marginnote{13.18} an ascetic or brahmin with psychic power, who has achieved mastery of the mind, could reduce ten, twenty, thirty, forty, or fifty \textsanskrit{Nāḷandās} to ashes with a single malevolent act of will. How impressive is one measly \textsanskrit{Nāḷandā}?” 

“Think\marginnote{13.20} about it, householder! You should think before answering. What you said before and what you said after don’t match up. But you said that you would debate on the basis of truth.” 

“Even\marginnote{13.25} though the Buddha says this, still the physical rod is the most blameworthy for performing bad deeds, not so much the verbal rod or the mental rod.” 

“What\marginnote{13.26} do you think, householder? Have you heard how the wildernesses of \textsanskrit{Daṇḍaka}, \textsanskrit{Kaliṅga}, Mejjha, and \textsanskrit{Mātaṅga} came to be that way?”\footnote{This passage is referenced at \href{https://suttacentral.net/mil5.1.6/en/sujato\#16.3}{Mil 5.1.6:16.3}. | \textsanskrit{Daṇḍaka} was a vast wilderness covering much of the Deccan; it features heavily in the \textsanskrit{Rāmāyaṇa}. Vettam Mani’s \emph{Puranic Encyclopedia} informs us that \textsanskrit{Daṇḍa} was a son of \textsanskrit{Ikṣvāku}, founder of the solar dynasty, who gave \textsanskrit{Daṇḍa} the country between the mountains \textsanskrit{Himālaya} and Vindhya. But \textsanskrit{Daṇḍa} raped \textsanskrit{Arā}, the daughter of the hermit Śukra. Śukra destroyed the country of \textsanskrit{Daṇḍa} in a rain of fire. See too \textsanskrit{Vālmīki} \textsanskrit{Rāmāyaṇa} 3.5. Buddhist accounts may be found in the \textsanskrit{Sarabhaṅga} \textsanskrit{Jātaka} at \href{https://suttacentral.net/ja522/en/sujato}{Ja 522} and \textsanskrit{Mahāvastu} 99. This sage is possibly noted by name at \href{https://suttacentral.net/mn116/en/sujato\#6.13}{MN 116:6.13}. | \textsanskrit{Kaliṅga} was a coastal realm in modern Odisha and Andhra Pradesh. Its downfall is related in \href{https://suttacentral.net/ja522/en/sujato}{Ja 522}, although that was a kammic result of vile sin against ascetics rather than by psychic power. | Mejjha’s destruction is told in \href{https://suttacentral.net/ja497/en/sujato}{Ja 497}. There, \textsanskrit{Mātaṅga} features as a wise sage, and it is not clear to me how this relates to the wilderness of that name. In the suttas, the person \textsanskrit{Mātaṅga} was regarded as an enlightened sage of the past (\href{https://suttacentral.net/mn116/en/sujato\#6.17}{MN 116:6.17}, \href{https://suttacentral.net/snp1.7/en/sujato\#26.1}{Snp 1.7:26.1}). } 

“I\marginnote{13.28} have, sir.” 

“What\marginnote{14.1} have you heard?” 

“I\marginnote{14.2} heard that it was because of a malevolent act of will by seers that the wildernesses of \textsanskrit{Daṇḍaka}, \textsanskrit{Kaliṅga}, Mejjha, and \textsanskrit{Mātaṅga} came to be that way.” 

“Think\marginnote{14.3} about it, householder! You should think before answering. What you said before and what you said after don’t match up. But you said that you would debate on the basis of truth.” 

“Sir,\marginnote{15.1} I was already delighted and satisfied by the Buddha’s very first simile.\footnote{As at \href{https://suttacentral.net/dn23/en/sujato\#30.1}{DN 23:30.1}. } Nevertheless, I wanted to hear the Buddha’s various solutions to the problem, so I thought I’d oppose you in this way. 

Excellent,\marginnote{15.3} sir! Excellent! As if he were righting the overturned, or revealing the hidden, or pointing out the path to the lost, or lighting a lamp in the dark so people with clear eyes can see what’s there, the Buddha has made the teaching clear in many ways. I go for refuge to the Buddha, to the teaching, and to the mendicant \textsanskrit{Saṅgha}. From this day forth, may the Buddha remember me as a lay follower who has gone for refuge for life.” 

“Householder,\marginnote{16.1} you should act after careful consideration. It’s good for well-known people such as yourself to act after careful consideration.”\footnote{As at \href{https://suttacentral.net/an8.12/en/sujato\#26.1}{AN 8.12:26.1}. } 

“Now\marginnote{16.2} I’m even more delighted and satisfied with the Buddha, since he tells me to act after careful consideration. For if the followers of other religions were to gain me as a disciple, they’d carry a banner all over \textsanskrit{Nāḷandā}, saying: ‘The householder \textsanskrit{Upāli} has become our disciple!’ And yet the Buddha says: ‘Householder, you should act after careful consideration. It’s good for well-known people such as yourself to act after careful consideration.’ 

For\marginnote{16.7} a second time, I go for refuge to the Buddha, to the teaching, and to the mendicant \textsanskrit{Saṅgha}. From this day forth, may the Buddha remember me as a lay follower who has gone for refuge for life.” 

“For\marginnote{17.1} a long time now, householder, your family has been a well-spring of support for the Jain ascetics. You should consider giving to them when they come.” 

“Now\marginnote{17.2} I’m even more delighted and satisfied with the Buddha, since he tells me to consider giving to the Jain ascetics when they come. I have heard, sir, that the ascetic Gotama says this: ‘Gifts should only be given to me, not to others.\footnote{The same calumny is reported at \href{https://suttacentral.net/an8.12/en/sujato\#27.6}{AN 8.12:27.6} and \href{https://suttacentral.net/an3.57/en/sujato\#1.4}{AN 3.57:1.4}. } Gifts should only be given to my disciples, not to the disciples of others. Only what is given to me is very fruitful, not what is given to others. Only what is given to my disciples is very fruitful, not what is given to the disciples of others.’ Yet the Buddha encourages me to give to the Jain ascetics. Well, sir, we’ll know the proper time for that. 

For\marginnote{17.10} a third time, I go for refuge to the Buddha, to the teaching, and to the mendicant \textsanskrit{Saṅgha}. From this day forth, may the Buddha remember me as a lay follower who has gone for refuge for life.” 

Then\marginnote{18.1} the Buddha taught the householder \textsanskrit{Upāli} step by step, with a talk on giving, ethical conduct, and heaven. He explained the drawbacks of sensual pleasures, so sordid and corrupt, and the benefit of renunciation. And when he knew that \textsanskrit{Upāli}’s mind was ready, pliable, rid of hindrances, elated, and confident he explained the special teaching of the Buddhas: suffering, its origin, its cessation, and the path. Just as a clean cloth rid of stains would properly absorb dye, in that very seat the stainless, immaculate vision of the Dhamma arose in \textsanskrit{Upāli}: “Everything that has a beginning has an end.” Then \textsanskrit{Upāli} saw, attained, understood, and fathomed the Dhamma. He went beyond doubt, got rid of indecision, and became self-assured and independent of others regarding the Teacher’s instructions. 

He\marginnote{18.9} said to the Buddha, “Well, now, sir, I must go. I have many duties, and much to do.” 

“Please,\marginnote{18.10} householder, go at your convenience.” 

And\marginnote{19.1} then the householder \textsanskrit{Upāli} approved and agreed with what the Buddha said. He got up from his seat, bowed, and respectfully circled the Buddha, keeping him on his right. Then he went back to his own home, where he addressed the gatekeeper, “My good gatekeeper, from this day forth close the gate to Jain monks and nuns, and open it for the Buddha’s monks, nuns, laymen, and laywomen. If any Jain ascetics come, say this to them: ‘Wait, sir, do not enter.\footnote{This is even harsher than it appears, for one of the Jains’ rules is not to wait when invited for food (\textit{\textsanskrit{natiṭṭhabhaddantiko}}, eg. \href{https://suttacentral.net/mn12/en/sujato\#45.1}{MN 12:45.1}). } From now on the householder \textsanskrit{Upāli} has become a disciple of the ascetic Gotama. His gate is closed to Jain monks and nuns, and opened for the Buddha’s monks, nuns, laymen, and laywomen. If you require almsfood, wait here, they will bring it to you.’” 

“Yes,\marginnote{19.8} sir,” replied the gatekeeper. 

\textsanskrit{Dīgha}\marginnote{20.1} \textsanskrit{Tapassī} heard that \textsanskrit{Upāli} had become a disciple of the ascetic Gotama. He went to the Jain \textsanskrit{Ñātika} and said to him, “Sir, they say that the householder \textsanskrit{Upāli} has become a disciple of the ascetic Gotama.” 

“It\marginnote{20.5} is impossible, \textsanskrit{Tapassī}, it cannot happen that \textsanskrit{Upāli} could become Gotama’s disciple. But it is possible that Gotama could become \textsanskrit{Upāli}’s disciple.” 

For\marginnote{20.7} a second time … and a third time, \textsanskrit{Dīgha} \textsanskrit{Tapassī} said to the Jain \textsanskrit{Ñātika}, “Sir, they say that the householder \textsanskrit{Upāli} has become a disciple of the ascetic Gotama.” 

“It\marginnote{20.10} is impossible, \textsanskrit{Tapassī}, it cannot happen that \textsanskrit{Upāli} could become Gotama’s disciple. But it is possible that Gotama could become \textsanskrit{Upāli}’s disciple.” 

“Well,\marginnote{20.12} sir, I’d better go and find out whether or not \textsanskrit{Upāli} has become Gotama’s disciple.” 

“Go,\marginnote{20.13} \textsanskrit{Tapassī}, and find out whether or not \textsanskrit{Upāli} has become Gotama’s disciple.” 

Then\marginnote{21.1} \textsanskrit{Dīgha} \textsanskrit{Tapassī} went to \textsanskrit{Upāli}’s home. The gatekeeper saw him coming off in the distance and said to him, “Wait, sir, do not enter. From now on the householder \textsanskrit{Upāli} has become a disciple of the ascetic Gotama. His gate is closed to Jain monks and nuns, and opened for the Buddha’s monks, nuns, laymen, and laywomen. If you require almsfood, wait here, they will bring it to you.” 

Saying,\marginnote{21.8} “No, mister, I do not require almsfood,” he turned back and went to the Jain \textsanskrit{Ñātika} and said to him, “Sir, it’s really true that \textsanskrit{Upāli} has become Gotama’s disciple. Sir, I couldn’t get you to accept that it wasn’t a good idea for the householder \textsanskrit{Upāli} to rebut the ascetic Gotama’s doctrine. For the ascetic Gotama is a magician. He knows a conversion magic, and uses it to convert the disciples of other religions. The householder \textsanskrit{Upāli} has been converted by the ascetic Gotama’s conversion magic!” 

“It\marginnote{21.13} is impossible, \textsanskrit{Tapassī}, it cannot happen that \textsanskrit{Upāli} could become Gotama’s disciple. But it is possible that Gotama could become \textsanskrit{Upāli}’s disciple.” 

For\marginnote{21.15} a second time … and a third time, \textsanskrit{Dīgha} \textsanskrit{Tapassī} told the Jain \textsanskrit{Ñātika} that it was really true. 

“It\marginnote{21.20} is impossible … 

Well,\marginnote{21.22} \textsanskrit{Tapassī}, I’d better go and find out for myself whether or not \textsanskrit{Upāli} has become Gotama’s disciple.” 

Then\marginnote{22.1} the Jain \textsanskrit{Ñātika} went to \textsanskrit{Upāli}’s home together with a large following of Jain ascetics. The gatekeeper saw him coming off in the distance and said to him: “Wait, sir, do not enter. From now on the householder \textsanskrit{Upāli} has become a disciple of the ascetic Gotama. His gate is closed to Jain monks and nuns, and opened for the Buddha’s monks, nuns, laymen, and laywomen. If you require almsfood, wait here, they will bring it to you.” 

“Well\marginnote{22.8} then, my good gatekeeper, go to \textsanskrit{Upāli} and say: ‘Sir, the Jain \textsanskrit{Ñātika} is waiting outside the gates together with a large following of Jain ascetics. He wishes to see you.’” 

“Yes,\marginnote{22.11} sir,” replied the gatekeeper. He went to \textsanskrit{Upāli} and relayed what was said. \textsanskrit{Upāli} said to him, “Well, then, my good gatekeeper, prepare seats in the hall of the middle gate.” 

“Yes,\marginnote{22.15} sir,” replied the gatekeeper. He did as he was asked, then returned to \textsanskrit{Upāli} and said, “Sir, seats have been prepared in the hall of the middle gate. Please go at your convenience.” 

Then\marginnote{23.1} \textsanskrit{Upāli} went to the hall of the middle gate, where he sat on the highest and finest seat. He addressed the gatekeeper, “Well then, my good gatekeeper, go to the Jain \textsanskrit{Ñātika} and say to him: ‘Sir, \textsanskrit{Upāli} says you may enter if you wish.’” 

“Yes,\marginnote{23.5} sir,” replied the gatekeeper. He went to the Jain \textsanskrit{Ñātika} and relayed what was said. 

Then\marginnote{23.8} the Jain \textsanskrit{Ñātika} went to the hall of the middle gate together with a large following of Jain ascetics. Previously, when \textsanskrit{Upāli} saw the Jain \textsanskrit{Ñātika} coming, he would go out to greet him and, having wiped off the highest and finest seat with his upper robe, he would embrace him and sit him down.\footnote{\textit{\textsanskrit{Pariggahetvā}} (“embrace”) is used in only one other passage, where it refers to a nurse cradling a baby’s head (\href{https://suttacentral.net/an5.7/en/sujato\#1.9}{AN 5.7:1.9}, \href{https://suttacentral.net/mn58/en/sujato\#7.6}{MN 58:7.6}). } But today, having seated himself on the highest and finest seat, he said to the Jain \textsanskrit{Ñātika}, “There are seats, sir. Please sit if you wish.” 

When\marginnote{25.1} he said this, the Jain \textsanskrit{Ñātika} said to him: “You’re mad, householder! You’re a moron! You said: ‘I’ll go and refute the ascetic Gotama’s doctrine.’ But you come back caught in the vast net of his doctrine. Suppose a man went to deliver a pair of testicles, but came back castrated. Or they went to deliver eyes, but came back blinded. In the same way, you said: ‘I’ll go and refute the ascetic Gotama’s doctrine.’ But you come back caught in the vast net of his doctrine. You’ve been converted by the ascetic Gotama’s conversion magic!” 

“Sir,\marginnote{26.1} this conversion magic is excellent. This conversion magic is lovely! If my loved ones—relatives and kin—were to be converted by this, it would be for their lasting welfare and happiness. If all the aristocrats, brahmins, peasants, and menials were to be converted by this, it would be for their lasting welfare and happiness. If the whole world—with its gods, \textsanskrit{Māras}, and divinities, this population with its ascetics and brahmins, gods and humans—were to be converted by this, it would be for their lasting welfare and happiness. Well then, sir, I shall give you a simile. For by means of a simile some sensible people understand the meaning of what is said. 

Once\marginnote{27.1} upon a time there was an old brahmin, elderly and senior. His wife was a young brahmin lady who was pregnant and about to give birth.\footnote{The age at which girls were considered ready for marriage was sixteen. } Then she said to the brahmin, ‘Go, brahmin, buy a baby monkey from the market and bring it back so it can be a playmate for my child.’\footnote{Though the word is not used, this is the \textit{\textsanskrit{dohaḷa}}, the desire of a (usually pregnant) woman that is as unstoppable as it is irrational. In numerous \textsanskrit{Jātaka} stories as well as across Indian storytelling generally, it serves to send a hero on his journey. For a developed \textit{\textsanskrit{dohaḷa}} story in the canon, see the legend of \textsanskrit{Dīghāvu} (\href{https://suttacentral.net/pli-tv-kd10/en/sujato\#2.3.1}{Kd 10:2.3.1}). } 

When\marginnote{27.4} she said this, the brahmin said to her, ‘Wait, my dear, until you give birth. If your child is a boy, I’ll buy you a male monkey, but if it’s a girl, I’ll buy a female monkey.’ 

For\marginnote{27.8} a second time, and a third time she said to the brahmin, ‘Go, brahmin, buy a baby monkey from the market and bring it back so it can be a playmate for my child.’ 

Then\marginnote{27.11} that brahmin, because of his love for the brahmin lady, bought a male baby monkey at the market, brought it to her, and said, ‘I’ve bought this male baby monkey for you so it can be a playmate for your child.’ 

When\marginnote{27.13} he said this, she said to him, ‘Go, brahmin, take this monkey to \textsanskrit{Rattapāṇi} the dyer and say, “Mister \textsanskrit{Rattapāṇi}, I wish to have this monkey dyed the color of yellow greasepaint, pounded and re-pounded, and pressed on both sides.”’ 

Then\marginnote{27.16} that brahmin, because of his love for the brahmin lady, took the monkey to \textsanskrit{Rattapāṇi} the dyer and said, ‘Mister \textsanskrit{Rattapāṇi}, I wish to have this monkey dyed the color of yellow greasepaint, pounded and re-pounded, and pressed on both sides.’ 

When\marginnote{27.18} he said this, \textsanskrit{Rattapāṇi} said to him, ‘Sir, this monkey can withstand a dying, but not a pounding or a pressing.’ 

In\marginnote{27.20} the same way, the doctrine of the foolish Jains looks fine initially—for fools, not for the astute—but can’t withstand being scrutinized or pressed. 

Then\marginnote{27.21} some time later that brahmin took a new pair of garments to \textsanskrit{Rattapāṇi} the dyer and said, ‘Mister \textsanskrit{Rattapāṇi}, I wish to have this new pair of garments dyed the color of yellow greasepaint, pounded and re-pounded, and pressed on both sides.’ 

When\marginnote{27.23} he said this, \textsanskrit{Rattapāṇi} said to him, ‘Sir, this pair of garments can withstand a dying, a pounding, and a pressing.’ 

In\marginnote{27.25} the same way, the doctrine of the Buddha looks fine initially—for the astute, not for fools—and it can withstand being scrutinized and pressed.” 

“Householder,\marginnote{28.1} the king and his retinue know you as a disciple of the Jain \textsanskrit{Ñātika}. Whose disciple should we remember you as?” 

When\marginnote{29.1} he had spoken, the householder \textsanskrit{Upāli} got up from his seat, arranged his robe over one shoulder, raised his joined palms in the direction of the Buddha, and said to the Jain \textsanskrit{Ñātika}, “Well then, sir, hear whose disciple I am: 

\begin{verse}%
The\marginnote{29.3} attentive one, free of delusion, \\
with hard-heartedness dissolved, victor in battle;\footnote{Read \textit{\textsanskrit{khīla}} as \textit{khila}, a reference to the five kinds of hard-heartedness. For similar idioms see \href{https://suttacentral.net/snp4.16/en/sujato\#19.2}{Snp 4.16:19.2} and \href{https://suttacentral.net/sn8.8/en/sujato\#8.2}{SN 8.8:8.2} = \href{https://suttacentral.net/thag21.1/en/sujato\#34.2}{Thag 21.1:34.2}. } \\
he’s untroubled and very even-minded, \\
with the virtue of an elder and the wisdom of a saint, \\
immaculate in the midst of it all:\footnote{“In the midst of it all” is \textit{vessantara}, as in the famous \textsanskrit{Jātaka} of that name (Ja 547). MA 133 here has \langlzh{安隱, where \langlzh{隱} represents \textit{antara}, and \langlzh{安} may be \textit{sama}. \textit{Vessa} in the sense of “all” is rare in Pali. }} \\
he is the Buddha, and I am his disciple. 

He\marginnote{29.9} has no indecision, he’s content, \\
joyful, he has spat out the world’s bait; \\
he has completed the ascetic’s task as a human, \\
a man who bears his final body; \\
he’s beyond compare, he’s stainless: \\
he is the Buddha, and I am his disciple. 

He’s\marginnote{29.15} free of doubt, he’s skillful, \\
he’s a trainer, an excellent charioteer; \\
supreme, with brilliant qualities, \\
confident, his light shines forth; \\
he has cut off conceit, he’s a hero: \\
he is the Buddha, and I am his disciple. 

The\marginnote{29.21} chief bull, immeasurable, \\
profound, sagacious; \\
he is the maker of sanctuary, knowledgeable, \\
firm in principle and restrained; \\
he has slipped his chains and is liberated: \\
he is the Buddha, and I am his disciple. 

He’s\marginnote{29.27} a giant, living remotely, \\
he’s ended the fetters and is liberated; \\
he’s skilled in debate and cleansed, \\
with banner lowered, desireless;\footnote{“Banner lowered” is explained at \href{https://suttacentral.net/mn22/en/sujato\#35.1}{MN 22:35.1}. } \\
he’s tamed, and doesn’t proliferate: \\
he is the Buddha, and I am his disciple. 

He\marginnote{29.33} is the seventh sage, free of deceit,\footnote{The idea of the seven sages, long proverbial in Vedic culture, was adapted for Gotama in \href{https://suttacentral.net/dn14/en/sujato}{DN 14}, where he is the seventh recorded Buddha. Readings of this phrase in various texts and commentaries also allow the sense, “best of seers”. } \\
with three knowledges, he has attained to divinity, \\
he has bathed, he knows philology,\footnote{The terms here are all Brahmanical, so we should take \textit{padaka} in the usual sense of “philology” as in eg. \href{https://suttacentral.net/mn91/en/sujato\#2.1}{MN 91:2.1}. } \\
he’s tranquil, he understands what is known; \\
he is Sakka the Able, Purindada the Firstgiver:\footnote{\textit{Sakka} (\textit{\textsanskrit{śakra}}) was originally an epithet of Indra, the “able one”. | \textit{Purandara} (“Fortbreaker”) is another epithet (eg. Rig Veda 1.102.7). The Buddha reforms it to \textit{Purindada} (“Firstgiver”) at \href{https://suttacentral.net/sn11.12/en/sujato\#2.1}{SN 11.12:2.1}. But Indra’s generosity is long renowned, eg. Rig Veda 1.10.6c: “He is the able one, and he will be able for us—Indra who distributes the goods” (\textit{sa \textsanskrit{śakra} uta naḥ \textsanskrit{śakad} indro vasu \textsanskrit{dayamānaḥ}}). } \\
he is the Buddha, and I am his disciple. 

The\marginnote{29.39} noble one, evolved, \\
he has attained the goal and explains it; \\
he is mindful, discerning, \\
neither leaning forward nor pulling back,\footnote{These terms refer to \textit{\textsanskrit{samādhi}} at \href{https://suttacentral.net/sn1.38/en/sujato\#8.2}{SN 1.38:8.2} and \href{https://suttacentral.net/an9.37/en/sujato\#8.2}{AN 9.37:8.2}. } \\
he’s unstirred, attained to mastery: \\
he is the Buddha, and I am his disciple. 

Transcendent,\marginnote{29.45} he practices absorption,\footnote{At \href{https://suttacentral.net/thig14.1/en/sujato\#14.1}{Thig 14.1:14.1} \textit{samuggata} refers to a lily rising above the waters; at \href{https://suttacentral.net/sn10.7/en/sujato\#10.2}{SN 10.7:10.2} to rising above suffering. } \\
no inner thoughts arising, he is pure,\footnote{See \href{https://suttacentral.net/ud4.1/en/sujato\#23.2}{Ud 4.1:23.2}, where \textit{anu(g)gata} refers to thoughts that “arise” in the mind. } \\
independent and fearless;\footnote{Readings are unclear; I follow Thai \textit{\textsanskrit{appabhītassa}}. } \\
secluded, he has reached the peak, \\
crossed over, he helps others across: \\
he is the Buddha, and I am his disciple. 

He’s\marginnote{29.51} peaceful, his wisdom is vast, \\
with great wisdom, he’s free of greed; \\
he is the Realized One, the Holy One, \\
unrivaled, unequaled, \\
assured, and subtle: \\
he is the Buddha, and I am his disciple. 

He\marginnote{29.57} has cut off craving and is awakened, \\
free of fuming, unsullied; \\
a mighty spirit worthy of offerings,\footnote{The Chinese here evidently reads \textit{akkha} (“eye”), but in the context of offerings I think \textit{yakkha} (“spirit”) fits better. Conversion legends of \textit{yakkhas} tell how the cruel offerings of sacrifice are replaced with the harmless offerings to the Buddha. Perhaps it was later adjusted to \textit{akkha} as the \textit{yakkha} developed a poor reputation in later Buddhism. } \\
best of men, inestimable, \\
grand, he has reached the peak of glory: \\
he is the Buddha, and I am his disciple.” 

%
\end{verse}

“But\marginnote{30.1} when did you compose these praises of the ascetic Gotama’s beautiful qualities, householder?”\footnote{“When did you compose” is also at \href{https://suttacentral.net/dn21/en/sujato\#1.6.4}{DN 21:1.6.4}. } 

“Sir,\marginnote{30.2} suppose there was a large heap of many different flowers. A deft garland-maker or their apprentice could tie them into a colorful garland. In the same way, the Buddha has many beautiful qualities to praise, many hundreds of such qualities.\footnote{\textsanskrit{Upāli} is playing on the dual senses of \textit{\textsanskrit{vaṇṇa}} as both “praise” and “beauty”. In doing so, he implicitly answers \textsanskrit{Mahāvīra}’s question: he is adept at poetic improvisation (\textit{\textsanskrit{paṭibhāna}}). } Who, sir, would not praise the praiseworthy?” 

Unable\marginnote{31.1} to bear this honor paid to the Buddha, the Jain \textsanskrit{Ñātika} spewed hot blood from his mouth there and then.\footnote{The Pali commentary says this was the cause of his death in \textsanskrit{Pāvā} soon after, which supports the contention that \textsanskrit{Pāvā} is \textsanskrit{Pāvārika}. The Jain \textsanskrit{Kalpasūtra}, however, tells a very different story, of a triumphant leader who, having lived a long and exemplary life, mindfully departed this world in honor and dignity, to the consternation of the gods. } 

%
\section*{{\suttatitleacronym MN 57}{\suttatitletranslation The Ascetic Who Behaved Like a Dog }{\suttatitleroot Kukkuravatikasutta}}
\addcontentsline{toc}{section}{\tocacronym{MN 57} \toctranslation{The Ascetic Who Behaved Like a Dog } \tocroot{Kukkuravatikasutta}}
\markboth{The Ascetic Who Behaved Like a Dog }{Kukkuravatikasutta}
\extramarks{MN 57}{MN 57}

\scevam{So\marginnote{1.1} I have heard. }At one time the Buddha was staying in the land of the Koliyans, where they have a town named Haliddavasana.\footnote{The Koliyans were south-eastern neighbors of the Sakyans, with whom they had close ties in marriage and customs. | Haliddavasana (“yellow-clothes”) is also the scene of \href{https://suttacentral.net/sn46.54/en/sujato}{SN 46.54}, which discusses similarities and differences between Buddhism and wanderers when it comes to meditation. } 

Then\marginnote{2.1} \textsanskrit{Puṇṇa} the Koliyan, a cow votary, and the naked ascetic Seniya, a dog votary, went to see the Buddha. \textsanskrit{Puṇṇa} bowed to the Buddha and sat down to one side, while Seniya exchanged greetings and polite conversation with him before sitting down to one side curled up like a dog.\footnote{The cow votary (\textit{govatika}) is attested in Sanskrit (eg. \textsanskrit{Mahābhārata} 5.97.13). Elsewhere, non-Buddhist texts mention cat votaries (Manu 4.30), snake votaries, fish votaries, those who live like beasts (\textit{\textsanskrit{mṛgacaryā}}), and others, but not, so far as I am aware, dog votaries. The cow and dog observances are found in later Buddhist texts (\textsanskrit{Yogacārabhūmi} 157.10, \textsanskrit{Mahākarmavibhaṅga} 44.19, \textsanskrit{Śikṣāsamuccaya} 332.3, etc.) evidently drawing from this sutta. Thus, while no parallel of this sutta has survived, it was known in northern Buddhist traditions and must have been lost due to a failure of transmission. } 

\textsanskrit{Puṇṇa}\marginnote{2.2} said to the Buddha, “Sir, this naked ascetic Seniya the dog votary does a hard thing: he eats food placed on the ground. For a long time he has undertaken that observance to behave like a dog. Where will he be reborn in his next life?” 

“Enough,\marginnote{2.6} \textsanskrit{Puṇṇa}, let it be. Don’t ask me that.”\footnote{Also at \href{https://suttacentral.net/sn42.2/en/sujato}{SN 42.2} and \href{https://suttacentral.net/sn42.3/en/sujato}{SN 42.3}. } 

For\marginnote{2.7} a second time … and a third time, \textsanskrit{Puṇṇa} said to the Buddha, “Sir, this naked ascetic Seniya does a hard thing: he eats food placed on the ground. For a long time he has undertaken that observance to behave like a dog. Where will he be reborn in his next life?” 

“Clearly,\marginnote{2.12} \textsanskrit{Puṇṇa}, I’m not getting through to you when I say: ‘Enough, \textsanskrit{Puṇṇa}, let it be. Don’t ask me that.’ Nevertheless, I will answer you. 

Take\marginnote{3.1} someone who develops the dog observance fully and uninterruptedly. They develop a dog’s ethics, a dog’s mentality, and a dog’s behavior fully and uninterruptedly. When their body breaks up, after death, they’re reborn in the company of dogs. But if they have such a view: ‘By this precept or observance or fervent austerity or spiritual life, may I become one of the gods!’ This is their wrong view. An individual with wrong view is reborn in one of two places, I say: hell or the animal realm. So if the dog observance succeeds it leads to rebirth in the company of dogs, but if it fails it leads to hell.” 

When\marginnote{4.1} he said this, Seniya cried and burst out in tears. 

The\marginnote{4.2} Buddha said to \textsanskrit{Puṇṇa}, “This is what I didn’t get through to you when I said: ‘Enough, \textsanskrit{Puṇṇa}, let it be. Don’t ask me that.’” 

Seniya\marginnote{4.5} said, “Sir, I’m not crying because of what the Buddha said. But, sir, for a long time I have undertaken this observance to behave like a dog. Sir, this \textsanskrit{Puṇṇa} is a cow votary. For a long time he has undertaken that observance to behave like a cow.\footnote{From a text of a later date, the cow observance is described thus: “One should bathe in the urine of a cow, maintain himself with milk, and move with the cows, eating after they had eaten. It is known as \textit{govrata}. One would become free from the sins in a month. He would attain the world of cows and reach heaven” (Agni \textsanskrit{Purāna} 292:12–13a, Gangadharan’s translation). This is not a penance for burning off sin, but rather a way of elevating oneself through the sacred nature of the cow. It is not clear, however, if this is the same as the observance in the Buddha’s day. } Where will he be reborn in his next life?” 

“Enough,\marginnote{4.10} Seniya, let it be. Don’t ask me that.” 

For\marginnote{4.11} a second time … and a third time Seniya said to the Buddha, “Sir, this \textsanskrit{Puṇṇa} is a cow votary. For a long time he has undertaken that observance to behave like a cow. Where will he be reborn in his next life?” 

“Clearly,\marginnote{4.16} Seniya, I’m not getting through to you when I say: ‘Enough, Seniya, let it be. Don’t ask me that.’ Nevertheless, I will answer you. 

Take\marginnote{5.1} someone who develops the cow observance fully and uninterruptedly. They develop a cow’s ethics, a cow’s mentality, and a cow’s behavior fully and uninterruptedly. When their body breaks up, after death, they’re reborn in the company of cows.\footnote{This means to be reborn as a cow, rather than in the \textit{goloka}, which in Hinduism is a heaven for cows. } But if they have such a view: ‘By this precept or observance or fervent austerity or spiritual life, may I become one of the gods!’ This is their wrong view. An individual with wrong view is reborn in one of two places, I say: hell or the animal realm. So if the cow observance succeeds it leads to rebirth in the company of cows, but if it fails it leads to hell.” 

When\marginnote{6.1} he said this, \textsanskrit{Puṇṇa} cried and burst out in tears. 

The\marginnote{6.2} Buddha said to Seniya, “This is what I didn’t get through to you when I said: ‘Enough, Seniya, let it be. Don’t ask me that.’” 

\textsanskrit{Puṇṇa}\marginnote{6.5} said, “Sir, I’m not crying because of what the Buddha said. But, sir, for a long time I have undertaken this observance to behave like a cow. I am quite confident that the Buddha is capable of teaching me so that I can give up this cow observance, and the naked ascetic Seniya can give up that dog observance.” 

“Well\marginnote{6.9} then, \textsanskrit{Puṇṇa}, listen and apply your mind well, I will speak.” 

“Yes,\marginnote{6.10} sir,” he replied. The Buddha said this: 

“\textsanskrit{Puṇṇa},\marginnote{7.1} I declare these four kinds of deeds, having realized them with my own insight.\footnote{See \href{https://suttacentral.net/an4.233/en/sujato}{AN 4.233} ff. } What four? 

\begin{enumerate}%
\item There are dark deeds with dark results; %
\item bright deeds with bright results; %
\item dark and bright deeds with dark and bright results; and\footnote{By treating ethical decisions via a tetralemma, the Buddha rejects the “law of the excluded middle” and the consequent belief that acts must be either right or wrong. } %
\item neither dark nor bright deeds with neither dark nor bright results, which lead to the ending of deeds.\footnote{These are the intentions associated with the noble eightfold path. } %
\end{enumerate}

And\marginnote{8.1} what are dark deeds with dark results? It’s when someone makes hurtful choices by way of body, speech, and mind. Having made these choices, they’re reborn in a hurtful world, where hurtful contacts strike them. Touched by hurtful contacts, they experience hurtful feelings that are exclusively painful—like the beings in hell. This is how a being is born from a being. For your deeds determine your rebirth, and when you’re reborn contacts strike you. This is why I say that sentient beings are heirs to their deeds. These are called dark deeds with dark results. 

And\marginnote{9.1} what are bright deeds with bright results? It’s when someone makes pleasing choices by way of body, speech, and mind. Having made these choices, they are reborn in a pleasing world, where pleasing contacts strike them. Touched by pleasing contacts, they experience pleasing feelings of perfect happiness—like the gods of universal beauty. This is how a being is born from a being. For your deeds determine your rebirth, and when you’re reborn contacts strike you. This is why I say that sentient beings are heirs to their deeds. These are called bright deeds with bright results. 

And\marginnote{10.1} what are dark and bright deeds with dark and bright results? It’s when someone makes both hurtful and pleasing choices by way of body, speech, and mind. Having made these choices, they are reborn in a world that is both hurtful and pleasing, where hurtful and pleasing contacts strike them. Touched by both hurtful and pleasing contacts, they experience both hurtful and pleasing feelings that are a mixture of pleasure and pain—like humans, some gods, and some beings in the underworld. This is how a being is born from a being. For what you do brings about your rebirth, and when you’re reborn contacts strike you. This is why I say that sentient beings are heirs to their deeds. These are called dark and bright deeds with dark and bright results. 

And\marginnote{11.1} what are neither dark nor bright deeds with neither dark nor bright results, which lead to the ending of deeds? It’s the intention to give up dark deeds with dark results, bright deeds with bright results, and both dark and bright deeds with both dark and bright results. These are called neither dark nor bright deeds with neither dark nor bright results, which lead to the ending of deeds. 

These\marginnote{11.4} are the four kinds of deeds that I declare, having realized them with my own insight.” 

When\marginnote{12.1} he had spoken, \textsanskrit{Puṇṇa} the Koliyan said to the Buddha, “Excellent, sir! Excellent! … From this day forth, may the Buddha remember me as a lay follower who has gone for refuge for life.” 

And\marginnote{13.1} Seniya the naked ascetic said to the Buddha, “Excellent, sir! Excellent! … I go for refuge to the Buddha, to the teaching, and to the mendicant \textsanskrit{Saṅgha}. Sir, may I receive the going forth, the ordination in the Buddha’s presence?”\footnote{\textsanskrit{Puṇṇa} the layman goes for refuge, while Seniya the ascetic goes forth. } 

“Seniya,\marginnote{14.1} if someone formerly ordained in another sect wishes to take the going forth, the ordination in this teaching and training, they must spend four months on probation. When four months have passed, if the mendicants are satisfied, they’ll give the going forth, the ordination into monkhood.\footnote{This probation is laid down in the Vinaya at \href{https://suttacentral.net/pli-tv-kd1/en/sujato\#38.1.5}{Kd 1:38.1.5}. The candidate shaves, dons the robes, takes refuge, and asks for probation. They must show good conduct and restraint, diligence in duties, and enthusiasm for the Buddha’s teachings and practice. } However, I have recognized individual differences in this matter.”\footnote{In addition to individual exceptions, there are general exceptions for dreadlocked ascetics, since they believe in kamma, and for the Buddha’s relatives. } 

“Sir,\marginnote{14.3} if four months probation are required in such a case, I’ll spend four years on probation. When four years have passed, if the mendicants are satisfied, let them give me the going forth, the ordination into monkhood.” 

And\marginnote{15.1} the naked ascetic Seniya received the going forth, the ordination in the Buddha’s presence. Not long after his ordination, Venerable Seniya, living alone, withdrawn, diligent, keen, and resolute, soon realized the supreme end of the spiritual path in this very life. He lived having achieved with his own insight the goal for which gentlemen rightly go forth from the lay life to homelessness. 

He\marginnote{15.3} understood: “Rebirth is ended; the spiritual journey has been completed; what had to be done has been done; there is nothing further for this place.” And Venerable Seniya became one of the perfected. 

%
\section*{{\suttatitleacronym MN 58}{\suttatitletranslation With Prince Abhaya }{\suttatitleroot Abhayarājakumārasutta}}
\addcontentsline{toc}{section}{\tocacronym{MN 58} \toctranslation{With Prince Abhaya } \tocroot{Abhayarājakumārasutta}}
\markboth{With Prince Abhaya }{Abhayarājakumārasutta}
\extramarks{MN 58}{MN 58}

\scevam{So\marginnote{1.1} I have heard. }At one time the Buddha was staying near \textsanskrit{Rājagaha}, in the Bamboo Grove, the squirrels’ feeding ground.\footnote{This sutta is framed as a response to a challenge by \textsanskrit{Mahāvīra}, the Jain leader. It shows the Buddha’s characteristic dry humor to undermine religious pretensions (see eg. \href{https://suttacentral.net/mn76/en/sujato\#21.4}{MN 76:21.4}, \href{https://suttacentral.net/dn11/en/sujato\#82.7}{DN 11:82.7}). The dialogue slyly shows the Buddha defeating \textsanskrit{Mahāvīra} by using one of \textsanskrit{Mahāvīra}’s key methods against him. } 

Then\marginnote{2.1} Prince Abhaya went up to the Jain ascetic of the \textsanskrit{Ñātika} clan, bowed, and sat down to one side. The Jain \textsanskrit{Ñātika} said to him,\footnote{Prince Abhaya (“fearless”) raised \textsanskrit{Jīvaka} after discovering him abandoned \href{https://suttacentral.net/pli-tv-kd8/en/sujato\#1.4.5}{Kd 8:1.4.5}. In this sutta he appears as an acolyte of \textsanskrit{Mahāvīra}, while in \href{https://suttacentral.net/sn46.56/en/sujato}{SN 46.56} he asks the Buddha about \textsanskrit{Pūraṇa} Kassapa. Apart from these details he is unknown in the early texts. Commentaries say he was the son of King \textsanskrit{Bimbisāra} by one of his queens, \textsanskrit{Padumavatī} of \textsanskrit{Ujjenī}, making him half-brother to \textsanskrit{Ajātasattu}. They also say that following \textsanskrit{Bimbisāra}’s death, Abhaya went forth and spoke the verse at \href{https://suttacentral.net/thag1.26/en/sujato}{Thag 1.26}. | Not to be confused with Abhaya the \textsanskrit{Licchavī} of \href{https://suttacentral.net/an3.74/en/sujato}{AN 3.74} and \href{https://suttacentral.net/an4.196/en/sujato}{AN 4.196}; nor with the author of \href{https://suttacentral.net/thag1.98/en/sujato}{Thag 1.98}; nor is \href{https://suttacentral.net/thig2.8/en/sujato}{Thig 2.8} addressed to his mother, but that of the nun \textsanskrit{Abhayā} (\href{https://suttacentral.net/thig2.9/en/sujato}{Thig 2.9}). } “Come, prince, refute the ascetic Gotama’s doctrine. Then you will get a good reputation: ‘Prince Abhaya refuted the doctrine of the ascetic Gotama, so mighty and powerful!’” 

“But\marginnote{3.4} sir, how am I to do this?” 

“Here,\marginnote{3.5} prince, go to the ascetic Gotama and say to him: ‘Sir, may the Realized One utter speech that is disliked by others?’\footnote{The Jain doctrine \textit{\textsanskrit{anekantavāda}} (“doctrine of many-sidedness”, “non-categorical doctrine”) posits that in metaphysical propositions, one must conditionally affirm multiple possibilities so as to illuminate the complex, multifacted nature of reality. An aspect of \textit{\textsanskrit{anekantavāda}} is the \textit{syadvada}, which frames all statements with \textit{\textsanskrit{syā}}, “it may be”. This uses the optative tense to conditionally affirm or deny without making absolute statements. The same tense is employed here by \textsanskrit{Mahāvīra} in order to trap the Buddha on the horns of a dilemma (\textit{\textsanskrit{ubhatokoṭika}}). Needless to say, this is not how he is presented in Jain texts. } When he’s asked this, if he answers: ‘He may, prince,’ say this to him, ‘Then, sir, what exactly is the difference between you and an ordinary person?\footnote{When springing the trap, the voice shifts from the respectful third person to the familiar second person. } For even an ordinary person may utter speech that is disliked by others.’ But if he answers, ‘He may not, prince,’ say this to him: ‘Then, sir, why exactly did you declare of Devadatta: “Devadatta is going to a place of loss, to hell, there to remain for an eon, irredeemable”?\footnote{\textsanskrit{Mahāvīra} is quoting from the Buddha at \href{https://suttacentral.net/an8.7/en/sujato\#1.7}{AN 8.7:1.7} = \href{https://suttacentral.net/pli-tv-kd17/en/sujato\#4.7.1}{Kd 17:4.7.1}. This dates this sutta to the last years of the Buddha’s life, after the rebellion of Devadatta and the ascension of \textsanskrit{Ajātasattu}, and shortly before \textsanskrit{Mahāvīra}’s passing. } Devadatta was angry and upset with what you said.’ 

When\marginnote{3.16} you put this dilemma to him, the Buddha won’t be able to either spit it out or swallow it down. He’ll be like a man with an iron cross stuck in his throat, unable to either spit it out or swallow it down.” 

“Yes,\marginnote{4.1} sir,” replied Abhaya. He got up from his seat, bowed, and respectfully circled the Jain \textsanskrit{Ñātika}, keeping him on his right. Then he went to the Buddha, bowed, and sat down to one side. 

Then\marginnote{4.2} he looked up at the sun and thought, “It’s too late to refute the Buddha’s doctrine today. I shall refute his doctrine in my own home tomorrow.” He said to the Buddha, “Sir, may the Buddha please accept tomorrow’s meal from me, together with three other monks.” The Buddha consented with silence. 

Then,\marginnote{5.1} knowing that the Buddha had consented, Abhaya got up from his seat, bowed, and respectfully circled the Buddha, keeping him on his right, before leaving. 

Then\marginnote{5.2} when the night had passed, the Buddha robed up in the morning and, taking his bowl and robe, went to Abhaya’s home, and sat down on the seat spread out. Then Abhaya served and satisfied the Buddha with his own hands with delicious fresh and cooked foods. 

When\marginnote{5.4} the Buddha had eaten and washed his hand and bowl, Abhaya took a low seat, sat to one side, and said to him, “Sir, may the Realized One utter speech that is disliked by others?” 

“This\marginnote{6.3} matter is not categorical, prince.”\footnote{The Buddha solves the dilemma by avoiding a categorical statement, thus beating the Jains at their own game. Unlike the Jains, the Buddha did not advocate giving non-categorical answers as a general principle; different questions require different approaches when answering (\href{https://suttacentral.net/an3.67/en/sujato}{AN 3.67}, \href{https://suttacentral.net/mn136/en/sujato\#5.3}{MN 136:5.3}). Cf. \href{https://suttacentral.net/an3.78/en/sujato}{AN 3.78} and \href{https://suttacentral.net/an3.21/en/sujato\#9.1}{AN 3.21:9.1}. } 

“Then\marginnote{6.4} the Jains have lost in this, sir.” 

“But\marginnote{6.5} prince, why do you say that the Jains have lost in this?” 

Then\marginnote{6.7} Abhaya told the Buddha all that had happened. 

Now\marginnote{7.1} at that time a little baby boy was sitting in Prince Abhaya’s lap. Then the Buddha said to Abhaya, “What do you think, prince? If—because of your negligence or his nursemaid’s negligence—your boy were to put a stick or stone in his mouth, what would you do to him?” 

“I’d\marginnote{7.5} try to take it out, sir. If that didn’t work, I’d cradle his head with my left hand and take it out using a hooked finger of my right hand, even if it drew blood. Why is that? Because I have sympathy for the boy, sir.” 

“In\marginnote{8.1} the same way, prince, the Realized One does not utter speech that he knows to be untrue, false, and pointless, and which is disliked by others. The Realized One does not utter speech that he knows to be true and correct, but which is harmful and disliked by others. The Realized One knows the right time to speak so as to explain what he knows to be true, correct, and beneficial, but which is disliked by others.\footnote{For the phrase “speak so as to explain” (\textit{\textsanskrit{vācāya} \textsanskrit{veyyākaraṇāya}}) I follow the commentary. } The Realized One does not utter speech that he knows to be untrue, false, and pointless, but which is liked by others. The Realized One does not utter speech that he knows to be true and correct, but which is harmful, even if it is liked by others. The Realized One knows the right time to speak so as to explain what he knows to be true, correct, and beneficial, and which is liked by others. Why is that? Because the Realized One has sympathy for sentient beings.”\footnote{Here the optatives are gone and the Buddha states his views directly, showing that just because a question is not “categorical” does not mean that it cannot be given a definitive answer. However, it cannot be answered all-at-once or “categorically”, but must be broken into parts by “analysis” before answering. } 

“Sir,\marginnote{9.1} there are clever aristocrats, brahmins, householders, or ascetics who come to see you with a question already planned. Do you think beforehand that if they ask you like this, you’ll answer like that, or does the answer just strike you on the spot?” 

“Well\marginnote{10.1} then, prince, I’ll ask you about this in return, and you can answer as you like. What do you think, prince? Are you skilled in the various parts of a chariot?” 

“I\marginnote{10.4} am, sir.” 

“What\marginnote{10.5} do you think, prince? When they come to you and ask: ‘What’s the name of this chariot part?’ Do you think beforehand that if they ask you like this, you’ll answer like that, or does the answer appear to you on the spot?” 

“Sir,\marginnote{10.9} I’m well-known as a charioteer skilled in a chariot’s parts. All the major and minor parts are well-known to me. The answer just appears to me on the spot.” 

“In\marginnote{11.1} the same way, when clever aristocrats, brahmins, householders, or ascetics come to see me with a question already planned, the answer just strikes me on the spot. Why is that? Because the Realized One has clearly comprehended the principle of the teachings, so that the answer just strikes him on the spot.” 

When\marginnote{12.1} he had spoken, Prince Abhaya said to the Buddha, “Excellent, sir! Excellent! … From this day forth, may the Buddha remember me as a lay follower who has gone for refuge for life.” 

%
\section*{{\suttatitleacronym MN 59}{\suttatitletranslation The Many Kinds of Feeling }{\suttatitleroot Bahuvedanīyasutta}}
\addcontentsline{toc}{section}{\tocacronym{MN 59} \toctranslation{The Many Kinds of Feeling } \tocroot{Bahuvedanīyasutta}}
\markboth{The Many Kinds of Feeling }{Bahuvedanīyasutta}
\extramarks{MN 59}{MN 59}

\scevam{So\marginnote{1.1} I have heard. }At one time the Buddha was staying near \textsanskrit{Sāvatthī} in Jeta’s Grove, \textsanskrit{Anāthapiṇḍika}’s monastery. 

Then\marginnote{1.3} the chamberlain \textsanskrit{Pañcakaṅga} went up to Venerable \textsanskrit{Udāyī}, bowed, sat down to one side, and said to him,\footnote{\textsanskrit{Pañcakaṅga} features as a prominent devotee of the Buddha in \href{https://suttacentral.net/mn78/en/sujato}{MN 78}, \href{https://suttacentral.net/mn127/en/sujato}{MN 127}, and in a parallel to the current sutta at \href{https://suttacentral.net/sn36.19/en/sujato}{SN 36.19}. As \textit{thapati} (“chamberlain”) he was the ranking official under Pasenadi. Śatapatha \textsanskrit{Brāhmaṇa} 5.4.4.17–18 tells us that royal authority goes from the king, to the king’s brother, to the \textit{thapati}, to the village head. Similarly, Atharva Veda 2.32.4a and 5.23.11a mention the \textit{thapati} directly after the king. \href{https://suttacentral.net/sn55.6/en/sujato\#15.1}{SN 55.6:15.1} shows the \textit{thapati} was trusted with the most intimate and sensitive duties of the court. Elsewhere we learn that they shared the meal and carriage with the king, accompanying him on military campaigns (\href{https://suttacentral.net/mn89/en/sujato\#18.1}{MN 89:18.1}), while \href{https://suttacentral.net/sn55.6/en/sujato\#1.4}{SN 55.6:1.4} shows them engaged in governing the provinces. Thus the \textit{thapati} was the highest non-royal official in the court, entrusted with governing, advising, and security of the household. | There are several individuals called \textsanskrit{Udāyī}, and it is not possible to distinguish them all. } “Sir, how many feelings has the Buddha spoken of?” 

“Chamberlain,\marginnote{1.5} the Buddha has spoken of three feelings: pleasant, painful, and neutral. The Buddha has spoken of these three feelings.” 

When\marginnote{1.8} he said this, \textsanskrit{Pañcakaṅga} said to \textsanskrit{Udāyī}, “Sir, \textsanskrit{Udāyī}, the Buddha hasn’t spoken of three feelings. He’s spoken of two feelings: pleasant and painful. The Buddha said that neutral feeling is included as a peaceful and subtle kind of pleasure.” 

For\marginnote{2.1} a second time, \textsanskrit{Udāyī} said to \textsanskrit{Pañcakaṅga}, “The Buddha hasn’t spoken of two feelings, he’s spoken of three.” For a second time, \textsanskrit{Pañcakaṅga} said to \textsanskrit{Udāyī}, “The Buddha hasn’t spoken of three feelings, he’s spoken of two.” 

And\marginnote{3.1} for a third time, \textsanskrit{Udāyī} said to \textsanskrit{Pañcakaṅga}, “The Buddha hasn’t spoken of two feelings, he’s spoken of three.” 

And\marginnote{3.6} for a third time, \textsanskrit{Pañcakaṅga} said to \textsanskrit{Udāyī}, “The Buddha hasn’t spoken of three feelings, he’s spoken of two.” 

But\marginnote{3.11} neither was able to persuade the other. 

Venerable\marginnote{4.1} Ānanda heard this discussion between \textsanskrit{Udāyī} and \textsanskrit{Pañcakaṅga}. Then he went up to the Buddha, bowed, sat down to one side, and informed the Buddha of all they had discussed. When he had spoken, the Buddha said to him, 

“Ānanda,\marginnote{5.1} the explanation by the mendicant \textsanskrit{Udāyī}, which the chamberlain \textsanskrit{Pañcakaṅga} didn’t agree with, was quite correct. But the explanation by \textsanskrit{Pañcakaṅga}, which \textsanskrit{Udāyī} didn’t agree with, was also quite correct.\footnote{The Buddha’s reply here develops the theme of “non-categorical” statements introduced in the previous sutta, showing that the same reality can be spoken of in many different ways. } In one explanation I’ve spoken of two feelings. In another explanation I’ve spoken of three feelings, or five, six, eighteen, thirty-six, or a hundred and eight feelings.\footnote{These are all explained in \href{https://suttacentral.net/sn36.22/en/sujato}{SN 36.22}. } I’ve explained the teaching in all these different ways. This being so, you can expect that those who don’t concede, approve, or agree with what has been well spoken will argue, quarrel, and fight, continually wounding each other with barbed words. I’ve explained the teaching in all these different ways. This being so, you can expect that those who do concede, approve, or agree with what has been well spoken will live in harmony, appreciating each other, without quarreling, blending like milk and water, and regarding each other with kindly eyes. 

There\marginnote{6.1} are these five kinds of sensual stimulation. What five? Sights known by the eye, which are likable, desirable, agreeable, pleasant, sensual, and arousing. Sounds known by the ear … Smells known by the nose … Tastes known by the tongue … Touches known by the body, which are likable, desirable, agreeable, pleasant, sensual, and arousing. These are the five kinds of sensual stimulation. The pleasure and happiness that arise from these five kinds of sensual stimulation is called sensual pleasure. 

There\marginnote{7.1} are those who would say that this is the highest pleasure and happiness that sentient beings experience. But I don’t grant them that. Why is that? Because there is another pleasure that is finer than that.\footnote{In discussing the path, the Buddha focuses on pleasure. } And what is that pleasure? It’s when a mendicant, quite secluded from sensual pleasures, secluded from unskillful qualities, enters and remains in the first absorption, which has the rapture and bliss born of seclusion, while placing the mind and keeping it connected. This is a pleasure that is finer than that. 

There\marginnote{8.1} are those who would say that this is the highest pleasure and happiness that sentient beings experience. But I don’t grant them that. Why is that? Because there is another pleasure that is finer than that. And what is that pleasure? It’s when, as the placing of the mind and keeping it connected are stilled, a mendicant enters and remains in the second absorption, which has the rapture and bliss born of immersion, with internal clarity and mind at one, without placing the mind and keeping it connected. … 

There\marginnote{9.1} is another pleasure that is finer than that. And what is that pleasure? It’s when, with the fading away of rapture, a mendicant enters and remains in the third absorption, where they meditate with equanimity, mindful and aware, personally experiencing the bliss of which the noble ones declare, ‘Equanimous and mindful, one meditates in bliss.’ … 

There\marginnote{10.1} is another pleasure that is finer than that. And what is that pleasure? It’s when, giving up pleasure and pain, and ending former happiness and sadness, a mendicant enters and remains in the fourth absorption, without pleasure or pain, with pure equanimity and mindfulness. …\footnote{This validates \textsanskrit{Pañcakaṅga}’s view that neutral feeling can be reckoned as a subtle kind of pleasure. } 

There\marginnote{11.1} is another pleasure that is finer than that. And what is that pleasure? It’s when a mendicant, going totally beyond perceptions of form, with the ending of perceptions of impingement, not focusing on perceptions of diversity, aware that ‘space is infinite’, enters and remains in the dimension of infinite space. … 

There\marginnote{12.1} is another pleasure that is finer than that. And what is that pleasure? It’s when a mendicant, going totally beyond the dimension of infinite space, aware that ‘consciousness is infinite’, enters and remains in the dimension of infinite consciousness. … 

There\marginnote{13.1} is another pleasure that is finer than that. And what is that pleasure? It’s when a mendicant, going totally beyond the dimension of infinite consciousness, aware that ‘there is nothing at all’, enters and remains in the dimension of nothingness. … 

There\marginnote{14.1} is another pleasure that is finer than that. And what is that pleasure? It’s when a mendicant, going totally beyond the dimension of nothingness, enters and remains in the dimension of neither perception nor non-perception. This is a pleasure that is finer than that. 

There\marginnote{15.1} are those who would say that this is the highest pleasure and happiness that sentient beings experience. But I don’t grant them that. Why is that? Because there is another pleasure that is finer than that. And what is that pleasure? It’s when a mendicant, going totally beyond the dimension of neither perception nor non-perception, enters and remains in the cessation of perception and feeling. This is a pleasure that is finer than that.\footnote{Since all feeling is conditioned, and what is conditioned is suffering, the cessation of feeling is reckoned as pleasurable. } 

It’s\marginnote{16.1} possible that wanderers of other religions might say, ‘The ascetic Gotama spoke of the cessation of perception and feeling, and he includes it in happiness. What’s up with that?’ 

When\marginnote{16.4} wanderers of other religions say this, you should say to them, ‘Reverends, when the Buddha describes what’s included in happiness, he’s not just referring to pleasant feeling. The Realized One describes pleasure as included in happiness wherever it is found, and in whatever context.’” 

That\marginnote{16.7} is what the Buddha said. Satisfied, Venerable Ānanda approved what the Buddha said. 

%
\section*{{\suttatitleacronym MN 60}{\suttatitletranslation Unfailing }{\suttatitleroot Apaṇṇakasutta}}
\addcontentsline{toc}{section}{\tocacronym{MN 60} \toctranslation{Unfailing } \tocroot{Apaṇṇakasutta}}
\markboth{Unfailing }{Apaṇṇakasutta}
\extramarks{MN 60}{MN 60}

\scevam{So\marginnote{1.1} I have heard.\footnote{This sutta subverts the imagery of gambling by reasoning towards logical certitude in a world of apparently random chances. Gambling was a major feature of Vedic culture, with the moving confession of Rig Veda 10.34 recounting the thrill and loss of the game. The gods were invoked to ensure success (Atharvaveda 7.109). Even kings bowed to the dice (Rig Veda 10.34.8), so that the \textsanskrit{Rājasūya} consecration ceremony is secured with a (loaded) game of dice (Śatapatha \textsanskrit{Brāhmaṇa} 5.4.4.6, 23). Numerous tales of kings losing their realms at dice, most notably \textsanskrit{Yudhiṣṭhira} in the \textsanskrit{Mahābhārata}, show that the power and danger of gambling was not merely symbolic. } }At one time the Buddha was wandering in the land of the Kosalans together with a large \textsanskrit{Saṅgha} of mendicants when he arrived at a village of the Kosalan brahmins named \textsanskrit{Sālā}.\footnote{The same framing narrative recurs at \href{https://suttacentral.net/mn41/en/sujato}{MN 41}, where the topic of rebirth is also at stake. } 

The\marginnote{2.1} brahmins and householders of \textsanskrit{Sālā} heard: 

“It\marginnote{2.2} seems the ascetic Gotama—a Sakyan, gone forth from a Sakyan family—wandering in the land of the Kosalans has arrived at \textsanskrit{Sālā}, together with a large \textsanskrit{Saṅgha} of mendicants. He has this good reputation: ‘That Blessed One is perfected, a fully awakened Buddha, accomplished in knowledge and conduct, holy, knower of the world, supreme guide for those who wish to train, teacher of gods and humans, awakened, blessed.’ He has realized with his own insight this world—with its gods, \textsanskrit{Māras}, and divinities, this population with its ascetics and brahmins, gods and humans—and he makes it known to others. He proclaims a teaching that is good in the beginning, good in the middle, and good in the end, meaningful and well-phrased. And he reveals a spiritual practice that’s entirely full and pure. It’s good to see such perfected ones.” 

Then\marginnote{3.1} the brahmins and householders of \textsanskrit{Sālā} went up to the Buddha. Before sitting down to one side, some bowed, some exchanged greetings and polite conversation, some held up their joined palms toward the Buddha, some announced their name and clan, while some kept silent. The Buddha said to them: 

“So,\marginnote{4.1} householders, is there any teacher you’re happy with, in whom you have acquired grounded faith?”\footnote{The Brahmanical community was in a period of questioning and transition, with multiple different perspectives offered by both Brahmins and ascetics. } 

“No,\marginnote{4.2} sir.” 

“Since\marginnote{4.3} you haven’t found a teacher you’re happy with, you should undertake and implement this unfailing teaching.\footnote{While the contextual sense of the key term \textit{\textsanskrit{apaṇṇaka}} (“unfailing”) is clear, the etymology has proven a puzzle. I propose that it is a gambling term, and \textit{\textsanskrit{paṇṇa}} appears as a form of the number five. While a set of four dice was “perfect”, the fifth “losing” (\textit{kali}) die was all-powerful, since just one extra die meant you lose everything (Rig Veda 10.34.2, Śatapatha \textsanskrit{Brāhmaṇa} 5.4.4.6). \textit{\textsanskrit{Apaṇṇaka}} is therefore literally a “set without a fifth” and metaphorically “without fail”. } For when the unfailing teaching is undertaken, it will be for your lasting welfare and happiness. And what is the unfailing teaching? 

There\marginnote{5.1} are some ascetics and brahmins who have this doctrine and view: ‘There’s no meaning in giving, sacrifice, or offerings. There’s no fruit or result of good and bad deeds. There’s no afterlife. There’s no such thing as mother and father, or beings that are reborn spontaneously. And there’s no ascetic or brahmin who is rightly comported and rightly practiced, and who describes the afterlife after realizing it with their own insight.’\footnote{At \href{https://suttacentral.net/dn2/en/sujato\#23.2}{DN 2:23.2} this is presented as the moral view of Ajita of the Hair Blanket, who justifies it with a materialist theory. } 

And\marginnote{6.1} there are some ascetics and brahmins whose doctrine directly contradicts this. They say: ‘There is meaning in giving, sacrifice, and offerings. There are fruits and results of good and bad deeds. There is an afterlife. There are such things as mother and father, and beings that are reborn spontaneously. And there are ascetics and brahmins who are rightly comported and rightly practiced, and who describe the afterlife after realizing it with their own insight.’\footnote{The positive side is of course the Buddha’s view. } 

What\marginnote{6.4} do you think, householders? Don’t these doctrines directly contradict each other?” 

“Yes,\marginnote{6.6} sir.” 

“Since\marginnote{7.1} this is so, consider those ascetics and brahmins whose view is that there’s no meaning in giving, etc. You can expect that they will reject good conduct by way of body, speech, and mind, and undertake and implement bad conduct by way of body, speech, and mind. Why is that? Because those ascetics and brahmins don’t see that unskillful qualities are full of drawbacks, sordidness, and corruption, or that skillful qualities have the benefit and cleansing power of renunciation. 

Moreover,\marginnote{8.1} since there actually is another world, their view that there is no other world is wrong view. Since there actually is another world, their thought that there is no other world is wrong thought. Since there actually is another world, their speech that there is no other world is wrong speech. Since there actually is another world, in saying that there is no other world they contradict those perfected ones who know the other world. Since there actually is another world, in convincing another that there is no other world they are convincing them to accept an untrue teaching. And on account of that they glorify themselves and put others down. So they give up their former ethical conduct and are established in unethical conduct. And that is how these many bad, unskillful qualities come to be with wrong view as condition—wrong view, wrong thought, wrong speech, contradicting the noble ones, convincing others to accept untrue teachings, and glorifying oneself and putting others down. 

A\marginnote{9.1} sensible person reflects on this matter in this way: ‘If there is no other world, when this individual’s body breaks up they will keep themselves safe.\footnote{Given that they no longer exist, how are they safe? Bodhi’s “will have made himself safe” is unlikely, as it is future perfect, a rare sense that requires the past participle. Anyway he hasn’t “made himself” anything, he just happen to live in a world without moral consequences. Horner notes the problem without offering a solution, there are no parallels, and commentaries are silent. I think the text is corrupt. } And if there is another world, when their body breaks up, after death, they will be reborn in a place of loss, a bad place, the underworld, hell. But let us grant that those who say that there is no other world are correct.\footnote{The syntax of this sentence is tricky. \textit{\textsanskrit{Kāmaṁ}} is “willingly”, which together with \textit{hotu} conveys the sense “let us grant …”. \textit{\textsanskrit{Māhu}} resolves to \textit{\textsanskrit{mā} ahu}, a simple negation rather than a prohibition. For \textit{\textsanskrit{kāmaṁ}} with \textit{\textsanskrit{mā}} and imperative, compare \href{https://suttacentral.net/sn11.4/en/sujato\#7.1}{SN 11.4:7.1}. } Regardless, that individual is still criticized by sensible people in this very life as being an immoral individual of wrong view, a nihilist.’\footnote{“Nihilist” is \textit{\textsanskrit{natthikavāda}}, “one whose doctrine is that there is nothing”. While this view is discussed many times in the suttas, the word \textit{\textsanskrit{natthikavāda}} is elsewhere only applied to a specific pair of persons, Vassa and \textsanskrit{Bhañña} of \textsanskrit{Ukkalā} (\href{https://suttacentral.net/mn117/en/sujato\#38.1}{MN 117:38.1}, \href{https://suttacentral.net/sn22.62/en/sujato\#13.2}{SN 22.62:13.2}, \href{https://suttacentral.net/an4.30/en/sujato\#8.2}{AN 4.30:8.2}). } But if there really is another world, they hold a losing hand on both counts.\footnote{“Losing hand” is \textit{\textsanskrit{kaliggāha}}. } For they are criticized by sensible people in this very life, and when their body breaks up, after death, they will be reborn in a place of loss, a bad place, the underworld, hell. They have wrongly undertaken this unfailing teaching in such a way that it encompasses the positive outcomes of one side only, leaving out the skillful premise.\footnote{It is difficult to choose between “skillful” and “wholesome” as renderings for \textit{kusala}. “Skillful” sticks closer to the root sense, while “wholesome” is more idiomatic. But the context here well illustrates that the root sense is still alive in the language, as the “skillful” option is that which wisely aligns cause and effect with one’s own well-being. } 

Since\marginnote{10.1} this is so, consider those ascetics and brahmins whose view is that there is meaning in giving, etc. You can expect that they will reject bad conduct by way of body, speech, and mind, and undertake and implement good conduct by way of body, speech, and mind. Why is that? Because those ascetics and brahmins see that unskillful qualities are full of drawbacks, sordidness, and corruption, and that skillful qualities have the benefit and cleansing power of renunciation. 

Moreover,\marginnote{11.1} since there actually is another world, their view that there is another world is right view. Since there actually is another world, their thought that there is another world is right thought. Since there actually is another world, their speech that there is another world is right speech. Since there actually is another world, in saying that there is another world they don’t contradict those perfected ones who know the other world. Since there actually is another world, in convincing another that there is another world they are convincing them to accept a true teaching. And on account of that they don’t glorify themselves or put others down. So they give up their former unethical conduct and are established in ethical conduct. And that is how these many skillful qualities come to be with right view as condition—right view, right thought, right speech, not contradicting the noble ones, convincing others to accept true teachings, and not glorifying oneself or putting others down. 

A\marginnote{12.1} sensible person reflects on this matter in this way: ‘If there is another world, when this individual’s body breaks up, after death, they will be reborn in a good place, a heavenly realm. But let us grant that those who say that there is no other world are correct. Regardless, that individual is still praised by sensible people in this very life as being a moral individual of right view, who affirms a positive teaching.’ So if there really is another world, they hold a perfect hand on both counts.\footnote{“Perfect hand” (\textit{\textsanskrit{kaṭaggaha}}) is an expression from the classical Vedic game of chance. The details are obscure and would have changed over time, but it seems that a large quantity of \textit{\textsanskrit{vibhītaka}} (“bedda”) nuts were cast in a hollow, from which players took a handful. If the number of nuts was divisible by four, it was said to be a “perfect” (\textit{\textsanskrit{kaṭa}}; see Rig Veda 1.132.1, etc.) “hand” (\textit{\textsanskrit{gāha}}). | \textit{\textsanskrit{Apaṇṇaka}} and \textit{\textsanskrit{kaṭaggāha}} are also connected at \href{https://suttacentral.net/sn42.13/en/sujato\#23.5}{SN 42.13:23.5}; compare the “gem” thrown like a loaded die at \href{https://suttacentral.net/an3.118/en/sujato\#4.6}{AN 3.118:4.6} and \href{https://suttacentral.net/an10.217/en/sujato\#17.1}{AN 10.217:17.1}. Good rebirth is further compared to a “perfect hand” at \href{https://suttacentral.net/mn129/en/sujato\#49.1}{MN 129:49.1}. And at \href{https://suttacentral.net/ja1/en/sujato\#2.4}{Ja 1:2.4}, a wise person would “take up” (\textit{\textsanskrit{gaṇhe}}) what is unfailing, namely a winning hand. } For they are praised by sensible people in this very life, and when their body breaks up, after death, they will be reborn in a good place, a heavenly realm. They have rightly undertaken this unfailing teaching in such a way that it encompasses the positive outcomes of both sides, leaving out the unskillful premise. 

There\marginnote{13.1} are some ascetics and brahmins who have this doctrine and view: ‘The one who acts does nothing wrong when they punish, mutilate, torture, aggrieve, oppress, intimidate, or when they encourage others to do the same. They do nothing wrong when they kill, steal, break into houses, plunder wealth, steal from isolated buildings, commit highway robbery, commit adultery, and lie.\footnote{At \href{https://suttacentral.net/dn2/en/sujato\#17.2}{DN 2:17.2}, this denial of the doctrine of kamma is attributed to \textsanskrit{Pūraṇa} Kassapa. He may have subscribed to hard determinism, so that we have no choice in what we do. He may also have believed that we should keep moral rules as a social contract, but that this had no effect on the afterlife. | In such contexts, \textit{kar-} means “punish, inflict” (\href{https://suttacentral.net/mn129/en/sujato\#29.2}{MN 129:29.2}). } If you were to reduce all the living creatures of this earth to one heap and mass of flesh with a razor-edged chakram, no evil comes of that, and no outcome of evil. If you were to go along the south bank of the Ganges killing, mutilating, and torturing, and encouraging others to do the same, no evil comes of that, and no outcome of evil. If you were to go along the north bank of the Ganges giving and sacrificing and encouraging others to do the same, no merit comes of that, and no outcome of merit. In giving, self-control, restraint, and truthfulness there is no merit or outcome of merit.’ 

And\marginnote{14.1} there are some ascetics and brahmins whose doctrine directly contradicts this. They say: ‘The one who acts does a bad deed when they punish, mutilate, torture, aggrieve, oppress, intimidate, or when they encourage others to do the same. They do a bad deed when they kill, steal, break into houses, plunder wealth, steal from isolated buildings, commit highway robbery, commit adultery, and lie. If you were to reduce all the living creatures of this earth to one heap and mass of flesh with a razor-edged chakram, evil comes of that, and an outcome of evil. If you were to go along the south bank of the Ganges killing, mutilating, and torturing, and encouraging others to do the same, evil comes of that, and an outcome of evil. If you were to go along the north bank of the Ganges giving and sacrificing and encouraging others to do the same, merit comes of that, and an outcome of merit. In giving, self-control, restraint, and truthfulness there is merit and outcome of merit.’ 

What\marginnote{14.7} do you think, householders? Don’t these doctrines directly contradict each other?” 

“Yes,\marginnote{14.9} sir.” 

“Since\marginnote{15.1} this is so, consider those ascetics and brahmins whose view is that the one who acts does nothing wrong when they punish, etc. You can expect that they will reject good conduct by way of body, speech, and mind, and undertake and implement bad conduct by way of body, speech, and mind. Why is that? Because those ascetics and brahmins don’t see that unskillful qualities are full of drawbacks, sordidness, and corruption, or that skillful qualities have the benefit and cleansing power of renunciation. 

Moreover,\marginnote{16.1} since action actually does have an effect, their view that action is ineffective is wrong view. Since action actually does have an effect, their thought that action is ineffective is wrong thought. Since action actually does have an effect, their speech that action is ineffective is wrong speech. Since action actually does have an effect, in saying that action is ineffective they contradict those perfected ones who teach that action is effective. Since action actually does have an effect, in convincing another that action is ineffective they are convincing them to accept an untrue teaching. And on account of that they glorify themselves and put others down. So they give up their former ethical conduct and are established in unethical conduct. And that is how these many bad, unskillful qualities come to be with wrong view as condition—wrong view, wrong thought, wrong speech, contradicting the noble ones, convincing others to accept untrue teachings, and glorifying oneself and putting others down. 

A\marginnote{17.1} sensible person reflects on this matter in this way: ‘If there is no effective action, when this individual’s body breaks up they will keep themselves safe. And if there is effective action, when their body breaks up, after death, they will be reborn in a place of loss, a bad place, the underworld, hell. But let us grant that those who say that there is no effective action are correct. Regardless, that individual is still criticized by sensible people in this very life as being an immoral individual of wrong view, one who denies the efficacy of action.’\footnote{“One who denies the efficacy of action” is \textit{\textsanskrit{akiriyavāda}}. } But if there really is effective action, they hold a losing hand on both counts. For they are criticized by sensible people in this very life, and when their body breaks up, after death, they will be reborn in a place of loss, a bad place, the underworld, hell. They have wrongly undertaken this unfailing teaching in such a way that it encompasses the positive outcomes of one side only, leaving out the skillful premise. 

Since\marginnote{18.1} this is so, consider those ascetics and brahmins whose view is that the one who acts does a bad deed when they punish, etc. You can expect that they will reject bad conduct by way of body, speech, and mind, and undertake and implement good conduct by way of body, speech, and mind. Why is that? Because those ascetics and brahmins see that unskillful qualities are full of drawbacks, sordidness, and corruption, and that skillful qualities have the benefit and cleansing power of renunciation. 

Moreover,\marginnote{19.1} since action actually does have an effect, their view that action is effective is right view. Since action actually does have an effect, their thought that action is effective is right thought. Since action actually does have an effect, their speech that action is effective is right speech. Since action actually does have an effect, in saying that action is effective they don’t contradict those perfected ones who teach that action is effective. Since action actually does have an effect, in convincing another that action is effective they are convincing them to accept a true teaching. And on account of that they don’t glorify themselves or put others down. So they give up their former unethical conduct and are established in ethical conduct. And that is how these many skillful qualities come to be with right view as condition—right view, right thought, right speech, not contradicting the noble ones, convincing others to accept true teachings, and not glorifying oneself or putting others down. 

A\marginnote{20.1} sensible person reflects on this matter in this way: ‘If there is effective action, when this individual’s body breaks up, after death, they will be reborn in a good place, a heavenly realm. But let us grant that those who say that there is no effective action are correct. Regardless, that individual is still praised by sensible people in this very life as being a moral individual of right view, who affirms the efficacy of action.’ So if there really is effective action, they hold a perfect hand on both counts. For they are praised by sensible people in this very life, and when their body breaks up, after death, they will be reborn in a good place, a heavenly realm. They have rightly undertaken this unfailing teaching in such a way that it encompasses the positive outcomes of both sides, leaving out the unskillful premise. 

There\marginnote{21.1} are some ascetics and brahmins who have this doctrine and view: ‘There is no cause or reason for the corruption of sentient beings.\footnote{At \href{https://suttacentral.net/dn2/en/sujato\#20.2}{DN 2:20.2} this view is attributed to the Bamboo-staffed Ascetic \textsanskrit{Gosāla}, the founder of the \textsanskrit{Ājīvikas} and teacher of fatalism. } Sentient beings are corrupted without cause or reason. There’s no cause or reason for the purification of sentient beings. Sentient beings are purified without cause or reason. There is no power, no energy, no human strength or vigor. All sentient beings, all living creatures, all beings, all souls lack control, power, and energy. Molded by destiny, circumstance, and nature, they experience pleasure and pain in the six classes of rebirth.’ 

And\marginnote{22.1} there are some ascetics and brahmins whose doctrine directly contradicts this. They say: ‘There is a cause and reason for the corruption of sentient beings. Sentient beings are corrupted with cause and reason. There is a cause and reason for the purification of sentient beings. Sentient beings are purified with cause and reason. There is power, energy, human strength and vigor. It is not the case that all sentient beings, all living creatures, all beings, all souls lack control, power, and energy, or that, molded by destiny, circumstance, and nature, they experience pleasure and pain in the six classes of rebirth.’ 

What\marginnote{22.9} do you think, householders? Don’t these doctrines directly contradict each other?” 

“Yes,\marginnote{22.11} sir.” 

“Since\marginnote{23.1} this is so, consider those ascetics and brahmins whose view is that there’s no cause or reason for the corruption of sentient beings, etc. You can expect that they will reject good conduct by way of body, speech, and mind, and undertake and implement bad conduct by way of body, speech, and mind. Why is that? Because those ascetics and brahmins don’t see that unskillful qualities are full of drawbacks, sordidness, and corruption, or that skillful qualities have the benefit and cleansing power of renunciation. 

Moreover,\marginnote{24.1} since there actually is causality, their view that there is no causality is wrong view. Since there actually is causality, their thought that there is no causality is wrong thought. Since there actually is causality, their speech that there is no causality is wrong speech. Since there actually is causality, in saying that there is no causality they contradict those perfected ones who teach that there is causality. Since there actually is causality, in convincing another that there is no causality they are convincing them to accept an untrue teaching. And on account of that they glorify themselves and put others down. So they give up their former ethical conduct and are established in unethical conduct. And that is how these many bad, unskillful qualities come to be with wrong view as condition—wrong view, wrong thought, wrong speech, contradicting the noble ones, convincing others to accept untrue teachings, and glorifying oneself and putting others down. 

A\marginnote{25.1} sensible person reflects on this matter in this way: ‘If there is no causality, when this individual’s body breaks up they will keep themselves safe. And if there is causality, when their body breaks up, after death, they will be reborn in a place of loss, a bad place, the underworld, hell. But let us grant that those who say that there is no causality are correct. Regardless, that individual is still criticized by sensible people in this very life as being an immoral individual of wrong view, one who denies causality.’\footnote{“One who denies causality” is \textit{\textsanskrit{ahetukavāda}}. } But if there really is causality, they hold a losing hand on both counts. For they are criticized by sensible people in this very life, and when their body breaks up, after death, they will be reborn in a place of loss, a bad place, the underworld, hell. They have wrongly undertaken this unfailing teaching in such a way that it encompasses the positive outcomes of one side only, leaving out the skillful premise. 

Since\marginnote{26.1} this is so, consider those ascetics and brahmins whose view is that there is a cause and reason for the corruption of sentient beings, etc. You can expect that they will reject bad conduct by way of body, speech, and mind, and undertake and implement good conduct by way of body, speech, and mind. Why is that? Because those ascetics and brahmins see that unskillful qualities are full of drawbacks, sordidness, and corruption, and that skillful qualities have the benefit and cleansing power of renunciation. 

Moreover,\marginnote{27.1} since there actually is causality, their view that there is causality is right view. Since there actually is causality, their thought that there is causality is right thought. Since there actually is causality, their speech that there is causality is right speech. Since there actually is causality, in saying that there is causality they don’t contradict those perfected ones who teach that there is causality. Since there actually is causality, in convincing another that there is causality they are convincing them to accept a true teaching. And on account of that they don’t glorify themselves or put others down. So they give up their former unethical conduct and are established in ethical conduct. And that is how these many skillful qualities come to be with right view as condition—right view, right thought, right speech, not contradicting the noble ones, convincing others to accept true teachings, and not glorifying oneself or putting others down. 

A\marginnote{28.1} sensible person reflects on this matter in this way: ‘If there is causality, when this individual’s body breaks up, after death, they will be reborn in a good place, a heavenly realm. But let us grant that those who say that there is no causality are correct. Regardless, that individual is still praised by sensible people in this very life as being a moral individual of right view, who affirms causality.’ So if there really is causality, they hold a perfect hand on both counts. For they are praised by sensible people in this very life, and when their body breaks up, after death, they will be reborn in a good place, a heavenly realm. They have rightly undertaken this unfailing teaching in such a way that it encompasses the positive outcomes of both sides, leaving out the unskillful premise. 

There\marginnote{29.1} are some ascetics and brahmins who have this doctrine and view: ‘There are no totally formless states of meditation.’\footnote{The four “formless states” (\textit{\textsanskrit{āruppā}}) are attained with the “total” (\textit{sabbaso}) surmounting of form. While the previous views characterize prominent ascetic movements, the final view relates to the formless meditations, the cardinal teaching of the contemplative movement of the Kosalan brahmins, represented by such figures as \textsanskrit{Āḷāra} \textsanskrit{Kālāma} and Uddaka \textsanskrit{Rāmaputta} (\href{https://suttacentral.net/mn36/en/sujato\#14.9}{MN 36:14.9}) or the sixteen brahmins of the \textsanskrit{Pārāyanavagga} (\href{https://suttacentral.net/snp5.1/en/sujato}{Snp 5.1} ff.). The Buddha elevates his teaching by directly addressing the highest and best of contemporary Brahmanical practices. } 

And\marginnote{30.1} there are some ascetics and brahmins whose doctrine directly contradicts this. They say: ‘There are totally formless states of meditation.’ 

What\marginnote{30.4} do you think, householders? Don’t these doctrines directly contradict each other?” 

“Yes,\marginnote{30.6} sir.” 

“A\marginnote{31.1} sensible person reflects on this matter in this way: ‘Some ascetics and brahmins say that there are no totally formless meditations, but I have not seen that. Some ascetics and brahmins say that there are totally formless meditations, but I have not known that. Without knowing or seeing, it would not be appropriate for me to take one side and declare, ‘This is the only truth, anything else is futile.’ If those ascetics and brahmins who say that there are no totally formless meditations are correct, it is possible that I will be unfailingly reborn among the gods who are formed and made of mind. If those ascetics and brahmins who say that there are totally formless meditations are correct, it is possible that I will be unfailingly reborn among the gods who are formless and made of perception. Now, owing to form, bad things are seen: taking up the rod and the sword, quarrels, arguments, and disputes, accusations, divisive speech, and lies. But those things don’t exist where it is totally formless.’ Reflecting like this, they simply practice for disillusionment, dispassion, and cessation regarding forms. 

There\marginnote{32.1} are some ascetics and brahmins who have this doctrine and view: ‘There is no such thing as the total cessation of continued existence.’\footnote{This denies the basic tenet of dependent origination. \textit{Bhava} means “being”, but in the pregnant philosophical sense of “continued existence”, which often takes the form of “future lives”. It can be broader than that, however, as with the \textsanskrit{Upaniṣadic} doctrine of eternal unity of the contingent personal self with the absolute universal divinity. Since the Brahmins saw the cosmos as an expression of divinity, to them, existence must be inherently good and the cessation of existence abhorrent. } 

And\marginnote{33.1} there are some ascetics and brahmins whose doctrine directly contradicts this. They say: ‘There is such a thing as the total cessation of continued existence.’ 

What\marginnote{33.4} do you think, householders? Don’t these doctrines directly contradict each other?” 

“Yes,\marginnote{33.6} sir.” 

“A\marginnote{34.1} sensible person reflects on this matter in this way: ‘Some ascetics and brahmins say that there is no such thing as the total cessation of continued existence, but I have not seen that. Some ascetics and brahmins say that there is such a thing as the total cessation of continued existence, but I have not known that. Without knowing or seeing, it would not be appropriate for me to take one side and declare, ‘This is the only truth, anything else is futile.’ If those ascetics and brahmins who say that there is no such thing as the total cessation of continued existence are correct, it is possible that I will be unfailingly reborn among the gods who are formless and made of perception. If those ascetics and brahmins who say that there is such a thing as the total cessation of continued existence are correct, it is possible that I will be fully extinguished in this very life. The view of those ascetics and brahmins who say that there is no such thing as the total cessation of continued existence is close to greed, yoking, relishing, attachment, and grasping. The view of those ascetics and brahmins who say that there is such a thing as the total cessation of continued existence is close to non-greed, non-yoking, non-relishing, non-attachment, and non-grasping.’ Reflecting like this, they simply practice for disillusionment, dispassion, and cessation regarding future lives. 

Householders,\marginnote{35.1} these four people are found in the world. What four? 

\begin{enumerate}%
\item One person mortifies themselves, committed to the practice of mortifying themselves. %
\item One person mortifies others, committed to the practice of mortifying others.\footnote{See \href{https://suttacentral.net/mn51/en/sujato\#8.1}{MN 51:8.1}. } %
\item One person mortifies themselves and others, committed to the practice of mortifying themselves and others. %
\item One person doesn’t mortify either themselves or others, committed to the practice of not mortifying themselves or others. They live without wishes in this very life, quenched, cooled, experiencing bliss, with self become divine. %
\end{enumerate}

And\marginnote{36.1} what person mortifies themselves, committed to the practice of mortifying themselves? It’s when a person goes naked, ignoring conventions. … And so they live committed to practicing these various ways of mortifying and tormenting the body. This is called a person who mortifies themselves, being committed to the practice of mortifying themselves. 

And\marginnote{37.1} what person mortifies others, committed to the practice of mortifying others? It’s when a person is a butcher of sheep, pigs, poultry, or deer, a hunter or fisher, a bandit, an executioner, a butcher of cattle, a jailer, or has some other cruel livelihood. This is called a person who mortifies others, being committed to the practice of mortifying others. 

And\marginnote{38.1} what person mortifies themselves and others, being committed to the practice of mortifying themselves and others? It’s when a person is an anointed aristocratic king or a well-to-do brahmin. He has a new ceremonial hall built to the east of the citadel. He shaves off his hair and beard, dresses in a rough antelope hide, and smears his body with ghee and oil. …\footnote{Pali here is abbreviated, and I have expanded partially for clarity. For the full text, see \href{https://suttacentral.net/mn51/en/sujato\#10.8}{MN 51:10.8}. } His bondservants, servants, and workers do their jobs under threat of punishment and danger, weeping, with tearful faces. This is called a person who mortifies themselves and others, being committed to the practice of mortifying themselves and others. 

And\marginnote{39.1} what person doesn’t mortify either themselves or others, committed to the practice of not mortifying themselves or others, living without wishes in this very life, quenched, cooled, experiencing bliss, with self become divine? 

It’s\marginnote{40{-}54.1} when a Realized One arises in the world, perfected, a fully awakened Buddha … A householder hears that teaching, or a householder’s child, or someone reborn in some good family. … They give up these five hindrances, corruptions of the heart that weaken wisdom. Then, quite secluded from sensual pleasures, secluded from unskillful qualities, they enter and remain in the first absorption … second absorption … third absorption … fourth absorption. 

When\marginnote{55.1} their mind has become immersed in \textsanskrit{samādhi} like this—purified, bright, flawless, rid of corruptions, pliable, workable, steady, and imperturbable—they extend it toward recollection of past lives. … They recollect their many kinds of past lives, with features and details. 

When\marginnote{55.3} their mind has become immersed in \textsanskrit{samādhi} like this—purified, bright, flawless, rid of corruptions, pliable, workable, steady, and imperturbable—they extend it toward knowledge of the death and rebirth of sentient beings. With clairvoyance that is purified and superhuman, they see sentient beings passing away and being reborn—inferior and superior, beautiful and ugly, in a good place or a bad place. … They understand how sentient beings are reborn according to their deeds. 

When\marginnote{55.5} their mind has become immersed in \textsanskrit{samādhi} like this—purified, bright, flawless, rid of corruptions, pliable, workable, steady, and imperturbable—they extend it toward knowledge of the ending of defilements. They truly understand: ‘This is suffering’ … ‘This is the origin of suffering’ … ‘This is the cessation of suffering’ … ‘This is the practice that leads to the cessation of suffering’. They truly understand: ‘These are defilements’ … ‘This is the origin of defilements’ … ‘This is the cessation of defilements’ … ‘This is the practice that leads to the cessation of defilements’. Knowing and seeing like this, their mind is freed from the defilements of sensuality, desire to be reborn, and ignorance. When they’re freed, they know they’re freed. 

They\marginnote{55.10} understand: ‘Rebirth is ended, the spiritual journey has been completed, what had to be done has been done, there is nothing further for this place.’ 

This\marginnote{56.1} is called a person who neither mortifies themselves or others, being committed to the practice of not mortifying themselves or others. They live without wishes in this very life, quenched, cooled, experiencing bliss, with self become divine.” 

When\marginnote{57.1} he had spoken, the brahmins and householders of \textsanskrit{Sālā} said to the Buddha, “Excellent, Mister Gotama! Excellent! As if he were righting the overturned, or revealing the hidden, or pointing out the path to the lost, or lighting a lamp in the dark so people with clear eyes can see what’s there, Mister Gotama has made the teaching clear in many ways. We go for refuge to Mister Gotama, to the teaching, and to the mendicant \textsanskrit{Saṅgha}.\footnote{They are also said to have gone forth at \href{https://suttacentral.net/mn41/en/sujato\#44.4}{MN 41:44.4}. } From this day forth, may Mister Gotama remember us as lay followers who have gone for refuge for life.” 

%
\addtocontents{toc}{\let\protect\contentsline\protect\nopagecontentsline}
\chapter*{The Chapter on Mendicants }
\addcontentsline{toc}{chapter}{\tocchapterline{The Chapter on Mendicants }}
\addtocontents{toc}{\let\protect\contentsline\protect\oldcontentsline}

%
\section*{{\suttatitleacronym MN 61}{\suttatitletranslation Advice to Rāhula at Ambalaṭṭhika }{\suttatitleroot Ambalaṭṭhikarāhulovādasutta}}
\addcontentsline{toc}{section}{\tocacronym{MN 61} \toctranslation{Advice to Rāhula at Ambalaṭṭhika } \tocroot{Ambalaṭṭhikarāhulovādasutta}}
\markboth{Advice to Rāhula at Ambalaṭṭhika }{Ambalaṭṭhikarāhulovādasutta}
\extramarks{MN 61}{MN 61}

\scevam{So\marginnote{1.1} I have heard.\footnote{The Buddha teaches his son \textsanskrit{Rāhula} the dangers of lying, using gestures to illustrate his point. The simple subject matter and child-friendly methods support the commentary’s claim that this was when \textsanskrit{Rāhula} was still a boy, soon after he ordained (\href{https://suttacentral.net/pli-tv-kd1/en/sujato\#54.1.1}{Kd 1:54.1.1}). | This sutta is among those recommended for study by the \textsanskrit{Saṅgha} by King Ashoka in Minor Rock Edict No. 3, the Calcutta-\textsanskrit{Bairāṭ} rock inscription from Viratnagar in Rajasthan. There it is described as “the exhortation to \textsanskrit{Rāhula} spoken by the Lord Buddha concerning falsehood” (\textit{laghulo-\textsanskrit{vāde} \textsanskrit{musā}-\textsanskrit{vādaṁ} adhigicya \textsanskrit{bhagavatā} budhena \textsanskrit{bhāsite}}). } }At one time the Buddha was staying near \textsanskrit{Rājagaha}, in the Bamboo Grove, the squirrels’ feeding ground. 

Now\marginnote{2.1} at that time Venerable \textsanskrit{Rāhula} was staying at \textsanskrit{Ambalaṭṭhikā}.\footnote{The commentary says this is a building next to the Bamboo Grove rather than the site of the royal rest-house mentioned at \href{https://suttacentral.net/dn1/en/sujato\#1.2.1}{DN 1:1.2.1}. } Then in the late afternoon, the Buddha came out of retreat and went to \textsanskrit{Ambalaṭṭhika} to see Venerable \textsanskrit{Rāhula}. \textsanskrit{Rāhula} saw the Buddha coming off in the distance. He spread out a seat and placed water for washing the feet. The Buddha sat on the seat spread out, and washed his feet. \textsanskrit{Rāhula} bowed to the Buddha and sat down to one side. 

Then\marginnote{3.1} the Buddha, leaving a little water in the pot, addressed \textsanskrit{Rāhula}, “\textsanskrit{Rāhula}, do you see this little bit of water left in the pot?” 

“Yes,\marginnote{3.3} sir.” 

“That’s\marginnote{3.4} how little of the ascetic’s nature is left in those who are not ashamed to tell a deliberate lie.” 

Then\marginnote{4.1} the Buddha, tossing away what little water was left in the pot, said to \textsanskrit{Rāhula}, “Do you see this little bit of water that was tossed away?” 

“Yes,\marginnote{4.3} sir.” 

“That’s\marginnote{4.4} how the ascetic’s nature is tossed away in those who are not ashamed to tell a deliberate lie.” 

Then\marginnote{5.1} the Buddha, turning the pot upside down, said to \textsanskrit{Rāhula}, “Do you see how this pot is turned upside down?” 

“Yes,\marginnote{5.3} sir.” 

“That’s\marginnote{5.4} how the ascetic’s nature is turned upside down in those who are not ashamed to tell a deliberate lie.” 

Then\marginnote{6.1} the Buddha, turning the pot right side up, said to \textsanskrit{Rāhula}, “Do you see how this pot is vacant and hollow?” 

“Yes,\marginnote{6.3} sir.” 

“That’s\marginnote{6.4} how vacant and hollow the ascetic’s nature is in those who are not ashamed to tell a deliberate lie. 

Suppose\marginnote{7.1} there was a royal bull elephant with tusks like chariot-poles, able to draw a heavy load, pedigree and battle-hardened. In battle it uses its fore-feet and hind-feet, its fore-quarters and hind-quarters, its head, ears, tusks, and tail, but it still protects its trunk.\footnote{The Buddha shows his mastery of pedagogy. He begins with movement and gestures to illustrate what for a boy could easily have been a dull and abstract teaching. Then he continues to engage the boy’s interest by comparing the mendicant life to war-elephants. } So its rider thinks: ‘This royal bull elephant still protects its trunk. It has not yet given its life.’ But when that royal bull elephant … in battle uses its fore-feet and hind-feet, its fore-quarters and hind-quarters, its head, ears, tusks, and tail, and its trunk, its rider thinks: ‘This royal bull elephant … in battle uses its fore-feet and hind-feet, its fore-quarters and hind-quarters, its head, ears, tusks, and tail, and its trunk. It has given its life. Now there is nothing that royal bull elephant would not do.’ 

In\marginnote{7.9} the same way, when someone is not ashamed to tell a deliberate lie, there is no bad deed they would not do, I say. So you should train like this: ‘I will not tell a lie, even for a joke.’\footnote{This does not mean that one cannot tell jokes, or even that the jokes cannot speak of things that are not true. To be guilty of lying, one must try to deliberately create a false belief in the other person. If I say, “A monk, a priest, and a rabbi brought a donkey into a bar”, it is obviously not something I expect people to believe. This point is illustrated in the background story for this sutta in the \textsanskrit{Mūlasarvāstivāda} \textsanskrit{Vinayavibhaṅga} and \textsanskrit{Mahāprajñāpāramitopadeśaśāstra}, which say that \textsanskrit{Rāhula} had been mischievously telling visitors the Buddha was at the Vulture’s Peak when in reality he was in the Bamboo Grove. In this case, he deliberately created a false belief in order to send people in the wrong direction. } 

What\marginnote{8.1} do you think, \textsanskrit{Rāhula}? What is the purpose of a mirror?”\footnote{Mirrors of polished bronze were the selfies of the day, a vanity of youth (\href{https://suttacentral.net/mn77/en/sujato\#33.18}{MN 77:33.18}). They were forbidden for monastics except when ill (\href{https://suttacentral.net/pli-tv-kd15/en/sujato\#2.4.1}{Kd 15:2.4.1}). } 

“It’s\marginnote{8.3} for checking your reflection, sir.” 

“In\marginnote{8.4} the same way, deeds of body, speech, and mind should be done only after repeated checking.\footnote{The Buddha illustrates the abstract concept of introspection with what is perhaps the most fundamental metaphor: the sight of one’s own face. See also \textsanskrit{Chāndogya} \textsanskrit{Upaniṣad} 8.7.4 and \textsanskrit{Kaṭha} \textsanskrit{Upaniṣad} 2.3.5. } 

When\marginnote{9.1} you want to act with the body, you should check on that same deed: ‘Does this act with the body that I want to do lead to hurting myself, hurting others, or hurting both? Is it unskillful, with suffering as its outcome and result?’ If, while checking in this way, you know: ‘This act with the body that I want to do leads to hurting myself, hurting others, or hurting both. It’s unskillful, with suffering as its outcome and result.’ To the best of your ability, \textsanskrit{Rāhula}, you should not do such a deed.\footnote{\textit{\textsanskrit{Sasakkaṁ}} is from \textit{sa} (“own”) + \textit{sakka} (“ability”). } But if, while checking in this way, you know: ‘This act with the body that I want to do doesn’t lead to hurting myself, hurting others, or hurting both. It’s skillful, with happiness as its outcome and result.’ Then, \textsanskrit{Rāhula}, you should do such a deed. 

While\marginnote{10.1} you are acting with the body, you should check on that same act: ‘Does this act with the body that I am doing lead to hurting myself, hurting others, or hurting both? Is it unskillful, with suffering as its outcome and result?’ If, while checking in this way, you know: ‘This act with the body that I am doing leads to hurting myself, hurting others, or hurting both. It’s unskillful, with suffering as its outcome and result.’ Then, \textsanskrit{Rāhula}, you should desist from such a deed. But if, while checking in this way, you know: ‘This act with the body that I am doing doesn’t lead to hurting myself, hurting others, or hurting both. It’s skillful, with happiness as its outcome and result.’ Then, \textsanskrit{Rāhula}, you should continue doing such a deed. 

After\marginnote{11.1} you have acted with the body, you should check on that same act: ‘Does this act with the body that I have done lead to hurting myself, hurting others, or hurting both? Is it unskillful, with suffering as its outcome and result?’ If, while checking in this way, you know: ‘This act with the body that I have done leads to hurting myself, hurting others, or hurting both. It’s unskillful, with suffering as its outcome and result.’ Then, \textsanskrit{Rāhula}, you should confess, reveal, and clarify such a deed to the Teacher or a sensible spiritual companion. And having revealed it you should restrain yourself in future. But if, while checking in this way, you know: ‘This act with the body that I have done doesn’t lead to hurting myself, hurting others, or hurting both. It’s skillful, with happiness as its outcome and result.’ Then, \textsanskrit{Rāhula}, you should live in rapture and joy because of this, training day and night in skillful qualities. 

When\marginnote{12.1} you want to act with speech, you should check on that same deed:\footnote{The threefold awareness of actions before, during, and after is emphasized in the Vinaya analysis of the rule against deliberate lying (\href{https://suttacentral.net/pli-tv-bu-vb-pc1/en/sujato\#2.2.2}{Bu Pc 1:2.2.2}). } ‘Does this act of speech that I want to do lead to hurting myself, hurting others, or hurting both? …’ … 

If,\marginnote{14.1} while checking in this way, you know: ‘This act of speech that I have done leads to hurting myself, hurting others, or hurting both. It’s unskillful, with suffering as its outcome and result.’ Then, \textsanskrit{Rāhula}, you should confess, reveal, and clarify such a deed to the Teacher or a sensible spiritual companion. And having revealed it you should restrain yourself in future. But if, while checking in this way, you know: ‘This act of speech that I have done doesn’t lead to hurting myself, hurting others, or hurting both. It’s skillful, with happiness as its outcome and result.’ Then, \textsanskrit{Rāhula}, you should live in rapture and joy because of this, training day and night in skillful qualities. 

When\marginnote{15.1} you want to act with the mind, you should check on that same deed: ‘Does this act of mind that I want to do lead to hurting myself, hurting others, or hurting both? …’ … 

If,\marginnote{17.1} while checking in this way, you know: ‘This act of mind that I have done leads to hurting myself, hurting others, or hurting both. It’s unskillful, with suffering as its outcome and result.’ Then, \textsanskrit{Rāhula}, you should be horrified, repelled, and disgusted by that deed.\footnote{Acts of mind are not covered under the Vinaya and so need not be confessed, unlike those of body (\href{https://suttacentral.net/mn61/en/sujato\#11.6}{MN 61:11.6}) and speech (\href{https://suttacentral.net/mn61/en/sujato\#14.6}{MN 61:14.6}). They are matters of personal integrity rather than social reinforcement. } And being repelled, you should restrain yourself in future. But if, while checking in this way, you know: ‘This act with the mind that I have done doesn’t lead to hurting myself, hurting others, or hurting both. It’s skillful, with happiness as its outcome and result.’ Then, \textsanskrit{Rāhula}, you should live in rapture and joy because of this, training day and night in skillful qualities. 

All\marginnote{18.1} the ascetics and brahmins of the past, future, and present who purify their physical, verbal, and mental actions do so after repeatedly checking. So \textsanskrit{Rāhula}, you should train yourself like this: ‘I will purify my physical, verbal, and mental actions after repeatedly checking.’” 

That\marginnote{18.6} is what the Buddha said. Satisfied, Venerable \textsanskrit{Rāhula} approved what the Buddha said. 

%
\section*{{\suttatitleacronym MN 62}{\suttatitletranslation The Longer Advice to Rāhula }{\suttatitleroot Mahārāhulovādasutta}}
\addcontentsline{toc}{section}{\tocacronym{MN 62} \toctranslation{The Longer Advice to Rāhula } \tocroot{Mahārāhulovādasutta}}
\markboth{The Longer Advice to Rāhula }{Mahārāhulovādasutta}
\extramarks{MN 62}{MN 62}

\scevam{So\marginnote{1.1} I have heard.\footnote{Compared to \href{https://suttacentral.net/mn61/en/sujato}{MN 61}, now the Buddha is assigning advanced practices to \textsanskrit{Rāhula}, recommending a diverse range of different meditations. | The unusual structure raises questions as to its historicity, which I will discuss in my final note, having noted relevant details along the way. } }At one time the Buddha was staying near \textsanskrit{Sāvatthī} in Jeta’s Grove, \textsanskrit{Anāthapiṇḍika}’s monastery. 

Then\marginnote{2.1} the Buddha robed up in the morning and, taking his bowl and robe, entered \textsanskrit{Sāvatthī} for alms. And Venerable \textsanskrit{Rāhula} also robed up and followed behind the Buddha.\footnote{According to the commentary, \textsanskrit{Rāhula} was now eighteen years old, situating this sutta several years after \href{https://suttacentral.net/mn61/en/sujato}{MN 61}. } 

Then\marginnote{3.1} the Buddha looked back at \textsanskrit{Rāhula} and said, “\textsanskrit{Rāhula}, you should truly see any kind of form at all—past, future, or present; internal or external; solid or subtle; inferior or superior; far or near: \emph{all} form—with right understanding: ‘This is not mine, I am not this, this is not my self.’” 

“Only\marginnote{3.3} form, Blessed One? Only form, Holy One?”\footnote{The commentary says that the Buddha admonished him because he was admiring his father’s beauty, thinking that he looked similar. See \href{https://suttacentral.net/mn61/en/sujato\#8.2}{MN 61:8.2}, which similarly implies that vanity was a weakness of \textsanskrit{Rāhula}. } 

“Form,\marginnote{3.4} \textsanskrit{Rāhula}, as well as feeling and perception and choices and consciousness.” 

Then\marginnote{4.1} \textsanskrit{Rāhula} thought, “Who would go to the village for alms today after being advised directly by the Buddha?” Turning back, he sat down cross-legged at the root of a certain tree, setting his body straight, and establishing mindfulness in his presence.\footnote{He decides to meditate rather than eat for that day. } 

Venerable\marginnote{5.1} \textsanskrit{Sāriputta} saw him sitting there,\footnote{\textsanskrit{Rāhula} would have still been a novice (\textit{\textsanskrit{sāmaṇera}}) as he was not yet twenty. It was \textsanskrit{Sāriputta} who ordained him (\href{https://suttacentral.net/pli-tv-kd1/en/sujato\#54.2.7}{Kd 1:54.2.7}). } and addressed him, “\textsanskrit{Rāhula}, develop mindfulness of breathing. When mindfulness of breathing is developed and cultivated it’s very fruitful and beneficial.” 

Then\marginnote{6.1} in the late afternoon, \textsanskrit{Rāhula} came out of retreat, went to the Buddha, bowed, sat down to one side, and said to him: 

“Sir,\marginnote{7.1} how is mindfulness of breathing developed and cultivated to be very fruitful and beneficial?”\footnote{The Pali text has a striking narrative form. The Buddha encourages \textsanskrit{Rāhula} to contemplate form; then \textsanskrit{Sāriputta} urges him in breath meditation; but when \textsanskrit{Rāhula} asks the Buddha about that, the Buddha ignores him (until much later in the sutta) and instead expands on his instructions on form. The Chinese parallel (EA 17.1 at T ii 581c–582c) paints quite a different picture. \textsanskrit{Sāriputta} does not appear at all, and it is the Buddha who instructs \textsanskrit{Rāhula} when he is meditating, teaching him not just mindfulness of breathing but also the meditations on the ugliness of the body and the divine abidings. Then, when \textsanskrit{Rāhula} approaches him later to ask further about mindfulness of breathing, the Buddha teaches him this right away, omitting entirely the long section on the elements. Both suttas agree on teaching \textsanskrit{Rāhula} a range of meditations, but the Pali implies that he needed to mature his mind with a range of other meditations before attempting mindfulness of breathing, a situation of which \textsanskrit{Sāriputta} was not aware. } 

“\textsanskrit{Rāhula},\marginnote{8.1} the interior earth element is anything internal, pertaining to an individual, that’s hard, solid, and appropriated. This includes:\footnote{This largely follows \href{https://suttacentral.net/mn140/en/sujato\#14.4}{MN 140:14.4} and \href{https://suttacentral.net/mn28/en/sujato\#4.1}{MN 28:4.1}, the latter of which was taught by \textsanskrit{Sāriputta}. There the expanded teaching is clearly situated within the contemplation of form, whereas here it is somewhat abrupt. } head hair, body hair, nails, teeth, skin, flesh, sinews, bones, bone marrow, kidneys, heart, liver, diaphragm, spleen, lungs, intestines, mesentery, undigested food, feces; or anything else internal, pertaining to an individual, that’s hard, solid, and appropriated. This is called the interior earth element. The interior earth element and the exterior earth element are just the earth element. This should be truly seen with right understanding like this: ‘This is not mine, I am not this, this is not my self.’ When you truly see with right understanding, you grow disillusioned with the earth element, detaching the mind from the earth element. 

And\marginnote{9.1} what is the water element? The water element may be interior or exterior. And what is the interior water element? Anything internal, pertaining to an individual, that’s water, watery, and appropriated. This includes: bile, phlegm, pus, blood, sweat, fat, tears, grease, saliva, snot, synovial fluid, urine; or anything else internal, pertaining to an individual, that’s water, watery, and appropriated. This is called the interior water element. The interior water element and the exterior water element are just the water element. This should be truly seen with right understanding like this: ‘This is not mine, I am not this, this is not my self.’ When you truly see with right understanding, you grow disillusioned with the water element, detaching the mind from the water element. 

And\marginnote{10.1} what is the fire element? The fire element may be interior or exterior. And what is the interior fire element? Anything internal, pertaining to an individual, that’s fire, fiery, and appropriated. This includes: that which warms, that which ages, that which heats you up when feverish, that which properly digests food and drink; or anything else internal, pertaining to an individual, that’s fire, fiery, and appropriated. This is called the interior fire element. The interior fire element and the exterior fire element are just the fire element. This should be truly seen with right understanding like this: ‘This is not mine, I am not this, this is not my self.’ When you truly see with right understanding, you grow disillusioned with the fire element, detaching the mind from the fire element. 

And\marginnote{11.1} what is the air element? The air element may be interior or exterior. And what is the interior air element? Anything internal, pertaining to an individual, that’s air, airy, and appropriated. This includes: winds that go up or down, winds in the belly or the bowels, winds that flow through the limbs, in-breaths and out-breaths; or anything else internal, pertaining to an individual, that’s air, airy, and appropriated. This is called the interior air element. The interior air element and the exterior air element are just the air element. This should be truly seen with right understanding like this: ‘This is not mine, I am not this, this is not my self.’ When you truly see with right understanding, you grow disillusioned with the air element, detaching the mind from the air element. 

And\marginnote{12.1} what is the space element? The space element may be interior or exterior. And what is the interior space element? Anything internal, pertaining to an individual, that’s space, spacious, and appropriated. This includes: the ear canals, nostrils, and mouth; and the space for swallowing what is eaten and drunk, the space where it stays, and the space for excreting it from the nether regions; or anything else internal, pertaining to an individual, that’s space, spacious, and appropriated.\footnote{The MS edition has an extended passage here that appears to be imported from the Abhidhamma (eg, \href{https://suttacentral.net/vb3/en/sujato\#10.5}{Vb 3:10.5}). It would translate as: “space, spacious, void, voidness, opening, openness, untouched, and appropriated by flesh and blood.” } This is called the interior space element. The interior space element and the exterior space element are just the space element. This should be truly seen with right understanding like this: ‘This is not mine, I am not this, this is not my self.’ When you truly see with right understanding, you grow disillusioned with the space element, detaching the mind from the space element. 

\textsanskrit{Rāhula},\marginnote{13.1} meditate like the earth. For when you meditate like the earth, pleasant and unpleasant contacts will not occupy your mind. Suppose they were to toss both clean and unclean things on the earth, like feces, urine, spit, pus, and blood. The earth isn’t horrified, repelled, and disgusted because of this.\footnote{This simile and those that follow on water, fire, and wind are found at \href{https://suttacentral.net/an9.11/en/sujato\#4.1}{AN 9.11:4.1}, where they are also spoken by \textsanskrit{Sāriputta}. } In the same way, meditate like the earth. For when you meditate like the earth, pleasant and unpleasant contacts will not occupy your mind. 

Meditate\marginnote{14.1} like water. For when you meditate like water, pleasant and unpleasant contacts will not occupy your mind. Suppose they were to wash both clean and unclean things in the water, like feces, urine, spit, pus, and blood. The water isn’t horrified, repelled, and disgusted because of this. In the same way, meditate like water. For when you meditate like water, pleasant and unpleasant contacts will not occupy your mind. 

Meditate\marginnote{15.1} like fire. For when you meditate like fire, pleasant and unpleasant contacts will not occupy your mind. Suppose a fire were to burn both clean and unclean things, like feces, urine, spit, pus, and blood. The fire isn’t horrified, repelled, and disgusted because of this. In the same way, meditate like fire. For when you meditate like fire, pleasant and unpleasant contacts will not occupy your mind. 

Meditate\marginnote{16.1} like wind. For when you meditate like wind, pleasant and unpleasant contacts will not occupy your mind. Suppose the wind were to blow on both clean and unclean things, like feces, urine, spit, pus, and blood. The wind isn’t horrified, repelled, and disgusted because of this. In the same way, meditate like the wind. For when you meditate like wind, pleasant and unpleasant contacts will not occupy your mind. 

Meditate\marginnote{17.1} like space. For when you meditate like space, pleasant and unpleasant contacts will not occupy your mind. Just as space is not established anywhere,\footnote{This phrase is unique in the early texts. It is quoted by name in \href{https://suttacentral.net/mil7.4.6/en/sujato\#5.4}{Mil 7.4.6:5.4}. } in the same way, meditate like space. For when you meditate like space, pleasant and unpleasant contacts will not occupy your mind. 

Meditate\marginnote{18.1} on love. For when you meditate on love any ill will will be given up. 

Meditate\marginnote{19.1} on compassion. For when you meditate on compassion any cruelty will be given up. 

Meditate\marginnote{20.1} on rejoicing. For when you meditate on rejoicing any discontent will be given up. 

Meditate\marginnote{21.1} on equanimity. For when you meditate on equanimity any repulsion will be given up. 

Meditate\marginnote{22.1} on ugliness. For when you meditate on ugliness any lust will be given up. 

Meditate\marginnote{23.1} on impermanence. For when you meditate on impermanence any conceit ‘I am’ will be given up. 

Develop\marginnote{24.1} mindfulness of breathing. When mindfulness of breathing is developed and cultivated it’s very fruitful and beneficial. And how is mindfulness of breathing developed and cultivated to be very fruitful and beneficial? 

It’s\marginnote{25.1} when a mendicant—gone to a wilderness, or to the root of a tree, or to an empty hut—sits down cross-legged, sets their body straight, and establishes mindfulness in their presence. Just mindful, they breath in. Mindful, they breath out.\footnote{For notes on this practice, see \href{https://suttacentral.net/mn118/en/sujato\#17.2}{MN 118:17.2}. } 

Breathing\marginnote{26.1} in heavily they know: ‘I’m breathing in heavily.’ Breathing out heavily they know: ‘I’m breathing out heavily.’ When breathing in lightly they know: ‘I’m breathing in lightly.’ Breathing out lightly they know: ‘I’m breathing out lightly.’ They practice like this: ‘I’ll breathe in experiencing the whole body.’ They practice like this: ‘I’ll breathe out experiencing the whole body.’They practice like this: ‘I’ll breathe in stilling the physical process.’ They practice like this: ‘I’ll breathe out stilling the physical process.’ 

They\marginnote{27.1} practice like this: ‘I’ll breathe in experiencing rapture.’ They practice like this: ‘I’ll breathe out experiencing rapture.’ They practice like this: ‘I’ll breathe in experiencing bliss.’ They practice like this: ‘I’ll breathe out experiencing bliss.’ They practice like this: ‘I’ll breathe in experiencing mental processes.’ They practice like this: ‘I’ll breathe out experiencing mental processes.’They practice like this: ‘I’ll breathe in stilling mental processes.’ They practice like this: ‘I’ll breathe out stilling mental processes.’

They\marginnote{28.1} practice like this: ‘I’ll breathe in experiencing the mind.’ They practice like this: ‘I’ll breathe out experiencing the mind.’ They practice like this: ‘I’ll breathe in gladdening the mind.’ They practice like this: ‘I’ll breathe out gladdening the mind.’ They practice like this: ‘I’ll breathe in immersing the mind in \textsanskrit{samādhi}.’ They practice like this: ‘I’ll breathe out immersing the mind in \textsanskrit{samādhi}.’ They practice like this: ‘I’ll breathe in freeing the mind.’ They practice like this: ‘I’ll breathe out freeing the mind.’ 

They\marginnote{29.1} practice like this: ‘I’ll breathe in observing impermanence.’ They practice like this: ‘I’ll breathe out observing impermanence.’ They practice like this: ‘I’ll breathe in observing fading away.’ They practice like this: ‘I’ll breathe out observing fading away.’ They practice like this: ‘I’ll breathe in observing cessation.’ They practice like this: ‘I’ll breathe out observing cessation.’ They practice like this: ‘I’ll breathe in observing letting go.’ They practice like this: ‘I’ll breathe out observing letting go.’ 

Mindfulness\marginnote{30.1} of breathing, when developed and cultivated in this way, is very fruitful and beneficial. When mindfulness of breathing is developed and cultivated in this way, even as the final breaths cease they are known, not unknown.” 

That\marginnote{30.3} is what the Buddha said. Satisfied, Venerable \textsanskrit{Rāhula} approved what the Buddha said.\footnote{The Chinese parallel presents a fuller picture of \textsanskrit{Rāhula}’s progress at this point, saying he went away, developed \textit{\textsanskrit{jhānas}}, and became an arahant, before returning to announce this to the Buddha, who praised him as the foremost of those desiring training. In the Pali, \textsanskrit{Rāhula}’s awakening occurs differently in \href{https://suttacentral.net/mn147/en/sujato}{MN 147}, while he is praised for desiring training at \href{https://suttacentral.net/an1.209/en/sujato\#1.1}{AN 1.209:1.1}. | It is not easy to assess the historical authenticity of this sutta. The Chinese version presents a more coherent narrative that may indicate less corruption. Yet it expands the ending, making the sutta a comprehensive account of \textsanskrit{Rāhula}’s progress. The collection from which it stems, the \textsanskrit{Ekottarikāgama}, is the latest and least orthodox of extant Āgama collections, and it could be that it has streamlined a knotty narrative. The Pali, with its abrupt narrative transition and its implication that \textsanskrit{Sāriputta} got his meditation instructions wrong, has more of the randomness of natural dialogue. Yet the fact that the portions extra to the Chinese include not only the person of \textsanskrit{Sāriputta}, but also teachings associated with \textsanskrit{Sāriputta}, is hard to explain as a loss of text. Perhaps both texts have been expanded from a simpler original where the Buddha teaches meditation to his son. } 

%
\section*{{\suttatitleacronym MN 63}{\suttatitletranslation The Shorter Discourse With Māluṅkyaputta }{\suttatitleroot Cūḷamālukyasutta}}
\addcontentsline{toc}{section}{\tocacronym{MN 63} \toctranslation{The Shorter Discourse With Māluṅkyaputta } \tocroot{Cūḷamālukyasutta}}
\markboth{The Shorter Discourse With Māluṅkyaputta }{Cūḷamālukyasutta}
\extramarks{MN 63}{MN 63}

\scevam{So\marginnote{1.1} I have heard. }At one time the Buddha was staying near \textsanskrit{Sāvatthī} in Jeta’s Grove, \textsanskrit{Anāthapiṇḍika}’s monastery. 

Then\marginnote{2.1} as Venerable \textsanskrit{Māluṅkyaputta} was in private retreat this thought came to his mind:\footnote{\textsanskrit{Māluṅkyaputta} was, according to Pali commentaries, the son of the lady \textsanskrit{Māluṅkyā} of Kosala. Alternatively, Sanskrit dictionaries give \textit{\textsanskrit{māluka}} as the name of a people or a mixed caste, although I have not been able to trace any further details. His conversion is not recorded as he appears always as a monk in Pali. This discourse and \href{https://suttacentral.net/mn64/en/sujato}{MN 64} present him as somewhat of a fool, but it seems that he persisted, and late in life sought teachings in meditation and ultimately realized arahantship, of which we have two different accounts (\href{https://suttacentral.net/an4.257/en/sujato}{AN 4.257}, \href{https://suttacentral.net/sn35.95/en/sujato}{SN 35.95}). The latter contains an extensive set of verses attributed to him, which are repeated at \href{https://suttacentral.net/thag16.5/en/sujato}{Thag 16.5}. A second set of his verses are at \href{https://suttacentral.net/thag6.5/en/sujato}{Thag 6.5}. } 

“There\marginnote{2.2} are several convictions that the Buddha has left undeclared; he has set them aside and refused to comment on them. For example: the cosmos is eternal, or not eternal, or finite, or infinite; the soul and the body are the same thing, or they are different things; after death, a realized one still exists, or no longer exists, or both still exists and no longer exists, or neither still exists nor no longer exists. The Buddha does not explain these points to me. I don’t endorse that, and do not accept it. I’ll go to him and ask him about this. If he gives me a straight answer on any of these points, I will lead the spiritual life under him. If he does not explain these points to me, I shall reject the training and return to a lesser life.” 

Then\marginnote{3.1} in the late afternoon, \textsanskrit{Māluṅkyaputta} came out of retreat and went to the Buddha. He bowed, sat down to one side, and told the Buddha of his thoughts. He then continued: 

“If\marginnote{3.2} the Buddha knows that the cosmos is eternal, please tell me. If you know that the cosmos is not eternal, tell me. If you don’t know whether the cosmos is eternal or not, then it is straightforward to simply say: ‘I neither know nor see.’ If you know that the cosmos is finite, or infinite; that the soul and the body are the same thing, or they are different things; that after death, a realized one still exists, or no longer exists, or both still exists and no longer exists, or neither still exists nor no longer exists, please tell me. If you don’t know any of these things, then it is straightforward to simply say: ‘I neither know nor see.’” 

“What,\marginnote{4.1} \textsanskrit{Māluṅkyaputta}, did I ever say to you: ‘Come, \textsanskrit{Māluṅkyaputta}, lead the spiritual life under me, and I will declare these things to you’?” 

“No,\marginnote{4.4} sir.” 

“Or\marginnote{4.5} did you ever say to me: ‘Sir, I will lead the spiritual life under the Buddha, and the Buddha will declare these things to me’?” 

“No,\marginnote{4.8} sir.” 

“So\marginnote{4.9} it seems that I did not say to you: ‘Come, \textsanskrit{Māluṅkyaputta}, lead the spiritual life under me, and I will declare these things to you.’ And you never said to me: ‘Sir, I will lead the spiritual life under the Buddha, and the Buddha will declare these things to me.’ In that case, you futile man, who are you and what are you rejecting? 

Suppose\marginnote{5.1} someone were to say this: ‘I will not lead the spiritual life under the Buddha until the Buddha declares to me that the cosmos is eternal, or that the cosmos is not eternal … or that after death a realized one neither still exists nor no longer exists.’ That would still remain undeclared by the Realized One, and meanwhile that person would die. 

Suppose\marginnote{5.6} a man was struck by an arrow thickly smeared with poison. His friends and colleagues, relatives and kin would get a surgeon to treat him. But the man would say: ‘I won’t extract this arrow as long as I don’t know whether the man who wounded me was an aristocrat, a brahmin, a peasant, or a menial.’\footnote{“Extract this arrow”(\textit{\textsanskrit{imaṁ} \textsanskrit{sallaṁ} \textsanskrit{āharissāmi}}): in his Ayurvedic text \textsanskrit{Aṣṭāṅgahṛdayasaṁhitā}, \textsanskrit{Vāgbhaṭa} devotes the entirety of chapter 1.28 to this procedure (\textit{\textsanskrit{śalyāharaṇavidhi}}). } He’d say: ‘I won’t extract this arrow as long as I don’t know the following things about the man who wounded me: his name and clan; whether he’s tall, short, or medium; whether his skin is black, brown, or tawny; and what village, town, or city he comes from. I won’t extract this arrow as long as I don’t know whether the bow that wounded me was straight or recurved;\footnote{It would appear that \textit{dhanu} is the general word for “bow”, of which \textit{\textsanskrit{cāpa}} and \textit{\textsanskrit{kodaṇḍa}} are specialized varieties. These bows appear with others in a list of weapons at \textsanskrit{Arthaśāstra} 2.18.8. While it is not clear what these were exactly, \textit{\textsanskrit{kodaṇḍa}} is sometimes used for “eyebrows”, suggesting that the bow of that name was of the recurve type favored by horsemen, and the \textit{\textsanskrit{cāpa}} by contrast may have been straight. } whether the bow-string is made of swallow-wort fibre, sunn hemp fibre, sinew, sanseveria fibre, or spurge fibre;\footnote{The identifications for these are from \textsanskrit{Ñāṇatusita}’s notes to Bodhi’s translation. } whether the shaft is made from a bush or a plantation tree; whether the shaft was fitted with feathers from a vulture, a heron, a hawk, a peacock, or a stork; whether the shaft was bound with sinews of a cow, a buffalo, a black lion, or an ape;\footnote{The animals \textit{bherava} (“terrifier”, variant \textit{roruva} “roarer”) and \textit{\textsanskrit{semhāra}} are otherwise unknown. The commentary explains them as “black lion” and “monkey”. } and whether the arrowhead was spiked, razor-tipped, barbed, made of iron or a calf’s tooth, or lancet-shaped.’\footnote{Again from \textsanskrit{Ñāṇatusita}. Note that calves lose their teeth, and the roots are evidently quite sharp. } That man would still not have learned these things, and meanwhile they’d die. 

In\marginnote{5.31} the same way, suppose someone was to say: ‘I will not lead the spiritual life under the Buddha until the Buddha declares to me that the cosmos is eternal, or that the cosmos is not eternal … or that after death a realized one neither still exists nor no longer exists.’ That would still remain undeclared by the Realized One, and meanwhile that person would die. 

It’s\marginnote{6.1} not true that if there were the view ‘the cosmos is eternal’ there would be the living of the spiritual life.\footnote{This is a double conditional, the sense being that if the pre-condition is met (locative absolute), the subsequent act would follow (conditional). The form echoes \textsanskrit{Māluṅkyaputta}’s conditions, which is why the syntax here is more convoluted than in similar cases elsewhere. } It’s not true that if there were the view ‘the cosmos is not eternal’ there would be the living of the spiritual life. When there is the view that the cosmos is eternal or that the cosmos is not eternal, there is rebirth, there is old age, there is death, and there is sorrow, lamentation, pain, sadness, and distress. And it is the defeat of these things in this very life that I advocate. It’s not true that if there were the view ‘the cosmos is finite’ … ‘the cosmos is infinite’ … ‘the soul and the body are the same thing’ … ‘the soul and the body are different things’ … ‘a realized one still exists after death’ … ‘A realized one no longer exists after death’ … ‘a realized one both still exists and no longer exists after death’ … ‘a realized one neither still exists nor no longer exists after death’ there would be the living of the spiritual life. When there are any of these views there is rebirth, there is old age, there is death, and there is sorrow, lamentation, pain, sadness, and distress. And it is the defeat of these things in this very life that I advocate. 

So,\marginnote{7.1} \textsanskrit{Māluṅkyaputta}, you should remember what I have not declared as undeclared, and what I have declared as declared. And what have I not declared? I have not declared the following: ‘the cosmos is eternal,’ ‘the cosmos is not eternal,’ ‘the cosmos is finite,’ ‘the world is infinite,’ ‘the soul and the body are the same thing,’ ‘the soul and the body are different things,’ ‘a realized one still exists after death,’ ‘A realized one no longer exists after death,’ ‘a realized one both still exists and no longer exists after death,’ ‘a realized one neither still exists nor no longer exists after death.’ 

And\marginnote{8.1} why haven’t I declared these things? Because they aren’t beneficial or relevant to the fundamentals of the spiritual life. They don’t lead to disillusionment, dispassion, cessation, peace, insight, awakening, and extinguishment. That’s why I haven’t declared them. 

And\marginnote{9.1} what have I declared? I have declared the following: ‘this is suffering,’ ‘this is the origin of suffering,’ ‘this is the cessation of suffering,’ ‘this is the practice that leads to the cessation of suffering.’ 

And\marginnote{10.1} why have I declared these things? Because they are beneficial and relevant to the fundamentals of the spiritual life. They lead to disillusionment, dispassion, cessation, peace, insight, awakening, and extinguishment. That’s why I have declared them. So, \textsanskrit{Māluṅkyaputta}, you should remember what I have not declared as undeclared, and what I have declared as declared.” 

That\marginnote{10.6} is what the Buddha said. Satisfied, Venerable \textsanskrit{Māluṅkyaputta} approved what the Buddha said. 

%
\section*{{\suttatitleacronym MN 64}{\suttatitletranslation The Longer Discourse With Māluṅkya }{\suttatitleroot Mahāmālukyasutta}}
\addcontentsline{toc}{section}{\tocacronym{MN 64} \toctranslation{The Longer Discourse With Māluṅkya } \tocroot{Mahāmālukyasutta}}
\markboth{The Longer Discourse With Māluṅkya }{Mahāmālukyasutta}
\extramarks{MN 64}{MN 64}

\scevam{So\marginnote{1.1} I have heard. }At one time the Buddha was staying near \textsanskrit{Sāvatthī} in Jeta’s Grove, \textsanskrit{Anāthapiṇḍika}’s monastery. There the Buddha addressed the mendicants, “Mendicants!” 

“Venerable\marginnote{1.5} sir,” they replied. The Buddha said this: 

“Mendicants,\marginnote{2.1} do you remember the five lower fetters taught by me?” 

When\marginnote{2.2} he said this, Venerable \textsanskrit{Māluṅkyaputta} said to him, “Sir, I remember them.” 

“But\marginnote{2.4} how do you remember them?” 

“I\marginnote{2.5} remember the lower fetters taught by the Buddha as follows: substantialist view,\footnote{As at \href{https://suttacentral.net/sn45.179/en/sujato}{SN 45.179}, \href{https://suttacentral.net/an10.13/en/sujato}{AN 10.13}, etc. They are “lower fetters” because they bind sentient beings to rebirth in lower planes. } doubt, misapprehension of precepts and observances, sensual desire, and ill will. That’s how I remember the five lower fetters taught by the Buddha.” 

“Who\marginnote{3.1} on earth do you remember being taught the five lower fetters in that way?\footnote{Given that \textsanskrit{Māluṅkyaputta} got the answer right, it is not clear why the Buddha adopted the reproving tone that he would normally use in response to a dangerously mistaken view (\href{https://suttacentral.net/mn22/en/sujato\#6.1}{MN 22:6.1}). The commentary explains that he had the view that fetters were only present when active, hence the Buddha’s simile to follow. Regardless, as compared to \href{https://suttacentral.net/mn63/en/sujato}{MN 63}, \textsanskrit{Māluṅkyaputta} is making progress, as he gives a serious answer to an important question and appears to no longer be on the verge of disrobing. } Wouldn’t the wanderers of other religions fault you using the simile of the infant? For a little baby doesn’t even have a concept of ‘substantial reality’, so how could substantialist view possibly arise in them? Yet the underlying tendency to substantialist view still lies within them. A little baby doesn’t even have a concept of ‘teachings’, so how could doubt about the teachings possibly arise in them? Yet the underlying tendency to doubt still lies within them. A little baby doesn’t even have a concept of ‘precepts’, so how could misapprehension of precepts and observances possibly arise in them? Yet the underlying tendency to misapprehension of precepts and observances still lies within them. A little baby doesn’t even have a concept of ‘sensual pleasures’, so how could desire for sensual pleasures possibly arise in them? Yet the underlying tendency to sensual desire still lies within them. A little baby doesn’t even have a concept of ‘sentient beings’, so how could ill will for sentient beings possibly arise in them? Yet the underlying tendency to ill will still lies within them. Wouldn’t the wanderers of other religions fault you using the simile of the infant?”\footnote{For other accounts of infant development in relation to defilements, see \href{https://suttacentral.net/an10.99/en/sujato\#6.2}{AN 10.99:6.2} and \href{https://suttacentral.net/mn38/en/sujato\#26.1}{MN 38:26.1}. } 

When\marginnote{4.1} he said this, Venerable Ānanda said to the Buddha, “Now is the time, Blessed One! Now is the time, Holy One! May the Buddha teach the five lower fetters. The mendicants will listen and remember it.” 

“Well\marginnote{4.4} then, Ānanda, listen and apply your mind well, I will speak.” 

“Yes,\marginnote{4.5} sir,” Ānanda replied. The Buddha said this: 

“Ānanda,\marginnote{5.1} take an unlearned ordinary person who has not seen the noble ones, and is neither skilled nor trained in the teaching of the noble ones. They’ve not seen true persons, and are neither skilled nor trained in the teaching of the true persons. Their heart is overcome and mired in substantialist view, and they don’t truly understand the escape from substantialist view that has arisen. That substantialist view is entrenched in them, not eliminated: it is a lower fetter. 

Their\marginnote{5.5} heart is overcome and mired in doubt, and they don’t truly understand the escape from doubt that has arisen. That doubt is entrenched in them, not eliminated: it is a lower fetter. 

Their\marginnote{5.8} heart is overcome and mired in misapprehension of precepts and observances, and they don’t truly understand the escape from misapprehension of precepts and observances that has arisen. That misapprehension of precepts and observances is entrenched in them, not eliminated: it is a lower fetter. 

Their\marginnote{5.11} heart is overcome and mired in sensual desire, and they don’t truly understand the escape from sensual desire that has arisen. That sensual desire is entrenched in them, not eliminated: it is a lower fetter. 

Their\marginnote{5.14} heart is overcome and mired in ill will, and they don’t truly understand the escape from ill will that has arisen. That ill will is entrenched in them, not eliminated: it is a lower fetter. 

But\marginnote{6.1} a learned noble disciple has seen the noble ones, and is skilled and trained in the teaching of the noble ones. They’ve seen true persons, and are skilled and trained in the teaching of the true persons. Their heart is not overcome and mired in substantialist view, and they truly understand the escape from substantialist view that has arisen. That substantialist view, along with any underlying tendency to it, is given up in them. 

Their\marginnote{6.4} heart is not overcome and mired in doubt, and they truly understand the escape from doubt that has arisen. That doubt, along with any underlying tendency to it, is given up in them. 

Their\marginnote{6.7} heart is not overcome and mired in misapprehension of precepts and observances, and they truly understand the escape from misapprehension of precepts and observances that has arisen. That misapprehension of precepts and observances, along with any underlying tendency to it, is given up in them. 

Their\marginnote{6.10} heart is not overcome and mired in sensual desire, and they truly understand the escape from sensual desire that has arisen. That sensual desire, along with any underlying tendency to it, is given up in them. 

Their\marginnote{6.13} heart is not overcome and mired in ill will, and they truly understand the escape from ill will that has arisen. That ill will, along with any underlying tendency to it, is given up in them. 

There\marginnote{6.16} is a path and a practice for giving up the five lower fetters. It’s not possible to know or see or give up the five lower fetters without relying on that path and that practice. Suppose there was a large tree standing with heartwood. It’s not possible to cut out the heartwood without having cut through the bark and the softwood. In the same way, there is a path and a practice for giving up the five lower fetters. It’s not possible to know or see or give up the five lower fetters without relying on that path and that practice. 

But\marginnote{7.1} it is possible to know and see and give up the five lower fetters by relying on that path and that practice. 

Suppose\marginnote{8.1} there was a large tree standing with heartwood. It is possible to cut out the heartwood after having cut through the bark and the softwood. In the same way, there is a path and a practice for giving up the five lower fetters. It is possible to know and see and give up the five lower fetters by relying on that path and that practice. Suppose the river Ganges was full to the brim so a crow could drink from it. Then along comes a feeble person, who thinks: ‘By swimming with my arms I’ll safely cross over to the far shore of the Ganges.’ But they’re not able to do so. In the same way, when the Dhamma is being taught for the cessation of substantial reality, someone whose mind does not leap forth, gain confidence, settle down, and become decided should be regarded as being like that feeble person. Suppose the river Ganges was full to the brim so a crow could drink from it. Then along comes a strong person, who thinks: ‘By swimming with my arms I’ll safely cross over to the far shore of the Ganges.’ And they are able to do so. 

In\marginnote{8.13} the same way, when the Dhamma is being taught for the cessation of substantial reality, someone whose mind leaps forth, gains confidence, settles down, and becomes decided should be regarded as being like that strong person. 

And\marginnote{9.1} what, Ānanda, is the path and the practice for giving up the five lower fetters? It’s when a mendicant—due to the seclusion from attachments, the giving up of unskillful qualities, and the complete settling of physical discomforts—quite secluded from sensual pleasures, secluded from unskillful qualities, enters and remains in the first absorption, which has the rapture and bliss born of seclusion, while placing the mind and keeping it connected.\footnote{The opening clause is a unique addition to the standard \textit{\textsanskrit{jhāna}} formula. | In later texts, “seclusion from attachments” (\textit{upadhiviveka}) signified arahantship (\href{https://suttacentral.net/mnd14/en/sujato\#10.1}{Mnd 14:10.1}), but here it must have a less exalted sense. Perhaps it refers to leaving behind the material things to which one is attached (\href{https://suttacentral.net/sn1.2/en/sujato}{SN 1.2}). | For “bodily discomfort”, see \href{https://suttacentral.net/mn127/en/sujato\#16.7}{MN 127:16.7}. } They contemplate the phenomena there—included in form, feeling, perception, choices, and consciousness—as impermanent, as suffering, as diseased, as a boil, as a dart, as misery, as an affliction, as alien, as falling apart, as empty, as not-self.\footnote{This shows the development of insight based directly on the absorption. } They turn their mind away from those things, and apply it to the element free of death: ‘This is peaceful; this is sublime—that is, the stilling of all activities, the letting go of all attachments, the ending of craving, fading away, cessation, extinguishment.’ Abiding in that they attain the ending of defilements. If they don’t attain the ending of defilements, with the ending of the five lower fetters they’re reborn spontaneously, because of their passion and love for that meditation. They are extinguished there, and are not liable to return from that world. This is the path and the practice for giving up the five lower fetters. 

Furthermore,\marginnote{10{-}12.1} as the placing of the mind and keeping it connected are stilled, a mendicant enters and remains in the second absorption … third absorption … fourth absorption. They contemplate the phenomena there—included in form, feeling, perception, choices, and consciousness—as impermanent … They turn their mind away from those things … If they don’t attain the ending of defilements, they’re reborn spontaneously … and are not liable to return from that world. This too is the path and the practice for giving up the five lower fetters. 

Furthermore,\marginnote{13.1} a mendicant, going totally beyond perceptions of form, with the ending of perceptions of impingement, not focusing on perceptions of diversity, aware that ‘space is infinite’, enters and remains in the dimension of infinite space. They contemplate the phenomena there—included in feeling, perception, choices, and consciousness—as impermanent … They turn their mind away from those things … If they don’t attain the ending of defilements, they’re reborn spontaneously … and are not liable to return from that world. This too is the path and the practice for giving up the five lower fetters. 

Furthermore,\marginnote{14.1} a mendicant, going totally beyond the dimension of infinite space, aware that ‘consciousness is infinite’, enters and remains in the dimension of infinite consciousness. They contemplate the phenomena there—included in feeling, perception, choices, and consciousness—as impermanent … They turn their mind away from those things … If they don’t attain the ending of defilements, they’re reborn spontaneously … and are not liable to return from that world. This too is the path and the practice for giving up the five lower fetters. 

Furthermore,\marginnote{15.1} a mendicant, going totally beyond the dimension of infinite consciousness, aware that ‘there is nothing at all’, enters and remains in the dimension of nothingness. They contemplate the phenomena there—included in feeling, perception, choices, and consciousness—as impermanent … They turn their mind away from those things … If they don’t attain the ending of defilements, they’re reborn spontaneously … and are not liable to return from that world. This too is the path and the practice for giving up the five lower fetters.”\footnote{The final formless attainment is omitted as at \href{https://suttacentral.net/mn52/en/sujato\#14.7}{MN 52:14.7}. } 

“Sir,\marginnote{16.1} if this is the path and the practice for giving up the five lower fetters, how come some mendicants here are released in heart while others are released by wisdom?”\footnote{“Released in heart“ and “released by wisdom” indicate an emphasis on either immersion or wisdom respectively. } 

“In\marginnote{16.2} that case, I say it is the diversity of their faculties.”\footnote{The Buddha makes the same point at \href{https://suttacentral.net/sn48.13/en/sujato}{SN 48.13} and \href{https://suttacentral.net/sn48/en/sujato\#16}{SN 48:16}. For more on the “diversity of faculties”, see \href{https://suttacentral.net/mn66/en/sujato\#14.6}{MN 66:14.6}. } 

That\marginnote{16.3} is what the Buddha said. Satisfied, Venerable Ānanda approved what the Buddha said. 

%
\section*{{\suttatitleacronym MN 65}{\suttatitletranslation With Bhaddāli }{\suttatitleroot Bhaddālisutta}}
\addcontentsline{toc}{section}{\tocacronym{MN 65} \toctranslation{With Bhaddāli } \tocroot{Bhaddālisutta}}
\markboth{With Bhaddāli }{Bhaddālisutta}
\extramarks{MN 65}{MN 65}

\scevam{So\marginnote{1.1} I have heard. }At one time the Buddha was staying near \textsanskrit{Sāvatthī} in Jeta’s Grove, \textsanskrit{Anāthapiṇḍika}’s monastery. There the Buddha addressed the mendicants, “Mendicants!” 

“Venerable\marginnote{1.5} sir,” they replied. The Buddha said this: 

“Mendicants,\marginnote{2.1} I eat my food in one sitting per day.\footnote{The practice of eating one meal a day is encouraged but not required by Vinaya. At \href{https://suttacentral.net/mn21/en/sujato\#7.4}{MN 21:7.4} the Buddha reminisced about the days when he merely needed to mention it and monks would follow suit. | In the related story at \href{https://suttacentral.net/mn70/en/sujato\#1.3}{MN 70:1.3}, the Buddha encouraged not eating at night, to which monks also objected, occasioning the laying down of a Vinaya rule (\href{https://suttacentral.net/pli-tv-bu-vb-pc37/en/sujato}{Bu Pc 37}), although at least some of the monks later came to regret their objections, recognizing that the Buddha had acted for their welfare (\href{https://suttacentral.net/mn66/en/sujato\#6.4}{MN 66:6.4}). } Doing so, I find that I’m healthy and well, nimble, strong, and living comfortably. You too should eat your food in one sitting per day. Doing so, you’ll find that you’re healthy and well, nimble, strong, and living comfortably.” 

When\marginnote{3.1} he said this, Venerable \textsanskrit{Bhaddāli} said to the Buddha,\footnote{\textsanskrit{Bhaddāli} does not appear elsewhere in early texts. } “Sir, I’m not going to try to eat my food in one sitting per day. For when eating once a day I might feel remorse and regret.”\footnote{The commentary says that he feared he would not be able to sustain his monkhood. Chinese parallels at this point indicate that he feared discomfort (EA 49.7 at T ii 800b29, T 1425 at T xxii 359b14). } 

“Well\marginnote{4.1} then, \textsanskrit{Bhaddāli}, eat one part of the meal in the place where you’re invited, and bring the rest back to eat. Eating this way, too, you will sustain yourself.”\footnote{\textit{\textsanskrit{Ekāsano}} here in \textsanskrit{Mahāsaṅgīti} edition appears superfluous, as it is missing in most editions. } 

“Sir,\marginnote{4.3} I’m not going to try to eat that way, either. For when eating that way I might also feel remorse and regret.” Then, as this rule was being laid down by the Buddha and the \textsanskrit{Saṅgha} was undertaking it, \textsanskrit{Bhaddāli} announced he would not try to keep it. Then for the whole of that three months \textsanskrit{Bhaddāli} would not see the Buddha face to face, as happens when someone doesn’t fulfill the training according to the Teacher’s instructions.\footnote{“Would not see the Buddha face to face” is a loose rendering of \textit{na bhagavato \textsanskrit{sammukhībhāvaṁ} \textsanskrit{adāsi}}. The idiom can be captured more literally at \href{https://suttacentral.net/dn3/en/sujato\#2.6.2}{DN 3:2.6.2}; when King Pasenadi meets \textsanskrit{Pokkharasāti}, he does not “grant him an audience face to face” (\textit{tassa … \textsanskrit{sammukhībhāvampi} na \textsanskrit{dadāti}}) but only behind a curtain. } 

At\marginnote{5.1} that time several mendicants were making a robe for the Buddha, thinking that when his robe was finished and the three months of the rains residence had passed the Buddha would set out wandering. 

Then\marginnote{6.1} \textsanskrit{Bhaddāli} went up to those mendicants, and exchanged greetings with them. When the greetings and polite conversation were over, he sat down to one side. The mendicants said to \textsanskrit{Bhaddāli}, “Reverend \textsanskrit{Bhaddāli}, this robe is being made for the Buddha. When it’s finished and the three months of the rains residence have passed the Buddha will set out wandering. Come on, \textsanskrit{Bhaddāli}, learn your lesson. Don’t make it hard for yourself later on.” 

“Yes,\marginnote{7.1} reverends,” \textsanskrit{Bhaddāli} replied. He went to the Buddha, bowed, sat down to one side, and said to him, “I have made a mistake, sir. It was foolish, stupid, and unskillful of me that, as this rule was being laid down by the Buddha and the \textsanskrit{Saṅgha} was undertaking it, I announced I would not try to keep it. Please, sir, accept my mistake for what it is, so I will restrain myself in future.” 

“Indeed,\marginnote{8.1} \textsanskrit{Bhaddāli}, you made a mistake. It was foolish, stupid, and unskillful of you that, as this rule was being laid down by the Buddha and the \textsanskrit{Saṅgha} was undertaking it, you announced you would not try to keep it.\footnote{When someone genuinely confesses, the Buddha makes a point to acknowledge what they had done, without dismissing it and thereby discounting the inner work they had achieved to get to this point. } 

And\marginnote{9.1} you didn’t realize this situation: ‘The Buddha is staying in \textsanskrit{Sāvatthī}, and he’ll know me as the mendicant named \textsanskrit{Bhaddāli} who doesn’t fulfill the training according to the Teacher’s instructions.’ 

And\marginnote{9.5} you didn’t realize this situation: ‘Several monks have commenced the rains retreat in \textsanskrit{Sāvatthī} … several nuns have commenced the rains retreat in \textsanskrit{Sāvatthī} … several laymen reside in \textsanskrit{Sāvatthī} … several laywomen reside in \textsanskrit{Sāvatthī}, and they’ll know me as the mendicant named \textsanskrit{Bhaddāli} who doesn’t fulfill the training according to the Teacher’s instructions. … 

Several\marginnote{9.21} ascetics and brahmins who follow various other religions have commenced the rains retreat in \textsanskrit{Sāvatthī}, and they’ll know me as the mendicant named \textsanskrit{Bhaddāli}, one of the senior disciples of Gotama, who doesn’t fulfill the training according to the Teacher’s instructions.’ You also didn’t realize this situation.” 

“I\marginnote{10.1} made a mistake, sir. It was foolish, stupid, and unskillful of me that, as this rule was being laid down by the Buddha and the \textsanskrit{Saṅgha} was undertaking it, I announced I would not try to keep it. Please, sir, accept my mistake for what it is, so I will restrain myself in future.” 

“Indeed,\marginnote{10.3} \textsanskrit{Bhaddāli}, you made a mistake. It was foolish, stupid, and unskillful of you that, as this rule was being laid down by the Buddha and the \textsanskrit{Saṅgha} was undertaking it, you announced you would not try to keep it. 

What\marginnote{11.1} do you think, \textsanskrit{Bhaddāli}? Suppose I was to say this to a mendicant who is freed both ways:\footnote{This set of seven noble disciples is detailed at \href{https://suttacentral.net/mn70/en/sujato\#14.1}{MN 70:14.1}. } ‘Please, mendicant, be a bridge for me to cross over the mud.’ Would they cross over themselves, or struggle to get out of it, or just say no?” 

“No,\marginnote{11.4} sir.” 

“What\marginnote{11.5} do you think, \textsanskrit{Bhaddāli}? Suppose I was to say the same thing to a mendicant who is freed by wisdom, or a direct witness, or attained to view, or freed by faith, or a follower of teachings, or a follower by faith: ‘Please, mendicant, be a bridge for me to cross over the mud.’ Would they cross over themselves, or struggle to get out of it, or just say no?” 

“No,\marginnote{11.13} sir.” 

“What\marginnote{12.1} do you think, \textsanskrit{Bhaddāli}? At that time were you freed both ways, freed by wisdom, a direct witness, attained to view, freed by faith, a follower of teachings, or a follower by faith?” 

“No,\marginnote{12.3} sir.” 

“Weren’t\marginnote{12.4} you vacuous, hollow, and mistaken?” 

“Yes,\marginnote{13.1} sir. 

I\marginnote{13.2} made a mistake … Please, sir, accept my mistake for what it is, so I will restrain myself in future.” 

“Indeed,\marginnote{13.4} \textsanskrit{Bhaddāli}, you made a mistake. … But since you have recognized your mistake for what it is, and have dealt with it properly, I accept it. For it is growth in the training of the Noble One to recognize a mistake for what it is, deal with it properly, and commit to restraint in the future. 

\textsanskrit{Bhaddāli},\marginnote{14.1} take a mendicant who doesn’t fulfill the training according to the Teacher’s instructions. They think, ‘Why don’t I frequent a secluded lodging—a wilderness, the root of a tree, a hill, a ravine, a mountain cave, a charnel ground, a forest, the open air, a heap of straw. Hopefully I’ll realize a superhuman distinction in knowledge and vision worthy of the noble ones.’ So they frequent a secluded lodging. While they’re living withdrawn, they’re reprimanded by the Teacher, by sensible spiritual companions after examination, by deities, and by themselves. Being reprimanded in this way, they don’t realize any superhuman distinction in knowledge and vision worthy of the noble ones. Why is that? Because that’s how it is when someone doesn’t fulfill the training according to the Teacher’s instructions. 

But\marginnote{15.1} take a mendicant who does fulfill the training according to the Teacher’s instructions. They think, ‘Why don’t I frequent a secluded lodging—a wilderness, the root of a tree, a hill, a ravine, a mountain cave, a charnel ground, a forest, the open air, a heap of straw. Hopefully I’ll realize a superhuman distinction in knowledge and vision worthy of the noble ones.’ They frequent a secluded lodging—a wilderness, the root of a tree, a hill, a ravine, a mountain cave, a charnel ground, a forest, the open air, a heap of straw. While they’re living withdrawn, they’re not reprimanded by the Teacher, by sensible spiritual companions after examination, by deities, or by themselves. Not being reprimanded in this way, they realize a superhuman distinction in knowledge and vision worthy of the noble ones. 

Quite\marginnote{16.1} secluded from sensual pleasures, secluded from unskillful qualities, they enter and remain in the first absorption, which has the rapture and bliss born of seclusion, while placing the mind and keeping it connected. Why is that? Because that’s what happens when someone fulfills the training according to the Teacher’s instructions. 

Furthermore,\marginnote{17.1} as the placing of the mind and keeping it connected are stilled, a mendicant enters and remains in the second absorption, which has the rapture and bliss born of immersion, with internal clarity and mind at one, without placing the mind and keeping it connected. Why is that? Because that’s what happens when someone fulfills the training according to the Teacher’s instructions. 

Furthermore,\marginnote{17.4} with the fading away of rapture, a mendicant enters and remains in the third absorption, where they meditate with equanimity, mindful and aware, personally experiencing the bliss of which the noble ones declare, ‘Equanimous and mindful, one meditates in bliss.’ Why is that? Because that’s what happens when someone fulfills the training according to the Teacher’s instructions. 

Furthermore,\marginnote{17.7} giving up pleasure and pain, and ending former happiness and sadness, a mendicant enters and remains in the fourth absorption, without pleasure or pain, with pure equanimity and mindfulness. Why is that? Because that’s what happens when someone fulfills the training according to the Teacher’s instructions. 

When\marginnote{18.1} their mind has become immersed in \textsanskrit{samādhi} like this—purified, bright, flawless, rid of corruptions, pliable, workable, steady, and imperturbable—they extend it toward recollection of past lives. They recollect many kinds of past lives, that is, one, two, three, four, five, ten, twenty, thirty, forty, fifty, a hundred, a thousand, a hundred thousand rebirths; many eons of the world contracting, many eons of the world expanding, many eons of the world contracting and expanding. … They recollect their many kinds of past lives, with features and details. Why is that? Because that’s what happens when someone fulfills the training according to the Teacher’s instructions. 

When\marginnote{19.1} their mind has become immersed in \textsanskrit{samādhi} like this—purified, bright, flawless, rid of corruptions, pliable, workable, steady, and imperturbable—they extend it toward knowledge of the death and rebirth of sentient beings. With clairvoyance that is purified and superhuman, they see sentient beings passing away and being reborn—inferior and superior, beautiful and ugly, in a good place or a bad place. They understand how sentient beings are reborn according to their deeds: ‘These dear beings did bad things by way of body, speech, and mind. … They’re reborn in the underworld, hell. These dear beings, however, did good things by way of body, speech, and mind. … they’re reborn in a good place, a heavenly realm.’ And so, with clairvoyance that is purified and superhuman … they understand how sentient beings are reborn according to their deeds. Why is that? Because that’s what happens when someone fulfills the training according to the Teacher’s instructions. 

When\marginnote{20.1} their mind has become immersed in \textsanskrit{samādhi} like this—purified, bright, flawless, rid of corruptions, pliable, workable, steady, and imperturbable—they extend it toward knowledge of the ending of defilements. They truly understand: ‘This is suffering’ … ‘This is the origin of suffering’ … ‘This is the cessation of suffering’ … ‘This is the practice that leads to the cessation of suffering’. They truly understand: ‘These are defilements’ … ‘This is the origin of defilements’ … ‘This is the cessation of defilements’ … ‘This is the practice that leads to the cessation of defilements’. 

Knowing\marginnote{21.1} and seeing like this, their mind is freed from the defilements of sensuality, desire to be reborn, and ignorance. When they’re freed, they know they’re freed. 

They\marginnote{21.3} understand: ‘Rebirth is ended, the spiritual journey has been completed, what had to be done has been done, there is nothing further for this place.’ Why is that? Because that’s what happens when someone fulfills the training according to the Teacher’s instructions.” 

When\marginnote{22.1} he said this, Venerable \textsanskrit{Bhaddāli} said to the Buddha, “What is the cause, sir, what is the reason why they punish some monk, repeatedly pressuring him?\footnote{Read \textit{pavayha} per \href{https://suttacentral.net/mn122/en/sujato\#27.3}{MN 122:27.3} for \textsanskrit{Mahāsaṅgīti}’s \textit{pasayha} here. | Vinaya matters are dealt with separately by the monks and nuns, hence I use “monk” rather than “mendicant” here. } And what is the cause, what is the reason why they don’t similarly punish another monk, repeatedly pressuring him?” 

“Take\marginnote{23.1} a monk who is a frequent offender with many offenses. When admonished by the monks, he dodges the issue, distracting the discussion with irrelevant points. He displays annoyance, hate, and bitterness. He doesn’t proceed properly, he doesn’t fall in line, he doesn’t proceed to get past it, and he doesn’t say: ‘I’ll do what pleases the \textsanskrit{Saṅgha}.’ In such a case, the monks say: ‘Reverends, this monk is a frequent offender, with many offenses. When admonished by the monks, he dodges the issue, distracting the discussion with irrelevant points. He displays annoyance, hate, and bitterness. He doesn’t proceed properly, he doesn’t fall in line, he doesn’t proceed to get past it, and he doesn’t say: “I’ll do what pleases the \textsanskrit{Saṅgha}.” It’d be good for the venerables to examine this monk in such a way that this disciplinary issue is not quickly settled.’ And that’s what they do. 

Take\marginnote{24.1} some other monk who is a frequent offender with many offenses. When admonished by the monks, he doesn’t dodge the issue, distracting the discussion with irrelevant points. He doesn’t display annoyance, hate, and bitterness. He proceeds properly, he falls in line, he proceeds to get past it, and he says: ‘I’ll do what pleases the \textsanskrit{Saṅgha}.’ In such a case, the monks say: ‘Reverends, this monk is a frequent offender, with many offenses. When admonished by the monks, he doesn’t dodge the issue, distracting the discussion with irrelevant points. He doesn’t display annoyance, hate, and bitterness. He proceeds properly, he falls in line, he proceeds to get past it, and he says: “I’ll do what pleases the \textsanskrit{Saṅgha}.” It’d be good for the venerables to examine this monk in such a way that this disciplinary issue is quickly settled.’ And that’s what they do. 

Take\marginnote{25.1} some other monk who is an occasional offender without many offenses. When admonished by the monks, he dodges the issue … In such a case, the monks say: ‘Reverends, this monk is an occasional offender without many offenses. When admonished by the monks, he dodges the issue … It’d be good for the venerables to examine this monk in such a way that this disciplinary issue is not quickly settled.’ And that’s what they do. 

Take\marginnote{26.1} some other monk who is an occasional offender without many offenses. When admonished by the monks, he doesn’t dodge the issue … In such a case, the monks say: ‘Reverends, this monk is an occasional offender without many offenses. When admonished by the monks, he doesn’t dodge the issue … It’d be good for the venerables to examine this monk in such a way that this disciplinary issue is quickly settled.’ And that’s what they do. 

Take\marginnote{27.1} some other monk who gets by with mere faith and love. In such a case, the monks say: ‘Reverends, this monk gets by with mere faith and love. Suppose we punish him, repeatedly pressuring him—\footnote{This reads like a broken sentence; they got half way through, then realized the implications of what they were saying. } no, let him not lose what little faith and love he has!’ 

Suppose\marginnote{28.1} there was a person with one eye. Their friends and colleagues, relatives and kin would protect that one eye: ‘Let them not lose the one eye that they have!’ In the same way, some monk gets by with mere faith and love. In such a case, the monks say: ‘Reverends, this monk gets by with mere faith and love. Suppose we punish him, repeatedly pressuring him—no, let him not lose what little faith and love he has!’ This is the cause, this is the reason why they punish some monk, repeatedly pressuring him. And this is the cause, this is the reason why they don’t similarly punish another monk, repeatedly pressuring him.” 

“What\marginnote{29.1} is the cause, sir, what is the reason why there used to be fewer training rules but more enlightened mendicants?\footnote{This question was also asked by Kassapa at \href{https://suttacentral.net/sn16.13/en/sujato}{SN 16.13}. } And what is the cause, what is the reason why these days there are more training rules and fewer enlightened mendicants?” 

“That’s\marginnote{30.1} how it is, \textsanskrit{Bhaddāli}. When sentient beings are in decline and the true teaching is disappearing there are more training rules and fewer enlightened mendicants. The Teacher does not lay down training rules for disciples as long as certain defiling influences have not appeared in the \textsanskrit{Saṅgha}.\footnote{This echoes the Buddha’s words to \textsanskrit{Sāriputta} at \href{https://suttacentral.net/pli-tv-bu-vb-pj1/en/sujato\#3.4.6}{Bu Pj 1:3.4.6}. } But when such defiling influences appear in the \textsanskrit{Saṅgha}, the Teacher lays down training rules for disciples to protect against them. 

And\marginnote{31.1} they don’t appear until the \textsanskrit{Saṅgha} has attained a great size, an abundance of material support and fame, learning, and seniority. But when the \textsanskrit{Saṅgha} has attained these things, then such defiling influences appear in the \textsanskrit{Saṅgha}, and the Teacher lays down training rules for disciples to protect against them. 

There\marginnote{32.1} were only a few of you there at the time when I taught the exposition on the simile of the thoroughbred colt.\footnote{While there are many suttas on a thoroughbred, the simile taught here does not appear elsewhere in the Pali canon. Perhaps this is in reference to \href{https://suttacentral.net/mn107/en/sujato\#3.2}{MN 107:3.2}, which, however, only has one line in common. } Do you remember that, \textsanskrit{Bhaddāli}?” 

“No,\marginnote{32.3} sir.” 

“What\marginnote{32.4} do you believe the reason for that is?” 

“Sir,\marginnote{32.5} it’s surely because for a long time now I haven’t fulfilled the training according to the Teacher’s instructions.” 

“That’s\marginnote{32.6} not the only reason, \textsanskrit{Bhaddāli}. Rather, for a long time I have comprehended your mind and known: ‘While I’m teaching, this futile man doesn’t pay attention, apply the mind, concentrate wholeheartedly, or actively listen to the teaching.’ Still, \textsanskrit{Bhaddāli}, I shall teach the exposition on the simile of the thoroughbred colt. Listen and apply your mind well, I will speak.” 

“Yes,\marginnote{32.11} sir,” \textsanskrit{Bhaddāli} replied. The Buddha said this: 

“Suppose\marginnote{33.1} a deft horse trainer were to obtain a fine thoroughbred. First of all he’d make it get used to wearing the bit. Because it has not done this before, it still resorts to some twists, ducks, and dodges. But with regular and gradual practice its bad behavior is extinguished.\footnote{For the phrase \textit{\textsanskrit{tasmiṁ} \textsanskrit{ṭhāne} \textsanskrit{parinibbāyati}} (“its bad behavior is extinguished”, literally “it is extinguished regarding that state”) I follow the commentary, which specifies that \textit{\textsanskrit{tasmiṁ} \textsanskrit{ṭhāne}} refers to the twists and dodges that cease rather than the good habits that are developed. } 

When\marginnote{33.4} it has done this, the horse trainer next makes it get used to wearing the harness. Because it has not done this before, it still resorts to some twists, ducks, and dodges. But with regular and gradual practice its bad behavior is extinguished. 

When\marginnote{33.7} it has done this, the horse trainer next makes it get used to walking in procession, circling, prancing, galloping, charging, the protocols and traditions of court, and in the very best speed, fleetness, and friendliness.\footnote{\textit{\textsanskrit{Rājaguṇe}} occurs in \href{https://suttacentral.net/mil5.1.4/en/sujato}{Mil 5.1.4}, where it distinguishes royal officials from ordinary ones. I take it and \textit{\textsanskrit{rājavaṁse}} to refer to “the protocols and traditions of court” rather than “kingly qualities”. } Because it has not done this before, it still resorts to some twists, ducks, and dodges. But with regular and gradual practice its bad behavior is extinguished. 

When\marginnote{33.10} it has done this, the horse trainer next rewards it with a grooming and a rub down. A fine royal thoroughbred with these ten factors is worthy of a king, fit to serve a king, and reckoned as a factor of kingship. 

In\marginnote{34.1} the same way, a mendicant with ten qualities is worthy of offerings dedicated to the gods, worthy of hospitality, worthy of a religious donation, worthy of veneration with joined palms, and is the supreme field of merit for the world. What ten? It’s when a mendicant has an adept’s right view, right thought, right speech, right action, right livelihood, right effort, right mindfulness, right immersion, right knowledge, and right freedom. A mendicant with these ten qualities is worthy of offerings dedicated to the gods, worthy of hospitality, worthy of a religious donation, worthy of veneration with joined palms, and is the supreme field of merit for the world.” 

That\marginnote{34.5} is what the Buddha said. Satisfied, Venerable \textsanskrit{Bhaddāli} approved what the Buddha said. 

%
\section*{{\suttatitleacronym MN 66}{\suttatitletranslation The Simile of the Quail }{\suttatitleroot Laṭukikopamasutta}}
\addcontentsline{toc}{section}{\tocacronym{MN 66} \toctranslation{The Simile of the Quail } \tocroot{Laṭukikopamasutta}}
\markboth{The Simile of the Quail }{Laṭukikopamasutta}
\extramarks{MN 66}{MN 66}

\scevam{So\marginnote{1.1} I have heard. }At one time the Buddha was staying in the land of the \textsanskrit{Aṅguttarāpans}, near the town of theirs named \textsanskrit{Āpaṇa}. 

Then\marginnote{2.1} the Buddha robed up in the morning and, taking his bowl and robe, entered \textsanskrit{Āpaṇa} for alms. He wandered for alms in \textsanskrit{Āpaṇa}. After the meal, on his return from almsround, he went to a certain forest grove for the day’s meditation. Having plunged deep into it, he sat at the root of a certain tree to meditate. 

Venerable\marginnote{3.1} \textsanskrit{Udāyī} also robed up in the morning and, taking his bowl and robe, entered \textsanskrit{Āpaṇa} for alms. He wandered for alms in \textsanskrit{Āpaṇa}. After the meal, on his return from almsround, he went to that forest grove for the day’s meditation. Having plunged deep into it, he sat at the root of a certain tree to meditate. Then as Venerable \textsanskrit{Udāyī} was in private retreat this thought came to his mind: 

“The\marginnote{4.1} Buddha has rid us of so many things that bring suffering and gifted us so many things that bring happiness! He has rid us of so many unskillful things and gifted us so many skillful things!” 

Then\marginnote{5.1} in the late afternoon, \textsanskrit{Udāyī} came out of retreat and went to the Buddha. He bowed, sat down to one side, and said to him: 

“Just\marginnote{6.1} now, sir, as I was in private retreat this thought came to mind: ‘The Buddha has rid us of so many things that bring suffering and gifted us so many things that bring happiness! He has rid us of so many unskillful things and gifted us so many skillful things!’ 

For\marginnote{6.4} we used to eat in the evening, the morning, and at the wrong time of day.\footnote{This sutta distinguishes the “wrong time of day”, i.e. the afternoon, and the “wrong time of night”, i.e. any time when it is dark. The common expression “eating at night and at the wrong time” (\href{https://suttacentral.net/mn27/en/sujato\#13.9}{MN 27:13.9}) thus refers to both these periods. The Vinaya rule defines \textit{\textsanskrit{vikāla}} for the purpose of this rule to include both afternoon and nighttime (\href{https://suttacentral.net/pli-tv-bu-vb-pc37/en/sujato\#2.1.5}{Bu Pc 37:2.1.5}), and it is this understanding that prevails in the tradition. } But then there came a time when the Buddha addressed the mendicants, saying, ‘Please, mendicants, give up that meal at the wrong time of day.’ At that, sir, we became sad and upset, ‘But these faithful householders give us delicious fresh and cooked foods at the wrong time of day. And the Blessed One tells us to give it up! The Holy One tells us to let it go!’ But when we considered our love and respect for the Buddha, and our sense of conscience and prudence, we gave up that meal at the wrong time of day. Then we ate in the evening and the morning. 

But\marginnote{6.11} then there came a time when the Buddha addressed the mendicants, saying, ‘Please, mendicants, give up that meal at the wrong time of night.’\footnote{Recorded at \href{https://suttacentral.net/mn70/en/sujato\#1.3}{MN 70:1.3}. } At that, sir, we became sad and upset, ‘But that’s considered the more delicious of the two meals. And the Blessed One tells us to give it up! The Holy One tells us to let it go!’ Once it so happened that a certain person got some soup during the day. He said, ‘Come, let’s set this aside; we’ll enjoy it together this evening.’ Nearly all meals are prepared at night, only a few in the day. But when we considered our love and respect for the Buddha, and our sense of conscience and prudence, we gave up that meal at the wrong time of night. 

In\marginnote{6.19} the past, mendicants went wandering for alms in the dark of the night. They walked into a swamp, or fell into a sewer, or collided with a thorn bush, or collided with a sleeping cow, or encountered youths escaping a crime or on their way to commit one, or were invited by a female to commit a lewd act.\footnote{“Youths escaping a crime or on their way to commit one” (\textit{katakammehipi akatakammehipi}) follows the commentary. See also \href{https://suttacentral.net/an5.77/en/sujato\#5.3}{AN 5.77:5.3}. } 

Once\marginnote{6.20} it so happened that I was wandering for alms in the dark of the night. A woman washing a pot saw me by a flash of lightning. Startled, she cried out, ‘Woe is me! It’s a damn goblin!’\footnote{Goblins (\textit{\textsanskrit{pisāca}}) and other fiends roamed freely at night, kept at bay only by means of potent spells (eg. Rig Veda 1.133.5a, Atharva Veda 1.16, 5.29, 6.32, 8.6). | This interjection is found at \href{https://suttacentral.net/ja525/en/sujato\#2.1}{Ja 525:2.1}, \href{https://suttacentral.net/ja547/en/sujato\#62.3}{Ja 547:62.3}, and \href{https://suttacentral.net/pli-tv-kd15/en/sujato\#10.2.4}{Kd 15:10.2.4}. In all cases read \textit{\textsanskrit{abhuṁ} me}; also accept the Vinaya reading \textit{\textsanskrit{pisāco} \textsanskrit{vatāyan}}. Rig Vedic \textit{abhva} means “formless, void”, while in Atharva Veda 4.17.5 and 7.23.1 it describes a fiend who is to be banished. } 

When\marginnote{6.24} she said this, I said to her, ‘Sister, I am no goblin. I’m a mendicant waiting for alms.’ 

‘Die,\marginnote{6.27} mendicant’s father! Die, mendicant’s mother!\footnote{\textit{\textsanskrit{Mārī}} is obscure, but the Chinese parallel is  framed as a curse, which makes sense for the Pali too (MA 192 at T i 741b9). Agni \textsanskrit{Purāna} 137 provides a \textit{\textsanskrit{mahāmārī}} spell for bringing death. | \textit{Ātu} is explained in the commentary as “father”. The Chinese parallel agrees that both parents are meant. | For the uttering of a curse upon the relatives of the cursed, see Atharva Veda 2.32.4 (\textit{\textsanskrit{hatamātā}}), 5.29.6. } Better to have your belly sliced open with a sharp meat cleaver than to wander for alms in the dark of night for the sake of your belly!’ 

Recollecting\marginnote{6.29} that, I thought, ‘The Buddha has rid us of so many things that bring suffering and gifted us so many things that bring happiness! He has rid us of so many unskillful things and gifted us so many skillful things!’” 

“This\marginnote{7.1} is exactly what happens when some silly people are told by me to give something up. They say,\footnote{The idiom \textit{evameva pana} has an adversarial sense. } ‘What, such a trivial, insignificant thing as this? This ascetic is much too strict!’ They don’t give it up, and they nurse bitterness towards me; and for the mendicants who want to train, that becomes a strong, firm, stout bond, a tie that has not rotted, and a heavy yoke.\footnote{The “mendicants who want to train” are those who are good-hearted and sincere in their practice, but still struggle to overcome defilements. } 

Suppose\marginnote{8.1} a quail was tied with a vine, and was waiting there to be injured, caged, or killed. Would it be right to say that, for that quail, that vine is weak, feeble, rotten, and insubstantial?” 

“No,\marginnote{8.5} sir. For that quail, that vine is a strong, firm, stout bond, a tie that has not rotted, and a heavy yoke.” 

“In\marginnote{8.7} the same way, when some silly people are told by me to give something up, they say, ‘What, such a trivial, insignificant thing as this? This ascetic is much too strict!’ They don’t give it up, and they nurse bitterness towards me; and for the mendicants who want to train, that becomes a strong, firm, stout bond, a tie that has not rotted, and a heavy yoke. 

But\marginnote{9.1} when some gentlemen are told by me to give something up, they say, ‘What, we just have to give up such a trivial, insignificant thing as this, when the Blessed One tells us to give it up, the Holy One tells us to let it go?’ They give it up, and they don’t nurse bitterness towards me; and when the mendicants who want to train have given that up, they live relaxed, unruffled, surviving on charity, their hearts free as a wild deer. For them, that bond is weak, feeble, rotten, and insubstantial. 

Suppose\marginnote{10.1} there was a royal bull elephant with tusks like chariot-poles, able to draw a heavy load, pedigree and battle-hardened. And it was bound with a strong harness. But just by twisting its body a little, it would break apart its bonds and go wherever it wants. Would it be right to say that, for that bull elephant, that strong harness is a strong, firm, stout bond, a tie that has not rotted, and a heavy yoke?” 

“No,\marginnote{10.5} sir. For that bull elephant, that strong harness is weak, feeble, rotten, and insubstantial.” 

“In\marginnote{10.7} the same way, when some gentlemen are told by me to give something up, they say, ‘What, we just have to give up such a trivial, insignificant thing as this, when the Blessed One tells us to give it up, the Holy One tells us to let it go?’ They give it up, and they don’t nurse bitterness towards me; and when the mendicants who want to train have given that up, they live relaxed, unruffled, surviving on charity, their hearts free as a wild deer. For them, that bond is weak, feeble, rotten, and insubstantial. 

Suppose\marginnote{11.1} there was a poor man, with few possessions and little wealth. He had a single broken-down hovel open to the crows, not the best sort; a single broken-down couch, not the best sort; a single pot for storing grain, not the best sort; and a single wifey, not the best sort. He’d see a mendicant sitting in meditation in the cool shade, their hands and feet well washed after eating a delectable meal.\footnote{A similar phrase is attributed to \textsanskrit{Sāriputta} in the Jain \textsanskrit{Isibhāsiyāiṁ} 38.2. } He’d think, ‘The ascetic life is so very pleasant! The ascetic life is so very healthy! If only I could shave off my hair and beard, dress in ocher robes, and go forth from the lay life to homelessness.’ But he’s not able to give up his broken-down hovel, his broken-down couch, his pot for storing grain, or his wifey—none of which are the best sort—in order to go forth. Would it be right to say that, for that man, those bonds are weak, feeble, rotten, and insubstantial?” 

“No,\marginnote{11.12} sir. For that man, they are a strong, firm, stout bond, a tie that has not rotted, and a heavy yoke.” 

“In\marginnote{11.15} the same way, when some silly people are told by me to give something up, they say, ‘What, such a trivial, insignificant thing as this? This ascetic is much too strict!’ They don’t give it up, and they nurse bitterness towards me; and for the mendicants who want to train, that becomes a strong, firm, stout bond, a tie that has not rotted, and a heavy yoke. 

Suppose\marginnote{12.1} there was a rich householder or householder’s child, affluent, and wealthy. He had a vast amount of gold ingots, grain, fields, lands, wives, and male and female bondservants.\footnote{Several sizes of gold coins were reckoned in ancient India, of which the \textit{nikkha} (“gold ingot”) was the largest. } He’d see a mendicant sitting in meditation in the cool shade, their hands and feet well washed after eating a delectable meal. He’d think, ‘The ascetic life is so very pleasant! The ascetic life is so very healthy! If only I could shave off my hair and beard, dress in ocher robes, and go forth from the lay life to homelessness.’ And he is able to give up his vast amount of gold ingots, grain, fields, lands, wives, and male and female bondservants in order to go forth. Would it be right to say that, for that householder, they are a strong, firm, stout bond, a tie that has not rotted, and a heavy yoke?” 

“No,\marginnote{12.10} sir. For that householder, those bonds are weak, feeble, rotten, and insubstantial.” 

“In\marginnote{12.13} the same way, when some gentlemen are told by me to give something up, they say, ‘What, we just have to give up such a trivial, insignificant thing as this, when the Blessed One tells us to give it up, the Holy One tells us to let it go?’ They give it up, and they don’t nurse bitterness towards me; and when the mendicants who want to train have given that up, they live relaxed, unruffled, surviving on charity, their hearts free as a wild deer. For them, that bond is weak, feeble, rotten, and insubstantial. 

\textsanskrit{Udāyī},\marginnote{13.1} these four people are found in the world. What four? 

Take\marginnote{14.1} a certain person practicing to give up and let go of attachments. As they do so, memories and thoughts connected with attachments beset them. They tolerate them and don’t give them up, get rid of them, eliminate them, and obliterate them. I call this person ‘fettered’, not ‘detached’. Why is that? Because I understand the diversity of faculties as it applies to this person.\footnote{“Diversity of faculties” refers to the differing capabilities of individuals, so \textit{\textsanskrit{imasmiṁ} puggale} must be locative of reference. } 

Take\marginnote{15.1} another person practicing to give up and let go of attachments. As they do so, memories and thoughts connected with attachments beset them. They don’t tolerate them, but give them up, get rid of them, eliminate them, and obliterate them. I call this person ‘fettered’, not ‘detached’. Why is that? Because I understand the diversity of faculties as it applies to this person. 

Take\marginnote{16.1} another person practicing to give up and let go of attachments. As they do so, every so often they lose mindfulness, and memories and thoughts connected with attachments beset them. Their mindfulness is slow to come up, but they quickly give up, get rid of, eliminate, and obliterate those thoughts. Suppose there was an iron cauldron that had been heated all day, and a person let two or three drops of water fall onto it. The drops would be slow to fall, but they’d quickly dry up and evaporate. 

In\marginnote{16.7} the same way, take a person practicing to give up and let go of attachments. As they do so, every so often they lose mindfulness, and memories and thoughts connected with attachments beset them. Their mindfulness is slow to come up, but they quickly give them up, get rid of, eliminate, and obliterate those thoughts. I also call this person ‘fettered’, not ‘detached’. Why is that? Because I understand the diversity of faculties as it applies to this person. 

Take\marginnote{17.1} another person who, understanding that attachment is the root of suffering, is freed with the ending of attachments. I call this person ‘detached’, not ‘fettered’. Why is that? Because I understand the diversity of faculties as it applies to this person. These are the four people found in the world. 

\textsanskrit{Udāyī},\marginnote{18.1} these are the five kinds of sensual stimulation. What five? Sights known by the eye, which are likable, desirable, agreeable, pleasant, sensual, and arousing. Sounds known by the ear … Smells known by the nose … Tastes known by the tongue … Touches known by the body, which are likable, desirable, agreeable, pleasant, sensual, and arousing. These are the five kinds of sensual stimulation. 

The\marginnote{19.1} pleasure and happiness that arise from these five kinds of sensual stimulation is called sensual pleasure—a filthy, ordinary, ignoble pleasure. Such pleasure should not be cultivated or developed, but should be feared, I say. 

Take\marginnote{20.1} a mendicant who, quite secluded from sensual pleasures, secluded from unskillful qualities, enters and remains in the first absorption … second absorption … third absorption … fourth absorption. 

This\marginnote{21.1} is called the pleasure of renunciation, the pleasure of seclusion, the pleasure of peace, the pleasure of awakening. Such pleasure should be cultivated and developed, and should not be feared, I say. 

Take\marginnote{22.1} a mendicant who, quite secluded from sensual pleasures, secluded from unskillful qualities, enters and remains in the first absorption. This belongs to the perturbable, I say. And what there belongs to the perturbable? Whatever placing of the mind and keeping it connected has not ceased there is what belongs to the perturbable. 

Take\marginnote{23.1} a mendicant who, as the placing of the mind and keeping it connected are stilled, enters and remains in the second absorption. This belongs to the perturbable, I say. And what there belongs to the perturbable? Whatever rapture and bliss has not ceased there is what belongs to the perturbable. 

Take\marginnote{24.1} a mendicant who, with the fading away of rapture, enters and remains in the third absorption. This belongs to the perturbable. And what there belongs to the perturbable? Whatever bliss with equanimity has not ceased there is what belongs to the perturbable. 

Take\marginnote{25.1} a mendicant who, giving up pleasure and pain, enters and remains in the fourth absorption. This belongs to the imperturbable, I say. 

Take\marginnote{26.1} a mendicant who, quite secluded from sensual pleasures, secluded from unskillful qualities, enters and remains in the first absorption. But this is not enough, I say: give it up, go beyond it. And what goes beyond it? 

Take\marginnote{27.1} a mendicant who, as the placing of the mind and keeping it connected are stilled, enters and remains in the second absorption. That goes beyond it. But this too is not enough, I say: give it up, go beyond it. And what goes beyond it? 

Take\marginnote{28.1} a mendicant who, with the fading away of rapture, enters and remains in the third absorption. That goes beyond it. But this too is not enough, I say: give it up, go beyond it. And what goes beyond it? 

Take\marginnote{29.1} a mendicant who, giving up pleasure and pain, enters and remains in the fourth absorption. That goes beyond it. But this too is not enough, I say: give it up, go beyond it. And what goes beyond it? 

Take\marginnote{30.1} a mendicant who, going totally beyond perceptions of form, with the ending of perceptions of impingement, not focusing on perceptions of diversity, aware that ‘space is infinite’, enters and remains in the dimension of infinite space. That goes beyond it. But this too is not enough, I say: give it up, go beyond it. And what goes beyond it? 

Take\marginnote{31.1} a mendicant who, going totally beyond the dimension of infinite space, aware that ‘consciousness is infinite’, enters and remains in the dimension of infinite consciousness. That goes beyond it. But this too is not enough, I say: give it up, go beyond it. And what goes beyond it? 

Take\marginnote{32.1} a mendicant who, going totally beyond the dimension of infinite consciousness, aware that ‘there is nothing at all’, enters and remains in the dimension of nothingness. That goes beyond it. But this too is not enough, I say: give it up, go beyond it. And what goes beyond it? 

Take\marginnote{33.1} a mendicant who, going totally beyond the dimension of nothingness, enters and remains in the dimension of neither perception nor non-perception. That goes beyond it. But this too is not enough, I say: give it up, go beyond it. And what goes beyond it? 

Take\marginnote{34.1} a mendicant who, going totally beyond the dimension of neither perception nor non-perception, enters and remains in the cessation of perception and feeling. That goes beyond it. 

So,\marginnote{34.2} \textsanskrit{Udāyī}, I even recommend giving up the dimension of neither perception nor non-perception. Do you see any fetter, large or small, that I don’t recommend giving up?” 

“No,\marginnote{34.4} sir.” 

That\marginnote{34.5} is what the Buddha said. Satisfied, Venerable \textsanskrit{Udāyī} approved what the Buddha said. 

%
\section*{{\suttatitleacronym MN 67}{\suttatitletranslation At Cātumā }{\suttatitleroot Cātumasutta}}
\addcontentsline{toc}{section}{\tocacronym{MN 67} \toctranslation{At Cātumā } \tocroot{Cātumasutta}}
\markboth{At Cātumā }{Cātumasutta}
\extramarks{MN 67}{MN 67}

\scevam{So\marginnote{1.1} I have heard. }At one time the Buddha was staying near \textsanskrit{Cātumā} in a myrobalan grove.\footnote{\textsanskrit{Cātumā} is a Sakyan town mentioned only here. | The introduction is similar to \href{https://suttacentral.net/ud3.3/en/sujato}{Ud 3.3}. } 

Now\marginnote{2.1} at that time five hundred mendicants headed by \textsanskrit{Sāriputta} and \textsanskrit{Moggallāna} arrived at \textsanskrit{Cātumā} to see the Buddha. And the visiting mendicants, while exchanging pleasantries with the resident mendicants, preparing their lodgings, and putting away their bowls and robes, made a dreadful racket. 

Then\marginnote{3.1} the Buddha said to Venerable Ānanda, “Ānanda, who’s making that dreadful racket? You’d think it was fishermen hauling in a catch!” 

And\marginnote{3.3} Ānanda told him what had happened. 

“Well\marginnote{4.1} then, Ānanda, in my name tell those mendicants that the teacher summons them.” 

“Yes,\marginnote{4.3} sir,” Ānanda replied. He went to those mendicants and said, “Venerables, the teacher summons you.” 

“Yes,\marginnote{4.5} reverend,” replied those mendicants. Then they rose from their seats and went to the Buddha, bowed, and sat down to one side. The Buddha said to them: 

“Mendicants,\marginnote{4.6} what’s with that dreadful racket? You’d think it was fishermen hauling in a catch!” 

And\marginnote{4.7} they told him what had happened. 

“Go\marginnote{5.1} away, mendicants, I dismiss you. You are not to stay in my presence.” 

“Yes,\marginnote{5.2} sir,” replied those mendicants. They got up from their seats, bowed, and respectfully circled the Buddha, keeping him on their right. They set their lodgings in order and left, taking their bowls and robes. 

Now\marginnote{6.1} at that time the Sakyans of \textsanskrit{Cātumā} were sitting together at the town hall on some business. Seeing those mendicants coming off in the distance, they went up to them and said, “Hello venerables, where are you going?” 

“Sirs,\marginnote{6.5} the mendicant \textsanskrit{Saṅgha} has been dismissed by the Buddha.” 

“Well\marginnote{6.6} then, venerables, sit here for an hour. Hopefully we’ll be able to restore the Buddha’s confidence.” 

“Yes,\marginnote{6.7} sirs,” replied the mendicants. 

Then\marginnote{7.1} the Sakyans of \textsanskrit{Cātumā} went up to the Buddha, bowed, sat down to one side, and said to him: 

“May\marginnote{7.2} the Buddha be happy with the mendicant \textsanskrit{Saṅgha}! May the Buddha welcome the mendicant \textsanskrit{Saṅgha}! May the Buddha support the mendicant \textsanskrit{Saṅgha} now as he did in the past! There are mendicants here who are junior, recently gone forth, newly come to this teaching and training. If they don’t get to see the Buddha they may change and fall apart. If young seedlings don’t get water they may change and fall apart. In the same way, there are mendicants here who are junior, recently gone forth, newly come to this teaching and training. If they don’t get to see the Buddha they may change and fall apart. If a young calf doesn’t see its mother it may change and fall apart. In the same way, there are mendicants here who are junior, recently gone forth, newly come to this teaching and training. If they don’t get to see the Buddha they may change and fall apart. May the Buddha be happy with the mendicant \textsanskrit{Saṅgha}! May the Buddha welcome the mendicant \textsanskrit{Saṅgha}! May the Buddha support the mendicant \textsanskrit{Saṅgha} now as he did in the past!” 

Then\marginnote{8.1} the divinity Sahampati knew what the Buddha was thinking. As easily as a strong person would extend or contract their arm, he vanished from the realm of divinity and reappeared in front of the Buddha. He arranged his robe over one shoulder, raised his joined palms toward the Buddha, and said: 

“May\marginnote{9.1} the Buddha be happy with the mendicant \textsanskrit{Saṅgha}! May the Buddha welcome the mendicant \textsanskrit{Saṅgha}! May the Buddha support the mendicant \textsanskrit{Saṅgha} now as he did in the past! There are mendicants here who are junior, recently gone forth, newly come to this teaching and training. If they don’t get to see the Buddha they may change and fall apart. If young seedlings don’t get water they may change and fall apart. … If a young calf doesn’t see its mother it may change and fall apart. In the same way, there are mendicants here who are junior, recently gone forth, newly come to this teaching and training. If they don’t get to see the Buddha they may change and fall apart. May the Buddha be happy with the mendicant \textsanskrit{Saṅgha}! May the Buddha welcome the mendicant \textsanskrit{Saṅgha}! May the Buddha support the mendicant \textsanskrit{Saṅgha} now as he did in the past!” 

The\marginnote{10.1} Sakyans of \textsanskrit{Cātumā} and the divinity Sahampati were able to restore the Buddha’s confidence with the similes of the seedlings and the calf. 

Then\marginnote{11.1} Venerable \textsanskrit{Mahāmoggallāna} addressed the mendicants, “Get up, reverends, and pick up your bowls and robes. The Buddha’s confidence has been restored by the Sakyans of \textsanskrit{Cātumā} and the divinity Sahampati with the similes of the seedlings and the calf.” 

“Yes,\marginnote{12.1} reverend,” replied those mendicants. Then they rose from their seats and, taking their bowls and robes, went to the Buddha, bowed, and sat down to one side. The Buddha said to Venerable \textsanskrit{Sāriputta}, “\textsanskrit{Sāriputta}, what did you think when the mendicant \textsanskrit{Saṅgha} was dismissed by me?” 

“Sir,\marginnote{12.3} I thought this: ‘The Buddha has dismissed the mendicant \textsanskrit{Saṅgha}. Now he will remain passive, dwelling in blissful meditation in this life, and so will we.’” 

“Hold\marginnote{12.6} on, \textsanskrit{Sāriputta}, hold on! Don’t you ever think such a thing again!”\footnote{The PTS reading is preferable: \textit{na kho te \textsanskrit{Sāriputta} puna pi \textsanskrit{evarūpaṁ} \textsanskrit{cittaṁ} \textsanskrit{uppādetabbanti}}. } 

Then\marginnote{13.1} the Buddha addressed Venerable \textsanskrit{Mahāmoggallāna}, “\textsanskrit{Moggallāna}, what did you think when the mendicant \textsanskrit{Saṅgha} was dismissed by me?” 

“Sir,\marginnote{13.3} I thought this: ‘The Buddha has dismissed the mendicant \textsanskrit{Saṅgha}. Now he will remain passive, dwelling in blissful meditation in this life. Meanwhile, Venerable \textsanskrit{Sāriputta} and I shall lead the mendicant \textsanskrit{Saṅgha}.’” 

“Good,\marginnote{13.6} good, \textsanskrit{Moggallāna}! For either I should lead the mendicant \textsanskrit{Saṅgha}, or else \textsanskrit{Sāriputta} and \textsanskrit{Moggallāna}.” 

Then\marginnote{14.1} the Buddha said to the mendicants: 

“Mendicants,\marginnote{14.2} when you go into the water you should anticipate four dangers.\footnote{Also at \href{https://suttacentral.net/an4.122/en/sujato}{AN 4.122}. } What four? The dangers of waves, gharials, whirlpools, and sharks. These are the four dangers that you should anticipate when you go into the water. 

In\marginnote{15.1} the same way, a person gone forth from the lay life to homelessness in this teaching and training should anticipate four dangers. What four? The dangers of waves, gharials, whirlpools, and sharks. 

And\marginnote{16.1} what, mendicants, is the danger of waves?\footnote{Indian literature uses the wave as metaphor for a “surge” of disturbing emotion. } It’s when a gentleman has gone forth out of faith from the lay life to homelessness, thinking: ‘I’m swamped by rebirth, old age, and death; by sorrow, lamentation, pain, sadness, and distress. I’m swamped by suffering, mired in suffering. Hopefully I can find an end to this entire mass of suffering.’ When they’ve gone forth, their spiritual companions advise and instruct them: ‘You should go out like this, and come back like that. You should look to the front like this, and to the side like that. You should contract your limbs like this, and extend them like that. This is how you should bear your outer robe, bowl, and robes.’\footnote{These activities are elsewhere framed as the practice of “situational awareness” (\href{https://suttacentral.net/mn10/en/sujato\#8.1}{MN 10:8.1}). } They think: ‘Formerly, as laypeople, we advised and instructed others.\footnote{When challenged, he uses the plural as a defense mechanism, disguising his personal reaction as an undefined “us”, implicating the \textsanskrit{Saṅgha} as a whole. } And now these mendicants—who you’d think were our children or grandchildren—imagine they can advise and instruct us!’ They resign the training and return to a lesser life. This is called a mendicant who resigns the training and returns to a lesser life for fear of the danger of waves. ‘Danger of waves’ is a term for anger and distress. 

And\marginnote{17.1} what, mendicants, is the danger of gharials?\footnote{The gharial is a large crocodilian of the Ganges. The adult male has a distinctive “pot” (\textit{ghara}) at the tip of its snout, from which it takes its name. The Pali \textit{\textsanskrit{kumbhīla}} (Sanskrit \textit{\textsanskrit{kumbhīra}}) similarly stems from \textit{kumbha}, “pot”. Gharials can be seen breaking water to swallow fish, their long toothy snouts making them an apt symbol for gluttony. | Referenced by the nun \textsanskrit{Sumedhā} in her verses at \href{https://suttacentral.net/thig16.1/en/sujato\#55.3}{Thig 16.1:55.3}. } It’s when a gentleman has gone forth out of faith from the lay life to homelessness, thinking: ‘I’m swamped by rebirth, old age, and death; by sorrow, lamentation, pain, sadness, and distress. I’m swamped by suffering, mired in suffering. Hopefully I can find an end to this entire mass of suffering.’ When they’ve gone forth, their spiritual companions advise and instruct them: ‘You may eat, consume, taste, and drink these things, but not those. You may eat what’s allowable, but not what’s unallowable. You may eat at the right time, but not at the wrong time.’ They think: ‘Formerly, as laypeople, we used to eat, consume, taste, and drink what we wanted, not what we didn’t want. We ate and drank both allowable and unallowable things, at the right time and the wrong time. And these faithful householders give us delicious fresh and cooked foods at the wrong time of day. But these guys imagine they can gag our mouths!’ They resign the training and return to a lesser life. This is called one who rejects the training and returns to a lesser life because they’re afraid of the danger of gharials. ‘Danger of gharials’ is a term for gluttony. 

And\marginnote{18.1} what, mendicants, is the danger of whirlpools?\footnote{\textit{\textsanskrit{Āvaṭṭa}} means a “whirlpool”, but also to “revert”. Cf. the “return to a lesser life” (\textit{\textsanskrit{hīnāyāvattati}}). } It’s when a gentleman has gone forth out of faith from the lay life to homelessness, thinking: ‘I’m swamped by rebirth, old age, and death; by sorrow, lamentation, pain, sadness, and distress. I’m swamped by suffering, mired in suffering. Hopefully I can find an end to this entire mass of suffering.’ When they’ve gone forth, they robe up in the morning and, taking their bowl and robe, enter a village or town for alms without guarding body, speech, and mind, without establishing mindfulness, and without restraining the sense faculties. There they see a householder or their child amusing themselves, supplied and provided with the five kinds of sensual stimulation. They think: ‘Formerly, as laypeople, we amused ourselves, supplied and provided with the five kinds of sensual stimulation. And it’s true that my family is wealthy.\footnote{Here he reverts to the singular in his reflections. Obviously not all of his spiritual companions enjoy the same cushion. } I can both enjoy my wealth and make merit.’ They resign the training and return to a lesser life. This is called one who rejects the training and returns to a lesser life for fear of the danger of whirlpools. ‘Danger of whirlpools’ is a term for the five kinds of sensual stimulation. 

And\marginnote{19.1} what, mendicants, is the danger of sharks?\footnote{The \textit{\textsanskrit{susukā}} is said to be a carnivorous fish (\textit{\textsanskrit{caṇḍamaccha}}). The Sanskrit form \textit{\textsanskrit{śuṣkala}} is said to mean both a species of fish and also “flesh”. The pun appears to be between \textit{\textsanskrit{susukā}} and \textit{\textsanskrit{susukāḷakesā}}, a stock description of attractive young people with “pristine black hair”. The bull shark, which is one of several shark species prowling the Ganges, has black-tipped fins. } It’s when a gentleman has gone forth out of faith from the lay life to homelessness, thinking: ‘I’m swamped by rebirth, old age, and death; by sorrow, lamentation, pain, sadness, and distress. I’m swamped by suffering, mired in suffering. Hopefully I can find an end to this entire mass of suffering.’ When they’ve gone forth, they robe up in the morning and, taking their bowl and robe, enter a village or town for alms without guarding body, speech, and mind, without establishing mindfulness, and without restraining the sense faculties. There they see a female scantily clad, with revealing clothes. Lust infects their mind, so they resign the training and return to a lesser life. This is called one who rejects the training and returns to a lesser life for fear of the danger of sharks. ‘Danger of sharks’ is a term for females. 

These\marginnote{20.1} are the four dangers that a person who gone forth from the lay life to homelessness in this teaching and training should anticipate.” 

That\marginnote{20.2} is what the Buddha said. Satisfied, the mendicants approved what the Buddha said. 

%
\section*{{\suttatitleacronym MN 68}{\suttatitletranslation At Naḷakapāna }{\suttatitleroot Naḷakapānasutta}}
\addcontentsline{toc}{section}{\tocacronym{MN 68} \toctranslation{At Naḷakapāna } \tocroot{Naḷakapānasutta}}
\markboth{At Naḷakapāna }{Naḷakapānasutta}
\extramarks{MN 68}{MN 68}

\scevam{So\marginnote{1.1} I have heard. }At one time the Buddha was staying in the land of the Kosalans near \textsanskrit{Naḷakapāna} in the grove of flame-of-the-forest trees.\footnote{The Buddha is recorded as visiting this grove in two other discourses, which share a common narrative; both deal with the growth of spiritual qualities (\href{https://suttacentral.net/an10.67/en/sujato}{AN 10.67}, \href{https://suttacentral.net/an10.68/en/sujato}{AN 10.68}). } 

Now\marginnote{2.1} at that time several very well-known gentlemen had gone forth out of faith from the lay life to homelessness in the Buddha’s name—\footnote{This makes it sound like they had gone forth recently, which would date this discourse to the early years of the dispensation. } The venerables Anuruddha, Bhaddiya, Kimbila, Bhagu, \textsanskrit{Koṇḍañña}, Revata, Ānanda, and other very well-known gentlemen.\footnote{Pali editions vary a little in the exact monks mentioned. } 

Now\marginnote{3.1} at that time the Buddha was sitting in the open, surrounded by the mendicant \textsanskrit{Saṅgha}. Then the Buddha spoke to the mendicants about those gentlemen: “Mendicants, those gentlemen who have gone forth out of faith from the lay life to homelessness in my name—I trust they’re satisfied with the spiritual life?” When this was said, the mendicants kept silent. 

For\marginnote{3.5} a second and a third time the Buddha asked the same question. For a third time, the mendicants kept silent. 

Then\marginnote{4.1} it occurred to the Buddha, “Why don’t I question just those gentlemen?”\footnote{Reading \textit{te va} as in PTS. Here the Buddha switches from addressing the monks as a whole to addressing just Anuruddha and his friends. } Then the Buddha said to Venerable Anuruddha, “Anuruddha and friends, I hope you’re satisfied with the spiritual life?” 

“Indeed,\marginnote{4.5} sir, we are satisfied with the spiritual life.” 

“Good,\marginnote{5.1} good, Anuruddha and friends! It’s appropriate for gentlemen like yourselves, who have gone forth out of faith from the lay life to homelessness, to be satisfied with the spiritual life. Since you’re blessed with youth, in the prime of life, with pristine black hair, you could have enjoyed sensual pleasures; yet you have gone forth from the lay life to homelessness. But you didn’t go forth to escape a summons by a king or a summons for a bandit, or because you were in debt or in fear, or in order to make a living.\footnote{Also at \href{https://suttacentral.net/sn22.80/en/sujato\#6.4}{SN 22.80:6.4} and \href{https://suttacentral.net/iti91/en/sujato\#2.4}{Iti 91:2.4}. The terms here follow the same sequence as in the Vinaya account of ordination. Generally, ordination should not be given in such cases, but if it is given, those performing it incur an offense of wrong-doing. | Soldiers joined the Sangha to escape military service, so the Buddha said one should not ordain those in service to a king (\href{https://suttacentral.net/pli-tv-kd1/en/sujato\#40.1.1}{Kd 1:40.1.1}). | Several rules regarding ordination of criminals were passed, the thrust of which is that wanted outlaws should not be ordained (\href{https://suttacentral.net/pli-tv-kd1/en/sujato\#41.1.1}{Kd 1:41.1.1}). | Another man ordained to escape debt (\href{https://suttacentral.net/pli-tv-kd1/en/sujato\#46.1.1}{Kd 1:46.1.1}). | “In fear” (\textit{\textsanskrit{bhayaṭṭa}}) is the only item that does not straightforwardly correspond to the Vinaya sequence, where instead the ordination of slaves appears at this point (\href{https://suttacentral.net/pli-tv-kd1/en/sujato\#47.1.1}{Kd 1:47.1.1}). But a connection is suggested by the verses of the water-carrier \textsanskrit{Puṇṇikā}, who speaks of living “in fear” of her masters’ abuse and beatings. | Seventeen boys ordained as novices to get a nice livelihood (\href{https://suttacentral.net/pli-tv-kd1/en/sujato\#49.1.1}{Kd 1:49.1.1}). } Rather, didn’t you go forth thinking: ‘I’m swamped by rebirth, old age, and death; by sorrow, lamentation, pain, sadness, and distress. I’m swamped by suffering, mired in suffering. Hopefully I can find an end to this entire mass of suffering’?” 

“Yes,\marginnote{5.8} sir.” 

“But,\marginnote{6.1} Anuruddha and friends, when a gentleman has gone forth like this, what should he do? Take someone who doesn’t achieve the rapture and bliss that are secluded from sensual pleasures and unskillful qualities, or something even more peaceful than that. Their mind is still occupied by desire, ill will, dullness and drowsiness, restlessness and remorse, doubt, discontent, and sloth.\footnote{This is the first absorption, lacking which the mind is full of hindrances. It appears that here the Buddha is explaining the benefits of absorption to Anuruddha, making this the first of several suttas concerned with Anuruddha’s meditative progress (\href{https://suttacentral.net/mn128/en/sujato}{MN 128}, \href{https://suttacentral.net/an3.130 /en/sujato}{AN 3.130 }, \href{https://suttacentral.net/mn31/en/sujato}{MN 31}). } That’s someone who doesn’t achieve the rapture and bliss that are secluded from sensual pleasures and unskillful qualities, or something even more peaceful than that. 

Take\marginnote{6.4} someone who does achieve the rapture and bliss that are secluded from sensual pleasures and unskillful qualities, or something even more peaceful than that. Their mind is not occupied by desire, ill will, dullness and drowsiness, restlessness and remorse, doubt, discontent, and sloth. That’s someone who does achieve the rapture and bliss that are secluded from sensual pleasures and unskillful qualities, or something even more peaceful than that. 

Is\marginnote{7.1} this what you think of me? ‘The Realized One has not given up the defilements that are corrupting, leading to future lives, hurtful, resulting in suffering and future rebirth, old age, and death. That’s why, after appraisal, he uses some things, endures some things, avoids some things, and dispels some things.’”\footnote{Of the seven methods given at \href{https://suttacentral.net/mn2/en/sujato\#4.1}{MN 2:4.1}, this omits “seeing”, “restraint”, and “developing”. The Buddha is asking whether his use of the four remaining methods gives the impression he is secretly acting from defilements. It seems to me that this question is speaking to a common tendency among young, idealistic practitioners to expect the teacher to completely transcend worldly activities. But then they see the teacher eating food (like someone desiring taste) or avoiding dangerous situations (like someone worried about their body). When they do such things, it comes from a place of attachment, so it is only natural to wonder if the same applies to the teacher as well. Someone who has experienced, at minimum, the first absorption has known what it is like for the mind to be temporarily free of such attachments, hence would not have such thoughts. } 

“No\marginnote{7.4} sir, we don’t think of you that way. We think of you this way: ‘The Realized One has given up the defilements that are corrupting, leading to future lives, hurtful, resulting in suffering and future rebirth, old age, and death. That’s why, after appraisal, he uses some things, endures some things, avoids some things, and dispels some things.’” 

“Good,\marginnote{7.10} good, Anuruddha and friends! The Realized One has given up the defilements that are corrupting, leading to future lives, hurtful, resulting in suffering and future rebirth, old age, and death. He has cut them off at the root, made them like a palm stump, obliterated them so they are unable to arise in the future. Just as a palm tree with its crown cut off is incapable of further growth, in the same way, the Realized One has given up the defilements so they are unable to arise in the future. That’s why, after appraisal, he uses some things, endures some things, avoids some things, and dispels some things. 

What\marginnote{8.1} do you think, Anuruddha and friends? What advantage does the Realized One see in declaring the rebirth of his disciples who have passed away: ‘This one is reborn here, while that one is reborn there’?”\footnote{At \href{https://suttacentral.net/sn44.9/en/sujato}{SN 44.9} the six ascetic teachers are each said to do this, and the Buddha does it at \href{https://suttacentral.net/dn16/en/sujato\#2.5.1}{DN 16:2.5.1} = \href{https://suttacentral.net/dn18/en/sujato\#1.4}{DN 18:1.4}. | This can be seen as a preemptive teaching to Anuruddha, who was to master the clairvoyance that makes such insights possible. The Buddha wants him to know that such abilities should only be used for pure reasons. } 

“Our\marginnote{8.4} teachings are rooted in the Buddha. He is our guide and our refuge. Sir, may the Buddha himself please clarify the meaning of this. The mendicants will listen and remember it.” 

“The\marginnote{9.1} Realized One does not declare such things for the sake of deceiving people or flattering them, nor for the benefit of possessions, honor, or popularity, nor thinking, ‘So let people know about me!’ Rather, there are gentlemen of faith who are full of sublime joy and gladness. When they hear that, they apply their minds to that end. That is for their lasting welfare and happiness. 

Take\marginnote{10.1} a monk who hears this: ‘The monk named so-and-so has passed away. The Buddha has declared that, he was enlightened.’ And he’s either seen for himself, or heard from someone else, that that venerable had such ethics, such qualities, such wisdom, such meditation, or such freedom. Recollecting that monk’s faith, ethics, learning, generosity, and wisdom, he applies his mind to that end. That’s how a monk lives at ease. 

Take\marginnote{11.1} a monk who hears this: ‘The monk named so-and-so has passed away. The Buddha has declared that, with the ending of the five lower fetters, he’s been reborn spontaneously and will become extinguished there, not liable to return from that world.’ And he’s either seen for himself, or heard from someone else, that that venerable had such ethics, such qualities, such wisdom, such meditation, or such freedom. Recollecting that monk’s faith, ethics, learning, generosity, and wisdom, he applies his mind to that end. That too is how a monk lives at ease. 

Take\marginnote{12.1} a monk who hears this: ‘The monk named so-and-so has passed away. The Buddha has declared that, with the ending of three fetters, and the weakening of greed, hate, and delusion, he’s a once-returner. He’ll come back to this world once only, then make an end of suffering.’ And he’s either seen for himself, or heard from someone else, that that venerable had such ethics, such qualities, such wisdom, such meditation, or such freedom. Recollecting that monk’s faith, ethics, learning, generosity, and wisdom, he applies his mind to that end. That too is how a monk lives at ease. 

Take\marginnote{13.1} a monk who hears this: ‘The monk named so-and-so has passed away. The Buddha has declared that, with the ending of three fetters he’s a stream-enterer, not liable to be reborn in the underworld, bound for awakening.’ And he’s either seen for himself, or heard from someone else, that that venerable had such ethics, such qualities, such wisdom, such meditation, or such freedom. Recollecting that monk’s faith, ethics, learning, generosity, and wisdom, he applies his mind to that end. That too is how a monk lives at ease. 

Take\marginnote{14.1} a nun who hears this: ‘The nun named so-and-so has passed away. The Buddha has declared that, she was enlightened.’ And she’s either seen for herself, or heard from someone else, that that sister had such ethics, such qualities, such wisdom, such meditation, or such freedom. Recollecting that nun’s faith, ethics, learning, generosity, and wisdom, she applies her mind to that end. That’s how a nun lives at ease. 

Take\marginnote{15.1} a nun who hears this: ‘The nun named so-and-so has passed away. The Buddha has declared that, with the ending of the five lower fetters, she’s been reborn spontaneously and will become extinguished there, not liable to return from that world.’ And she’s either seen for herself, or heard from someone else, that that sister had such ethics, such qualities, such wisdom, such meditation, or such freedom. Recollecting that nun’s faith, ethics, learning, generosity, and wisdom, she applies her mind to that end. That too is how a nun lives at ease. 

Take\marginnote{16.1} a nun who hears this: ‘The nun named so-and-so has passed away. The Buddha has declared that, with the ending of three fetters, and the weakening of greed, hate, and delusion, she’s a once-returner. She’ll come back to this world once only, then make an end of suffering.’ And she’s either seen for herself, or heard from someone else, that that sister had such ethics, such qualities, such wisdom, such meditation, or such freedom. Recollecting that nun’s faith, ethics, learning, generosity, and wisdom, she applies her mind to that end. That too is how a nun lives at ease. 

Take\marginnote{17.1} a nun who hears this: ‘The nun named so-and-so has passed away. The Buddha has declared that, with the ending of three fetters she’s a stream-enterer, not liable to be reborn in the underworld, bound for awakening.’ And she’s either seen for herself, or heard from someone else, that that sister had such ethics, such qualities, such wisdom, such meditation, or such freedom. Recollecting that nun’s faith, ethics, learning, generosity, and wisdom, she applies her mind to that end. That too is how a nun lives at ease. 

Take\marginnote{18.1} a layman who hears this: ‘The layman named so-and-so has passed away. The Buddha has declared that, with the ending of the five lower fetters, he’s been reborn spontaneously and will become extinguished there, not liable to return from that world.’ And he’s either seen for himself, or heard from someone else, that that venerable had such ethics, such qualities, such wisdom, such meditation, or such freedom. Recollecting that layman’s faith, ethics, learning, generosity, and wisdom, he applies his mind to that end. That’s how a layman lives at ease. 

Take\marginnote{19.1} a layman who hears this: ‘The layman named so-and-so has passed away. The Buddha has declared that, with the ending of three fetters, and the weakening of greed, hate, and delusion, he’s a once-returner. He’ll come back to this world once only, then make an end of suffering.’ And he’s either seen for himself, or heard from someone else, that that venerable had such ethics, such qualities, such wisdom, such meditation, or such freedom. Recollecting that layman’s faith, ethics, learning, generosity, and wisdom, he applies his mind to that end. That too is how a layman lives at ease. 

Take\marginnote{20.1} a layman who hears this: ‘The layman named so-and-so has passed away. The Buddha has declared that, with the ending of three fetters he’s a stream-enterer, not liable to be reborn in the underworld, bound for awakening.’ And he’s either seen for himself, or heard from someone else, that that venerable had such ethics, such qualities, such wisdom, such meditation, or such freedom. Recollecting that layman’s faith, ethics, learning, generosity, and wisdom, he applies his mind to that end. That too is how a layman lives at ease. 

Take\marginnote{21.1} a laywoman who hears this: ‘The laywoman named so-and-so has passed away. The Buddha has declared that, with the ending of the five lower fetters, she’s been reborn spontaneously and will become extinguished there, not liable to return from that world.’ And she’s either seen for herself, or heard from someone else, that that sister had such ethics, such qualities, such wisdom, such meditation, or such freedom. Recollecting that laywoman’s faith, ethics, learning, generosity, and wisdom, she applies her mind to that end. That’s how a laywoman lives at ease. 

Take\marginnote{22.1} a laywoman who hears this: ‘The laywoman named so-and-so has passed away. The Buddha has declared that, with the ending of three fetters, and the weakening of greed, hate, and delusion, she’s a once-returner. She’ll come back to this world once only, then make an end of suffering.’ And she’s either seen for herself, or heard from someone else, that that sister had such ethics, such qualities, such wisdom, such meditation, or such freedom. Recollecting that laywoman’s faith, ethics, learning, generosity, and wisdom, she applies her mind to that end. That too is how a laywoman lives at ease. 

Take\marginnote{23.1} a laywoman who hears this: ‘The laywoman named so-and-so has passed away. The Buddha has declared that, with the ending of three fetters she’s a stream-enterer, not liable to be reborn in the underworld, bound for awakening.’ And she’s either seen for herself, or heard from someone else, that that sister had such ethics, such qualities, such wisdom, such meditation, or such freedom. Recollecting that laywoman’s faith, ethics, learning, generosity, and wisdom, she applies her mind to that end. That too is how a laywoman lives at ease. 

So\marginnote{24.1} it’s not for the sake of deceiving people or flattering them, nor for the benefit of possessions, honor, or popularity, nor thinking, ‘So let people know about me!’ that the Realized One declares the rebirth of his disciples who have passed away: ‘This one is reborn here, while that one is reborn there.’ Rather, there are gentlemen of faith who are full of joy and gladness. When they hear that, they apply their minds to that end. That is for their lasting welfare and happiness.” 

That\marginnote{24.6} is what the Buddha said. Satisfied, Venerable Anuruddha approved what the Buddha said. 

%
\section*{{\suttatitleacronym MN 69}{\suttatitletranslation With Gulissāni }{\suttatitleroot Goliyānisutta}}
\addcontentsline{toc}{section}{\tocacronym{MN 69} \toctranslation{With Gulissāni } \tocroot{Goliyānisutta}}
\markboth{With Gulissāni }{Goliyānisutta}
\extramarks{MN 69}{MN 69}

\scevam{So\marginnote{1.1} I have heard. }At one time the Buddha was staying near \textsanskrit{Rājagaha}, in the Bamboo Grove, the squirrels’ feeding ground. 

Now\marginnote{2.1} at that time a wilderness mendicant of boorish behavior named \textsanskrit{Gulissāni} had come down to the midst of the \textsanskrit{Saṅgha} on some business.\footnote{\textsanskrit{Gulissāni} (variant \textsanskrit{Goliyāni}) is only known here. | The term \textit{\textsanskrit{padasamācaro}} (\textsanskrit{Mahāsaṅgīti} editions) or \textit{\textsanskrit{padarasamācāro}} (Buddha Jayanthi, PTS, and Thai editions) is unique and uncertain. The commentary does not gloss it linguistically, but explains, “of weak conduct, of coarse conduct”. The reading \textit{padara} is preferred in Cone’s \emph{Dictionary of Pali} in the sense “cleft”. But I propose we read with Sanskrit \textit{padra}, “village”. This explains the confusion in Pali readings, as the Sanskrit is midway between \textit{pada} and \textit{padara}. The point is that he was supposed to be a forest monk, yet he behaved “boorishly” as if he were from the village. } There Venerable \textsanskrit{Sāriputta} spoke to the mendicants about \textsanskrit{Gulissāni}: 

“Reverends,\marginnote{3.1} a wilderness monk who has come to stay in the \textsanskrit{Saṅgha} should have respect and reverence for his spiritual companions. If he doesn’t, there’ll be some who say: ‘What’s the point of this wilderness venerable’s staying alone and autonomous in the wilderness, since he has no respect and reverence for his spiritual companions?’\footnote{“Autonomous” is \textit{seri}, literally “self-moving”, i.e. living at one’s own pleasure like a deer (\href{https://suttacentral.net/thag19.1/en/sujato\#54.1}{Thag 19.1:54.1}; see \href{https://suttacentral.net/mn75/en/sujato\#13.4}{MN 75:13.4}). } That’s why a wilderness monk who has come to stay in the \textsanskrit{Saṅgha} should have respect and reverence for his spiritual companions. 

A\marginnote{4.1} wilderness monk who has come to stay in the \textsanskrit{Saṅgha} should be careful where he sits, thinking:\footnote{“Careful” is \textit{kusala}, literally “skillful”. } ‘I shall sit so that I don’t intrude on the senior monks and I don’t block the junior monks from a seat.’ If he doesn’t, there’ll be some who say: ‘What’s the point of this wilderness venerable’s staying alone and autonomous in the wilderness, since he’s not careful where he sits?’\footnote{The PTS reading \textit{\textsanskrit{ābhisamācārika}} here is spurious. } That’s why a wilderness monk who has come to stay in the \textsanskrit{Saṅgha} should be careful where he sits. 

A\marginnote{4.7} wilderness monk who has come to stay in the \textsanskrit{Saṅgha} should know even the supplementary regulations.\footnote{“Supplementary regulations” (\textit{\textsanskrit{ābhisamācārika}}) is a Vinaya term for minor rules of training as distinguished from those that are fundamental to the spiritual life (\href{https://suttacentral.net/pli-tv-kd1/en/sujato\#36.12.2}{Kd 1:36.12.2}). In the \textsanskrit{Mahāsanghika} \textsanskrit{Lokuttaravāda} school, these became collected as a treatise named \textsanskrit{Abhisamācārikadharma}. } If he doesn’t, there’ll be some who say: ‘What’s the point of this wilderness venerable’s staying alone and autonomous in the wilderness, since he doesn’t even know the supplementary regulations?’ That’s why a wilderness monk who has come to stay in the \textsanskrit{Saṅgha} should know even the supplementary regulations. 

A\marginnote{5.1} wilderness monk who has come to stay in the \textsanskrit{Saṅgha} shouldn’t enter the village too early or return too late in the day.\footnote{The perils of wandering the village at night are illustrated at \href{https://suttacentral.net/mn66/en/sujato\#6.20}{MN 66:6.20}; see also \href{https://suttacentral.net/dn31/en/sujato\#9.1}{DN 31:9.1}. } If he does so, there’ll be some who say: ‘What’s the point of this wilderness venerable’s staying alone and autonomous in the wilderness, since he enters the village too early or returns too late in the day?’ That’s why a wilderness monk who has come to stay in the \textsanskrit{Saṅgha} shouldn’t enter the village too early or return too late in the day. 

A\marginnote{6.1} wilderness monk who has come to stay in the \textsanskrit{Saṅgha} shouldn’t visit families before or after the meal.\footnote{In the Vinaya rule \href{https://suttacentral.net/pli-tv-bu-vb-pc46/en/sujato}{Bu Pc 46}, Upananda the Sakyan was in the habit of visiting several families before and after the meal, upsetting the donors and monastics at the scheduled offering. The rule is not meant to be absolute, as the Vinaya grants many exceptions. } If he does so, there’ll be some who say: ‘This wilderness venerable, staying alone and autonomous in the wilderness, must be used to wandering about at the wrong time, since he behaves like this when he’s come to the \textsanskrit{Saṅgha}.’ That’s why a wilderness monk who has come to stay in the \textsanskrit{Saṅgha} shouldn’t visit families before or after the meal. 

A\marginnote{7.1} wilderness monk who has come to stay in the \textsanskrit{Saṅgha} shouldn’t be restless and fickle. If he is, there’ll be some who say: ‘This wilderness venerable, staying alone and autonomous in the wilderness, must be used to being restless and fickle, since he behaves like this when he’s come to the \textsanskrit{Saṅgha}.’ That’s why a wilderness monk who has come to stay in the \textsanskrit{Saṅgha} shouldn’t be restless and fickle. 

A\marginnote{8.1} wilderness monk who has come to stay in the \textsanskrit{Saṅgha} shouldn’t be scurrilous and loose-tongued. If he is, there’ll be some who say: ‘What’s the point of this wilderness venerable’s staying alone and autonomous in the wilderness, since he’s scurrilous and loose-tongued?’ That’s why a wilderness monk who has come to stay in the \textsanskrit{Saṅgha} shouldn’t be scurrilous and loose-tongued. 

A\marginnote{9.1} wilderness monk who has come to stay in the \textsanskrit{Saṅgha} should be easy to admonish, with good friends. If he’s hard to admonish, with bad friends, there’ll be some who say: ‘What’s the point of this wilderness venerable’s staying alone and autonomous in the wilderness, since he’s hard to admonish, with bad friends?’ That’s why a wilderness monk who has come to stay in the \textsanskrit{Saṅgha} should be easy to admonish, with good friends. 

A\marginnote{10.1} wilderness monk should guard the sense doors. If he doesn’t, there’ll be some who say: ‘What’s the point of this wilderness venerable’s staying alone and autonomous in the wilderness, since he doesn’t guard the sense doors?’ That’s why a wilderness monk should guard the sense doors. 

A\marginnote{11.1} wilderness monk should eat in moderation. If he doesn’t, there’ll be some who say: ‘What’s the point of this wilderness venerable’s staying alone and autonomous in the wilderness, since he eats too much?’ That’s why a wilderness monk should eat in moderation. 

A\marginnote{12.1} wilderness monk should be committed to wakefulness. If he isn’t, there’ll be some who say: ‘What’s the point of this wilderness venerable’s staying alone and autonomous in the wilderness, since he’s not committed to wakefulness?’ That’s why a wilderness monk should be committed to wakefulness. 

A\marginnote{13.1} wilderness monk should be energetic. If he isn’t, there’ll be some who say: ‘What’s the point of this wilderness venerable’s staying alone and autonomous in the wilderness, since he’s not energetic?’ That’s why a wilderness monk should be energetic. 

A\marginnote{14.1} wilderness monk should be mindful. If he isn’t, there’ll be some who say: ‘What’s the point of this wilderness venerable’s staying alone and autonomous in the wilderness, since he’s not mindful?’ That’s why a wilderness monk should be mindful. 

A\marginnote{15.1} wilderness monk should have immersion. If he doesn’t, there’ll be some who say: ‘What’s the point of this wilderness venerable’s staying alone and autonomous in the wilderness, since he doesn’t have immersion?’ That’s why a wilderness monk should have immersion. 

A\marginnote{16.1} wilderness monk should be wise. If he isn’t, there’ll be some who say: ‘What’s the point of this wilderness venerable’s staying alone and autonomous in the wilderness, since he’s not wise?’ That’s why a wilderness monk should be wise. 

A\marginnote{17.1} wilderness monk should make an effort in regards to the teaching and training.\footnote{In the suttas, \textit{abhivinaya} means “about the training” just as \textit{abhidhamma} means “about the teaching” (eg. \href{https://suttacentral.net/an3.140/en/sujato\#4.4}{AN 3.140:4.4}). Over time, such discussions “about” the fundamentals came to be thought of as a separate body of analytical texts called \textit{abhidhamma} and \textit{abhivinaya}. Thus in the late canonical \textsanskrit{Parivāra}, compiled in Sri Lanka, the analysis (\textit{vibhatti}) of a Vinaya rule is called \textit{abhivinaya} (\href{https://suttacentral.net/pli-tv-pvr1.1/en/sujato\#3.33}{Pvr 1.1:3.33}). } There are those who will question a wilderness monk about the teaching and training. If he is stumped, there’ll be some who say: ‘What’s the point of this wilderness venerable’s staying alone and autonomous in the wilderness, since he is stumped by a question about the teaching and training?’ That’s why a wilderness monk should make an effort to learn the teaching and training. 

A\marginnote{18.1} wilderness monk should practice meditation to realize the peaceful liberations that are formless, transcending form. There are those who will question a wilderness monk about the formless liberations. If he is stumped, there’ll be some who say: ‘What’s the point of this wilderness venerable’s staying alone and autonomous in the wilderness, since he is stumped by a question about the formless liberations?’ That’s why a wilderness monk should practice meditation to realize the peaceful liberations that are formless, transcending form. 

A\marginnote{19.1} wilderness monk should practice meditation to realize the superhuman state. There are those who will question a wilderness monk about the superhuman state. If he is stumped, there’ll be some who say: ‘What’s the point of this wilderness venerable’s staying alone and autonomous in the wilderness, since he doesn’t know the goal for which he went forth?’\footnote{Normally the “superhuman states” are in plural as they include absorptions, various advanced meditative abilities, and the stages of realization. Unusually, the form is singular here, denoting the “goal for which one goes forth”, namely arahantship. This point is confirmed in the Chinese parallel, which speaks of the ending of defilements (MA 26 at T i 456a3). } That’s why a wilderness monk should practice meditation to realize the superhuman state.” 

When\marginnote{19.7} Venerable \textsanskrit{Sāriputta} said this, Venerable \textsanskrit{Mahāmoggallāna} said to him, “Reverend \textsanskrit{Sāriputta}, should these things be undertaken and followed only by wilderness monks, or by those who live within a village as well?” 

“Reverend\marginnote{19.9} \textsanskrit{Moggallāna}, these things should be undertaken and followed by wilderness monks, and still more by those who live within a village.”\footnote{Those who live in a village are closer to temptations and need to be even more on their guard. } 

%
\section*{{\suttatitleacronym MN 70}{\suttatitletranslation At Kīṭāgiri }{\suttatitleroot Kīṭāgirisutta}}
\addcontentsline{toc}{section}{\tocacronym{MN 70} \toctranslation{At Kīṭāgiri } \tocroot{Kīṭāgirisutta}}
\markboth{At Kīṭāgiri }{Kīṭāgirisutta}
\extramarks{MN 70}{MN 70}

\scevam{So\marginnote{1.1} I have heard. }At one time the Buddha was wandering in the land of the \textsanskrit{Kāsis} together with a large \textsanskrit{Saṅgha} of mendicants. There the Buddha addressed the mendicants: 

“Mendicants,\marginnote{2.1} I abstain from eating at night.\footnote{While I have not been able to trace a Jain rule regarding eating in the afternoon, not eating at night (\textit{\textsanskrit{rattibhojanā}}) was a standard practice of Jain ascetics (\textsanskrit{Uttarādhyayana} 19.30, 13.2;  \textsanskrit{Dasaveyāliya} 4.6, 6.26; \textsanskrit{Sūyagaḍa} 1.2.3.3). It was adopted early into the Gradual Training (\href{https://suttacentral.net/mn27/en/sujato\#13.9}{MN 27:13.9}). The failure of mendicants to comply led to the laying down of a formal Vinaya rule (\href{https://suttacentral.net/pli-tv-bu-vb-pc37/en/sujato}{Bu Pc 37}). But some came to regret their objections, recognizing that the Buddha had acted for their welfare (\href{https://suttacentral.net/mn66/en/sujato\#6.4}{MN 66:6.4}). | The related practice of eating in one sitting is, by contrast, not required in the Vinaya, but was encouraged (\href{https://suttacentral.net/mn21/en/sujato\#7.4}{MN 21:7.4}, \href{https://suttacentral.net/mn65/en/sujato\#2.1}{MN 65:2.1}). } Doing so, I find that I’m healthy and well, nimble, strong, and living comfortably. You too should abstain from eating at night. Doing so, you’ll find that you’re healthy and well, nimble, strong, and living comfortably.” 

“Yes,\marginnote{2.5} sir,” they replied. 

Then\marginnote{3.1} the Buddha, traveling stage by stage in the land of the \textsanskrit{Kāsis}, arrived at a town of the \textsanskrit{Kāsis} named \textsanskrit{Kīṭāgiri}, and stayed there. 

Now\marginnote{4.1} at that time the mendicants who followed Assaji and Punabbasuka were residing at \textsanskrit{Kīṭāgiri}.\footnote{Assaji and Punabbasuka were a pair of shameless monks known for corrupting lay folk with their superficial charm (\href{https://suttacentral.net/pli-tv-bu-vb-ss13/en/sujato\#1.1.2}{Bu Ss 13:1.1.2}, \href{https://suttacentral.net/pli-tv-kd11/en/sujato\#13.1.1}{Kd 11:13.1.1}) and for misdirecting Sangha lodgings (\href{https://suttacentral.net/pli-tv-kd16/en/sujato\#16.1.1}{Kd 16:16.1.1}), while leading others down the same path. | Assaji is not to be confused with the good monk who was one of the first five mendicants (\href{https://suttacentral.net/mn35/en/sujato\#3.1}{MN 35:3.1}). } Then several mendicants went up to them and said, “Reverends, the Buddha abstains from eating at night, and so does the mendicant \textsanskrit{Saṅgha}. Doing so, they find that they’re healthy and well, nimble, strong, and living comfortably. You too should abstain from eating at night. Doing so, you’ll find that you’re healthy and well, nimble, strong, and living comfortably.” 

When\marginnote{4.7} they said this, the mendicants who followed Assaji and Punabbasuka said to them, “Reverends, we eat in the evening, the morning, and at the wrong time of day. Doing so, we find that we’re healthy and well, nimble, strong, and living comfortably. Why should we give up what is apparent in the present to chase after what takes effect over time?\footnote{They are twisting the meaning of well-known properties of the Dhamma, which is “apparent in the present” and “immediately effective” (\textit{\textsanskrit{sandiṭṭhiko} \textsanskrit{akāliko}}, \href{https://suttacentral.net/mn7/en/sujato\#6.2}{MN 7:6.2}). In this they are following the example of \textsanskrit{Māra}, who makes the exact same argument (\href{https://suttacentral.net/sn1.20/en/sujato\#4.4}{SN 1.20:4.4}, \href{https://suttacentral.net/sn4.21/en/sujato\#1.7}{SN 4.21:1.7}). The fallacy is that it is sensual pleasures that take time—for they bind you to rebirth—whereas the Dhamma may be realized in this life. } We shall eat in the evening, the morning, and at the wrong time of day.” 

Since\marginnote{5.1} those mendicants were unable to persuade the mendicants who were followers of Assaji and Punabbasuka, they approached the Buddha, bowed, sat down to one side, and told him what had happened. 

So\marginnote{6.1} the Buddha addressed one of the monks, “Please, monk, in my name tell the mendicants who follow Assaji and Punabbasuka that the teacher summons them.” 

“Yes,\marginnote{6.4} sir,” that monk replied. He went to those mendicants and said, “Venerables, the teacher summons you.” 

“Yes,\marginnote{6.6} reverend,” those mendicants replied. They went to the Buddha, bowed, and sat down to one side. 

The\marginnote{6.7} Buddha said to them, “Is it really true, mendicants, that several mendicants went to you and said: ‘Reverends, the Buddha abstains from eating at night, and so does the mendicant \textsanskrit{Saṅgha}. Doing so, they find that they’re healthy and well, nimble, strong, and living comfortably. You too should abstain from eating at night. Doing so, you’ll find that you’re healthy and well, nimble, strong, and living comfortably.’ When they said this, did you really say to them: ‘Reverends, we eat in the evening, the morning, and at the wrong time of day. Doing so, we find that we’re healthy and well, nimble, strong, and living comfortably. Why should we give up what is apparent in the present to chase after what takes effect over time? We shall eat in the evening, the morning, and at the wrong time of day.’” 

“Yes,\marginnote{6.17} sir.” 

“Mendicants,\marginnote{6.18} have you ever known me to teach the Dhamma like this: no matter what this individual experiences—pleasurable, painful, or neutral—their unskillful qualities decline and their skillful qualities grow?”\footnote{The Buddha, even in the face of such intransigence, begins by establishing common ground and giving encouragement. } 

“No,\marginnote{6.19} sir.” 

“Haven’t\marginnote{7.1} you known me to teach the Dhamma like this: ‘When someone feels this kind of pleasant feeling, unskillful qualities grow and skillful qualities decline. But when someone feels that kind of pleasant feeling, unskillful qualities decline and skillful qualities grow. When someone feels this kind of painful feeling, unskillful qualities grow and skillful qualities decline. But when someone feels that kind of painful feeling, unskillful qualities decline and skillful qualities grow. When someone feels this kind of neutral feeling, unskillful qualities grow and skillful qualities decline. But when someone feels that kind of neutral feeling, unskillful qualities decline and skillful qualities grow’?”\footnote{See \href{https://suttacentral.net/mn137/en/sujato\#9.3}{MN 137:9.3}. } 

“Yes,\marginnote{7.2} sir.” 

“Good,\marginnote{8.1} mendicants! Now, suppose I hadn’t known, seen, understood, realized, and experienced this with wisdom:\footnote{Having found an initial common ground, the Buddha takes time to establish that his teaching is grounded on experience. } ‘When someone feels this kind of pleasant feeling, unskillful qualities grow and skillful qualities decline.’ Not knowing this, would it be appropriate for me to say: ‘You should give up this kind of pleasant feeling’?” 

“No,\marginnote{8.5} sir.” 

“But\marginnote{8.6} I have known, seen, understood, realized, and experienced this with wisdom: ‘When someone feels this kind of pleasant feeling, unskillful qualities grow and skillful qualities decline.’ Since this is so, that’s why I say: ‘You should give up this kind of pleasant feeling.’ Now, suppose I hadn’t known, seen, understood, realized, and experienced this with wisdom: ‘When someone feels that kind of pleasant feeling, unskillful qualities decline and skillful qualities grow.’ Not knowing this, would it be appropriate for me to say: ‘You should enter and remain in that kind of pleasant feeling’?” 

“No,\marginnote{8.11} sir.” 

“But\marginnote{8.12} I have known, seen, understood, realized, and experienced this with wisdom: ‘When someone feels that kind of pleasant feeling, unskillful qualities decline and skillful qualities grow.’ Since this is so, that’s why I say: ‘You should enter and remain in that kind of pleasant feeling.’ 

Now,\marginnote{8.14} suppose I hadn’t known, seen, understood, realized, and experienced this with wisdom: ‘When someone feels this kind of painful feeling, unskillful qualities grow and skillful qualities decline.’ Not knowing this, would it be appropriate for me to say: ‘You should give up this kind of painful feeling’?” 

“No,\marginnote{8.17} sir.” 

“But\marginnote{9.1} I have known, seen, understood, realized, and experienced this with wisdom: ‘When someone feels this kind of painful feeling, unskillful qualities grow and skillful qualities decline.’ Since this is so, that’s why I say: ‘You should give up this kind of painful feeling.’ Now, suppose I hadn’t known, seen, understood, realized, and experienced this with wisdom: ‘When someone feels that kind of painful feeling, unskillful qualities decline and skillful qualities grow.’ Not knowing this, would it be appropriate for me to say: ‘You should enter and remain in that kind of painful feeling’?” 

“No,\marginnote{9.6} sir.” 

“But\marginnote{9.7} I have known, seen, understood, realized, and experienced this with wisdom: ‘When someone feels that kind of painful feeling, unskillful qualities decline and skillful qualities grow.’ Since this is so, that’s why I say: ‘You should enter and remain in that kind of painful feeling.’ 

Now,\marginnote{10.1} suppose I hadn’t known, seen, understood, realized, and experienced this with wisdom: ‘When someone feels this kind of neutral feeling, unskillful qualities grow and skillful qualities decline.’ Not knowing this, would it be appropriate for me to say: ‘You should give up this kind of neutral feeling’?” 

“No,\marginnote{10.4} sir.” 

“But\marginnote{10.5} I have known, seen, understood, realized, and experienced this with wisdom: ‘When someone feels this kind of neutral feeling, unskillful qualities grow and skillful qualities decline.’ Since this is so, that’s why I say: ‘You should give up this kind of neutral feeling.’ Now, suppose I hadn’t known, seen, understood, realized, and experienced this with wisdom: ‘When someone feels that kind of neutral feeling, unskillful qualities decline and skillful qualities grow.’ Not knowing this, would it be appropriate for me to say: ‘You should enter and remain in that kind of neutral feeling’?” 

“No,\marginnote{10.10} sir.” 

“But\marginnote{11.1} I have known, seen, understood, realized, and experienced this with wisdom: ‘When someone feels that kind of neutral feeling, unskillful qualities decline and skillful qualities grow.’ Since this is so, that’s why I say: ‘You should enter and remain in that kind of neutral feeling.’ 

Mendicants,\marginnote{12.1} I don’t say that all these mendicants still have work to do with diligence.\footnote{Now the Buddha shows that it is not just him, but others also, who have benefited from sincere practice. } Nor do I say that all these mendicants have no work to do with diligence. I say that mendicants don’t have work to do with diligence if they are perfected, with defilements ended, having completed the spiritual journey, done what had to be done, laid down the burden, achieved their own goal, utterly ended the fetter of continued existence, and become rightly freed through enlightenment. Why is that? They’ve done their work with diligence. They’re incapable of being negligent. 

I\marginnote{13.1} say that mendicants still have work to do with diligence if they are trainees, who haven’t achieved their heart’s desire, but live aspiring to the supreme sanctuary from the yoke. Why is that? Thinking: ‘Hopefully this venerable will frequent appropriate lodgings, associate with good friends, and control their faculties. Then they might realize the supreme culmination of the spiritual path in this very life, and live having achieved with their own insight the goal for which gentlemen rightly go forth from the lay life to homelessness.’ Seeing this fruit of diligence for those mendicants, I say that they still have work to do with diligence. 

Mendicants,\marginnote{14.1} these seven people are found in the world. What seven? One freed both ways, one freed by wisdom, a direct witness, one attained to view, one freed by faith, a follower of teachings, and a follower by faith. 

And\marginnote{15.1} what person is freed both ways? It’s a person who has direct meditative experience of the peaceful liberations that are formless, transcending form. And, having seen with wisdom, their defilements have come to an end.\footnote{The first two of the seven are both fully perfected arahants. The arahant “freed both ways” is defined by their mastery of formless meditations, which they understand with wisdom (\href{https://suttacentral.net/dn15/en/sujato\#36.3}{DN 15:36.3}, \href{https://suttacentral.net/an9.45/en/sujato}{AN 9.45}). | “Direct meditative experience” is an oblique rendering of \textit{\textsanskrit{kāyena} \textsanskrit{phusitvā}}, since in the “formless” meditations, \textit{\textsanskrit{kāya}} cannot mean “body”. Like other meditative terms, the sense of the word gradually grows more subtle as meditation deepens. In preliminary passages it simply means “the body” as contemplated in meditation. As meditation deepens it takes more of a sense of experience as it happens in the body. Finally, as physical perception fades, \textit{\textsanskrit{kāya}} loses any sense of materiality and simply means direct personal experience of meditative states or even of Nibbana. } This person is called freed both ways. And I say that this mendicant has no work to do with diligence. Why is that? They’ve done their work with diligence. They’re incapable of being negligent. 

And\marginnote{16.1} what person is freed by wisdom? It’s a person who does not have direct meditative experience of the peaceful liberations that are formless, transcending form. Nevertheless, having seen with wisdom, their defilements have come to an end.\footnote{Of course these arahants have practiced absorption, which is an essential part of the eightfold path. But because of their strong insight, they have not needed to further develop the ultimate refinement of the formless attainments. } This person is called freed by wisdom. I say that this mendicant has no work to do with diligence. Why is that? They’ve done their work with diligence. They’re incapable of being negligent. 

And\marginnote{17.1} what person is a direct witness?\footnote{The next three may be once-returners, non-returners, or on the path to perfection (\href{https://suttacentral.net/an3.21/en/sujato\#9.3}{AN 3.21:9.3}). | A “direct witness” (\textit{\textsanskrit{kāyasakkhi}}) is distinguished by their mastery of immersion including formless attainments (\href{https://suttacentral.net/an9.43/en/sujato}{AN 9.43}). } It’s a person who has direct meditative experience of the peaceful liberations that are formless, transcending form. And, having seen with wisdom, some of their defilements have come to an end. This person is called a direct witness. I say that this mendicant still has work to do with diligence. Why is that? Thinking: ‘Hopefully this venerable will frequent appropriate lodgings, associate with good friends, and control their faculties. Then they might realize the supreme culmination of the spiritual path in this very life, and live having achieved with their own insight the goal for which gentlemen rightly go forth from the lay life to homelessness.’ Seeing this fruit of diligence for this mendicant, I say that they still have work to do with diligence. 

And\marginnote{18.1} what person is attained to view? It’s a person who doesn’t have direct meditative experience of the peaceful liberations that are formless, transcending form. Nevertheless, having seen with wisdom, some of their defilements have come to an end. And they have clearly seen and clearly contemplated with wisdom the teaching and training proclaimed by the Realized One. This person is called attained to view. I say that this mendicant also still has work to do with diligence. Why is that? Thinking: ‘Hopefully this venerable will frequent appropriate lodgings, associate with good friends, and control their faculties. Then they might realize the supreme culmination of the spiritual path in this very life, and live having achieved with their own insight the goal for which gentlemen rightly go forth from the lay life to homelessness.’ Seeing this fruit of diligence for this mendicant, I say that they still have work to do with diligence. 

And\marginnote{19.1} what person is freed by faith? It’s a person who doesn’t have direct meditative experience of the peaceful liberations that are formless, transcending form. Nevertheless, having seen with wisdom, some of their defilements have come to an end. And their faith is settled, rooted, and planted in the Realized One. This person is called freed by faith. I say that this mendicant also still has work to do with diligence. Why is that? Thinking: ‘Hopefully this venerable will frequent appropriate lodgings, associate with good friends, and control their faculties. Then they might realize the supreme culmination of the spiritual path in this very life, and live having achieved with their own insight the goal for which gentlemen rightly go forth from the lay life to homelessness.’ Seeing this fruit of diligence for this mendicant, I say that they still have work to do with diligence. 

And\marginnote{20.1} what person is a follower of teachings? It’s a person who doesn’t have direct meditative experience of the peaceful liberations that are formless, transcending form. And having seen with wisdom, their defilements have not come to an end. And they accept the teachings proclaimed by the Realized One after deliberating them with a degree of wisdom. And they have the following qualities:\footnote{For “and having seen with wisdom, their defilements have not come to an end”, accept the reading of BJT and PTS,  (\textit{\textsanskrit{paññāya} cassa \textsanskrit{disvā} \textsanskrit{āsavā} \textsanskrit{aparikkhīṇā} honti}). The \textsanskrit{Mahāsaṅgīti} says that “certain” defilements have ended, but it is stream-entry that marks the initial breaking of fetters, while the follower of teachings (and the follower by faith below) are still on the path to stream-entry. } the faculties of faith, energy, mindfulness, immersion, and wisdom. This person is called a follower of teachings. I say that this mendicant also still has work to do with diligence. Why is that? Thinking: ‘Hopefully this venerable will frequent appropriate lodgings, associate with good friends, and control their faculties. Then they might realize the supreme culmination of the spiritual path in this very life, and live having achieved with their own insight the goal for which gentlemen rightly go forth from the lay life to homelessness.’ Seeing this fruit of diligence for this mendicant, I say that they still have work to do with diligence. 

And\marginnote{21.1} what person is a follower by faith? It’s a person who doesn’t have direct meditative experience of the peaceful liberations that are formless, transcending form. And having seen with wisdom, their defilements have not come to an end. And they have a degree of faith and love for the Realized One. And they have the following qualities: the faculties of faith, energy, mindfulness, immersion, and wisdom.\footnote{Even the least of the noble ones possesses the five spiritual faculties, which include absorption. } This person is called a follower by faith. I say that this mendicant also still has work to do with diligence. Why is that? Thinking: ‘Hopefully this venerable will frequent appropriate lodgings, associate with good friends, and control their faculties. Then they might realize the supreme culmination of the spiritual path in this very life, and live having achieved with their own insight the goal for which gentlemen rightly go forth from the lay life to homelessness.’ Seeing this fruit of diligence for this mendicant, I say that they still have work to do with diligence. 

Mendicants,\marginnote{22.1} I don’t say that enlightenment is achieved right away. Rather, enlightenment is achieved by gradual training, progress, and practice. 

And\marginnote{23.1} how is enlightenment achieved by gradual training, progress, and practice? It’s when someone in whom faith has arisen approaches a teacher. They pay homage, actively listen, hear the teachings, remember the teachings, reflect on their meaning, and accept them after deliberation. Then enthusiasm springs up; they make an effort, weigh up, and persevere. Persevering, they directly realize the ultimate truth, and see it with penetrating wisdom.\footnote{Parallel passages at \href{https://suttacentral.net/an8.82/en/sujato\#2.1}{AN 8.82:2.1} and \href{https://suttacentral.net/an10.83/en/sujato\#4.1}{AN 10.83:4.1} add “ask questions” after “pay homage”, which seems preferable. } 

Mendicants,\marginnote{24.1} there has not been that faith,\footnote{Finally the Buddha admonishes them. } that approaching, that paying homage, that listening, that hearing the teachings, that remembering the teachings, that reflecting on their meaning, that acceptance after deliberation, that enthusiasm, that making an effort, that weighing up, or that striving. You’re on a bad path, mendicants! You’re practicing the wrong way!\footnote{The \textit{vi-} in \textit{\textsanskrit{vippaṭipanna}} indicates not separation (“lost your way”) but (moral) depravity (eg. \href{https://suttacentral.net/pli-tv-bu-vb-pj1/en/sujato\#9.7.11}{Bu Pj 1:9.7.11}, \href{https://suttacentral.net/sn3.6/en/sujato\#1.4}{SN 3.6:1.4}). | The suffix \textit{-attha} is second plural present of \textit{atthi} (“you are”), whereas the same form at \href{https://suttacentral.net/dn9/en/sujato\#34.19}{DN 9:34.19} is imperative (“you should”). } How far these futile men have strayed from this teaching and training! 

There\marginnote{25.1} is an exposition in four statements, which a sensible person would quickly understand when it is recited.\footnote{This refers to the four injunctions below that begin with “for a faithful disciple”. } I shall recite it for you, mendicants. Try to understand it.” 

“Sir,\marginnote{25.3} who are we next to those who understand the teaching?”\footnote{This idiomatic expression is paralleled at \href{https://suttacentral.net/mn95/en/sujato\#34.12}{MN 95:34.12}. } 

“Even\marginnote{26.1} with a teacher who values things of the flesh, is an heir in things of the flesh, who lives caught up in things of the flesh, you wouldn’t get into such haggling: ‘If we get this, we’ll do that. If we don’t get this, we won’t do it.’ What then of the Realized One, who lives utterly detached from things of the flesh? 

For\marginnote{27.1} a faithful disciple who is practicing to fathom the Teacher’s instructions, this is in line with the teaching: ‘The Buddha is my Teacher, I am his disciple. The Buddha knows, I do not know.’ For a faithful disciple who is practicing to fathom the Teacher’s instructions, the Teacher’s instructions are nourishing and nutritious. For a faithful disciple who is practicing to fathom the Teacher’s instructions, this is in line with the teaching: ‘Gladly, let only skin, sinews, and bones remain! Let the flesh and blood waste away in my body! I will not relax my energy until I have achieved what is possible by human strength, energy, and vigor.’ For a faithful disciple who is practicing to fathom the Teacher’s instructions, one of two results can be expected: enlightenment in this very life, or if there’s something left over, non-return.” 

That\marginnote{27.9} is what the Buddha said. Satisfied, the mendicants approved what the Buddha said. 

%
\addtocontents{toc}{\let\protect\contentsline\protect\nopagecontentsline}
\chapter*{The Chapter on Wanderers }
\addcontentsline{toc}{chapter}{\tocchapterline{The Chapter on Wanderers }}
\addtocontents{toc}{\let\protect\contentsline\protect\oldcontentsline}

%
\section*{{\suttatitleacronym MN 71}{\suttatitletranslation To Vacchagotta on the Three Knowledges }{\suttatitleroot Tevijjavacchasutta}}
\addcontentsline{toc}{section}{\tocacronym{MN 71} \toctranslation{To Vacchagotta on the Three Knowledges } \tocroot{Tevijjavacchasutta}}
\markboth{To Vacchagotta on the Three Knowledges }{Tevijjavacchasutta}
\extramarks{MN 71}{MN 71}

\scevam{So\marginnote{1.1} I have heard.\footnote{This begins the “Vacchagotta trilogy”, a series of three suttas tracing Vacchagotta’s spiritual development. In this sutta he shows an attitude of curiosity and respect, and gains a degree of faith in the Buddha. | This is not the same person as the lay brahmin Vacchagotta of \textsanskrit{Venāgapura}  at \href{https://suttacentral.net/an3.63/en/sujato\#2.1}{AN 3.63:2.1}. } }At one time the Buddha was staying near \textsanskrit{Vesālī}, at the Great Wood, in the hall with the peaked roof. 

Now\marginnote{2.1} at that time the wanderer Vacchagotta was residing in the Single Lotus Monastery of the wanderers.\footnote{Vacchagotta is a clan name; his personal name is not recorded. The Vaccha clan stems from the Vedic seer Vatsa, author of Rig Veda 8.6–11. Several notable figures of this clan are known, including a lineage master in \textsanskrit{Bṛhadāraṇyaka} \textsanskrit{Upaniṣad} 4.5.3, but there is nothing to identify them with the Vacchagotta of the suttas. Given Vacchagotta’s evident interest in the ascetic traditions, it is noteworthy that a certain Śayyambha Vatsagotra, father of Manaka, was said to be a direct disciple of \textsanskrit{Mahāvīra} (\textsanskrit{Kalpasūtra} 10.3). } 

Then\marginnote{3.1} the Buddha robed up in the morning and, taking his bowl and robe, entered \textsanskrit{Vesālī} for alms. Then it occurred to him, “It’s too early to wander for alms in \textsanskrit{Vesālī}. Why don’t I visit the wanderer Vacchagotta at the Single Lotus Monastery?” So that’s what he did. 

Vacchagotta\marginnote{4.1} saw the Buddha coming off in the distance, and said to him, “Let the Blessed One come, sir! Welcome to the Blessed One, sir! It’s been a long time since you took the opportunity to come here. Please, sir, sit down, this seat is ready.” 

The\marginnote{4.7} Buddha sat on the seat spread out, while Vacchagotta took a low seat and sat to one side. Then Vacchagotta said to the Buddha: 

“Sir,\marginnote{5.1} I have heard this:\footnote{At \href{https://suttacentral.net/an3.57/en/sujato}{AN 3.57} he similarly asks the Buddha to confirm his teachings on generosity. At \href{https://suttacentral.net/sn44.9/en/sujato}{SN 44.9}, meanwhile, he asks a somewhat related question about the six ascetic teachers and their divining of afterlife destinies. } ‘The ascetic Gotama claims to be all-knowing and all-seeing, to know and see everything without exception, thus:\footnote{Such extraordinary claims were not unprecedented, as both the Jain leader \textsanskrit{Mahāvīra} (\href{https://suttacentral.net/mn14/en/sujato\#17.2}{MN 14:17.2}, \href{https://suttacentral.net/mn79/en/sujato\#6.7}{MN 79:6.7}) and \textsanskrit{Pūraṇa} Kassapa (\href{https://suttacentral.net/an9.38/en/sujato\#2.1}{AN 9.38:2.1}) are said to claim omniscience in the Pali canon. Extant Jain texts have no shortage of such claims (eg. \textsanskrit{Sūyagaḍa1}.6, \textsanskrit{Kalpasūtra} 1, \textsanskrit{Bhagavatisūtra} 4.10, etc.), and it seems likely that this was part of \textsanskrit{Mahāvīra}’s teachings. Sources for \textsanskrit{Pūraṇa} Kassapa are slim, but he was regarded as a great teacher of the \textsanskrit{Ājīvakas}, whose founder the bamboo-staffed ascetic \textsanskrit{Gosāla} was said to be omniscient (Basham, \emph{History and Doctrines of the \textsanskrit{Ājīvikas}}, 275–6). } “Knowledge and vision are constantly and continually present to me, while walking, standing, sleeping, and waking.”’ I trust that those who say this repeat what the Buddha has said, and do not misrepresent him with an untruth? Is their explanation in line with the teaching? Are there any legitimate grounds for rebuttal and criticism?” 

“Vaccha,\marginnote{5.5} those who say this do not repeat what I have said. They misrepresent me with what is false and untrue.”\footnote{At \href{https://suttacentral.net/an4.24/en/sujato\#2.1}{AN 4.24:2.1} the Buddha claims to know whatever is “seen, heard, thought, known, attained, sought, and explored by the mind”, which is a complete knowledge of subjective experience. In the curious dialogue at \href{https://suttacentral.net/mn90/en/sujato\#8.5}{MN 90:8.5} the Buddha denies that anyone “will know” all things simultaneously. The \textsanskrit{Milindapañha} reversed this to the positive statement that the Buddha could know all things, just not simultaneously (\href{https://suttacentral.net/mil5.1.2/en/sujato\#1.2}{Mil 5.1.2:1.2}). The Buddha’s omniscience became the standard view of the later books of the Pali canon (\href{https://suttacentral.net/cnd10/en/sujato\#5.2}{Cnd 10:5.2}, \href{https://suttacentral.net/bv2/en/sujato\#55.3}{Bv 2:55.3}, \href{https://suttacentral.net/pp1.6/en/sujato\#2.1}{Pp 1.6:2.1}, \href{https://suttacentral.net/kv3.1/en/sujato\#5.3}{Kv 3.1:5.3}, etc.). } 

“So\marginnote{6.1} how should we answer so as to repeat what the Buddha has said, and not misrepresent him with an untruth? How should we explain in line with his teaching, with no legitimate grounds for rebuttal and criticism?” 

“‘The\marginnote{6.2} ascetic Gotama has the three knowledges.’ Answering like this you would repeat what I have said, and not misrepresent me with an untruth. You would explain in line with my teaching, and there would be no legitimate grounds for rebuttal and criticism.\footnote{The Buddha’s knowledge is specific and functional: it serves the purpose of liberation. } 

For,\marginnote{7.1} Vaccha, whenever I want, I recollect my many kinds of past lives. That is: one, two, three, four, five, ten, twenty, thirty, forty, fifty, a hundred, a thousand, a hundred thousand rebirths; many eons of the world contracting, many eons of the world expanding, many eons of the world contracting and expanding. I remember: ‘There, I was named this, my clan was that, I looked like this, and that was my food. This was how I felt pleasure and pain, and that was how my life ended. When I passed away from that place I was reborn somewhere else. There, too, I was named this, my clan was that, I looked like this, and that was my food. This was how I felt pleasure and pain, and that was how my life ended. When I passed away from that place I was reborn here.’ And so I recollect my many kinds of past lives, with features and details. 

And\marginnote{8.1} whenever I want, with clairvoyance that is purified and superhuman, I see sentient beings passing away and being reborn—inferior and superior, beautiful and ugly, in a good place or a bad place. I understand how sentient beings are reborn according to their deeds. 

And\marginnote{9.1} I have realized the undefiled freedom of heart and freedom by wisdom in this very life. I live having realized it with my own insight due to the ending of defilements. 

‘The\marginnote{10.1} ascetic Gotama has the three knowledges.’ Answering like this you would repeat what I have said, and not misrepresent me with an untruth. You would explain in line with my teaching, and there would be no legitimate grounds for rebuttal and criticism.” 

When\marginnote{11.1} he said this, the wanderer Vacchagotta said to the Buddha, “Mister Gotama, are there any laypeople who, without giving up the fetter of lay life, make an end of suffering when the body breaks up?”\footnote{Vacchagotta uses the phrase \textit{\textsanskrit{gihisaṁyojana}} (“fetter of lay life”) for the inner attachment to lay life, which is not overcome by the mere act of joining a religious order. Compare the Buddha’s term \textit{gehasita}, the thoughts and memories “of the lay life” (\href{https://suttacentral.net/mn21/en/sujato\#6.2}{MN 21:6.2}, \href{https://suttacentral.net/mn119/en/sujato\#4.7}{MN 119:4.7}, \href{https://suttacentral.net/mn125/en/sujato\#23.5}{MN 125:23.5}, \href{https://suttacentral.net/mn137/en/sujato\#9.3}{MN 137:9.3}). See also \href{https://suttacentral.net/mn54/en/sujato}{MN 54}, where the external “signs” of a householder are connected with the fetters. } 

“No,\marginnote{11.3} Vaccha.” 

“But\marginnote{12.1} are there any laypeople who, without giving up the fetter of lay life, go to heaven when the body breaks up?” 

“There’s\marginnote{12.2} not just one hundred laypeople, Vaccha, or two or three or four or five hundred, but many more than that who, without giving up the fetter of lay life, go to heaven when the body breaks up.” 

“Mister\marginnote{13.1} Gotama, are there any \textsanskrit{Ājīvaka} ascetics who make an end of suffering when the body breaks up?” 

“No,\marginnote{13.2} Vaccha.” 

“But\marginnote{14.1} are there any \textsanskrit{Ājīvaka} ascetics who go to heaven when the body breaks up?” 

“Vaccha,\marginnote{14.2} when I recollect the past ninety-one eons, I can’t find any \textsanskrit{Ājīvaka} ascetics who have gone to heaven, except one;\footnote{This goes back to the time of the Buddha \textsanskrit{Vipassī} (\href{https://suttacentral.net/dn14/en/sujato\#1.4.1}{DN 14:1.4.1}). } and he taught the efficacy of deeds and action.”\footnote{Thus denying the basic doctrines of both \textsanskrit{Gosāla} (\href{https://suttacentral.net/dn2/en/sujato\#20.2}{DN 2:20.2}) and \textsanskrit{Pūraṇa} Kassapa (\href{https://suttacentral.net/dn2/en/sujato\#17.2}{DN 2:17.2}). } 

“In\marginnote{15.1} that case, Mister Gotama, that sectarian fold is empty even of the chance to go to heaven.”\footnote{The Buddha is speaking of the \textsanskrit{Ājīvakas}, who denied moral causality, not of non-Buddhist paths generally. } 

“Yes,\marginnote{15.2} Vaccha, that sectarian fold is empty even of the chance to go to heaven.” 

That\marginnote{15.3} is what the Buddha said. Satisfied, the wanderer Vacchagotta approved what the Buddha said. 

%
\section*{{\suttatitleacronym MN 72}{\suttatitletranslation With Vacchagotta on Fire }{\suttatitleroot Aggivacchasutta}}
\addcontentsline{toc}{section}{\tocacronym{MN 72} \toctranslation{With Vacchagotta on Fire } \tocroot{Aggivacchasutta}}
\markboth{With Vacchagotta on Fire }{Aggivacchasutta}
\extramarks{MN 72}{MN 72}

\scevam{So\marginnote{1.1} I have heard.\footnote{This sutta depicts Vacchagotta, having gained a degree of faith as depicted in \href{https://suttacentral.net/mn71/en/sujato}{MN 71}, question the Buddha regarding the undeclared points. This is just a taste of Vacchagotta’s curiosity, as the 55 suttas of SN 33 all deal with the same topic, as do \href{https://suttacentral.net/sn44.7/en/sujato}{SN 44.7}, \href{https://suttacentral.net/sn44.8/en/sujato}{SN 44.8}, and \href{https://suttacentral.net/sn44.11/en/sujato}{SN 44.11}. Underlying all these questions is the assumption of a self, which Vacchagotta asks directly about at \href{https://suttacentral.net/sn44.10/en/sujato}{SN 44.10}. } }At one time the Buddha was staying near \textsanskrit{Sāvatthī} in Jeta’s Grove, \textsanskrit{Anāthapiṇḍika}’s monastery. 

Then\marginnote{2.1} the wanderer Vacchagotta went up to the Buddha and exchanged greetings with him. When the greetings and polite conversation were over, he sat down to one side and said to the Buddha: 

“Mister\marginnote{3.1} Gotama, is this your view: ‘The cosmos is eternal. This is the only truth, anything else is futile’?” 

“That’s\marginnote{3.3} not my view, Vaccha.” 

“Then\marginnote{4.1} is this your view: ‘The cosmos is not eternal. This is the only truth, anything else is futile’?” 

“That’s\marginnote{4.3} not my view, Vaccha.” 

“Then\marginnote{5.1} is this your view: ‘The cosmos is finite. This is the only truth, anything else is futile’?” 

“That’s\marginnote{5.3} not my view, Vaccha.” 

“Then\marginnote{6.1} is this your view: ‘The cosmos is infinite. This is the only truth, anything else is futile’?” 

“That’s\marginnote{6.3} not my view, Vaccha.” 

“Then\marginnote{7.1} is this your view: ‘The soul and the body are the same thing. This is the only truth, anything else is futile’?” 

“That’s\marginnote{7.3} not my view, Vaccha.” 

“Then\marginnote{8.1} is this your view: ‘The soul and the body are different things. This is the only truth, anything else is futile’?” 

“That’s\marginnote{8.3} not my view, Vaccha.” 

“Then\marginnote{9.1} is this your view: ‘A realized one still exists after death. This is the only truth, anything else is futile’?” 

“That’s\marginnote{9.3} not my view, Vaccha.” 

“Then\marginnote{10.1} is this your view: ‘A realized one no longer exists after death. This is the only truth, anything else is futile’?” 

“That’s\marginnote{10.3} not my view, Vaccha.” 

“Then\marginnote{11.1} is this your view: ‘A realized one both still exists and no longer exists after death. This is the only truth, anything else is futile’?” 

“That’s\marginnote{11.3} not my view, Vaccha.” 

“Then\marginnote{12.1} is this your view: ‘A realized one neither still exists nor no longer exists after death. This is the only truth, anything else is futile’?” 

“That’s\marginnote{12.3} not my view, Vaccha.” 

“Mister\marginnote{13.1} Gotama, when asked these ten questions, you say: ‘That’s not my view.’ Seeing what drawback do you avoid all these convictions?” 

“Each\marginnote{14.1} of these ten convictions is the thicket of views, the desert of views, the twist of views, the dodge of views, the fetter of views. They’re beset with suffering, distress, anguish, and fever. They don’t lead to disillusionment, dispassion, cessation, peace, insight, awakening, and extinguishment. Seeing this drawback I avoid all these convictions.” 

“But\marginnote{15.1} does Mister Gotama have any convictions at all?” 

“A\marginnote{15.2} realized one has done away with convictions.\footnote{The Buddha, or indeed any arahant, has no “convictions” (\textit{\textsanskrit{diṭṭhigata}}), i.e. theories to attach to, since his “right view” (\textit{\textsanskrit{sammādiṭṭhi}}) is grounded in what he has “seen” (\textit{\textsanskrit{diṭṭha}}). } For a realized one has seen: ‘Such is form, such is the origin of form, such is the ending of form. Such is feeling, such is the origin of feeling, such is the ending of feeling. Such is perception, such is the origin of perception, such is the ending of perception. Such are choices, such is the origin of choices, such is the ending of choices. Such is consciousness, such is the origin of consciousness, such is the ending of consciousness.’ That’s why a Realized One is freed with the ending, fading away, cessation, giving up, and letting go of all conceiving, all churning, and all I-making, mine-making, or underlying tendency to conceit, I say.”\footnote{At \href{https://suttacentral.net/thag14.2/en/sujato}{Thag 14.2}, \textit{mathita} means “oppressed, weighed down”, and the same sense probably applies at \href{https://suttacentral.net/thag1.102/en/sujato}{Thag 1.102}. At \href{https://suttacentral.net/dhp349/en/sujato}{Dhp 349}, however, the root sense of “churn” fits the context of “thoughts”, while at \href{https://suttacentral.net/mn90/en/sujato\#12.14}{MN 90:12.14} it refers to starting a fire by “churning” with a fire-drill, a common Vedic usage (Rig Veda 3.23.1a, 3.29.12a, 8.48.6a; Śatapatha \textsanskrit{Brāhmaṇa} 2.1.4.8, 3.7.3.3, etc.). Its unique appearance in this context anticipates the simile of fire. } 

“But\marginnote{16.1} Mister Gotama, when a mendicant’s mind is freed like this, where are they reborn?” 

“‘They’re\marginnote{16.2} reborn’ doesn’t apply, Vaccha.” 

“Well\marginnote{16.3} then, are they not reborn?” 

“‘They’re\marginnote{16.4} not reborn’ doesn’t apply, Vaccha.” 

“Well\marginnote{16.5} then, are they both reborn and not reborn?” 

“‘They’re\marginnote{16.6} both reborn and not reborn’ doesn’t apply, Vaccha.” 

“Well\marginnote{16.7} then, are they neither reborn nor not reborn?” 

“‘They’re\marginnote{16.8} neither reborn nor not reborn’ doesn’t apply, Vaccha.” 

“Mister\marginnote{17.1} Gotama, when asked all these questions, you say: ‘It doesn’t apply.’ I fail to understand this point, Mister Gotama; I’ve fallen into confusion. And I’ve now lost even the degree of clarity I had from previous discussions with Mister Gotama.” 

“No\marginnote{18.1} wonder you don’t understand, Vaccha, no wonder you’re confused.\footnote{Vacchagotta’s confusion on this same point is addressed at \href{https://suttacentral.net/sn44.9/en/sujato\#5.3}{SN 44.9:5.3}. } For this principle is deep, hard to see, hard to understand, peaceful, sublime, beyond the scope of logic, subtle, comprehensible to the astute. It’s hard for you to understand, since you have a different view, creed, and belief, unless you dedicate yourself to practice with the guidance of tradition. 

Well\marginnote{18.4} then, Vaccha, I’ll ask you about this in return, and you can answer as you like.\footnote{When Vacchagotta is out of his depth, the Buddha helps him by coming back to a simple, grounded metaphor. } 

What\marginnote{19.1} do you think, Vaccha? Suppose a fire was burning in front of you. Would you know: ‘This fire is burning in front of me’?” 

“Yes,\marginnote{19.4} I would, Mister Gotama.” 

“But\marginnote{19.6} Vaccha, suppose they were to ask you: ‘This fire burning in front of you: what does it depend on to burn?’ How would you answer?”\footnote{The language in this passage echoes key Buddhist doctrinal terms. Here, “depend” is \textit{\textsanskrit{paṭicca}}. } 

“I\marginnote{19.9} would answer like this: ‘This fire burning in front of me burns in dependence on grass and logs as fuel.’”\footnote{“Fuel” is \textit{\textsanskrit{upādāna}}, normally translated as “grasping”. } 

“Suppose\marginnote{19.11} that fire burning in front of you was extinguished. Would you know:\footnote{“Extinguished” is \textit{\textsanskrit{nibbāyeyya}}, a verbal form of \textsanskrit{Nibbāna}. } ‘This fire in front of me is quenched’?” 

“Yes,\marginnote{19.13} I would, Mister Gotama.” 

“But\marginnote{19.15} Vaccha, suppose they were to ask you: ‘This fire in front of you that is quenched: in what direction did it go—east, south, west, or north?’ How would you answer?” 

“It\marginnote{19.18} doesn’t apply, Mister Gotama. The fire depended on grass and logs as fuel. When that runs out, and no more fuel is added, the fire is reckoned to have become quenched due to lack of fuel.” 

“In\marginnote{20.1} the same way, Vaccha, any form by which a realized one might be described has been given up, cut off at the root, made like a palm stump, obliterated, and unable to arise in the future.\footnote{Here “realized one” (\textit{\textsanskrit{tathāgata}}) pertains to any arahant, as at \href{https://suttacentral.net/sn54.12/en/sujato}{SN 54.12}. The Pali commentary and one of the Chinese parallels here (SA2 196 at T ii 445a18) agree in saying that a “living being” (\textit{satta}) is meant. However, \textit{\textsanskrit{tathāgata}} and \textit{satta} are exact verbal parallels: \textit{\textsanskrit{tathā}} (“real”) + \textit{gata} (“come to the state of”) and \textit{sa} (“real”) + \textit{tta} (“state of”). Thus it seems likely that this was originally intended as a mere verbal gloss to resolve the compound. } A realized one is freed from reckoning in terms of form. They’re deep, immeasurable, and hard to fathom, like the ocean. ‘They’re reborn’, ‘they’re not reborn’, ‘they’re both reborn and not reborn’, ‘they’re neither reborn nor not reborn’—none of these apply. 

Any\marginnote{20.5} feeling … perception … choices … consciousness by which a realized one might be described has been given up, cut off at the root, made like a palm stump, obliterated, and unable to arise in the future. A realized one is freed from reckoning in terms of consciousness. They’re deep, immeasurable, and hard to fathom, like the ocean. ‘They’re reborn’, ‘they’re not reborn’, ‘they’re both reborn and not reborn’, ‘they’re neither reborn nor not reborn’—none of these apply.” 

When\marginnote{21.1} he said this, the wanderer Vacchagotta said to the Buddha: 

“Mister\marginnote{21.2} Gotama, suppose there was a large sal tree not far from a town or village. And because it’s impermanent, its branches and foliage, bark and shoots, and softwood would fall off. After some time it would be rid of branches and foliage, bark and shoots, and softwood, pure, and consolidated in the core. In the same way, Mister Gotama’s dispensation is rid of branches and foliage, bark and shoots, and softwood, pure, and consolidated in the core. 

Excellent,\marginnote{22.1} Mister Gotama! … From this day forth, may Mister Gotama remember me as a lay follower who has gone for refuge for life.”\footnote{Vacchagotta had already gone forth as a wanderer, so it is surprising that he calls himself a “lay follower” (\textit{\textsanskrit{upāsaka}}). The Chinese parallels simply say that he left without taking refuge (SA 962 at T ii 246a17 and SA2 196 at T ii 445c8). Given that the Pali tradition repeats him going for refuge in the next sutta (\href{https://suttacentral.net/mn73/en/sujato\#15.2}{MN 73:15.2}), it seems likely that this detail was added by mistake here. } 

%
\section*{{\suttatitleacronym MN 73}{\suttatitletranslation The Longer Discourse With Vacchagotta }{\suttatitleroot Mahāvacchasutta}}
\addcontentsline{toc}{section}{\tocacronym{MN 73} \toctranslation{The Longer Discourse With Vacchagotta } \tocroot{Mahāvacchasutta}}
\markboth{The Longer Discourse With Vacchagotta }{Mahāvacchasutta}
\extramarks{MN 73}{MN 73}

\scevam{So\marginnote{1.1} I have heard.\footnote{The final episode in the Vacchagotta trilogy sees our hero rise above his useless metaphysical quandaries and seek a better way to live. The Buddha’s patience is rewarded as Vacchagotta finally achieves his goal. } }At one time the Buddha was staying near \textsanskrit{Rājagaha}, in the Bamboo Grove, the squirrels’ feeding ground. 

Then\marginnote{2.1} the wanderer Vacchagotta went up to the Buddha and exchanged greetings with him. When the greetings and polite conversation were over, he sat down to one side and said to the Buddha, “For a long time I have had discussions with Mister Gotama. Please teach me in brief what is skillful and what is unskillful.” 

“Vaccha,\marginnote{3.3} I can teach you what is skillful and what is unskillful in brief or in detail. Still, let me do so in brief. Listen and apply your mind well, I will speak.” 

“Yes,\marginnote{3.6} sir,” Vaccha replied. The Buddha said this: 

“Greed\marginnote{4.1} is unskillful, contentment is skillful.\footnote{Elsewhere (eg. \href{https://suttacentral.net/mn9/en/sujato\#4.1}{MN 9:4.1}) a distinction is drawn between greed, hate, and delusion as the \emph{root} of the unskillful, while the ten kinds of action are the unskillful (and the same applies to the skillful). } Hate is unskillful, love is skillful. Delusion is unskillful, understanding is skillful. So there are these three unskillful things and three that are skillful. 

Killing\marginnote{5.1} living creatures, stealing, and sexual misconduct; speech that’s false, divisive, harsh, or nonsensical; covetousness, ill will, and wrong view: these things are unskillful. Refraining from killing living creatures, stealing, and sexual misconduct; refraining from speech that’s false, divisive, harsh, or nonsensical; contentment, kind-heartedness, and right view: these things are skillful. So there are these ten unskillful things and ten that are skillful. 

When\marginnote{6.1} a mendicant has given up craving so it is cut off at the root, made like a palm stump, obliterated, and unable to arise in the future, that mendicant is perfected. They’ve ended the defilements, completed the spiritual journey, done what had to be done, laid down the burden, achieved their own true goal, utterly ended the fetter of continued existence, and are rightly freed through enlightenment.” 

“Leaving\marginnote{7.1} aside Mister Gotama, is there even a single monk disciple of Mister Gotama who has realized the undefiled freedom of heart and freedom by wisdom in this very life, and lives having realized it with their own insight due to the ending of defilements?”\footnote{Two devas mentally ask \textsanskrit{Mahāvīra} a similar question at \textsanskrit{Bhagavatisūtra} 4.5.59; the answer was seven hundred. } 

“There\marginnote{7.3} are not just one hundred such monks who are my disciples, Vaccha, or two or three or four or five hundred, but many more than that.” 

“Leaving\marginnote{8.1} aside Mister Gotama and the monks, is there even a single nun disciple of Mister Gotama who has realized the undefiled freedom of heart and freedom by wisdom in this very life, and lives having realized it with their own insight due to the ending of defilements?” 

“There\marginnote{8.3} are not just one hundred such nuns who are my disciples, Vaccha, or two or three or four or five hundred, but many more than that.” 

“Leaving\marginnote{9.1} aside Mister Gotama, the monks, and the nuns, is there even a single layman disciple of Mister Gotama—white-clothed and celibate—who, with the ending of the five lower fetters, is reborn spontaneously, to be extinguished there, not liable to return from that world?”\footnote{Devoted layfolk would regularly keep the five precepts, which forbid sexual misconduct, and undertake the eight precepts, including celibacy, on the sabbath. However it seems that then, as today, there is a less formalized class of layfolk who would undertake the eight precepts continuously. White-robed, celibate layfolk are mentioned in \href{https://suttacentral.net/dn29/en/sujato\#12.11}{DN 29:12.11}, where they are a property of a fully-developed and prosperous religion. Celibate laypeople are also referred to at \href{https://suttacentral.net/an5.180/en/sujato}{AN 5.180} and \href{https://suttacentral.net/an10.75/en/sujato\#2.2}{AN 10.75:2.2}. Such layfolk have the potential to realize non-return, since that requires the complete letting go of all forms of desire. } 

“There\marginnote{9.3} are not just one hundred such celibate laymen who are my disciples, Vaccha, or two or three or four or five hundred, but many more than that.” 

“Leaving\marginnote{10.1} aside Mister Gotama, the monks, the nuns, and the celibate laymen, is there even a single layman disciple of Mister Gotama—white-clothed, enjoying sensual pleasures, following instructions, and responding to advice—who has gone beyond doubt, got rid of indecision, and lives self-assured and independent of others regarding the Teacher’s instruction?”\footnote{That is, they are stream-enterers. } 

“There\marginnote{10.3} are not just one hundred such laymen enjoying sensual pleasures who are my disciples, Vaccha, or two or three or four or five hundred, but many more than that.” 

“Leaving\marginnote{11.1} aside Mister Gotama, the monks, the nuns, the celibate laymen, and the laymen enjoying sensual pleasures, is there even a single laywoman disciple of Mister Gotama—white-clothed and celibate—who, with the ending of the five lower fetters, is reborn spontaneously, to be extinguished there, not liable to return from that world?” 

“There\marginnote{11.3} are not just one hundred such celibate laywomen who are my disciples, Vaccha, or two or three or four or five hundred, but many more than that.” 

“Leaving\marginnote{12.1} aside Mister Gotama, the monks, the nuns, the celibate laymen, the laymen enjoying sensual pleasures, and the celibate laywomen, is there even a single laywoman disciple of Mister Gotama—white-clothed, enjoying sensual pleasures, following instructions, and responding to advice—who has gone beyond doubt, got rid of indecision, and lives self-assured and independent of others regarding the Teacher’s instruction?” 

“There\marginnote{12.3} are not just one hundred such laywomen enjoying sensual pleasures who are my disciples, Vaccha, or two or three or four or five hundred, but many more than that.” 

“If\marginnote{13.1} Mister Gotama was the only one to succeed in this teaching, not any monks, then this spiritual path would be incomplete in that respect. But because both Mister Gotama and monks have succeeded in this teaching, this spiritual path is complete in that respect. 

If\marginnote{13.5} Mister Gotama and the monks were the only ones to succeed in this teaching, not any nuns … celibate laymen … laymen enjoying sensual pleasures … celibate laywomen … 

laywomen\marginnote{13.21} enjoying sensual pleasures, then this spiritual path would be incomplete in that respect. But because Mister Gotama, monks, nuns, celibate laymen, laymen enjoying sensual pleasures, celibate laywomen, and laywomen enjoying sensual pleasures have all succeeded in this teaching, this spiritual path is complete in that respect. 

Just\marginnote{14.1} as the Ganges river slants, slopes, and inclines towards the ocean, and keeps pushing into the ocean, in the same way Mister Gotama’s assembly—with both laypeople and renunciates—slants, slopes, and inclines towards extinguishment, and keeps pushing into extinguishment. 

Excellent,\marginnote{15.1} Mister Gotama! … I go for refuge to Mister Gotama, to the teaching, and to the mendicant \textsanskrit{Saṅgha}. May I receive the going forth, the ordination in Mister Gotama’s presence?” 

“Vaccha,\marginnote{16.1} if someone formerly ordained in another sect wishes to take the going forth, the ordination in this teaching and training, they must spend four months on probation. When four months have passed, if the mendicants are satisfied, they’ll give the going forth, the ordination into monkhood. However, I have recognized individual differences in this matter.” 

“Sir,\marginnote{16.3} if four months probation are required in such a case, I’ll spend four years on probation. When four years have passed, if the mendicants are satisfied, let them give me the going forth, the ordination into monkhood.” And the wanderer Vaccha received the going forth, the ordination in the Buddha’s presence. 

Not\marginnote{17.1} long after his ordination, a fortnight later, Venerable Vacchagotta went to the Buddha, bowed, sat down to one side, and said to him, “Sir, I’ve reached as far as possible with the knowledge and understanding of a trainee. Please teach me further.” 

“Well\marginnote{18.1} then, Vaccha, further develop two things: serenity and discernment. When you have further developed these two things, they’ll lead to the penetration of many elements. 

Whenever\marginnote{19.1} you want, you will be capable of realizing these things, since each and every one is within range: ‘May I wield the many kinds of psychic power: multiplying myself and becoming one again; appearing and disappearing; going unobstructed through a wall, a rampart, or a mountain as if through space; diving in and out of the earth as if it were water; walking on water as if it were earth; flying cross-legged through the sky like a bird; touching and stroking with my hand the sun and moon, so mighty and powerful; controlling my body as far as the realm of divinity.’ 

Whenever\marginnote{20.1} you want, you will be capable of realizing these things, since each and every one is within range: ‘With clairaudience that is purified and superhuman, may I hear both kinds of sounds, human and heavenly, whether near or far.’ 

Whenever\marginnote{21.1} you want, you will be capable of realizing these things, since each and every one is within range: ‘May I understand the minds of other beings and individuals, having comprehended them with my mind. May I understand mind with greed as “mind with greed”, and mind without greed as “mind without greed”; mind with hate as “mind with hate”, and mind without hate as “mind without hate”; mind with delusion as “mind with delusion”, and mind without delusion as “mind without delusion”; constricted mind as “constricted mind”, and scattered mind as “scattered mind”; expansive mind as “expansive mind”, and unexpansive mind as “unexpansive mind”; mind that is not supreme as “mind that is not supreme”, and mind that is supreme as “mind that is supreme”; mind immersed in \textsanskrit{samādhi} as “mind immersed in \textsanskrit{samādhi}”, and mind not immersed in \textsanskrit{samādhi} as “mind not immersed in \textsanskrit{samādhi}”; freed mind as “freed mind”, and unfreed mind as “unfreed mind”.’ 

Whenever\marginnote{22.1} you want, you will be capable of realizing these things, since each and every one is within range: ‘May I recollect many kinds of past lives. That is: one, two, three, four, five, ten, twenty, thirty, forty, fifty, a hundred, a thousand, a hundred thousand rebirths; many eons of the world contracting, many eons of the world expanding, many eons of the world contracting and expanding. May I remember: “There, I was named this, my clan was that, I looked like this, and that was my food. This was how I felt pleasure and pain, and that was how my life ended. When I passed away from that place I was reborn somewhere else. There, too, I was named this, my clan was that, I looked like this, and that was my food. This was how I felt pleasure and pain, and that was how my life ended. When I passed away from that place I was reborn here.” May I recollect my many past lives, with features and details.’ 

Whenever\marginnote{23.1} you want, you will be capable of realizing these things, since each and every one is within range: ‘With clairvoyance that is purified and superhuman, may I see sentient beings passing away and being reborn—inferior and superior, beautiful and ugly, in a good place or a bad place—and understand how sentient beings are reborn according to their deeds: “These dear beings did bad things by way of body, speech, and mind. They denounced the noble ones; they had wrong view; and they chose to act out of that wrong view. When their body breaks up, after death, they’re reborn in a place of loss, a bad place, the underworld, hell. These dear beings, however, did good things by way of body, speech, and mind. They never denounced the noble ones; they had right view; and they chose to act out of that right view. When their body breaks up, after death, they’re reborn in a good place, a heavenly realm.” And so, with clairvoyance that is purified and superhuman, may I see sentient beings passing away and being reborn—inferior and superior, beautiful and ugly, in a good place or a bad place. And may I understand how sentient beings are reborn according to their deeds.’ 

Whenever\marginnote{24.1} you want, you will be capable of realizing these things, since each and every one is within range: ‘May I realize the undefiled freedom of heart and freedom by wisdom in this very life, and live having realized it with my own insight due to the ending of defilements.’” 

And\marginnote{25.1} then Venerable Vacchagotta approved and agreed with what the Buddha said. He got up from his seat, bowed, and respectfully circled the Buddha, keeping him on his right, before leaving. 

Then\marginnote{26.1} Vacchagotta, living alone, withdrawn, diligent, keen, and resolute, soon realized the supreme end of the spiritual path in this very life. He lived having achieved with his own insight the goal for which gentlemen rightly go forth from the lay life to homelessness. 

He\marginnote{26.2} understood: “Rebirth is ended; the spiritual journey has been completed; what had to be done has been done; there is nothing further for this place.” And Venerable Vacchagotta became one of the perfected. 

Now\marginnote{27.1} at that time several mendicants were going to see the Buddha. Vacchagotta saw them coming off in the distance, went up to them, and said, “Hello venerables, where are you going?” 

“Reverend,\marginnote{27.5} we are going to see the Buddha.” 

“Well\marginnote{27.6} then, reverends, in my name please bow with your head at the Buddha’s feet and say: ‘Sir, the mendicant Vacchagotta bows with his head to your feet and says, “I have served the Blessed One! I have served the Holy One!”’” 

“Yes,\marginnote{28.1} reverend,” they replied. Then those mendicants went up to the Buddha, bowed, sat down to one side, and said to him, “Sir, the mendicant Vacchagotta bows with his head to your feet and says: ‘I have served the Blessed One! I have served the Holy One!’” 

“I’ve\marginnote{28.5} already comprehended Vacchagotta’s mind and understood that he has the three knowledges, and is very mighty and powerful. And deities also told me about this.”\footnote{Vacchagotta’s attainment is also confirmed in his verse at \href{https://suttacentral.net/thag1.112/en/sujato}{Thag 1.112}. } 

That\marginnote{28.9} is what the Buddha said. Satisfied, the mendicants approved what the Buddha said. 

%
\section*{{\suttatitleacronym MN 74}{\suttatitletranslation With Dīghanakha }{\suttatitleroot Dīghanakhasutta}}
\addcontentsline{toc}{section}{\tocacronym{MN 74} \toctranslation{With Dīghanakha } \tocroot{Dīghanakhasutta}}
\markboth{With Dīghanakha }{Dīghanakhasutta}
\extramarks{MN 74}{MN 74}

\scevam{So\marginnote{1.1} I have heard.\footnote{This sutta records the occasion of the full enlightenment of the Buddha’s greatest disciple, \textsanskrit{Sāriputta}, as well as the conversion of \textsanskrit{Dīghanakha}. The commentary says this became the occasion for a unique four-factored gathering: it is the full moon in the month of \textsanskrit{Māgha}; 1,250 mendicants spontaneously assemble; all were arahants with the six high knowledges; all were ordained by the “come, mendicant” formula. Today this is celebrated in the Makha Puja festival in Thailand. } }At one time the Buddha was staying near \textsanskrit{Rājagaha}, on the Vulture’s Peak Mountain in the Boar’s Cave. 

Then\marginnote{2.1} the wanderer \textsanskrit{Dīghanakha} went up to the Buddha, and exchanged greetings with him.\footnote{\textsanskrit{Dīghanakha} means “Long-nails”; it must have been a nickname. Long nails are forbidden in the Vinaya (\href{https://suttacentral.net/pli-tv-kd15/en/sujato\#27.1.21}{Kd 15:27.1.21}) and also to the one performing a Brahmanical consecration (\textit{\textsanskrit{dīkṣā}}, Śatapatha \textsanskrit{Brāhmaṇa} 3.1.2). Further, EA 34.4 says that long nails (\langlzh{爪長}), uncut hair, and unwashed robes were disgraceful characteristics of ascetics. \textsanskrit{Ācārāṅgasūtra} 2.8.5 speaks of the ascetic practice of abandoning the care of the hair of the head, beard, and nails. | \textsanskrit{Dīghanakha} was \textsanskrit{Sāriputta}’s nephew according to the Pali commentaries (not here, but in \href{https://suttacentral.net/mn111/en/sujato}{MN 111}, \href{https://suttacentral.net/sn11.2/en/sujato}{SN 11.2}, etc.) and northern texts such as \textsanskrit{Mahāprajñāpāramitā}-\textsanskrit{upadeśa}-\textsanskrit{śāstra}, \textsanskrit{Mahāvibhāṣā}, and \textsanskrit{Avadānaśataka}. } When the greetings and polite conversation were over, he stood to one side and said to the Buddha, “Mister Gotama, this is my doctrine and view: ‘Nothing is acceptable to me.’”\footnote{\textsanskrit{Dīghanakha}’s view “nothing is acceptable to me” (\textit{\textsanskrit{sabbaṁ} me nakkhamati}) is obscure, which seems to be the point. Sanskrit and Tibetan parallels to this passage are identical to the Pali, so when some Chinese translations say “no view” or “no teaching” is acceptable, this is probably because the translators felt the need to clarify, echoing the common expression, “the acceptance of a view after consideration” (\textit{\textsanskrit{diṭṭhinijjhānakkhanti}}). This reading fits the text, but the sense may be broader than that. The closest phrasing in Pali is at \href{https://suttacentral.net/dn25/en/sujato\#10.11}{DN 25:10.11}, where “this is not acceptable to me” (\textit{\textsanskrit{idaṁ} me nakkhamati}) refers to food that a picky monk rejects. Views about acceptable food are recorded in detail in Jain texts, and we may presume that other orders had similar systems. Strictness in such matters is seen as a sign of virtue, escalating ultimately to the refusal of all food. } 

“This\marginnote{2.5} view of yours, Aggivessana—\footnote{Aggivessana is his Brahmanical clan name. } is that acceptable to you?”\footnote{While this displays the Buddha’s ready wit, like much good humor it plays on a deeper anxiety. For it is easy to say that one believes in nothing, but the state of unknowing is psychologically unstable, as minds seek answers. When someone claims to hold no views, their skepticism is how they hide their views from themselves. } 

“If\marginnote{2.7} I were to accept this view, Mister Gotama, it would make no difference, it would make no difference!” 

“Well,\marginnote{3.1} Aggivessana, there are many more in the world who say, ‘It would make no difference! It would make no difference!’ But they don’t give up that view, and they grasp another view. And there are a scant few in the world who say, ‘It would make no difference! It would make no difference!’ And they give up that view by not grasping another view. 

There\marginnote{4.1} are some ascetics and brahmins who have this doctrine and view: ‘Everything is acceptable to me.’ There are some ascetics and brahmins who have this doctrine and view: ‘Nothing is acceptable to me.’ There are some ascetics and brahmins who have this doctrine and view: ‘Some things are acceptable to me and some things are not.’ Regarding this, the view of the ascetics and brahmins to whom everything is acceptable is close to greed, bondage, approving, attachment, and grasping. The view of the ascetics and brahmins to whom nothing is acceptable is far from greed, bondage, approving, attachment, and grasping.” 

When\marginnote{5.1} he said this, the wanderer \textsanskrit{Dīghanakha} said to the Buddha, “Mister Gotama commends my conviction! He recommends my conviction!” 

“Now,\marginnote{5.3} regarding the ascetics and brahmins to whom some things are acceptable and some things are not. Their view of what is acceptable is close to greed, bondage, approving, attachment, and grasping. Their view of what is not acceptable is far from greed, bondage, approving, attachment, and grasping. 

When\marginnote{6.1} it comes to the view of the ascetics and brahmins to whom everything is acceptable, a sensible person reflects like this: ‘I have the view that everything is acceptable. Suppose I were to obstinately stick to this view and insist, “This is the only truth, anything else is futile.” Then I’d argue with two people—an ascetic or brahmin to whom nothing is acceptable, and an ascetic or brahmin to whom some things are acceptable and some things are not. And when there’s arguing, there’s quarreling; when there’s quarreling there’s distress; and when there’s anguish there’s harm.’ So, considering in themselves the potential for arguing, quarreling, distress, and harm, they give up that view by not grasping another view. That’s how those views are given up and let go. 

When\marginnote{7.1} it comes to the view of the ascetics and brahmins to whom nothing is acceptable, a sensible person reflects like this: ‘I have the view that nothing is acceptable. Suppose I were to obstinately stick to this view and insist, “This is the only truth, anything else is futile.” Then I’d argue with two people—an ascetic or brahmin to whom everything is acceptable, and an ascetic or brahmin to whom some things are acceptable and some things are not. And when there’s arguing, there’s quarreling; when there’s quarreling there’s distress; and when there’s anguish there’s harm.’ So, considering in themselves the potential for arguing, quarreling, distress, and harm, they give up that view by not grasping another view. That’s how those views are given up and let go. 

When\marginnote{8.1} it comes to the view of the ascetics and brahmins to whom some things are acceptable to me and some things are not, a sensible person reflects like this: ‘I have the view that some things are acceptable and some things are not. Suppose I were to obstinately stick to this view and insist, “This is the only truth, anything else is futile.” Then I’d argue with two people—an ascetic or brahmin to whom everything is acceptable, and an ascetic or brahmin to whom nothing is acceptable. And when there’s arguing, there’s quarreling; when there’s quarreling there’s distress; and when there’s anguish there’s harm.’ So, considering in themselves the potential for arguing, quarreling, distress, and harm, they give up that view by not grasping another view. That’s how those views are given up and let go. 

Aggivessana,\marginnote{9.1} this body is formed. It’s made up of the four principal states, produced by mother and father, built up from rice and porridge, liable to impermanence, to wearing away and erosion, to breaking up and destruction. You should see it as impermanent, as suffering, as diseased, as a boil, as a dart, as misery, as an affliction, as alien, as falling apart, as empty, as not-self. Doing so, you’ll give up desire, affection, and subservience to the body.\footnote{For \textit{\textsanskrit{kāyanvayatā}} (“subservience to the body”) see \href{https://suttacentral.net/mn36/en/sujato\#4.4}{MN 36:4.4}. The inclusion of this rare phrase here suggests that Jain-like practices such as starvation may be indicated. } 

There\marginnote{10.1} are these three feelings: pleasant, painful, and neutral. At a time when you feel a pleasant feeling, you don’t feel a painful or neutral feeling; you only feel a pleasant feeling. At a time when you feel a painful feeling, you don’t feel a pleasant or neutral feeling; you only feel a painful feeling. At a time when you feel a neutral feeling, you don’t feel a pleasant or painful feeling; you only feel a neutral feeling. 

Pleasant,\marginnote{11.1} painful, and neutral feelings are impermanent, conditioned, dependently originated, liable to end, vanish, fade away, and cease. 

Seeing\marginnote{12.1} this, a learned noble disciple grows disillusioned with pleasant, painful, and neutral feelings. Being disillusioned, desire fades away. When desire fades away they’re freed. When they’re freed, they know they’re freed. 

They\marginnote{12.3} understand: ‘Rebirth is ended, the spiritual journey has been completed, what had to be done has been done, there is nothing further for this place.’ 

A\marginnote{13.1} mendicant whose mind is freed like this doesn’t side with anyone or dispute with anyone. They use the language of the world to communicate without getting stuck on it.”\footnote{See \href{https://suttacentral.net/dn9/en/sujato\#53.5}{DN 9:53.5}. } 

Now\marginnote{14.1} at that time Venerable \textsanskrit{Sāriputta} was standing behind the Buddha fanning him. Then he thought, “It seems the Buddha speaks of giving up and letting go all these things through direct knowledge.” Reflecting like this, Venerable \textsanskrit{Sāriputta}’s mind was freed from the defilements by not grasping.\footnote{\textsanskrit{Sāriputta} became a stream-enterer on hearing a short teaching from the monk Assaji (\href{https://suttacentral.net/pli-tv-kd1/en/sujato\#23.5.6}{Kd 1:23.5.6}), whereupon he told his friend \textsanskrit{Moggallāna}. They both promptly left their former teacher \textsanskrit{Sañjaya}, a traumatic loss to \textsanskrit{Sañjaya} and his community. \href{https://suttacentral.net/an4.172/en/sujato\#2.1}{AN 4.172:2.1} says that \textsanskrit{Sāriputta} became enlightened a fortnight after he ordained, a detail confirmed in multiple parallels to this passage; \href{https://suttacentral.net/mn111/en/sujato\#2.8}{MN 111:2.8} details the practice he undertook in that fortnight. This means that the current discussion must have taken place only a fortnight after leaving \textsanskrit{Sañjaya}. Now, \textsanskrit{Sañjaya}’s philosophy was to systematically avoid taking any position (\href{https://suttacentral.net/dn2/en/sujato\#32.2}{DN 2:32.2}), which is similar to \textsanskrit{Dīghanakha}’s doctrine. It seems likely, then, that \textsanskrit{Dīghanakha} was a disciple of \textsanskrit{Sañjaya}. This supposition finds support in several parallels, which say \textsanskrit{Dīghanakha} was annoyed to learn that \textsanskrit{Sāriputta} had rejected other teachers and gone over to the Buddha (eg. \textsanskrit{Avadānaśataka}, Vaidya 1958a 256: \textit{sarve \textsanskrit{tīrthakarā} \textsanskrit{nigṛhītāḥ}}; also in \textsanskrit{Mahāprajñāpāramitā}-\textsanskrit{upadeśa}-\textsanskrit{śāstra} and \textsanskrit{Mahāvibhāṣā}). These sources do not specify \textsanskrit{Sañjaya}, but who else could it be? } 

And\marginnote{15.1} the stainless, immaculate vision of the Dhamma arose in the wanderer \textsanskrit{Dīghanakha}: “Everything that has a beginning has an end.” Then \textsanskrit{Dīghanakha} saw, attained, understood, and fathomed the Dhamma. He went beyond doubt, got rid of indecision, and became self-assured and independent of others regarding the Teacher’s instructions. He said to the Buddha: 

“Excellent,\marginnote{16.1} Mister Gotama! Excellent! As if he were righting the overturned, or revealing the hidden, or pointing out the path to the lost, or lighting a lamp in the dark so people with clear eyes can see what’s there, Mister Gotama has made the teaching clear in many ways. I go for refuge to Mister Gotama, to the teaching, and to the mendicant \textsanskrit{Saṅgha}. From this day forth, may Mister Gotama remember me as a lay follower who has gone for refuge for life.”\footnote{The Pali is alone in saying \textsanskrit{Dīghanakha} became a lay follower, as all parallels report that he ordained and became an arahant. As with Vacchagotta at \href{https://suttacentral.net/mn72/en/sujato\#22.2}{MN 72:22.2}, it is unlikely that a wanderer would become a lay follower and the Pali is probably in error. | The \textsanskrit{Mahāprajñāpāramitā}-\textsanskrit{upadeśa}-\textsanskrit{śāstra}, Tibetan \textsanskrit{Mūla}-\textsanskrit{Sarvāstivāda}-Vinaya, and \textsanskrit{Avadānaśataka} say that, having cut his long nails before ordination, he was known by his actual name, \textsanskrit{Koṣṭhila}, that is, the monk known in Pali as \textsanskrit{Mahākoṭṭhita}. This contradicts the Pali tradition, but as \textsanskrit{Anālayo} shows, there are several inconsistencies in the Pali accounts, so the northern tradition is credible (\emph{Comparative Study}, vol. 1, p. 406, note 88). } 

%
\section*{{\suttatitleacronym MN 75}{\suttatitletranslation With Māgaṇḍiya }{\suttatitleroot Māgaṇḍiyasutta}}
\addcontentsline{toc}{section}{\tocacronym{MN 75} \toctranslation{With Māgaṇḍiya } \tocroot{Māgaṇḍiyasutta}}
\markboth{With Māgaṇḍiya }{Māgaṇḍiyasutta}
\extramarks{MN 75}{MN 75}

\scevam{So\marginnote{1.1} I have heard.\footnote{The wanderer \textsanskrit{Māgaṇḍiya} appears outraged at the presence of the Buddha, who he thinks stands in opposition to life itself since he opposes sensual enjoyment. It soon turns out that he misunderstands the Buddha’s teachings on sensuality, as the Buddha’s aim is not to prevent people from experiencing pleasure but to elevate them so they can enjoy a higher form of pleasure. } }At one time the Buddha was staying in the land of the Kurus, near the Kuru town named \textsanskrit{Kammāsadamma}, on a grass mat in the fire chamber of a brahmin of the \textsanskrit{Bhāradvāja} clan. 

Then\marginnote{2.1} the Buddha robed up in the morning and, taking his bowl and robe, entered \textsanskrit{Kammāsadamma} for alms. He wandered for alms in \textsanskrit{Kammāsadamma}. After the meal, on his return from almsround, he went to a certain forest grove for the day’s meditation. Having plunged deep into it, he sat at the root of a certain tree to meditate. 

Then\marginnote{3.1} as the wanderer \textsanskrit{Māgaṇḍiya} was going for a walk he approached that fire chamber.\footnote{A \textsanskrit{Māgaṇḍiya} appears here and in \href{https://suttacentral.net/snp4.9/en/sujato}{Snp 4.9}, where he uses his beautiful daughter to tempt the Buddha (cited and discussed at \href{https://suttacentral.net/sn22.3/en/sujato}{SN 22.3}). As to the protagonist of \href{https://suttacentral.net/snp4.9/en/sujato}{Snp 4.9}, the Pali commentary, \textsanskrit{Divyāvadāna} 36—which calls him \textsanskrit{Mākandika}—and the Chinese parallel in the Arthapada (T iv 180) say he too was from the Kuru country. Now, the \textsanskrit{Mahābhārata} and other Sanskrit sources name \textsanskrit{Mākandī} as a city or province on the banks of the Ganges, in the \textsanskrit{Pañcāla} kingdom south of \textsanskrit{Hastināpura}. This more-or-less agrees with \textsanskrit{Māgaṇḍiya}’s home in the Kuru country, and suggests his name means “the man from \textsanskrit{Mākandī}”. \textsanskrit{Divyāvadāna} 36 and Sanskrit fragment say the \textsanskrit{Mākandika} of \href{https://suttacentral.net/snp4.9/en/sujato}{Snp 4.9} was also a wanderer, which does not contradict the position of the Pali commentaries and the Arthapada that he was a brahmin. At \href{https://suttacentral.net/an5.294-302/en/sujato\#1.1}{AN 5.294–302:1.1}, \textit{\textsanskrit{māgaṇḍika}} is listed as a type of ascetic; these could well be students of \textsanskrit{Māgaṇḍiya}. \textsanskrit{Kathāsaritsāgara} 15 tells the comic farce of an ascetic from \textsanskrit{Mākandikā} on the banks of the Ganges, who fell in love with the daughter of a merchant and went to ridiculous lengths to obtain her, only to end up a laughing-stock. This story could well have begun life as a comic inversion of \href{https://suttacentral.net/snp4.9/en/sujato}{Snp 4.9}. Reading all these sources together, we can posit that there was a brahmin wanderer from \textsanskrit{Mākandī} on the Ganges in Kuru-\textsanskrit{Pañcāla}, who gained a certain fame and following as a teacher, but whose hedonism was his downfall. The Pali commentaries further specify that the \textsanskrit{Māgaṇḍiya} of the current sutta is the nephew of the one at \href{https://suttacentral.net/snp4.9/en/sujato}{Snp 4.9}. } He saw the grass mat spread out there and asked the brahmin of the \textsanskrit{Bhāradvāja} clan, “Mister \textsanskrit{Bhāradvāja}, who has this grass mat been spread out for? It looks like an ascetic’s bed.” 

“There\marginnote{4.1} is the ascetic Gotama, a Sakyan, gone forth from a Sakyan family. He has this good reputation: ‘That Blessed One is perfected, a fully awakened Buddha, accomplished in knowledge and conduct, holy, knower of the world, supreme guide for those who wish to train, teacher of gods and humans, awakened, blessed.’ This bed has been spread for that Mister Gotama.” 

“Well,\marginnote{5.1} it’s a sad sight, Mister \textsanskrit{Bhāradvāja}, a very sad sight indeed, to see a bed for Mister Gotama, that baby-killer!”\footnote{\textit{\textsanskrit{Bhūnahan}} is Sanskrit \textit{\textsanskrit{bhrūṇahan}}. The earliest sense is “fetus-killer”, found in Rig Veda 10.155.2a where a grotesque demoness is cursed as one who “slays all fetuses” (\textit{\textsanskrit{sarvā} \textsanskrit{bhrūṇāny} \textsanskrit{āruṣī}}), and also at Atharvaveda 6.1112.3c (\textit{\textsanskrit{bhrūṇaghni}}). Later it took on the sense “murderer of a brahmin”. It is used in Pali for a depraved person, especially one who destroys the life of a developed fetus or child (see \href{https://suttacentral.net/an7.64/en/sujato\#22.1}{AN 7.64:22.1}, \href{https://suttacentral.net/ja358/en/sujato\#1.1}{Ja 358:1.1}, \href{https://suttacentral.net/ja544/en/sujato\#105.2}{Ja 544:105.2}). | The brahmin of \textsanskrit{Verañja} calls the Buddha a variety of names at \href{https://suttacentral.net/an8.11/en/sujato}{AN 8.11}. } 

“Be\marginnote{5.4} careful what you say, \textsanskrit{Māgaṇḍiya}, be careful what you say. Many astute aristocrats, brahmins, householders, and ascetics are devoted to Mister Gotama. They’ve been guided by him in the noble system, the skillful teaching.” 

“Even\marginnote{5.7} if I was to see Mister Gotama face to face, Mister \textsanskrit{Bhāradvāja}, I would say to his face: ‘The ascetic Gotama is a baby-killer.’ Why is that? For so it came to me while sleeping.”\footnote{\textit{Sutta} can mean “discourse” or “sleeping”. I do not know anywhere else in the Pali where non-Buddhist texts are referred to as \textit{sutta}. Further, it would seem unlikely that they had such a prophecy of the Buddha, and no early Brahmanical texts have anything like this. On the other hand, the \textsanskrit{Upaniṣads} feature many passages on sleep and dream, and the importance of the visions therein (eg. \textsanskrit{Bṛhadāraṇyaka} \textsanskrit{Upaniṣad} 2.1.18). Also, \textit{avacarati} is not, so far as I know, used for the “transmission” of scripture, but it is used for the “descent” from heaven (Rig Veda 10.59.9a), a usage fitting well the appearance of a vision in a dream. | The plural \textit{no} (“our”) here is not significant, as \textsanskrit{Māgaṇḍiya} refers to himself with the royal “we”. } 

“If\marginnote{5.11} you don’t mind, I’ll tell the ascetic Gotama about this.” 

“Don’t\marginnote{5.12} worry, Mister \textsanskrit{Bhāradvāja}. You may tell him exactly what I’ve said.” 

With\marginnote{6.1} clairaudience that is purified and superhuman, the Buddha heard this discussion between the brahmin of the \textsanskrit{Bhāradvāja} clan and the wanderer \textsanskrit{Māgaṇḍiya}. Coming out of retreat, he went to the brahmin’s fire chamber and sat on the grass mat. Then the brahmin of the \textsanskrit{Bhāradvāja} clan went to the Buddha and exchanged greetings with him. When the greetings and polite conversation were over, he sat down to one side. The Buddha said to him, “\textsanskrit{Bhāradvāja}, did you have a discussion with the wanderer \textsanskrit{Māgaṇḍiya} about this grass mat?” 

When\marginnote{6.6} he said this, the brahmin, shocked and awestruck, said to the Buddha, “I wanted to mention this very thing to Mister Gotama, but you brought it up before I had a chance.” 

But\marginnote{7.1} this conversation between the Buddha and the brahmin was left unfinished. Then as the wanderer \textsanskrit{Māgaṇḍiya} was going for a walk he approached that fire chamber. He went up to the Buddha, and exchanged greetings with him. When the greetings and polite conversation were over, he sat down to one side, and the Buddha said to him: 

“\textsanskrit{Māgaṇḍiya},\marginnote{8.1} the eye likes sights, it loves them and enjoys them. That’s been tamed, guarded, protected and restrained by the Realized One, and he teaches Dhamma for its restraint. Is that what you were referring to when you called me a baby-killer?” 

“That’s\marginnote{8.5} exactly what I was referring to. Why is that? For so it came to me while sleeping.” 

“The\marginnote{8.9} ear likes sounds … The nose likes smells … The tongue likes tastes … The body likes touches … The mind likes ideas, it loves them and enjoys them. That’s been tamed, guarded, protected and restrained by the Realized One, and he teaches Dhamma for its restraint. Is that what you were referring to when you called me a baby-killer?” 

“That’s\marginnote{8.24} exactly what I was referring to. Why is that? For so it came to me while sleeping.” 

“What\marginnote{9.1} do you think, \textsanskrit{Māgaṇḍiya}? Take someone who used to amuse themselves with sights known by the eye, which are likable, desirable, agreeable, pleasant, sensual, and arousing. Some time later—having truly understood the origin, ending, gratification, drawback, and escape of sights, and having given up craving and dispelled passion for sights—they would live rid of thirst, their mind peaceful inside. What would you have to say to them, \textsanskrit{Māgaṇḍiya}?” 

“Nothing,\marginnote{9.4} Mister Gotama.” 

“What\marginnote{9.5} do you think, \textsanskrit{Māgaṇḍiya}? Take someone who used to amuse themselves with sounds known by the ear … smells known by the nose … tastes known by the tongue … touches known by the body, which are likable, desirable, agreeable, pleasant, sensual, and arousing. Some time later—having truly understood the origin, ending, gratification, drawback, and escape of touches, and having given up craving and dispelled passion for touches—they would live rid of thirst, their mind peaceful inside. What would you have to say to them, \textsanskrit{Māgaṇḍiya}?” 

“Nothing,\marginnote{9.12} Mister Gotama.” 

“Well,\marginnote{10.1} when I was still a layperson I used to amuse myself, supplied and provided with sights known by the eye … sounds known by the ear … smells known by the nose … tastes known by the tongue … touches known by the body, which are likable, desirable, agreeable, pleasant, sensual, and arousing. I had three stilt longhouses—one for the rainy season, one for the winter, and one for the summer. I stayed in a stilt longhouse without coming downstairs for the four months of the rainy season, where I was entertained by musicians—none of them men. Some time later—having truly understood the origin, ending, gratification, drawback, and escape of sensual pleasures, and having given up craving and dispelled passion for sensual pleasures—I live rid of thirst, my mind peaceful inside. I see other sentient beings who are not free from sensual pleasures being consumed by craving for sensual pleasures, burning with passion for sensual pleasures, indulging in sensual pleasures. I don’t envy them, nor do I hope to enjoy that. Why is that? Because there is a satisfaction that is apart from sensual pleasures and unskillful qualities, which even equals heavenly pleasure.\footnote{\textit{Samadhigayha \textsanskrit{tiṭṭhati}} means “stands at the same level”, “equals” not “surpasses” (\href{https://suttacentral.net/iti23/en/sujato\#2.1}{Iti 23:2.1}, \href{https://suttacentral.net/mn78/en/sujato\#8.8}{MN 78:8.8}). } Enjoying that satisfaction, I don’t envy what is inferior, nor do I hope to enjoy it. 

Suppose\marginnote{11.1} there was a householder or a householder’s child who was rich, affluent, and wealthy. And they would amuse themselves, supplied and provided with the five kinds of sensual stimulation. That is, sights known by the eye … sounds … smells … tastes … touches known by the body, which are likable, desirable, agreeable, pleasant, sensual, and arousing. Having practiced good conduct by way of body, speech, and mind, when their body breaks up, after death, they’d be reborn in a good place, a heavenly realm, in the company of the gods of the thirty-three. There they’d amuse themselves in the Garden of Delight, escorted by a band of nymphs, supplied and provided with the five kinds of heavenly sensual stimulation. Then they’d see a householder or a householder’s child amusing themselves, supplied and provided with the five kinds of sensual stimulation. 

What\marginnote{11.7} do you think, \textsanskrit{Māgaṇḍiya}? Would that god—amusing themselves in the Garden of Delight, escorted by a band of nymphs, supplied and provided with the five kinds of heavenly sensual stimulation—envy that householder or householder’s child their five kinds of human sensual stimulation, or return to human sensual pleasures?” 

“No,\marginnote{11.8} Mister Gotama. Why is that? Because heavenly sensual pleasures are better than human sensual pleasures.” 

“In\marginnote{12.1} the same way, \textsanskrit{Māgaṇḍiya}, when I was still a layperson I used to entertain myself with sights … sounds … smells … tastes … touches known by the body, which are likable, desirable, agreeable, pleasant, sensual, and arousing. Some time later—having truly understood the origin, ending, gratification, drawback, and escape of sensual pleasures, and having given up craving and dispelled passion for sensual pleasures—I live rid of thirst, my mind peaceful inside. I see other sentient beings who are not free from sensual pleasures being consumed by craving for sensual pleasures, burning with passion for sensual pleasures, indulging in sensual pleasures. I don’t envy them, nor do I hope to enjoy that. Why is that? Because there is a satisfaction that is apart from sensual pleasures and unskillful qualities, which even equals heavenly pleasure. Enjoying that satisfaction, I don’t envy what is inferior, nor do I hope to enjoy it. 

Suppose\marginnote{13.1} there was a person affected by leprosy, with sores and blisters on their limbs. Being devoured by worms, scratching with their nails at the opening of their wounds, they’d cauterize their body over a pit of glowing coals.\footnote{“Person affected by leprosy” is \textit{\textsanskrit{kuṭṭhī} puriso} } Their friends and colleagues, relatives and kin would get a surgeon to treat them. The surgeon would make medicine for them, and by using that they’d be cured of leprosy. They’d be healthy, happy, autonomous, master of themselves, able to go where they wanted. Then they’d see another person affected by leprosy, with sores and blisters on their limbs, being devoured by worms, scratching with their nails at the opening of their wounds, cauterizing their body over a pit of glowing coals. 

What\marginnote{13.6} do you think, \textsanskrit{Māgaṇḍiya}? Would that person envy that other person affected by leprosy for their pit of glowing coals or for taking medicine?” 

“No,\marginnote{13.8} Mister Gotama. Why is that? Because you need to take medicine only when there’s a disease. When there’s no disease, there’s no need for medicine.” 

“In\marginnote{14.1} the same way, \textsanskrit{Māgaṇḍiya}, when I was still a layperson I used to entertain myself with sights … sounds … smells … tastes … touches known by the body, which are likable, desirable, agreeable, pleasant, sensual, and arousing. Some time later—having truly understood the origin, ending, gratification, drawback, and escape of sensual pleasures, and having given up craving and dispelled passion for sensual pleasures—I live rid of thirst, my mind peaceful inside. I see other sentient beings who are not free from sensual pleasures being consumed by craving for sensual pleasures, burning with passion for sensual pleasures, indulging in sensual pleasures. I don’t envy them, nor do I hope to enjoy that. Why is that? Because there is a satisfaction that is apart from sensual pleasures and unskillful qualities, which even equals heavenly pleasure. Enjoying that satisfaction, I don’t envy what is inferior, nor do I hope to enjoy it. 

Suppose\marginnote{15.1} there was a person affected by leprosy, with sores and blisters on their limbs. Being devoured by worms, scratching with their nails at the opening of their wounds, they’d cauterize their body over a pit of glowing coals. Their friends and colleagues, relatives and kin would get a surgeon to treat them. The surgeon would make medicine for them, and by using that they’d be cured of leprosy. They’d be healthy, happy, autonomous, master of themselves, able to go where they wanted. Two strong men would grab them by the arms and drag them towards the pit of glowing coals. 

What\marginnote{15.6} do you think, \textsanskrit{Māgaṇḍiya}? Wouldn’t that person writhe and struggle to and fro?” 

“Yes,\marginnote{15.8} Mister Gotama. Why is that? Because that fire is really painful to touch, fiercely burning and scorching.” 

“What\marginnote{15.11} do you think, \textsanskrit{Māgaṇḍiya}? Is it only now that the fire is really painful to touch, fiercely burning and scorching, or was it painful previously as well?” 

“That\marginnote{15.13} fire is painful now and it was also painful previously. That person was affected by leprosy, with sores and blisters on their limbs. Being devoured by worms, scratching with their nails at the opening of their wounds, their sense faculties were impaired. So even though the fire was actually painful to touch, they had a distorted perception that it was pleasant.” 

“In\marginnote{16.1} the same way, sensual pleasures of the past, future, and present are painful to touch, fiercely burning and scorching. These sentient beings who are not free from sensual pleasures—being consumed by craving for sensual pleasures, burning with passion for sensual pleasures—have impaired sense faculties. So even though sensual pleasures are actually painful to touch, they have a distorted perception that they are pleasant. 

Suppose\marginnote{17.1} there was a person affected by leprosy, with sores and blisters on their limbs. Being devoured by worms, scratching with their nails at the opening of their wounds, they’re cauterizing their body over a pit of glowing coals. The more they scratch their wounds and cauterize their body, the more their wounds become foul, stinking, and infected. But still, they derive a degree of pleasure and gratification from the itchiness of their wounds. In the same way, I see other sentient beings who are not free from sensual pleasures being consumed by craving for sensual pleasures, burning with passion for sensual pleasures, indulging in sensual pleasures. The more they indulge in sensual pleasures, the more their craving for sensual pleasures grows, and the more they burn with passion for sensual pleasures. But still, they derive a degree of pleasure and gratification from the five kinds of sensual stimulation. 

What\marginnote{18.1} do you think, \textsanskrit{Māgaṇḍiya}? Have you seen or heard of a king or a royal minister of the past, future, or present, amusing themselves supplied and provided with the five kinds of sensual stimulation, who—without giving up craving for sensual pleasures and dispelling passion for sensual pleasures—lives rid of thirst, their mind peaceful inside?” 

“No,\marginnote{18.3} Mister Gotama.” 

“Good,\marginnote{18.4} \textsanskrit{Māgaṇḍiya}. Neither have I. On the contrary, all the ascetics or brahmins of the past, future, or present who live rid of thirst, their minds peaceful inside, do so after truly understanding the origin, ending, gratification, drawback, and escape of sensual pleasures, and after giving up craving and dispelling passion for sensual pleasures.” 

Then\marginnote{19.1} on that occasion the Buddha expressed this heartfelt sentiment: 

\begin{verse}%
“Health\marginnote{19.2} is the ultimate blessing;\footnote{The first two lines appear also as the first and last lines of \href{https://suttacentral.net/dhp204/en/sujato}{Dhp 204}, as well as in many parallels for that verse. } \\
extinguishment, the ultimate happiness. \\
Of paths, the ultimate is eightfold—\\
it’s safe, and leads to freedom from death.” 

%
\end{verse}

When\marginnote{19.6} he said this, \textsanskrit{Māgaṇḍiya} said to him, “It’s incredible, Mister Gotama, it’s amazing! How well said this was by Mister Gotama! ‘Health is the ultimate blessing; extinguishment, the ultimate happiness.’ I’ve also heard that wanderers of the past, the tutors of tutors, said: ‘Health is the ultimate blessing; extinguishment, the ultimate happiness.’\footnote{I have not been able to trace such a saying. \textsanskrit{Kaṭha} \textsanskrit{Upaniṣad} 2.4.2 identifies the supreme divinity as “ultimate happiness”. } And it agrees, Mister Gotama.” 

“But\marginnote{19.13} \textsanskrit{Māgaṇḍiya}, when you heard that wanderers of the past said this, what is that health? And what is that extinguishment?” When he said this, \textsanskrit{Māgaṇḍiya} stroked his own limbs with his hands, saying: 

“This\marginnote{19.16} is that health, Mister Gotama, this is that extinguishment! For I am now healthy and happy, and have no afflictions.”\footnote{Compare \href{https://suttacentral.net/dn1/en/sujato\#3.20.2}{DN 1:3.20.2}. } 

“\textsanskrit{Māgaṇḍiya},\marginnote{20.1} suppose a person was blind from birth. They couldn’t see sights that are dark or bright, or blue, yellow, red, or magenta. They couldn’t see even and uneven ground, or the stars, or the moon and sun. They might hear a sighted person saying: ‘White cloth is really nice, it’s attractive, stainless, and clean.’ They’d go in search of white cloth. But someone would cheat them with a dirty, soiled garment, saying: ‘Sir, here is a white cloth for you, it’s attractive, stainless, and clean.’ They’d take it and put it on, expressing their gladness: ‘White cloth is really nice, it’s attractive, stainless, and clean.’ 

What\marginnote{20.10} do you think, \textsanskrit{Māgaṇḍiya}? Did that person blind from birth do this knowing and seeing, or out of faith in the sighted person?” 

“They\marginnote{20.13} did so not knowing or seeing, but out of faith in the sighted person.” 

“In\marginnote{21.1} the same way, the wanderers of other religions are blind and sightless. Not knowing health and not seeing extinguishment, they still recite this verse: ‘Health is the ultimate blessing; extinguishment, the ultimate happiness.’ For this verse was recited by the perfected ones, fully awakened Buddhas of the past: 

\begin{verse}%
‘Health\marginnote{21.4} is the ultimate blessing; \\
extinguishment, the ultimate happiness. \\
Of paths, the ultimate is eightfold—\\
it’s safe, and leads to freedom from death.' 

%
\end{verse}

These\marginnote{21.8} days it has gradually become a verse used by ordinary people. But \textsanskrit{Māgaṇḍiya}, this body is a disease, a boil, a dart, a misery, an affliction. Yet you say of this body: ‘This is that health, this is that extinguishment!’ \textsanskrit{Māgaṇḍiya}, you don’t have the noble vision by which you might know health and see extinguishment.” 

“I\marginnote{22.1} am quite confident that Mister Gotama is capable of teaching me so that I can know health and see extinguishment.” 

“\textsanskrit{Māgaṇḍiya},\marginnote{22.3} suppose a person was blind from birth. They couldn’t see sights that are dark or bright, or blue, yellow, red, or magenta. They couldn’t see even and uneven ground, or the stars, or the moon and sun. Their friends and colleagues, relatives and kin would get a surgeon to treat them. The surgeon would make medicine for them, but when they used it their eyes were not cured and they still could not see clearly. What do you think, \textsanskrit{Māgaṇḍiya}? Wouldn’t that doctor just get weary and frustrated?” 

“Yes,\marginnote{22.10} Mister Gotama.” 

“In\marginnote{22.11} the same way, suppose I were to teach you the Dhamma, saying: ‘This is that health, this is that extinguishment.’ But you might not know health or see extinguishment, which would be wearying and troublesome for me.” 

“I\marginnote{23.1} am quite confident that Mister Gotama is capable of teaching me so that I can know health and see extinguishment.” 

“\textsanskrit{Māgaṇḍiya},\marginnote{23.3} suppose a person was blind from birth. They couldn’t see sights that are dark or bright, or blue, yellow, red, or magenta. They couldn’t see even and uneven ground, or the stars, or the moon and sun. They might hear a sighted person saying: ‘White cloth is really nice, it’s attractive, stainless, and clean.’ 

They’d\marginnote{23.7} go in search of white cloth. But someone would cheat them with a dirty, soiled garment, saying: ‘Sir, here is a white cloth for you, it’s attractive, stainless, and clean.’ They’d take it and put it on. Their friends and colleagues, relatives and kin would get a surgeon to treat them. The surgeon would make medicine for them: emetics, purgatives, ointment, counter-ointment, or nasal treatment. And when they used it their eyes would be cured so that they could see clearly. As soon as their eyes were cured they’d lose all desire for that dirty, soiled garment. Then they would consider that person to be no friend, but an enemy, and might even think of murdering them:\footnote{For “consider (\textit{daheyya}) as a friend”, see \href{https://suttacentral.net/sn22.85/en/sujato\#14.15}{SN 22.85:14.15}. } ‘For such a long time I’ve been cheated, tricked, and deceived by that person with this dirty, soiled garment when he said, “Sir, here is a white cloth for you, it’s attractive, stainless, and clean.”’ 

In\marginnote{24.1} the same way, \textsanskrit{Māgaṇḍiya}, suppose I were to teach you the Dhamma, saying: ‘This is that health, this is that extinguishment.’ You might know health and see extinguishment. And as soon as your vision arises you might give up desire for the five grasping aggregates. And you might even think: ‘For such a long time I’ve been cheated, tricked, and deceived by this mind. For what I have been grasping is only form, feeling, perception, choices, and consciousness. My grasping is a condition for continued existence. Continued existence is a condition for rebirth. Rebirth is a condition for old age and death, sorrow, lamentation, pain, sadness, and distress to come to be. That is how this entire mass of suffering originates.’” 

“I\marginnote{25.1} am quite confident that Mister Gotama is capable of teaching me so that I can rise from this seat cured of blindness.” 

“Well\marginnote{25.3} then, \textsanskrit{Māgaṇḍiya}, you should associate with true persons. When you associate with true persons, you will hear the true teaching. When you hear the true teaching, you’ll practice in line with the teaching. When you practice in line with the teaching, you’ll know and see for yourself: ‘These are diseases, boils, and darts. And here is where diseases, boils, and darts cease without anything left over.’ When my grasping ceases, continued existence ceases. When continued existence ceases, rebirth ceases. When rebirth ceases, old age and death, sorrow, lamentation, pain, sadness, and distress cease. That is how this entire mass of suffering ceases.” 

When\marginnote{26.1} he said this, \textsanskrit{Māgaṇḍiya} said to him, “Excellent, Mister Gotama! Excellent! As if he were righting the overturned, or revealing the hidden, or pointing out the path to the lost, or lighting a lamp in the dark so people with clear eyes can see what’s there, Mister Gotama has made the teaching clear in many ways. I go for refuge to Mister Gotama, to the teaching, and to the mendicant \textsanskrit{Saṅgha}. May I receive the going forth, the ordination in Mister Gotama’s presence?” 

“\textsanskrit{Māgaṇḍiya},\marginnote{27.1} if someone formerly ordained in another sect wishes to take the going forth, the ordination in this teaching and training, they must spend four months on probation. When four months have passed, if the mendicants are satisfied, they’ll give the going forth, the ordination into monkhood. However, I have recognized individual differences in this matter.” 

“Sir,\marginnote{27.3} if four months probation are required in such a case, I’ll spend four years on probation. When four years have passed, if the mendicants are satisfied, let them give me the going forth, the ordination into monkhood.” 

And\marginnote{28.1} the wanderer \textsanskrit{Māgaṇḍiya} received the going forth, the ordination in the Buddha’s presence. Not long after his ordination, Venerable \textsanskrit{Māgaṇḍiya}, living alone, withdrawn, diligent, keen, and resolute, realized the supreme culmination of the spiritual path in this very life. He lived having achieved with his own insight the goal for which gentlemen rightly go forth from the lay life to homelessness. 

He\marginnote{28.3} understood: “Rebirth is ended; the spiritual journey has been completed; what had to be done has been done; there is nothing further for this place.” And Venerable \textsanskrit{Māgaṇḍiya} became one of the perfected. 

%
\section*{{\suttatitleacronym MN 76}{\suttatitletranslation With Sandaka }{\suttatitleroot Sandakasutta}}
\addcontentsline{toc}{section}{\tocacronym{MN 76} \toctranslation{With Sandaka } \tocroot{Sandakasutta}}
\markboth{With Sandaka }{Sandakasutta}
\extramarks{MN 76}{MN 76}

\scevam{So\marginnote{1.1} I have heard. }At one time the Buddha was staying near \textsanskrit{Kosambī}, in Ghosita’s Monastery. 

Now\marginnote{2.1} at that time the wanderer Sandaka was residing at the cave of the wavy leaf fig tree together with a large assembly of around five hundred wanderers.\footnote{Neither Sandaka nor this location appear to be known elsewhere. } 

Then\marginnote{3.1} in the late afternoon, Venerable Ānanda came out of retreat and addressed the mendicants: “Come, reverends, let’s go to the Devakata Pool to see the cave.”\footnote{Evidently an excursion for sightseeing was popular. As well as simply enjoying nature, such trips give the mendicants a chance to check out possible places to stay for meditation. } 

“Yes,\marginnote{3.3} reverend,” they replied. Then Ānanda together with several mendicants went to the Devakata Pool. 

Now\marginnote{4.1} at that time, Sandaka and the large assembly of wanderers were sitting together making an uproar, a dreadful racket. They engaged in all kinds of low talk, such as talk about kings, bandits, and chief ministers; talk about armies, threats, and wars; talk about food, drink, clothes, and beds; talk about garlands and fragrances; talk about family, vehicles, villages, towns, cities, and countries; talk about women and heroes; street talk and well talk; talk about the departed; motley talk; tales of land and sea; and talk about being reborn in this or that state of existence. 

Sandaka\marginnote{4.3} saw Ānanda coming off in the distance, and hushed his own assembly, “Be quiet, good sirs, don’t make a sound. The ascetic Ānanda, a disciple of the ascetic Gotama, is coming. He is included among the disciples of the ascetic Gotama, who is residing near \textsanskrit{Kosambī}.\footnote{This sentence is redundant as it stands. But wherever this idiom occurs elsewhere, it specifies that the person is a lay disciple (\href{https://suttacentral.net/dn25/en/sujato\#3.5}{DN 25:3.5}, \href{https://suttacentral.net/mn78/en/sujato\#3.7}{MN 78:3.7}, \href{https://suttacentral.net/an10.93/en/sujato\#2.7}{AN 10.93:2.7}, \href{https://suttacentral.net/an10.94/en/sujato\#3.4}{AN 10.94:3.4}). Perhaps the text here has been corrupted and originally specified that Ānanda was a \emph{monastic} disciple. } Such venerables like the quiet, are educated to be quiet, and praise the quiet. Hopefully if he sees that our assembly is quiet he’ll see fit to approach.” Then those wanderers fell silent. 

Then\marginnote{5.1} Venerable Ānanda went up to the wanderer Sandaka, who said to him, “Come, Mister Ānanda! Welcome, Mister Ānanda! It’s been a long time since you took the opportunity to come here. Please sit down, this seat is ready.” Ānanda sat down on the seat spread out, while Sandaka took a low seat and sat to one side. Ānanda said to Sandaka, “Sandaka, what were you sitting talking about just now? What conversation was left unfinished?” 

“Mister\marginnote{5.10} Ānanda, leave aside what we were sitting talking about just now. It won’t be hard for you to hear about that later. It’d be great if Mister Ānanda himself would give a Dhamma talk explaining his own tradition.” 

“Well\marginnote{5.13} then, Sandaka, listen and apply your mind well, I will speak.” 

“Yes,\marginnote{5.14} good sir,” replied Sandaka. Venerable Ānanda said this: 

“Sandaka,\marginnote{6.1} these things have been explained by the Blessed One, who knows and sees, the perfected one, the fully awakened Buddha: four ways that negate the spiritual life, and four kinds of unreliable spiritual life. A sensible person would, to the best of their ability, not practice such spiritual paths, and if they did practice them, they wouldn’t succeed in the system of the skillful teaching.” 

“But\marginnote{6.2} Mister Ānanda, what are the four ways that negate the spiritual life, and the four kinds of unreliable spiritual life?” 

“Sandaka,\marginnote{7.1} take a certain teacher who has this doctrine and view: ‘There’s no meaning in giving, sacrifice, or offerings. There’s no fruit or result of good and bad deeds. There’s no afterlife. There’s no such thing as mother and father, or beings that are reborn spontaneously. And there’s no ascetic or brahmin who is rightly comported and rightly practiced, and who describes the afterlife after realizing it with their own insight.\footnote{Attributed to Ajita of the hair blanket at \href{https://suttacentral.net/dn2/en/sujato\#23.2}{DN 2:23.2}. | The denial of “mother and father” is usually interpreted as the denial of moral duty towards one’s parents. However, I think it is a doctrine of conception which denies that a child is created by the mother and father. Rather, the child is produced by the four elements, with parents as mere instigators and incubators. } This person is made up of the four principal states. When they die, the earth in their body merges and coalesces with the substance of earth. The water in their body merges and coalesces with the substance of water. The fire in their body merges and coalesces with the substance of fire. The air in their body merges and coalesces with the substance of air. The faculties are transferred to space.\footnote{This is a materialist analysis of the person. | The word \textit{\textsanskrit{kāya}} (“substance”) is central to Jainism. \textsanskrit{Ācārāṅgasūtra} 8.1.11 speaks of the “substances” of earth, water, fire, and air as being imbued with life so one should avoid damaging them. } Four men with a bier carry away the corpse. Their footprints show the way to the cemetery. The bones become bleached. Offerings dedicated to the gods end in ashes. Giving is a doctrine of morons. When anyone affirms a positive teaching it’s just hollow, false nonsense. Both the foolish and the astute are annihilated and destroyed when their body breaks up, and they don’t exist after death.’ 

A\marginnote{8.1} sensible person reflects on this matter in this way: ‘This teacher has such a doctrine and view. If what that teacher says is true, both I who have not accomplished this and one who has accomplished it have attained exactly the same level. Yet I’m not one who says that both of us are annihilated and destroyed when our body breaks up, and we don’t exist after death. But it’s superfluous for this teacher to go naked, shaven, persisting in squatting, tearing out their hair and beard. For I’m living at home with my children, using sandalwood imported from \textsanskrit{Kāsi}, wearing garlands, fragrance, and makeup, and accepting gold and currency. Yet I’ll have exactly the same destiny in the next life as this teacher. What do I know or see that I should lead the spiritual life under this teacher? This negates the spiritual life.’ Realizing this, they leave disappointed. 

This\marginnote{9.1} is the first way that negates the spiritual life. 

Furthermore,\marginnote{10.1} take a certain teacher who has this doctrine and view: ‘The one who acts does nothing wrong when they punish, mutilate, torture, aggrieve, oppress, intimidate, or when they encourage others to do the same. Nothing bad is done when they kill, steal, break into houses, plunder wealth, steal from isolated buildings, commit highway robbery, commit adultery, and lie.\footnote{Attributed to \textsanskrit{Pūraṇa} Kassapa  at \href{https://suttacentral.net/dn2/en/sujato\#17.2}{DN 2:17.2}. | This is a denial of the doctrine of kamma. While his doctrine appears to be morally nihilistic, it seems unlikely this was \textsanskrit{Pūraṇa} Kassapa’s full teaching. He may have subscribed to hard determinism, so that we have no choice in what we do. He may also have believed that we should keep moral rules as a social contract, but that this had no effect on the afterlife. | In such contexts, \textit{kar-} means “punish, inflict” (\href{https://suttacentral.net/mn129/en/sujato\#29.2}{MN 129:29.2}). } If you were to reduce all the living creatures of this earth to one heap and mass of flesh with a razor-edged chakram, no evil comes of that, and no outcome of evil. If you were to go along the south bank of the Ganges killing, mutilating, and torturing, and encouraging others to do the same, no evil comes of that, and no outcome of evil. If you were to go along the north bank of the Ganges giving and sacrificing and encouraging others to do the same, no merit comes of that, and no outcome of merit. In giving, self-control, restraint, and truthfulness there is no merit or outcome of merit.’ 

A\marginnote{11.1} sensible person reflects on this matter in this way: ‘This teacher has such a doctrine and view. If what that teacher says is true, both I who have not accomplished this and one who has accomplished it have attained exactly the same level. Yet I’m not one who says that when both of us act, nothing wrong is done. But it’s superfluous for this teacher to go naked, shaven, persisting in squatting, tearing out their hair and beard. For I’m living at home with my children, using sandalwood imported from \textsanskrit{Kāsi}, wearing garlands, fragrance, and makeup, and accepting gold and currency. Yet I’ll have exactly the same destiny in the next life as this teacher. What do I know or see that I should lead the spiritual life under this teacher? This negates the spiritual life.’ Realizing this, they leave disappointed. 

This\marginnote{12.1} is the second way that negates the spiritual life. 

Furthermore,\marginnote{13.1} take a certain teacher who has this doctrine and view: ‘There is no cause or reason for the corruption of sentient beings.\footnote{This is the first part of the view attributed to Bamboo-staffed Ascetic \textsanskrit{Gosāla} at \href{https://suttacentral.net/dn2/en/sujato\#20.2}{DN 2:20.2}. | This denies the principle of causality and the efficacy of action. The fatalistic teachings of the \textsanskrit{Ājīvikas} led to them becoming popular as prognosticators. } Sentient beings are corrupted without cause or reason. There’s no cause or reason for the purification of sentient beings. Sentient beings are purified without cause or reason. There is no power, no energy, no human strength or vigor.\footnote{The otherwise parallel passage has a different phrasing at \href{https://suttacentral.net/dn2/en/sujato\#20.5}{DN 2:20.5}. } All sentient beings, all living creatures, all beings, all souls lack control, power, and energy. Molded by destiny, circumstance, and nature, they experience pleasure and pain in the six classes of rebirth.’ 

A\marginnote{14.1} sensible person reflects on this matter in this way: ‘This teacher has such a doctrine and view. If what that teacher says is true, both I who have not accomplished this and one who has accomplished it have attained exactly the same level. Yet I’m not one who says that both of us are purified without cause or reason. But it’s superfluous for this teacher to go naked, shaven, persisting in squatting, tearing out their hair and beard. For I’m living at home with my children, using sandalwood imported from \textsanskrit{Kāsi}, wearing garlands, fragrance, and makeup, and accepting gold and currency. Yet I’ll have exactly the same destiny in the next life as this teacher. What do I know or see that I should lead the spiritual life under this teacher? This negates the spiritual life.’ Realizing this, they leave disappointed. 

This\marginnote{15.1} is the third way that negates the spiritual life. 

Furthermore,\marginnote{16.1} take a certain teacher who has this doctrine and view: ‘There are these seven substances that are not made, not derived, not created, without a creator, barren, steady as a mountain peak, standing firm like a pillar.\footnote{This combines the view attributed to Pakudha \textsanskrit{Kaccāyana} (\href{https://suttacentral.net/dn2/en/sujato\#26.2}{DN 2:26.2}) with the second part of \textsanskrit{Gosāla}’s view (\href{https://suttacentral.net/dn2/en/sujato\#20.7}{DN 2:20.7}). There are several inconsistencies in the presentation of these views in Pali, and still more when parallels are considered. These were closely-related teachers, who often practiced together for a time, and it is hardly surprising that they should share some views in common, or that their exact views have become confused. } They don’t move or deteriorate or obstruct each other. They’re unable to cause pleasure, pain, or both pleasure and pain to each other. What seven? The substances of earth, water, fire, air; pleasure, pain, and the soul is the seventh. These seven substances are not made, not derived, not created, without a creator, barren, steady as a mountain peak, standing firm like a pillar. They don’t move or deteriorate or obstruct each other. They’re unable to cause pleasure, pain, or both pleasure and pain to each other. And here there is no-one who kills or who makes others kill; no-one who learns or who educates others; no-one who understands or who helps others understand.\footnote{Compare \href{https://suttacentral.net/an8.16/en/sujato\#1.3}{AN 8.16:1.3}. } If you chop off someone’s head with a sharp sword, you don’t take anyone’s life. The sword simply passes through the gap between the seven substances. There are 1.4 million main wombs, and 6,000, and 600. There are 500 deeds, and five, and three. There are deeds and half-deeds. There are 62 paths, 62 sub-eons, six classes of rebirth, and eight stages in a person’s life. There are 4,900 \textsanskrit{Ājīvaka} ascetics, 4,900 wanderers, and 4,900 naked ascetics. There are 2,000 faculties, 3,000 hells, and 36 realms of dust. There are seven percipient embryos, seven non-percipient embryos, and seven knotless embryos. There are seven gods, seven humans, and seven goblins. There are seven lakes, seven rivers, 700 rivers, seven cliffs, and 700 cliffs. There are seven dreams and 700 dreams. There are 8.4 million great eons through which the foolish and the astute transmigrate before making an end of suffering. And here there is no such thing as this: “By this precept or observance or fervent austerity or spiritual life I shall force unripened deeds to bear their fruit, or eliminate old deeds by experiencing their results little by little”—for that cannot be. Pleasure and pain are allotted. Transmigration lasts only for a limited period, so there’s no increase or decrease, no getting better or worse. It’s like how, when you toss a ball of string, it rolls away unraveling. In the same way, after transmigrating the foolish and the astute will make an end of suffering.’ 

A\marginnote{17.1} sensible person reflects on this matter in this way: ‘This teacher has such a doctrine and view. If what that teacher says is true, both I who have not accomplished this and one who has accomplished it have attained exactly the same level. Yet I’m not one who says that after transmigrating both of us will make an end of suffering. But it’s superfluous for this teacher to go naked, shaven, persisting in squatting, tearing out their hair and beard. For I’m living at home with my children, using sandalwood imported from \textsanskrit{Kāsi}, wearing garlands, fragrance, and makeup, and accepting gold and currency. Yet I’ll have exactly the same destiny in the next life as this teacher. What do I know or see that I should lead the spiritual life under this teacher? This negates the spiritual life.’ Realizing this, they leave disappointed. 

This\marginnote{18.1} is the fourth way that negates the spiritual life. 

These\marginnote{19.1} are the four ways that negate the spiritual life that have been explained by the Blessed One, who knows and sees, the perfected one, the fully awakened Buddha. A sensible person would, to the best of their ability, not practice such spiritual paths, and if they did practice them, they wouldn’t succeed in the system of the skillful teaching.” 

“It’s\marginnote{20.1} incredible, Mister Ānanda, it’s amazing, how these four ways that negate the spiritual life have been explained by the Buddha. But Mister Ānanda, what are the four kinds of unreliable spiritual life?”\footnote{This section deals with religious claims to truth, which sound confident but turn out to be unreliable. \textit{\textsanskrit{Anassāsika}} (“unreliable”) normally describes “conditions”, where it is a synonym of \textit{anicca} and \textit{adhuva} (\href{https://suttacentral.net/an7.66/en/sujato\#2.3}{AN 7.66:2.3}, \href{https://suttacentral.net/dn17/en/sujato\#2.16.4}{DN 17:2.16.4}, \href{https://suttacentral.net/sn15.20/en/sujato\#2.12}{SN 15.20:2.12}, \href{https://suttacentral.net/sn22.96/en/sujato\#4.14}{SN 22.96:4.14}). For its opposite \textit{\textsanskrit{assāsa}} as “certainty” see \href{https://suttacentral.net/mn11/en/sujato\#3.2}{MN 11:3.2} and \href{https://suttacentral.net/mn93/en/sujato\#6.7}{MN 93:6.7}. } 

“Sandaka,\marginnote{21.1} take a certain teacher who claims to be all-knowing and all-seeing, to know and see everything without exception, thus: ‘Knowledge and vision are constantly and continually present to me, while walking, standing, sleeping, and waking.’ They enter an empty house; they get no almsfood; a dog bites them; they encounter a wild elephant, a wild horse, and a wild cow; they ask the name and clan of a woman or man; they ask the name and path to a village or town. When asked, ‘Why is this?’ they answer: ‘I had to enter an empty house, that’s why I entered it. I had to get no almsfood, that’s why I got none. I had to get bitten by a dog, that’s why I was bitten. I had to encounter a wild elephant, a wild horse, and a wild cow, that’s why I encountered them. I had to ask the name and clan of a woman or man, that’s why I asked. I had to ask the name and path to a village or town, that’s why I asked.’\footnote{Not only did the Buddha deny this kind of strong omniscience, he took pains to point out its absurdity. } 

A\marginnote{22.1} sensible person reflects on this matter in this way: ‘This teacher makes such a claim, but they answer in such a way. This spiritual life is unreliable.’ Realizing this, they leave disappointed. 

This\marginnote{23.1} is the first kind of unreliable spiritual life. 

Furthermore,\marginnote{24.1} take another teacher who is an oral transmitter, who takes oral transmission to be the truth.\footnote{As do some brahmins (\href{https://suttacentral.net/mn95/en/sujato\#14.1}{MN 95:14.1}). } They teach by oral transmission, by the lineage of testament, by canonical authority. But when a teacher takes oral transmission to be the truth, some of that is well learned, some poorly learned, some true, and some otherwise. 

A\marginnote{25.1} sensible person reflects on this matter in this way: ‘This teacher takes oral transmission to be the truth. He teaches by oral transmission, by the lineage of testament, by canonical authority. But when a teacher takes oral transmission to be the truth, some of that is well learned, some poorly learned, some true, and some otherwise. This spiritual life is unreliable.’ Realizing this, they leave disappointed. 

This\marginnote{26.1} is the second kind of unreliable spiritual life. 

Furthermore,\marginnote{27.1} take another teacher who relies on logic and inquiry. They teach what they have worked out by logic, following a line of inquiry, expressing their own perspective. But when a teacher relies on logic and inquiry, some of that is well reasoned, some poorly reasoned, some true, and some otherwise. 

A\marginnote{28.1} sensible person reflects on this matter in this way: ‘This teacher relies on logic and inquiry. They teach what they have worked out by logic, following a line of inquiry, expressing their own perspective. But when a teacher relies on logic and inquiry, some of that is well reasoned, some poorly reasoned, some true, and some otherwise. This spiritual life is unreliable.’ Realizing this, they leave disappointed. 

This\marginnote{29.1} is the third kind of unreliable spiritual life. 

Furthermore,\marginnote{30.1} take another teacher who is dull and stupid. Because of that, whenever they’re asked a question, they resort to verbal flip-flops and endless flip-flops:\footnote{\textit{Vikkhepa} is “flip-flopping”. | \textit{\textsanskrit{Amarā}} is explained in the commentary as either “undying” or “eel-like”. However, \textit{\textsanskrit{amarā}} in the sense of “eel” is 
found only in the commentary to this term so is probably spurious. } ‘I don’t say it’s like this. I don’t say it’s like that. I don’t say it’s otherwise. I don’t say it’s not so. And I don’t deny it’s not so.’ 

A\marginnote{31.1} sensible person reflects on this matter in this way: ‘This teacher is dull and stupid. Because of that, whenever they’re asked a question, they resort to verbal flip-flops and endless flip-flops: “I don’t say it’s like this. I don’t say it’s like that. I don’t say it’s otherwise. I don’t say it’s not so. And I don’t deny it’s not so.” This spiritual life is unreliable.’ Realizing this, they leave disappointed. 

This\marginnote{32.1} is the fourth kind of unreliable spiritual life. 

These\marginnote{33.1} are the four kinds of unreliable spiritual life that have been explained by the Blessed One, who knows and sees, the perfected one, the fully awakened Buddha. A sensible person would, to the best of their ability, not practice such spiritual paths, and if they did practice them, they wouldn’t complete the system of the skillful teaching.” 

“It’s\marginnote{34.1} incredible, Mister Ānanda, it’s amazing, how these four kinds of unreliable spiritual life have been explained by the Buddha. But, Mister Ānanda, what would a teacher say and explain so that a sensible person would, to the best of their ability, practice such a spiritual path, and once practicing it, they would complete the system of the skillful teaching?” 

“Sandaka,\marginnote{35{-}42.1} it’s when a Realized One arises in the world, perfected, a fully awakened Buddha, accomplished in knowledge and conduct, holy, knower of the world, supreme guide for those who wish to train, teacher of gods and humans, awakened, blessed. … He gives up these five hindrances, corruptions of the heart that weaken wisdom. Then, quite secluded from sensual pleasures, secluded from unskillful qualities, he enters and remains in the first absorption, which has the rapture and bliss born of seclusion, while placing the mind and keeping it connected. A sensible person would, to the best of their ability, lead the spiritual life with a teacher under whom they achieve such a high distinction, and, once practicing it, they would complete the system of the skillful teaching. 

Furthermore,\marginnote{43.1} as the placing of the mind and keeping it connected are stilled, a mendicant … enters and remains in the second absorption …\footnote{\textit{Bhikkhu} here seems improbable, yet it is attested in the manuscripts. } third absorption … fourth absorption. A sensible person would, to the best of their ability, lead the spiritual life under a teacher who achieves such a high distinction, and, once practicing it, they would complete the system of the skillful teaching. 

When\marginnote{47.1} their mind has become immersed in \textsanskrit{samādhi} like this—purified, bright, flawless, rid of corruptions, pliable, workable, steady, and imperturbable—they extend it toward recollection of past lives. They recollect many kinds of past lives. That is: one, two, three, four, five, ten, twenty, thirty, forty, fifty, a hundred, a thousand, a hundred thousand rebirths; many eons of the world contracting, many eons of the world expanding, many eons of the world contracting and expanding. … They recollect their many kinds of past lives, with features and details. A sensible person would, to the best of their ability, lead the spiritual life with a teacher under whom they achieve such a high distinction, and, once practicing it, they would complete the system of the skillful teaching. 

When\marginnote{48.1} their mind has become immersed in \textsanskrit{samādhi} like this—purified, bright, flawless, rid of corruptions, pliable, workable, steady, and imperturbable—they extend it toward knowledge of the death and rebirth of sentient beings. With clairvoyance that is purified and superhuman, they see sentient beings passing away and being reborn—inferior and superior, beautiful and ugly, in a good place or a bad place. … They understand how sentient beings are reborn according to their deeds. A sensible person would, to the best of their ability, lead the spiritual life with a teacher under whom they achieve such a high distinction, and, once practicing it, they would complete the system of the skillful teaching. 

When\marginnote{49.1} their mind has become immersed in \textsanskrit{samādhi} like this—purified, bright, flawless, rid of corruptions, pliable, workable, steady, and imperturbable—they extend it toward knowledge of the ending of defilements. They truly understand: ‘This is suffering’ … ‘This is the origin of suffering’ … ‘This is the cessation of suffering’ … ‘This is the practice that leads to the cessation of suffering’. They truly understand: ‘These are defilements’ … ‘This is the origin of defilements’ … ‘This is the cessation of defilements’ … ‘This is the practice that leads to the cessation of defilements’. 

Knowing\marginnote{50.1} and seeing like this, their mind is freed from the defilements of sensuality, desire to be reborn, and ignorance. When they’re freed, they know they’re freed. 

They\marginnote{50.3} understand: ‘Rebirth is ended, the spiritual journey has been completed, what had to be done has been done, there is nothing further for this place.’ A sensible person would, to the best of their ability, lead the spiritual life with a teacher under whom they achieve such a high distinction, and, once practicing it, they would complete the system of the skillful teaching.” 

“But\marginnote{51.1} Mister Ānanda, when a mendicant is perfected—with defilements ended, who has completed the spiritual journey, done what had to be done, laid down the burden, achieved their own true goal, utterly ended the fetter of continued existence, and is rightly freed through enlightenment—could they still enjoy sensual pleasures?” 

“Sandaka,\marginnote{51.2} a mendicant who is perfected—with defilements ended, who has completed the spiritual journey, done what had to be done, laid down the burden, achieved their own true goal, utterly ended the fetter of continued existence, and is rightly freed through enlightenment—can’t transgress in five respects. A mendicant with defilements ended can’t deliberately take the life of a living creature, take something with the intention to steal, have sex, tell a deliberate lie, or store up goods for their own enjoyment like they used to as a lay person. A mendicant who is perfected can’t transgress in these five respects.” 

“But\marginnote{52.1} Mister Ānanda, when a mendicant is perfected, would the knowledge and vision that their defilements are ended be constantly and continually present to them, while walking, standing, sleeping, and waking?” 

“Well\marginnote{52.3} then, Sandaka, I shall give you a simile. For by means of a simile some sensible people understand the meaning of what is said. Suppose there was a person whose hands and feet had been amputated. Would they be aware that their hands and feet had been amputated constantly and continually, while walking, standing, sleeping, and waking? Or would they be aware of it only when they checked it?” 

“They\marginnote{52.9} wouldn’t be aware of it constantly, only when they checked it.” 

“In\marginnote{52.13} the same way, when a mendicant is perfected, the knowledge and vision that their defilements are ended is not constantly and continually present to them, while walking, standing, sleeping, and waking. Rather, they are aware of it only when they checked it.” 

“But\marginnote{53.1} Reverend Ānanda, in this teaching and training, how many are emancipated?”\footnote{“(Those who are) emancipated” is \textit{\textsanskrit{niyyātāro}}, a unique occurrence of an agent noun from \textit{\textsanskrit{niyyāti}}, “to go out”. This was probably a technical term in Sandaka’s system. } 

“There\marginnote{53.2} are not just one hundred who are emancipated, Sandaka, or two or three or four or five hundred, but many more than that in this teaching and training.” 

“It’s\marginnote{53.3} incredible, Mister Ānanda, it’s amazing! Namely, that there’s no glorifying one’s own teaching and putting down the teaching of others. The Dhamma is taught in its own field, and so many emancipated are recognized. But these \textsanskrit{Ājīvakas}, sons of she whose sons are dead, glorify themselves and put others down. And they only recognize three who have been emancipated:\footnote{The commentary takes \textit{mata} as “dead” and the compound \textit{\textsanskrit{puttamatāya}} as genitive feminine, yielding the sense, “she whose sons are dead”, explaining that Sandaka thought, “Those \textsanskrit{Ājīvakas} are surely dead; their mother is she whose sons are dead”. Perhaps, however, “she whose sons are dead” is an allusion to a child-killing \textit{\textsanskrit{yakkhinī}} such as \textsanskrit{Hārītī} of \textsanskrit{Madhurā} or \textsanskrit{Kālī} in the commentary to \href{https://suttacentral.net/dhp5/en/sujato}{Dhp 5}, thus having the force, “sons of a she-devil”. Either way this is unsatisfactory, a curse or slur intruding into a passage whose intent is to highlight the Buddha’s fair-minded treatment of others. Nonetheless, lacking a compelling alternative I follow the commentary. } Nanda Vaccha, Kisa \textsanskrit{Saṅkicca}, and the bamboo-staffed ascetic \textsanskrit{Gosāla}.” 

Then\marginnote{54.1} the wanderer Sandaka addressed his own assembly, “Go, good sirs. The spiritual life is lived under the ascetic Gotama. It’s not easy for me to forsake possessions, honor, or popularity now.” And that’s how the wanderer Sandaka sent his own assembly to lead the spiritual life under the Buddha. 

%
\section*{{\suttatitleacronym MN 77}{\suttatitletranslation The Longer Discourse with Sakuludāyī }{\suttatitleroot Mahāsakuludāyisutta}}
\addcontentsline{toc}{section}{\tocacronym{MN 77} \toctranslation{The Longer Discourse with Sakuludāyī } \tocroot{Mahāsakuludāyisutta}}
\markboth{The Longer Discourse with Sakuludāyī }{Mahāsakuludāyisutta}
\extramarks{MN 77}{MN 77}

\scevam{So\marginnote{1.1} I have heard.\footnote{Beginning as a discussion on why certain teachers are given due respect, the Buddha creates one of the most comprehensive accounts of his system of meditation, drawing in almost all the different perspectives and approaches to \textit{\textsanskrit{samādhi}}. } }At one time the Buddha was staying near \textsanskrit{Rājagaha}, in the Bamboo Grove, the squirrels’ feeding ground. 

Now\marginnote{2.1} at that time several very well-known wanderers were residing in the monastery of the wanderers in the peacocks’ feeding ground. They included \textsanskrit{Annabhāra}, Varadhara, \textsanskrit{Sakuludāyī}, and other very well-known wanderers.\footnote{The Buddha addresses these wanderers at \href{https://suttacentral.net/an4.30/en/sujato}{AN 4.30}, where he teaches four fundamental principles of any holy life, and at \href{https://suttacentral.net/an4.185/en/sujato}{AN 4.185}, where he teaches four “truths of the brahmins”. These are all respectful engagements that highlight common ground. | \textsanskrit{Annabhāra} means “food-carrier” and indicates a humble origin (\href{https://suttacentral.net/thag16.9/en/sujato\#19.1}{Thag 16.9:19.1}). | Varadhara means “bearer of good fortune”; it is probably a name for the earth. | \textsanskrit{Sakuludāyī} also appears in \href{https://suttacentral.net/mn79/en/sujato}{MN 79}. His name (“one who rises up with his own family”) is ironic, given that his attempt to go forth is frustrated by his followers at \href{https://suttacentral.net/mn79/en/sujato\#46.2}{MN 79:46.2}. } 

Then\marginnote{3.1} the Buddha robed up in the morning and, taking his bowl and robe, entered \textsanskrit{Rājagaha} for alms. Then it occurred to him, “It’s too early to wander for alms in \textsanskrit{Rājagaha}. Why don’t I visit the wanderer \textsanskrit{Sakuludāyī} at the monastery of the wanderers in the peacocks’ feeding ground?” 

So\marginnote{4.1} the Buddha went to the monastery of the wanderers. 

Now\marginnote{4.2} at that time, \textsanskrit{Sakuludāyī} was sitting together with a large assembly of wanderers making an uproar, a dreadful racket. They engaged in all kinds of low talk, such as talk about kings, bandits, and chief ministers; talk about armies, threats, and wars; talk about food, drink, clothes, and beds; talk about garlands and fragrances; talk about family, vehicles, villages, towns, cities, and countries; talk about women and heroes; street talk and well talk; talk about the departed; motley talk; tales of land and sea; and talk about being reborn in this or that state of existence. 

\textsanskrit{Sakuludāyī}\marginnote{4.4} saw the Buddha coming off in the distance, and hushed his own assembly, “Be quiet, good sirs, don’t make a sound. Here comes the ascetic Gotama. The venerable likes quiet and praises quiet. Hopefully if he sees that our assembly is quiet he’ll see fit to approach.” Then those wanderers fell silent. 

Then\marginnote{5.1} the Buddha approached \textsanskrit{Sakuludāyī}, who said to him, “Let the Blessed One come, sir! Welcome to the Blessed One, sir! It’s been a long time since you took the opportunity to come here. Please, sir, sit down, this seat is ready.” The Buddha sat on the seat spread out, while \textsanskrit{Sakuludāyī} took a low seat and sat to one side. 

The\marginnote{5.10} Buddha said to him, “\textsanskrit{Udāyī}, what were you sitting talking about just now? What conversation was left unfinished?” 

“Sir,\marginnote{6.1} leave aside what we were sitting talking about just now. It won’t be hard for you to hear about that later. 

Sir,\marginnote{6.3} a few days ago several ascetics and brahmins who follow various other religions were sitting together at the debating hall, and this discussion came up among them: ‘The people of \textsanskrit{Aṅga} and Magadha are so fortunate, so very fortunate! For there are these ascetics and brahmins who lead an order and a community, and tutor a community. They’re well-known and famous religious founders, deemed holy by many people. And they have come down for the rainy season residence at \textsanskrit{Rājagaha}. They include \textsanskrit{Pūraṇa} Kassapa, the bamboo-staffed ascetic \textsanskrit{Gosāla}, Ajita of the hair blanket, Pakudha \textsanskrit{Kaccāyana}, \textsanskrit{Sañjaya} \textsanskrit{Belaṭṭhiputta}, and the Jain ascetic of the \textsanskrit{Ñātika} clan. This ascetic Gotama also leads an order and a community, and tutors a community. He’s a well-known and famous religious founder, deemed holy by many people. And he too has come down for the rains residence at \textsanskrit{Rājagaha}. Which of these ascetics and brahmins is honored, respected, revered, and venerated by their disciples? And how do their disciples, honoring and respecting them, remain loyal?’ 

Some\marginnote{6.17} of them said: ‘This \textsanskrit{Pūraṇa} Kassapa leads an order and a community, and tutors a community. He’s a well-known and famous religious founder, deemed holy by many people. But he’s not honored, respected, revered, and venerated by his disciples. And his disciples, not honoring and respecting him, don’t remain loyal to him. Once it so happened that he was teaching an assembly of many hundreds. Then one of his disciples made a noise, “My good sirs, don’t ask \textsanskrit{Pūraṇa} Kassapa about that. He doesn’t know that. I know it. Ask me about it, and I’ll answer you.” Once it so happened that \textsanskrit{Pūraṇa} Kassapa didn’t get his way, though he called out with raised arms, “Be quiet, good sirs, don’t make a sound. They’re not asking you, they’re asking me! I’ll answer you!” Indeed, many of his disciples have left him after refuting his doctrine: “You don’t understand this teaching and training. I understand this teaching and training. What, you understand this teaching and training? You’re practicing wrong. I’m practicing right. I stay on topic, you don’t. You said last what you should have said first. You said first what you should have said last. What you’ve thought so much about has been disproved. Your doctrine is refuted. Go on, save your doctrine! You’re trapped; get yourself out of this—if you can!” That’s how \textsanskrit{Pūraṇa} Kassapa is not honored, respected, revered, venerated, and esteemed by his disciples. On the contrary, his disciples, not honoring and respecting him, don’t remain loyal to him. Rather, he’s reviled, and rightly so.’\footnote{Taking \textit{dhamma} here in the sense of “right, legitimate”, rather than “teaching”, as the example criticizes him not his teaching. } 

Others\marginnote{6.34} said: 'This bamboo-staffed ascetic \textsanskrit{Gosāla} … Ajita of the hair blanket … Pakudha \textsanskrit{Kaccāyana} … \textsanskrit{Sañjaya} \textsanskrit{Belaṭṭhiputta} … The Jain ascetic of the \textsanskrit{Ñātika} clan leads an order and a community, and tutors a community. He’s a well-known and famous religious founder, deemed holy by many people. But he’s not honored, respected, revered, and venerated by his disciples. And his disciples, not honoring and respecting him, don’t remain loyal to him. Once it so happened that he was teaching an assembly of many hundreds. Then one of his disciples made a noise, “My good sirs, don’t ask the Jain \textsanskrit{Ñātika} about that. He doesn’t know that. I know it. Ask me about it, and I’ll answer you.” Once it so happened that the Jain \textsanskrit{Ñātika} didn’t get his way, though he called out with raised arms, “Be quiet, good sirs, don’t make a sound. They’re not asking you, they’re asking me! I’ll answer you!” Indeed, many of his disciples have left him after refuting his doctrine: “You don’t understand this teaching and training. I understand this teaching and training. What, you understand this teaching and training? You’re practicing wrong. I’m practicing right. I stay on topic, you don’t. You said last what you should have said first. You said first what you should have said last. What you’ve thought so much about has been disproved. Your doctrine is refuted. Go on, save your doctrine! You’re trapped; get yourself out of this—if you can!” That’s how the Jain \textsanskrit{Ñātika} is not honored, respected, revered, and venerated by his disciples. On the contrary, his disciples, not honoring and respecting him, don’t remain loyal to him. Rather, he’s reviled, and rightly so.’ 

Others\marginnote{6.55} said: ‘This ascetic Gotama leads an order and a community, and tutors a community. He’s a well-known and famous religious founder, deemed holy by many people. He’s honored, respected, revered, and venerated by his disciples. And his disciples, honoring and respecting him, remain loyal to him. Once it so happened that he was teaching an assembly of many hundreds. Then one of his disciples cleared their throat. And one of their spiritual companions nudged them with their knee, to indicate, “Hush, venerable, don’t make sound! Our teacher, the Blessed One, is teaching!” While the ascetic Gotama is teaching an assembly of many hundreds, there is no sound of his disciples coughing or clearing their throats. That large crowd is poised expectantly at the ready, thinking, “Whatever the Buddha teaches, we shall listen to it.” It’s like when there’s a person at the crossroads pressing out pure dwarf-bee honey,\footnote{\textit{\textsanskrit{Khuddā}} (literally “small one”) is said to be a species of small bee, also known in Sanskrit as \textit{\textsanskrit{kṣudrā}}. The “dwarf bee” (\textit{apis florea}) fits the bill, as it is a small wild honeybee found in India. } and a large crowd is poised expectantly at the ready. In the same way, while the ascetic Gotama is teaching an assembly of many hundreds, there is no sound of his disciples coughing or clearing their throats. That large crowd is poised expectantly at the ready, thinking, “Whatever the Buddha teaches, we shall listen to it.” Even when disciples of the ascetic Gotama, having clashed with their spiritual companions, reject the training and return to a lesser life, they speak only praise of the teacher, the teaching, and the \textsanskrit{Saṅgha}. They blame only themselves, not others: “We were unlucky, we had little merit. For even after going forth in such a well explained teaching and training we weren’t able to practice for life the perfectly full and pure spiritual life.”\footnote{For “having clashed” (\textit{\textsanskrit{sampayojetvā}}), see \href{https://suttacentral.net/sn11.24/en/sujato\#1.2}{SN 11.24:1.2}. } They become monastery workers or lay followers, and they proceed having undertaken the five precepts. That’s how the ascetic Gotama is honored, respected, revered, and venerated by his disciples. And that’s how his disciples, honoring and respecting him, remain loyal to him.’” 

“But\marginnote{7.1} \textsanskrit{Udāyī}, how many qualities do you see in me, because of which my disciples honor, respect, revere, and venerate me; and honoring and respecting me, they remain loyal to me?” 

“Sir,\marginnote{8.1} I see five such qualities in the Buddha.\footnote{This is an example of what the Buddha says in \href{https://suttacentral.net/dn1/en/sujato\#1.7.1}{DN 1:1.7.1}, “When an ordinary person speaks praise of the Realized One, they speak only of trivial, insignificant details of mere ethics.” } What five? 

The\marginnote{8.3} Buddha eats little and praises eating little. This is the first such quality I see in the Buddha. 

Furthermore,\marginnote{8.5} the Buddha is content with any kind of robe, and praises such contentment. This is the second such quality I see in the Buddha. 

Furthermore,\marginnote{8.7} the Buddha is content with any kind of almsfood, and praises such contentment. This is the third such quality I see in the Buddha. 

Furthermore,\marginnote{8.9} the Buddha is content with any kind of lodging, and praises such contentment. This is the fourth such quality I see in the Buddha. 

Furthermore,\marginnote{8.11} the Buddha is secluded, and praises seclusion. This is the fifth such quality I see in the Buddha. 

These\marginnote{8.13} are the five qualities I see in the Buddha, because of which his disciples honor, respect, revere, and venerate him; and honoring and respecting him, they remain loyal to him.” 

“Suppose,\marginnote{9.1} \textsanskrit{Udāyī}, my disciples were loyal to me because I eat little. Well, there are disciples of mine who eat a tumbler of food, or half a tumbler; they eat a wood apple, or half a wood apple. But sometimes I even eat this bowl full to the brim, or even more. So if it were the case that my disciples are loyal to me because I eat little, then those disciples who eat even less would not be loyal to me. 

Suppose\marginnote{9.4} my disciples were loyal to me because I’m content with any kind of robe. Well, there are disciples of mine who have rag robes, wearing shabby robes. They gather scraps from charnel grounds, rubbish dumps, and shops, make them into a patchwork robe and wear it. But sometimes I wear robes offered by householders that are strong, yet next to which bottle-gourd down is coarse.\footnote{For \textsanskrit{Mahāsaṅgīti} \textit{\textsanskrit{satthalūkhāni}} accept PTS and BJT reading \textit{yattha \textsanskrit{lūkhāni}}. | \textit{\textsanskrit{Alābu}} is “bottle gourd”, which is covered in a fine down. } So if it were the case that my disciples are loyal to me because I’m content with any kind of robe, then those disciples who wear rag robes would not be loyal to me. 

Suppose\marginnote{9.7} my disciples were loyal to me because I’m content with any kind of almsfood. Well, there are disciples of mine who eat only almsfood, wander indiscriminately for almsfood, happy to eat whatever they glean. When they’ve entered an inhabited area, they don’t consent when invited to sit down. But sometimes I even eat by invitation boiled fine rice with the dark grains picked out, served with many soups and sauces. So if it were the case that my disciples are loyal to me because I’m content with any kind of almsfood, then those disciples who eat only almsfood would not be loyal to me. 

Suppose\marginnote{9.10} my disciples were loyal to me because I’m content with any kind of lodging. Well, there are disciples of mine who stay at the root of a tree, in the open air. For eight months they don’t go under a roof. But sometimes I even stay in bungalows, plastered inside and out, draft-free, with doors fastened and windows shuttered. So if it were the case that my disciples are loyal to me because I’m content with any kind of lodging, then those disciples who stay at the root of a tree would not be loyal to me. 

Suppose\marginnote{9.13} my disciples were loyal to me because I’m secluded and I praise seclusion. Well, there are disciples of mine who live in the wilderness, in remote lodgings. Having ventured deep into remote lodgings in the wilderness and the forest, they live there, coming down to the midst of the \textsanskrit{Saṅgha} each fortnight for the recitation of the monastic code. But sometimes I live crowded by monks, nuns, laymen, and laywomen; by rulers and their chief ministers, and monastics of other religions and their disciples. So if it were the case that my disciples are loyal to me because I’m secluded and praise seclusion, then those disciples who live in the wilderness would not be loyal to me. 

So,\marginnote{9.16} \textsanskrit{Udāyī}, it’s not because of these five qualities that my disciples honor, respect, revere, and venerate me; and honoring and respecting me, they remain loyal to me. 

There\marginnote{10.1} are five other qualities because of which my disciples honor, respect, revere, and venerate me; and honoring and respecting me, they remain loyal to me. What five? 

Firstly,\marginnote{10.3} my disciples esteem me for the higher ethics: ‘The ascetic Gotama is ethical. He possesses the entire spectrum of ethical conduct to the highest degree.’ Since this is so, this is the first quality because of which my disciples are loyal to me. 

Furthermore,\marginnote{12.1} my disciples esteem me for my excellent knowledge and vision: ‘The ascetic Gotama only claims to know when he does in fact know. He only claims to see when he really does see. He teaches based on direct knowledge, not without direct knowledge. He teaches based on reason, not without reason. He teaches with a demonstrable basis, not without it.’\footnote{This is a good example of why \textit{\textsanskrit{sappāṭihāriya}} means “with demonstrable basis”, not “with miracles”. } Since this is so, this is the second quality because of which my disciples are loyal to me. 

Furthermore,\marginnote{13.1} my disciples esteem me for my higher wisdom: ‘The ascetic Gotama is wise. He possesses the entire spectrum of wisdom to the highest degree. It’s not possible that he would fail to foresee grounds for future criticism, or to legitimately and completely refute the doctrines of others that come up.’\footnote{\textit{\textsanskrit{Vādapatha}} is found here and at \href{https://suttacentral.net/an4.8/en/sujato\#5.1}{AN 4.8:5.1} in the sense of “criticism”. Compare \textit{vacanapatha}, which is always used in the sense of “harsh words”, “criticism” (eg. \href{https://suttacentral.net/an4.114/en/sujato\#9.2}{AN 4.114:9.2}). Compare English “give a talking to”, “have words with”. At \href{https://suttacentral.net/snp5.7/en/sujato\#8.5}{Snp 5.7:8.5}, however, \textit{\textsanskrit{vādapatha}} is used in the sense “ways of speech”. } What do you think, \textsanskrit{Udāyī}? Would my disciples, knowing and seeing this, break in and interrupt me?” 

“No,\marginnote{13.6} sir.” 

“That’s\marginnote{13.7} because I don’t expect to be instructed by my disciples. Invariably, my disciples expect instruction from me. 

Since\marginnote{13.9} this is so, this is the third quality because of which my disciples are loyal to me. 

Furthermore,\marginnote{14.1} my disciples come to me and ask how the noble truth of suffering applies to the suffering in which they are swamped and mired. And I provide them with a satisfying answer to their question.\footnote{The Buddha does not merely teach a satisfying theory, he applies it to address the specific suffering with which individuals are afflicted (\textit{yena dukkhena \textsanskrit{dukkhotiṇṇā}}). } They ask how the noble truths of the origin of suffering, the cessation of suffering, and the practice that leads to the cessation of suffering apply to the suffering  in which they are swamped and mired. And I provide them with satisfying answers to their questions. Since this is so, this is the fourth quality because of which my disciples are loyal to me. 

Furthermore,\marginnote{15.1} I have explained to my disciples a practice that they use to develop the four kinds of mindfulness meditation.\footnote{The fifth quality as presented here constitutes one of the most comprehensive presentations of meditation in any early Buddhist text. The Chinese parallel (MA 207 at T i 783b16) simply says that the Buddha taught disciples to transcend doubt and reach the far shore, which makes the same point far more economically. It seems likely that the Pali text has been expanded. | From here through to \href{https://suttacentral.net/mn77/en/sujato\#21.3}{MN 77:21.3} the Buddha presents the seven sets of practices later renowned as the 37 “qualities that lead to awakening” (\textit{\textsanskrit{bodhipakkhiyadhammā}}). } It’s when a mendicant meditates by observing an aspect of the body—keen, aware, and mindful, rid of covetousness and displeasure for the world. They meditate observing an aspect of feelings … mind … principles—keen, aware, and mindful, rid of covetousness and displeasure for the world. And many of my disciples meditate on that having attained perfection and consummation of insight.\footnote{“Perfection” is \textit{\textsanskrit{pāramī}}, here a term for arahantship. The concept of “perfections” as a set of qualities developed by a Bodhisattva over many lifetimes is not found in any early text. It was introduced several centuries after the Buddha in the \textsanskrit{Cariyāpiṭaka} (\href{https://suttacentral.net/cp1/en/sujato}{Cp 1}) and throughout the latest texts of the Pali canon (\href{https://suttacentral.net/bv1/en/sujato\#78.1}{Bv 1:78.1}, \href{https://suttacentral.net/thi-ap28/en/sujato\#64.2}{Thi Ap 28:64.2}, \href{https://suttacentral.net/kp8/en/sujato\#15.2}{Kp 8:15.2}, \href{https://suttacentral.net/pli-tv-pvr3/en/sujato\#3.2}{Pvr 3:3.2}, \href{https://suttacentral.net/mil5.1.4/en/sujato\#13.1}{Mil 5.1.4:13.1}). Similar conceptions were meanwhile developing in other schools, eventually becoming a key doctrine of \textsanskrit{Mahāyāna}. } 

Furthermore,\marginnote{16.1} I have explained to my disciples a practice that they use to develop the four right efforts. It’s when a mendicant generates enthusiasm, tries, makes an effort, exerts the mind, and strives so that bad, unskillful qualities don’t arise. They generate enthusiasm, try, make an effort, exert the mind, and strive so that bad, unskillful qualities that have arisen are given up. They generate enthusiasm, try, make an effort, exert the mind, and strive so that skillful qualities arise. They generate enthusiasm, try, make an effort, exert the mind, and strive so that skillful qualities that have arisen remain, are not lost, but increase, mature, and are fulfilled by development. And many of my disciples meditate on that having attained perfection and consummation of insight. 

Furthermore,\marginnote{17.1} I have explained to my disciples a practice that they use to develop the four bases of psychic power. It’s when a mendicant develops the basis of psychic power that has immersion due to enthusiasm, and active effort. They develop the basis of psychic power that has immersion due to energy, and active effort. They develop the basis of psychic power that has immersion due to mental development, and active effort. They develop the basis of psychic power that has immersion due to inquiry, and active effort. And many of my disciples meditate on that having attained perfection and consummation of insight. 

Furthermore,\marginnote{18.1} I have explained to my disciples a practice that they use to develop the five faculties. It’s when a mendicant develops the faculties of faith, energy, mindfulness, immersion, and wisdom, which lead to peace and awakening. And many of my disciples meditate on that having attained perfection and consummation of insight. 

Furthermore,\marginnote{19.1} I have explained to my disciples a practice that they use to develop the five powers. It’s when a mendicant develops the powers of faith, energy, mindfulness, immersion, and wisdom, which lead to peace and awakening. And many of my disciples meditate on that having attained perfection and consummation of insight. 

Furthermore,\marginnote{20.1} I have explained to my disciples a practice that they use to develop the seven awakening factors. It’s when a mendicant develops the awakening factors of mindfulness, investigation of principles, energy, rapture, tranquility, immersion, and equanimity, which rely on seclusion, fading away, and cessation, and ripen as letting go. And many of my disciples meditate on that having attained perfection and consummation of insight. 

Furthermore,\marginnote{21.1} I have explained to my disciples a practice that they use to develop the noble eightfold path. It’s when a mendicant develops right view, right thought, right speech, right action, right livelihood, right effort, right mindfulness, and right immersion. And many of my disciples meditate on that having attained perfection and consummation of insight. 

Furthermore,\marginnote{22.1} I have explained to my disciples a practice that they use to develop the eight liberations.\footnote{The eight liberations (\textit{\textsanskrit{vimokkhā}}) are an alternative way of describing the meditative experiences of \textit{\textsanskrit{jhāna}}. Elsewhere they are listed at \href{https://suttacentral.net/dn15/en/sujato\#35.1}{DN 15:35.1}, \href{https://suttacentral.net/dn16/en/sujato\#3.33.1}{DN 16:3.33.1}, \href{https://suttacentral.net/dn33/en/sujato\#3.1.168}{DN 33:3.1.168}, \href{https://suttacentral.net/dn34/en/sujato\#2.1.191}{DN 34:2.1.191}, \href{https://suttacentral.net/an/en/sujato\#8.66}{AN:8.66}, and referred to at \href{https://suttacentral.net/an4.189/en/sujato\#1.8}{AN 4.189:1.8} and \href{https://suttacentral.net/thag20.1/en/sujato\#33.1}{Thag 20.1:33.1}. At \href{https://suttacentral.net/an8.120/en/sujato}{AN 8.120} and \href{https://suttacentral.net/mn137/en/sujato\#27.1}{MN 137:27.1} they are listed but not called the eight liberations. } 

Having\marginnote{22.2} physical form, they see forms.\footnote{Someone sees a meditative vision based on the perception of their own body, such as through mindfulness of breathing or one’s own body parts. Such “visions” or “forms” (\textit{\textsanskrit{rūpā}}) are the lights or other meditation phenomena that today are sometimes called \textit{nimitta}. | The first three liberations all cover the four \textit{\textsanskrit{jhānas}}. } This is the first liberation. 

Not\marginnote{22.4} perceiving form internally, they see forms externally.\footnote{A meditator grounds their practice on some external focus, such as a light, the sight of a corpse, or an external element such as earth. } This is the second liberation. 

They’re\marginnote{22.6} focused only on beauty.\footnote{This is a practice based on wholly pure and exalted meditation, such as the meditation on love, or the sight of a pure brilliant color like the sky. } This is the third liberation. 

Going\marginnote{22.8} totally beyond perceptions of form, with the ending of perceptions of impingement, not focusing on perceptions of diversity, aware that ‘space is infinite’, they enter and remain in the dimension of infinite space. This is the fourth liberation. 

Going\marginnote{22.10} totally beyond the dimension of infinite space, aware that ‘consciousness is infinite’, they enter and remain in the dimension of infinite consciousness. This is the fifth liberation. 

Going\marginnote{22.12} totally beyond the dimension of infinite consciousness, aware that ‘there is nothing at all’, they enter and remain in the dimension of nothingness. This is the sixth liberation. 

Going\marginnote{22.14} totally beyond the dimension of nothingness, they enter and remain in the dimension of neither perception nor non-perception. This is the seventh liberation. 

Going\marginnote{22.16} totally beyond the dimension of neither perception nor non-perception, they enter and remain in the cessation of perception and feeling. This is the eighth liberation. 

And\marginnote{22.18} many of my disciples meditate on that having attained perfection and consummation of insight. 

Furthermore,\marginnote{23.1} I have explained to my disciples a practice that they use to develop the eight dimensions of mastery.\footnote{These are another way of describing the different experiences of \textit{\textsanskrit{jhāna}}. Also at \href{https://suttacentral.net/an8.65/en/sujato}{AN 8.65}, \href{https://suttacentral.net/an10.29/en/sujato\#6.1}{AN 10.29:6.1}, \href{https://suttacentral.net/dn16/en/sujato\#3.24.1}{DN 16:3.24.1}, \href{https://suttacentral.net/dn33/en/sujato\#3.1.142}{DN 33:3.1.142}, and \href{https://suttacentral.net/dn34/en/sujato\#2.1.160}{DN 34:2.1.160}. } 

Perceiving\marginnote{23.2} form internally, someone sees forms externally, limited, both pretty and ugly.\footnote{An “ugly” form is the mental image that arises in such contemplations as the parts of the body. A “beautiful” image arises from practices such as mindfulness of breathing. } Mastering them, they perceive: ‘I know and see.’ This is the first dimension of mastery. 

Perceiving\marginnote{23.5} form internally, someone sees forms externally, limitless, both pretty and ugly. Mastering them, they perceive: ‘I know and see.’ This is the second dimension of mastery. 

Not\marginnote{23.8} perceiving form internally, someone sees forms externally, limited, both pretty and ugly. Mastering them, they perceive: ‘I know and see.’ This is the third dimension of mastery. 

Not\marginnote{23.11} perceiving form internally, someone sees forms externally, limitless, both pretty and ugly. Mastering them, they perceive: ‘I know and see.’ This is the fourth dimension of mastery. 

Not\marginnote{23.14} perceiving form internally, someone sees forms externally, blue, with blue color and blue appearance.\footnote{This is the meditation where one contemplates an external color, either a prepared disk or a natural phenomena such as the sky or a flower, which eventually gives rise to a “counterpart” image. Today such meditations are called \textit{\textsanskrit{kasiṇa}} following the Visuddhimagga, but for the original sense of \textit{\textsanskrit{kasiṇa}} see below (\href{https://suttacentral.net/mn77/en/sujato\#24.1}{MN 77:24.1}). } They’re like a flax flower that’s blue, with blue color and blue appearance. Or a cloth from Varanasi that’s smoothed on both sides, blue, with blue color and blue appearance. In the same way, not perceiving form internally, someone sees forms externally, blue, with blue color and blue appearance. Mastering them, they perceive: ‘I know and see.’ This is the fifth dimension of mastery. 

Not\marginnote{23.19} perceiving form internally, someone sees forms externally that are yellow, with yellow color and yellow appearance. They’re like a champak flower that’s yellow, with yellow color and yellow appearance. Or a cloth from Varanasi that’s smoothed on both sides, yellow, with yellow color and yellow appearance. In the same way, not perceiving form internally, someone sees forms externally that are yellow, with yellow color and yellow appearance. Mastering them, they perceive: ‘I know and see.’ This is the sixth dimension of mastery. 

Not\marginnote{23.24} perceiving form internally, someone sees forms externally that are red, with red color and red appearance. They’re like a scarlet mallow flower that’s red, with red color and red appearance. Or a cloth from Varanasi that’s smoothed on both sides, red, with red color and red appearance. In the same way, not perceiving form internally, someone sees forms externally that are red, with red color and red appearance. Mastering them, they perceive: ‘I know and see.’ This is the seventh dimension of mastery. 

Not\marginnote{23.29} perceiving form internally, someone sees forms externally that are white, with white color and white appearance. They’re like the morning star that’s white, with white color and white appearance. Or a cloth from Varanasi that’s smoothed on both sides, white, with white color and white appearance. In the same way, not perceiving form internally, someone sees forms externally that are white, with white color and white appearance. Mastering them, they perceive: ‘I know and see.’ This is the eighth dimension of mastery. 

And\marginnote{23.34} many of my disciples meditate on that having attained perfection and consummation of insight. 

Furthermore,\marginnote{24.1} I have explained to my disciples a practice that they use to develop the ten universal dimensions of meditation. 

Someone\marginnote{24.2} perceives the meditation on universal earth above, below, across, undivided and limitless. 

They\marginnote{24.3} perceive the meditation on universal water … the meditation on universal fire … the meditation on universal air … the meditation on universal blue … the meditation on universal yellow … the meditation on universal red … the meditation on universal white … the meditation on universal space … the meditation on universal consciousness above, below, across, undivided and limitless.\footnote{That Pali \textit{\textsanskrit{kasiṇa}} (Sanskrit \textit{\textsanskrit{kṛtsna}}) means “universal”, “totality” is shown by this passage where it is “undivided and limitless”. It is a name for the state of absorption, not for the meditation disk used in preliminary practice. \textsanskrit{Yājñavalkya} says that, just as salt is “entirely” salty, the Self is an “entire mass of consciousness” (\textit{\textsanskrit{kṛtsnaḥ} \textsanskrit{prajñānaghana} eva}, \textsanskrit{Bṛhadāraṇyaka} \textsanskrit{Upaniṣad} 4.5.13). } 

And\marginnote{24.12} many of my disciples meditate on that having attained perfection and consummation of insight. 

Furthermore,\marginnote{25.1} I have explained to my disciples a practice that they use to develop the four absorptions.\footnote{From here, the text introduces the absorptions and higher knowledges, illustrated as in \href{https://suttacentral.net/dn2/en/sujato}{DN 2} with a few differing similes. The absorptions and three higher knowledges also appear with similes at \href{https://suttacentral.net/mn39/en/sujato}{MN 39}. } 

It’s\marginnote{25.2} when a mendicant, quite secluded from sensual pleasures, secluded from unskillful qualities, enters and remains in the first absorption, which has the rapture and bliss born of seclusion, while placing the mind and keeping it connected. They drench, steep, fill, and spread their body with rapture and bliss born of seclusion. There’s no part of the body that’s not spread with rapture and bliss born of seclusion.\footnote{As a meditator proceeds, their subjective experience of the “body” evolves from tactile sense impressions (\textit{\textsanskrit{phoṭṭhabba}}), to the interior mental experience of bliss and light (\textit{\textsanskrit{manomayakāya}}), to the direct personal realization of highest truth (\href{https://suttacentral.net/mn70/en/sujato\#23.2}{MN 70:23.2}: \textit{\textsanskrit{kāyena} ceva \textsanskrit{paramasaccaṁ} sacchikaroti}). } It’s like when a deft bathroom attendant or their apprentice pours bath powder into a bronze dish, sprinkling it little by little with water. They knead it until the ball of bath powder is soaked and saturated with moisture, spread through inside and out; yet no moisture oozes out.\footnote{The kneading is the “placing the mind and keeping it connected”, the water is bliss, while the lack of leaking speaks to the contained interiority of the experience. } In the same way, a mendicant drenches, steeps, fills, and spreads their body with rapture and bliss born of seclusion. There’s no part of the body that’s not spread with rapture and bliss born of seclusion. 

Furthermore,\marginnote{26.1} as the placing of the mind and keeping it connected are stilled, a mendicant enters and remains in the second absorption. It has the rapture and bliss born of immersion, with internal clarity and mind at one, without placing the mind and keeping it connected. They drench, steep, fill, and spread their body with rapture and bliss born of immersion. There’s no part of the body that’s not spread with rapture and bliss born of immersion. It’s like a deep lake fed by spring water. There’s no inlet to the east, west, north, or south, and the heavens would not properly bestow showers from time to time.\footnote{Again the simile emphasizes the water as bliss, while the lack of inflow expresses containment and unification. } But the stream of cool water welling up in the lake drenches, steeps, fills, and spreads throughout the lake. There’s no part of the lake that’s not spread through with cool water.\footnote{The water welling up is the rapture, which is the uplifting emotional response to the experience of bliss. } In the same way, a mendicant drenches, steeps, fills, and spreads their body with rapture and bliss born of immersion. There’s no part of the body that’s not spread with rapture and bliss born of immersion. 

Furthermore,\marginnote{27.1} with the fading away of rapture, a mendicant enters and remains in the third absorption. They meditate with equanimity, mindful and aware, personally experiencing the bliss of which the noble ones declare, ‘Equanimous and mindful, one meditates in bliss.’ They drench, steep, fill, and spread their body with bliss free of rapture. There’s no part of the body that’s not spread with bliss free of rapture. It’s like a pool with blue water lilies, or pink or white lotuses. Some of them sprout and grow in the water without rising above it, thriving underwater. From the tip to the root they’re drenched, steeped, filled, and soaked with cool water. There’s no part of them that’s not soaked with cool water.\footnote{The meditator is utterly immersed in stillness and bliss. } In the same way, a mendicant drenches, steeps, fills, and spreads their body with bliss free of rapture. There’s no part of the body that’s not spread with bliss free of rapture. 

Furthermore,\marginnote{28.1} giving up pleasure and pain, and ending former happiness and sadness, a mendicant enters and remains in the fourth absorption. It is without pleasure or pain, with pure equanimity and mindfulness. They sit spreading their body through with pure bright mind. There’s no part of the body that’s not spread with pure bright mind.\footnote{The equanimity of the fourth \textit{\textsanskrit{jhāna}} is not dullness and indifference, but a brilliant and radiant awareness. } It’s like someone sitting wrapped from head to foot with white cloth. There’s no part of the body that’s not spread over with white cloth.\footnote{The white cloth is the purity and brightness of equanimity. The commentary explains this as a person who has just got out of a bath and sits perfectly dry and content. } In the same way, they sit spreading their body through with pure bright mind. There’s no part of the body that’s not spread with pure bright mind. And many of my disciples meditate on that having attained perfection and consummation of insight. 

Furthermore,\marginnote{29.1} I have explained to my disciples a practice that they use to understand this:\footnote{Here these practices are simply listed one after another, but normally the following knowledges are said to be made possible by the fourth absorption (eg. \href{https://suttacentral.net/dn2/en/sujato\#83.1}{DN 2:83.1}). | The first two knowledges—“knowledge and vision” of the mind attached to body, and the mind-made body—are found only here, \href{https://suttacentral.net/dn2/en/sujato\#83.1}{DN 2:83.1}, and \href{https://suttacentral.net/dn10/en/sujato\#2.21.3}{DN 10:2.21.3}. } ‘This body of mine is formed. It’s made up of the four principal states, produced by mother and father, built up from rice and porridge, liable to impermanence, to wearing away and erosion, to breaking up and destruction.\footnote{This is the “coarse” (\textit{\textsanskrit{olārika}}) body. The obvious impermanence of the body invites the tempting but fallacious notion that the mind or soul is permanent, which is dispelled by deeper insight. } And this consciousness of mine is attached to it, tied to it.’\footnote{This distinction should not be mistaken for mind-body dualism. These are not fundamental substances but experiences of a meditator. } Suppose there was a beryl gem that was naturally beautiful, eight-faceted, well-worked, transparent and clear, endowed with all good qualities. And it was strung with a thread of blue, yellow, red, white, or golden brown. And someone with clear eyes were to take it in their hand and check it: ‘This beryl gem is naturally beautiful, eight-faceted, well-worked, transparent and clear, endowed with all good qualities. And it’s strung with a thread of blue, yellow, red, white, or golden brown.’\footnote{Strung gems were loved in India from the time in the Harappan civilization, millennia before the Buddha. } 

In\marginnote{29.9} the same way, I have explained to my disciples a practice that they use to understand this: ‘This body of mine is formed. It’s made up of the four principal states, produced by mother and father, built up from rice and porridge, liable to impermanence, to wearing away and erosion, to breaking up and destruction. And this consciousness of mine is attached to it, tied to it.’ 

And\marginnote{29.12} many of my disciples meditate on that having attained perfection and consummation of insight. 

Furthermore,\marginnote{30.1} I have explained to my disciples a practice that they use to create from this body another body—formed, mind-made, whole in its major and minor limbs, not deficient in any faculty.\footnote{The “mind-made body” is the interior mental representation of the physical body. In ordinary consciousness it is proprioception, which here is enhanced by the power of meditation. The higher powers in Buddhism are regarded as extensions and evolutions of aspects of ordinary experience, not as metaphysical realities separate from the world of mundane experience. Note that it is still “physical” (\textit{\textsanskrit{rūpī}}) even though it is mind-made. This is the subtle (\textit{sukhuma}) body, which is an energetic experience of physical properties by the mind. | \textit{\textsanskrit{Sabbaṅga}} has the sense “whole and healthy of limb” (Rig Veda 10.161.5c, Atharva Veda 8.2.8c, 11.3.32 ff.). One is reborn with “whole body” (\textit{sarvatanu}, Atharva Veda 5.6.11c, Śatapatha \textsanskrit{Brāhmaṇa} 11.1.8.60, 12.8.3.31). | \textit{\textsanskrit{Paccaṅga}} means “minor limb”, for example the fingers or internal organs. } Suppose a person was to draw a reed out from its sheath. They’d think: ‘This is the reed, this is the sheath. The reed and the sheath are different things. The reed has been drawn out from the sheath.’ Or suppose a person was to draw a sword out from its scabbard. They’d think: ‘This is the sword, this is the scabbard. The sword and the scabbard are different things. The sword has been drawn out from the scabbard.’ Or suppose a person was to draw a snake out from its slough. They’d think: ‘This is the snake, this is the slough. The snake and the slough are different things. The snake has been drawn out from the slough.’ In the same way, I have explained to my disciples a practice that they use to create from this body another body—formed, mind-made, whole in its major and minor limbs, not deficient in any faculty. 

And\marginnote{30.12} many of my disciples meditate on that having attained perfection and consummation of insight. 

Furthermore,\marginnote{31.1} I have explained to my disciples a practice that they use to wield the many kinds of psychic power: multiplying themselves and becoming one again; appearing and disappearing; going unobstructed through a wall, a rampart, or a mountain as if through space; diving in and out of the earth as if it were water; walking on water as if it were earth; flying cross-legged through the sky like a bird; touching and stroking with the hand the sun and moon, so mighty and powerful. They control the body as far as the realm of divinity.\footnote{“Psychic powers” (\textit{iddhi}) were much cultivated in the Buddha’s day, but the means to acquire them varied: devotion to a god, brutal penances, or magic rituals. The Buddha taught that the mind developed in \textit{\textsanskrit{samādhi}} was capable of things that are normally incomprehensible. Only a few of these are attested as events in the early texts. The most common is the ability to disappear and reappear, exhibited by the Buddha (\href{https://suttacentral.net/an8.30/en/sujato\#2.1}{AN 8.30:2.1}), some disciples (\href{https://suttacentral.net/mn37/en/sujato\#6.1}{MN 37:6.1}), and deities (\href{https://suttacentral.net/mn67/en/sujato\#8.1}{MN 67:8.1}). } Suppose a deft potter or their apprentice had some well-prepared clay. They could produce any kind of pot that they like.\footnote{These similes hark back to the descriptions of the purified mind as pliable and workable. } Or suppose a deft ivory-carver or their apprentice had some well-prepared ivory. They could produce any kind of ivory item that they like. Or suppose a deft goldsmith or their apprentice had some well-prepared gold. They could produce any kind of gold item that they like.\footnote{This simile is extended in detail at \href{https://suttacentral.net/an3.101/en/sujato}{AN 3.101}. } In the same way, I have explained to my disciples a practice that they use to wield the many kinds of psychic power … 

And\marginnote{31.6} many of my disciples meditate on that having attained perfection and consummation of insight. 

Furthermore,\marginnote{32.1} I have explained to my disciples a practice that they use so that, with clairaudience that is purified and superhuman, they hear both kinds of sounds, human and heavenly, whether near or far.\footnote{“Clairaudience” is a literal rendition of \textit{dibbasota}. The root sense of \textit{dibba} is to “shine” like the bright sky or a divine being. The senses of clarity and divinity are both present. The Buddha occasionally used this ability for teaching, as at \href{https://suttacentral.net/mn75/en/sujato\#6.1}{MN 75:6.1}. } Suppose there was a powerful horn blower. They’d easily make themselves heard in the four quarters.\footnote{This simile appears misplaced (see eg. \href{https://suttacentral.net/mn99/en/sujato\#24.4}{MN 99:24.4}). At \href{https://suttacentral.net/dn2/en/sujato\#90.1}{DN 2:90.1} we find the simile of being able to clearly identify sounds along the road, which fits better. } In the same way, I have explained to my disciples a practice that they use so that, with clairaudience that is purified and superhuman, they hear both kinds of sounds, human and heavenly, whether near or far. 

And\marginnote{32.4} many of my disciples meditate on that having attained perfection and consummation of insight. 

Furthermore,\marginnote{33.1} I have explained to my disciples a practice that they use to understand the minds of other beings and individuals, having comprehended them with their own mind.\footnote{Note that the Indic idiom is not the “reading” of minds, which suggests hearing the words spoken in inner dialogue. While this is exhibited by the Buddha (eg. \href{https://suttacentral.net/an8.30/en/sujato\#2.1}{AN 8.30:2.1}), the main emphasis is on the comprehension of the overall state of mind. } They understand mind with greed as ‘mind with greed’, and mind without greed as ‘mind without greed’; mind with hate as ‘mind with hate’, and mind without hate as ‘mind without hate’; mind with delusion as ‘mind with delusion’, and mind without delusion as ‘mind without delusion’; constricted mind as ‘constricted mind’, and scattered mind as ‘scattered mind’; expansive mind as ‘expansive mind’, and unexpansive mind as ‘unexpansive mind’; mind that is not supreme as ‘mind that is not supreme’, and mind that is supreme as ‘mind that is supreme’; mind immersed in \textsanskrit{samādhi} as ‘mind immersed in \textsanskrit{samādhi}’, and mind not immersed in \textsanskrit{samādhi} as ‘mind not immersed in \textsanskrit{samādhi}’; freed mind as ‘freed mind’, and unfreed mind as ‘unfreed mind’. Suppose there was a woman or man who was young, youthful, and fond of adornments, and they check their own reflection in a clean bright mirror or a clear bowl of water. If they had a spot they’d know ‘I have a spot’, and if they had no spots they’d know ‘I have no spots’.\footnote{Again the simile emphasizes how clear and direct the experience is. Without deep meditation, we have some intuitive sense for the minds of others, but it is far from clear. } In the same way, I have explained to my disciples a practice that they use to understand the minds of other beings and individuals, having comprehended them with their own mind … 

And\marginnote{33.36} many of my disciples meditate on that having attained perfection and consummation of insight. 

Furthermore,\marginnote{34.1} I have explained to my disciples a practice that they use to recollect the many kinds of past lives. That is: one, two, three, four, five, ten, twenty, thirty, forty, fifty, a hundred, a thousand, a hundred thousand rebirths; many eons of the world contracting, many eons of the world expanding, many eons of the world contracting and expanding. ‘There, I was named this, my clan was that, I looked like this, and that was my food. This was how I felt pleasure and pain, and that was how my life ended. When I passed away from that place I was reborn somewhere else. There, too, I was named this, my clan was that, I looked like this, and that was my food. This was how I felt pleasure and pain, and that was how my life ended. When I passed away from that place I was reborn here.’ And so they recollect their many kinds of past lives, with features and details.\footnote{Here begins the “three knowledges” (\textit{\textsanskrit{tevijjā}}), a subset of the six direct knowledges. The first two of these play an important role in deepening understanding of the nature of suffering in \textit{\textsanskrit{saṁsāra}}. | Empowered by the fourth \textit{\textsanskrit{jhāna}}, memory breaks through the veil of birth and death, revealing the vast expanse of time and dispelling the illusion that there is any place of eternal rest or sanctuary in the cycle of transmigration. The knowledge of these events is not hazy or murky, but clear and precise, illuminated by the brilliance of purified consciousness. } Suppose a person was to leave their home village and go to another village. From that village they’d go to yet another village. And from that village they’d return to their home village. They’d think: ‘I went from my home village to another village. There I stood like this, sat like that, spoke like this, or kept silent like that. From that village I went to yet another village. There too I stood like this, sat like that, spoke like this, or kept silent like that. And from that village I returned to my home village.’\footnote{The word for “past life” is \textit{\textsanskrit{pubbanivāsa}}, literally “former home”, and the imagery of houses is found in the second of the three knowledges as well. Recollection of past lives is as fresh and clear as the memory of a recent journey. } In the same way, I have explained to my disciples a practice that they use to recollect the many kinds of past lives. 

And\marginnote{34.4} many of my disciples meditate on that having attained perfection and consummation of insight. 

Furthermore,\marginnote{35.1} I have explained to my disciples a practice that they use so that, with clairvoyance that is purified and superhuman, they see sentient beings passing away and being reborn—inferior and superior, beautiful and ugly, in a good place or a bad place. They understand how sentient beings are reborn according to their deeds: ‘These dear beings did bad things by way of body, speech, and mind. They denounced the noble ones; they had wrong view; and they chose to act out of that wrong view. When their body breaks up, after death, they’re reborn in a place of loss, a bad place, the underworld, hell. These dear beings, however, did good things by way of body, speech, and mind. They never denounced the noble ones; they had right view; and they chose to act out of that right view. When their body breaks up, after death, they’re reborn in a good place, a heavenly realm.’ And so, with clairvoyance that is purified and superhuman, they see sentient beings passing away and being reborn—inferior and superior, beautiful and ugly, in a good place or a bad place. They understand how sentient beings are reborn according to their deeds.\footnote{Here knowledge extends to the rebirths of others as well as oneself. Even more significant, it brings in the understanding of cause and effect; \emph{why} rebirth happens the way it does. Such knowledge, however, is not infallible, as the Buddha warns in \href{https://suttacentral.net/dn1/en/sujato\#2.5.3}{DN 1:2.5.3} and \href{https://suttacentral.net/mn136/en/sujato}{MN 136}. The experience is one thing; the inferences drawn from it are another. One should draw conclusions only tentatively, after long experience. | “Clairvoyance” renders \textit{dibbacakkhu} (“celestial eye”), for which see \textsanskrit{Chāndogya} \textsanskrit{Upaniṣad} 8.12.5, “the mind is its [the self’s] celestial eye” (\textit{mano'sya \textsanskrit{daivaṁ} \textsanskrit{cakṣuḥ}}). } Suppose there were two houses with doors. A person with clear eyes standing in between them would see people entering and leaving a house and wandering to and fro.\footnote{This simile is found in Majjhima at \href{https://suttacentral.net/mn39/en/sujato\#20.3}{MN 39:20.3} and \href{https://suttacentral.net/mn130/en/sujato\#2.1}{MN 130:2.1}. A slightly different simile is found in the \textsanskrit{Dīgha} at \href{https://suttacentral.net/dn2/en/sujato\#96.1}{DN 2:96.1} and \href{https://suttacentral.net/dn10/en/sujato\#2.33.1}{DN 10:2.33.1}. } In the same way, I have explained to my disciples a practice that they use so that, with clairvoyance that is purified and superhuman, they see sentient beings passing away and being reborn … 

And\marginnote{35.4} many of my disciples meditate on that having attained perfection and consummation of insight. 

Furthermore,\marginnote{36.1} I have explained to my disciples a practice that they use to realize the undefiled freedom of heart and freedom by wisdom in this very life. And they live having realized it with their own insight due to the ending of defilements.\footnote{This is the experience of awakening that is the true goal of the Buddhist path. The defilements—properties of the mind that create suffering—have been curbed by the practice of ethics and suppressed by the power of \textit{\textsanskrit{jhāna}}. Here they are eliminated forever. } Suppose that in a mountain glen there was a lake that was transparent, clear, and unclouded. A person with clear eyes standing on the bank would see the clams and mussels, and pebbles and gravel, and schools of fish swimming about or staying still. They’d think: ‘This lake is transparent, clear, and unclouded. And here are the clams and mussels, and pebbles and gravel, and schools of fish swimming about or staying still.’\footnote{Once again the pool of water represents the mind, but now the meditator is not immersed in the experience, but looks back and reviews it objectively.  } 

In\marginnote{36.3} the same way, I have explained to my disciples a practice that they use to realize the undefiled freedom of heart and freedom by wisdom in this very life. And they live having realized it with their own insight due to the ending of defilements. 

And\marginnote{36.4} many of my disciples meditate on that having attained perfection and consummation of insight. 

This\marginnote{37.1} is the fifth quality because of which my disciples are loyal to me. 

These\marginnote{38.1} are the five qualities because of which my disciples honor, respect, revere, and venerate me; and honoring and respecting me, they remain loyal to me.” 

That\marginnote{38.2} is what the Buddha said. Satisfied, the wanderer \textsanskrit{Sakuludāyī} approved what the Buddha said. 

%
\section*{{\suttatitleacronym MN 78}{\suttatitletranslation With Uggāhamāna Samaṇamaṇḍikāputta }{\suttatitleroot Samaṇamuṇḍikasutta}}
\addcontentsline{toc}{section}{\tocacronym{MN 78} \toctranslation{With Uggāhamāna Samaṇamaṇḍikāputta } \tocroot{Samaṇamuṇḍikasutta}}
\markboth{With Uggāhamāna Samaṇamaṇḍikāputta }{Samaṇamuṇḍikasutta}
\extramarks{MN 78}{MN 78}

\scevam{So\marginnote{1.1} I have heard. }At one time the Buddha was staying near \textsanskrit{Sāvatthī} in Jeta’s Grove, \textsanskrit{Anāthapiṇḍika}’s monastery. 

Now\marginnote{1.3} at that time the wanderer \textsanskrit{Uggāhamāna} \textsanskrit{Samaṇamaṇḍikāputta} was residing together with around five hundred wanderers in \textsanskrit{Mallikā}’s single-halled monastery for philosophical debates, hedged by pale-moon ebony trees.\footnote{\textsanskrit{Uggāhamāna} appears to have the sense “argumentative” found in the Sanskrit \textit{\textsanskrit{udgrāha}} but not otherwise attested in Pali. Both sense and reading of \textsanskrit{Samaṇamaṇḍikā} are unclear. He appears nowhere else. | \textsanskrit{Mallikā} was the chief queen of Pasenadi, and her place is mentioned in a similar context at \href{https://suttacentral.net/dn9/en/sujato\#1.3}{DN 9:1.3}. The commentary explains that the brahmins, Jains, and others would assemble there to “debate their beliefs” (\textit{\textsanskrit{samayaṁ} pavadanti}), a rare early usage of \textit{samaya} in this sense. | Read \textit{\textsanskrit{ācīra}} in the sense of “boundary, hedge” (Commentary: \textit{\textsanskrit{timbarūrukkhapantiyā} \textsanskrit{parikkhittattā}}; cf. Sanskrit \textit{\textsanskrit{prācīra}}, “enclosure, hedge, fence, wall”). } 

Then\marginnote{2.1} the chamberlain \textsanskrit{Pañcakaṅga} left \textsanskrit{Sāvatthī} in the middle of the day to see the Buddha.\footnote{Previously featuring in \href{https://suttacentral.net/mn59/en/sujato\#1.3}{MN 59:1.3}. } It occurred to him, “It’s the wrong time to see the Buddha, as he’s in retreat. And it’s the wrong time to see the esteemed mendicants, as they’re in retreat. Why don’t I go to \textsanskrit{Mallikā}’s monastery to visit the wanderer \textsanskrit{Uggāhamāna}?” So that’s what he did. 

Now\marginnote{3.1} at that time, \textsanskrit{Uggāhamāna} was sitting together with a large assembly of wanderers making an uproar, a dreadful racket. They engaged in all kinds of low talk, such as talk about kings, bandits, and chief ministers; talk about armies, threats, and wars; talk about food, drink, clothes, and beds; talk about garlands and fragrances; talk about family, vehicles, villages, towns, cities, and countries; talk about women and heroes; street talk and well talk; talk about the departed; motley talk; tales of land and sea; and talk about being reborn in this or that state of existence. 

\textsanskrit{Uggāhamāna}\marginnote{3.3} saw \textsanskrit{Pañcakaṅga} coming off in the distance, and hushed his own assembly, “Be quiet, good sirs, don’t make a sound. Here comes \textsanskrit{Pañcakaṅga}, a disciple of the ascetic Gotama. He is included among the white-clothed lay disciples of the ascetic Gotama, who is residing in \textsanskrit{Sāvatthī}. Such venerables like the quiet, are educated to be quiet, and praise the quiet. Hopefully if he sees that our assembly is quiet he’ll see fit to approach.” Then those wanderers fell silent. 

Then\marginnote{4.1} \textsanskrit{Pañcakaṅga} approached \textsanskrit{Uggāhamāna}, and exchanged greetings with him. When the greetings and polite conversation were over, he sat down to one side. \textsanskrit{Uggāhamāna} said to him: 

“Householder,\marginnote{5.1} when an individual has four qualities I describe them as an invincible ascetic—accomplished in the skillful, excelling in the skillful, attained to the highest attainment.\footnote{The “invincible ascetic” (\textit{\textsanskrit{samaṇaṁ} \textsanskrit{ayojjhaṁ}}) appears only here. The Sanskrit \textit{ayodhya} occurs in the Atharva Veda as an epithet of Indra (19.13.7) and of the gods’ “impregnable” fortress (10.2.31). It seems to be a special term for an enlightened being in \textsanskrit{Uggāhamāna}’s system; compare Upaka’s “Infinite Victor” (\textit{anantajina}) of \href{https://suttacentral.net/mn26/en/sujato\#25.22}{MN 26:25.22}. } What four? It’s when they do no bad deeds with their body; speak no bad words; think no bad thoughts; and don’t earn a living by bad livelihood. When an individual has these four qualities I describe them as an invincible ascetic.” 

Then\marginnote{6.1} \textsanskrit{Pañcakaṅga} neither approved nor dismissed the wanderer \textsanskrit{Uggāhamāna}’s statement. He got up from his seat, thinking, “I will learn the meaning of this statement from the Buddha himself.” 

Then\marginnote{7.1} he went to the Buddha, bowed, sat down to one side, and informed the Buddha of all that had been discussed. 

When\marginnote{8.1} he had spoken, the Buddha said to him, “Chamberlain, if what \textsanskrit{Uggāhamāna} says is true, a little baby boy is an invincible ascetic—accomplished in the skillful, excelling in the skillful, attained to the highest attainment. For a little baby doesn’t even have a concept of ‘a body’, so how could they possibly do a bad deed with their body, aside from just wriggling? And a little baby doesn’t even have a concept of ‘speech’, so how could they possibly speak bad words, aside from just crying? And a little baby doesn’t even have a concept of ‘thought’, so how could they possibly think bad thoughts, aside from just whimpering? And a little baby doesn’t even have a concept of ‘livelihood’, so how could they possibly earn a living by bad livelihood, aside from their mother’s breast? If what \textsanskrit{Uggāhamāna} says is true, a little baby boy is an invincible ascetic—accomplished in the skillful, excelling in the skillful, attained to the highest attainment. 

When\marginnote{8.8} an individual has four qualities I describe them, not as an invincible ascetic—accomplished in the skillful, excelling in the skillful, attained to the highest attainment—but as having achieved the same level as a little baby. What four? It’s when they do no bad deeds with their body; speak no bad words; think no bad thoughts; and don’t earn a living by bad livelihood. When an individual has these four qualities I describe them, not as an invincible ascetic, but as having achieved the same level as a little baby. 

When\marginnote{9.1} an individual has ten qualities, chamberlain, I describe them as an invincible ascetic—accomplished in the skillful, excelling in the skillful, attained to the highest attainment. 

The\marginnote{9.2} following things must be understood, I say. ‘These are unskillful behaviors.’ ‘Unskillful behaviors stem from this.’ ‘Here unskillful behaviors cease without anything left over.’ ‘Someone practicing like this is practicing for the cessation of unskillful behaviors.’ 

‘These\marginnote{9.10} are skillful behaviors.’ ‘Skillful behaviors stem from this.’ ‘Here skillful behaviors cease without anything left over.’ ‘Someone practicing like this is practicing for the cessation of skillful behaviors.’ 

‘These\marginnote{9.18} are unskillful thoughts.’ ‘Unskillful thoughts stem from this.’ ‘Here unskillful thoughts cease without anything left over.’ ‘Someone practicing like this is practicing for the cessation of unskillful thoughts.’ 

‘These\marginnote{9.26} are skillful thoughts.’ ‘Skillful thoughts stem from this.’ ‘Here skillful thoughts cease without anything left over.’ ‘Someone practicing like this is practicing for the cessation of skillful thoughts.’ 

And\marginnote{10.1} what, chamberlain, are unskillful behaviors? Unskillful deeds by way of body and speech, and bad livelihood. These are called unskillful behaviors. 

And\marginnote{10.4} where do these unskillful behaviors stem from? Where they stem from has been stated. You should say that they stem from the mind. What mind? The mind takes many and diverse forms. But unskillful behaviors stem from a mind that has greed, hate, and delusion. 

And\marginnote{10.10} where do these unskillful behaviors cease without anything left over? Their cessation has also been stated. It’s when a mendicant gives up bad conduct by way of body, speech, and mind, and develops good conduct by way of body, speech, and mind; they give up wrong livelihood and earn a living by right livelihood. This is where these unskillful behaviors cease without anything left over. 

And\marginnote{10.14} how is someone practicing for the cessation of unskillful behaviors? It’s when a mendicant generates enthusiasm, tries, makes an effort, exerts the mind, and strives so that bad, unskillful qualities don’t arise. They generate enthusiasm, try, make an effort, exert the mind, and strive so that bad, unskillful qualities that have arisen are given up. They generate enthusiasm, try, make an effort, exert the mind, and strive so that skillful qualities arise. They generate enthusiasm, try, make an effort, exert the mind, and strive so that skillful qualities that have arisen remain, are not lost, but increase, mature, and are completed by development. Someone practicing like this is practicing for the cessation of unskillful behaviors. 

And\marginnote{11.1} what are skillful behaviors? Skillful deeds by way of body and speech, and purified livelihood are included in behavior, I say. These are called skillful behaviors. 

And\marginnote{11.4} where do these skillful behaviors stem from? Where they stem from has been stated. You should say that they stem from the mind. What mind? The mind takes many and diverse forms. But skillful behaviors stem from a mind that is free from greed, hate, and delusion. 

And\marginnote{11.10} where do these skillful behaviors cease without anything left over? Their cessation has also been stated. It’s when a mendicant behaves ethically, but they are not determined by ethical behavior.\footnote{Taking the suffix \textit{-maya} in the same sense as \textit{tammaya}; see my note on \href{https://suttacentral.net/mn47/en/sujato\#13.4}{MN 47:13.4}. The point is that they are not formed by and hence limited by the scope of ethics. } And they truly understand the freedom of heart and freedom by wisdom where these skillful behaviors cease without anything left over. 

And\marginnote{11.14} how is someone practicing for the cessation of skillful behaviors? It’s when a mendicant generates enthusiasm, tries, makes an effort, exerts the mind, and strives so that bad, unskillful qualities don’t arise … so that unskillful qualities are given up … so that skillful qualities arise … so that skillful qualities that have arisen remain, are not lost, but increase, mature, and are fulfilled by development. Someone practicing like this is practicing for the cessation of skillful behaviors. 

And\marginnote{12.1} what are unskillful thoughts? Thoughts of sensuality, of malice, and of cruelty. These are called unskillful thoughts. 

And\marginnote{12.4} where do these unskillful thoughts stem from? Where they stem from has been stated. You should say that they stem from perception. What perception? Perception takes many and diverse forms. Perceptions of sensuality, malice, and cruelty—unskillful thoughts stem from this. 

And\marginnote{12.11} where do these unskillful thoughts cease without anything left over? Their cessation has also been stated. It’s when a mendicant, quite secluded from sensual pleasures, secluded from unskillful qualities, enters and remains in the first absorption, which has the rapture and bliss born of seclusion, while placing the mind and keeping it connected. This is where these unskillful thoughts cease without anything left over. 

And\marginnote{12.15} how is someone practicing for the cessation of unskillful thoughts? It’s when a mendicant generates enthusiasm, tries, makes an effort, exerts the mind, and strives so that bad, unskillful qualities don’t arise … so that unskillful qualities are given up … so that skillful qualities arise … so that skillful qualities that have arisen remain, are not lost, but increase, mature, and are fulfilled by development. Someone practicing like this is practicing for the cessation of unskillful thoughts. 

And\marginnote{13.1} what are skillful thoughts? Thoughts of renunciation, good will, and harmlessness. These are called skillful thoughts. 

And\marginnote{13.4} where do these skillful thoughts stem from? Where they stem from has been stated. You should say that they stem from perception. What perception? Perception takes many and diverse forms. Perceptions of renunciation, good will, and harmlessness—skillful thoughts stem from this. 

And\marginnote{13.11} where do these skillful thoughts cease without anything left over? Their cessation has also been stated. It’s when, as the placing of the mind and keeping it connected are stilled, a mendicant enters and remains in the second absorption, which has the rapture and bliss born of immersion, with internal clarity and mind at one, without placing the mind and keeping it connected. This is where these skillful thoughts cease without anything left over. 

And\marginnote{13.15} how is someone practicing for the cessation of skillful thoughts? It’s when a mendicant generates enthusiasm, tries, makes an effort, exerts the mind, and strives so that bad, unskillful qualities don’t arise … so that unskillful qualities are given up … so that skillful qualities arise … so that skillful qualities that have arisen remain, are not lost, but increase, mature, and are fulfilled by development. Someone practicing like this is practicing for the cessation of skillful thoughts. 

Chamberlain,\marginnote{14.1} when an individual has what ten qualities do I describe them as an invincible ascetic—accomplished in the skillful, excelling in the skillful, attained to the highest attainment? It’s when a mendicant has an adept’s right view, right thought, right speech, right action, right livelihood, right effort, right mindfulness, right immersion, right knowledge, and right freedom. When an individual has these ten qualities, I describe them as an invincible ascetic—accomplished in the skillful, excelling in the skillful, attained to the highest attainment.” 

That\marginnote{14.4} is what the Buddha said. Satisfied, \textsanskrit{Pañcakaṅga} the chamberlain approved what the Buddha said. 

%
\section*{{\suttatitleacronym MN 79}{\suttatitletranslation The Shorter Discourse With Sakuludāyī }{\suttatitleroot Cūḷasakuludāyisutta}}
\addcontentsline{toc}{section}{\tocacronym{MN 79} \toctranslation{The Shorter Discourse With Sakuludāyī } \tocroot{Cūḷasakuludāyisutta}}
\markboth{The Shorter Discourse With Sakuludāyī }{Cūḷasakuludāyisutta}
\extramarks{MN 79}{MN 79}

\scevam{So\marginnote{1.1} I have heard. }At one time the Buddha was staying near \textsanskrit{Rājagaha}, in the Bamboo Grove, the squirrels’ feeding ground. Now at that time the wanderer \textsanskrit{Sakuludāyī} was residing together with a large assembly of wanderers in the monastery of the wanderers in the peacocks’ feeding ground. 

Then\marginnote{2.1} the Buddha robed up in the morning and, taking his bowl and robe, entered \textsanskrit{Rājagaha} for alms. Then it occurred to him, “It’s too early to wander for alms in \textsanskrit{Rājagaha}. Why don’t I visit the wanderer \textsanskrit{Sakuludāyī} at the monastery of the wanderers in the peacocks’ feeding ground?” Then the Buddha went to the monastery of the wanderers. 

Now\marginnote{3.1} at that time, \textsanskrit{Sakuludāyī} was sitting together with a large assembly of wanderers making an uproar, a dreadful racket. They engaged in all kinds of low talk, such as talk about kings, bandits, and chief ministers; talk about armies, threats, and wars; talk about food, drink, clothes, and beds; talk about garlands and fragrances; talk about family, vehicles, villages, towns, cities, and countries; talk about women and heroes; street talk and well talk; talk about the departed; motley talk; tales of land and sea; and talk about being reborn in this or that state of existence. 

\textsanskrit{Sakuludāyī}\marginnote{3.3} saw the Buddha coming off in the distance, and hushed his own assembly, “Be quiet, good sirs, don’t make a sound. Here comes the ascetic Gotama. The venerable likes quiet and praises quiet. Hopefully if he sees that our assembly is quiet he’ll see fit to approach.” Then those wanderers fell silent. 

Then\marginnote{4.1} the Buddha approached \textsanskrit{Sakuludāyī}, who said to him, “Let the Blessed One come, sir! Welcome to the Blessed One, sir! It’s been a long time since you took the opportunity to come here. Please, sir, sit down, this seat is ready.” The Buddha sat on the seat spread out, while \textsanskrit{Sakuludāyī} took a low seat and sat to one side. The Buddha said to him, “\textsanskrit{Udāyī}, what were you sitting talking about just now? What conversation was left unfinished?” 

“Sir,\marginnote{5.1} leave aside what we were sitting talking about just now. It won’t be hard for you to hear about that later. When I don’t come to the assembly, they sit and engage in all kinds of low talk. But when I have come to the assembly, they sit gazing up at my face alone, thinking,\footnote{Cp. \href{https://suttacentral.net/sn56.39/en/sujato\#1.2}{SN 56.39:1.2}. At \href{https://suttacentral.net/dn2/en/sujato\#35.2}{DN 2:35.2} this is the attitude of a servant to the king. } ‘Whatever the ascetic \textsanskrit{Udāyī} teaches, we shall listen to it.’\footnote{The same phrase is used of the Buddha’s assembly at \href{https://suttacentral.net/mn77/en/sujato\#6.64}{MN 77:6.64}, but with a different attitude: the Buddha’s assembly is poised expectantly, while \textsanskrit{Udāyī}’s is gazing up at his face. } But when the Buddha has come to the assembly, both myself and the assembly sit gazing up at your face, thinking, ‘Whatever the Buddha teaches, we shall listen to it.’” 

“Well\marginnote{6.1} then, \textsanskrit{Udāyī}, suggest something for me to talk about.” 

“Mister\marginnote{6.2} Gotama, a few days ago someone was claiming to be all-knowing and all-seeing, to know and see everything without exception, thus: ‘Knowledge and vision are constantly and continually present to me, while walking, standing, sleeping, and waking.’ When I asked them a question about the past, they dodged the issue, distracted the discussion with irrelevant points, and displayed annoyance, hate, and bitterness. That reminded me of the Buddha:\footnote{For the \textsanskrit{Mahāsaṅgīti} \textit{sati} (“reminded”), PTS reads \textit{\textsanskrit{pīti}} here, but the phrase occurs at \href{https://suttacentral.net/mn89/en/sujato\#4.2}{MN 89:4.2} and \href{https://suttacentral.net/dn9/en/sujato\#6.32}{DN 9:6.32}, both times with \textit{sati}. } ‘Surely it must be the Blessed One, the Holy One who is so skilled in such matters.’” 

“But\marginnote{6.6} \textsanskrit{Udāyī}, who was it that made such a claim and behaved in such a way?” 

“It\marginnote{6.7} was the Jain ascetic of the \textsanskrit{Ñātika} clan, sir.”\footnote{The same claim is recorded at \href{https://suttacentral.net/mn14/en/sujato\#17.2}{MN 14:17.2}. } 

“\textsanskrit{Udāyī},\marginnote{7.1} someone who can recollect their many kinds of past lives, with features and details, might ask me a question about the past, or I might ask them a question about the past. And they might satisfy me with their answer, or I might satisfy them with my answer. 

Someone\marginnote{7.3} who, with clairvoyance that is purified and superhuman, understands how sentient beings are reborn according to their deeds might ask me a question about the future, or I might ask them a question about the future. And they might satisfy me with their answer, or I might satisfy them with my answer. 

Nevertheless,\marginnote{7.5} \textsanskrit{Udāyī}, leave aside the past and the future. I shall teach you the Dhamma: ‘When this exists, that is; due to the arising of this, that arises.\footnote{The Buddha does not claim to see every fact and detail, but he comprehends the principle that underlies all suffering. } When this doesn’t exist, that is not; due to the cessation of this, that ceases.’” 

“Well\marginnote{8.1} sir, I can’t even recall with features and details what I’ve undergone in this incarnation.\footnote{\textit{\textsanskrit{Attabhāva}} is literally “the state of the self”, and refers to the body in which consciousness has been incarnated. While \textit{\textsanskrit{attā}} usually has a psychological or metaphysical sense, it sometimes means the “body”, a usage found also in the Sanskrit \textit{\textsanskrit{ātman}} (eg. \textsanskrit{Bṛhadāraṇyaka} \textsanskrit{Upaniṣad} 1.2.4). } How should I possibly recollect my many kinds of past lives with features and details, like the Buddha? For I can’t even see a mud-goblin right now.\footnote{The curious mud-goblin (\textit{\textsanskrit{paṁsupisācaka}}) is similarly mentioned at \href{https://suttacentral.net/ud4.4/en/sujato\#7.3}{Ud 4.4:7.3}. } How should I possibly, with clairvoyance that is purified and superhuman, see sentient beings passing away and being reborn, like the Buddha? But then the Buddha told me, ‘Nevertheless, \textsanskrit{Udāyī}, leave aside the past and the future. I shall teach you the Dhamma: 

“When\marginnote{8.8} this exists, that is; due to the arising of this, that arises. When this doesn’t exist, that is not; due to the cessation of this, that ceases.”’ But that is even more unclear to me. Perhaps I might satisfy the Buddha by answering a question about my own tradition.” 

“But\marginnote{9.1} \textsanskrit{Udāyī}, what is your own tradition?” 

“Sir,\marginnote{9.2} it’s this: ‘This is the ultimate splendor, this is the ultimate splendor.’”\footnote{For more on this theory see note on \href{https://suttacentral.net/mn80/en/sujato\#2.4}{MN 80:2.4}. } 

“But\marginnote{9.4} what is that ultimate splendor?” 

“Sir,\marginnote{9.6} the ultimate splendor is the splendor compared to which no other splendor is finer.” 

“But\marginnote{9.7} what is that ultimate splendor compared to which no other splendor is finer?” 

“Sir,\marginnote{9.8} the ultimate splendor is the splendor compared to which no other splendor is finer.” 

“\textsanskrit{Udāyī},\marginnote{10.1} you could draw this out for a long time. You say, ‘The ultimate splendor is the splendor compared to which no other splendor is finer.’ But you don’t describe that splendor. Suppose a man was to say, ‘Whoever the finest lady in the land is, it is her that I want, her I desire!’ They’d say to him, ‘Mister, that finest lady in the land who you desire—do you know whether she’s an aristocrat, a brahmin, a peasant, or a menial?’ Asked this, he’d say, ‘No.’ They’d say to him, ‘Mister, that finest lady in the land who you desire—do you know her name or clan? Whether she’s tall or short or medium? Whether her skin is black, brown, or tawny? What village, town, or city she comes from?’ Asked this, he’d say, ‘No.’ They’d say to him, ‘Mister, do you desire someone who you’ve never even known or seen?’ Asked this, he’d say, ‘Yes.’ 

What\marginnote{10.14} do you think, \textsanskrit{Udāyī}? This being so, doesn’t that man’s statement turn out to have no demonstrable basis?” 

“Clearly\marginnote{10.16} that’s the case, sir.” 

“In\marginnote{10.17} the same way, you say, ‘The ultimate splendor is the splendor compared to which no other splendor is finer.’ But you don’t describe that splendor.” 

“Sir,\marginnote{11.1} suppose there was a beryl gem that was naturally beautiful, eight-faceted, well-worked. When placed on a cream rug it would shine and glow and radiate. Such is the splendor of the self that is healthy after death.”\footnote{The statement on the “splendor of the self after death” is not found in the parallel at MA 208. } 

“What\marginnote{12.1} do you think, \textsanskrit{Udāyī}? Which of these two has a finer splendor: such a beryl gem, or a firefly in the dark of night?” 

“A\marginnote{12.3} firefly in the dark of night, sir.” 

“What\marginnote{13.1} do you think, \textsanskrit{Udāyī}? Which of these two has a finer splendor: a firefly in the dark of night, or an oil lamp in the dark of night?” 

“An\marginnote{13.3} oil lamp in the dark of night, sir.” 

“What\marginnote{14.1} do you think, \textsanskrit{Udāyī}? Which of these two has a finer splendor: an oil lamp in the dark of night, or a great mass of fire in the dark of night?” 

“A\marginnote{14.3} great mass of fire in the dark of night, sir.” 

“What\marginnote{15.1} do you think, \textsanskrit{Udāyī}? Which of these two has a finer splendor: a great mass of fire in the dark of night, or the Morning Star in the clear and cloudless heavens at the crack of dawn?” 

“The\marginnote{15.3} Morning Star in the clear and cloudless heavens at the crack of dawn, sir.” 

“What\marginnote{16.1} do you think, \textsanskrit{Udāyī}? Which of these two has a finer splendor: the Morning Star in the clear and cloudless heavens at the crack of dawn, or the full moon at midnight in the clear and cloudless heavens on the fifteenth day sabbath?” 

“The\marginnote{16.3} full moon at midnight in the clear and cloudless heavens on the fifteenth day sabbath, sir.” 

“What\marginnote{17.1} do you think, \textsanskrit{Udāyī}? Which of these two has a finer splendor: the full moon at midnight in the clear and cloudless heavens on the fifteenth day sabbath, or the sun at midday in the clear and cloudless heavens in the last month of the rainy season, in autumn?” 

“The\marginnote{17.3} sun at midday in the clear and cloudless heavens in the last month of the rainy season, in autumn, sir.” 

“Beyond\marginnote{18.1} this, \textsanskrit{Udāyī}, I know very many gods on whom the light of the sun and moon makes no impression.\footnote{\textit{\textsanskrit{Ābhā} nanubhonti} (“light makes no impression”) is glossed as \textit{\textsanskrit{obhāsaṁ} na \textsanskrit{vaḷañjanti}}, “does not partake of the light”. At \href{https://suttacentral.net/an4.127/en/sujato\#1.4}{AN 4.127:1.4} the same phrase, in regard to the space where the light of the sun does not reach, is glossed \textit{\textsanskrit{pabhā} nappahonti}, i.e. “light is ineffective”. Compare \textit{\textsanskrit{ananubhūtaṁ}} at \href{https://suttacentral.net/mn49/en/sujato\#11.1}{MN 49:11.1}. } Nevertheless, I do not say: ‘The splendor compared to which no other splendor is finer.’ But of the splendor inferior to a firefly you say, ‘This is the ultimate splendor.’ And you don’t describe that splendor.” 

“The\marginnote{19.1} Blessed One has cut short the discussion! The Holy One has cut short the discussion!” 

“But\marginnote{19.2} \textsanskrit{Udāyī}, why do you say this?” 

“Sir,\marginnote{19.4} it says this in our own tradition: ‘This is the ultimate splendor, this is the ultimate splendor.’ But when pursued, pressed, and grilled on our own tradition, we turned out to be vacuous, hollow, and mistaken.” 

“But\marginnote{20.1} \textsanskrit{Udāyī}, is there a world of perfect happiness? And is there a grounded practice for realizing a world of perfect happiness?” 

“Sir,\marginnote{20.2} it says this in our own tradition: ‘There is a world of perfect happiness. And there is a grounded practice for realizing a world of perfect happiness.’” 

“Well,\marginnote{21.1} what is that grounded practice for realizing a world of perfect happiness?” 

“Sir,\marginnote{21.2} it’s when someone gives up killing living creatures, stealing, sexual misconduct, and lying. And they proceed having undertaken some kind of mortification.\footnote{\textit{\textsanskrit{Vā}} here is conjunctive. } This is the grounded practice for realizing a world of perfect happiness.” 

“What\marginnote{22.1} do you think, \textsanskrit{Udāyī}? On an occasion when someone refrains from killing living creatures, is their self perfectly happy at that time, or does it have both pleasure and pain?” 

“It\marginnote{22.3} has both pleasure and pain.” 

“What\marginnote{22.4} do you think, \textsanskrit{Udāyī}? On an occasion when someone refrains from stealing … sexual misconduct … lying, is their self perfectly happy at that time, or does it have both pleasure and pain?” 

“It\marginnote{22.10} has both pleasure and pain.” 

“What\marginnote{22.11} do you think, \textsanskrit{Udāyī}? On an occasion when someone undertakes and follows some kind of mortification, is their self perfectly happy at that time, or does it have both pleasure and pain?” 

“It\marginnote{22.13} has both pleasure and pain.” 

“What\marginnote{22.14} do you think, \textsanskrit{Udāyī}? Is a perfectly happy world realized by relying on a practice of mixed pleasure and pain?” 

“The\marginnote{23.1} Blessed One has cut short the discussion! The Holy One has cut short the discussion!” 

“But\marginnote{23.2} \textsanskrit{Udāyī}, why do you say this?” 

“Sir,\marginnote{23.4} it says this in our own tradition: ‘There is a world of perfect happiness. And there is a grounded practice for realizing a world of perfect happiness.’\footnote{“Grounded path” is \textit{\textsanskrit{ākāravatī} \textsanskrit{paṭipadā}}, for which compare the “grounded faith” (\textit{\textsanskrit{ākāravatī} \textsanskrit{saddhā}}) of \href{https://suttacentral.net/mn47/en/sujato\#16.1}{MN 47:16.1} and \href{https://suttacentral.net/mn60/en/sujato\#4.1}{MN 60:4.1}. } But when pursued, pressed, and grilled on our own tradition, we turned out to be vacuous, hollow, and mistaken. 

But\marginnote{23.7} sir, is there a world of perfect happiness? And is there a grounded practice for realizing a world of perfect happiness?” 

“There\marginnote{24.1} is a world of perfect happiness, \textsanskrit{Udāyī}. And there is a grounded practice for realizing a world of perfect happiness.” 

“Well\marginnote{24.2} sir, what is that grounded practice for realizing a world of perfect happiness?” 

“It’s\marginnote{25.1} when a mendicant, quite secluded from sensual pleasures, secluded from unskillful qualities, enters and remains in the first absorption. As the placing of the mind and keeping it connected are stilled, they enter and remain in the second absorption. With the fading away of rapture, they enter and remain in the third absorption. This is the grounded practice for realizing a world of perfect happiness.” 

“Sir,\marginnote{25.5} that’s not the grounded practice for realizing a world of perfect happiness. At that point a perfectly happy world has already been realized.” 

“No,\marginnote{25.6} \textsanskrit{Udāyī}, at that point a perfectly happy world has not been realized. This is the grounded practice for realizing a world of perfect happiness.” 

When\marginnote{26.1} he said this, \textsanskrit{Sakuludāyī}’s assembly made an uproar, a dreadful racket, “In that case, we’re lost, and so are our traditional teachings! We’re lost, and so are our traditional teachings! We know nothing higher than this!” 

Then\marginnote{26.4} \textsanskrit{Sakuludāyī}, having quieted those wanderers, said to the Buddha, “Well sir, at what point is a perfectly happy world realized?” 

“It’s\marginnote{27.2} when, giving up pleasure and pain, and ending former happiness and sadness, a mendicant enters and remains in the fourth absorption. There are deities who have been reborn in a perfectly happy world. That mendicant associates with them, converses, and engages in discussion. It’s at this point that a perfectly happy world has been realized.” 

“Surely\marginnote{28.1} the mendicants must lead the spiritual life under the Buddha for the sake of realizing this perfectly happy world?” 

“No,\marginnote{28.2} \textsanskrit{Udāyī}, the mendicants don’t lead the spiritual life under me for the sake of realizing this perfectly happy world. There are other things that are finer, for the sake of which the mendicants lead the spiritual life under me.” 

“But\marginnote{28.4} what are those finer things?” 

“It’s\marginnote{29{-}36.1} when a Realized One arises in the world, perfected, a fully awakened Buddha, accomplished in knowledge and conduct, holy, knower of the world, supreme guide for those who wish to train, teacher of gods and humans, awakened, blessed. … 

They\marginnote{37.1} give up these five hindrances, corruptions of the heart that weaken wisdom. Then, quite secluded from sensual pleasures, secluded from unskillful qualities, they enter and remain in the first absorption. This is one of the finer things for the sake of which the mendicants lead the spiritual life under me. 

Furthermore,\marginnote{38{-}40.1} as the placing of the mind and keeping it connected are stilled, a mendicant enters and remains in the second absorption … third absorption … fourth absorption. This too is one of the finer things. 

When\marginnote{41.1} their mind has become immersed in \textsanskrit{samādhi} like this—purified, bright, flawless, rid of corruptions, pliable, workable, steady, and imperturbable—they extend it toward recollection of past lives. They recollect many kinds of past lives. That is: one, two, three, four, five, ten, twenty, thirty, forty, fifty, a hundred, a thousand, a hundred thousand rebirths; many eons of the world contracting, many eons of the world expanding, many eons of the world contracting and expanding. They recollect their many kinds of past lives, with features and details. This too is one of the finer things. 

When\marginnote{42.1} their mind has become immersed in \textsanskrit{samādhi} like this—purified, bright, flawless, rid of corruptions, pliable, workable, steady, and imperturbable—they extend it toward knowledge of the death and rebirth of sentient beings. With clairvoyance that is purified and superhuman, they see sentient beings passing away and being reborn—inferior and superior, beautiful and ugly, in a good place or a bad place. They understand how sentient beings are reborn according to their deeds. This too is one of the finer things. 

When\marginnote{43.1} their mind has become immersed in \textsanskrit{samādhi} like this—purified, bright, flawless, rid of corruptions, pliable, workable, steady, and imperturbable—they extend it toward knowledge of the ending of defilements. They truly understand: ‘This is suffering’ … ‘This is the origin of suffering’ … ‘This is the cessation of suffering’ … ‘This is the practice that leads to the cessation of suffering’. They truly understand: ‘These are defilements’ … ‘This is the origin of defilements’ … ‘This is the cessation of defilements’ … ‘This is the practice that leads to the cessation of defilements’. 

Knowing\marginnote{44.1} and seeing like this, their mind is freed from the defilements of sensuality, desire to be reborn, and ignorance. When they’re freed, they know they’re freed. 

They\marginnote{44.3} understand: ‘Rebirth is ended, the spiritual journey has been completed, what had to be done has been done, there is nothing further for this place.’ This too is one of the finer things. These are the finer things for the sake of which the mendicants lead the spiritual life under me.” 

When\marginnote{45.1} he had spoken, \textsanskrit{Sakuludāyī} said to the Buddha, “Excellent, sir! Excellent! As if he were righting the overturned, or revealing the hidden, or pointing out the path to the lost, or lighting a lamp in the dark so people with clear eyes can see what’s there, the Buddha has made the teaching clear in many ways. I go for refuge to the Buddha, to the teaching, and to the mendicant \textsanskrit{Saṅgha}. Sir, may I receive the going forth, the ordination in the Buddha’s presence?” 

When\marginnote{46.1} he said this, \textsanskrit{Sakuludāyī}’s assembly said to him, “Mister \textsanskrit{Udāyī}, don’t lead the spiritual life under the ascetic Gotama. You have been a tutor; don’t live as a pupil. You’ll end up like a water jar that turns into a water ladle.\footnote{“Water ladle” (\textit{\textsanskrit{udañcanika}}, variant \textit{uddekanika}, commentary \textit{\textsanskrit{udakavāraka}}) is a unique term. Read \textit{\textsanskrit{udañchanika}} in the sense “(ladle) for drawing water” (cp. Sanskrit \textit{\textsanskrit{ākarṣa}} for Pali \textit{\textsanskrit{añchana}}). | The terms \textit{\textsanskrit{udakamaṇika}} (“water jar”) and \textit{\textsanskrit{udañchanika}} (“water ladle”) pun on \textsanskrit{Udāyī}’s name. } Mister \textsanskrit{Udāyī}, don’t lead the spiritual life under the ascetic Gotama. You have been a tutor; don’t live as a pupil.” And that’s how the wanderer \textsanskrit{Sakuludāyī}’s own assembly prevented him from leading the spiritual life under the Buddha. 

%
\section*{{\suttatitleacronym MN 80}{\suttatitletranslation With Vekhanasa }{\suttatitleroot Vekhanasasutta}}
\addcontentsline{toc}{section}{\tocacronym{MN 80} \toctranslation{With Vekhanasa } \tocroot{Vekhanasasutta}}
\markboth{With Vekhanasa }{Vekhanasasutta}
\extramarks{MN 80}{MN 80}

\scevam{So\marginnote{1.1} I have heard. }At one time the Buddha was staying near \textsanskrit{Sāvatthī} in Jeta’s Grove, \textsanskrit{Anāthapiṇḍika}’s monastery. 

Then\marginnote{2.1} the wanderer Vekhanasa went up to the Buddha, and exchanged greetings with him.\footnote{Vekhanasa, the “descendant of \textsanskrit{Vikhānasa}” was, according to the commentary, the teacher of \textsanskrit{Sakaludāyī} and came to defend his teaching. The Vekhanasas were Vedic poets. They became known in later Sanskrit texts as an order of forest-dwelling ascetics, who evolved into the modern \textsanskrit{Vaikhānasa} Vishnavite tradition. This sutta may be the earliest reference to a tradition of Vekhanasa ascetics, creating a link between the Vedic poets and the later tradition. The Buddha calls him by his personal or clan name \textsanskrit{Kaccāna}. } When the greetings and polite conversation were over, he stood to one side, and expressed this heartfelt sentiment: “This is the ultimate splendor, this is the ultimate splendor.”\footnote{The Rig Veda features two \textsanskrit{Vaikhānasa} poems, one of which describes Soma as “most resplendent” (\textit{\textsanskrit{śubhraśastamaḥ}}, Rig Veda 9.66.26), a phrase notably parallel in meaning to the Pali “ultimate splendor” (\textit{paramo \textsanskrit{vaṇṇo}}). These verses are rich with similar imagery: Soma is a “gleaming light” (24) whose “glittering drops” “flash with radiance” (25), “pervading with rays” (27), a “heaven-bright milk” (30). This poem is ascribed to “a hundred \textsanskrit{Vaikhānasas}”, suggesting that an order already existed at the time. Perhaps the Vekhanasa lineage developed their philosophy from these verses, which they had recited over so many centuries that the original referent, Soma, was forgotten. } 

“But\marginnote{2.5} \textsanskrit{Kaccāna}, why do you say: ‘This is the ultimate splendor, this is the ultimate splendor.’ What is that ultimate splendor?” 

“Mister\marginnote{2.8} Gotama, the ultimate splendor is the splendor compared to which no other splendor is finer.” 

“But\marginnote{2.9} what is that ultimate splendor compared to which no other splendor is finer?” 

“Mister\marginnote{2.10} Gotama, the ultimate splendor is the splendor compared to which no other splendor is finer.” 

“\textsanskrit{Kaccāna},\marginnote{3.1} you could draw this out for a long time. You say, ‘The ultimate splendor is the splendor compared to which no other splendor is finer.’ But you don’t describe that splendor. 

Suppose\marginnote{3.3} a man was to say, ‘Whoever the finest lady in the land is, it is her that I want, her I desire!’ 

They’d\marginnote{3.5} say to him, ‘Mister, that finest lady in the land who you desire—do you know whether she’s an aristocrat, a brahmin, a peasant, or a menial?’ 

Asked\marginnote{3.7} this, he’d say, ‘No.’ 

They’d\marginnote{3.8} say to him, ‘Mister, that finest lady in the land who you desire—do you know her name or clan? Whether she’s tall or short or medium? Whether her skin is black, brown, or tawny? What village, town, or city she comes from?’ 

Asked\marginnote{3.10} this, he’d say, ‘No.’ 

They’d\marginnote{3.11} say to him, ‘Mister, do you desire someone who you’ve never even known or seen?’ 

Asked\marginnote{3.13} this, he’d say, ‘Yes.’ 

What\marginnote{3.14} do you think, \textsanskrit{Kaccāna}? This being so, doesn’t that man’s statement turn out to have no demonstrable basis?” 

“Clearly\marginnote{3.16} that’s the case, Mister Gotama.” 

“In\marginnote{3.17} the same way, you say, ‘The ultimate splendor is the splendor compared to which no other splendor is finer.’ But you don’t describe that splendor.” 

“Mister\marginnote{4.1} Gotama, suppose there was a beryl gem that was naturally beautiful, eight-faceted, well-worked. When placed on a cream rug it would shine and glow and radiate. Such is the splendor of the self that is healthy after death.” 

“What\marginnote{5.1} do you think, \textsanskrit{Kaccāna}? Which of these two has a finer splendor: such a beryl gem, or a firefly in the dark of night?” 

“A\marginnote{5.3} firefly in the dark of night.” 

“What\marginnote{6.1} do you think, \textsanskrit{Kaccāna}? Which of these two has a finer splendor: a firefly in the dark of night, or an oil lamp in the dark of night?” 

“An\marginnote{6.3} oil lamp in the dark of night.” 

“What\marginnote{7.1} do you think, \textsanskrit{Kaccāna}? Which of these two has a finer splendor: an oil lamp in the dark of night, or a great mass of fire in the dark of night?” 

“A\marginnote{7.3} great mass of fire in the dark of night.” 

“What\marginnote{8.1} do you think, \textsanskrit{Kaccāna}? Which of these two has a finer splendor: a great mass of fire in the dark of night, or the Morning Star in the clear and cloudless heavens at the crack of dawn?” 

“The\marginnote{8.3} Morning Star in the clear and cloudless heavens at the crack of dawn.” 

“What\marginnote{9.1} do you think, \textsanskrit{Kaccāna}? Which of these two has a finer splendor: the Morning Star in the clear and cloudless heavens at the crack of dawn, or the full moon at midnight in the clear and cloudless heavens on the fifteenth day sabbath?” 

“The\marginnote{9.3} full moon at midnight in the clear and cloudless heavens on the fifteenth day sabbath.” 

“What\marginnote{10.1} do you think, \textsanskrit{Kaccāna}? Which of these two has a finer splendor: the full moon at midnight in the clear and cloudless heavens on the fifteenth day sabbath, or the sun at midday in the clear and cloudless heavens in the last month of the rainy season, in autumn?” 

“The\marginnote{10.3} sun at midday in the clear and cloudless heavens in the last month of the rainy season, in autumn.” 

“Beyond\marginnote{11.1} this, \textsanskrit{Kaccāna}, I know very many gods on whom the light of the sun and moon makes no impression. Nevertheless, I do not say: ‘The splendor compared to which no other splendor is finer.’ But of the splendor inferior to a firefly you say, ‘This is the ultimate splendor.’ And you don’t describe that splendor. 

\textsanskrit{Kaccāna},\marginnote{12.1} there are these five kinds of sensual stimulation. What five? Sights known by the eye, which are likable, desirable, agreeable, pleasant, sensual, and arousing. Sounds known by the ear … Smells known by the nose … Tastes known by the tongue … Touches known by the body, which are likable, desirable, agreeable, pleasant, sensual, and arousing. These are the five kinds of sensual stimulation. 

The\marginnote{13.1} pleasure and happiness that arises from these five kinds of sensual stimulation is called sensual pleasure. ‘From the senses comes sensual pleasure. Beyond sensual pleasure is the pleasure that surmounts the sensual, which is said to be the best of these.’\footnote{The introductory \textit{iti} (“so they say”) indicates that the Buddha is quoting a saying, which may not have originated with him. | The “pleasure that surmounts the sensual” (\textit{\textsanskrit{kāmaggasukha}}) is defined by the commentary as \textsanskrit{Nibbāna}, which explains why the Buddha says below that only arahants can truly understand this saying (\href{https://suttacentral.net/mn80/en/sujato\#14.8}{MN 80:14.8}). | Compare with the passage on extracting cream of ghee, which uses parallel syntax (\href{https://suttacentral.net/dn9/en/sujato\#52.1}{DN 9:52.1}, \href{https://suttacentral.net/sn34.1/en/sujato\#1.9}{SN 34.1:1.9}, \href{https://suttacentral.net/an4.95/en/sujato\#4.1}{AN 4.95:4.1}). } 

When\marginnote{14.1} he said this, Vekhanasa said to the Buddha, “It’s incredible, Mister Gotama, it’s amazing! How well said this was by Mister Gotama! ‘From the senses comes sensual pleasure. Beyond sensual pleasure is the pleasure that surmounts the sensual, which is said to be the best of these.’ Mister Gotama, from the senses comes sensual pleasure. Beyond sensual pleasure is the pleasure that surmounts the sensual, which is said to be the best of these.” 

“\textsanskrit{Kaccāna},\marginnote{14.6} since you have a different view, creed, and belief, then, unless you dedicate yourself to practice with the guidance of tradition, it’s hard for to understand the senses, sensual pleasure, and the pleasure that surmounts the sensual. There are mendicants who are perfected, who have ended the defilements, completed the spiritual journey, done what had to be done, laid down the burden, achieved their own goal, utterly ended the fetter of continued existence, and are rightly freed through enlightenment. They can understand the senses, sensual pleasure, and the pleasure that surmounts the sensual.” 

When\marginnote{15.1} he said this, Vekhanasa became angry and upset. He even attacked and badmouthed the Buddha himself, saying, “The ascetic Gotama will be worsted!” He said to the Buddha, “This is exactly what happens with some ascetics and brahmins. Not knowing the past or seeing the future, they nevertheless claim: ‘We understand: “Rebirth is ended, the spiritual journey has been completed, what had to be done has been done, there is nothing further for this place.’” Their statement turns out to be a joke—mere words, vacuous and hollow.” 

“\textsanskrit{Kaccāna},\marginnote{16.1} there are some ascetics and brahmins who, not knowing the past or seeing the future, nevertheless claim: ‘We understand: “Rebirth is ended, the spiritual journey has been completed, what had to be done has been done, there is nothing further for this place.’” There is a legitimate refutation of them. Nevertheless, \textsanskrit{Kaccāna}, leave aside the past and the future. Let a sensible person come—neither devious nor deceitful, a person of integrity. I teach and instruct them. Practicing as instructed they will soon know and see for themselves, ‘So this is how to be rightly released from the bond, that is, the bond of ignorance.’ Suppose there was a little baby bound with swaddling up to the neck. As they grow up and their senses mature, they’re accordingly released from those bonds. They’d know ‘I’m released,’ and there would be no more bonds. 

In\marginnote{16.11} the same way, let a sensible person come—neither devious nor deceitful, a person of integrity. I teach and instruct them. Practicing as instructed they will soon know and see for themselves, ‘So this is how to be rightly released from the bond, that is, the bond of ignorance.’” 

When\marginnote{17.1} he said this, Vekhanasa said to the Buddha, “Excellent, Mister Gotama! … From this day forth, may Mister Gotama remember me as a lay follower who has gone for refuge for life.” 

%
\addtocontents{toc}{\let\protect\contentsline\protect\nopagecontentsline}
\chapter*{The Chapter on Kings }
\addcontentsline{toc}{chapter}{\tocchapterline{The Chapter on Kings }}
\addtocontents{toc}{\let\protect\contentsline\protect\oldcontentsline}

%
\section*{{\suttatitleacronym MN 81}{\suttatitletranslation With Ghaṭīkāra }{\suttatitleroot Ghaṭikārasutta}}
\addcontentsline{toc}{section}{\tocacronym{MN 81} \toctranslation{With Ghaṭīkāra } \tocroot{Ghaṭikārasutta}}
\markboth{With Ghaṭīkāra }{Ghaṭikārasutta}
\extramarks{MN 81}{MN 81}

\scevam{So\marginnote{1.1} I have heard.\footnote{This lively story is one of the most developed of the early canonical \textsanskrit{Jātakas}. The popularity of the story is attested by the many parallels, which include a retelling of the story by the deity who had been \textsanskrit{Ghaṭīkāra} (\href{https://suttacentral.net/sn1.50/en/sujato}{SN 1.50}). \textsanskrit{Ghaṭīkāra}’s virtues as an ideal lay follower are extolled; his kindness, faith, and contentment, and especially his practice of economic harmlessness through rejection of the contractual market economy. } }At one time the Buddha was wandering in the land of the Kosalans together with a large \textsanskrit{Saṅgha} of mendicants. Then the Buddha left the road, and at a certain spot he smiled. 

Then\marginnote{2.2} Venerable Ānanda thought, “What is the cause, what is the reason why the Buddha smiled? Realized Ones do not smile for no reason.” 

So\marginnote{2.5} Ānanda arranged his robe over one shoulder, raised his joined palms toward the Buddha, and said, “What is the cause, what is the reason why the Buddha smiled? Realized Ones do not smile for no reason.” 

“Once\marginnote{3.1} upon a time, Ānanda, there was a market town in this spot named \textsanskrit{Vebhaliṅga}. It was successful and prosperous, populous, full of people.\footnote{Each of the occasions when the Buddha smiled in this way served as an introduction to an uplifting story of the past. In \href{https://suttacentral.net/mn83/en/sujato}{MN 83}, the Buddha tells of the fabled king Maghadeva of \textsanskrit{Mithilā}, who went forth and practice the divine meditations. \href{https://suttacentral.net/an5.180/en/sujato}{AN 5.180}, like the current sutta, tells the story of a virtuous layperson in the time of Kassapa Buddha. Both the Kassapa stories are set in Kosala. | \textsanskrit{Vebhaliṅga} is known only from this context and there are many variant readings. | Here the compound \textit{\textsanskrit{gāmanigama}}, elsewhere “village or town” clearly means “market town”. The same idiom is found in Sanskrit versions of this text (\textit{\textsanskrit{vaibhiḍiṅgī} \textsanskrit{nāma} \textsanskrit{grāmanigamo}}, \textsanskrit{Mūlasarvāstivāda} Vinaya 17.390). } And Kassapa, a blessed one, a perfected one, a fully awakened Buddha, lived supported by \textsanskrit{Vebhaliṅga}. It was here, in fact, that he had his monastery, where he advised the mendicant \textsanskrit{Saṅgha} while seated.” 

Then\marginnote{4.1} Ānanda spread out his outer robe folded in four and said to the Buddha, “Well then, sir, may the Blessed One sit here! Then this piece of land will have been occupied by two perfected ones, fully awakened Buddhas.” The Buddha sat on the seat spread out. When he was seated he said to Venerable Ānanda: 

“Once\marginnote{5.1} upon a time, Ānanda, there was a market town in this spot named \textsanskrit{Vebhaliṅga}. It was successful and prosperous, populous, full of people. And Kassapa, a blessed one, a perfected one, a fully awakened Buddha, lived supported by \textsanskrit{Vebhaliṅga}. It was here, in fact, that he had his monastery, where he advised the mendicant \textsanskrit{Saṅgha} while seated. 

The\marginnote{6.1} Buddha Kassapa had as chief supporter in \textsanskrit{Vebhaliṅga} a potter named \textsanskrit{Ghaṭīkāra}. \textsanskrit{Ghaṭīkāra} had a dear friend named \textsanskrit{Jotipāla}, a student.\footnote{Various texts of the northern \textsanskrit{Sarvāstivāda} movement, including the Madhyama-\textsanskrit{āgama} and the \textsanskrit{Saṅghabhedavastu}, speak instead of the potter \textsanskrit{Nandīpāla} and the Brahmin youth Uttara. | \textsanskrit{Ghaṭīkāra} (“potter”) \textsanskrit{Nandīpāla} is also referred to by his clan name Bhaggava. | \textsanskrit{Jotipāla} means “guardian of the sacred flame”, i.e. someone who maintains the Vedic fire ritual, as attested by another \textsanskrit{Jotipāla} of the past (\href{https://suttacentral.net/dn19/en/sujato\#47.26}{DN 19:47.26}). Yet another \textsanskrit{Jotipāla} was a religious founder of the past (\href{https://suttacentral.net/an6.54/en/sujato\#18.1}{AN 6.54:18.1}, \href{https://suttacentral.net/an7.73/en/sujato\#2.5}{AN 7.73:2.5}). The name seems to appear only in Buddhist sources. Given that it is always in legendary contexts, it is probably a vocational epithet like \textsanskrit{Ghaṭīkāra}. } Then \textsanskrit{Ghaṭīkāra} addressed \textsanskrit{Jotipāla}, ‘Come, dear \textsanskrit{Jotipāla}, let’s go to see the Blessed One Kassapa, the perfected one, the fully awakened Buddha.\footnote{Kassapa (Sanskrit \textsanskrit{Kaśyapa}) means “tortoise”. It is a common Brahmanical clan name, stemming from an ancient figure reckoned as the eldest of the “seven sages”, to whom some Vedic verses are attributed. Further details of the Buddha Kassapa’s time are found at \href{https://suttacentral.net/dn14/en/sujato\#1.4.6}{DN 14:1.4.6}, \href{https://suttacentral.net/sn15.20/en/sujato\#4.1}{SN 15.20:4.1}, and \href{https://suttacentral.net/sn48.57/en/sujato\#3.1}{SN 48.57:3.1}. } For I deem it holy to see that Blessed One.’\footnote{\textit{\textsanskrit{Sādhusammata}} is normally an epithet of (real or supposed) saints. Here it is the “sight” (\textit{dassana}) of the saint that is deemed holy. This is an early example of the belief known in Hinduism as \textit{darshana}, that the sight of a holy person, place, or object was auspicious and conveyed blessings on the seer. This topic is also treated at \href{https://suttacentral.net/snp4.4/en/sujato}{Snp 4.4} and \href{https://suttacentral.net/an6.30/en/sujato}{AN 6.30}. } 

When\marginnote{6.6} he said this, \textsanskrit{Jotipāla} said to him, ‘Enough, dear \textsanskrit{Ghaṭīkāra}. What’s the use of seeing that shaveling, that fake ascetic?’ 

For\marginnote{6.9} a second time … and a third time, \textsanskrit{Ghaṭīkāra} addressed \textsanskrit{Jotipāla}, ‘Come, dear \textsanskrit{Jotipāla}, let’s go to see the Blessed One Kassapa, the perfected one, the fully awakened Buddha. For I deem it holy to see that Blessed One.’ 

For\marginnote{6.13} a third time, \textsanskrit{Jotipāla} said to him, ‘Enough, dear \textsanskrit{Ghaṭīkāra}. What’s the use of seeing that shaveling, that fake ascetic?’ 

‘Well\marginnote{6.16} then, dear \textsanskrit{Jotipāla}, let’s take some bathing cleanser and go to the river to bathe.’\footnote{\textit{Sotti} is uncertain. I take it as from the root \textit{suc} in the sense “cleanser”. However, it may be related to Sanskrit \textit{\textsanskrit{śukti}} (from the same root), which is mentioned by \textsanskrit{Varāhamihira} as an ingredient in perfumed bath powders fit for a king (\textsanskrit{Bṛhat} \textsanskrit{Saṁhitā} 77). There, it is probably either powdered or ashen oyster shell, which is included as an exfoliant in some modern soaps, or else perfume from a plant whose leaves or flowers resemble oysters. The commentary, identifying it with the \textit{kuruvindakasutti} that was banned for mendicants at \href{https://suttacentral.net/pli-tv-kd15/en/sujato\#1.3.18}{Kd 15:1.3.18}, evidently derives it from \textit{sutta} (“string”) and explains it as a string of resin-balls compounded with \textit{kuruvindaka} (itself uncertain, possibly a herb). However, these explanations all seem dubious, as Pali texts treat \textit{sotti} as a normal means of washing, and should not require specialized ingredients or methods (\href{https://suttacentral.net/an3.70/en/sujato\#8.2}{AN 3.70:8.2}, \href{https://suttacentral.net/mn93/en/sujato\#10.5}{MN 93:10.5}). } 

‘Yes,\marginnote{6.17} dear,’ replied \textsanskrit{Jotipāla}. So that’s what they did. 

Then\marginnote{7.1} \textsanskrit{Ghaṭīkāra} addressed \textsanskrit{Jotipāla}, ‘Dear \textsanskrit{Jotipāla}, the Buddha Kassapa’s monastery is not far away. Let’s go to see the Blessed One Kassapa, the perfected one, the fully awakened Buddha. For I deem it holy to see that Blessed One.’ 

When\marginnote{7.5} he said this, \textsanskrit{Jotipāla} said to him, ‘Enough, dear \textsanskrit{Ghaṭīkāra}. What’s the use of seeing that shaveling, that fake ascetic?’ 

For\marginnote{7.8} a second time … and a third time, \textsanskrit{Ghaṭīkāra} addressed \textsanskrit{Jotipāla}, ‘Dear \textsanskrit{Jotipāla}, the Buddha Kassapa’s monastery is not far away. Let’s go to see the Blessed One Kassapa, the perfected one, the fully awakened Buddha. For I deem it holy to see that Blessed One.’ 

For\marginnote{7.13} a third time, \textsanskrit{Jotipāla} said to him, ‘Enough, dear \textsanskrit{Ghaṭīkāra}. What’s the use of seeing that shaveling, that fake ascetic?’ 

Then\marginnote{8.1} \textsanskrit{Ghaṭīkāra} grabbed \textsanskrit{Jotipāla} by the skirt-hem and said,\footnote{The \textit{\textsanskrit{ovaṭṭikā}} is the turned-over hem of a skirt or sarong that holds it up (see \href{https://suttacentral.net/pli-tv-bu-vb-np10/en/sujato\#1.2.7}{Bu NP 10:1.2.7}). } ‘Dear \textsanskrit{Jotipāla}, the Buddha Kassapa’s monastery is not far away. Let’s go to see the Blessed One Kassapa, the perfected one, the fully awakened Buddha. For I deem it holy to see that Blessed One.’ 

So\marginnote{8.5} \textsanskrit{Jotipāla} undid his skirt-hem and said to \textsanskrit{Ghaṭīkāra},\footnote{Leaving him, apparently, naked standing in the water. } ‘Enough, dear \textsanskrit{Ghaṭīkāra}. What’s the use of seeing that shaveling, that fake ascetic?’ 

Then\marginnote{9.1} \textsanskrit{Ghaṭīkāra} grabbed \textsanskrit{Jotipāla} by the hair of his freshly-washed head and said,\footnote{\textsanskrit{Manusmṛti} 4.83 forbids grabbing hair (\textit{\textsanskrit{keśagraha}}), which was an act of dominance in wrestling. It is forbidden in anger, notes \textsanskrit{Medhātithi}’s commentary, but permitted in sex. \textsanskrit{Mahābhārata} 5.91.11a beautifully explains how such rules are meant to be understood in context: “The learned regard him to be a wretch who does not by his solicitation seek to save a friend who is about to sink in calamity. Striving to the best of his might, even to the extent of seizing him by the hair, one should seek to dissuade a friend from an improper act. In that case, he that acts so, instead of incurring blame, reaps praise” (Kisari Mohan Ganguli’s translation, Section 93, Bhagavat-\textsanskrit{yāna} Parva). } ‘Dear \textsanskrit{Jotipāla}, the Buddha Kassapa’s monastery is not far away. Let’s go to see the Blessed One Kassapa, the perfected one, the fully awakened Buddha. For I deem it holy to see that Blessed One.’ 

Then\marginnote{9.5} \textsanskrit{Jotipāla} thought, ‘Oh, how incredible, how amazing, how this potter \textsanskrit{Ghaṭīkāra}, though of lowly birth, should presume to grab me by the hair of my freshly-washed head!\footnote{Pottery is said to be a low craft at \href{https://suttacentral.net/pli-tv-bu-vb-pc2/en/sujato\#2.1.30}{Bu Pc 2:2.1.30}. } This must be no ordinary matter.’ He said to \textsanskrit{Ghaṭīkāra}, ‘You’d even milk it to this extent, dear \textsanskrit{Ghaṭīkāra}?’\footnote{\textit{\textsanskrit{Dohī}} is “milking”, as in \textit{\textsanskrit{anavasesadohī}} “milk dry”, an idiom which is used in the sense of “pushing things too far” (\href{https://suttacentral.net/mn33/en/sujato\#13.1}{MN 33:13.1}, \href{https://suttacentral.net/an11.17/en/sujato\#12.1}{AN 11.17:12.1}). } 

‘I\marginnote{9.11} even milk it to this extent, dear \textsanskrit{Jotipāla}. For that is how holy I deem it to see that Blessed One.’ 

‘Well\marginnote{9.13} then, dear \textsanskrit{Ghaṭīkāra}, release me, we shall go.’ 

Then\marginnote{10.1} \textsanskrit{Ghaṭīkāra} the potter and \textsanskrit{Jotipāla} the student went to the Buddha Kassapa. \textsanskrit{Ghaṭīkāra} bowed and sat down to one side, but \textsanskrit{Jotipāla} exchanged greetings with the Buddha and sat down to one side. 

\textsanskrit{Ghaṭīkāra}\marginnote{10.2} said to the Buddha Kassapa, ‘Sir, this is the student \textsanskrit{Jotipāla}, my dear friend. Please teach him the Dhamma.’ Then the Buddha Kassapa educated, encouraged, fired up, and inspired \textsanskrit{Ghaṭīkāra} and \textsanskrit{Jotipāla} with a Dhamma talk. Then they got up from their seat, bowed, and respectfully circled the Buddha Kassapa, keeping him on their right, before leaving. 

Then\marginnote{11.1} \textsanskrit{Jotipāla} said to \textsanskrit{Ghatīkāra}, ‘Dear \textsanskrit{Ghaṭīkāra}, you have heard this teaching, so why don’t you go forth from the lay life to homelessness?’ 

‘Don’t\marginnote{11.3} you know, dear \textsanskrit{Jotipāla}, that I  provide for my blind old parents?’ 

‘Well\marginnote{11.4} then, dear \textsanskrit{Ghaṭīkāra}, I shall go forth from the lay life to homelessness.’ 

Then\marginnote{12.1} \textsanskrit{Ghaṭīkāra} and \textsanskrit{Jotipāla} went to the Buddha Kassapa, bowed and sat down to one side. \textsanskrit{Ghaṭīkāra} said to the Buddha Kassapa, ‘Sir, this is the student \textsanskrit{Jotipāla}, my dear friend. Please give him the going forth.’ And \textsanskrit{Jotipāla} the student received the going forth, the ordination in the Buddha’s presence. 

Not\marginnote{13.1} long after \textsanskrit{Jotipāla}’s ordination, a fortnight later, the Buddha Kassapa—having stayed in \textsanskrit{Vebhaliṅga} as long as he pleased—set out for Varanasi.\footnote{Varanasi, one of the oldest cities in the world, was the capital of the \textsanskrit{Kāsī} kingdom. It lost its status as an independent kingdom shortly before the Buddha, when it was taken over by Kosala. It appears in countless Buddhist stories of the past as the dominant city of the region in what appears to be a timeless and ageless past. However, despite its great antiquity, it is a historical settlement, of which the discovered remains date back to perhaps 1200 BCE; it was a capital city from perhaps 800 BCE. It would have enjoyed its status as a major city for a period of a few hundred years before the Buddha, during which time all these stories were set. } Traveling stage by stage, he arrived at Varanasi, where he stayed near Varanasi, in the deer park at Isipatana. King \textsanskrit{Kikī} of \textsanskrit{Kāsi} heard that he had arrived.\footnote{\textsanskrit{Kikī} (“blue jay”) is also the king in the time of Kassapa at \href{https://suttacentral.net/dn14/en/sujato\#1.12.20}{DN 14:1.12.20}. No king of this name appears to be attested in Puranic lineages. By the time of the Buddha, the much-conquered city of Varanasi was subject to first Kosala and then Magadha. This sutta establishes, however, that the true ancient king of the city was in fact a disciple of the past Buddha. | The suttas of this chapter all deal with kings in one way or another. Leaving aside the final five, all of which which Pasenadi of Kosala, we encounter royalty of of Kuru, \textsanskrit{Kāsī}, Videha, \textsanskrit{Sūrasena}, \textsanskrit{Bhaggā}, and indirectly of Avanti and \textsanskrit{Kosambī}. Thus the chapter serves to show that not just the well-known cases of Pasenadi and \textsanskrit{Bimbisāra}, as well as the republican nations such as \textsanskrit{Vajjī} and Sakya, were devoted to the Buddha, but the overwhelming majority of the kings of the time. Many, perhaps all, of these kings belong to what the Puranas later called the “lunar” dynasty, in contrast with the Buddha’s “solar” lineage. } He had the finest carriages harnessed. He then mounted a fine carriage and, along with other fine carriages, set out in full royal pomp from Varanasi to see the Buddha Kassapa. He went by carriage as far as the terrain allowed, then descended and approached the Buddha Kassapa on foot. He bowed and sat down to one side. The Buddha educated, encouraged, fired up, and inspired him with a Dhamma talk. 

Then\marginnote{14.5} King \textsanskrit{Kikī} said to the Buddha, ‘Sir, would the Buddha together with the mendicant \textsanskrit{Saṅgha} please accept tomorrow’s meal from me?’ The Buddha Kassapa consented with silence. 

Then,\marginnote{15.1} knowing that the Buddha had consented, King \textsanskrit{Kikī} got up from his seat, bowed, and respectfully circled the Buddha, keeping him on his right, before leaving. And when the night had passed, King \textsanskrit{Kikī} had delicious fresh and cooked foods prepared in his own home—soft saffron rice with the dark grains picked out, served with many soups and sauces. Then he had the Buddha informed of the time, saying,\footnote{For “soft saffron rice” prefer the reading \textit{\textsanskrit{paṇḍumudikassa}}. } ‘Sir, it’s time. The meal is ready.’ 

Then\marginnote{17.1} Kassapa Buddha robed up in the morning and, taking his bowl and robe, went to the home of King \textsanskrit{Kikī}, where he sat on the seat spread out, together with the \textsanskrit{Saṅgha} of mendicants. Then King \textsanskrit{Kikī} served and satisfied the mendicant \textsanskrit{Saṅgha} headed by the Buddha with his own hands with delicious fresh and cooked foods. 

When\marginnote{17.3} the Buddha Kassapa had eaten and washed his hand and bowl, King \textsanskrit{Kikī} took a low seat and sat to one side. There he said to the Buddha Kassapa, ‘Sir, may the Buddha please accept my invitation to reside in Varanasi for the rainy season. The \textsanskrit{Saṅgha} will be looked after in the same style.’\footnote{The king is promising that the offerings for the whole Sangha will be as luxurious as the ones just offered. } 

‘Enough,\marginnote{17.7} great king. I have already accepted an invitation for the rains residence.’ 

For\marginnote{17.9} a second time … and a third time King \textsanskrit{Kikī} said to the Buddha Kassapa, ‘Sir, may the Buddha please accept my invitation to reside in Varanasi for the rainy season. The \textsanskrit{Saṅgha} will be looked after in the same style.’ 

‘Enough,\marginnote{17.13} Great King. I have already accepted an invitation for the rains residence.’ 

Then\marginnote{17.15} King \textsanskrit{Kikī}, thinking, ‘The Buddha does not accept my invitation to reside for the rains in Varanasi,’ became sad and upset. Then King \textsanskrit{Kikī} said to the Buddha Kassapa, ‘Sir, do you have another supporter better than me?’\footnote{The Buddha does not take the bait of describing someone  as “better”; rather, he describes \textsanskrit{Ghaṭīkāra}’s qualities. } 

‘Great\marginnote{18.1} king, there is a market town named \textsanskrit{Vebhaliṅga}, where there is a potter named \textsanskrit{Ghaṭīkāra}. He is my chief supporter. Now, great king, when you thought, “The Buddha does not accept my invitation to reside for the rains in Varanasi,” you became sad and upset. But \textsanskrit{Ghaṭīkāra} doesn’t get upset, nor will he. 

\textsanskrit{Ghaṭīkāra}\marginnote{18.6} has gone for refuge to the Buddha, the teaching, and the \textsanskrit{Saṅgha}. He doesn’t kill living creatures, steal, commit sexual misconduct, lie, or consume beer, wine, and liquor intoxicants. He has experiential confidence in the Buddha, the teaching, and the \textsanskrit{Saṅgha}, and has the ethics loved by the noble ones. He is free of doubt regarding suffering, its origin, its cessation, and the practice that leads to its cessation.\footnote{This shows him to be one of the noble ones. } He eats in one part of the day; he’s celibate, ethical, and of good character.\footnote{This is equivalent to the eight precepts. } He has set aside gems and gold, and rejected gold and currency.\footnote{The same is said of Buddhist mendicants at \href{https://suttacentral.net/sn42.10/en/sujato\#2.4}{SN 42.10:2.4}. This parallels the ten precepts of the \textit{\textsanskrit{sāmaṇera}}, which forbid using money (\href{https://suttacentral.net/pli-tv-kd1/en/sujato\#56.1.14}{Kd 1:56.1.14}). } He has put down the shovel and doesn’t dig the earth with his own hands.\footnote{Also said of ancient Brahmanical hermits at \href{https://suttacentral.net/dn27/en/sujato\#22.6}{DN 27:22.6}. Digging the earth is an offence for mendicants (\href{https://suttacentral.net/pli-tv-bu-vb-pc10/en/sujato}{Bu Pc 10}). } He takes what has crumbled off by a riverbank or been dug up by mice, and brings it back in a carrier. When he has made a pot, he says, “Anyone may leave bagged sesame, mung beans, or chickpeas here and take what they wish.”\footnote{He made a living by purely voluntary exchange rather than contractual obligations. } He provided for his blind old parents. And since he has ended the five lower fetters, \textsanskrit{Ghaṭīkāra} will be reborn spontaneously and will become extinguished there, not liable to return from that world.\footnote{He is a non-returner. } 

This\marginnote{19.1} one time, great king, I was staying near the market town of \textsanskrit{Vebhaliṅga}. Then I robed up in the morning and, taking my bowl and robe, went to the home of \textsanskrit{Ghaṭīkāra}’s parents, where I said to them, “Excuse me, where has Bhaggava gone?”\footnote{The Buddha uses \textsanskrit{Ghaṭīkāra}’s clan name, as he does at \href{https://suttacentral.net/sn1.50/en/sujato\#11.2}{SN 1.50:11.2}. } 

“Your\marginnote{19.4} supporter has gone out, sir. But take rice from the pot and sauce from the pan and eat.” So that’s what I did. And after eating I got up from my seat and left.\footnote{This appears to technically violate \href{https://suttacentral.net/pli-tv-bu-vb-pc40/en/sujato}{Bu Pc 40}, which requires that food be given with the hands, not simply by speech. However, not all Buddhas lay down the exact same Vinaya rules. Several parallels explain this by saying that Kassapa was following the customs of the mystical land of Uttarakuru (MA 63 at T i 502a19; T i 502b8; \textsanskrit{Saṅghabhedavastu} 17:391 at Gnoli 1978a:27,14 and D (1) \emph{’dul ba}, \emph{ga} 8a3), where no-one had any possessions (\href{https://suttacentral.net/dn32/en/sujato\#7.4}{DN 32:7.4}). } 

Then\marginnote{19.6} \textsanskrit{Ghaṭīkāra} went up to his parents and said, “Who took rice from the pot and sauce from the pan, ate it, and left?” 

“It\marginnote{19.8} was the Buddha Kassapa, my dear.” 

Then\marginnote{19.9} \textsanskrit{Ghaṭīkāra} thought, “I’m so fortunate, so very fortunate, in that the Buddha Kassapa trusts me so much!” Then joy and happiness did not leave him for a fortnight, or his parents for a week. 

Another\marginnote{20.1} time, great king, I was staying near that same market town of \textsanskrit{Vebhaliṅga}. Then I robed up in the morning and, taking my bowl and robe, went to the home of \textsanskrit{Ghaṭīkāra}’s parents, where I said to them, “Excuse me, where has Bhaggava gone?” 

“Your\marginnote{20.4} supporter has gone out, sir. But take porridge from the pot and sauce from the pan and eat.” So that’s what I did. And after eating I got up from my seat and left. 

Then\marginnote{20.6} \textsanskrit{Ghaṭīkāra} went up to his parents and said, “Who took porridge from the pot and sauce from the pan, ate it, and left?” 

“It\marginnote{20.8} was the Buddha Kassapa, my dear.” 

Then\marginnote{20.9} \textsanskrit{Ghaṭīkāra} thought, “I’m so fortunate, so very fortunate, to be trusted so much by the Buddha Kassapa!” Then joy and happiness did not leave him for a fortnight, or his parents for a week. 

Another\marginnote{21.1} time, great king, I was staying near that same market town of \textsanskrit{Vebhaliṅga}. Now at that time my hut leaked. So I addressed the mendicants, 

“Mendicants,\marginnote{21.4} go to \textsanskrit{Ghaṭīkāra}’s home and find some grass.” 

When\marginnote{21.5} I said this, those mendicants said to me, “Sir, there’s no grass there, but his workshop has a grass roof.”\footnote{\textit{Āvesana} is found here and at \href{https://suttacentral.net/mn140/en/sujato\#2.1}{MN 140:2.1} of a potter’s workshop, at \href{https://suttacentral.net/pli-tv-kd15/en/sujato\#11.5.15}{Kd 15:11.5.15} of a sewing hut, and is mentioned at Manu 9.264 as an “artisan’s workshop” (\textit{\textsanskrit{kārukāveśana}}). } 

“Then\marginnote{21.7} go to the workshop and strip the grass.” So that’s what they did. 

Then\marginnote{21.9} \textsanskrit{Ghaṭīkāra}’s parents said to those mendicants, “Who’s stripping the grass from the workshop?” 

“It’s\marginnote{21.11} the mendicants, sister. The Buddha’s hut is leaking.” 

“Take\marginnote{21.12} it, sirs! Take it, dearest ones!” 

Then\marginnote{21.13} \textsanskrit{Ghaṭīkāra} went up to his parents and said, “Who stripped the grass from the workshop?” 

“It\marginnote{21.15} was the mendicants, dear. It seems the Buddha’s hut is leaking.” 

Then\marginnote{21.16} \textsanskrit{Ghaṭīkāra} thought, “I’m so fortunate, so very fortunate, to be trusted so much by the Buddha Kassapa!” Then joy and happiness did not leave him for a fortnight, or his parents for a week. 

Then\marginnote{21.20} the workshop remained with the atmosphere for a roof for the whole three months, but the heavens did not rain on it. And that, great king, is what \textsanskrit{Ghaṭīkāra} the potter is like.’ 

‘\textsanskrit{Ghaṭīkāra}\marginnote{21.22} the potter is fortunate, very fortunate, to be so trusted by the Buddha Kassapa.’ 

Then\marginnote{22.1} King \textsanskrit{Kikī} sent around five hundred cartloads of rice, soft saffron rice, and suitable sauce to \textsanskrit{Ghaṭīkāra}.\footnote{“Suitable sauce” (\textit{\textsanskrit{tadupiyañca} \textsanskrit{sūpeyyaṁ}}) is also at \href{https://suttacentral.net/dn17/en/sujato\#2.15.10}{DN 17:2.15.10} and \href{https://suttacentral.net/sn22.96/en/sujato\#4.10}{SN 22.96:4.10}, where it describes a similarly luxurious meal. | A secondary theme of this parable is to highlight the different attitudes to offering. \textsanskrit{Kikī} is generous, but his offering is extravagant and indulgent, and ultimately serves his desires, whereas \textsanskrit{Ghaṭīkāra} is concerned only to help with what is needed at the time. } Then one of the king’s men approached \textsanskrit{Ghaṭīkāra} and said, ‘Sir, these five hundred cartloads of rice, soft saffron rice, and suitable sauce have been sent to you by King \textsanskrit{Kikī} of \textsanskrit{Kāsi}. Please accept them.’ 

‘The\marginnote{22.5} king has many duties, and much to do. I have enough.\footnote{The two words \textit{\textsanskrit{alaṁ} me} encapsulate the radical basis of a Buddhist economy. } Let this be for the king himself.’ 

Ānanda,\marginnote{23.1} you might think: ‘Surely the student \textsanskrit{Jotipāla} must have been someone else at that time?’\footnote{This phrase marks canonical \textsanskrit{Jātaka} stories at \href{https://suttacentral.net/dn17/en/sujato\#2.14.2}{DN 17:2.14.2}, \href{https://suttacentral.net/mn83/en/sujato\#21.2}{MN 83:21.2}, \href{https://suttacentral.net/an3.15/en/sujato\#5.2}{AN 3.15:5.2}, and \href{https://suttacentral.net/an9.20/en/sujato\#5.2}{AN 9.20:5.2}. } But you should not see it like this. I myself was the student \textsanskrit{Jotipāla} at that time.” 

That\marginnote{23.5} is what the Buddha said. Satisfied, Venerable Ānanda approved what the Buddha said. 

%
\section*{{\suttatitleacronym MN 82}{\suttatitletranslation With Raṭṭhapāla }{\suttatitleroot Raṭṭhapālasutta}}
\addcontentsline{toc}{section}{\tocacronym{MN 82} \toctranslation{With Raṭṭhapāla } \tocroot{Raṭṭhapālasutta}}
\markboth{With Raṭṭhapāla }{Raṭṭhapālasutta}
\extramarks{MN 82}{MN 82}

\scevam{So\marginnote{1.1} I have heard.\footnote{The story of \textsanskrit{Raṭṭhapāla}, the “guardian of the nation”, has become one of the most famous and beloved parables from early Buddhism. It is a testament to a young man’s resolution to go forth out of faith, a quality for which he was later honored (\href{https://suttacentral.net/an1.210/en/sujato\#1.1}{AN 1.210:1.1}). But the power of \textsanskrit{Raṭṭhapāla}’s determination should not overshadow the moving conversation he has with King Koravya at the end. } }At one time the Buddha was wandering in the land of the Kurus together with a large \textsanskrit{Saṅgha} of mendicants when he arrived at a town of the Kurus named \textsanskrit{Thullakoṭṭhika}.\footnote{\textsanskrit{Thullakoṭṭhika} (“place of fat granaries”; Sanskrit \textit{\textsanskrit{sthūlakoṣṭhaka}}) only appears in the story of \textsanskrit{Raṭṭhapāla}. } 

The\marginnote{2.1} brahmins and householders of \textsanskrit{Thullakoṭṭhika} heard: 

“It\marginnote{2.2} seems the ascetic Gotama—a Sakyan, gone forth from a Sakyan family—has arrived at \textsanskrit{Thullakoṭṭhika}, together with a large \textsanskrit{Saṅgha} of mendicants. He has this good reputation: ‘That Blessed One is perfected, a fully awakened Buddha, accomplished in knowledge and conduct, holy, knower of the world, supreme guide for those who wish to train, teacher of gods and humans, awakened, blessed.’ He has realized with his own insight this world—with its gods, \textsanskrit{Māras}, and divinities, this population with its ascetics and brahmins, gods and humans—and he makes it known to others. He proclaims a teaching that is good in the beginning, good in the middle, and good in the end, meaningful and well-phrased. And he reveals a spiritual practice that’s entirely full and pure. It’s good to see such perfected ones.” 

Then\marginnote{3.1} the brahmins and householders of \textsanskrit{Thullakoṭṭhika} went up to the Buddha. Before sitting down to one side, some bowed, some exchanged greetings and polite conversation, some held up their joined palms toward the Buddha, some announced their name and clan, while some kept silent. When they were seated, the Buddha educated, encouraged, fired up, and inspired them with a Dhamma talk. 

Now\marginnote{4.1} at that time a gentleman named \textsanskrit{Raṭṭhapāla}, the son of the leading clan in \textsanskrit{Thullakoṭṭhika}, was sitting in the assembly. He thought, “As I understand the Buddha’s teaching, it’s not easy for someone living at home to lead the spiritual life utterly full and pure, like a polished shell.\footnote{From this point, the narrative is largely echoed in the story of Sudinna (\href{https://suttacentral.net/pli-tv-bu-vb-pj1/en/sujato\#5.1.7}{Bu Pj 1:5.1.7}), with, however, a very different outcome. While the exact relationship between these shared narratives is not easy to determine, it seems likely that Sudinna’s story adopted elements of \textsanskrit{Raṭṭhapāla}’s. } Why don’t I shave off my hair and beard, dress in ocher robes, and go forth from lay life to homelessness?” 

Then,\marginnote{5.1} having approved and agreed with what the Buddha said, the brahmins and householders of \textsanskrit{Thullakoṭṭhika} got up from their seat, bowed, and respectfully circled the Buddha, keeping him on their right, before leaving. 

Soon\marginnote{6.1} after they left, \textsanskrit{Raṭṭhapāla} went up to the Buddha, bowed, sat down to one side, and said to him, “Sir, as I understand the Buddha’s teaching, it’s not easy for someone living at home to lead the spiritual life utterly full and pure, like a polished shell. I wish to shave off my hair and beard, dress in ocher robes, and go forth from the lay life to homelessness. Sir, may I receive the going forth, the ordination in the Buddha’s presence? May the Buddha please give me the going forth!” 

“But,\marginnote{6.6} \textsanskrit{Raṭṭhapāla}, do you have your parents’ permission?”\footnote{This Vinaya rule is not mentioned elsewhere in the suttas, although it was required also of Sudinna (\href{https://suttacentral.net/pli-tv-bu-vb-pj1/en/sujato\#5.1.14}{Bu Pj 1:5.1.14}). It was instituted on the emotional request of the Buddha’s own father, who was distraught when his grandson \textsanskrit{Rāhula} went forth while still a boy (\href{https://suttacentral.net/pli-tv-kd1/en/sujato\#54.4.2}{Kd 1:54.4.2}). It originally applied to novices, who went forth as young as fifteen or “when old enough to scare crows” (\href{https://suttacentral.net/pli-tv-kd1/en/sujato\#51.1.14}{Kd 1:51.1.14}). Over time the requirements for ordination tend to become stricter as things required in a particular context are applied generally. If this is the case here, the process must have happened quickly, so that even in the canonical full ordination procedure a candidate, who must be twenty years at least, must have permission of their parents (\href{https://suttacentral.net/pli-tv-kd1/en/sujato\#76.1.12}{Kd 1:76.1.12}). However, there are many people who cannot get such permission, such as orphans. Thus if one does give the going-forth to someone without permission of the parents, it is a minor offence of wrong-doing for the mendicant performing the ordination, but it does not invalidate it (\href{https://suttacentral.net/pli-tv-kd1/en/sujato\#54.6.5}{Kd 1:54.6.5}). } 

“No,\marginnote{6.7} sir.” 

“\textsanskrit{Raṭṭhapāla},\marginnote{6.8} Buddhas don’t give the going forth to the child of parents who haven’t given their permission.” 

“I’ll\marginnote{6.9} make sure, sir, to get my parents’ permission.” 

Then\marginnote{7.1} \textsanskrit{Raṭṭhapāla} got up from his seat, bowed, and respectfully circled the Buddha. Then he went to his parents and said, “Mum and dad, as I understand the Buddha’s teachings, it’s not easy for someone living at home to lead the spiritual life utterly full and pure, like a polished shell. I wish to shave off my hair and beard, dress in ocher robes, and go forth from the lay life to homelessness. Please give me permission to go forth.” 

When\marginnote{7.5} he said this, \textsanskrit{Raṭṭhapāla}’s parents said to him, “But, dear \textsanskrit{Raṭṭhapāla}, you’re our only child. You’re dear to us and we love you. You’re dainty and raised in comfort. You know nothing of suffering. Even when you die we will lose you against our wishes. So how can we allow you to go forth while you’re still alive?” 

For\marginnote{7.10} a second time, and a third time, \textsanskrit{Raṭṭhapāla} asked his parents for permission, but got the same reply. 

Then\marginnote{7.20} \textsanskrit{Raṭṭhapāla} thought, “My parents don’t allow me to go forth.” He laid down right there on the bare ground, saying,\footnote{A similar method was employed by \textsanskrit{Bhaddā} \textsanskrit{Kuṇḍalakesā}, according to the commentary to \href{https://suttacentral.net/thig5.9/en/sujato}{Thig 5.9}. She lay face down until her parents allowed her to marry the man she desired, who was, unfortunately, a convicted criminal. The marriage did not end well, but she persevered and ultimately triumphed. } “I’ll either die right here or go forth.” And he refused to eat, up to the seventh meal. 

Then\marginnote{8.1} \textsanskrit{Raṭṭhapāla}’s parents said to him, “Dear \textsanskrit{Raṭṭhapāla}, you’re our only child. You’re dear to us and we love you. You’re dainty and raised in comfort. You know nothing of suffering. When you die we will lose you against our wishes. So how can we allow you to go forth from lay life to homelessness while you’re still living? Get up, \textsanskrit{Raṭṭhapāla}! Eat, drink, and amuse yourself. While enjoying sensual pleasures, delight in making merit. We don’t allow you to go forth. When you die we will lose you against our wishes. So how can we allow you to go forth while you’re still alive?” 

When\marginnote{8.11} they said this, \textsanskrit{Raṭṭhapāla} kept silent. 

For\marginnote{8.12} a second time, and a third time, \textsanskrit{Raṭṭhapāla}’s parents made the same request. 

And\marginnote{9.1} for a third time, \textsanskrit{Raṭṭhapāla} kept silent. \textsanskrit{Raṭṭhapāla}’s parents then went to see his friends. They told them of the situation and asked for their help. 

Then\marginnote{10.1} \textsanskrit{Raṭṭhapāla}’s friends went to him and said, “Our friend \textsanskrit{Raṭṭhapāla}, you are your parents’ only child. You’re dear to them and they love you. You’re dainty and raised in comfort. You know nothing of suffering. When you die your parents will lose you against their wishes. So how can they allow you to go forth while you’re still alive? Get up, \textsanskrit{Raṭṭhapāla}! Eat, drink, and amuse yourself. While enjoying sensual pleasures, delight in making merit. Your parents will not allow you to go forth. Even when you die your parents will lose you against their wishes. So how can they allow you to go forth while you’re still alive?” 

When\marginnote{10.11} they said this, \textsanskrit{Raṭṭhapāla} kept silent. 

For\marginnote{10.12} a second time, and a third time, \textsanskrit{Raṭṭhapāla}’s friends made the same request. And for a third time, \textsanskrit{Raṭṭhapāla} kept silent. 

Then\marginnote{11.1} \textsanskrit{Raṭṭhapāla}’s friends went to his parents and said, “Mum and dad, \textsanskrit{Raṭṭhapāla} is lying there on the bare ground saying: ‘I’ll either die right here or go forth.’ If you don’t allow him to go forth, he’ll die there. But if you do allow him to go forth, you’ll see him again afterwards. And if he doesn’t enjoy the renunciate life, where else will he have to go? He’ll come right back here. Please give \textsanskrit{Raṭṭhapāla} permission to go forth.” 

“Then,\marginnote{11.8} dears, we give \textsanskrit{Raṭṭhapāla} permission to go forth. But once gone forth he must visit his parents.” 

Then\marginnote{11.10} \textsanskrit{Raṭṭhapāla}’s friends went to him and said, “Get up, \textsanskrit{Raṭṭhapāla}! Your parents have given you permission to go forth from lay life to homelessness. But once gone forth you must visit your parents.” 

\textsanskrit{Raṭṭhapāla}\marginnote{12.1} got up and regained his strength. He went to the Buddha, bowed, sat down to one side, and said to him, “Sir, I have my parents’ permission to go forth from the lay life to homelessness. May the Buddha please give me the going forth.” 

And\marginnote{13.1} \textsanskrit{Raṭṭhapāla} received the going forth, the ordination in the Buddha’s presence. Not long after Venerable \textsanskrit{Raṭṭhapāla}’s ordination, a fortnight later, the Buddha—having stayed in \textsanskrit{Thullakoṭṭhika} as long as he pleased—set out for \textsanskrit{Sāvatthī}. Traveling stage by stage, he arrived at \textsanskrit{Sāvatthī}, where he stayed in Jeta’s Grove, \textsanskrit{Anāthapiṇḍika}’s monastery. 

Then\marginnote{14.1} Venerable \textsanskrit{Raṭṭhapāla}, living alone, withdrawn, diligent, keen, and resolute, soon realized the supreme end of the spiritual path in this very life. He lived having achieved with his own insight the goal for which gentlemen rightly go forth from the lay life to homelessness.\footnote{While the conventional phrase “soon” (\textit{nacirasseva}) is used, the commentary says it was twelve years, which would explain why his family did not immediately recognize him. } 

He\marginnote{14.2} understood: “Rebirth is ended; the spiritual journey has been completed; what had to be done has been done; there is nothing further for this place.” And Venerable \textsanskrit{Raṭṭhapāla} became one of the perfected. 

Then\marginnote{15.1} he went up to the Buddha, bowed, sat down to one side, and said to him, “Sir, I would like to visit my parents, if the Buddha allows it.” 

Then\marginnote{15.3} the Buddha focused on comprehending \textsanskrit{Raṭṭhapāla}’s mind. When he knew that it was impossible for \textsanskrit{Raṭṭhapāla} to resign the training and return to a lesser life, he said, “Please, \textsanskrit{Raṭṭhapāla}, go at your convenience.” 

And\marginnote{16.1} then \textsanskrit{Raṭṭhapāla} got up from his seat, bowed, and respectfully circled the Buddha, keeping him on his right. Then he set his lodgings in order and, taking his bowl and robe, set out for \textsanskrit{Thullakoṭṭhika}. Traveling stage by stage, he arrived at \textsanskrit{Thullakoṭṭhika}, where he stayed in King Koravya’s deer range. Then \textsanskrit{Raṭṭhapāla} robed up in the morning and, taking his bowl and robe, entered \textsanskrit{Thullakoṭṭhika} for alms. Wandering indiscriminately for almsfood, he approached his own father’s house. 

Now\marginnote{17.1} at that time \textsanskrit{Raṭṭhapāla}’s father was having his hair dressed in the hall of the middle gate.\footnote{This is a sign of the dandified life expected of \textsanskrit{Raṭṭhapāla}. Hairstyles for upper class men of the time were quite fancy, or at least they are depicted so in the early sculptures at Sanchi. } He saw \textsanskrit{Raṭṭhapāla} coming off in the distance and said, 

“Our\marginnote{17.4} dear and beloved only son was made to go forth by these shavelings, these fake ascetics!” And at his own father’s home \textsanskrit{Raṭṭhapāla} received neither alms nor a polite refusal, but only abuse. 

Now\marginnote{18.1} at that time a family bondservant wanted to throw away the previous night’s porridge. So \textsanskrit{Raṭṭhapāla} said to her, “If that’s to be thrown away, sister, pour it here in my bowl.” As she was pouring the porridge into his bowl, she recognized the features of his hands, feet, and voice.\footnote{This detail is remarkably similar to the \emph{Odyssey} 19:361–475, where due to the marks on his feet, the maidservant Eurycleia recognizes Odysseus on his return even though his own wife did not. } 

She\marginnote{18.5} then went to his mother and said, “Please, madam, you should know this. Master \textsanskrit{Raṭṭhapāla} has arrived.” 

“If\marginnote{18.8} you speak the truth, wench, I’ll make you a free woman!” 

Then\marginnote{18.9} \textsanskrit{Raṭṭhapāla}’s mother went to his father and said, “Please householder, you should know this. It seems our son \textsanskrit{Raṭṭhapāla} has arrived.” 

Now\marginnote{19.1} at that time \textsanskrit{Raṭṭhapāla} was eating last night’s porridge by a wall. Then \textsanskrit{Raṭṭhapāla}’s father went up to him and said, “Dear \textsanskrit{Raṭṭhapāla}! How can you be eating last night’s porridge?\footnote{The same syntax recurs at \href{https://suttacentral.net/an5.166/en/sujato\#11.2}{AN 5.166:11.2}, where it is clear the future tense expresses indignation rather than future time. } Why not go to your own home?”\footnote{In the Vinaya, the exchange between \textsanskrit{Raṭṭhapāla} and his father is expressed in verse (\href{https://suttacentral.net/pli-tv-bu-vb-ss6/en/sujato\#1.5.1}{Bu Ss 6:1.5.1}). } 

“Householder,\marginnote{19.5} how could those of us who have gone forth from the lay life to homelessness have a house? We are homeless, householder. I came to your house, but there I received neither alms nor a polite refusal, but only abuse.” 

“Come,\marginnote{19.9} dear \textsanskrit{Raṭṭhapāla}, let’s go to the house.” 

“Enough,\marginnote{19.10} householder. My meal is finished for today.” 

“Well\marginnote{19.11} then, dear \textsanskrit{Raṭṭhapāla}, please accept tomorrow’s meal from me.” \textsanskrit{Raṭṭhapāla} consented with silence. 

Then,\marginnote{20.1} knowing that \textsanskrit{Raṭṭhapāla} had consented, his father went back to his own house. He made a heap of gold, both coined and uncoined, and hid it under mats. Then he addressed \textsanskrit{Raṭṭhapāla}’s former wives, “Please, daughters-in-law, adorn yourselves in the way that our son \textsanskrit{Raṭṭhapāla} found you most adorable.” 

And\marginnote{21.1} when the night had passed \textsanskrit{Raṭṭhapāla}’s father had delicious fresh and cooked foods prepared in his own home, and announced the time to the Venerable \textsanskrit{Raṭṭhapāla}, saying, “Sir, it’s time. The meal is ready.” 

Then\marginnote{22.1} \textsanskrit{Raṭṭhapāla} robed up in the morning and, taking his bowl and robe, went to his father’s home, and sat down on the seat spread out. \textsanskrit{Raṭṭhapāla}’s father, revealing the heap of gold, both coined and uncoined, said to him, “Dear \textsanskrit{Raṭṭhapāla}, this is your maternal fortune. There’s another paternal fortune, and an ancestral one. You can both enjoy your wealth and make merit. Come, return to a lesser life, enjoy wealth, and make merit!” 

“If\marginnote{22.5} you’d follow my advice, householder, you’d have this heap of gold, both coined and uncoined, loaded on a cart and carried away to be dumped in the middle of the Ganges river. Why is that? Because this will bring you nothing but sorrow, lamentation, pain, sadness, and distress.” 

Then\marginnote{23.1} \textsanskrit{Raṭṭhapāla}’s former wives each clasped his feet and said, “What are they like, master, the nymphs for whom you lead the spiritual life?” 

“Sisters,\marginnote{23.3} I don’t lead the spiritual life for the sake of nymphs.”\footnote{\textsanskrit{Raṭṭhapāla} refers to his former wives as sisters, an indication that he sees them in accordance with the advice in \href{https://suttacentral.net/sn35.127/en/sujato\#1.6}{SN 35.127:1.6}, to regard women of one’s one age as a sister. Notably, he has already used the same respectful term when addressing the household maid, whereas his mother uses the disrespectful “wench” (\textit{je}). } 

Saying,\marginnote{23.4} “Our master \textsanskrit{Raṭṭhapāla} addresses us as sisters!” they fainted right away. 

Then\marginnote{24.1} \textsanskrit{Raṭṭhapāla} said to his father, “If there is food to be given, householder, please give it. But don’t harass me.” 

“Eat,\marginnote{24.4} dear \textsanskrit{Raṭṭhapāla}. The meal is ready.” Then \textsanskrit{Raṭṭhapāla}’s father served and satisfied Venerable \textsanskrit{Raṭṭhapāla} with his own hands with delicious fresh and cooked foods. 

When\marginnote{24.6} he had eaten and washed his hand and bowl, he recited this verse while standing right there: 

\begin{verse}%
“See\marginnote{25.1} this fancy puppet,\footnote{The following verses are also found in \textsanskrit{Raṭṭhapāla}’s verses at \href{https://suttacentral.net/thag16.4/en/sujato}{Thag 16.4}. } \\
a body built of sores, \\
diseased, obsessed over, \\
in which nothing lasts at all. 

See\marginnote{25.5} this fancy figure, \\
with its gems and earrings; \\
it is bones encased in skin, \\
made pretty by its clothes. 

Rouged\marginnote{25.9} feet \\
and powdered face \\
may be enough to beguile a fool, \\
but not a seeker of the far shore. 

Hair\marginnote{25.13} in eight braids \\
and eyeshadow \\
may be enough to beguile a fool, \\
but not a seeker of the far shore. 

A\marginnote{25.17} rotting body all adorned \\
like a freshly painted makeup box \\
may be enough to beguile a fool, \\
but not a seeker of the far shore. 

The\marginnote{25.21} hunter laid his snare, \\
but the deer didn’t spring the trap. \\
I’ve eaten the bait and now I go, \\
leaving the trapper to lament.” 

%
\end{verse}

Then\marginnote{26.1} \textsanskrit{Raṭṭhapāla}, having recited this verse while standing, went to King Koravya’s deer range and sat at the root of a tree for the day’s meditation. 

Then\marginnote{27.1} King Koravya addressed his gamekeeper,\footnote{Koravya (Sanskrit \textit{kauravya} or \textit{kaurava}) is the conventional name for descendants of the legendary Kuru, founding king of the Kurus. } “My good gamekeeper, tidy up the park of the deer range. We will go to see the scenery.” 

“Yes,\marginnote{27.4} Your Majesty,” replied the gamekeeper. While tidying the deer range he saw \textsanskrit{Raṭṭhapāla} sitting in meditation. Seeing this, he went to the king, and said, “The deer range is tidy, sire. And the gentleman named \textsanskrit{Raṭṭhapāla}, the son of the leading clan in \textsanskrit{Thullakoṭṭhika}, of whom you have often spoken highly, is meditating there at the root of a tree.” 

“Well\marginnote{27.8} then, my good gamekeeper, that’s enough of the park for today. Now I shall pay homage to Mister \textsanskrit{Raṭṭhapāla}.” 

And\marginnote{28.1} then King Koravya said, “Give away the fresh and cooked foods that have been prepared there.” He had the finest carriages harnessed. Then he mounted a fine carriage and, along with other fine carriages, set out in full royal pomp from \textsanskrit{Thullakoṭṭhika} to see \textsanskrit{Raṭṭhapāla}. He went by carriage as far as the terrain allowed, then descended and approached \textsanskrit{Raṭṭhapāla} on foot, together with a group of eminent officials. They exchanged greetings, and, when the greetings and polite conversation were over, he stood to one side and said to \textsanskrit{Raṭṭhapāla}: 

“Here,\marginnote{28.3} Mister \textsanskrit{Raṭṭhapāla}, sit on this elephant rug.” 

“Enough,\marginnote{28.4} great king, you sit on it. I’m sitting on my own seat.” 

So\marginnote{28.6} the king sat down on the seat spread out, and said: 

“Mister\marginnote{29.1} \textsanskrit{Raṭṭhapāla}, there are these four kinds of decay. Because of these, some people shave off their hair and beard, dress in ocher robes, and go forth from the lay life to homelessness. What four? Decay due to old age, decay due to sickness, decay of wealth, and decay of relatives. 

And\marginnote{30.1} what is decay due to old age? It’s when someone is old, elderly, and senior, advanced in years, and has reached the final stage of life. They reflect: ‘I’m now old, elderly, and senior. I’m advanced in years and have reached the final stage of life. It’s not easy for me to acquire more wealth or to increase the wealth I’ve already acquired. Why don’t I shave off my hair and beard, dress in ocher robes, and go forth from the lay life to homelessness?’ So because of that decay due to old age they go forth. This is called decay due to old age. But Mister \textsanskrit{Raṭṭhapāla} is now a youth, young, with pristine black hair, blessed with youth, in the prime of life.\footnote{While \textsanskrit{Raṭṭhapāla}’s age is not given, he seems quite young when he went forth. He addresses his parents with the familiar “mum and dad”; permission of his parents is required; his friends play a major role influencing his decisions; and he lies on the ground refusing to move until he gets what he wants from his parents. All this sounds like a teenager. True, he was married, but such marriages could be arranged young. And now here, twelve years later according to the commentary, he is still a young man. } You have no decay due to old age. So what did you know or see or hear that made you go forth? 

And\marginnote{31.1} what is decay due to sickness? It’s when someone is sick, suffering, gravely ill. They reflect: ‘I’m now sick, suffering, gravely ill. It’s not easy for me to acquire more wealth or to increase the wealth I’ve already acquired. Why don’t I go forth from the lay life to homelessness?’ So because of that decay due to sickness they go forth. This is called decay due to sickness. But Mister \textsanskrit{Raṭṭhapāla} is now rarely ill or unwell. Your stomach digests well, being neither too hot nor too cold. You have no decay due to sickness. So what did you know or see or hear that made you go forth? 

And\marginnote{32.1} what is decay of wealth? It’s when someone is rich, affluent, and wealthy. But gradually their wealth dwindles away. They reflect: ‘I used to be rich, affluent, and wealthy. But gradually my wealth has dwindled away. It’s not easy for me to acquire more wealth or to increase the wealth I’ve already acquired. Why don’t I go forth from the lay life to homelessness?’ So because of that decay of wealth they go forth. This is called decay of wealth. But Mister \textsanskrit{Raṭṭhapāla} is the son of the leading clan here in \textsanskrit{Thullakoṭṭhika}. You have no decay of wealth. So what did you know or see or hear that made you go forth? 

And\marginnote{33.1} what is decay of relatives? It’s when someone has many friends and colleagues, relatives and kin. But gradually their relatives dwindle away. They reflect: ‘I used to have many friends and colleagues, relatives and kin. But gradually they’ve dwindled away. It’s not easy for me to acquire more wealth or to increase the wealth I’ve already acquired. Why don’t I shave off my hair and beard, dress in ocher robes, and go forth from the lay life to homelessness?’ So because of that decay of relatives they go forth. This is called decay of relatives. But Mister \textsanskrit{Raṭṭhapāla} has many friends and colleagues, relatives and kin right here in \textsanskrit{Thullakoṭṭhika}. You have no decay of relatives. So what did you know or see or hear that made you go forth? 

There\marginnote{34.1} are these four kinds of decay. Because of these, some people shave off their hair and beard, dress in ocher robes, and go forth from the lay life to homelessness. Mister \textsanskrit{Raṭṭhapāla} has none of these. So what did you know or see or hear that made you go forth?” 

“Great\marginnote{35.1} king, the Blessed One who knows and sees, the perfected one, the fully awakened Buddha has taught these four summaries of the teaching for recitation. It was after knowing and seeing and hearing these that I went forth from the lay life to homelessness.\footnote{While these four summaries reflect key teachings of the Buddha, they are not found as such in early texts. } 

What\marginnote{35.2} four? 

‘The\marginnote{36.1} world is unstable and swept away.’ This is the first summary. 

‘The\marginnote{36.3} world has no shelter and no savior.’ This is the second summary. 

‘The\marginnote{36.5} world has no owner—you must leave it all behind and pass on.’ This is the third summary. 

‘The\marginnote{36.7} world is wanting, insatiable, the slave of craving.’ This is the fourth summary. 

The\marginnote{37.1} Blessed One who knows and sees, the perfected one, the fully awakened Buddha taught these four summaries of the teaching. It was after knowing and seeing and hearing these that I went forth from the lay life to homelessness.” 

“‘The\marginnote{38.1} world is unstable and swept away.’ So Mister \textsanskrit{Raṭṭhapāla} said. How should I see the meaning of this statement?” 

“What\marginnote{38.4} do you think, great king? When you were twenty or twenty-five years of age, were you proficient at riding elephants, horses, and chariots, and at archery and swordsmanship? Were you strong in thigh and arm, capable, and battle-hardened?”\footnote{For \textit{alamatto} read \textit{alamattho}, “capable”. } 

“I\marginnote{38.6} was, Mister \textsanskrit{Raṭṭhapāla}. Sometimes it seems as if I had superpowers then. I don’t see anyone who could have equalled me in strength.” 

“What\marginnote{38.8} do you think, great king? These days are you just as strong in thigh and arm, capable, and battle-hardened?” 

“No,\marginnote{38.10} Mister \textsanskrit{Raṭṭhapāla}. For now I am old, elderly, and senior, I’m advanced in years and have reached the final stage of life. I am eighty years old. Sometimes I intend to step in one place, but my foot goes somewhere else.” 

“This\marginnote{38.13} is what the Buddha was referring to when he said: ‘The world is unstable and swept away.’” 

“It’s\marginnote{38.16} incredible, Mister \textsanskrit{Raṭṭhapāla}, it’s amazing, how well said this was by the Buddha. For the world is indeed unstable and swept away. 

In\marginnote{39.1} this royal court you can find divisions of elephants, cavalry, chariots, and infantry. They will serve to defend us from any threats. Yet you said: ‘The world has no shelter and no savior.’ How should I see the meaning of this statement?” 

“What\marginnote{39.5} do you think, great king? Do you have any chronic ailments?” 

“Yes,\marginnote{39.7} I do. Sometimes my friends and colleagues, relatives and kin surround me, thinking: ‘Now the king will die! Now the king will die!’” 

“What\marginnote{39.10} do you think, great king? Can you get your friends and colleagues, relatives and kin to help: ‘Please, my dear friends and colleagues, relatives and kin, all of you here share my pain so that I may feel less pain.’ Or must you alone feel that pain?” 

“I\marginnote{39.14} can’t get my friends to share my pain. Rather, I alone must feel it.” 

“This\marginnote{39.17} is what the Buddha was referring to when he said: ‘The world has no shelter and no savior.’” 

“It’s\marginnote{39.20} incredible, Mister \textsanskrit{Raṭṭhapāla}, it’s amazing, how well said this was by the Buddha. For the world indeed has no shelter and no savior. 

In\marginnote{40.1} this royal court you can find abundant gold, both coined and uncoined, stored in dungeons and towers. Yet you said: ‘The world has no owner—you must leave it all behind and pass on.’ How should I see the meaning of this statement?” 

“What\marginnote{40.5} do you think, great king? These days you amuse yourself, supplied and provided with the five kinds of sensual stimulation. But is there any way to ensure that in the next life you will continue to amuse yourself in the same way, supplied and provided with the same five kinds of sensual stimulation? Or will others make use of this property, while you pass on according to your deeds?” 

“There’s\marginnote{40.8} no way to ensure that I will continue to amuse myself in the same way. Rather, others will take over this property, while I pass on according to my deeds.” 

“This\marginnote{40.11} is what the Buddha was referring to when he said: ‘The world has no owner—you must leave it all behind and pass on.’” 

“It’s\marginnote{40.14} incredible, Mister \textsanskrit{Raṭṭhapāla}, it’s amazing, how well said this was by the Buddha. For the world indeed has no owner—you must leave it all behind and pass on. 

You\marginnote{41.1} also said this: ‘The world is wanting, insatiable, the slave of craving.’ How should I see the meaning of this statement?” 

“What\marginnote{41.4} do you think, great king? Do you reign over the prosperous land of Kuru?” 

“Indeed\marginnote{41.6} I do.” 

“What\marginnote{41.7} do you think, great king? Suppose a trustworthy and dependable man were to come from the east. He’d approach you and say: ‘Please great king, you should know this. I come from the east. There I saw a large country that is successful and prosperous, populous, full of people. They have many divisions of elephants, cavalry, chariots, and infantry. And there’s plenty of money and grain, plenty of gold, coined and uncoined, worked and unworked, and plenty of women for the taking. With your current forces you can conquer it. Conquer it, great king!’ What would you do?” 

“I\marginnote{41.18} would conquer it and reign over it.” 

“What\marginnote{41.19} do you think, great king? Suppose a trustworthy and dependable man were to come from the west, north, south, or from over the ocean.\footnote{This is a rare acknowledgement of the existence of lands over the seas. } He’d approach you and say the same thing. What would you do?” 

“I\marginnote{41.33} would conquer it and reign over it.” 

“This\marginnote{41.34} is what the Buddha was referring to when he said: ‘The world is wanting, insatiable, the slave of craving.’\footnote{This passage is a counterweight to the ideal of the Wheel-turning Monarch, who “after conquering this land girt by sea, reigns by principle, without rod or sword” (eg. \href{https://suttacentral.net/mn91/en/sujato\#5.8}{MN 91:5.8}). The Wheel-turning Monarch is in turn a Buddhist recasting of the Brahmanical notion of a king who rules by favor of the gods as demonstrated in the horse sacrifice. The idea of a righteous king whom all obey without coercion belongs rightly to legend, while as \textsanskrit{Raṭṭhapāla} so eloquently, not to mention bravely, points out, the very desire to rule is founded on craving. } And it was after knowing and seeing and hearing this that I went forth from the lay life to homelessness.” 

“It’s\marginnote{41.37} incredible, Mister \textsanskrit{Raṭṭhapāla}, it’s amazing, how well said this was by the Buddha. For the world is indeed wanting, insatiable, the slave of craving.” 

This\marginnote{42.1} is what Venerable \textsanskrit{Raṭṭhapāla} said. Then he went on to say: 

\begin{verse}%
“I\marginnote{42.3} see rich people in the world who, \\
because of delusion, give not \\>the wealth they’ve earned. \\
Greedily, they hoard their riches, \\
yearning for ever more sensual pleasures. 

A\marginnote{42.7} king who conquered the earth by force, \\
ruling the land from sea to sea, \\
unsatisfied with the near shore of the ocean, \\
would still yearn for the further shore. 

Not\marginnote{42.11} just the king, but others too, \\
reach death not rid of craving. \\
They leave the body still wanting, \\
for in this world sensual pleasures never satisfy. 

Relatives\marginnote{42.15} lament, their hair disheveled, \\
saying ‘Ah! Alas! They’re not immortal!’ \\
They take out the body wrapped in a shroud, \\
heap up a pyre, and burn it there. 

It’s\marginnote{42.19} poked with stakes while being burnt, \\
in just a single cloth, all wealth gone. \\
Relatives, friends, and companions \\
can’t help you when you’re dying. 

Heirs\marginnote{42.23} take your riches, \\
while beings fare on according to their deeds. \\
Riches don’t follow you when you die; \\
nor do children, wife, wealth, nor kingdom. 

Longevity\marginnote{42.27} isn’t gained by riches, \\
nor does wealth banish old age; \\
for the attentive say this life is short, \\
it’s perishable and not eternal. 

The\marginnote{42.31} rich and the poor feel its touch; \\
the fool and the attentive one feel it too. \\
But the fool lies stricken by their own folly, \\
while the attentive one does not tremble at the touch. 

Therefore\marginnote{42.35} wisdom’s much better than wealth, \\
since by wisdom \\>you reach consummation in this life. \\
But if because of delusion \\>you don’t reach consummation, \\
you’ll do evil deeds in life after life. 

One\marginnote{42.39} who enters a womb and the world beyond, \\
will transmigrate from one life to the next. \\
While someone of little wisdom, \\>placing faith in them, \\
also enters a womb and the world beyond. 

As\marginnote{42.43} a bandit caught in a window \\
is punished for his own bad deeds; \\
so after departing, in the world beyond, \\
people are punished for their own bad deeds. 

Sensual\marginnote{42.47} pleasures are diverse, sweet, delightful; \\
appearing in disguise they disturb the mind.\footnote{\textit{\textsanskrit{Virūparūpena}} appears at \href{https://suttacentral.net/dn24/en/sujato\#1.17.8}{DN 24:1.17.8}, \href{https://suttacentral.net/ja522/en/sujato\#18.3}{Ja 522:18.3}, and \href{https://suttacentral.net/ja526/en/sujato\#56.2}{Ja 526:56.2}, where it always has the meaning “in disguise”. } \\
Seeing danger in sensual stimulations, \\
I went forth, O King. 

As\marginnote{42.51} fruit falls from a tree, so people fall, \\
young and old, when the body breaks up. \\
Seeing this, too, I went forth, O King; \\
the ascetic life is unfailingly better.” 

%
\end{verse}

%
\section*{{\suttatitleacronym MN 83}{\suttatitletranslation About King Maghadeva }{\suttatitleroot Maghadevasutta}}
\addcontentsline{toc}{section}{\tocacronym{MN 83} \toctranslation{About King Maghadeva } \tocroot{Maghadevasutta}}
\markboth{About King Maghadeva }{Maghadevasutta}
\extramarks{MN 83}{MN 83}

\scevam{So\marginnote{1.1} I have heard.\footnote{Versions of this story are told in the Maghadeva \textsanskrit{Jātaka} (\href{https://suttacentral.net/ja9/en/sujato}{Ja 9}) and Nimi \textsanskrit{Jātaka} (\href{https://suttacentral.net/ja541/en/sujato}{Ja 541}). } }At one time the Buddha was staying near \textsanskrit{Mithilā} in the Maghadeva Mango Grove.\footnote{\textsanskrit{Mithilā} was the capital of Videha, to the north east of the Vajjian federation, nestled against the Himalayas. \textsanskrit{Mithilā} was a dominant kingdom before the Buddha, its king Janaka featuring prominently in early \textsanskrit{Upaniṣads}. It features rarely in the suttas (\href{https://suttacentral.net/mn91/en/sujato}{MN 91}, \href{https://suttacentral.net/thig6.2/en/sujato}{Thig 6.2}, \href{https://suttacentral.net/thig13.4/en/sujato}{Thig 13.4}) and had apparently declined in importance. It is unclear whether \textsanskrit{Mithilā} had been subsumed into the Vajjian federation or remained as a minor independent kingdom until conquered by Magadha some years later. \href{https://suttacentral.net/dn19/en/sujato\#36.7}{DN 19:36.7} establishes the mythic origins of the land from a Buddhist perspective, whereas Śatapatha \textsanskrit{Brāhmaṇa} 1.4.1.10–19 depicts its origins in terms of the spread of fire-worship from the west by the founding king \textsanskrit{Māthava} Videgha with his priest Gotama \textsanskrit{Rāhūgaṇa}. } Then the Buddha smiled at a certain spot. 

Then\marginnote{2.2} Venerable Ānanda thought, “What is the cause, what is the reason why the Buddha smiled? Realized Ones do not smile for no reason.” 

So\marginnote{2.5} Ānanda arranged his robe over one shoulder, raised his joined palms toward the Buddha, and said, “What is the cause, what is the reason why the Buddha smiled? Realized Ones do not smile for no reason.” 

“Once\marginnote{3.1} upon a time, Ānanda, right here in \textsanskrit{Mithilā} there was a just and principled king named Maghadeva, a great king who stood by his duty.\footnote{Maghadeva (“bounteous god”) is from the Vedic \textit{magha} (“bounty”; see \href{https://suttacentral.net/sn11.12/en/sujato\#1.3}{SN 11.12:1.3}). Pali variants are \textit{\textsanskrit{makhādeva}} and \textit{\textsanskrit{māghadeva}}, while Sanskrit sources usually have \textit{\textsanskrit{mahādeva}}; a Bharhut \textsanskrit{stūpa} inscription has \textit{\textsanskrit{maghādeva}}.  | For “stood by his duty” (\textit{dhamme \textsanskrit{ṭhito}}) compare “standing by the tree’s duty” (\textit{rukkhadhamme \textsanskrit{ṭhito}}) at \href{https://suttacentral.net/an6.54/en/sujato\#12.4}{AN 6.54:12.4}. In such instances, \textit{dhamma} anticipates the later Hindu notion of \textit{svadharma}, i.e. the duty appropriate for ones’ station. } He justly treated brahmins and householders, and people of town and country. And he observed the sabbath on the fourteenth, fifteenth, and eighth of the fortnight. 

Then,\marginnote{4.1} after many years, many hundred years, many thousand years had passed, King Maghadeva addressed his barber,\footnote{Maghadeva’s story of renunciation shares much in common with that of \textsanskrit{Daḷhanemi} (\href{https://suttacentral.net/dn26/en/sujato\#2.1}{DN 26:2.1}). } ‘My dear barber, when you see grey hairs growing on my head, please tell me.’ 

‘Yes,\marginnote{4.3} Your Majesty,’ replied the barber. 

When\marginnote{4.4} many thousand years had passed, the barber saw grey hairs growing on the king’s head. He said to the king, ‘The messengers of the gods have shown themselves to you. Grey hairs can be seen growing on your head.’\footnote{Such “messengers of the gods” appear as reminders for the dangers of mortality in \href{https://suttacentral.net/mn130/en/sujato}{MN 130} and \href{https://suttacentral.net/an3.36/en/sujato}{AN 3.36}. } 

‘Well\marginnote{4.7} then, my dear barber, carefully pull them out with tweezers and place them in my cupped hands.’ 

‘Yes,\marginnote{4.8} Your Majesty,’ replied the barber, and he did as the king said.\footnote{This event is depicted in a Barhut relief. } 

The\marginnote{4.9} king gave the barber a prize village, then summoned the crown prince and said, ‘Dear prince, the messengers of the gods have shown themselves to me. Grey hairs can be seen growing on my head. I have enjoyed human pleasures. Now it is time to seek heavenly pleasures.\footnote{It is an expected virtue of a good ruler to retire when signs of aging appear. | The notion that spiritual practice is for the elderly is expressed at \href{https://suttacentral.net/pli-tv-bi-vb-pc21/en/sujato\#1.4}{Bi Pc 21:1.4}, where sex workers teased nuns, suggesting they enjoy themselves while young and ordain when old. The Buddhist position is that Dhamma can be practiced at any age. } Come, dear prince, rule the realm.\footnote{\textit{\textsanskrit{Rajjaṁ}} means ”realm” rather than “kingship”. } I shall shave off my hair and beard, dress in ocher robes, and go forth from the lay life to homelessness.\footnote{The signs of the renunciate predate Buddhism. } 

For\marginnote{4.16} dear prince, you too will one day see grey hairs growing on your head. When this happens, after giving a prize village to the barber and carefully instructing the crown prince in kingship, you should shave off your hair and beard, dress in ocher robes, and go forth from the lay life to homelessness. Keep up this good practice that I have founded. Do not be my final man. When a pair of men are living, the one who breaks such good practice is their final man. Therefore I say to you, “Keep up this good practice that I have founded. Do not be my final man.”’ 

And\marginnote{5.1} so, after giving a prize village to the barber and carefully instructing the crown prince in kingship, King Maghadeva shaved off his hair and beard, dressed in ocher robes, and went forth from the lay life to homelessness here in this mango grove. He meditated spreading a heart full of love to one direction, and to the second, and to the third, and to the fourth. In the same way above, below, across, everywhere, all around, he spread a heart full of love to the whole world—abundant, expansive, limitless, free of enmity and ill will. He meditated spreading a heart full of compassion … rejoicing … equanimity to one direction, and to the second, and to the third, and to the fourth. In the same way above, below, across, everywhere, all around, he spread a heart full of equanimity to the whole world—abundant, expansive, limitless, free of enmity and ill will. 

For\marginnote{6.1} 84,000 years King Maghadeva played games as a child, for 84,000 years he acted as viceroy, for 84,000 years he ruled the realm, and for 84,000 years he led the spiritual life after going forth here in this mango grove.\footnote{84,000 is 12 times 7 times 1000, where 12 is the months of the year (a full cycle) and in addition, a multiple of 4, the number of fullness and balance. 7 signifies the entire cycle of birth and death, being the days of the week and the visible “planets” (Sun, Moon, Mercury, Venus, Mars, Jupiter, and Saturn). 1000 is a major multiplier. Thus the overall sense of 84,000 is an abundant fullness (especially of time), in which sense it became a popular mystic number in Buddhism. } And having developed the four divine meditations, when his body broke up, after death, he was reborn in a good place, a realm of divinity.\footnote{Living the “spiritual (divine) life” (\textit{brahmacariya}), he practiced the “divine meditations” (\textit{\textsanskrit{brahmavihāra}}) and was reborn in the “divine realm of \textsanskrit{Brahmā}” (\textit{brahmaloka}). The Buddha is presenting an idealized vision of the ancient Brahmanical path as a coherent spiritual practice guided by the notion of \textit{brahma} as spiritual or divine excellence (see too \href{https://suttacentral.net/mn99/en/sujato\#22.14}{MN 99:22.14}, \href{https://suttacentral.net/dn13/en/sujato\#38.2}{DN 13:38.2}). | As the exemplar of an ideal Brahmanized king, Maghadeva is perhaps to be identified with \textsanskrit{Māthava} Videgha, the founder and Brahmanizer of \textsanskrit{Mithilā}. His name ought probably be read \textit{\textsanskrit{mādhava}} (“possessor of \textit{madhu}”) where \textit{madhu} (“honey”) is a common term for Soma. \textsanskrit{Māthava} Videgha was the first king who carried out the Soma rites in the region, bringing Agni to “burn over” (civilize) the newly-brahmanized land. } 

Then,\marginnote{7.1} after many years, many hundred years, many thousand years had passed, King Maghadeva’s son addressed his barber, ‘My dear barber, when you see grey hairs growing on my head, please tell me.’ And all unfolded as in the case of his father. And having developed the four divine meditations, when his body broke up, after death, Maghadeva’s son was reborn in a good place, a realm of divinity. 

And\marginnote{10.1} a lineage of 84,000 kings, sons of sons of King Maghadeva, shaved off their hair and beard, dressed in ocher robes, and went forth from the lay life to homelessness here in this mango grove. They meditated spreading a heart full of love … compassion … rejoicing … equanimity to one direction, and to the second, and to the third, and to the fourth. In the same way above, below, across, everywhere, all around, they spread a heart full of equanimity to the whole world—abundant, expansive, limitless, free of enmity and ill will. For 84,000 years they played games as a child, for 84,000 years they acted as viceroy, for 84,000 years they ruled the realm, and for 84,000 years they led the spiritual life after going forth here in this mango grove.\footnote{This totals 28.224 billion years, while the current estimate of the age of the Universe is half that at about 13.8 billion years; we are estimated to be less than half way through the current Universe. Thus the Buddhist scope of time falls within the same order of magnitude as modern cosmology. Similar durations emerge from different contexts in the suttas. The lifespan of the lowest Brahma realm is implied to be 37 billion years (\href{https://suttacentral.net/an3.70/en/sujato\#34.4}{AN 3.70:34.4}), while the duration in hell is about 25 billion (\href{https://suttacentral.net/an10.89/en/sujato\#15.3}{AN 10.89:15.3} = \href{https://suttacentral.net/snp3.10/en/sujato\#8.1}{Snp 3.10:8.1}). } And having developed the four divine meditations, when their bodies broke up, after death, they were reborn in a good place, a realm of divinity. 

Nimi\marginnote{12.1} was the last of those kings, a just and principled king, a great king who stood by his duty.\footnote{Nimi is widely known in Sanskrit literature as one of the three main sons of \textsanskrit{Ikṣvāku} (\textsanskrit{Okkāka}), the legendary ancestor of the solar dynasty and the source of the Sakyan royal lineage (\textsanskrit{Bhāgavata} \textsanskrit{Purāṇa} 9.6.4). As the king of Videha, Nimi was the direct ancestor of \textsanskrit{Mahājanaka} (\textsanskrit{Rāmāyaṇa} 1.71.3). While his legends underwent proliferation, it seems the kernel of the story is common, as \textsanskrit{Mahābhārata} says that “Nimi, the ruler of the Videhas, gave away his kingdom” (14.234 of Kisari Mohan Ganguli’s translation). } He justly treated brahmins and householders, and people of town and country. And he observed the sabbath on the fourteenth, fifteenth, and eighth of the fortnight. 

Once\marginnote{13.1} upon a time, Ānanda, while the gods of the thirty-three were sitting together in the Hall of Justice, this discussion came up among them: ‘The people of Videha are so fortunate, so very fortunate to have Nimi as their king. He is a just and principled king, a great king who stands by his duty. He justly treats brahmins and householders, and people of town and country. And he observes the sabbath on the fourteenth, fifteenth, and eighth of the fortnight.’ 

Then\marginnote{13.6} Sakka, lord of gods, addressed the gods of the thirty-three, ‘Good sirs, would you like to see King Nimi?’ 

‘We\marginnote{13.8} would.’ 

Now\marginnote{13.9} at that time it was the fifteenth day sabbath, and King Nimi had bathed his head and was sitting upstairs in the royal longhouse to observe the sabbath. Then, as easily as a strong person would extend or contract their arm, Sakka vanished from the thirty-three gods and reappeared in front of King Nimi. He said to the king, ‘You’re fortunate, great king, so very fortunate. The gods of the thirty-three were sitting together in the Hall of Justice, where they were singing your praises.\footnote{For “sing his praises” (\textit{\textsanskrit{kittayamānarūpā}}) see \href{https://suttacentral.net/dn18/en/sujato\#4.8}{DN 18:4.8}. } They would like to see you. I shall send a chariot harnessed with a thousand thoroughbreds for you, great king. Mount the heavenly chariot, great king, without wavering!’ King Nimi consented with silence. 

Then,\marginnote{13.22} knowing that the king had consented, as easily as a strong person would extend or contract their arm, Sakka vanished from King Nimi and reappeared among the gods of the thirty-three. 

Then\marginnote{14.1} Sakka, lord of gods, addressed his charioteer \textsanskrit{Mātali}, ‘Come, dear \textsanskrit{Mātali}, harness the chariot with a thousand thoroughbreds. Then go to King Nimi and say, “Great king, this chariot has been sent for you by Sakka, lord of gods. Mount the heavenly chariot, great king, without wavering!”’ 

‘Yes,\marginnote{14.5} lord,’ replied \textsanskrit{Mātali}. He did as Sakka asked, and said to the king, ‘Great king, this chariot has been sent for you by Sakka, lord of gods. Mount the heavenly chariot, great king, without wavering! But which way should we go—the way of those who experience the result of bad deeds, or the way of those who experience the result of good deeds?’ 

‘Take\marginnote{14.9} me both ways, \textsanskrit{Mātali}.’\footnote{Despite this, no journey to the lower realms is mentioned here, which perhaps indicates textual loss. It is, however, told in detail in \href{https://suttacentral.net/ja541/en/sujato}{Ja 541}. } 

\textsanskrit{Mātali}\marginnote{15.1} brought King Nimi to the Hall of Justice. Sakka saw King Nimi coming off in the distance, and said to him: ‘Come, great king! Welcome, great king! The gods of the thirty-three who wanted to see you were sitting together in the Hall of Justice, where they were singing your praises. The gods of the thirty-three would like to see you. Enjoy heavenly glory among the gods!’ 

‘Enough,\marginnote{15.13} good sir. Send me back to \textsanskrit{Mithilā} right away. That way I shall justly treat brahmins and householders, and people of town and country. And I shall observe the sabbath on the fourteenth, fifteenth, and eighth of the fortnight.’\footnote{PTS and BJT editions have the expected \textit{\textsanskrit{upavasissāmi}} (“I shall observe”) for \textsanskrit{Mahāsaṅgīti}’s \textit{\textsanskrit{upavasāmi}} (“I observe”). } 

Then\marginnote{16.1} Sakka, lord of gods, addressed his charioteer \textsanskrit{Mātali}, ‘Come, dear \textsanskrit{Mātali}, harness the chariot with a thousand thoroughbreds and send King Nimi back to \textsanskrit{Mithilā} right away.’ 

‘Yes,\marginnote{16.3} lord,’ replied \textsanskrit{Mātali}, and did as Sakka asked. And there King Nimi justly treated his people, and observed the sabbath. 

Then,\marginnote{16.5} after many years, many hundred years, many thousand years had passed, King Nimi addressed his barber, ‘My dear barber, when you see grey hairs growing on my head, please tell me.’ And all unfolded as before. 

And\marginnote{17{-}19.1} having developed the four divine meditations, when his body broke up, after death, King Nimi was reborn in a good place, a realm of divinity. 

But\marginnote{20.1} King Nimi had a son named \textsanskrit{Kaḷārajanaka}.\footnote{The Janakas were the famed kingly house of Videha, here framed as the source of the kingdom’s decline. A King Janaka was \textsanskrit{Sītā}’s father in the \textsanskrit{Rāmāyaṇa}, and one discussed philosophy with \textsanskrit{Yājñavalkya} in the \textsanskrit{Bṛhadāraṇyaka} \textsanskrit{Upaniṣad}. A Janaka is also featured in the \textsanskrit{Jātakas} (\href{https://suttacentral.net/ja539/en/sujato}{Ja 539}). It seems impossible to determine which, if any, are the same person. } He didn’t go forth from the lay life to homelessness. He broke that good practice. He was their final man. 

Ānanda,\marginnote{21.1} you might think, ‘Surely King Maghadeva, by whom that good practice was founded, must have been someone else at that time?’ But you should not see it like this. I myself was King Maghadeva at that time. I was the one who founded that good practice, which was kept up by those who came after. 

But\marginnote{21.7} that good practice doesn’t lead to disillusionment, dispassion, cessation, peace, insight, awakening, and extinguishment. It only leads as far as rebirth in the realm of divinity.\footnote{According to the doctrine of the “perfections” (\textit{\textsanskrit{pāramī}}), which emerged around two to four centuries after the Buddha’s passing, the practices he undertook in past lives laid the foundation for awakening in this life. Here, however, the Buddha states that his former practices did not lead to awakening. Rather, since they were based on the wrong view of eternal bliss in the \textsanskrit{Brahmā} realm, they only led to a good rebirth so long as that kamma lasted. It is the eightfold path, which the Buddha discovered in his final life, which leads to awakening. The same saying in a similar context is found at \href{https://suttacentral.net/dn19/en/sujato\#61.4}{DN 19:61.4}. } But now I have founded a good practice that does lead to disillusionment, dispassion, cessation, peace, insight, awakening, and extinguishment. 

And\marginnote{21.9} what is that good practice? It is simply this noble eightfold path, that is: right view, right thought, right speech, right action, right livelihood, right effort, right mindfulness, and right immersion. This is the good practice I have now founded that leads to disillusionment, dispassion, cessation, peace, insight, awakening, and extinguishment. 

Ānanda,\marginnote{21.13} I say to you: ‘You all should keep up this good practice that I have founded. Do not be my final men.’\footnote{The Buddha changes to the plural so as to include the whole \textsanskrit{Saṅgha}. } When a pair of men are living, the one who breaks such good practice is their final man. Ānanda, I say to you: ‘You all should keep up this good practice that I have founded. Do not be my final men.’” 

That\marginnote{21.18} is what the Buddha said. Satisfied, Venerable Ānanda approved what the Buddha said. 

%
\section*{{\suttatitleacronym MN 84}{\suttatitletranslation At Madhurā }{\suttatitleroot Madhurasutta}}
\addcontentsline{toc}{section}{\tocacronym{MN 84} \toctranslation{At Madhurā } \tocroot{Madhurasutta}}
\markboth{At Madhurā }{Madhurasutta}
\extramarks{MN 84}{MN 84}

\scevam{So\marginnote{1.1} I have heard.\footnote{The fact that the Buddha is not mentioned usually signifies a discourse set after his passing, which is confirmed towards the end of this sutta. } }At one time Venerable \textsanskrit{Mahākaccāna} was staying near \textsanskrit{Madhurā}, in Gunda’s Grove.\footnote{\textsanskrit{Madhurā} (Sanskrit \textsanskrit{Mathurā}), located  on the north-western trade route, between modern Agra and Delhi, was the capital of \textsanskrit{Sūrasena}. Painted Grey Ware at the site attests to its settlement in the Vedic era, long before the Buddha’s day with its characteristic Northen Black Polished Ware, which is also found there. Early suttas place the Buddha on the road to \textsanskrit{Madhurā} (\href{https://suttacentral.net/an4.43/en/sujato}{AN 4.43}), and attribute one short sutta where he complained about how bad it was (\href{https://suttacentral.net/an5.220/en/sujato}{AN 5.220}). SA 36 in Chinese translation places him at the Umbrella Mango Tree Park on the bank of the Bhadra River, where he gave a discourse whose Pali parallel is in \textsanskrit{Sāvatthī} (\href{https://suttacentral.net/sn22.43/en/sujato}{SN 22.43}). Despite its minor role in early Buddhism, it soon became a major Buddhist center, and as spiritual home of the great arahants \textsanskrit{Śaṇavāsin}, Upagupta, and Moggaliputtatissa played a major role in spreading Buddhism. } 

King\marginnote{2.1} Avantiputta of \textsanskrit{Madhurā} heard,\footnote{Avanti’s northern capital Ujjeni lay 600km south of \textsanskrit{Madhurā}. The commentary explains that the former king of \textsanskrit{Madhurā} (perhaps the \textsanskrit{Subāhu} of Lalitavistara 3.24 = Vaidya 16 who ruled when the Bodhisatta was born) had married the sister of Pajjota, king of Avanti, hence their son was named “Son of Avanti”. The unification of their houses was, according to the \textsanskrit{Purānas}, in fact a reunification, as both descended from the lunar Yadava lineage. The Yadavas were said to have settled as far as the key port on the Arabian sea at \textsanskrit{Dvāraka}. Avantiputta, however, is not attested outside of this sutta, for which the Chinese parallel says rather, “the king of \textsanskrit{Mathurā}, a western prince”. } “It seems the ascetic \textsanskrit{Kaccāna} is staying near \textsanskrit{Madhurā}, in Gunda’s Grove. He has this good reputation: ‘He is astute, competent, clever, learned, a brilliant speaker, eloquent, mature, a perfected one.’ It’s good to see such perfected ones.” 

And\marginnote{3.1} then King Avantiputta had the finest carriages harnessed. He mounted a fine carriage and, along with other fine carriages, set out in full royal pomp from \textsanskrit{Madhurā} to see \textsanskrit{Mahākaccāna}. He went by carriage as far as the terrain allowed, then descended and approached \textsanskrit{Mahākaccāna} on foot. They exchanged greetings, and when the greetings and polite conversation were over, the king sat down to one side and said to \textsanskrit{Mahākaccāna}: 

“Mister\marginnote{4.1} \textsanskrit{Kaccāna}, the brahmins say:\footnote{\textsanskrit{Kaccāna} appears in \textsanskrit{Madhurā} also at \href{https://suttacentral.net/an2.38/en/sujato}{AN 2.38}, where he has a discussion on a similar topic with the brahmin \textsanskrit{Kandarāyana}. \textsanskrit{Kaccāna} often stayed at Avanti, both before and after the Buddha’s passing, where he also dealt with the question of the status of the brahimns (\href{https://suttacentral.net/sn35.132/en/sujato}{SN 35.132}). The commentary says he was in fact the son of the chief priest (\textit{purohita}) of Avanti, which would explain his connection with Avanti, his discussions on the topic of brahminhood, and his skill in analyzing scripture. \textsanskrit{Kaccāna} ultimately established Buddhism in the region, from where it was later brought to Sri Lanka by another “son of Avanti”, the great monk Mahinda. } ‘Only brahmins are the best class; other classes are inferior. Only brahmins are the light class; other classes are dark. Only brahmins are purified, not others. Only brahmins are the true-born sons of divinity, born from his mouth, born of divinity, created by divinity, heirs of divinity.’\footnote{“True-born” is \textit{orasa}, signifying a natural heir to a legitimately married couple (\textsanskrit{Manusmṛti} 9.159, \textsanskrit{Skandapurāṇa} 223.5, \textsanskrit{Kauṭilya}’s \textsanskrit{Arthaśāstra} 3.7). The literal translation “bosom sons” is tempting, but “bosom” in English conveys intimacy and comfort rather than legitimacy. While \textit{putta} can mean “son” or “child”, in this case the gendered sense is required, since in Brahmanical context the concept is intended to explain how the male child takes precedence in inheritance (\textsanskrit{Manusmṛti} 9.134). } What does Mister \textsanskrit{Kaccāna} have to say about this?” 

“Great\marginnote{5.1} king, that’s just hearsay in the world. And here’s a way to understand that it’s just hearsay in the world. 

What\marginnote{5.9} do you think, great king? Suppose an aristocrat prospers in money, grain, silver, or gold. Wouldn’t there be aristocrats, brahmins, peasants, and menials who would get up before him and go to bed after him, and be obliging, behaving nicely and speaking politely?” 

“There\marginnote{5.14} would, Mister \textsanskrit{Kaccāna}.” 

“What\marginnote{5.18} do you think, great king? Suppose a brahmin … a peasant … a menial prospers in money, grain, silver, or gold. Wouldn’t there be menials, aristocrats, brahmins, and peasants who would get up before him and go to bed after him, and be obliging, behaving nicely and speaking politely?”\footnote{Then as now, it would seem, wealth trumps birth. } 

“There\marginnote{5.39} would, Mister \textsanskrit{Kaccāna}.” 

“What\marginnote{5.43} do you think, great king? If this is so, are the four classes equal or not? Or how do you see this?” 

“Certainly,\marginnote{5.46} Mister \textsanskrit{Kaccāna}, in this case these four classes are equal. I can’t see any difference between them.” 

“And\marginnote{5.48} here’s another way to understand that the claims of the brahmins are just hearsay in the world. 

What\marginnote{6.1} do you think, great king? Take an aristocrat who kills living creatures, steals, and commits sexual misconduct; uses speech that’s false, divisive, harsh, or nonsensical; and is covetous, malicious, and has wrong view. When their body breaks up, after death, would they be reborn in a place of loss, a bad place, the underworld, hell, or not? Or how do you see this?” 

“Such\marginnote{6.4} an aristocrat would be reborn in a bad place. That’s what I think, but I’ve also heard it from the perfected ones.”\footnote{\textit{\textsanskrit{Arahataṁ}} is genitive plural. } 

“Good,\marginnote{6.6} good, great king! It’s good that you think so, and it’s good that you’ve heard it from the perfected ones. What do you think, great king? Take a brahmin … a peasant … a menial who kills living creatures, steals, and commits sexual misconduct; uses speech that’s false, divisive, harsh, or nonsensical; and is covetous, malicious, and has wrong view. When their body breaks up, after death, would they be reborn in a place of loss, a bad place, the underworld, hell, or not? Or how do you see this?” 

“Such\marginnote{6.13} a brahmin, peasant, or menial would be reborn in a bad place. That’s what I think, but I’ve also heard it from the perfected ones.” 

“Good,\marginnote{6.15} good, great king! It’s good that you think so, and it’s good that you’ve heard it from the perfected ones. What do you think, great king? If this is so, are the four classes equal or not? Or how do you see this?” 

“Certainly,\marginnote{6.20} Mister \textsanskrit{Kaccāna}, in this case these four classes are equal. I can’t see any difference between them.” 

“And\marginnote{6.22} here’s another way to understand that the claims of the brahmins are just hearsay in the world. 

What\marginnote{7.1} do you think, great king? Take an aristocrat who doesn’t kill living creatures, steal, or commit sexual misconduct. They don’t use speech that’s false, divisive, harsh, or nonsensical. And they’re contented, kind-hearted, with right view. When their body breaks up, after death, would they be reborn in a good place, a heavenly realm, or not? Or how do you see this?” 

“Such\marginnote{7.4} an aristocrat would be reborn in a good place. That’s what I think, but I’ve also heard it from the perfected ones.” 

“Good,\marginnote{7.6} good, great king! It’s good that you think so, and it’s good that you’ve heard it from the perfected ones. What do you think, great king? Take a brahmin, peasant, or menial who doesn’t kill living creatures, steal, or commit sexual misconduct. They don’t use speech that’s false, divisive, harsh, or nonsensical. And they’re contented, kind-hearted, with right view. When their body breaks up, after death, would they be reborn in a good place, a heavenly realm, or not? Or how do you see this?” 

“Such\marginnote{7.11} a brahmin, peasant, or menial would be reborn in a good place. That’s what I think, but I’ve also heard it from the perfected ones.” 

“Good,\marginnote{7.13} good, great king! It’s good that you think so, and it’s good that you’ve heard it from the perfected ones. What do you think, great king? If this is so, are the four classes equal or not? Or how do you see this?” 

“Certainly,\marginnote{7.18} Mister \textsanskrit{Kaccāna}, in this case these four classes are equal. I can’t see any difference between them.” 

“And\marginnote{7.20} here’s another way to understand that the claims of the brahmins are just hearsay in the world. 

What\marginnote{8.1} do you think, great king? Take an aristocrat who breaks into houses, plunders wealth, steals from isolated buildings, commits highway robbery, and commits adultery. Suppose your men arrest him and present him to you, saying: ‘Your Majesty, this man is a bandit, a criminal. Punish him as you will.’ What would you do to him?” 

“I\marginnote{8.6} would have him executed, fined, or banished, or dealt with as befits the crime. Why is that? Because he’s lost his former status as an aristocrat, and is just reckoned as a bandit.” 

“What\marginnote{8.9} do you think, great king? Take a brahmin, peasant, or menial who breaks into houses, plunders wealth, steals from isolated buildings, commits highway robbery, and commits adultery. Suppose your men arrest him and present him to you, saying: ‘Your Majesty, this man is a bandit, a criminal. Punish him as you will.’ What would you do to him?” 

“I\marginnote{8.14} would have him executed, fined, or banished, or dealt with as befits the crime. Why is that? Because he’s lost his former status as a brahmin, peasant, or menial, and is just reckoned as a bandit.” 

“What\marginnote{8.17} do you think, great king? If this is so, are the four classes equal or not? Or how do you see this?” 

“Certainly,\marginnote{8.20} Mister \textsanskrit{Kaccāna}, in this case these four classes are equal. I can’t see any difference between them.” 

“And\marginnote{8.22} here’s another way to understand that the claims of the brahmins are just hearsay in the world. 

What\marginnote{9.1} do you think, great king? Take an aristocrat who shaves off their hair and beard, dresses in ocher robes, and goes forth from the lay life to homelessness. They refrain from killing living creatures, stealing, and lying. They abstain from eating at night, eat in one part of the day, and are celibate, ethical, and of good character. How would you treat them?” 

“I\marginnote{9.4} would bow to them, rise in their presence, or offer them a seat. I’d invite them to accept robes, almsfood, lodgings, and medicines and supplies for the sick. And I’d organize their lawful guarding and protection. Why is that? Because they’ve lost their former status as an aristocrat, and are just reckoned as an ascetic.” 

“What\marginnote{9.7} do you think, great king? Take a brahmin, peasant, or menial who shaves off their hair and beard, dresses in ocher robes, and goes forth from the lay life to homelessness. They refrain from killing living creatures, stealing, and lying. They abstain from eating at night, eat in one part of the day, and are celibate, ethical, and of good character. How would you treat them?” 

“I\marginnote{9.10} would bow to them, rise in their presence, or offer them a seat. I’d invite them to accept robes, almsfood, lodgings, and medicines and supplies for the sick. And I’d organize their lawful guarding and protection. Why is that? Because they’ve lost their former status as a brahmin, peasant, or menial, and are just reckoned as an ascetic.” 

“What\marginnote{9.13} do you think, great king? If this is so, are the four classes equal or not? Or how do you see this?” 

“Certainly,\marginnote{9.16} Mister \textsanskrit{Kaccāna}, in this case these four classes are equal. I can’t see any difference between them.” 

“This\marginnote{9.18} is another way to understand that this is just hearsay in the world: ‘Only brahmins are the best class; other classes are inferior.\footnote{This verse is also at \href{https://suttacentral.net/dn27/en/sujato\#3.6}{DN 27:3.6} and \href{https://suttacentral.net/mn93/en/sujato\#5.4}{MN 93:5.4}. | For “best class” (\textit{\textsanskrit{seṭṭho} \textsanskrit{vaṇṇo}}) see \textit{\textsanskrit{varṇaśreṣṭhaḥ}} at \textsanskrit{Mahābhārata} 1.24.4c. } Only brahmins are the light class; other classes are dark.\footnote{\textsanskrit{Mahābhārata} 12.181.5 says that brahmins are the white class, aristocrats red, peasants yellow, and menials black. Underlying this is an assumption of the superiority of the fair-skinned northerners of Indo-Aryan descent over the dark-skinned natives (see \href{https://suttacentral.net/dn3/en/sujato\#1.10.3}{DN 3:1.10.3}). } Only brahmins are purified, not others.\footnote{I cannot trace this statement to Brahmanical texts, and indeed they have a strong tendency to push back against the idea that only brahmins can find purity. } Only brahmins are divinity’s true-born sons, born from his mouth, born of divinity, created by divinity, heirs of divinity.’”\footnote{An allusion to the famous \textsanskrit{Puruṣasūkta} (Rig Veda 10.90.12): “His (the cosmic Man’s) mouth became the brahmin, his arms became the ruler (\textit{\textsanskrit{rājanya}}, an alternate name for the \textit{khattiyas}), his thighs became the peasant (\textit{\textsanskrit{vaiśya}}); the menial (\textit{\textsanskrit{sūdra}}) was born from his feet.” This belongs to the latest portion of the Rig Veda, and probably represents a time when the system of four castes was established in the late Vedic period, some centuries before the Buddha. Apart from this, the Rig Veda hardly mentions any of these as castes, let alone all four together. } 

When\marginnote{10.1} he had spoken, King Avantiputta of \textsanskrit{Madhurā} said to \textsanskrit{Mahākaccāna}, “Excellent, Mister \textsanskrit{Kaccāna}! Excellent! As if he were righting the overturned, or revealing the hidden, or pointing out the path to the lost, or lighting a lamp in the dark so people with clear eyes can see what’s there, Mister \textsanskrit{Kaccāna} has made the teaching clear in many ways. I go for refuge to Mister \textsanskrit{Kaccāna}, to the teaching, and to the mendicant \textsanskrit{Saṅgha}. From this day forth, may Mister \textsanskrit{Kaccāna} remember me as a lay follower who has gone for refuge for life.” 

“Great\marginnote{10.6} king, don’t go for refuge to me. You should go for refuge to that same Blessed One to whom I have gone for refuge.” 

“But\marginnote{10.8} where is that Blessed One at present, the perfected one, the fully awakened Buddha?” 

“Great\marginnote{10.9} king, the Buddha has already become fully quenched.” 

“Mister\marginnote{11.1} \textsanskrit{Kaccāna}, if I heard that the Buddha was within ten leagues, or twenty, or even up to a hundred leagues away, I’d go a hundred leagues to see him. But since the Buddha has become fully quenched, I go for refuge to that fully quenched Buddha, to the teaching, and to the mendicant \textsanskrit{Saṅgha}. From this day forth, may Mister \textsanskrit{Kaccāna} remember me as a lay follower who has gone for refuge for life.” 

%
\section*{{\suttatitleacronym MN 85}{\suttatitletranslation With Prince Bodhi }{\suttatitleroot Bodhirājakumārasutta}}
\addcontentsline{toc}{section}{\tocacronym{MN 85} \toctranslation{With Prince Bodhi } \tocroot{Bodhirājakumārasutta}}
\markboth{With Prince Bodhi }{Bodhirājakumārasutta}
\extramarks{MN 85}{MN 85}

\scevam{So\marginnote{1.1} I have heard. }At one time the Buddha was staying in the land of the Bhaggas at Crocodile Hill, in the deer park at \textsanskrit{Bhesakaḷā}’s Wood. 

Now\marginnote{2.1} at that time a new stilt longhouse named Pink Lotus had recently been constructed for Prince Bodhi. It had not yet been occupied by an ascetic or brahmin or any person at all.\footnote{The commentary says that Bodhi’s father was Udena, king of \textsanskrit{Kosambī}, while his mother was \textsanskrit{Vāsuladattā}, daughter of King Pajjota of Avanti, whose marriage became one of the great romantic legends of Indian story. Bodhi would then be the first cousin once removed of King Avantiputta of \textsanskrit{Madhurā} (\href{https://suttacentral.net/mn84/en/sujato}{MN 84}). It seems that the Bhaggas were vassals of the Vacchas of \textsanskrit{Kosambī}. The ending of our text shows that Bodhi was conceived in \textsanskrit{Kosambī} and raised at Crocodile Hill, where he was currently dwelling as suzerain. } 

Then\marginnote{3.1} Prince Bodhi addressed the student \textsanskrit{Sañjikāputta}, “Please, dear \textsanskrit{Sañjikāputta}, go to the Buddha, and in my name bow with your head to his feet. Ask him if he is healthy and well, nimble, strong, and living comfortably. And then ask him whether he might accept tomorrow’s meal from me together with the mendicant \textsanskrit{Saṅgha}.” 

“Yes,\marginnote{4.1} sir,” \textsanskrit{Sañjikāputta} replied. He did as Prince Bodhi asked, and the Buddha consented with silence. 

Then,\marginnote{4.7} knowing that the Buddha had consented, \textsanskrit{Sañjikāputta} got up from his seat, went to Prince Bodhi, and said, “I gave your message to Mister Gotama, and he accepted.” 

And\marginnote{5.1} when the night had passed Prince Bodhi had delicious fresh and cooked foods prepared in his own home. He also had the Pink Lotus longhouse spread with white cloth down to the last step of the staircase. Then he said to \textsanskrit{Sañjikāputta}, “Please, dear \textsanskrit{Sañjikāputta}, go to the Buddha, and announce the time, saying, ‘Sir, it’s time. The meal is ready.’” 

“Yes,\marginnote{5.4} sir,” \textsanskrit{Sañjikāputta} replied, and he did as he was asked. 

Then\marginnote{6.1} the Buddha robed up in the morning and, taking his bowl and robe, went to Prince Bodhi’s home. 

Now\marginnote{7.1} at that time Prince Bodhi was standing outside the gates waiting for the Buddha. Seeing the Buddha coming off in the distance, he went out to greet him. After bowing and inviting the Buddha to go first, he approached the Pink Lotus longhouse. But the Buddha stopped by the last step of the staircase. 

Then\marginnote{7.5} Prince Bodhi said to him, “Sir, let the Blessed One ascend on the cloth! Let the Holy One ascend on the cloth! It will be for my lasting welfare and happiness.” But when he said this, the Buddha kept silent. 

For\marginnote{7.9} a second time … and a third time, Prince Bodhi said to him, “Sir, let the Blessed One ascend on the cloth! Let the Holy One ascend on the cloth! It will be for my lasting welfare and happiness.” 

Then\marginnote{7.13} the Buddha glanced at Venerable Ānanda.\footnote{With just a glance Ānanda knew what to do. } So Ānanda said to Prince Bodhi, “Fold up the cloth, Prince. The Buddha will not step upon white cloth.\footnote{According to the commentary, Bodhi was childless. He thought that if the Buddha stepped on the cloth it would show that he would have a child in the future, whereas if he did not then he would have no child. In the commentary to \href{https://suttacentral.net/dhp157/en/sujato}{Dhp 157}, the Buddha confirms to Bodhi that he will have no child. Under this explanation, then, by refusing to step on the cloth the Buddha was giving Bodhi an answer. The canonical account, however, reads as if the Buddha is simply refusing to indulge Bodhi’s superstition. Nonetheless, the fact that the Vinaya account (see below) concerns a woman who cannot have a child confirms that the blessing concerns fertility. } The Realized One has sympathy for future generations.”\footnote{Pali editions vary between “has regard” (\textit{apaloketi}) and “has compassion” (\textit{anukampati}), while the Chinese and Sanskrit fragments agree on “compassion”. The reading \textit{apaloketi} (or \textit{oloketi} in the Dhammapada commentary) seems to have arisen to conform with the idea that the Buddha was looking into the future to see if Bodhi would have a child. } 

So\marginnote{8.1} Prince Bodhi had the cloth folded up and the seats spread out upstairs in the longhouse. Then the Buddha ascended the longhouse and sat on the seats spread out together with the \textsanskrit{Saṅgha} of mendicants. 

Then\marginnote{9.1} Prince Bodhi served and satisfied the mendicant \textsanskrit{Saṅgha} headed by the Buddha with his own hands with delicious fresh and cooked foods.\footnote{The discourse thus far is also found at \href{https://suttacentral.net/pli-tv-kd15/en/sujato\#21.1.1}{Kd 15:21.1.1}. The Vinaya version omits the following teaching, and instead provides an occasion for the Buddha to make it a minor offence to step on a cloth covering (\href{https://suttacentral.net/pli-tv-kd15/en/sujato\#21.3.6}{Kd 15:21.3.6}). But when a woman who is either barren or has miscarried (\textit{apagatagabbha}) complains, the Buddha made a compassionate exception for stepping on a cloth for superstitious laypeople (\textit{\textsanskrit{maṅgalikā}}) seeking blessings (\href{https://suttacentral.net/pli-tv-kd15/en/sujato\#21.4.11}{Kd 15:21.4.11}). The commentary says this only applies for women without child or with a difficult pregnancy. } When the Buddha had eaten and washed his hand and bowl, Prince Bodhi took a low seat, sat to one side, and said to him, “Sir, this is what I think: ‘Pleasure is not gained through pleasure; pleasure is gained through pain.’”\footnote{This is a Jain view (\href{https://suttacentral.net/mn14/en/sujato\#20.1}{MN 14:20.1}). This lends support to the Jain tradition that claims Bodhi’s grandfather Pajjota as a disciple. | Bodhi’s spirituality was inclusive: he is close to the brahmin \textsanskrit{Sañjikāputta}, inquires into Buddhism, asserts a Jain view, and practices superstitions. } 

“Prince,\marginnote{10.1} before my awakening—when I was still unawakened but intent on awakening—I too thought: ‘Pleasure is not gained through pleasure; pleasure is gained through pain.’\footnote{The Buddha confirms that this Jain view drove his practices before awakening. } 

Some\marginnote{11.1} time later, while still with pristine black hair, blessed with youth, in the prime of life—though my mother and father wished otherwise, weeping with tearful faces—I shaved off my hair and beard, dressed in ocher robes, and went forth from the lay life to homelessness. Once I had gone forth I set out to discover what is skillful, seeking the supreme state of sublime peace. I approached \textsanskrit{Āḷāra} \textsanskrit{Kālāma} and said to him,\footnote{See \href{https://suttacentral.net/mn26/en/sujato\#15.1}{MN 26:15.1} for notes. } ‘Reverend \textsanskrit{Kālāma}, I wish to lead the spiritual life in this teaching and training.’ 

\textsanskrit{Āḷāra}\marginnote{11.4} \textsanskrit{Kālāma} replied, ‘Stay, venerable. This teaching is such that a sensible person can soon realize their own tradition with their own insight and live having achieved it.’ 

I\marginnote{11.7} quickly memorized that teaching. As far as lip-recital and verbal repetition went, I spoke the doctrine of knowledge, the elder doctrine. I claimed to know and see, and so did others. Then it occurred to me, ‘It is not solely by mere faith that \textsanskrit{Āḷāra} \textsanskrit{Kālāma} declares: “I realize this teaching with my own insight, and live having achieved it.” Surely he meditates knowing and seeing this teaching.’ 

So\marginnote{12.1} I approached \textsanskrit{Āḷāra} \textsanskrit{Kālāma} and said to him, ‘Reverend \textsanskrit{Kālāma}, to what extent do you say you’ve realized this teaching with your own insight?’ When I said this, he declared the dimension of nothingness. 

Then\marginnote{12.4} it occurred to me, ‘It’s not just \textsanskrit{Āḷāra} \textsanskrit{Kālāma} who has faith, energy, mindfulness, immersion, and wisdom; I too have these things. Why don’t I make an effort to realize the same teaching that \textsanskrit{Āḷāra} \textsanskrit{Kālāma} says he has realized with his own insight?’ I quickly realized that teaching with my own insight, and lived having achieved it. 

So\marginnote{12.12} I approached \textsanskrit{Āḷāra} \textsanskrit{Kālāma} and said to him, ‘Reverend \textsanskrit{Kālāma}, is it up to this point that you realized this teaching with your own insight, and declare having achieved it?’ 

‘I\marginnote{12.14} have, reverend.’ 

‘I\marginnote{12.15} too have realized this teaching with my own insight up to this point, and live having achieved it.’ 

‘We\marginnote{12.16} are fortunate, reverend, so very fortunate to see a venerable such as yourself as one of our spiritual companions! So the teaching that I’ve realized with my own insight, and declare having achieved it, you’ve realized with your own insight, and dwell having achieved it. The teaching that you’ve realized with your own insight, and dwell having achieved it, I’ve realized with my own insight, and declare having achieved it. So the teaching that I know, you know, and the teaching you know, I know. I am like you and you are like me. Come now, reverend! We should both lead this community together.’ And that is how my tutor \textsanskrit{Āḷāra} \textsanskrit{Kālāma} placed me, his pupil, on the same position as him, and honored me with lofty praise. 

Then\marginnote{12.24} it occurred to me, ‘This teaching doesn’t lead to disillusionment, dispassion, cessation, peace, insight, awakening, and extinguishment. It only leads as far as rebirth in the dimension of nothingness.’ Realizing that this teaching was inadequate, I left disappointed. 

I\marginnote{12.27} set out to discover what is skillful, seeking the supreme state of sublime peace. I approached Uddaka son of \textsanskrit{Rāma} and said to him, ‘Reverend, I wish to lead the spiritual life in this teaching and training.’ 

Uddaka\marginnote{12.29} replied, ‘Stay, venerable. This teaching is such that a sensible person can soon realize their own tradition with their own insight and live having achieved it.’ 

I\marginnote{12.32} quickly memorized that teaching. As far as lip-recital and verbal repetition went, I spoke the doctrine of knowledge, the elder doctrine. I claimed to know and see, and so did others. 

Then\marginnote{12.34} it occurred to me, ‘It is not solely by mere faith that \textsanskrit{Rāma} declared: “I realize this teaching with my own insight, and live having achieved it.” Surely he meditated knowing and seeing this teaching.’ 

So\marginnote{13.1} I approached Uddaka son of \textsanskrit{Rāma} and said to him, ‘Reverend, to what extent did \textsanskrit{Rāma} say he’d realized this teaching with his own insight?’ When I said this, Uddaka son of \textsanskrit{Rāma} declared the dimension of neither perception nor non-perception. 

Then\marginnote{13.4} it occurred to me, ‘It’s not just \textsanskrit{Rāma} who had faith, energy, mindfulness, immersion, and wisdom; I too have these things. Why don’t I make an effort to realize the same teaching that \textsanskrit{Rāma} said he had realized with his own insight?’ I quickly realized that teaching with my own insight, and lived having achieved it. 

So\marginnote{13.12} I approached Uddaka son of \textsanskrit{Rāma} and said to him, ‘Reverend, had \textsanskrit{Rāma} realized this teaching with his own insight up to this point, and declared having achieved it?’ 

‘He\marginnote{13.14} had, reverend.’ 

‘I\marginnote{13.15} too have realized this teaching with my own insight up to this point, and live having achieved it.’ 

‘We\marginnote{13.16} are fortunate, reverend, so very fortunate to see a venerable such as yourself as one of our spiritual companions! So the teaching that \textsanskrit{Rāma} had realized with his own insight, and declared having achieved it, you've realized with your own insight, and dwell having achieved it. The teaching that you’ve realized with your own insight, and dwell having achieved it, \textsanskrit{Rāma} had realized with his own insight, and declared having achieved it. So the teaching that \textsanskrit{Rāma} directly knew, you know, and the teaching you know, \textsanskrit{Rāma} directly knew. \textsanskrit{Rāma} was like you and you are like \textsanskrit{Rāma}. Come now, reverend! You should lead this community.’ And that is how my spiritual companion Uddaka son of \textsanskrit{Rāma} placed me in the position of a tutor and honored me with lofty praise. 

Then\marginnote{13.24} it occurred to me, ‘This teaching doesn’t lead to disillusionment, dispassion, cessation, peace, insight, awakening, and extinguishment. It only leads as far as rebirth in the dimension of neither perception nor non-perception.’ Realizing that this teaching was inadequate, I left disappointed. 

I\marginnote{14.1} set out to discover what is skillful, seeking the supreme state of sublime peace. Traveling stage by stage in the Magadhan lands, I arrived at \textsanskrit{Senānigama} in \textsanskrit{Uruvelā}. There I saw a delightful park, a lovely grove with a flowing river that was clean and charming, with smooth banks. And nearby was a village to resort to for alms. 

Then\marginnote{14.3} it occurred to me, ‘This park is truly delightful, a lovely grove with a flowing river that’s clean and charming, with smooth banks. And nearby there’s a village to resort to for alms. This is good enough for striving for a gentleman wanting to strive.’ So I sat down right there, thinking, ‘This is good enough for striving.’ 

And\marginnote{15.1} then these three similes, which were neither supernaturally inspired, nor learned before in the past, occurred to me. 

Suppose\marginnote{16.1} there was a green, sappy log, and it was lying in water. Then a person comes along with a drill-stick, thinking to light a fire and produce heat. What do you think, Prince? By drilling the stick against that green, sappy log lying in water, could they light a fire and produce heat?” 

“No,\marginnote{16.6} sir. Why is that? Because it’s a green, sappy log, and it’s lying in the water. That person will eventually get weary and frustrated.” 

“In\marginnote{17.1} the same way, there are ascetics and brahmins who don’t live withdrawn in body and mind from sensual pleasures. They haven’t internally given up or stilled desire, affection, infatuation, thirst, and passion for sensual pleasures. Regardless of whether or not they suffer painful, sharp, severe, acute feelings because of their efforts, they are incapable of knowledge and vision, of supreme awakening. This was the first example that occurred to me. 

Then\marginnote{18.1} a second example occurred to me. 

Suppose\marginnote{18.2} there was a green, sappy log, and it was lying on dry land far from the water. Then a person comes along with a drill-stick, thinking to light a fire and produce heat. What do you think, Prince? By drilling the stick against that green, sappy log on dry land far from water, could they light a fire and produce heat?” 

“No,\marginnote{18.7} sir. Why is that? Because it’s still a green, sappy log, despite the fact that it’s lying on dry land far from water. That person will eventually get weary and frustrated.” 

“In\marginnote{18.11} the same way, there are ascetics and brahmins who live withdrawn in body and mind from sensual pleasures. But they haven’t internally given up or stilled desire, affection, infatuation, thirst, and passion for sensual pleasures. Regardless of whether or not they suffer painful, sharp, severe, acute feelings because of their efforts, they are incapable of knowledge and vision, of supreme awakening. This was the second example that occurred to me. 

Then\marginnote{19.1} a third example occurred to me. 

Suppose\marginnote{19.2} there was a dried up, withered log, and it was lying on dry land far from the water. Then a person comes along with a drill-stick, thinking to light a fire and produce heat. What do you think, Prince? By drilling the stick against that dried up, withered log on dry land far from water, could they light a fire and produce heat?” 

“Yes,\marginnote{19.7} sir. Why is that? Because it’s a dried up, withered log, and it’s lying on dry land far from water.” 

“In\marginnote{19.10} the same way, there are ascetics and brahmins who live withdrawn in body and mind from sensual pleasures. And they have internally given up and stilled desire, affection, infatuation, thirst, and passion for sensual pleasures. Regardless of whether or not they suffer painful, sharp, severe, acute feelings because of their efforts, they are capable of knowledge and vision, of supreme awakening. This was the third example that occurred to me. These are the three similes, which were neither supernaturally inspired, nor learned before in the past, that occurred to me. 

Then\marginnote{20.1} it occurred to me, ‘Why don’t I, with teeth clenched and tongue pressed against the roof of my mouth, squeeze, squash, and scorch mind with mind?’ So that’s what I did, until sweat ran from my armpits. It was like when a strong man grabs a weaker man by the head or throat or shoulder and squeezes, squashes, and crushes them. In the same way, with teeth clenched and tongue pressed against the roof of my mouth, I squeezed, squashed, and crushed mind with mind until sweat ran from my armpits. My energy was roused up and unflagging, and my mindfulness was established and lucid, but my body was disturbed, not tranquil, because I’d pushed too hard with that painful striving. 

Then\marginnote{21.1} it occurred to me, ‘Why don’t I practice the breathless absorption?’ So I cut off my breathing through my mouth and nose. But then winds came out my ears making a loud noise, like the puffing of a blacksmith’s bellows. My energy was roused up and unflagging, and my mindfulness was established and lucid, but my body was disturbed, not tranquil, because I’d pushed too hard with that painful striving. 

Then\marginnote{22.1} it occurred to me, ‘Why don’t I keep practicing the breathless absorption?’ So I cut off my breathing through my mouth and nose and ears. But then strong winds ground my head, like a strong man was drilling into my head with a sharp point. My energy was roused up and unflagging, and my mindfulness was established and lucid, but my body was disturbed, not tranquil, because I’d pushed too hard with that painful striving. 

Then\marginnote{23.1} it occurred to me, ‘Why don’t I keep practicing the breathless absorption?’ So I cut off my breathing through my mouth and nose and ears. But then I got a severe headache, like a strong man was tightening a tough leather strap around my head. My energy was roused up and unflagging, and my mindfulness was established and lucid, but my body was disturbed, not tranquil, because I’d pushed too hard with that painful striving. 

Then\marginnote{24.1} it occurred to me, ‘Why don’t I keep practicing the breathless absorption?’ So I cut off my breathing through my mouth and nose and ears. But then strong winds carved up my belly, like a deft butcher or their apprentice was slicing my belly open with a sharp meat cleaver. My energy was roused up and unflagging, and my mindfulness was established and lucid, but my body was disturbed, not tranquil, because I’d pushed too hard with that painful striving. 

Then\marginnote{25.1} it occurred to me, ‘Why don’t I keep practicing the breathless absorption?’ So I cut off my breathing through my mouth and nose and ears. But then there was an intense burning in my body, like two strong men grabbing a weaker man by the arms to burn and scorch him on a pit of glowing coals. My energy was roused up and unflagging, and my mindfulness was established and lucid, but my body was disturbed, not tranquil, because I’d pushed too hard with that painful striving. 

Then\marginnote{26.1} some deities saw me and said, ‘The ascetic Gotama is dead.’ Others said, ‘He’s not dead, but he’s dying.’ Others said, ‘He’s not dead or dying. The ascetic Gotama is a perfected one, for that is how the perfected ones live.’ 

Then\marginnote{27.1} it occurred to me, ‘Why don’t I practice completely cutting off food?’ But deities came to me and said, ‘Good sir, don’t practice totally cutting off food. If you do, we’ll infuse heavenly nectar into your pores and you will live on that.’ Then it occurred to me, ‘If I claim to be completely fasting while these deities are infusing heavenly nectar in my pores, that would be a lie on my part.’ So I dismissed those deities, saying, ‘There’s no need.’ 

Then\marginnote{28.1} it occurred to me, ‘Why don’t I just take a little bit of food each time, a handful of broth made from mung beans, horse gram, chickpeas, or green gram?’ So that’s what I did, until my body became extremely emaciated. Due to eating so little, my major and minor limbs became like the joints of an eighty-year-old or a dying man, my bottom became like a camel’s hoof, my vertebrae stuck out like beads on a string, and my ribs were as gaunt as the broken-down rafters on an old barn. Due to eating so little, the gleam of my eyes sank deep in their sockets, like the gleam of water sunk deep down a well. Due to eating so little, my scalp shriveled and withered like a green bitter-gourd in the wind and sun. Due to eating so little, the skin of my belly stuck to my backbone, so that when I tried to rub the skin of my belly I grabbed my backbone, and when I tried to rub my backbone I rubbed the skin of my belly. Due to eating so little, when I tried to urinate or defecate I fell face down right there. Due to eating so little, when I tried to relieve my body by rubbing my limbs with my hands, the hair, rotted at its roots, fell out. 

Then\marginnote{29.1} some people saw me and said, ‘The ascetic Gotama is black.’ Some said, ‘He’s not black, he’s brown.’ Some said, ‘He’s neither black nor brown. The ascetic Gotama has tawny skin.’ That’s how far the pure, bright complexion of my skin had been ruined by taking so little food. 

Then\marginnote{30.1} it occurred to me, ‘Whatever ascetics and brahmins have experienced painful, sharp, severe, acute feelings due to overexertion—whether in the past, future, or present—this is as far as it goes, no-one has done more than this. But I have not achieved any superhuman distinction in knowledge and vision worthy of the noble ones by this severe, gruelling work. Could there be another path to awakening?’ 

Then\marginnote{31.1} it occurred to me, ‘I recall sitting in the cool shade of a black plum tree while my father the Sakyan was off working. Quite secluded from sensual pleasures, secluded from unskillful qualities, I entered and remained in the first absorption, which has the rapture and bliss born of seclusion, while placing the mind and keeping it connected. Could that be the path to awakening?’ Stemming from that memory came the realization: ‘\emph{That} is the path to awakening!’ 

Then\marginnote{32.1} it occurred to me, ‘Why am I afraid of that pleasure, for it has nothing to do with sensual pleasures or unskillful qualities?’ Then it occurred to me, ‘I’m not afraid of that pleasure, for it has nothing to do with sensual pleasures or unskillful qualities.’ 

Then\marginnote{33.1} it occurred to me, ‘I can’t achieve that pleasure with a body so excessively emaciated. Why don’t I eat some solid food, some rice and porridge?’ So I ate some solid food. Now at that time the group of five mendicants were attending on me, thinking, ‘The ascetic Gotama will tell us of any truth that he realizes.’ But when I ate some solid food, they left disappointed in me, saying, ‘The ascetic Gotama has become indulgent; he has strayed from the struggle and returned to indulgence.’ 

After\marginnote{34{-}37.1} eating solid food and gathering my strength, quite secluded from sensual pleasures, secluded from unskillful qualities, I entered and remained in the first absorption … second absorption … third absorption … fourth absorption. When my mind had immersed in \textsanskrit{samādhi} like this—purified, bright, flawless, rid of corruptions, pliable, workable, steady, and imperturbable—I extended it toward recollection of past lives. I recollected many past lives. That is: one, two, three, four, five, ten, twenty, thirty, forty, fifty, a hundred, a thousand, a hundred thousand rebirths; many eons of the world contracting, many eons of the world expanding, many eons of the world contracting and expanding. And so I recollected my many kinds of past lives, with features and details. This was the first knowledge, which I achieved in the first watch of the night. Ignorance was destroyed and knowledge arose; darkness was destroyed and light arose, as happens for a meditator who is diligent, keen, and resolute. 

When\marginnote{38.1} my mind had immersed in \textsanskrit{samādhi} like this—purified, bright, flawless, rid of corruptions, pliable, workable, steady, and imperturbable—I extended it toward knowledge of the death and rebirth of sentient beings. With clairvoyance that is purified and superhuman, I saw sentient beings passing away and being reborn—inferior and superior, beautiful and ugly, in a good place or a bad place. I understood how sentient beings are reborn according to their deeds. 

This\marginnote{39.1} was the second knowledge, which I achieved in the middle watch of the night. Ignorance was destroyed and knowledge arose; darkness was destroyed and light arose, as happens for a meditator who is diligent, keen, and resolute. 

When\marginnote{40.1} my mind had immersed in \textsanskrit{samādhi} like this—purified, bright, flawless, rid of corruptions, pliable, workable, steady, and imperturbable—I extended it toward knowledge of the ending of defilements. I truly understood: ‘This is suffering’ … ‘This is the origin of suffering’ … ‘This is the cessation of suffering’ … ‘This is the practice that leads to the cessation of suffering’. I truly understood: ‘These are defilements’ … ‘This is the origin of defilements’ … ‘This is the cessation of defilements’ … ‘This is the practice that leads to the cessation of defilements’. 

Knowing\marginnote{41.1} and seeing like this, my mind was freed from the defilements of sensuality, desire to be reborn, and ignorance. When it was freed, I knew it was freed. 

I\marginnote{41.3} understood: ‘Rebirth is ended; the spiritual journey has been completed; what had to be done has been done; there is nothing further for this place.’ 

This\marginnote{42.1} was the third knowledge, which I achieved in the last watch of the night. Ignorance was destroyed and knowledge arose; darkness was destroyed and light arose, as happens for a meditator who is diligent, keen, and resolute. 

Then\marginnote{43.1} it occurred to me, ‘This principle I have discovered is deep, hard to see, hard to understand, peaceful, sublime, beyond the scope of logic, subtle, comprehensible to the astute. But people like clinging, they love it and enjoy it. It’s hard for them to see this topic; that is, specific conditionality, dependent origination. It’s also hard for them to see this topic; that is, the stilling of all activities, the letting go of all attachments, the ending of craving, fading away, cessation, extinguishment. And if I were to teach the Dhamma, others might not understand me, which would be wearying and troublesome for me.’ And then these verses, which were neither supernaturally inspired, nor learned before in the past, occurred to me: 

\begin{verse}%
‘I’ve\marginnote{43.8} struggled hard to realize this, \\
enough with trying to explain it! \\
Those mired in greed and hate \\
can’t really understand this teaching. 

It\marginnote{43.12} goes against the stream, subtle, \\
deep, obscure, and very fine. \\
Those besotted by greed cannot see, \\
for they’re shrouded in a mass of darkness.’ 

%
\end{verse}

And\marginnote{43.16} as I reflected like this, my mind inclined to remaining passive, not to teaching the Dhamma. 

Then\marginnote{44.1} the divinity Sahampati, knowing my train of thought, thought, ‘Alas! The world will be lost, the world will perish! For the mind of the Realized One, the perfected one, the fully awakened Buddha, inclines to remaining passive, not to teaching the Dhamma.’ 

Then\marginnote{44.3} the divinity Sahampati, as easily as a strong person would extend or contract their arm, vanished from the realm of divinity and reappeared in front of me. He arranged his robe over one shoulder, raised his joined palms toward me, and said, ‘Sir, let the Blessed One teach the Dhamma! Let the Holy One teach the Dhamma! There are beings with little dust in their eyes. They’re in decline because they haven’t heard the teaching. There will be those who understand the teaching!’ 

That’s\marginnote{44.8} what the divinity Sahampati said. Then he went on to say: 

\begin{verse}%
‘Among\marginnote{44.10} the Magadhans there appeared in the past \\
an impure teaching thought up by the stained. \\
Fling open the door to freedom from death! \\
Let them hear the teaching \\>the immaculate one discovered. 

Standing\marginnote{44.14} high on a rocky mountain, \\
you can see the people all around. \\
In just the same way, All-seer, so intelligent, \\
having ascended the Temple of Truth, 

rid\marginnote{44.18} of sorrow, look upon the people \\
swamped with sorrow,  \\>oppressed by rebirth and old age. \\
Rise, hero! Victor in battle, leader of the caravan, \\
wander the world free of debt. \\
Let the Blessed One teach the Dhamma! \\
There will be those who understand!’ 

%
\end{verse}

Then,\marginnote{45.1} understanding the Divinity’s invitation, I surveyed the world with the eye of a Buddha, because of my compassion for sentient beings. And I saw sentient beings with little dust in their eyes, and some with much dust in their eyes; with keen faculties and with weak faculties, with good qualities and with bad qualities, easy to teach and hard to teach. And some of them lived seeing the danger in the fault to do with the next world, while others did not. It’s like a pool with blue water lilies, or pink or white lotuses. Some of them sprout and grow in the water without rising above it, thriving underwater. Some of them sprout and grow in the water reaching the water’s surface. And some of them sprout and grow in the water but rise up above the water and stand with no water clinging to them. Then I replied in verse to the divinity Sahampati: 

\begin{verse}%
‘Flung\marginnote{45.6} open are the doors to freedom from death! \\
Let those with ears to hear commit to faith. \\
Thinking it would be troublesome, Divinity, \\>I did not teach \\
the sophisticated, sublime Dhamma among humans.’ 

%
\end{verse}

Then\marginnote{45.10} the divinity Sahampati, knowing that his request for me to teach the Dhamma had been granted, bowed and respectfully circled me, keeping me on his right, before vanishing right there. 

Then\marginnote{46.1} it occurred to me, ‘Who should I teach first of all? Who will quickly understand the teaching?’ Then it occurred to me, ‘That \textsanskrit{Āḷāra} \textsanskrit{Kālāma} is astute, competent, clever, and has long had little dust in his eyes. Why don’t I teach him first of all? He’ll quickly understand the teaching.’ But a deity came to me and said, ‘Sir, \textsanskrit{Āḷāra} \textsanskrit{Kālāma} passed away seven days ago.’ 

And\marginnote{46.10} knowledge and vision arose in me, ‘\textsanskrit{Āḷāra} \textsanskrit{Kālāma} passed away seven days ago.’ Then it occurred to me, ‘This is a great loss for \textsanskrit{Āḷāra} \textsanskrit{Kālāma}. If he had heard the teaching, he would have understood it quickly.’ 

Then\marginnote{47.1} it occurred to me, ‘Who should I teach first of all? Who will quickly understand the teaching?’ Then it occurred to me, ‘That Uddaka son of \textsanskrit{Rāma} is astute, competent, clever, and has long had little dust in his eyes. Why don’t I teach him first of all? He’ll quickly understand the teaching.’ But a deity came to me and said, ‘Sir, Uddaka son of \textsanskrit{Rāma}, passed away just last night.’ 

And\marginnote{47.10} knowledge and vision arose in me, ‘Uddaka son of \textsanskrit{Rāma}, passed away just last night.’ Then it occurred to me, ‘This is a great loss for Uddaka. If he had heard the teaching, he would have understood it quickly.’ 

Then\marginnote{48.1} it occurred to me, ‘Who should I teach first of all? Who will quickly understand the teaching?’ Then it occurred to me, ‘The group of five mendicants were very helpful to me. They looked after me during my time of resolute striving. Why don’t I teach them first of all?’ Then it occurred to me, ‘Where are the group of five mendicants staying these days?’ With clairvoyance that is purified and superhuman I saw that the group of five mendicants were staying near Varanasi, in the deer park at Isipatana. 

So,\marginnote{49.1} when I had stayed in \textsanskrit{Uruvelā} as long as I pleased, I set out for Varanasi. 

While\marginnote{49.2} I was traveling along the road between \textsanskrit{Gayā} and Bodhgaya, the \textsanskrit{Ājīvaka} ascetic Upaka saw me and said, ‘Reverend, your faculties are so very clear, and your complexion is pure and bright. In whose name have you gone forth, reverend? Who is your Teacher? Whose teaching do you believe in?’ 

I\marginnote{49.6} replied to Upaka in verse: 

\begin{verse}%
‘I\marginnote{49.7} am the champion, the knower of all, \\
unsullied in the midst of all things. \\
I’ve given up all, freed in the ending of craving. \\
Since I know for myself, whose follower should I be? 

I\marginnote{49.11} have no tutor. \\
There is no-one like me. \\
In the world with its gods, \\
I have no rival. 

For\marginnote{49.15} in this world, I am the perfected one; \\
I am the supreme Teacher. \\
I alone am fully awakened, \\
cooled, quenched. 

I\marginnote{49.19} am going to the city of \textsanskrit{Kāsi} \\
to roll forth the Wheel of Dhamma. \\
In this world that is so blind, \\
I’ll beat the drum of freedom from death!’ 

%
\end{verse}

‘According\marginnote{49.23} to what you claim, reverend, you ought to be the Infinite Victor.’ 

\begin{verse}%
‘The\marginnote{49.24} victors are those who, like me, \\
have reached the ending of defilements. \\
I have conquered bad qualities, Upaka—\\
that’s why I’m a victor.’ 

%
\end{verse}

When\marginnote{49.28} I had spoken, Upaka said: ‘If you say so, reverend.’ Shaking his head, he took a wrong turn and left. 

Traveling\marginnote{50.1} stage by stage, I arrived at Varanasi, and went to see the group of five mendicants in the deer park at Isipatana. The group of five mendicants saw me coming off in the distance and stopped each other, saying, ‘Here comes the ascetic Gotama. He’s so indulgent; he strayed from the struggle and returned to indulgence. We shouldn’t bow to him or rise for him or receive his bowl and robe. But we can set out a seat; he can sit if he likes.’ 

Yet\marginnote{50.7} as I drew closer, the group of five mendicants were unable to stop themselves as they had agreed. Some came out to greet me and receive my bowl and robe, some spread out a seat, while others set out water for washing my feet. But they still addressed me by name and as ‘reverend’. 

So\marginnote{51.1} I said to them, ‘Mendicants, don’t address me by name and as “reverend”. The Realized One is Perfected, a fully awakened Buddha. Listen up, mendicants: I have achieved freedom from death! I shall instruct you, I will teach you the Dhamma. By practicing as instructed you will soon realize the supreme end of the spiritual path in this very life. You will live having achieved with your own insight the goal for which gentlemen rightly go forth from the lay life to homelessness.’ 

But\marginnote{51.6} they said to me, ‘Reverend Gotama, even by that conduct, that practice, that grueling work you did not achieve any superhuman distinction in knowledge and vision worthy of the noble ones. How could you have achieved such a state now that you’ve become indulgent, strayed from the struggle and fallen into indulgence?’ 

So\marginnote{51.8} I said to them, ‘The Realized One has not become indulgent, strayed from the struggle and fallen into indulgence. The Realized One is Perfected, a fully awakened Buddha. Listen up, mendicants: I have achieved freedom from death! I shall instruct you, I will teach you the Dhamma. By practicing as instructed you will soon realize the supreme end of the spiritual path in this very life. You will live having achieved with your own insight the goal for which gentlemen rightly go forth from the lay life to homelessness.’ 

But\marginnote{51.13} for a second time they said to me, ‘Reverend Gotama … you’ve fallen into indulgence.’ 

So\marginnote{51.15} for a second time I said to them, ‘The Realized One has not become indulgent …’ 

But\marginnote{51.20} for a third time they said to me, ‘Reverend Gotama … you’ve fallen into indulgence.’ 

So\marginnote{52.1} I said to them, ‘Mendicants, have you ever known me to speak like this before?’ 

‘No,\marginnote{52.3} sir.’ 

‘The\marginnote{52.4} Realized One is Perfected, a fully awakened Buddha. Listen up, mendicants: I have achieved freedom from death! I shall instruct you, I will teach you the Dhamma. By practicing as instructed you will soon realize the supreme end of the spiritual path in this very life. You will live having achieved with your own insight the goal for which gentlemen rightly go forth from the lay life to homelessness.’ 

I\marginnote{53.1} was able to persuade the group of five mendicants. Then sometimes I advised two mendicants, while the other three went for alms. Then those three would feed all six of us with what they brought back. Sometimes I advised three mendicants, while the other two went for alms. Then those two would feed all six of us with what they brought back. 

As\marginnote{54.1} the group of five mendicants were being advised and instructed by me like this, they soon realized the supreme end of the spiritual path in this very life. They lived having achieved with their own insight the goal for which gentlemen rightly go forth from the lay life to homelessness.” 

When\marginnote{55.1} he had spoken, Prince Bodhi said to the Buddha, “Sir, when a mendicant has the Realized One as trainer, how long would it take for them to realize the supreme end of the spiritual path in this very life?” 

“Well\marginnote{55.3} then, prince, I’ll ask you about this in return, and you can answer as you like. What do you think, prince? Are you skilled in the art of wielding a hooked goad while riding an elephant?”\footnote{According to the commentary, it was Pajjota’s desire to learn the art of training elephants from Bodhi’s father Udena that, after a colorful series of events, ultimately led to the joining of their houses and the birth of Bodhi. While the story is framed as a romantic adventure, it shows the strategic importance of trained elephants for royal power. } 

“Yes,\marginnote{55.6} sir.” 

“What\marginnote{56.1} do you think, prince? Suppose a man were to come along thinking, ‘Prince Bodhi knows the art of wielding a hooked goad while riding an elephant. I’ll train in that art under him.’ If he’s faithless, he wouldn’t achieve what he could with faith. If he’s unhealthy, he wouldn’t achieve what he could with good health. If he’s devious or deceitful, he wouldn’t achieve what he could with honesty and integrity. If he’s lazy, he wouldn’t achieve what he could with energy. If he’s stupid, he wouldn’t achieve what he could with wisdom. What do you think, prince? Could that man still train under you in the art of wielding a hooked goad while riding an elephant?” 

“Sir,\marginnote{56.17} if he had even a single one of these factors he couldn’t train under me, let alone all five.” 

“What\marginnote{57.1} do you think, prince? Suppose a man were to come along thinking, ‘Prince Bodhi knows the art of wielding a hooked goad while riding an elephant. I’ll train in that art under him.’ If he’s faithful, he’d achieve what he could with faith. If he’s healthy, he’d achieve what he could with good health. If he’s honest and has integrity, he’d achieve what he could with honesty and integrity. If he’s energetic, he’d achieve what he could with energy. If he’s wise, he’d achieve what he could with wisdom. What do you think, prince? Could that man still train under you in the art of wielding a hooked goad while riding an elephant?” 

“Sir,\marginnote{57.17} if he had even a single one of these factors he could train under me, let alone all five.” 

“In\marginnote{58.1} the same way, prince, there are these five factors that support meditation.\footnote{\textit{\textsanskrit{Padhāna}} is literally to “fix” or “place” and hence to “try” or “strive”. Its core application is to apply oneself to spiritual practices, and hence in Buddhism it overlaps in meaning with “meditation”; a \textit{\textsanskrit{padhānasāla}} is a “meditation hall”. Here the five factors are qualities that are essential foundations for meditation. } What five? It’s when a noble disciple has faith in the Realized One’s awakening: ‘That Blessed One is perfected, a fully awakened Buddha, accomplished in knowledge and conduct, holy, knower of the world, supreme guide for those who wish to train, teacher of gods and humans, awakened, blessed.’ They are rarely ill or unwell. Their stomach digests well, being neither too hot nor too cold, but just right, and fit for meditation. They’re not devious or deceitful. They reveal themselves honestly to the Teacher or sensible spiritual companions. They live with energy roused up for giving up unskillful qualities and embracing skillful qualities. They’re strong, staunchly vigorous, not slacking off when it comes to developing skillful qualities. They’re wise. They have the wisdom of arising and passing away which is noble, penetrative, and leads to the complete ending of suffering. These are the five factors that support meditation. 

When\marginnote{59.1} a mendicant with these five factors that support meditation has the Realized One as trainer, they could realize the supreme end of the spiritual path in seven years. Let alone seven years, they could realize the supreme end of the spiritual path in six years, or as little as one year. Let alone one year, when a mendicant with these five factors that support meditation has the Realized One as trainer, they could realize the supreme end of the spiritual path in seven months, or as little as one day. Let alone one day, when a mendicant with these five factors that support meditation has the Realized One as trainer, they could be instructed In the evening and achieve distinction in the morning, or be instructed in the morning and achieve distinction in the evening.” 

When\marginnote{60.1} he had spoken, Prince Bodhi said to the Buddha, “Oh, the Buddha! Oh, the teaching! Oh, how well explained is the teaching! For someone could be instructed in the evening and achieve distinction in the morning, or be instructed in the morning and achieve distinction in the evening.” 

When\marginnote{61.1} he said this, \textsanskrit{Sañjikāputta} said to Prince Bodhi, “Though Mister Bodhi speaks like this, you don’t go for refuge to Mister Gotama, to the teaching, and to the mendicant \textsanskrit{Saṅgha}.” 

“Don’t\marginnote{61.5} say that, dear \textsanskrit{Sañjikāputta}, don’t say that! I have heard and learned this in the presence of my lady mother.\footnote{His “lady mother” (\textit{\textsanskrit{ayyā}}), as noted above, was Queen \textsanskrit{Vāsuladattā}, daughter of Pajjota and wife of Udena. | This is the only place in the Pali canon where the idiom “heard and learned this in the presence” is applied to anyone other than the Buddha. } This one time the Buddha was staying near \textsanskrit{Kosambī}, in Ghosita’s Monastery. Then my pregnant lady mother went up to the Buddha, bowed, sat down to one side, and said to him, ‘Sir, the prince or princess in my womb goes for refuge to the Buddha, the teaching, and the mendicant \textsanskrit{Saṅgha}. From this day forth, may the Buddha remember them as a lay follower who has gone for refuge for life.’ 

Another\marginnote{61.11} time the Buddha was staying here in the land of the Bhaggas at Crocodile Hill, in the deer park at \textsanskrit{Bhesakaḷā}’s Wood. Then my nursemaid, carrying me on her hip, went to the Buddha, bowed, stood to one side, and said to him, ‘Sir, this Prince Bodhi goes for refuge to the Buddha, to the teaching, and to the mendicant \textsanskrit{Saṅgha}. From this day forth, may the Buddha remember him as a lay follower who has gone for refuge for life.’ 

Now\marginnote{61.15} for a third time I go for refuge to the Buddha, to the teaching, and to the mendicant \textsanskrit{Saṅgha}. From this day forth, may the Buddha remember me as a lay follower who has gone for refuge for life.” 

%
\section*{{\suttatitleacronym MN 86}{\suttatitletranslation With Aṅgulimāla }{\suttatitleroot Aṅgulimālasutta}}
\addcontentsline{toc}{section}{\tocacronym{MN 86} \toctranslation{With Aṅgulimāla } \tocroot{Aṅgulimālasutta}}
\markboth{With Aṅgulimāla }{Aṅgulimālasutta}
\extramarks{MN 86}{MN 86}

\scevam{So\marginnote{1.1} I have heard. }At one time the Buddha was staying near \textsanskrit{Sāvatthī} in Jeta’s Grove, \textsanskrit{Anāthapiṇḍika}’s monastery. 

Now\marginnote{2.1} at that time in the realm of King Pasenadi of Kosala there was a bandit named \textsanskrit{Aṅgulimāla}. He was violent, bloody-handed, a hardened killer, merciless to living beings.\footnote{This is a rare sutta where the Buddha’s words take a back seat to dramatic narrative. The reform of the serial killer \textsanskrit{Aṅgulimāla} became one of the most popular stories in Buddhism, with is its message of redemption even for the most wicked. The narrative with extensive backstory has been elaborated in the Pali commentaries and several parallels, with various overlaps and divergences, where events become inflated to an implausible degree. The tendency to exaggeration is found even here, for example in the claim that \textsanskrit{Aṅgulimāla} laid waste to entire countries. } He laid waste to villages, towns, and countries. He was constantly murdering people, and he wore their fingers as a necklace.\footnote{\textsanskrit{Aṅgulimāla}’s eponymous grotesquerie of a “finger garland” is explained by the commentary and some parallels as being a count of his victims killed by order of his former Brahmanical teacher, who assigned him this task out of jealousy. This unlikely scenario was rejected by Richard Gombrich (\emph{How Buddhism Began}, pg. 151), who proposed that \textsanskrit{Aṅgulimāla} belonged to a violent cult. The shared core of canonical accounts, however, ascribe no religious motivation to his acts: he is just a killer. He is driven by untrammeled will to power, the need to show he is stronger and better. } 

Then\marginnote{3.1} the Buddha robed up in the morning and, taking his bowl and robe, entered \textsanskrit{Sāvatthī} for alms. Then, after the meal, on his return from almsround, he set his lodgings in order and, taking his bowl and robe, he walked down the road that led to \textsanskrit{Aṅgulimāla}. 

The\marginnote{3.3} cowherds, shepherds, farmers, and travelers saw him on the road, and said to him, “Don’t take this road, ascetic. On this road there is a bandit named \textsanskrit{Aṅgulimāla}. He is violent, bloody-handed, a hardened killer, merciless to living beings. He has laid waste to villages, towns, and countries. He is constantly murdering people, and he wears their fingers as a necklace. People travel along this road only after banding closely together in groups of ten, twenty, thirty, forty, or fifty. Still they meet their end by \textsanskrit{Aṅgulimāla}’s hand.”\footnote{\textit{Attha} in \textit{hatthattha} means “death, end” (commentary: \textit{\textsanskrit{atthaṁ} \textsanskrit{vināsaṁ}}). } But when they said this, the Buddha went on in silence. 

For\marginnote{3.12} a second time … and a third time, they urged the Buddha to turn back. 

But\marginnote{4.1} when they said this, the Buddha went on in silence. 

The\marginnote{4.2} bandit \textsanskrit{Aṅgulimāla} saw the Buddha coming off in the distance, and thought, “Oh, how incredible, how amazing! People travel along this road only after banding closely together in groups of ten, twenty, thirty, forty, or fifty. Still they meet their end by my hand. But still this ascetic comes along alone and unaccompanied, like a conqueror.\footnote{For “like a conqueror” (\textit{pasayha \textsanskrit{maññe}}) compare how the “fierce-fanged lion, king of beasts” wanders as conqueror at \href{https://suttacentral.net/snp1.3/en/sujato\#38.1}{Snp 1.3:38.1}. \textsanskrit{Aṅgulimāla} sees the Buddha’s strength as a challenge; he needs to prove himself the mightiest. } Why don’t I take his life?” 

Then\marginnote{5.1} \textsanskrit{Aṅgulimāla} donned his sword and shield, fastened his bow and arrows, and followed behind the Buddha. But the Buddha used his psychic power to will that \textsanskrit{Aṅgulimāla} could not catch up with him no matter how hard he tried, even though the Buddha kept walking at a normal speed. 

Then\marginnote{5.3} \textsanskrit{Aṅgulimāla} thought, “Oh, how incredible, how amazing! Previously, even when I’ve chased a speeding elephant, horse, chariot or deer, I’ve always caught up with them. But I can’t catch up with this ascetic no matter how hard I try, even though he’s walking at a normal speed.”\footnote{The decisive turn comes when \textsanskrit{Aṅgulimāla} must admit he is not the strongest. } 

He\marginnote{5.7} stood still and said, “Stop, stop, ascetic!” 

“I’ve\marginnote{5.9} stopped, \textsanskrit{Aṅgulimāla}—now you stop.” 

Then\marginnote{5.10} \textsanskrit{Aṅgulimāla} thought, “These ascetics who follow the Sakyan speak the truth. Yet while walking the ascetic Gotama says: ‘I’ve stopped, \textsanskrit{Aṅgulimāla}—now you stop.’ Why don’t I ask him about this?” 

Then\marginnote{6.1} he addressed the Buddha in verse: 

\begin{verse}%
“While\marginnote{6.2} walking, ascetic, you say ‘I’ve stopped.’\footnote{These verses in this sutta are found in \textsanskrit{Aṅgulimāla}’s verses at \href{https://suttacentral.net/thag16.8/en/sujato}{Thag 16.8}. } \\
And I have stopped, but you tell me I’ve not. \\
I’m asking you this, ascetic: \\
how is it you’ve stopped and I have not?” 

“\textsanskrit{Aṅgulimāla},\marginnote{6.6} I have forever stopped—\\
I’ve laid aside violence towards all creatures.\footnote{This phrase is shared with the Jains, \textsanskrit{Sūyagaḍa} 14.23: \textit{savvehi \textsanskrit{pāṇehi} \textsanskrit{nihāya} \textsanskrit{daṇḍaṁ}}. } \\
But you can’t stop yourself \\>from harming living creatures; \\
that’s why I’ve stopped, but you have not.” 

“Oh,\marginnote{6.10} at long last a renowned great seer, \\
an ascetic has followed me into this deep wood.\footnote{Read \textit{\textsanskrit{samaṇo} \textsanskrit{paccapādi}} per \href{https://suttacentral.net/thag16.8/en/sujato\#3.2}{Thag 16.8:3.2}. } \\
Now that I’ve heard your verse on Dhamma, \\
I shall live without evil.” 

With\marginnote{6.14} these words, \\>the bandit hurled his sword and weapons \\
down a cliff into an abyss. \\
He venerated the Holy One’s feet, \\
and asked him for the going forth right away. 

Then\marginnote{6.18} the Buddha, the compassionate great seer, \\
the teacher of the world with its gods, \\
said to him, “Come, monk!” \\
And with that he became a monk. 

%
\end{verse}

Then\marginnote{7.1} the Buddha set out for \textsanskrit{Sāvatthī} with Venerable \textsanskrit{Aṅgulimāla} as his second monk. Traveling stage by stage, he arrived at \textsanskrit{Sāvatthī}, where he stayed in Jeta’s Grove, \textsanskrit{Anāthapiṇḍika}’s monastery. 

Now\marginnote{8.1} at that time a crowd had gathered by the gate of King Pasenadi’s royal compound making a dreadful racket, “In your realm, Your Majesty, there is a bandit named \textsanskrit{Aṅgulimāla}. He is violent, bloody-handed, a hardened killer, merciless to living beings. He has laid waste to villages, towns, and countries. He is constantly murdering people, and he wears their fingers as a necklace. Your Majesty must put a stop to him!” 

Then\marginnote{9.1} King Pasenadi drove out from \textsanskrit{Sāvatthī} in the middle of the day with around five hundred horses, heading for the monastery. He went by carriage as far as the terrain allowed, then descended and approached the Buddha on foot. He bowed and sat down to one side. The Buddha said to him, 

“What\marginnote{9.4} is it, great king? Is King Seniya \textsanskrit{Bimbisāra} of Magadha angry with you, or the Licchavis of \textsanskrit{Vesālī}, or some other opposing ruler?” 

“No,\marginnote{10.1} sir. In my realm there is a bandit named \textsanskrit{Aṅgulimāla}. He is violent, bloody-handed, a hardened killer, merciless to living beings. … I shall put a stop to him.”\footnote{It is Pasenadi’s duty, not the Buddha’s, to defend his citizens. Yet while the Buddha had already identified the problem, taken action, and solved it, Pasenadi is just now coming to the Buddha to announce his intention to do something about it. A certain criticism of Pasanadi’s weakness can be discerned here. } 

“But\marginnote{11.1} great king, suppose you were to see that \textsanskrit{Aṅgulimāla} had shaved off his hair and beard, dressed in ocher robes, and gone forth from the lay life to homelessness. And that he was refraining from killing living creatures, stealing, and lying; that he was eating in one part of the day, and was celibate, ethical, and of good character. What would you do to him?” 

“I\marginnote{11.2} would bow to him, rise in his presence, or offer him a seat. I’d invite him to accept robes, almsfood, lodgings, and medicines and supplies for the sick. And I’d organize his lawful guarding and protection.\footnote{As \textsanskrit{Ajātasattu} would for a runaway slave who ordained at \href{https://suttacentral.net/dn2/en/sujato\#36.2}{DN 2:36.2}. } But sir, how could such an immoral, evil man ever have such virtue and restraint?” 

Now\marginnote{12.1} at that time Venerable \textsanskrit{Aṅgulimāla} was sitting not far from the Buddha. Then the Buddha pointed with his right arm and said to the king, “Great king, this is \textsanskrit{Aṅgulimāla}.” 

Then\marginnote{12.4} the king became frightened, scared, his hair standing on end. Knowing this, the Buddha said to him, “Do not fear, great king. You have nothing to fear from him.”\footnote{Also per \textsanskrit{Ajātasattu} (\href{https://suttacentral.net/dn2/en/sujato\#10.8}{DN 2:10.8}). } Then the king’s fear died down. 

Then\marginnote{12.8} the king went over to \textsanskrit{Aṅgulimāla} and said, “Sir, is the master really \textsanskrit{Aṅgulimāla}?” 

“Yes,\marginnote{12.10} great king.” 

“What\marginnote{12.11} clans were your father and mother from?” 

“My\marginnote{12.12} father was a Gagga, and my mother a \textsanskrit{Mantāṇī}.”\footnote{Gagga is a clan descended from the Vedic seer Garga \textsanskrit{Bhāradvāja} (Rig Veda 6.47). Clan members feature prominently throughout Brahmanical literature (\textsanskrit{Bṛhadāraṇyaka} \textsanskrit{Upaniṣad} 2.1, 3.6, 3.8; \textsanskrit{Praśna} \textsanskrit{Upaniṣad} 4, etc.), none of whom, however, share \textsanskrit{Aṅgulimāla}’s depravity. | \textsanskrit{Mantāṇī}, spelled \textsanskrit{Maitrāyaṇī} in Sanskrit (\textsanskrit{Mahāvastu} 3.377, cf. \textsanskrit{Puṇṇa} son of \textsanskrit{Mantāṇī} at \href{https://suttacentral.net/mn24/en/sujato\#2.4}{MN 24:2.4}), is shared with the \textsanskrit{Maitrāyaṇī} \textsanskrit{Saṁhitā}, the oldest ritual text of the Black Yajurveda. } 

“Sir,\marginnote{12.13} may Master Gagga son of \textsanskrit{Mantāṇī} be happy. I’ll make sure that you’re provided with robes, almsfood, lodgings, and medicines and supplies for the sick.” 

But\marginnote{13.1} at that time Venerable \textsanskrit{Aṅgulimāla} lived in the wilderness, ate only almsfood, and owned just three robes. So he said to the king, “Enough, great king. My robes are complete.” 

Then\marginnote{13.4} the king went back to the Buddha, bowed, sat down to one side, and said to him, “It’s incredible, sir, it’s amazing! How the Buddha tames those who are wild, pacifies those who are violent, and extinguishes those who are unquenched! For I was not able to tame him with the rod and the sword, but the Buddha tamed him without rod or sword.\footnote{This heads off a potential criticism of Buddhism that it would allow criminals to flourish. Non-violent approaches are not just morally superior but more effective. } Well, now, sir, I must go. I have many duties, and much to do.” 

“Please,\marginnote{13.10} great king, go at your convenience.” Then King Pasenadi got up from his seat, bowed, and respectfully circled the Buddha, keeping him on his right, before leaving. 

Then\marginnote{14.1} Venerable \textsanskrit{Aṅgulimāla} robed up in the morning and, taking his bowl and robe, entered \textsanskrit{Sāvatthī} for alms. Then as he was wandering indiscriminately for almsfood he saw a woman undergoing a distressing obstructed labor.\footnote{\textit{\textsanskrit{Mūḷha}} here has the sense of “gone astray” in reference to obstructed labor that prevents birth; see \href{https://suttacentral.net/ud2.8/en/sujato\#1.4}{Ud 2.8:1.4}. } Seeing this, it occurred to him, “Oh, beings undergo such travail!\footnote{\textit{Kilissanti} here has the sense “undergo distress or travail”. Compare \textit{\textsanskrit{garbhakleśa}} in Mārkaṇḍeya Purāṇa 22.45 with the sense, “travails of childbirth”. } Oh, beings undergo such travail!” 

Then\marginnote{14.6} after wandering for alms in \textsanskrit{Sāvatthī}, after the meal, on his return from almsround, he went to the Buddha, bowed, sat down to one side, and told him what had happened. The Buddha said to him, “Well then, \textsanskrit{Aṅgulimāla}, go to that woman and say this: 

‘Ever\marginnote{15.2} since I was born, sister, I don’t recall having intentionally taken the life of a living creature. By this truth, may both you and your baby be safe.’”\footnote{The asseveration of truth—the belief that “by this truth” (\textit{tena saccena}, also at \href{https://suttacentral.net/snp2.1/en/sujato}{Snp 2.1}) effective results could be won—can be traced to Rig Veda 1.21.6 (\textit{tena satyena}). } 

“But\marginnote{15.3} sir, wouldn’t that be telling a deliberate lie? For I have intentionally killed many living creatures.” 

“In\marginnote{15.5} that case, \textsanskrit{Aṅgulimāla}, go to that woman and say this: 

‘Ever\marginnote{15.6} since I was born in the noble birth, sister, I don’t recall having intentionally taken the life of a living creature. By this truth, may both you and your baby be safe.’”\footnote{This is still used as a blessing for pregnant women in Theravada. | “Noble birth” is a unique way of referring to ordination. It echoes the \textit{upanayana} initiation, where the teacher is said to become pregnant with the student and give birth to them as a brahmin (Śatapatha \textsanskrit{Brāhmaṇa} 11.5.4; also see notes to \href{https://suttacentral.net/mn26/en/sujato\#15.2}{MN 26:15.2} ff.). } 

“Yes,\marginnote{15.7} sir,” replied \textsanskrit{Aṅgulimāla}. He went to that woman and said: 

“Ever\marginnote{15.8} since I was born in the noble birth, sister, I don’t recall having intentionally taken the life of a living creature. By this truth, may both you and your baby be safe.” 

Then\marginnote{15.9} that woman was safe, and so was her baby. 

Then\marginnote{16.1} \textsanskrit{Aṅgulimāla}, living alone, withdrawn, diligent, keen, and resolute, soon realized the supreme end of the spiritual path in this very life. He lived having achieved with his own insight the goal for which gentlemen rightly go forth from the lay life to homelessness. 

He\marginnote{16.2} understood: “Rebirth is ended; the spiritual journey has been completed; what had to be done has been done; there is nothing further for this place.” And Venerable \textsanskrit{Aṅgulimāla} became one of the perfected. 

Then\marginnote{17.1} Venerable \textsanskrit{Aṅgulimāla} robed up in the morning and, taking his bowl and robe, entered \textsanskrit{Sāvatthī} for alms. Now at that time someone threw a stone that hit \textsanskrit{Aṅgulimāla}, someone else threw a stick, and someone else threw gravel. Then \textsanskrit{Aṅgulimāla}—with cracked head, bleeding, his bowl broken, and his outer robe torn—went to the Buddha. 

The\marginnote{17.4} Buddha saw him coming off in the distance, and said to him, “Endure it, brahmin! Endure it, brahmin! You’re experiencing in this life the result of deeds that might have caused you to be tormented in hell for many years, many hundreds or thousands of years.”\footnote{For the residual effects of kamma, compare such suttas as \href{https://suttacentral.net/sn56.49/en/sujato}{SN 56.49}, where for a stream-enterer the suffering that has disappeared is like a great mountain, while that which remains is like seven pebbles. For \textsanskrit{Aṅgulimāla} as an arahant there is even less, but not none at all. } 

Later,\marginnote{18.1} Venerable \textsanskrit{Aṅgulimāla} was experiencing the bliss of release while in private retreat. On that occasion he expressed this heartfelt sentiment: 

\begin{verse}%
“He\marginnote{18.3} who once was heedless, \\
but turned to heedfulness, \\
lights up the world, \\
like the moon freed from clouds. 

Someone\marginnote{18.7} whose bad deed \\
is supplanted by the good, \\
lights up the world, \\
like the moon freed from clouds. 

A\marginnote{18.11} young mendicant \\
devoted to the Buddha’s teaching, \\
lights up the world, \\
like the moon freed from clouds. 

May\marginnote{18.15} even my enemies \\>hear a Dhamma talk! \\
May even my enemies \\>devote themselves to the Buddha’s teaching! \\
May even my enemies \\>associate with those good people \\
who establish others in the Dhamma! 

May\marginnote{18.19} even my enemies \\>hear Dhamma at the right time, \\
from those who teach acceptance, \\
praising acquiescence; \\
and may they follow that path! 

For\marginnote{18.23} then they’d never wish harm \\
upon myself or others. \\
Having arrived at ultimate peace,\footnote{\textit{Pappuyya} is absolutive. } \\
they’d look after creatures firm and frail.\footnote{Creatures “firm and frail” (\textit{\textsanskrit{tasathāvare}}) is a common Jain idiom. Examples include \textit{tasesu \textsanskrit{thāvaresu} ya} (\textsanskrit{Uttarādhyayana} 5.8); \textit{je \textsanskrit{keī} \textsanskrit{tasathāvarā}} (\textsanskrit{Sūyagaḍa} 3.4.20); \textit{\textsanskrit{tasaṁ} \textsanskrit{vā} \textsanskrit{thāvaraṁ} \textsanskrit{vā}} (\textsanskrit{Dasaveyāliya} 4.1.42); \textit{tasa-\textsanskrit{jīvā} ya \textsanskrit{thāvarattāe}} (\textsanskrit{Ācāraṅgasūtra} 9.1.14). } 

For\marginnote{18.27} irrigators guide the water, \\
and fletchers straighten arrows; \\
carpenters carve timber—\\
but the astute tame themselves. 

Some\marginnote{18.31} tame by using the rod, \\
some with goads, and some with whips. \\
But the unaffected one tamed me \\
without rod or sword. 

My\marginnote{18.35} name is ‘Harmless’,\footnote{\textsanskrit{Aṅgulimāla} became known as “Harmless” (\textsanskrit{Ahiṁsaka}) after his awakening. } \\
though I used to be harmful. \\
The name I bear today is true, \\
for I do no harm to anyone. 

I\marginnote{18.39} used to be a bandit, \\
the notorious \textsanskrit{Aṅgulimāla}. \\
Swept away in a great flood, \\
I went to the Buddha as a refuge. 

I\marginnote{18.43} used to have blood on my hands, \\
the notorious \textsanskrit{Aṅgulimāla}. \\
See the refuge I’ve found—\\
the conduit to rebirth is eradicated. 

I’ve\marginnote{18.47} done many of the sort of deeds \\
that lead to a bad destination. \\
The result of my deeds has already struck me, \\
so I enjoy my food free of debt. 

Fools\marginnote{18.51} and simpletons \\
devote themselves to negligence. \\
But the wise protect diligence \\
as their best treasure. 

Don’t\marginnote{18.55} devote yourself to negligence, \\
or delight in erotic intimacy. \\
For if you’re diligent and practice absorption, \\
you’ll attain abundant happiness. 

It\marginnote{18.59} was welcome, not unwelcome, \\
the advice I got was good. \\
Of the well-explained teachings, \\
I arrived at the best. 

It\marginnote{18.63} was welcome, not unwelcome, \\
the advice I got was good. \\
I’ve attained the three knowledges \\
and fulfilled the Buddha’s instructions.” 

%
\end{verse}

%
\section*{{\suttatitleacronym MN 87}{\suttatitletranslation Born From the Beloved }{\suttatitleroot Piyajātikasutta}}
\addcontentsline{toc}{section}{\tocacronym{MN 87} \toctranslation{Born From the Beloved } \tocroot{Piyajātikasutta}}
\markboth{Born From the Beloved }{Piyajātikasutta}
\extramarks{MN 87}{MN 87}

\scevam{So\marginnote{1.1} I have heard.\footnote{This sutta records King Pasenadi’s conversion due to the wise counsel of his chief queen, \textsanskrit{Mallikā} (“Jasmine”). Similar conversations with \textsanskrit{Mallikā} on attachment to loved ones are found at \href{https://suttacentral.net/sn3.8/en/sujato}{SN 3.8} and \href{https://suttacentral.net/ud5.1/en/sujato}{Ud 5.1}. King Pasenadi’s committment to non-attachment was sorely tested with the birth of a daughter (\href{https://suttacentral.net/sn3.16/en/sujato}{SN 3.16}), and when \textsanskrit{Mallikā} passed away (\href{https://suttacentral.net/an5.49/en/sujato}{AN 5.49}). } }At one time the Buddha was staying near \textsanskrit{Sāvatthī} in Jeta’s Grove, \textsanskrit{Anāthapiṇḍika}’s monastery. 

Now\marginnote{2.1} at that time a certain householder’s dear and beloved only child passed away. After their death he didn’t feel like working or eating.\footnote{An unusual use of \textit{\textsanskrit{paṭibhāti}} (“feel like”); commentary glosses \textit{ruccati}. } He would go to the cremation ground and wail, “Where are you, my only child? Where are you, my only child?” 

Then\marginnote{3.1} he went to the Buddha, bowed, and sat down to one side. The Buddha said to him, “Your faculties, householder, are those of one who is unstable in their own mind; there is a deterioration in your faculties.” 

“And\marginnote{3.3} how, sir, could there be no deterioration of my faculties? For my dear and beloved only child has passed away. Since their death I haven’t felt like working or eating. I go to the cremation ground and wail: ‘Where are you, my only child? Where are you, my only child?’” 

“That’s\marginnote{3.8} so true, householder! That’s so true, householder! For our loved ones are a source of sorrow, lamentation, pain, sadness, and distress.” 

“Sir,\marginnote{3.10} who on earth could ever think such a thing! For our loved ones are a source of joy and happiness.” Disagreeing with the Buddha’s statement, rejecting it, he got up from his seat and left. 

Now\marginnote{4.1} at that time several gamblers were playing dice not far from the Buddha. That householder approached them and told them what had happened. 

“That’s\marginnote{4.17} so true, householder! That’s so true, householder! For our loved ones are a source of joy and happiness.” 

Thinking,\marginnote{4.19} “The gamblers and I are in agreement,” the householder left. 

Eventually\marginnote{5.1} that topic of discussion reached the royal compound.\footnote{Also a scene for gossip in \href{https://suttacentral.net/an3.60/en/sujato}{AN 3.60} and \href{https://suttacentral.net/an10.45/en/sujato}{AN 10.45}. } Then King Pasenadi addressed Queen \textsanskrit{Mallikā},\footnote{In addition to the conversations on the theme of attachemnt to loved ones, unrelated incidents with \textsanskrit{Mallikā} appear at \href{https://suttacentral.net/an4.197/en/sujato}{AN 4.197}, \href{https://suttacentral.net/pli-tv-bu-vb-pc53/en/sujato\#1.1}{Bu Pc 53:1.1}, and \href{https://suttacentral.net/pli-tv-bu-vb-pc83/en/sujato\#1.2.7}{Bu Pc 83:1.2.7}. She is listed as an eminent lay disciple (\href{https://suttacentral.net/an8.91-117/en/sujato\#1.1}{AN 8.91–117:1.1} and maintained a monastery for philosophical debates (\href{https://suttacentral.net/mn78/en/sujato\#1.3}{MN 78:1.3}). This is what we know of her from early texts, while later \textsanskrit{Jātakas} and other scriptures tell many colorful tales about her and her unique marriage. | \textsanskrit{Śvetāmbara} Jains hold that, in the far distant past, the nineteenth founding teacher (\textit{\textsanskrit{tīrthaṅkara}}) was the lady Malli (\textsanskrit{Triṣaṣṭiśalākāpuruṣacaritra} 6.6). Perhaps the Buddhist stories of the wise lady \textsanskrit{Mallikā} helped inspire this legend. } “Mallika, your ascetic Gotama said this: ‘Our loved ones are a source of sorrow, lamentation, pain, sadness, and distress.’” 

“If\marginnote{5.5} that’s what the Buddha said, great king, then that’s how it is.” 

“No\marginnote{5.6} matter what the ascetic Gotama says, \textsanskrit{Mallikā} agrees with him: ‘If that’s what the Buddha said, great king, then that’s how it is.’ You’re just like a pupil who agrees with everything their tutor says. Go away, \textsanskrit{Mallikā}, get out of here!” 

Then\marginnote{6.1} Queen \textsanskrit{Mallikā} addressed the brahmin \textsanskrit{Nāḷijaṅgha}, “Please, brahmin, go to the Buddha, and in my name bow with your head to his feet. Ask him if he is healthy and well, nimble, strong, and living comfortably. And then say: ‘Sir, did the Buddha make this statement: “Our loved ones are a source of sorrow, lamentation, pain, sadness, and distress”?’ Remember well how the Buddha answers and tell it to me. For Realized Ones say nothing that is not so.” 

“Yes,\marginnote{6.9} ma’am,” he replied. He went to the Buddha and exchanged greetings with him. When the greetings and polite conversation were over, he sat down to one side and said to the Buddha, “Mister Gotama, Queen \textsanskrit{Mallikā} bows with her head to your feet. She asks if you are healthy and well, nimble, strong, and living comfortably. And she asks whether the Buddha made this statement: ‘Our loved ones are a source of sorrow, lamentation, pain, sadness, and distress.’” 

“That’s\marginnote{7.1} right, brahmin, that’s right! For our loved ones are a source of sorrow, lamentation, pain, sadness, and distress. 

And\marginnote{8.1} here’s a way to understand how our loved ones are a source of sorrow, lamentation, pain, sadness, and distress. Once upon a time right here in \textsanskrit{Sāvatthī} a certain woman’s mother passed away. And because of that she went mad and lost her mind. She went from street to street and from square to square saying, ‘Has anyone seen my mother? Has anyone seen my mother?’ 

And\marginnote{9{-}14.1} here’s another way to understand how our loved ones are a source of sorrow, lamentation, pain, sadness, and distress. 

Once\marginnote{9{-}14.2} upon a time right here in \textsanskrit{Sāvatthī} a certain woman’s father … brother … sister … son … daughter … husband passed away. And because of that she went mad and lost her mind. She went from street to street and from square to square saying, ‘Has anyone seen my husband? Has anyone seen my husband?’ 

And\marginnote{15{-}21.1} here’s another way to understand how our loved ones are a source of sorrow, lamentation, pain, sadness, and distress. 

Once\marginnote{15{-}21.2} upon a time right here in \textsanskrit{Sāvatthī} a certain man’s mother … father … brother … sister … son … daughter … wife passed away. And because of that he went mad and lost his mind. He went from street to street and from square to square saying, ‘Has anyone seen my wife? Has anyone seen my wife?’ 

And\marginnote{15{-}21.14} here’s another way to understand how our loved ones are a source of sorrow, lamentation, pain, sadness, and distress. 

Once\marginnote{22.1} upon a time right here in \textsanskrit{Sāvatthī} a certain woman went to live with her relative’s family. But her relatives wanted to divorce her from her husband and give her to another, who she didn’t want. So she told her husband about this. But he cut her in two and disemboweled himself, thinking, ‘We shall be together after death.’ That’s another way to understand how our loved ones are a source of sorrow, lamentation, pain, sadness, and distress.” 

Then\marginnote{23.1} \textsanskrit{Nāḷijaṅgha} the brahmin, having approved and agreed with what the Buddha said, got up from his seat, went to Queen \textsanskrit{Mallikā}, and told her of all they had discussed. Then Queen \textsanskrit{Mallikā} approached King Pasenadi and said to him, “What do you think, great king? Do you love Princess \textsanskrit{Vajirī}?” 

“Indeed\marginnote{24.3} I do, \textsanskrit{Mallikā}.” 

“What\marginnote{24.4} do you think, great king? If she were to decay and perish, would sorrow, lamentation, pain, sadness, and distress arise in you?” 

“If\marginnote{24.6} she were to decay and perish, my life would fall apart. How could sorrow, lamentation, pain, sadness, and distress not arise in me?” 

“This\marginnote{24.7} is what the Buddha was referring to when he said: ‘Our loved ones are a source of sorrow, lamentation, pain, sadness, and distress.’ 

What\marginnote{25.1} do you think, great king? Do you love Lady \textsanskrit{Vāsabhā}? … 

Do\marginnote{26.1} you love your son, General \textsanskrit{Viḍūḍabha}? … 

Do\marginnote{27.1} you love me?” 

“Indeed\marginnote{27.2} I do love you, \textsanskrit{Mallikā}.” 

“What\marginnote{27.3} do you think, great king? If I were to decay and perish, would sorrow, lamentation, pain, sadness, and distress arise in you?” 

“If\marginnote{27.5} you were to decay and perish, my life would fall apart. How could sorrow, lamentation, pain, sadness, and distress not arise in me?” 

“This\marginnote{27.6} is what the Buddha was referring to when he said: ‘Our loved ones are a source of sorrow, lamentation, pain, sadness, and distress.’ 

What\marginnote{28.1} do you think, great king? Do you love the realms of \textsanskrit{Kāsi} and Kosala?”\footnote{Kosala is the native realm of Pasenadi. \textsanskrit{Kāsi} had formerly been an independent kingdom, but was taken over by Pasenadi’s father \textsanskrit{Mahākosala}. Towards the end of the Buddha’s life it was contested between Kosala and Magadha (\href{https://suttacentral.net/sn3.14/en/sujato}{SN 3.14}, \href{https://suttacentral.net/sn3.15/en/sujato}{SN 3.15}). Ultimately it became part of the greater Magadhan empire. } 

“Indeed\marginnote{28.3} I do, \textsanskrit{Mallikā}. It’s due to the bounty of \textsanskrit{Kāsi} and Kosala that we use sandalwood imported from \textsanskrit{Kāsi} and wear garlands, fragrance, and makeup.” 

“What\marginnote{28.5} do you think, great king? If these realms were to decay and perish, would sorrow, lamentation, pain, sadness, and distress arise in you?” 

“If\marginnote{28.7} they were to decay and perish, my life would fall apart. How could sorrow, lamentation, pain, sadness, and distress not arise in me?” 

“This\marginnote{28.8} is what the Buddha was referring to when he said: ‘Our loved ones are a source of sorrow, lamentation, pain, sadness, and distress.’” 

“It’s\marginnote{29.1} incredible, \textsanskrit{Mallikā}, it’s amazing, how far the Buddha sees with penetrating wisdom, it seems to me. Come, \textsanskrit{Mallikā}, rinse my hands.”\footnote{I think the king held his hands over a bowl and \textsanskrit{Mallikā} poured water over them. } 

Then\marginnote{29.4} King Pasenadi got up from his seat, arranged his robe over one shoulder, raised his joined palms toward the Buddha, and expressed this heartfelt sentiment three times: 

“Homage\marginnote{29.5} to that Blessed One, the perfected one, the fully awakened Buddha! 

Homage\marginnote{29.6} to that Blessed One, the perfected one, the fully awakened Buddha! 

Homage\marginnote{29.7} to that Blessed One, the perfected one, the fully awakened Buddha!” 

%
\section*{{\suttatitleacronym MN 88}{\suttatitletranslation The Imported Cloth }{\suttatitleroot Bāhitikasutta}}
\addcontentsline{toc}{section}{\tocacronym{MN 88} \toctranslation{The Imported Cloth } \tocroot{Bāhitikasutta}}
\markboth{The Imported Cloth }{Bāhitikasutta}
\extramarks{MN 88}{MN 88}

\scevam{So\marginnote{1.1} I have heard. }At one time the Buddha was staying near \textsanskrit{Sāvatthī} in Jeta’s Grove, \textsanskrit{Anāthapiṇḍika}’s monastery. 

Then\marginnote{2.1} Venerable Ānanda robed up in the morning and, taking his bowl and robe, entered \textsanskrit{Sāvatthī} for alms. He wandered for alms in \textsanskrit{Sāvatthī}. After the meal, on his return from almsround, he went to the Eastern Monastery, the stilt longhouse of \textsanskrit{Migāra}’s mother, for the day’s meditation. 

Now\marginnote{3.1} at that time King Pasenadi of Kosala mounted the Single Lotus Elephant and drove out from \textsanskrit{Sāvatthī} in the middle of the day.\footnote{“Single lotus” (\textit{\textsanskrit{ekapuṇḍarīka}}) appears as an auspicious epithet in a prayer for wealth at \textsanskrit{Bṛhadāraṇyaka} \textsanskrit{Upaniṣad} 6.3.6. } He saw Ānanda coming off in the distance and said to the chief minister \textsanskrit{Sirivaḍḍha},\footnote{Different persons of the same name appear in \href{https://suttacentral.net/sn47.29/en/sujato}{SN 47.29} and \href{https://suttacentral.net/thag1.41/en/sujato}{Thag 1.41}. } “My dear \textsanskrit{Sirivaḍḍha}, isn’t that Venerable Ānanda?” 

“Indeed\marginnote{3.5} it is, great king.” 

Then\marginnote{4.1} King Pasenadi addressed a man, “Please, mister, go to Venerable Ānanda, and in my name bow with your head to his feet. Say to him: ‘Sir, King Pasenadi of Kosala bows with his head at your feet.’ And then say: ‘Sir, if you have no urgent business, please wait an hour out of sympathy.’” 

“Yes,\marginnote{5.1} Your Majesty,” that man replied. He did as the king asked. 

Ānanda\marginnote{6.1} consented with silence. 

Then\marginnote{6.2} King Pasenadi rode on the elephant as far as the terrain allowed, then descended and approached Ānanda on foot. He bowed, stood to one side, and said to Ānanda, “Sir, if you have no urgent business, it would be nice of you to go to the bank of the \textsanskrit{Aciravatī} river out of sympathy.” 

Ānanda\marginnote{7.1} consented with silence. 

He\marginnote{7.2} went to the river bank and sat at the root of a certain tree on a seat spread out. Then King Pasenadi rode on the elephant as far as the terrain allowed, then descended and approached Ānanda on foot. He bowed, stood to one side, and said to Ānanda, “Here, Venerable Ānanda, sit on this elephant rug.” 

“Enough,\marginnote{7.5} great king, you sit on it. I’m sitting on my own seat.” 

So\marginnote{8.1} the king sat down on the seat spread out, and said, “Honorable Ānanda, would the Buddha engage in the sort of behavior—by way of body, speech, or mind—that is faulted by ascetics and brahmins?”\footnote{The commentary says that Pasenadi’s actions were occasioned by the brutal murder of the wanderer \textsanskrit{Sundarī}. Unscrupulous ascetics urged her to visit the Jeta Grove, then killed her so that blame would fall on the Buddha and his community (\href{https://suttacentral.net/ud8.4/en/sujato}{Ud 8.4}). However, nothing in this sutta or its parallel at MA 214 supports this connection. } 

“No,\marginnote{8.4} great king, the Buddha would not engage in the sort of behavior that is faulted by sensible ascetics and brahmins.” 

“It’s\marginnote{9.1} incredible, sir, it’s amazing! For I couldn’t fully express the question, but Venerable Ānanda’s answer completed it for me.\footnote{Ānanda completed Pasenadi’s phrasing by adding “sensible” (\textit{\textsanskrit{viññūhi}}), since the Buddha’s conduct was in fact faulted by foolish ascetics and brahmins. } I don’t believe that praise or criticism of others spoken by incompetent fools, without examining or scrutinizing, is the most important thing. Rather, I believe that praise or criticism of others spoken by competent and intelligent people after examining and scrutinizing is the most important thing. 

But\marginnote{10.1} Honorable Ānanda, what kind of bodily behavior is faulted by sensible ascetics and brahmins?”\footnote{This passage describes bad actions from five perspectives, showing that the Buddha’s analysis of morality was not based on a single factor. } 

“Unskillful\marginnote{10.2} behavior.”\footnote{“Unskillful” (\textit{akusala}) is an unwholesome quality of the mind, so this is the psychological dimension of immorality. What is unskillful is that which leads to the result opposite to that desired: we want happiness, but it leads to suffering. } 

“But\marginnote{10.3} what kind of bodily behavior is unskillful?” 

“Blameworthy\marginnote{10.4} behavior.”\footnote{“Blameworthy” (\textit{\textsanskrit{sāvajja}}) behavior is that which makes one liable to criticism by others. This is the social dimension of immorality. } 

“But\marginnote{10.5} what kind of bodily behavior is blameworthy?” 

“Hurtful\marginnote{10.6} behavior.”\footnote{“Hurtful” (\textit{\textsanskrit{sabyābajjha}}) refers to the fact that bad conduct is characterized by hurtful affliction both in its action and its consequences. } 

“But\marginnote{10.7} what kind of bodily behavior is hurtful?” 

“Behavior\marginnote{10.8} that results in suffering.”\footnote{“Results in suffering” (\textit{\textsanskrit{dukkhavipāka}}) through painful kammic consequences in this life or the future. } 

“But\marginnote{10.9} what kind of bodily behavior results in suffering?” 

“Bodily\marginnote{10.10} behavior that leads to hurting yourself, hurting others, and hurting both,\footnote{Bad conduct does not just affect oneself, but reverberates in space, by harming others, and in time, by promoting the growth of even more unskillful qualities. } and which makes unskillful qualities grow while skillful qualities decline. That kind of bodily behavior is faulted by sensible ascetics and brahmins.” 

“But\marginnote{11.1} what kind of verbal behavior … mental behavior is faulted by sensible ascetics and brahmins?” … 

“Mental\marginnote{12.4} behavior that leads to hurting yourself, hurting others, and hurting both, and which makes unskillful qualities grow while skillful qualities decline. That kind of mental behavior is faulted by sensible ascetics and brahmins.” 

“Sir,\marginnote{13.1} does the Buddha praise giving up all these unskillful things?” 

“Great\marginnote{13.2} king, the Realized One has given up all unskillful things and possesses skillful things.” 

“But\marginnote{14.1} sir, what kind of bodily behavior is not faulted by sensible ascetics and brahmins?”\footnote{The analysis is inverted for good conduct. } 

“Skillful\marginnote{14.2} behavior.” 

“But\marginnote{14.3} what kind of bodily behavior is skillful?” 

“Blameless\marginnote{14.4} behavior.” 

“But\marginnote{14.5} what kind of bodily behavior is blameless?” 

“Pleasing\marginnote{14.6} behavior.” 

“But\marginnote{14.7} what kind of bodily behavior is pleasing?” 

“Behavior\marginnote{14.8} that results in happiness.” 

“But\marginnote{14.9} what kind of bodily behavior results in happiness?” 

“Bodily\marginnote{14.10} behavior that leads to pleasing yourself, pleasing others, and pleasing both, and which makes unskillful qualities decline while skillful qualities grow. That kind of bodily behavior is not faulted by sensible ascetics and brahmins.” 

“But\marginnote{15.1} what kind of verbal behavior … mental behavior is not faulted by sensible ascetics and brahmins?” … 

“Mental\marginnote{16.6} behavior that leads to pleasing yourself, pleasing others, and pleasing both, and which makes unskillful qualities decline while skillful qualities grow. That kind of mental behavior is not faulted by sensible ascetics and brahmins.” 

“Sir,\marginnote{17.1} does the Buddha praise embracing all these skillful things?” 

“Great\marginnote{17.2} king, the Realized One has given up all unskillful things and possesses skillful things.”\footnote{Ānanda emphasizes that the Buddha does not just praise these things, he embodies them fully. } 

“It’s\marginnote{18.1} incredible, sir, it’s amazing! How well this was said by Venerable Ānanda! I’m delighted and satisfied with what you’ve expressed so well. So much so that if an elephant-treasure was suitable for you, I would give you one. If a horse-treasure was suitable for you, I would give you one. If a prize village was suitable for you, I would give you one. But, sir, I know that these things are not suitable for you.\footnote{See eg. \href{https://suttacentral.net/mn94/en/sujato\#16.19}{MN 94:16.19}. } This imported cloth was sent to me by King \textsanskrit{Ajātasattu} of Magadha, son of the princess of Videha, packed in a parasol case. It’s exactly sixteen measures long and eight wide.\footnote{The Chinese parallel at MA 214 has \langlzh{繖}, which supports the reading \textit{chatta} (“parasol”). | For \textit{\textsanskrit{bāhitikā}}, I accept the commentary’s explanation that it means “foreign” or “imported” (cf. Sanskrit \textit{\textsanskrit{bāhya}}). } May Venerable Ānanda please accept it out of sympathy.” 

“Enough,\marginnote{18.12} great king. My three robes are complete.” 

“Sir,\marginnote{19.1} we have both seen this river \textsanskrit{Aciravatī} when it has rained heavily in the mountains, and the river overflows both its banks. In the same way, Venerable Ānanda can make a set of three robes for himself from this imported cloak, and you can share your old robes with your fellow monks. In this way my religious donation will come to overflow, it seems to me. Please accept the imported cloth.” 

So\marginnote{20.1} Ānanda accepted it. 

Then\marginnote{20.2} King Pasenadi said to him, “Well, now, sir, I must go. I have many duties, and much to do.” 

“Please,\marginnote{20.5} great king, go at your convenience.” Then King Pasenadi approved and agreed with what Ānanda said. He got up from his seat, bowed, and respectfully circled Ānanda, keeping him on his right, before leaving. 

Soon\marginnote{21.1} after he left, Ānanda went to the Buddha, bowed, sat down to one side, and told him what had happened. He presented the cloth to the Buddha. 

Then\marginnote{22.1} the Buddha said to the mendicants, 

“Mendicants,\marginnote{22.2} King Pasenadi is lucky, so very lucky, to get to see Ānanda and pay homage to him.” 

That\marginnote{22.4} is what the Buddha said. Satisfied, the mendicants approved what the Buddha said. 

%
\section*{{\suttatitleacronym MN 89}{\suttatitletranslation Shrines to the Teaching }{\suttatitleroot Dhammacetiyasutta}}
\addcontentsline{toc}{section}{\tocacronym{MN 89} \toctranslation{Shrines to the Teaching } \tocroot{Dhammacetiyasutta}}
\markboth{Shrines to the Teaching }{Dhammacetiyasutta}
\extramarks{MN 89}{MN 89}

\scevam{So\marginnote{1.1} I have heard. }At one time the Buddha was staying in the land of the Sakyans, near the Sakyan town named \textsanskrit{Medaḷumpa}.\footnote{This town, the spelling of which is uncertain, is not mentioned elsewhere. } 

Now\marginnote{2.1} at that time King Pasenadi of Kosala had arrived at Townsville on some business.\footnote{“Townsville” is a literal rendering of \textit{nagaraka}. I translate it in homage to the town in northern Queensland where a chance encounter with a joey set me on the path to the Dhamma. } 

Then\marginnote{2.2} he addressed \textsanskrit{Dīgha} \textsanskrit{Kārāyana}, “My good \textsanskrit{Kārāyana}, harness the finest chariots. We will go to a park and see the scenery.” 

“Yes,\marginnote{2.4} Your Majesty,” replied \textsanskrit{Dīgha} \textsanskrit{Kārāyana}. He harnessed the chariots and informed the king, “Sire, the finest chariots are harnessed. Please go at your convenience.” 

Then\marginnote{3.1} King Pasenadi mounted a fine carriage and, along with other fine carriages, set out in full royal pomp from Townsville, heading for the park grounds.\footnote{The word \textit{\textsanskrit{ārāma}}, from a root meaning “to enjoy”, originally referred to a cultivated and pleasant parkland enjoyed for recreation. Such places were favorite dwelling places for ascetics, and became the sites for developed monasteries. In this sutta we see both senses. Here Pasenadi enjoys an \textit{\textsanskrit{ārāma}} (“park”) for recreation. Below we see the Buddha staying in an \textit{\textsanskrit{ārāma}} with a community of mendicants, featuring substantial private dwellings (\textit{\textsanskrit{vihāra}}) with lockable doors; in other words, a “monastery” (\href{https://suttacentral.net/mn89/en/sujato\#8.1}{MN 89:8.1}). } He went by carriage as far as the terrain allowed, then descended and entered the park on foot. 

As\marginnote{4.1} he was going for a walk in the park he saw roots of trees that were impressive and inspiring, quiet and still, far from the madding crowd, remote from human settlements, and fit for retreat. The sight reminded him right away of the Buddha: “These roots of trees, so impressive and inspiring, are like those where we used to pay homage to the Blessed One, the perfected one, the fully awakened Buddha.”\footnote{Evidently the Buddha used to meditate on emptiness in these very woods (\href{https://suttacentral.net/mn121/en/sujato\#3.1}{MN 121:3.1}). } 

He\marginnote{4.4} addressed \textsanskrit{Dīgha} \textsanskrit{Kārāyana}, “These roots of trees, so impressive and inspiring, are like those where we used to pay homage to the Blessed One, the perfected one, the fully awakened Buddha. My good \textsanskrit{Kārāyana}, where is that Buddha at present?” 

“Great\marginnote{5.1} king, there is a Sakyan town named \textsanskrit{Medaḷumpa}. That’s where the Buddha is now staying.” 

“But\marginnote{5.3} how far away is that town?” 

“Not\marginnote{5.4} far, great king, it’s three leagues. We can get there while it’s still light.” 

“Well\marginnote{5.7} then, harness the chariots, and we shall go to see the Buddha.” 

“Yes,\marginnote{5.8} Your Majesty,” replied \textsanskrit{Dīgha} \textsanskrit{Kārāyana}. He harnessed the chariots and informed the king, “Sire, the finest chariots are harnessed. Please go at your convenience.” 

Then\marginnote{6.1} King Pasenadi mounted a fine carriage and, along with other fine carriages, set out from Townsville to \textsanskrit{Medaḷumpa}. He reached the town while it was still light and headed for the monastery grounds. He went by carriage as far as the terrain allowed, then descended and entered the monastery on foot. 

At\marginnote{7.1} that time several mendicants were walking mindfully in the open air. King Pasenadi of Kosala went up to them and said, “Sirs, where is the Blessed One at present, the perfected one, the fully awakened Buddha? For I want to see him.” 

“Great\marginnote{8.1} king, that’s his dwelling, with the door closed. Approach it quietly, without hurrying; go onto the porch, clear your throat, and knock on the door-panel. The Buddha will open the door.” The king right away presented his sword and turban to \textsanskrit{Dīgha} \textsanskrit{Kārāyana}, who thought, “Now the king seeks privacy. I should wait here.”\footnote{\textsanskrit{Kārāyana}’s response gives no hint of anything other than loyal service. } 

Then\marginnote{8.5} the king approached the Buddha’s dwelling, cleared his throat and knocked on the door-panel, and the Buddha opened the door. 

King\marginnote{9.1} Pasenadi entered the dwelling, and bowed with his head at the Buddha’s feet, caressing them and covering them with kisses, and pronounced his name:\footnote{For similar gestures, see \textsanskrit{Koṇḍañña} at \href{https://suttacentral.net/sn8.9/en/sujato\#1.2}{SN 8.9:1.2}, the brahmin \textsanskrit{Mānatthaddha} (“Stuck-up”) at \href{https://suttacentral.net/sn7.15/en/sujato\#3.2}{SN 7.15:3.2}, the brahmin \textsanskrit{Brahmāyu} at \href{https://suttacentral.net/mn91/en/sujato\#34.1}{MN 91:34.1}, and Pasenadi again at \href{https://suttacentral.net/an10.30/en/sujato\#2.2}{AN 10.30:2.2}. } “Sir, I am Pasenadi, king of Kosala! I am Pasenadi, king of Kosala!” 

“But\marginnote{9.4} great king, for what reason do you demonstrate such utmost devotion for this body, conveying your manifest love?” 

“Sir,\marginnote{10.1} I infer about the Buddha from the teaching: ‘The Blessed One is a fully awakened Buddha. The teaching is well explained. The \textsanskrit{Saṅgha} is practicing well.’\footnote{This is an abbreviated form of the full devotions to the Triple Gem. } It happens, sir, that I see some ascetics and brahmins leading the spiritual life only for a limited period: ten, twenty, thirty, or forty years. Some time later—nicely bathed and anointed, with hair and beard dressed—they amuse themselves, supplied and provided with the five kinds of sensual stimulation.\footnote{He knows this because at least some of these worked for him as spies (\href{https://suttacentral.net/sn3.11/en/sujato\#7.3}{SN 3.11:7.3}, \href{https://suttacentral.net/ud6.2/en/sujato\#8.3}{Ud 6.2:8.3}). } But here I see the mendicants leading the spiritual life entirely full and pure as long as they live, to their last breath.\footnote{While Buddhist mendicants may disrobe at any time, their intention is to practice for life. } I don’t see any other spiritual life elsewhere so full and pure. That’s why I infer this about the Buddha from the teaching: ‘The Blessed One is a fully awakened Buddha. The teaching is well explained. The \textsanskrit{Saṅgha} is practicing well.’ 

Furthermore,\marginnote{11.1} kings fight with kings, aristocrats fight with aristocrats, brahmins fight with brahmins, householders fight with householders. A mother fights with her child, child with mother, father with child, and child with father. Brother fights with brother, brother with sister, sister with brother, and friend fights with friend.\footnote{PTS text includes brother with brother. } But here I see the mendicants living in harmony, appreciating each other, without quarreling, blending like milk and water, and regarding each other with kindly eyes. I don’t see any other assembly elsewhere so harmonious. So I infer this about the Buddha from the teaching: ‘The Blessed One is a fully awakened Buddha. The teaching is well explained. The \textsanskrit{Saṅgha} is practicing well.’ 

Furthermore,\marginnote{12.1} I have walked and wandered from monastery to monastery and from park to park. There I’ve seen some ascetics and brahmins who are thin, haggard, pale, and veiny—hardly a captivating sight for people, you’d think.\footnote{This is a standard description of monastics who are unhappy and dissatisfied for whatever reason, eg. \href{https://suttacentral.net/pli-tv-bu-vb-ss1/en/sujato\#1.1.3}{Bu Ss 1:1.1.3}. } It occurred to me: ‘Clearly these venerables lead the spiritual life dissatisfied, or they’re hiding some bad deed they’ve done. That’s why they’re thin, haggard, pale, and veiny—hardly a captivating sight for people, you’d think.’ I went up to them and said: ‘Venerables, why are you so thin, haggard, pale, and veiny—hardly a captivating sight for people, you’d think?’ They say: ‘We have jaundice, great king.’\footnote{The commentary glosses \textit{kularogo} (“family sickness”), showing it must accept the reading \textit{bandhu} (“relative”). However, the parallel at MA 213 has \langlzh{白病}, which supports the variant \textit{\textsanskrit{paṇḍukarogino}} (“jaundice”). } But here I see mendicants always smiling and joyful, obviously happy, with cheerful faces, living relaxed, unruffled, surviving on charity, their hearts free as a wild deer. It occurred to me: ‘Clearly these venerables have realized a higher distinction in the Buddha’s instructions than they had before. That’s why these venerables are always smiling and joyful, obviously happy, with cheerful faces, living relaxed, unruffled, surviving on charity, their hearts free as a wild deer.’ So I infer this about the Buddha from the teaching: ‘The Blessed One is a fully awakened Buddha. The teaching is well explained. The \textsanskrit{Saṅgha} is practicing well.’ 

Furthermore,\marginnote{13.1} as an anointed aristocratic king I am able to execute, fine, or banish those who are guilty. Yet when I’m sitting in judgment they interrupt me. And I can’t get them to stop interrupting me and wait until I’ve finished speaking.\footnote{Pasenadi speaks of his frustrations in judging at \href{https://suttacentral.net/sn3.7/en/sujato}{SN 3.7}. He was getting old and passing his duties to his son. } But here I’ve seen the mendicants while the Buddha is teaching an assembly of many hundreds, and there is no sound of his disciples coughing or clearing their throats. Once it so happened that the Buddha was teaching an assembly of many hundreds. Then one of his disciples cleared their throat. And one of their spiritual companions nudged them with their knee, to indicate: ‘Hush, venerable, don’t make a sound! Our teacher, the Blessed One, is teaching!’ It occurred to me: ‘Oh, how incredible, how amazing, how an assembly can be so well trained without rod or sword!’\footnote{The future tense \textit{bhavissati} here expresses surprise or wonder, for which compare \href{https://suttacentral.net/an4.36/en/sujato\#1.6}{AN 4.36:1.6}. } I don’t see any other assembly elsewhere so well trained.\footnote{For the poor behavior of other ascetics, see \href{https://suttacentral.net/mn77/en/sujato\#6.21}{MN 77:6.21}. } So I infer this about the Buddha from the teaching: ‘The Blessed One is a fully awakened Buddha. The teaching is well explained. The \textsanskrit{Saṅgha} is practicing well.’ 

Furthermore,\marginnote{14.1} I’ve seen some clever aristocrats who are subtle, accomplished in the doctrines of others, hair-splitters. You’d think they live to demolish convictions with their intellect.\footnote{This passage on conversion of disputants is also found at \href{https://suttacentral.net/mn27/en/sujato\#4.2}{MN 27:4.2}. } They hear: ‘So, gentlemen, that ascetic Gotama will come down to such and such village or town.’ They formulate a question, thinking: ‘We’ll approach the ascetic Gotama and ask him this question. If he answers like this, we’ll refute him like that; and if he answers like that, we’ll refute him like this.’ When they hear that he has come down they approach him. The Buddha educates, encourages, fires up, and inspires them with a Dhamma talk. They don’t even get around to asking their question to the Buddha, so how could they refute his answer? Invariably, they become his disciples. So I infer this about the Buddha from the teaching: ‘The Blessed One is a fully awakened Buddha. The teaching is well explained. The \textsanskrit{Saṅgha} is practicing well.’ 

Furthermore,\marginnote{15.1} I see some clever brahmins … some clever householders … some clever ascetics who are subtle, accomplished in the doctrines of others, hair-splitters. … They don’t even get around to asking their question to the Buddha, so how could they refute his answer? Invariably, they ask the ascetic Gotama for the chance to go forth. And he gives them the going-forth. Soon after going forth, living withdrawn, diligent, keen, and resolute, they realize the supreme end of the spiritual path in this very life. They live having achieved with their own insight the goal for which gentlemen rightly go forth from the lay life to homelessness. They say: ‘We were almost lost! We almost perished! For we used to claim that we were ascetics, brahmins, and perfected ones, but we were none of these things. But now we really are ascetics, brahmins, and perfected ones!’ So I infer this about the Buddha from the teaching: ‘The Blessed One is a fully awakened Buddha. The teaching is well explained. The \textsanskrit{Saṅgha} is practicing well.’ 

Furthermore,\marginnote{18.1} these chamberlains Isidatta and \textsanskrit{Purāṇa} share my meals and my carriages. I give them a livelihood and bring them renown. And yet they don’t show me the same level of devotion that they show to the Buddha. Once it so happened that while I was leading a military campaign and testing Isidatta and \textsanskrit{Purāṇa} I took up residence in a cramped house. They spent much of the night discussing the teaching, then they lay down with their heads towards where the Buddha was and their feet towards me. It occurred to me: ‘Oh, how incredible, how amazing! These chamberlains Isidatta and \textsanskrit{Purāṇa} share my meals and my carriages. I give them a livelihood and bring them renown. And yet they don’t show me the same level of devotion that they show to the Buddha. Clearly these venerables have realized a higher distinction in the Buddha’s instructions than they had before.’\footnote{They were at least stream-enterers (\href{https://suttacentral.net/sn55.6/en/sujato\#18.1}{SN 55.6:18.1}, \href{https://suttacentral.net/an6.120-139/en/sujato\#1.12}{AN 6.120–139:1.12}), and died as once-returners (\href{https://suttacentral.net/an6.44/en/sujato\#2.3}{AN 6.44:2.3}, \href{https://suttacentral.net/an10.75/en/sujato\#2.2}{AN 10.75:2.2}). | The king refers to his subordinates with \textit{\textsanskrit{āyasmā}}, normally reserved for addressing mendicants. When used by lay people of other lay people it indicates deep respect, often, as here, in reference to noble individuals on the path (\href{https://suttacentral.net/mn68/en/sujato\#18.5}{MN 68:18.5}, \href{https://suttacentral.net/sn55.54/en/sujato\#2.2}{SN 55.54:2.2}), or else used by a criminal beseeching the mercy of the crowd (\href{https://suttacentral.net/an4.244/en/sujato\#2.2}{AN 4.244:2.2}). } So I infer this about the Buddha from the teaching: ‘The Blessed One is a fully awakened Buddha. The teaching is well explained. The \textsanskrit{Saṅgha} is practicing well.’ 

Furthermore,\marginnote{19.1} the Buddha is an aristocrat, and so am I. The Buddha is Kosalan, and so am I.\footnote{The Sakyans were vassals of Kosala (\href{https://suttacentral.net/snp3.1/en/sujato\#18.4}{Snp 3.1:18.4}). } The Buddha is eighty years old, and so am I.\footnote{This sets these events in the last year of the Buddha’s life. } Since this is so, it’s proper for me to show the Buddha such utmost devotion and demonstrate such friendship. 

Well,\marginnote{20.1} now, sir, I must go. I have many duties, and much to do.” 

“Please,\marginnote{20.3} great king, go at your convenience.” Then King Pasenadi got up from his seat, bowed, and respectfully circled the Buddha, keeping him on his right, before leaving.\footnote{While the sutta and its parallels do not hint at anything other than a respectful visit to the Buddha, there is a tradition that this event led to Pasenadi’s downfall. Apparently, \textsanskrit{Dīgha} \textsanskrit{Kārāyana} blamed Pasenadi for the death of his uncle, Pasenadi’s former general Bandhula. Seeking revenge, he took the royal insignia and crowned Pasanadi’s dubious son \textsanskrit{Viḍūḍabha} (\href{https://suttacentral.net/mn90/en/sujato\#6.1}{MN 90:6.1}) in his stead. Abandoned and alone, Pasenadi sought shelter in \textsanskrit{Rājagaha} with his nephew \textsanskrit{Ajātasattu}, but arrived too late and died while spending the night outside the walls. The details are improbable—why flee his friends the Sakyans for the untrustworthy \textsanskrit{Ajātasattu}, 500 kms distant? And why would a city watch bar a king in need? Nonetheless, the outline of this story is found widely, not just in the Pali commentary (see \href{https://suttacentral.net/ja465/en/sujato}{Ja 465}, T 1451 at T xxiv 238c1, T 211 at T iv 583a16, D (6) \textit{’dul ba}, \textit{tha} 86a7). } 

Soon\marginnote{21.1} after the king had left, the Buddha addressed the mendicants: “Mendicants, before he got up and left, King Pasenadi spoke shrines to the teaching. Learn these shrines to the teaching! Memorize these shrines to the teaching! Remember these shrines to the teaching! These shrines to the teaching are beneficial and relate to the fundamentals of the spiritual life.” 

That\marginnote{21.7} is what the Buddha said. Satisfied, the mendicants approved what the Buddha said. 

%
\section*{{\suttatitleacronym MN 90}{\suttatitletranslation At Kaṇṇakatthala }{\suttatitleroot Kaṇṇakatthalasutta}}
\addcontentsline{toc}{section}{\tocacronym{MN 90} \toctranslation{At Kaṇṇakatthala } \tocroot{Kaṇṇakatthalasutta}}
\markboth{At Kaṇṇakatthala }{Kaṇṇakatthalasutta}
\extramarks{MN 90}{MN 90}

\scevam{So\marginnote{1.1} I have heard.\footnote{The narrative of this sutta is entertainingly chaotic, with characters appearing and disappearing at random, many of whom are unknown or little-known elsewhere. This is a lifelike impression of how hard it is to maintain a rational dialogue among the comings and goings. Historically it gives a brief but telling character portrait of \textsanskrit{Viḍūḍabha}, the ill-starred prince of Kosala. } }At one time the Buddha was staying near \textsanskrit{Ujuññā}, in the deer park at \textsanskrit{Kaṇṇakatthala}.\footnote{\textsanskrit{Ujuññā} was a Kosalan town at which discussed asceticism with the naked ascetic Kassapa in \href{https://suttacentral.net/dn8/en/sujato}{DN 8}. | “Deer parks” were nature reservations where the animals were safe from hunters. } 

Now\marginnote{2.1} at that time King Pasenadi of Kosala had arrived at \textsanskrit{Ujuññā} on some business. Then he addressed a man, “Please, mister, go to the Buddha, and in my name bow with your head to his feet. Ask him if he is healthy and well, nimble, strong, and living comfortably. And then say: ‘Sir, King Pasenadi of Kosala will come to see you today when he has finished breakfast.’” 

“Yes,\marginnote{2.7} Your Majesty,” that man replied. He did as the king asked. 

The\marginnote{3.1} sisters \textsanskrit{Somā} and \textsanskrit{Sakulā} heard this.\footnote{These sisters were apparently both married to the King and, like his chief queen \textsanskrit{Mallikā}, were devoted to the Buddha. \textsanskrit{Somā} is perhaps the devout laywoman mentioned at \href{https://suttacentral.net/an8.91-117/en/sujato\#1.1}{AN 8.91–117:1.1}, but is probably not the \textsanskrit{bhikkhunī} of the same name mentioned at \href{https://suttacentral.net/sn5.2/en/sujato}{SN 5.2} and \href{https://suttacentral.net/thig3.8/en/sujato}{Thig 3.8}. \textsanskrit{Sakulā} likewise is probably not the same person as the \textsanskrit{bhikkhunī} mentioned in \href{https://suttacentral.net/thig5.7/en/sujato}{Thig 5.7} and \href{https://suttacentral.net/an1.242/en/sujato\#1.1}{AN 1.242:1.1}. } While the meal was being served, they approached the king and said, “Great king, since you are going to the Buddha, please bow in our name with your head to his feet. Ask him if he is healthy and well, nimble, strong, and living comfortably.” 

When\marginnote{4.1} he had finished breakfast, King Pasenadi went to the Buddha, bowed, sat down to one side, and said to him, “Sir, the sisters \textsanskrit{Somā} and \textsanskrit{Sakulā} bow with their heads to your feet. They ask if you are healthy and well, nimble, strong, and living comfortably.” 

“But,\marginnote{4.3} great king, couldn’t they get any other messenger?” 

So\marginnote{4.4} Pasenadi explained the circumstances of the message. The Buddha said, “May the sisters \textsanskrit{Somā} and \textsanskrit{Sakulā} be happy, great king.” 

Then\marginnote{5.1} the king said to the Buddha, “I have heard, sir, that the ascetic Gotama says this: ‘There is no ascetic or brahmin who will ever claim to be all-knowing and all-seeing, to know and see everything without exception: that is not possible.’\footnote{The Buddha was well aware of the fact that people did, in fact, make such claims, as he discussed them several times (eg. \href{https://suttacentral.net/mn76/en/sujato\#21.1}{MN 76:21.1}). He just thought the claims were wrong. One unique detail of the phrasing here is that it is in future tense (also used below at \href{https://suttacentral.net/mn90/en/sujato\#8.5}{MN 90:8.5}). The implication seems to be that not only is it the case that no-one currently makes such claims, but that no-one ever will in the future. | Compare with other statements on omniscience at \href{https://suttacentral.net/mn14/en/sujato\#17.3}{MN 14:17.3} and \href{https://suttacentral.net/mn71/en/sujato\#5.2}{MN 71:5.2}. } Do those who say this repeat what the Buddha has said, and not misrepresent him with an untruth? Is their explanation in line with the teaching? Are there any legitimate grounds for rebuttal and criticism?” 

“Great\marginnote{5.5} king, those who say this do not repeat what I have said. They misrepresent me with what is false and untrue.” 

Then\marginnote{6.1} King Pasenadi addressed General \textsanskrit{Viḍūḍabha},\footnote{The King’s son and heir, whose disastrous reign heralded the destruction of the Sakyans and the collapse of the Kosalan empire. } “General, who introduced this topic of discussion to the royal compound?” 

“It\marginnote{6.3} was \textsanskrit{Sañjaya}, great king, the brahmin of the \textsanskrit{Ākāsa} clan.”\footnote{\textit{\textsanskrit{Ākāsa}} here is the Pali form of Sanskrit \textit{\textsanskrit{āgastya}}, the clan descending from the Vedic seer Agastya. } 

Then\marginnote{7.1} the king addressed a man, “Please, mister, in my name tell \textsanskrit{Sañjaya} that King Pasenadi summons him.” 

“Yes,\marginnote{7.4} Your Majesty,” that man replied. He did as the king asked. 

Then\marginnote{8.1} the king said to the Buddha, “Sir, might the Buddha have spoken in reference to one thing, but that person believed it was something else? How then do you recall making this statement?” 

“Great\marginnote{8.4} king, I recall making this statement: ‘There is no ascetic or brahmin who will ever know all and see all simultaneously: that is not possible.’”\footnote{Consciousness is conditioned, so any instance of knowing depends on a particular stimulus. At \href{https://suttacentral.net/mn76/en/sujato\#52.5}{MN 76:52.5} the Buddha discusses the related point of whether knowledge is continuous, illustrating it with a simile of an amputee. } 

“What\marginnote{8.6} the Buddha says appears reasonable. 

Sir,\marginnote{9.1} there are these four classes: aristocrats, brahmins, peasants, and menials. Is there any difference between them?” 

“Of\marginnote{9.4} the four classes, two are said to be preeminent—the aristocrats and the brahmins. That is, when it comes to bowing down, rising up, greeting with joined palms, and observing proper etiquette.”\footnote{The Buddha acknowledges the social distinctions between the classes. } 

“Sir,\marginnote{10.1} I am not asking you about this life, but about the life to come.” 

“Great\marginnote{10.6} king, there are these five factors that support meditation. What five? It’s when a mendicant has faith in the Realized One’s awakening: ‘That Blessed One is perfected, a fully awakened Buddha, accomplished in knowledge and conduct, holy, knower of the world, supreme guide for those who wish to train, teacher of gods and humans, awakened, blessed.’ They are rarely ill or unwell. Their stomach digests well, being neither too hot nor too cold, but just right, and fit for meditation. They’re not devious or deceitful. They reveal themselves honestly to the Teacher or sensible spiritual companions. They live with energy roused up for giving up unskillful qualities and embracing skillful qualities. They’re strong, staunchly vigorous, not slacking off when it comes to developing skillful qualities. They’re wise. They have the wisdom of arising and passing away which is noble, penetrative, and leads to the complete ending of suffering. These are the five factors that support meditation. There are these four classes: aristocrats, brahmins, peasants, and menials. If they had these five factors that support meditation, that would be for their lasting welfare and happiness.” 

“Sir,\marginnote{11.1} there are these four classes: aristocrats, brahmins, peasants, and menials. If they had these five factors that support meditation, would there be any difference between them?” 

“In\marginnote{11.5} that case, I say it is the diversity of their efforts in meditation. Suppose there was a pair of elephants or horses or oxen in training who were well tamed and well trained. And there was a pair who were not tamed or trained. What do you think, great king? Wouldn’t the pair that was well tamed and well trained perform the tasks of the tamed, and reach the level of the tamed?” 

“Yes,\marginnote{11.9} sir.” 

“But\marginnote{11.10} would the pair that was not tamed and trained perform the tasks of the tamed and reach the level of the tamed, just like the tamed pair?” 

“No,\marginnote{11.11} sir.” 

“In\marginnote{11.12} the same way, there are things that must be attained by someone with faith, health, integrity, energy, and wisdom. It’s not possible for a faithless, unhealthy, deceitful, lazy, witless person to attain them.” 

“What\marginnote{12.1} the Buddha says appears reasonable. Sir, there are these four classes: aristocrats, brahmins, peasants, and workers. If they had these five factors that support meditation, and if they practiced rightly, would there be any difference between them?” 

“In\marginnote{12.6} that case, I say that there is no difference between the freedom of one and the freedom of the other. Suppose a person took dry teak wood and lit a fire and produced heat. Then another person did the same using \textsanskrit{sāl} wood, another used mango wood, while another used wood of the cluster fig. What do you think, great king? Would there be any difference between the fires produced by these different kinds of wood, that is, in the flame, color, or light?” 

“No,\marginnote{12.13} sir.” 

“In\marginnote{12.14} the same way, when fire has been churned by energy and produced by effort, I say that there is no difference between the freedom of one and the freedom of the other.”\footnote{\textit{Nimmathita} refers to the process whereby a flame is produced by “churning” with a fire-drill. See Rig Veda 3.23.1a, 3.29.12a, 6.48.5, 8.48.6a, etc. } 

“What\marginnote{13.1} the Buddha says appears reasonable. But sir, do gods survive?”\footnote{The sense of \textit{atthi \textsanskrit{devā}} (“do gods survive?”) is made clear by the following discussion. Compare \textit{atthi \textsanskrit{attā}} (“the self survives”) at \href{https://suttacentral.net/sn44.10/en/sujato\#1.3}{SN 44.10:1.3} and \href{https://suttacentral.net/mn2/en/sujato\#8.2}{MN 2:8.2} in the same sense. In such cases the verb \textit{atthi} implies continued existence in a future life. We find a similar usage in \textsanskrit{Kaṭha} \textsanskrit{Upaniṣad} 1.20, where Naciketa wonders what happens to a man after he dies, as “some say he survives, while others say he does not survive” (\textit{\textsanskrit{astīty} eke \textsanskrit{nāyam} \textsanskrit{astīti} caike}). } 

“But\marginnote{13.3} what exactly are you asking?”\footnote{Questions with \textit{atthi} (“exists”, “survives”) about the self and rebirth are routinely met indirectly. Here the Buddha asks for clarification; at \href{https://suttacentral.net/sn44.10/en/sujato\#1.3}{SN 44.10:1.3} he avoids answering; at \href{https://suttacentral.net/mn100/en/sujato\#42.5}{MN 100:42.5} he answers in an oblique fashion. The reason for this seems to be that the verb \textit{atthi} conveys the metaphysical implication of “eternal existence”. The king clarifies that he wants to know what happens when a god passes away, showing that he is asking about rebirth rather than eternal existence. } 

“Whether\marginnote{13.5} those gods come back to this place or not.” 

“Those\marginnote{13.6} gods who are afflicted come back to this place, but those who are unafflicted do not come back.”\footnote{This builds on the discussion with Pasenadi in \href{https://suttacentral.net/mn88/en/sujato\#10.6}{MN 88:10.6}, where “hurtful” or “afflicted” (\textit{\textsanskrit{sabyābajjha}}) behavior leads to suffering, elsewhere described as rebirth in a “hurtful” realm (eg. \href{https://suttacentral.net/mn57/en/sujato\#8.2}{MN 57:8.2}). } 

When\marginnote{14.1} he said this, General \textsanskrit{Viḍūḍabha} said to the Buddha, “Sir, will the gods who are afflicted topple or expel from their place the gods who are unafflicted?”\footnote{In a few words we learn a lot about how \textsanskrit{Viḍūḍabha} sees himself: he is an afflicted god. } 

Then\marginnote{14.3} Venerable Ānanda thought, “This General \textsanskrit{Viḍūḍabha} is King Pasenadi’s son, and I am the Buddha’s son. Now is the time for one son to confer with another.” So Ānanda addressed General \textsanskrit{Viḍūḍabha}, “Well then, general, I’ll ask you about this in return, and you can answer as you like. What do you think, general? As far as the dominion of King Pasenadi of Kosala extends, where he rules as sovereign lord, can he topple or expel from that place any ascetic or brahmin, regardless of whether they are good or bad, or whether or not they are genuine spiritual practitioners?” 

“He\marginnote{14.11} can, mister.”\footnote{\textit{Bho} (“mister”) is typically used by brahmins; he also uses it in reference to the king below. } 

“What\marginnote{14.12} do you think, general? As far as the dominion of King Pasenadi does not extend, where he does not rule as sovereign lord, can he topple or expel from that place any ascetic or brahmin, regardless of whether they are good or bad, or whether or not they are genuine spiritual practitioners?” 

“He\marginnote{14.14} cannot, mister.” 

“What\marginnote{14.15} do you think, general? Have you heard of the gods of the thirty-three?” 

“Yes,\marginnote{14.17} mister, I’ve heard of them, and so has the good King Pasenadi.” 

“What\marginnote{14.20} do you think, general? Can King Pasenadi topple or expel from their place the gods of the thirty-three?” 

“King\marginnote{14.22} Pasenadi can’t even see the gods of the thirty-three, so how could he possibly topple or expel them from their place?” 

“In\marginnote{14.23} the same way, general, the gods who are afflicted can’t even see the gods who are unafflicted, so how could they possibly topple or expel them from their place?”\footnote{Compare the discussion in the Jain \textsanskrit{Viyāhapaṇṇatti} 6.9.144, where pure gods are beyond the scope of impure gods. } 

Then\marginnote{15.1} the king said to the Buddha, “Sir, what is this mendicant’s name?”\footnote{If Pasenadi does not know Ānanda, this must be a very early encounter, perhaps his first. } 

“Ānanda,\marginnote{15.3} great king.” 

“A\marginnote{15.4} joy he is, and a joy he seems! What Venerable Ānanda says seems reasonable. But sir, does a divinity survive?” 

“But\marginnote{15.7} what exactly are you asking?” 

“Whether\marginnote{15.9} that Divinity comes back to this place or not.” 

“A\marginnote{15.10} divinity who is afflicted comes back to this place, but one who is unafflicted does not come back.” 

Then\marginnote{16.1} a certain man said to the king, “Great king, \textsanskrit{Sañjaya}, the brahmin of the \textsanskrit{Ākāsa} clan, has come.” 

Then\marginnote{16.3} King Pasenadi asked \textsanskrit{Sañjaya}, “Brahmin, who introduced this topic of discussion to the royal compound?” 

“It\marginnote{16.5} was General \textsanskrit{Viḍūḍabha}, great king.” 

But\marginnote{16.6} \textsanskrit{Viḍūḍabha} said, “It was \textsanskrit{Sañjaya}, great king, the brahmin of the \textsanskrit{Ākāsa} clan.”\footnote{Not only does \textsanskrit{Viḍūḍabha} exhibit poor leadership by blaming a subordinate, he is evidently lying about it. } 

Then\marginnote{17.1} a certain man said to the king, “It’s time to depart, great king.” 

So\marginnote{17.3} the king said to the Buddha, “Sir, I asked you about omniscience, and you answered. I endorse and accept this, and am satisfied with it. I asked you about purification in the four classes, about the gods, and about divinities, and you answered in each case. Whatever I asked the Buddha about, he answered. I endorse and accept this, and am satisfied with it. Well, now, sir, I must go. I have many duties, and much to do.” 

“Please,\marginnote{17.16} great king, go at your convenience.” 

Then\marginnote{18.1} King Pasenadi approved and agreed with what the Buddha said. Then he got up from his seat, bowed, and respectfully circled the Buddha, keeping him on his right, before leaving. 

%
\addtocontents{toc}{\let\protect\contentsline\protect\nopagecontentsline}
\chapter*{The Chapter on Brahmins }
\addcontentsline{toc}{chapter}{\tocchapterline{The Chapter on Brahmins }}
\addtocontents{toc}{\let\protect\contentsline\protect\oldcontentsline}

%
\section*{{\suttatitleacronym MN 91}{\suttatitletranslation With Brahmāyu }{\suttatitleroot Brahmāyusutta}}
\addcontentsline{toc}{section}{\tocacronym{MN 91} \toctranslation{With Brahmāyu } \tocroot{Brahmāyusutta}}
\markboth{With Brahmāyu }{Brahmāyusutta}
\extramarks{MN 91}{MN 91}

So\marginnote{1.1} I have heard. At one time the Buddha was wandering in the land of the Videhans together with a large \textsanskrit{Saṅgha} of five hundred mendicants.\footnote{Videha was famous as the site of philosophical conversations in the \textsanskrit{Bṛhadāraṇyaka} \textsanskrit{Upaniṣad}, especially between \textsanskrit{Yājñavalkya} and King Janaka. In my view, this sutta is constructed to show the Buddha’s conversion of a senior brahmin in \textsanskrit{Yājñavalkya}’s tradition, possibly even a direct student. } 

Now\marginnote{2.1} at that time the brahmin \textsanskrit{Brahmāyu} was residing in \textsanskrit{Mithilā}. He was old, elderly, and senior, advanced in years, having reached the final stage of life; he was a hundred and twenty years old. He had mastered the three Vedas, together with their vocabularies and ritual performance, their phonology and word classification, and the testaments as fifth. He knew them word-by-word, and their grammar. He was well versed in cosmology and the marks of a great man.\footnote{This brahmin seems to only be known from this sutta. | A few other people are said to have reached the advanced age of 120: Pasenadi’s grandmother (\href{https://suttacentral.net/sn3.22/en/sujato\#2.1}{SN 3.22:2.1}), some brahmins (\href{https://suttacentral.net/an3.51/en/sujato}{AN 3.51}, \href{https://suttacentral.net/an3.52/en/sujato}{AN 3.52}), and Dhammasavapitu who went forth at 120 (\href{https://suttacentral.net/thag1.108/en/sujato}{Thag 1.108}). | “Vocabularies” is \textit{\textsanskrit{nighaṇḍu}} (Sanskrit \textit{\textsanskrit{nighaṇṭu}}), known from the Nirukta of \textsanskrit{Yāska}. | \textit{\textsanskrit{Keṭubha}} lacks an obvious Sanskrit form. The commentary explains, “The study of proper and improper actions for the assistance of poets.” This suggests a connection with ritual performance, which is the special area of the Śatapatha \textsanskrit{Brāhmaṇa}. There we often find phrases such as \textit{\textsanskrit{kṛtam} bhavati}, “it is performed”, of which \textit{\textsanskrit{keṭubha}} is perhaps a contraction. | \textit{Akkhara} (literally “syllable”) is explained by the commentary as \textit{\textsanskrit{sikkhā}} (Sanskrit \textit{\textsanskrit{śikṣā}}), which is the study of pronunciation. This can be traced back to \textsanskrit{Pāṇinī}, and is sometimes referred to as \textit{\textsanskrit{akṣara}-\textsanskrit{samāmnāya}}, “collation of syllables”. | \textit{Pabheda} is found in Buddhist Sanskrit texts as \textit{padaprabheda}, “classification of words”, such as into the different parts of speech. The commentary identifies it with \textit{nirutti}. | \textit{Padaka} is one skilled in the \textit{\textsanskrit{padapāṭha}} recitation of Vedas, which separates the individual words. | For “testaments” (\textit{\textsanskrit{itihāsa}}) see \textit{\textsanskrit{itihāsa}-\textsanskrit{purāṇa}} in Śatapatha \textsanskrit{Brāhmaṇa} 11.5.6.8, explained by the commentator there as legends of creation and olden times. | “Cosmology” (\textit{\textsanskrit{lokāyata}}) here is a branch of Vedic learning, not the heterodox school known by this name in later times (\href{https://suttacentral.net/an9.38/en/sujato}{AN 9.38}, \href{https://suttacentral.net/sn12.48/en/sujato}{SN 12.48}). } 

He\marginnote{3.1} heard: “It seems the ascetic Gotama—a Sakyan, gone forth from a Sakyan family—is wandering in the land of the Videhans, together with a large \textsanskrit{Saṅgha} of five hundred mendicants. He has this good reputation: ‘That Blessed One is perfected, a fully awakened Buddha, accomplished in knowledge and conduct, holy, knower of the world, supreme guide for those who wish to train, teacher of gods and humans, awakened, blessed.’ He has realized with his own insight this world—with its gods, \textsanskrit{Māras}, and divinities, this population with its ascetics and brahmins, gods and humans—and he makes it known to others. He proclaims a teaching that is good in the beginning, good in the middle, and good in the end, meaningful and well-phrased. And he reveals a spiritual practice that’s entirely full and pure. It’s good to see such perfected ones.” 

Now\marginnote{4.1} at that time the brahmin \textsanskrit{Brahmāyu} had a pupil named Uttara. He too had mastered the Vedic curriculum.\footnote{Uttara is a common name and it is unclear if this is the same person as any of the other Uttaras. } \textsanskrit{Brahmāyu} told Uttara of the Buddha’s presence in the land of the Videhans, and added: “Please, dear Uttara, go to the ascetic Gotama and find out whether or not he lives up to his reputation.\footnote{Similar quests were enjoined by \textsanskrit{Pokkharasāti} on \textsanskrit{Ambaṭṭha} (\href{https://suttacentral.net/dn3/en/sujato\#dn3:1.48}{DN 3:DN 3:1.48}), and by the elderly brahmin \textsanskrit{Bāvari} on his sixteen students \href{https://suttacentral.net/snp5.1/en/sujato\#22.1}{Snp 5.1:22.1}. } Through you I shall learn about Mister Gotama.”\footnote{Following PTS and BJT editions, which read \textit{\textsanskrit{tayā}} for \textit{\textsanskrit{tathā}}. } 

“But\marginnote{5.1} sir, how shall I find out whether or not the ascetic Gotama lives up to his reputation?” 

“Dear\marginnote{5.3} Uttara, the thirty-two marks of a great man have been handed down in our hymns. A great man who possesses these has only two possible destinies, no other.\footnote{In Buddhist texts the marks are presented as the fulfillment of Brahmanical prophecy, but they are not found in any Brahmanical texts of the Buddha’s time. However, later astrological texts such as the \textsanskrit{Gārgīyajyotiṣa} (1st century BCE?) and \textsanskrit{Bṛhatsaṁhitā} (6th century CE?) contain references to many of these marks, albeit in a different context, so it seems likely the Buddhist texts are drawing on now-lost Brahmanical scriptures. | The notion of a two-fold course for a great hero—worldly success or spiritual—can be traced back as far as the epic of Gilgamesh. } If he stays at home he becomes a king, a wheel-turning monarch, a just and principled king. His dominion extends to all four sides, he achieves stability in the country, and he possesses the seven treasures.\footnote{The idea of the wheel-turning monarch draws from the Vedic horse sacrifice, which establishes the authority of a king from sea to sea. The Buddhist telling is divested of all coarse and violent elements. The wheeled chariot gave military supremacy to the ancient Indo-Europeans, allowing them to spread from their ancient homeland north of the Black Sea. In Buddhism, the wheel, which also has solar connotations, symbolizes unstoppable power. For a legendary account of such a king, see the \textsanskrit{Mahāsudassanasutta} \href{https://suttacentral.net/dn17/en/sujato}{DN 17}. } He has the following seven treasures:\footnote{Various “treasures” (\textit{ratana}) or “gems” of a king are discussed in such texts as Śatapatha \textsanskrit{Brāhmaṇa} 5.3.1. They are described in detail at \href{https://suttacentral.net/mn129/en/sujato\#34.1}{MN 129:34.1} ff. } the wheel, the elephant, the horse, the jewel, the woman, the householder, and the commander as the seventh treasure. He has over a thousand sons who are valiant and heroic, crushing the armies of his enemies.\footnote{The sacrificial horse on its journey across the land is protected by a hundred sons. } After conquering this land girt by sea, he reigns by principle, without rod or sword. But if he goes forth from the lay life to homelessness, he becomes a perfected one, a fully awakened Buddha, who draws back the veil from the world. But, dear Uttara, I am the one who gives the hymns, and you are the one who receives them.” 

“Yes,\marginnote{6.1} sir,” replied Uttara. He got up from his seat, bowed, and respectfully circled \textsanskrit{Brahmāyu} before setting out for the land of the Videhans where the Buddha was wandering. Traveling stage by stage, he came to the Buddha and exchanged greetings with him. When the greetings and polite conversation were over, he sat down to one side, and scrutinized his body for the thirty-two marks of a great man. He saw all of them except for two, which he had doubts about: whether the private parts are covered in a foreskin, and the largeness of the tongue. 

Then\marginnote{6.8} it occurred to the Buddha, “This student Uttara sees all the marks except for two, which he has doubts about: whether the private parts are covered in a foreskin, and the largeness of the tongue.” 

So\marginnote{7.1} the Buddha used his psychic power to will that Uttara would see his private parts covered in a foreskin. And he stuck out his tongue and stroked back and forth on his ear holes and nostrils, and covered his entire forehead with his tongue. 

Then\marginnote{8.1} Uttara thought, “The ascetic Gotama possesses the thirty-two marks. Why don’t I follow him and observe his deportment?” So Uttara followed the Buddha like a shadow for seven months.\footnote{In \href{https://suttacentral.net/an4.192/en/sujato}{AN 4.192} the Buddha says it is only possible to really know a person’s character by living with them for a long time. } 

When\marginnote{8.5} seven months had passed he set out wandering towards \textsanskrit{Mithilā}. There he approached the brahmin \textsanskrit{Brahmāyu}, bowed, and sat down to one side. \textsanskrit{Brahmāyu} said to him, “Well, dear Uttara, does Mister Gotama live up to his reputation or not?” 

“He\marginnote{8.9} does, sir. Mister Gotama possesses the thirty-two marks.\footnote{The marks are also listed at \href{https://suttacentral.net/dn14/en/sujato\#1.32.7}{DN 14:1.32.7} and analyzed in kammic terms at \href{https://suttacentral.net/dn30/en/sujato\#1.2.4}{DN 30:1.2.4}. Here I list the related marks in the \textsanskrit{Bṛhatsaṁhitā} as identified by Nathan McGovern (\emph{On the Origins of the 32 Marks of a Great Man}, Journal of the International Association of Buddhist Studies, 2016, vol. 39, pp. 207–247). } 

He\marginnote{9.1} has well-planted feet.\footnote{This echoes the posture of the newborn bodhisatta, and has the same meaning: that he will become awakened by “standing on his own two feet”. } 

On\marginnote{9.3} the soles of his feet there are thousand-spoked wheels, with rims and hubs, complete in every detail.\footnote{These leave marks that were seen by \textsanskrit{Doṇa} (\href{https://suttacentral.net/an4.36/en/sujato\#1.3}{AN 4.36:1.3}). They are often depicted in Buddhist art, signifying the perfection and completeness of the traces that the Buddha leaves behind in his teachings and practice. \textsanskrit{Bṛhatsaṁhitā} 69.17 lists several auspicious marks, including the wheel. } 

He\marginnote{9.4} has stretched heels.\footnote{Described as “abundantly long” at \href{https://suttacentral.net/dn30/en/sujato\#1.12.8}{DN 30:1.12.8}. } 

He\marginnote{9.5} has long fingers.\footnote{Same at \textsanskrit{Bṛhatsaṁhitā} 68.36. } 

His\marginnote{9.6} hands and feet are tender.\footnote{Tender feet at \textsanskrit{Bṛhatsaṁhitā} 68.2. } 

He\marginnote{9.7} has serried hands and feet.\footnote{\textsanskrit{Bṛhatsaṁhitā} 68.2 has \textit{\textsanskrit{śliṣtāṅgulī}} (“compact or sticky fingers”). The commentary denies that the Pali \textit{\textsanskrit{jāla}} means a physical web. I think it means that the fingers and toes were usually held together rather than splayed, hence not letting things slip through the fingers. } 

The\marginnote{9.8} tops of his feet are arched.\footnote{\textit{\textsanskrit{Ussaṅkha}} means “(curved) up like a shell”, while \textsanskrit{Bṛhatsaṁhitā} 68.2 says “curved up like a tortoise”. The descriptive verse at \href{https://suttacentral.net/dn30/en/sujato\#1.21.12}{DN 30:1.21.12} shows that it refers to the tops of the feet. } 

His\marginnote{9.9} calves are like those of an antelope.\footnote{These are presumably the long, elegant rear calves of the Indian Blackbuck. } 

When\marginnote{9.10} standing upright and not bending over, the palms of both hands touch the knees.\footnote{This agrees with \textsanskrit{Bṛhatsaṁhitā} 68.35. } 

His\marginnote{9.11} private parts are covered in a foreskin.\footnote{Same at \textsanskrit{Bṛhatsaṁhitā} 68.8. } 

He\marginnote{9.12} is golden colored; his skin shines like lustrous gold.\footnote{\textsanskrit{Bṛhatsaṁhitā} 68.102 says kings have a shining complexion. } 

He\marginnote{9.13} has delicate skin, so delicate that dust and dirt don’t stick to his body.\footnote{\textsanskrit{Bṛhatsaṁhitā} 68.102 mentions a “clean complexion” (\textit{\textsanskrit{śuddha}}). } 

His\marginnote{9.14} hairs grow one per pore.\footnote{Same at \textsanskrit{Bṛhatsaṁhitā} 68.5. } 

His\marginnote{9.15} hairs stand up; they’re blue-black and curl clockwise.\footnote{\textsanskrit{Bṛhatsaṁhitā} 68.26 says those with hairs turning right become kings. } 

His\marginnote{9.16} body is tall and straight-limbed.\footnote{Here \textit{brahm-} is an adjective from √\textit{brah} + \textit{ma}, equivalent to the Sanskrit \textit{\textsanskrit{bṛṁh}}, having the sense “grown, extended”. The Sanskrit form here is \textit{\textsanskrit{bṛhadṛjugātra}}. } 

He\marginnote{9.17} has bulging muscles in seven places.\footnote{Hands, feet, shoulders, and chest (\href{https://suttacentral.net/dn30/en/sujato\#1.13.5}{DN 30:1.13.5}). } 

His\marginnote{9.18} chest is like that of a lion.\footnote{\textsanskrit{Bṛhatsaṁhitā} 68.18 compares not the chest but the hips with a lion. } 

He\marginnote{9.19} is filled out between the shoulders.\footnote{\textsanskrit{Bṛhatsaṁhitā} 68.27 says the heart is raised and muscular. } 

He\marginnote{9.20} has the proportional circumference of a banyan tree: the span of his arms equals the height of his body.\footnote{\textsanskrit{Bṛhatsaṁhitā} 69.13 has the same proportions without the simile. These are the normal human proportions, yet we cannot touch our knees without bending. The only way these marks could be reconciled is if the arms were extra long and the length of the legs below the knees was extra long as well. And this is exactly what we are told: the ankles are stretched and long, and the calves are like those of an antelope, whose rear calves are long proportionate to the thigh. Thus in this regard the marks appear to be internally consistent, though not describing normal human anatomy. } 

His\marginnote{9.21} torso is cylindrical. 

He\marginnote{9.22} has ridged taste buds.\footnote{“Ridged taste buds” is \textit{\textsanskrit{rasaggasaggī}}. \textit{Rasa} can mean either “taste” or “nutrition”, but the use of \textit{\textsanskrit{ojā}} in \href{https://suttacentral.net/dn30/en/sujato\#2.9.8}{DN 30:2.9.8} confirms the latter. \textit{Gasa} is “swallow” and per \href{https://suttacentral.net/dn30/en/sujato\#2.7.4}{DN 30:2.7.4} it is the “conveyance of savor” (\textit{\textsanskrit{rasaharaṇīyo}}). \textit{Agga} often means “best”, but this is derived from the primary sense of “peak”. The descriptors \textit{uddhagga} (“raised”) at \href{https://suttacentral.net/dn30/en/sujato\#2.7.4}{DN 30:2.7.4} and \textit{\textsanskrit{susaṇṭhitā}} (“prominent”) at \href{https://suttacentral.net/dn30/en/sujato\#2.9.8}{DN 30:2.9.8} confirm that the latter is meant. The mark refers to taste buds raised in noticeable ridges on the tongue that absorb nutrition and aid digestion. It has often been interpreted as “excellent (\textit{\textsanskrit{aggī}}) sense (\textit{gasa}) of taste (\textit{rasa})”, but this, being imperceptible to others, is rather a secondary consequence of the mark. } 

His\marginnote{9.23} jaw is like that of a lion. 

He\marginnote{9.24} has forty teeth. 

His\marginnote{9.25} teeth are even.\footnote{Even, gapless, and white teeth are at \textsanskrit{Bṛhatsaṁhitā} 68.52. } 

His\marginnote{9.26} teeth have no gaps. 

His\marginnote{9.27} teeth are perfectly white. 

He\marginnote{9.28} has a large tongue.\footnote{Same at \textsanskrit{Bṛhatsaṁhitā} 68.53. } 

He\marginnote{9.29} has the voice of the Divinity, like a cuckoo’s call. 

His\marginnote{9.30} eyes are indigo.\footnote{At \href{https://suttacentral.net/thig13.1/en/sujato\#6.2}{Thig 13.1:6.2} \textsanskrit{Ambapālī} describes her eyes as \textit{\textsanskrit{abhinīla}}. While some Indians do indeed have blue eyes, this probably describes a black so deep it appears blue. } 

He\marginnote{9.31} has eyelashes like a cow’s.\footnote{Cows have long and elegant eyelashes. } 

Between\marginnote{9.32} his eyebrows there grows a tuft, soft and white like cotton-wool. 

The\marginnote{9.33} crown of his head is like a turban.\footnote{The \textit{\textsanskrit{uṇhīsa}} is depicted as a bulge on the Buddha’s crown. } 

These\marginnote{9.34} are the thirty-two marks of a great man possessed by Mister Gotama. 

When\marginnote{10.1} he’s walking he takes the first step with the right foot.\footnote{From here Uttara describes the Buddha’s conduct, showing his keen observation of details. } He doesn’t lift his foot too far or place it too near. He doesn’t walk too slow or too fast. He walks without knocking his knees or ankles together.\footnote{\textit{Adduva} means “knee” according to the commentary. } When he’s walking he keeps his thighs neither too straight nor too bent, neither too tight nor too loose. When he walks, only the lower half of his body moves, and he walks effortlessly. When he turns to look he does so with the whole body.\footnote{This is the so-called “elephant look” (\href{https://suttacentral.net/dn16/en/sujato\#4.1.2}{DN 16:4.1.2}). } He doesn’t look directly up or down. He doesn’t look all around while walking, but focuses a plough’s length in front. Beyond that he has unhindered knowledge and vision.\footnote{As at \href{https://suttacentral.net/snp3.1/en/sujato\#6.4}{Snp 3.1:6.4}. } 

When\marginnote{11.1} entering an inhabited area he keeps his body neither too straight nor too bent, neither too tight nor too loose. 

He\marginnote{12.1} turns around neither too far nor too close to the seat. He doesn’t lean on his hand when sitting down. And he doesn’t just plonk his body down on the seat. When sitting in inhabited areas he doesn’t fidget with his hands or feet. He doesn’t sit with his knees or ankles crossed. He doesn’t sit with his hand holding his chin. When sitting in inhabited areas he doesn’t shake, tremble, quake, or get nervous, and so he is not nervous at all. When sitting in inhabited areas he still practices seclusion. 

When\marginnote{13.1} receiving water for rinsing the bowl, he holds the bowl neither too straight nor too bent, neither too tight nor too loose. 

He\marginnote{14.1} receives neither too little nor too much water. He rinses the bowl without making a sloshing noise, or spinning it around. He doesn’t put the bowl on the ground to rinse his hands; his hands and bowl are rinsed at the same time. He doesn’t throw the bowl rinsing water away too far or too near, or splash it about. When receiving rice, he holds the bowl neither too straight nor too bent, neither too close nor too loose. He receives neither too little nor too much rice. He eats sauce in a moderate proportion, and doesn’t put too much sauce on his portions. He chews over each portion two or three times before swallowing. But no grain of rice enters his body unchewed, and none remain in his mouth. Only then does he raise another portion to his lips. He eats experiencing the taste, but without experiencing greed for the taste. 

He\marginnote{14.11} eats food thinking of eight reasons: ‘Not for fun, indulgence, adornment, or decoration, but only to sustain this body, to avoid harm, and to support spiritual practice. In this way, I shall put an end to old discomfort and not give rise to new discomfort, and I will have the means to keep going, blamelessness, and a comfortable abiding.’\footnote{This is the traditional reflection on the meal (\href{https://suttacentral.net/mn2/en/sujato\#14.2}{MN 2:14.2}). } 

After\marginnote{15.1} eating, when receiving water for washing the bowl, he holds the bowl neither too straight nor too bent, neither too tight nor too loose. He receives neither too little nor too much water. He washes the bowl without making a sloshing noise, or spinning it around. He doesn’t put the bowl on the ground to wash his hands; his hands and bowl are washed at the same time. He doesn’t throw the bowl washing water away too far or too near, or splash it about. 

After\marginnote{16.1} eating he doesn’t put the bowl on the ground too far away or too close. He’s not careless with his bowl, nor does he spend too much time on it. 

After\marginnote{17.1} eating he sits a while in silence, but doesn’t wait too long to give the verses of appreciation. After eating he expresses appreciation without criticizing the meal or expecting another one.\footnote{This is the \textit{\textsanskrit{anumodanā}} recited for the meal offering. Examples are found at \href{https://suttacentral.net/dn16/en/sujato\#1.31.1}{DN 16:1.31.1}, \href{https://suttacentral.net/snp3.7/en/sujato\#34.3}{Snp 3.7:34.3} = \href{https://suttacentral.net/mn92/en/sujato\#25.6}{MN 92:25.6}, \href{https://suttacentral.net/sn55.26/en/sujato\#20.1}{SN 55.26:20.1}, \href{https://suttacentral.net/pli-tv-kd1/en/sujato\#15.14.4}{Kd 1:15.14.4}, and \href{https://suttacentral.net/pli-tv-kd1/en/sujato\#1.5.1}{Kd 1:1.5.1}. | It is worth noting that no early \textit{\textsanskrit{anomodanā}} uses the imperative verb form \textit{-tu} signifying giving a blessing (eg. \textit{bhavatu \textsanskrit{sabbamaṅgalaṁ}}, “may all blessings be”). They strictly use the indicative \textit{-ti} to teach cause and effect: if you do this, that happens. } Invariably, he educates, encourages, fires up, and inspires that assembly with a Dhamma talk. Then he gets up from his seat and leaves. 

He\marginnote{18.1} walks neither too fast nor too slow, without wanting to get out of there. 

He\marginnote{19.1} wears his robe on his body neither too high nor too low, neither too clinging nor too loose. The wind doesn’t blow his robe off his body. And dust and dirt don’t stick to his body. 

When\marginnote{20.1} he has gone to the monastery he sits on a seat spread out and washes his feet. But he doesn’t waste time with pedicures. When he has washed his feet, he sits down cross-legged, sets his body straight, and establishes mindfulness in his presence. He has no intention to hurt himself, hurt others, or hurt both. He only wishes for the welfare of himself, of others, of both, and of the whole world. In the monastery when he teaches Dhamma to an assembly, he neither flatters them nor rebukes them. Invariably, he educates, encourages, fires up, and inspires that assembly with a Dhamma talk. 

The\marginnote{21.3} sound of his voice has eight qualities: it is clear, comprehensible, charming, audible, lucid, undistorted, deep, and resonant. He makes sure his voice is intelligible as far as the assembly goes, but the sound doesn’t extend outside the assembly. And when they’ve been inspired with a Dhamma talk by Mister Gotama they get up from their seats and leave looking back at him alone, and not forgetting their lesson. 

I\marginnote{22.1} have seen Mister Gotama walking and standing; entering inhabited areas, and sitting and eating there; sitting silently after eating, and expressing appreciation; going to the monastery, sitting silently there, and teaching Dhamma to an assembly there. Such is Mister Gotama; such he is and more than that.” 

When\marginnote{23.1} he had spoken, the brahmin \textsanskrit{Brahmāyu} got up from his seat, arranged his robe over one shoulder, raised his joined palms toward the Buddha, and uttered this aphorism three times: 

“Homage\marginnote{23.2} to that Blessed One, the perfected one, the fully awakened Buddha! 

Homage\marginnote{23.3} to that Blessed One, the perfected one, the fully awakened Buddha! 

Homage\marginnote{23.4} to that Blessed One, the perfected one, the fully awakened Buddha! 

Hopefully,\marginnote{23.5} some time or other I’ll get to meet him, and we can have a discussion.” 

And\marginnote{24.1} then the Buddha, traveling stage by stage in the Videhan lands, arrived at \textsanskrit{Mithilā}, where he stayed in the Maghadeva Mango Grove.\footnote{The story of Maghadeva is told in \href{https://suttacentral.net/mn83/en/sujato}{MN 83}. } 

The\marginnote{24.3} brahmins and householders of \textsanskrit{Mithilā} heard: “It seems the ascetic Gotama—a Sakyan, gone forth from a Sakyan family—has arrived at \textsanskrit{Mithilā}, where he is staying in the Maghadeva Mango Grove. He has this good reputation: ‘That Blessed One is perfected, a fully awakened Buddha, accomplished in knowledge and conduct, holy, knower of the world, supreme guide for those who wish to train, teacher of gods and humans, awakened, blessed.’ He has realized with his own insight this world—with its gods, \textsanskrit{Māras}, and divinities, this population with its ascetics and brahmins, gods and humans—and he makes it known to others. He proclaims a teaching that is good in the beginning, good in the middle, and good in the end, meaningful and well-phrased. And he reveals a spiritual practice that’s entirely full and pure. It’s good to see such perfected ones.” 

Then\marginnote{25.1} the brahmins and householders of \textsanskrit{Mithilā} went up to the Buddha. Before sitting down to one side, some bowed, some exchanged greetings and polite conversation, some held up their joined palms toward the Buddha, some announced their name and clan, while some kept silent. 

The\marginnote{26.1} brahmin \textsanskrit{Brahmāyu} also heard that the Buddha had arrived. So he went to the Maghadeva Mango Grove together with several disciples. 

Not\marginnote{26.3} far from the grove he thought, “It wouldn’t be appropriate for me to go to see the ascetic Gotama without first letting him know.”\footnote{This shows his deep respect and politeness. } 

So\marginnote{26.5} he addressed one of his young students: “Here, young student, go to the ascetic Gotama and in my name bow with your head to his feet. Ask him if he is healthy and well, nimble, strong, and living comfortably. And then say: ‘Mister Gotama, the brahmin \textsanskrit{Brahmāyu} is old, elderly, and senior, advanced in years, having reached the final stage of life; he is a hundred and twenty years old. He has mastered the three Vedas, together with their vocabularies and ritual performance, their phonology and word classification, and the testaments as fifth. He knows them word-by-word, and their grammar. He is well versed in cosmology and the marks of a great man. Of all the brahmins and householders residing in \textsanskrit{Mithilā}, \textsanskrit{Brahmāyu} is said to be the foremost in wealth, hymns, lifespan, and fame. He wants to see Mister Gotama.’” 

“Yes,\marginnote{26.17} sir,” that young student replied. He did as he was asked, and the Buddha said, “Please, student, let \textsanskrit{Brahmāyu} come when he’s ready.” 

The\marginnote{27.1} young student went back to \textsanskrit{Brahmāyu} and said to him, “Your request for an audience with the ascetic Gotama has been granted. Please go at your convenience.” 

Then\marginnote{28.1} the brahmin \textsanskrit{Brahmāyu} went up to the Buddha. The assembly saw him coming off in the distance, and made way for him, as he was well-known and famous. 

\textsanskrit{Brahmāyu}\marginnote{28.4} said to that retinue, “Enough, gentlemen. Please sit on your own seats. I shall sit here by the ascetic Gotama.” 

Then\marginnote{29.1} the brahmin \textsanskrit{Brahmāyu} went up to the Buddha, and exchanged greetings with him. When the greetings and polite conversation were over, he sat down to one side, and scrutinized the Buddha’s body for the thirty-two marks of a great man. He saw all of them except for two, which he had doubts about: whether the private parts are covered in a foreskin, and the largeness of the tongue. Then \textsanskrit{Brahmāyu} addressed the Buddha in verse: 

\begin{verse}%
“I\marginnote{29.8} have learned of the thirty-two \\
marks of a great man. \\
There are two that I don’t see \\
on the body of Mister Gotama. 

Are\marginnote{29.12} the private parts covered in a foreskin, \\
O supreme person? \\
Though called by a word of the feminine gender,\footnote{“Tongue” in Pali is the feminine \textit{\textsanskrit{jivhā}}. } \\
perhaps your tongue is a manly one?\footnote{Read \textit{\textsanskrit{narassikā}}. } 

Perhaps\marginnote{29.16} your tongue is large, \\
as we have been informed. \\
Please stick it out in its full extent, \\
and so, O seer, dispel my doubt. 

For\marginnote{29.20} my welfare and benefit in this life, \\
and happiness in the next. \\
And I ask you to grant the opportunity \\
to ask whatever I desire.” 

%
\end{verse}

Then\marginnote{30.1} the Buddha thought, “\textsanskrit{Brahmāyu} sees all the marks except for two, which he has doubts about: whether the private parts are covered in a foreskin, and the largeness of the tongue.” 

So\marginnote{30.5} the Buddha used his psychic power to will that Divinityyu would see his private parts covered in a foreskin. And he stuck out his tongue and stroked back and forth on his ear holes and nostrils, and covered his entire forehead with his tongue. 

Then\marginnote{30.7} the Buddha replied to \textsanskrit{Brahmāyu} in verse: 

\begin{verse}%
“The\marginnote{31.1} thirty-two marks of a great man \\
that you have learned \\
are all found on my body: \\
so do not doubt, brahmin. 

I\marginnote{31.5} have known what should be known,\footnote{This four-line verse encompasses the four noble truths with their duties: suffering is to be known; the origin of suffering is to be given up; the path is to be developed. The third noble truth, the witnessing of Nibbana, is implied in the final line. } \\
and developed what should be developed, \\
and given up what should be given up: \\
and so, brahmin, I am a Buddha. 

For\marginnote{31.9} your welfare and benefit in this life, \\
and happiness in the next: \\
I grant you the opportunity \\
to ask whatever you desire.” 

%
\end{verse}

Then\marginnote{32.1} \textsanskrit{Brahmāyu} thought: 

“My\marginnote{32.2} request has been granted. Should I ask him about what is beneficial in this life or the next?” Then he thought, “I’m well versed in the benefits that apply to this life, and others ask me about this. Why don’t I ask the ascetic Gotama about the benefit that specifically applies to lives to come?” 

So\marginnote{32.9} \textsanskrit{Brahmāyu} addressed the Buddha in verse: 

\begin{verse}%
“How\marginnote{32.10} do you become a brahmin?\footnote{For some reason, the Buddha does not actually answer these questions, except the final two on the sage and the awakened one. Most of the others are technical terms in Brahmanism, and they are elsewhere defined by the Buddha in his own terms (eg. \href{https://suttacentral.net/mn39/en/sujato\#23.1}{MN 39:23.1} ff.). The Chinese parallels at MA 161 and T 76 have fewer items than the Pali, but do a better job of matching the answers to the questions. } \\
And how do you become a knowledge master? \\
How a master of the three knowledges? \\
And how is one called a scholar? 

How\marginnote{32.14} do you become a perfected one? \\
And how a consummate one? \\
How do you become a sage? \\
And how is one declared to be awakened?” 

%
\end{verse}

Then\marginnote{33.1} the Buddha replied to \textsanskrit{Brahmāyu} in verse: 

\begin{verse}%
“One\marginnote{33.2} who knows their past lives, \\
sees heaven and places of loss, \\
and has attained the end of rebirth: \\
such a sage has perfect insight. 

They\marginnote{33.6} know their mind is pure, \\
completely freed from greed; \\
they’ve given up birth and death, \\
and have completed the spiritual journey. \\
Gone beyond all things, \\
such a one is declared to be awakened.” 

%
\end{verse}

When\marginnote{34.1} he said this, \textsanskrit{Brahmāyu} got up from his seat and arranged his robe on one shoulder. He bowed with his head at the Buddha’s feet, caressing them and covering them with kisses, and pronounced his name: “I am the brahmin \textsanskrit{Brahmāyu}, Mister Gotama! I am the brahmin \textsanskrit{Brahmāyu}!” 

Then\marginnote{35.1} that assembly, their minds full of wonder and amazement, thought, “Oh, how incredible, how amazing, that Divinityyu, who is so well-known and famous, should show the Buddha such utmost devotion.” Then the Buddha said to \textsanskrit{Brahmāyu}, “Enough, brahmin. Get up, and sit in your own seat, since your mind has such confidence in me.” So \textsanskrit{Brahmāyu} got up and sat in his own seat. 

Then\marginnote{36.1} the Buddha taught him step by step, with a talk on giving, ethical conduct, and heaven. He explained the drawbacks of sensual pleasures, so sordid and corrupt, and the benefit of renunciation. And when the Buddha knew that Divinityyu’s mind was ready, pliable, rid of hindrances, elated, and confident he explained the special teaching of the Buddhas: suffering, its origin, its cessation, and the path. Just as a clean cloth rid of stains would properly absorb dye, in that very seat the stainless, immaculate vision of the Dhamma arose in the brahmin \textsanskrit{Brahmāyu}: “Everything that has a beginning has an end.” 

Then\marginnote{36.9} \textsanskrit{Brahmāyu} saw, attained, understood, and fathomed the Dhamma. He went beyond doubt, got rid of indecision, and became self-assured and independent of others regarding the Teacher’s instructions. He said to the Buddha: 

“Excellent,\marginnote{37.1} Mister Gotama! Excellent! As if he were righting the overturned, or revealing the hidden, or pointing out the path to the lost, or lighting a lamp in the dark so people with clear eyes can see what’s there, Mister Gotama has made the teaching clear in many ways. I go for refuge to Mister Gotama, to the teaching, and to the mendicant \textsanskrit{Saṅgha}. From this day forth, may Mister Gotama remember me as a lay follower who has gone for refuge for life. Would you and the mendicant \textsanskrit{Saṅgha} please accept a meal from me tomorrow?” The Buddha consented with silence. Then, knowing that the Buddha had consented, \textsanskrit{Brahmāyu} got up from his seat, bowed, and respectfully circled the Buddha, keeping him on his right, before leaving. 

And\marginnote{38.1} when the night had passed \textsanskrit{Brahmāyu} had delicious fresh and cooked foods prepared in his own home. Then he had the Buddha informed of the time, saying, “It’s time, Mister Gotama, the meal is ready.” 

Then\marginnote{38.3} the Buddha robed up in the morning and, taking his bowl and robe, went to the home of the brahmin \textsanskrit{Brahmāyu}, where he sat on the seat spread out, together with the \textsanskrit{Saṅgha} of mendicants. For seven days, \textsanskrit{Brahmāyu} served and satisfied the mendicant \textsanskrit{Saṅgha} headed by the Buddha with his own hands with delicious fresh and cooked foods. 

When\marginnote{39.1} the seven days had passed, the Buddha departed to wander in the Videhan lands. Not long after the Buddha left, \textsanskrit{Brahmāyu} passed away. 

Then\marginnote{39.3} several mendicants went up to the Buddha, bowed, sat down to one side, and said to him, “Sir, \textsanskrit{Brahmāyu} has passed away. Where has he been reborn in his next life?” 

“Mendicants,\marginnote{39.6} the brahmin \textsanskrit{Brahmāyu} was astute. He practiced in line with the teachings, and did not trouble me about the teachings. With the ending of the five lower fetters, he’s been reborn spontaneously and will become extinguished there, not liable to return from that world.” 

That\marginnote{39.8} is what the Buddha said. Satisfied, the mendicants approved what the Buddha said. 

%
\section*{{\suttatitleacronym MN 92}{\suttatitletranslation With Sela }{\suttatitleroot Selasutta}}
\addcontentsline{toc}{section}{\tocacronym{MN 92} \toctranslation{With Sela } \tocroot{Selasutta}}
\markboth{With Sela }{Selasutta}
\extramarks{MN 92}{MN 92}

\scevam{So\marginnote{1.1} I have heard.\footnote{This entire sutta is repeated at \href{https://suttacentral.net/snp3.7/en/sujato}{Snp 3.7}. } }At one time the Buddha was wandering together with a large \textsanskrit{Saṅgha} of 1,250 mendicants in the land of the \textsanskrit{Aṅguttarāpans} when he arrived at a town of theirs named \textsanskrit{Āpaṇa}. 

The\marginnote{2.1} matted-hair ascetic \textsanskrit{Keṇiya} heard: “It seems the ascetic Gotama—a Sakyan, gone forth from a Sakyan family—has arrived at \textsanskrit{Āpaṇa}, together with a large \textsanskrit{Saṅgha} of 1,250 mendicants. He has this good reputation: ‘That Blessed One is perfected, a fully awakened Buddha, accomplished in knowledge and conduct, holy, knower of the world, supreme guide for those who wish to train, teacher of gods and humans, awakened, blessed.’ He has realized with his own insight this world—with its gods, \textsanskrit{Māras}, and divinities, this population with its ascetics and brahmins, gods and humans—and he makes it known to others. He proclaims a teaching that is good in the beginning, good in the middle, and good in the end, meaningful and well-phrased. And he reveals a spiritual practice that’s entirely full and pure. It’s good to see such perfected ones.” 

So\marginnote{3.1} \textsanskrit{Keṇiya} approached the Buddha and exchanged greetings with him.\footnote{\textsanskrit{Keṇiya} features also in the Vinaya, where at this point he offered the Buddha and the Sangha a drink in the evening, reasoning that the Brahmanical sages of the past did likewise (\href{https://suttacentral.net/pli-tv-kd6/en/sujato\#35.1.2}{Kd 6:35.1.2}). As a result, the Buddha allowed fruit juice and other drinks in the evening. } When the greetings and polite conversation were over, he sat down to one side. The Buddha educated, encouraged, fired up, and inspired him with a Dhamma talk. 

Then\marginnote{3.4} he said to the Buddha, “Would Mister Gotama together with the mendicant \textsanskrit{Saṅgha} please accept tomorrow’s meal from me?” 

When\marginnote{3.6} he said this, the Buddha said to him, “The \textsanskrit{Saṅgha} is large, \textsanskrit{Keṇiya}; there are 1,250 mendicants. And you are devoted to the brahmins.” 

For\marginnote{3.8} a second time, \textsanskrit{Keṇiya} asked the Buddha to accept a meal offering. “Never mind that the \textsanskrit{Saṅgha} is large, with 1,250 mendicants, and that I am devoted to the brahmins. Would Mister Gotama together with the mendicant \textsanskrit{Saṅgha} please accept tomorrow’s meal from me?” And for a second time, the Buddha gave the same reply. For a third time, \textsanskrit{Keṇiya} asked the Buddha to accept a meal offering. “Never mind that the \textsanskrit{Saṅgha} is large, with 1,250 mendicants, and that I am devoted to the brahmins. Would Mister Gotama together with the mendicant \textsanskrit{Saṅgha} please accept tomorrow’s meal from me?” The Buddha consented with silence. 

Then,\marginnote{4.1} knowing that the Buddha had consented, \textsanskrit{Keṇiya} got up from his seat and went to his own hermitage. There he addressed his friends and colleagues, relatives and kin, “My friends and colleagues, relatives and kin: please listen! The ascetic Gotama together with the mendicant \textsanskrit{Saṅgha} has been invited by me for tomorrow’s meal. Please help me out with the manual preparations.”\footnote{“Manual preparations” is \textit{\textsanskrit{kāyaveyyāvaṭikaṁ}}. Ten kinds of service are enumerated in the Jain \textsanskrit{Tattvārthasūtra} 9.24. } 

“Yes,\marginnote{4.5} sir,” they replied. Some dug ovens, some chopped wood, some washed dishes, some set out a water jar, and some spread out seats. Meanwhile, \textsanskrit{Keṇiya} set up the pavilion himself. 

Now\marginnote{5.1} at that time the brahmin Sela was residing in \textsanskrit{Āpaṇa}. He had mastered the three Vedas, together with their vocabularies and ritual performance, their phonology and word classification, and the testaments as fifth. He knew them word-by-word, and their grammar. He was well versed in cosmology and the marks of a great man. And he was teaching three hundred young students to recite the hymns. 

And\marginnote{6.1} at that time \textsanskrit{Keṇiya} was devoted to Sela. Then Sela, while going for a walk escorted by the three hundred young students, approached \textsanskrit{Keṇiya}’s hermitage. He saw the preparations going on, and said to \textsanskrit{Keṇiya}, “\textsanskrit{Keṇiya}, is your son or daughter being married? Or are you setting up a big sacrifice? Or has King Seniya \textsanskrit{Bimbisāra} of Magadha been invited for tomorrow’s meal?”\footnote{\textsanskrit{Aṅga} was subject to Magadha at that time. | This passage is partially echoed in the account of \textsanskrit{Anāthapiṇḍika}’s conversion (\href{https://suttacentral.net/pli-tv-kd16/en/sujato\#4.1.10}{Kd 16:4.1.10}). } 

“There\marginnote{8.1} is no marriage, Sela, and the king is not coming. Rather, I am setting up a big sacrifice. The ascetic Gotama has arrived at \textsanskrit{Āpaṇa}, together with a large \textsanskrit{Saṅgha} of 1,250 mendicants. He has this good reputation: ‘That Blessed One is perfected, a fully awakened Buddha, accomplished in knowledge and conduct, holy, knower of the world, supreme guide for those who wish to train, teacher of gods and humans, awakened, blessed.’ He has been invited by me for tomorrow’s meal together with the mendicant \textsanskrit{Saṅgha}.” 

“Mister\marginnote{9.1} \textsanskrit{Keṇiya}, did you say ‘the awakened one’?”\footnote{\textsanskrit{Anāthapiṇḍika} responded in the same way when hearing of the Buddha (\href{https://suttacentral.net/pli-tv-kd16/en/sujato\#4.2.10}{Kd 16:4.2.10}). Also compare \textsanskrit{Bāvari}’s elation at \href{https://suttacentral.net/snp5.1/en/sujato\#19.1}{Snp 5.1:19.1}. } 

“I\marginnote{9.2} said ‘the awakened one’.” 

“Did\marginnote{9.3} you say ‘the awakened one’?” 

“I\marginnote{9.4} said ‘the awakened one’.” 

Then\marginnote{10.1} Sela thought, “It’s hard to even find the word ‘awakened one’ in the world. The thirty-two marks of a great man have been handed down in our hymns. A great man who possesses these has only two possible destinies, no other. If he stays at home he becomes a king, a wheel-turning monarch, a just and principled king. His dominion extends to all four sides, he achieves stability in the country, and he possesses the seven treasures. He has the following seven treasures: the wheel, the elephant, the horse, the jewel, the woman, the householder, and the commander as the seventh treasure. He has over a thousand sons who are valiant and heroic, crushing the armies of his enemies. After conquering this land girt by sea, he reigns by principle, without rod or sword. But if he goes forth from the lay life to homelessness, he becomes a perfected one, a fully awakened Buddha, who draws back the veil from the world.” 

“But\marginnote{11.1} \textsanskrit{Keṇiya}, where is the Blessed One at present, the perfected one, the fully awakened Buddha?” 

When\marginnote{11.2} he said this, \textsanskrit{Keṇiya} pointed with his right arm and said, “There, Mister Sela, at that line of blue forest.” 

Then\marginnote{12.1} Sela, together with his young students, approached the Buddha. He said to his young students, “Come quietly, gentlemen, tread gently. For the Buddhas are intimidating, like a lion living alone.\footnote{\textit{\textsanskrit{Durāsada}} (“intimidating”) is also at \href{https://suttacentral.net/an4.42/en/sujato\#4.1}{AN 4.42:4.1}, which shows that the sense is that they are hard to defeat in debate. } When I’m consulting with the ascetic Gotama, don’t interrupt. Wait until I’ve finished speaking.” 

Then\marginnote{13.1} Sela went up to the Buddha, and exchanged greetings with him. When the greetings and polite conversation were over, he sat down to one side, and scrutinized the Buddha’s body for the thirty-two marks of a great man. 

He\marginnote{13.4} saw all of them except for two, which he had doubts about: whether the private parts are covered in a foreskin, and the largeness of the tongue. 

Then\marginnote{13.7} it occurred to the Buddha, “Sela sees all the marks except for two, which he has doubts about: whether the private parts are covered in a foreskin, and the largeness of the tongue.” 

The\marginnote{14.1} Buddha used his psychic power to will that Sela would see his private parts covered in a foreskin. And he stuck out his tongue and stroked back and forth on his ear holes and nostrils, and covered his entire forehead with his tongue. 

Then\marginnote{15.1} Sela thought, “The ascetic Gotama possesses the thirty-two marks completely, lacking none. But I don’t know whether or not he is an awakened one.\footnote{As his verses make clear, Sela has not ruled out the possibility that Gotama’s destiny is to become a wheel-turning monarch rather than a Buddha. } I have heard that brahmins of the past who were elderly and senior, the tutors of tutors, said, ‘Those who are perfected ones, fully awakened Buddhas reveal themselves when praised.’\footnote{This idea is not found elsewhere in the Pali. } Why don’t I extoll him in his presence with fitting verses?” 

Then\marginnote{15.7} Sela extolled the Buddha in his presence with fitting verses: 

\begin{verse}%
“O\marginnote{16.1} Blessed One, your body’s perfect,\footnote{These are also found in Sela’s verses at \href{https://suttacentral.net/thag16.6/en/sujato}{Thag 16.6}. } \\
you’re radiant, handsome, lovely to behold; \\
golden colored, \\
with teeth so white; you’re strong. 

The\marginnote{16.5} characteristics \\
of a handsome man, \\
the marks of a great man, \\
are all found on your body. 

Your\marginnote{16.9} eyes are clear, your face is fair, \\
you’re formidable, sincere, majestic. \\
In the midst of the \textsanskrit{Saṅgha} of ascetics, \\
you shine like the sun. 

You’re\marginnote{16.13} a mendicant fine to see, \\
with skin that shines like lustrous gold. \\
But with such excellent appearance, \\
what do you want with the ascetic life? 

You’re\marginnote{16.17} fit to be a king, \\
a wheel-turning monarch, chief of charioteers, \\
victorious in the four quarters, \\
lord of the Black Plum Tree Land. 

Aristocrats,\marginnote{16.21} nobles, and kings \\
ought follow your rule. \\
Gotama, may you reign \\
as king of kings, lord of men!” 

“I\marginnote{17.1} am a king, Sela, \\
the supreme king of the teaching. \\
By the teaching I roll forth the wheel \\
which cannot be rolled back.”\footnote{This is of course a reference to the first sermon (\href{https://suttacentral.net/sn56.11/en/sujato}{SN 56.11}). } 

“You\marginnote{18.1} claim to be awakened, \\
the supreme king of the teaching. \\
‘I roll forth the teaching’: \\
so you say, Gotama. 

Then\marginnote{18.5} who is your general,\footnote{The title “General of the Dhamma” belongs to \textsanskrit{Sāriputta} (\href{https://suttacentral.net/ud2.8/en/sujato\#16.2}{Ud 2.8:16.2}, \href{https://suttacentral.net/thag18.1/en/sujato\#33.1}{Thag 18.1:33.1}). } \\
the disciple who follows the Teacher’s way? \\
Who keeps rolling the wheel \\
of teaching you rolled forth?” 

“By\marginnote{19.1} me the wheel was rolled forth,” \\
\scspeaker{said the Buddha, }\\
“the supreme wheel of teaching. \\
\textsanskrit{Sāriputta}, taking after the Realized One, \\
keeps it rolling on.\footnote{\textit{\textsanskrit{Anujāta}} is said at \href{https://suttacentral.net/iti74/en/sujato\#4.1}{Iti 74:4.1} to be a child who “takes after” the good qualities of their parents. } 

I\marginnote{19.6} have known what should be known,\footnote{As at \href{https://suttacentral.net/mn91/en/sujato\#31.5}{MN 91:31.5}. } \\
and developed what should be developed, \\
and given up what should be given up: \\
and so, brahmin, I am a Buddha. 

Dispel\marginnote{19.10} your doubt in me—\\
make up your mind, brahmin! \\
The sight of a Buddha \\
is hard to find again.\footnote{\textit{\textsanskrit{Abhiṇhaso}} means “repeatedly”. Here the force of the saying is, I think, “It is hard enough to encounter a Buddha even once, let alone repeatedly.” } 

I\marginnote{19.14} am a Buddha, brahmin, \\
the supreme surgeon, \\
one of those whose appearance in the world \\
is hard to find again. 

A\marginnote{19.18} manifestation of divinity, unequaled, \\
crusher of \textsanskrit{Māra}’s army; \\
having subdued all my opponents, \\
I rejoice, fearing nothing from any quarter.” 

“Pay\marginnote{20.1} heed, sirs, to what \\
is spoken by the Clear-eyed One. \\
The surgeon, the great hero, \\
roars like a lion in the jungle. 

A\marginnote{20.5} manifestation of divinity, unequaled, \\
crusher of \textsanskrit{Māra}’s army; \\
who would not be inspired by him, \\
even one born in a dark class?\footnote{The “dark class” refers to those born in an unfortunate state (\href{https://suttacentral.net/an6.57/en/sujato\#11.1}{AN 6.57:11.1}). Sela is saying that the Buddha’s path is for everyone, not just the fortunate. } 

Those\marginnote{20.9} who wish may follow me; \\
those who don’t may go. \\
Right here, I’ll go forth in his presence, \\
the one of such splendid wisdom.” 

“Sir,\marginnote{21.1} if you endorse \\
the teaching of the Buddha, \\
we’ll also go forth in his presence, \\
the one of such splendid wisdom.” 

“These\marginnote{22.1} three hundred brahmins \\
with joined palms held up, ask: \\
‘May we lead the spiritual life \\
in your presence, Blessed One?’” 

“The\marginnote{23.1} spiritual life is well explained,” \\
\scspeaker{said the Buddha, }\\
“apparent in the present life, immediately effective. \\
Here the going forth isn’t in vain \\
for one who trains with diligence.” 

%
\end{verse}

And\marginnote{24.1} the brahmin Sela together with his assembly received the going forth, the ordination in the Buddha’s presence. 

And\marginnote{25.1} when the night had passed \textsanskrit{Keṇiya} had delicious fresh and cooked foods prepared in his own hermitage. Then he had the Buddha informed of the time, saying, “It’s time, Mister Gotama, the meal is ready.” 

Then\marginnote{25.3} the Buddha robed up in the morning and, taking his bowl and robe, went to \textsanskrit{Keṇiya}’s hermitage, where he sat on the seat spread out, together with the \textsanskrit{Saṅgha} of mendicants. Then \textsanskrit{Keṇiya} served and satisfied the mendicant \textsanskrit{Saṅgha} headed by the Buddha with his own hands with delicious fresh and cooked foods. When the Buddha had eaten and washed his hand and bowl, \textsanskrit{Keṇiya} took a low seat and sat to one side. The Buddha expressed his appreciation with these verses: 

\begin{verse}%
“The\marginnote{26.1} foremost of sacrifices \\>is the offering to the sacred flame;\footnote{Here \textit{\textsanskrit{yaññā}} is a truncated genitive plural (“of sacrifices”). } \\
the \textsanskrit{Sāvitrī} Mantra \\>is the foremost of poetic meters;\footnote{The \textsanskrit{Sāvitrī} (or \textsanskrit{Gāyatrī}) is the “head” of metres (\textit{\textsanskrit{śiro} \textsanskrit{gāyatryaḥ}}, Śatapatha \textsanskrit{Brāhmaṇa}  8.6.2.3, 8.6.2.6), and was “praised as the foremost of recitations” (\textit{\textsanskrit{gāyatrāṇy} uktha mukhanai \textsanskrit{śaṁsanti}}, \textsanskrit{Kauṣītaki} \textsanskrit{Brāhmaṇa} 17.7.1). It was the first verse taught a new initiate (Śatapatha \textsanskrit{Brāhmaṇa} 11.5.4.13). } \\
of humans, the king is the foremost; \\
the ocean’s the foremost of rivers; 

the\marginnote{26.5} foremost of stars is the moon; \\
the sun is the foremost of lights; \\
for those who sacrifice seeking merit, \\
the \textsanskrit{Saṅgha} is the foremost.” 

%
\end{verse}

When\marginnote{26.9} the Buddha had expressed his appreciation to \textsanskrit{Keṇiya} the matted-hair ascetic with these verses, he got up from his seat and left. 

Then\marginnote{27.1} Venerable Sela and his assembly, living alone, withdrawn, diligent, keen, and resolute, soon realized the supreme end of the spiritual path in this very life. They lived having achieved with their own insight the goal for which gentlemen rightly go forth from the lay life to homelessness. 

They\marginnote{27.2} understood: “Rebirth is ended; the spiritual journey has been completed; what had to be done has been done; there is nothing further for this place.” And Venerable Sela together with his assembly became perfected. 

Then\marginnote{28.1} Sela with his assembly went to see the Buddha. He arranged his robe over one shoulder, raised his joined palms toward the Buddha, and said: 

\begin{verse}%
“This\marginnote{28.2} is the eighth day since \\
we went for refuge, O Clear-eyed One. \\
In these seven days, Blessed One, \\
we’ve become tamed in your teaching. 

You\marginnote{28.6} are the Buddha, you are the Teacher,\footnote{This verse and the next are also at \href{https://suttacentral.net/snp3.6/en/sujato\#49.1}{Snp 3.6:49.1}. } \\
you are the sage who has overcome \textsanskrit{Māra}; \\
you have cut off the underlying tendencies, \\
you’ve crossed over, and you bring humanity across. 

You\marginnote{28.10} have transcended attachments, \\
your defilements are shattered; \\
by not grasping, like a lion, \\
you’ve given up fear and dread. 

These\marginnote{28.14} three hundred mendicants \\
stand with joined palms raised. \\
Stretch out your feet, great hero: \\
let these giants bow to the Teacher.” 

%
\end{verse}

%
\section*{{\suttatitleacronym MN 93}{\suttatitletranslation With Assalāyana }{\suttatitleroot Assalāyanasutta}}
\addcontentsline{toc}{section}{\tocacronym{MN 93} \toctranslation{With Assalāyana } \tocroot{Assalāyanasutta}}
\markboth{With Assalāyana }{Assalāyanasutta}
\extramarks{MN 93}{MN 93}

\scevam{So\marginnote{1.1} I have heard.\footnote{This discourse records how the brahmin student \textsanskrit{Assalāyana} took on the Buddha on the question of the four classes, an attempt that \textsanskrit{Assalāyana} knew was doomed from the start. Like similar contests such as the \textsanskrit{Ambaṭṭhasutta} (\href{https://suttacentral.net/dn3/en/sujato}{DN 3}), it is rich in historical connections. } }At one time the Buddha was staying near \textsanskrit{Sāvatthī} in Jeta’s Grove, \textsanskrit{Anāthapiṇḍika}’s monastery. 

Now\marginnote{2.1} at that time around five hundred brahmins from abroad were residing in \textsanskrit{Sāvatthī} on some business.\footnote{Similar gatherings were noted at \textsanskrit{Campā} (\href{https://suttacentral.net/dn4/en/sujato\#4.1}{DN 4:4.1}) and \textsanskrit{Opāsāda} (\href{https://suttacentral.net/mn95/en/sujato\#7.1}{MN 95:7.1}). } Then those brahmins thought, “This ascetic Gotama advocates purification for all four classes. Who is capable of debating with him about this?” 

Now\marginnote{3.1} at that time the student \textsanskrit{Assalāyana} was residing in \textsanskrit{Sāvatthī}. He was young, tonsured, and sixteen years old. He had mastered the three Vedas, together with their vocabularies and ritual performance, their phonology and word classification, and the testaments as fifth. He knew them word-by-word, and their grammar. He was well versed in cosmology and the marks of a great man.\footnote{While this description is similar to that of brahmin students such as \textsanskrit{Ambaṭṭha} (\href{https://suttacentral.net/dn3/en/sujato\#1.3.1}{DN 3:1.3.1}), it is only here and at \href{https://suttacentral.net/mn95/en/sujato\#11.2}{MN 95:11.2} that we find the additional details that they are tonsured and sixteen years of age. | “Tonsured” (\textit{vuttasira}) literally means “circlehead” and it refers to the practice of shaving off all but a topknot (cf. \textit{\textsanskrit{nivṛttacūḍaka}} at \textsanskrit{Manusmṛiti} 5.67). According to \textsanskrit{Manusmṛiti} 2.65, the ceremony of tonsure (there called \textit{\textsanskrit{keśānta}}) is to be given to brahmins at sixteen, so this must be meant here. Contrast the usual brahmanical pejorative of \textit{\textsanskrit{muṇḍaka}} (“shaveling”, \href{https://suttacentral.net/mn81/en/sujato\#6.8}{MN 81:6.8}). } 

Then\marginnote{3.2} those brahmins thought, “This \textsanskrit{Assalāyana} is capable of debating with the ascetic Gotama about this.”\footnote{\textsanskrit{Assalāyana} may be identified with \textsanskrit{Kauśalyaṣcāśvalāyana} (“\textsanskrit{Assalāyana} of Kosala”) who asked the sage \textsanskrit{Pippalāda} about the nature of \textit{\textsanskrit{prāṇa}} (breath/life energy) in \textsanskrit{Praśna} \textsanskrit{Upaniṣad} 3. He was probably a grandson or further descendant (as indicated by the \textit{\textsanskrit{naḍādi}} formation of his name) of the \textsanskrit{Aśvala} of Videha who debated \textsanskrit{Yājñavalkya} (see note below). } 

So\marginnote{4.1} they approached \textsanskrit{Assalāyana} and said to him, “This ascetic Gotama advocates purification for all four classes. Please, Mister \textsanskrit{Assalāyana}, debate with the ascetic Gotama about this.” 

When\marginnote{4.4} they said this, \textsanskrit{Assalāyana} said to them, “They say that the ascetic Gotama is a speaker of principle. But speakers of principle are hard to debate. I’m not capable of debating with the ascetic Gotama about this.”\footnote{This dialogue is modeled after the debate at \textsanskrit{Bṛhadāraṇyaka} \textsanskrit{Upaniṣad} 3.1 between \textsanskrit{Yājñavalkya} and \textsanskrit{Aśvala}. Foreign brahmins had come to Videha from Kuru and \textsanskrit{Pañcāla}, the older-established and more prestigious Brahmanical lands (cf. the “western brahmins” of \href{https://suttacentral.net/sn42.6/en/sujato\#2.1}{SN 42.6:2.1} and \href{https://suttacentral.net/an10.176/en/sujato\#1.5}{AN 10.176:1.5}). King Janaka offered an extravagant prize of a thousand cows with gold-tipped horns for the greatest brahmin among them. But none dared step forward, so \textsanskrit{Yājñavalkya} started to drive the cows home. \textsanskrit{Aśvala}, the priest in charge of litany (\textit{\textsanskrit{hotṛ}}) for Janaka, challenged him to a debate, but was ultimately silenced. } 

For\marginnote{4.8} a second time, those brahmins said to him “This ascetic Gotama advocates purification for all four classes. Please, Mister \textsanskrit{Assalāyana}, debate with the ascetic Gotama about this. For you have lived as a wanderer.”\footnote{This agrees with \textsanskrit{Praśna} \textsanskrit{Upaniṣad} 1.2, where \textsanskrit{Assalāyana}, along with his fellow-seekers, is asked to live another year in fervor, spiritual practice, and faith (\textit{\textsanskrit{bhūya} eva \textsanskrit{tapasā} \textsanskrit{brahmacaryeṇa} \textsanskrit{śraddhayā} \textsanskrit{saṁvatsaraṁ}}). } And for a second time, \textsanskrit{Assalāyana} refused. 

For\marginnote{4.16} a third time, those brahmins said to him, “This ascetic Gotama advocates purification for all four classes. Please, Mister \textsanskrit{Assalāyana}, debate with the ascetic Gotama about this. For you have lived as a wanderer. Don’t admit defeat before going into battle!” 

When\marginnote{4.21} they said this, \textsanskrit{Assalāyana} said to them, “Clearly, good sirs, I’m not getting through to you when I say: ‘They say that the ascetic Gotama is a speaker of principle. But speakers of principle are hard to debate. I’m not capable of debating with the ascetic Gotama about this.’ Nevertheless, I shall go at your bidding.” 

Then\marginnote{5.1} \textsanskrit{Assalāyana} together with a large group of brahmins went to the Buddha and exchanged greetings with him. When the greetings and polite conversation were over, he sat down to one side and said to the Buddha: 

“Mister\marginnote{5.3} Gotama, the brahmins say: ‘Only brahmins are the best class; other classes are inferior.\footnote{This verse is also at \href{https://suttacentral.net/dn27/en/sujato\#3.6}{DN 27:3.6} and \href{https://suttacentral.net/mn84/en/sujato\#9.19}{MN 84:9.19}. } Only brahmins are the light class; other classes are dark. Only brahmins are purified, not others. Only brahmins are true-born sons of divinity, born from his mouth, born of divinity, created by divinity, heirs of divinity.’ What do you say about this?” 

“But\marginnote{5.9} \textsanskrit{Assalāyana}, brahmin women are seen menstruating, being pregnant, giving birth, and breastfeeding. Yet even though they’re born from a brahmin womb they say: ‘Only brahmins are the best class; other classes are inferior. Only brahmins are the light class; other classes are dark. Only brahmins are purified, not others. Only brahmins are true-born sons of divinity, born from his mouth, born of divinity, created by divinity, heirs of divinity.’” 

“Even\marginnote{6.1} though you say this, still the brahmins maintain their belief.” 

“What\marginnote{6.4} do you think, \textsanskrit{Assalāyana}? Have you heard that in Greece and Persia and other foreign lands there are only two classes, masters and bonded servants; and that masters may become servants, and servants masters?”\footnote{“Greece” (\textit{yona}, from “Ionia”) and “Persia” (\textit{kamboja}) are loose exonyms indicating the regions dominated by Greek and Iranian culture. | The Greeks and other foreigners are mentioned in Ashokan edicts, but this is the only reference to Greece in an early Buddhist text. The Sanskrit \textit{yavana} appears shortly after the Buddha in \textsanskrit{Pāṇini} 4.1.49. The Indic terms are derived from the Persian \textit{yauna}, first attested in an inscription of Darius I (522–486 BCE), who conquered three Indian realms: \textit{\textsanskrit{gandāra}} (\textsanskrit{Gandhāra}), \textit{hindush} (Indus valley), and \textit{thataguš} (Sattagydia, around modern Bannu in Pakistan). Soldiers from these realms served under Achaemenid kings from this time. Moreover, several Greek and Achaemenid coins dating back as far as Darius I have been found in the Kabul hoard as well as the Shaikhan Dheri hoard at the site of \textsanskrit{Puṣkalāvatī}, capital of \textsanskrit{Gandhāra}. | “Persia” is one of the sixteen great nations (\href{https://suttacentral.net/an3.70/en/sujato\#28.2}{AN 3.70:28.2}). It was the home of horses, lying to the west of \textsanskrit{Gandhāra} (Pakistan) in modern Afghanistan. This was the easternmost extent of Iranian culture, and the passage here shows how different it was considered compared to the other nations. At this date, I think \textit{kamboja} meant not just the specific location in Afghanistan, but more generally the Iranian peoples west of \textsanskrit{Gandhāra}, just as \textit{yona} (“Ionian”) is used for all people of Greek culture. | The perceived difference in slave culture mirrors the Greek ambassador Megasthenes who, writing over a century later, said there were no slaves in India. } 

“Yes,\marginnote{6.6} I have heard that.” 

“Then\marginnote{6.7} what is the source of the brahmins’ certainty and forcefulness in this matter that they make this claim?” 

“Even\marginnote{7.1} though you say this, still the brahmins maintain their belief.” 

“What\marginnote{7.4} do you think, \textsanskrit{Assalāyana}? Suppose an aristocrat were to kill living creatures, steal, and commit sexual misconduct; to use speech that’s false, divisive, harsh, or nonsensical; and to be covetous, malicious, with wrong view. When their body breaks up, after death, they’d be reborn in a place of loss, a bad place, the underworld, hell. Would this happen only to an aristocrat, and not to a brahmin? Or suppose a peasant, or a menial were to act in the same way. Would that result befall only a peasant or a menial, and not to a brahmin?” 

“No,\marginnote{7.8} Mister Gotama. If they acted the same way, the same result would befall an aristocrat, a brahmin, a peasant, or a menial. For if any of the four classes were to kill living creatures, steal, and commit sexual misconduct; to use speech that’s false, divisive, harsh, or nonsensical; and to be covetous, malicious, with wrong view, then, when their body breaks up, after death, they’d be reborn in a place of loss, a bad place, the underworld, hell.” 

“Then\marginnote{7.14} what is the source of the brahmins’ certainty and forcefulness in this matter that they make this claim?” 

“Even\marginnote{8.1} though you say this, still the brahmins maintain their belief.” 

“What\marginnote{8.4} do you think, \textsanskrit{Assalāyana}? Suppose a brahmin were to refrain from killing living creatures, stealing, and committing sexual misconduct; from using speech that’s false, divisive, harsh, or nonsensical; and from covetousness, malice, and wrong view. When their body breaks up, after death, they’d be reborn in a good place, a heavenly realm. Would this happen only to a brahmin, and not to an aristocrat, a peasant, or a menial?” 

“No,\marginnote{8.6} Mister Gotama. If they acted the same way, the same result would befall an aristocrat, a brahmin, a peasant, or a menial. For if any of the four classes were to refrain from killing living creatures, stealing, and committing sexual misconduct; from using speech that’s false, divisive, harsh, or nonsensical; and from covetousness, malice, and wrong view, then, when their body breaks up, after death, they’d be reborn in a good place, a heavenly realm.” 

“Then\marginnote{8.12} what is the source of the brahmins’ certainty and forcefulness in this matter that they make this claim?” 

“Even\marginnote{9.1} though you say this, still the brahmins maintain their belief.” 

“What\marginnote{9.4} do you think, \textsanskrit{Assalāyana}? Is only a brahmin capable of developing a heart of love, free of enmity and ill will for this region, and not an aristocrat, peasant, or menial?”\footnote{The next three examples are (1) meditating on love, (2) bathing, and (3) starting a fire. All of these reflect activities characteristic of brahmins: (1) the “divine meditations” (\textit{\textsanskrit{brahmavihāra}}) that originated as contemplation on the qualities of divinity, (2) ritual bathing to wash away sin, and (3) worship of Agni on the fire altar. } 

“No,\marginnote{9.6} Mister Gotama. Aristocrats, brahmins, peasants, and menials can all do so. For all four classes are capable of developing a heart of love, free of enmity and ill will for this region.” 

“Then\marginnote{9.12} what is the source of the brahmins’ certainty and forcefulness in this matter that they make this claim?” 

“Even\marginnote{10.1} though you say this, still the brahmins maintain their belief.” 

“What\marginnote{10.4} do you think, \textsanskrit{Assalāyana}? Is only a brahmin capable of taking some bathing cleanser, going to the river, and washing off dust and dirt, and not an aristocrat, peasant, or menial?” 

“No,\marginnote{10.6} Mister Gotama. All four classes are capable of doing this.” 

“Then\marginnote{10.11} what is the source of the brahmins’ certainty and forcefulness in this matter that they make this claim?” 

“Even\marginnote{11.1} though you say this, still the brahmins maintain their belief.” 

“What\marginnote{11.4} do you think, \textsanskrit{Assalāyana}? Suppose an anointed aristocratic king were to gather a hundred people of various births and say to them:\footnote{Śatapatha \textsanskrit{Brāhmaṇa} 11.5 relates how \textsanskrit{Purūravas}, longing to be reunited with his queen, the nymph \textsanskrit{Urvaśī}, was required by the Gandharvas to make a special fire. It must be generated by wood from a tree that grew from the ashes of celestial fire provided by themselves, the upper drill-stick of \textit{\textsanskrit{aśvattha}} (i.e. Bodhi tree) and the lower stick of \textit{\textsanskrit{śamī}}. In that legend, the essence of the fire is determined by its divine origin, against which the Buddha poses a naturalistic explanation. } ‘Please gentlemen, let anyone here who was born in a family of aristocrats, brahmins, or chieftains take a drill-stick made of teak, sal, frankincense wood, sandalwood, or cherry wood, light a fire and produce heat.\footnote{This seems to be the only mention of the \textit{padumaka} tree in Pali. It may be the Sanskrit \textit{padmaka}, for which one possible identification is \emph{Prunus cerasoides}, the wild Himalayan cherry. | Here we find both \textit{\textsanskrit{sāka}} and \textit{\textsanskrit{sāla}}, so I translate \textit{\textsanskrit{sāka}} as “teak”, whereas at \href{https://suttacentral.net/dn3/en/sujato\#1.15.10}{DN 3:1.15.10} I have sakhua, an alternate name for the sal tree. } And let anyone here who was born in a family of corpse-workers, hunters, bamboo-workers, chariot-makers, or scavengers take a drill-stick made from a dog’s drinking trough, a pig’s trough, a dustbin, or castor-oil wood, light a fire and produce heat.’\footnote{The peoples listed here probably all originated as various aboriginal tribes. | \textit{\textsanskrit{Caṇḍālas}} (“corpse-workers”) are frequently depicted as the lowest of untouchables as seen by Brahmins. However, the usual rendering of “outcast” is unsatisfactory, as all the groups here are defined positively by their occupation, and moreover some, such as the scavengers, would also have been untouchable. The Pali commentaries equate the \textit{\textsanskrit{caṇḍāla}} with the \textit{chavaka} (or \textit{chapaka}) of \href{https://suttacentral.net/pli-tv-bu-vb-sk69/en/sujato\#1.12.1}{Bu Sk 69:1.12.1}, a “corpse disposer” (\textit{\textsanskrit{chavachaḍḍakacaṇḍālā}}; see also \textit{\textsanskrit{chavakacaṇḍālo}} at \href{https://suttacentral.net/mil5.4.7/en/sujato\#2.4}{Mil 5.4.7:2.4}). They were known for their “bone-washing” ceremony (\href{https://suttacentral.net/dn1/en/sujato\#1.13.2}{DN 1:1.13.2}, \href{https://suttacentral.net/an10.107/en/sujato\#1.1}{AN 10.107:1.1}). This agrees with \textsanskrit{Rāmāyana} 1.58.10, which describes a \textit{\textsanskrit{caṇḍāla}} as dark-complexioned, dirty, with disheveled hair, his body smeared with graveyard ash. \textsanskrit{Mahāvaṁsa} 10.91–94 tells how King \textsanskrit{Paṇḍukābhaya} employed \textit{\textsanskrit{caṇḍālas}} for cleaning streets and toilets, bearing the dead, and watching over cemeteries, with their village adjoining the cemetery. } 

What\marginnote{11.8} do you think, \textsanskrit{Assalāyana}? Would only the fire produced by the high class people with good quality wood have flames, color, and radiance, and be usable as fire, and not the fire produced by the low class people with poor quality wood?” 

“No,\marginnote{11.11} Mister Gotama. The fire produced by the high class people with good quality wood would have flames, color, and radiance, and be usable as fire, and so would the fire produced by the low class people with poor quality wood. For all fire has flames, color, and radiance, and is usable as fire.”\footnote{A similar argument is made regarding consciousness at \href{https://suttacentral.net/mn93/en/sujato\#11.5}{MN 93:11.5}. } 

“Then\marginnote{11.15} what is the source of the brahmins’ certainty and forcefulness in this matter that they make this claim?” 

“Even\marginnote{12.1} though you say this, still the brahmins maintain their belief.” 

“What\marginnote{12.4} do you think, \textsanskrit{Assalāyana}?\footnote{The complex question of intercaste children is usually governed by the assumption that the father supplies the “seed” which the mother merely incubates. \textsanskrit{Kauṭilya}, however, notes a variety of opinions on this question (\textsanskrit{Arthaśāstra} 3.7.1–3). | \textsanskrit{Assalāyana}’s views do not agree with the more developed doctrine found in later texts. } Suppose an aristocrat boy was to sleep with a brahmin girl, and they had a child.\footnote{Since the brahmins declared their own class the highest, such a union is considered to be \textit{pratiloma} (“contrary to the natural order”). \textsanskrit{Arthaśāstra} 3.7.28 says such a child is a \textit{\textsanskrit{sūta}}; \textsanskrit{Manusmṛti} 10.11 agrees, adding that their livelihood is a charioteer (10.47). } Would that child be called an aristocrat after the father or a brahmin after the mother?” 

“They\marginnote{12.7} could be called either.” 

“What\marginnote{13.1} do you think, \textsanskrit{Assalāyana}? Suppose a brahmin boy was to sleep with an aristocrat girl, and they had a child.\footnote{\textsanskrit{Baudhāyana} \textsanskrit{Dharmasūtra} 1.9.17 says that the child of a brahmin father and aristocrat mother is a brahmin. \textsanskrit{Manusmṛti} 10.6 admits the same, though due to the “defect” of the mother they are considered “like” a brahmin. } Would that child be called an aristocrat after the mother or a brahmin after the father?” 

“They\marginnote{13.4} could be called either.” 

“What\marginnote{14.1} do you think, \textsanskrit{Assalāyana}? Suppose a mare were to mate with a donkey, and she gave birth to a mule. Would that mule be called a horse after the mother or a donkey after the father?” 

“It’s\marginnote{14.4} a mule, as it is a crossbreed. I see the difference in this case, but not in the previous cases.” 

“What\marginnote{15.1} do you think, \textsanskrit{Assalāyana}? Suppose there were two young students who were brothers who had shared a womb. One was an educated reciter, while the other was not an educated reciter.\footnote{“Shared a womb” is \textit{sodariya}, the only occurrence of this term in early Pali. | This thought experiment anticipates the modern method of studying twins to eliminate variables in development. } Who would the brahmins feed first at an offering of food for ancestors, an offering of a dish of milk-rice, a sacrifice, or a feast for guests?” 

“They’d\marginnote{15.4} first feed the young student who was an educated reciter. For how could an offering to someone who is not an educated reciter be very fruitful?” 

“What\marginnote{16.1} do you think, \textsanskrit{Assalāyana}? Suppose there were two young students who were brothers who had shared a womb. One was an educated reciter, but was unethical, of bad character, while the other was not an educated reciter, but was ethical and of good character. Who would the brahmins feed first?” 

“They’d\marginnote{16.4} first feed the young student who was not an educated reciter, but was ethical and of good character. For how could an offering to someone who is unethical and of bad character be very fruitful?” 

“Firstly\marginnote{17.1} you relied on birth, \textsanskrit{Assalāyana}, then you switched to education, then you switched to abstemious behavior.\footnote{Here \textit{tape} (“fervent austerity”) stands for ethical conduct, hence I translate “abstemious behavior”. Some manuscripts omit it, but I think it is required by the sense. } Now you’ve come around to believing in purification for the four classes, just as I advocate.” When he said this, \textsanskrit{Assalāyana} sat silent, dismayed, shoulders drooping, downcast, depressed, with nothing to say. 

Knowing\marginnote{18.1} this, the Buddha said to him: 

“Once\marginnote{18.2} upon a time, \textsanskrit{Assalāyana}, seven brahmin seers settled in leaf huts in a wilderness region. They had the following harmful misconception:\footnote{The “seven brahmin seers”  (Sanskrit \textit{\textsanskrit{saptarṣi}}) were renowned sages to whom the Vedic lineages (\textit{\textsanskrit{gottā}}) were attributed. As the sutta’s conclusion will show, they are introduced here to demonstrate that the fallacy of geneology has been there since the beginning. | A similar practice of living in wilderness is described at \href{https://suttacentral.net/mn25/en/sujato\#9.5}{MN 25:9.5}, \href{https://suttacentral.net/sn11.9/en/sujato}{SN 11.9}, \href{https://suttacentral.net/dn23/en/sujato\#21.3}{DN 23:21.3}, \href{https://suttacentral.net/dn3/en/sujato\#2.3.3}{DN 3:2.3.3}, and \href{https://suttacentral.net/dn27/en/sujato\#22.9}{DN 27:22.9}. \textsanskrit{Chāndogya} \textsanskrit{Upaniṣad} 5.10.1–5 compares the forest contemplatives destined for the \textsanskrit{Brahmā} realm with the ritualists who are reborn on the moon before returning to earth. } ‘Only brahmins are the best class; other classes are inferior. Only brahmins are the light class; other classes are dark. Only brahmins are purified, not others.\footnote{\textsanskrit{Mahābhārata} 12.181.5 says that brahmins are the white class, aristocrats red, peasants yellow, and menials black. } Only brahmins are true-born sons of divinity, born from his mouth, born of divinity, created by divinity, heirs of divinity.’ 

The\marginnote{18.6} seer Devala the Dark heard about this.\footnote{Asita Devala is probably meant to be the legendary seer known as Asita or Devala son of \textsanskrit{Kāśyapa} who composed Rig Veda 9.5–24, although they seem to share little but the name. He is an example of the “dark hermit” archetype, irrupting within Brahmanical traditions to subvert from within, relying on his mysterious power of otherness. See also Asita \textsanskrit{Kaṇhasiri} (or \textsanskrit{Kāladevala}, \href{https://suttacentral.net/snp3.11/en/sujato\#1.1}{Snp 3.11:1.1}), \textsanskrit{Kaṇha} (\href{https://suttacentral.net/dn3/en/sujato\#1.23.6}{DN 3:1.23.6}), \textsanskrit{Kaṇhadīpāyana} (\href{https://suttacentral.net/cp31/en/sujato\#1.2}{Cp 31:1.2}), and \textsanskrit{Sāma} (\href{https://suttacentral.net/cp33/en/sujato\#1.1}{Cp 33:1.1}). Such encounters show how the contention between Vedic ritualism and native contemplatives challenged and elevated both traditions. In this way they invert the stereotype of Brahmanical colorism, paving the way for the ascension of the dark god Krishna. } So he did up his hair and beard, dressed in magenta robes, put on his boots, grasped a golden staff, and appeared in the courtyard of the seven brahmin seers.\footnote{His behavior is disrespectful, like that of Vepacitti, and unlike that of Sakka at \href{https://suttacentral.net/sn11.9/en/sujato}{SN 11.9}. But the context is different, for there the seers were of good character and deserving of respect, whereas here they embrace wrong views. } Then he wandered about the yard saying, ‘Where, oh where have those brahmin seers gone? Where, oh where have those brahmin seers gone?’ 

Then\marginnote{18.14} those brahmin seers said, ‘Who’s this wandering about our courtyard like a village lout? Let’s curse him!’ 

So\marginnote{18.19} they cursed Devala the Dark, ‘Be ashes, lowlife! Be ashes, lowlife!’ But the more the seers cursed him, the more attractive, good-looking, and lovely Devala the Dark became.\footnote{Compare \href{https://suttacentral.net/sn11.22/en/sujato\#1.7}{SN 11.22:1.7}. } 

Then\marginnote{18.23} those brahmin seers said, ‘Our fervor is in vain! Our spiritual path is fruitless! For when we used to curse someone to become ashes, ashes they became. But the more we curse this one, the more attractive, good-looking, and lovely he becomes.’ 

‘Gentlemen,\marginnote{18.29} your fervor is not in vain; your spiritual path is not fruitless. Please let go of your malevolence towards me.’ 

‘We\marginnote{18.31} let go of our malevolence towards you. But who are you, sir?’ 

‘Have\marginnote{18.33} you heard of the seer Devala the Dark?’ 

‘Yes,\marginnote{18.35} sir.’ 

‘I\marginnote{18.36} am he, sirs.’ Then they approached Devala and bowed to him. 

Devala\marginnote{18.38} said to them, ‘I heard that when the seven brahmin seers had settled in leaf huts in a wilderness region, they had the following harmful misconception:\footnote{This passage accurately conveys that caste was not just criticized by Buddhism, it was actively contested within Brahmanical circles. As just one of many examples, \textsanskrit{Yudhiṣṭhira} said, “It is not by reason of family or study or learning that one is of high caste, but because of behavior alone” (\textsanskrit{Mahābhārata} 3.297.61). } “Only brahmins are the best class; other classes are inferior. Only brahmins are the light class; other classes are dark. Only brahmins are purified, not others. Only brahmins are true-born sons of divinity, born from his mouth, born of divinity, created by divinity, heirs of divinity.”’ 

‘That’s\marginnote{18.44} right, sir.’ 

‘But\marginnote{18.45} do you know whether your birth mother only had relations with a brahmin and not with a non-brahmin?’ 

‘We\marginnote{18.47} don’t know that.’ 

‘But\marginnote{18.48} do you know whether your birth mother’s mothers back to the seventh generation only had relations with brahmins and not with non-brahmins?’ 

‘We\marginnote{18.50} don’t know that.’ 

‘But\marginnote{18.51} do you know whether your birth father only had relations with a brahmin woman and not with a non-brahmin?’ 

‘We\marginnote{18.53} don’t know that.’ 

‘But\marginnote{18.54} do you know whether your birth father’s fathers back to the seventh generation only had relations with brahmins and not with non-brahmins?’ 

‘We\marginnote{18.56} don’t know that.’ 

‘But\marginnote{18.57} do you know how an embryo is conceived?’ 

‘We\marginnote{18.59} do know that, sir. An embryo is conceived when these three things come together—the mother and father come together, the mother is in the fertile phase of her menstrual cycle, and the virile spirit is ready.’\footnote{The \textit{gandhabba} (normally “centaur”, here “virile spirit”) is, per the commentary, the being to be reborn. He represents the element of male sexuality in procreation. He is “father, begetter, kinsman”, by whose knowledge one becomes the “father’s father” (Arthavaveda 2.1.2–3). He is said to guard (Rig Veda 9.83.4) or steal (Śatapatha \textsanskrit{Brāhmaṇa} 3.2.4) or actually be (Rig Veda 9.86.36) the vitalizing liquid Soma, who is extracted on an altar as semen is extracted in the vagina (\textsanskrit{Bṛhadāraṇyaka} \textsanskrit{Upaniṣad} 6.4.3). He thus lies within the waters (Rig Veda 9.86.36, 10.10.4) or within the womb (Rig Veda 10.177.2). It seems that, since semen is “a man’s essence” (\textsanskrit{Bṛhadāraṇyaka} \textsanskrit{Upaniṣad} 6.4.1, Aitareya \textsanskrit{Upaniṣad} 2.1), the seed of past lovers—curses against whom are helpfully provided (\textsanskrit{Bṛhadāraṇyaka} \textsanskrit{Upaniṣad} 6.4.9–12)—remains in the womb. This is why the \textit{gandhabba} possesses women (\textsanskrit{Bṛhadāraṇyaka} \textsanskrit{Upaniṣad} 1.7.1, 3.3.1) and, unless placated with proper worship, might become a “hairy one” who devours embryos (Artharva Veda 8.6.23, see 8.6.18–19). To ward against such risks, in the rites of marriage and procreation, the \textit{gandharva} king \textsanskrit{Viśvāvasa}, a playboy and deadbeat father, is asked to “rise up” out of the woman (Rig Veda 10.85.21–22, Atharva Veda 14.2.33–6, \textsanskrit{Bṛhadāraṇyaka} \textsanskrit{Upaniṣad} 6.4.19) and find his pleasures with another, so that the husband may father the child. This contrasts with the Pali passage, where the \textit{gandhabba} “is ready” for conception; in other words, a being is driven by their karma to be born there. } 

‘But\marginnote{18.62} do you know for sure whether that virile spirit is an aristocrat, a brahmin, a peasant, or a menial?’\footnote{\textsanskrit{Chāndogya} \textsanskrit{Upaniṣad} 5.10.3–7 details the “path of the fathers” after death; they follow a winding path to the moon Soma, only to return as rain, grow as plants, be eaten, and ultimately be emitted as semen, thus determining the caste of the child. It seems from this passage, however, that this doctrine, convincing as it may seem, was not sufficient to allay male anxiety over paternity. Indeed, the very baroqueness of these conceptions show the doctrinal and ritual lengths required to fix paternity. } 

‘We\marginnote{18.64} don’t know that.’ 

‘In\marginnote{18.66} that case, sirs, don’t you know what you are?’ 

‘In\marginnote{18.68} that case, sir, we don’t know what we are.’ 

Given\marginnote{18.70} that even those seven brahmin seers could not prevail when pursued, pressed, and grilled by the seer Devala on their own genealogy,\footnote{The syntax of this passage mirrors \href{https://suttacentral.net/an6.18/en/sujato\#4.12}{AN 6.18:4.12} and \href{https://suttacentral.net/pli-tv-kd16/en/sujato\#6.4.1}{Kd 16:6.4.1}. In each case, \textit{te hi \textsanskrit{nāma}} introduces a passage referring to events of the past whose outcome is framed in future tense (from the past point of view). } how could you prevail now being grilled by me on your own genealogy when you and your tradition do not so much as pick up the last spoonful?”\footnote{\textit{\textsanskrit{Puṇṇo} \textsanskrit{dabbigāho}} is a highly specific reference to a detail of the \textsanskrit{Sākamedha} ritual that marks the beginning of winter. From the rice offering on the first day, a pot is set aside until the next morning, when the very last “full spoon” (\textit{\textsanskrit{pūrṇā} darvi}) is scraped up, with the invocation, “Full, O spoon, fly off, and fly back to us well filled!” (Śatapatha \textsanskrit{Brāhmaṇa} 2.5.3.16–17, \textsanskrit{Kauṣītaki} \textsanskrit{Brāhmaṇa} 5.6.20; invocation found at Atharva Veda 3.10.7c, \textsanskrit{Maitrāyaṇī} \textsanskrit{Saṁhitā} 1.10.2). This ensures that present prosperity will be renewed after the barren cold season (just as life is renewed after death, or as last season’s grain is sown on the field). The implication is that present-day brahmins don’t even fulfill the rites of continuity in their own tradition, so how can they assert the continuity of caste over generations? | \textit{\textsanskrit{Yesaṁ}} refers back to \textit{te} of the previous portion, i.e. the seven brahmin seers, whose tradition modern brahmins are supposed to continue. } 

When\marginnote{19.1} he had spoken, \textsanskrit{Assalāyana} said to him, “Excellent, Mister Gotama! … From this day forth, may Mister Gotama remember me as a lay follower who has gone for refuge for life.” 

%
\section*{{\suttatitleacronym MN 94}{\suttatitletranslation With Ghoṭamukha }{\suttatitleroot Ghoṭamukhasutta}}
\addcontentsline{toc}{section}{\tocacronym{MN 94} \toctranslation{With Ghoṭamukha } \tocroot{Ghoṭamukhasutta}}
\markboth{With Ghoṭamukha }{Ghoṭamukhasutta}
\extramarks{MN 94}{MN 94}

\scevam{So\marginnote{1.1} I have heard.\footnote{As usual, the lack of mention of the Buddha at the start suggests that this sutta is set after his passing away, a detail confirmed at the end. This sutta attests to the continued influence of Buddhism among the wealthy and influential as they adapted to the new realities after the Buddha’s passing. } }At one time Venerable Udena was staying near Varanasi in the Khemiya Mango Grove.\footnote{Neither monk nor monastery are mentioned elsewhere in early texts. } 

Now\marginnote{2.1} at that time the brahmin \textsanskrit{Ghoṭamukha} had arrived at Varanasi on some business.\footnote{This may be the \textsanskrit{Ghoṭamukha} (“Beard-face”) of \textsanskrit{Kauṭilya}’s \textsanskrit{Arthaśāstra} 5.5.11. On the question of when a courtier should know that his king is displeased with him, a series of expert opinions are cited, each one bewilderingly cryptic. \textsanskrit{Ghoṭamukha} says such a king is like a “wet cloth”. The key to these clues is provided in a particular story, which is deliberately unmentioned, presumably because of the delicacy of the subject. } Then as he was going for a walk he went to the Khemiya Mango Grove. At that time Venerable Udena was walking mindfully in the open air. \textsanskrit{Ghoṭamukha} approached and exchanged greetings with him. 

Walking\marginnote{2.5} alongside Udena, he said, “Mister ascetic, there is no such thing as a principled renunciate life;\footnote{\textit{Ambho \textsanskrit{samaṇa}}  (“mister ascetic”) is a unique form of address. I think it is deliberately clumsy, a sign that Ghotamukha, having never met a monk before, doesn’t really know the right way to address them. } that’s what I think. Yet I have not seen honorable ones such as yourself, or a relevant teaching.” 

When\marginnote{3.1} he said this, Udena stepped down from the walking path, entered his dwelling, and sat down on the seat spread out.\footnote{Udena takes care to ensure the setting is appropriate for a serious discussion. } \textsanskrit{Ghoṭamukha} also stepped down from the walking path and entered the dwelling, where he stood to one side. Udena said to him, “There are seats, brahmin. Please sit if you wish.” 

“I\marginnote{3.6} was just waiting for you to sit down. For how could one such as I presume to sit first without being invited?” 

Then\marginnote{4.1} he took a low seat and sat to one side, where he said, “Mister ascetic, there is no such thing as a principled renunciate life; that’s what I think. Yet I have not seen honorable ones such as yourself, or a relevant teaching.” 

“Brahmin,\marginnote{4.6} we can discuss this. But only if you allow what should be allowed, and reject what should be rejected. And if you ask me the meaning of anything you don’t understand, saying:\footnote{\textsanskrit{Ghoṭamukha} was unfamiliar with the forms of address, admits he has never seen a mendicant, talks to Udena while he is walking meditation, and does not know when he should sit down. Udena, noticing his awkwardness, makes sure that \textsanskrit{Ghoṭamukha} understands that he can engage in a conversation without impropriety. } ‘Sir, why is this? What does that mean?’” 

“Let\marginnote{4.8} us discuss this. I will do as you say.” 

“Brahmin,\marginnote{5.1} these four people are found in the world. What four? 

\begin{enumerate}%
\item One person mortifies themselves, committed to the practice of mortifying themselves. %
\item One person mortifies others, committed to the practice of mortifying others. %
\item One person mortifies themselves and others, committed to the practice of mortifying themselves and others. %
\item One person doesn’t mortify either themselves or others, committed to the practice of not mortifying themselves or others. They live without wishes in this very life, quenched, cooled, experiencing bliss, with self become divine. %
\end{enumerate}

Which\marginnote{5.8} one of these four people do you like the sound of?” 

“Sir,\marginnote{5.9} I don’t like the sound of the first three people. I only like the sound of the last person, who doesn’t mortify either themselves or others.” 

“But\marginnote{6.1} why don’t you like the sound of those three people?” 

“Sir,\marginnote{6.2} the person who mortifies themselves does so even though they want to be happy and recoil from pain. That’s why I don’t like the sound of that person. The person who mortifies others does so even though others want to be happy and recoil from pain. That’s why I don’t like the sound of that person. The person who mortifies themselves and others does so even though both themselves and others want to be happy and recoil from pain. That’s why I don’t like the sound of that person. The person who doesn’t mortify either themselves or others—living without wishes, quenched, cooled, experiencing bliss, with self become divine—does not torment themselves or others, both of whom want to be happy and recoil from pain. That’s why I like the sound of that person.” 

“There\marginnote{7.1} are, brahmin, these two groups of people. What two? There’s one group of people who, being obsessed with jeweled earrings, seeks partners and children, male and female bondservants, fields and lands, and gold and currency. 

And\marginnote{7.4} there’s another group of people who, not being obsessed with jeweled earrings, has given up partner and children, male and female bondservants, fields and lands, and gold and currency, and goes forth from the lay life to homelessness. 

Now,\marginnote{7.5} brahmin, that person who doesn’t mortify either themselves or others—in which of these two groups of people do you usually find such a person?” 

“I\marginnote{7.9} usually find such a person in the group that has gone forth from the lay life to homelessness.” 

“Just\marginnote{8.1} now I understood you to say: ‘Mister ascetic, there is no such thing as a principled renunciate life; that’s what I think. Yet I have not seen honorable ones such as yourself, or a relevant teaching.” 

“Well,\marginnote{8.5} I obviously had my reasons for saying that, master Udena.\footnote{The commentary glosses \textit{\textsanskrit{sānuggahā}} with \textit{\textsanskrit{sakāraṇā}}, “with reason”. } But there is such a thing as a principled renunciate life. That’s what I think, and that’s how you should remember me. Now, these four people that you’ve spoken of in a brief summary: please explain them to me in detail, out of sympathy.” 

“Well\marginnote{9.1} then, brahmin, listen and apply your mind well, I will speak.” 

“Yes,\marginnote{9.2} sir,” replied \textsanskrit{Ghoṭamukha}. Udena said this: 

“What\marginnote{10.1} person mortifies themselves, committed to the practice of mortifying themselves?\footnote{For details, see notes at \href{https://suttacentral.net/mn12/en/sujato\#45.1}{MN 12:45.1}, \href{https://suttacentral.net/mn27/en/sujato\#11.1}{MN 27:11.1}, and \href{https://suttacentral.net/mn51/en/sujato\#8.1}{MN 51:8.1}. } It’s when a person goes naked, ignoring conventions. They lick their hands, and don’t come or wait when called. They don’t consent to food brought to them, or food prepared on their behalf, or an invitation for a meal. They don’t receive anything from a pot or bowl; or from someone who keeps sheep, or who has a weapon or a shovel in their home; or where a couple is eating; or where there is a woman who is pregnant, breastfeeding, or who lives with a man; or where there’s a dog waiting or flies buzzing. They accept no fish or meat or beer or wine, and drink no fermented gruel. They go to just one house for alms, taking just one mouthful, or two houses and two mouthfuls, up to seven houses and seven mouthfuls. They feed on one saucer a day, two saucers a day, up to seven saucers a day. They eat once a day, once every second day, up to once a week, and so on, even up to once a fortnight. They live committed to the practice of eating food at set intervals. They eat herbs, millet, wild rice, poor rice, water lettuce, rice bran, scum from boiling rice, sesame flour, grass, or cow dung. They survive on forest roots and fruits, or eating fallen fruit. They wear robes of sunn hemp, mixed hemp, corpse-wrapping cloth, rags, lodh tree bark, antelope hide (whole or in strips), kusa grass, bark, wood-chips, human hair, horse-tail hair, or owls’ wings. They tear out their hair and beard, committed to this practice. They constantly stand, refusing seats. They squat, committed to persisting in the squatting position. They lie on a mat of thorns, making a mat of thorns their bed. They’re devoted to ritual bathing three times a day, including the evening. And so they live committed to practicing these various ways of mortifying and tormenting the body. This is called a person who mortifies themselves, being committed to the practice of mortifying themselves. 

And\marginnote{11.1} what person mortifies others, committed to the practice of mortifying others? It’s when a person is a slaughterer of sheep, pigs, poultry, or deer, a hunter or fisher, a bandit, an executioner, a butcher of cattle, a jailer, or has some other cruel livelihood. This is called a person who mortifies others, being committed to the practice of mortifying others. 

And\marginnote{12.1} what person mortifies themselves and others, being committed to the practice of mortifying themselves and others? It’s when a person is an anointed aristocratic king or a well-to-do brahmin. He has a new ceremonial hall built to the east of the citadel. He shaves off his hair and beard, dresses in a rough antelope hide, and smears his body with ghee and oil. Scratching his back with antlers, he enters the hall with his chief queen and the brahmin high priest. There he lies on the bare ground strewn with grass. The king feeds on the milk from one teat of a cow that has a calf of the same color. The chief queen feeds on the milk from the second teat. The brahmin high priest feeds on the milk from the third teat. The milk from the fourth teat is served to the sacred flame. The calf feeds on the remainder. He says: ‘Slaughter this many bulls, bullocks, heifers, goats, rams, and horses for the sacrifice! Fell this many trees and reap this much grass for the sacrificial equipment!’ His bondservants, servants, and workers do their jobs under threat of punishment and danger, weeping with tearful faces. This is called a person who mortifies themselves and others, being committed to the practice of mortifying themselves and others. 

And\marginnote{13.1} what person doesn’t mortify either themselves or others, committed to the practice of not mortifying themselves or others, living without wishes in this very life, quenched, cooled, experiencing bliss, with self become divine? 

It’s\marginnote{14.1} when a Realized One arises in the world, perfected, a fully awakened Buddha, accomplished in knowledge and conduct, holy, knower of the world, supreme guide for those who wish to train, teacher of gods and humans, awakened, blessed. He has realized with his own insight this world—with its gods, \textsanskrit{Māras}, and divinities, this population with its ascetics and brahmins, gods and humans—and he makes it known to others. He proclaims a teaching that is good in the beginning, good in the middle, and good in the end, meaningful and well-phrased. And he reveals a spiritual practice that’s entirely full and pure. 

A\marginnote{15.1} householder hears that teaching, or a householder’s child, or someone reborn in a good family. They gain faith in the Realized One, and reflect: ‘Life at home is cramped and dirty, life gone forth is wide open. It’s not easy for someone living at home to lead the spiritual life utterly full and pure, like a polished shell. Why don’t I shave off my hair and beard, dress in ocher robes, and go forth from the lay life to homelessness?’ After some time they give up a large or small fortune, and a large or small family circle. They shave off hair and beard, dress in ocher robes, and go forth from the lay life to homelessness. Once they’ve gone forth, they take up the training and livelihood of the mendicants. They give up killing living creatures, renouncing the rod and the sword. They’re scrupulous and kind, living full of sympathy for all living beings. 

They\marginnote{16.1} give up stealing. They take only what’s given, and expect only what’s given. They keep themselves clean by not thieving. 

They\marginnote{16.2} give up unchastity. They are celibate, set apart, avoiding the vulgar act of sex. 

They\marginnote{16.3} give up lying. They speak the truth and stick to the truth. They’re honest and dependable, and don’t trick the world with their words. 

They\marginnote{16.4} give up divisive speech. They don’t repeat in one place what they heard in another so as to divide people against each other. Instead, they reconcile those who are divided, supporting unity, delighting in harmony, loving harmony, speaking words that promote harmony. 

They\marginnote{16.5} give up harsh speech. They speak in a way that’s mellow, pleasing to the ear, lovely, going to the heart, polite, likable and agreeable to the people. 

They\marginnote{16.6} give up talking nonsense. Their words are timely, true, and meaningful, in line with the teaching and training. They say things at the right time which are valuable, reasonable, succinct, and beneficial. 

They\marginnote{16.7} refrain from injuring plants and seeds. They eat in one part of the day, abstaining from eating at night and food at the wrong time. They refrain from seeing shows of dancing, singing, and music . They refrain from beautifying and adorning themselves with garlands, fragrance, and makeup. They refrain from high and luxurious beds. They refrain from receiving gold and currency, raw grains, raw meat, women and girls, male and female bondservants, goats and sheep, chickens and pigs, elephants, cows, horses, and mares, and fields and land. They refrain from running errands and messages; buying and selling; falsifying weights, metals, or measures; bribery, fraud, cheating, and duplicity; mutilation, murder, abduction, banditry, plunder, and violence. 

They’re\marginnote{17.1} content with robes to look after the body and almsfood to look after the belly. Wherever they go, they set out taking only these things. They’re like a bird: wherever it flies, wings are its only burden. In the same way, a mendicant is content with robes to look after the body and almsfood to look after the belly. Wherever they go, they set out taking only these things. When they have this entire spectrum of noble ethics, they experience a blameless happiness inside themselves. 

When\marginnote{18.1} they see a sight with their eyes, they don’t get caught up in the features and details. If the faculty of sight were left unrestrained, bad unskillful qualities of covetousness and displeasure would become overwhelming. For this reason, they practice restraint, protecting the faculty of sight, and achieving its restraint. When they hear a sound with their ears … When they smell an odor with their nose … When they taste a flavor with their tongue … When they feel a touch with their body … When they know an idea with their mind, they don’t get caught up in the features and details. If the faculty of mind were left unrestrained, bad unskillful qualities of covetousness and displeasure would become overwhelming. For this reason, they practice restraint, protecting the faculty of mind, and achieving its restraint. When they have this noble sense restraint, they experience an unsullied bliss inside themselves. 

They\marginnote{19.1} act with situational awareness when going out and coming back; when looking ahead and aside; when bending and extending the limbs; when bearing the outer robe, bowl and robes; when eating, drinking, chewing, and tasting; when urinating and defecating; when walking, standing, sitting, sleeping, waking, speaking, and keeping silent. 

When\marginnote{20.1} they have this entire spectrum of noble ethics, this noble contentment, this noble sense restraint, and this noble mindfulness and situational awareness, they frequent a secluded lodging—a wilderness, the root of a tree, a hill, a ravine, a mountain cave, a charnel ground, a forest, the open air, a heap of straw. 

After\marginnote{21.1} the meal, they return from almsround, sit down cross-legged, set their body straight, and establish mindfulness in their presence. Giving up covetousness for the world, they meditate with a heart rid of covetousness, cleansing the mind of covetousness. Giving up ill will and malevolence, they meditate with a mind rid of ill will, full of sympathy for all living beings, cleansing the mind of ill will. Giving up dullness and drowsiness, they meditate with a mind rid of dullness and drowsiness, perceiving light, mindful and aware, cleansing the mind of dullness and drowsiness. Giving up restlessness and remorse, they meditate without restlessness, their mind peaceful inside, cleansing the mind of restlessness and remorse. Giving up doubt, they meditate having gone beyond doubt, not undecided about skillful qualities, cleansing the mind of doubt. 

They\marginnote{22.1} give up these five hindrances, corruptions of the heart that weaken wisdom. Then, quite secluded from sensual pleasures, secluded from unskillful qualities, they enter and remain in the first absorption, which has the rapture and bliss born of seclusion, while placing the mind and keeping it connected. 

As\marginnote{23.1} the placing of the mind and keeping it connected are stilled, they enter and remain in the second absorption, which has the rapture and bliss born of immersion, with internal clarity and mind at one, without placing the mind and keeping it connected. 

And\marginnote{24.1} with the fading away of rapture, they enter and remain in the third absorption, where they meditate with equanimity, mindful and aware, personally experiencing the bliss of which the noble ones declare, ‘Equanimous and mindful, one meditates in bliss.’ 

Giving\marginnote{25.1} up pleasure and pain, and ending former happiness and sadness, they enter and remain in the fourth absorption, without pleasure or pain, with pure equanimity and mindfulness. 

When\marginnote{26.1} their mind has become immersed in \textsanskrit{samādhi} like this—purified, bright, flawless, rid of corruptions, pliable, workable, steady, and imperturbable—they extend it toward recollection of past lives. They recollect many kinds of past lives. That is: one, two, three, four, five, ten, twenty, thirty, forty, fifty, a hundred, a thousand, a hundred thousand rebirths; many eons of the world contracting, many eons of the world expanding, many eons of the world contracting and expanding. They remember: ‘There, I was named this, my clan was that, I looked like this, and that was my food. This was how I felt pleasure and pain, and that was how my life ended. When I passed away from that place I was reborn somewhere else. There, too, I was named this, my clan was that, I looked like this, and that was my food. This was how I felt pleasure and pain, and that was how my life ended. When I passed away from that place I was reborn here.’ And so they recollect their many kinds of past lives, with features and details. 

When\marginnote{27.1} their mind has become immersed in \textsanskrit{samādhi} like this—purified, bright, flawless, rid of corruptions, pliable, workable, steady, and imperturbable—they extend it toward knowledge of the death and rebirth of sentient beings. With clairvoyance that is purified and superhuman, they see sentient beings passing away and being reborn—inferior and superior, beautiful and ugly, in a good place or a bad place. They understand how sentient beings are reborn according to their deeds: ‘These dear beings did bad things by way of body, speech, and mind. They denounced the noble ones; they had wrong view; and they chose to act out of that wrong view. When their body breaks up, after death, they’re reborn in a place of loss, a bad place, the underworld, hell. These dear beings, however, did good things by way of body, speech, and mind. They never denounced the noble ones; they had right view; and they chose to act out of that right view. When their body breaks up, after death, they’re reborn in a good place, a heavenly realm.’ And so, with clairvoyance that is purified and superhuman, they see sentient beings passing away and being reborn—inferior and superior, beautiful and ugly, in a good place or a bad place. They understand how sentient beings are reborn according to their deeds. 

When\marginnote{28.1} their mind has become immersed in \textsanskrit{samādhi} like this—purified, bright, flawless, rid of corruptions, pliable, workable, steady, and imperturbable—they extend it toward knowledge of the ending of defilements. They truly understand: ‘This is suffering’ … ‘This is the origin of suffering’ … ‘This is the cessation of suffering’ … ‘This is the practice that leads to the cessation of suffering’. They truly understand: ‘These are defilements’ … ‘This is the origin of defilements’ … ‘This is the cessation of defilements’ … ‘This is the practice that leads to the cessation of defilements’. 

Knowing\marginnote{29.1} and seeing like this, their mind is freed from the defilements of sensuality, desire to be reborn, and ignorance. When they’re freed, they know they’re freed. 

They\marginnote{29.3} understand: ‘Rebirth is ended, the spiritual journey has been completed, what had to be done has been done, there is nothing further for this place.’ 

This\marginnote{30.1} is called a person who neither mortifies themselves or others, being committed to the practice of not mortifying themselves or others. They live without wishes in this very life, quenched, cooled, experiencing bliss, with self become divine.” 

When\marginnote{31.1} he had spoken, \textsanskrit{Ghoṭamukha} said to him, “Excellent, Mister Udena! Excellent! As if he were righting the overturned, or revealing the hidden, or pointing out the path to the lost, or lighting a lamp in the dark so people with clear eyes can see what’s there, Mister Udena has made the teaching clear in many ways. I go for refuge to Mister Udena, to the teaching, and to the mendicant \textsanskrit{Saṅgha}. From this day forth, may Mister Udena remember me as a lay follower who has gone for refuge for life.” 

“Brahmin,\marginnote{32.1} don’t go for refuge to me.\footnote{As at \href{https://suttacentral.net/mn84/en/sujato\#10.6}{MN 84:10.6}. } You should go for refuge to that same Blessed One to whom I have gone for refuge.” 

“But\marginnote{32.3} Mister Udena, where is the Blessed One at present, the perfected one, the fully awakened Buddha?” 

“Brahmin,\marginnote{32.4} the Buddha has already become fully quenched.” 

“Mister\marginnote{32.5} Udena, if I heard that the Buddha was within ten leagues, or twenty, or even up to a hundred leagues away, I’d go a hundred leagues to see him. 

But\marginnote{32.11} since the Buddha has become fully quenched, I go for refuge to that fully quenched Buddha, to the teaching, and to the \textsanskrit{Saṅgha}. From this day forth, may Mister Udena remember me as a lay follower who has gone for refuge for life. Mister Udena, the king of \textsanskrit{Aṅga} gives me a regular daily allowance. I will give you one portion of that.”\footnote{The mention of the king of \textsanskrit{Aṅga} is unique. \textsanskrit{Aṅga} had been conquered by \textsanskrit{Bimbisāra} long ago and became part of Magadha. There is a faint memory of a time when \textsanskrit{Rājagaha} was a city of \textsanskrit{Aṅga} (\href{https://suttacentral.net/ja546/en/sujato\#36.2}{Ja 546:36.2}), but \textsanskrit{Aṅga}’s glory days were long past. Perhaps “king of \textsanskrit{Aṅga}” was a title of the king of Magadha, or else of a vassal ruler. | \textit{\textsanskrit{Niccabhikkhā}} at \href{https://suttacentral.net/ja398/en/sujato\#4.1}{Ja 398:4.1} refers to the daily offering of food to a \textit{yakkha}, which must be the early meaning. But here and at \href{https://suttacentral.net/sn3.13/en/sujato\#4.4}{SN 3.13:4.4} it refers to money. } 

“But\marginnote{33.2} brahmin, what does the king of \textsanskrit{Aṅga} give you as a regular daily allowance?” 

“Five\marginnote{33.3} hundred dollars.”\footnote{A \textit{\textsanskrit{kahāpaṇa}} is in the ballpark of a US dollar. At \href{https://suttacentral.net/an10.46 /en/sujato}{AN 10.46 } the wages for an honest day’s work start at half a \textit{\textsanskrit{kahāpaṇa}}, which compares with the poverty line in modern India of 38 rupees/day or 48 US cents. The upper level income is a thousand \textit{\textsanskrit{kahāpaṇas}} per day, so \textsanskrit{Ghoṭamukha}’s allowance made him a rich man. } 

“It’s\marginnote{33.4} not proper for us to receive gold and currency.” 

“If\marginnote{33.5} that’s not proper, I will have a dwelling built for Mister Udena.”\footnote{A \textit{\textsanskrit{vihāra}} was a “dwelling” in a monastery, not, as in later usage, an entire monastery. } 

“If\marginnote{33.6} you want to build me a dwelling, then build an assembly hall for the \textsanskrit{Saṅgha} at \textsanskrit{Pāṭaliputta}.”\footnote{This reflects the central role of the new capital of \textsanskrit{Pāṭaliputta} after the Buddha died (\href{https://suttacentral.net/dn16/en/sujato\#1.28.7}{DN 16:1.28.7}). Compare the similar offering at \href{https://suttacentral.net/mn52/en/sujato\#16.1}{MN 52:16.1}. } 

“Now\marginnote{33.7} I’m even more delighted and satisfied with Mister Udena, since he encourages me to give to the \textsanskrit{Saṅgha}. So with this allowance and another one I will have an assembly hall built for the \textsanskrit{Saṅgha} at \textsanskrit{Pāṭaliputta}.” 

And\marginnote{33.9} so he had that hall built. And these days it’s called the “\textsanskrit{Ghoṭamukhī}”. 

%
\section*{{\suttatitleacronym MN 95}{\suttatitletranslation With Caṅkī }{\suttatitleroot Caṅkīsutta}}
\addcontentsline{toc}{section}{\tocacronym{MN 95} \toctranslation{With Caṅkī } \tocroot{Caṅkīsutta}}
\markboth{With Caṅkī }{Caṅkīsutta}
\extramarks{MN 95}{MN 95}

\scevam{So\marginnote{1.1} I have heard. }At one time the Buddha was wandering in the land of the Kosalans together with a large \textsanskrit{Saṅgha} of mendicants when he arrived at a village of the Kosalan brahmins named \textsanskrit{Opāsāda}.\footnote{\textsanskrit{Opāsāda} does not appear anywhere else. We later hear of \textsanskrit{Caṅkī}’s longhouse (\textit{\textsanskrit{pāsāda}}), a luxurious residence to the north of the town on the way to the Godswood. Perhaps, then, \textsanskrit{Opāsāda} means “below the mansion”. } He stayed in the Godswood of sal trees to the north of \textsanskrit{Opāsāda}.\footnote{A grove in which offerings were made to the gods. } 

Now\marginnote{2.1} at that time the brahmin \textsanskrit{Caṅkī} was living in \textsanskrit{Opāsāda}. It was a crown property given by King Pasenadi of Kosala, teeming with living creatures, full of hay, wood, water, and grain, a royal park endowed to a brahmin.\footnote{\textsanskrit{Caṅkī} is regularly mentioned along with other senior brahmins \textsanskrit{Tārukkha}, \textsanskrit{Pokkharasāti}, \textsanskrit{Jānussoṇi}, and Todeyya (\href{https://suttacentral.net/dn13/en/sujato\#2.2}{DN 13:2.2}, \href{https://suttacentral.net/mn98/en/sujato\#2.2}{MN 98:2.2}, \href{https://suttacentral.net/mn99/en/sujato\#13.5}{MN 99:13.5}, \href{https://suttacentral.net/snp3.9/en/sujato\#1.4}{Snp 3.9:1.4}). His name is obscure; spelled \textit{\textsanskrit{caṅgī}} in Sanskrit, it is perhaps related to the \textit{\textsanskrit{gatāgate} \textsanskrit{caṅgimā}} of \textsanskrit{Mahā}-\textsanskrit{subhāṣita}-\textsanskrit{Saṁgraha} 7235-1, where it refers to the sensual walk of a young woman; dictionaries record the sense “beautiful” for Kannada \textit{\textsanskrit{caṁgi}} and Sanskrit \textit{\textsanskrit{caṅgiman}}. | “Royal park” is \textit{\textsanskrit{rājadāya}} (cp. \textit{\textsanskrit{migadāya}}, “deer park”). | A \textit{brahmadeyya} is a gift of land by a king to a brahmin, which was an outstanding feature of Indian feudalism. Similar grants are mentioned in the \textsanskrit{Dīgha} \textsanskrit{Nikāya} at \href{https://suttacentral.net/dn3/en/sujato\#1.2.1}{DN 3:1.2.1}, \href{https://suttacentral.net/dn5/en/sujato\#1.4}{DN 5:1.4}, \href{https://suttacentral.net/dn12/en/sujato\#1.3}{DN 12:1.3}, and \href{https://suttacentral.net/dn23/en/sujato\#1.4}{DN 23:1.4}. } 

The\marginnote{3.1} brahmins and householders of \textsanskrit{Opāsāda} heard: “It seems the ascetic Gotama—a Sakyan, gone forth from a Sakyan family—has arrived at \textsanskrit{Opāsāda} together with a large \textsanskrit{Saṅgha} of mendicants. He is staying in the Godswood to the north. He has this good reputation: ‘That Blessed One is perfected, a fully awakened Buddha, accomplished in knowledge and conduct, holy, knower of the world, supreme guide for those who wish to train, teacher of gods and humans, awakened, blessed.’ He has realized with his own insight this world—with its gods, \textsanskrit{Māras}, and divinities, this population with its ascetics and brahmins, gods and humans—and he makes it known to others. He proclaims a teaching that is good in the beginning, good in the middle, and good in the end, meaningful and well-phrased. And he reveals a spiritual practice that’s entirely full and pure. It’s good to see such perfected ones.” 

Then,\marginnote{4.1} having departed \textsanskrit{Opāsāda}, they formed into companies and headed north to the Godswood. 

Now\marginnote{5.1} at that time the brahmin \textsanskrit{Caṅkī} had retired to the upper floor of his stilt longhouse for his midday nap.\footnote{Also a favored siesta location for \textsanskrit{Soṇadaṇḍa} (\href{https://suttacentral.net/dn5/en/sujato\#3.1}{DN 5:3.1}), \textsanskrit{Kūṭadanta} (\href{https://suttacentral.net/dn4/en/sujato\#3.1}{DN 4:3.1}), and \textsanskrit{Pāyāsi} (\href{https://suttacentral.net/dn23/en/sujato\#3.1}{DN 23:3.1}). } He saw the brahmins and householders heading for the Godswood, and addressed his butler,\footnote{The \textit{khatta} (“butler”; Sanskrit \textit{\textsanskrit{kṣattṛ}}) was a senior member of the household staff, responsible for management of activities (Śatapatha \textsanskrit{Brāhmaṇa} 5.3.1.7). } “My butler, why are the brahmins and householders heading north for the Godswood?” 

“The\marginnote{6.1} ascetic Gotama has arrived at \textsanskrit{Opāsāda} together with a large \textsanskrit{Saṅgha} of mendicants. He is staying in the Godswood to the north. He has this good reputation: ‘That Blessed One is perfected, a fully awakened Buddha, accomplished in knowledge and conduct, holy, knower of the world, supreme guide for those who wish to train, teacher of gods and humans, awakened, blessed.’ They’re going to see that Mister Gotama.” 

“Well\marginnote{6.5} then, go to the brahmins and householders and say to them: ‘Sirs, the brahmin \textsanskrit{Caṅkī} asks you to wait, as he will also go to see the ascetic Gotama.’” 

“Yes,\marginnote{6.8} sir,” replied the butler, and did as he was asked. 

Now\marginnote{7.1} at that time around five hundred brahmins from abroad were residing in \textsanskrit{Opāsāda} on some business. They heard that the brahmin \textsanskrit{Caṅkī} was going to see the ascetic Gotama. They approached \textsanskrit{Caṅkī} and said to him, “Is it really true that you are going to see the ascetic Gotama?” 

“Yes,\marginnote{7.6} gentlemen, it is true.” 

“Please\marginnote{8.1} don’t! It’s not appropriate for you to go to see the ascetic Gotama;\footnote{Compare \href{https://suttacentral.net/dn4/en/sujato\#5.2}{DN 4:5.2} and \href{https://suttacentral.net/dn5/en/sujato\#6.2}{DN 5:6.2}. } it’s appropriate that he comes to see you. 

You\marginnote{8.4} are well born on both your mother’s and father’s side, of pure descent, with irrefutable and impeccable genealogy back to the seventh paternal generation.\footnote{\textit{\textsanskrit{Jātivāda}} is sometimes translated as “doctrine of birth”, but the context here shows this cannot be the case. It refers to the genealogical records of the family lineage. } For this reason it’s not appropriate for you to go to see the ascetic Gotama; it’s appropriate that he comes to see you. 

You’re\marginnote{8.7} rich, affluent, and wealthy. … 

You\marginnote{8.8} recite and remember the hymns, and have mastered the three Vedas, together with their vocabularies and ritual performance, their phonology and word classification, and the testaments as fifth. You know them word-by-word, and their grammar. You are well versed in cosmology and the marks of a great man. … 

You\marginnote{8.9} are attractive, good-looking, lovely, of surpassing beauty. You are magnificent and splendid as the Divinity, remarkable to behold. …\footnote{For \textsanskrit{Mahāsaṅgīti} \textit{\textsanskrit{vacchasī}} read \textit{\textsanskrit{vaccasī}} (Sanskrit \textit{varcasin}), “possessing splendor”. } 

You\marginnote{8.10} are ethical, mature in ethical conduct. … 

You’re\marginnote{8.11} a good speaker who enunciates well, with a polished, clear, and articulate voice that expresses the meaning. … 

You\marginnote{8.12} teach the tutors of many, and teach three hundred young students to recite the hymns. …\footnote{Notice that the royal endowment was not just for a luxury residence, it was the site of a major international college. Kings invested in education. } 

You’re\marginnote{8.13} honored, respected, revered, venerated, and esteemed by King Pasenadi of Kosala and the brahmin \textsanskrit{Pokkharasāti}. … 

You\marginnote{8.15} live in \textsanskrit{Opāsāda}, a crown property given by King Pasenadi of Kosala, teeming with living creatures, full of hay, wood, water, and grain, a royal park endowed to a brahmin. 

For\marginnote{8.16} all these reasons it’s not appropriate for you to go to see the ascetic Gotama; it’s appropriate that he comes to see you.” 

When\marginnote{9.1} they had spoken, \textsanskrit{Caṅkī} said to those brahmins: 

“Well\marginnote{9.2} then, gentlemen, listen to why it’s appropriate for me to go to see the ascetic Gotama, and it’s not appropriate for him to come to see me. 

He\marginnote{9.4} is well born on both his mother’s and father’s side, of pure descent, with irrefutable and impeccable genealogy back to the seventh paternal generation. For this reason it’s not appropriate for the ascetic Gotama to come to see me; rather, it’s appropriate for me to go to see him. 

When\marginnote{9.7} he went forth he abandoned abundant gold, both coined and uncoined, stored in dungeons and towers. … 

He\marginnote{9.8} went forth from the lay life to homelessness while still a youth, young, with pristine black hair, blessed with youth, in the prime of life. … 

Though\marginnote{9.9} his mother and father wished otherwise, weeping with tearful faces, he shaved off his hair and beard, dressed in ocher robes, and went forth from the lay life to homelessness. … 

He\marginnote{9.10} is attractive, good-looking, lovely, of surpassing beauty. He is magnificent and splendid as the Divinity, remarkable to behold. … 

He\marginnote{9.11} is ethical, possessing ethical conduct that is noble and skillful. … 

He’s\marginnote{9.12} a good speaker who enunciates well, with a polished, clear, and articulate voice that expresses the meaning. … 

He’s\marginnote{9.13} a tutor of tutors. … 

He\marginnote{9.14} has ended sensual desire, and is rid of caprice. … 

He\marginnote{9.15} teaches the efficacy of deeds and action. He doesn’t wish any harm upon the community of brahmins. … 

He\marginnote{9.16} went forth from an eminent family of unbroken aristocratic lineage. … 

He\marginnote{9.17} went forth from a rich, affluent, and wealthy family. … 

People\marginnote{9.18} come from distant lands and distant countries to question him. …\footnote{“Distant land” (\textit{\textsanskrit{tiroraṭṭha}}) is defined at \href{https://suttacentral.net/pli-tv-bi-vb-pc38/en/sujato\#2.6}{Bi Pc 38:2.6} as “any land other than where one is living”. } 

Many\marginnote{9.19} thousands of deities have gone for refuge for life to him. … 

He\marginnote{9.20} has this good reputation: ‘That Blessed One is perfected, a fully awakened Buddha, accomplished in knowledge and conduct, holy, knower of the world, supreme guide for those who wish to train, teacher of gods and humans, awakened, blessed.’ … 

He\marginnote{9.22} has the thirty-two marks of a great man. … 

King\marginnote{9.23} Seniya \textsanskrit{Bimbisāra} of Magadha and his wives and children have gone for refuge for life to the ascetic Gotama. …\footnote{\textsanskrit{Bimbisāra}’s refuge is at \href{https://suttacentral.net/pli-tv-kd1/en/sujato\#22.11.4}{Kd 1:22.11.4}. } 

King\marginnote{9.24} Pasenadi of Kosala and his wives and children have gone for refuge for life to the ascetic Gotama. …\footnote{Pasenadi’s refuge is at \href{https://suttacentral.net/sn3.1/en/sujato\#14.5}{SN 3.1:14.5}. } 

The\marginnote{9.25} brahmin \textsanskrit{Pokkharasāti} and his wives and children have gone for refuge for life to the ascetic Gotama. …\footnote{Refuge of \textsanskrit{Pokkharasāti} and family is at \href{https://suttacentral.net/dn3/en/sujato\#2.22.4}{DN 3:2.22.4}. } 

The\marginnote{9.26} ascetic Gotama has arrived to stay in the Godswood to the north of \textsanskrit{Opāsāda}. Any ascetic or brahmin who comes to stay in our village district is our guest, and should be honored and respected as such. For this reason, too, it’s not appropriate for Mister Gotama to come to see me, rather, it’s appropriate for me to go to see him. 

This\marginnote{9.33} is the extent of Mister Gotama’s praise that I have memorized. But his praises are not confined to this, for the praise of Mister Gotama is limitless. The possession of even a single one of these factors makes it inappropriate for Mister Gotama to come to see me, rather, it’s appropriate for me to go to see him. Well then, gentlemen, let’s all go to see the ascetic Gotama.” 

Then\marginnote{10.1} \textsanskrit{Caṅkī} together with a large group of brahmins went to the Buddha and exchanged greetings with him. When the greetings and polite conversation were over, he sat down to one side. 

Now\marginnote{11.1} at that time the Buddha was sitting engaged in some polite conversation together with some very senior brahmins. And the student \textsanskrit{Kāpaṭika} was sitting in that assembly. He was young, tonsured, and sixteen years old. He had mastered the three Vedas, together with their vocabularies and ritual performance, their phonology and word classification, and the testaments as fifth. He knew them word-by-word, and their grammar. He was well versed in cosmology and the marks of a great man.\footnote{This student’s education is like that of \textsanskrit{Assalāyana} (\href{https://suttacentral.net/mn93/en/sujato\#3.1}{MN 93:3.1}), but his character could not be more different. His name has a variety of spellings in Pali and Sanskrit, but the correct form is \textit{\textsanskrit{kāpaṭika}} from \textit{\textsanskrit{kapaṭa}}, “fraud”. \textsanskrit{Kauṭilya} explains that fraudulent students (\textit{\textsanskrit{chātraḥ} \textsanskrit{kāpaṭikaḥ}}) were spies installed by kings to test the loyalty of subjects (\textsanskrit{Arthaśāstra} 1.10.11, 1.11.1–2), for which they were paid handsomely (5.3.22). Note that the Buddha calls him \textsanskrit{Bhāradvāja}; this follows the convention in Pali that brahmins of that name are distinguished by epithets, often unflattering: \textsanskrit{Bhāradvāja} the Farmer (\href{https://suttacentral.net/snp1.4/en/sujato\#1.3}{Snp 1.4:1.3}), \textsanskrit{Bhāradvāja} the Fire-Worshiper (\href{https://suttacentral.net/snp1.7/en/sujato\#1.4}{Snp 1.7:1.4}), \textsanskrit{Bhāradvāja} the Alms-Gatherer (\href{https://suttacentral.net/sn35.127/en/sujato\#1.1}{SN 35.127:1.1}), \textsanskrit{Bhāradvāja} the Builder (\href{https://suttacentral.net/sn7.17/en/sujato\#1.2}{SN 7.17:1.2}), \textsanskrit{Bhāradvāja} the Rude (\href{https://suttacentral.net/sn7.2/en/sujato\#1.2}{SN 7.2:1.2}), \textsanskrit{Bhāradvāja} the Fiend (\href{https://suttacentral.net/sn7.3/en/sujato\#1.2}{SN 7.3:1.2}), \textsanskrit{Bhāradvāja} the Bitter (\href{https://suttacentral.net/sn7.4/en/sujato\#1.2}{SN 7.4:1.2}), and so on. Thus \textsanskrit{Kāpaṭika} would have been an epithet recognizing that he was the king’s spy, a practice to which Pasenadi openly admits (\href{https://suttacentral.net/sn3.11/en/sujato\#7.1}{SN 3.11:7.1}, \href{https://suttacentral.net/ud6.2/en/sujato\#8.1}{Ud 6.2:8.1}). This was the downside of building a religious institution on the king’s largess. Whether he was recognized as a spy at the time, or the epithet was applied later, is uncertain, but at some point it seems the meaning was forgotten and the epithet was taken to be his proper name. } While the senior brahmins were conversing together with the Buddha, he interrupted. 

Then\marginnote{11.4} the Buddha rebuked \textsanskrit{Kāpaṭika}, “Venerable \textsanskrit{Bhāradvāja}, don’t interrupt the senior brahmins.\footnote{It is rare for the Buddha to address a lay person as “venerable” (\textit{\textsanskrit{āyasmā}}). It seems the form is excessively polite to soften the rebuke. } Wait until they’ve finished speaking.” 

When\marginnote{11.7} he had spoken, \textsanskrit{Caṅkī} said to the Buddha, “Mister Gotama, don’t rebuke the student \textsanskrit{Kāpaṭika}. He’s a gentleman, learned and astute, who enunciates well. He is capable of debating with Mister Gotama about this.” 

Then\marginnote{12.1} it occurred to the Buddha,\footnote{This passage is unique to to this sutta. } “Clearly the student \textsanskrit{Kāpaṭika} will talk about the scriptural heritage of the three Vedas. That’s why the brahmins put him at the front.” 

Then\marginnote{12.4} \textsanskrit{Kāpaṭika} thought, “When the ascetic Gotama looks at me, I’ll ask him a question.” Then the Buddha, knowing \textsanskrit{Kāpaṭika}’s train of thought, looked at him. 

Then\marginnote{12.7} \textsanskrit{Kāpaṭika} thought, “The ascetic Gotama is engaging with me. Why don’t I ask him a question?” Then he said, “Mister Gotama, regarding that which by the lineage of testament and by canonical authority is the ancient hymnal of the brahmins, the brahmins come to the categorical conclusion:\footnote{The “lineage of testament” (\textit{\textsanskrit{itihitihaparamparā}}); “canonical authority” (\textit{\textsanskrit{piṭakasampadā}}); “ancient hymnal” (\textit{\textsanskrit{porāṇaṁ} \textsanskrit{mantapadaṁ}}): these refer to the Vedas. } ‘This is the only truth, anything else is futile.’ What do you say about this?” 

“Well,\marginnote{13.1} \textsanskrit{Bhāradvāja}, is there even a single one of the brahmins who says this:\footnote{Compare \href{https://suttacentral.net/mn99/en/sujato\#9.8}{MN 99:9.8}. } ‘I know this, I see this: this is the only truth, anything else is futile’?” 

“No,\marginnote{13.4} Mister Gotama.” 

“Well,\marginnote{13.5} is there even a single tutor of the brahmins, or a tutors’ tutor, or anyone back to the seventh generation of tutors, who says this:\footnote{Compare \href{https://suttacentral.net/dn13/en/sujato\#12.1}{DN 13:12.1}. } ‘I know this, I see this: this is the only truth, anything else is futile’?” 

“No,\marginnote{13.8} Mister Gotama.” 

“Well,\marginnote{13.9} what of the ancient seers of the brahmins, namely \textsanskrit{Aṭṭhaka}, \textsanskrit{Vāmaka}, \textsanskrit{Vāmadeva}, \textsanskrit{Vessāmitta}, Yamadaggi, \textsanskrit{Aṅgīrasa}, \textsanskrit{Bhāradvāja}, \textsanskrit{Vāseṭṭha}, Kassapa, and Bhagu? They were the authors and propagators of the hymns. Their hymnal was sung and propagated and compiled in ancient times; and these days, brahmins continue to sing and chant it, chanting what was chanted and teaching what was taught.\footnote{The ten names in Pali include the seven authors of the so-called “family books” of the Rig Veda (\textsanskrit{Maṇḍalas} 2–8). As founders of poetic lineages, we often find works by their descendants, which are not always confined to their dedicated family book. Poems by the other three authors are mostly outside the family books. Thus the sages listed here cover most of the Rig Veda, although the Vedic tradition records many other authors as well. | Atri Bhauma (\textsanskrit{Maṇḍala} 5, rather than \textsanskrit{Aṣṭaka} \textsanskrit{Vaiśvāmitra} of 10.104); Vamra(ka) \textsanskrit{Vaikhānasa} (10.99; see 9.66); \textsanskrit{Vāmadeva} Gautama (\textsanskrit{Maṇḍala} 4); \textsanskrit{Viśvāmitra} \textsanskrit{Gāthina} (\textsanskrit{Maṇḍala} 3); Jamadagni \textsanskrit{Bhārgava} was a descendant of \textsanskrit{Bhṛgu} (several hymns mostly in \textsanskrit{Maṇḍalas} 9 and 10); \textsanskrit{Aṅgirasa} is identified with Agni as the founder of a lineage of poet-singers (\textsanskrit{Maṇḍala} 8); \textsanskrit{Bharadvāja} \textsanskrit{Bārhaspatya} (\textsanskrit{Maṇḍala} 6); \textsanskrit{Vasiṣṭha} \textsanskrit{Maitrāvaruṇi} (\textsanskrit{Maṇḍala} 7); \textsanskrit{Kaśyapa} \textsanskrit{Mārīca} (several hymns mostly in \textsanskrit{Maṇḍalas} 9 and 10); \textsanskrit{Bhṛgu} was the bringer of fire from heaven whose adoptive descendant was \textsanskrit{Gṛtsamada} \textsanskrit{Bhārgava} Śaunaka (\textsanskrit{Maṇḍala} 2). } Did even they say: ‘We know this, we see this: this is the only truth, anything else is futile’?” 

“No,\marginnote{13.13} Mister Gotama.” 

“So,\marginnote{13.14} \textsanskrit{Bhāradvāja}, it seems that there is not a single one of the brahmins, not even anyone back to the seventh generation of tutors, nor even the ancient seers of the brahmins who say: ‘We know this, we see this: this is the only truth, anything else is futile.’ 

Suppose\marginnote{13.23} there was a queue of blind men, each holding the one in front: the first one does not see, the middle one does not see, and the last one does not see.\footnote{The “blind following the blind” is also at \href{https://suttacentral.net/mn95/en/sujato\#13.24}{MN 95:13.24} and \href{https://suttacentral.net/mn99/en/sujato\#9.25}{MN 99:9.25}. \textsanskrit{Maitrī} \textsanskrit{Upaniṣad} 7.8–9 turns it around, saying that the blind teach false doctrines aimed at destroying the Vedas, “the doctrine of not-self” (\textit{\textsanskrit{nairātmyavāda}}), an obvious reference to Buddhists. We also find it at \textsanskrit{Kaṭha} \textsanskrit{Upaniṣad} 1.2.5, \textsanskrit{Mahābhārata} 2.38.3, and the Jain \textsanskrit{Sūyagaḍa} 1.1.2.19. } In the same way, it seems to me that the brahmins’ statement turns out to be like a queue of blind men: the first one does not see, the middle one does not see, and the last one does not see. What do you think, \textsanskrit{Bhāradvāja}? This being so, doesn’t the brahmins’ faith turn out to be baseless?” 

“The\marginnote{14.1} brahmins don’t just honor this because of faith, but also because of oral transmission.”\footnote{“Faith” is \textit{\textsanskrit{saddhā}} (see note on \href{https://suttacentral.net/mn26/en/sujato\#15.9}{MN 26:15.9}) |  “Oral transmission” is \textit{anussava}. While the idea of oral tradition was of course central to Vedism, we don’t seem to find the term until later (eg. \textsanskrit{Bhāgavata} \textsanskrit{Purāṇa} 5.8.29; \textsanskrit{Śrīdhara}’s commentary on Bhagavad-\textsanskrit{Gītā} 18.3). | See too \href{https://suttacentral.net/mn76/en/sujato\#24.1}{MN 76:24.1}. } 

“First\marginnote{14.2} you relied on faith, now you speak of oral transmission. These five things can be seen to turn out in two different ways. What five? Faith, endorsement, oral transmission, reasoned train of thought, and acceptance of a view after deliberation.\footnote{\textit{Parivitakka} means “train of thought”, as when the Buddha reads \textsanskrit{Kāpaṭika}’s thoughts above (\href{https://suttacentral.net/mn95/en/sujato\#12.6}{MN 95:12.6}). \textit{\textsanskrit{Ākāra}} in this context means “reason”, as at \href{https://suttacentral.net/mn47/en/sujato\#10.2}{MN 47:10.2}; the current sutta expands on the ideas there. | \textit{Khanti} is usually better rendered “acceptance” than the common “patience”. } Even though you have full faith in something, it may be vacuous, hollow, and false. And even if you don’t have full faith in something, it may be true and real, not otherwise. Even though you fully endorse something …\footnote{\textit{Ruci} means “liking, preference”, and from there takes the sense of “endorsing” an idea or belief, in which case it can be translated as “opinion”. } something may be well transmitted … something may be well thought out … something may be well deliberated, it may be vacuous, hollow, and false. And even if something is not well deliberated, it may be true and real, not otherwise. For a sensible person who is preserving truth this is not sufficient to come to the categorical conclusion:\footnote{This sutta introduces the important epistemological distinction between the “preservation of truth” (\textit{\textsanskrit{saccānurakkhaṇā}}), the “awakening to the truth” (\textit{\textsanskrit{saccānubodha}}), and the “attainment of the truth” (\textit{\textsanskrit{saccānuppatti}}). } ‘This is the only truth, anything else is futile.’” 

“But\marginnote{15.1} Mister Gotama, how do you define the preservation of truth?” 

“If\marginnote{15.3} a person has faith, they preserve truth by saying, ‘Such is my faith.’ But they don’t yet come to the categorical conclusion: ‘This is the only truth, anything else is futile.’ If a person has a belief … or has received an oral transmission … or has a reasoned reflection about something … or has accepted a view after contemplation, they preserve truth by saying, ‘Such is the view I have accepted after contemplation.’ But they don’t yet come to the categorical conclusion: ‘This is the only truth, anything else is futile.’ That’s how the preservation of truth is defined, \textsanskrit{Bhāradvāja}. I describe the preservation of truth as defined in this way.\footnote{This is essentially the standard used in references today: one should accurately represent one’s sources. } But this is not yet the awakening to the truth.” 

“That’s\marginnote{16.1} how the preservation of truth is defined, Mister Gotama. We regard the preservation of truth as defined in this way. But Mister Gotama, how do you define awakening to the truth?” 

“\textsanskrit{Bhāradvāja},\marginnote{17.1} take the case of a mendicant living supported by a town or village. A householder or their child approaches and scrutinizes them for three kinds of things: things that arouse greed, things that provoke hate, and things that promote delusion.\footnote{For these phrases, compare \href{https://suttacentral.net/an4.117/en/sujato}{AN 4.117} and \href{https://suttacentral.net/an5.144/en/sujato}{AN 5.144}. } ‘Does this venerable have any qualities that arouse greed? Such qualities that, were their mind to be overwhelmed by them, they might say that they know, even though they don’t know, or that they see, even though they don’t see; or that they might encourage others to do what is for their lasting harm and suffering?’ Scrutinizing them they find: ‘This venerable has no such qualities that arouse greed. Rather, that venerable has bodily and verbal behavior like that of someone without greed. And the principle that they teach is deep, hard to see, hard to understand, peaceful, sublime, beyond the scope of logic, subtle, comprehensible to the astute. It’s not easy for someone with greed to teach this.’ 

Scrutinizing\marginnote{18.1} them in this way they see that they are purified of qualities that arouse greed. Next, they search them for qualities that provoke hate. ‘Does this venerable have any qualities that provoke hate? Such qualities that, were their mind to be overwhelmed by them, they might say that they know, even though they don’t know, or that they see, even though they don’t see; or that they might encourage others to do what is for their lasting harm and suffering?’ Scrutinizing them they find: ‘This venerable has no such qualities that provoke hate. Rather, that venerable has bodily and verbal behavior like that of someone without hate. And the principle that they teach is deep, hard to see, hard to understand, peaceful, sublime, beyond the scope of logic, subtle, comprehensible to the astute. It’s not easy for someone with hate to teach this.’ 

Scrutinizing\marginnote{19.1} them in this way they see that they are purified of qualities that provoke hate. Next, they scrutinize them for qualities that promote delusion. ‘Does this venerable have any qualities that promote delusion? Such qualities that, were their mind to be overwhelmed by them, they might say that they know, even though they don’t know, or that they see, even though they don’t see; or that they might encourage others to do what is for their lasting harm and suffering?’ Scrutinizing them they find: ‘This venerable has no such qualities that promote delusion. Rather, that venerable has bodily and verbal behavior like that of someone without delusion. And the principle that they teach is deep, hard to see, hard to understand, peaceful, sublime, beyond the scope of logic, subtle, comprehensible to the astute. It’s not easy for someone with delusion to teach this.’ 

Scrutinizing\marginnote{20.1} them in this way they see that they are purified of qualities that promote delusion. Next, they place faith in them. When faith has arisen they approach the teacher. They pay homage, actively listen, hear the teachings, remember the teachings, reflect on their meaning, and accept them after deliberation. Then enthusiasm springs up; they apply zeal, weigh up, and strive. Striving, they directly realize the ultimate truth, and see it with penetrating wisdom.\footnote{Here we see the meaning of the Buddhist “grounded faith” (\textit{\textsanskrit{ākāravatī} \textsanskrit{saddhā}}, \href{https://suttacentral.net/mn47/en/sujato\#16.1}{MN 47:16.1}, \href{https://suttacentral.net/mn60/en/sujato\#4.1}{MN 60:4.1}) as opposed to “blind faith”. But faith is only the beginning of the long and demanding process described here. } That’s how the awakening to truth is defined, \textsanskrit{Bhāradvāja}. I describe the awakening to truth as defined in this way. But this is not yet the attainment of truth.” 

“That’s\marginnote{21.1} how the awakening to truth is defined, Mister Gotama. I regard the awakening to truth as defined in this way. But Mister Gotama, how do you define the attainment of truth?” 

“By\marginnote{21.4} the cultivation, development, and making much of these very same things there is the attainment of truth.\footnote{Typically this distinction refers to the stream-enterer and the arahant. } That’s how the attainment of truth is defined, \textsanskrit{Bhāradvāja}. I describe the attainment of truth as defined in this way.” 

“That’s\marginnote{22.1} how the attainment of truth is defined, Mister Gotama. I regard the attainment of truth as defined in this way. But what quality is helpful for arriving at the truth?” 

“Striving\marginnote{22.4} is helpful for arriving at the truth. If you don’t strive, you won’t arrive at the truth. You arrive at the truth because you strive. That’s why striving is helpful for arriving at the truth.” 

“But\marginnote{23.1} what quality is helpful for striving?” 

“Weighing\marginnote{23.3} up the teachings is helpful for striving …\footnote{\textit{\textsanskrit{Tulanā}} is literally “weighing”, here in the applied sense of “evaluating”. } 

Zeal\marginnote{24.1} is helpful for weighing up the teachings …\footnote{Here, “zeal” (\textit{\textsanskrit{ussāha}}) is making an effort to understand the teachings, while “striving” (\textit{\textsanskrit{padhāna}}) is making an effort in meditation. } 

Enthusiasm\marginnote{25.1} is helpful for zeal … 

Acceptance\marginnote{26.1} of the teachings after deliberation is helpful for enthusiasm … 

Reflecting\marginnote{27.1} on the meaning of the teachings is helpful for accepting them after deliberation … 

Remembering\marginnote{28.1} the teachings is helpful for reflecting on their meaning … 

Hearing\marginnote{29.1} the teachings is helpful for remembering the teachings … 

Active\marginnote{30.1} listening is helpful for hearing the teachings … 

Paying\marginnote{31.1} homage is helpful for active listening … 

Approaching\marginnote{32.1} is helpful for paying homage … 

Faith\marginnote{33.1} is helpful for approaching a teacher. If you don’t give rise to faith, you won’t approach a teacher. You approach a teacher because you have faith. That’s why faith is helpful for approaching a teacher.” 

“I’ve\marginnote{34.1} asked Mister Gotama about the preservation of truth, and he has answered me. I endorse and accept this, and am satisfied with it. I’ve asked Mister Gotama about awakening to the truth, and he has answered me. I endorse and accept this, and am satisfied with it. I’ve asked Mister Gotama about the attainment of truth, and he has answered me. I endorse and accept this, and am satisfied with it. I’ve asked Mister Gotama about the things that are helpful for the attainment of truth, and he has answered me. I endorse and accept this, and am satisfied with it. Whatever I have asked Mister Gotama about he has answered me. I endorse and accept this, and am satisfied with it. 

Mister\marginnote{34.11} Gotama, I used to think this: ‘Who are these shavelings, fake ascetics, primitives, black spawn from the feet of our kinsman next to those who understand the teaching?’ The Buddha has inspired me to have love, confidence, and respect for ascetics! 

Excellent,\marginnote{35.1} Mister Gotama! … From this day forth, may Mister Gotama remember me as a lay follower who has gone for refuge for life.”\footnote{That \textsanskrit{Kāpaṭika} should sincerely go forth does not conflict with him being a spy; it is a conversion story. In authoritarian countries today, spies are regularly placed in monasteries for the same reason. There are stories of how, after being placed for many years and seeing not a hotbed of sedition but a place for wisdom and goodness, such spies feel shame and genuinely convert to Buddhism. } 

%
\section*{{\suttatitleacronym MN 96}{\suttatitletranslation With Esukārī }{\suttatitleroot Esukārīsutta}}
\addcontentsline{toc}{section}{\tocacronym{MN 96} \toctranslation{With Esukārī } \tocroot{Esukārīsutta}}
\markboth{With Esukārī }{Esukārīsutta}
\extramarks{MN 96}{MN 96}

\scevam{So\marginnote{1.1} I have heard. }At one time the Buddha was staying near \textsanskrit{Sāvatthī} in Jeta’s Grove, \textsanskrit{Anāthapiṇḍika}’s monastery. 

Then\marginnote{2.1} \textsanskrit{Esukārī} the brahmin went up to the Buddha, and exchanged greetings with him.\footnote{This brahmin appears only here, and there is little to distinguish him personally. A king of the same name features in a \textsanskrit{Jātaka} (\href{https://suttacentral.net/ja509/en/sujato}{Ja 509}). Could \textit{\textsanskrit{esukārī}} be equivalent to \textit{\textsanskrit{ayaskārī}}, “blacksmith”? } When the greetings and polite conversation were over, he sat down to one side and said to the Buddha: 

“Mister\marginnote{3.1} Gotama, the brahmins prescribe four kinds of service:\footnote{To “serve” can be to look after, nurse, or care for (\href{https://suttacentral.net/an5.76/en/sujato\#3.5}{AN 5.76:3.5}, \href{https://suttacentral.net/dn31/en/sujato\#29.2}{DN 31:29.2}), to offer religious devotions (\href{https://suttacentral.net/mn12/en/sujato\#61.2}{MN 12:61.2}), or even provide sexual favors (\href{https://suttacentral.net/pli-tv-bu-vb-ss4/en/sujato\#1.2.7}{Bu Ss 4:1.2.7}). } for a brahmin, an aristocrat, a peasant, and a menial. This is the service they prescribe for a brahmin: ‘A brahmin, an aristocrat, a peasant, and a menial may all serve a brahmin.’\footnote{\textsanskrit{Manusmṛti} 2.238 ff. allows that in abnormal times a brahmin may receive teachings from lower castes, except menials. } This is the service they prescribe for an aristocrat: ‘An aristocrat, a peasant, and a menial may all serve an aristocrat.’ This is the service they prescribe for a peasant: ‘A peasant or a menial may serve a peasant.’ This is the service they prescribe for a menial: ‘Only a menial may serve a menial. For who else will serve a menial?’ These are the four kinds of service that the brahmins prescribe. What do you say about this?” 

“But\marginnote{4.1} brahmin, did the whole world authorize the brahmins to prescribe these four kinds of service?” 

“No,\marginnote{4.2} Mister Gotama.” 

“It’s\marginnote{4.3} as if they were to force a chop on a poor, penniless person, telling them, ‘Eat this meat and pay for it!’ In the same way, the brahmins have prescribed these four kinds of service without the consent of those ascetics and brahmins.\footnote{The reference to “these” ascetics and brahmins intrudes here, since none have been mentioned. In the parallel passage at \href{https://suttacentral.net/an6.57/en/sujato\#9.4}{AN 6.57:9.4} it fits, so it may well have come from there. } 

Brahmin,\marginnote{5.1} I don’t say that you should serve everyone, nor do I say that you shouldn’t serve anyone. I say that you shouldn’t serve someone if serving them makes you worse, not better. And I say that you should serve someone if serving them makes you better, not worse. 

If\marginnote{6.1} they were to ask an aristocrat this, ‘Who should you serve? Someone in whose service you get worse, or someone in whose service you get better?’ Answering rightly, an aristocrat would say, ‘Someone in whose service I get better.’ 

If\marginnote{7.1} they were to ask a brahmin … a peasant … or a menial this, ‘Who should you serve? Someone in whose service you get worse, or someone in whose service you get better?’ Answering rightly, a menial would say, ‘Someone in whose service I get better.’ 

Brahmin,\marginnote{7.7} I don’t say that coming from an eminent family makes you a better or worse person. I don’t say that being very beautiful makes you a better or worse person.\footnote{The commentary explains (\textit{\textsanskrit{uḷāravaṇṇatā}}) as “high class”. That may be correct, but normally in Pali it refers to beauty (\href{https://suttacentral.net/dn18/en/sujato\#10.1}{DN 18:10.1}, \href{https://suttacentral.net/ja458/en/sujato\#15.3}{Ja 458:15.3}, \href{https://suttacentral.net/vv29/en/sujato\#1.1}{Vv 29:1.1}). } I don’t say that being very wealthy makes you a better or worse person. 

For\marginnote{8.1} some people from eminent families kill living creatures, steal, and commit sexual misconduct. They use speech that’s false, divisive, harsh, or nonsensical. And they’re covetous, malicious, with wrong view. That’s why I don’t say that coming from an eminent family makes you an even truer person. 

But\marginnote{8.3} some people from eminent families also refrain from killing living creatures, stealing, and committing sexual misconduct. They refrain from using speech that’s false, divisive, harsh, or nonsensical. And they’re not covetous or malicious, and they have right view. That’s why I don’t say that coming from an eminent family makes you a worse person. 

People\marginnote{8.5} who are very beautiful, or not very beautiful, who are very wealthy, or not very wealthy, may also behave in the same ways. That’s why I don’t say that any of these things makes you a better or worse person. 

Brahmin,\marginnote{9.1} I don’t say that you should serve everyone, nor do I say that you shouldn’t serve anyone. And I say that you should serve someone if serving them makes you grow in faith, ethics, learning, generosity, and wisdom. I say that you shouldn’t serve someone if serving them doesn’t make you grow in faith, ethics, learning, generosity, and wisdom.” 

When\marginnote{10.1} he had spoken, \textsanskrit{Esukārī} said to him: 

“Mister\marginnote{10.2} Gotama, the brahmins prescribe four kinds of wealth: for a brahmin, an aristocrat, a peasant, and a menial.\footnote{Within limits, one from a higher caste may adopt the livelihood of a lower, but not the other way around (\textsanskrit{Vāsiṣṭha} \textsanskrit{Dharmasūtra} 2.22–3). } The wealth they prescribe for a brahmin is living on alms.\footnote{“Living on alms” (\textit{\textsanskrit{bhikkhācariya}}, Sanskrit \textit{\textsanskrit{bhikṣācarya}}) is acknowledged as the best livelihood for the sages of old at \textsanskrit{Bṛhadāraṇyaka} \textsanskrit{Upaniṣad} 3.5.1, 4.4.22. The later \textsanskrit{Vāsiṣṭha} \textsanskrit{Dharmasūtra} 2.13 says there are six occupations for a brahmin: study, recitation, sacrificing for oneself, or for others, giving, and receiving gifts. \textsanskrit{Manusmṛti} 1.88 is similar. } A brahmin who scorns his own wealth, living on alms, fails in his duty like a guard who steals. The wealth they prescribe for an aristocrat is the bow and quiver.\footnote{\textsanskrit{Vāsiṣṭha} \textsanskrit{Dharmasūtra} 2.16 and 2.18 say aristocrats and peasants share three the same as brahmins—recital, sacrifice, and giving. Their special livelihood is the use of weapons. \textsanskrit{Manusmṛti} 1.89 says more diplomatically “protection of the people” (\textit{\textsanskrit{prajānāṁ} \textsanskrit{rakṣaṇaṁ}}). } An aristocrat who scorns his own wealth, the bow and quiver, fails in his duty like a guard who steals. The wealth they prescribe for a peasant is farming and animal husbandry.\footnote{\textsanskrit{Vāsiṣṭha} \textsanskrit{Dharmasūtra} 2.19 says farming, trade, husbandry, and money-lending. This reflects a more developed economy in this later text. \textsanskrit{Manusmṛti} 1.90 is similar. } A peasant who scorns his own wealth, farming and animal husbandry, fails in his duty like a guard who steals. The wealth they prescribe for a menial is the scythe and flail.\footnote{\textsanskrit{Vāsiṣṭha} \textsanskrit{Dharmasūtra} 2.20 says that the only work of a menial is to serve the other castes (\textit{\textsanskrit{teṣāṁ} \textsanskrit{paricaryā}}). \textsanskrit{Manusmṛti} 1.91 is similar, emphasizing “unbegrudged obedience” (\textit{\textsanskrit{śuśrūṣāmanasūyayā}}). } A menial who scorns his own wealth, the scythe and flail, fails in his duty like a guard who steals. These are the four kinds of wealth that the brahmins prescribe. What do you say about this?” 

“But\marginnote{11.1} brahmin, did the whole world authorize the brahmins to prescribe these four kinds of wealth?” 

“No,\marginnote{11.2} Mister Gotama.” 

“It’s\marginnote{11.3} as if they were to force a chop on a poor, penniless person, telling them, ‘Eat this meat and pay for it!’ 

In\marginnote{11.4} the same way, the brahmins have prescribed these four kinds of wealth without the consent of these ascetics and brahmins. 

I\marginnote{12.1} declare that a person’s own wealth is the noble, transcendent teaching.\footnote{For \textit{\textsanskrit{lokuttaraṁ} \textsanskrit{dhammaṁ}} as “transcendental teaching”, see \href{https://suttacentral.net/sn20.7/en/sujato\#1.6}{SN 20.7:1.6}, \href{https://suttacentral.net/an2.47/en/sujato\#1.5}{AN 2.47:1.5}, \href{https://suttacentral.net/an5.79/en/sujato\#5.2}{AN 5.79:5.2}. } But they are reckoned by recollecting the traditional family lineage of their mother and father wherever they are incarnated. If they incarnate in a family of aristocrats they are reckoned as an aristocrat. If they incarnate in a family of brahmins they are reckoned as a brahmin. If they incarnate in a family of peasants they are reckoned as a peasant. If they incarnate in a family of menials they are reckoned as a menial. 

It’s\marginnote{12.7} like fire, which is reckoned according to the very same condition dependent upon which it burns. A fire that burns dependent on logs is reckoned as a log fire. A fire that burns dependent on twigs is reckoned as a twig fire. A fire that burns dependent on grass is reckoned as a grass fire. A fire that burns dependent on cow-dung is reckoned as a cow-dung fire. 

In\marginnote{12.12} the same way, I declare that a person’s own wealth is the noble, transcendent teaching. But they are reckoned by recollecting the traditional family lineage of their mother and father wherever they are incarnated. 

Suppose\marginnote{13.1} someone from a family of aristocrats goes forth from the lay life to homelessness. Relying on the teaching and training proclaimed by the Realized One they refrain from killing living creatures, stealing, and sex. They refrain from using speech that’s false, divisive, harsh, or nonsensical. And they’re not covetous or malicious, and they have right view. They succeed in the system of the skillful teaching. 

Suppose\marginnote{13.2} someone from a family of brahmins … peasants … menials goes forth from the lay life to homelessness. Relying on the teaching and training proclaimed by the Realized One … they succeed in the system of the skillful teaching. 

What\marginnote{14.1} do you think, brahmin? Is only a brahmin capable of developing a heart of love free of enmity and ill will for this region, and not an aristocrat, peasant, or menial?” 

“No,\marginnote{14.3} Mister Gotama. Aristocrats, brahmins, peasants, and menials can all do so. For all four classes are capable of developing a heart of love free of enmity and ill will for this region.” 

“In\marginnote{14.9} the same way, suppose someone from a family of aristocrats, brahmins, peasants, or menials goes forth from the lay life to homelessness. Relying on the teaching and training proclaimed by the Realized One … they succeed in the system of the skillful teaching. 

What\marginnote{15.1} do you think, brahmin? Is only a brahmin capable of taking some bathing cleanser, going to the river, and washing off dust and dirt, and not an aristocrat, peasant, or menial?”\footnote{This is an allusion to the Brahmanical practice of ritual bathing. See \href{https://suttacentral.net/mn7/en/sujato\#18.5}{MN 7:18.5} and notes there. } 

“No,\marginnote{15.3} Mister Gotama. All four classes are capable of doing this.” 

“In\marginnote{15.9} the same way, suppose someone from a family of aristocrats, brahmins, peasants, or menials goes forth from the lay life to homelessness. Relying on the teaching and training proclaimed by the Realized One … they succeed in the system of the skillful teaching. 

What\marginnote{16.1} do you think, brahmin? Suppose an anointed aristocratic king were to gather a hundred people born in different classes and say to them: ‘Please gentlemen, let anyone here who was born in a family of aristocrats, brahmins, or chieftains take a drill-stick made of teak, sal, frankincense wood, sandalwood, or cherry wood, light a fire and produce heat. And let anyone here who was born in a family of corpse-workers, hunters, bamboo-workers, chariot-makers, or scavengers take a drill-stick made from a dog’s drinking trough, a pig’s trough, a dustbin, or castor-oil wood, light a fire and produce heat.’ 

What\marginnote{16.5} do you think, brahmin? Would only the fire produced by the high class people with good quality wood have flames, color, and radiance, and be usable as fire, and not the fire produced by the low class people with poor quality wood?” 

“No,\marginnote{16.8} Mister Gotama. The fire produced by the high class people with good quality wood would have flames, color, and radiance, and be usable as fire, and so would the fire produced by the low class people with poor quality wood. For all fire has flames, color, and radiance, and is usable as fire.” 

“In\marginnote{16.12} the same way, suppose someone from a family of aristocrats, brahmins, peasants, or menials goes forth from the lay life to homelessness. Relying on the teaching and training proclaimed by the Realized One they refrain from killing living creatures, stealing, and sex. They refrain from using speech that’s false, divisive, harsh, or nonsensical. And they’re not covetous or malicious, and they have right view. They succeed in the system of the skillful teaching.” 

When\marginnote{17.1} he had spoken, \textsanskrit{Esukārī} said to him, “Excellent, Mister Gotama! Excellent! … From this day forth, may Mister Gotama remember me as a lay follower who has gone for refuge for life.” 

%
\section*{{\suttatitleacronym MN 97}{\suttatitletranslation With Dhanañjāni }{\suttatitleroot Dhanañjānisutta}}
\addcontentsline{toc}{section}{\tocacronym{MN 97} \toctranslation{With Dhanañjāni } \tocroot{Dhanañjānisutta}}
\markboth{With Dhanañjāni }{Dhanañjānisutta}
\extramarks{MN 97}{MN 97}

\scevam{So\marginnote{1.1} I have heard. }At one time the Buddha was staying near \textsanskrit{Rājagaha}, in the Bamboo Grove, the squirrels’ feeding ground. 

Now\marginnote{2.1} at that time Venerable \textsanskrit{Sāriputta} was wandering in the Southern Hills together with a large \textsanskrit{Saṅgha} of mendicants.\footnote{The Southern Hills (\textit{\textsanskrit{dakkhiṇāgiri}}, “Deccan”) is the Vindhya Range that lies on the southern border of Magadha, leading southwest to Avanti. The Buddha and his disciples visited there so rarely that people complained (\href{https://suttacentral.net/pli-tv-kd1/en/sujato\#53.1.3}{Kd 1:53.1.3}). Nonetheless, the present narrative suggests that \textsanskrit{Sāriputta} had stayed there for an extended period of time, long enough to lose touch with his friends back home. } Then a certain mendicant who had completed the rainy season residence in \textsanskrit{Rājagaha} went to the Southern Hills, where he approached Venerable \textsanskrit{Sāriputta}, and exchanged greetings with him. When the greetings and polite conversation were over, he sat down to one side. \textsanskrit{Sāriputta} said to him, “Reverend, I hope the Buddha is healthy and strong?” 

“He\marginnote{2.5} is, reverend.” 

“And\marginnote{2.6} I hope that the mendicant \textsanskrit{Saṅgha} is healthy and strong.” 

“It\marginnote{2.7} is.” 

“Reverend,\marginnote{2.8} at the rice checkpoint at the city gate there is a brahmin named \textsanskrit{Dhanañjāni}.\footnote{The “rice checkpoint at the city gate” (\textit{\textsanskrit{taṇḍulapālidvārā}}) was a gate of \textsanskrit{Rājagaha} where rice would be inspected and tolls collected. Here \textit{\textsanskrit{pāli}} has the sense “guard” (cp. Sanskrit \textit{\textsanskrit{pālin}}). This was not \textsanskrit{Dhanañjāni}’s residence, but where he worked and, it would seem, skimmed off funds for his own benefit. | \textsanskrit{Kauṭilya} discusses corruption among government servants (\textsanskrit{Arthaśāstra} 2.9), sets out the duties of the toll collector (2.21), and lists the various tolls (2.22). } I hope that he is healthy and strong?” 

“He\marginnote{2.10} too is well.” 

“But\marginnote{2.11} is he diligent?” 

“How\marginnote{2.12} could he possibly be diligent? \textsanskrit{Dhanañjāni} robs the brahmins and householders in the name of the king, and he robs the king in the name of the brahmins and householders. His wife, a lady of faith who he married from a family of faith, has passed away.\footnote{A faithful laywoman named \textsanskrit{Dhanañjānī} features in \href{https://suttacentral.net/mn100/en/sujato}{MN 100} and \href{https://suttacentral.net/sn7.1/en/sujato}{SN 7.1}, but this was a different person. The name appears to be a Pali variation of the Vedic \textit{\textsanskrit{dhanañjaya}}, “winner of wealth”. } And he has taken a new wife who has no faith.” 

“Oh,\marginnote{2.16} it’s bad news to hear that \textsanskrit{Dhanañjāni} is negligent. Hopefully, some time or other I’ll get to meet him, and we can have a discussion.” 

When\marginnote{3.1} \textsanskrit{Sāriputta} had stayed in the Southern Hills as long as he pleased, he set out for \textsanskrit{Rājagaha}. Traveling stage by stage, he arrived at \textsanskrit{Rājagaha}, where he stayed in the Bamboo Grove, the squirrels’ feeding ground. 

Then\marginnote{4.1} he robed up in the morning and, taking his bowl and robe, entered \textsanskrit{Rājagaha} for alms. Now at that time \textsanskrit{Dhanañjāni} was having his cows milked in a cow-shed outside the city. Then \textsanskrit{Sāriputta} wandered for alms in \textsanskrit{Rājagaha}. After the meal, on his return from almsround, he approached \textsanskrit{Dhanañjāni}. 

Seeing\marginnote{4.4} \textsanskrit{Sāriputta} coming off in the distance, \textsanskrit{Dhanañjāni} went to him and said, “Here, Mister \textsanskrit{Sāriputta}, drink some fresh milk before the meal time.” 

“Enough,\marginnote{4.7} brahmin, I’ve finished eating for today. I shall be at the root of that tree for the day’s meditation. Come see me there.” 

“Yes,\marginnote{4.11} sir,” replied \textsanskrit{Dhanañjāni}. 

When\marginnote{5.1} \textsanskrit{Dhanañjāni} had finished breakfast he went to \textsanskrit{Sāriputta} and exchanged greetings with him. When the greetings and polite conversation were over, he sat down to one side. \textsanskrit{Sāriputta} said to him, “I hope you’re diligent, \textsanskrit{Dhanañjāni}?” 

“How\marginnote{5.4} can I possibly be diligent, Mister \textsanskrit{Sāriputta}? I have to provide for my mother and father, my wives and children, and my bondservants and workers. And I have to make the proper offerings to friends and colleagues, relatives and kin, guests, ancestors, deities, and king. And then this body must also be fattened and built up.”\footnote{The five “proper offerings” correspond with the five to whom \textit{bali} offerings are made (\href{https://suttacentral.net/an4.61/en/sujato\#15.2}{AN 4.61:15.2}, \href{https://suttacentral.net/an5.41/en/sujato\#4.2}{AN 5.41:4.2}). | In agreement with \textit{\textsanskrit{brūhetabbo}} (“to be built up”), \textit{\textsanskrit{pīṇetabbo}} is from \textit{\textsanskrit{pīṇa}} in the sense “fatten” rather than “gladden”. } 

“What\marginnote{6.1} do you think, \textsanskrit{Dhanañjāni}? Suppose someone was to behave in an unprincipled and unjust way for the sake of their parents. Because of this the wardens of hell would drag them to hell. Could they get out of being dragged to hell by pleading that they had acted for the sake of their parents? Or could their parents save them by pleading that the acts had been done for their sake?” 

“No,\marginnote{6.4} Mister \textsanskrit{Sāriputta}. Rather, even as they were wailing the wardens of hell would cast them down into hell.” 

“What\marginnote{7.1} do you think, \textsanskrit{Dhanañjāni}? Suppose someone was to behave in an unprincipled and unjust way for the sake of their wives and children … bondservants and workers … friends and colleagues … relatives and kin … guests … ancestors … deities … king … fattening and building up their body. Because of this the wardens of hell would drag them to hell. Could they get out of being dragged to hell by pleading that they had acted for the sake of fattening and building up their body? Or could anyone else save them by pleading that the acts had been done for that reason?” 

“No,\marginnote{15.3} Mister \textsanskrit{Sāriputta}. Rather, even as they were wailing the wardens of hell would cast them down into hell.” 

“Who\marginnote{16.1} do you think is better, \textsanskrit{Dhanañjāni}? Someone who, for the sake of their parents, behaves in an unprincipled and unjust manner, or someone who behaves in a principled and just manner?” 

“Someone\marginnote{16.3} who behaves in a principled and just manner for the sake of their parents. For principled and moral conduct is better than unprincipled and immoral conduct.” 

“\textsanskrit{Dhanañjāni},\marginnote{16.6} others have livelihoods that are both profitable and legitimate. By means of these it’s possible to provide for your parents, avoid bad deeds, and practice the path of goodness. 

Who\marginnote{17.1} do you think is better, \textsanskrit{Dhanañjāni}? Someone who, for the sake of their wives and children … bondservants and workers … friends and colleagues … relatives and kin … guests … ancestors … deities … king … fattening and building up their body, behaves in an unprincipled and unjust manner, or someone who behaves in a principled and just manner?” 

“Someone\marginnote{25.3} who behaves in a principled and just manner. For principled and moral conduct is better than unprincipled and immoral conduct.” 

“\textsanskrit{Dhanañjāni},\marginnote{25.6} others have livelihoods that are both profitable and legitimate. By means of these it’s possible to fatten and build up your body, avoid bad deeds, and practice the path of goodness.” 

Then\marginnote{26.1} \textsanskrit{Dhanañjāni} the brahmin, having approved and agreed with what Venerable \textsanskrit{Sāriputta} said, got up from his seat and left. 

Some\marginnote{27.1} time later \textsanskrit{Dhanañjāni} became sick, suffering, gravely ill. Then he addressed a man, “Please, mister, go to the Buddha, and in my name bow with your head to his feet. Say to him: ‘Sir, the brahmin \textsanskrit{Dhanañjāni} is sick, suffering, gravely ill. He bows with his head to your feet.’ Then go to Venerable \textsanskrit{Sāriputta}, and in my name bow with your head to his feet. Say to him: ‘Sir, the brahmin \textsanskrit{Dhanañjāni} is sick, suffering, gravely ill. He bows with his head to your feet.’ And then say: ‘Sir, please visit \textsanskrit{Dhanañjāni} at his home out of sympathy.’” 

“Yes,\marginnote{27.11} sir,” that man replied. He did as \textsanskrit{Dhanañjāni} asked. \textsanskrit{Sāriputta} consented with silence. 

He\marginnote{28.1} robed up, and, taking his bowl and robe, went to \textsanskrit{Dhanañjāni}’s home, where he sat on the seat spread out and said to \textsanskrit{Dhanañjāni}, “I hope you’re keeping well, \textsanskrit{Dhanañjāni}; I hope you’re all right. And I hope the pain is fading, not growing, that its fading is evident, not its growing.” 

“I’m\marginnote{29.1} not keeping well, Mister \textsanskrit{Sāriputta}, I’m not getting by. The pain is terrible and growing, not fading; its growing is evident, not its fading. The winds piercing my head are so severe, it feels like a strong man drilling into my head with a sharp point. I’m not keeping well. The pain in my head is so severe, it feels like a strong man tightening a tough leather strap around my head. I’m not keeping well. The winds slicing my belly are so severe, like a deft butcher or their apprentice were slicing open a cows’s belly with a sharp meat cleaver. I’m not keeping well. The burning in my body is so severe, it feels like two strong men grabbing a weaker man by the arms to burn and scorch him on a pit of glowing coals. I’m not keeping well, Mister \textsanskrit{Sāriputta}, I’m not getting by. The pain is terrible and growing, not fading; its growing is evident, not its fading.” 

“\textsanskrit{Dhanañjāni},\marginnote{30.1} which do you think is better: hell or the animal realm?” 

“The\marginnote{30.3} animal realm is better.” 

“Which\marginnote{30.4} do you think is better: the animal realm or the ghost realm?” 

“The\marginnote{30.6} ghost realm is better.” 

“Which\marginnote{30.7} do you think is better: the ghost realm or human life?” 

“Human\marginnote{30.9} life is better.” 

“Which\marginnote{30.10} do you think is better: human life or as one of the gods of the four great kings?” 

“The\marginnote{30.12} gods of the four great kings.”\footnote{The “gods of the four great kings” (\textit{\textsanskrit{cātumahārājikā} \textsanskrit{devā}}) are deities born in a realm subject to the overlords known as the Four Great Kings: Kuvera (also called \textsanskrit{Vessavaṇa}) lord of spirits (\textit{yakkha}) in the north; \textsanskrit{Dhataraṭṭha} lord of centaurs (\textit{gandhabba}) in the east; \textsanskrit{Virūḷha} lord of goblins (\textit{\textsanskrit{kumbhaṇḍa}}) in the south; and \textsanskrit{Virūpakkha} lord of dragons (\textit{\textsanskrit{nāga}}) in the west (see \href{https://suttacentral.net/dn32/en/sujato}{DN 32}). Sanskrit sources sometimes multiply their number to eight (eg. \textsanskrit{Śivapurāṇa} 2.2.22.43). } 

“Which\marginnote{30.13} do you think is better: the gods of the four great kings or the gods of the thirty-three?” 

“The\marginnote{30.15} gods of the thirty-three.”\footnote{The “gods of the thirty-three” (\textit{\textsanskrit{tāvatiṁsā} \textsanskrit{devā}}), who are subjects of Sakka (also called Indra), enjoy refined sensual delights. The number is a reduplication of the trinity. In Buddhist texts they are not enumerated, but \textsanskrit{Yājñavalkya} reckons them as eight Vasus, eleven Rudras, twelve Ādityas, plus Indra and \textsanskrit{Prajāpati} (\textsanskrit{Bṛhadāraṇyaka} \textsanskrit{Upaniṣad} 3.9.2). The final pair are elsewhere said to be Dyaus (“Heaven” = Zeus) and \textsanskrit{Pṛthivī} (“Earth”), or the twin \textsanskrit{Aśvins}. } 

“Which\marginnote{30.16} do you think is better: the gods of the thirty-three or the gods of Yama?” 

“The\marginnote{30.18} gods of Yama.”\footnote{Gods in this realm (spelled \textit{\textsanskrit{yāma}}, “of Yama”) are subjects of Yama, the god who guards the paths to the land of the dead (\href{https://suttacentral.net/sn1.33/en/sujato\#10.3}{SN 1.33:10.3}, \href{https://suttacentral.net/mn130/en/sujato\#5.1}{MN 130:5.1}). He was the son of Vivasvant (the sun) and brother of Manu, and as the first person to die, Rig Veda 10.4 tells us that he discovered the paths to the lands of the dead that he now guards with his pair of hounds. Yama is a righteous god, but sinners see him as horrifying due to their own guilt (Brahma \textsanskrit{Purāṇa} 106.46). } 

“Which\marginnote{30.19} do you think is better: the gods of Yama or the joyful gods?” 

“The\marginnote{30.21} joyful gods.”\footnote{The “joyful gods” (\textit{\textsanskrit{tusitā} \textsanskrit{devā}}) were known to the Jains, who classed them among the gods of the world’s end (\textit{\textsanskrit{lokāntika}}; \textsanskrit{Tattvārthasūtra} 4.25; \textsanskrit{Viyāhapaṇṇatti} 5.3.95). Later Hindu tradition says they were the sons of \textsanskrit{Kaśyapa} through Aditi, otherwise known as the twelve Ādityas: \textsanskrit{Viṣṇu}, Śakra, \textsanskrit{Tvaṣṭṛ}, \textsanskrit{Dhātṛ}, Aryaman, \textsanskrit{Pūṣan}, Vivasvat, \textsanskrit{Savitṛ}, Mitra, \textsanskrit{Varuṇa}, Bhaga, and \textsanskrit{Aṁśu}. (\textsanskrit{Agnipurāṇa} 19.1–3, \textsanskrit{Liṅgapurāṇa} 63.22b–26). In Buddhism, this realm is famed as the final home of Bodhisattas before they descend to the human realm to become Buddhas. } 

“Which\marginnote{30.22} do you think is better: the joyful gods or the gods who love to imagine?” 

“The\marginnote{30.24} gods who love to imagine.”\footnote{These deities (\textit{\textsanskrit{nimmānaratī} \textsanskrit{devā}}) are said to create whatever delights they wish and enjoy them, living in a world of their own projection. They seem to be primarily Buddhist, perhaps influenced by such passages as \textsanskrit{Bṛhadāraṇyaka} \textsanskrit{Upaniṣad} 4.3.9, which says that in dreams one creates a new body that is revealed by one’s inner light. They are, however, mentioned in later texts such as \textsanskrit{Viṣṇupurāṇa} 3.2.30 as one of three groups of thirty, and \textsanskrit{Mārkaṇḍeyapurāṇa} 44, which adds that they preside over the months, seasons, and days. } 

“Which\marginnote{30.25} do you think is better: the gods who love to imagine or the gods who control what is imagined by others?” 

“The\marginnote{30.27} gods who control what is imagined by others.”\footnote{Deities in this, the highest of the sensual realms, control the visions seen by others (\textit{\textsanskrit{paranimmitavasavattī}}). Their most famous member is \textsanskrit{Māra}, who uses this skill for wicked ends. Again they seem primarily Buddhist, although we probably hear an echo of them in \textsanskrit{Mahābhārata} 12,325.004E, which lists the \textit{parinirmita} and \textit{\textsanskrit{vaśavartin}} as two separate groups. } 

“Which\marginnote{31.1} do you think is better: the gods who control what is imagined by others or the realm of divinity?” 

“Mister\marginnote{31.3} \textsanskrit{Sāriputta} speaks of the realm of divinity! Mister \textsanskrit{Sāriputta} speaks of the realm of divinity!” 

Then\marginnote{31.5} \textsanskrit{Sāriputta} thought: 

“These\marginnote{31.6} brahmins are devoted to the realm of divinity. Why don’t I teach him a path to the company of Divinity?” 

“\textsanskrit{Dhanañjāni},\marginnote{31.8} I shall teach you a path to the company of Divinity. Listen and apply your mind well, I will speak.” 

“Yes,\marginnote{31.10} sir,” replied \textsanskrit{Dhanañjāni}. Venerable \textsanskrit{Sāriputta} said this: 

“And\marginnote{32.1} what is a path to company with Divinity?\footnote{While it is possible to be reborn in even the highest of the sensual realms through the power of ordinary goodness, rebirth in the \textsanskrit{Brahmā} and higher realms requires the development of absorption meditation. } Firstly, a mendicant meditates spreading a heart full of love to one direction, and to the second, and to the third, and to the fourth. In the same way above, below, across, everywhere, all around, they spread a heart full of love to the whole world—abundant, expansive, limitless, free of enmity and ill will. This is a path to company with Divinity. 

Furthermore,\marginnote{33{-}35.1} a mendicant meditates spreading a heart full of compassion … 

They\marginnote{33{-}35.2} meditate spreading a heart full of rejoicing … 

They\marginnote{33{-}35.3} meditate spreading a heart full of equanimity to one direction, and to the second, and to the third, and to the fourth. In the same way above, below, across, everywhere, all around, they spread a heart full of equanimity to the whole world—abundant, expansive, limitless, free of enmity and ill will. This is a path to company with Divinity.” 

“Well\marginnote{36.1} then, Mister \textsanskrit{Sāriputta}, in my name bow with your head at the Buddha’s feet. Say to him: ‘Sir, the brahmin \textsanskrit{Dhanañjāni} is sick, suffering, gravely ill. He bows with his head to your feet.’” Then \textsanskrit{Sāriputta}, after establishing \textsanskrit{Dhanañjāni} in the inferior realm of divinity, got up from his seat and left while there was still more left to do. Not long after \textsanskrit{Sāriputta} had departed, \textsanskrit{Dhanañjāni} passed away and was reborn in the realm of divinity. 

Then\marginnote{37.1} the Buddha said to the mendicants, “Mendicants, \textsanskrit{Sāriputta}, after establishing \textsanskrit{Dhanañjāni} in the inferior realm of divinity, got up from his seat and left while there was still more left to do.” 

Then\marginnote{38.1} \textsanskrit{Sāriputta} went to the Buddha, bowed, sat down to one side, and said, “Sir, the brahmin \textsanskrit{Dhanañjāni} is sick, suffering, gravely ill. He bows with his head to your feet.” 

“But\marginnote{38.4} \textsanskrit{Sāriputta}, after establishing \textsanskrit{Dhanañjāni} in the inferior realm of divinity, why did you get up from your seat and leave while there was still more left to do?” 

“Sir,\marginnote{38.5} I thought: ‘These brahmins are devoted to the realm of divinity. Why don’t I teach him a path to the company of Divinity?’” 

“And\marginnote{38.7} \textsanskrit{Sāriputta}, the brahmin \textsanskrit{Dhanañjāni} has passed away and been reborn in the realm of divinity.”\footnote{This is a rare instance of someone taking rebirth due to deathbed kamma. Presumably \textsanskrit{Dhanañjāni} had developed absorption based on the divine meditations, which overpowered the limited bad kamma he had made previously (\href{https://suttacentral.net/mn99/en/sujato\#24.3}{MN 99:24.3}). Another instance of kamma at death influencing rebirth is found at \href{https://suttacentral.net/sn42.3/en/sujato\#2.7}{SN 42.3:2.7}, where a warrior who dies in battle goes to hell. In Theravada circles today, it is commonly held that one’s mental state at death determines one’s rebirth. These examples show that this is true only in exceptional cases where one is actively making strong good or bad kamma at that time. } 

%
\section*{{\suttatitleacronym MN 98}{\suttatitletranslation With Vāseṭṭha }{\suttatitleroot Vāseṭṭhasutta}}
\addcontentsline{toc}{section}{\tocacronym{MN 98} \toctranslation{With Vāseṭṭha } \tocroot{Vāseṭṭhasutta}}
\markboth{With Vāseṭṭha }{Vāseṭṭhasutta}
\extramarks{MN 98}{MN 98}

\scevam{So\marginnote{1.1} I have heard.\footnote{This sutta is repeated verbatim at \href{https://suttacentral.net/snp3.9/en/sujato\#69.3}{Snp 3.9:69.3}. } }At one time the Buddha was staying in a forest near \textsanskrit{Icchānaṅgala}. 

Now\marginnote{2.1} at that time several very well-known well-to-do brahmins were residing in \textsanskrit{Icchānaṅgala}. They included the brahmins \textsanskrit{Caṅkī}, \textsanskrit{Tārukkha}, \textsanskrit{Pokkharasāti}, \textsanskrit{Jānussoṇi}, Todeyya, and others.\footnote{Also at \href{https://suttacentral.net/dn13/en/sujato\#2.2}{DN 13:2.2} and \href{https://suttacentral.net/mn99/en/sujato\#13.5}{MN 99:13.5}. } 

Then\marginnote{3.1} as the students \textsanskrit{Vāseṭṭha} and \textsanskrit{Bhāradvāja} were going for a walk they began to discuss the question:\footnote{We meet \textsanskrit{Vāseṭṭha} and \textsanskrit{Bhāradvāja} as the students of \textsanskrit{Pokkharasāti} and \textsanskrit{Tārukkha} respectively in \href{https://suttacentral.net/dn13/en/sujato}{DN 13}, at the end of which they went for refuge. Some time later they must have sought the monkhood, as in \href{https://suttacentral.net/dn27/en/sujato\#1.3}{DN 27:1.3} they are living in the Sangha awaiting ordination. } “How do you become a brahmin?” 

\textsanskrit{Bhāradvāja}\marginnote{3.3} said this: “When you’re well born on both your mother’s and father’s side, of pure descent, with irrefutable and impeccable genealogy back to the seventh paternal generation—\footnote{The ideal of brahminhood was a long-standing one, and so was the conflict over what exactly made someone a brahmin. \textsanskrit{Bhāradvāja} expresses the most obvious and socially widespread understanding that it is simply a matter of birth. } then you’re a brahmin.” 

\textsanskrit{Vāseṭṭha}\marginnote{3.6} said this: “When you’re ethical and accomplished in doing your duties—\footnote{\textit{Vattasampanno} (Sanskrit \textit{\textsanskrit{vṛttasaṁpanna}}) means “one of good conduct¸one who has fulfilled their duties” rather than having a technical sense of fulfilling religious observances or rites. For example, \textsanskrit{Manusmṛti} 8.179 advises that one should deposit money with someone who is “of good conduct, knows the law, and speaks the truth” (\textit{\textsanskrit{vṛttasampanne} \textsanskrit{dharmajñe} \textsanskrit{satyavādini}}. See also eg. \textsanskrit{Mahābhārata} 3.188.90b, 13.61.029a, \textsanskrit{Rāmāyaṇa} 1.47.25c). | Compare the discussion at \href{https://suttacentral.net/dn4/en/sujato\#21.1}{DN 4:21.1}, where the Buddha leads \textsanskrit{Soṇadaṇḍa} to agree that the essential qualities of a brahmin were ethical conduct and wisdom. } then you’re a brahmin.” 

But\marginnote{4.1} neither was able to persuade the other. 

So\marginnote{5.1} \textsanskrit{Vāseṭṭha} said to \textsanskrit{Bhāradvāja}, “Mister \textsanskrit{Bhāradvāja}, the ascetic Gotama—a Sakyan, gone forth from a Sakyan family—is staying in a forest near \textsanskrit{Icchānaṅgala}. He has this good reputation: ‘That Blessed One is perfected, a fully awakened Buddha, accomplished in knowledge and conduct, holy, knower of the world, supreme guide for those who wish to train, teacher of gods and humans, awakened, blessed.’ Come, let’s go to see him and ask him about this matter. As he answers, so we’ll remember it.” 

“Yes,\marginnote{5.7} sir,” replied \textsanskrit{Bhāradvāja}. 

So\marginnote{6.1} they went to the Buddha and exchanged greetings with him. When the greetings and polite conversation were over, they sat down to one side, and \textsanskrit{Vāseṭṭha} addressed the Buddha in verse: 

\begin{verse}%
“We’re\marginnote{7.1} both authorized masters \\
of the three Vedas. \\
I’m a student of \textsanskrit{Pokkharasāti},\footnote{\textsanskrit{Pokkharasāti}’s conversion to Buddhism is recorded in \href{https://suttacentral.net/dn3/en/sujato}{DN 3}, which marks a turning point in the embracing of Buddhism by leading brahmins. The suttas of the \textsanskrit{Dīgha} \textsanskrit{Nikāya} that follow reverberate with the consequences of this encounter. He was one of the most influential brahmins of his time, although the Buddha elsewhere denied that he had any special knowledge (\href{https://suttacentral.net/mn99/en/sujato\#15.5}{MN 99:15.5}). Brahmanical texts know him as an influential teacher around the time of the Buddha; his name is spelled \textsanskrit{Pauṣkarasādi} in Sanskrit. He is cited on grammar by \textsanskrit{Kātyāyana} and \textsanskrit{Patañjali}, and in the \textsanskrit{Taittirīya}-\textsanskrit{prātiśākhya}; on allowable food and theft in the Āpastamba Dharmasūtra; and on Vedic ritual in the \textsanskrit{Śāṅkhāyana}-\textsanskrit{Āraṇyaka}. \href{https://suttacentral.net/mn99/en/sujato\#10.3}{MN 99:10.3} clarifies that he is of the \textsanskrit{Upamañña} lineage. } \\
and he of \textsanskrit{Tārukkha}.\footnote{In Pali we never meet \textsanskrit{Tārukkha} and he is only mentioned in his absence. \textsanskrit{Bhāradvāja} advocates his path at \href{https://suttacentral.net/dn13/en/sujato\#5.2}{DN 13:5.2} but without details. There is a \textsanskrit{Tārukṣya} of Aitareya \textsanskrit{Āraṇyaka} 3.1.6.1 whose view was that union (with \textsanskrit{Brahmā}) arose with the conjunction of speech and breath; this was discussed alongside the views of many other brahmins. In Rig Veda 8.46.32 a certain \textsanskrit{Balbūtha} \textsanskrit{Tarukṣa} the \textsanskrit{Dāsa} makes an offering to a sage. \textsanskrit{Sāyaṇa}, the Vedic commentator, says that \textsanskrit{Balbūtha} \textsanskrit{Tarukṣa} was a guardian of cows, evidently alluding to the Aitareya \textsanskrit{Āraṇyaka}, which describes \textsanskrit{Tārukṣya} as a guardian of his teacher’s cows, thus locating \textsanskrit{Tārukṣya} in the lineage of \textsanskrit{Tarukṣa}. \textsanskrit{Hiraṇyakeśīgṛhyasūtra} 2.8.19 also mentions him as a teacher, there spelled \textsanskrit{Tarukṣa}. } 

We’re\marginnote{7.5} fully qualified \\
in all the Vedic experts teach. \\
As philologists and grammarians, \\
we match our tutors in recitation. \\
We have a dispute \\
regarding genealogy. 

For\marginnote{7.11} \textsanskrit{Bhāradvāja} says that \\
one is a brahmin due to birth, \\
but I declare it’s because of one’s deeds. \\
Oh Clear-eyed One, know this as our debate. 

Since\marginnote{7.15} neither of us was able \\
to convince the other, \\
we’ve come to ask you, sir, \\
renowned as the awakened one. 

As\marginnote{7.19} people honor with joined palms \\
the moon on the cusp of waxing, \\
bowing, they revere \\
Gotama in the world. 

We\marginnote{7.23} ask this of Gotama,\footnote{The Buddha as “eye” evokes the common (eg. Rig Veda 1.164.14, 5.40.8, 5.59.5, 10.10.9) Vedic image of the Sun as the “eye of all” (\textit{\textsanskrit{viśvacakṣāḥ}}, 7.63.1), the “eye” for “eyes to see” (10.158.4), moving as an unaging wheel through the sky (1.164.14). See \href{https://suttacentral.net/dn16/en/sujato\#5.6.3}{DN 16:5.6.3}. } \\
the Eye arisen in the world: \\
is one a brahmin due to birth, \\
or else because of deeds? \\
We don’t know, please tell us, \\
so we can recognize a brahmin.” 

“I\marginnote{8.1} shall explain to you,” \\
\scspeaker{said the Buddha, }\\
“accurately and in sequence, \\
the taxonomy of living creatures, \\
for species are indeed diverse. 

Know\marginnote{8.6} the grass and trees, \\
though they lack self-awareness. \\
They’re defined by birth, \\
for species are indeed diverse.\footnote{Notice how the Buddha responds clearly, yet without immediately contradicting either of them. As usual, he begins by identifying common ground, pointing out things that they will all agree on. } 

Next\marginnote{8.10} there are bugs and moths, \\
and so on, to ants and termites. \\
They’re defined by birth, \\
for species are indeed diverse. 

Know\marginnote{8.14} the quadrupeds, too, \\
both small and large. \\
They’re defined by birth, \\
for species are indeed diverse. 

Know,\marginnote{8.18} too, the long-backed snakes, \\
crawling on their bellies. \\
They’re defined by birth, \\
for species are indeed diverse. 

Next\marginnote{8.22} know the fish, \\
whose range is the water. \\
They’re defined by birth, \\
for species are indeed diverse. 

Next\marginnote{8.26} know the birds, \\
winged chariots in flight. \\
They’re defined by birth, \\
for species are indeed diverse. 

While\marginnote{9.1} the differences between these species \\
are defined by birth, \\
the differences between humans \\
are not defined by birth.\footnote{Having established common ground, the Buddha does not hesitate to disagree with \textsanskrit{Bhāradvāja}. } 

Not\marginnote{9.5} by hair nor by head, \\
not by ear nor by eye, \\
not by mouth nor by nose, \\
not by lips nor by eyebrow, 

not\marginnote{9.9} by shoulder nor by neck, \\
not by belly nor by back, \\
not by buttocks nor by breast, \\
not by groin nor by genitals, 

not\marginnote{9.13} by hands nor by feet, \\
not by fingers nor by nails, \\
not by knees nor by thighs, \\
not by color nor by voice: \\
none of these are defined by birth \\
as it is for other species. 

In\marginnote{9.19} individual human bodies \\
you can’t find such distinctions. \\
The distinctions among humans \\
are spoken of by convention.\footnote{The Buddha rejects any essentialist theory of human differentiation. } 

Anyone\marginnote{10.1} among humans \\
who lives off keeping cattle: \\
know them, \textsanskrit{Vāseṭṭha}, \\
as a farmer, not a brahmin. 

Anyone\marginnote{10.5} among humans \\
who lives off various professions:\footnote{\textit{Sippa} here is usually translated “craft”, but this refers primarily to occupations making things by hand. In DN 2 there is a long list of \textit{sippa}, only a few of which are “crafts”. “Profession” fits better. } \\
know them, \textsanskrit{Vāseṭṭha}, \\
as a professional, not a brahmin. 

Anyone\marginnote{10.9} among humans \\
who lives off trade: \\
know them, \textsanskrit{Vāseṭṭha}, \\
as a trader, not a brahmin. 

Anyone\marginnote{10.13} among humans \\
who lives off serving others: \\
know them, \textsanskrit{Vāseṭṭha}, \\
as a servant, not a brahmin. 

Anyone\marginnote{10.17} among humans \\
who lives off stealing: \\
know them, \textsanskrit{Vāseṭṭha}, \\
as a bandit, not a brahmin. 

Anyone\marginnote{10.21} among humans \\
who lives off archery: \\
know them, \textsanskrit{Vāseṭṭha}, \\
as a soldier, not a brahmin. 

Anyone\marginnote{10.25} among humans \\
who lives off priesthood: \\
know them, \textsanskrit{Vāseṭṭha}, \\
as a sacrificer, not a brahmin.\footnote{The Buddha has been carefully leading up to this rhetorical thrust: a true brahmin is not one who performs the traditional rituals revered by the brahmins. } 

Anyone\marginnote{10.29} among humans \\
who taxes village and nation,\footnote{Read \textit{\textsanskrit{bhuñjati}} at \href{https://suttacentral.net/mn98/en/sujato\#10.30}{MN 98:10.30} with \textit{\textsanskrit{yathābhuttañca} \textsanskrit{bhuñjatha}} at \href{https://suttacentral.net/dn17/en/sujato\#1.9.4}{DN 17:1.9.4}, \href{https://suttacentral.net/dn26/en/sujato\#6.7}{DN 26:6.7}, and \href{https://suttacentral.net/mn129/en/sujato\#35.7}{MN 129:35.7}. Translators have rendered these with “eat”, “enjoy”, or “govern”. But compare the archaic English “use” meaning “the benefit or profit of lands”. Thus \textit{\textsanskrit{yathābhuttañca} \textsanskrit{bhuñjatha}} means “use as has been used”, i.e. “maintain the current level of taxation”. } \\
know them, \textsanskrit{Vāseṭṭha}, \\
as a ruler, not a brahmin. 

I\marginnote{11.1} don’t call someone a brahmin \\
after the mother’s womb they’re born from. \\
If they still have attachments, \\
they’re just someone who says ‘mister’.\footnote{\textit{Bho} (“mister”) is a term of address used by brahmins when speaking to one another or to mendicants. } \\
Having nothing, taking nothing: \\
that’s who I call a brahmin.\footnote{The Buddha categorically rejects \textsanskrit{Bhāradvāja}’s genealogical understanding of a brahmin and endorses \textsanskrit{Vāseṭṭha}’s understanding in terms of good conduct. Nonetheless, while a brahmin must have good conduct, the true meaning of the word lies in liberation and transcendence. In this, the Buddha’s position approaches that of \textsanskrit{Yājñavalkya}, who argued that a brahmin, knowing the Self, gives up desire for sons, wealth, and heaven and wanders seeking alms (\textsanskrit{Bṛhadāraṇyaka} \textsanskrit{Upaniṣad} 3.5.1); ultimately, only one who knows the imperishable is a brahmin (3.8.10). } 

Having\marginnote{11.7} cut off all fetters \\
they have no anxiety; \\
they’ve slipped their chains and are detached: \\
that’s who I call a brahmin. 

They’ve\marginnote{11.11} cut the strap and harness, \\
the reins and bridle too; \\
with cross-bar lifted, they’re awakened:\footnote{For an explanation of the “cross-bar”, see \href{https://suttacentral.net/mn22/en/sujato\#31.1}{MN 22:31.1} and note. } \\
that’s who I call a brahmin. 

Abuse,\marginnote{11.15} killing, caging: \\
they endure these without anger. \\
Patience is their powerful army: \\
that’s who I call a brahmin. 

Not\marginnote{11.19} irritable or pretentious, \\
dutiful in precepts and observances, \\
tamed, bearing their final body: \\
that’s who I call a brahmin. 

Like\marginnote{11.23} rain off a lotus leaf, \\
like a mustard seed off the point of a pin, \\
sensual pleasures slip off them: \\
that’s who I call a brahmin. 

They\marginnote{11.27} understand for themselves \\
the end of suffering in this life; \\
with burden put down, detached: \\
that’s who I call a brahmin. 

Deep\marginnote{11.31} in wisdom, intelligent, \\
expert in what is the path \\>and what is not the path;\footnote{The compound \textit{\textsanskrit{maggāmagga}} can be read either as “what is the path and what is not the path”, or as “the variety of paths” (compare \textit{\textsanskrit{phalāphala}}, “all sorts of fruit”). However, this was the central topic of conversation between \textsanskrit{Vāseṭṭha} and \textsanskrit{Bhāradvāja} at \href{https://suttacentral.net/dn13/en/sujato\#3.1}{DN 13:3.1}, where they were concerned to distinguish one path as correct. } \\
arrived at the highest goal: \\
that’s who I call a brahmin. 

Mixing\marginnote{11.35} with neither \\
householders nor the homeless; \\
a migrant with no shelter, few in wishes:\footnote{The “migrant with no shelter” (\textit{\textsanskrit{anokasārī}}) is defined at \href{https://suttacentral.net/sn22.3/en/sujato\#5.1}{SN 22.3:5.1} as one free of attachments. } \\
that’s who I call a brahmin. 

They’ve\marginnote{11.39} laid aside violence \\
against creatures firm and frail; \\
not killing or making others kill: \\
that’s who I call a brahmin.\footnote{A brahmin is identified with Mitra, the “friendly” god of alliances (Śatapatha \textsanskrit{Brāhmaṇa} 4.1.4). The Brahmanical tradition endorses harmlessness except in the sacrifice (\textsanskrit{Chāndogya} \textsanskrit{Upaniṣad} 8.15.1), which can be read as either sheer hypocrisy or as a sacred and effective limitation on the scope of killing. Either way, the Buddha made no such exception. } 

Not\marginnote{11.43} fighting among those who fight, \\
quenched among those who are armed, \\
not grasping among those who grasp: \\
that’s who I call a brahmin. 

They’ve\marginnote{11.47} discarded greed and hate, \\
along with conceit and contempt, \\
like a mustard seed off the point of a pin: \\
that’s who I call a brahmin. 

The\marginnote{11.51} words they utter \\
are sweet, informative, and true, \\
and don’t offend anyone: \\
that’s who I call a brahmin. 

They\marginnote{11.55} don’t steal anything in the world, \\
long or short, \\
fine or coarse, beautiful or ugly: \\
that’s who I call a brahmin. 

They\marginnote{11.59} have no hope\footnote{\textit{\textsanskrit{Āsā}} is probably the Pali word that comes closest to “hope” in the sense of longing for a positive destiny in the future. Hope is not a virtue in Buddhism, which is focused on what is visible in the present. } \\
for this world or the next; \\
with no need for hope, detached: \\
that’s who I call a brahmin. 

They\marginnote{11.63} have no clinging, \\
knowledge has freed them of indecision, \\
they’ve arrived at the culmination of freedom from death: \\
that’s who I call a brahmin. 

They’ve\marginnote{11.67} escaped the snare \\
of both good and bad deeds; \\
sorrowless, stainless, pure: \\
that’s who I call a brahmin. 

Pure\marginnote{11.71} as the spotless moon, \\
clear and undisturbed, \\
they’ve ended desire to be reborn: \\
that’s who I call a brahmin. 

They’ve\marginnote{11.75} got past this grueling swamp \\
of delusion, transmigration. \\
Meditating in stillness, free of indecision, \\
they have crossed over to the far shore. \\
They’re quenched by not grasping: \\
that’s who I call a brahmin. 

They’ve\marginnote{11.81} given up sensual stimulations, \\
and have gone forth from lay life; \\
they’ve ended rebirth in the sensual realm: \\
that’s who I call a brahmin. 

They’ve\marginnote{11.85} given up craving, \\
and have gone forth from lay life; \\
they’ve ended craving to be reborn: \\
that’s who I call a brahmin. 

They’ve\marginnote{11.89} thrown off the human yoke, \\
and slipped out of the heavenly yoke; \\
unyoked from all yokes: \\
that’s who I call a brahmin. 

Giving\marginnote{11.93} up discontent and desire, \\
they’re cooled and free of attachments; \\
a hero, master of the whole world: \\
that’s who I call a brahmin. 

They\marginnote{11.97} know the passing away \\
and rebirth of all beings; \\
unattached, holy, awakened: \\
that’s who I call a brahmin. 

Gods,\marginnote{11.101} centaurs, and humans \\
don’t know their destiny; \\
the perfected ones with defilements ended: \\
that’s who I call a brahmin. 

They\marginnote{11.105} have nothing before or after, \\
or even in between. \\
Having nothing, taking nothing: \\
that’s who I call a brahmin. 

Leader\marginnote{11.109} of the herd, excellent hero, \\
great seer and victor; \\
unstirred, washed, awakened: \\
that’s who I call a brahmin. 

They\marginnote{11.113} who know their past lives, \\
see heaven and places of loss, \\
and have attained the end of rebirth: \\
that’s who I call a brahmin. 

For\marginnote{12.1} name and clan are formulated \\
as mere convention in the world. \\
Produced by mutual agreement, \\
they’re formulated for each individual. 

For\marginnote{12.5} a long time this misconception \\
has prejudiced those who don’t understand. \\
Ignorant, they declare \\
that one is a brahmin by birth. 

You’re\marginnote{12.9} not a brahmin by birth, \\
nor by birth a non-brahmin. \\
You’re a brahmin by your deeds, \\
and by deeds a non-brahmin.\footnote{The Buddha comes down on \textsanskrit{Vāseṭṭha}’s side, as also at \href{https://suttacentral.net/snp1.7/en/sujato\#24.4}{Snp 1.7:24.4}. Despite this definitive statement, the Buddha goes on to indicate the limits of deeds. } 

You’re\marginnote{12.13} a farmer by your deeds, \\
by deeds you’re a professional; \\
you’re a trader by your deeds, \\
by deeds are you a servant; 

you’re\marginnote{12.17} a bandit by your deeds, \\
by deeds you’re a soldier; \\
you’re a sacrificer by your deeds, \\
by deeds you’re a ruler. 

In\marginnote{13.1} this way the astute regard deeds \\
in accord with truth. \\
Seeing dependent origination, \\
they’re expert in deeds and their results. 

Deeds\marginnote{13.5} make the world go on, \\
deeds make people go on; \\
sentient beings are bound by deeds, \\
like a moving chariot’s linchpin. 

By\marginnote{13.9} fervor and spiritual practice, \\
by restraint and by self-control: \\
that’s how to become a brahmin, \\
this is the supreme brahmin.\footnote{The text implies that a “supreme brahmin” is one who has transcended all deeds, for they bind one to transmigration. } 

Accomplished\marginnote{13.13} in the three knowledges, \\
peaceful, with rebirth ended, \\
know them, \textsanskrit{Vāseṭṭha}, \\
as the Divinity and Sakka to the wise.”\footnote{A liberated person is revered in Buddhism like God in theistic religions. } 

%
\end{verse}

When\marginnote{14.1} he had spoken, \textsanskrit{Vāseṭṭha} and \textsanskrit{Bhāradvāja} said to him, “Excellent, Mister Gotama! Excellent! As if he were righting the overturned, or revealing the hidden, or pointing out the path to the lost, or lighting a lamp in the dark so people with clear eyes can see what’s there, Mister Gotama has made the teaching clear in many ways. We go for refuge to Mister Gotama, to the teaching, and to the mendicant \textsanskrit{Saṅgha}. From this day forth, may Mister Gotama remember us as lay followers who have gone for refuge for life.” 

%
\section*{{\suttatitleacronym MN 99}{\suttatitletranslation With Subha }{\suttatitleroot Subhasutta}}
\addcontentsline{toc}{section}{\tocacronym{MN 99} \toctranslation{With Subha } \tocroot{Subhasutta}}
\markboth{With Subha }{Subhasutta}
\extramarks{MN 99}{MN 99}

\scevam{So\marginnote{1.1} I have heard. }At one time the Buddha was staying near \textsanskrit{Sāvatthī} in Jeta’s Grove, \textsanskrit{Anāthapiṇḍika}’s monastery. 

Now\marginnote{2.1} at that time the student Subha, Todeyya’s son, was residing in \textsanskrit{Sāvatthī} at a certain householder’s home on some business.\footnote{The same Subha later met the Buddha in \href{https://suttacentral.net/mn135/en/sujato}{MN 135}, where he asked about kamma, and, following the Buddha’s passing, met with Ānanda in \href{https://suttacentral.net/dn10/en/sujato}{DN 10}. | Subha’s father Todeyya was a prominent brahmin, often mentioned alongside \textsanskrit{Pokkharasāti}. The two apparently lived not far from each other, as, according to the commentary, Todeyya was named for his village of Tudi outside of \textsanskrit{Sāvatthī} (see \textsanskrit{Pāṇini}’s \textsanskrit{Aṣṭādhyāyī} 4.3.94). } Then Subha said to that householder, “Householder, I have heard that\footnote{This passage suggests this was Subha’s first meeting with the Buddha. } \textsanskrit{Sāvatthī} does not lack for perfected ones. What ascetic or brahmin might we pay homage to today?” 

“Sir,\marginnote{2.6} the Buddha is staying near \textsanskrit{Sāvatthī} in Jeta’s Grove, \textsanskrit{Anāthapiṇḍika}’s monastery. You can pay homage to him.” 

Acknowledging\marginnote{3.1} that householder, Subha went to the Buddha and exchanged greetings with him. When the greetings and polite conversation were over, he sat down to one side and said to the Buddha: 

“Mister\marginnote{4.1} Gotama, the brahmins say: ‘Laypeople succeed in the system of the skillful teaching, not renunciates.’\footnote{The “the system of the skillful teaching” (or perhaps “principle”, \textit{\textsanskrit{ñāyaṁ} \textsanskrit{dhammaṁ} \textsanskrit{kusalaṁ}}), though attributed to the brahmins, is a very Buddhist-sounding phrase; I have not been able to identify anything similar in Brahmanical texts. The sentiment, however, is Brahmanical, since Vedism was for a long time practiced by householders, and the renunciate path advocated by \textsanskrit{Yājñavalkya} was still controversial (see eg. \textsanskrit{Bṛhadāraṇyaka} \textsanskrit{Upaniṣad} 4.4.22). } What do you say about this?” 

“On\marginnote{4.4} this point, student, I speak after analyzing the question,\footnote{“One who speaks after analyzing the question” is \textit{\textsanskrit{vibhajjavāda}}, a term that in later years characterized the Buddha’s teachings as a whole, or else a school or group of schools of whom the Sri Lankan \textsanskrit{Theravāda} was one. The term is also used in Jainism (\textsanskrit{Sūyagaḍa} 14.22). } not one-sidedly. I don’t praise wrong practice for either laypeople or renunciates. Because of wrong practice, neither laypeople nor renunciates succeed in the system of the skillful teaching. I praise right practice for both laypeople and renunciates. Because of right practice, both laypeople and renunciates succeed in the system of the skillful teaching.” 

“Mister\marginnote{5.1} Gotama, the brahmins say: ‘Since the work of the lay life has many requirements, duties, issues, and undertakings it is very fruitful.\footnote{Compare passages such as \href{https://suttacentral.net/an3.60/en/sujato\#4.2}{AN 3.60:4.2}, \href{https://suttacentral.net/dn5/en/sujato\#23.1}{DN 5:23.1}, and \href{https://suttacentral.net/an5.96/en/sujato\#1.3}{AN 5.96:1.3}. } But since the work of the renunciate has few requirements, duties, issues, and undertakings it is not very fruitful.’ What do you say about this?” 

“On\marginnote{5.5} this point, too, I speak after analyzing the question, not one-sidedly. Some work has many requirements, duties, issues, and undertakings, and when it fails it’s not very fruitful. Some work has many requirements, duties, issues, and undertakings, and when it succeeds it is very fruitful. Some work has few requirements, duties, issues, and undertakings, and when it fails it’s not very fruitful. Some work has few requirements, duties, issues, and undertakings, and when it succeeds it is very fruitful. 

And\marginnote{6.1} what work has many requirements, duties, issues, and undertakings, and when it fails it’s not very fruitful? Farming. And what work has many requirements, duties, issues, and undertakings, and when it succeeds it is very fruitful? Again, it is farming. And what work has few requirements, duties, issues, and undertakings, and when it fails it’s not very fruitful? Trade. And what work has few requirements, duties, issues, and undertakings, and when it succeeds it is very fruitful? Again, it’s trade. 

The\marginnote{7.1} lay life is like farming in that it’s work with many requirements and when it fails it’s not very fruitful; but when it succeeds it is very fruitful. The renunciate life is like trade in that it’s work with few requirements and when it fails it’s not very fruitful; but when it succeeds it is very fruitful.” 

“Mister\marginnote{8.1} Gotama, the brahmins prescribe five things for making merit and succeeding in the skillful.” 

“If\marginnote{8.3} you don’t mind, please explain these in this assembly.” 

“It’s\marginnote{8.5} no trouble when good sirs such as yourself are sitting here.” 

“Well,\marginnote{8.6} speak then, student.” 

“Mister\marginnote{9.1} Gotama, truth is the first thing. Fervor is the second thing.\footnote{Given that this is said to be a practice for householders, it would seem that \textit{tapas} here means “fervor” rather than “mortification”. | A somewhat similar list of four things is found at \textsanskrit{Bṛhadāraṇyaka} \textsanskrit{Upaniṣad} 4.4.22: a brahmin realizes the Self by means of study of Vedas, sacrifice, generosity, and the fervor of fasting (\textit{\textsanskrit{vedānuvacanena} … \textsanskrit{yajñena} \textsanskrit{dānena} \textsanskrit{tapasānāśakena}}). \textsanskrit{Chāndogya} \textsanskrit{Upaniṣad} 8.5.3 further identifies fasting with \textit{brahmacariya} and living in the forest. } Celibacy is the third thing. Recitation is the fourth thing. Generosity is the fifth thing. These are the five things that the brahmins prescribe for making merit and succeeding in the skillful. What do you say about this?” 

“Well,\marginnote{9.8} student, is there even a single one of the brahmins who says this:\footnote{Compare \href{https://suttacentral.net/mn95/en/sujato\#13.1}{MN 95:13.1}. } ‘I declare the result of these five things after realizing it with my own insight’?” 

“No,\marginnote{9.10} Mister Gotama.” 

“Well,\marginnote{9.11} is there even a single tutor of the brahmins, or a tutor’s tutor, or anyone back to the seventh generation of tutors, who says this: ‘I declare the result of these five things after realizing it with my own insight’?” 

“No,\marginnote{9.13} Mister Gotama.” 

“Well,\marginnote{9.14} what of the ancient seers of the brahmins, namely \textsanskrit{Aṭṭhaka}, \textsanskrit{Vāmaka}, \textsanskrit{Vāmadeva}, \textsanskrit{Vessāmitta}, Yamadaggi, \textsanskrit{Aṅgīrasa}, \textsanskrit{Bhāradvāja}, \textsanskrit{Vāseṭṭha}, Kassapa, and Bhagu? They were the authors and propagators of the hymns. Their hymnal was sung and propagated and compiled in ancient times; and these days, brahmins continue to sing and chant it, chanting what was chanted and teaching what was taught. Did even they say: ‘We declare the result of these five things after realizing it with our own insight’?” 

“No,\marginnote{9.17} Mister Gotama.” 

“So,\marginnote{9.18} student, it seems that there is not a single one of the brahmins, not even anyone back to the seventh generation of tutors, nor even the ancient seers of the brahmins who says: ‘We declare the result of these five things after realizing it with our own insight.’ 

Suppose\marginnote{9.25} there was a queue of blind men, each holding the one in front: the first one does not see, the middle one does not see, and the last one does not see. In the same way, it seems to me that the brahmins’ statement turns out to be comparable to a queue of blind men: the first one does not see, the middle one does not see, and the last one does not see.” 

When\marginnote{10.1} he said this, Subha became angry and upset with the Buddha because of the simile of the queue of blind men. He even attacked and badmouthed the Buddha himself, saying, “The ascetic Gotama will be worsted!” He said to the Buddha: 

“Mister\marginnote{10.3} Gotama, the brahmin \textsanskrit{Pokkharasāti} \textsanskrit{Upamañña} of the Subhaga Forest says:\footnote{The Subhaga Forest outside \textsanskrit{Ukkaṭṭha} was the scene of cosmic drama demonstrating the superiority of the Buddha over the brahmins, likely chosen because it was \textsanskrit{Pokkharasāti}’s home (\href{https://suttacentral.net/mn1/en/sujato\#1.2}{MN 1:1.2}, \href{https://suttacentral.net/mn49/en/sujato\#2.1}{MN 49:2.1}; see also \href{https://suttacentral.net/dn14/en/sujato\#3.29}{DN 14:3.29}). | The \textsanskrit{Upamañña} clan (\textit{gotra}) traces to Upamanyu \textsanskrit{Vāsiṣṭha}, the author of three verses in homage of Indra and Soma (Rig Veda 9.97.13–15). The association with Soma worship continued with \textsanskrit{Prācīnaśāla} son of Upamanyu, who worshiped the Self in the sky only, which, though limited, brings an abundance of Soma libations and food in the family (\textsanskrit{Chāndogya} \textsanskrit{Upaniṣad} 5.12; cf. \textsanskrit{Bṛhadāraṇyaka} \textsanskrit{Upaniṣad} 2.1.3). An Upamanyu descendant hailed from Kamboja (Persia); his teacher \textsanskrit{Madragāra} \textsanskrit{Śauṅgāyani} was from the neighboring Madra (\textsanskrit{Vaṁśa}-\textsanskrit{Brāhmaṇa} 1.17). It is perhaps this Upamanyu who is cited for his linguistic expertise in \textsanskrit{Yāska}’s Nirukta (1.1, etc.). Insofar as any conclusions can be drawn from all this, it agrees with the sutta depiction of \textsanskrit{Pokkharasāti} as a family man, devoted to Brahmanical traditions, and learned in the Vedas. Later, the \textsanskrit{Purāṇas} would depict an Upamanyu as a devotee of Śiva. } ‘This is exactly what happens with some ascetics and brahmins. They claim to have a superhuman distinction in knowledge and vision worthy of the noble ones. But their statement turns out to be a joke—mere words, vacuous and hollow. For how on earth can a human being know or see or realize a superhuman distinction in knowledge and vision worthy of the noble ones? That is not possible.’”\footnote{This attitude sharply contrasts with \href{https://suttacentral.net/dn3/en/sujato\#1.2.6}{DN 3:1.2.6}, where he is inspired to hear of the Buddha’s reputation for meditative insight. Presumably this was earlier. He may have grown more interested in the Buddha over time, as his reputation spread. } 

“But\marginnote{11.1} student, does \textsanskrit{Pokkharasāti} understand the minds of all these ascetics and brahmins, having comprehended them with his mind?” 

“Mister\marginnote{11.2} Gotama, \textsanskrit{Pokkharasāti} doesn’t even know the mind of his own bonded maid \textsanskrit{Puṇṇikā}, so how could he know all those ascetics and brahmins?” 

“Suppose\marginnote{12.1} there was a person blind from birth. They couldn’t see sights that are dark or bright, or blue, yellow, red, or magenta. They couldn’t see even and uneven ground, or the stars, or the moon and sun. They’d say: ‘There’s no such thing as dark and bright sights, and no-one who sees them. There’s no such thing as blue, yellow, red, magenta, even and uneven ground, stars, moon and sun, and no-one who sees these things. I don’t know it or see it, therefore it doesn’t exist.’\footnote{A logic still all too familiar today. } Would they be speaking rightly?” 

“No,\marginnote{12.14} Mister Gotama. There are such things as dark and bright sights, and one who sees them. There is blue, yellow, red, magenta, even and uneven ground, stars, moon and sun, and one who sees these things. So it’s not right to say this: ‘I don’t know it or see it, therefore it doesn’t exist.’” 

“In\marginnote{13.1} the same way, \textsanskrit{Pokkharasāti} is blind and sightless. It is quite impossible for him to know or see or realize a superhuman distinction in knowledge and vision worthy of the noble ones. 

What\marginnote{13.3} do you think, student? There are well-to-do brahmins of Kosala such as the brahmins \textsanskrit{Caṅkī}, \textsanskrit{Tārukkha}, \textsanskrit{Pokkharasāti}, \textsanskrit{Jānussoṇi}, and your father Todeyya. What’s better for them: that their speech agrees or disagrees with the consensus of opinion?” 

“That\marginnote{13.6} it agrees, Mister Gotama.” 

“What’s\marginnote{13.7} better for them: that their speech is thoughtful or thoughtless?” 

“That\marginnote{13.8} it is thoughtful.” 

“What’s\marginnote{13.9} better for them: that their speech follows reflection or is unreflective?” 

“That\marginnote{13.10} it follows reflection.” 

“What’s\marginnote{13.11} better for them: that their speech is beneficial or worthless?” 

“That\marginnote{13.12} it’s beneficial.” 

“What\marginnote{14.1} do you think, student? If this is so, does \textsanskrit{Pokkharasāti}’s speech agree or disagree with the consensus of opinion?” 

“It\marginnote{14.3} disagrees, Mister Gotama.” 

“Is\marginnote{14.4} it thoughtful or thoughtless?” 

“Thoughtless.”\marginnote{14.5} 

“Is\marginnote{14.6} it reflective or unreflective?” 

“Unreflective.”\marginnote{14.7} 

“Is\marginnote{14.8} it beneficial or worthless?” 

“Worthless.”\marginnote{14.9} 

“Student,\marginnote{15.1} there are these five hindrances. What five? The hindrances of sensual desire, ill will, dullness and drowsiness, restlessness and remorse, and doubt. These are the five hindrances. \textsanskrit{Pokkharasāti} is veiled, shrouded, covered, and engulfed by these five hindrances.\footnote{For \textit{\textsanskrit{ophuṭo}} compare \textit{\textsanskrit{sabbavāriphuṭo}} at \href{https://suttacentral.net/mn56/en/sujato\#12.2}{MN 56:12.2}. In both cases \textit{\textsanskrit{phuṭ}} appears in a string of terms from the root \textit{var}, and is possibly a corrupted form, or at least has the same meaning. } It is quite impossible for him to know or see or realize a superhuman distinction in knowledge and vision worthy of the noble ones. 

There\marginnote{16.1} are these five kinds of sensual stimulation. What five? There are sights known by the eye, which are likable, desirable, agreeable, pleasant, sensual, and arousing. There are sounds known by the ear … smells known by the nose … tastes known by the tongue … touches known by the body, which are likable, desirable, agreeable, pleasant, sensual, and arousing. These are the five kinds of sensual stimulation. 

\textsanskrit{Pokkharasāti}\marginnote{16.9} enjoys himself with these five kinds of sensual stimulation, tied, infatuated, attached, blind to the drawbacks, and not understanding the escape. It is quite impossible for him to know or see or realize a superhuman distinction in knowledge and vision worthy of the noble ones. 

What\marginnote{17.1} do you think, student? Which would have better flames, color, and radiance: a fire that depends on grass and logs as fuel, or one that does not?” 

“If\marginnote{17.3} it were possible for a fire to burn without depending on grass and logs as fuel, that would have better flames, color, and radiance.” 

“But\marginnote{17.4} it isn’t possible, except by psychic power. Rapture that depends on the five kinds of sensual stimulation is like a fire that depends on grass and logs as fuel. Rapture that’s apart from sensual pleasures and unskillful qualities is like a fire that doesn’t depend on grass and logs as fuel. 

And\marginnote{17.7} what is rapture that’s apart from sensual pleasures and unskillful qualities? It’s when a mendicant, quite secluded from sensual pleasures, secluded from unskillful qualities, enters and remains in the first absorption, which has the rapture and bliss born of seclusion, while placing the mind and keeping it connected. This is rapture that’s apart from sensual pleasures and unskillful qualities. 

Furthermore,\marginnote{17.10} as the placing of the mind and keeping it connected are stilled, they enter and remain in the second absorption, which has the rapture and bliss born of immersion, with internal clarity and mind at one, without placing the mind and keeping it connected. This too is rapture that’s apart from sensual pleasures and unskillful qualities. 

Of\marginnote{18.1} the five things that the brahmins prescribe for making merit and succeeding in the skillful, which do they say is the most fruitful?” 

“Generosity.”\marginnote{18.2} 

“What\marginnote{19.1} do you think, student? Suppose a brahmin was setting up a big sacrifice. Then two brahmins came along, thinking to participate. Then one of those brahmins thought: ‘Oh, I hope that I alone get the best seat, the best drink, and the best almsfood in the refectory, not some other brahmin.’\footnote{The same thought can occur to mendicants too (\href{https://suttacentral.net/mn5/en/sujato\#15.2}{MN 5:15.2}). } But it’s possible that some other brahmin gets the best seat, the best drink, and the best almsfood in the refectory. Thinking, ‘Some other brahmin has got the best seat, the best drink, the best almsfood,’ they get angry and bitter. What do the brahmins say is the result of this?” 

“Mister\marginnote{19.11} Gotama, brahmins don’t give gifts so that others will get angry and upset. Rather, they give only out of sympathy.” 

“In\marginnote{19.14} that case, isn’t compassion a sixth ground for making merit?” 

“In\marginnote{19.16} that case, compassion is a sixth ground for making merit.”\footnote{Accepting his initial idea of generosity in deed, the Buddha leads Subha to agree on the importance of inner motivation as well. } 

“Of\marginnote{20.1} the five things that the brahmins prescribe for making merit and succeeding in the skillful, where do you usually find them: among laypeople or renunciates?” 

“Mostly\marginnote{20.3} among renunciates, and less so among lay people. For a lay person has many requirements, duties, issues, and undertakings, and they can’t always tell the truth, be fervent, be celibate, do lots of recitation, or be very generous. But a renunciate has few requirements, duties, issues, and undertakings, and they can always tell the truth, be fervent, be celibate, do lots of recitation, and be very generous. Of the five things that the brahmins prescribe for making merit and succeeding in the skillful, I usually find them among renunciates, and less so among laypeople.” 

“I\marginnote{21.1} say that the five things prescribed by the brahmins for making merit are prerequisites of the mind for developing a mind free of enmity and ill will. 

Take\marginnote{21.3} a mendicant who speaks the truth. Thinking, ‘I’m truthful,’ they find inspiration in the meaning and the teaching, and find joy connected with the teaching. And I say that joy connected with the skillful is a prerequisite of the mind for developing a mind free of enmity and ill will. 

Take\marginnote{21.7} a mendicant who is fervent … is celibate … does lots of recitation … and is very generous. Thinking, ‘I’m very generous,’ they find inspiration in the meaning and the teaching, and find joy connected with the teaching. And I say that joy connected with the skillful is a prerequisite of the mind for developing a mind free of enmity and ill will. I say that these five things prescribed by the brahmins for making merit are prerequisites of the mind for developing a mind free of enmity and ill will.” 

When\marginnote{22.1} he had spoken, Subha said to him, “Mister Gotama, I have heard that the ascetic Gotama knows a path to company with divinity.” 

“What\marginnote{22.4} do you think, student? Is the village of \textsanskrit{Naḷakāra} nearby?” 

“Yes\marginnote{22.6} it is, sir.” 

“What\marginnote{22.7} do you think, student? Suppose a person was born and raised in \textsanskrit{Naḷakāra}. And as soon as they left the town some people asked them for the road to \textsanskrit{Naḷakāra}. Would they be slow or hesitant to answer?” 

“No,\marginnote{22.10} Mister Gotama. Why is that? Because they were born and raised in \textsanskrit{Naḷakāra}. They’re well acquainted with all the roads to the village.” 

“Still,\marginnote{22.13} it’s possible they might be slow or hesitant to answer. But the Realized One is never slow or hesitant when questioned about the realm of divinity or the practice that leads to the realm of divinity. I understand divinity, the realm of divinity, and the practice that leads to the realm of divinity, practicing in accordance with which one is reborn in the realm of divinity.” 

“Mister\marginnote{23.1} Gotama, I have heard that the ascetic Gotama teaches a path to company with divinity. Please teach me that path.”\footnote{As at \href{https://suttacentral.net/dn13/en/sujato\#39.2}{DN 13:39.2}. \textsanskrit{Sāriputta} teaches a similar path at \href{https://suttacentral.net/mn97/en/sujato\#31.3}{MN 97:31.3}, but without being asked. } 

“Well\marginnote{23.4} then, student, listen and apply your mind well, I will speak.” 

“Yes,\marginnote{23.5} sir,” replied Subha. The Buddha said this: 

“And\marginnote{24.1} what is a path to company with divinity? Firstly, a mendicant meditates spreading a heart full of love to one direction, and to the second, and to the third, and to the fourth. In the same way above, below, across, everywhere, all around, they spread a heart full of love to the whole world—abundant, expansive, limitless, free of enmity and ill will. When the heart’s release by love has been developed like this, any limited deeds they’ve done don’t remain or persist there. Suppose there was a powerful horn blower. They’d easily make themselves heard in the four quarters. In the same way, when the heart’s release by love has been developed like this, any limited deeds they’ve done don’t remain or persist there. This is a path to company with divinity. 

Furthermore,\marginnote{25{-}27.1} a mendicant meditates spreading a heart full of compassion … 

They\marginnote{25{-}27.2} meditate spreading a heart full of rejoicing … 

They\marginnote{25{-}27.3} meditate spreading a heart full of equanimity to one direction, and to the second, and to the third, and to the fourth. In the same way above, below, across, everywhere, all around, they spread a heart full of equanimity to the whole world—abundant, expansive, limitless, free of enmity and ill will. When the heart’s release by equanimity has been developed and cultivated like this, any limited deeds they’ve done don’t remain or persist there. Suppose there was a powerful horn blower. They’d easily make themselves heard in the four quarters. In the same way, when the heart’s release by equanimity has been developed and cultivated like this, any limited deeds they’ve done don’t remain or persist there. This too is a path to company with divinity.” 

When\marginnote{28.1} he had spoken, Subha said to him, “Excellent, Mister Gotama! Excellent! As if he were righting the overturned, or revealing the hidden, or pointing out the path to the lost, or lighting a lamp in the dark so people with clear eyes can see what’s there, Mister Gotama has made the teaching clear in many ways. I go for refuge to Mister Gotama, to the teaching, and to the mendicant \textsanskrit{Saṅgha}.\footnote{He is said to have gone for refuge again in \href{https://suttacentral.net/mn135/en/sujato\#21.4}{MN 135:21.4} and \href{https://suttacentral.net/dn10/en/sujato\#2.37.9}{DN 10:2.37.9}. } From this day forth, may Mister Gotama remember me as a lay follower who has gone for refuge for life. Well, now, Mister Gotama, I must go. I have many duties, and much to do.”\footnote{Echoing his earlier acknowledgment of the many duties of lay folk at \href{https://suttacentral.net/mn99/en/sujato\#5.2}{MN 99:5.2}. } 

“Please,\marginnote{29.1} student, go at your convenience.” And then Subha approved and agreed with what the Buddha said. He got up from his seat, bowed, and respectfully circled the Buddha, keeping him on his right, before leaving. 

Now\marginnote{30.1} at that time the brahmin \textsanskrit{Jānussoṇi} drove out from \textsanskrit{Sāvatthī} in the middle of the day in an all-white chariot drawn by mares. He saw the student Subha coming off in the distance, and said to him, “So, Mister \textsanskrit{Bhāradvāja}, where are you coming from in the middle of the day?” 

“Just\marginnote{30.5} now, good sir, I’ve come from the presence of the ascetic Gotama.” 

“What\marginnote{30.6} do you think of the ascetic Gotama’s lucidity of wisdom? Do you think he’s astute?” 

“My\marginnote{30.7} good man, who am I to judge the ascetic Gotama’s lucidity of wisdom? You’d really have to be on the same level to judge his lucidity of wisdom.” 

“Mister\marginnote{30.9} \textsanskrit{Bhāradvāja} praises the ascetic Gotama with lofty praise indeed.” 

“Who\marginnote{30.10} am I to praise the ascetic Gotama? He is praised by the praised as the best among gods and humans. The five things that the brahmins prescribe for making merit and succeeding in the skillful he says are prerequisites of the mind for developing a mind free of enmity and ill will.” 

When\marginnote{31.1} he had spoken, \textsanskrit{Jānussoṇi} got down from his chariot, arranged his robe over one shoulder, raised his joined palms toward the Buddha, and expressed this heartfelt sentiment three times, “King Pasenadi of Kosala is lucky, so very lucky that the Realized One, the perfected one, the fully awakened Buddha is living in his realm!” 

%
\section*{{\suttatitleacronym MN 100}{\suttatitletranslation With Saṅgārava }{\suttatitleroot Saṅgāravasutta}}
\addcontentsline{toc}{section}{\tocacronym{MN 100} \toctranslation{With Saṅgārava } \tocroot{Saṅgāravasutta}}
\markboth{With Saṅgārava }{Saṅgāravasutta}
\extramarks{MN 100}{MN 100}

\scevam{So\marginnote{1.1} I have heard. }At one time the Buddha was wandering in the land of the Kosalans together with a large \textsanskrit{Saṅgha} of mendicants. Now at that time a brahmin lady named \textsanskrit{Dhanañjānī} was residing at \textsanskrit{Caṇḍalakappa}. She was devoted to the Buddha, the teaching, and the \textsanskrit{Saṅgha}. Once, she tripped and expressed this heartfelt sentiment three times:\footnote{A brahmin lady of the same name also trips and utters praise of the Buddha in \href{https://suttacentral.net/sn7.1/en/sujato}{SN 7.1}. Nonetheless, most other details of the stories are different, including the location and the subsequent teaching. There, her husband gets angry, goes to confront by the Buddha, but is converted by his teaching on the harm of anger. The historical relation between these two texts is unclear. Regardless, they do attest to the use of Buddhist invocations from an early date. A similar invocation is expressed for the Great Steward at \href{https://suttacentral.net/dn19/en/sujato\#58.5}{DN 19:58.5}. } 

“Homage\marginnote{2.3} to that Blessed One, the perfected one, the fully awakened Buddha! 

Homage\marginnote{2.4} to that Blessed One, the perfected one, the fully awakened Buddha! 

Homage\marginnote{2.5} to that Blessed One, the perfected one, the fully awakened Buddha!” 

Now\marginnote{3.1} at that time the student \textsanskrit{Saṅgārava} was residing in \textsanskrit{Caṇḍalakappa}. He was young, newly tonsured; he was sixteen years old. He had mastered the three Vedas, together with their vocabularies and ritual performance, their phonology and word classification, and the testaments as fifth. He knew them word-by-word, and their grammar. He was well versed in cosmology and the marks of a great man.\footnote{A “brahmin” \textsanskrit{Saṅgārava} of \textsanskrit{Sāvatthī} appears in a number of discourses, and his spiritual progression can be traced. He practiced purification in water (\href{https://suttacentral.net/sn7.21/en/sujato}{SN 7.21}) and (apparently later) asked the Buddha about sacrifice (\href{https://suttacentral.net/an3.60/en/sujato}{AN 3.60}). He further asked how hymns are remembered (\href{https://suttacentral.net/sn46.55/en/sujato}{SN 46.55}, \href{https://suttacentral.net/an5.193/en/sujato}{AN 5.193}). In all these discourses he was said to have gone for refuge at the end, as does the \textsanskrit{Saṅgārava} of the current discourse. Finally he asks the more meaningful spiritual question about the near shore and far shore (\href{https://suttacentral.net/an10.117/en/sujato}{AN 10.117}, \href{https://suttacentral.net/an10.169/en/sujato}{AN 10.169}). It is possible that the teenage “student” (\textit{\textsanskrit{māṇava}}) \textsanskrit{Saṅgārava} living at \textsanskrit{Caṇḍalakappa} graduated and moved to nearby \textsanskrit{Sāvatthī} where he became known as the “brahmin” \textsanskrit{Saṅgārava}. This would only make sense, however, if we disregard the statements on going for refuge, as he clearly followed the Brahmanical path for quite some time. } 

Hearing\marginnote{3.2} \textsanskrit{Dhanañjānī}’s exclamation, he said to her, “The brahmin lady \textsanskrit{Dhanañjānī} is a disgrace! Though brahmins proficient in the three Vedas are found, she praises that shaveling, that fake ascetic.”\footnote{The relationship between \textsanskrit{Saṅgārava} and \textsanskrit{Dhanañjānī} is not specified in the text, but she was probably the wife of his brahmin teacher. } 

“But\marginnote{3.5} dearest boy, you don’t understand the Buddha’s ethics and wisdom. If you did, you’d never think of abusing or insulting him.” 

“Well\marginnote{3.7} then, ma’am, let me know when the Buddha arrives in \textsanskrit{Caṇḍalakappa}.” 

“I\marginnote{3.8} will, dearest,” she replied. 

And\marginnote{4.1} then the Buddha, traveling stage by stage in the Kosalan lands, arrived at \textsanskrit{Caṇḍalakappa}, where he stayed in the mango grove of the Todeyya brahmins. 

\textsanskrit{Dhanañjānī}\marginnote{5.1} heard that he had arrived. So she went to \textsanskrit{Saṅgārava} and told him, adding, “Please, dearest boy, go at your convenience.” 

“Yes,\marginnote{5.5} ma’am,” replied \textsanskrit{Saṅgārava}. He went to the Buddha and exchanged greetings with him.\footnote{Text drops a \textit{-ti}, apparently by mistake; \textit{\textsanskrit{evaṁ} bhoti} should be \textit{\textsanskrit{evaṁ} \textsanskrit{bhotīti}}. } When the greetings and polite conversation were over, he sat down to one side and said to the Buddha: 

“Mister\marginnote{6.1} Gotama, there are some ascetics and brahmins who claim to have mastered the fundamentals of the spiritual life having attained perfection and consummation of insight in this life.\footnote{For the phrase “having attained perfection and consummation of insight”, see \href{https://suttacentral.net/mn77/en/sujato\#15.6}{MN 77:15.6}. } Where do you stand regarding these?” 

“I\marginnote{7.1} say there is a diversity among those who claim to have mastered the fundamentals of the spiritual life having attained perfection and consummation of insight in this life. There are some ascetics and brahmins who are oral transmitters. Through oral transmission they claim to have mastered the fundamentals of the spiritual life. For example, the brahmins who are proficient in the three Vedas. There are some ascetics and brahmins who solely by mere faith claim to have mastered the fundamentals of the spiritual life. For example, those who rely on logic and inquiry.\footnote{Elsewhere, reason and faith are regarded as two distinct means of knowledge (eg. \href{https://suttacentral.net/mn95/en/sujato\#14.5}{MN 95:14.5}), and their conflation here is probably a redaction error. In the Sanskrit parallel, faith applies to all the means of knowledge, in the sense that they have faith in those means. (Lixiang Zhang, \emph{Das \textsanskrit{Śaṁkarasūtra}}, 2004, cited in \textsanskrit{Anālayo}, \emph{Comparative Study} volume 2.) } There are some ascetics and brahmins who, having directly known for themselves the principle regarding teachings not learned before from another, claim to have mastered the fundamentals of the spiritual life. I am one of those. And here’s a way to understand that I am one of them. 

Before\marginnote{9.1} my awakening—when I was still unawakened but intent on awakening—I thought: ‘Life at home is cramped and dirty, life gone forth is wide open. It’s not easy for someone living at home to lead the spiritual life utterly full and pure, like a polished shell. Why don’t I shave off my hair and beard, dress in ocher robes, and go forth from the lay life to homelessness?’ Some time later, while still with pristine black hair, blessed with youth, in the prime of life—though my mother and father wished otherwise, weeping with tearful faces—I shaved off my hair and beard, dressed in ocher robes, and went forth from the lay life to homelessness. 

Once\marginnote{9.6} I had gone forth I set out to discover what is skillful, seeking the supreme state of sublime peace. I approached \textsanskrit{Āḷāra} \textsanskrit{Kālāma} and said to him,\footnote{See \href{https://suttacentral.net/mn26/en/sujato\#14.1}{MN 26:14.1}ff. for notes. } ‘Reverend \textsanskrit{Kālāma}, I wish to lead the spiritual life in this teaching and training.’ 

\textsanskrit{Āḷāra}\marginnote{9.8} \textsanskrit{Kālāma} replied, ‘Stay, venerable. This teaching is such that a sensible person can soon realize their own tradition with their own insight and live having achieved it.’ 

I\marginnote{9.11} quickly memorized that teaching. As far as lip-recital and verbal repetition went, I spoke the doctrine of knowledge, the elder doctrine. I claimed to know and see, and so did others. 

Then\marginnote{9.13} it occurred to me, ‘It is not solely by mere faith that \textsanskrit{Āḷāra} \textsanskrit{Kālāma} declares: “I realize this teaching with my own insight, and live having achieved it.” Surely he meditates knowing and seeing this teaching.’ 

So\marginnote{10.1} I approached \textsanskrit{Āḷāra} \textsanskrit{Kālāma} and said to him: ‘Reverend \textsanskrit{Kālāma}, to what extent do you say you’ve realized this teaching with your own insight?’ When I said this, he declared the dimension of nothingness. 

Then\marginnote{10.4} it occurred to me, ‘It’s not just \textsanskrit{Āḷāra} \textsanskrit{Kālāma} who has faith, energy, mindfulness, immersion, and wisdom; I too have these things. Why don’t I make an effort to realize the same teaching that \textsanskrit{Āḷāra} \textsanskrit{Kālāma} says he has realized with his own insight?’ I quickly realized that teaching with my own insight, and lived having achieved it. 

So\marginnote{10.12} I approached \textsanskrit{Āḷāra} \textsanskrit{Kālāma} and said to him, ‘Reverend \textsanskrit{Kālāma}, is it up to this point that you realized this teaching with your own insight, and declare having achieved it?’ 

‘I\marginnote{10.14} have, reverend.’ 

‘I\marginnote{10.15} too have realized this teaching with my own insight up to this point, and live having achieved it.’ 

‘We\marginnote{10.16} are fortunate, reverend, so very fortunate to see a venerable such as yourself as one of our spiritual companions! So the teaching that I’ve realized with my own insight, and declare having achieved it, you’ve realized with your own insight, and dwell having achieved it. The teaching that you’ve realized with your own insight, and dwell having achieved it, I’ve realized with my own insight, and declare having achieved it. So the teaching that I know, you know, and the teaching you know, I know. I am like you and you are like me. Come now, reverend! We should both lead this community together.’ 

And\marginnote{10.23} that is how my tutor \textsanskrit{Āḷāra} \textsanskrit{Kālāma} placed me, his pupil, on the same position as him, and honored me with lofty praise. 

Then\marginnote{10.24} it occurred to me, ‘This teaching doesn’t lead to disillusionment, dispassion, cessation, peace, insight, awakening, and extinguishment. It only leads as far as rebirth in the dimension of nothingness.’ Realizing that this teaching was inadequate, I left disappointed. 

I\marginnote{11.1} set out to discover what is skillful, seeking the supreme state of sublime peace. I approached Uddaka son of \textsanskrit{Rāma} and said to him, ‘Reverend, I wish to lead the spiritual life in this teaching and training.’ 

Uddaka\marginnote{11.3} replied, ‘Stay, venerable. This teaching is such that a sensible person can soon realize their own tradition with their own insight and live having achieved it.’ 

I\marginnote{11.6} quickly memorized that teaching. As far as lip-recital and verbal repetition went, I spoke the doctrine of knowledge, the elder doctrine. I claimed to know and see, and so did others. 

Then\marginnote{11.8} it occurred to me, ‘It is not solely by mere faith that \textsanskrit{Rāma} declared: “I realize this teaching with my own insight, and live having achieved it.” Surely he meditated knowing and seeing this teaching.’ 

So\marginnote{11.11} I approached Uddaka son of \textsanskrit{Rāma} and said to him, ‘Reverend, to what extent did \textsanskrit{Rāma} say he’d realized this teaching with his own insight?’ When I said this, Uddaka son of \textsanskrit{Rāma} declared the dimension of neither perception nor non-perception. 

Then\marginnote{11.14} it occurred to me, ‘It’s not just \textsanskrit{Rāma} who had faith, energy, mindfulness, immersion, and wisdom; I too have these things. Why don’t I make an effort to realize the same teaching that \textsanskrit{Rāma} said he had realized with his own insight?’ I quickly realized that teaching with my own insight, and lived having achieved it. 

So\marginnote{12.1} I approached Uddaka son of \textsanskrit{Rāma} and said to him, ‘Reverend, had \textsanskrit{Rāma} realized this teaching with his own insight up to this point, and declared having achieved it?’ 

‘He\marginnote{12.3} had, reverend.’ 

‘I\marginnote{12.4} too have realized this teaching with my own insight up to this point, and live having achieved it.’ 

‘We\marginnote{12.5} are fortunate, reverend, so very fortunate to see a venerable such as yourself as one of our spiritual companions! So the teaching that \textsanskrit{Rāma} had realized with his own insight, and declared having achieved it, you’ve realized with your own insight, and dwell having achieved it. The teaching that you’ve realized with your own insight, and dwell having achieved it, \textsanskrit{Rāma} had realized with his own insight, and declared having achieved it. So the teaching that \textsanskrit{Rāma} directly knew, you know, and the teaching you know, \textsanskrit{Rāma} directly knew. \textsanskrit{Rāma} was like you and you are like \textsanskrit{Rāma}. Come now, reverend! You should lead this community.’ And that is how my spiritual companion Uddaka son of \textsanskrit{Rāma} placed me in the position of a tutor, and honored me with lofty praise. 

Then\marginnote{12.13} it occurred to me, ‘This teaching doesn’t lead to disillusionment, dispassion, cessation, peace, insight, awakening, and extinguishment. It only leads as far as rebirth in dimension of neither perception nor non-perception.’ Realizing that this teaching was inadequate, I left disappointed. 

I\marginnote{13.1} set out to discover what is skillful, seeking the supreme state of sublime peace. Traveling stage by stage in the Magadhan lands, I arrived at \textsanskrit{Senānigama} in \textsanskrit{Uruvelā}. There I saw a delightful park, a lovely grove with a flowing river that was clean and charming, with smooth banks. And nearby was a village to resort to for alms. Then it occurred to me, ‘This park is truly delightful, a lovely grove with a flowing river that’s clean and charming, with smooth banks. And nearby there’s a village to resort to for alms. This is good enough for striving for a gentleman wanting to strive.’ So I sat down right there, thinking: ‘This is good enough for striving.’ 

And\marginnote{14.1} then these three similes, which were neither supernaturally inspired, nor learned before in the past, occurred to me.\footnote{See \href{https://suttacentral.net/mn36/en/sujato\#17.1}{MN 36:17.1}ff. for notes. } Suppose there was a green, sappy log, and it was lying in water. Then a person comes along with a drill-stick, thinking to light a fire and produce heat. What do you think, \textsanskrit{Bhāradvāja}? By drilling the stick against that green, sappy log lying in water, could they light a fire and produce heat?” 

“No,\marginnote{14.7} Mister Gotama. Why is that? Because it’s a green, sappy log, and it’s lying in the water. That person will eventually get weary and frustrated.” 

“In\marginnote{14.11} the same way, there are ascetics and brahmins who don’t live withdrawn in body and mind from sensual pleasures. They haven’t internally given up or stilled desire, affection, infatuation, thirst, and passion for sensual pleasures. Regardless of whether or not they suffer painful, sharp, severe, acute feelings due to overexertion, they are incapable of knowledge and vision, of supreme awakening. This was the first example that occurred to me. 

Then\marginnote{15.1} a second example occurred to me. Suppose there was a green, sappy log, and it was lying on dry land far from the water. Then a person comes along with a drill-stick, thinking to light a fire and produce heat. What do you think, \textsanskrit{Bhāradvāja}? By drilling the stick against that green, sappy log on dry land far from water, could they light a fire and produce heat?” 

“No,\marginnote{15.7} Mister Gotama. Why is that? Because it’s still a green, sappy log, despite the fact that it’s lying on dry land far from water. That person will eventually get weary and frustrated.” “In the same way, there are ascetics and brahmins who live withdrawn in body and mind from sensual pleasures. But they haven’t internally given up or stilled desire, affection, infatuation, thirst, and passion for sensual pleasures. Regardless of whether or not they suffer painful, sharp, severe, acute feelings due to overexertion, they are incapable of knowledge and vision, of supreme awakening. This was the second example that occurred to me. 

Then\marginnote{16.1} a third example occurred to me. Suppose there was a dried up, withered log, and it was lying on dry land far from the water. Then a person comes along with a drill-stick, thinking to light a fire and produce heat. What do you think, \textsanskrit{Bhāradvāja}? By drilling the stick against that dried up, withered log on dry land far from water, could they light a fire and produce heat?” 

“Yes,\marginnote{16.7} Mister Gotama. Why is that? Because it’s a dried up, withered log, and it’s lying on dry land far from water.” 

“In\marginnote{16.10} the same way, there are ascetics and brahmins who live withdrawn in body and mind from sensual pleasures. And they have internally given up and stilled desire, affection, infatuation, thirst, and passion for sensual pleasures. Regardless of whether or not they suffer painful, sharp, severe, acute feelings due to overexertion, they are capable of knowledge and vision, of supreme awakening. This was the third example that occurred to me. These are the three similes, which were neither supernaturally inspired, nor learned before in the past, that occurred to me. 

Then\marginnote{17.1} it occurred to me, ‘Why don’t I, with teeth clenched and tongue pressed against the roof of my mouth, squeeze, squash, and scorch mind with mind?’ So that’s what I did, until sweat ran from my armpits. It was like when a strong man grabs a weaker man by the head or throat or shoulder and squeezes, squashes, and crushes them. In the same way, with teeth clenched and tongue pressed against the roof of my mouth, I squeezed, squashed, and crushed mind with mind until sweat ran from my armpits. My energy was roused up and unflagging, and my mindfulness was established and lucid, but my body was disturbed, not tranquil, because I’d pushed too hard with that painful striving. 

Then\marginnote{18.1} it occurred to me, ‘Why don’t I practice the breathless absorption?’ So I cut off my breathing through my mouth and nose. But then winds came out my ears making a loud noise, like the puffing of a blacksmith’s bellows. My energy was roused up and unflagging, and my mindfulness was established and lucid, but my body was disturbed, not tranquil, because I’d pushed too hard with that painful striving. 

Then\marginnote{19.1} it occurred to me, ‘Why don’t I keep practicing the breathless absorption?’ So I cut off my breathing through my mouth and nose. But then strong winds ground my head, like a strong man was drilling into my head with a sharp point. My energy was roused up and unflagging, and my mindfulness was established and lucid, but my body was disturbed, not tranquil, because I’d pushed too hard with that painful striving. 

Then\marginnote{20.1} it occurred to me, ‘Why don’t I keep practicing the breathless absorption?’ So I cut off my breathing through my mouth and nose. But then I got a severe headache, like a strong man was tightening a tough leather strap around my head. My energy was roused up and unflagging, and my mindfulness was established and lucid, but my body was disturbed, not tranquil, because I’d pushed too hard with that painful striving. 

Then\marginnote{21.1} it occurred to me, ‘Why don’t I keep practicing the breathless absorption?’ So I cut off my breathing through my mouth and nose. But then strong winds carved up my belly, like a deft butcher or their apprentice was slicing my belly open with a sharp meat cleaver. My energy was roused up and unflagging, and my mindfulness was established and lucid, but my body was disturbed, not tranquil, because I’d pushed too hard with that painful striving. 

Then\marginnote{22.1} it occurred to me, ‘Why don’t I keep practicing the breathless absorption?’ So I cut off my breathing through my mouth and nose. But then there was an intense burning in my body, like two strong men grabbing a weaker man by the arms to burn and scorch him on a pit of glowing coals. My energy was roused up and unflagging, and my mindfulness was established and lucid, but my body was disturbed, not tranquil, because I’d pushed too hard with that painful striving. 

Then\marginnote{23.1} some deities saw me and said, ‘The ascetic Gotama is dead.’ Others said, ‘He’s not dead, but he’s dying.’ Others said, ‘He’s not dead or dying. The ascetic Gotama is a perfected one, for that is how the perfected ones live.’ 

Then\marginnote{24.1} it occurred to me, ‘Why don’t I practice completely cutting off food?’ 

But\marginnote{24.3} deities came to me and said, ‘Good sir, don’t practice totally cutting off food. If you do, we’ll infuse heavenly nectar into your pores and you will live on that.’ 

Then\marginnote{24.7} it occurred to me, ‘If I claim to be completely fasting while these deities are infusing heavenly nectar in my pores, that would be a lie on my part.’ So I dismissed those deities, saying, ‘There’s no need.’ 

Then\marginnote{25.1} it occurred to me, ‘Why don’t I just take a little bit of food each time, a handful of broth made from mung beans, horse gram, chickpeas, or green gram?’ So that’s what I did, until my body became extremely emaciated. Due to eating so little, my major and minor limbs became like the joints of an eighty-year-old or a dying man, my bottom became like a camel’s hoof, my vertebrae stuck out like beads on a string, and my ribs were as gaunt as the broken-down rafters on an old barn. Due to eating so little, the gleam of my eyes sank deep in their sockets, like the gleam of water sunk deep down a well. Due to eating so little, my scalp shriveled and withered like a green bitter-gourd in the wind and sun. Due to eating so little, the skin of my belly stuck to my backbone, so that when I tried to rub the skin of my belly I grabbed my backbone, and when I tried to rub my backbone I rubbed the skin of my belly. Due to eating so little, when I tried to urinate or defecate I fell face down right there. Due to eating so little, when I tried to relieve my body by rubbing my limbs with my hands, the hair, rotted at its roots, fell out. 

Then\marginnote{26.1} some people saw me and said: ‘The ascetic Gotama is black.’ Some said: ‘He’s not black, he’s brown.’ Some said: ‘He’s neither black nor brown. The ascetic Gotama has tawny skin.’ That’s how far the pure, bright complexion of my skin had been ruined by taking so little food. 

Then\marginnote{27.1} it occurred to me, ‘Whatever ascetics and brahmins have experienced painful, sharp, severe, acute feelings due to overexertion—whether in the past, future, or present—this is as far as it goes, no-one has done more than this. But I have not achieved any superhuman distinction in knowledge and vision worthy of the noble ones by this severe, grueling work. Could there be another path to awakening?’ 

Then\marginnote{28.1} it occurred to me, ‘I recall sitting in the cool shade of a black plum tree while my father the Sakyan was off working. Quite secluded from sensual pleasures, secluded from unskillful qualities, I entered and remained in the first absorption, which has the rapture and bliss born of seclusion, while placing the mind and keeping it connected. Could that be the path to awakening?’ Stemming from that memory came the realization: ‘\emph{That} is the path to awakening!’ 

Then\marginnote{29.1} it occurred to me, ‘Why am I afraid of that pleasure, for it has nothing to do with sensual pleasures or unskillful qualities?’ I thought, ‘I’m not afraid of that pleasure, for it has nothing to do with sensual pleasures or unskillful qualities.’ 

Then\marginnote{30.1} it occurred to me, ‘I can’t achieve that pleasure with a body so excessively emaciated. Why don’t I eat some solid food, some rice and porridge?’ So I ate some solid food. 

Now\marginnote{30.5} at that time the five mendicants were attending on me, thinking, ‘The ascetic Gotama will tell us of any truth that he realizes.’ But when I ate some solid food, they left disappointed in me, saying, ‘The ascetic Gotama has become indulgent; he has strayed from the struggle and returned to indulgence.’ 

After\marginnote{31{-}33.1} eating solid food and gathering my strength, quite secluded from sensual pleasures, secluded from unskillful qualities, I entered and remained in the first absorption … As the placing of the mind and keeping it connected were stilled, I entered and remained in the second absorption … third absorption … fourth absorption. 

When\marginnote{34.1} my mind had immersed in \textsanskrit{samādhi} like this—purified, bright, flawless, rid of corruptions, pliable, workable, steady, and imperturbable—I extended it toward recollection of past lives. I recollected many past lives. That is: one, two, three, four, five, ten, twenty, thirty, forty, fifty, a hundred, a thousand, a hundred thousand rebirths; many eons of the world contracting, many eons of the world expanding, many eons of the world contracting and expanding. And so I recollected my many kinds of past lives, with features and details. 

This\marginnote{35.1} was the first knowledge, which I achieved in the first watch of the night. Ignorance was destroyed and knowledge arose; darkness was destroyed and light arose, as happens for a meditator who is diligent, keen, and resolute. 

When\marginnote{36{-}38.1} my mind had immersed in \textsanskrit{samādhi} like this—purified, bright, flawless, rid of corruptions, pliable, workable, steady, and imperturbable—I extended it toward knowledge of the death and rebirth of sentient beings. With clairvoyance that is purified and superhuman, I saw sentient beings passing away and being reborn—inferior and superior, beautiful and ugly, in a good place or a bad place. I understood how sentient beings are reborn according to their deeds … 

This\marginnote{36{-}38.3} was the second knowledge, which I achieved in the middle watch of the night. Ignorance was destroyed and knowledge arose; darkness was destroyed and light arose, as happens for a meditator who is diligent, keen, and resolute. 

When\marginnote{39.1} my mind had immersed in \textsanskrit{samādhi} like this—purified, bright, flawless, rid of corruptions, pliable, workable, steady, and imperturbable—I extended it toward knowledge of the ending of defilements. I truly understood: ‘This is suffering’ … ‘This is the origin of suffering’ … ‘This is the cessation of suffering’ … ‘This is the practice that leads to the cessation of suffering’. I truly understood: ‘These are defilements’ … ‘This is the origin of defilements’ … ‘This is the cessation of defilements’ … ‘This is the practice that leads to the cessation of defilements’. 

Knowing\marginnote{40.1} and seeing like this, my mind was freed from the defilements of sensuality, desire to be reborn, and ignorance. When it was freed, I knew it was freed. 

I\marginnote{40.3} understood: ‘Rebirth is ended; the spiritual journey has been completed; what had to be done has been done; there is nothing further for this place.’ 

This\marginnote{41.1} was the third knowledge, which I achieved in the last watch of the night. Ignorance was destroyed and knowledge arose; darkness was destroyed and light arose, as happens for a meditator who is diligent, keen, and resolute.” 

When\marginnote{42.1} he had spoken, \textsanskrit{Saṅgārava} said to the Buddha,\footnote{This final passage has been widely misunderstood. K.R. Norman (“The Buddha’s View of Devas”) argued that the passage is corrupt, and Bodhi, while not endorsing Norman’s emendations, agrees. However, while there are difficulties in readings and interpretations, for the most part the Pali makes good sense. The opening showed \textsanskrit{Saṅgārava} being overreactive and dogmatic, an impression reinforced here by his sudden changes of topic and misunderstandings. Note too that the \textsanskrit{Saṅgārava} of \href{https://suttacentral.net/an3.60/en/sujato}{AN 3.60} is evasive and a little rude, reminiscent of the prickly and unpredictable personality on display in the current sutta. I think the bulk of the difficulties simply represent a lively but somewhat chaotic exchange, although textual corruption cannot be ruled out. However, note that Sanskrit fragments indicate that in that version this discussion happened before the bulk of the sutta teachings. I agree with \textsanskrit{Anālayo} that the Sanskrit form makes better sense (\emph{Comparative Study}, volume ii, page 583). } “The striving of Mister Gotama was indeed assiduous and that of a true person,\footnote{Read \textit{\textsanskrit{aṭṭhita}} as \textit{\textsanskrit{ā}+\textsanskrit{ṭhita}}, “assiduous”; see \href{https://suttacentral.net/ja242/en/sujato\#2.1}{Ja 242:2.1} in the same sense, and compare Sanskrit \textit{\textsanskrit{āsthita}}. | I follow the variant \textit{vata} rather than \textit{\textsanskrit{vataṁ}}, as exclamations with repeated \textit{vata} are idiomatic. } since he is a perfected one, a fully awakened Buddha. But Mister Gotama, do gods survive?”\footnote{The sense of \textit{atthi \textsanskrit{devā}} is established from such passages as \href{https://suttacentral.net/mn90/en/sujato\#13.2}{MN 90:13.2}. } 

“I’ve\marginnote{42.5} understood about gods in terms of causes.”\footnote{\textit{\textsanskrit{Ṭhānaso}} here means “in terms of cause” or “by way of cause”. It typically appears in the context of kamma and rebirth, followed by a verb of knowing, eg. \textit{\textsanskrit{ṭhānaso} hetuso \textsanskrit{vipākaṁ} \textsanskrit{yathābhūtaṁ} \textsanskrit{pajānāti}}, “understands results (of kamma) in terms of grounds and causes” (\href{https://suttacentral.net/mn12/en/sujato\#11.2}{MN 12:11.2}). The Buddha’s point is that the survival of gods is not a metaphysical absolute but depends on their kamma. | As usual, the Buddha answers such questions framed with \textit{atthi} indirectly; see note on \href{https://suttacentral.net/mn90/en/sujato\#13.2}{MN 90:13.2}. | That \textit{adhideva} means “about the gods” is established at \href{https://suttacentral.net/mn90/en/sujato\#17.8}{MN 90:17.8}. } 

“But\marginnote{42.7} Mister Gotama, when asked ‘Do gods survive?’ why did you say that you have understood about gods in terms of causes? That being so, is it not hollow and false?”\footnote{Normally such a phrase would include a reference to the Buddha’s speech (as eg. \href{https://suttacentral.net/mn80/en/sujato\#3.15}{MN 80:3.15}). But this is lacking in the Pali so I omit it. } 

“When\marginnote{42.9} asked ‘Do gods survive’, whether you reply ‘Gods survive’ or ‘I’ve understood in terms of causes’ a sensible person would come to the categorical conclusion in this matter that gods survive.”\footnote{This should be read in light of \href{https://suttacentral.net/sn12.15/en/sujato\#2.2}{SN 12.15:2.2}, where “survival” or “existence” (\textit{atthita}) is understood by seeing “origin” (\textit{samudaya}), while non-existence is understood by seeing “cessation”. Here too the survival of gods is understood in terms of cause. } 

“But\marginnote{42.12} why didn’t you say that in the first place?”\footnote{The Buddha’s answer relied on the implication that since gods depend on causes they “survive” in the sense of being reborn, but not in the sense of eternal existence. \textsanskrit{Saṅgārava} didn’t follow the argument. } 

“It\marginnote{42.13} is agreed by the eminent in the world that gods survive.”\footnote{I think the Buddha felt that since it is agreed “by the eminent” (\textit{uccena})—which would include the brahmanical teachers of \textsanskrit{Saṅgārava}—that gods are real, this could be assumed. So the Buddha’s response was aimed at clarifying the \emph{kind} of existence that gods have, namely, a conditioned and impermanent existence determined by the kamma. When faced with the same question at \href{https://suttacentral.net/mn90/en/sujato\#13.2}{MN 90:13.2}, the Buddha’s immediate response was to ask for a clarification, so the confusion did not arise. } 

When\marginnote{43.1} he had spoken, \textsanskrit{Saṅgārava} said to the Buddha, “Excellent, Mister Gotama! Excellent! As if he were righting the overturned, or revealing the hidden, or pointing out the path to the lost, or lighting a lamp in the dark so people with clear eyes can see what’s there, Mister Gotama has made the Teaching clear in many ways. I go for refuge to Mister Gotama, to the teaching, and to the mendicant \textsanskrit{Saṅgha}. From this day forth, may Mister Gotama remember me as a lay follower who has gone for refuge for life.” 

%
\backmatter%
\chapter*{Colophon}
\addcontentsline{toc}{chapter}{Colophon}
\markboth{Colophon}{Colophon}

\section*{The Translator}

Bhikkhu Sujato was born as Anthony Aidan Best on 4/11/1966 in Perth, Western Australia. He grew up in the pleasant suburbs of Mt Lawley and Attadale alongside his sister Nicola, who was the good child. His mother, Margaret Lorraine Huntsman née Pinder, said “he’ll either be a priest or a poet”, while his father, Anthony Thomas Best, advised him to “never do anything for money”. He attended Aquinas College, a Catholic school, where he decided to become an atheist. At the University of WA he studied philosophy, aiming to learn what he wanted to do with his life. Finding that what he wanted to do was play guitar, he dropped out. His main band was named Martha’s Vineyard, which achieved modest success in the indie circuit. 

A seemingly random encounter with a roadside joey took him to Thailand, where he entered his first meditation retreat at Wat Ram Poeng, Chieng Mai in 1992. Feeling the call to the Buddha’s path, he took full ordination in Wat Pa Nanachat in 1994, where his teachers were Ajahn Pasanno and Ajahn Jayasaro. In 1997 he returned to Perth to study with Ajahn Brahm at Bodhinyana Monastery. 

He spent several years practicing in seclusion in Malaysia and Thailand before establishing Santi Forest Monastery in Bundanoon, NSW, in 2003. There he was instrumental in supporting the establishment of the Theravada bhikkhuni order in Australia and advocating for women’s rights. He continues to teach in Australia and globally, with a special concern for the moral implications of climate change and other forms of environmental destruction. He has published a series of books of original and groundbreaking research on early Buddhism. 

In 2005 he founded SuttaCentral together with Rod Bucknell and John Kelly. In 2015, seeing the need for a complete, accurate, plain English translation of the Pali texts, he undertook the task, spending nearly three years in isolation on the isle of Qi Mei off the coast of the nation of Taiwan. He completed the four main \textsanskrit{Nikāyas} in 2018, and the early books of the Khuddaka \textsanskrit{Nikāya} were complete by 2021. All this work is dedicated to the public domain and is entirely free of copyright encumbrance. 

In 2019 he returned to Sydney where he established Lokanta Vihara (The Monastery at the End of the World). 

\section*{Creation Process}

Primary source was the digital \textsanskrit{Mahāsaṅgīti} edition of the Pali \textsanskrit{Tipiṭaka}. Translated from the Pali, with reference to several English translations, especially those of Bhikkhu Bodhi.

\section*{The Translation}

This translation was part of a project to translate the four Pali \textsanskrit{Nikāyas} with the following aims: plain, approachable English; consistent terminology; accurate rendition of the Pali; free of copyright. It was made during 2016–2018 while Bhikkhu Sujato was staying in Qimei, Taiwan.

\section*{About SuttaCentral}

SuttaCentral publishes early Buddhist texts. Since 2005 we have provided root texts in Pali, Chinese, Sanskrit, Tibetan, and other languages, parallels between these texts, and translations in many modern languages. Building on the work of generations of scholars, we offer our contribution freely.

SuttaCentral is driven by volunteer contributions, and in addition we employ professional developers. We offer a sponsorship program for high quality translations from the original languages. Financial support for SuttaCentral is handled by the SuttaCentral Development Trust, a charitable trust registered in Australia.

\section*{About Bilara}

“Bilara” means “cat” in Pali, and it is the name of our Computer Assisted Translation (CAT) software. Bilara is a web app that enables translators to translate early Buddhist texts into their own language. These translations are published on SuttaCentral with the root text and translation side by side.

\section*{About SuttaCentral Editions}

The SuttaCentral Editions project makes high quality books from selected Bilara translations. These are published in formats including HTML, EPUB, PDF, and print.

You are welcome to print any of our Editions.

%
\end{document}