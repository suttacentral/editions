\documentclass[12pt,openany]{book}%
\usepackage{lastpage}%
%
\usepackage{ragged2e}
\usepackage{verse}
\usepackage[a-3u]{pdfx}
\usepackage[inner=1in, outer=1in, top=.7in, bottom=1in, papersize={6in,9in}, headheight=13pt]{geometry}
\usepackage{polyglossia}
\usepackage[12pt]{moresize}
\usepackage{soul}%
\usepackage{microtype}
\usepackage{tocbasic}
\usepackage{realscripts}
\usepackage{epigraph}%
\usepackage{setspace}%
\usepackage{sectsty}
\usepackage{fontspec}
\usepackage{marginnote}
\usepackage[bottom]{footmisc}
\usepackage{enumitem}
\usepackage{fancyhdr}
\usepackage{emptypage}
\usepackage{extramarks}
\usepackage{graphicx}
\usepackage{relsize}
\usepackage{etoolbox}

% improve ragged right headings by suppressing hyphenation and orphans. spaceskip plus and minus adjust interword spacing; increase rightskip stretch to make it want to push a word on the first line(s) to the next line; reduce parfillskip stretch to make line length more equal . spacefillskip and xspacefillskip can be deleted to use defaults.
\protected\def\BalancedRagged{
\leftskip     0pt
\rightskip    0pt plus 10em
\spaceskip=1\fontdimen2\font plus .5\fontdimen3\font minus 1.5\fontdimen4\font
\xspaceskip=1\fontdimen2\font plus 1\fontdimen3\font minus 1\fontdimen4\font
\parfillskip  0pt plus 15em
\relax
}

\hypersetup{
colorlinks=true,
urlcolor=black,
linkcolor=black,
citecolor=black,
allcolors=black
}

% use a small amount of tracking on small caps
\SetTracking[ spacing = {25*,166, } ]{ encoding = *, shape = sc }{ 25 }

% add a blank page
\newcommand{\blankpage}{
\newpage
\thispagestyle{empty}
\mbox{}
\newpage
}

% define languages
\setdefaultlanguage[]{english}
\setotherlanguage[script=Latin]{sanskrit}

%\usepackage{pagegrid}
%\pagegridsetup{top-left, step=.25in}

% define fonts
% use if arno sanskrit is unavailable
%\setmainfont{Gentium Plus}
%\newfontfamily\Marginalfont[]{Gentium Plus}
%\newfontfamily\Allsmallcapsfont[RawFeature=+c2sc]{Gentium Plus}
%\newfontfamily\Noligaturefont[Renderer=Basic]{Gentium Plus}
%\newfontfamily\Noligaturecaptionfont[Renderer=Basic]{Gentium Plus}
%\newfontfamily\Fleuronfont[Ornament=1]{Gentium Plus}

% use if arno sanskrit is available. display is applied to \chapter and \part, subhead to \section and \subsection.
\setmainfont[
  FontFace={sb}{n}{Font = {Arno Pro Semibold}},
  FontFace={sb}{it}{Font = {Arno  Pro Semibold Italic}}
]{Arno Pro}

% create commands for using semibold
\DeclareRobustCommand{\sbseries}{\fontseries{sb}\selectfont}
\DeclareTextFontCommand{\textsb}{\sbseries}

\newfontfamily\Marginalfont[RawFeature=+subs]{Arno Pro Regular}
\newfontfamily\Allsmallcapsfont[RawFeature=+c2sc]{Arno Pro}
\newfontfamily\Noligaturefont[Renderer=Basic]{Arno Pro}
\newfontfamily\Noligaturecaptionfont[Renderer=Basic]{Arno Pro Caption}

% chinese fonts
\newfontfamily\cjk{Noto Serif TC}
\newcommand*{\langlzh}[1]{\cjk{#1}\normalfont}%

% logo
\newfontfamily\Logofont{sclogo.ttf}
\newcommand*{\sclogo}[1]{\large\Logofont{#1}}

% use subscript numerals for margin notes
\renewcommand*{\marginfont}{\Marginalfont}

% ensure margin notes have consistent vertical alignment
\renewcommand*{\marginnotevadjust}{-.17em}

% use compact lists
\setitemize{noitemsep,leftmargin=1em}
\setenumerate{noitemsep,leftmargin=1em}
\setdescription{noitemsep, style=unboxed, leftmargin=1em}

% style ToC
\DeclareTOCStyleEntries[
  raggedentrytext,
  linefill=\hfill,
  pagenumberwidth=.5in,
  pagenumberformat=\normalfont,
  entryformat=\normalfont
]{tocline}{chapter,section}


  \setlength\topsep{0pt}%
  \setlength\parskip{0pt}%

% define new \centerpars command for use in ToC. This ensures centering, proper wrapping, and no page break after
\def\startcenter{%
  \par
  \begingroup
  \leftskip=0pt plus 1fil
  \rightskip=\leftskip
  \parindent=0pt
  \parfillskip=0pt
}
\def\stopcenter{%
  \par
  \endgroup
}
\long\def\centerpars#1{\startcenter#1\stopcenter}

% redefine part, so that it adds a toc entry without page number
\let\oldcontentsline\contentsline
\newcommand{\nopagecontentsline}[3]{\oldcontentsline{#1}{#2}{}}

    \makeatletter
\renewcommand*\l@part[2]{%
  \ifnum \c@tocdepth >-2\relax
    \addpenalty{-\@highpenalty}%
    \addvspace{0em \@plus\p@}%
    \setlength\@tempdima{3em}%
    \begingroup
      \parindent \z@ \rightskip \@pnumwidth
      \parfillskip -\@pnumwidth
      {\leavevmode
       \setstretch{.85}\large\scshape\centerpars{#1}\vspace*{-1em}\llap{#2}}\par
       \nobreak
         \global\@nobreaktrue
         \everypar{\global\@nobreakfalse\everypar{}}%
    \endgroup
  \fi}
\makeatother

\makeatletter
\def\@pnumwidth{2em}
\makeatother

% define new sectioning command, which is only used in volumes where the pannasa is found in some parts but not others, especially in an and sn

\newcommand*{\pannasa}[1]{\clearpage\thispagestyle{empty}\begin{center}\vspace*{14em}\setstretch{.85}\huge\itshape\scshape\MakeLowercase{#1}\end{center}}

    \makeatletter
\newcommand*\l@pannasa[2]{%
  \ifnum \c@tocdepth >-2\relax
    \addpenalty{-\@highpenalty}%
    \addvspace{.5em \@plus\p@}%
    \setlength\@tempdima{3em}%
    \begingroup
      \parindent \z@ \rightskip \@pnumwidth
      \parfillskip -\@pnumwidth
      {\leavevmode
       \setstretch{.85}\large\itshape\scshape\lowercase{\centerpars{#1}}\vspace*{-1em}\llap{#2}}\par
       \nobreak
         \global\@nobreaktrue
         \everypar{\global\@nobreakfalse\everypar{}}%
    \endgroup
  \fi}
\makeatother

% don't put page number on first page of toc (relies on etoolbox)
\patchcmd{\chapter}{plain}{empty}{}{}

% global line height
\setstretch{1.05}

% allow linebreak after em-dash
\catcode`\—=13
\protected\def—{\unskip\textemdash\allowbreak}

% style headings with secsty. chapter and section are defined per-edition
\partfont{\setstretch{.85}\normalfont\centering\textsc}
\subsectionfont{\setstretch{.95}\normalfont\BalancedRagged}%
\subsubsectionfont{\setstretch{1}\normalfont\itshape\BalancedRagged}

% style elements of suttatitle
\newcommand*{\suttatitleacronym}[1]{\smaller[2]{#1}\vspace*{.3em}}
\newcommand*{\suttatitletranslation}[1]{\linebreak{#1}}
\newcommand*{\suttatitleroot}[1]{\linebreak\smaller[2]\itshape{#1}}

\DeclareTOCStyleEntries[
  indent=3.3em,
  dynindent,
  beforeskip=.2em plus -2pt minus -1pt,
]{tocline}{section}

\DeclareTOCStyleEntries[
  indent=0em,
  dynindent,
  beforeskip=.4em plus -2pt minus -1pt,
]{tocline}{chapter}

\newcommand*{\tocacronym}[1]{\hspace*{-3.3em}{#1}\quad}
\newcommand*{\toctranslation}[1]{#1}
\newcommand*{\tocroot}[1]{(\textit{#1})}
\newcommand*{\tocchapterline}[1]{\bfseries\itshape{#1}}


% redefine paragraph and subparagraph headings to not be inline
\makeatletter
% Change the style of paragraph headings %
\renewcommand\paragraph{\@startsection{paragraph}{4}{\z@}%
            {-2.5ex\@plus -1ex \@minus -.25ex}%
            {1.25ex \@plus .25ex}%
            {\noindent\normalfont\itshape\small}}

% Change the style of subparagraph headings %
\renewcommand\subparagraph{\@startsection{subparagraph}{5}{\z@}%
            {-2.5ex\@plus -1ex \@minus -.25ex}%
            {1.25ex \@plus .25ex}%
            {\noindent\normalfont\itshape\footnotesize}}
\makeatother

% use etoolbox to suppress page numbers on \part
\patchcmd{\part}{\thispagestyle{plain}}{\thispagestyle{empty}}
  {}{\errmessage{Cannot patch \string\part}}

% and to reduce margins on quotation
\patchcmd{\quotation}{\rightmargin}{\leftmargin 1.2em \rightmargin}{}{}
\AtBeginEnvironment{quotation}{\small}

% titlepage
\newcommand*{\titlepageTranslationTitle}[1]{{\begin{center}\begin{large}{#1}\end{large}\end{center}}}
\newcommand*{\titlepageCreatorName}[1]{{\begin{center}\begin{normalsize}{#1}\end{normalsize}\end{center}}}

% halftitlepage
\newcommand*{\halftitlepageTranslationTitle}[1]{\setstretch{2.5}{\begin{Huge}\uppercase{\so{#1}}\end{Huge}}}
\newcommand*{\halftitlepageTranslationSubtitle}[1]{\setstretch{1.2}{\begin{large}{#1}\end{large}}}
\newcommand*{\halftitlepageFleuron}[1]{{\begin{large}\Fleuronfont{{#1}}\end{large}}}
\newcommand*{\halftitlepageByline}[1]{{\begin{normalsize}\textit{{#1}}\end{normalsize}}}
\newcommand*{\halftitlepageCreatorName}[1]{{\begin{LARGE}{\textsc{#1}}\end{LARGE}}}
\newcommand*{\halftitlepageVolumeNumber}[1]{{\begin{normalsize}{\Allsmallcapsfont{\textsc{#1}}}\end{normalsize}}}
\newcommand*{\halftitlepageVolumeAcronym}[1]{{\begin{normalsize}{#1}\end{normalsize}}}
\newcommand*{\halftitlepageVolumeTranslationTitle}[1]{{\begin{Large}{\textsc{#1}}\end{Large}}}
\newcommand*{\halftitlepageVolumeRootTitle}[1]{{\begin{normalsize}{\Allsmallcapsfont{\textsc{\itshape #1}}}\end{normalsize}}}
\newcommand*{\halftitlepagePublisher}[1]{{\begin{large}{\Noligaturecaptionfont\textsc{#1}}\end{large}}}

% epigraph
\renewcommand{\epigraphflush}{center}
\renewcommand*{\epigraphwidth}{.85\textwidth}
\newcommand*{\epigraphTranslatedTitle}[1]{\vspace*{.5em}\footnotesize\textsc{#1}\\}%
\newcommand*{\epigraphRootTitle}[1]{\footnotesize\textit{#1}\\}%
\newcommand*{\epigraphReference}[1]{\footnotesize{#1}}%

% map
\newsavebox\IBox

% custom commands for html styling classes
\newcommand*{\scnamo}[1]{\begin{Center}\textit{#1}\end{Center}\bigskip}
\newcommand*{\scendsection}[1]{\begin{Center}\begin{small}\textit{#1}\end{small}\end{Center}\addvspace{1em}}
\newcommand*{\scendsutta}[1]{\begin{Center}\textit{#1}\end{Center}\addvspace{1em}}
\newcommand*{\scendbook}[1]{\bigskip\begin{Center}\uppercase{#1}\end{Center}\addvspace{1em}}
\newcommand*{\scendkanda}[1]{\begin{Center}\textbf{#1}\end{Center}\addvspace{1em}} % use for ending vinaya rule sections and also samyuttas %
\newcommand*{\scend}[1]{\begin{Center}\begin{small}\textit{#1}\end{small}\end{Center}\addvspace{1em}}
\newcommand*{\scendvagga}[1]{\begin{Center}\textbf{#1}\end{Center}\addvspace{1em}}
\newcommand*{\scrule}[1]{\textsb{#1}}
\newcommand*{\scadd}[1]{\textit{#1}}
\newcommand*{\scevam}[1]{\textsc{#1}}
\newcommand*{\scspeaker}[1]{\hspace{2em}\textit{#1}}
\newcommand*{\scbyline}[1]{\begin{flushright}\textit{#1}\end{flushright}\bigskip}
\newcommand*{\scexpansioninstructions}[1]{\begin{small}\textit{#1}\end{small}}
\newcommand*{\scuddanaintro}[1]{\medskip\noindent\begin{footnotesize}\textit{#1}\end{footnotesize}\smallskip}

\newenvironment{scuddana}{%
\setlength{\stanzaskip}{.5\baselineskip}%
  \vspace{-1em}\begin{verse}\begin{footnotesize}%
}{%
\end{footnotesize}\end{verse}
}%

% custom command for thematic break = hr
\newcommand*{\thematicbreak}{\begin{center}\rule[.5ex]{6em}{.4pt}\begin{normalsize}\quad\Fleuronfont{•}\quad\end{normalsize}\rule[.5ex]{6em}{.4pt}\end{center}}

% manage and style page header and footer. "fancy" has header and footer, "plain" has footer only

\pagestyle{fancy}
\fancyhf{}
\fancyfoot[RE,LO]{\thepage}
\fancyfoot[LE,RO]{\footnotesize\lastleftxmark}
\fancyhead[CE]{\setstretch{.85}\Noligaturefont\MakeLowercase{\textsc{\firstrightmark}}}
\fancyhead[CO]{\setstretch{.85}\Noligaturefont\MakeLowercase{\textsc{\firstleftmark}}}
\renewcommand{\headrulewidth}{0pt}
\fancypagestyle{plain}{ %
\fancyhf{} % remove everything
\fancyfoot[RE,LO]{\thepage}
\fancyfoot[LE,RO]{\footnotesize\lastleftxmark}
\renewcommand{\headrulewidth}{0pt}
\renewcommand{\footrulewidth}{0pt}}
\fancypagestyle{plainer}{ %
\fancyhf{} % remove everything
\fancyfoot[RE,LO]{\thepage}
\renewcommand{\headrulewidth}{0pt}
\renewcommand{\footrulewidth}{0pt}}

% style footnotes
\setlength{\skip\footins}{1em}

\makeatletter
\newcommand{\@makefntextcustom}[1]{%
    \parindent 0em%
    \thefootnote.\enskip #1%
}
\renewcommand{\@makefntext}[1]{\@makefntextcustom{#1}}
\makeatother

% hang quotes (requires microtype)
\microtypesetup{
  protrusion = true,
  expansion  = true,
  tracking   = true,
  factor     = 1000,
  patch      = all,
  final
}

% Custom protrusion rules to allow hanging punctuation
\SetProtrusion
{ encoding = *}
{
% char   right left
  {-} = {    , 500 },
  % Double Quotes
  \textquotedblleft
      = {1000,     },
  \textquotedblright
      = {    , 1000},
  \quotedblbase
      = {1000,     },
  % Single Quotes
  \textquoteleft
      = {1000,     },
  \textquoteright
      = {    , 1000},
  \quotesinglbase
      = {1000,     }
}

% make latex use actual font em for parindent, not Computer Modern Roman
\AtBeginDocument{\setlength{\parindent}{1em}}%
%

% Default values; a bit sloppier than normal
\tolerance 1414
\hbadness 1414
\emergencystretch 1.5em
\hfuzz 0.3pt
\clubpenalty = 10000
\widowpenalty = 10000
\displaywidowpenalty = 10000
\hfuzz \vfuzz
 \raggedbottom%

\title{Middle Discourses}
\author{Bhikkhu Sujato}
\date{}%
% define a different fleuron for each edition
\newfontfamily\Fleuronfont[Ornament=4]{Arno Pro}

% Define heading styles per edition for chapter, section, and subsection. Suttatitle can be any one of these, depending on the volume. 

\let\oldfrontmatter\frontmatter
\renewcommand{\frontmatter}{%
\chapterfont{\setstretch{.85}\normalfont\centering}%
\sectionfont{\setstretch{.85}\normalfont\BalancedRagged}%
\oldfrontmatter}

\let\oldmainmatter\mainmatter
\renewcommand{\mainmatter}{%
\chapterfont{\thispagestyle{empty}\vspace*{4em}\setstretch{.85}\LARGE\normalfont\itshape\scshape\centering\MakeLowercase}
\sectionfont{\clearpage\thispagestyle{plain}\vspace*{2em}\setstretch{.85}\normalfont\centering}%
\oldmainmatter}

\let\oldbackmatter\backmatter
\renewcommand{\backmatter}{%
\chapterfont{\setstretch{.85}\normalfont\centering}%
\sectionfont{\setstretch{.85}\normalfont\BalancedRagged}%
\pagestyle{plainer}%
\oldbackmatter}
%
%
\begin{document}%
\normalsize%
\frontmatter%
\setlength{\parindent}{0cm}

\pagestyle{empty}

\maketitle

\blankpage%
\begin{center}

\vspace*{2.2em}

\halftitlepageTranslationTitle{Middle Discourses}

\vspace*{1em}

\halftitlepageTranslationSubtitle{A lucid translation of the Majjhima Nikāya}

\vspace*{2em}

\halftitlepageFleuron{•}

\vspace*{2em}

\halftitlepageByline{translated and introduced by}

\vspace*{.5em}

\halftitlepageCreatorName{Bhikkhu Sujato}

\vspace*{4em}

\halftitlepageVolumeNumber{Volume 3}

\smallskip

\halftitlepageVolumeAcronym{MN 101–152}

\smallskip

\halftitlepageVolumeTranslationTitle{The Final Fifty}

\smallskip

\halftitlepageVolumeRootTitle{Uparipaṇṇāsa}

\vspace*{\fill}

\sclogo{0}
 \halftitlepagePublisher{SuttaCentral}

\end{center}

\newpage
%
\setstretch{1.05}

\begin{footnotesize}

\textit{Middle Discourses} is a translation of the Majjhimanikāya by Bhikkhu Sujato.

\medskip

Creative Commons Zero (CC0)

To the extent possible under law, Bhikkhu Sujato has waived all copyright and related or neighboring rights to \textit{Middle Discourses}.

\medskip

This work is published from Australia.

\begin{center}
\textit{This translation is an expression of an ancient spiritual text that has been passed down by the Buddhist tradition for the benefit of all sentient beings. It is dedicated to the public domain via Creative Commons Zero (CC0). You are encouraged to copy, reproduce, adapt, alter, or otherwise make use of this translation. The translator respectfully requests that any use be in accordance with the values and principles of the Buddhist community.}
\end{center}

\medskip

\begin{description}
    \item[Web publication date] 2018
    \item[This edition] 2025-01-13 01:01:43
    \item[Publication type] hardcover
    \item[Edition] ed3
    \item[Number of volumes] 3
    \item[Publication ISBN] 978-1-76132-058-3
    \item[Volume ISBN] 978-1-76132-061-3
    \item[Publication URL] \href{https://suttacentral.net/editions/mn/en/sujato}{https://suttacentral.net/editions/mn/en/sujato}
    \item[Source URL] \href{https://github.com/suttacentral/bilara-data/tree/published/translation/en/sujato/sutta/mn}{https://github.com/suttacentral/bilara-data/tree/published/translation/en/sujato/sutta/mn}
    \item[Publication number] scpub3
\end{description}

\medskip

Map of Jambudīpa is by Jonas David Mitja Lang, and is released by him under Creative Commons Zero (CC0).

\medskip

Published by SuttaCentral

\medskip

\textit{SuttaCentral,\\
c/o Alwis \& Alwis Pty Ltd\\
Kaurna Country,\\
Suite 12,\\
198 Greenhill Road,\\
Eastwood,\\
SA 5063,\\
Australia}

\end{footnotesize}

\newpage

\setlength{\parindent}{1em}%%
\tableofcontents
\newpage
\pagestyle{fancy}
%
\chapter*{Summary of Contents}
\addcontentsline{toc}{chapter}{Summary of Contents}
\markboth{Summary of Contents}{Summary of Contents}

\begin{description}%
\item[The Chapter Beginning With Devadaha (\textit{\textsanskrit{Devadahavagga}})] Diverse teachings.%
\item[MN 101: At Devadaha (\textit{\textsanskrit{Devadahasutta}})] The Buddha tackles a group of Jain ascetics, pressing them on their claim to be practicing to end all suffering by self-mortification. He points out a series of fallacies in their logic, and explains his own middle way.%
\item[MN 102: The Five and Three (\textit{\textsanskrit{Pañcattayasutta}})] A middle length version of the more famous \textsanskrit{Brahmajāla} Sutta (DN1), this surveys a range of speculative views and dismisses them all.%
\item[MN 103: Is This What You Think Of Me? (\textit{\textsanskrit{Kintisutta}})] The Buddha teaches the monks to not dispute about the fundamental teachings, but to always strive for harmony.%
\item[MN 104: At \textsanskrit{Sāmagāma} (\textit{\textsanskrit{Sāmagāmasutta}})] Hearing of the death of the Jain leader \textsanskrit{Nigaṇṭha} \textsanskrit{Nātaputta}, the Buddha encourages the \textsanskrit{Saṅgha} to swiftly resolve any disputes. He lays down a series of seven methods for resolving disputes. These form the foundation for the monastic code.%
\item[MN 105: With Sunakkhatta (\textit{\textsanskrit{Sunakkhattasutta}})] Not all of those who claim to be awakened are genuine. The Buddha teaches how true spiritual progress depends on an irreversible letting go of the forces that lead to suffering.%
\item[MN 106: Conducive to the Imperturbable (\textit{\textsanskrit{Āneñjasappāyasutta}})] Beginning with profound meditation absorption, the Buddha goes on to deeper and deeper levels, showing how insight on this basis leads to the detaching of consciousness from any form of rebirth.%
\item[MN 107: With \textsanskrit{Moggallāna} the Accountant (\textit{\textsanskrit{Gaṇakamoggallānasutta}})] The Buddha compares the training of an accountant with the step by step spiritual path of his followers. But even with such a well explained path, the Buddha can only show the way, and it is up to us to walk it.%
\item[MN 108: With \textsanskrit{Moggallāna} the Guardian (\textit{\textsanskrit{Gopakamoggallānasutta}})] Amid rising military tensions after the Buddha’s death, Venerable Ānanda is questioned about how the \textsanskrit{Saṅgha} planned to continue in their teacher’s absence. As the Buddha refused to appoint a successor, the teaching and practice that he laid down become the teacher, and the \textsanskrit{Saṅgha} resolves issues by consensus.%
\item[MN 109: The Longer Discourse on the Full-Moon Night (\textit{\textsanskrit{Mahāpuṇṇamasutta}})] On a lovely full moon night, one of the mendicants presents the Buddha with a series of questions that go to the heart of the teaching. But when he hears of the doctrine of not-self, another mendicant is unable to grasp the meaning.%
\item[MN 110: The Shorter Discourse on the Full-Moon Night (\textit{\textsanskrit{Cūḷapuṇṇamasutta}})] A good person is able to understand a bad person, but not vice versa.%
\item[The Chapter Beginning with One By One (\textit{\textsanskrit{Anupadavagga}})] Many of the discourses in this chapter delve into complex and analytical presentations of core teachings. It includes important discourses on meditation.%
\item[MN 111: One by One (\textit{\textsanskrit{Anupadasutta}})] The Buddha describes in technical detail the process of insight of Venerable \textsanskrit{Sāriputta}. Many ideas and terms in this text anticipate the Abhidhamma.%
\item[MN 112: The Sixfold Purification (\textit{\textsanskrit{Chabbisodhanasutta}})] If someone claims to be awakened, their claim should be interrogated with a detailed series of detailed questions. Only if they can answer them clearly should the claim be accepted.%
\item[MN 113: A True Person (\textit{\textsanskrit{Sappurisasutta}})] The Buddha explains that a truly good person does not disparage others or feel superior because of their attainment.%
\item[MN 114: What Should and Should Not Be Cultivated (\textit{\textsanskrit{Sevitabbāsevitabbasutta}})] The Buddha sets up a framework on things to be cultivated or avoided, and Venerable \textsanskrit{Sāriputta} volunteers to elaborate.%
\item[MN 115: Many Elements (\textit{\textsanskrit{Bahudhātukasutta}})] Beginning by praising a wise person, the Buddha goes on to explain that one becomes wise by inquiring into the elements, sense fields, dependent origination, and what is possible and impossible.%
\item[MN 116: At Isigili (\textit{\textsanskrit{Isigilisutta}})] Reflecting on the changes that even geographical features undergo, the Buddha then recounts the names of sages of the past who have lived in Mount Isigili near \textsanskrit{Rājagaha}.%
\item[MN 117: The Great Forty (\textit{\textsanskrit{Mahācattārīsakasutta}})] A discourse on the prerequisites of right \textsanskrit{samādhi} that emphasizes the interrelationship and mutual support of all the factors of the eightfold path.%
\item[MN 118: Mindfulness of Breathing (\textit{\textsanskrit{Ānāpānassatisutta}})] Surrounded by many well-practiced mendicants, the Buddha teaches mindfulness of breathing in detail, showing how it relates to the four kinds of mindfulness meditation.%
\item[MN 119: Mindfulness of the Body (\textit{\textsanskrit{Kāyagatāsatisutta}})] This focuses on the first aspect of mindfulness meditation, the observation of the body. This set of practices, simple as they seem, have far-reaching benefits.%
\item[MN 120: Rebirth by Choice (\textit{\textsanskrit{Saṅkhārupapattisutta}})] The Buddha explains how one can make a wish to be reborn in different realms.%
\item[The Chapter Beginning with Emptiness (\textit{\textsanskrit{Suññatavagga}})] Named after the first discourses, which deal with emptiness, this chapter presents less analytical and more narrative texts.%
\item[MN 121: The Shorter Discourse on Emptiness (\textit{\textsanskrit{Cūḷasuññatasutta}})] The Buddha describes his own practice of the meditation on emptiness.%
\item[MN 122: The Longer Discourse on Emptiness (\textit{\textsanskrit{Mahāsuññatasutta}})] A group of mendicants have taken to socializing too much, so the Buddha teaches on the importance of seclusion in order to enter fully into emptiness.%
\item[MN 123: Incredible and Amazing (\textit{\textsanskrit{Acchariyaabbhutasutta}})] Venerable Ānanda is invited by the Buddha to speak on the Buddha’s amazing qualities, and proceeds to list a series of apparently miraculous events accompanying his birth. The Buddha caps it off by explaining what he thinks is really amazing about himself.%
\item[MN 124: With Bakkula (\textit{\textsanskrit{Bākulasutta}})] Venerable Bakkula, regarded as the healthiest of the mendicants, explains to an old friend his strict and austere practice. The unusual form of this discourse suggests it was added to the canon some time after the Buddha’s death.%
\item[MN 125: The Level of the Tamed (\textit{\textsanskrit{Dantabhūmisutta}})] A young monk is unable to persuade a prince of the blessings of peace of mind. The Buddha offers similes based on training an elephant that would have been successful, as this was a field the prince was familiar with.%
\item[MN 126: With \textsanskrit{Bhūmija} (\textit{\textsanskrit{Bhūmijasutta}})] Success in the spiritual life does not depend on any vows you may or may not make, but on whether you practice well.%
\item[MN 127: With Anuruddha (\textit{\textsanskrit{Anuruddhasutta}})] A lay person becomes confused when encouraged to develop the “limitless” and “expansive” liberations, and asks Venerable Anuruddha to explain whether they are the same or different.%
\item[MN 128: Corruptions (\textit{\textsanskrit{Upakkilesasutta}})] A second discourse set at the quarrel of Kosambi, this depicts the Buddha, having failed to achieve reconciliation between the disputing mendicants, leaving the monastery. He spends time in the wilderness before encountering an inspiring community of practicing monks. There he discusses in detail obstacles to meditation that he encountered before awakening.%
\item[MN 129: The Foolish and the Astute (\textit{\textsanskrit{Bālapaṇḍitasutta}})] A fool suffers both in this life and the next, while the astute benefits in both respects.%
\item[MN 130: Messengers of the Gods (\textit{\textsanskrit{Devadūtasutta}})] Expanding on the previous, this discourse contains the most detailed descriptions of the horrors of hell.%
\item[The Chapter on Analysis (\textit{\textsanskrit{Vibhaṅgavagga}})] A series of discourses presented as technical analyses of shorter teachings.%
\item[MN 131: One Fine Night (\textit{\textsanskrit{Bhaddekarattasutta}})] This discourse opens with a short but powerful set of verses extolling the benefits of insight into the here and now, followed by an explanation.%
\item[MN 132: Ānanda and One Fine Night (\textit{\textsanskrit{Ānandabhaddekarattasutta}})] The same discourse as MN 131, but spoken by Venerable Ānanda.%
\item[MN 133: \textsanskrit{Mahākaccāna} and One Fine Night (\textit{\textsanskrit{Mahākaccānabhaddekarattasutta}})] The verses from MN 131 are explained in a different way by Venerable \textsanskrit{Mahakaccāna}.%
\item[MN 134: \textsanskrit{Lomasakaṅgiya} and One Fine Night (\textit{\textsanskrit{Lomasakaṅgiyabhaddekarattasutta}})] A monk who does not know the verses from MN 131 is encouraged by a deity to learn them.%
\item[MN 135: The Shorter Analysis of Deeds (\textit{\textsanskrit{Cūḷakammavibhaṅgasutta}})] The Buddha explains to a brahmin how your deeds in past lives affect you in this life.%
\item[MN 136: The Longer Analysis of Deeds (\textit{\textsanskrit{Mahākammavibhaṅgasutta}})] Confronted with an overly simplistic version of his own teachings, the Buddha emphasizes the often overlooked nuances and qualifications in how karma plays out.%
\item[MN 137: The Analysis of the Six Sense Fields (\textit{\textsanskrit{Saḷāyatanavibhaṅgasutta}})] A detailed analysis of the six senses and the relation to emotional and cognitive processes.%
\item[MN 138: A Summary Recital and its Analysis (\textit{\textsanskrit{Uddesavibhaṅgasutta}})] The Buddha gives a brief and enigmatic statement on the ways consciousness may become attached. Venerable \textsanskrit{Mahākaccāna} is invited by the mendicants to draw out the implications.%
\item[MN 139: The Analysis of Non-Conflict (\textit{\textsanskrit{Araṇavibhaṅgasutta}})] Achieving peace is no simple matter. The Buddha explains how to avoid conflict through contentment, right speech, understanding pleasure, and not insisting on local conventions.%
\item[MN 140: The Analysis of the Elements (\textit{\textsanskrit{Dhātuvibhaṅgasutta}})] While staying overnight in a potter’s workshop, the Buddha has a chance encounter with a monk who does not recognize him. They have a long and profound discussion based on the four elements. This is one of the most insightful and moving discourses in the canon.%
\item[MN 141: The Analysis of the Truths (\textit{\textsanskrit{Saccavibhaṅgasutta}})] Expanding on the Buddha’s first sermon, Venerable \textsanskrit{Sāriputta} gives a detailed explanation of the four noble truths.%
\item[MN 142: The Analysis of Religious Donations (\textit{\textsanskrit{Dakkhiṇāvibhaṅgasutta}})] When his step-mother \textsanskrit{Mahāpajāpatī} wishes to offer him a robe for his personal use, the Buddha encourages her to offer it to the entire \textsanskrit{Saṅgha} instead. He goes on to explain that the best kind of offering to the \textsanskrit{Saṅgha} is one given to the dual community of monks and nuns, headed by the Buddha.%
\item[The Chapter on the Six Senses(\textit{\textsanskrit{Saḷāyatanavagga}})] Most discourses in this chapter deal with the six sense fields.%
\item[MN 143: Advice to \textsanskrit{Anāthapiṇḍika} (\textit{\textsanskrit{Anāthapiṇḍikovādasutta}})] As the great lay disciple \textsanskrit{Anāthapiṇḍika} lies dying, Venerable \textsanskrit{Sāriputta} visits him and gives a powerful teaching on non-attachment.%
\item[MN 144: Advice to Channa (\textit{\textsanskrit{Channovādasutta}})] The monk Channa is suffering a painful terminal illness and wishes to take his own life.%
\item[MN 145: Advice to \textsanskrit{Puṇṇa} (\textit{\textsanskrit{Puṇṇovādasutta}})] On the eve of his departure to a distant country, full of wild and unpredictable people, Venerable \textsanskrit{Puṇṇa} is asked by the Buddha how he would respond if attacked there.%
\item[MN 146: Advice from Nandaka (\textit{\textsanskrit{Nandakovādasutta}})] When asked to teach the nuns, Venerable Nandaka proceeds by inviting them to engage with his discourse and ask if there is anything that needs further explanation.%
\item[MN 147: The Shorter Advice to \textsanskrit{Rāhula} (\textit{\textsanskrit{Cūḷarāhulovādasutta}})] The Buddha takes \textsanskrit{Rāhula} with him to a secluded spot in order to lead him on to liberation.%
\item[MN 148: Six By Six (\textit{\textsanskrit{Chachakkasutta}})] The Buddha analyzes the six senses from six perspectives, and demonstrates the emptiness of all of them.%
\item[MN 149: The Great Discourse on What Relates to the Six Sense Fields (\textit{\textsanskrit{Mahāsaḷāyatanikasutta}})] Explains how insight into the six senses is integrated with the eightfold path and leads to liberation.%
\item[MN 150: With the People of Nagaravinda (\textit{\textsanskrit{Nagaravindeyyasutta}})] In discussion with a group of householders, the Buddha helps them to distinguish those spiritual practitioners who are truly worthy of respect.%
\item[MN 151: The Purification of Alms (\textit{\textsanskrit{Piṇḍapātapārisuddhisutta}})] The Buddha notices Venerable \textsanskrit{Sāriputta}’s glowing complexion, which is the result of his deep meditation. He then presents a series of reflections by which a mendicant can be sure that they are worthy of their alms-food.%
\item[MN 152: The Development of the Faculties (\textit{\textsanskrit{Indriyabhāvanāsutta}})] A brahmin teacher advocates that purification of the senses consists in simply avoiding seeing and hearing things. The Buddha explains that it is not about avoiding sense experience, but understanding it and learning to not be affected by sense experience.%
\end{description}

%
\mainmatter%
\pagestyle{fancy}%
\addtocontents{toc}{\let\protect\contentsline\protect\nopagecontentsline}
\part*{The Final Fifty }
\addcontentsline{toc}{part}{The Final Fifty }
\markboth{}{}
\addtocontents{toc}{\let\protect\contentsline\protect\oldcontentsline}

%
\addtocontents{toc}{\let\protect\contentsline\protect\nopagecontentsline}
\chapter*{The Chapter Beginning With Devadaha }
\addcontentsline{toc}{chapter}{\tocchapterline{The Chapter Beginning With Devadaha }}
\addtocontents{toc}{\let\protect\contentsline\protect\oldcontentsline}

%
\section*{{\suttatitleacronym MN 101}{\suttatitletranslation At Devadaha }{\suttatitleroot Devadahasutta}}
\addcontentsline{toc}{section}{\tocacronym{MN 101} \toctranslation{At Devadaha } \tocroot{Devadahasutta}}
\markboth{At Devadaha }{Devadahasutta}
\extramarks{MN 101}{MN 101}

\scevam{So\marginnote{1.1} I have heard. }At one time the Buddha was staying in the land of the Sakyans, near the Sakyan town named Devadaha.\footnote{Devadaha also appears in \href{https://suttacentral.net/sn22.2/en/sujato}{SN 22.2} and \href{https://suttacentral.net/sn35.134/en/sujato}{SN 35.134}. The name is explained in the commentary as “royal lake”, taking \textit{deva-} as a term for kings, and \textit{-daha} as equivalent to the Sanskrit \textit{hrada}, “lake”. However it is spelled \textit{\textsanskrit{devadṛśa}} in the \textsanskrit{Mūlasarvāstivāda} Vinaya (\href{https://suttacentral.net/san-mu-kd17/san/gbm\#sc16}{San Mu Kd 17:16}). \textit{\textsanskrit{Dṛśa}} means “sight, appearance”, thus occupying a completely different semantic space than “lake”. It is a Vedic term that commonly describes Agni’s fiery gleam (Rig Veda 3.17.4, 6.10.4, 7.1.1, etc.). This suggests a Pali derivation from √\textit{\textsanskrit{ḍah}}, “to burn”. This in turn recalls Śatapatha \textsanskrit{Brāhmaṇa} 1.4.1.14, which allegorically depicts the spread of Aryan culture eastward as Agni “burning” (\textit{\textsanskrit{dadāha}}) the lands, i.e. introducing the civilized practice of Vedic fire worship. I think the meaning of \textit{devadaha} is, therefore, “burned by the god”, or “place of the god’s flame”. It became famous in later Buddhism as the birthplace of the sisters \textsanskrit{Māyā} and \textsanskrit{Mahāpajāpatī} of the Koliyan clan, the Buddha’s birth mother and foster mother. These details are not mentioned in early texts, however, where the town is said to be Sakyan rather than Koliyan. It is identified with Devdaha in the Rupandehi District of Nepal. } There the Buddha addressed the mendicants, “Mendicants!” 

“Venerable\marginnote{2.2} sir,” they replied. The Buddha said this: 

“Mendicants,\marginnote{2.4} there are some ascetics and brahmins who have this doctrine and view: ‘Everything this individual experiences—pleasurable, painful, or neutral—is because of past deeds.\footnote{The view that all experiences are caused by past kamma is consistently rejected in the suttas (eg. \href{https://suttacentral.net/an3.61/en/sujato\#2.1}{AN 3.61:2.1}, \href{https://suttacentral.net/sn36.21/en/sujato}{SN 36.21}). This is reflected in the later Theravada commentaries, which describe five interacting \textit{\textsanskrit{niyāmas}}, or systems of natural law: seasons (\textit{utu}), genetics (\textit{\textsanskrit{bīja}}), moral or immoral deeds (\textit{kamma}), mind (\textit{citta}), and principles of the teaching (\textit{dhamma}). Nonetheless, the view that all experiences are caused by kamma is extremely common among modern Buddhists, and is often heard even from learned scholars. This is a harmful misunderstanding, as it has led to discrimination against the poor and disabled in Buddhist countries. One point of confusion is that rebirth into a particular life is caused by kamma, so in that general sense, kamma is a necessary condition for having a life in which one can experience anything. But the feelings experienced in that life are caused by a variety of things, not necessarily by kamma. } So, due to eliminating past deeds by fervent mortification, and not doing any new deeds, there’s nothing to come up in the future.\footnote{Jains believed that karma was material, made up of particles. The practice of \textit{tapas} (“fervent mortification”) creates an intense bodily heat which burns off these karmic particles, freeing the soul (\textit{\textsanskrit{jīva}}) to shine in its intrinsic purity. | This passage also at \href{https://suttacentral.net/mn14/en/sujato\#17.7}{MN 14:17.7}. } With no future consequence, deeds end. With the ending of deeds, suffering ends. With the ending of suffering, feeling ends. And with the ending of feeling, all suffering will have been worn away.’\footnote{The mention of “feeling” after “suffering” here is found consistently in Pali sources, but not in their parallels. | \textit{\textsanskrit{Nijiṇṇa}} (“worn away”) is a characteristic Jain term that is sometimes applied to kamma in the suttas as well. } Such is the doctrine of the Jain ascetics.\footnote{The idea that the elimination of karma leads to freedom from suffering is indeed a fundamental tenet of Jain thought. } 

I’ve\marginnote{3.1} gone up to the Jain ascetics who say this and said, ‘Is it really true that this is the venerables’ view?’ They admitted that it is. 

I\marginnote{4.1} said to them, ‘But reverends, do you know for sure that you existed in the past, and it is not the case that you did not exist?’ 

‘No\marginnote{4.4} we don’t, reverend.’ 

‘But\marginnote{4.5} reverends, do you know for sure that you did bad deeds in the past?’ 

‘No\marginnote{4.7} we don’t, reverend.’ 

‘But\marginnote{4.8} reverends, do you know that you did such and such bad deeds?’ 

‘No\marginnote{4.10} we don’t, reverend.’ 

‘But\marginnote{4.11} reverends, do you know that so much suffering has already been worn away? Or that so much suffering still remains to be worn away? Or that when so much suffering is worn away all suffering will have been worn away?’ 

‘No\marginnote{4.13} we don’t, reverend.’ 

‘But\marginnote{4.14} reverends, do you know about giving up unskillful qualities in this very life and embracing skillful qualities?’ 

‘No\marginnote{4.16} we don’t, reverend.’ 

‘So\marginnote{5.1} it seems that you don’t know any of these things. In that case, it’s not appropriate for the Jain venerables to declare this. 

Now,\marginnote{6.1} supposing you did know these things. In that case, it would be appropriate for the Jain venerables to declare this. 

Suppose\marginnote{7.1} a man was struck by an arrow thickly smeared with poison, causing painful feelings, sharp and severe. Their friends and colleagues, relatives and kin would get a surgeon to treat them. The surgeon would cut open the wound with a scalpel, causing painful feelings, sharp and severe. They’d probe for the arrow, causing painful feelings, sharp and severe. They’d extract the arrow, causing painful feelings, sharp and severe. They’d apply cauterizing medicine to the wound, causing painful feelings, sharp and severe. After some time that wound would be healed and the skin regrown. They’d be healthy, happy, autonomous, master of themselves, able to go where they wanted. 

They’d\marginnote{7.13} think, “Earlier I was struck by an arrow thickly smeared with poison, causing painful feelings, sharp and severe. My friends and colleagues, relatives and kin got a surgeon to treat me. At each step, the treatment was painful. But these days that wound is healed and the skin regrown. I’m healthy, happy, autonomous, my own master, able to go where I want.” 

In\marginnote{8.1} the same way, reverends, if you knew about these things, it would be appropriate for the Jain venerables to declare this. 

But\marginnote{9.1} since you don’t know any of these things, it’s not appropriate for the Jain venerables to declare this.’ 

When\marginnote{10.1} I said this, those Jain ascetics said to me, ‘Reverend, the Jain ascetic of the \textsanskrit{Ñātika} clan claims to be all-knowing and all-seeing, to know and see everything without exception, thus: “Knowledge and vision are constantly and continually present to me, while walking, standing, sleeping, and waking.” 

He\marginnote{10.4} says: “O reverend Jain ascetics, you have done bad deeds in a past life. Wear them away with these severe and grueling austerities. And when in the present you are restrained in body, speech, and mind, you’re not doing any bad deeds for the future. So, due to eliminating past deeds by fervent mortification, and not doing any new deeds, there’s nothing to come up in the future. With no future consequence, deeds end. With the ending of deeds, suffering ends. With the ending of suffering, feeling ends. And with the ending of feeling, all suffering will have been worn away.” We endorse and accept this, and we are satisfied with it.’ 

When\marginnote{11.1} they said this, I said to them, ‘These five things can be seen to turn out in two different ways.\footnote{Also at \href{https://suttacentral.net/mn95/en/sujato\#14.3}{MN 95:14.3}. } What five? Faith, endorsement, oral transmission, reasoned train of thought, and acceptance of a view after deliberation. These are the five things that can be seen to turn out in two different ways. In this case, what faith in your teacher do you have when it comes to the past? What endorsement, oral transmission, reasoned train of thought, or acceptance of a view after deliberation?’ When I said this, I did not see any legitimate defense of their doctrine from the Jains. 

Furthermore,\marginnote{12.1} I said to those Jain ascetics, ‘What do you think, reverends? At a time of intense exertion and striving do you suffer painful, sharp, severe, acute feelings due to overexertion? Whereas at a time without intense exertion and striving do you not suffer painful, sharp, severe, acute feelings due to overexertion?’ 

‘Reverend\marginnote{12.5} Gotama, at a time of intense exertion we suffer painful, sharp feelings due to overexertion, not without intense exertion.’ 

‘So\marginnote{13.1} it seems that only at a time of intense exertion do you suffer painful, sharp feelings due to overexertion, not without intense exertion. In that case, it’s not appropriate for the Jain venerables to declare: “Everything this individual experiences—pleasurable, painful, or neutral—is because of past deeds. …” 

If\marginnote{14.1} at a time of intense exertion you did not suffer painful, sharp feelings due to overexertion, and if without intense exertion you did experience such feelings, it would be appropriate for the Jain venerables to declare this. 

But\marginnote{15.1} since this is not the case, aren’t you experiencing painful, sharp feelings due only to your own exertion, which out of ignorance, unknowing, and confusion you misconstrue to imply:\footnote{This is a difficult sentence, and different translators handle it in quite different ways. I take \textit{vipaccetha} as second plural optative, intended as a rhetorical question. The sense, I think, plays on the common meaning of “ripen”, but in the sense “make something out of it that it isn’t”, i.e. (mis)construe. } “Everything this individual experiences—pleasurable, painful, or neutral—is because of past deeds. …”?’ When I said this, I did not see any legitimate defense of their doctrine from the Jains. 

Furthermore,\marginnote{16.1} I said to those Jain ascetics, ‘What do you think, reverends? If a deed is to be experienced in this life, can exertion make it be experienced in lives to come?’\footnote{Just as different deeds have differing characteristics, the results of those deeds also have different characteristics, which constrain the result in certain ways. This does not mean that they are fully deterministic, however. Compare the sprouting of a seed. A mango seed cannot give rise to an orange tree. And it has certain conditions under which it cannot sprout—if it is out of season, or the soil is inadequate, or there is no water, etc. Still, there are many different ways that it \emph{can} sprout, and the tree and its fruit that result will vary according to conditions. } 

‘No,\marginnote{16.3} reverend.’ 

‘But\marginnote{16.4} if a deed is to be experienced in lives to come, can exertion make it be experienced in this life?’ 

‘No,\marginnote{16.5} reverend.’ 

‘What\marginnote{17.1} do you think, reverends? If a deed is to be experienced as pleasure, can exertion make it be experienced as pain?’ 

‘No,\marginnote{17.2} reverend.’ 

‘But\marginnote{17.3} if a deed is to be experienced as pain, can exertion make it be experienced as pleasure?’ 

‘No,\marginnote{17.4} reverend.’ 

‘What\marginnote{18.1} do you think, reverends? If a deed is to be experienced when fully ripened, can exertion make it be experienced when not fully ripened?’ 

‘No,\marginnote{18.2} reverend.’ 

‘But\marginnote{18.3} if a deed is to be experienced when not fully ripened, can exertion make it be experienced when fully ripened?’ 

‘No,\marginnote{18.4} reverend.’ 

‘What\marginnote{19.1} do you think, reverends? If a deed is to be experienced strongly, can exertion make it be experienced weakly?’ 

‘No,\marginnote{19.2} reverend.’ 

‘But\marginnote{19.3} if a deed is to be experienced weakly, can exertion make it be experienced strongly?’ 

‘No,\marginnote{19.4} reverend.’ 

‘What\marginnote{20.1} do you think, reverends? If a deed is to be experienced, can exertion make it not be experienced?’ 

‘No,\marginnote{20.2} reverend.’ 

‘But\marginnote{20.3} if a deed is not to be experienced, can exertion make it be experienced?’\footnote{This is a deed that has no opportunity to bear fruit, like a good seed on barren soil. } 

‘No,\marginnote{20.4} reverend.’ 

‘So\marginnote{21.1} it seems that exertion cannot change the way deeds are experienced in any of these ways. This being so, your exertion and striving are fruitless.’ 

Such\marginnote{22.1} is the doctrine of the Jain ascetics. Saying this, the Jain ascetics deserve rebuttal and criticism on ten legitimate grounds.\footnote{The “ten legitimate grounds” are, as explained below, the five grounds and their inversions. Whether or whether not the ground is true, the Jains deserve censure. } 

If\marginnote{22.3} sentient beings experience pleasure and pain because of past deeds, clearly the Jains have done bad deeds in the past, since they now experience such intense pain. If sentient beings experience pleasure and pain because of God Almighty’s creation,\footnote{“God Almighty” is \textit{issara}, the creator deity. See also \href{https://suttacentral.net/an3.61/en/sujato}{AN 3.61}. } clearly the Jains were created by a bad God, since they now experience such intense pain. If sentient beings experience pleasure and pain because of circumstance and nature,\footnote{This is the \textsanskrit{Ājīvika} doctrine of the Bamboo-staffed Ascetic \textsanskrit{Gosāla}, for which see \href{https://suttacentral.net/dn2/en/sujato\#20.6}{DN 2:20.6}, \href{https://suttacentral.net/mn60/en/sujato\#21.7}{MN 60:21.7}, and \href{https://suttacentral.net/an3.61/en/sujato}{AN 3.61}. The mention of \textsanskrit{Ājīvika} doctrines here and in the next section reflect their closeness with the Jains. } clearly the Jains arise from bad circumstances, since they now experience such intense pain. If sentient beings experience pleasure and pain because of the class of rebirth,\footnote{See \href{https://suttacentral.net/an6.57/en/sujato}{AN 6.57} for a fuller discussion of this \textsanskrit{Ājīvika} doctrine. } clearly the Jains have been reborn in a bad class, since they now experience such intense pain. If sentient beings experience pleasure and pain because of exertion in this life, clearly the Jains exert themselves badly in this life, since they now experience such intense pain.\footnote{PTS and BJT editions have the expected \textit{\textsanskrit{pāpadiṭṭhadhammūpakkamā}}. } 

The\marginnote{22.13} Jains deserve criticism whether or not sentient beings experience pleasure and pain because of past deeds, or God Almighty’s creation, or circumstance and nature, or class of rebirth, or exertion in this life. Such is the doctrine of the Jain ascetics. The Jain ascetics who say this deserve rebuttal and criticism on these ten legitimate grounds. That’s how exertion and striving is fruitless. 

And\marginnote{23.1} how is exertion and striving fruitful? It’s when a mendicant doesn’t bring suffering upon themselves; and they don’t forsake legitimate pleasure, but they’re not besotted with that pleasure. They understand: ‘When I actively strive I become dispassionate towards this source of suffering. But when I develop equanimity I become dispassionate towards this other source of suffering.’ So they either actively strive or develop equanimity as appropriate. Through active striving they become dispassionate towards that specific source of suffering, and so that suffering is worn away. Through developing equanimity they become dispassionate towards that other source of suffering, and so that suffering is worn away. 

Suppose\marginnote{24.1} a man is in love with a woman, full of intense desire and lust. Then he sees her standing together with another man, chatting, giggling, and laughing. 

What\marginnote{24.3} do you think, mendicants? Would that give rise to sorrow, lamentation, pain, sadness, and distress for him?” 

“Yes,\marginnote{24.5} sir. Why is that? Because that man is in love with that woman, full of intense desire and lust.” 

“Then\marginnote{25.1} that man might think: ‘I’m in love with that woman, full of intense desire and lust. When I saw her standing together with another man, chatting, giggling, and laughing, it gave rise to sorrow, lamentation, pain, sadness, and distress for me. Why don’t I give up that desire and lust for that woman?’ So that’s what he did. Some time later he sees her again standing together with another man, chatting, giggling, and laughing. 

What\marginnote{25.7} do you think, mendicants? Would that give rise to sorrow, lamentation, pain, sadness, and distress for him?” 

“No,\marginnote{25.9} sir. Why is that? Because he no longer desires that woman.” 

“In\marginnote{26.1} the same way, a mendicant doesn’t bring suffering upon themselves; and they don’t forsake legitimate pleasure, but they’re not besotted with that pleasure. They understand: ‘When I actively strive I become dispassionate towards this source of suffering. But when I develop equanimity I become dispassionate towards this other source of suffering.’ So they either actively strive or develop equanimity as appropriate. Through active striving they become dispassionate towards that specific source of suffering, and so that suffering is worn away. Through developing equanimity they become dispassionate towards that other source of suffering, and so that suffering is worn away. That’s how exertion and striving is fruitful. 

Furthermore,\marginnote{27.1} a mendicant reflects: ‘When I live as I please, unskillful qualities grow and skillful qualities decline. But when I strive painfully, unskillful qualities decline and skillful qualities grow. Why don’t I strive painfully?’ So that’s what they do, and as they do so unskillful qualities decline and skillful qualities grow. After some time, they no longer strive painfully. Why is that? Because they have accomplished the goal for which they strived painfully. 

Suppose\marginnote{28.1} an arrowsmith was heating an arrow shaft between two firebrands, making it straight and fit for use. After it’s been made straight and fit for use, they’d no longer heat it to make it straight and fit for use. Why is that? Because they have accomplished the goal for which they heated it. 

In\marginnote{29.5} the same way, a mendicant reflects: ‘When I live as I please, unskillful qualities grow and skillful qualities decline. But when I strive painfully, unskillful qualities decline and skillful qualities grow. Why don’t I strive painfully?’ … After some time, they no longer strive painfully. That too is how exertion and striving is fruitful. 

Furthermore,\marginnote{30.1} a Realized One arises in the world, perfected, a fully awakened Buddha, accomplished in knowledge and conduct, holy, knower of the world, supreme guide for those who wish to train, teacher of gods and humans, awakened, blessed. He has realized with his own insight this world—with its gods, \textsanskrit{Māras}, and divinities, this population with its ascetics and brahmins, gods and humans—and he makes it known to others. He proclaims a teaching that is good in the beginning, good in the middle, and good in the end, meaningful and well-phrased. And he reveals a spiritual practice that’s entirely full and pure. 

A\marginnote{31.1} householder hears that teaching, or a householder’s child, or someone reborn in a good family. They gain faith in the Realized One, and reflect: ‘Life at home is cramped and dirty, life gone forth is wide open. It’s not easy for someone living at home to lead the spiritual life utterly full and pure, like a polished shell. Why don’t I shave off my hair and beard, dress in ocher robes, and go forth from the lay life to homelessness?’ After some time they give up a large or small fortune, and a large or small family circle. They shave off hair and beard, dress in ocher robes, and go forth from the lay life to homelessness. 

Once\marginnote{32.1} they’ve gone forth, they take up the training and livelihood of the mendicants. They give up killing living creatures, renouncing the rod and the sword. They’re scrupulous and kind, living full of sympathy for all living beings. They give up stealing. They take only what’s given, and expect only what’s given. They keep themselves clean by not thieving. They give up unchastity. They are celibate, set apart, avoiding the vulgar act of sex. They give up lying. They speak the truth and stick to the truth. They’re honest and dependable, and don’t trick the world with their words. They give up divisive speech. They don’t repeat in one place what they heard in another so as to divide people against each other. Instead, they reconcile those who are divided, supporting unity, delighting in harmony, loving harmony, speaking words that promote harmony. They give up harsh speech. They speak in a way that’s mellow, pleasing to the ear, lovely, going to the heart, polite, likable and agreeable to the people. They give up talking nonsense. Their words are timely, true, and meaningful, in line with the teaching and training. They say things at the right time which are valuable, reasonable, succinct, and beneficial. They refrain from injuring plants and seeds. They eat in one part of the day, abstaining from eating at night and food at the wrong time. They refrain from seeing shows of dancing, singing, and music . They refrain from beautifying and adorning themselves with garlands, fragrance, and makeup. They refrain from high and luxurious beds. They refrain from receiving gold and currency, raw grains, raw meat, women and girls, male and female bondservants, goats and sheep, chickens and pigs, elephants, cows, horses, and mares, and fields and land. They refrain from running errands and messages; buying and selling; falsifying weights, metals, or measures; bribery, fraud, cheating, and duplicity; mutilation, murder, abduction, banditry, plunder, and violence. 

They’re\marginnote{33.1} content with robes to look after the body and almsfood to look after the belly. Wherever they go, they set out taking only these things. They’re like a bird: wherever it flies, wings are its only burden. In the same way, a mendicant is content with robes to look after the body and almsfood to look after the belly. Wherever they go, they set out taking only these things. When they have this entire spectrum of noble ethics, they experience a blameless happiness inside themselves. 

When\marginnote{34.1} they see a sight with their eyes, they don’t get caught up in the features and details. If the faculty of sight were left unrestrained, bad unskillful qualities of covetousness and displeasure would become overwhelming. For this reason, they practice restraint, protecting the faculty of sight, and achieving its restraint. When they hear a sound with their ears … When they smell an odor with their nose … When they taste a flavor with their tongue … When they feel a touch with their body … When they know an idea with their mind, they don’t get caught up in the features and details. If the faculty of mind were left unrestrained, bad unskillful qualities of covetousness and displeasure would become overwhelming. For this reason, they practice restraint, protecting the faculty of mind, and achieving its restraint. When they have this noble sense restraint, they experience an unsullied bliss inside themselves. 

They\marginnote{35.1} act with situational awareness when going out and coming back; when looking ahead and aside; when bending and extending the limbs; when bearing the outer robe, bowl and robes; when eating, drinking, chewing, and tasting; when urinating and defecating; when walking, standing, sitting, sleeping, waking, speaking, and keeping silent. 

When\marginnote{36.1} they have this entire spectrum of noble ethics, this noble contentment, this noble sense restraint, and this noble mindfulness and situational awareness, they frequent a secluded lodging—a wilderness, the root of a tree, a hill, a ravine, a mountain cave, a charnel ground, a forest, the open air, a heap of straw. After the meal, they return from almsround, sit down cross-legged, set their body straight, and establish mindfulness in their presence. 

Giving\marginnote{37.1} up covetousness for the world, they meditate with a heart rid of covetousness, cleansing the mind of covetousness. Giving up ill will and malevolence, they meditate with a mind rid of ill will, full of sympathy for all living beings, cleansing the mind of ill will. Giving up dullness and drowsiness, they meditate with a mind rid of dullness and drowsiness, perceiving light, mindful and aware, cleansing the mind of dullness and drowsiness. Giving up restlessness and remorse, they meditate without restlessness, their mind peaceful inside, cleansing the mind of restlessness and remorse. Giving up doubt, they meditate having gone beyond doubt, not undecided about skillful qualities, cleansing the mind of doubt. 

They\marginnote{38.1} give up these five hindrances, corruptions of the heart that weaken wisdom. Then, quite secluded from sensual pleasures, secluded from unskillful qualities, they enter and remain in the first absorption, which has the rapture and bliss born of seclusion, while placing the mind and keeping it connected. That too is how exertion and striving is fruitful. 

Furthermore,\marginnote{39.1} as the placing of the mind and keeping it connected are stilled, they enter and remain in the second absorption, which has the rapture and bliss born of immersion, with internal clarity and mind at one, without placing the mind and keeping it connected. That too is how exertion and striving is fruitful. 

Furthermore,\marginnote{40.1} with the fading away of rapture, a mendicant enters and remains in the third absorption, where they meditate with equanimity, mindful and aware, personally experiencing the bliss of which the noble ones declare, ‘Equanimous and mindful, one meditates in bliss.’ That too is how exertion and striving is fruitful. 

Furthermore,\marginnote{41.1} giving up pleasure and pain, and ending former happiness and sadness, they enter and remain in the fourth absorption, without pleasure or pain, with pure equanimity and mindfulness. That too is how exertion and striving is fruitful. 

When\marginnote{42.1} their mind has become immersed in \textsanskrit{samādhi} like this—purified, bright, flawless, rid of corruptions, pliable, workable, steady, and imperturbable—they extend it toward recollection of past lives. They recollect many kinds of past lives, that is, one, two, three, four, five, ten, twenty, thirty, forty, fifty, a hundred, a thousand, a hundred thousand rebirths; many eons of the world contracting, many eons of the world expanding, many eons of the world contracting and expanding. They remember: ‘There, I was named this, my clan was that, I looked like this, and that was my food. This was how I felt pleasure and pain, and that was how my life ended. When I passed away from that place I was reborn somewhere else. There, too, I was named this, my clan was that, I looked like this, and that was my food. This was how I felt pleasure and pain, and that was how my life ended. When I passed away from that place I was reborn here.’ And so they recollect their many kinds of past lives, with features and details. That too is how exertion and striving is fruitful. 

When\marginnote{43.1} their mind has become immersed in \textsanskrit{samādhi} like this—purified, bright, flawless, rid of corruptions, pliable, workable, steady, and imperturbable—they extend it toward knowledge of the death and rebirth of sentient beings. With clairvoyance that is purified and superhuman, they see sentient beings passing away and being reborn—inferior and superior, beautiful and ugly, in a good place or a bad place. They understood how sentient beings are reborn according to their deeds: ‘These dear beings did bad things by way of body, speech, and mind. They denounced the noble ones; they had wrong view; and they chose to act out of that wrong view. When their body breaks up, after death, they’re reborn in a place of loss, a bad place, the underworld, hell. These dear beings, however, did good things by way of body, speech, and mind. They never denounced the noble ones; they had right view; and they chose to act out of that right view. When their body breaks up, after death, they’re reborn in a good place, a heavenly realm.’ And so, with clairvoyance that is purified and superhuman, they see sentient beings passing away and being reborn—inferior and superior, beautiful and ugly, in a good place or a bad place. They understand how sentient beings are reborn according to their deeds. That too is how exertion and striving is fruitful. 

When\marginnote{44.1} their mind has become immersed in \textsanskrit{samādhi} like this—purified, bright, flawless, rid of corruptions, pliable, workable, steady, and imperturbable—they extend it toward knowledge of the ending of defilements. They truly understand: ‘This is suffering’ … ‘This is the origin of suffering’ … ‘This is the cessation of suffering’ … ‘This is the practice that leads to the cessation of suffering’. They truly understand: ‘These are defilements’ … ‘This is the origin of defilements’ … ‘This is the cessation of defilements’ … ‘This is the practice that leads to the cessation of defilements’. 

Knowing\marginnote{45.1} and seeing like this, their mind is freed from the defilements of sensuality, desire to be reborn, and ignorance. When they’re freed, they know they’re freed. 

They\marginnote{45.3} understand: ‘Rebirth is ended, the spiritual journey has been completed, what had to be done has been done, there is nothing further for this place.’ That too is how exertion and striving is fruitful. Such is the doctrine of the Realized One. Saying this, the Realized One deserves praise on ten legitimate grounds. 

If\marginnote{46.3} sentient beings experience pleasure and pain because of past deeds, clearly the Realized One has done good deeds in the past, since he now experiences such undefiled pleasure. If sentient beings experience pleasure and pain because of God Almighty’s creation, clearly the Realized One was created by a good God, since he now experiences such undefiled pleasure. If sentient beings experience pleasure and pain because of circumstance and nature, clearly the Realized One arises from good circumstances, since he now experiences such undefiled pleasure. If sentient beings experience pleasure and pain because of the class of rebirth, clearly the Realized One was reborn in a good class, since he now experiences such undefiled pleasure. If sentient beings experience pleasure and pain because of exertion in this life, clearly the Realized One exerts himself well in this life, since he now experiences such undefiled pleasure. 

The\marginnote{46.13} Realized One deserves praise whether or not sentient beings experience pleasure and pain because of past deeds, or God Almighty’s creation, or circumstance and nature, or class of rebirth, or exertion in this life. Such is the doctrine of the Realized One. Saying this, the Realized One deserves praise on these ten legitimate grounds.” 

That\marginnote{46.25} is what the Buddha said. Satisfied, the mendicants approved what the Buddha said. 

%
\section*{{\suttatitleacronym MN 102}{\suttatitletranslation The Five and Three }{\suttatitleroot Pañcattayasutta}}
\addcontentsline{toc}{section}{\tocacronym{MN 102} \toctranslation{The Five and Three } \tocroot{Pañcattayasutta}}
\markboth{The Five and Three }{Pañcattayasutta}
\extramarks{MN 102}{MN 102}

\scevam{So\marginnote{1.1} I have heard.\footnote{This sutta is something of a “middle length” version of the better-known \textsanskrit{Brahmajālasutta} (\href{https://suttacentral.net/dn1/en/sujato}{DN 1}). Both these texts emphasize the importance of understanding the views of others. This is important, because we live and communicate in a cultural context, and our views are shared with those of others. Some of the views defined here are still current today, while others have disappeared in history. But we can still learn from the Buddha’s method to learn how to approach different views today. | A Tibetan translation of the \textsanskrit{Mūlasarvāstivāda} \textsanskrit{Pañcatraya} \textsanskrit{Sūtra} is discussed by Peter Skilling in \emph{\textsanskrit{Mahāsūtras} II}, pp. 469–511. } }At one time the Buddha was staying near \textsanskrit{Sāvatthī} in Jeta’s Grove, \textsanskrit{Anāthapiṇḍika}’s monastery. There the Buddha addressed the mendicants, “Mendicants!” 

“Venerable\marginnote{1.5} sir,” they replied. The Buddha said this: 

“Mendicants,\marginnote{2.1} there are some ascetics and brahmins who speculate and theorize about the future, and assert various hypotheses concerning the future. Some propose this: ‘The self is percipient and healthy after death.’\footnote{The first three are varieties of eternalism. | \textit{Aroga} (“free of disease”) is explained by the commentary as “permanent” (\textit{nicca}), drawing on the root sense of the word, “unbroken”. However, \textit{aroga} is always used in the sense “free of disease, well, healthy” (eg. \href{https://suttacentral.net/mn97/en/sujato\#2.4}{MN 97:2.4}), and this applies to the Brahmanical tradition as well as the Buddhist. Chandogya \textsanskrit{Upaniṣad} 7.26.2 says that one who sees (the self) does not see death, they have no disease or pain. \textsanskrit{Bṛhadāraṇyaka} \textsanskrit{Upaniṣad} 4.4.12 similarly says that one who sees the self will not suffer in the wake of the body, which \textsanskrit{Śaṅkāra} explains, “Struggling with desires for himself, for his son, for his wife, and so on, he is born and dies again and again, and is diseased when his body is diseased.” } Some propose this: ‘The self is non-percipient and healthy after death.’ Some propose this: ‘The self is neither percipient nor non-percipient and healthy after death.’ But some assert the annihilation, eradication, and obliteration of an existing being, while others propose extinguishment in this life.\footnote{“Extinguishment in the present life” sounds like the Buddhist view, but since it is collected here with wrong views, it refers to misapprehension of absorption as being \textsanskrit{Nibbāna} (\href{https://suttacentral.net/mn102/en/sujato\#24.1}{MN 102:24.1}). This is similar to the five views of \href{https://suttacentral.net/dn1/en/sujato\#3.19.1}{DN 1:3.19.1}. DN 1 adds sensual pleasures as a kind of false \textsanskrit{Nibbāna}; it also makes it explicit that it is the extinguishment of “an existing being”. | In both suttas “extinguishment in the present life” is included with the views about the future, for unclear reasons. Skilling notes, however, that in the \textsanskrit{Pañcatraya} \textsanskrit{Sūtra}, “extinguishment in the present life” constitutes a separate category. } Thus they assert an existent self that is free of disease after death; or they assert the annihilation of an existing being; while some propose extinguishment in this life. In this way five become three, and three become five.\footnote{That is, there are five total propositions, of which the first three are varieties of eternalism, which if treated as a group, gives three kinds of views. } This is the summary recital of the five and three. 

Now,\marginnote{3.1} the ascetics and brahmins who assert a self that is percipient and healthy after death describe it as formed, or formless, or both formed and formless, or neither formed nor formless. Or they describe it as of unified perception, or of diverse perception, or of limited perception, or of limitless perception. Or some among those who go beyond this propose universal consciousness, limitless and imperturbable.\footnote{The Buddha subdivides the first kind of eternalist according to the kind of self that is envisaged. | The “form” of the self may be either “coarse” form such as a human body, or “subtle form” like the heaven realms. | A formless self is associated with formless attainments. | “Having form and being formless” is not meant as a paradox. It could mean that the self evolves or transitions through different states, sometimes having form, sometimes not. Or it could refer to something like different layers of sheaths of the self, where everything is the self, but may be lesser attributes of self. | Unified perception results from \textit{\textsanskrit{jhāna}}, diverse perception from sense experience. | Limited perception is associated with sense consciousness or a limited attainment of absorption, while limitless perception is associated with a limitless absorption. This seems to refer to the first formless attainment. | “Universal consciousness” (\textit{\textsanskrit{viññāṇakasiṇa}}) is equivalent to the second formless attainment, which is mentioned by both the Pali commentary and the \textsanskrit{Pañcatraya} \textsanskrit{Sūtra}. } 

The\marginnote{4.1} Realized One understands this as follows. There are ascetics and brahmins who assert a self that is percipient and healthy after death, describing it as formed, or formless, or both formed and formless, or neither formed nor formless. Or they describe it as of unified perception, or of diverse perception, or of limited perception, or of limitless perception. Or some, aware that ‘there is nothing at all’, propose the dimension of nothingness, limitless and imperturbable. They declare that this is the purest, highest, best, and supreme of all those perceptions, whether of form or of formlessness or of unity or of diversity.\footnote{This is the third formless attainment. | The fourth formless attainment, “neither perception nor non-perception” is so subtle that it cannot be considered an attainment with perception. } ‘That much is conditioned and crude. But there is the cessation of conditions—\emph{that} is real.’\footnote{This describes the development of insight based on deep absorption. Even the most refined states are still conditioned and therefore “coarse” compared to \textsanskrit{Nibbāna}. } Understanding this and seeing the escape from it, the Realized One has gone beyond all that. 

Now,\marginnote{5.1} the ascetics and brahmins who assert a self that is non-percipient and healthy after death describe it as formed, or formless, or both formed and formless, or neither formed nor formless.\footnote{The suttas occasionally mention the “non-percipient beings” (\href{https://suttacentral.net/dn1/en/sujato\#2.31.1}{DN 1:2.31.1}). The commentaries explain this as a realm of pure form, where consciousness is suspended for a long time. This is tricky to interpret in line with the Buddhist classes of existence, since such a state could not be considered “formless”. A similar problem comes when the dimension of neither perception nor non-perception is considered to have form (\href{https://suttacentral.net/mn102/en/sujato\#8.1}{MN 102:8.1}). Such cases must refer to sectarian views that do not fit within the Buddhist system. } 

So\marginnote{6.1} they reject those who assert a self that is percipient and healthy after death. Why is that? Because they believe that perception is a disease, a boil, a dart, and that the state of non-perception is peaceful and sublime. 

The\marginnote{7.1} Realized One understands this as follows. There are ascetics and brahmins who assert a self that is non-percipient and healthy after death, describing it as formed, or formless, or both formed and formless, or neither formed nor formless. But if any ascetic or brahmin should say this: ‘Apart from form, feeling, perception, and choices, I will describe the coming and going of consciousness, its passing away and reappearing, its growth, increase, and maturity.’\footnote{Also at \href{https://suttacentral.net/sn22.53/en/sujato\#2.2}{SN 22.53:2.2}, etc. | The process of rebirth is often described in terms of consciousness, as in dependent origination or, say, the “stream of consciousness” (\href{https://suttacentral.net/dn28/en/sujato\#7.13}{DN 28:7.13}). However, consciousness never exists by itself, but only in relation to the other aggregates. | The case of the formless realm might seem to be somewhat of an exception. However even formless rebirth is dependent on form of the past, since a meditator must undertake the practice, and also the four “form” absorptions are the basis for the formless attainments. Thus while the formless realms can be described without reference to \emph{present} form they cannot be described without reference to form at all. } That is not possible. ‘All that is conditioned and crude. But there is the cessation of conditions—\emph{that} is real.’ Understanding this and seeing the escape from it, the Realized One has gone beyond all that. 

Now,\marginnote{8.1} the ascetics and brahmins who assert a self that is neither percipient nor non-percipient and healthy after death describe it as formed, or formless, or both formed and formless, or neither formed nor formless. 

So\marginnote{9.1} they reject those who assert a self that is percipient and healthy after death, as well as those who assert a self that is non-percipient and healthy after death. Why is that? Because they believe that perception is a disease, a boil, a dart; that non-perception is a stupor; and that the state of neither perception nor non-perception is peaceful and sublime.\footnote{This is the highest of the formless attainments. | “Stupor” is \textit{sammoha}. } 

The\marginnote{10.1} Realized One understands this as follows. There are ascetics and brahmins who assert a self that is neither percipient nor non-percipient and healthy after death, describing it as formed, or formless, or both formed and formless, or neither formed nor formless. Some ascetics or brahmins assert the embracing of that dimension merely through the conditioned phenomena of what is able to be seen, heard, thought, and known. But that is said to be a disastrous approach.\footnote{The commentary resolves the compound \textit{\textsanskrit{diṭṭhasutamutaviññātabba}} by applying the future passive participle \textit{\textsanskrit{viññātabba}} to the preceding items: \textit{\textsanskrit{diṭṭhaviññātabbamattena} \textsanskrit{sutaviññātabbamattena} \textsanskrit{mutaviññātabbamattena}}. However the same phrase occurs at \href{https://suttacentral.net/sn35.95/en/sujato\#10.1}{SN 35.95:10.1} and \href{https://suttacentral.net/an4.24/en/sujato\#7.5}{AN 4.24:7.5}, and in both cases the future passive sense is clearly meant to be distributed to each item. Thus it means “what is seeable, hearable, thinkable, and knowable”. } For that dimension is said to be not attainable by means of conditioned phenomena,\footnote{Cp. \href{https://suttacentral.net/sn14.11/en/sujato}{SN 14.11}. } but only with a residue of conditioned phenomena.\footnote{This “residue” is the dimension of neither perception nor non-perception itself. } ‘All that is conditioned and crude. But there is the cessation of conditions—\emph{that} is real.’ Understanding this and seeing the escape from it, the Realized One has gone beyond all that. 

Now,\marginnote{11.1} the ascetics and brahmins who assert the annihilation, eradication, and obliteration of an existing being reject those who assert a self that is healthy after death, whether percipient or non-percipient or neither percipient non-percipient. Why is that? Because they think: ‘All of those ascetics and brahmins only assert their attachment to heading upstream:\footnote{For \textit{\textsanskrit{uddhaṁ} \textsanskrit{saraṁ}} (“heading upstream”) in the sense “heading for rebirth in a higher realm”, compare \textit{\textsanskrit{uddhaṁsoto}} (“one who heads upstream”), a term for a non-returner. } “After death we shall be like this! After death we shall be like that!” 

Suppose\marginnote{12.1} a trader was going to market, thinking: “With this, that shall be mine! This way, I shall get that!” In the same way, those ascetics and brahmins seem to be like traders when they say:\footnote{This is a pointed criticism of those who approach spiritual life with a materialist mindset. } “After death we shall be like this! After death we shall be like that!”’ The Realized One understands this as follows. The ascetics and brahmins who assert the annihilation, eradication, and obliteration of an existing being; from fear and disgust with substantial reality, they just keep running and circling around substantial reality.\footnote{“Fear and disgust” is what in the second Noble Truth is called “craving for non-existence” (\textit{\textsanskrit{vibhavataṇhā}}). | “Substantial reality” (\textit{\textsanskrit{sakkāya}}) is the five aggregates that make up the three realms of existence, and which are grasped at as an “existing being” (\textit{sato sattassa}). } Suppose a dog on a leash was tethered to a strong post or pillar. It would just keeping running and circling around that post or pillar.\footnote{\textit{\textsanskrit{Sā}} (Sanskrit \textit{\textsanskrit{śvan}}) is a short form for “dog” (in masculine). This simile is also at \href{https://suttacentral.net/sn22.99/en/sujato\#2.1}{SN 22.99:2.1}. } In the same way, those ascetics and brahmins, from fear and disgust with substantial reality, just keep running and circling around substantial reality. ‘All that is conditioned and crude. But there is the cessation of conditions—\emph{that} is real.’ Understanding this and seeing the escape from it, the Realized One has gone beyond all that. 

Whatever\marginnote{13.1} ascetics and brahmins speculate and theorize about the future, and propose various hypotheses concerning the future, all of them propose one or other of these five theses.\footnote{This statement seems misplaced here, as the introduction included “extinguishment in the present life” which has not yet been discussed. Probably it should occur further down, perhaps at \href{https://suttacentral.net/mn102/en/sujato\#25.1}{MN 102:25.1}. } 

There\marginnote{14.1} are some ascetics and brahmins who theorize about the past, and propose various hypotheses concerning the past. They propose the following, each insisting that theirs is the only truth and that everything else is wrong.\footnote{“Theories of the past” are not mentioned in the introduction. In the \textsanskrit{Brahmajālasutta}, such theories are logically described \emph{before} theories of the future. The theories of the past are similar to those of the \textsanskrit{Brahmajālasutta}, but not identical, so they cannot have been simply copied over. These theories may have originally been mentioned in the introduction and listed at the start of the sutta, but displaced due to textual corruption. We have already seen in the preceding segment that corruption is likely at this point. Analayo discusses this issue in detail (\emph{Comparative Study}, vol. ii, p. 597ff). } ‘The self and the cosmos are eternal.’\footnote{See the four kinds of eternalism at \href{https://suttacentral.net/dn1/en/sujato\#1.30.1}{DN 1:1.30.1}. | The phrase “self and cosmos” (\textit{\textsanskrit{attā} ca loko ca}) refers to the \textsanskrit{Upaniṣadic} view, where the Self is ultimately a contingent aspect of the divinity that is the cosmos. That is why they occur together, as their nature and fate is inseparable. This is not true of other philosophies, as for example where a “soul” is created in a pre-existing universe, or when the universe is felt to be impermanent but the soul is eternal. } ‘The self and the cosmos are not eternal.’\footnote{This implies that at some point in the past, the self and the world appeared, which the commentary equates with the two kinds of “origination by chance” of \href{https://suttacentral.net/dn1/en/sujato\#2.30.1}{DN 1:2.30.1}. } ‘The self and the cosmos are both eternal and not eternal.’\footnote{Compare the four kinds of “partial eternalism” (\href{https://suttacentral.net/dn1/en/sujato\#2.1.1}{DN 1:2.1.1}). } ‘The self and the cosmos are neither eternal nor not eternal.’\footnote{The commentary suggests that these are the “endless flip-floppers” of \href{https://suttacentral.net/dn1/en/sujato\#2.23.1}{DN 1:2.23.1}. That may be so, but they are characterized by the refusal to make a statement, whereas here we have a definite statement, even if the exact sense is unclear. As a rule, constructions of the “neither … nor” type point to a subtle state that does not fit a dualistic description. Perhaps it refers to a philosophy that words such as “eternal” are inadequate, because “eternal” means “live for an infinite time” whereas they believed in a destiny that transcends time. } ‘The self and the cosmos are finite.’\footnote{These four views correspond to the same four “theories of the past” at \href{https://suttacentral.net/dn1/en/sujato\#2.16.1}{DN 1:2.16.1}. In addition, all the following items in this list are “theories of the future” at \href{https://suttacentral.net/dn1/en/sujato\#2.38.7}{DN 1:2.38.7}. Clearly it is possible to have similar views in relation to both the past and future. The classification of the finitude of the cosmos in relation to time seems puzzling, although in a relativistic universe time and distance are inextricably linked by the speed of light. } ‘The self and the cosmos are infinite.’ ‘The self and the cosmos are both finite and infinite.’ ‘The self and the cosmos are neither finite nor infinite.’ ‘The self and the cosmos are unified in perception.’ ‘The self and the cosmos are diverse in perception.’ ‘The self and the cosmos have limited perception.’ ‘The self and the cosmos have limitless perception.’ ‘The self and the cosmos experience nothing but happiness.’ ‘The self and the cosmos experience nothing but suffering.’ ‘The self and the cosmos experience both happiness and suffering.’ ‘The self and the cosmos experience neither happiness nor suffering.’ 

Now,\marginnote{15.1} consider the ascetics and brahmins whose view is as follows. ‘The self and the cosmos are eternal. This is the only truth, anything else is futile.’ It’s simply not possible for them to have personal knowledge of this that is pure and bright, apart from faith, endorsement, oral transmission, reasoned train of thought, or acceptance of a view after deliberation. But in the absence of personal knowledge that is pure and bright, even the portion of knowledge they illuminate is said to be grasping on their part.\footnote{The phrase “pure and bright” (\textit{parisuddhe \textsanskrit{pariyodāte}}) normally describes the fourth absorption. The argument here may be parsed as follows. Even in the highest form of consciousness it is not possible to directly know such things as whether the cosmos is infinite. Any such doctrine, therefore, must be inferred from experience through the five forms of unreliable knowledge starting with faith. They are therefore confidently asserting things of which they have no direct knowledge. Because of this, even the portion of genuine knowledge and insight their meditation has brought them is corrupted by attachment. | \textit{\textsanskrit{Pariyodāta}} (“bright”) is glossed by the commentary as \textit{pabhassara} (“radiant”). But “brighten” doesn’t quite work here for the verbal form \textit{pariyodapenti}, so I use “illuminate”. } ‘All that is conditioned and crude. But there is the cessation of conditions—\emph{that} is real.’ Understanding this and seeing the escape from it, the Realized One has gone beyond all that. 

Now,\marginnote{16.1} consider the ascetics and brahmins whose view is as follows. The self and the cosmos are not eternal, or both eternal and not eternal, or neither eternal nor not-eternal, or finite, or infinite, or both finite and infinite, or neither finite nor infinite, or of unified perception, or of diverse perception, or of limited perception, or of limitless perception, or experience nothing but happiness, or experience nothing but suffering, or experience both happiness and suffering, or experience neither happiness nor suffering. It’s simply not possible for them to have personal knowledge of this that is pure and bright, apart from faith, endorsement, oral transmission, reasoned train of thought, or acceptance of a view after deliberation. But in the absence of personal knowledge that is pure and bright, even the portion of knowledge they illuminate is said to be grasping on their part. ‘All that is conditioned and crude. But there is the cessation of conditions—\emph{that} is real.’ Understanding this and seeing the escape from it, the Realized One has gone beyond all that. 

Now,\marginnote{17.1} some ascetics and brahmins, letting go of theories about the past and the future, not fixating on the fetters of sensuality, enter and remain in the rapture of seclusion:\footnote{The following section illustrates how genuine experiences of deep meditation become misapprehended as “extinguishment in the present life”. It begins with a lengthy and unique description of the absorptions, before specifying how grasping intervenes. | The “rapture of seclusion” includes the first two absorptions, which are characterized by rapture (\textit{\textsanskrit{pīti}}). } ‘This is peaceful, this is sublime, that is, entering and remaining in the rapture of seclusion.’ But that rapture of seclusion of theirs ceases. When the rapture of seclusion ceases, sadness arises; and when sadness ceases, the rapture of seclusion arises.\footnote{Upon emerging from the bliss of absorption, the mind experiences anew all the flavors of life, including sadness. Typically a meditator is happy and refreshed for a long time after a deep meditation. Still, there can be a sense of loss of the blissful state, and a melancholy reflection that that too is impermanent. } 

It’s\marginnote{18.1} like how the sunlight fills the space when the shadow leaves, or the shadow fills the space when the sunshine leaves. In the same way, when the rapture of seclusion ceases, sadness arises; and when sadness ceases, the rapture of seclusion arises. The Realized One understands this as follows. This good ascetic or brahmin, letting go of theories about the past and the future, not fixating on the fetters of sensuality, enters and remains in the rapture of seclusion: ‘This is peaceful, this is sublime, that is, entering and remaining in the rapture of seclusion.’ But that rapture of seclusion of theirs ceases. When the rapture of seclusion ceases, sadness arises; and when sadness ceases, the rapture of seclusion arises. ‘All that is conditioned and crude. But there is the cessation of conditions—\emph{that} is real.’ Understanding this and seeing the escape from it, the Realized One has gone beyond all that. 

Now,\marginnote{19.1} some ascetics and brahmins, letting go of theories about the past and the future, not fixating on the fetters of sensuality, going beyond the rapture of seclusion, enter and remain in pleasure not of the flesh.\footnote{The “pleasure not of the flesh” (\textit{\textsanskrit{nirāmisaṁ} \textsanskrit{sukhaṁ}}) is the third absorption, characterized by the ending of rapture and the persistence of pleasure (\textit{sukha}). At \href{https://suttacentral.net/sn36.31/en/sujato\#5.1}{SN 36.31:5.1} it is described as all the \textit{\textsanskrit{jhānas}} leading up to the third. The same sense is also at \href{https://suttacentral.net/an6.45/en/sujato\#24.3}{AN 6.45:24.3}. } ‘This is peaceful, this is sublime, that is, entering and remaining in pleasure not of the flesh.’ But that pleasure not of the flesh of theirs ceases. When pleasure not of the flesh ceases, the rapture of seclusion arises; and when the rapture of seclusion ceases, pleasure not of the flesh arises. 

It’s\marginnote{20.1} like how the sunlight fills the space when the shadow leaves, or the shadow fills the space when the sunshine leaves. … The Realized One understands this as follows. This good ascetic or brahmin, letting go of theories about the past and the future, not fixating on the fetters of sensuality, going beyond the rapture of seclusion, enters and remains in pleasure not of the flesh. ‘This is peaceful, this is sublime, that is, entering and remaining in pleasure not of the flesh.’ But that pleasure not of the flesh of theirs ceases. When pleasure not of the flesh ceases, the rapture of seclusion arises; and when the rapture of seclusion ceases, pleasure not of the flesh arises. ‘All that is conditioned and crude. But there is the cessation of conditions—\emph{that} is real.’ Understanding this and seeing the escape from it, the Realized One has gone beyond all that. 

Now,\marginnote{21.1} some ascetics and brahmins, letting go of theories about the past and the future, not fixating on the fetters of sensuality, going beyond the rapture of seclusion and pleasure not of the flesh, enter and remain in neutral feeling.\footnote{This is the fourth absorption. } ‘This is peaceful, this is sublime, that is, entering and remaining in neutral feeling.’ Then that neutral feeling ceases. When neutral feeling ceases, pleasure not of the flesh arises; and when pleasure not of the flesh ceases, neutral feelings arises. 

It’s\marginnote{22.1} like how the sunlight fills the space when the shadow leaves, or the shadow fills the space when the sunshine leaves. … The Realized One understands this as follows. This good ascetic or brahmin, letting go of theories about the past and the future, not fixating on the fetters of sensuality, going beyond the rapture of seclusion and pleasure not of the flesh, enters and remains in neutral feeling. ‘This is peaceful, this is sublime, that is, entering and remaining in neutral feeling.’ Then that neutral feeling ceases. When neutral feeling ceases, pleasure not of the flesh arises; and when pleasure not of the flesh ceases, neutral feelings arises. ‘All that is conditioned and crude. But there is the cessation of conditions—\emph{that} is real.’ Understanding this and seeing the escape from it, the Realized One has gone beyond all that. 

Now,\marginnote{23.1} some ascetics and brahmins, letting go of theories about the past and the future, not fixating on the fetters of sensuality, go beyond the rapture of seclusion, pleasure not of the flesh, and neutral feeling. They regard themselves like this: ‘I am at peace; I am quenched; I am free of grasping.’\footnote{The repeated “I am” (\textit{ahamasmi}) emphasizes the conceit that reinforces the sense of self. } 

The\marginnote{24.1} Realized One understands this as follows. This good ascetic or brahmin, letting go of theories about the past and the future, not fixating on the fetters of sensuality, goes beyond the rapture of seclusion, pleasure not of the flesh, and neutral feeling. They regard themselves like this: ‘I am at peace; I am quenched; I am free of grasping.’ Clearly this venerable speaks of a practice that’s conducive to extinguishment.\footnote{The practice of absorption leads to \textsanskrit{Nibbāna}, but if the state of absorption itself is misapprehended as being \textsanskrit{Nibbāna}, the misapprehension is a form of attachment. } Nevertheless, they still grasp at theories about the past or the future, or the fetters of sensuality, or the rapture of seclusion, or pleasure not of the flesh, or neutral feeling.\footnote{This contrasts with the earlier statement that they had let go of views of the past and future before entering absorption. Perhaps they had had temporarily let them go, only for them to re-arise later. Or perhaps there is an editorial issue. } And when they regard themselves like this: ‘I am at peace; I am quenched; I am free of grasping,’ that’s also said to be grasping on their part. ‘All that is conditioned and crude. But there is the cessation of conditions—\emph{that} is real.’ Understanding this and seeing the escape from it, the Realized One has gone beyond all that. 

But\marginnote{25.1} the Realized One has awakened to the supreme state of sublime peace, that is, liberation by not grasping after truly understanding the six fields of contact’s origin, ending, gratification, drawback, and escape.”\footnote{The \textsanskrit{Brahmajālasutta} also emphasizes how liberation from the “sixty-two” wrong views is achieved through insight into the “six fields of contact” (\href{https://suttacentral.net/dn13.71.12/en/sujato}{DN 13.71.12}). } 

That\marginnote{25.3} is what the Buddha said. Satisfied, the mendicants approved what the Buddha said. 

%
\section*{{\suttatitleacronym MN 103}{\suttatitletranslation Is This What You Think Of Me? }{\suttatitleroot Kintisutta}}
\addcontentsline{toc}{section}{\tocacronym{MN 103} \toctranslation{Is This What You Think Of Me? } \tocroot{Kintisutta}}
\markboth{Is This What You Think Of Me? }{Kintisutta}
\extramarks{MN 103}{MN 103}

\scevam{So\marginnote{1.1} I have heard. }At one time the Buddha was staying near \textsanskrit{Kusinārā}, in the Forest of Offerings. There the Buddha addressed the mendicants, “Mendicants!” 

“Venerable\marginnote{1.5} sir,” they replied. The Buddha said this: 

“Mendicants,\marginnote{2.1} is this what you think of me? ‘The ascetic Gotama teaches the Dhamma for the sake of robes, almsfood, lodgings, or rebirth in this or that state.’”\footnote{\textit{\textsanskrit{Bhavābhava}} does not mean “existence and non-existence” but is a distributive compound, “this or that state of existence”. Indian religious texts are full of discussions about different heavens and hells. The Buddha’s goal is liberation, not rebirth in heaven. } 

“No\marginnote{2.3} sir, we don’t think of you that way.” 

“If\marginnote{2.5} you don’t think of me that way, then what exactly do you think of me?” 

“We\marginnote{2.9} think of you this way: ‘The Buddha is sympathetic and wants what’s best for us. He teaches out of sympathy.’” 

“So\marginnote{2.12} it seems you think that I teach out of sympathy. 

In\marginnote{3.1} that case, each and every one of you should train in the things I have taught from my direct knowledge, that is: the four kinds of mindfulness meditation, the four right efforts, the four bases of psychic power, the five faculties, the five powers, the seven awakening factors, and the noble eightfold path. You should train in these things in harmony, appreciating each other, without quarreling.\footnote{These are the sets of practices later called the 37 \textit{\textsanskrit{bodhiyapakkhiyā} \textsanskrit{dhammā}}, the “wings to awakening”. Here they are presented as an essential summary of the Buddha’s teachings. They form the backbone of the final book of the \textsanskrit{Saṁyuttanikāya}, the Maggavagga (or \textsanskrit{Mahāvagga}). It is likely that the Buddha was referring to the earliest recension of this text. They are found as a summary of the Buddha’s teachings at \href{https://suttacentral.net/dn16/en/sujato\#3.50.5}{DN 16:3.50.5}, \href{https://suttacentral.net/dn29/en/sujato\#17.3}{DN 29:17.3}, \href{https://suttacentral.net/mn104/en/sujato\#5.3}{MN 104:5.3}, and \href{https://suttacentral.net/an8.19/en/sujato\#17.2}{AN 8.19:17.2} = \href{https://suttacentral.net/ud5.5/en/sujato\#25.2}{Ud 5.5:25.2}. } 

As\marginnote{4.1} you do so, it may happen that two mendicants disagree about the teaching. 

Now,\marginnote{5.1} you might think, ‘These two venerables disagree on both the meaning and the phrasing.’ So you should approach whichever mendicant you think is most amenable and say to them:\footnote{“Meaning” is \textit{attha}, the “purpose”, “goal”, or “benefit” of the teaching. “Phrasing is \textit{\textsanskrit{byañjana}} (also spelled \textit{\textsanskrit{vyañjana}}), the “letter”, “syllable”, or “expression”, with the root sense of “differentiating”. Such passages show that concern for both aspects were present from the earliest times. | “Most amenable” is \textit{suvacatara}, literally, “more easy to speak to”. } ‘The venerables disagree on the meaning and the phrasing. But the venerables should know that this is how such disagreement on the meaning and the phrasing comes to be. Please don’t get into a dispute about this.’ Then they should approach whichever mendicant they think is most amenable among those who side with the other party and say to them: ‘The venerables disagree on the meaning and the phrasing. But the venerables should know that this is how such disagreement on the meaning and the phrasing comes to be. Please don’t get into a dispute about this.’ So you should remember what has been incorrectly memorized as incorrectly memorized and what has been correctly memorized as correctly memorized.\footnote{When the source of the problem has been identified, it should be remembered so that it does not recur. } Remembering this, you should speak on the teaching and the training. 

Now,\marginnote{6.1} you might think, ‘These two venerables disagree on the meaning but agree on the phrasing.’ So you should approach whichever mendicant you think is most amenable and say to them:\footnote{At \href{https://suttacentral.net/an6.61/en/sujato\#10.5}{AN 6.61:10.5} we see a group of monks discussing a verse of the Dhamma, with different interpretations, and the Buddha praises them as having all spoken well. Thus it is not that \emph{any} different interpretation is wrong, but that \emph{some} different interpretations conflict with the fundamental purpose of the teaching, the \textit{attha}. } ‘The venerables disagree on the meaning but agree on the phrasing. But the venerables should know that this is how such disagreement on the meaning and agreement on the phrasing comes to be. Please don’t get into a dispute about this.’ Then they should approach whichever mendicant they think is most amenable among those who side with the other party and say to them: ‘The venerables disagree on the meaning but agree on the phrasing. But the venerables should know that this is how such disagreement on the meaning and agreement on the phrasing comes to be. Please don’t get into a dispute about this.’ So you should remember what has been incorrectly memorized as incorrectly memorized and what has been correctly memorized as correctly memorized. Remembering this, you should speak on the teaching and the training. 

Now,\marginnote{7.1} you might think, ‘These two venerables agree on the meaning but disagree on the phrasing.’ So you should approach whichever mendicant you think is most amenable and say to them: ‘The venerables agree on the meaning but disagree on the phrasing. But the venerables should know that this is how such agreement on the meaning and disagreement on the phrasing comes to be. But the phrasing is a minor matter.\footnote{One of the reasons why the phrasing is considered less important is that the Buddhist texts are constructed with a massive amount of redundancy. In any question of significance, a variation in a phrase or a word in one place is almost always clarified with reference to another passage. } Please don’t get into a dispute about something so minor.’ Then they should approach whichever mendicant they think is most amenable among those who side with the other party and say to them: ‘The venerables agree on the meaning but disagree on the phrasing. But the venerables should know that this is how such agreement on the meaning and disagreement on the phrasing comes to be. But the phrasing is a minor matter. Please don’t get into a dispute about something so minor.’ So you should remember what has been correctly memorized as correctly memorized and what has been incorrectly memorized as incorrectly memorized. Remembering this, you should speak on the teaching and the training. 

Now,\marginnote{8.1} you might think, ‘These two venerables agree on both the meaning and the phrasing.’ So you should approach whichever mendicant you think is most amenable and say to them: ‘The venerables agree on both the meaning and the phrasing. But the venerables should know that this is how they come to agree on the meaning and the phrasing. Please don’t get into a dispute about this.’ Then they should approach whichever mendicant they think is most amenable among those who side with the other party and say to them: ‘The venerables agree on both the meaning and the phrasing. But the venerables should know that this is how they come to agree on the meaning and the phrasing. Please don’t get into a dispute about this.’ So you should remember what has been correctly memorized as correctly memorized. Remembering this, you should speak on the teaching and the training. 

As\marginnote{9.1} you train in harmony, appreciating each other, without quarreling, one of the mendicants might commit an offense or transgression.\footnote{Namely, a  transgression against the Monastic Code, the Vinaya. Such offences are a collective ethical standard that require other monastics to resolve, at least by hearing a confession. } In such a case, you should not be in a hurry to accuse them. The individual should be examined like this:\footnote{Vinaya is not meant to be applied legalistically or cruelly, but to uplift and reform. | Punctuation in \textsanskrit{Mahāsaṅgīti} is incorrect. The following \textit{iti} belongs on this line, as shown in the repetition below. } ‘I won’t be troubled and the other individual won’t be hurt, for they’re not irritable and acrimonious. They don’t hold fast to their views, but let them go easily. I can draw them away from the unskillful and establish them in the skillful.’ If that’s what you think, then it’s appropriate to speak to them.\footnote{If the problem can be resolved without resorting to legal proceedings, it should. } 

But\marginnote{11.1} suppose you think this: ‘I will be troubled and the other individual will be hurt, for they’re irritable and acrimonious. However, they don’t hold fast to their views, but let them go easily. I can draw them away from the unskillful and establish them in the skillful. But for the other individual to get hurt is a minor matter. It’s more important that I can draw them away from the unskillful and establish them in the skillful.’ If that’s what you think, then it’s appropriate to speak to them. 

But\marginnote{12.1} suppose you think this: ‘I will be troubled but the other individual won’t be hurt, for they’re not irritable and acrimonious. However, they hold fast to their views, refusing to let go. Nevertheless, I can draw them away from the unskillful and establish them in the skillful. But for me to be troubled is a minor matter. It’s more important that I can draw them away from the unskillful and establish them in the skillful.’ If that’s what you think, then it’s appropriate to speak to them. 

But\marginnote{13.1} suppose you think this: ‘I will be troubled and the other individual will be hurt, for they’re irritable and acrimonious. And they hold fast to their views, refusing to let go. Nevertheless, I can draw them away from the unskillful and establish them in the skillful. But for me to be troubled and the other individual to get hurt is a minor matter. It’s more important that I can draw them away from the unskillful and establish them in the skillful.’ If that’s what you think, then it’s appropriate to speak to them. 

But\marginnote{14.1} suppose you think this: ‘I will be troubled and the other individual will be hurt, for they’re irritable and acrimonious. And they hold fast to their views, refusing to let go. I cannot draw them away from the unskillful and establish them in the skillful.’ Don’t underestimate the value of equanimity regarding such a person.\footnote{Wisdom knows that sometimes there is nothing to be done. That doesn’t mean one gives up, but rather, that sometimes patience is the best way. Note that this is the last option, however, and gentle intervention is always preferred. } 

As\marginnote{15.1} you train in harmony, appreciating each other, without quarreling, the sides might continue to bring up settled issues with each other, with contempt for each other’s views, resentful, bitter, and exasperated.\footnote{Readings include \textit{\textsanskrit{vacīsaṁsāro}}, \textit{\textsanskrit{vacīsaṁhāro}}, and \textit{\textsanskrit{vacīsaṅkhāro}}. \textit{\textsanskrit{Vacīsaṁsāro}} is found, without recorded variants, in the parallel passage at \href{https://suttacentral.net/an2.63/en/sujato\#1.1}{AN 2.63:1.1}. The commentary here glosses with \textit{\textsanskrit{vacanasañcāro}}, while the subcommentary to AN 2.63 has \textit{\textsanskrit{pavattamānā}}, both of which support the dominant reading \textit{\textsanskrit{vacīsaṁsāro}}. Their explanations differ, however as the commentary here explains as spreading rumors from one place to another, while at AN 2.63 it is said to be ongoing arguing. Since the normal meaning of \textit{\textsanskrit{saṁsāra}} is “endless ongoing”, and since, in both MN and AN, this follows a passage where disputes were supposed to have been settled, perhaps the sense is “continuing to bring up matters that have been settled”. } In this case you should approach whichever mendicant you think is most amenable among those who side with one party and say to them: ‘Reverend, as we were training, the sides continued to bring up settled issues with each other. If the Ascetic knew about this, would he criticize it?’ Answering rightly, the mendicant should say: ‘Yes, reverend, he would.’ ‘But without giving that up, reverend, can one realize extinguishment?’ Answering rightly, the mendicant should say: ‘No, reverend, one cannot.’ 

Then\marginnote{16.1} they should approach whichever mendicant they think is most amenable among those who side with the other party and say to them: ‘Reverend, as we were training, the sides continued to bring up settled issues with each other. If the Ascetic knew about this, would he criticize it?’ Answering rightly, the mendicant should say: ‘Yes, reverend, he would.’ ‘But without giving that up, reverend, can one realize extinguishment?’ Answering rightly, the mendicant should say: ‘No, reverend, one cannot.’ 

If\marginnote{17.1} others should ask that mendicant: ‘Were you the venerable who drew those mendicants away from the unskillful and established them in the skillful?’ Answering rightly, the mendicant should say: ‘Well, reverends, I approached the Buddha. He taught me the Dhamma. After hearing that teaching I explained it to those mendicants. When those mendicants heard that teaching they were drawn away from the unskillful and established in the skillful.’ Answering in this way, that mendicant doesn’t glorify themselves or put others down. They answer in line with the teaching, with no legitimate grounds for rebuttal and criticism.” 

That\marginnote{17.7} is what the Buddha said. Satisfied, the mendicants approved what the Buddha said. 

%
\section*{{\suttatitleacronym MN 104}{\suttatitletranslation At Sāmagāma }{\suttatitleroot Sāmagāmasutta}}
\addcontentsline{toc}{section}{\tocacronym{MN 104} \toctranslation{At Sāmagāma } \tocroot{Sāmagāmasutta}}
\markboth{At Sāmagāma }{Sāmagāmasutta}
\extramarks{MN 104}{MN 104}

\scevam{So\marginnote{1.1} I have heard. }At one time the Buddha was staying in the land of the Sakyans near the village of \textsanskrit{Sāma}. 

Now\marginnote{2.1} at that time the Jain ascetic of the \textsanskrit{Ñātika} clan had recently passed away at \textsanskrit{Pāvā}.\footnote{When this event is mentioned at \href{https://suttacentral.net/dn29/en/sujato\#1.3}{DN 29:1.3} the Buddha is also in the Sakyan lands, but in a mango grove belonging to a group of Sakyans named \textsanskrit{Vedhaññā}. Both texts tell the story of Cunda conveying the news via Ānanda at \textsanskrit{Sāma}, which was presumably near the mango grove. It does seem strange that two distinct discourses are recorded from the same prompt, but then, why shouldn’t the Buddha give more than one teaching on such an important topic? At \href{https://suttacentral.net/dn33/en/sujato\#1.6.1}{DN 33:1.6.1} the Buddha is at \textsanskrit{Pāvā} in the Mallian lands, and the discourse is spoken by \textsanskrit{Sāriputta} there. Given the evident lateness of DN 33, this is a less convincing framework. A parallel to MN 104 (MA 196 at T i 752c12) says he was in the Vajjian lands at the time; both Sakya and Vajji border on \textsanskrit{Mallā}. } With his passing the Jain ascetics split, dividing into two factions, arguing, quarreling, and disputing, continually wounding each other with barbed words:\footnote{While this description of the Jains might seem like sheer sectarian calumny, it is a fact that the Jain tradition is split into two sects, the “sky-clad” Digambara, whose male ascetics went naked, and the “white-clad” \textsanskrit{Śvetāmbara} who wear an unstitched cloth. Jain tradition holds that the split occurred about a century later, in the reign of Candragupta Maurya. } “You don’t understand this teaching and training. I understand this teaching and training. What, you understand this teaching and training? You’re practicing wrong. I’m practicing right. I stay on topic, you don’t. You said last what you should have said first. You said first what you should have said last. What you’ve thought so much about has been disproved. Your doctrine is refuted. Go on, save your doctrine! You’re trapped; get yourself out of this—if you can!” You’d think there was nothing but slaughter going on among the Jain ascetics.\footnote{A slight on the movement whose signature virtue was non-violence. } And the Jain \textsanskrit{Ñātika}’s white-clothed lay disciples were disillusioned, dismayed, and disappointed in the Jain ascetics. They were equally disappointed with a teaching and training so poorly explained and poorly propounded, not emancipating, not leading to peace, proclaimed by someone who is not a fully awakened Buddha, with broken monument and without a refuge.\footnote{“With broken monument” (\textit{bhinnathupe}) is used only in this context. When a great teacher or leader died, a “monument” was built to keep their memory alive. The breaking of a monument—whether physical or symbolic—was, in a way, truly killing them. } 

And\marginnote{3.1} then, after completing the rainy season residence near \textsanskrit{Pāvā}, the novice Cunda went to see Venerable Ānanda at \textsanskrit{Sāma} village. He bowed, sat down to one side, and told him what had happened.\footnote{Cunda took the time to complete his rains residence before conveying the news, reminding us of the speed with which news traveled in those days—slowly. For Cunda’s identity, see the note to \href{https://suttacentral.net/mn8/en/sujato\#2.1}{MN 8:2.1} | Apart from the events described here, we hear of \textsanskrit{Sāma} only once (\href{https://suttacentral.net/an6.21/en/sujato}{AN 6.21}). The topic there is the decline of the \textsanskrit{Saṅgha}, hinting at a connection with these events. } 

Ānanda\marginnote{4.1} said to him, “Reverend Cunda, we should see the Buddha about this matter. Come, let’s go to the Buddha and inform him about this.” 

“Yes,\marginnote{4.4} sir,” replied Cunda. 

Then\marginnote{4.5} Ānanda and Cunda went to the Buddha, bowed, sat down to one side, and Ānanda informed him of what Cunda had said. He went on to say, “Sir, it occurs to me: ‘When the Buddha has passed away, let no dispute arise in the \textsanskrit{Saṅgha}. For such a dispute would be for the detriment and suffering of the people, against the people, for the harm, detriment, and suffering of gods and humans.’” 

“What\marginnote{5.1} do you think, Ānanda? Do you see even two mendicants whose opinion differs regarding the things I have taught from my direct knowledge, that is,\footnote{Two brahmins have a “difference of opinion” (\textit{\textsanskrit{nānāvāda}}) regarding their path at \href{https://suttacentral.net/dn13/en/sujato\#8.8}{DN 13:8.8}. } the four kinds of mindfulness meditation, the four right efforts, the four bases of psychic power, the five faculties, the five powers, the seven awakening factors, and the noble eightfold path?” 

“No,\marginnote{5.4} sir, I do not. Nevertheless, there are some individuals who appear to live obedient to the Buddha, but when the Buddha has passed away they might create a dispute in the \textsanskrit{Saṅgha} regarding livelihood or the monastic code.\footnote{As in the quarrel at \textsanskrit{Kosambī} (\href{https://suttacentral.net/mn48/en/sujato}{MN 48}, \href{https://suttacentral.net/pli-tv-kd10/en/sujato}{Kd 10}). } Such a dispute would be for the detriment and suffering of the people, against the people, for the harm, detriment, and suffering of gods and humans.” 

“Ānanda,\marginnote{5.8} dispute about livelihood or the monastic code is a minor matter.\footnote{Just as the “phrasing” of the text is a minor matter at \href{https://suttacentral.net/mn103/en/sujato\#7.12}{MN 103:7.12}. } But should a dispute arise in the \textsanskrit{Saṅgha} concerning the path or the practice, that would be for the detriment and suffering of the people, against the people, for the harm, detriment, and suffering of gods and humans.\footnote{This refers back to the teachings the Buddha has summarized above. } 

Ānanda,\marginnote{6.1} there are these six roots of arguments.\footnote{The Buddha is more concerned with the unskillful roots of division than with the surface details. The first four pairs of “roots” are also found as “corruptions of mind” at \href{https://suttacentral.net/mn7/en/sujato\#3.1}{MN 7:3.1}. } What six? Firstly, a mendicant is irritable and acrimonious. Such a mendicant lacks respect and reverence for the teacher, the teaching, and the \textsanskrit{Saṅgha}, and they don’t fulfill the training. They create a dispute in the \textsanskrit{Saṅgha}, which is for the detriment and suffering of the people, against the people, for the harm, detriment, and suffering of gods and humans. If you see such a root of arguments in yourselves or others, you should try to give up this bad thing. If you don’t see it, you should practice so that it doesn’t come up in the future. That’s how to give up this bad root of arguments, so it doesn’t come up in the future. 

Furthermore,\marginnote{7{-}11.1} a mendicant is offensive and contemptuous … They’re jealous and stingy … They’re devious and deceitful … They have corrupt wishes and wrong view … They’re attached to their own views, holding them tight, and refusing to let go. Such a mendicant lacks respect and reverence for the teacher, the teaching, and the \textsanskrit{Saṅgha}, and they don’t fulfill the training. They create a dispute in the \textsanskrit{Saṅgha}, which is for the detriment and suffering of the people, against the people, for the harm, detriment, and suffering of gods and humans. If you see such a root of arguments in yourselves or others, you should try to give up this bad thing. If you don’t see it, you should practice so that it doesn’t come up in the future. That’s how to give up this bad root of arguments, so it doesn’t come up in the future. These are the six roots of arguments. 

There\marginnote{12.1} are four kinds of disciplinary issues. What four? Disciplinary issues due to disputes, accusations, offenses, or business.\footnote{These are formal legal proceedings in the Vinaya, explained at \href{https://suttacentral.net/pli-tv-kd14/en/sujato\#14.2.1}{Kd 14:14.2.1}. | Here and below I follow the renderings of Brahmali for legal terms. } These are the four kinds of disciplinary issues. There are seven methods for the settlement of any disciplinary issues that might arise.\footnote{Explained in more detail in the Vinaya (\href{https://suttacentral.net/pli-tv-kd14/en/sujato\#14.16.1}{Kd 14:14.16.1}). } Resolution face-to-face to be applied. Resolution through recollection to be granted. Resolution because of past insanity to be granted. Acting according to what has been admitted. Majority decision. Further penalty. Covering over as if with grass. 

And\marginnote{14.1} how is there resolution face-to-face? It’s when mendicants are disputing: ‘This is the teaching,’ ‘This is not the teaching,’ ‘This is the monastic law,’ ‘This is not the monastic law.’ Those mendicants should all sit together in harmony and thoroughly go over the guidelines of the teaching.\footnote{\textit{Samanumajjati} occurs only here. The root \textit{majj} has the general sense “stroke, massage”. Here it means that they should go over and over until it is settled. | \textit{Dhammanetti} (“the guidelines of the teaching”) is another rare term, occurring elsewhere only in the late canonical \href{https://suttacentral.net/mil6.3.12/en/sujato\#7.5}{Mil 6.3.12:7.5}. } They should settle that disciplinary issue in agreement with the guidelines. That’s how there is resolution face-to-face. And that’s how certain disciplinary issues are settled, that is, by resolution face-to-face. 

And\marginnote{15.1} how is there majority decision?\footnote{This explanation is out of order, as it is listed as the second item here, but it is normally the fifth item, as in the summary above. } If those mendicants are not able to settle that issue in that monastery,\footnote{Normally issues are settled by consensus, so this applies in cases where consensus cannot be reached. } they should go to another monastery with more mendicants. There they should all sit together in harmony and thoroughly go over the guidelines of the teaching. They should settle that disciplinary issue in agreement with the guidelines. That’s how there is majority decision. And that’s how certain disciplinary issues are settled, that is, by majority decision. 

And\marginnote{16.1} how is there resolution through recollection? It’s when mendicants accuse a mendicant of a serious offense; one entailing expulsion, or close to it: ‘Venerable, do you recall committing the kind of serious offense that entails expulsion or close to it?’ They say: ‘No, reverends, I don’t recall committing such an offense.’\footnote{The Vinaya operates on good faith, so that the testimony of a clear-headed mendicant is sufficient to clear them from offences in most cases. } The resolution through recollection should be granted to them. That’s how there is the resolution through recollection. And that’s how certain disciplinary issues are settled, that is, by resolution through recollection. 

And\marginnote{17.1} how is there resolution because of past insanity? It’s when a mendicant accuses a mendicant of the kind of serious offense that entails expulsion, or close to it:\footnote{The offenses entailing expulsion for a mendicant are: willingly engaging in sexual intercourse; intentional killing of a human being; theft of something of value; and deliberately making a false claim to exalted spiritual attainments such as awakening or absorption. | Pali editions vary as to whether the accuser is singular or plural. However, even those editions that use the plural here (\textsanskrit{Mahāsaṅgīti}, PTS) shift to singular below (\href{https://suttacentral.net/mn104/en/sujato\#17.6}{MN 104:17.6}, \href{https://suttacentral.net/mn104/en/sujato\#19.2}{MN 104:19.2}), while in the Vinaya it is also singular (\href{https://suttacentral.net/pli-tv-kd14/en/sujato\#14.29.7}{Kd 14:14.29.7}). } ‘Venerable, do you recall committing the kind of serious offense that entails expulsion or close to it?’ They say: ‘No, reverend, I don’t recall committing such an offense.’ But though they try to get out of it, the mendicant pursues the issue: ‘Surely the venerable must know perfectly well if you recall committing an offense that entails expulsion or close to it!’ They say: ‘Reverend, I had gone mad, I was out of my mind. And while I was mad I did and said many things that are not proper for an ascetic. I don’t remember any of that, I was mad when I did it.’\footnote{This evidently refers to an episode of psychosis, during which time a person is subject to hallucinations and delusions, and may exhibit bizarre behavior. There is a blanket exemption in Vinaya for all acts committed in such a state. } The resolution because of past insanity should be granted to them. That’s how there is resolution because of past insanity. And that’s how certain disciplinary issues are settled, that is, by resolution because of past insanity. 

And\marginnote{18.1} how is there acting according to what has been admitted? It’s when a mendicant, whether accused or not, recalls an offense and clarifies it and reveals it. After approaching a more senior mendicant, that mendicant should arrange his robe over one shoulder, bow to that mendicant’s feet, squat on their heels, raise their joined palms, and say:\footnote{This is the standard way of clearing offenses by confessing them as soon as they are recalled and recognized. Mendicants still confess with this formula today. Most offenses can be cleared in this way, but others require the relinquishment of a certain object (such as money), or undergoing a period of probation, while expulsion offenses cannot be cleared. } ‘Sir, I have fallen into such-and-such an offense. I confess it.’ The senior mendicant says: ‘Do you see it?’ ‘Yes, I see it.’ ‘Then restrain yourself in future.’ ‘I shall restrain myself.’ That’s how there is acting according to what has been admitted. And that’s how certain disciplinary issues are settled, that is, by acting according to what has been admitted. 

And\marginnote{19.1} how is there a verdict of further penalty?\footnote{“Further penalty” renders \textit{\textsanskrit{tassapāpiyasikā}}, more literally “what is worse for him”. In this case, a mendicant tries to deny the accusation but ends up just making it worse. } It’s when a mendicant accuses a mendicant of the kind of serious offense that entails expulsion, or close to it: ‘Venerable, do you recall committing the kind of serious offense that entails expulsion or close to it?’ They say: ‘No, reverends, I don’t recall committing such an offense.’ But though they try to get out of it, the mendicants pursue the issue:\footnote{Normally the mendicant’s denial is sufficient, but the case might be pressed if there is substantial evidence. } ‘Surely the venerable must know perfectly well if you recall committing an offense that entails expulsion or close to it!’ They say: ‘Reverends, I don’t recall committing a serious offense of that nature. But I do recall committing a light offense.’ But though they try to get out of it, the mendicants pursue the issue: ‘Surely the venerable must know perfectly well if you recall committing an offense that entails expulsion or close to it!’ They say: ‘Reverends, I’ll go so far as to acknowledge this light offense even when not asked. Why wouldn’t I acknowledge a serious offense when asked?’ They say: ‘You wouldn’t have acknowledged that light offense without being asked, so why would you acknowledge a serious offense? Surely the venerable must know perfectly well if you recall committing an offense that entails expulsion or close to it!’ They say: ‘Reverend, I do recall committing the kind of serious offense that entails expulsion or close to it. I spoke too hastily when I said that I didn’t recall it.’ That’s how there is further penalty. And that’s how certain disciplinary issues are settled, that is, by further penalty. 

And\marginnote{20.1} how is there the covering over as if with grass? It’s when the mendicants continually argue, quarrel, and dispute, doing and saying many things that are not proper for an ascetic. Those mendicants should all sit together in harmony. A competent mendicant of one party, having got up from their seat, arranged their robe over one shoulder, and raised their joined palms, should inform the \textsanskrit{Saṅgha}: 

‘Sir,\marginnote{20.5} let the \textsanskrit{Saṅgha} listen to me. We have been continually arguing, quarreling, and disputing, doing and saying many things that are not proper for an ascetic. If it seems appropriate to the \textsanskrit{Saṅgha}, then—for the benefit of these venerables and myself—I disclose in the middle of the \textsanskrit{Saṅgha} by means of covering over as if with grass any offenses committed by these venerables and by myself, excepting only those that are gravely blameworthy and those connected with laypeople.’ 

Then\marginnote{20.8} a competent mendicant of the other party, having got up from their seat, arranged their robe over one shoulder, and raising their joined palms, should inform the \textsanskrit{Saṅgha}: 

‘Sir,\marginnote{20.9} let the \textsanskrit{Saṅgha} listen to me. We have been continually arguing, quarreling, and disputing, doing and saying many things that are not proper for an ascetic. If it seems appropriate to the \textsanskrit{Saṅgha}, then—for the benefit of these venerables and myself—I disclose in the middle of the \textsanskrit{Saṅgha} by means of covering over as if with grass any offenses committed by these venerables and by myself, excepting only those that are gravely blameworthy and those connected with laypeople.’\footnote{Those that are “gravely blameworthy” include offenses of expulsion (\textit{\textsanskrit{pārājika}}) and suspension (\textit{\textsanskrit{saṅghādisesa}}), while those “connected with laypeople” include cases where the mendicant abuses the laity. } 

That’s\marginnote{20.12} how there is the covering over as if with grass. And that’s how certain disciplinary issues are settled, that is,\footnote{This final method is a reminder that it is not always necessary to dig down to the bottom of every dispute, especially when blame accrues to both sides. Sometimes the most important thing is to move forward. } by covering over as if with grass. 

Ānanda,\marginnote{21.1} these six warm-hearted qualities make for fondness and respect, conducing to inclusion, harmony, and unity, without quarreling.\footnote{Also taught at \href{https://suttacentral.net/mn48/en/sujato\#6.2}{MN 48:6.2}, \href{https://suttacentral.net/an6.11/en/sujato\#1.1}{AN 6.11:1.1}, and \href{https://suttacentral.net/an6.12/en/sujato\#1.1}{AN 6.12:1.1}, and collected at \href{https://suttacentral.net/dn33/en/sujato\#2.2.37}{DN 33:2.2.37} and \href{https://suttacentral.net/dn34/en/sujato\#1.7.3}{DN 34:1.7.3}. } What six? Firstly, a mendicant consistently treats their spiritual companions with bodily kindness, both in public and in private. This warm-hearted quality makes for fondness and respect, conducing to inclusion, harmony, and unity, without quarreling. 

Furthermore,\marginnote{21.5} a mendicant consistently treats their spiritual companions with verbal kindness … This too is a warm-hearted quality. 

Furthermore,\marginnote{21.7} a mendicant consistently treats their spiritual companions with mental kindness … This too is a warm-hearted quality. 

Furthermore,\marginnote{21.9} a mendicant shares without reservation any material things they have gained by legitimate means, even the food placed in the alms-bowl, using them in common with their ethical spiritual companions. This too is a warm-hearted quality. 

Furthermore,\marginnote{21.11} a mendicant lives according to the precepts shared with their spiritual companions, both in public and in private. Those precepts are intact, impeccable, spotless, and unmarred, liberating, praised by sensible people, not mistaken, and leading to immersion. This too is a warm-hearted quality. 

Furthermore,\marginnote{21.13} a mendicant lives according to the view shared with their spiritual companions, both in public and in private. That view is noble and emancipating, and delivers one who practices it to the complete ending of suffering. This too is a warm-hearted quality. 

These\marginnote{21.15} six warm-hearted qualities make for fondness and respect, conducing to inclusion, harmony, and unity, without quarreling. 

If\marginnote{22.1} you should undertake and follow these six warm-hearted qualities, do you see any criticism, large or small, that you could not endure?” 

“No,\marginnote{22.2} sir.” 

“That’s\marginnote{22.3} why, Ānanda, you should undertake and follow these six warm-hearted qualities. That will be for your lasting welfare and happiness.” 

That\marginnote{22.5} is what the Buddha said. Satisfied, Venerable Ānanda approved what the Buddha said. 

%
\section*{{\suttatitleacronym MN 105}{\suttatitletranslation With Sunakkhatta }{\suttatitleroot Sunakkhattasutta}}
\addcontentsline{toc}{section}{\tocacronym{MN 105} \toctranslation{With Sunakkhatta } \tocroot{Sunakkhattasutta}}
\markboth{With Sunakkhatta }{Sunakkhattasutta}
\extramarks{MN 105}{MN 105}

\scevam{So\marginnote{1.1} I have heard. }At one time the Buddha was staying near \textsanskrit{Vesālī}, at the Great Wood, in the hall with the peaked roof. 

Now\marginnote{2.1} at that time several mendicants had declared their enlightenment in the Buddha’s presence:\footnote{Also at \href{https://suttacentral.net/sn12.70/en/sujato\#4.1}{SN 12.70:4.1}, which concerns the dubious ordination of the wanderer \textsanskrit{Susīma}. | “Enlightenment” is \textit{\textsanskrit{aññā}}, a technical term that refers to the knowledge of perfection (\textit{\textsanskrit{arahattā}}). | A Buddhist mendicant incurs a penalty of expulsion if they deliberately lay false claim to awakening (\href{https://suttacentral.net/pli-tv-bu-vb-pj4/en/sujato}{Bu Pj 4}), and a penalty of confession if they truthfully claim awakening to someone who is not ordained (\href{https://suttacentral.net/pli-tv-bu-vb-pc8/en/sujato}{Bu Pc 8}). These rules directly concern laying claim regarding oneself, but the background stories apply when making claims with regards to another. Mendicants may declare their attainments among the community of monks and nuns who are fully ordained. Typically such a solemn matter is spoken of in a formal and respectful manner with one’s teacher. } 

“We\marginnote{2.2} understand: ‘Rebirth is ended, the spiritual journey has been completed, what had to be done has been done, there is nothing further for this place.’” 

Sunakkhatta\marginnote{2.3} the Licchavi heard about this.\footnote{This is apparently Sunakkhatta’s first meeting with the Buddha. In \href{https://suttacentral.net/dn6/en/sujato\#5.3}{DN 6:5.3} we learn that, after being ordained three years, he spoke of his limited success in meditation. \href{https://suttacentral.net/mn12/en/sujato}{MN 12} and \href{https://suttacentral.net/dn24/en/sujato}{DN 24} deal with his bitter criticisms of the Buddha shortly after his disrobal. } 

He\marginnote{3.1} went to the Buddha, bowed, sat down to one side, and said to him, “Sir, I have heard that several mendicants have declared their enlightenment in the Buddha’s presence. I trust they did so rightly—or are there some who declared enlightenment out of overestimation?”\footnote{By a loophole in the wording of the Vinaya, it is not an offence to tell anyone of your attainments if you are genuinely deluded and falsely believe that you have an attainment that you do not have. The Buddha dismissed such “overestimation” (\textit{\textsanskrit{adhimāna}}) as “negligible” (\href{https://suttacentral.net/pli-tv-bu-vb-pj4/en/sujato\#2.1}{Bu Pj 4:2.1}). } 

“Some\marginnote{5.1} of them did so rightly, Sunakkhatta, while others did so out of overestimation. Now, when mendicants declare enlightenment rightly, that is their truth. But when mendicants declare enlightenment out of overestimation, the Realized One thinks: ‘I should teach them the Dhamma.’ If the Realized One thinks he should teach them the Dhamma, but then certain futile men, having carefully planned a question, approach the Realized One and ask it, then the Realized One changes his mind.”\footnote{That is, if they plan to trap the Buddha with sophistry, he realizes that they are not ready to be taught. } 

“Now\marginnote{6.1} is the time, Blessed One! Now is the time, Holy One! Let the Buddha teach the Dhamma. The mendicants will listen and remember it.” 

“Well\marginnote{6.3} then, Sunakkhatta, listen and apply your mind well, I will speak.” 

“Yes,\marginnote{6.4} sir,” replied Sunakkhatta. The Buddha said this: 

“Sunakkhatta,\marginnote{7.1} there are these five kinds of sensual stimulation. What five? Sights known by the eye, which are likable, desirable, agreeable, pleasant, sensual, and arousing. Sounds known by the ear … Smells known by the nose … Tastes known by the tongue … Touches known by the body, which are likable, desirable, agreeable, pleasant, sensual, and arousing. These are the five kinds of sensual stimulation. 

It’s\marginnote{8.1} possible that a certain individual may be intent on worldly pleasures of the flesh. Such an individual engages in pertinent conversation, thinking and considering in line with that. They associate with that kind of person, and they find it satisfying. But when talk connected with the imperturbable is going on they don’t want to listen. They don’t actively listen or try to understand. They don’t associate with that kind of person, and they don’t find it satisfying.\footnote{The “imperturbable” refers to high meditative attainments, typically the fourth absorption and above. } 

Suppose\marginnote{9.1} a person had left their own village or town long ago, and they saw another person who had only recently left there. They would ask about whether their village was safe, with plenty of food and little disease, and the other person would tell them the news. What do you think, Sunakkhatta? Would that person want to listen to that other person? Would they actively listen and try to understand? Would they associate with that person, and find it satisfying?” 

“Yes,\marginnote{9.7} sir.” 

“In\marginnote{9.8} the same way, it’s possible that a certain individual may be intent on worldly pleasures of the flesh. Such an individual engages in pertinent conversation, thinking and considering in line with that. They associate with that kind of person, and they find it satisfying. But when talk connected with the imperturbable is going on they don’t want to listen. They don’t actively listen or try to understand. They don’t associate with that kind of person, and they don’t find it satisfying. You should know of them: ‘That individual is intent on pleasures of the flesh, for they’re detached from things connected with the imperturbable.’\footnote{The reading \textit{\textsanskrit{āneñjasaṁyojanena} hi kho \textsanskrit{visaṁyutto}} is problematic, as it suggests they are free of the fetter of the imperturbable, which would mean they are an arahant. Several manuscripts omit this phrase, but it is glossed in the commentary. I think it harks back to the immediately prior \textit{\textsanskrit{āneñjapaṭisaṁyuttāya}}; possibly the \textit{\textsanskrit{paṭi}} dropped out in an old misreading. They’re not detached from the “fetter” of the imperturbable, but from “what is connected with” the imperturbable, i.e. they are disinterested in the talk about it. } 

It’s\marginnote{10.1} possible that a certain individual may be intent on the imperturbable. Such an individual engages in pertinent conversation, thinking and considering in line with that. They associate with that kind of person, and they find it satisfying. But when talk connected with worldly pleasures of the flesh is going on they don’t want to listen. They don’t actively listen or try to understand. They don’t associate with that kind of person, and they don’t find it satisfying. 

Suppose\marginnote{11.1} there was a fallen, withered leaf. It’s incapable of becoming green again. In the same way, an individual intent on the imperturbable has dropped the connection with worldly pleasures of the flesh.\footnote{Here and in the following examples, the verb in the explanation follows the metaphor: a leaf is “dropped”, a stone is “broken”, etc. } You should know of them: ‘That individual is intent on the imperturbable, for they’re detached from things connected with pleasures of the flesh.’ 

It’s\marginnote{12.1} possible that a certain individual may be intent on the dimension of nothingness. Such an individual engages in pertinent conversation, thinking and considering in line with that. They associate with that kind of person, and they find it satisfying.\footnote{This case and the next would apply to the Buddha’s former teachers, who taught the dimensions of nothingness and neither perception nor non-perception respectively (\href{https://suttacentral.net/mn26/en/sujato\#15.1}{MN 26:15.1}). } But when talk connected with the imperturbable is going on they don’t want to listen. They don’t actively listen or try to understand. They don’t associate with that kind of person, and they don’t find it satisfying.\footnote{In this case, the “imperturbable” must refer to the fourth absorption and the first two formless attainments. } 

Suppose\marginnote{13.1} there was a broad rock that had been broken in half, so that it could not be put back together again. In the same way, an individual intent on the dimension of nothingness has broken the connection with the imperturbable. You should know of them: ‘That individual is intent on the dimension of nothingness, for they’re detached from things connected with the imperturbable.’ 

It’s\marginnote{14.1} possible that a certain individual may be intent on the dimension of neither perception nor non-perception. Such an individual engages in pertinent conversation, thinking and considering in line with that. They associate with that kind of person, and they find it satisfying. But when talk connected with the dimension of nothingness is going on they don’t want to listen. They don’t actively listen or try to understand. They don’t associate with that kind of person, and they don’t find it satisfying. 

Suppose\marginnote{15.1} someone had eaten some delectable food and thrown it up. What do you think, Sunakkhatta? Would that person want to eat that food again?” 

“No,\marginnote{15.4} sir. Why is that? Because that food is deemed repulsive.” 

“In\marginnote{15.7} the same way, an individual intent on the dimension of neither perception nor non-perception has vomited the connection with the dimension of nothingness. You should know of them: ‘That individual is intent on the dimension of neither perception nor non-perception, for they’re detached from things connected with the dimension of nothingness.’ 

It’s\marginnote{16.1} possible that a certain individual may be rightly intent on extinguishment. Such an individual engages in pertinent conversation, thinking and considering in line with that. They associate with that kind of person, and they find it satisfying. But when talk connected with the dimension of neither perception nor non-perception is going on they don’t want to listen. They don’t actively listen or try to understand. They don’t associate with that kind of person, and they don’t find it satisfying. 

Suppose\marginnote{17.1} there was a palm tree with its crown cut off. It’s incapable of further growth. In the same way, an individual rightly intent on extinguishment has cut off the connection with the dimension of neither perception nor non-perception at the root, made it like a palm stump, obliterated it, so it’s unable to arise in the future. You should know of them: ‘That individual is rightly intent on extinguishment, for they’re detached from things connected with the dimension of neither perception nor non-perception.’ 

It’s\marginnote{18.1} possible that a certain mendicant might think:\footnote{The Buddha goes on to discuss a case of overestimation. } ‘The Ascetic has said that craving is a dart; and that the poison of ignorance is inflicted by desire and ill will.\footnote{Here, the “Ascetic” (\textit{\textsanskrit{samaṇa}}) is the Buddha. } I have given up the dart of craving and expelled the poison of ignorance; I am rightly intent on extinguishment.’ Having such conceit, though it’s not based in fact, they would engage in things unconducive to extinguishment: unsuitable sights, sounds, smells, tastes, touches, and ideas. Doing so, lust infects their mind, resulting in death or deadly pain. 

Suppose\marginnote{19.1} a man was struck by an arrow thickly smeared with poison. Their friends and colleagues, relatives and kin would get a surgeon to treat them. The surgeon would cut open the wound with a scalpel, probe for the arrow, extract it, and expel the poison, leaving some residue behind.\footnote{The term \textit{\textsanskrit{saupādisesa}} is better known as a description of those who have attained Nibbana “with residue”, namely the five aggregates that persist so long as life lasts. The prefix \textit{\textsanskrit{upādi}} is usually taken as derived from \textit{\textsanskrit{upādāna}}, normally “grasping”, but here in its secondary sense of “material cause, fuel”. The “residue” of poison is the material cause of potential harm, just as the five aggregates still entail a residue of physical suffering for a perfected one. } Imagining that no residue remained, the surgeon would say:\footnote{For “imagining that no residue remained”, I follow PTS and Siamese editions, which have \textit{\textsanskrit{anupādiseso} ti \textsanskrit{maññamāno}}. Bodhi’s note 1003 to his \emph{Middle Length Discourses} points out that earlier manuscripts support this reading, which seems to have been “corrected” to \textit{\textsanskrit{saupādisesoti} \textsanskrit{jānamāno}} in the Burmese Sixth Council edition on which SuttaCentral’s \textsanskrit{Mahāsaṅgīti} edition is based. A Sanskrit fragment also supports the PTS/Siamese reading: \textit{(\textsanskrit{sā})va(\textsanskrit{ś})\textsanskrit{eṣaṁ} nira(\textsanskrit{vaśe})\textsanskrit{ṣaṁ} (iti \textsanskrit{manyamānaḥ})} (SHT IV 500 folio 3V3, cited in Analayo, \emph{Comparative Study}, p. 611, note 136). } ‘My good man, the dart has been extracted and the poison expelled without residue. It is still capable of harming you.\footnote{Most editions have “not capable” of harming here, which seems odd given that the doctor immediately goes on to describe how to mitigate possible harm. I follow the PTS edition, which is alone in reading \textit{\textsanskrit{alañ} ca te \textsanskrit{antarāyāya}}. This is supported by a Sanskrit fragment, which reads \textit{ala(\textsanskrit{ṁ})te-t(r)-\textsanskrit{ānta}(\textsanskrit{rā})\textsanskrit{yāya}} (SHT IV 500 folio 3V4, cited in Analayo, ibid.). } Eat only suitable food. Don’t eat unsuitable food, or else the wound may get infected. Regularly wash the wound and anoint the opening, or else it’ll get covered with pus and blood. Don’t walk too much in the wind and sun, or else dust and dirt will infect the wound. Take care of the wound, my good sir, heal it.’ 

They’d\marginnote{20.1} think: ‘The dart has been extracted and the poison expelled with no residue. It’s not capable of harming me.’\footnote{Pali editions all agree that the patient thinks it cannot cause harm, which, if we adopt the PTS reading as I have above, implies that the patient misunderstood the doctor’s instructions. A Sanskrit fragment says that he did understand that it could cause harm, suggesting that he did what was unsuitable despite knowing this: \textit{\textsanskrit{niravaśeṣaḥ} a(\textsanskrit{panīto}) \textsanskrit{viṣa}-\textsanskrit{doṣaḥ} \textsanskrit{alaṁ} (me-tr)-\textsanskrit{ān}(ta)\textsanskrit{rā}(\textsanskrit{yāya})} (SHT IV 500 folio 3R3, cited in Analayo, ibid.). I don’t follow this as it is not found in any Pali edition. } They’d eat unsuitable food, and the wound would get infected. And they wouldn’t regularly wash and anoint the opening, so it would get covered in pus and blood. And they’d walk too much in the wind and sun, so dust and dirt infected the wound. And they wouldn’t take care of the wound or heal it. Then both because they did what was unsuitable, and because of the residue of unclean poison, the wound would spread, resulting in death or deadly pain. 

In\marginnote{21.1} the same way, it’s possible that a certain mendicant might think: ‘The Ascetic has said that craving is a dart; and that the poison of ignorance is inflicted by desire and ill will. I have given up the dart of craving and expelled the poison of ignorance; I am rightly intent on extinguishment.’ Having such conceit, though it’s not based in fact, they would engage in things unconducive to extinguishment: unsuitable sights, sounds, smells, tastes, touches, and ideas. Doing so, lust infects their mind, resulting in death or deadly pain. 

For\marginnote{22.1} it is death in the training of the Noble One to reject the training and return to a lesser life. And it is deadly pain to commit one of the corrupt offenses. 

It’s\marginnote{23.1} possible that a certain mendicant might think: ‘The Ascetic has said that craving is a dart; and that the poison of ignorance is inflicted by desire and ill will. I have given up the dart of craving and expelled the poison of ignorance; I am rightly intent on extinguishment.’ Being rightly intent on extinguishment, they wouldn’t engage in things unconducive to extinguishment: unsuitable sights, sounds, smells, tastes, touches, and ideas. Doing so, lust wouldn’t infect their mind, so no death or deadly pain would result. 

Suppose\marginnote{24.1} a man was struck by an arrow thickly smeared with poison. Their friends and colleagues, relatives and kin would get a surgeon to treat them. The surgeon would cut open the wound with a scalpel, probe for the arrow, extract it, and expel the poison, leaving no residue behind. Knowing that no residue remained, the surgeon would say: ‘My good man, the dart has been extracted and the poison expelled with no residue. It’s not capable of harming you. Eat only suitable food. Don’t eat unsuitable food, or else the wound may get infected. Regularly wash the wound and anoint the opening, or else it’ll get covered with pus and blood. Don’t walk too much in the wind and sun, or else dust and dirt will infect the wound. Take care of the wound, my good sir, heal it.’ 

They’d\marginnote{25.1} think: ‘The dart has been extracted and the poison expelled with no residue. It’s not capable of harming me.’ They’d eat suitable food, and the wound wouldn’t get infected. And they’d regularly wash and anoint the opening, so it wouldn’t get covered in pus and blood. And they wouldn’t walk too much in the wind and sun, so dust and dirt wouldn’t infect the wound. And they’d take care of the wound and heal it. Then both because they did what was suitable, and the unclean poison had left no residue, the wound would heal, and no death or deadly pain would result. 

In\marginnote{26.1} the same way, it’s possible that a certain mendicant might think: ‘The Ascetic has said that craving is a dart; and that the poison of ignorance is inflicted by desire and ill will. I have given up the dart of craving and expelled the poison of ignorance; I am rightly intent on extinguishment.’ Being rightly intent on extinguishment, they wouldn’t engage in things unconducive to extinguishment: unsuitable sights, sounds, smells, tastes, touches, and ideas. Doing so, lust wouldn’t infect their mind, so no death or deadly pain would result. 

I’ve\marginnote{27.1} made up this simile to make a point. And this is the point: ‘Wound’ is a term for the six interior sense fields. ‘Poison’ is a term for ignorance. ‘Dart’ is a term for craving. ‘Probing’ is a term for mindfulness. ‘Scalpel’ is a term for noble wisdom. ‘Surgeon’ is a term for the Realized One, the perfected one, the fully awakened Buddha. 

Truly,\marginnote{28.1} Sunakkhatta, that mendicant practices restraint regarding the six fields of contact. Understanding that attachment is the root of suffering, they are freed with the ending of attachments. It’s not possible that they would apply their body or interest their mind in any attachment.\footnote{“Attachment” here is \textit{upadhi}, which refers both to the things of the world to which we cling, and the inner clinging. } 

Suppose\marginnote{29.1} there was a bronze goblet of beverage that had a nice color, aroma, and flavor. But it was mixed with poison. Then a person would come along who wants to live and doesn’t want to die, who wants to be happy and recoils from pain. What do you think, Sunakkhatta? Would that person drink that beverage if they knew that it would result in death or deadly suffering?” 

“No,\marginnote{29.7} sir.” 

“In\marginnote{29.8} the same way, Sunakkhatta, that mendicant practices restraint regarding the six fields of contact. Understanding that attachment is the root of suffering, they are freed with the ending of attachments. It’s not possible that they would apply their body or interest their mind in any attachment. 

Suppose\marginnote{30.1} there was a lethal viper. Then a person would come along who wants to live and doesn’t want to die, who wants to be happy and recoils from pain. What do you think, Sunakkhatta? Would that person give that lethal viper their hand or finger if they knew that it would result in death or deadly suffering?” 

“No,\marginnote{30.6} sir.” 

“In\marginnote{30.7} the same way, Sunakkhatta, that mendicant practices restraint regarding the six fields of contact. Understanding that attachment is the root of suffering, they are freed with the ending of attachments. It’s not possible that they would apply their body or interest their mind in any attachment.” 

That\marginnote{30.10} is what the Buddha said. Satisfied, Sunakkhatta of the Licchavi clan approved what the Buddha said. 

%
\section*{{\suttatitleacronym MN 106}{\suttatitletranslation Conducive to the Imperturbable }{\suttatitleroot Āneñjasappāyasutta}}
\addcontentsline{toc}{section}{\tocacronym{MN 106} \toctranslation{Conducive to the Imperturbable } \tocroot{Āneñjasappāyasutta}}
\markboth{Conducive to the Imperturbable }{Āneñjasappāyasutta}
\extramarks{MN 106}{MN 106}

\scevam{So\marginnote{1.1} I have heard.\footnote{This sutta describes practices that lead to attaining the formless meditations: three ways to attain the “imperturbable”, three ways to continue to the dimension of nothingness, and finally a way to attain neither perception nor non-perception. One theme of the text is that deep states of absorption are supported by reflective wisdom. The sutta concludes by teaching what is beyond all these, namely not grasping. } }At one time the Buddha was staying in the land of the Kurus, near the Kuru town named \textsanskrit{Kammāsadamma}. There the Buddha addressed the mendicants, “Mendicants!” 

“Venerable\marginnote{1.5} sir,” they replied. The Buddha said this: 

“Mendicants,\marginnote{2.1} sensual pleasures are impermanent, hollow, false, and deceptive. This is made by illusion, mendicants, lamented by fools.\footnote{This phrase is sometimes translated as if it was joined to the previous. However, the repeated use of \textit{bhikkhave}, the shift from plural to singular, and the independent occurrence of the former portion (\href{https://suttacentral.net/an10.46/en/sujato\#6.5}{AN 10.46:6.5}) all suggest it is a distinct sentence. | As for \textit{\textsanskrit{bālalāpana}}, at \href{https://suttacentral.net/sn22.95/en/sujato\#13.2}{SN 22.95:13.2} we find \textit{\textsanskrit{bālalāpinī}} in a similar context, which there follows right after the death of the physical body. This suggests that \textit{\textsanskrit{lāpinī}} there has the sense “lament” which is attested in Sanskrit. \textit{\textsanskrit{Bālalāpana}} is found at \href{https://suttacentral.net/thig5.2/en/sujato\#2.2}{Thig 5.2:2.2} of a beautiful body; see too \href{https://suttacentral.net/ja421/en/sujato\#5.3}{Ja 421:5.3}. In later Sanskrit, \textit{\textsanskrit{bālalapitaṁ}} is used by \textsanskrit{Jayarāśi} \textsanskrit{Bhaṭṭa} in his \textsanskrit{Tattvopaplavasiṁha} in the sense “prattle of fools”. } Sensual pleasures in this life and in lives to come, sensual perceptions in this life and in lives to come; both of these are \textsanskrit{Māra}’s dominion, \textsanskrit{Māra}’s domain, \textsanskrit{Māra}’s lair, and \textsanskrit{Māra}’s range. They conduce to bad, unskillful qualities such as desire, ill will, and aggression. And they create an obstacle for a noble disciple training here. 

A\marginnote{3.1} noble disciple reflects on this: ‘Sensual pleasures in this life and in lives to come, sensual perceptions in this life and in lives to come; both of these are \textsanskrit{Māra}’s dominion, \textsanskrit{Māra}’s domain, \textsanskrit{Māra}’s lair, and \textsanskrit{Māra}’s range. They conduce to bad, unskillful qualities such as desire, ill will, and aggression. And they create an obstacle for a noble disciple training here. Why don’t I meditate with an abundant, expansive heart, having mastered the world and stabilized the mind? Then I will have no more bad, unskillful qualities such as desire, ill will, and aggression. And by giving them up my mind, no longer limited, will become limitless and well developed.’ 

Practicing\marginnote{3.10} in this way and meditating on it often their mind becomes confident in this dimension. Being confident, they either attain the imperturbable now, or are freed by wisdom.\footnote{The Pali has \textit{\textsanskrit{paññāya} \textsanskrit{vā} adhimuccati} (“they resolve with wisdom”). This appears to be an old confusion between \textit{adhimuccati} (“resolves”) and \textit{vimuccati} (“is freed”). Elsewhere in Pali we find one who is “freed by wisdom” (eg. \href{https://suttacentral.net/sn22.45/en/sujato\#1.6}{SN 22.45:1.6}) but we don’t find \textit{adhimuccati} used in this way. The commentary accepts the reading \textit{adhimuccati}, but explains it as referring to either the realization of arahantship or the development of the path to arahantship, which agrees with the idea of “freedom”. Finally, the Chinese parallel (MA 75 at T i 542b23) has \langlzh{解}, while the Tibetan parallel in Śamathadeva’s commentary on the \textsanskrit{Abhidharmakośabhāṣya} (D (4094) mngon pa, ju 228a2) has \textit{rnam par grol ba}, both of which appear to translate \textit{vimuccati}. If this reading is accepted, it also adds greater coherence to the sutta as a whole, for after discussing the various ways to the imperturbable, the final section is devoted to the “noble liberation”, where \textit{vimokkha} is a synonym of \textit{vimutti} (\href{https://suttacentral.net/mn106/en/sujato\#13.3}{MN 106:13.3}). } When their body breaks up, after death, it’s possible that that conducive consciousness will be reborn in the imperturbable.\footnote{\textit{\textsanskrit{Saṁvattanika}} means “conducive, leading to”. The 
 state of consciousness developed in meditation leads to a rebirth in a plane where the normal level of consciousness is that of the corresponding meditation attainment. Thus if they have developed the fourth absorption, they will be reborn in a corresponding \textsanskrit{Brahmā} realm. } This is said to be the first way of practice suitable for attaining the imperturbable. 

Furthermore,\marginnote{4.1} a noble disciple reflects: ‘Sensual pleasures in this life and in lives to come, sensual perceptions in this life and in lives to come; whatever form there is, all form is the four principal states, or form derived from the four principal states.’ Practicing in this way and meditating on it often their mind becomes confident in this dimension. Being confident, they either attain the imperturbable now, or are freed by wisdom. When their body breaks up, after death, it’s possible that that conducive consciousness will be reborn in the imperturbable. This is said to be the second way of practice suitable for attaining the imperturbable. 

Furthermore,\marginnote{5.1} a noble disciple reflects: ‘Sensual pleasures in this life and in lives to come, sensual perceptions in this life and in lives to come, visions in this life and in lives to come,\footnote{Here \textit{\textsanskrit{rūpa}} is neither “matter” nor “sights” but “visions” seen in meditation. } perceptions of visions in this life and in lives to come; all of these are impermanent. And what’s impermanent is not worth approving, welcoming, or clinging to.’ Practicing in this way and meditating on it often their mind becomes confident in this dimension. Being confident, they either attain the imperturbable now, or are freed by wisdom. When their body breaks up, after death, it’s possible that that conducive consciousness will be reborn in the imperturbable. This is said to be the third way of practice suitable for attaining the imperturbable. 

Furthermore,\marginnote{6.1} a noble disciple reflects: ‘Sensual pleasures in this life and in lives to come, sensual perceptions in this life and in lives to come, visions in this life and in lives to come, perceptions of visions in this life and in lives to come, and perceptions of the imperturbable; all are perceptions. Where they cease without anything left over, that is peaceful, that is sublime, namely the dimension of nothingness.’\footnote{As in \href{https://suttacentral.net/mn105/en/sujato\#12.3}{MN 105:12.3}, the “imperturbable” here appears to refer to the fourth absorption and first two formless attainments. } Practicing in this way and meditating on it often their mind becomes confident in this dimension. Being confident, they either attain the dimension of nothingness now, or are freed by wisdom. When their body breaks up, after death, it’s possible that that conducive consciousness will be reborn in the dimension of nothingness. This is said to be the first way of practice suitable for attaining the dimension of nothingness. 

Furthermore,\marginnote{7.1} a noble disciple has gone to a wilderness, or to the root of a tree, or to an empty hut, and reflects like this: ‘This is empty of a self or what belongs to a self.’ Practicing in this way and meditating on it often their mind becomes confident in this dimension. Being confident, they either attain the dimension of nothingness now, or are freed by wisdom. When their body breaks up, after death, it’s possible that that conducive consciousness will be reborn in the dimension of nothingness. This is said to be the second way of practice suitable for attaining the dimension of nothingness. 

Furthermore,\marginnote{8.1} a noble disciple reflects: ‘I don’t belong to anyone anywhere! And nothing belongs to me anywhere!’\footnote{This reflection occurs three times in the Pali. At \href{https://suttacentral.net/an4.185/en/sujato\#7.2}{AN 4.185:7.2} it leads to the dimension of nothingness for a brahmin. It leads to the same attainment in a Buddhist context here (compare \href{https://suttacentral.net/an5.144/en/sujato\#7.2}{AN 5.144:7.2}). At \href{https://suttacentral.net/an3.70/en/sujato\#3.11}{AN 3.70:3.11} it is said to be a Jain practice. } Practicing in this way and meditating on it often their mind becomes confident in this dimension. Being confident, they either attain the dimension of nothingness now, or are freed by wisdom. When their body breaks up, after death, it’s possible that that conducive consciousness will be reborn in the dimension of nothingness. This is said to be the third way of practice suitable for attaining the dimension of nothingness. 

Furthermore,\marginnote{9.1} a noble disciple reflects: ‘Sensual pleasures in this life and in lives to come, sensual perceptions in this life and in lives to come, visions in this life and in lives to come, perceptions of visions in this life and in lives to come, perceptions of the imperturbable, and perceptions of the dimension of nothingness; all are perceptions. Where they cease without anything left over, that is peaceful, that is sublime, namely the dimension of neither perception nor non-perception.’ Practicing in this way and meditating on it often their mind becomes confident in this dimension. Being confident, they either attain the dimension of neither perception nor non-perception now, or are freed by wisdom. When their body breaks up, after death, it’s possible that that conducive consciousness will be reborn in the dimension of neither perception nor non-perception. This is said to be the way of practice suitable for attaining the dimension of neither perception nor non-perception.” 

When\marginnote{10.1} he said this, Venerable Ānanda said to the Buddha: “Sir, take a mendicant who practices like this: ‘It might not be, and it might not be mine. It will not be, and it will not be mine. I am giving up what exists, what has come to be.’\footnote{See Bodhi’s extensive discussion in his translation of \href{https://suttacentral.net/sn22.55/en/sujato\#1.3}{SN 22.55:1.3} (\emph{Connected Discourses}, vol. iii, note 75). This formula appears several times throughout the suttas, in two forms: the personal form used by non-Buddhists  (“I might not be …”, eg. \href{https://suttacentral.net/sn22.81/en/sujato\#12.8}{SN 22.81:12.8}, \href{https://suttacentral.net/an10.29/en/sujato\#19.1}{AN 10.29:19.1}), and the impersonal form as adapted by the Buddha here (“It might not be …”). The formula seems to have a shared among different sectarian groups, like the similar reflection above (\href{https://suttacentral.net/mn106/en/sujato\#8.2}{MN 106:8.2}. But whereas that reflection is associated with Brahmans and Jains—both classified as “eternalists”—this view is said to be that of the “annihilationists” (\href{https://suttacentral.net/sn22.81/en/sujato\#11.10}{SN 22.81:11.10}). } In this way they gain equanimity. Would that mendicant become extinguished or not?” 

“One\marginnote{10.6} such mendicant might become extinguished, Ānanda, while another might not.” 

“What\marginnote{10.7} is the cause, sir, what is the reason for this?” 

“Ānanda,\marginnote{10.8} take a mendicant who practices like this: ‘It might not be, and it might not be mine. It will not be, and it will not be mine. I am giving up what exists, what has come to be.’ In this way they gain equanimity. They approve, welcome, and keep clinging to that equanimity. Their consciousness has that as support and fuel for grasping. A mendicant with fuel for grasping does not become extinguished.” 

“But\marginnote{11.1} sir, what is that mendicant grasping?” 

“The\marginnote{11.2} dimension of neither perception nor non-perception.” 

“Sir,\marginnote{11.3} it seems that mendicant is grasping the best thing to grasp!” 

“Indeed,\marginnote{11.4} Ānanda. For the best thing to grasp is the dimension of neither perception nor non-perception. 

Take\marginnote{12.1} a mendicant who practices like this: ‘It might not be, and it might not be mine. It will not be, and it will not be mine. I am giving up what exists, what has come to be.’ In this way they gain equanimity. They don’t approve, welcome, or keep clinging to that equanimity. So their consciousness doesn’t have that as support and fuel for grasping. A mendicant free of grasping becomes extinguished.” 

“It’s\marginnote{13.1} incredible, sir, it’s amazing! It seems the Buddha has explained to us how to cross over the flood by relying on one support or another.\footnote{This passage shows how the method of attaining such states is secondary, as one might practice it well and yet still have clinging. Any of these “supports”, i.e. the various methods taught above, suffices to develop the mind, but what matters is letting go. | “One support or another” (\textit{\textsanskrit{nissāya} \textsanskrit{nissāya}}) appears at \href{https://suttacentral.net/dn16/en/sujato\#4.27.2}{DN 16:4.27.2}, where it means “right beside”. For the sense “support” in this context, compare \href{https://suttacentral.net/iti107/en/sujato\#2.3}{Iti 107:2.3}: “supporting each other … in order to cross over the flood” (\textit{\textsanskrit{aññamaññaṁ} \textsanskrit{nissāya} …  oghassa \textsanskrit{nittharaṇatthāya}}). } But sir, what is noble liberation?” 

“Ānanda,\marginnote{13.4} it’s when a noble disciple reflects like this: ‘Sensual pleasures in this life and in lives to come, sensual perceptions in this life and in lives to come, visions in this life and in lives to come, perceptions of visions in this life and in lives to come, perceptions of the imperturbable, perceptions of the dimension of nothingness, perceptions of the dimension of neither perception nor non-perception; that is substantial reality as far as substantial reality extends. But this is freedom from death, namely the liberation of the mind through not grasping. 

So,\marginnote{14.1} Ānanda, I have taught the ways of practice suitable for attaining the imperturbable, the dimension of nothingness, and the dimension of neither perception nor non-perception. I have taught how to cross the flood by relying on one support or another, and I have taught noble liberation. 

Out\marginnote{15.1} of sympathy, I’ve done what a teacher should do who wants what’s best for their disciples. Here are these roots of trees, and here are these empty huts. Practice absorption, Ānanda! Don’t be negligent! Don’t regret it later! This is my instruction to you.” 

That\marginnote{15.3} is what the Buddha said. Satisfied, Venerable Ānanda approved what the Buddha said. 

%
\section*{{\suttatitleacronym MN 107}{\suttatitletranslation With Moggallāna the Accountant }{\suttatitleroot Gaṇakamoggallānasutta}}
\addcontentsline{toc}{section}{\tocacronym{MN 107} \toctranslation{With Moggallāna the Accountant } \tocroot{Gaṇakamoggallānasutta}}
\markboth{With Moggallāna the Accountant }{Gaṇakamoggallānasutta}
\extramarks{MN 107}{MN 107}

\scevam{So\marginnote{1.1} I have heard. }At one time the Buddha was staying near \textsanskrit{Sāvatthī} in the stilt longhouse of \textsanskrit{Migāra}’s mother in the Eastern Monastery. Then the brahmin \textsanskrit{Moggallāna} the Accountant went up to the Buddha, and exchanged greetings with him. When the greetings and polite conversation were over, he sat down to one side and said to the Buddha: 

“Mister\marginnote{2.1} Gotama, in this stilt longhouse we can see gradual progress down to the last step of the staircase.\footnote{The Buddha stops on the last step at \href{https://suttacentral.net/mn85/en/sujato\#7.4}{MN 85:7.4}. } Among the brahmins we can see gradual progress in recitation.\footnote{Memorizing the Vedic texts, a key skill of the brahmins, was so difficult that they sometimes asked the Buddha for advice (\href{https://suttacentral.net/an5.193/en/sujato\#8.4}{AN 5.193:8.4}). Details on the gradual memorization of texts are found at \href{https://suttacentral.net/pli-tv-bu-vb-pc4/en/sujato\#2.1.7}{Bu Pc 4:2.1.7}. Texts were learned by line (teacher and student start and finish together), by going after the line (one starts, they finish together), by going after the syllable (the teacher prompts with the first syllable of the line), and by phrase (the teacher says the first phrase, the student the second). } Among archers we can see gradual progress in archery.\footnote{Archery (\textit{issattha}) is regularly listed as a craft or livelihood (\href{https://suttacentral.net/mn14/en/sujato\#7.3}{MN 14:7.3}), which took skill in training (\href{https://suttacentral.net/sn56.45/en/sujato\#1.3}{SN 56.45:1.3}), and to which a mendicant is compared (\href{https://suttacentral.net/an4.196/en/sujato\#10.3}{AN 4.196:10.3}). } Among us accountants, who earn a living by accounting, we can see gradual progress in mathematics.\footnote{The complexities of accounting are detailed in \textsanskrit{Kauṭilya}’s \textsanskrit{Arthaśāstra} 2.7. There, the “accounts” are called \textit{\textsanskrit{gāṇanikya}} (2.7.16). } For when we get an apprentice we first make them count: ‘One one, two twos, three threes, four fours, five fives, six sixes, seven sevens, eight eights, nine nines, ten tens.’\footnote{The method of listing things up to tens is the framework of the Dasuttarasutta (\href{https://suttacentral.net/dn34/en/sujato}{DN 34}). More generally, it seems to underlie the “\textit{\textsanskrit{aṅguttara}} principle” of organizing teachings by number. } We even make them count up to a hundred. Is it possible to similarly describe a gradual training, gradual progress, and gradual practice in this teaching and training?” 

“It\marginnote{3.1} is possible, brahmin. Suppose a deft horse trainer were to obtain a fine thoroughbred. First of all he’d make it get used to wearing the bit.\footnote{See \href{https://suttacentral.net/mn65/en/sujato\#33.1}{MN 65:33.1}. } In the same way, when the Realized One gets a person for training they first guide them like this: ‘Come, mendicant, be ethical and restrained in the monastic code, conducting yourself well and resorting for alms in suitable places. Seeing danger in the slightest fault, keep the rules you’ve undertaken.’ 

When\marginnote{4.1} they have ethical conduct, the Realized One guides them further: ‘Come, mendicant, guard your sense doors. When you see a sight with your eyes, don’t get caught up in the features and details. If the faculty of sight were left unrestrained, bad unskillful qualities of covetousness and displeasure would become overwhelming. For this reason, practice restraint, protect the faculty of sight, and achieve restraint over it. When you hear a sound with your ears … When you smell an odor with your nose … When you taste a flavor with your tongue … When you feel a touch with your body … When you know an idea with your mind, don’t get caught up in the features and details. If the faculty of mind were left unrestrained, bad unskillful qualities of covetousness and displeasure would become overwhelming. For this reason, practice restraint, protect the faculty of mind, and achieve its restraint.’ 

When\marginnote{5.1} they guard their sense doors, the Realized One guides them further: ‘Come, mendicant, eat in moderation. Reflect rationally on the food that you eat: ‘Not for fun, indulgence, adornment, or decoration, but only to sustain this body, to avoid harm, and to support spiritual practice. In this way, I shall put an end to old discomfort and not give rise to new discomfort, and I will have the means to keep going, blamelessness, and a comfortable abiding.’ 

When\marginnote{6.1} they eat in moderation, the Realized One guides them further: ‘Come, mendicant, be committed to wakefulness. Practice walking and sitting meditation by day, purifying your mind from obstacles. In the first watch of the night, continue to practice walking and sitting meditation. In the middle watch, lie down in the lion’s posture—on the right side, placing one foot on top of the other—mindful and aware, and focused on the time of getting up. In the last watch, get up and continue to practice walking and sitting meditation, purifying your mind from obstacles.’ 

When\marginnote{7.1} they are committed to wakefulness, the Realized One guides them further: ‘Come, mendicant, have mindfulness and situational awareness. Act with situational awareness when going out and coming back; when looking ahead and aside; when bending and extending the limbs; when bearing the outer robe, bowl and robes; when eating, drinking, chewing, and tasting; when urinating and defecating; when walking, standing, sitting, sleeping, waking, speaking, and keeping silent.’ 

When\marginnote{8.1} they have mindfulness and situational awareness, the Realized One guides them further: ‘Come, mendicant, frequent a secluded lodging—a wilderness, the root of a tree, a hill, a ravine, a mountain cave, a charnel ground, a forest, the open air, a heap of straw.’ And they do so. 

After\marginnote{9.2} the meal, they return from almsround, sit down cross-legged, set their body straight, and establish mindfulness in their presence. Giving up covetousness for the world, they meditate with a heart rid of covetousness, cleansing the mind of covetousness. Giving up ill will and malevolence, they meditate with a mind rid of ill will, full of sympathy for all living beings, cleansing the mind of ill will. Giving up dullness and drowsiness, they meditate with a mind rid of dullness and drowsiness, perceiving light, mindful and aware, cleansing the mind of dullness and drowsiness. Giving up restlessness and remorse, they meditate without restlessness, their mind peaceful inside, cleansing the mind of restlessness and remorse. Giving up doubt, they meditate having gone beyond doubt, not undecided about skillful qualities, cleansing the mind of doubt. 

They\marginnote{10.1} give up these five hindrances, corruptions of the heart that weaken wisdom. Then, quite secluded from sensual pleasures, secluded from unskillful qualities, they enter and remain in the first absorption, which has the rapture and bliss born of seclusion, while placing the mind and keeping it connected. As the placing of the mind and keeping it connected are stilled, they enter and remain in the second absorption, which has the rapture and bliss born of immersion, with internal clarity and mind at one, without placing the mind and keeping it connected. And with the fading away of rapture, they enter and remain in the third absorption, where they meditate with equanimity, mindful and aware, personally experiencing the bliss of which the noble ones declare, ‘Equanimous and mindful, one meditates in bliss.’ Giving up pleasure and pain, and ending former happiness and sadness, they enter and remain in the fourth absorption, without pleasure or pain, with pure equanimity and mindfulness. 

That’s\marginnote{11.1} how I instruct the mendicants who are trainees—who haven’t achieved their heart’s desire, but live aspiring to the supreme sanctuary from the yoke.\footnote{The “trainee” (\textit{sekha}) in the strict sense is restricted to those who have entered the ranks of the Noble Ones (\textit{ariyapuggala}) through the realization of the four noble truths (eg. \href{https://suttacentral.net/sn48.53/en/sujato\#3.1}{SN 48.53:3.1}); that is to say, the seven Noble Ones excluding the arahant, who is an “adept” beyond training (\textit{asekha}). However, in cases such as this it would be over-strict to insist that the teaching applies only to Noble Ones, as the Gradual Training is the recommended practice for all new monastics. Indeed, the burden of the text is to show how practice is taken up gradually, so above it says that this is how the Buddha “first guides them” (\textit{\textsanskrit{paṭhamaṁ} \textsanskrit{evaṁ} vineti}, \href{https://suttacentral.net/mn107/en/sujato\#3.3}{MN 107:3.3}). This agrees with the Chinese parallel, which here mentions a “young monk” (MA 144 at T i 652a29). Late canonical Pali texts introduce the idea of the “ordinary person of good character” (\textit{\textsanskrit{puthujjanakalyāṇaka}}, eg. \href{https://suttacentral.net/cnd8/en/sujato\#81.2}{Cnd 8:81.2}, \href{https://suttacentral.net/ps1.1/en/sujato\#206.2}{Ps 1.1:206.2}), who is said to “train” like the trainees (\href{https://suttacentral.net/pli-tv-pvr1.1/en/sujato\#3.46}{Pvr 1.1:3.46}). The commentaries say they may be counted along with the seven Noble Ones as a “trainee” (\textit{sekkhoti \textsanskrit{puthujjanakalyāṇakena} \textsanskrit{saddhiṁ} satta \textsanskrit{ariyā}}, commentaries to \textsanskrit{Pārājika} 1 and \textsanskrit{Jhānavibhaṅga}). } But for those mendicants who are perfected—who have ended the defilements, completed the spiritual journey, done what had to be done, laid down the burden, achieved their own goal, utterly ended the fetter of continued existence, and are rightly freed through enlightenment—these things lead to blissful meditation in this life, and to mindfulness and awareness.” 

When\marginnote{12.1} he had spoken, \textsanskrit{Moggallāna} the Accountant said to the Buddha, “When his disciples are instructed and advised like this by Mister Gotama, do all of them achieve the ultimate goal, extinguishment, or do some of them fail?” 

“Some\marginnote{12.3} succeed, while others fail.” 

“What\marginnote{13.1} is the cause, Mister Gotama, what is the reason why, though extinguishment is present, the path leading to extinguishment is present, and Mister Gotama is present to encourage them, still some succeed while others fail?” 

“Well\marginnote{14.1} then, brahmin, I’ll ask you about this in return, and you can answer as you like. What do you think, brahmin? Are you skilled in the road to \textsanskrit{Rājagaha}?”\footnote{\textsanskrit{Chāndogya} \textsanskrit{Upaniṣad} 6.14.1–2 illustrates the role of a teacher with the story of a man kidnapped, blindfolded, and abandoned in a wilderness. A kind person looses his bandage and shows him the way to \textsanskrit{Gandhāra}. See too Śatapatha \textsanskrit{Brāhmaṇa} 13.2.3–2, where the horse knows the way to heaven that humans do not, like one who knows the country. } 

“Yes,\marginnote{14.4} I am.” 

“What\marginnote{14.5} do you think, brahmin? Suppose a person was to come along who wanted to go to \textsanskrit{Rājagaha}. He’d approach you and say: ‘Sir, I wish to go to \textsanskrit{Rājagaha}. Please point out the road to \textsanskrit{Rājagaha}.’ You’d say to them: ‘Here, mister, this road goes to \textsanskrit{Rājagaha}. Go along it for an hour, and you’ll see a certain village. Go along an hour further, and you’ll see a certain town. Go along an hour further and you’ll see \textsanskrit{Rājagaha} with its delightful parks, woods, meadows, and lotus ponds.’ Instructed like this by you, they might still take the wrong road, heading west.\footnote{\textit{\textsanskrit{Pacchāmukha}} means “heading west”. The present sutta is set in \textsanskrit{Sāvatthī}, so for someone wanting to get to \textsanskrit{Rājagaha} in the south-east, west is the wrong way. The word also appears at \href{https://suttacentral.net/thag10.1/en/sujato\#3.4}{Thag 10.1:3.4} of the Buddha crossing the \textsanskrit{Rohiṇī} river, which runs north to south; the background story says he was coming from \textsanskrit{Rājagaha}, which again implies he was heading west from Koliya to Sakya. } But a second person might come with the same question and receive the same instructions. Instructed by you, they might safely arrive at \textsanskrit{Rājagaha}. What is the cause, brahmin, what is the reason why, though \textsanskrit{Rājagaha} is present, the path leading to \textsanskrit{Rājagaha} is present, and you are there to encourage them, one person takes the wrong path and heads west, while another arrives safely at \textsanskrit{Rājagaha}?” 

“What\marginnote{14.27} can I do about that, Mister Gotama? I am the one who shows the way.”\footnote{The one who “shows (or explains) the way” is also discussed at \href{https://suttacentral.net/snp1.5/en/sujato\#3.3}{Snp 1.5:3.3}, along with other good and bad ascetics. See too \href{https://suttacentral.net/dhp276/en/sujato\#2}{Dhp 276:2}, “the Realized Ones show the way” (\textit{\textsanskrit{akkhātāro} \textsanskrit{tathāgatā}}). } 

“In\marginnote{14.29} the same way, though extinguishment is present, the path leading to extinguishment is present, and I am present to encourage them, still some of my disciples, instructed and advised like this, achieve the ultimate goal, extinguishment, while some of them fail. What can I do about that, brahmin? The Realized One is the one who shows the way.” 

When\marginnote{15.1} he had spoken, \textsanskrit{Moggallāna} the Accountant said to the Buddha, “Mister Gotama, there are those faithless people who went forth from the lay life to homelessness not out of faith but to earn a livelihood. They’re devious, deceitful, and sneaky. They’re restless, insolent, fickle, scurrilous, and loose-tongued. They do not guard their sense doors or eat in moderation, and they are not committed to wakefulness. They don’t care about the ascetic life, and don’t keenly respect the training. They’re indulgent and slack, leaders in backsliding, neglecting seclusion, lazy, and lacking energy. They’re unmindful, lacking situational awareness and immersion, with straying minds, witless and idiotic. Mister Gotama does not live together with these.\footnote{Also at \href{https://suttacentral.net/mn5/en/sujato\#32.1}{MN 5:32.1}. } 

But\marginnote{15.3} there are those gentlemen who went forth from the lay life to homelessness out of faith. They’re not devious, deceitful, and sneaky. They’re not restless, insolent, fickle, scurrilous, and loose-tongued. They guard their sense doors and eat in moderation, and they are committed to wakefulness. They care about the ascetic life, and keenly respect the training. They’re not indulgent or slack, nor are they leaders in backsliding, neglecting seclusion. They’re energetic and determined. They’re mindful, with situational awareness, immersion, and unified minds; wise and clever. Mister Gotama does live together with these. 

Of\marginnote{16.1} all kinds of fragrant root, spikenard is said to be the best. Of all kinds of fragrant heartwood, red sandalwood is said to be the best. Of all kinds of fragrant flower, jasmine is said to be the best. In the same way, Mister Gotama’s advice is the best of contemporary teachings. 

Excellent,\marginnote{17.1} Mister Gotama! Excellent! As if he were righting the overturned, or revealing the hidden, or pointing out the path to the lost, or lighting a lamp in the dark so people with clear eyes can see what’s there, Mister Gotama has made the Teaching clear in many ways. I go for refuge to Mister Gotama, to the teaching, and to the mendicant \textsanskrit{Saṅgha}. From this day forth, may Mister Gotama remember me as a lay follower who has gone for refuge for life.” 

%
\section*{{\suttatitleacronym MN 108}{\suttatitletranslation With Moggallāna the Guardian }{\suttatitleroot Gopakamoggallānasutta}}
\addcontentsline{toc}{section}{\tocacronym{MN 108} \toctranslation{With Moggallāna the Guardian } \tocroot{Gopakamoggallānasutta}}
\markboth{With Moggallāna the Guardian }{Gopakamoggallānasutta}
\extramarks{MN 108}{MN 108}

\scevam{So\marginnote{1.1} I have heard. }At one time Venerable Ānanda was staying near \textsanskrit{Rājagaha}, in the Bamboo Grove, the squirrels’ feeding ground. It was not long after the Buddha had become fully quenched. 

Now\marginnote{2.1} at that time King \textsanskrit{Ajātasattu} of Magadha, son of the princess of Videha, being suspicious of King Pajjota, was having \textsanskrit{Rājagaha} fortified.\footnote{As to the politics of the situation, Pajjota had established a formidable set of alliances by marriage. He had taken to wife \textsanskrit{Śivā}, daughter of \textsanskrit{Ceṭaka} of the \textsanskrit{Licchavīs}. Prince Bodhi of the Bhaggas was Pajjota’s grandson via his mother \textsanskrit{Vāsuladattā}, queen of the powerful King Udena of \textsanskrit{Kosambī} (\href{https://suttacentral.net/mn85/en/sujato\#2.1}{MN 85:2.1}). Meanwhile, \textsanskrit{Subāhu} of \textsanskrit{Madhurā} had married Pajjota’s sister (\href{https://suttacentral.net/mn84/en/sujato\#2.1}{MN 84:2.1}). Pajjota therefore had allies in Avanti, Bhagga, Vaccha, and \textsanskrit{Sūrasena}, all nations to the west of Magadha, as well as \textsanskrit{Vajjī}, against which Magadha was preparing war to the north. He was also friendly enough with \textsanskrit{Ajātasattu}’s father \textsanskrit{Bimbisāra} to ask him for help in a time of illness (\href{https://suttacentral.net/pli-tv-kd8/en/sujato\#1.23.4}{Kd 8:1.23.4}). The Vinaya relates how Pajjota’s illness was treated by \textsanskrit{Jīvaka}, the doctor of the Buddha and of \textsanskrit{Bimbisāra}. \textsanskrit{Jīvaka}, knowing Pajjota would dislike the medicine, escaped his wrath by fleeing after administering the cure. But when the cure worked, Pajjota favored \textsanskrit{Jīvaka} with a gift of fine cloth. \textsanskrit{Ajātasattu} subsequently murdered \textsanskrit{Bimbisāra} and threatened Pajjota’s allies in \textsanskrit{Vajjī}. Racked by guilt, \textsanskrit{Ajātasattu} saw enemies on all sides; he was even afraid of mendicants meditating in a hall (\href{https://suttacentral.net/dn2/en/sujato\#10.8}{DN 2:10.8}). His fear of Pajjota was not idle, for Pajjota was known for the power of his elephants as much as for his temperamental character. However, we do not hear that any direct conflict flared up at that time. Roughly half a century later, the matter was settled when the \textsanskrit{Licchavī} \textsanskrit{Śiśunāga}, having usurped the throne of Magadha from the \textsanskrit{Haryaṅka} dynasty founded by \textsanskrit{Bimbisāra}, conquered Avanti, Kosala, and Vaccha. } 

Then\marginnote{3.1} Venerable Ānanda robed up in the morning and, taking his bowl and robe, entered \textsanskrit{Rājagaha} for alms. 

Then\marginnote{3.2} it occurred to him, “It’s too early to wander for alms in \textsanskrit{Rājagaha}. Why don’t I go to see the brahmin \textsanskrit{Moggallāna} the Guardian at his place of work?” 

So\marginnote{4.1} that’s what he did. \textsanskrit{Moggallāna} the Guardian saw Ānanda coming off in the distance and said to him, “Come, Mister Ānanda! Welcome, Mister Ānanda! It’s been a long time since you took the opportunity to come here. Please, sir, sit down, this seat is ready.” 

Ānanda\marginnote{4.8} sat down on the seat spread out, while \textsanskrit{Moggallāna} took a low seat and sat to one side. Then he said to Ānanda, “Mister Ānanda, is there even a single mendicant who has all the same qualities in each and every way as possessed by Mister Gotama, the perfected one, the fully awakened Buddha?” 

“No,\marginnote{5.2} brahmin, there is not. For the Blessed One gave rise to the unarisen path, gave birth to the unborn path, and explained the unexplained path. He is the knower of the path, the discoverer of the path, the expert on the path.\footnote{This links back to the Buddha’s claim in the previous sutta that he is the one who shows the way (\href{https://suttacentral.net/mn107/en/sujato\#14.28}{MN 107:14.28}). } And now the disciples live following the path; they acquire it later.” 

But\marginnote{6.1} this conversation between Ānanda and \textsanskrit{Moggallāna} the Guardian was left unfinished. 

For\marginnote{6.2} just then the brahmin \textsanskrit{Vassakāra}, a minister of Magadha, while supervising the work at \textsanskrit{Rājagaha}, approached Ānanda at \textsanskrit{Moggallāna}’s place of work and exchanged greetings with him.\footnote{\textsanskrit{Vassakāra} must have returned from building \textsanskrit{Pāṭaliputta} to the north on the Ganges, which would eventually become the capital of the expanded Magadhan empire (\href{https://suttacentral.net/dn16/en/sujato\#1.26.1}{DN 16:1.26.1}). | \textsanskrit{Rājagaha} is surrounded by hills, on which remnants of ancient fortifications can be seen. } When the greetings and polite conversation were over, he sat down to one side and said to Ānanda, “Mister Ānanda, what were you sitting talking about just now? What conversation was left unfinished?” 

So\marginnote{6.5} Ānanda told him of the conversation that they were having when \textsanskrit{Vassakāra} arrived. \textsanskrit{Vassakāra} said: 

“Mister\marginnote{7.1} Ānanda, is there even a single mendicant who was appointed by Mister Gotama, saying: ‘This one will be your refuge when I have passed away,’ to whom you would now turn?”\footnote{Accept reading \textit{\textsanskrit{paṭidhāveyyāthāti}} (BJT, PTS, Siamese and Khmer editions) against \textsanskrit{Mahāsaṅgīti}’s \textit{\textsanskrit{paṭipādeyyāthāti}}. For this term, see \href{https://suttacentral.net/mn38/en/sujato\#23.4}{MN 38:23.4} and note there, and \href{https://suttacentral.net/sn12.20/en/sujato\#5.1}{SN 12.20:5.1}. } 

“No,\marginnote{7.3} there is not.”\footnote{The Buddha made this clear to Ānanda shortly before his death (\href{https://suttacentral.net/dn16/en/sujato\#2.24.4}{DN 16:2.24.4}). } 

“But\marginnote{8.1} is there even a single mendicant who has been elected to such a position by the \textsanskrit{Saṅgha} and appointed by several senior mendicants?”\footnote{The Vinaya sets up procedures whereby the \textsanskrit{Saṅgha} can appoint officials to carry out necessary duties. For example, a monk might be in charge of assigning dwellings (\href{https://suttacentral.net/pli-tv-bu-vb-ss8/en/sujato\#1.1.1}{Bu Ss 8:1.1.1}) or teaching the novices (\href{https://suttacentral.net/pli-tv-kd16/en/sujato\#21.3.30}{Kd 16:21.3.30}). Such officials have authority within the roles prescribed to them under the Vinaya as implemented by the \textsanskrit{Saṅgha}. However, they have no special authority outside of that role, and there is no provision for a general role of authority over the \textsanskrit{Saṅgha} as a whole. The \textsanskrit{Saṅgha} followed this precedent at the First Council, where Ānanda was appointed to recite the suttas, and \textsanskrit{Upāli} to recite the Vinaya. \textsanskrit{Mahākassapa} led the proceedings, but decisions were made by the \textsanskrit{Saṅgha} in consensus. } 

“No,\marginnote{8.3} there is not.” 

“But\marginnote{9.1} since you lack a refuge, Mister Ānanda, what’s the reason for your harmony?”\footnote{The Buddha had, not long before, taught \textsanskrit{Vassakāra} the eight principles of non-decline by which the Vajjis—as well as the \textsanskrit{Saṅgha}—remained strong due to harmony. There, \textsanskrit{Vassakāra} pointedly said that the Vajjis could only be defeated “by bribery or by sowing dissension” (\href{https://suttacentral.net/dn16/en/sujato\#1.5.8}{DN 16:1.5.8}). Here he returns to the question of how harmony creates strength. We can imagine that this question was equally relevant in both weakening their enemies as well as strengthening the Magadhans, who cannot have been happy to live under their brutal and patricidal king. } 

“We\marginnote{9.2} don’t lack a refuge, brahmin, we have a refuge. The teaching is our refuge.”\footnote{As explained to Ānanda at \href{https://suttacentral.net/dn16/en/sujato\#4.7.3}{DN 16:4.7.3}. As will become clear, the “teaching” here includes the Dhamma and Vinaya. } 

“But\marginnote{10.1} Mister Ānanda, when asked whether there was even a single mendicant—either appointed by the Buddha, or elected by the \textsanskrit{Saṅgha} and appointed by several senior mendicants—who serves as your refuge after the Buddha passed away, to whom you now turn, you replied, ‘No, there is not.’ But you say that the reason for your harmony is that you have the teaching as a refuge. How should I see the meaning of this statement?” 

“The\marginnote{10.14} Blessed One, who knows and sees, the perfected one, the fully awakened Buddha laid down training rules and recited the monastic code for the mendicants.\footnote{This is the \textit{\textsanskrit{pātimokkha}}, which in early days had “over a hundred and fifty rules” and today, in the Pali, has 227 for monks (\href{https://suttacentral.net/an3.84/en/sujato\#1.3}{AN 3.84:1.3}). The story of how the Buddha came to lay down the first rule is related in the Vinaya (\href{https://suttacentral.net/pli-tv-bu-vb-pj1/en/sujato}{Bu Pj 1}). Before this, however, the joint recitation consisted of the verses known as \textsanskrit{Ovāda} \textsanskrit{Pātimokkha} (\href{https://suttacentral.net/dn14/en/sujato\#3.28.1}{DN 14:3.28.1}). See too \href{https://suttacentral.net/dn2/en/sujato\#42.1}{DN 2:42.1} and note there. The primary purpose of the recitation was to create a unified community through shared allegiance to a particular code of ethics. } On the day of the sabbath all of us who live in dependence on one village district gather together as one.\footnote{The “sabbath” (\textit{uposatha}) falls on the eighth and fourteenth or fifteenth days of the lunar cycle, that is, roughly each week. However the \textit{\textsanskrit{pātimokkha}} is recited only twice a month, leaving aside the eighth days. See the detailed account in the Vinaya (\href{https://suttacentral.net/pli-tv-kd2/en/sujato}{Kd 2}). | A “village district” (\textit{\textsanskrit{gāmakhetta}}) is the village and surrounding farmlands. The Vinaya account specifies how to create a “boundary” (\textit{\textsanskrit{sīmā}}) within which all resident mendicants must attend. This may or may not correspond with the village district. But it seems reasonable to suppose that the common practice was, in fact, for mendicants from the same village district to join together. } We invite one who transmits the code to recite it.\footnote{The choice of the verb \textit{pavattati} (“transmits, rolls forth”) is deliberate. It echoes the first sermon, where the Buddha “rolled forth” the wheel of the Dhamma. All mendicants are supposed to memorize the rules and be ready to recite them when called upon. The commentary explains the phrase as “arrives having rehearsed”. } If anyone remembers an offense or transgression while they’re reciting, we make them act in line with the teachings and in line with the instructions.\footnote{Offences should be confessed as soon as practicable, but if they have not been previously confessed due to lapse of memory and the like, the recitation may be interrupted to confess them. These days, mendicants normally confess before the recitation to avoid interruptions. } It’s not the good sirs who make us act, it’s the teaching that makes us act.”\footnote{The \textsanskrit{Saṅgha} proceeds according to the guidelines of the Dhamma and Vinaya, rather than the authority of any individuals. This is again in accordance with some of the Buddha’s final words to Ānanda: “The teaching and training that I have taught and pointed out for you shall be your Teacher after my passing” (\href{https://suttacentral.net/dn16/en/sujato\#6.1.1}{DN 16:6.1.1}). } 

“Mister\marginnote{11.1} Ānanda, is there even a single mendicant who you honor, respect, revere, venerate, and rely on?” 

“There\marginnote{11.2} is not, brahmin.”\footnote{The \textsanskrit{Mahāsaṅgīti} and BJT editions read \textit{natthi} here (“there is not”). The PTS edition reads \textit{atthi} (“there is”). The commentary is silent, and none of these editions record any variants. I suspect the PTS edition has amended the text to create a contrast between the “single mendicant” who is respected and the one who is taken as refuge. But the contrast per the MS and BJT editions, rather, is between respecting a “single mendicant”, and respecting “whoever” has the qualities that make them worthy of respect. } 

“Mister\marginnote{11.3} Ānanda, when asked whether there was even a single mendicant—either appointed by the Buddha, or elected by the \textsanskrit{Saṅgha} and appointed by several senior mendicants—who serves as your refuge after the Buddha passed away, to whom you now turn, you replied, ‘No, there is not.’ And when asked whether there is even a single mendicant who you honor, respect, revere, venerate, and rely on, you replied, ‘There is not.’ How should I see the meaning of this statement?” 

“There\marginnote{13.1} are ten inspiring things explained by the Blessed One, who knows and sees, the perfected one, the fully awakened Buddha.\footnote{The ten things are ethics, learning, contentment, absorption, and the six direct knowledges. While all of these are taught often in the suttas, I do not think we find this exact set elsewhere. } We honor anyone in whom these things are found. What ten? 

It’s\marginnote{14.1} when a mendicant is ethical, restrained in the monastic code, conducting themselves well and resorting for alms in suitable places. Seeing danger in the slightest fault, they keep the rules they’ve undertaken. 

They’re\marginnote{15.1} very learned, remembering and keeping what they’ve learned. These teachings are good in the beginning, good in the middle, and good in the end, meaningful and well-phrased, describing a spiritual practice that’s entirely full and pure. They are very learned in such teachings, remembering them, rehearsing them, mentally scrutinizing them, and comprehending them theoretically. 

They’re\marginnote{16.1} content with robes, almsfood, lodgings, and medicines and supplies for the sick. 

They\marginnote{17.1} get the four absorptions—blissful meditations in this life that belong to the higher mind—when they want, without trouble or difficulty. 

They\marginnote{18.1} wield the many kinds of psychic power: multiplying themselves and becoming one again; appearing and disappearing; going unobstructed through a wall, a rampart, or a mountain as if through space; diving in and out of the earth as if it were water; walking on water as if it were earth; flying cross-legged through the sky like a bird; touching and stroking with the hand the sun and moon, so mighty and powerful. They control the body as far as the realm of divinity. 

With\marginnote{19.1} clairaudience that is purified and superhuman, they hear both kinds of sounds, human and heavenly, whether near or far. 

They\marginnote{20.1} understand the minds of other beings and individuals, having comprehended them with their own mind. They understand mind with greed as ‘mind with greed’, and mind without greed as ‘mind without greed’. They understand mind with hate … mind without hate … mind with delusion … mind without delusion … constricted mind … scattered mind … expansive mind … unexpansive mind … mind that is not supreme … mind that is supreme … mind immersed in \textsanskrit{samādhi} … mind not immersed in \textsanskrit{samādhi} … freed mind … They understand unfreed mind as ‘unfreed mind’. 

They\marginnote{21.1} recollect many kinds of past lives. That is: one, two, three, four, five, ten, twenty, thirty, forty, fifty, a hundred, a thousand, a hundred thousand rebirths; many eons of the world contracting, many eons of the world expanding, many eons of the world contracting and expanding. They remember: ‘There, I was named this, my clan was that, I looked like this, and that was my food. This was how I felt pleasure and pain, and that was how my life ended. When I passed away from that place I was reborn somewhere else. There, too, I was named this, my clan was that, I looked like this, and that was my food. This was how I felt pleasure and pain, and that was how my life ended. When I passed away from that place I was reborn here.’ And so they recollect their many kinds of past lives, with features and details. 

With\marginnote{22.1} clairvoyance that is purified and superhuman, they see sentient beings passing away and being reborn—inferior and superior, beautiful and ugly, in a good place or a bad place. They understand how sentient beings are reborn according to their deeds. 

They\marginnote{23.1} realize the undefiled freedom of heart and freedom by wisdom in this very life. And they live having realized it with their own insight due to the ending of defilements. 

These\marginnote{23.2} are the ten inspiring things explained by the Blessed One, who knows and sees, the perfected one, the fully awakened Buddha. We honor anyone in whom these things are found, and rely on them.”\footnote{The suttas record many monks and nuns of such qualities. } 

When\marginnote{24.1} he had spoken, \textsanskrit{Vassakāra} addressed General Upananda,\footnote{This Upananda does not appear elsewhere and the commentary is silent. } “What do you think, general?\footnote{Accept the punctuation in \textsanskrit{Mahāsaṅgīti} rather than PTS. } Do these venerables honor, respect, revere, and venerate those who are worthy?” 

“Indeed\marginnote{24.4} they do. For if these venerables were not to honor, respect, revere, and venerate them, then who exactly would they honor?” 

Then\marginnote{25.1} \textsanskrit{Vassakāra} said to Ānanda, “Where are you staying at present?” 

“In\marginnote{25.3} the Bamboo Grove, brahmin.” 

“I\marginnote{25.4} hope the Bamboo Grove is delightful, quiet and still, far from the madding crowd, remote from human settlements, and fit for retreat?” 

“Indeed\marginnote{25.5} it is, brahmin. And it is like that owing to such protectors and guardians as yourself.”\footnote{Ānanda graciously recognizes the protection offered by the Magadhan state. Even a morally compromised king such as \textsanskrit{Ajātasattu} honored his duty to protect religious seekers (\href{https://suttacentral.net/dn2/en/sujato\#36.2}{DN 2:36.2}). } 

“Surely,\marginnote{25.6} Mister Ānanda, it is owing to the good sirs who meditate, making a habit of meditating.\footnote{“Habit of meditating” is \textit{\textsanskrit{jhānasīlī}}. Throughout this passage, “meditation” renders \textit{\textsanskrit{jhāna}}. This is the only sutta that speaks of a kind of \textit{\textsanskrit{jhāna}} with hindrances. My choice to render the term differently in this passage was prefigured by the translator of the Chinese parallel, which here uses the character that normally renders \textit{\textsanskrit{vicāra}} (MA 145 at T i 655b23). } For the good sirs do in fact meditate and make a habit of meditating. 

This\marginnote{25.8} one time, Mister Ānanda, Mister Gotama was staying near \textsanskrit{Vesālī}, at the Great Wood, in the hall with the peaked roof. So I went there to see him. And there he spoke about meditation in many ways. He meditated, and made a habit of meditating. And he praised all kinds of meditation.” 

“No,\marginnote{26.1} brahmin, the Buddha did not praise all kinds of meditation, nor did he dispraise all kinds of meditation. And what kind of meditation did he not praise? It’s when someone’s heart is overcome and mired in sensual desire, and they don’t truly understand the escape from sensual desire that has arisen. Hiding sensual desire within, they meditate and concentrate and contemplate and ruminate. Their heart is overcome and mired in ill will … dullness and drowsiness … restlessness and remorse … doubt, and they don’t truly know and see the escape from doubt that has arisen. Hiding doubt within, they meditate and concentrate and contemplate and ruminate. The Buddha didn’t praise this kind of meditation. 

And\marginnote{27.1} what kind of meditation did he praise? It’s when a mendicant, quite secluded from sensual pleasures, secluded from unskillful qualities, enters and remains in the first absorption, which has the rapture and bliss born of seclusion, while placing the mind and keeping it connected. As the placing of the mind and keeping it connected are stilled, they enter and remain in the second absorption, which has the rapture and bliss born of immersion, with internal clarity and mind at one, without placing the mind and keeping it connected. And with the fading away of rapture, they enter and remain in the third absorption, where they meditate with equanimity, mindful and aware, personally experiencing the bliss of which the noble ones declare, ‘Equanimous and mindful, one meditates in bliss.’ Giving up pleasure and pain, and ending former happiness and sadness, they enter and remain in the fourth absorption, without pleasure or pain, with pure equanimity and mindfulness. The Buddha praised this kind of meditation.” 

“Well,\marginnote{28.1} Mister Ānanda, it seems that Mister Gotama criticized the kind of meditation that deserves criticism and praised that deserving of praise.\footnote{Normally, of course, when the Buddha teaches “meditation” he means the kind of meditation he praises, namely the four \textit{\textsanskrit{jhānas}}, or “absorptions”. } Well, now, Mister Ānanda, I must go. I have many duties, and much to do.” 

“Please,\marginnote{28.4} brahmin, go at your convenience.” 

Then\marginnote{28.5} \textsanskrit{Vassakāra} the brahmin, having approved and agreed with what Venerable Ānanda said, got up from his seat and left. 

Soon\marginnote{29.1} after he had left, \textsanskrit{Moggallāna} the Guardian said to Ānanda, “Mister Ānanda, you still haven’t answered my question.” 

“But\marginnote{29.3} brahmin, didn’t I say: ‘There is no single mendicant who has all the same qualities in each and every way as possessed by Mister Gotama, the perfected one, the fully awakened Buddha.\footnote{The Chinese parallel differs here, emphasizing instead that there is no difference between the liberation of the Buddha and his arahant disciples (MA 145 at T i 655c27). } For the Blessed One gave rise to the unarisen path, gave birth to the unborn path, and explained the unexplained path. He is the knower of the path, the discoverer of the path, the expert on the path. And now the disciples live following the path; they acquire it later.’”\footnote{Ānanda is clarifying that his original statement was complete, namely, this is the quality that distinguishes the Buddha from his followers. | The Pali lacks the normal ending, whereas the Chinese ends with \textsanskrit{Moggallāna} inviting Ānanda for a meal. } 

%
\section*{{\suttatitleacronym MN 109}{\suttatitletranslation The Longer Discourse on the Full-Moon Night }{\suttatitleroot Mahāpuṇṇamasutta}}
\addcontentsline{toc}{section}{\tocacronym{MN 109} \toctranslation{The Longer Discourse on the Full-Moon Night } \tocroot{Mahāpuṇṇamasutta}}
\markboth{The Longer Discourse on the Full-Moon Night }{Mahāpuṇṇamasutta}
\extramarks{MN 109}{MN 109}

\scevam{So\marginnote{1.1} I have heard. }At one time the Buddha was staying near \textsanskrit{Sāvatthī} in the stilt longhouse of \textsanskrit{Migāra}’s mother in the Eastern Monastery.\footnote{After the Jetavana, this was the best-known monastery in \textsanskrit{Sāvatthī}. It was offered by the lady \textsanskrit{Visākhā}, known as \textsanskrit{Migāra}’s mother. } 

Now,\marginnote{2.1} at that time it was the sabbath—the full moon on the fifteenth day—and the Buddha was sitting in the open surrounded by the \textsanskrit{Saṅgha} of monks. 

Then\marginnote{3.1} one of the mendicants got up from their seat, arranged their robe over one shoulder, raised their joined palms toward the Buddha, and said, “I’d like to ask the Buddha about a certain point, if you’d take the time to answer.”\footnote{The commentary says that he asked these questions in order to educate his students. This seems probable in view of the fact that the sequence of questions is structured in a way that is clearly purposeful. } 

“Well\marginnote{3.3} then, mendicant, take your own seat and ask what you wish.” 

That\marginnote{3.4} mendicant took his seat and said to the Buddha: 

“Sir,\marginnote{4.1} are these the five grasping aggregates: form, feeling, perception, choices, and consciousness?” 

“Yes,\marginnote{4.3} they are,” replied the Buddha. 

Saying\marginnote{4.5} “Good, sir”, that mendicant approved and agreed with what the Buddha said. Then he asked another question: 

“But\marginnote{5.1} sir, what is the root of these five grasping aggregates?” 

“These\marginnote{5.2} five grasping aggregates are rooted in desire.”\footnote{“Desire” (\textit{chanda}) is the fundamental driving force underlying all manifestations of the aggregates. It is the same as “craving” (\textit{\textsanskrit{taṇhā}}). } 

“But\marginnote{6.1} sir, is that grasping the exact same thing as the five grasping aggregates? Or is grasping one thing and the five grasping aggregates another?” 

“Neither.\marginnote{6.2} The desire and greed for the five grasping aggregates is the grasping there.”\footnote{See \href{https://suttacentral.net/mn44/en/sujato\#6.3}{MN 44:6.3} and note there. } 

“But\marginnote{7.1} sir, can there be different kinds of desire and greed for the five grasping aggregates?” 

“There\marginnote{7.2} can,” said the Buddha. “It’s when someone thinks: ‘In the future, may I be of such form, such feeling, such perception, such choices, and such consciousness!’\footnote{This assumes the persistence of a “self” through time. } That’s how there can be different kinds of desire and greed for the five grasping aggregates.” 

“Sir,\marginnote{8.1} what is the scope of the term ‘aggregates’ as applied to the aggregates?” 

“Any\marginnote{8.2} kind of form at all—past, future, or present; internal or external; solid or subtle; inferior or superior; far or near: this is called the aggregate of form. Any kind of feeling at all … Any kind of perception at all … Any kind of choices at all … Any kind of consciousness at all—past, future, or present; internal or external; solid or subtle; inferior or superior; far or near: this is called the aggregate of consciousness. That’s the scope of the term ‘aggregates’ as applied to the aggregates.”\footnote{That is to say, the term “aggregate” (\textit{khandha}) is a collective term for all instances of this kind of phenomenon. } 

“What\marginnote{9.1} is the cause, sir, what is the reason why the aggregate of form is found?\footnote{While the “root” of all aggregates is desire, the question now is the immediate basis of each particular aggregate. } What is the cause, what is the reason why the aggregate of feeling … perception … choices … consciousness is found?” 

“The\marginnote{9.6} four principal states are the reason why the aggregate of form is found. Contact is the reason why the aggregates of feeling …\footnote{In dependent origination, “contact” (\textit{phassa}) is the immediate condition for “feeling” (\textit{\textsanskrit{vedanā}}). But contact can also be regarded as the immediate condition for all three: feeling, perception, and choices (eg. \href{https://suttacentral.net/sn35.93/en/sujato\#1.14}{SN 35.93:1.14}). These things are always active in consciousness, so they can be treated as simultaneous, as arising in sequence, or as mutually conditioning. } perception …\footnote{Feeling and “perception” (\textit{\textsanskrit{saññā}}) are closely connected, since how we feel conditions how we recognize and interpret. While contact directly underlies feeling, perception works so quickly that is can be included as a direct consequence as well. } and choices are found.\footnote{Likewise with “choices” (\textit{\textsanskrit{saṅkhārā}}), which are also part of the complex of reactivity that normally take effect immediately. For example, suppose we try eating a berry in the forest. “Contact” with its bitter taste “feels” unpleasant, so we “perceive” it as poisonous, and “choose” to spit it out. } Name and form are the reasons why the aggregate of consciousness is found.”\footnote{Normally in dependent origination, it is consciousness that is a condition for name and form. The presentation here assumes their mutual conditioning, as spelled out at \href{https://suttacentral.net/dn15/en/sujato\#22.1}{DN 15:22.1}. Name and form are the embryo that supports consciousness in the process of rebirth, while the physical body with its senses, and the psychological functions included under “name” continue to evolve and grow with consciousness through life. } 

“But\marginnote{10.1} sir, how does substantialist view come about?”\footnote{As at \href{https://suttacentral.net/mn44/en/sujato\#7.1}{MN 44:7.1}. } 

“It’s\marginnote{10.2} when an unlearned ordinary person has not seen the noble ones, and is neither skilled nor trained in the teaching of the noble ones. They’ve not seen true persons, and are neither skilled nor trained in the teaching of the true persons. They regard form as self, self as having form, form in self, or self in form. They regard feeling as self, self as having feeling, feeling in self, or self in feeling. They regard perception as self, self as having perception, perception in self, or self in perception. They regard choices as self, self as having choices, choices in self, or self in choices. They regard consciousness as self, self as having consciousness, consciousness in self, or self in consciousness. That’s how substantialist view comes about.”\footnote{That is to say, the “substantialist” view takes the aggregates, or one of them, to be a “substantial reality” that is identified with the self. Such views are created each time we think or attach to the aggregates as a self. } 

“But\marginnote{11.1} sir, how does substantialist view not come about?” 

“It’s\marginnote{11.2} when a learned noble disciple has seen the noble ones, and is skilled and trained in the teaching of the noble ones. They’ve seen true persons, and are skilled and trained in the teaching of the true persons. They don’t regard form as self, self as having form, form in self, or self in form. They don’t regard feeling as self, self as having feeling, feeling in self, or self in feeling. They don’t regard perception as self, self as having perception, perception in self, or self in perception. They don’t regard choices as self, self as having choices, choices in self, or self in choices. They don’t regard consciousness as self, self as having consciousness, consciousness in self, or self in consciousness. That’s how substantialist view does not come about.”\footnote{Since substantialist view is itself a conditioned phenomenon, it can only continue to exist if it is continually reinforced. Without it, like a fire without a constant supply of fuel, it falters and goes out. } 

“Sir,\marginnote{12.1} what’s the gratification, the drawback, and the escape when it comes to form,\footnote{We have met these questions before at \href{https://suttacentral.net/mn13/en/sujato\#6.2}{MN 13:6.2}. Here they build on the previous section by showing a strategy for how to stop identifying them as the substantial reality underlying the self. This can be used as the basis for a contemplative insight meditation. } feeling, perception, choices, and consciousness?” 

“The\marginnote{12.6} pleasure and happiness that arise from form: this is its gratification. That form is impermanent, suffering, and perishable: this is its drawback. Removing and giving up desire and greed for form: this is its escape. The pleasure and happiness that arise from feeling … perception … choices … consciousness: this is its gratification. That consciousness is impermanent, suffering, and perishable: this is its drawback. Removing and giving up desire and greed for consciousness: this is its escape.” 

“Sir,\marginnote{13.1} how does one know and see so that there’s no I-making, mine-making, or underlying tendency to conceit for this conscious body and all external stimuli?”\footnote{From meditative contemplation, the question now turns to realization. The Buddha’s response makes it clear he is now speaking of the penetrative understanding of the aggregates that arises at stream-entry at least. The “underlying tendency to conceit”, however, is only fully overcome by the arahant. | “I-making” (\textit{\textsanskrit{ahaṅkāra}}) emphasizes the active role of the mind in creating a sense of self through its thoughts, habits, and desires. This is in contrast with the English notion of “ego”, which presents self as a given construct. | “Mine-making” (\textit{\textsanskrit{mamaṅkāra}}) is the propensity of the mind to appropriate things as possessions. | “All external stimuli” (\textit{sabbanimitta}) are the “features” (also \textit{nimitta}) that characterize sensory stimulation (eg. \href{https://suttacentral.net/mn112/en/sujato\#15.1}{MN 112:15.1}). } 

“One\marginnote{13.2} truly sees any kind of form at all—past, future, or present; internal or external; solid or subtle; inferior or superior; far or near: \emph{all} form—with right understanding: ‘This is not mine, I am not this, this is not my self.’ One truly sees any kind of feeling … perception … choices … consciousness at all—past, future, or present; internal or external; solid or subtle; inferior or superior; far or near, \emph{all} consciousness—with right understanding: ‘This is not mine, I am not this, this is not my self.’ That’s how to know and see so that there’s no I-making, mine-making, or underlying tendency to conceit for this conscious body and all external stimuli.” 

Now\marginnote{14.1} at that time one of the mendicants had the thought, “So it seems, good sir, that form, feeling, perception, choices, and consciousness are not-self. Then what self will the deeds done by not-self affect?”\footnote{This monk, missing the point, thinks not-self means he can escape moral responsibility. A theory of self assumes a continuous entity, so that deeds done at one time will affect the same entity at another time. The Buddhist concept, on the other hand, is of a flow, so that, for example, pollution dumped in a stream will affect the river and the sea below, even though they are not materially the “same”. } 

But\marginnote{14.4} the Buddha, knowing that mendicant’s train of thought, addressed the mendicants: “It’s possible that some futile person here—unknowing and ignorant, their mind dominated by craving—thinks they can overstep the teacher’s instructions. They think: ‘So it seems, good sir, that form, feeling, perception, choices, and consciousness are not-self. Then what self will the deeds done by not-self affect?’ Now, mendicants, you have been educated by me in questioning with regard to all these things in all such cases.\footnote{Read \textit{\textsanskrit{paṭipucchāvinītā}} with \href{https://suttacentral.net/sn22.82/en/sujato\#13.4}{SN 22.82:13.4}. It is the name for the method of questioning the Buddha is about to use. The Buddha does not just state doctrine for his students, he has “educated” (\textit{\textsanskrit{vinīta}}) them in the methods for learning. Active questioning breaks up long discourses and encourages reflection and articulation. } 

What\marginnote{15.1} do you think, mendicants? Is form permanent or impermanent?” 

“Impermanent,\marginnote{15.3} sir.” 

“But\marginnote{15.4} if it’s impermanent, is it suffering or happiness?” 

“Suffering,\marginnote{15.5} sir.” 

“But\marginnote{15.6} if it’s impermanent, suffering, and perishable, is it fit to be regarded thus: ‘This is mine, I am this, this is my self’?” 

“No,\marginnote{15.8} sir.” 

“What\marginnote{16.1} do you think, mendicants? Is feeling … perception … choices … consciousness permanent or impermanent?” 

“Impermanent,\marginnote{16.6} sir.” 

“But\marginnote{16.7} if it’s impermanent, is it suffering or happiness?” 

“Suffering,\marginnote{16.8} sir.” 

“But\marginnote{16.9} if it’s impermanent, suffering, and perishable, is it fit to be regarded thus: ‘This is mine, I am this, this is my self’?” 

“No,\marginnote{16.11} sir.” 

“So\marginnote{16.12} you should truly see any kind of form at all—past, future, or present; internal or external; solid or subtle; inferior or superior; far or near: \emph{all} form—with right understanding: ‘This is not mine, I am not this, this is not my self.’ 

You\marginnote{17.1} should truly see any kind of feeling … perception … choices … consciousness at all—past, future, or present; internal or external; solid or subtle; inferior or superior; far or near, \emph{all} consciousness—with right understanding: ‘This is not mine, I am not this, this is not my self.’ 

Seeing\marginnote{17.5} this, a learned noble disciple grows disillusioned with form, feeling, perception, choices, and consciousness. 

Being\marginnote{18.1} disillusioned, desire fades away. When desire fades away they’re freed. When they’re freed, they know they’re freed. 

They\marginnote{18.2} understand: ‘Rebirth is ended, the spiritual journey has been completed, what had to be done has been done, there is nothing further for this place.’” 

That\marginnote{18.3} is what the Buddha said. Satisfied, the mendicants approved what the Buddha said. And while this discourse was being spoken, the minds of sixty mendicants were freed from defilements by not grasping. 

%
\section*{{\suttatitleacronym MN 110}{\suttatitletranslation The Shorter Discourse on the Full-Moon Night }{\suttatitleroot Cūḷapuṇṇamasutta}}
\addcontentsline{toc}{section}{\tocacronym{MN 110} \toctranslation{The Shorter Discourse on the Full-Moon Night } \tocroot{Cūḷapuṇṇamasutta}}
\markboth{The Shorter Discourse on the Full-Moon Night }{Cūḷapuṇṇamasutta}
\extramarks{MN 110}{MN 110}

\scevam{So\marginnote{1.1} I have heard. }At one time the Buddha was staying near \textsanskrit{Sāvatthī} in the stilt longhouse of \textsanskrit{Migāra}’s mother in the Eastern Monastery. 

Now,\marginnote{2.1} at that time it was the sabbath—the full moon on the fifteenth day—and the Buddha was sitting in the open surrounded by the \textsanskrit{Saṅgha} of monks. Then the Buddha looked around the \textsanskrit{Saṅgha} of mendicants, who were so very silent. He addressed them, “Mendicants, could an untrue person know of an untrue person:\footnote{\textsanskrit{Vassakāra} asks the Buddha the same questions at \href{https://suttacentral.net/an4.187/en/sujato\#2.1}{AN 4.187:2.1}, with an interesting example given from the court gossip. } ‘This fellow is an untrue person’?” 

“No,\marginnote{3.3} sir.” 

“Good,\marginnote{3.4} mendicants! It’s impossible, it can’t happen, that an untrue person could know of an untrue person: ‘This fellow is an untrue person.’ But could an untrue person know of a true person: ‘This fellow is a true person’?” 

“No,\marginnote{3.9} sir.” 

“Good,\marginnote{3.10} mendicants! That too is impossible. A untrue person has bad qualities, is devoted to untrue persons, and has the intentions, counsel, speech, actions, views, and giving of an untrue person.\footnote{\textit{Bhatti} is given in the Abhidhamma \textsanskrit{Vibhaṅga} as a synonym of \textit{\textsanskrit{sevanā}}, “association” (\href{https://suttacentral.net/vb17/en/sujato\#93.2}{Vb 17:93.2}). While this is correct, the early Pali contexts are more warmly emotional than the rather austere “association”, connecting it with words such as “faith” (\textit{\textsanskrit{saddhā}}) and “fondness” (\textit{pema}) (\href{https://suttacentral.net/thag5.12/en/sujato\#1.3}{Thag 5.12:1.3}, \href{https://suttacentral.net/thig15.1/en/sujato\#14.1}{Thig 15.1:14.1}, \href{https://suttacentral.net/an5.141/en/sujato\#5.2}{AN 5.141:5.2}). This anticipates the later use of \textit{bhakti} as “(ecstatic religious) devotion”. | “Devotion to untrue persons” is said to be a quality of the Jains (\href{https://suttacentral.net/an10.78/en/sujato\#1.7}{AN 10.78:1.7}). } 

And\marginnote{5.1} how does an untrue person have bad qualities? It’s when an untrue person is faithless, shameless, imprudent, unlearned, lazy, unmindful, and witless. That’s how an untrue person has bad qualities. 

And\marginnote{6.1} how is an untrue person devoted to untrue persons? It’s when an untrue person is a friend and companion of ascetics and brahmins who are faithless, shameless, imprudent, unlearned, lazy, unmindful, and witless.\footnote{Where \textit{assaddhamma} appears in \href{https://suttacentral.net/an10.62/en/sujato\#3.1}{AN 10.62:3.1}, it means listening to “untrue teachings”, whereas here it means “bad qualities”. } That’s how an untrue person is devoted to untrue persons. 

And\marginnote{7.1} how does an untrue person have the intentions of an untrue person? It’s when an untrue person intends to hurt themselves, hurt others, and hurt both. That’s how an untrue person has the intentions of an untrue person. 

And\marginnote{8.1} how does an untrue person offer the counsel of an untrue person? It’s when an untrue person offers counsel that hurts themselves, hurts others, and hurts both. That’s how an untrue person offers the counsel of an untrue person. 

And\marginnote{9.1} how does an untrue person have the speech of an untrue person? It’s when an untrue person uses speech that’s false, divisive, harsh, and nonsensical. That’s how an untrue person has the speech of an untrue person. 

And\marginnote{10.1} how does an untrue person have the action of an untrue person? It’s when an untrue person kills living creatures, steals, and commits sexual misconduct. That’s how an untrue person has the actions of an untrue person. 

And\marginnote{11.1} how does an untrue person have the view of an untrue person? It’s when an untrue person has such a view: ‘There’s no meaning in giving, sacrifice, or offerings. There’s no fruit or result of good and bad deeds. There’s no afterlife. There’s no such thing as mother and father, or beings that are reborn spontaneously. And there’s no ascetic or brahmin who is rightly comported and rightly practiced, and who describes the afterlife after realizing it with their own insight.’ That’s how an untrue person has the view of an untrue person. 

And\marginnote{12.1} how does an untrue person give the gifts of an untrue person?\footnote{Also at \href{https://suttacentral.net/an5.147/en/sujato}{AN 5.147}; see too \href{https://suttacentral.net/an9.20/en/sujato}{AN 9.20}. } It’s when an untrue person gives a gift carelessly, not with their own hand, and thoughtlessly. They give the dregs, and they give without consideration for consequences. That’s how an untrue person gives the gifts of an untrue person. 

That\marginnote{13.1} untrue person—who has such bad qualities, frequents untrue persons, and has the intentions, counsel, speech, actions, views, and giving of an untrue person—when their body breaks up, after death, is reborn in the place where untrue persons are reborn. And what is the place where untrue persons are reborn? Hell or the animal realm.\footnote{At \href{https://suttacentral.net/an2.26/en/sujato}{AN 2.26}–29 these destinies are said to await one who hides misdeeds, who has wrong view, and who is unethical. At \href{https://suttacentral.net/mn97/en/sujato\#30.3}{MN 97:30.3} it is said that, of the two, the animal realm is better. } 

Mendicants,\marginnote{14.1} could a true person know of a true person: ‘This fellow is a true person’?” 

“Yes,\marginnote{14.3} sir.” 

“Good,\marginnote{14.4} mendicants! It is possible that a true person could know of a true person: ‘This fellow is a true person.’ But could a true person know of an untrue person: ‘This fellow is an untrue person’?” 

“Yes,\marginnote{14.9} sir.” 

“Good,\marginnote{15.1} mendicants! That too is possible. A true person has good qualities, is devoted to true persons, and has the intentions, counsel, speech, actions, views, and giving of a true person. 

And\marginnote{16.1} how does a true person have good qualities? It’s when a true person is faithful, conscientious, prudent, learned, energetic, mindful, and wise. That’s how a true person has good qualities. 

And\marginnote{17.1} how is a true person devoted to true persons? It’s when a true person is a friend and companion of ascetics and brahmins who are faithful, conscientious, prudent, learned, energetic, mindful, and wise. That’s how a true person is devoted to true persons. 

And\marginnote{18.1} how does a true person have the intentions of a true person? It’s when a true person doesn’t intend to hurt themselves, hurt others, and hurt both. That’s how a true person has the intentions of a true person. 

And\marginnote{19.1} how does a true person offer the counsel of a true person? It’s when a true person offers counsel that doesn’t hurt themselves, hurt others, and hurt both. That’s how a true person offers the counsel of a true person. 

And\marginnote{20.1} how does a true person have the speech of a true person? It’s when a true person refrains from speech that’s false, divisive, harsh, or nonsensical. That’s how a true person has the speech of a true person. 

And\marginnote{21.1} how does a true person have the action of a true person? It’s when a true person refrains from killing living creatures, stealing, and committing sexual misconduct. That’s how a true person has the action of a true person. 

And\marginnote{22.1} how does a true person have the view of a true person? It’s when a true person has such a view: ‘There is meaning in giving, sacrifice, and offerings. There are fruits and results of good and bad deeds. There is an afterlife. There are such things as mother and father, and beings that are reborn spontaneously. And there are ascetics and brahmins who are rightly comported and rightly practiced, and who describe the afterlife after realizing it with their own insight.’ That’s how a true person has the view of a true person. 

And\marginnote{23.1} how does a true person give the gifts of a true person? It’s when a true person gives a gift carefully, with their own hand, and thoughtfully. They don’t give the dregs, and they give with consideration for consequences. That’s how a true person gives the gifts of a true person. 

That\marginnote{24.1} true person—who has such good qualities, is devoted to true persons, and has the intentions, counsel, speech, actions, views, and giving of a true person—when their body breaks up, after death, is reborn in the place where true persons are reborn. And what is the place where true persons are reborn? A state of greatness among gods or humans.” 

That\marginnote{25.1} is what the Buddha said. Satisfied, the mendicants approved what the Buddha said. 

%
\addtocontents{toc}{\let\protect\contentsline\protect\nopagecontentsline}
\chapter*{The Chapter Beginning with One By One }
\addcontentsline{toc}{chapter}{\tocchapterline{The Chapter Beginning with One By One }}
\addtocontents{toc}{\let\protect\contentsline\protect\oldcontentsline}

%
\section*{{\suttatitleacronym MN 111}{\suttatitletranslation One by One }{\suttatitleroot Anupadasutta}}
\addcontentsline{toc}{section}{\tocacronym{MN 111} \toctranslation{One by One } \tocroot{Anupadasutta}}
\markboth{One by One }{Anupadasutta}
\extramarks{MN 111}{MN 111}

\scevam{So\marginnote{1.1} I have heard.\footnote{This discourse describes \textsanskrit{Sāriputta}’s practice of discernment leading to awakening. It fills in the blanks between realizing stream entry and arahantship, reminding us that when a narrative depicts a person realizing the truth while listing to a teaching—as \textsanskrit{Sāriputta} does—there is usually an extensive background of practice behind that. A number of indications suggest that this is among the later discourses: there are no parallels; the exposition is unusually elaborate; there is editorial clumsiness; and a number of terms are unique or characteristic of later texts. } }At one time the Buddha was staying near \textsanskrit{Sāvatthī} in Jeta’s Grove, \textsanskrit{Anāthapiṇḍika}’s monastery. There the Buddha addressed the mendicants, “Mendicants!” 

“Venerable\marginnote{1.5} sir,” they replied. The Buddha said this: 

“\textsanskrit{Sāriputta}\marginnote{2.1} is astute, mendicants. He has great wisdom,\footnote{These are qualities of the Buddha at \href{https://suttacentral.net/dn30/en/sujato\#1.26.6}{DN 30:1.26.6} and of \textsanskrit{Sāriputta} again at \href{https://suttacentral.net/sn8.7/en/sujato\#3.2}{SN 8.7:3.2}. } widespread wisdom,\footnote{“Widespread wisdom” is defined at \href{https://suttacentral.net/an3.30/en/sujato\#3.1}{AN 3.30:3.1} as one who often listens to teachings, applies their mind while listening, and continues to do so afterwards. } laughing wisdom, swift wisdom, sharp wisdom, and penetrating wisdom. For a fortnight he practiced discernment of phenomena one by one.\footnote{The fortnight between realizing stream-entry while still a student of \textsanskrit{Sañjaya} (\href{https://suttacentral.net/pli-tv-kd1/en/sujato\#23.5.6}{Kd 1:23.5.6}) and arahantship while overhearing the Buddha teach \textsanskrit{Dīghanakha} (\href{https://suttacentral.net/mn74/en/sujato\#5.1}{MN 74:5.1}). | \textit{Anupada} elsewhere means a “following statement” (\href{https://suttacentral.net/pli-tv-bu-vb-pc4/en/sujato\#2.1.11}{Bu Pc 4:2.1.11}, \href{https://suttacentral.net/vv35/en/sujato\#7.4}{Vv 35:7.4}, \href{https://suttacentral.net/vv53/en/sujato\#9.4}{Vv 53:9.4}). Here it refers to the “discernment” (\textit{\textsanskrit{vipassanā}}) of each individual phenomena one by one as they occur. This methodical style of contemplation is reflected in the similarly detailed and thorough teachings characteristic of \textsanskrit{Sāriputta}. } And this is how he did it. 

Quite\marginnote{3.1} secluded from sensual pleasures, secluded from unskillful qualities, he entered and remained in the first absorption, which has the rapture and bliss born of seclusion, while placing the mind and keeping it connected. And he distinguished the phenomena of the first absorption one by one: placing and keeping and rapture and bliss and unification of mind; contact, feeling, perception, intention, mind, enthusiasm, decision, energy, mindfulness, equanimity, and application of mind.\footnote{\textit{Vavatthita} (“distinguished”) occurs elsewhere in the suttas in only one late verse (\href{https://suttacentral.net/dn20/en/sujato\#6.10}{DN 20:6.10}). The Vinaya gives it as an antonym to \textit{sambhinna}, “mixed” (\href{https://suttacentral.net/pli-tv-kd13/en/sujato\#33.1.7}{Kd 13:33.1.7}). | The first five items here are the five so-called “\textsanskrit{jhāna} factors”, which in the suttas appear here and at \href{https://suttacentral.net/mn43/en/sujato\#19.3}{MN 43:19.3}. The list summarizes the normal depiction of the first absorption, as the first four factors are all part of the standard first absorption formula, and all absorption or \textit{\textsanskrit{samādhi}} is characterized by unification of mind (eg. \href{https://suttacentral.net/mn20/en/sujato\#3.3}{MN 20:3.3}). | There is a striking syntactic break between the first five factors and the remainder of the list starting with “contact”. The five factors are articulated in the colloquial Pali fashion connected with \textit{ca}, while the remainder are simply listed without \textit{ca}. This, together with the fact that the first five appear elsewhere as a set and the remainder do not, as well as the fact that the second list is partly redundant (eg. “feeling” is already covered in the five factors), makes it certain that the latter items were appended to the first five. Indeed, the appended items appear elaborated even compared with similar lists in the Abhidhamma (eg. \href{https://suttacentral.net/dt1.2/en/sujato\#16.1}{Dt 1.2:16.1}, \href{https://suttacentral.net/kv15.10/en/sujato\#2.3}{Kv 15.10:2.3}; see too \href{https://suttacentral.net/mil3.3.6/en/sujato\#7.2}{Mil 3.3.6:7.2}). } He knew those phenomena as they arose, as they remained, and as they went away.\footnote{He was mindfully aware as the attainment characterized by these qualities arose, persisted, and fell away. As \textsanskrit{Anālayo} points out, this cannot refer to observing the cessation of phenomena \emph{inside} the absorption, as the absorption itself is defined by their persistence (\emph{Early Buddhist Meditation Studies}, pp. 117 ff.). In other words, he was aware of the impermanence \emph{of} absorption, not of impermanence \emph{in} absorption. | Compare the “development of immersion” that contemplates feelings, perceptions, and thoughts in the same way (\href{https://suttacentral.net/an4.41/en/sujato\#4.3}{AN 4.41:4.3}). According to the suttas, for all conditioned phenomena, “arising is evident, vanishing is evident, and change while persisting is evident” (\href{https://suttacentral.net/an3.47/en/sujato\#1.3}{AN 3.47:1.3}). } He understood: ‘So it seems that these phenomena, not having been, come to be; and having come to be, they flit away.’\footnote{In the previous passage he \emph{observed} them, whereas now he \emph{reflects} on what he has seen. This course of practice anticipates the Niddesa’s notion of three kinds of full knowledge (\href{https://suttacentral.net/cnd11/en/sujato\#51.4}{Cnd 11:51.4}). First comes “full knowledge of the known” (\textit{\textsanskrit{ñātapariññā}}), where the meditator clearly observes each phenomena in experience. Then they reflect and scrutinize the “known” in light of the three characteristics of impermanence, suffering, and not-self, as \textsanskrit{Sāriputta} is doing here (\textit{\textsanskrit{tīraṇapariññā}}). Finally they let go all attachments (\textit{\textsanskrit{pahānapariññā}}), as \textsanskrit{Sāriputta} does below. | “Flit away” (\textit{\textsanskrit{paṭiventi}}) is unique. } Regarding those phenomena, he meditated without going near or going away, independent, untied, liberated, detached, his mind free of limits.\footnote{“Without going near or going away” (\textit{\textsanskrit{anupāyo} \textsanskrit{anapāyo}}) is explained by the commentary as not being swayed by the power of lust or aversion. It means having a poised and unreactive attitude. It is a rare phrase that also occurs, with the same syntactic context, in the next sutta, creating a link between the two (\href{https://suttacentral.net/mn112/en/sujato\#4.2}{MN 112:4.2}). } He understood: ‘There is an escape beyond.’\footnote{The “escape beyond” (\textit{uttari \textsanskrit{nissaraṇaṁ}}) normally refers to full awakening (\href{https://suttacentral.net/mn7/en/sujato\#17.1}{MN 7:17.1}, \href{https://suttacentral.net/mn49/en/sujato\#2.3}{MN 49:2.3}, \href{https://suttacentral.net/an3.66/en/sujato\#46.6}{AN 3.66:46.6}, \href{https://suttacentral.net/an7.56/en/sujato\#6.1}{AN 7.56:6.1}, \href{https://suttacentral.net/an10.93/en/sujato\#14.4}{AN 10.93:14.4}), but the commentary says that here it refers to the next higher attainment. } Repeated practice of that confirmed this for him.\footnote{The compound \textit{atthitvevassa} resolves to: \textit{atthi iti} (“that it is real”) \textit{eva} (“indeed”) \textit{assa} (“for him”). The commentary glosses, “That it is real is reinforced for that monk.” } 

Furthermore,\marginnote{5.1} as the placing of the mind and keeping it connected were stilled, he entered and remained in the second absorption, which has the rapture and bliss born of immersion, with internal clarity and mind at one, without placing the mind and keeping it connected. 

And\marginnote{6.1} he distinguished the phenomena of the second absorption one by one: internal confidence and rapture and bliss and unification of mind; contact, feeling, perception, intention, mind, enthusiasm, decision, energy, mindfulness, equanimity, and application of mind.\footnote{The initial list representing the absorption factors is adjusted according to the level of absorption, while the remainder of the list stays the same. } He knew those phenomena as they arose, as they remained, and as they went away. He understood: ‘So it seems that these phenomena, not having been, come to be; and having come to be, they flit away.’ Regarding those phenomena, he meditated without going near or going away, independent, untied, liberated, detached, his mind free of limits. He understood: ‘There is an escape beyond.’ Repeated practice of that confirmed this for him. 

Furthermore,\marginnote{7.1} with the fading away of rapture, he entered and remained in the third absorption, where he meditated with equanimity, mindful and aware, personally experiencing the bliss of which the noble ones declare, ‘Equanimous and mindful, one meditates in bliss.’ 

And\marginnote{8.1} he distinguished the phenomena of the third absorption one by one: bliss and mindfulness and awareness and unification of mind; contact, feeling, perception, intention, mind, enthusiasm, decision, energy, mindfulness, equanimity, and application of mind. He knew those phenomena as they arose, as they remained, and as they went away. He understood: ‘So it seems that these phenomena, not having been, come to be; and having come to be, they flit away.’ Regarding those phenomena, he meditated without going near or going away, independent, untied, liberated, detached, his mind free of limits. He understood: ‘There is an escape beyond.’ Repeated practice of that confirmed this for him. 

Furthermore,\marginnote{9.1} with the giving up of pleasure and pain, and the ending of former happiness and sadness, he entered and remained in the fourth absorption, without pleasure or pain, with pure equanimity and mindfulness. 

And\marginnote{10.1} he distinguished the phenomena of the fourth absorption one by one: equanimity and neutral feeling and mental unconcern due to tranquility and pure mindfulness and unification of mind; contact, feeling, perception, intention, mind, enthusiasm, decision, energy, mindfulness, equanimity, and application of mind.\footnote{“Unconcern” (\textit{\textsanskrit{anābhoga}}) is a unique term in the suttas. In the Abhidhamma \textsanskrit{Vibhaṅga}, it is said to be a characteristic of the five kinds of sense consciousness, which according to that system do not feel pleasure or pain (\href{https://suttacentral.net/vb16/en/sujato\#13.1}{Vb 16:13.1}). } He knew those phenomena as they arose, as they remained, and as they went away. He understood: ‘So it seems that these phenomena, not having been, come to be; and having come to be, they flit away.’ Regarding those phenomena, he meditated without going near or going away, independent, untied, liberated, detached, his mind free of limits. He understood: ‘There is an escape beyond.’ Repeated practice of that confirmed this for him. 

Furthermore,\marginnote{11.1} going totally beyond perceptions of form, with the ending of perceptions of impingement, not focusing on perceptions of diversity, aware that ‘space is infinite’, he entered and remained in the dimension of infinite space. 

And\marginnote{12.1} he distinguished the phenomena of the dimension of infinite space one by one: the perception of the dimension of infinite space and unification of mind; contact, feeling, perception, intention, mind, enthusiasm, decision, energy, mindfulness, equanimity, and application of mind. He knew those phenomena as they arose, as they remained, and as they went away. He understood: ‘So it seems that these phenomena, not having been, come to be; and having come to be, they flit away.’ Regarding those phenomena, he meditated without going near or going away, independent, untied, liberated, detached, his mind free of limits. He understood: ‘There is an escape beyond.’ Repeated practice of that confirmed this for him. 

Furthermore,\marginnote{13.1} going totally beyond the dimension of infinite space, aware that ‘consciousness is infinite’, he entered and remained in the dimension of infinite consciousness. 

And\marginnote{14.1} he distinguished the phenomena of the dimension of infinite consciousness one by one: the perception of the dimension of infinite consciousness and unification of mind; contact, feeling, perception, intention, mind, enthusiasm, decision, energy, mindfulness, equanimity, and application of mind. He knew those phenomena as they arose, as they remained, and as they went away. He understood: ‘So it seems that these phenomena, not having been, come to be; and having come to be, they flit away.’ Regarding those phenomena, he meditated without going near or going away, independent, untied, liberated, detached, his mind free of limits. He understood: ‘There is an escape beyond.’ Repeated practice of that confirmed this for him. 

Furthermore,\marginnote{15.1} going totally beyond the dimension of infinite consciousness, aware that ‘there is nothing at all’, he entered and remained in the dimension of nothingness. 

And\marginnote{16.1} he distinguished the phenomena of the dimension of nothingness one by one: the perception of the dimension of nothingness and unification of mind; contact, feeling, perception, intention, mind, enthusiasm, decision, energy, mindfulness, equanimity, and application of mind. He knew those phenomena as they arose, as they remained, and as they went away. He understood: ‘So it seems that these phenomena, not having been, come to be; and having come to be, they flit away.’ Regarding those phenomena, he meditated without going near or going away, independent, untied, liberated, detached, his mind free of limits. He understood: ‘There is an escape beyond.’ Repeated practice of that confirmed this for him. 

Furthermore,\marginnote{17.1} going totally beyond the dimension of nothingness, he entered and remained in the dimension of neither perception nor non-perception. 

And\marginnote{18.1} he emerged from that attainment with mindfulness. Then he contemplated the phenomena of that attainment that had passed, ceased, and perished: ‘So it seems that these phenomena, not having been, come to be; and having come to be, they flit away.’\footnote{In this, the highest of the eight attainments, there is no individual discernment of mental factors, as the function of “perception” is liminal; compare \href{https://suttacentral.net/mn52/en/sujato\#14.7}{MN 52:14.7}, \href{https://suttacentral.net/mn64/en/sujato\#15.5}{MN 64:15.5}, and \href{https://suttacentral.net/mn102/en/sujato\#4.4}{MN 102:4.4}. And rather than observing the arising, persisting, and ceasing of the absorption phenomena, as he had done previously, he reflects on the fact of their come and gone. As \textsanskrit{Anālayo} (\emph{ibid.}, 122) puts it, “there is no continuity of perceptual awareness from before the time one entered these attainments to emergence from them … practitioners could only know that at some time in the past they entered and that by now they have emerged”. } Regarding those phenomena, he meditated without going near or going away, independent, untied, liberated, detached, his mind free of limits. He understood: ‘There is an escape beyond.’ Repeated practice of that confirmed this for him. 

Furthermore,\marginnote{19.1} going totally beyond the dimension of neither perception nor non-perception, he entered and remained in the cessation of perception and feeling. And, having seen with wisdom, his defilements came to an end. 

And\marginnote{20.1} he emerged from that attainment with mindfulness. Then he contemplated the phenomena of that attainment that had passed, ceased, and perished: ‘So it seems that these phenomena, not having been, come to be; and having come to be, they flit away.’\footnote{It is not clear what this refers to, as there are no mental states arising or ceasing in the state of cessation. The commentary says it refers to either the material states present (i.e. the body originated by kamma) or the states of the previous attainment. Perhaps it is simply an editorial slip. } Regarding those phenomena, he meditated without going near or going away, independent, untied, liberated, detached, his mind free of limits. He understood: ‘There is no escape beyond.’\footnote{He has fully realized Nibbana, beyond which there is nothing higher. } By cultivating that he confirmed that there is not. 

And\marginnote{21.1} if there’s anyone of whom it may be rightly said that they have attained mastery and perfection in noble ethics, immersion, wisdom, and freedom, it’s \textsanskrit{Sāriputta}.\footnote{Compare \href{https://suttacentral.net/mn77/en/sujato\#15.6}{MN 77:15.6} and \href{https://suttacentral.net/mn100/en/sujato\#6.1}{MN 100:6.1}. } 

And\marginnote{22.1} if there’s anyone of whom it may be rightly said that they’re the Buddha’s true-born son, born from his mouth, born of the teaching, created by the teaching, heir to the teaching, not the heir in things of the flesh, it’s \textsanskrit{Sāriputta}.\footnote{Also said of \textsanskrit{Sāriputta} at \href{https://suttacentral.net/sn8.7/en/sujato\#3.2}{SN 8.7:3.2} and of \textsanskrit{Mahākassapa} at \href{https://suttacentral.net/sn16.11/en/sujato\#15.2}{SN 16.11:15.2}. The phrase was adopted from the Brahmins’ claim to be born from the mouth of \textsanskrit{Brahmā} (\href{https://suttacentral.net/mn84/en/sujato\#4.5}{MN 84:4.5}, \href{https://suttacentral.net/dn27/en/sujato\#3.9}{DN 27:3.9}). } 

\textsanskrit{Sāriputta}\marginnote{23.1} rightly keeps rolling the supreme Wheel of Dhamma that was rolled forth by the Realized One.”\footnote{Also at \href{https://suttacentral.net/an1.187/en/sujato\#1.2}{AN 1.187:1.2}, \href{https://suttacentral.net/an5.132/en/sujato\#4.3}{AN 5.132:4.3}, and \href{https://suttacentral.net/sn8.7/en/sujato\#3.4}{SN 8.7:3.4}. This is a reference to the Buddha’s first teaching (\href{https://suttacentral.net/sn56.11/en/sujato}{SN 56.11}). } 

That\marginnote{23.2} is what the Buddha said. Satisfied, the mendicants approved what the Buddha said. 

%
\section*{{\suttatitleacronym MN 112}{\suttatitletranslation The Sixfold Purification }{\suttatitleroot Chabbisodhanasutta}}
\addcontentsline{toc}{section}{\tocacronym{MN 112} \toctranslation{The Sixfold Purification } \tocroot{Chabbisodhanasutta}}
\markboth{The Sixfold Purification }{Chabbisodhanasutta}
\extramarks{MN 112}{MN 112}

\scevam{So\marginnote{1.1} I have heard.\footnote{This sutta lays out a method for assessing a mendicant’s claim to realize perfection (\textit{arahatta}). They are taken through a series of questions of increasing complexity, culminating in the entire practice of the Gradual Training. } }At one time the Buddha was staying near \textsanskrit{Sāvatthī} in Jeta’s Grove, \textsanskrit{Anāthapiṇḍika}’s monastery. There the Buddha addressed the mendicants, “Mendicants!” 

“Venerable\marginnote{1.5} sir,” they replied. The Buddha said this: 

“Take\marginnote{2.1} a mendicant who declares enlightenment: ‘I understand: “Rebirth is ended, the spiritual journey has been completed, what had to be done has been done, there is nothing further for this place.”’ 

You\marginnote{3.1} should neither approve nor dismiss that mendicant’s statement.\footnote{As at \href{https://suttacentral.net/dn29/en/sujato\#18.4}{DN 29:18.4} and \href{https://suttacentral.net/dn16/en/sujato\#4.11.5}{DN 16:4.11.5} = \href{https://suttacentral.net/an4.180/en/sujato\#2.5}{AN 4.180:2.5}. } Rather, you should question them: ‘Reverend, these four kinds of expression have been rightly explained by the Blessed One, who knows and sees, the perfected one, the fully awakened Buddha. What four? One speaks of the seen as seen, the heard as heard, the thought as thought, and the known as known.\footnote{As usual, this group of four is not meant to stand for the six senses—which are discussed later—but for the ways in which one learns spiritual truths. That is why one “speaks of” what one has learned. See \href{https://suttacentral.net/mn1/en/sujato\#19.1}{MN 1:19.1} and note there. } These are the four kinds of expression rightly explained by the Blessed One, who knows and sees, the perfected one, the fully awakened Buddha. How does the venerable know and see regarding these four kinds of expression so that your mind is freed from defilements by not grasping?’ 

For\marginnote{4.1} a mendicant with defilements ended—who has completed the spiritual journey, done what had to be done, laid down the burden, achieved their own goal, utterly ended the fetter of continued existence, and is rightly freed through enlightenment—it is in line with the teaching to answer: ‘Regarding what is seen, reverends, I live without going near or going away, independent, untied, liberated, detached, my mind free of limits.\footnote{Also at \href{https://suttacentral.net/mn111/en/sujato\#4.5}{MN 111:4.5}. } Regarding what is heard … thought … or known, I live without going near or going away, independent, untied, liberated, detached, my mind free of limits. That is how I know and see regarding these four kinds of expression so that my mind is freed from defilements by not grasping.’ 

Saying\marginnote{5.1} ‘Good!’ you should applaud and cheer that mendicant’s statement, then ask a further question: 

‘Reverend,\marginnote{5.3} these five grasping aggregates have been rightly explained by the Buddha. What five? That is: the grasping aggregates of form, feeling, perception, choices, and consciousness. These are the five grasping aggregates that have been rightly explained by the Buddha. How does the venerable know and see regarding these five grasping aggregates so that your mind is freed from defilements by not grasping?’ 

For\marginnote{6.1} a mendicant with defilements ended it is in line with the teaching to answer: ‘Reverends, knowing that form is powerless, faded, and unreliable, I understand that my mind is freed through the ending, fading away, cessation, giving away, and letting go of attraction, grasping, mental fixation, insistence, and underlying tendency for form.\footnote{\textit{\textsanskrit{Virāgunaṁ}} is a secondary derivation from \textit{\textsanskrit{virāga}} with the \textit{-una} suffix. Also found at \href{https://suttacentral.net/iti77/en/sujato\#3.2}{Iti 77:3.2}. | \textit{\textsanskrit{Rāga}} can mean “color” or “desire”, so its opposite \textit{\textsanskrit{virāga}} has the dual connotations of “dispassion” and “colorless” or metaphorically “fading away”. \textsanskrit{Kauṭilya}’s \textsanskrit{Arthaśāstra} 2.10.58 gives an example in the context of writing. In those days, the letters would be etched in a palm leaf with a stylus, then “colored” by applying dye or ink. The surface would then be wiped clean, leaving just the ink staining the etched letters. An example of poorly done writing is \textit{\textsanskrit{virāga}}, “colorless”, i.e. the ink does not take hold, leaving only the near-invisible etching. } Knowing that feeling … perception … choices … consciousness is powerless, faded, and unreliable, I understand that my mind is freed through the ending, fading away, cessation, giving away, and letting go of attraction, grasping, mental fixation, insistence, and underlying tendency for consciousness. That is how I know and see regarding these five grasping aggregates so that my mind is freed from defilements by not grasping.’ 

Saying\marginnote{7.1} ‘Good!’ you should applaud and cheer that mendicant’s statement, then ask a further question: 

‘Reverend,\marginnote{7.3} these six elements have been rightly explained by the Buddha. What six? The elements of earth, water, fire, air, space, and consciousness. These are the six elements that have been rightly explained by the Buddha. How does the venerable know and see regarding these six elements so that your mind is freed from defilements by not grasping?’ 

For\marginnote{8.1} a mendicant with defilements ended it is in line with the teaching to answer: ‘Reverends, I’ve not taken the earth element as self, nor is there a self based on the earth element.\footnote{To take (\textit{upagacchi}) the earth element as self is to identify the self with the earth element. A self “based on” (\textit{nissita}) the earth element is one that is not itself the earth element, but which is dependent upon it, like a river on its bed. This would include identifying the self with other physical or mental properties. The logic is that, since the earth element is impermanent, etc., any self dependent on it must also be impermanent, etc. } And I understand that my mind is freed through the ending, fading away, cessation, giving away, and letting go of attraction, grasping, mental fixation, insistence, and underlying tendency based on the earth element. I’ve not taken the water element … fire element … air element … space element … consciousness element as self, nor is there a self based on the consciousness element. And I understand that my mind is freed through the ending of attraction based on the consciousness element. That is how I know and see regarding these six elements so that my mind is freed from defilements by not grasping.’ 

Saying\marginnote{9.1} ‘Good!’ you should applaud and cheer that mendicant’s statement, then ask a further question: 

‘Reverend,\marginnote{9.3} these six interior and exterior sense fields have been rightly explained by the Buddha. What six? The eye and sights, the ear and sounds, the nose and smells, the tongue and tastes, the body and touches, and the mind and ideas. These are the six interior and exterior sense fields that have been rightly explained by the Buddha. How does the venerable know and see regarding these six interior and exterior sense fields so that your mind is freed from defilements by not grasping?’ 

For\marginnote{10.1} a mendicant with defilements ended it is in line with the teaching to answer: ‘I understand that my mind is freed through the ending, fading away, cessation, giving away, and letting go of desire and greed and relishing and craving; attraction, grasping, mental fixation, insistence, and underlying tendency for the eye, sights, eye consciousness, and things knowable by eye consciousness.\footnote{The phrase “eye, sights, eye consciousness, and things knowable by eye consciousness” appears redundant, as “sights” (\textit{\textsanskrit{rūpa}}) are precisely what is “knowable by eye consciousness”. Where the phrase occurs elsewhere, \textit{\textsanskrit{rūpa}} may appear (\href{https://suttacentral.net/sn35.27/en/sujato\#1.3}{SN 35.27:1.3}, \href{https://suttacentral.net/sn35.65/en/sujato\#2.1}{SN 35.65:2.1}, \href{https://suttacentral.net/sn35.68/en/sujato\#1.3}{SN 35.68:1.3}) or it may not (\href{https://suttacentral.net/mn144/en/sujato\#9.1}{MN 144:9.1}, \href{https://suttacentral.net/sn35.87/en/sujato\#7.1}{SN 35.87:7.1}). Bodhi summarizes the commentary’s explanations, which he describes as “contrived” (\emph{Middle Length Discourses}, note 1080, and \emph{Connected Discourses} note 12 on book 4). The Chinese parallel omits \textit{\textsanskrit{rūpa}} here, thus avoiding the redundancy (MA 187 at T i 732c19). It is likely the original passage did not include \textit{\textsanskrit{rūpa}} and it was added in an old editorial oversight. } I understand that my mind is freed through the ending of desire for the ear … nose … tongue … body … mind, ideas, mind consciousness, and things knowable by mind consciousness. That is how I know and see regarding these six interior and exterior sense fields so that my mind is freed from defilements by not grasping.’ 

Saying\marginnote{11.1} ‘Good!’ you should applaud and cheer that mendicant’s statement, then ask a further question: 

‘Sir,\marginnote{11.3} how does the venerable know and see so that he has eradicated I-making, mine-making, and the underlying tendency to conceit for this conscious body and all external stimuli?’ 

For\marginnote{12.1} a mendicant with defilements ended it is in line with the teaching to answer: ‘Formerly, reverends, when I was still a layperson, I was ignorant. Then the Realized One or one of his disciples taught me the Dhamma. I gained faith in the Realized One, and reflected: 

“Life\marginnote{12.6} at home is cramped and dirty, life gone forth is wide open. It’s not easy for someone living at home to lead the spiritual life utterly full and pure, like a polished shell. Why don’t I shave off my hair and beard, dress in ocher robes, and go forth from lay life to homelessness?” 

After\marginnote{13.1} some time I gave up a large or small fortune, and a large or small family circle. I shaved off hair and beard, dressed in ocher robes, and went forth from the lay life to homelessness. Once I had gone forth, I took up the training and livelihood of the mendicants. I gave up killing living creatures, renouncing the rod and the sword. I was scrupulous and kind, living full of sympathy for all living beings. I gave up stealing. I took only what’s given, and expected only what’s given. I kept myself clean by not thieving. I gave up unchastity. I became celibate, set apart, avoiding the vulgar act of sex. I gave up lying. I spoke the truth and stuck to the truth. I was honest and dependable, not tricking the world with my words. I gave up divisive speech. I didn’t repeat in one place what I heard in another so as to divide people against each other. Instead, I reconciled those who are divided, supporting unity, delighting in harmony, loving harmony, speaking words that promote harmony. I gave up harsh speech. I spoke in a way that’s mellow, pleasing to the ear, lovely, going to the heart, polite, likable and agreeable to the people. I gave up talking nonsense. My words were timely, true, and meaningful, in line with the teaching and training. I said things at the right time which are valuable, reasonable, succinct, and beneficial. 

I\marginnote{14.1} refrained from injuring plants and seeds. I ate in one part of the day, abstaining from eating at night and food at the wrong time. I refrained from seeing shows of dancing, singing, and music . I refrained from beautifying and adorning myself with garlands, fragrance, and makeup. I refrained from high and luxurious beds. I refrained from receiving gold and currency, raw grains, raw meat, women and girls, male and female bondservants, goats and sheep, chicken and pigs, elephants, cows, horses, and mares, and fields and land. I refrained from running errands and messages; buying and selling; falsifying weights, metals, or measures; bribery, fraud, cheating, and duplicity; mutilation, murder, abduction, banditry, plunder, and violence. 

I\marginnote{14.19} became content with robes to look after the body and almsfood to look after the belly. Wherever I went, I set out taking only these things. Like a bird: wherever it flies, wings are its only burden. In the same way, I became content with robes to look after the body and almsfood to look after the belly. Wherever I went, I set out taking only these things. When I had this entire spectrum of noble ethics, I experienced a blameless happiness inside myself. 

When\marginnote{15.1} I saw a sight with my eyes, I didn’t get caught up in the features and details. If the faculty of sight were left unrestrained, bad unskillful qualities of covetousness and displeasure would become overwhelming. For this reason, I practiced restraint, protecting the faculty of sight, and achieving its restraint. When I heard a sound with my ears … When I smelled an odor with my nose … When I tasted a flavor with my tongue … When I felt a touch with my body … When I knew an idea with my mind, I didn’t get caught up in the features and details. If the faculty of the mind were left unrestrained, bad unskillful qualities of covetousness and displeasure would become overwhelming. For this reason, I practiced restraint, protecting the faculty of the mind, and achieving its restraint. When I had this noble sense restraint, I experienced an unsullied bliss inside myself. 

I\marginnote{16.1} acted with situational awareness when going out and coming back; when looking ahead and aside; when bending and extending the limbs; when bearing the outer robe, bowl and robes; when eating, drinking, chewing, and tasting; when urinating and defecating; when walking, standing, sitting, sleeping, waking, speaking, and keeping silent. 

When\marginnote{16.2} I had this entire spectrum of noble ethics, this noble contentment, this noble sense restraint, and this noble mindfulness and situational awareness, I frequented a secluded lodging—a wilderness, the root of a tree, a hill, a ravine, a mountain cave, a charnel ground, a forest, the open air, a heap of straw. After the meal, I returned from almsround, sat down cross-legged, set my body straight, and established mindfulness in his presence. 

Giving\marginnote{17.1} up covetousness for the world, I meditated with a heart rid of covetousness, cleansing the mind of covetousness. Giving up ill will and malevolence, I meditated with a mind rid of ill will, full of sympathy for all living beings, cleansing the mind of ill will. Giving up dullness and drowsiness, I meditated with a mind rid of dullness and drowsiness, perceiving light, mindful and aware, cleansing the mind of dullness and drowsiness. Giving up restlessness and remorse, I meditated without restlessness, my mind peaceful inside, cleansing the mind of restlessness and remorse. Giving up doubt, I meditated having gone beyond doubt, not undecided about skillful qualities, cleansing the mind of doubt. 

I\marginnote{18.1} gave up these five hindrances, corruptions of the heart that weaken wisdom. Then, quite secluded from sensual pleasures, secluded from unskillful qualities, I entered and remained in the first absorption, which has the rapture and bliss born of seclusion, while placing the mind and keeping it connected. As the placing of the mind and keeping it connected were stilled, I entered and remained in the second absorption … third absorption … fourth absorption. 

When\marginnote{19.1} my mind had immersed in \textsanskrit{samādhi} like this—purified, bright, flawless, rid of corruptions, pliable, workable, steady, and imperturbable—I extended it toward knowledge of the ending of defilements.\footnote{The text omits the usual description of the recollection of past lives and of clairvoyance here. } I truly understood: “This is suffering” … “This is the origin of suffering” … “This is the cessation of suffering” … “This is the practice that leads to the cessation of suffering”. I truly understood: “These are defilements”… “This is the origin of defilements” … “This is the cessation of defilements” … “This is the practice that leads to the cessation of defilements”. 

Knowing\marginnote{20.1} and seeing like this, my mind was freed from the defilements of sensuality, desire to be reborn, and ignorance. When it was freed, I knew it was freed. I understood: “Rebirth is ended; the spiritual journey has been completed; what had to be done has been done; there is nothing further for this place.” That is how I know and see so that I have eradicated I-making, mine-making, and the underlying tendency to conceit for this conscious body and all external stimuli.’ 

Saying\marginnote{21.1} ‘Good!’ you should applaud and cheer that mendicant’s statement, and then say to them: ‘We are fortunate, reverend, so very fortunate to see a venerable such as yourself as one of our spiritual companions!’” 

That\marginnote{21.5} is what the Buddha said. Satisfied, the mendicants approved what the Buddha said.\footnote{The title of the discourse refers to a “sixfold purity”, whereas the Pali text only has five items. The commentary offers a couple of explanations, one of which, attributed to “overseas Elders”, says the text should include a passage on the four nutriments. As it happens, the Chinese parallel does indeed include the four nutriments (MA 187 at T i 732b18), confirming the suggestion of the “overseas Elders” (\textsanskrit{Anālayo}, \emph{Comparative Study}, vol. ii, p. 638). Thus an accurate memory of the text persisted in the commentarial tradition, even when the text itself suffered loss. As to the identity of these Elders, the commentary to the \textsanskrit{Ghaṭikārasutta} (\href{https://suttacentral.net/mn81/en/sujato\#10.5}{MN 81:10.5}) says they taught the ten perfections (\textit{\textsanskrit{pāramī}}). These are a characteristic doctrine of the \textsanskrit{Mahāvihāra} (or \textsanskrit{Theravāda}) school, as other sectarian accounts of the perfections rarely number ten. So we can assume the Elders were not of a different school. Inscriptions at \textsanskrit{Amarāvatī} and \textsanskrit{Nāgārjunikoṇḍa} reveal that there was a \textsanskrit{Mahāvihāra} branch monastery in Andhra, so they may have lived there. } 

%
\section*{{\suttatitleacronym MN 113}{\suttatitletranslation A True Person }{\suttatitleroot Sappurisasutta}}
\addcontentsline{toc}{section}{\tocacronym{MN 113} \toctranslation{A True Person } \tocroot{Sappurisasutta}}
\markboth{A True Person }{Sappurisasutta}
\extramarks{MN 113}{MN 113}

\scevam{So\marginnote{1.1} I have heard. }At one time the Buddha was staying near \textsanskrit{Sāvatthī} in Jeta’s Grove, \textsanskrit{Anāthapiṇḍika}’s monastery. There the Buddha addressed the mendicants, “Mendicants!” 

“Venerable\marginnote{1.5} sir,” they replied. The Buddha said this: 

“Mendicants,\marginnote{2.1} I will teach you the qualities of a true person and the qualities of an untrue person. Listen and apply your mind well, I will speak.” 

“Yes,\marginnote{2.3} sir,” they replied. The Buddha said this: 

“And\marginnote{3.1} what is a quality of an untrue person? Take an untrue person who has gone forth from an eminent family. They reflect: ‘I have gone forth from an eminent family, unlike these other mendicants.’ And they glorify themselves and put others down on account of that. This is a quality of an untrue person. A true person reflects: ‘It’s not because of one’s eminent family that thoughts of greed, hate, or delusion come to an end. Even if someone has not gone forth from an eminent family, if they practice in line with the teaching, practice properly, and live in line with the teaching, they are worthy of honor and praise for that.’ Keeping the practice to themselves, they don’t glorify themselves and put others down on account of their eminent family.\footnote{By rejecting social measures of worth and focusing on inner purity, they are the opposite of the meditator at \href{https://suttacentral.net/mn108/en/sujato\#26.4}{MN 108:26.4}, who hides defilements within. } This is a quality of a true person. 

Furthermore,\marginnote{4{-}6.1} take an untrue person who has gone forth from a great family … from a wealthy family … from an extremely wealthy family. They reflect: ‘I have gone forth from an extremely wealthy family, unlike these other mendicants.’ And they glorify themselves and put others down on account of that. This too is a quality of an untrue person. A true person reflects: ‘It’s not because of one’s extremely wealthy family that thoughts of greed, hate, or delusion come to an end. Even if someone has not gone forth from an extremely wealthy family, if they practice in line with the teaching, practice properly, and live in line with the teaching, they are worthy of honor and praise for that.’ Keeping the practice to themselves, they don’t glorify themselves and put others down on account of their extremely wealthy family. This too is a quality of a true person. 

Furthermore,\marginnote{7.1} take an untrue person who is well-known and famous. They reflect: ‘I’m well-known and famous. These other mendicants are obscure and insignificant.’ And they glorify themselves and put others down on account of that. This too is a quality of an untrue person. A true person reflects: ‘It’s not because of one’s fame that thoughts of greed, hate, or delusion come to an end. Even if someone is not well-known and famous, if they practice in line with the teaching, practice properly, and live in line with the teaching, they are worthy of honor and praise for that.’ Keeping the practice to themselves, they don’t glorify themselves and put others down on account of their fame. This too is a quality of a true person. 

Furthermore,\marginnote{8.1} take an untrue person who receives robes, almsfood, lodgings, and medicines and supplies for the sick. They reflect: ‘I receive robes, almsfood, lodgings, and medicines and supplies for the sick, unlike these other mendicants.’ And they glorify themselves and put others down on account of that. This too is a quality of an untrue person. A true person reflects: ‘It’s not because of one’s material things that thoughts of greed, hate, or delusion come to an end. Even if someone doesn’t receive robes, almsfood, lodgings, and medicines and supplies for the sick, if they practice in line with the teaching, practice properly, and live in line with the teaching, they are worthy of honor and praise for that.’ Keeping the practice to themselves, they don’t glorify themselves and put others down on account of their material things. This too is a quality of a true person. 

Furthermore,\marginnote{9.1} take an untrue person who is very learned … who is an expert in the monastic law … who is a Dhamma teacher … who dwells in the wilderness … who is a rag robe wearer … who eats only almsfood … who stays at the root of a tree … who stays in a charnel ground … who stays in the open air … who never lies down … who sleeps wherever they lay their mat … who eats in one sitting per day. They reflect: ‘I eat in one sitting per day, unlike these other mendicants.’ And they glorify themselves and put others down on account of that. This too is a quality of an untrue person. A true person reflects: ‘It’s not because of eating in one sitting per day that thoughts of greed, hate, or delusion come to an end. Even if someone eats in more than one sitting per day, if they practice in line with the teaching, practice properly, and live in line with the teaching, they are worthy of honor and praise for that.’ Keeping the practice to themselves, they don’t glorify themselves and put others down on account of their eating in one sitting per day. This too is a quality of a true person. 

Furthermore,\marginnote{21.1} take an untrue person who, quite secluded from sensual pleasures, secluded from unskillful qualities, enters and remains in the first absorption, which has the rapture and bliss born of seclusion, while placing the mind and keeping it connected. They reflect: ‘I have attained the first absorption, unlike these other mendicants.’ And they glorify themselves and put others down on account of that. This too is a quality of an untrue person. A true person reflects: ‘The Buddha has spoken of not being determined even by the attainment of the first absorption.\footnote{For \textit{\textsanskrit{atammayatā}} see \href{https://suttacentral.net/mn47/en/sujato\#13.4}{MN 47:13.4} and my note there. Here it means that one’s spiritual goal is not limited to this attainment. } For whatever they imagine it is, it turns out to be something else.’\footnote{This saying also appears at \href{https://suttacentral.net/snp3.8/en/sujato\#15.1}{Snp 3.8:15.1} and \href{https://suttacentral.net/snp3.12/en/sujato\#52.1}{Snp 3.12:52.1}, where it is not connected with absorption practice. } Keeping to themselves the fact that they are not determined by that, they don’t glorify themselves and put others down on account of their attainment of the first absorption. This too is a quality of a true person. 

Furthermore,\marginnote{22{-}24.1} take an untrue person who, as the placing of the mind and keeping it connected are stilled, enters and remains in the second absorption … third absorption … fourth absorption. They reflect: ‘I have attained the fourth absorption, unlike these other mendicants.’ And they glorify themselves and put others down on account of that. This too is a quality of an untrue person. A true person reflects: ‘The Buddha has spoken of not being determined even by the attainment of the fourth absorption. For whatever they imagine it is, it turns out to be something else.’ Keeping to themselves the fact that they are not determined by that, they don’t glorify themselves and put others down on account of their attainment of the fourth absorption. This too is a quality of a true person. 

Furthermore,\marginnote{25.1} take an untrue person who, going totally beyond perceptions of form, with the ending of perceptions of impingement, not focusing on perceptions of diversity, aware that ‘space is infinite’, enters and remains in the dimension of infinite space … the dimension of infinite consciousness … the dimension of nothingness … the dimension of neither perception nor non-perception. They reflect: ‘I have attained the dimension of neither perception nor non-perception, unlike these other mendicants.’ And they glorify themselves and put others down on account of that. This too is a quality of an untrue person. A true person reflects: ‘The Buddha has spoken of not being determined even by the attainment of the dimension of neither perception nor non-perception. For whatever they imagine it is, it turns out to be something else.’ Keeping to themselves the fact that they are not determined by that, they don’t glorify themselves and put others down on account of their attainment of the dimension of neither perception nor non-perception. This too is a quality of a true person. 

Furthermore,\marginnote{29.1} take a true person who, going totally beyond the dimension of neither perception nor non-perception, enters and remains in the cessation of perception and feeling. And, having seen with wisdom, their defilements come to an end. This is a mendicant who does not conceive with anything, does not conceive regarding anything, does not conceive through anything.” 

That\marginnote{29.3} is what the Buddha said. Satisfied, the mendicants approved what the Buddha said. 

%
\section*{{\suttatitleacronym MN 114}{\suttatitletranslation What Should and Should Not Be Cultivated }{\suttatitleroot Sevitabbāsevitabbasutta}}
\addcontentsline{toc}{section}{\tocacronym{MN 114} \toctranslation{What Should and Should Not Be Cultivated } \tocroot{Sevitabbāsevitabbasutta}}
\markboth{What Should and Should Not Be Cultivated }{Sevitabbāsevitabbasutta}
\extramarks{MN 114}{MN 114}

\scevam{So\marginnote{1.1} I have heard.\footnote{The Buddha begins by speaking of what should and should not be cultivated. Then he goes on to offer three distinct analyses of this, each laid out in a series of dichotomies. After each one, \textsanskrit{Sāriputta} offers a lengthy elaboration, which the Buddha then approves and recapitulates. } }At one time the Buddha was staying near \textsanskrit{Sāvatthī} in Jeta’s Grove, \textsanskrit{Anāthapiṇḍika}’s monastery. There the Buddha addressed the mendicants, “Mendicants!” 

“Venerable\marginnote{1.5} sir,” they replied. The Buddha said this: 

“Mendicants,\marginnote{2.1} I will teach you an exposition of the teaching on what should and should not be cultivated.\footnote{To “cultivate” is to foster or nurture through repeated association or practice. } Listen and apply your mind well, I will speak.” 

“Yes,\marginnote{2.3} sir,” they replied. The Buddha said this: 

“I\marginnote{3.1} say that there are two kinds of bodily behavior:\footnote{The first analysis consists of seven items: bodily, verbal, and mental conduct; arising of thought; and acquisition of perception, views, and reincarnation. } that which you should cultivate, and that which you should not cultivate. And each of these is a kind of bodily behavior. 

I\marginnote{3.4} say that there are two kinds of verbal behavior: that which you should cultivate, and that which you should not cultivate. And each of these is a kind of verbal behavior. 

I\marginnote{3.7} say that there are two kinds of mental behavior: that which you should cultivate, and that which you should not cultivate. And each of these is a kind of mental behavior. 

I\marginnote{3.10} say that there are two arisings of thought: that which you should cultivate, and that which you should not cultivate. And each of these is an arising of thought. 

I\marginnote{3.13} say that there are two acquisitions of perception: that which you should cultivate, and that which you should not cultivate. And each of these is an acquisition of perception. 

I\marginnote{3.16} say that there are two acquisitions of views: that which you should cultivate, and that which you should not cultivate. And each of these is an acquisition of views. 

I\marginnote{3.19} say that there are two kinds of reincarnation in a life-form:\footnote{“Life-form” is \textit{\textsanskrit{attabhāva}}, literally the “state of the self”, where the “self” comes close in meaning to “body”—whether material or immaterial—which is a secondary sense of \textit{\textsanskrit{ātman}} in the \textsanskrit{Upaniṣads}. Accordingly, the commentary to \href{https://suttacentral.net/an5.100/en/sujato}{AN 5.100} glosses as \textit{\textsanskrit{sarīrapaṭilābha}}. | “Reincarnation” is \textit{\textsanskrit{paṭilābha}}. In the 20th century, Buddhist writers shied away from using “reincarnation”, which they associated with the Hindu idea of a transmigrating soul. But there is nothing in the word “reincarnation” that implies a soul. Rather, it simply refers to going into another body, which is precisely what is meant by \textit{\textsanskrit{attabhāva}}. It is unwise to hang a heavy philosophical cloak upon the feeble hook of a single word. } that which you should cultivate, and that which you should not cultivate. And these are equally reincarnation in a life-form.” 

When\marginnote{4.1} he said this, Venerable \textsanskrit{Sāriputta} said to the Buddha,\footnote{\textsanskrit{Sāriputta} proceeds to elaborate each of the seven items in turn. } “Sir, this is how I understand the detailed meaning of the Buddha’s brief statement. 

‘I\marginnote{5.1} say that there are two kinds of bodily behavior: that which you should cultivate, and that which you should not cultivate. And each of these is a kind of bodily behavior.’ That’s what the Buddha said, but why did he say it? You should not cultivate the kind of bodily behavior which causes unskillful qualities to grow while skillful qualities decline. And you should cultivate the kind of bodily behavior which causes unskillful qualities to decline while skillful qualities grow. 

And\marginnote{5.8} what kind of bodily behavior causes unskillful qualities to grow while skillful qualities decline?\footnote{This includes the first three precepts, or else the first three of the ten ways of skillful deeds. } It’s when someone kills living creatures. They’re violent, bloody-handed, a hardened killer, merciless to living beings. They steal. With the intention to commit theft, they take the wealth or belongings of others from village or wilderness. They commit sexual misconduct. They have sexual relations with women who have their mother, father, both mother and father, brother, sister, relatives, or clan as guardian. They have sexual relations with a woman who is protected on principle, or who has a husband, or whose violation is punishable by law, or even one who has been garlanded as a token of betrothal. That kind of bodily behavior causes unskillful qualities to grow while skillful qualities decline. 

And\marginnote{5.13} what kind of bodily behavior causes unskillful qualities to decline while skillful qualities grow? It’s when someone gives up killing living creatures. They renounce the rod and the sword. They’re scrupulous and kind, living full of sympathy for all living beings. They give up stealing. They don’t, with the intention to commit theft, take the wealth or belongings of others from village or wilderness. They give up sexual misconduct. They don’t have sexual relations with women who have their mother, father, both mother and father, brother, sister, relatives, or clan as guardian. They don’t have sexual relations with a woman who is protected on principle, or who has a husband, or whose violation is punishable by law, or even one who has been garlanded as a token of betrothal. That kind of bodily behavior causes unskillful qualities to decline while skillful qualities grow. ‘I say that there are two kinds of bodily behavior: that which you should cultivate, and that which you should not cultivate. And each of these is a kind of bodily behavior.’ That’s what the Buddha said, and this is why he said it. 

‘I\marginnote{6.1} say that there are two kinds of verbal behavior: that which you should cultivate, and that which you should not cultivate. And each of these is a kind of verbal behavior.’ That’s what the Buddha said, but why did he say it? You should not cultivate the kind of verbal behavior which causes unskillful qualities to grow while skillful qualities decline. And you should cultivate the kind of verbal behavior which causes unskillful qualities to decline while skillful qualities grow. 

And\marginnote{6.8} what kind of verbal behavior causes unskillful qualities to grow while skillful qualities decline?\footnote{This is the four verbal ways of skillful deeds. } It’s when someone lies. They’re summoned to a council, an assembly, a family meeting, a guild, or to the royal court, and asked to bear witness: ‘Please, mister, say what you know.’ Not knowing, they say ‘I know.’ Knowing, they say ‘I don’t know.’ Not seeing, they say ‘I see.’ And seeing, they say ‘I don’t see.’ So they deliberately lie for the sake of themselves or another, or for some trivial worldly reason. They speak divisively. They repeat in one place what they heard in another so as to divide people against each other. And so they divide those who are harmonious, supporting division, delighting in division, loving division, speaking words that promote division. They speak harshly. They use the kinds of words that are cruel, nasty, hurtful, offensive, bordering on anger, not leading to immersion. They talk nonsense. Their speech is untimely, and is neither factual nor beneficial. It has nothing to do with the teaching or the training. Their words have no value, and are untimely, unreasonable, rambling, and pointless. That kind of verbal behavior causes unskillful qualities to grow while skillful qualities decline. 

And\marginnote{6.14} what kind of verbal behavior causes unskillful qualities to decline while skillful qualities grow? It’s when a certain person gives up lying. They’re summoned to a council, an assembly, a family meeting, a guild, or to the royal court, and asked to bear witness: ‘Please, mister, say what you know.’ Not knowing, they say ‘I don’t know.’ Knowing, they say ‘I know.’ Not seeing, they say ‘I don’t see.’ And seeing, they say ‘I see.’ So they don’t deliberately lie for the sake of themselves or another, or for some trivial worldly reason. They give up divisive speech. They don’t repeat in one place what they heard in another so as to divide people against each other. Instead, they reconcile those who are divided, supporting unity, delighting in harmony, loving harmony, speaking words that promote harmony. They give up harsh speech. They speak in a way that’s mellow, pleasing to the ear, lovely, going to the heart, polite, likable and agreeable to the people. They give up talking nonsense. Their words are timely, true, and meaningful, in line with the teaching and training. They say things at the right time which are valuable, reasonable, succinct, and beneficial. That kind of verbal behavior causes unskillful qualities to decline while skillful qualities grow. ‘I say that there are two kinds of verbal behavior: that which you should cultivate, and that which you should not cultivate. And each of these is a kind of verbal behavior.’ That’s what the Buddha said, and this is why he said it. 

‘I\marginnote{7.1} say that there are two kinds of mental behavior: that which you should cultivate, and that which you should not cultivate. And each of these is a kind of mental behavior.’ That’s what the Buddha said, but why did he say it? You should not cultivate the kind of mental behavior which causes unskillful qualities to grow while skillful qualities decline. And you should cultivate the kind of mental behavior which causes unskillful qualities to decline while skillful qualities grow. 

And\marginnote{7.8} what kind of mental behavior causes unskillful qualities to grow while skillful qualities decline?\footnote{This is the final three ways of skillful deeds. } It’s when someone is covetous. They covet the wealth and belongings of others: ‘Oh, if only their belongings were mine!’ They have ill will and malicious intentions: ‘May these sentient beings be killed, slaughtered, slain, destroyed, or annihilated!’ That kind of mental behavior causes unskillful qualities to grow while skillful qualities decline. 

And\marginnote{7.12} what kind of mental behavior causes unskillful qualities to decline while skillful qualities grow? It’s when someone is content. They don’t covet the wealth and belongings of others: ‘Oh, if only their belongings were mine!’ They have a kind heart and loving intentions: ‘May these sentient beings live free of enmity and ill will, untroubled and happy!’ That kind of mental behavior causes unskillful qualities to decline while skillful qualities grow. ‘I say that there are two kinds of mental behavior: that which you should cultivate, and that which you should not cultivate. And each of these is a kind of mental behavior.’ That’s what the Buddha said, and this is why he said it. 

‘I\marginnote{8.1} say that there are two arisings of thought: that which you should cultivate, and that which you should not cultivate. And each of these is an arising of thought.’ That’s what the Buddha said, but why did he say it? You should not cultivate the arising of thought which causes unskillful qualities to grow while skillful qualities decline. And you should cultivate the arising of thought which causes unskillful qualities to decline while skillful qualities grow. 

And\marginnote{8.8} what arising of thought causes unskillful qualities to grow while skillful qualities decline?\footnote{The discussions at \href{https://suttacentral.net/mn8/en/sujato\#13.1}{MN 8:13.1} and \href{https://suttacentral.net/an7.53/en/sujato}{AN 7.53} show that this refers to the initial thought that precedes action. } It’s when someone is covetous, and lives with their heart full of covetousness. They are malicious, and live with their heart full of ill will. They’re hurtful, and live with their heart intent on harm. That arising of thought causes unskillful qualities to grow while skillful qualities decline. 

And\marginnote{8.13} what arising of thought causes unskillful qualities to decline while skillful qualities grow? It’s when someone is content, and lives with their heart full of contentment. They have good will, and live with their heart full of good will. They’re kind, and live with their heart full of kindness. That arising of thought causes unskillful qualities to decline while skillful qualities grow. ‘I say that there are two arisings of thought: that which you should cultivate, and that which you should not cultivate. And each of these is an arising of thought.’ That’s what the Buddha said, and this is why he said it. 

‘I\marginnote{9.1} say that there are two acquisitions of perception: that which you should cultivate, and that which you should not cultivate. And each of these is an acquisition of perception.’ That’s what the Buddha said, but why did he say it? You should not cultivate the acquisition of perception which causes unskillful qualities to grow while skillful qualities decline. And you should cultivate the acquisition of perception which causes unskillful qualities to decline while skillful qualities grow. 

And\marginnote{9.8} what acquisition of perception causes unskillful qualities to grow while skillful qualities decline?\footnote{Perception precedes even thought. } It’s when someone is covetous, and lives with their perception full of covetousness. They are malicious, and live with their perception full of ill will. They’re hurtful, and live with their perception intent on harm. That acquisition of perception causes unskillful qualities to grow while skillful qualities decline. 

And\marginnote{9.13} what acquisition of perception causes unskillful qualities to decline while skillful qualities grow? It’s when someone is content, and lives with their perception full of contentment. They have good will, and live with their perception full of good will. They’re kind, and live with their perception full of kindness. That acquisition of perception causes unskillful qualities to decline while skillful qualities grow. ‘I say that there are two acquisitions of perception: that which you should cultivate, and that which you should not cultivate. And each of these is an acquisition of perception.’ That’s what the Buddha said, and this is why he said it. 

‘I\marginnote{10.1} say that there are two acquisitions of views: that which you should cultivate, and that which you should not cultivate. And each of these is an acquisition of views.’ That’s what the Buddha said, but why did he say it? You should not cultivate the acquisition of views which causes unskillful qualities to grow while skillful qualities decline. And you should cultivate the acquisition of views which causes unskillful qualities to decline while skillful qualities grow. 

And\marginnote{10.8} what acquisition of views causes unskillful qualities to grow while skillful qualities decline?\footnote{A view is a perception solidified into a persistent pattern of thought, which becomes a way of seeing the world. View offer explanatory power, provide us with a convenient template for understanding, and often dismissing, what we encounter. Once a view has set in, it tends to reinforce itself, and hence is very slow to change. } It’s when someone has such a view: ‘There’s no meaning in giving, sacrifice, or offerings. There’s no fruit or result of good and bad deeds. There’s no afterlife. There’s no such thing as mother and father, or beings that are reborn spontaneously. And there’s no ascetic or brahmin who is rightly comported and rightly practiced, and who describes the afterlife after realizing it with their own insight.’ That acquisition of views causes unskillful qualities to grow while skillful qualities decline. 

And\marginnote{10.12} what acquisition of views causes unskillful qualities to decline while skillful qualities grow? It’s when someone has such a view: ‘There is meaning in giving, sacrifice, and offerings. There are fruits and results of good and bad deeds. There is an afterlife. There are such things as mother and father, and beings that are reborn spontaneously. And there are ascetics and brahmins who are rightly comported and rightly practiced, and who describe the afterlife after realizing it with their own insight.’ That acquisition of views causes unskillful qualities to decline while skillful qualities grow. ‘I say that there are two acquisitions of views: that which you should cultivate, and that which you should not cultivate. And each of these is an acquisition of views.’ That’s what the Buddha said, and this is why he said it. 

‘I\marginnote{11.1} say that there are two kinds of reincarnation in a life-form: that which you should cultivate, and that which you should not cultivate. And these are equally reincarnation in a life-form.’ That’s what the Buddha said, but why did he say it? Reincarnation in a life-form that causes unskillful qualities to grow while skillful qualities decline: you should not cultivate reincarnation in such a life-form. Reincarnation in a life-form that causes unskillful qualities to decline while skillful qualities grow: you should cultivate reincarnation such in a life-form. 

And reincarnation\marginnote{11.10} in what kind of life-form causes unskillful qualities to grow while skillful qualities decline?\footnote{As the examples will show, this refers to a kind of rebirth that is not conducive to developing wholesome qualities. } For one who generates reincarnation in a hurtful life-form not for the sake of the state of perfection, unskillful qualities grow while skillful qualities decline.\footnote{\textit{\textsanskrit{Apariniṭṭhitabhāvāya}} (“not for the sake of the state of perfection”) is found only here. The commentary explains it in terms of individuals who are reborn in a painful state where they are unable to realize the accomplishment of liberation. } And reincarnation in what kind of life-form causes unskillful qualities to decline while skillful qualities grow? For one who generates rebirth in a pleasing life-form for the sake of the state of perfection, unskillful qualities decline while skillful qualities grow.\footnote{The question is absent in the \textsanskrit{Mahāsaṅgīti} edition, but present in the PTS and BJT editions. } ‘I say that there are two kinds of reincarnation in a life-form: that which you should cultivate, and that which you should not cultivate. And these are equally reincarnation in a life-form.’ That’s what the Buddha said, and this is why he said it. 

Sir,\marginnote{12.1} that’s how I understand the detailed meaning of the Buddha’s brief statement.”\footnote{This completes the elaboration of the seven items. } 

“Good,\marginnote{13.1} good, \textsanskrit{Sāriputta}! It’s good that you understand the detailed meaning of my brief statement in this way.” 

And\marginnote{14.1} the Buddha went on to repeat and endorse Venerable \textsanskrit{Sāriputta}’s explanation in full. Then he went on to explain further: 

“I\marginnote{22.1} say that there are two kinds of sight known by the eye:\footnote{Now the Buddha introduces the second analysis, in terms of the six senses. } that which you should cultivate, and that which you should not cultivate. I say that there are two kinds of sound known by the ear … two kinds of smell known by the nose … two kinds of taste known by the tongue … two kinds of touch known by the body … two kinds of idea known by the mind: that which you should cultivate, and that which you should not cultivate.” 

When\marginnote{23.1} he said this, Venerable \textsanskrit{Sāriputta} said to the Buddha: 

“Sir,\marginnote{23.2} this is how I understand the detailed meaning of the Buddha’s brief statement.\footnote{\textsanskrit{Sāriputta} proceeds to elaborate these six items in terms of what gives rise to the skillful and unskillful. } 

‘I\marginnote{24.1} say that there are two kinds of sight known by the eye: that which you should cultivate, and that which you should not cultivate.’ That’s what the Buddha said, but why did he say it? You should not cultivate the kind of sight known by the eye which causes unskillful qualities to grow while skillful qualities decline. And you should cultivate the kind of sight known by the eye which causes unskillful qualities to decline while skillful qualities grow. ‘I say that there are two kinds of sight known by the eye: that which you should cultivate, and that which you should not cultivate.’ That’s what the Buddha said, and this is why he said it. 

‘I\marginnote{25.1} say that there are two kinds of sound known by the ear … two kinds of smell known by the nose … two kinds of taste known by the tongue … two kinds of touch known by the body … two kinds of idea known by the mind: that which you should cultivate, and that which you should not cultivate.’ That’s what the Buddha said, but why did he say it? 

You\marginnote{30.1} should not cultivate the kind of idea known by the mind which causes unskillful qualities to grow while skillful qualities decline. And you should cultivate the kind of idea known by the mind which causes unskillful qualities to decline while skillful qualities grow. ‘I say that there are two kinds of idea known by the mind: that which you should cultivate, and that which you should not cultivate.’ That’s what the Buddha said, and this is why he said it. Sir, that’s how I understand the detailed meaning of the Buddha’s brief statement.” 

“Good,\marginnote{31.1} good, \textsanskrit{Sāriputta}! It’s good that you understand the detailed meaning of my brief statement in this way.” 

And\marginnote{32{-}36.1} the Buddha went on to repeat and endorse Venerable \textsanskrit{Sāriputta}’s explanation in full. Then he went on to explain further: 

“I\marginnote{39.1} say that there are two kinds of robes:\footnote{The Buddha introduces the third and final analysis, which shifts to the requisites and residence of a monastic. } that which you should cultivate, and that which you should not cultivate. I say that there are two kinds of almsfood … lodging … village … town … city … country … person: that which you should cultivate, and that which you should not cultivate.” 

When\marginnote{40.1} he said this, Venerable \textsanskrit{Sāriputta} said to the Buddha: 

“Sir,\marginnote{40.2} this is how I understand the detailed meaning of the Buddha’s brief statement.\footnote{Once again \textsanskrit{Sāriputta} elaborates. } ‘I say that there are two kinds of robes … almsfood … lodging … village … town … city … country … person: that which you should cultivate, and that which you should not cultivate.’ That’s what the Buddha said, but why did he say it? You should not cultivate the kind of person who causes unskillful qualities to grow while skillful qualities decline. And you should cultivate the kind of person who causes unskillful qualities to decline while skillful qualities grow. ‘I say that there are two kinds of person: those who you should cultivate, and those who you should not cultivate.’ That’s what the Buddha said, and this is why he said it. 

Sir,\marginnote{49.1} that’s how I understand the detailed meaning of the Buddha’s brief statement.” 

“Good,\marginnote{50.1} good, \textsanskrit{Sāriputta}! It’s good that you understand the detailed meaning of my brief statement in this way.” 

And\marginnote{51{-}58.1} the Buddha went on to repeat and endorse Venerable \textsanskrit{Sāriputta}’s explanation in full. Then he added: 

“If\marginnote{60.1} all the aristocrats, brahmins, peasants, and menials were to understand the detailed meaning of my brief statement in this way, it would be for their lasting welfare and happiness. If the whole world—with its gods, \textsanskrit{Māras}, and divinities, this population with its ascetics and brahmins, gods and humans—was to understand the detailed meaning of my brief statement in this way, it would be for the whole world’s lasting welfare and happiness.” 

That\marginnote{61.1} is what the Buddha said. Satisfied, Venerable \textsanskrit{Sāriputta} approved what the Buddha said. 

%
\section*{{\suttatitleacronym MN 115}{\suttatitletranslation Many Elements }{\suttatitleroot Bahudhātukasutta}}
\addcontentsline{toc}{section}{\tocacronym{MN 115} \toctranslation{Many Elements } \tocroot{Bahudhātukasutta}}
\markboth{Many Elements }{Bahudhātukasutta}
\extramarks{MN 115}{MN 115}

\scevam{So\marginnote{1.1} I have heard. }At one time the Buddha was staying near \textsanskrit{Sāvatthī} in Jeta’s Grove, \textsanskrit{Anāthapiṇḍika}’s monastery. There the Buddha addressed the mendicants, “Mendicants!” 

“Venerable\marginnote{1.5} sir,” they replied. The Buddha said this: 

“Whatever\marginnote{2.1} dangers there are, all come from the foolish, not from the astute.\footnote{\textit{\textsanskrit{Paṇḍita}} is normally used for a learned scholar, rather than a contemplative “sage”. Pali has many more terms for “a wise person” than does English, and it is not easy to distinguish the senses. “Astute” is meant to convey the sense of one who is observant and of sound judgment. } Whatever perils there are, all come from the foolish, not from the astute. Whatever hazards there are, all come from the foolish, not from the astute. It’s like a fire that spreads from a hut made of reeds or grass, and burns down even a bungalow, plastered inside and out, draft-free, with doors fastened and windows shuttered. In the same way, whatever dangers there are, all come from the foolish, not from the astute. Whatever perils there are, all come from the foolish, not from the astute. Whatever hazards there are, all come from the foolish, not from the astute. So, the fool is dangerous, but the astute person is safe. The fool is perilous, but the astute person is not. The fool is hazardous, but the astute person is not. There’s no danger, peril, or hazard that comes from the astute. So you should train like this: ‘We shall be astute, we shall be inquirers.’” 

When\marginnote{3.1} he said this, Venerable Ānanda said to the Buddha, “Sir, how is a mendicant qualified to be called ‘astute, an inquirer’?”\footnote{Notice the different approaches used by Ānanda and \textsanskrit{Sāriputta} in the previous discourse. \textsanskrit{Sāriputta} took up the Buddha’s teaching and elaborated it, whereas Ānanda uses inquiry to prompt the Buddha himself to speak further. } 

“Ānanda,\marginnote{3.3} it’s when a mendicant is skilled in the elements, in the sense fields, in dependent origination, and in the possible and the impossible. That’s how a mendicant is qualified to be called ‘astute, an inquirer’.” 

“But\marginnote{4.1} sir, how is a mendicant qualified to be called ‘skilled in the elements’?”\footnote{As emphasized in this sutta, the notion of an “element” (\textit{\textsanskrit{dhātu}}) is a flexible one, including a range of different properties, qualities, or aspects of existence. } 

“There\marginnote{4.2} are, Ānanda, these eighteen elements: the elements of the eye, sights, and eye consciousness; the ear, sounds, and ear consciousness; the nose, smells, and nose consciousness; the tongue, tastes, and tongue consciousness; the body, touches, and body consciousness; the mind, ideas, and mind consciousness.\footnote{The “mind element” (\textit{\textsanskrit{manodhātu}}) appears a number of times (\href{https://suttacentral.net/sn14.1/en/sujato\#2.2}{SN 14.1:2.2}, \href{https://suttacentral.net/sn35.129/en/sujato\#1.17}{SN 35.129:1.17}, \href{https://suttacentral.net/sn41.2/en/sujato\#3.11}{SN 41.2:3.11}), but is not defined. See \href{https://suttacentral.net/mn28/en/sujato\#37.1}{MN 28:37.1} and note there. } When a mendicant knows and sees these eighteen elements, they’re qualified to be called ‘skilled in the elements’.” 

“But\marginnote{5.1} sir, could there be another way in which a mendicant is qualified to be called ‘skilled in the elements’?”\footnote{The method of building a lengthy discourse by asking for “another way” of exposition is also used at \href{https://suttacentral.net/mn9/en/sujato}{MN 9}, \href{https://suttacentral.net/sn44.6/en/sujato}{SN 44.6}, \href{https://suttacentral.net/an5.106/en/sujato}{AN 5.106}, and \href{https://suttacentral.net/snp3.12/en/sujato}{Snp 3.12}. } 

“There\marginnote{5.2} could, Ānanda. There are these six elements: the elements of earth, water, fire, air, space, and consciousness. When a mendicant knows and sees these six elements, they’re qualified to be called ‘skilled in the elements’.” 

“But\marginnote{6.1} sir, could there be another way in which a mendicant is qualified to be called ‘skilled in the elements’?” 

“There\marginnote{6.2} could, Ānanda. There are these six elements: the elements of pleasure, pain, happiness, sadness, equanimity, and ignorance. When a mendicant knows and sees these six elements, they’re qualified to be called ‘skilled in the elements’.” 

“But\marginnote{7.1} sir, could there be another way in which a mendicant is qualified to be called ‘skilled in the elements’?” 

“There\marginnote{7.2} could, Ānanda. There are these six elements: the elements of sensuality and renunciation, malice and good will, and cruelty and harmlessness. When a mendicant knows and sees these six elements, they’re qualified to be called ‘skilled in the elements’.” 

“But\marginnote{8.1} sir, could there be another way in which a mendicant is qualified to be called ‘skilled in the elements’?” 

“There\marginnote{8.2} could, Ānanda. There are these three elements: the elements of the sensual realm, the realm of luminous form, and the formless realm. When a mendicant knows and sees these three elements, they’re qualified to be called ‘skilled in the elements’.” 

“But\marginnote{9.1} sir, could there be another way in which a mendicant is qualified to be called ‘skilled in the elements’?” 

“There\marginnote{9.2} could, Ānanda. There are these two elements: the conditioned element and the unconditioned element. When a mendicant knows and sees these two elements, they’re qualified to be called ‘skilled in the elements’.” 

“But\marginnote{10.1} sir, how is a mendicant qualified to be called ‘skilled in the sense fields’?”\footnote{Ānanda moves on to the next topic, the six senses. Perhaps this move was prompted by the mention of the “unconditioned element” of Nibbana, signifying that that method of analysis had reached its culmination. } 

“There\marginnote{10.2} are these six interior and exterior sense fields: the eye and sights, the ear and sounds, the nose and smells, the tongue and tastes, the body and touches, and the mind and ideas. When a mendicant knows and sees these six interior and exterior sense fields, they’re qualified to be called ‘skilled in the sense fields’.” 

“But\marginnote{11.1} sir, how is a mendicant qualified to be called ‘skilled in dependent origination’?” 

“It’s\marginnote{11.2} when a mendicant understands: ‘When this exists, that is; due to the arising of this, that arises. When this doesn’t exist, that is not; due to the cessation of this, that ceases. That is: ignorance is a condition for choices. Choices are conditions for consciousness. Consciousness is a condition for name and form. Name and form are conditions for the six sense fields. The six sense fields are conditions for contact. Contact is a condition for feeling. Feeling is a condition for craving. Craving is a condition for grasping. Grasping is a condition for continued existence. Continued existence is a condition for rebirth. Rebirth is a condition for old age and death, sorrow, lamentation, pain, sadness, and distress to come to be. That is how this entire mass of suffering originates. When ignorance fades away and ceases with nothing left over, choices cease. When choices cease, consciousness ceases. When consciousness ceases, name and form cease. When name and form cease, the six sense fields cease. When the six sense fields cease, contact ceases. When contact ceases, feeling ceases. When feeling ceases, craving ceases. When craving ceases, grasping ceases. When grasping ceases, continued existence ceases. When continued existence ceases, rebirth ceases. When rebirth ceases, old age and death, sorrow, lamentation, pain, sadness, and distress cease. That is how this entire mass of suffering ceases.’ That’s how a mendicant is qualified to be called ‘skilled in dependent origination’.” 

“But\marginnote{12.1} sir, how is a mendicant qualified to be called ‘skilled in the possible and impossible’?”\footnote{Said to be one of the ten powers of a Realized One (\href{https://suttacentral.net/mn12/en/sujato\#10.1}{MN 12:10.1}, \href{https://suttacentral.net/an10.22/en/sujato\#3.3}{AN 10.22:3.3}), and to be restricted to those with meditative immersion (\textit{\textsanskrit{samādhi}}, \href{https://suttacentral.net/an6.64/en/sujato\#13.1}{AN 6.64:13.1}). Each item here, split into separate suttas, is found at \href{https://suttacentral.net/an1.168/en/sujato}{AN 1.168}–295. Briefer examples of this trope are found at \href{https://suttacentral.net/mn122/en/sujato\#3.2}{MN 122:3.2}, \href{https://suttacentral.net/sn12.30/en/sujato}{SN 12.30}, \href{https://suttacentral.net/an6.70/en/sujato}{AN 6.70}, and \href{https://suttacentral.net/sn6.93/en/sujato}{SN 6.93}. These all focus on cause and effect as it pertains to development of the path. } 

“It’s\marginnote{12.2} when a mendicant understands: ‘It’s impossible for a person accomplished in view to take any condition as permanent. That is not possible.\footnote{A stream-enterer (or above) might unthinkingly act as if something were permanent (etc.), but on reflection they would know right away that this is not the case. } But it’s possible for an ordinary person to take some condition as permanent. That is possible.’ They understand: ‘It’s impossible for a person accomplished in view to take any condition as pleasant. But it’s possible for an ordinary person to take some condition as pleasant.’ They understand: ‘It’s impossible for a person accomplished in view to take anything as self. But it’s possible for an ordinary person to take something as self.’ 

They\marginnote{13.1} understand: ‘It’s impossible for a person accomplished in view to murder their mother.\footnote{Murder of mother, father, or an arahant, injuring a Buddha, and causing schism are said to be specially heinous deeds that result in rebirth in hell. } But it’s possible for an ordinary person to murder their mother.’ They understand: ‘It’s impossible for a person accomplished in view to murder their father … or murder a perfected one. But it’s possible for an ordinary person to murder their father … or a perfected one.’ They understand: ‘It’s impossible for a person accomplished in view to injure a Realized One with malicious intent. But it’s possible for an ordinary person to injure a Realized One with malicious intent.’ They understand: ‘It’s impossible for a person accomplished in view to cause a schism in the \textsanskrit{Saṅgha}. But it’s possible for an ordinary person to cause a schism in the \textsanskrit{Saṅgha}.’ They understand: ‘It’s impossible for a person accomplished in view to dedicate themselves to another teacher.\footnote{This does not mean that they cannot learn from another teacher; rather, they know that the Buddha’s path leads to liberation because they have realized it for themselves. } But it’s possible for an ordinary person to dedicate themselves to another teacher.’ 

They\marginnote{14.1} understand: ‘It’s impossible for two perfected ones, fully awakened Buddhas to arise in the same solar system at the same time.\footnote{A \textit{\textsanskrit{lokadhātu}} is a single system with an earth, a sun and moon, planets, and a set of heavenly realms. Thus with due allowance for the different context, it is similar to a “solar system”. While many Buddhas may arise in the same solar system, they may not arise in the same region at the same time. There may, however, be many Buddhas simultaneously throughout the galaxy, an idea developed in later texts (eg. \textsanskrit{Mahāvastu} 1.122). } But it is possible for just one perfected one, a fully awakened Buddha, to arise in one solar system.’ They understand: ‘It’s impossible for two wheel-turning monarchs to arise in the same solar system at the same time.\footnote{A wheel-turning monarch was conceived as establishing dominion over the entire land from ocean to ocean, which apparently was believed to be the civilized world at the time the myth was created. } But it is possible for just one wheel-turning monarch to arise in one solar system.’ 

They\marginnote{15.1} understand: ‘It’s impossible for a woman to be a perfected one, a fully awakened Buddha.\footnote{It is unclear why this is mentioned, and since the Chinese parallel MA 181 lacks this passage entirely, it may be a later addition. It is of no apparent spiritual relevance, since in early Buddhist texts there is no concept of aspiring to become a Buddha in future lives. A woman may, of course, realize full liberation as an arahant, so the issue is not spiritual prowess. As to the reasoning underlying the claim, in this passage the Buddha is followed by a wheel-turning monarch, Sakka, \textsanskrit{Māra}, and \textsanskrit{Brahmā}. What they all have in common is that they are leaders. Since leaders of the time were men, perhaps it was assumed that people would not follow a woman in that role. This reasoning would also seem to be behind the idea that the Buddha was born either an aristocrat or a brahmin, whichever was the leading class of the time (\href{https://suttacentral.net/dn14/en/sujato\#1.5.1}{DN 14:1.5.1}). } But it is possible for a man to be a perfected one, a fully awakened Buddha.’ They understand: ‘It’s impossible for a woman to be a wheel-turning monarch. But it is possible for a man to be a wheel-turning monarch.’ They understand: ‘It’s impossible for a woman to perform the role of Sakka, \textsanskrit{Māra}, or the Divinity.\footnote{While we encounter these divinities as individuals, they are understood as being stations or offices in the leading role of their particular realms. Thus, for example, \textsanskrit{Moggallāna} said that he used to be \textsanskrit{Māra} in a past life (\href{https://suttacentral.net/mn50/en/sujato\#8.1}{MN 50:8.1}). } But it is possible for a man to perform the role of Sakka, \textsanskrit{Māra}, or the Divinity.’ 

They\marginnote{16.1} understand: ‘It’s impossible for a likable, desirable, agreeable result to come from bad conduct of body, speech, and mind. But it is possible for an unlikable, undesirable, disagreeable result to come from bad conduct of body, speech, and mind.’ 

They\marginnote{17.1} understand: ‘It’s impossible for an unlikable, undesirable, disagreeable result to come from good conduct of body, speech, and mind. But it is possible for a likable, desirable, agreeable result to come from good conduct of body, speech, and mind.’ 

They\marginnote{18.1} understand: ‘It’s impossible that someone who has engaged in bad conduct of body, speech, and mind, could for that reason alone, when their body breaks up, after death, be reborn in a good place, a heavenly realm.\footnote{The qualification “for that reason” (\textit{\textsanskrit{taṁnidānā} \textsanskrit{tappaccayā}}) is essential, as it is quite possible for someone to engage in bad conduct and later take a good rebirth due to other deeds. This case is discussed in \href{https://suttacentral.net/mn136/en/sujato\#18.1}{MN 136:18.1}. } But it is possible that someone who has engaged in bad conduct of body, speech, and mind could, for that reason alone, when their body breaks up, after death, be reborn in a place of loss, a bad place, the underworld, hell.’ 

They\marginnote{19.1} understand: ‘It’s impossible that someone who has engaged in good conduct of body, speech, and mind could, for that reason alone, when their body breaks up, after death, be reborn in a place of loss, a bad place, the underworld, hell. But it is possible that someone who has engaged in good conduct of body, speech, and mind could, for that reason alone, when their body breaks up, after death, be reborn in a good place, a heavenly realm.’ That’s how a mendicant is qualified to be called ‘skilled in the possible and impossible’.” 

When\marginnote{20.1} he said this, Venerable Ānanda said to the Buddha, “It’s incredible, sir, it’s amazing! What is the name of this exposition of the teaching?” 

“Well\marginnote{20.4} then, Ānanda, you may remember this exposition of the teaching as ‘The Many Elements’, or else ‘The Four Rounds’, or else ‘The Mirror of the Teaching’, or else ‘The Drum of Freedom From Death’, or else ‘The Supreme Victory in Battle’.”\footnote{The “four rounds” are the four main topics (elements, sense fields, dependent origination, the possible and the impossible); elsewhere it is said to be the four noble truths (\href{https://suttacentral.net/sn22.56/en/sujato\#2.1}{SN 22.56:2.1}). | The “mirror of the teaching” is the ability of a noble one to see the truth of the Dhamma in themselves (\href{https://suttacentral.net/sn55.8/en/sujato\#5.1}{SN 55.8:5.1}, \href{https://suttacentral.net/dn16/en/sujato\#2.8.3}{DN 16:2.8.3}). | The “drum of freedom from death” was sounded by the Buddha after his awakening (\href{https://suttacentral.net/mn26/en/sujato\#25.21}{MN 26:25.21}). | The “supreme victory in battle” was also suggested as a title for the \textsanskrit{Brahmajālasutta} (\href{https://suttacentral.net/dn1/en/sujato\#3.74.3}{DN 1:3.74.3}), while elsewhere it is said to be a name for the noble eightfold path (\href{https://suttacentral.net/sn45.4/en/sujato\#4.3}{SN 45.4:4.3}). } 

That\marginnote{20.5} is what the Buddha said. Satisfied, Venerable Ānanda approved what the Buddha said. 

%
\section*{{\suttatitleacronym MN 116}{\suttatitletranslation At Isigili }{\suttatitleroot Isigilisutta}}
\addcontentsline{toc}{section}{\tocacronym{MN 116} \toctranslation{At Isigili } \tocroot{Isigilisutta}}
\markboth{At Isigili }{Isigilisutta}
\extramarks{MN 116}{MN 116}

\scevam{So\marginnote{1.1} I have heard.\footnote{This is a devotional sutta that lists the names of sages of the past. It is recited as a protection chant in Sri Lanka. The Chinese parallel shares only the opening passages, after which it is quite different. Moreover, only three of the names are clearly the same: \textsanskrit{Ariṭṭha}, \textsanskrit{Upariṭṭha}, and Sudassana (EA 38.7 at T ii 723a–c). | \textsanskrit{Ñāṇamoḷī} and Bodhi both rightly note that it is difficult to distinguish names from attributes without the commentary. Yet the commentary leaves many problems unsolved, as it omits many names. } }At one time the Buddha was staying near \textsanskrit{Rājagaha}, on the Isigili Mountain.\footnote{\textsanskrit{Rājagaha} is surrounded by an extensive range of hills, upon which ascetics of various kinds could be found striving. } There the Buddha addressed the mendicants, “Mendicants!” 

“Venerable\marginnote{1.5} sir,” they replied. The Buddha said this: 

“Mendicants,\marginnote{2.1} do you see that Mount \textsanskrit{Vebhāra}?” 

“Yes,\marginnote{2.2} sir.” 

“It\marginnote{2.3} used to have a different label and description. Do you see that Mount \textsanskrit{Paṇḍava}?” 

“Yes,\marginnote{2.5} sir.” 

“It\marginnote{2.6} too used to have a different label and description. Do you see that Mount Vepulla?” 

“Yes,\marginnote{2.8} sir.” 

“It\marginnote{2.9} too used to have a different label and description. Do you see that Mount Vulture’s Peak?” 

“Yes,\marginnote{2.11} sir.” 

“It\marginnote{2.12} too used to have a different label and description. Do you see that Mount Isigili?” 

“Yes,\marginnote{2.14} sir.” 

“It\marginnote{3.1} used to have exactly the same label and description. 

Once\marginnote{3.2} upon a time, five hundred Independent Buddhas dwelt for a long time on this Isigili.\footnote{The concept of the \textit{paccekabuddha} allows that there were awakened sages in the long periods of time when there is no \textit{sambuddha} such as Gotama. Both are Buddhas in the sense of being awakened, but they differ in their relation to others. It is sometimes said that they do not teach, but the tradition attributes to them a number of sayings (notably \href{https://suttacentral.net/snp1.3/en/sujato}{Snp 1.3}). They are sometimes said to be “solitary”, but here 500 dwell on the same hill, while \textit{pacceka} \textsanskrit{Brahmās} appear in pairs (\href{https://suttacentral.net/sn6.6/en/sujato\#1.3}{SN 6.6:1.3}). Compare \href{https://suttacentral.net/pli-tv-bu-vb-np9/en/sujato\#1.36.1}{Bu NP 9:1.36.1}, where “independent” funds are set up for “independent” robes (rather than combining them); the same idea occurs at \href{https://suttacentral.net/an11.16/en/sujato\#11.2}{AN 11.16:11.2}. Likewise, \href{https://suttacentral.net/sn1.37/en/sujato\#1.7}{SN 1.37:1.7} speaks of “independent” verses. Thus the burden of sense in the Pali is that \textit{pacceka} means “independent’, which here has the sense of not forming a fourfold community, but simply living and practicing independently. } They were seen entering the mountain, but after entering were seen no more. When people noticed this they said: ‘That mountain swallows these seers!’ That’s how it came to be known as Isigili, the “seer-swallower”.\footnote{This etymology is a deliberate pun, since the obvious explanation would be that \textit{gili} is simply \textit{giri} (“mountain”), as Pali \emph{r} becomes \emph{l} in the Magadhan dialect. This passage is not found in the Chinese parallel. } 

I\marginnote{3.7} shall declare the names of the Independent Buddhas;\footnote{\textsanskrit{Mahāvastu} 21 lists many past \textit{sambuddhas}, among whom we find names recorded here as \textit{paccekabuddhas}, including \textsanskrit{Aparājita}, \textsanskrit{Supatiṭṭhita}, \textsanskrit{Bandhumā}, \textsanskrit{Ariṭṭha}, Tagarasikhi, Tissa, Sikhin, \textsanskrit{Accutagāma}, Asayha, Uppala, Sudassana, Ketu, Jayanta, and Uttara. } I shall extol the names of the Independent Buddhas; I shall teach the names of the Independent Buddhas. Listen and apply your mind well, I will speak.”\footnote{I give the probable meaning of each name, and attempt, perhaps unwisely, to locate other sages of the past with similar names. In almost every case it is impossible to establish such links with any confidence, due to the lack of contextual details, the prevalence of common names used by many people, the use of names based on places or doctrinal concepts, and so on. Nonetheless, certain patterns emerge, especially the fact that several of the names seem to be shared with Jain \textit{\textsanskrit{tīrthaṅkaras}}. These are the  twenty-four founding fathers of Jainism through the ages, like the Buddhist concept of past Buddhas. As many as eight names might be shared with \textit{\textsanskrit{tīrthaṅkaras}}, while several others might be other Jain sages. Even the name of the sutta itself is the name of a sage Isigiri in \textsanskrit{Isibhāsiyāiṁ} 34. It could be that both Buddhism and Jainism drew on shared traditions of ancient sages, but again, these similarities are all tentative at best. } 

“Yes,\marginnote{3.11} sir,” they replied. The Buddha said this: 

“The\marginnote{4.1} independent Buddhas who dwelt for a long time on this Isigili were named \textsanskrit{Ariṭṭha},\footnote{An agent of Kuvera has this name (meaning “indestructible”) at \href{https://suttacentral.net/dn32/en/sujato\#7.47}{DN 32:7.47}, where it is next to Nemi (below at \href{https://suttacentral.net/mn116/en/sujato\#6.7}{MN 116:6.7}). \textsanskrit{Ariṭṭha} and Nemi often appear compounded in Sanskrit as \textsanskrit{Ariṣṭanemi}. This was the name of a past Buddha (\textsanskrit{Mahāvastu} 1.140), the 22nd Jain \textit{\textsanskrit{tīrthaṅkara}} (\textsanskrit{Kalpasūtra} 170–183), and a Brahmanical sage of remarkable power (\textsanskrit{Mahābhārata} 3.182.8c). \textsanskrit{Ariṣṭanemi} \textsanskrit{Tārkṣya}, although a mythical being, was said to have composed Rig Veda 10.178. } \textsanskrit{Upariṭṭha},\footnote{This was the name of Anuruddha in a past life as an ascetic (\href{https://suttacentral.net/thag16.9/en/sujato\#19.4}{Thag 16.9:19.4}), although it couldn’t be the same person, since Independent Buddhas are not reborn. It means “son of \textsanskrit{Ariṭṭha}”. } \textsanskrit{Tagarasikhī},\footnote{Also named in \href{https://suttacentral.net/sn3.20/en/sujato}{SN 3.20}, \href{https://suttacentral.net/ud5.3/en/sujato}{Ud 5.3}, and \href{https://suttacentral.net/ja390/en/sujato}{Ja 390}, which makes him somewhat of a celebrity by \textit{paccekabuddha} standards. His name means “jasmine-crested”. } Yasassin,\footnote{“Renowned”. } Sudassana,\footnote{“Good-looking”. } Piyadassin,\footnote{Meaning “one who looks kindly”, this was the favored epithet of King Ashoka. } \textsanskrit{Gandhāra},\footnote{Elsewhere known as a nation famed for horses, located around modern Afghanistan (“Kandahar”, \href{https://suttacentral.net/an3.70/en/sujato\#28.2}{AN 3.70:28.2}) and its capital (\href{https://suttacentral.net/dn16/en/sujato\#6.28.6}{DN 16:6.28.6}). It was named after its founder, but he is otherwise not known as a sage. This could be any sage from \textsanskrit{Gandhāra} (eg. \textsanskrit{Triṣaṣṭiśalākāpuruṣacaritra} 10.11.9; see \href{https://suttacentral.net/ja406/en/sujato}{Ja 406}). Below we find several place names, which likewise probably refer to sages from those places. } \textsanskrit{Piṇḍola},\footnote{Meaning “alms-gatherer”, a \textsanskrit{Bhāradvāja} of this name was a famous monk. } \textsanskrit{Upāsabha},\footnote{\textit{Āsabha} or \textit{usabha} (“bull”) is a common name; this is “son of Bull”. } \textsanskrit{Nīta}, Tatha, \textsanskrit{Sutavā}, and \textsanskrit{Bhāvitatta}. 

\begin{verse}%
Those\marginnote{5.1} saintly beings, untroubled, \\>with no need for hope,\footnote{The verses that follow are not found in the Chinese parallel. | “Saintly beings” renders \textit{\textsanskrit{sattasāra}}, a rare term otherwise only found in the late \textsanskrit{Apadānas} where, moreover, it is usually a variant reading (\href{https://suttacentral.net/tha-ap532/en/sujato\#24.1}{Tha Ap 532:24.1}, \href{https://suttacentral.net/tha-ap539/en/sujato\#7.3}{Tha Ap 539:7.3}, \href{https://suttacentral.net/tha-ap551/en/sujato\#15.1}{Tha Ap 551:15.1}, \href{https://suttacentral.net/thi-ap32/en/sujato\#23.1}{Thi Ap 32:23.1}). } \\
who each achieved awakening independently; \\
hear me extol their names, \\
the supreme persons, free of thorns. 

\textsanskrit{Ariṭṭha},\marginnote{5.5} \textsanskrit{Upariṭṭha}, \textsanskrit{Tagarasikhī}, Yasassin, \\
Sudassana, and Piyadassin the awakened; \\
\textsanskrit{Gandhāra}, \textsanskrit{Piṇḍola}, and \textsanskrit{Upāsabha}, \\
\textsanskrit{Nīta}, Tatha, \textsanskrit{Sutavā}, and \textsanskrit{Bhāvitatta}.\footnote{The names of these four occupy a similar semantic space: “Educated”, “True”, “Learned”, “Developed”. } 

Sumbha,\marginnote{6.1} Subha, Methula, and \textsanskrit{Aṭṭhama},\footnote{Sumbha is the name of a country, and this might be a sage from there. There is also a powerful \textit{asura} named Śumbha. | Subha (“beautiful”), Methula (“pair”; variant \textit{matula}), and \textsanskrit{Aṭṭhama} (“eighth”) are otherwise unknown. | This verse is omitted in the commentary. } \\
then Assumegha, \textsanskrit{Anīgha}, and \textsanskrit{Sudāṭha},\footnote{Assumegha (“tear-cloud”) should perhaps be read \textit{sumegha}, “lovely cloud”, the name of a mountain. | \textsanskrit{Anīgha} (“untroubled”) is otherwise unknown. | \textsanskrit{Sudāṭha} (‘strong-tooth”) and \textsanskrit{Subāhu} (below, “strong-arm”) are unknown as sages, but a lion and tiger of these names are identified with \textsanskrit{Sāriputta} and \textsanskrit{Moggallāna} in a past life (\href{https://suttacentral.net/ja361/en/sujato}{Ja 361}). } \\
Independent Buddhas, enders of the conduit to rebirth; \\
and \textsanskrit{Hiṅgū} and \textsanskrit{Hiṅga} the mighty.\footnote{\textsanskrit{Hiṅgū} (“asafoetida”, Hindi “hing”) and \textsanskrit{Hiṅga} (the name of a people in \textsanskrit{Mārkaṇḍeya} \textsanskrit{Purāṇa} 58) are otherwise unknown. } 

Two\marginnote{6.5} sages named \textsanskrit{Jāli}, and \textsanskrit{Aṭṭhaka};\footnote{\textsanskrit{Jāli} (“one who has a net”, sometimes “fisherman”) is the name of one of Vessantara’s children (\href{https://suttacentral.net/ja547/en/sujato}{Ja 547}). | \textsanskrit{Aṭṭhaka} was the name of a Vedic sage (see \href{https://suttacentral.net/mn95/en/sujato\#13.9}{MN 95:13.9} and note there). } \\
then the awakened one Kosala, then \textsanskrit{Subāhu};\footnote{Kosala is another place name. | \textsanskrit{Subāhu} is a common name for warriors or princes. } \\
Upanemi, Nemi, and Santacitta,\footnote{Nemi (“rim”) was the name of the 22nd Jain \textit{\textsanskrit{tīrthaṅkara}}, where it is an abbreviation of \textit{\textsanskrit{ariṣṭanemi}} (“indestructible rim”). | Take \textit{so} here as connective pronoun. } \\
right and true, stainless and astute.\footnote{This line is not commented on, so each of these might be taken as a name rather than as attributes. } 

\textsanskrit{Kāḷa}\marginnote{6.9} and \textsanskrit{Upakāḷa}, Vijita and Jita,\footnote{\textsanskrit{Kāḷa} (“dark/time”) is a common name; \textsanskrit{Upakāḷa} is his son. | We might read \textit{vijito jito} as \textit{vijito (a)jito} (“one who has conquered”, “one who is unconquered”), in which case the name Ajita would be shared with the second Jain \textit{\textsanskrit{tīrthaṅkara}}. } \\
\textsanskrit{Aṅga} and \textsanskrit{Paṅga}, and Guttijita too;\footnote{\textsanskrit{Aṅga} (“limb”) is a kingdom, and is sometimes found as a personal name. | \textsanskrit{Paṅga} may be a variant spelling of \textsanskrit{Vaṅga}, i.e. Bengal. | Guttijita means “guard and conqueror (of the senses)”. } \\
Passin gave up attachment, suffering’s root,\footnote{Passi most obviously means “seer”. But perhaps it could be from the root \textit{passa} (“side”), making this the 23rd Jain \textit{\textsanskrit{tīrthaṅkara}} \textsanskrit{Pārśva}, who immediately preceded \textsanskrit{Mahāvīra}. } \\
while \textsanskrit{Aparājita} defeated \textsanskrit{Māra}’s power.\footnote{\textsanskrit{Aparājita} (“unconquered by another”) is a common name. It was the name of \textsanskrit{Śāntinātha}, the 16th Jain \textit{\textsanskrit{tīrthaṅkara}}, in a past life. } 

Satthar,\marginnote{6.13} Pavattar, \textsanskrit{Sarabhaṅga}, \textsanskrit{Lomahaṁsa},\footnote{Satthar (“teacher”) is used in both Buddhism (usually for the Buddha) and Jainism. Pavattar (“roller-forth”, “proclaimer”) has a similar sense. | \textsanskrit{Sarabhaṅga} (“arrow-breaker”) was a forest hermit in \textsanskrit{Daṇḍaka}. See note at \href{https://suttacentral.net/mn56/en/sujato\#13.27}{MN 56:13.27}. | \textsanskrit{Lomahaṁsa} means “hair raising”. } \\
\textsanskrit{Uccaṅgamāya}, Asita, \textsanskrit{Anāsava},\footnote{Asita (“dark one”) was the name of the sage who attended Siddhattha after his birth (\href{https://suttacentral.net/snp3.11/en/sujato}{Snp 3.11}), while an Asita Devala also features in \href{https://suttacentral.net/mn93/en/sujato}{MN 93}. The Rig Veda attributes several hymns to Asita or Devala (sometimes said to be his son) of the \textsanskrit{Kāśyapa} clan. Asita Devala also appears in the Jain \textsanskrit{Isibhāsiyāiṁ} chapter 3. | \textsanskrit{Anāsava} means “undefiled”. Freedom from “defilement” is a key concept in Jainism. } \\
Manomaya, and \textsanskrit{Bandhumā} the cutter of conceit, \\
then Adhimutta, and \textsanskrit{Ketumā} the immaculate.\footnote{The commentary says \textit{\textsanskrit{tadādhimutto}} is the only name in this line, but I take \textit{tad(\textsanskrit{ā})} as a connective pronoun, leaving \textit{adhimutta} (“faithful”) as the name. Moreover, since the syntax echoes the previous line, I construe it the same way, making \textit{\textsanskrit{ketumā}} a proper name, meaning one who is “adorned with banners” (like the sun). There are several people and places of this name in Sanskrit, among them at least one sage; it corresponds to Ketu in \textsanskrit{Mahāvastu} 21. See also \textit{\textsanskrit{ketumatī}}, said to be an old name of Varanasi (\href{https://suttacentral.net/dn26/en/sujato\#23.8}{DN 26:23.8}). On the other hand, I follow the commentary in taking \textit{vimalo} as an attribute (“immaculate”), but it could be a proper name, in which case it would be shared with Vimala, the 13th Jain \textit{\textsanskrit{tīrthaṅkara}}. } 

\textsanskrit{Ketumbharāga},\marginnote{6.17} \textsanskrit{Mātaṅga}, and Ariya,\footnote{\textsanskrit{Ketumbharāga} is “cloud-colored”. | \textsanskrit{Mātaṅga} (“one who goes where they wish”, i.e. “elephant”) is the name of an outcast who became a sage (\textsanskrit{Skandapurāṇa} 5.2.60). In other Buddhist versions of this story, he ascends to the \textsanskrit{Brahmā} realm (\href{https://suttacentral.net/snp1.7/en/sujato\#26.1}{Snp 1.7:26.1}), in which case he is identified with the Bodhisatta (\href{https://suttacentral.net/ja497/en/sujato}{Ja 497}). This contradicts the present sutta, for a \textit{paccekabuddha} is not reborn anywhere. Either there were several sages of this name, or else the memory of a past sage was rather vaguely specified. In the Jain \textsanskrit{Isibhāsiyāiṁ} chapter 26, he criticizes so-called Brahmins who practice violence. See too note on \href{https://suttacentral.net/mn56/en/sujato\#13.27}{MN 56:13.27}. | A sage named Ariya is found in \textsanskrit{Isibhāsiyāiṁ} chapter 19, where he teaches a doctrine of “civility”. } \\
then Accuta, \textsanskrit{Accutagāma}, and \textsanskrit{Byāmaka},\footnote{Accuta (“unfallen”) is a name of Agni (Śatapatha \textsanskrit{Brāhmaṇa} 1.6.1.6) and later \textsanskrit{Viṣṇu}/\textsanskrit{Kṛṣṇa}. | If Accuta is \textsanskrit{Viṣṇu}/\textsanskrit{Kṛṣṇa}, perhaps \textsanskrit{Accutagāma} is a garbled form of his elder brother \textsanskrit{Balarāma}’s name \textsanskrit{Acyutāgraja} (“firstborn of the unfallen”). | \textsanskrit{Byāmaka} means “fathom-long”. } \\
\textsanskrit{Sumaṅgala}, Dabbila, \textsanskrit{Supatiṭṭhita},\footnote{\textsanskrit{Sumaṅgala} (“good fortune”) is a common name. A Jain text tells of a king of that name who became an ascetic in later life (\textsanskrit{Triṣaṣṭiśalākāpuruṣacaritra} 10.6.2). \href{https://suttacentral.net/ja420/en/sujato}{Ja 420} tells of how a park-keeper named \textsanskrit{Sumaṅgala} accidentally killed a \textit{paccekabuddha}. | Dabbila probably refers to one who uses the sacrificial “ladle” (\textit{darvi}, see \href{https://suttacentral.net/mn93/en/sujato\#18.71}{MN 93:18.71} and note). | \textsanskrit{Supatiṭṭhita} (“well-established”) might be an attribute; commentary is silent, but \textsanskrit{Mahāvastu} 21 takes it as a name. } \\
Asayha, \textsanskrit{Khemābhirata}, and Sorata.\footnote{“Unbeatable”, “lover of sanctuary”, and “gentle”. } 

Durannaya,\marginnote{6.21} \textsanskrit{Saṅgha}, and also Ujjaya,\footnote{Durannaya means “hard to follow”. | \textsanskrit{Saṅgha} (“community”) is a fairly common name in Pali. | Ujjaya (“victory”) in Pali is the name of a brahmin (\href{https://suttacentral.net/an4.39/en/sujato}{AN 4.39}, \href{https://suttacentral.net/an8.55/en/sujato}{AN 8.55}) and a mendicant (\href{https://suttacentral.net/thag1.47/en/sujato}{Thag 1.47}). The city of Ujjeni in \textsanskrit{Avantī} is sometimes spelled \textit{ujjaya}, so this is perhaps someone from there. } \\
another sage, Sayha of peerless effort.\footnote{Sayha is the name of a mountain district in which the \textsanskrit{Godhāvarī} arises (Trimbakeshwar Range). It is south of Ujjeni, so perhaps both of these are places. } \\
There are twelve Ānandas, Nandas, and Upanandas,\footnote{The commentary says there were four of each, making twelve in total. All are common names meaning “joy”. } \\
and \textsanskrit{Bhāradvāja}, bearing his final body.\footnote{\textsanskrit{Bhāradvāja} (“sky-lark”) was one of the seven great Vedic sages, who founded a storied lineage (\href{https://suttacentral.net/dn32/en/sujato\#10.3}{DN 32:10.3}, \href{https://suttacentral.net/mn95/en/sujato\#11.2}{MN 95:11.2}). } 

Bodhi,\marginnote{6.25} \textsanskrit{Mahānāma}, and also Uttara,\footnote{“Awakening”, “great name”, and “north/supreme” are all common names. } \\
\textsanskrit{Kesī}, \textsanskrit{Sikhī}, Sundara, and \textsanskrit{Bhāradvāja},\footnote{In Jainism, \textsanskrit{Keśi} is a noted ascetic disciple of \textsanskrit{Pārśva} (\textsanskrit{Uttarādhyayanasūtra} 23.2). | \textsanskrit{Sikhī} was a former Buddha (\href{https://suttacentral.net/sn6.14/en/sujato}{SN 6.14}), whose name is solar in connotation (\href{https://suttacentral.net/dn14/en/sujato\#1.12.4}{DN 14:1.12.4}): the streaming radiance of the sun. The sun is also said to be \textit{kesi} (“hairy”) for the same reason (eg. Rig Veda 10.136.1), so these two names go together. | Sundara (“beautiful”) is a common name. | The presence of a second \textsanskrit{Bhāradvāja} creates tension, since duplicates are mostly avoided. The \textsanskrit{Mahāsaṅgīti} resolves it by reading \textit{\textsanskrit{dvārabhāja}} here, which is otherwise unattested. In her translation, Horner took \textit{athopi uttaro} in the previous line as “also another” \textsanskrit{Bhāradvāja}, with the other items as attributes. However, I leave the tension unresolved, as there were many \textsanskrit{Bhāradvājas}. } \\
Tissa and Upatissa, \\>who’ve both cut the bonds to rebirth,\footnote{Tissa is a common name, meaning “born under the star Sirius”. Notably, Tissa follows \textsanskrit{Bhāradvāja}; both were names of the chief disciples of Kassapa Buddha, and were associated with Mount Vepulla of \textsanskrit{Rājagaha} in a former age (\href{https://suttacentral.net/sn15.20/en/sujato\#4.6}{SN 15.20:4.6}). } \\
\textsanskrit{Upasīdarī} and \textsanskrit{Sīdarī}, who’ve both cut off craving.\footnote{\textsanskrit{Sīdarī} perhaps means “dweller by the \textit{\textsanskrit{Sīdā}}”, which in Pali tradition is a river in the Himalayas beside which ascetics flourish (\href{https://suttacentral.net/ja541/en/sujato\#16.1}{Ja 541:16.1}). It is probably a variant spelling of \textit{\textsanskrit{sītā}} (“cold”), a common name for rivers. The tenth Jain \textit{\textsanskrit{tīrthaṅkara}} has the similar name \textsanskrit{Śītala}. } 

\textsanskrit{Maṅgala}\marginnote{6.29} was awakened, free of greed,\footnote{\textsanskrit{Maṅgala} (“(good) omen”) is a common name. } \\
Usabha cut the net, the root of suffering,\footnote{Usabha (“bull”) is \textit{\textsanskrit{ṛṣabha}} in Sanskrit. It was the name of several Brahmanical sages, as well as the first Jain \textit{\textsanskrit{tīrthaṅkara}}. } \\
\textsanskrit{Upanīta} who attained the state of peace,\footnote{\textsanskrit{Upanīta} means “one who has been initiated” in the \textit{upanayana} ritual, by which a boy undertakes Vedic studies. } \\
Uposatha, Sundara, and \textsanskrit{Saccanāma}.\footnote{Uposatha (“sabbath”) is the name of King Sudassana’s royal elephant (\href{https://suttacentral.net/dn17/en/sujato\#2.5.5}{DN 17:2.5.5}). | This is the second Sundara; one or both might be an attribute. | \textsanskrit{Saccanāma} (“one whose name is true”) recalls the famous sage \textsanskrit{Satyakāma} \textsanskrit{Jābāla} of \textsanskrit{Chāndogya} \textsanskrit{Upaniṣad} 4.4. } 

Jeta,\marginnote{6.33} Jayanta, Paduma, and Uppala;\footnote{All these are common names. Jeta and Jayanta both mean “victor”, while Paduma and Uppala both mean “lotus”. In Jainism, Paduma recalls the name of the sixth \textit{\textsanskrit{tīrthaṅkara}}, Padmaprabha (“lotus-shine”), while a sage named Jayanta practiced in the mountains during a previous incarnation of the \textit{\textsanskrit{tīrthaṅkara}} Śanti (\textsanskrit{Triṣaṣṭiśalākāpuruṣacaritra} 5.1.9). } \\
Padumuttara, Rakkhita, and Pabbata,\footnote{Padumuttara (“supreme lotus”) is well known in \textsanskrit{Theravāda} as the name of the thirteenth of the twenty-eight Buddhas. Jains recognize two sages of this name, as incarnations of the \textit{\textsanskrit{tīrthaṅkaras}} \textsanskrit{Śītala} and \textsanskrit{Vāsupūjya} (\textsanskrit{Triṣaṣṭiśalākāpuruṣacaritra} 3.8.1, 4.2.1). | Rakkhita (“protected”) is common name. Here it is followed by another place name, Pabbata (“mountain”). A sage Rakkhita of the mountains features in the related \textsanskrit{Jātaka} stories told at \textsanskrit{Mahāvastu} 28a and \href{https://suttacentral.net/ja505/en/sujato}{Ja 505}. Parvata is also the name of a Vedic poet, who legend says was a nephew of \textsanskrit{Nārada}. } \\
\textsanskrit{Mānatthaddha}, beautiful and free of greed,\footnote{\textsanskrit{Mānatthaddha} means “stiff with pride”; a brahmin of that name appears in \href{https://suttacentral.net/sn7.15/en/sujato}{SN 7.15}. Presumably it was an old nickname that stuck. } \\
and the Buddha \textsanskrit{Kaṇha}, his mind well freed.\footnote{\textsanskrit{Kaṇha} (“dark”) is better known by his Sanskrit name Krishna. While the god of that name is not prominent in the suttas, it does seem as if elements of his story were known (\href{https://suttacentral.net/dn3/en/sujato\#1.23.8}{DN 3:1.23.8}). } 

These\marginnote{7.1} and other mighty ones \\>awakened independently,\footnote{I count 103 named \textit{paccekabuddhas} in total. However, the number is not certain, given the ambiguity between names and attributes.  } \\
enders of the conduit to rebirth—\\
honor these great seers \\>who have slipped all chains, \\
fully quenched, limitless.” 

%
\end{verse}

%
\section*{{\suttatitleacronym MN 117}{\suttatitletranslation The Great Forty }{\suttatitleroot Mahācattārīsakasutta}}
\addcontentsline{toc}{section}{\tocacronym{MN 117} \toctranslation{The Great Forty } \tocroot{Mahācattārīsakasutta}}
\markboth{The Great Forty }{Mahācattārīsakasutta}
\extramarks{MN 117}{MN 117}

\scevam{So\marginnote{1.1} I have heard.\footnote{This is one of the two major treatments of the noble eightfold path in the Middle Discourses, the other being \href{https://suttacentral.net/mn141/en/sujato}{MN 141}, where each factor is defined in detail. This discourse emphasizes the interrelationships between the factors, especially as they support right immersion. The meaning of the title is explained in the final passage. } }At one time the Buddha was staying near \textsanskrit{Sāvatthī} in Jeta’s Grove, \textsanskrit{Anāthapiṇḍika}’s monastery. There the Buddha addressed the mendicants, “Mendicants!” 

“Venerable\marginnote{1.5} sir,” they replied. The Buddha said this: 

“Mendicants,\marginnote{2.1} I will teach you noble right immersion with its vital conditions and its prerequisites.\footnote{The first portion of this sutta is also at \href{https://suttacentral.net/sn45.28/en/sujato\#1.2}{SN 45.28:1.2}, to which the current sutta can be seen as a commentary. | “Vital condition” is \textit{upanisa}; this is a condition for something that is both necessary and strongly conducive. | “Prerequiste” is \textit{\textsanskrit{parikkhāra}}, the “equipment” or “requirement”. } Listen and apply your mind well, I will speak.” 

“Yes,\marginnote{2.3} sir,” they replied. The Buddha said this: 

“And\marginnote{3.1} what is noble right immersion with its vital conditions and its prerequisites? They are: right view, right thought, right speech, right action, right livelihood, right effort, and right mindfulness. Unification of mind with these seven factors as prerequisites is what is called noble right immersion ‘with its vital conditions’ and also ‘with its prerequisites’.\footnote{While it is possible to obtain immersion outside the context of the noble eightfold path, without all the factors of the path working together it does not result in liberation and hence is not “right” immersion. | This became known in the Abhidhamma as \textit{lokuttara \textsanskrit{jhāna}}, where it is used in the same sense as the sutta: a meditative absorption which, empowered by all the path factors in concert, leads to the realization of the Dhamma. This is contrasted with the \textit{lokiya \textsanskrit{jhāna}} that leads only to rebirth in the \textsanskrit{Brahmā} realm; see comment on \href{https://suttacentral.net/mn7/en/sujato\#19.6}{MN 7:19.6}. Later, the commentarial tradition analyzed it further as a series of mind moments that occur at the occasion of realization. } 

In\marginnote{4.1} this context, right view comes first.\footnote{In the Gradual Training, it is hearing the Dhamma that sets you on the path (\href{https://suttacentral.net/mn101/en/sujato\#31.1}{MN 101:31.1}). } And how does right view come first? When you understand wrong view as wrong view and right view as right view, that’s your right view.\footnote{It is not just learning by rote, but the ability to make critical distinctions. } 

And\marginnote{5.1} what is wrong view? ‘There’s no meaning in giving, sacrifice, or offerings. There’s no fruit or result of good and bad deeds. There’s no afterlife. There’s no such thing as mother and father, or beings that are reborn spontaneously. And there’s no ascetic or brahmin who is rightly comported and rightly practiced, and who describes the afterlife after realizing it with their own insight.’ This is wrong view. 

And\marginnote{6.1} what is right view? Right view is twofold, I say. There is right view that is accompanied by defilements, partakes of good deeds, and ripens in attachments.\footnote{This unique phrase refers to the kind of right view that results in a good rebirth but not liberation. This kind of right view is not specifically Buddhist. Compare \textit{\textsanskrit{sāsavaṁ} \textsanskrit{upādāniyaṁ}} (the five grasping aggregates “accompanied by defilements and prone to being grasped”, \href{https://suttacentral.net/sn22.48/en/sujato\#2.2}{SN 22.48:2.2}) and \textit{iddhi \textsanskrit{sāsavā} \textsanskrit{saupadhikā}, ‘no \textsanskrit{ariyā}’ti} (“psychic powers that are accompanied by defilements and attachments, and are said to be ignoble”, \href{https://suttacentral.net/dn28/en/sujato\#18.3}{DN 28:18.3}). | For “partakes of good deeds” (\textit{\textsanskrit{puññabhāgiyā}}), see \href{https://suttacentral.net/an6.63/en/sujato}{AN 6.63}, where it refers to a rebirth that results from “defilement”. } And there is right view that is noble, undefiled, transcendent, a factor of the path.\footnote{This is the right view of one on the noble path. The suttas do not elsewhere distinguish these two kinds of path factors, and the “transcendent” analysis is not found in the parallels in Chinese (MA 189 at T i 735b) or Tibetan (Up 6.080 at D 4094 mngon pa, nyu 43b7). A similar distinction, however, is made in the Chinese translation of the \textsanskrit{Saṁyukta} Āgama of the (\textsanskrit{Mūla})-\textsanskrit{Sarvāstivāda} at SA 785 and SA 789, which contains analyses of the transcendent factors that are similar to those in Pali. | “Transcendent” (\textit{lokuttara}) means “leading beyond the world (of transmigration)”. Elsewhere in early texts it is used of discourses (\href{https://suttacentral.net/sn20.7/en/sujato\#1.6}{SN 20.7:1.6}, \href{https://suttacentral.net/sn55.53/en/sujato\#2.2}{SN 55.53:2.2}, \href{https://suttacentral.net/an2.47/en/sujato\#1.5}{AN 2.47:1.5}, \href{https://suttacentral.net/an5.79/en/sujato\#5.2}{AN 5.79:5.2}; see too \href{https://suttacentral.net/mn96/en/sujato\#12.12}{MN 96:12.12}), and of certain reflective realizations of a noble one (\href{https://suttacentral.net/mn48/en/sujato\#14.7}{MN 48:14.7}, \href{https://suttacentral.net/mn122/en/sujato\#18.1}{MN 122:18.1}). } 

And\marginnote{7.1} what is right view that is accompanied by defilements, partakes of good deeds, and ripens in attachments? ‘There is meaning in giving, sacrifice, and offerings. There are fruits and results of good and bad deeds. There is an afterlife. There are such things as mother and father, and beings that are reborn spontaneously. And there are ascetics and brahmins who are rightly comported and rightly practiced, and who describe the afterlife after realizing it with their own insight.’ This is right view that is accompanied by defilements, partakes of good deeds, and ripens in attachments. 

And\marginnote{8.1} what is right view that is noble, undefiled, transcendent, a factor of the path? It’s the wisdom—the faculty of wisdom, the power of wisdom, the awakening factor of investigation of principles, and right view as a factor of the path—in one intent on the noble, intent on the undefiled, who possesses the noble path and develops the noble path.\footnote{Whereas elsewhere right view is explained in terms of \emph{what} you understand (the four noble truths \href{https://suttacentral.net/mn141/en/sujato\#24.1}{MN 141:24.1}), this analysis shifts focus to \emph{how} you understand: wisdom functions in the context of the path as a whole. Later texts expand on this (eg. \href{https://suttacentral.net/mnd2/en/sujato\#82.4}{Mnd 2:82.4}, \href{https://suttacentral.net/ne17/en/sujato\#17.3}{Ne 17:17.3}, \href{https://suttacentral.net/ds2.1.1/en/sujato\#28.2}{Ds 2.1.1:28.2}). | As to the phrase “intent on the noble, intent on the undefiled”, the expressions \textit{ariyacitta} and  \textit{\textsanskrit{anāsavacitta}} are unique. It is tempting to translate \textit{\textsanskrit{anāsavacitta}} as “undefiled mind”, but that would pertain only to the arahant who has completed the path, whereas this is talking about one still in the process of developing the path. Here \textit{citta} must mean “intention” (eg. \href{https://suttacentral.net/mn120/en/sujato\#3.4}{MN 120:3.4}; cf. the later term \textit{bodhicitta}). Commentary is silent. } This is called right view that is noble, undefiled, transcendent, a factor of the path. 

They\marginnote{9.1} make an effort to give up wrong view and embrace right view: that’s their right effort. Mindfully they give up wrong view and take up right view: that’s their right mindfulness. So these three things keep running and circling around right view, namely: right view, right effort, and right mindfulness.\footnote{The path is linear in that each factor depends on the previous (\href{https://suttacentral.net/mn117/en/sujato\#34.3}{MN 117:34.3}, see too eg. \href{https://suttacentral.net/sn45.1/en/sujato}{SN 45.1}). But it is not \emph{just} linear, as the factors are always interacting. Right view understands what to do; right effort does the work; and right mindfulness monitors and checks. } 

In\marginnote{10.1} this context, right view comes first.\footnote{Right view is the anterior condition for all the path factors. } And how does right view come first? When you understand wrong thought as wrong thought and right thought as right thought, that’s your right view.\footnote{See \href{https://suttacentral.net/mn19/en/sujato}{MN 19} for how the bodhisatta developed this practice. } 

And\marginnote{11.1} what is wrong thought? Thoughts of sensuality, of malice, and of cruelty. This is wrong thought. 

And\marginnote{12.1} what is right thought? Right thought is twofold, I say. There is right thought that is accompanied by defilements, partakes of good deeds, and ripens in attachments. And there is right thought that is noble, undefiled, transcendent, a factor of the path. 

And\marginnote{13.1} what is right thought that is accompanied by defilements, partakes of good deeds, and ripens in attachments? Thoughts of renunciation, good will, and harmlessness. This is right thought that is accompanied by defilements. 

And\marginnote{14.1} what is right thought that is noble, undefiled, transcendent, a factor of the path? It’s the thinking—the placing of the mind, thought, planting, implanting, embedding of the mind, verbal process—in one intent on the noble, intent on the undefiled, who possesses the noble path and develops the noble path.\footnote{This passage shows that the scope of \textit{\textsanskrit{saṅkappa}} or \textit{vitakka} is wider than discursive “thought”, especially in deep meditation. This definition is similar to that in the Abhidhamma (eg. \href{https://suttacentral.net/vb3/en/sujato\#23.2}{Vb 3:23.2}) | \textit{Takka} (“thinking”) is normally “logic” but here is a synonym of \textit{vitakka}. | \textit{\textsanskrit{Appanā}} (“planting”) is where a movement enters its goal, as a river the ocean (\href{https://suttacentral.net/sn15.8/en/sujato\#2.3}{SN 15.8:2.3}), or a peg the wood (\href{https://suttacentral.net/mil3.3.13/en/sujato}{Mil 3.3.13}, for which see also \href{https://suttacentral.net/mn20.3.4/en/sujato}{MN 20.3.4}). | \textit{\textsanskrit{Byappanā}} is an intensive form that appears only in this definition. | \textit{Abhiniropana} is likewise a unique term, with the sense of “sinking into” or “embedding”. Compare Sanskrit \textit{\textsanskrit{āropaṇa}}, “placing or fixing in or upon”. | For “verbal process” see \href{https://suttacentral.net/mn44/en/sujato\#15.3}{MN 44:15.3} = \href{https://suttacentral.net/sn41.6/en/sujato\#2.4}{SN 41.6:2.4}. } This is right thought that is noble. 

They\marginnote{15.1} make an effort to give up wrong thought and embrace right thought: that’s their right effort. Mindfully they give up wrong thought and take up right thought: that’s their right mindfulness. So these three things keep running and circling around right thought, namely: right view, right effort, and right mindfulness. 

In\marginnote{16.1} this context, right view comes first. And how does right view come first? When you understand wrong speech as wrong speech and right speech as right speech, that’s your right view. 

And\marginnote{17.1} what is wrong speech? Speech that’s false, divisive, harsh, or nonsensical. This is wrong speech. 

And\marginnote{18.1} what is right speech? Right speech is twofold, I say. There is right speech that is accompanied by defilements, partakes of good deeds, and ripens in attachments. And there is right speech that is noble, undefiled, transcendent, a factor of the path. 

And\marginnote{19.1} what is right speech that is accompanied by defilements, partakes of good deeds, and ripens in attachments? The refraining from lying, divisive speech, harsh speech, and talking nonsense.\footnote{Right speech is defined solely as abstinence here, whereas elsewhere it also includes actually speaking in a constructive way (eg. \href{https://suttacentral.net/mn114/en/sujato\#6.18}{MN 114:6.18}). The negative mode of definition is more technically precise, as it includes times when one is silent or in deep meditation. } This is right speech that is accompanied by defilements. 

And\marginnote{20.1} what is right speech that is noble, undefiled, transcendent, a factor of the path? It’s the desisting, abstaining, abstinence, and refraining from the four kinds of bad verbal conduct in one intent on the noble, intent on the undefiled, who possesses the noble path and develops the noble path.\footnote{This sequence of synonyms (\textit{\textsanskrit{ārati} virati \textsanskrit{paṭivirati} \textsanskrit{veramaṇī}}) is otherwise found only in late canonical texts (eg. \href{https://suttacentral.net/cnd6/en/sujato\#11.6}{Cnd 6:11.6}, \href{https://suttacentral.net/vb4/en/sujato\#61.2}{Vb 4:61.2}, \href{https://suttacentral.net/pli-tv-pvr4/en/sujato\#5.2}{Pvr 4:5.2}). } This is right speech that is noble. 

They\marginnote{21.1} make an effort to give up wrong speech and embrace right speech: that’s their right effort. Mindfully they give up wrong speech and take up right speech: that’s their right mindfulness. So these three things keep running and circling around right speech, namely: right view, right effort, and right mindfulness. 

In\marginnote{22.1} this context, right view comes first. And how does right view come first? When you understand wrong action as wrong action and right action as right action, that’s your right view. 

And\marginnote{23.1} what is wrong action? Killing living creatures, stealing, and sexual misconduct. This is wrong action. 

And\marginnote{24.1} what is right action? Right action is twofold, I say. There is right action that is accompanied by defilements, partakes of good deeds, and ripens in attachments. And there is right action that is noble, undefiled, transcendent, a factor of the path. 

And\marginnote{25.1} what is right action that is accompanied by defilements, partakes of good deeds, and ripens in attachments? Refraining from killing living creatures, stealing, and sexual misconduct. This is right action that is accompanied by defilements. 

And\marginnote{26.1} what is right action that is noble, undefiled, transcendent, a factor of the path? It’s the desisting, abstaining, abstinence, and refraining from the three kinds of bad bodily conduct in one intent on the noble, intent on the undefiled, who possesses the noble path and develops the noble path. This is right action that is noble. 

They\marginnote{27.1} make an effort to give up wrong action and embrace right action: that’s their right effort. Mindfully they give up wrong action and take up right action: that’s their right mindfulness. So these three things keep running and circling around right action, namely: right view, right effort, and right mindfulness. 

In\marginnote{28.1} this context, right view comes first. And how does right view come first? When you understand wrong livelihood as wrong livelihood and right livelihood as right livelihood, that’s your right view. 

And\marginnote{29.1} what is wrong livelihood? Deceit, flattery, hinting, and belittling, and using material things to chase after other material things.\footnote{This applies to everyone, but especially to monastics, in reference to unwholesome ways of seeking requisites and material profits. For laypeople, five kinds of wrong livelihood are specified: trade in weapons, living creatures, meat, intoxicants, and poison (\href{https://suttacentral.net/an5.177/en/sujato\#1.3}{AN 5.177:1.3}). } This is wrong livelihood. 

And\marginnote{30.1} what is right livelihood? Right livelihood is twofold, I say. There is right livelihood that is accompanied by defilements, partakes of good deeds, and ripens in attachments. And there is right livelihood that is noble, undefiled, transcendent, a factor of the path. 

And\marginnote{31.1} what is right livelihood that is accompanied by defilements, partakes of good deeds, and ripens in attachments? It’s when a noble disciple gives up wrong livelihood and earns a living by right livelihood.\footnote{Defined for a monastic in the long section on ethics in the \textsanskrit{Sāmaññaphalasutta} (\href{https://suttacentral.net/dn2/en/sujato\#56.1}{DN 2:56.1}) and for a lay person as trade in weapons, living creatures, meat, intoxicants, and poisons (\href{https://suttacentral.net/an5.177/en/sujato\#1.3}{AN 5.177:1.3}). } This is right livelihood that is accompanied by defilements. 

And\marginnote{32.1} what is right livelihood that is noble, undefiled, transcendent, a factor of the path? It’s the desisting, abstaining, abstinence, and refraining from wrong livelihood in one intent on the noble, intent on the undefiled, who possesses the noble path and develops the noble path. This is right livelihood that is noble. 

They\marginnote{33.1} make an effort to give up wrong livelihood and embrace right livelihood: that’s their right effort. Mindfully they give up wrong livelihood and take up right livelihood: that’s their right mindfulness. So these three things keep running and circling around right livelihood, namely: right view, right effort, and right mindfulness.\footnote{This, the fifth factor of the path, is the last factor described as encompassed by right view, effort, and mindfulness. I think a purpose of this discourse is to make it clear that they \emph{also} apply to the first five factors, the final three being taken as self-evident. } 

In\marginnote{34.1} this context, right view comes first. And how does right view come first? Right view gives rise to right thought. Right thought gives rise to right speech. Right speech gives rise to right action. Right action gives rise to right livelihood. Right livelihood gives rise to right effort. Right effort gives rise to right mindfulness. Right mindfulness gives rise to right immersion. Right immersion gives rise to right knowledge. Right knowledge gives rise to right freedom.\footnote{This extends the linear arising to two additional factors: right knowledge and right freedom (see also \href{https://suttacentral.net/dn18/en/sujato\#27.8}{DN 18:27.8}, \href{https://suttacentral.net/an10.103/en/sujato\#2.3}{AN 10.103:2.3}). These are not divided into two kinds, as they apply solely to the perfected ones. } So the trainee has eight factors, while the perfected one has ten factors.\footnote{Right knowledge is the wisdom perfected by the ending of defilements, while right freedom is the liberation from rebirth. } And here too, the eradication of many bad, unskillful qualities is fully developed due to right knowledge. 

In\marginnote{35.1} this context, right view comes first. And how does right view come first? For one of right view, wrong view is worn away.\footnote{The term \textit{\textsanskrit{nijjiṇṇa}} (“worn away”) is associated with the Jains (eg. \href{https://suttacentral.net/mn14/en/sujato\#17.8}{MN 14:17.8}, \href{https://suttacentral.net/mn101/en/sujato\#10.8}{MN 101:10.8}). } And the many bad, unskillful qualities that arise because of wrong view are worn away. And because of right view, many skillful qualities are fully developed. For one of right thought, wrong thought is worn away. … For one of right speech, wrong speech is worn away. … For one of right action, wrong action is worn away. … For one of right livelihood, wrong livelihood is worn away. … For one of right effort, wrong effort is worn away. … For one of right mindfulness, wrong mindfulness is worn away. … For one of right immersion, wrong immersion is worn away. … For one of right knowledge, wrong knowledge is worn away. … For one of right freedom, wrong freedom is worn away. And the many bad, unskillful qualities that arise because of wrong freedom are worn away. And because of right freedom, many skillful qualities are fully developed. 

So\marginnote{36.1} there are twenty on the side of the skillful, and twenty on the side of the unskillful.\footnote{The twenty are the ten factors, plus ten for the relevant qualities they promote. } This exposition of the teaching on the Great Forty has been rolled forth. And it cannot be rolled back by any ascetic or brahmin or god or \textsanskrit{Māra} or divinity or by anyone in the world. 

If\marginnote{37.1} any ascetic or brahmin imagines they can criticize and reject the exposition of the teaching on the Great Forty, they deserve rebuttal and criticism on ten legitimate grounds in this very life. If such a gentleman criticizes right view, they praise and honor the ascetics and brahmins who have wrong view. If they criticize right thought … right speech … right action … right livelihood … right effort … right mindfulness … right immersion … right knowledge … right freedom, they praise and honor the ascetics and brahmins who have wrong freedom. If any ascetic or brahmin imagines they can criticize and reject the exposition of the teaching on the Great Forty, they deserve rebuttal and criticism on these ten legitimate grounds in this very life. 

Even\marginnote{38.1} those wanderers of the past, Vassa and \textsanskrit{Bhañña} of \textsanskrit{Ukkalā}, who adhered to the doctrines of no-cause, inaction, and nihilism, didn’t imagine that the Great Forty should be criticized or rejected.\footnote{This pair is also mentioned in the same way regarding different teachings at \href{https://suttacentral.net/an4.30/en/sujato\#8.2}{AN 4.30:8.2} and \href{https://suttacentral.net/sn22.62/en/sujato\#13.2}{SN 22.62:13.2}. Nothing is known of them apart from these passages. | \textsanskrit{Ukkalā} lay on the Bay of Bengal in the region today called Odisha. It was the origin of the merchants Tapussa and Bhallika who offered the first meal to the Buddha (\href{https://suttacentral.net/pli-tv-kd1/en/sujato\#4.2.1}{Kd 1:4.2.1}, \href{https://suttacentral.net/an1.248/en/sujato\#1.1}{AN 1.248:1.1}). } Why is that? For fear of blame, attack, and condemnation.” 

That\marginnote{38.4} is what the Buddha said. Satisfied, the mendicants approved what the Buddha said. 

%
\section*{{\suttatitleacronym MN 118}{\suttatitletranslation Mindfulness of Breathing }{\suttatitleroot Ānāpānassatisutta}}
\addcontentsline{toc}{section}{\tocacronym{MN 118} \toctranslation{Mindfulness of Breathing } \tocroot{Ānāpānassatisutta}}
\markboth{Mindfulness of Breathing }{Ānāpānassatisutta}
\extramarks{MN 118}{MN 118}

\scevam{So\marginnote{1.1} I have heard.\footnote{Mindfulness of breathing is taught partially in the \textsanskrit{Satipaṭṭhānasutta} (\href{https://suttacentral.net/mn10/en/sujato\#4.3}{MN 10:4.3} = \href{https://suttacentral.net/dn22/en/sujato\#2.3}{DN 22:2.3}) and the \textsanskrit{Kāyagatāsatisutta} (\href{https://suttacentral.net/mn119/en/sujato\#4.2}{MN 119:4.2}), and fully in the \textsanskrit{Rāhulovādasutta} (\href{https://suttacentral.net/mn62/en/sujato\#24.3}{MN 62:24.3}), the \textsanskrit{Girimānandasutta} (\href{https://suttacentral.net/an10.60/en/sujato}{AN 10.60}), and throughout the relevant \textsanskrit{Saṁyutta} (\href{https://suttacentral.net/sn54.1/en/sujato}{SN 54.1} ff.); it is also in the Vinaya (\href{https://suttacentral.net/pli-tv-bu-vb-pj3/en/sujato\#1.3.6}{Bu Pj 3:1.3.6}) and in the quasi-Abhidhamma text \textsanskrit{Paṭisambhidāmagga} (\href{https://suttacentral.net/ps1.3/en/sujato\#60.3}{Ps 1.3:60.3}). It is regarded in the tradition as the method practiced by the Buddha himself, and has remained the most popular and universal meditation to this day, both in Buddhism and generally. } }At one time the Buddha was staying near \textsanskrit{Sāvatthī} in the stilt longhouse of \textsanskrit{Migāra}’s mother in the Eastern Monastery, together with several well-known senior disciples, such as the venerables \textsanskrit{Sāriputta}, \textsanskrit{Mahāmoggallāna}, \textsanskrit{Mahākassapa}, \textsanskrit{Mahākaccāna}, \textsanskrit{Mahākoṭṭhita}, \textsanskrit{Mahākappina}, \textsanskrit{Mahācunda}, Anuruddha, Revata, Ānanda, and others.\footnote{The similar list of great disciples is found at \href{https://suttacentral.net/ud1.5/en/sujato\#1.3}{Ud 1.5:1.3}, where the Buddha described them as true brahmins. } 

Now\marginnote{2.1} at that time the senior mendicants were advising and instructing the junior mendicants.\footnote{This reminds us that what is recorded in the suttas is not the totality of what was taught. There are several instances where more personal meditation instructions are given (eg. \href{https://suttacentral.net/an9.3/en/sujato}{AN 9.3}), and this must have been ongoing. } Some senior mendicants instructed ten mendicants, while some instructed twenty, thirty, or forty. Being instructed by the senior mendicants, the junior mendicants realized a higher distinction than they had before. 

Now,\marginnote{3.1} at that time it was the sabbath—the full moon on the fifteenth day—and the Buddha was sitting surrounded by a \textsanskrit{Saṅgha} of monks for the invitation to admonish.\footnote{This was the end of the rains retreat, during which mendicants stayed in one place. To conclude the retreat, the mendicants gather to invite admonition from each other, clearing the air of any conflicts or tensions that might have arisen during the retreat. This is done instead of the regular recitation of the \textsanskrit{Pātimokkha} rules. Discussed in detail at \href{https://suttacentral.net/pli-tv-kd4/en/sujato}{Kd 4}. } Then the Buddha looked around the \textsanskrit{Saṅgha} of mendicants, who were so very silent. He addressed them: 

“I\marginnote{4.1} am satisfied, mendicants, with this practice. My heart is satisfied with this practice. So you should rouse up even more energy for attaining the unattained, achieving the unachieved, and realizing the unrealized. I will wait here in \textsanskrit{Sāvatthī} for the \textsanskrit{Komudī} full moon of the fourth month.”\footnote{The “fourth month” of the rainy season (\textit{vassa}), of which the rains retreat (\textit{\textsanskrit{vassāvāsa}}) is normally the first three. When the retreat is finished, mendicants usually stay in the monastery for a while to complete the \textit{\textsanskrit{kaṭhina}} (\href{https://suttacentral.net/pli-tv-kd7/en/sujato}{Kd 7}), repair their robes, and otherwise make preparations for wandering. During this time, it was customary for mendicants who had spent the rains elsewhere to visit \textsanskrit{Sāvatthī} to see the Buddha (\href{https://suttacentral.net/pli-tv-kd4/en/sujato\#1.8.1}{Kd 4:1.8.1}, \href{https://suttacentral.net/pli-tv-bu-vb-pj4/en/sujato\#1.2.1}{Bu Pj 4:1.2.1}, \href{https://suttacentral.net/pli-tv-bu-vb-pc8/en/sujato\#1.2.1}{Bu Pc 8:1.2.1}). This month, which falls between the full moons of October and November, is called \textsanskrit{Kattikā} after the constellation Pleiades. The final day of \textsanskrit{Kattikā} is the full moon of \textsanskrit{Komudī}, named after the white lotuses said to bloom at that time, and after the brilliance of the moon in the clear skies following the rains (see eg. \href{https://suttacentral.net/mn79/en/sujato\#17.2}{MN 79:17.2}). } 

Mendicants\marginnote{5.1} from around the country heard about this, and came down to \textsanskrit{Sāvatthī} to see the Buddha. 

And\marginnote{6.1} those senior mendicants instructed the junior mendicants even more. Some senior mendicants instructed ten mendicants, while some instructed twenty, thirty, or forty. Being instructed by the senior mendicants, the junior mendicants realized a higher distinction than they had before. 

Now,\marginnote{7.1} at that time it was the sabbath—the \textsanskrit{Komudī} full moon on the fifteenth day of the fourth month—and the Buddha was sitting in the open surrounded by a \textsanskrit{Saṅgha} of monks. Then the Buddha looked around the \textsanskrit{Saṅgha} of mendicants, who were so very silent. He addressed them:\footnote{A similar passage is found at \href{https://suttacentral.net/an4.190/en/sujato}{AN 4.190}. } 

“This\marginnote{8.1} assembly has no chaff, mendicants, it is free of chaff, pure, and consolidated in the core. Such is this \textsanskrit{Saṅgha} of mendicants, such is this assembly! An assembly such as this is worthy of offerings dedicated to the gods, worthy of hospitality, worthy of a religious donation, worthy of greeting with joined palms, and is the supreme field of merit for the world. Such is this \textsanskrit{Saṅgha} of mendicants, such is this assembly! For an assembly such as this, giving little becomes much, while giving much becomes even more. Such is this \textsanskrit{Saṅgha} of mendicants, such is this assembly! An assembly such as this is rarely seen in the world. Such is this \textsanskrit{Saṅgha} of mendicants, such is this assembly! An assembly such as this is worth traveling many leagues to see, even if you have to carry your own provisions in a knapsack.\footnote{“Knapsack” is \textit{\textsanskrit{puṭosa}}, where \textit{\textsanskrit{puṭa}} is “bag” and \textit{osa} is “food” (Sanskrit \textit{avasa}). The compound is a dative-dependent \textit{tappurisa} with reversal of the usual order, yielding the sense “bag for food”. } 

For\marginnote{9.1} in this \textsanskrit{Saṅgha} there are perfected mendicants, who have ended the defilements, completed the spiritual journey, done what had to be done, laid down the burden, achieved their own goal, utterly ended the fetter of continued existence, and are rightly freed through enlightenment. There are such mendicants in this \textsanskrit{Saṅgha}. 

In\marginnote{10.1} this \textsanskrit{Saṅgha} there are mendicants who, with the ending of the five lower fetters are reborn spontaneously. They are extinguished there, and are not liable to return from that world. There are such mendicants in this \textsanskrit{Saṅgha}. 

In\marginnote{11.1} this \textsanskrit{Saṅgha} there are mendicants who, with the ending of three fetters, and the weakening of greed, hate, and delusion, are once-returners. They come back to this world once only, then make an end of suffering. There are such mendicants in this \textsanskrit{Saṅgha}. 

In\marginnote{12.1} this \textsanskrit{Saṅgha} there are mendicants who, with the ending of three fetters are stream-enterers, not liable to be reborn in the underworld, bound for awakening. There are such mendicants in this \textsanskrit{Saṅgha}. 

In\marginnote{13.1} this \textsanskrit{Saṅgha} there are mendicants who are committed to developing the four kinds of mindfulness meditation … the four right efforts … the four bases of psychic power … the five faculties … the five powers … the seven awakening factors … the noble eightfold path. There are such mendicants in this \textsanskrit{Saṅgha}. In this \textsanskrit{Saṅgha} there are mendicants who are committed to developing the meditation on love … compassion … rejoicing … equanimity … ugliness … impermanence. There are such mendicants in this \textsanskrit{Saṅgha}. In this \textsanskrit{Saṅgha} there are mendicants who are committed to developing the meditation on mindfulness of breathing. 

Mendicants,\marginnote{15.1} when mindfulness of breathing is developed and cultivated it is very fruitful and beneficial.\footnote{The main contents of the discourse are summarized in these four statements, each of which is expanded in the four sections that follow. } Mindfulness of breathing, when developed and cultivated, fulfills the four kinds of mindfulness meditation. The four kinds of mindfulness meditation, when developed and cultivated, fulfill the seven awakening factors. And the seven awakening factors, when developed and cultivated, fulfill knowledge and freedom. 

And\marginnote{16.1} how is mindfulness of breathing developed and cultivated to be very fruitful and beneficial? 

It’s\marginnote{17.1} when a mendicant—gone to a wilderness, or to the root of a tree, or to an empty hut—sits down cross-legged, sets their body straight, and establishes mindfulness in their presence.\footnote{The situation here—a mendicant gone to the forest—establishes that this practice takes place in the wider context of the Gradual Training. Indeed, this whole sutta can be understood as an expansion of this phrase, mentioned briefly at \href{https://suttacentral.net/dn2/en/sujato\#67.3}{DN 2:67.3}. } Just mindful, they breathe in. Mindful, they breathe out.\footnote{The most fundamental meditation instruction. Notice how the Buddha phrases it: not “concentrate on the breath” as an object, but rather “breathing” as an activity to which one brings mindfulness. The stages of breath meditation are not meant to be done deliberately, but to be observed and understood as the natural process of deepening meditation. } 

Breathing\marginnote{18.1} in heavily they know: ‘I’m breathing in heavily.’ Breathing out heavily they know: ‘I’m breathing out heavily.’\footnote{In the beginning, the breath is somewhat rough and coarse. The Pali idiom is “long” and “short” breath, but in English we usually say one breathes “heavily” or “lightly”. } When breathing in lightly they know: ‘I’m breathing in lightly.’ Breathing out lightly they know: ‘I’m breathing out lightly.’\footnote{Over time, the breath becomes more subtle and soft. } They practice like this: ‘I’ll breathe in experiencing the whole body.’ They practice like this: ‘I’ll breathe out experiencing the whole body.’\footnote{The “whole body” (\textit{\textsanskrit{sabbakāya}}) here refers to the breath, marking the fuller and more continuous awareness that arises with tranquility. Some, however, interpret it as the “whole physical body”, broadening awareness to encompass the movement and settling of energies throughout the body. } They practice like this: ‘I’ll breathe in stilling the physical process.’ They practice like this: ‘I’ll breathe out stilling the physical process.’\footnote{The “physical process” (\textit{\textsanskrit{kāyasaṅkhāra}}) is the breath itself (\href{https://suttacentral.net/sn41.6/en/sujato\#1.8}{SN 41.6:1.8}). This can become so soft as to be imperceptible. } 

They\marginnote{19.1} practice like this: ‘I’ll breathe in experiencing rapture.’ They practice like this: ‘I’ll breathe out experiencing rapture.’\footnote{“Rapture” (\textit{\textsanskrit{pīti}}) is a joyous emotional response to pleasure, usually a spiritual sense of elevation or uplift in meditation. Normally rapture is said to arise before tranquility as part of the process leading towards absorption (see below, \href{https://suttacentral.net/mn118/en/sujato\#34.1}{MN 118:34.1}). In mindfulness of breathing, however, the body typically becomes quite tranquil before rapture arises. } They practice like this: ‘I’ll breathe in experiencing bliss.’ They practice like this: ‘I’ll breathe out experiencing bliss.’\footnote{The uplifting quality of rapture becomes sublimated to a more subtle, pervasive pleasure or bliss. The meditator refines and stabilizes these positive emotions in their journey towards absorption, which occurs when all necessary factors are fully developed and perfectly balanced in a subtle union. } They practice like this: ‘I’ll breathe in experiencing mental processes.’ They practice like this: ‘I’ll breathe out experiencing mental processes.’\footnote{“Mental processes” are said to be “perception and feeling” (\href{https://suttacentral.net/mn44/en/sujato\#14.1}{MN 44:14.1}, \href{https://suttacentral.net/sn41.6/en/sujato\#1.7}{SN 41.6:1.7}). The “feelings” include include rapture and bliss. “Perception” in meditation includes the “light” (\href{https://suttacentral.net/mn128/en/sujato\#28.2}{MN 128:28.2}) or “forms” (\href{https://suttacentral.net/mn77/en/sujato\#23.14}{MN 77:23.14}) often experienced by meditators at this point. Later texts call this a \textit{nimitta}, a “sign” that the mind is becoming subtle and the hindrances quiet as the meditator nears absorption. } They practice like this: ‘I’ll breathe in stilling mental processes.’ They practice like this: ‘I’ll breathe out stilling mental processes.’\footnote{A paradox of meditation is that the stronger things are, the more subtle they become. In the initial stages of practice, it is common to have overwhelming experiences of rapture, with tears of joy, or dramatic and striking “visions”. Inexperienced meditators often take this to signify a profound realization. With patient mindfulness, however, the feelings and perceptions settle down and become unified, so they are no longer experienced as separate things. Just as the physical breath virtually disappeared, here too the feelings and perceptions can become so subtle as to feel like the simple, normal state of the mind. } 

They\marginnote{20.1} practice like this: ‘I’ll breathe in experiencing the mind.’ They practice like this: ‘I’ll breathe out experiencing the mind.’\footnote{With the stilling of “mental processes”, the mind itself becomes clear, the center of awareness that knows all of this. These four factors focus on the experience of absorption. } They practice like this: ‘I’ll breathe in gladdening the mind.’ They practice like this: ‘I’ll breathe out gladdening the mind.’\footnote{This applies especially to the first two absorptions. } They practice like this: ‘I’ll breathe in immersing the mind in \textsanskrit{samādhi}.’ They practice like this: ‘I’ll breathe out immersing the mind in \textsanskrit{samādhi}.’\footnote{“Immersion” (\textit{\textsanskrit{samādhi}}) is deep meditative stillness. The word conveys the sense of “gathered”, “collected”, with a secondary sense of “ignited”, “illuminated”. It emphasizes both the stability of the mind as well as the unification of all the mental factors present (see \href{https://suttacentral.net/mn44/en/sujato\#12.2}{MN 44:12.2} and note there). } They practice like this: ‘I’ll breathe in freeing the mind.’ They practice like this: ‘I’ll breathe out freeing the mind.’\footnote{The mind is liberated from hindrances in absorption. This is a state so very refined it is called the “bliss of awakening” (\href{https://suttacentral.net/mn139/en/sujato\#9.18}{MN 139:9.18}), albeit temporarily (\href{https://suttacentral.net/an5.149/en/sujato}{AN 5.149}). } 

They\marginnote{21.1} practice like this: ‘I’ll breathe in observing impermanence.’ They practice like this: ‘I’ll breathe out observing impermanence.’\footnote{The first twelve steps focused on serenity (\textit{samatha}), and now the final four steps shift focus to discernment (or “insight”, \textit{\textsanskrit{vipassanā}}). This exemplifies the most common course of practice, where serenity precedes discernment (\href{https://suttacentral.net/an4.170/en/sujato\#3.2}{AN 4.170:3.2}). Here, impermanence is contemplated in the context of mindfulness of breathing itself. The entire course of practice has seen the physical energies of the breath become subtle and fade, then the mental processes, then the mind itself is transformed. The meditator reflects on these changes and sees not simply that change happens, but how changes happen due to causes. } They practice like this: ‘I’ll breathe in observing fading away.’ They practice like this: ‘I’ll breathe out observing fading away.’\footnote{\textit{\textsanskrit{Virāga}} means both “dispassion” in the sense of letting go desire, and “fading away” in the sense of its gradual disappearance. } They practice like this: ‘I’ll breathe in observing cessation.’ They practice like this: ‘I’ll breathe out observing cessation.’\footnote{This refers to the permanent ending of the fetters or underlying tendencies. } They practice like this: ‘I’ll breathe in observing letting go.’ They practice like this: ‘I’ll breathe out observing letting go.’\footnote{This is the full renunciation of all worldly desires that occurs with arahantship (“perfection”). These final four stages illustrating the progress of insight can even be mapped on to the four stages of awakening: the stream-enterer sees that “whatever has a beginning also has an end”; for the once-returner, greed and hate are fading away; for the non-returner, greed and hate have ceased; and for the arahant, all defilements are relinquished. } 

Mindfulness\marginnote{22.1} of breathing, when developed and cultivated in this way, is very fruitful and beneficial. 

And\marginnote{23.1} how is mindfulness of breathing developed and cultivated so as to fulfill the four kinds of mindfulness meditation? 

Whenever\marginnote{24.1} a mendicant knows that they breathe heavily, or lightly, or experiencing the whole body, or stilling physical processes—at that time they’re meditating by observing an aspect of the body—keen, aware, and mindful, rid of covetousness and displeasure for the world. For I say that the in-breaths and out-breaths are an aspect of the body.\footnote{The Pali here is \textit{\textsanskrit{kāyesu} \textsanskrit{kāyaññataraṁ}}, literally, “a body among the bodies”. What it means is that the breath, as one of the four material elements, is a certain aspect of our physical embodiment which is selected here for contemplation, rather than, say, the parts of the body or the postures as we find in the full presentation of \textit{\textsanskrit{satipaṭṭhāna}} at \href{https://suttacentral.net/mn10/en/sujato}{MN 10} or \href{https://suttacentral.net/mn119/en/sujato}{MN 119}. } That’s why at that time a mendicant is meditating by observing an aspect of the body—keen, aware, and mindful, rid of covetousness and displeasure for the world. 

Whenever\marginnote{25.1} a mendicant practices breathing while experiencing rapture, or experiencing bliss, or experiencing mental processes, or stilling mental processes—at that time they meditate observing an aspect of feelings—keen, aware, and mindful, rid of covetousness and displeasure for the world. For I say that careful application of mind to the in-breaths and out-breaths is an aspect of feelings.\footnote{Likewise, the sense here is that the pleasant feelings of breath meditation are part of the full spectrum of feelings contemplated in \textit{\textsanskrit{satipaṭṭhāna}}, which include painful feelings (\href{https://suttacentral.net/mn10/en/sujato\#32.1}{MN 10:32.1}). The meditator “carefully” (\textit{\textsanskrit{sādhukaṁ}}) focuses on these feelings as they manifest in the breath. The point is that “breath” is not purely a physical phenomenon. The material property of “air” is its most obvious dimension, but we experience breath as a complex that includes emotional and cognitive dimensions. As the physical process becomes subtle, feeling becomes apparent as an aspect of the breath that has always been there, but not clearly perceived. } That’s why at that time a mendicant is meditating by observing an aspect of feelings—keen, aware, and mindful, rid of covetousness and displeasure for the world. 

Whenever\marginnote{26.1} a mendicant practices breathing while experiencing the mind, or gladdening the mind, or immersing the mind in \textsanskrit{samādhi}, or freeing the mind—at that time they meditate observing an aspect of the mind—keen, aware, and mindful, rid of covetousness and displeasure for the world. There is no development of mindfulness of breathing for someone who is unmindful and lacks awareness, I say.\footnote{Here the manner of explanation is changed, but the sense is not dissimilar. At \href{https://suttacentral.net/mn10/en/sujato\#34.1}{MN 10:34.1} the contemplation of mind includes states of mind characterized by greed, hate, and delusion, which are experienced by one lacking mindfulness and awareness. Mindfulness of breathing, however, focuses on the purified mind states of deep meditation. It is also perhaps noteworthy that the third absorption is said to be characterized by mindfulness and awareness, while the fourth absorption is characterized by equanimity, which is highlighted in the next step. } That’s why at that time a mendicant is meditating by observing an aspect of the mind—keen, aware, and mindful, rid of covetousness and displeasure for the world. 

Whenever\marginnote{27.1} a mendicant practices breathing while observing impermanence, or observing fading away, or observing cessation, or observing letting go—at that time they meditate observing an aspect of principles—keen, aware, and mindful, rid of covetousness and displeasure for the world. Having seen with wisdom the giving up of covetousness and displeasure, they watch over closely with equanimity.\footnote{“Covetousness and displeasure” are the defilements relinquished in \textit{\textsanskrit{satipaṭṭhāna}} itself (\href{https://suttacentral.net/mn10/en/sujato\#3.2}{MN 10:3.2}). This passage refers to reviewing or reflective knowledge, looking back over the course of meditation, seeing how the defilements are abandoned, and understanding how this happens according to the “principles” (\textit{\textsanskrit{dhammā}}) of cause and effect. With understanding comes equanimity, as one is no longer attached. } That’s why at that time a mendicant is meditating by observing an aspect of principles—keen, aware, and mindful, rid of covetousness and displeasure for the world. 

That’s\marginnote{28.1} how mindfulness of breathing, when developed and cultivated, fulfills the four kinds of mindfulness meditation. 

And\marginnote{29.1} how are the four kinds of mindfulness meditation developed and cultivated so as to fulfill the seven awakening factors?\footnote{The seven awakening factors are included in the fourth \textit{\textsanskrit{satipaṭṭhāna}} under contemplation of principles (\href{https://suttacentral.net/mn10/en/sujato\#42.1}{MN 10:42.1}). There, their primary function is to eliminate the five hindrances, the causal contemplation of which is a prominent feature of suttas on this topic (\href{https://suttacentral.net/sn46.2/en/sujato}{SN 46.2}). Thus this expands on the details of the “observation of principles”. } 

Whenever\marginnote{30.1} a mendicant meditates by observing an aspect of the body, at that time their mindfulness is established and lucid. At such a time, a mendicant has activated the awakening factor of mindfulness; they develop it and perfect it.\footnote{The “awakening factor of mindfulness” may, as here, refer to meditation, or else to the recollection of the teachings (\href{https://suttacentral.net/sn46.3/en/sujato\#1.9}{SN 46.3:1.9}). } 

As\marginnote{31.1} they live mindfully in this way they investigate, explore, and inquire into that principle with wisdom. At such a time, a mendicant has activated the awakening factor of investigation of principles; they develop it and perfect it.\footnote{Likewise, \textit{dhamma} can refer as here to the “principles” of cause and effect, or else the “teachings” that are recollected (\href{https://suttacentral.net/sn46.3/en/sujato\#2.4}{SN 46.3:2.4}). } 

As\marginnote{32.1} they investigate principles with wisdom in this way their energy is roused up and unflagging. At such a time, a mendicant has activated the awakening factor of energy; they develop it and perfect it. 

When\marginnote{33.1} they’re energetic, rapture not of the flesh arises. At such a time, a mendicant has activated the awakening factor of rapture; they develop it and perfect it. 

When\marginnote{34.1} the mind is full of rapture, the body and mind become tranquil. At such a time, a mendicant has activated the awakening factor of tranquility; they develop it and perfect it. 

When\marginnote{35.1} the body is tranquil and they feel bliss, the mind becomes immersed in \textsanskrit{samādhi}. At such a time, a mendicant has activated the awakening factor of immersion; they develop it and perfect it. 

They\marginnote{36.1} closely watch over that mind immersed in \textsanskrit{samādhi}. At such a time, a mendicant has activated the awakening factor of equanimity; they develop it and perfect it. 

Whenever\marginnote{37.1} a mendicant meditates by observing an aspect of feelings … mind … principles, at that time their mindfulness is established and lucid. At such a time, a mendicant has activated the awakening factor of mindfulness … investigation of principles … energy … rapture … tranquility … immersion … equanimity. 

That’s\marginnote{40.1} how the four kinds of mindfulness meditation, when developed and cultivated, fulfill the seven awakening factors. 

And\marginnote{41.1} how are the seven awakening factors developed and cultivated so as to fulfill knowledge and freedom? 

It’s\marginnote{42.1} when a mendicant develops the awakening factors of mindfulness, investigation of principles, energy, rapture, tranquility, immersion, 

and\marginnote{42.7} equanimity, which rely on seclusion, fading away, and cessation, and ripen as letting go.\footnote{The final three terms here are almost identical to the final three stages of mindfulness of breathing: fading away, cessation, and letting go. One of the purposes of this sutta has been to show the deep interconnectedness of all these practices. } 

That’s\marginnote{43.1} how the seven awakening factors, when developed and cultivated, fulfill knowledge and freedom.” 

That\marginnote{43.2} is what the Buddha said. Satisfied, the mendicants approved what the Buddha said. 

%
\section*{{\suttatitleacronym MN 119}{\suttatitletranslation Mindfulness of the Body }{\suttatitleroot Kāyagatāsatisutta}}
\addcontentsline{toc}{section}{\tocacronym{MN 119} \toctranslation{Mindfulness of the Body } \tocroot{Kāyagatāsatisutta}}
\markboth{Mindfulness of the Body }{Kāyagatāsatisutta}
\extramarks{MN 119}{MN 119}

\scevam{So\marginnote{1.1} I have heard.\footnote{This discourse features the same meditation practices found under the “observation of the body” in the \textsanskrit{Satipaṭṭhānasutta} (\href{https://suttacentral.net/mn10/en/sujato}{MN 10}). } }At one time the Buddha was staying near \textsanskrit{Sāvatthī} in Jeta’s Grove, \textsanskrit{Anāthapiṇḍika}’s monastery. 

Then\marginnote{2.1} after the meal, on return from almsround, several mendicants sat together in the assembly hall and this discussion came up among them. 

“It’s\marginnote{2.2} incredible, reverends, it’s amazing, how the Blessed One, who knows and sees, the perfected one, the fully awakened Buddha has said that mindfulness of the body, when developed and cultivated, is very fruitful and beneficial.” 

But\marginnote{2.4} their conversation was left unfinished. Then the Buddha came out of retreat and went to the assembly hall. He sat on the seat spread out and addressed the mendicants, “Mendicants, what were you sitting talking about just now? What conversation was left unfinished?” 

So\marginnote{2.7} the mendicants told him what they had been talking about. The Buddha said: 

“And\marginnote{3.1} how, mendicants, is mindfulness of the body developed and cultivated to be very fruitful and beneficial? 

It’s\marginnote{4.1} when a mendicant—gone to a wilderness, or to the root of a tree, or to an empty hut—sits down cross-legged, sets their body straight, and establishes mindfulness in their presence. Just mindful, they breathe in. Mindful, they breathe out. Breathing in heavily they know: ‘I’m breathing in heavily.’ Breathing out heavily they know: ‘I’m breathing out heavily.’ When breathing in lightly they know: ‘I’m breathing in lightly.’ Breathing out lightly they know: ‘I’m breathing out lightly.’ They practice like this: ‘I’ll breathe in experiencing the whole body.’ They practice like this: ‘I’ll breathe out experiencing the whole body.’They practice like this: ‘I’ll breathe in stilling the physical process.’ They practice like this: ‘I’ll breathe out stilling the physical process.’ As they meditate like this—diligent, keen, and resolute—memories and thoughts tied to domestic life are given up.\footnote{For “memories and thoughts tied to lay life” (\textit{\textsanskrit{gehasitā} \textsanskrit{sarasaṅkappā}}) see \href{https://suttacentral.net/mn125/en/sujato\#23.5}{MN 125:23.5} and \href{https://suttacentral.net/sn54.8/en/sujato\#5.2}{SN 54.8:5.2}. | \textit{\textsanskrit{Sarasaṅkappā}} may be translated in two ways. Bodhi reads it as a \textit{dvanda} compound (“memories and thoughts”), which takes \textit{sara} as “memory” (Sanskrit \textit{smara} or \textit{\textsanskrit{smṛti}}). The commentary, followed by Ānandajoti, explains it as \textit{\textsanskrit{dhāvanasaṅkappā}} (“rushing thoughts”), which assumes the sense “flow” (Sanskrit \textit{saras}). Support for the \textit{dvanda} interpretation comes from Sanskrit texts, where we frequently find \textit{\textsanskrit{smṛtisaṁkalpa}} as either two words (eg. Aitareya \textsanskrit{Upaniṣad} 3.2) or a compound (\textsanskrit{Nyāyasūtra} 4.2.34). } Their mind becomes stilled internally; it settles, unifies, and becomes immersed in \textsanskrit{samādhi}.\footnote{Each practice leads directly to immersion, reminding us that mindfulness meditation is “the basis of immersion” (\textit{\textsanskrit{samādhinimitta}}, \href{https://suttacentral.net/mn44/en/sujato\#12.3}{MN 44:12.3}). } That’s how a mendicant develops mindfulness of the body. 

Furthermore,\marginnote{5.1} when a mendicant is walking they know ‘I am walking’. When standing they know ‘I am standing’. When sitting they know ‘I am sitting’. And when lying down they know ‘I am lying down’. Whatever posture their body is in, they know it. As they meditate like this—diligent, keen, and resolute—memories and thoughts tied to domestic life are given up. Their mind becomes stilled internally; it settles, unifies, and becomes immersed in \textsanskrit{samādhi}. That too is how a mendicant develops mindfulness of the body. 

Furthermore,\marginnote{6.1} a mendicant acts with situational awareness when going out and coming back; when looking ahead and aside; when bending and extending the limbs; when bearing the outer robe, bowl and robes; when eating, drinking, chewing, and tasting; when urinating and defecating; when walking, standing, sitting, sleeping, waking, speaking, and keeping silent.\footnote{Obviously this passage is not suggesting that one enters deep meditation while going to the toilet. As per the Gradual Training, such practices form a foundation that prepares the mind for meditation. } As they meditate like this—diligent, keen, and resolute—memories and thoughts tied to domestic life are given up. Their mind becomes stilled internally; it settles, unifies, and becomes immersed in \textsanskrit{samādhi}. That too is how a mendicant develops mindfulness of the body. 

Furthermore,\marginnote{7.1} a mendicant examines their own body, up from the soles of the feet and down from the tips of the hairs, wrapped in skin and full of many kinds of filth. ‘In this body there is head hair, body hair, nails, teeth, skin, flesh, sinews, bones, bone marrow, kidneys, heart, liver, diaphragm, spleen, lungs, intestines, mesentery, undigested food, feces, bile, phlegm, pus, blood, sweat, fat, tears, grease, saliva, snot, synovial fluid, urine.’ 

It’s\marginnote{7.3} as if there were a bag with openings at both ends, filled with various kinds of grains, such as fine rice, wheat, mung beans, peas, sesame, and ordinary rice. And someone with clear eyes were to open it and examine the contents: ‘These grains are fine rice, these are wheat, these are mung beans, these are peas, these are sesame, and these are ordinary rice.’ In the same way, a mendicant examines their own body, up from the soles of the feet and down from the tips of the hairs, wrapped in skin and full of many kinds of filth. …\footnote{Such meditations, as also the contemplation on death that follow, lead to immersion by ridding the mind of desire. While the subject of meditation is perceived as unpleasant, when the mind is free of desire it gives rise to a pleasure imbued with lightness and freedom, and it is that pleasure that leads to immersion. } As they meditate like this—diligent, keen, and resolute—memories and thoughts tied to domestic life are given up. Their mind becomes stilled internally; it settles, unifies, and becomes immersed in \textsanskrit{samādhi}. That too is how a mendicant develops mindfulness of the body. 

Furthermore,\marginnote{8.1} a mendicant examines their own body, whatever its placement or posture, according to the elements: ‘In this body there is the earth element, the water element, the fire element, and the air element.’ 

It’s\marginnote{8.3} as if a deft butcher or butcher’s apprentice were to kill a cow and sit down at the crossroads with the meat cut into chops. In the same way, a mendicant examines their own body, whatever its placement or posture, according to the elements: ‘In this body there is the earth element, the water element, the fire element, and the air element.’ As they meditate like this—diligent, keen, and resolute—memories and thoughts tied to domestic life are given up. Their mind becomes stilled internally; it settles, unifies, and becomes immersed in \textsanskrit{samādhi}. That too is how a mendicant develops mindfulness of the body. 

Furthermore,\marginnote{9.1} suppose a mendicant were to see a corpse discarded in a charnel ground. And it had been dead for one, two, or three days, bloated, livid, and festering. They’d compare it with their own body: ‘This body is also of that same nature, that same kind, and cannot go beyond that.’ As they meditate like this—diligent, keen, and resolute—memories and thoughts tied to domestic life are given up. Their mind becomes stilled internally; it settles, unifies, and becomes immersed in \textsanskrit{samādhi}. That too is how a mendicant develops mindfulness of the body. 

Or\marginnote{10.1} suppose they were to see a corpse discarded in a charnel ground being devoured by crows, hawks, vultures, herons, dogs, tigers, leopards, jackals, and many kinds of little creatures. They’d compare it with their own body: ‘This body is also of that same nature, that same kind, and cannot go beyond that.’ That too is how a mendicant develops mindfulness of the body. 

Furthermore,\marginnote{11{-}14.1} suppose they were to see a corpse discarded in a charnel ground, a skeleton with flesh and blood, held together by sinews … A skeleton without flesh but smeared with blood, and held together by sinews … A skeleton rid of flesh and blood, held together by sinews … Bones rid of sinews scattered in every direction. Here a hand-bone, there a foot-bone, here an ankle bone, there a shin-bone, here a thigh-bone, there a hip-bone, here a rib-bone, there a back-bone, here an arm-bone, there a neck-bone, here a jaw-bone, there a tooth, here the skull. … 

White\marginnote{15{-}17.1} bones, the color of shells … Decrepit bones, heaped in a pile … Bones rotted and crumbled to powder. They’d compare it with their own body: ‘This body is also of that same nature, that same kind, and cannot go beyond that.’ As they meditate like this—diligent, keen, and resolute—memories and thoughts tied to lay life are given up. Their mind becomes stilled internally; it settles, unifies, and becomes immersed in \textsanskrit{samādhi}. That too is how a mendicant develops mindfulness of the body. 

Furthermore,\marginnote{18.1} a mendicant, quite secluded from sensual pleasures, secluded from unskillful qualities, enters and remains in the first absorption, which has the rapture and bliss born of seclusion, while placing the mind and keeping it connected.\footnote{In the \textsanskrit{Satipaṭṭhānasutta}, the practice of absorption is implicit in the contemplation of feelings (as spiritual pleasure), mind (as immersed mind, etc.), and principles (as the awakening factor of immersion), while the four absorptions are mentioned explicitly in the expanded version (\href{https://suttacentral.net/dn22/en/sujato\#21.32}{DN 22:21.32}). Here they take an even more central role. } They drench, steep, fill, and spread their body with rapture and bliss born of seclusion. There’s no part of the body that’s not spread with rapture and bliss born of seclusion.\footnote{As a meditator proceeds, their subjective experience of the “body” evolves from tactile sense impressions (\textit{\textsanskrit{phoṭṭhabba}}), to the interior mental experience of bliss and light (\textit{\textsanskrit{manomayakāya}}), to the direct personal realization of highest truth (\href{https://suttacentral.net/mn70/en/sujato\#23.2}{MN 70:23.2}: \textit{\textsanskrit{kāyena} ceva \textsanskrit{paramasaccaṁ} sacchikaroti}). } It’s like when a deft bathroom attendant or their apprentice pours bath powder into a bronze dish, sprinkling it little by little with water. They knead it until the ball of bath powder is soaked and saturated with moisture, spread through inside and out; yet no moisture oozes out.\footnote{The kneading is the “placing the mind and keeping it connected”, the water is bliss, while the lack of leaking speaks to the contained interiority of the experience. } In the same way, they drench, steep, fill, and spread their body with rapture and bliss born of seclusion. There’s no part of the body that’s not spread with rapture and bliss born of seclusion. As they meditate like this—diligent, keen, and resolute—memories and thoughts tied to lay life are given up. Their mind becomes stilled internally; it settles, unifies, and becomes immersed in \textsanskrit{samādhi}. That too is how a mendicant develops mindfulness of the body. 

Furthermore,\marginnote{19.1} as the placing of the mind and keeping it connected are stilled, a mendicant enters and remains in the second absorption, which has the rapture and bliss born of immersion, with internal clarity and mind at one, without placing the mind and keeping it connected. They drench, steep, fill, and spread their body with rapture and bliss born of immersion. There’s no part of the body that’s not spread with rapture and bliss born of immersion. It’s like a deep lake fed by spring water. There’s no inlet to the east, west, north, or south, and the heavens would not properly bestow showers from time to time. But the stream of cool water welling up in the lake drenches, steeps, fills, and spreads throughout the lake. There’s no part of the lake that’s not spread through with cool water.\footnote{The simile emphasizes the water as bliss, while the lack of inflow expresses containment and unification. The water welling up is the rapture, which is the uplifting emotional response to the experience of bliss. These similes are also at \href{https://suttacentral.net/dn2/en/sujato\#78.1}{DN 2:78.1}, \href{https://suttacentral.net/dn10/en/sujato\#2.16.1}{DN 10:2.16.1}, \href{https://suttacentral.net/mn39.16/en/sujato}{MN 39.16}, \href{https://suttacentral.net/mn77/en/sujato\#26.3}{MN 77:26.3}, and \href{https://suttacentral.net/an5.28/en/sujato\#3.4}{AN 5.28:3.4}. } In the same way, a mendicant drenches, steeps, fills, and spreads their body with rapture and bliss born of immersion. There’s no part of the body that’s not spread with rapture and bliss born of immersion. That too is how a mendicant develops mindfulness of the body. 

Furthermore,\marginnote{20.1} with the fading away of rapture, a mendicant enters and remains in the third absorption. They meditate with equanimity, mindful and aware, personally experiencing the bliss of which the noble ones declare, ‘Equanimous and mindful, one meditates in bliss.’ They drench, steep, fill, and spread their body with bliss free of rapture. There’s no part of the body that’s not spread with bliss free of rapture. It’s like a pool with blue water lilies, or pink or white lotuses. Some of them sprout and grow in the water without rising above it, thriving underwater. From the tip to the root they’re drenched, steeped, filled, and soaked with cool water. There’s no part of them that’s not soaked with cool water.\footnote{The meditator is utterly immersed in stillness and bliss. } In the same way, a mendicant drenches, steeps, fills, and spreads their body with bliss free of rapture. There’s no part of the body that’s not spread with bliss free of rapture. That too is how a mendicant develops mindfulness of the body. 

Furthermore,\marginnote{21.1} a mendicant, giving up pleasure and pain, and ending former happiness and sadness, enters and remains in the fourth absorption, without pleasure or pain, with pure equanimity and mindfulness. They sit spreading their body through with pure bright mind. There’s no part of the body that’s not filled with pure bright mind. It’s like someone sitting wrapped from head to foot with white cloth. There’s no part of the body that’s not spread over with white cloth.\footnote{The white cloth is the purity and brightness of equanimity. The commentary explains this as a person who has just got out of a bath and sits perfectly dry and content. } In the same way, they sit spreading their body through with pure bright mind. There’s no part of the body that’s not filled with pure bright mind. As they meditate like this—diligent, keen, and resolute—memories and thoughts tied to domestic life are given up. Their mind becomes stilled internally; it settles, unifies, and becomes immersed in \textsanskrit{samādhi}. That too is how a mendicant develops mindfulness of the body. 

Anyone\marginnote{22.1} who has developed and cultivated mindfulness of the body includes all of the skillful qualities that play a part in realization.\footnote{By this, the Buddha is asserting that mindfulness of the body is not just the first of the four kinds of mindfulness meditation, but encompasses all of them. The body is the arena within which feelings, mind, and principles become apparent. (Also at \href{https://suttacentral.net/an1.575/en/sujato}{AN 1.575}.) | The things that “play a part in realization” are enumerated as two at \href{https://suttacentral.net/an2.31/en/sujato}{AN 2.31}: serenity and discernment. Another group of six at \href{https://suttacentral.net/sn55.3/en/sujato\#4.1}{SN 55.3:4.1} and \href{https://suttacentral.net/an6.35/en/sujato}{AN 6.35} focus on the discernment side (observing impermanence, suffering in impermanence, not-self in suffering, giving up, fading away, and cessation). } Anyone who brings into their mind the great ocean includes all of the streams that run down into it.\footnote{A simile of the rivers becoming united in the ocean is also found at \textsanskrit{Bṛhadāraṇyaka} \textsanskrit{Upaniṣad} 2.4.11, \textsanskrit{Muṇḍaka} \textsanskrit{Upaniṣad} 3.2.8, and \textsanskrit{Praśna} \textsanskrit{Upaniṣad} 6.5. } In the same way, anyone who has developed and cultivated mindfulness of the body includes all of the skillful qualities that play a part in realization. 

When\marginnote{23.1} a mendicant has not developed or cultivated mindfulness of the body, \textsanskrit{Māra} finds a vulnerability and gets hold of them. Suppose a person were to throw a heavy stone ball at a mound of wet clay. 

What\marginnote{23.3} do you think, mendicants? Would that heavy stone ball find an entry into that mound of wet clay?” 

“Yes,\marginnote{23.5} sir.” 

“In\marginnote{23.6} the same way, when a mendicant has not developed or cultivated mindfulness of the body, \textsanskrit{Māra} finds a vulnerability and gets hold of them. 

Suppose\marginnote{24.1} there was a dried up, withered log.\footnote{Compare \href{https://suttacentral.net/mn36/en/sujato\#19.2}{MN 36:19.2} = \href{https://suttacentral.net/mn85/en/sujato\#19.2}{MN 85:19.2} = \href{https://suttacentral.net/mn100/en/sujato\#16.2}{MN 100:16.2}. } Then a person comes along with a drill-stick, thinking to light a fire and produce heat. 

What\marginnote{24.4} do you think, mendicants? By drilling the stick against that dried up, withered log, could they light a fire and produce heat?” 

“Yes,\marginnote{24.6} sir.” 

“In\marginnote{24.7} the same way, when a mendicant has not developed or cultivated mindfulness of the body, \textsanskrit{Māra} finds a vulnerability and gets hold of them. 

Suppose\marginnote{25.1} a water jar was placed on a stand, empty and hollow. Then a person comes along with a load of water. 

What\marginnote{25.3} do you think, mendicants? Could that person pour water into the jar?” 

“Yes,\marginnote{25.5} sir.” 

“In\marginnote{25.6} the same way, when a mendicant has not developed or cultivated mindfulness of the body, \textsanskrit{Māra} finds a vulnerability and gets hold of them. 

When\marginnote{26.1} a mendicant has developed and cultivated mindfulness of the body, \textsanskrit{Māra} cannot find a vulnerability and doesn’t get hold of them. 

Suppose\marginnote{26.2} a person were to throw a light ball of string at a door-panel made entirely of hardwood.\footnote{Compare the simile at \textsanskrit{Chāndogya} \textsanskrit{Upaniṣad} 1.2.7 and \textsanskrit{Bṛhadāraṇyaka} \textsanskrit{Upaniṣad} 1.3.7, where the titans (or forces of evil) meet their end in the breath, just like a clod of earth thrown against solid rock. } 

What\marginnote{26.3} do you think, mendicants? Would that light ball of string find an entry into that door-panel made entirely of hardwood?” 

“No,\marginnote{26.5} sir.” 

“In\marginnote{26.6} the same way, when a mendicant has developed and cultivated mindfulness of the body, \textsanskrit{Māra} cannot find a vulnerability and doesn’t get hold of them. 

Suppose\marginnote{27.1} there was a green, sappy log. Then a person comes along with a drill-stick, thinking to light a fire and produce heat. 

What\marginnote{27.4} do you think, mendicants? By drilling the stick against that green, sappy log on dry land far from water, could they light a fire and produce heat?” 

“No,\marginnote{27.6} sir.” 

“In\marginnote{27.7} the same way, when a mendicant has developed and cultivated mindfulness of the body, \textsanskrit{Māra} cannot find a vulnerability and doesn’t get hold of them. Suppose a water jar was placed on a stand, full to the brim so a crow could drink from it.\footnote{These similes recall the absorption similes above. } Then a person comes along with a load of water. 

What\marginnote{28.3} do you think, mendicants? Could that person pour water into the jar?” 

“No,\marginnote{28.5} sir.” 

“In\marginnote{28.6} the same way, when a mendicant has developed and cultivated mindfulness of the body, \textsanskrit{Māra} cannot find a vulnerability and doesn’t get hold of them. 

When\marginnote{29.1} a mendicant has developed and cultivated mindfulness of the body, they extend the mind to realize by insight each and every thing that can be realized by insight; and they are capable of realizing those things, since each and every one is within range. 

Suppose\marginnote{29.2} a water jar was placed on a stand, full to the brim so a crow could drink from it. If a strong man was to pour it on any side, would water pour out?” 

“Yes,\marginnote{29.4} sir.” 

“In\marginnote{29.5} the same way, when a mendicant has developed and cultivated mindfulness of the body, they extend the mind to realize by insight each and every thing that can be realized by insight; and they are capable of realizing those things, since each and every one is within range.\footnote{The phrase \textit{sati sati \textsanskrit{āyatane}} is tricky. Syntactically, it is a locative absolute construction, which Bodhi treats as conditional: “there being a suitable basis”. However this does not readily account for the repeated \textit{sati}, which he takes as the locative present participle from the verb \textit{atthi} (“being”). Ānandajoti, by contrast, takes the first \textit{sati} as “being” and the second as “mindfulness”, and treats the absolute construction as temporal: “while there is a basis for mindfulness”. Further, both accept the commentary’s gloss of \textit{\textsanskrit{āyatana}} as \textit{\textsanskrit{kāraṇa}} (“reason, basis”), although this sense of \textit{\textsanskrit{āyatana}} is rare in the suttas (but see \href{https://suttacentral.net/sn12.25/en/sujato\#14.3}{SN 12.25:14.3} = \href{https://suttacentral.net/an4.171/en/sujato\#5.3}{AN 4.171:5.3}). These readings of the absolute construction give it a limiting sense; the benefits apply only insofar as there is a “suitable basis” or a “basis for mindfulness”. But the expression is used by the Buddha of himself (\href{https://suttacentral.net/an5.68/en/sujato\#1.9}{AN 5.68:1.9}), so it seems odd that it would be limiting. Indeed, the tenor of the whole sentence is not limiting but expansive, as emphasized by the repeated reduplications: \textit{yassa yassa … tatra tatreva … sati sati}, which have a distributive sense. I therefore take \textit{sati} as “being” in both cases; the reduplication of \textit{sati} as aligned with the preceding pronouns, emphasizing universality; the locative absolute as causal; and \textit{\textsanskrit{āyatana}}  in its normal sense of “scope, range”. } 

Suppose\marginnote{30.1} there was a square, walled lotus pond on level ground, full to the brim so a crow could drink from it. If a strong man was to open the wall on any side, would water pour out?” 

“Yes,\marginnote{30.3} sir.” 

“In\marginnote{30.4} the same way, when a mendicant has developed and cultivated mindfulness of the body, they extend the mind to realize by insight each and every thing that can be realized by insight; and they are capable of realizing those things, since each and every one is within range. Suppose a chariot stood harnessed to thoroughbreds at a level crossroads, with a goad ready. A deft horse trainer, a master charioteer, might mount the chariot, taking the reins in his right hand and goad in the left. He’d drive out and back wherever he wishes, whenever he wishes. In the same way, when a mendicant has developed and cultivated mindfulness of the body, they extend the mind to realize by insight each and every thing that can be realized by insight; and they are capable of realizing those things, since each and every one is within range. 

You\marginnote{32.1} can expect ten benefits when mindfulness of the body has been cultivated, developed, and practiced, made a vehicle and a basis, kept up, consolidated, and properly implemented. 

They\marginnote{33.1} prevail over desire and discontent, and live having mastered desire and discontent whenever they arose. 

They\marginnote{34.1} prevail over fear and dread, and live having mastered fear and dread whenever they arose. 

They\marginnote{35.1} endure cold, heat, hunger, and thirst; the touch of flies, mosquitoes, wind, sun, and reptiles; rude and unwelcome criticism; and put up with physical pain—sharp, severe, acute, unpleasant, disagreeable, and life-threatening. 

They\marginnote{36.1} get the four absorptions—blissful meditations in this life that belong to the higher mind—when they want, without trouble or difficulty. 

They\marginnote{37.1} wield the many kinds of psychic power: multiplying themselves and becoming one again … They control the body as far as the realm of divinity. 

With\marginnote{38.1} clairaudience that is purified and superhuman, they hear both kinds of sounds, human and heavenly, whether near or far. … 

They\marginnote{39.1} understand the minds of other beings and individuals, having comprehended them with their own mind. … 

They\marginnote{40.1} recollect many kinds of past lives, with features and details. 

With\marginnote{41.1} clairvoyance that is purified and superhuman, they see sentient beings passing away and being reborn—inferior and superior, beautiful and ugly, in a good place or a bad place. They understand how sentient beings are reborn according to their deeds. 

They\marginnote{42.1} realize the undefiled freedom of heart and freedom by wisdom in this very life. And they live having realized it with their own insight due to the ending of defilements. 

You\marginnote{43.1} can expect these ten benefits when mindfulness of the body has been cultivated, developed, and practiced, made a vehicle and a basis, kept up, consolidated, and properly implemented.” 

That\marginnote{43.2} is what the Buddha said. Satisfied, the mendicants approved what the Buddha said. 

%
\section*{{\suttatitleacronym MN 120}{\suttatitletranslation Rebirth by Choice }{\suttatitleroot Saṅkhārupapattisutta}}
\addcontentsline{toc}{section}{\tocacronym{MN 120} \toctranslation{Rebirth by Choice } \tocroot{Saṅkhārupapattisutta}}
\markboth{Rebirth by Choice }{Saṅkhārupapattisutta}
\extramarks{MN 120}{MN 120}

\scevam{So\marginnote{1.1} I have heard.\footnote{This sutta describes a practice by which a mendicant may choose their place of rebirth. This stands in contrast with the normal approach, where a mendicant aims to be free of all rebirth. At \href{https://suttacentral.net/mn16/en/sujato\#12.1}{MN 16:12.1}, a mendicant who practices out of desire for rebirth is said to be bound by an “emotional shackle” (\textit{cetasovinibandha}). At \href{https://suttacentral.net/an3.18/en/sujato}{AN 3.18}, the mendicants are said to be horrified by the suggestion that they practice for rebirth as a god, while \href{https://suttacentral.net/an7.50/en/sujato\#9.1}{AN 7.50:9.1} suggests that such aspirations stem from unresolved sexual desire. The present sutta takes a more conciliatory stance, moving from desire for progressively more refined kinds of rebirth to ultimately letting go of rebirth altogether. } }At one time the Buddha was staying near \textsanskrit{Sāvatthī} in Jeta’s Grove, \textsanskrit{Anāthapiṇḍika}’s monastery. There the Buddha addressed the mendicants, “Mendicants!” 

“Venerable\marginnote{1.5} sir,” they replied. The Buddha said this: 

“I\marginnote{2.1} shall teach you rebirth by choice.\footnote{Here \textit{\textsanskrit{saṅkhāra}} has the same sense “choice” that it has in dependent origination or the five aggregates. In each case, it is the mental volition that is motivated by morally meaningful forces, either greed, hate, and delusion or their opposites, and which creates results according to that moral force. Often this works unconsciously, as the effects of our choices and deeds manifest with or without our comprehension. Here, someone who understands \textit{kamma} turns it to their advantage by consciously developing an aspiration for a particular desired result. | \textit{Upapatti} is one of several terms in Pali that regularly mean “rebirth”, such as \textit{\textsanskrit{jāti}}, \textit{(abhi)-nibbatti}, \textit{okkanti}, and \textit{punabbhava}. } Listen and apply your mind well, I will speak.” 

“Yes,\marginnote{2.3} sir,” they replied. The Buddha said this: 

“Take\marginnote{3.1} a mendicant who has faith, ethics, learning, generosity, and wisdom.\footnote{The aspiration alone is not enough. It only works if the person already has the kammic potential to realize their aims. | In this discourse it is a mendicant who is making the aspiration. The related passages at \href{https://suttacentral.net/dn33/en/sujato\#3.1.98}{DN 33:3.1.98} and \href{https://suttacentral.net/an8.35/en/sujato}{AN 8.35} focus on “rebirth by giving”, implying the aspirant is a lay person. } They think: ‘If only, when my body breaks up, after death, I would be reborn in the company of well-to-do aristocrats!’\footnote{See \href{https://suttacentral.net/dn33/en/sujato\#3.1.102}{DN 33:3.1.102}, \href{https://suttacentral.net/mn41/en/sujato\#15.2}{MN 41:15.2}, and \href{https://suttacentral.net/an8.35/en/sujato\#1.8}{AN 8.35:1.8}. } They settle on that thought, stabilize it, and develop it. Those choices and meditations of theirs, developed and cultivated like this, lead to rebirth there.\footnote{“Meditations” here is \textit{\textsanskrit{vihārā}}, literally “abidings”. } This is the path and the practice that leads to rebirth there. 

Furthermore,\marginnote{4{-}5.1} take a mendicant who has faith, ethics, learning, generosity, and wisdom. They think: ‘If only, when my body breaks up, after death, I would be reborn in the company of well-to-do brahmins … well-to-do householders.’ They settle on that thought, stabilize it, and develop it. Those choices and meditations of theirs, developed and cultivated like this, lead to rebirth there. This is the path and the practice that leads to rebirth there. 

Furthermore,\marginnote{6.1} take a mendicant who has faith, ethics, learning, generosity, and wisdom. And they’ve heard: ‘The gods of the four great kings are long-lived, beautiful, and very happy.’ They think: ‘If only, when my body breaks up, after death, I would be reborn in the company of the gods of the four great kings!’ They settle on that thought, stabilize it, and develop it. Those choices and meditations of theirs, developed and cultivated like this, lead to rebirth there. This is the path and the practice that leads to rebirth there. 

Furthermore,\marginnote{7{-}11.1} take a mendicant who has faith, ethics, learning, generosity, and wisdom. And they’ve heard: ‘The gods of the thirty-three … the gods of Yama … the joyful gods … the gods who love to imagine … the gods who control what is imagined by others are long-lived, beautiful, and very happy.’ They think: ‘If only, when my body breaks up, after death, I would be reborn in the company of the gods who control what is imagined by others!’ They settle on that thought, stabilize it, and develop it. Those choices and meditations of theirs, developed and cultivated like this, lead to rebirth there. This is the path and the practice that leads to rebirth there. 

Furthermore,\marginnote{12.1} take a mendicant who has faith, ethics, learning, generosity, and wisdom. And they’ve heard: ‘The Divinity of a thousand is long-lived, beautiful, and very happy.’ Now the Divinity of a thousand meditates focused on pervading a thousandfold galaxy,\footnote{The text does not say what “pervading” (\textit{\textsanskrit{pharitvā}}) means, but presumably it is the “divine meditations” (\textit{\textsanskrit{brahmavihāra}}), which are said to be “spread” or “pervaded” (eg. \href{https://suttacentral.net/mn7/en/sujato\#13.1}{MN 7:13.1}). See also \href{https://suttacentral.net/mn127/en/sujato\#8.2}{MN 127:8.2}. } as well as the sentient beings reborn there. As a person might pick up a gallnut in their hand and examine it, so too the Divinity of a thousand meditates focused on pervading a thousandfold galaxy, as well as the sentient beings reborn there. They think: ‘If only, when my body breaks up, after death, I would be reborn in the company of the Divinity of a thousand!’ They settle on that thought, stabilize it, and develop it. Those choices and meditations of theirs, developed and cultivated like this, lead to rebirth there.\footnote{This passage raises the question as to how such a reflection leads to rebirth in the \textsanskrit{Brahmā} realms, which requires the development of absorption (eg. \href{https://suttacentral.net/an4.123/en/sujato}{AN 4.123}). Similar passages take care to specify that in order to be reborn in the \textsanskrit{Brahmā} realm, one must not only be ethical, but “free of desire”, by which they imply the practice of absorption (\href{https://suttacentral.net/dn33/en/sujato\#3.1.136}{DN 33:3.1.136} = \href{https://suttacentral.net/an8.35/en/sujato\#4.11}{AN 8.35:4.11}). Related passages in Chinese translation relate the development of different levels of absorption to rebirth in the various realms (MA 168 at T i 700c7; Dharmaskandha, T 1537 at T xxvi 506b14), as does the commentary to this sutta. Now, while the current sutta does not mention absorption, it may be considered in light of the meditation practice of the six recollections. One of these is the recollection that the deities, including \textsanskrit{Brahmās}, were reborn due to their faith, ethics, learning, generosity, and wisdom—exactly the same qualities mentioned in the present sutta (\href{https://suttacentral.net/an6.25/en/sujato\#6.3}{AN 6.25:6.3}). Thus when the present sutta says to “settle on that thought, concentrate on it, and develop it”, the phrase \textit{\textsanskrit{cittaṁ} \textsanskrit{bhāveti}} implies more than simply “developing the thought” of the aspiration, but means “developing the mind”. In the suttas this implies deep meditation, as for example the case of a teacher of the past, Sunetta, who was “free of desire”; after “developing the mind of love” (\textit{\textsanskrit{mettaṁ} \textsanskrit{cittaṁ} \textsanskrit{bhāvetvā}}) he was reborn in the \textsanskrit{Brahmā} realm (\href{https://suttacentral.net/an7.66/en/sujato\#13.1}{AN 7.66:13.1}). Thus we can interpret the present passage as a way of practicing recollection of the deities, recalling that one’s virtuous qualities are shared with the deities, and developing deep meditation based on the joy that brings. } This is the path and the practice that leads to rebirth there. 

Furthermore,\marginnote{13{-}16.1} take a mendicant who has faith, ethics, learning, generosity, and wisdom. And they’ve heard: ‘The Divinity of two thousand … the Divinity of three thousand … the Divinity of four thousand … the Divinity of five thousand is long-lived, beautiful, and very happy.’ Now the Divinity of five thousand meditates focused on pervading a five-thousandfold galaxy, as well as the sentient beings reborn there. As a person might pick up five gallnuts in their hand and examine them, so too the Divinity of five thousand meditates focused on pervading a five-thousandfold galaxy, as well as the sentient beings reborn there. They think: ‘If only, when my body breaks up, after death, I would be reborn in the company of the Divinity of five thousand!’ They settle on that thought, stabilize it, and develop it. Those choices and meditations of theirs, developed and cultivated like this, lead to rebirth there. This is the path and the practice that leads to rebirth there. 

Furthermore,\marginnote{17.1} take a mendicant who has faith, ethics, learning, generosity, and wisdom. And they’ve heard: ‘The Divinity of ten thousand is long-lived, beautiful, and very happy.’ Now the Divinity of ten thousand meditates focused on pervading a ten-thousandfold galaxy, as well as the sentient beings reborn there. Suppose there was a beryl gem that was naturally beautiful, eight-faceted, well-worked. When placed on a cream rug it would shine and glow and radiate. In the same way the Divinity of ten thousand meditates focused on pervading a ten-thousandfold galaxy, as well as the sentient beings reborn there. They think: ‘If only, when my body breaks up, after death, I would be reborn in the company of the Divinity of ten thousand!’ They settle on that thought, stabilize it, and develop it. Those choices and meditations of theirs, developed and cultivated like this, lead to rebirth there. This is the path and the practice that leads to rebirth there. 

Furthermore,\marginnote{18.1} take a mendicant who has faith, ethics, learning, generosity, and wisdom. And they’ve heard: ‘The Divinity of a hundred thousand is long-lived, beautiful, and very happy.’ Now the Divinity of a hundred thousand meditates focused on pervading a hundred-thousandfold galaxy, as well as the sentient beings reborn there. Suppose there was a pendant of Black Plum River gold, fashioned by a deft smith, well wrought in the forge. When placed on a cream rug it would shine and glow and radiate.\footnote{Legend has it that on the slopes of Mount Meru grows the vast Jambu tree that gives the continent of India her name, “the land of the black plum tree”. The fruits of that tree are as big as elephants, and when they fall, their juice flows forth as a river named Jambu. The dried mud of that river yields gold nuggets whose unparalleled lustre is highly sought-after among the gods. This is that gold (Śiva \textsanskrit{Purāṇa} 17.16–19). } In the same way the Divinity of a hundred thousand meditates focused on pervading a hundred-thousandfold galaxy, as well as the sentient beings reborn there. They think: ‘If only, when my body breaks up, after death, I would be reborn in the company of the Divinity of a hundred thousand!’ They settle on that thought, stabilize it, and develop it. Those choices and meditations of theirs, developed and cultivated like this, lead to rebirth there. This is the path and the practice that leads to rebirth there. 

Furthermore,\marginnote{19.1} take a mendicant who has faith, ethics, learning, generosity, and wisdom. And they’ve heard: ‘The radiant gods … the gods of limited radiance … the gods of limitless radiance … the gods of streaming radiance … the gods of limited beauty … the gods of limitless beauty … the gods of universal beauty … the gods of abundant fruit … the gods of Aviha … the gods of Atappa … the gods fair to see … the fair seeing gods … the gods of \textsanskrit{Akaniṭṭha} … the gods of the dimension of infinite space … the gods of the dimension of infinite consciousness … the gods of the dimension of nothingness … the gods of the dimension of neither perception nor non-perception are long-lived, beautiful, and very happy.’ They think: ‘If only, when my body breaks up, after death, I would be reborn in the company of the gods of the dimension of neither perception nor non-perception!’ They settle on that thought, stabilize it, and develop it. Those choices and meditations of theirs, developed and cultivated like this, lead to rebirth there. This is the path and the practice that leads to rebirth there. 

Furthermore,\marginnote{37.1} take a mendicant who has faith, ethics, learning, generosity, and wisdom. They think: ‘If only I might realize the undefiled freedom of heart and freedom by wisdom in this very life, and live having realized it with my own insight due to the ending of defilements.’ They realize the undefiled freedom of heart and freedom by wisdom in this very life. And they live having realized it with their own insight due to the ending of defilements.\footnote{Just as rebirth in the realms of higher divinity implies the corresponding level of absorption, freedom from rebirth implies the whole eightfold path. } And, mendicants, that mendicant is not reborn anywhere.” 

That\marginnote{37.9} is what the Buddha said. Satisfied, the mendicants approved what the Buddha said. 

%
\addtocontents{toc}{\let\protect\contentsline\protect\nopagecontentsline}
\chapter*{The Chapter Beginning with Emptiness }
\addcontentsline{toc}{chapter}{\tocchapterline{The Chapter Beginning with Emptiness }}
\addtocontents{toc}{\let\protect\contentsline\protect\oldcontentsline}

%
\section*{{\suttatitleacronym MN 121}{\suttatitletranslation The Shorter Discourse on Emptiness }{\suttatitleroot Cūḷasuññatasutta}}
\addcontentsline{toc}{section}{\tocacronym{MN 121} \toctranslation{The Shorter Discourse on Emptiness } \tocroot{Cūḷasuññatasutta}}
\markboth{The Shorter Discourse on Emptiness }{Cūḷasuññatasutta}
\extramarks{MN 121}{MN 121}

\scevam{So\marginnote{1.1} I have heard.\footnote{This sutta gives a unique approach to meditation on emptiness, using the perception of relative emptiness as a foundation for absorption. | The topic of emptiness does not play the central role in Pali texts that it did in later Buddhism. This is perhaps at least partly due to different doctrinal developments in the early Buddhist schools. Passages that in the Pali mention impermanence, suffering, and not-self, in parallels of the (\textsanskrit{Mūla})-\textsanskrit{Sarvāstivāda} often include emptiness as well. This does not indicate any doctrinal contradiction, as emptiness is explained as meaning not-self. But it does suggest that the presentation in terms of emptiness was more popular in that group of schools. } }At one time the Buddha was staying near \textsanskrit{Sāvatthī} in the stilt longhouse of \textsanskrit{Migāra}’s mother in the Eastern Monastery. 

Then\marginnote{2.1} in the late afternoon, Venerable Ānanda came out of retreat and went to the Buddha. He bowed, sat down to one side, and said to him: 

“Sir,\marginnote{3.1} this one time the Buddha was staying in the land of the Sakyans where they have a town named Townsville. There I heard and learned this in the presence of the Buddha:\footnote{“I heard and learned this in the presence” (\textit{\textsanskrit{sammukhā} \textsanskrit{sutaṁ}, \textsanskrit{sammukhā} \textsanskrit{paṭiggahitaṁ}}) is the phrase used in the suttas when reporting a teaching heard directly from the Buddha, eg. \href{https://suttacentral.net/sn55.52/en/sujato\#5.1}{SN 55.52:5.1}, \href{https://suttacentral.net/sn22.90/en/sujato\#9.1}{SN 22.90:9.1}, \href{https://suttacentral.net/mn47/en/sujato\#10.7}{MN 47:10.7}, etc. } ‘Ānanda, these days I usually practice the meditation on emptiness.’\footnote{The same thing is said by \textsanskrit{Sāriputta} at \href{https://suttacentral.net/mn151/en/sujato\#2.3}{MN 151:2.3}, upon which the Buddha encourages him to reflect on the purity of his almsfood. Before the Second Council, the elder monk \textsanskrit{Sabbakāmī} revealed that emptiness was his main practice too (\href{https://suttacentral.net/pli-tv-kd22/en/sujato\#2.5.9}{Kd 22:2.5.9}), while at \href{https://suttacentral.net/thig3.3/en/sujato\#2.1}{Thig 3.3:2.1}, the \textsanskrit{bhikkhunī} \textsanskrit{Uttamā} also claims to practice emptiness meditation. The release through emptiness is classified along with absorption as a superhuman attainment (\href{https://suttacentral.net/pli-tv-bu-vb-pj4/en/sujato\#4.1.6}{Bu Pj 4:4.1.6}). At \href{https://suttacentral.net/mn43/en/sujato\#33.1}{MN 43:33.1} = \href{https://suttacentral.net/sn41.7/en/sujato\#4.1}{SN 41.7:4.1}, the “release of the heart through emptiness” is a reflection on not-self (see too \href{https://suttacentral.net/sn35.85/en/sujato\#1.4}{SN 35.85:1.4}, \href{https://suttacentral.net/an9.36/en/sujato\#2.4}{AN 9.36:2.4}, \href{https://suttacentral.net/snp5.16/en/sujato\#4.1}{Snp 5.16:4.1}, \href{https://suttacentral.net/thag19.1/en/sujato\#27.2}{Thag 19.1:27.2}). } I trust I properly heard, learned, applied the mind, and remembered that from the Buddha?”\footnote{Notice the process of learning teachings through active engagement. Simply “hearing” it is just the beginning. } 

“Indeed,\marginnote{3.5} Ānanda, you properly heard, learned, applied the mind, and remembered that. Now, as before, I usually practice the meditation on emptiness. 

Consider\marginnote{4.1} this stilt longhouse of \textsanskrit{Migāra}’s mother. It’s empty of elephants, cows, horses, and mares; of gold and silver; and of gatherings of men and women.\footnote{Emptiness is not a metaphysical property, but a relative state: something is empty of something. } There is only this that is not emptiness, namely, the oneness dependent on the mendicant \textsanskrit{Saṅgha}.\footnote{Oneness, like emptiness, is a relative concept. Those things that are more distracting and tempting are absent, leaving only that which is more peaceful. } In the same way, a mendicant—ignoring the perception of the village and the perception of people—focuses on the oneness dependent on the perception of wilderness. Their mind leaps forth, gains confidence, settles down, and becomes decided in that perception of wilderness. They understand: ‘Here there is no stress due to the perception of village or the perception of people.\footnote{“Stress” is \textit{\textsanskrit{darathā}}. } There is only this modicum of stress, namely the oneness dependent on the perception of wilderness.’ They understand: ‘This field of perception is empty of the perception of the village. It is empty of the perception of people. There is only this that is not emptiness, namely the oneness dependent on the perception of wilderness.’ And so they regard it as empty of what is not there, but as to what remains they understand that it is present.\footnote{That is to say, the “purity” of the emptiness is not because everything is empty, but because what is empty is accurately recognized as empty. } That’s how emptiness manifests in them—genuine, undistorted, and pure. 

Furthermore,\marginnote{5.1} a mendicant—ignoring the perception of people and the perception of wilderness—focuses on the oneness dependent on the perception of earth. Their mind leaps forth, gains confidence, settles down, and becomes decided in that perception of earth. As a bull’s hide is rid of folds when fully stretched out by a hundred pegs, so too, ignoring the hilly terrain, inaccessible riverlands, stumps and thorns, and rugged mountains, they focus on the oneness dependent on the perception of earth.\footnote{The perception of earth is an idealized inner conceptual image. This is the “form” (\textit{\textsanskrit{rūpa}}) that is the basis for “absorption on luminous form” (\textit{\textsanskrit{rūpajjhāna}}). The commentaries call this the \textit{nimitta} (“sign”). } Their mind leaps forth, gains confidence, settles down, and becomes decided in that perception of earth. They understand: ‘Here there is no stress due to the perception of people or the perception of wilderness. There is only this modicum of stress, namely the oneness dependent on the perception of earth.’ They understand: ‘This field of perception is empty of the perception of people. It is empty of the perception of wilderness. There is only this that is not emptiness, namely the oneness dependent on the perception of earth.’ And so they regard it as empty of what is not there, but as to what remains they understand that it is present. That’s how emptiness manifests in them—genuine, undistorted, and pure. 

Furthermore,\marginnote{6.1} a mendicant—ignoring the perception of wilderness and the perception of earth—focuses on the oneness dependent on the perception of the dimension of infinite space.\footnote{The meditator moves from a practice dependent on an aspect of “form” (\textit{\textsanskrit{rūpa}}), namely the idealized image of earth as the basis for the four absorptions, to the formless meditations. } Their mind leaps forth, gains confidence, settles down, and becomes decided in that perception of the dimension of infinite space. They understand: ‘Here there is no stress due to the perception of wilderness or the perception of earth. There is only this modicum of stress, namely the oneness dependent on the perception of the dimension of infinite space.’ They understand: ‘This field of perception is empty of the perception of wilderness. It is empty of the perception of earth. There is only this that is not emptiness, namely the oneness dependent on the perception of the dimension of infinite space.’ And so they regard it as empty of what is not there, but as to what remains they understand that it is present. That’s how emptiness manifests in them—genuine, undistorted, and pure. 

Furthermore,\marginnote{7.1} a mendicant—ignoring the perception of earth and the perception of the dimension of infinite space—focuses on the oneness dependent on the perception of the dimension of infinite consciousness. Their mind leaps forth, gains confidence, settles down, and becomes decided in that perception of the dimension of infinite consciousness. They understand: ‘Here there is no stress due to the perception of earth or the perception of the dimension of infinite space. There is only this modicum of stress, namely the oneness dependent on the perception of the dimension of infinite consciousness.’ They understand: ‘This field of perception is empty of the perception of earth. It is empty of the perception of the dimension of infinite space. There is only this modicum of stress, namely the oneness dependent on the perception of the dimension of infinite consciousness.’ And so they regard it as empty of what is not there, but as to what remains they understand that it is present. That’s how emptiness manifests in them—genuine, undistorted, and pure. 

Furthermore,\marginnote{8.1} a mendicant—ignoring the perception of the dimension of infinite space and the perception of the dimension of infinite consciousness—focuses on the oneness dependent on the perception of the dimension of nothingness. Their mind leaps forth, gains confidence, settles down, and becomes decided in that perception of the dimension of nothingness. They understand: ‘Here there is no stress due to the perception of the dimension of infinite space or the perception of the dimension of infinite consciousness. There is only this modicum of stress, namely the oneness dependent on the perception of the dimension of nothingness.’ They understand: ‘This field of perception is empty of the perception of the dimension of infinite space. It is empty of the perception of the dimension of infinite consciousness. There is only this that is not emptiness, namely the oneness dependent on the perception of the dimension of nothingness.’ And so they regard it as empty of what is not there, but as to what remains they understand that it is present. That’s how emptiness manifests in them—genuine, undistorted, and pure. 

Furthermore,\marginnote{9.1} a mendicant—ignoring the perception of the dimension of infinite consciousness and the perception of the dimension of nothingness—focuses on the oneness dependent on the perception of the dimension of neither perception nor non-perception. Their mind leaps forth, gains confidence, settles down, and becomes decided in that perception of the dimension of neither perception nor non-perception. They understand: ‘Here there is no stress due to the perception of the dimension of infinite consciousness or the perception of the dimension of nothingness. There is only this modicum of stress, namely the oneness dependent on the perception of the dimension of neither perception nor non-perception.’ They understand: ‘This field of perception is empty of the perception of the dimension of infinite consciousness. It is empty of the perception of the dimension of nothingness. There is only this that is not emptiness, namely the oneness dependent on the perception of the dimension of neither perception nor non-perception.’ And so they regard it as empty of what is not there, but as to what remains they understand that it is present. That’s how emptiness manifests in them—genuine, undistorted, and pure. 

Furthermore,\marginnote{10.1} a mendicant—ignoring the perception of the dimension of nothingness and the perception of the dimension of neither perception nor non-perception—focuses on the oneness dependent on the signless immersion of the heart.\footnote{The defining characteristic of the “signless immersion of the heart” is that consciousness does not “follow after signs” (\textit{\textsanskrit{nimittānusāri} \textsanskrit{viññāṇaṁ}}, eg. \href{https://suttacentral.net/an6.13/en/sujato\#5.3}{AN 6.13:5.3}). This is explained in \href{https://suttacentral.net/mn138/en/sujato\#10.2}{MN 138:10.2} as not being distracted or affected by the features of sense impressions, as the “signs” are created by greed, hate, and delusion (\href{https://suttacentral.net/mn43/en/sujato\#37.1}{MN 43:37.1}). By itself, however, it does not necessarily indicate that one is awakened (\href{https://suttacentral.net/an6.60/en/sujato\#8.2}{AN 6.60:8.2}), as good conduct is still required for progress (\href{https://suttacentral.net/an7.56/en/sujato\#13.1}{AN 7.56:13.1}). It was practiced by \textsanskrit{Moggallāna} before awakening, who had to be urged to “not follow any signs” (\href{https://suttacentral.net/sn40.9/en/sujato}{SN 40.9}). It may also be used to describe the meditation of an arahant (\href{https://suttacentral.net/sn41.7/en/sujato\#6.12}{SN 41.7:6.12}). } Their mind leaps forth, gains confidence, settles down, and becomes decided in that signless immersion of the heart. They understand: ‘Here there is no stress due to the perception of the dimension of nothingness or the perception of the dimension of neither perception nor non-perception. There is only this modicum of stress, namely that related to the six sense fields dependent on this body and conditioned by life.’\footnote{Unlike with deep absorption, where there is no experience of the physical body, the signless meditation is a state of advanced discernment or insight where the six senses are seen as they truly are. This explains why it was a way for the Buddha to ease the pains of old age (\href{https://suttacentral.net/dn16/en/sujato\#2.25.13}{DN 16:2.25.13}). It seems that in such a state, the Buddha was able to function normally while seeing through the pain in his body. } They understand: ‘This field of perception is empty of the perception of the dimension of nothingness. It is empty of the perception of the dimension of neither perception nor non-perception. There is only this that is not emptiness, namely that related to the six sense fields dependent on this body and conditioned by life.’ And so they regard it as empty of what is not there, but as to what remains they understand that it is present. That’s how emptiness manifests in them—genuine, undistorted, and pure. 

Furthermore,\marginnote{11.1} a mendicant—ignoring the perception of the dimension of nothingness and the perception of the dimension of neither perception nor non-perception—focuses on the oneness dependent on the signless immersion of the heart. Their mind leaps forth, gains confidence, settles down, and becomes decided in that signless immersion of the heart. They understand: ‘Even this signless immersion of the heart is produced by choices and intentions.’\footnote{Liberating insight comes from insight into the state of signless meditation, that is to say, it is the insight into insight itself. } They understand: ‘But whatever is produced by choices and intentions is impermanent and liable to cessation.’ Knowing and seeing like this, their mind is freed from the defilements of sensuality, desire to be reborn, and ignorance. When they’re freed, they know they’re freed. 

They\marginnote{11.8} understand: ‘Rebirth is ended, the spiritual journey has been completed, what had to be done has been done, there is nothing further for this place.’ 

They\marginnote{12.1} understand: ‘Here there is no stress due to the defilements of sensuality, desire to be reborn, or ignorance. There is only this modicum of stress, namely that related to the six sense fields dependent on this body and conditioned by life.’ They understand: ‘This field of perception is empty of the perception of the defilements of sensuality, desire to be reborn, and ignorance.\footnote{The “field of perception” (\textit{\textsanskrit{saññāgataṁ}}) is the scope of awareness. } There is only this that is not emptiness, namely that related to the six sense fields dependent on this body and conditioned by life.’ And so they regard it as empty of what is not there, but as to what remains they understand that it is present. That’s how emptiness manifests in them—genuine, undistorted, pure, and supreme. 

Whatever\marginnote{13.1} ascetics and brahmins enter and remain in the pure, ultimate, supreme emptiness—whether in the past, future, or present—all of them enter and remain in this same pure, ultimate, supreme emptiness.\footnote{Here the emptiness is distinguished as “supreme” (\textit{anuttara}). In early Pali, \textsanskrit{Nibbāna} is not described as either empty or as not-self, as these are properties of the path. This passage is probably the closest we get to that, but even this is relative, since the body and life persist. } So, Ānanda, you should train like this: ‘We will enter and remain in the pure, ultimate, supreme emptiness.’ That’s how you should train.” 

That\marginnote{13.6} is what the Buddha said. Satisfied, Venerable Ānanda approved what the Buddha said. 

%
\section*{{\suttatitleacronym MN 122}{\suttatitletranslation The Longer Discourse on Emptiness }{\suttatitleroot Mahāsuññatasutta}}
\addcontentsline{toc}{section}{\tocacronym{MN 122} \toctranslation{The Longer Discourse on Emptiness } \tocroot{Mahāsuññatasutta}}
\markboth{The Longer Discourse on Emptiness }{Mahāsuññatasutta}
\extramarks{MN 122}{MN 122}

\scevam{So\marginnote{1.1} I have heard.\footnote{In the discourse, as with the previous, emptiness is associated with physical seclusion and the development of serenity meditation leading to insight. } }At one time the Buddha was staying in the land of the Sakyans, near Kapilavatthu in the Banyan Tree Monastery. 

Then\marginnote{2.1} the Buddha robed up in the morning and, taking his bowl and robe, entered Kapilavatthu for alms. He wandered for alms in Kapilavatthu. After the meal, on his return from almsround, he went to the lodge of the Sakyan Khemaka the Dark for the day’s meditation.\footnote{The Sakyans Khemaka, known for his dark skin, and \textsanskrit{Ghaṭā} are otherwise unknown. Their “lodges” (\textit{\textsanskrit{vihāra}}) were, according to the commentary, buildings offered by them for the use of the \textsanskrit{Saṅgha} inside the Banyan Tree Monastery, rather than being their personal homes. } 

Now\marginnote{2.3} at that time many resting places had been spread out at the lodge of Khemaka the Dark.\footnote{The fact that they were readily noticeable suggests that the beds were in the open, a detail confirmed by Tibetan parallel. Bedding and furniture were typically property of the \textsanskrit{Saṅgha}, rather than individuals, so it is an offence to leave them exposed to the elements (\href{https://suttacentral.net/pli-tv-bu-vb-pc14/en/sujato}{Bu Pc 14}, \href{https://suttacentral.net/pli-tv-bu-vb-pc15/en/sujato}{Bu Pc 15}). } The Buddha saw this, and wondered, “Many resting places have been spread out; are there many mendicants living here?” 

Now\marginnote{2.8} at that time Venerable Ānanda, together with many other mendicants, was making robes at the lodge of the Sakyan \textsanskrit{Ghaṭā}.\footnote{Some translators (Chalmers, Horner, Siam Rath), as well as the Dictionary of Pali Proper Names, take this name as the masculine form \textit{\textsanskrit{ghaṭāya}}, assuming it is compounded with \textit{sakkassa}; but names are not usually compounded in this way (cf. \textit{\textsanskrit{kāḷakhemakassa} sakkassa} above), and \textit{\textsanskrit{ghaṭāya}} does not seem to be attested elsewhere. In his hand-written manuscript, \textsanskrit{Ñāṇamoḷī} had \textit{\textsanskrit{ghāṭā}}, which was followed by Bodhi and Thanissaro. This reads the masculine agent noun “slayer”; but the genitive form would be \textit{\textsanskrit{ghātassa}}, while PTS, BJT, MS, and commentary all have \textit{\textsanskrit{ghaṭāya}} with no variants. Rather, read per text as the feminine genitive of \textit{\textsanskrit{ghaṭā}}, suggesting that it was a woman who donated the building. She was perhaps born under the astrological sign known in India as \textit{kumbha} or \textit{\textsanskrit{ghaṭa}} (“water-pot”), and in the west as Aquarius (“water-bearer”). } Then in the late afternoon, the Buddha came out of retreat and went to \textsanskrit{Ghaṭā}’s lodge, where he sat on the seat spread out and said to Venerable Ānanda, “Many resting places have been spread out at the lodge of Khemaka the Dark; are many mendicants living there?” 

“Indeed\marginnote{2.14} there are, sir. It’s currently the time for making robes.”\footnote{This the period after the Rains Retreat, when cloth is offered by the lay community at the \textit{\textsanskrit{kaṭhiṇa}} ceremony, and monastics busy themselves with sewing and repairs before setting out wandering. To this day, it is a time when the pent-up energies of the retreat are relaxed and the \textsanskrit{Saṅgha} gathers for socializing in large numbers. } 

“Ānanda,\marginnote{3.1} a mendicant doesn’t shine who enjoys company and groups, who loves them and likes to enjoy them. It is quite impossible that such a mendicant will get the pleasure of renunciation, the pleasure of seclusion, the pleasure of peace, the pleasure of awakening when they want, without trouble or difficulty.\footnote{This is defined at \href{https://suttacentral.net/mn139/en/sujato\#9.18}{MN 139:9.18} as the four absorptions. } But you should expect that a mendicant who lives alone, withdrawn from the group, will get the pleasure of renunciation, the pleasure of seclusion, the pleasure of peace, the pleasure of awakening when they want, without trouble or difficulty. That is possible. 

Indeed,\marginnote{4.1} Ānanda, it is quite impossible that a mendicant who enjoys company will enter and remain in the freedom of heart—either that which is temporary and pleasant, or that which is irreversible and unshakable.\footnote{The “temporary and pleasant” liberation is the absorptions, while the “irreversible and unshakable” liberation is arahantship. } But it is possible that a mendicant who lives alone, withdrawn from the group will enter and remain in the freedom of heart—either that which is temporary and pleasant, or that which is irreversible and unshakable. 

Ānanda,\marginnote{5.1} I do not see even a single sight which, with its decay and perishing, would not give rise to sorrow, lamentation, pain, sadness, and distress in someone who has desire and lust for it.\footnote{Taken in isolation, \textit{\textsanskrit{rūpa}} here might be “physical form” (i.e. “body”), but in passages with similar phrasing it is “visible form” (i.e. “sight”; \href{https://suttacentral.net/an1.1/en/sujato\#2.1}{AN 1.1:2.1}, \href{https://suttacentral.net/an5.55/en/sujato\#3.3}{AN 5.55:3.3}). } 

But\marginnote{6.1} this meditation has been understood by the Realized One, namely to enter and remain in emptiness internally by not focusing on any signs.\footnote{Normally the meditations on emptiness and on signlessness are presented as separate  meditation practices (\href{https://suttacentral.net/sn41.7/en/sujato\#4.1}{SN 41.7:4.1}). But in the previous sutta, the meditation on signlessness was presented as one of the stages in the manifestation of emptiness (\href{https://suttacentral.net/mn121/en/sujato\#10.1}{MN 121:10.1}). The outcome of the signless meditation is a heart that is “empty” of greed, hate, and delusion (\href{https://suttacentral.net/mn43/en/sujato\#37.1}{MN 43:37.1}). } Now, suppose that while the Realized One is practicing this meditation, monks, nuns, laymen, laywomen, rulers and their chief ministers, monastics of other religions and their disciples go to visit him.\footnote{\textit{Tittha}, literally “ford”, is a path to salvation, used as a term for a non-Buddhist “religion”. \textit{Titthakara} is a “religious founder” (literally “ford-maker”); \textit{titthiya} is a “monastic of (another) religion” (for example at \href{https://suttacentral.net/pli-tv-bu-vb-np22/en/sujato\#1.2.5}{Bu NP 22:1.2.5}); \textit{\textsanskrit{titthiyasāvaka}} is a “disciple of a monastic of (another) religion”. } In that case, with a mind slanting, sloping, and inclining to seclusion, withdrawn, and loving renunciation, having totally eliminated defiling influences, he invariably gives each of them a talk emphasizing the topic of dismissal. 

Therefore,\marginnote{7.1} if a mendicant should wish, ‘May I enter and remain in emptiness internally!’ then they should still, settle, unify, and immerse their mind in \textsanskrit{samādhi} internally. 

And\marginnote{7.3} how does a mendicant still, settle, unify, and immerse their mind in \textsanskrit{samādhi} internally? 

It’s\marginnote{8.1} when a mendicant, quite secluded from sensual pleasures, secluded from unskillful qualities, enters and remains in the first absorption … second absorption … third absorption … fourth absorption. That’s how a mendicant stills, settles, unifies, and immerses their mind in \textsanskrit{samādhi} internally. 

They\marginnote{9.1} focus on emptiness internally, but their mind does not leap forth, gain confidence, settle down, and become decided. In that case, they understand: ‘I am focusing on emptiness internally, but my mind does not leap forth, gain confidence, settle down, and become decided.’ In this way they are aware of the situation.\footnote{This phrase indicates that they have “situational awareness” (\textit{\textsanskrit{sampajañña}}), namely a contextual overview of how their meditation is progressing, understood in terms of cause and effect. The phrase also occurs in the same sense at \href{https://suttacentral.net/an7.49/en/sujato\#3.3}{AN 7.49:3.3}. } They focus on emptiness externally … They focus on emptiness internally and externally … They focus on the imperturbable,\footnote{For imperturbable, see \href{https://suttacentral.net/mn105/en/sujato\#12.3}{MN 105:12.3}. In this mode of practice, they wanted to develop emptiness before the formless attainments. Contrast with \href{https://suttacentral.net/mn121/en/sujato\#10.1}{MN 121:10.1}, where they develop the signless meditation only after the formless attainments. } but their mind does not leap forth, gain confidence, settle down, and become decided. In that case, they understand: ‘I am focusing on the imperturbable internally, but my mind does not leap forth, gain confidence, settle down, and become decided.’ In this way they are aware of the situation. 

Then\marginnote{10.1} that mendicant should still, settle, unify, and immerse their mind in \textsanskrit{samādhi} internally using the same meditation subject as a basis of immersion that they used before.\footnote{Even though they had previously attained up to the fourth absorption, their practice is not yet fully stabilized. | For this phrase, see \href{https://suttacentral.net/mn36/en/sujato\#45.6}{MN 36:45.6} and \href{https://suttacentral.net/an6.28/en/sujato\#4.3}{AN 6.28:4.3}. } They focus on emptiness internally,\footnote{In this context, it seems that “emptiness” is a way of deepening absorption. } and their mind leaps forth, gains confidence, settles down, and becomes decided. In that case, they understand: ‘I am focusing on emptiness internally, and my mind leaps forth, gains confidence, settles down, and becomes decided.’ In this way they are aware of the situation. They focus on emptiness externally … They focus on emptiness internally and externally … They focus on the imperturbable, and their mind leaps forth, gains confidence, settles down, and becomes decided. In that case, they understand: ‘I am focusing on the imperturbable, and my mind leaps forth, gains confidence, settles down, and becomes decided.’ In this way they are aware of the situation. 

While\marginnote{11.1} a mendicant is practicing such a meditation, if their mind inclines to walking, they walk, thinking:\footnote{The Buddha goes on to explain who one with such advanced meditation is able to maintain their practice. | The phrase \textit{tassa … bhikkhuno \textsanskrit{iminā} \textsanskrit{vihārena} viharato} (“While a mendicant is practicing such a meditation”) sounds like it means they are actually in the meditation state itself, but this passage shows that this is not necessarily the case, as they are able to walk about and hold conversations. } ‘While I’m walking, bad, unskillful qualities of covetousness and displeasure will not overwhelm me.’ In this way they are aware of the situation. While a mendicant is practicing such a meditation, if their mind inclines to standing, they stand, thinking: ‘While I’m standing, bad, unskillful qualities of covetousness and displeasure will not overwhelm me.’ In this way they are aware of the situation. While a mendicant is practicing such a meditation, if their mind inclines to sitting, they sit, thinking: ‘While I’m sitting, bad, unskillful qualities of covetousness and displeasure will not overwhelm me.’ In this way they are aware of the situation. While a mendicant is practicing such a meditation, if their mind inclines to lying down, they lie down, thinking: ‘While I’m lying down, bad, unskillful qualities of covetousness and displeasure will not overwhelm me.’ In this way they are aware of the situation. 

While\marginnote{12.1} a mendicant is practicing such a meditation, if their mind inclines to talking, they think: ‘I will not engage in the kind of speech that is low, crude, ordinary, ignoble, and pointless. Such speech doesn’t lead to disillusionment, dispassion, cessation, peace, insight, awakening, and extinguishment. Namely: talk about kings, bandits, and chief ministers; talk about armies, threats, and wars; talk about food, drink, clothes, and beds; talk about garlands and fragrances; talk about family, vehicles, villages, towns, cities, and countries; talk about women and heroes; street talk and well talk; talk about the departed; motley talk; tales of land and sea; and talk about being reborn in this or that state of existence.’ In this way they are aware of the situation. ‘But I will take part in talk about self-effacement that helps open the heart and leads solely to disillusionment, dispassion, cessation, peace, insight, awakening, and extinguishment. That is, talk about fewness of wishes, contentment, seclusion, aloofness, arousing energy, ethics, immersion, wisdom, freedom, and the knowledge and vision of freedom.’ In this way they are aware of the situation. 

While\marginnote{13.1} a mendicant is practicing such a meditation, if their mind inclines to thinking, they think: ‘I will not think the kind of thought that is low, crude, ordinary, ignoble, and pointless. Such thoughts don’t lead to disillusionment, dispassion, cessation, peace, insight, awakening, and extinguishment. That is, sensual, malicious, or cruel thoughts.’ In this way they are aware of the situation. ‘But I will think the kind of thought that is noble and emancipating, and brings one who practices it to the complete ending of suffering. That is, thoughts of renunciation, good will, and harmlessness.’ In this way they are aware of the situation. 

There\marginnote{14.1} are these five kinds of sensual stimulation. What five? Sights known by the eye, which are likable, desirable, agreeable, pleasant, sensual, and arousing. Sounds known by the ear … Smells known by the nose … Tastes known by the tongue … Touches known by the body, which are likable, desirable, agreeable, pleasant, sensual, and arousing. These are the five kinds of sensual stimulation. 

So\marginnote{15.1} a mendicant should regularly check their own mind: ‘Does my mind take an interest in any of these five kinds of sensual stimulation?’ Suppose that, upon checking, a mendicant knows this: ‘My mind does take an interest.’ In that case, they understand: ‘I have not given up desire and greed for the five kinds of sensual stimulation.’ In this way they are aware of the situation.\footnote{Even such an advanced meditator might sill have baser desires beneath the surface. However, so long as they are aware of this they do not stray from the path. } But suppose that, upon checking, a mendicant knows this: ‘My mind does not take an interest.’ In that case, they understand: ‘I have given up desire and greed for the five kinds of sensual stimulation.’ In this way they are aware of the situation. 

A\marginnote{16.1} mendicant should meditate observing rise and fall in these five grasping aggregates: ‘Such is form, such is the origin of form, such is the ending of form. Such is feeling … Such is perception … Such are choices … Such is consciousness, such is the origin of consciousness, such is the ending of consciousness.’ 

As\marginnote{17.1} they do so, they give up the conceit ‘I am’ regarding the five grasping aggregates.\footnote{Finally the meditator undertakes discernment (\textit{\textsanskrit{vipassanā}}). } In that case, they understand: ‘I have given up the conceit “I am” regarding the five grasping aggregates.’ In this way they are aware of the situation. 

These\marginnote{18.1} principles are entirely skillful, with skillful outcomes; they are noble, transcendent, and inaccessible to the Wicked One. 

What\marginnote{19.1} do you think, Ānanda? For what reason would a disciple deem it worthwhile to follow the Teacher, even if sent away?” 

“Our\marginnote{19.3} teachings are rooted in the Buddha. He is our guide and our refuge. Sir, may the Buddha himself please clarify the meaning of this. The mendicants will listen and remember it.” 

“A\marginnote{20.1} disciple would not deem it worthwhile to follow the Teacher for the sake of this, namely statements, mixed prose \& verse, or discussions.\footnote{Here we find mentioned only the first three of the normal nine categories (\textit{\textsanskrit{aṅga}}) of the teaching (\href{https://suttacentral.net/mn22/en/sujato\#10.2}{MN 22:10.2}). This may be an earlier list or perhaps simply an abbreviation. Note that the PTS edition reads \textit{\textsanskrit{yadidaṁ} \textsanskrit{suttaṁ} \textsanskrit{geyyaṁ} \textsanskrit{veyyākaraṇassa} hetu}, which would imply only two categories: “for the sake of explanations, namely statements and mixed prose \& verse”. A passage at \href{https://suttacentral.net/an5.194/en/sujato\#3.3}{AN 5.194:3.3} mentions four categories, with the addition of “amazing stories”, but this too might be simply an artifact of abbreviation. } Why is that? Because for a long time you have learned the teachings, remembering them, rehearsing them, mentally scrutinizing them, and comprehending them theoretically.\footnote{The Buddha shifts to address Ānanda directly. This is in reference to the fact that Ānanda was renowned as the greatest master of scriptural memorization. } But a disciple would deem it worthwhile to follow the Teacher, even if sent away, for the sake of talk about self-effacement that helps open the heart and leads solely to disillusionment, dispassion, cessation, peace, insight, awakening, and extinguishment. That is, talk about fewness of wishes, contentment, seclusion, aloofness, arousing energy, ethics, immersion, wisdom, freedom, and the knowledge and vision of freedom.\footnote{That is, rather than simply memorizing more texts, one such as Ānanda should seek encouragement that supports practice. } 

This\marginnote{21.1} being so, Ānanda, there is a peril for the tutor, a peril for the pupil, and a peril for a spiritual practitioner. 

And\marginnote{22.1} how is there a peril for the tutor?\footnote{As is made clear by the example of the spiritual practitioner below, who is said to follow the Buddha, the “tutor” (\textit{\textsanskrit{ācariya}}, or \textit{\textsanskrit{satthā}} “teacher” in the next line) and the “pupil” (\textit{\textsanskrit{antevāsī}}) here are outside of  Buddhism. } It’s when some teacher frequents a secluded lodging—a wilderness, the root of a tree, a hill, a ravine, a mountain cave, a charnel ground, a forest, the open air, a heap of straw. While meditating withdrawn, they’re visited by a stream of brahmins and householders, and people of town and country.\footnote{The more a monastic has a reputation for seclusion, the more people seek them out. } When this happens, they enjoy infatuation, fall into greed, and return to indulgence. This is said to be the tutor’s peril.\footnote{Compare the similar phrase at \href{https://suttacentral.net/sn16.8/en/sujato\#7.7}{SN 16.8:7.7}. } They’re ruined by bad, unskillful qualities that are corrupting, leading to future lives, hurtful, resulting in suffering and future rebirth, old age, and death. That’s how there is a peril for the tutor. 

And\marginnote{23.1} how is there a peril for the pupil? It’s when the disciple of a teacher, emulating their teacher’s fostering of seclusion, frequents a secluded lodging—a wilderness, the root of a tree, a hill, a ravine, a mountain cave, a charnel ground, a forest, the open air, a heap of straw. While meditating withdrawn, they’re visited by a stream of brahmins and householders, and people of town and country. When this happens, they enjoy infatuation, fall into greed, and return to indulgence. This is said to be the pupil’s peril. They’re ruined by bad, unskillful qualities that are corrupting, leading to future lives, hurtful, resulting in suffering and future rebirth, old age, and death. That’s how there is a peril for the pupil. 

And\marginnote{24.1} how is there a peril for a spiritual practitioner? It’s when a Realized One arises in the world, perfected, a fully awakened Buddha, accomplished in knowledge and conduct, holy, knower of the world, supreme guide for those who wish to train, teacher of gods and humans, awakened, blessed. He frequents a secluded lodging—a wilderness, the root of a tree, a hill, a ravine, a mountain cave, a charnel ground, a forest, the open air, a heap of straw. While meditating withdrawn, he’s visited by a stream of brahmins and householders, and people of town and country. When this happens, he doesn’t enjoy infatuation, fall into greed, and return to indulgence. But a disciple of this teacher, emulating their teacher’s fostering of seclusion, frequents a secluded lodging—a wilderness, the root of a tree, a hill, a ravine, a mountain cave, a charnel ground, a forest, the open air, a heap of straw. While meditating withdrawn, they’re visited by a stream of brahmins and householders, and people of town and country. When this happens, they enjoy infatuation, fall into greed, and return to indulgence. This is said to be the spiritual practitioner’s peril. They’re ruined by bad, unskillful qualities that are corrupting, leading to future lives, hurtful, resulting in suffering and future rebirth, old age, and death. That’s how there is a peril for the spiritual practitioner. 

And\marginnote{24.13} in this context, Ānanda, as compared to the peril of the tutor or the pupil, the peril of the spiritual practitioner has more painful, bitter results, and even leads to the underworld. 

So,\marginnote{25.1} Ānanda, treat me as a friend, not as an enemy. That will be for your lasting welfare and happiness. 

And\marginnote{25.3} how do disciples treat their Teacher as an enemy, not a friend? It’s when the Teacher teaches Dhamma to his disciples out of kindness and sympathy: ‘This is for your welfare. This is for your happiness.’ But their disciples don’t want to listen. They don’t actively listen or try to understand. They proceed having turned away from the Teacher’s instruction. That’s how the disciples treat their Teacher as an enemy, not a friend. 

And\marginnote{26.1} how do disciples treat their Teacher as a friend, not an enemy? It’s when the Teacher teaches Dhamma to his disciples out of kindness and sympathy: ‘This is for your welfare. This is for your happiness.’ And their disciples want to listen. They actively listen and try to understand. They don’t proceed having turned away from the Teacher’s instruction. That’s how the disciples treat their Teacher as a friend, not an enemy. 

So,\marginnote{26.6} Ānanda, treat me as a friend, not as an enemy. That will be for your lasting welfare and happiness. I shall not mollycoddle you like a potter with their damp, unfired pots.\footnote{This passage seems to mean, “I shall not work you …”, which is the exact opposite of the sense required by the next sentence. A Sanskrit parallel yields the required sense with the verb \textit{\textsanskrit{dhanayiṣye}}, “to treasure, to mollycoddle” (\href{https://suttacentral.net/san-mu-kd17/san/gbm?lang=en&reference=main&highlight=false\#sc493}{Mūlasarvāstivāda Vinaya, Kd 17:493}). Perhaps an initial negatory \textit{a-} has been elided in the Pali \textit{\textsanskrit{parakkamissāmi}}, which would yield a similar sense, “I shall not stop working you …”. | The same Sanskrit passage has \textit{\textsanskrit{bhājanānā}} (“pots”) where the Pali has \textit{\textsanskrit{āmake} \textsanskrit{āmakamatte}}, literally “uncooked, a bit damp”. } I shall speak, correcting you again and again, pressing you again and again. The core will stand the test.” 

That\marginnote{27.5} is what the Buddha said. Satisfied, Venerable Ānanda approved what the Buddha said. 

%
\section*{{\suttatitleacronym MN 123}{\suttatitletranslation Incredible and Amazing }{\suttatitleroot Acchariyaabbhutasutta}}
\addcontentsline{toc}{section}{\tocacronym{MN 123} \toctranslation{Incredible and Amazing } \tocroot{Acchariyaabbhutasutta}}
\markboth{Incredible and Amazing }{Acchariyaabbhutasutta}
\extramarks{MN 123}{MN 123}

\scevam{So\marginnote{1.1} I have heard. }At one time the Buddha was staying near \textsanskrit{Sāvatthī} in Jeta’s Grove, \textsanskrit{Anāthapiṇḍika}’s monastery. 

Then\marginnote{2.1} after the meal, on return from almsround, several mendicants sat together in the assembly hall and this discussion came up among them: 

“It’s\marginnote{2.2} incredible, reverends, it’s amazing! The Realized One has such psychic power and might! For he is able to know the Buddhas of the past who have become completely quenched, cut off proliferation, cut off the track, finished off the cycle, and transcended all suffering. He knows their birth, names, clans, conduct, teaching, wisdom, meditation, and freedom.”\footnote{This is in reference to the \textsanskrit{Mahāpadānasutta}, which is where we learn the Buddha’s mother was named \textsanskrit{Māyā} (\href{https://suttacentral.net/dn14/en/sujato}{DN 14}). But whereas that sutta emphasizes the “normalcy” (\textit{\textsanskrit{dhammatā}}) of Buddhahood as a recurring cycle down through the eons, this sutta is devotional, emphasizing Gotama’s personal qualities. } 

When\marginnote{2.5} they said this, Venerable Ānanda said, “The Realized Ones are incredible, reverends, and they have incredible qualities. They’re amazing, and they have amazing qualities.”\footnote{This a key example of the genre of “amazing stories” (\textit{abbhutadhamma}), one of the nine categories (eg. \href{https://suttacentral.net/mn22/en/sujato\#10.2}{MN 22:10.2}), linking this sutta with the previous (\href{https://suttacentral.net/mn122/en/sujato\#20.1}{MN 122:20.1}). This genre is closely connected to Ānanda, so much so that he may be regarded as its originator (eg. \href{https://suttacentral.net/dn1/en/sujato\#3.74.2}{DN 1:3.74.2}, \href{https://suttacentral.net/dn15/en/sujato\#1.4}{DN 15:1.4}, \href{https://suttacentral.net/dn16/en/sujato\#3.11.2}{DN 16:3.11.2}, \href{https://suttacentral.net/sn51.22/en/sujato\#2.2}{SN 51.22:2.2}, \href{https://suttacentral.net/an4.76/en/sujato\#5.5}{AN 4.76:5.5}, etc.). When Ānanda was at a low point, the Buddha turned it around to speak of Ānanda’s amazing qualities (\href{https://suttacentral.net/dn16/en/sujato\#5.16.15}{DN 16:5.16.15} = \href{https://suttacentral.net/an4.129/en/sujato}{AN 4.129}). } But this conversation among those mendicants was left unfinished. 

Then\marginnote{2.9} in the late afternoon, the Buddha came out of retreat, went to the assembly hall, sat down on the seat spread out, and addressed the mendicants: “Mendicants, what were you sitting talking about just now? What conversation was left unfinished?” 

So\marginnote{2.12} the mendicants told him what they had been talking about. The Buddha said, “Well then, Ānanda, say some more about the incredible and amazing qualities of the Realized One.” 

“Sir,\marginnote{3.1} I have heard and learned this in the presence of the Buddha: ‘Mindful and aware, the being intent on awakening was reborn in the host of joyful gods.’\footnote{The first three items are unique to this sutta. | Normally in early Pali, the word \textit{bodhisatta} is reserved for the Buddha-to-be once he has left home and is practicing “intent on awakening” (eg. \href{https://suttacentral.net/mn26/en/sujato\#13.1}{MN 26:13.1}). \href{https://suttacentral.net/dn14/en/sujato\#1.17.1}{DN 14:1.17.1} and \href{https://suttacentral.net/an4.127/en/sujato\#1.3}{AN 4.127:1.3} extend the usage back as far as the end of the immediate past life. Here, he is called \textit{bodhisatta} from the beginning of the immediate past life. The Chinese parallel at MA 32 goes back even further, depicting the \textit{bodhisatta} making his initial vow to become a Buddha in the time of the Buddha Kassapa. Compare \href{https://suttacentral.net/mn81/en/sujato}{MN 81} and its parallel MA 63, where no such vow is mentioned. Only then does he take rebirth in the Joyful heaven. These differences indicate a development of the \textit{bodhisatta} doctrine even in early texts. } This I remember as an incredible quality of the Buddha. 

I\marginnote{4.1} have learned this in the presence of the Buddha: ‘Mindful and aware, the being intent on awakening remained in the host of joyful gods.’ This too I remember as an incredible quality of the Buddha. 

I\marginnote{5.1} have learned this in the presence of the Buddha: ‘For the whole of that life, the being intent on awakening remained in the host of joyful gods.’ This too I remember as an incredible quality of the Buddha. 

I\marginnote{6.1} have learned this in the presence of the Buddha: ‘Mindful and aware, the being intent on awakening passed away from the host of joyful gods and was conceived in his mother’s womb.’\footnote{From here on, the details are the same as \href{https://suttacentral.net/dn14/en/sujato\#1.17.1}{DN 14:1.17.1}. Additionally, the details of passing away from heaven and emerging from the womb are also found in \href{https://suttacentral.net/an4.127/en/sujato\#1.3}{AN 4.127:1.3}. } This too I remember as an incredible quality of the Buddha. 

I\marginnote{7.1} have learned this in the presence of the Buddha: ‘When the being intent on awakening passes away from the host of joyful gods, he is conceived in his mother’s womb. And then—in this world with its gods, \textsanskrit{Māras}, and divinities, this population with its ascetics and brahmins, gods and humans—an immeasurable, magnificent light appears, surpassing the glory of the gods. Even in the boundless void of interstellar space—so utterly dark that even the light of the moon and the sun, so mighty and powerful, makes no impression—an immeasurable, magnificent light appears, surpassing the glory of the gods.\footnote{The commentary identifies this realm of “utter darkness” (\textit{\textsanskrit{andhakāratimisā}}) with a cold hell realm. There is a corresponding \textsanskrit{Purāṇic} hell called \textit{\textsanskrit{andhatāmisra}}. | \textit{Agha} (“void”) is a synonym of \textit{\textsanskrit{ākāsa}} (“space”, \href{https://suttacentral.net/mn62/en/sujato\#12.5}{MN 62:12.5}). | \textit{\textsanskrit{Asaṁvuta}} was translated by \textsanskrit{Ñāṇamoḷī} as “abysmal”, but this relies on a commentarial cosmology that is not found in the suttas. The sense, rather, is “boundless”. The root harks back to the Vedic serpent \textsanskrit{Vṛtra} who wraps the world in darkness. | \textit{\textsanskrit{Nānubhonti}} (“makes no impression”) is glossed in the commentary to \href{https://suttacentral.net/an4.127/en/sujato}{AN 4.127} as \textit{nappahonti} “ineffective”. } And even the sentient beings reborn there recognize each other by that light: “So, it seems other sentient beings have been reborn here!”\footnote{The light is a physical one, not just a metaphor. From this, it appears that sentient beings may be spontaneously reborn in interstellar space. Compare the problem of the “Boltzmann brain” in physics. } And this ten-thousandfold galaxy shakes and rocks and trembles. And an immeasurable, magnificent light appears in the world, surpassing the glory of the gods.’ This too I remember as an incredible quality of the Buddha. 

I\marginnote{8.1} have learned this in the presence of the Buddha: ‘When the being intent on awakening is conceived in his mother’s belly, four gods approach to guard the four quarters, so that no human or non-human or anyone at all shall harm the being intent on awakening or his mother.’\footnote{These are the Four Great Kings, regarded as protector deities. | Eight items, from here until the detail of giving birth standing, are absent from the parallel at MA 32. Notably, these items mostly emphasize the amazing qualities of the mother rather than the child, so they fit uneasily in a discourse whose subject is the amazing qualities of the Buddha. } This too I remember as an incredible quality of the Buddha. 

I\marginnote{9.1} have learned this in the presence of the Buddha: ‘When the being intent on awakening is conceived in his mother’s belly, she becomes naturally ethical. She refrains from killing living creatures, stealing, sexual misconduct, lying, and beer, wine, and liquor intoxicants.’\footnote{The five precepts. } This too I remember as an incredible quality of the Buddha. 

I\marginnote{10.1} have learned this in the presence of the Buddha: ‘When the being intent on awakening is conceived in his mother’s belly, she no longer feels sexual desire for men, and she cannot be violated by a man of lustful intent.’ This too I remember as an incredible quality of the Buddha. 

I\marginnote{11.1} have learned this in the presence of the Buddha: ‘When the being intent on awakening is conceived in his mother’s belly, she obtains the five kinds of sensual stimulation and amuses herself, supplied and provided with them.’\footnote{While sensual pleasures provoke attachment, they are nonetheless a kind of pleasure and therefore a sign of virtue and good past kamma. } This too I remember as an incredible quality of the Buddha. 

I\marginnote{12.1} have learned this in the presence of the Buddha: ‘When the being intent on awakening is conceived in his mother’s belly, no afflictions beset her. She’s happy and free of bodily fatigue. And she sees the being intent on awakening in her womb, whole in his major and minor limbs, not deficient in any faculty. Suppose there was a beryl gem that was naturally beautiful, eight-faceted, well-worked. And it was strung with a thread of blue, yellow, red, white, or golden brown. And someone with clear eyes were to take it in their hand and examine it: “This beryl gem is naturally beautiful, eight-faceted, well-worked. And it’s strung with a thread of blue, yellow, red, white, or golden brown.” 

In\marginnote{12.5} the same way, when the being intent on awakening is conceived in his mother’s belly, no afflictions beset her. She’s happy and free of bodily fatigue. And she sees the being intent on awakening in her womb, whole in his major and minor limbs, not deficient in any faculty.’ This too I remember as an incredible quality of the Buddha. 

I\marginnote{13.1} have learned this in the presence of the Buddha: ‘Seven days after the being intent on awakening is born, his mother passes away and is reborn in the host of joyful gods.’\footnote{This tragic detail is also mentioned in \href{https://suttacentral.net/ud5.2/en/sujato}{Ud 5.2}. The Buddha was raised by his step-mother, \textsanskrit{Māyā}’s sister \textsanskrit{Mahāpajāpatī} (\href{https://suttacentral.net/an8.51/en/sujato\#9.8}{AN 8.51:9.8}, \href{https://suttacentral.net/mn142/en/sujato\#3.3}{MN 142:3.3}). } This too I remember as an incredible quality of the Buddha. 

I\marginnote{14.1} have learned this in the presence of the Buddha: ‘Other women carry the infant in the womb for nine or ten months before giving birth. Not so the mother of the being intent on awakening. She gives birth after exactly ten months.’\footnote{Ten signifies fullness and completion, as for example the “ten directions”. | The notion that the term of pregnancy was “nine or ten months” is also found at \textsanskrit{Chāndogya} \textsanskrit{Upaniṣad} 5.9.1. In the Rig Veda it is typically “in the tenth month” (5.78.7, 10.84.3). } This too I remember as an incredible quality of the Buddha. 

I\marginnote{15.1} have learned this in the presence of the Buddha: ‘Other women give birth while sitting or lying down. Not so the mother of the being intent on awakening. She only gives birth standing up.’\footnote{In illustrations she is depicted standing while holding a tree in the pose known as \textit{\textsanskrit{sālabhañjikā}}, a common motif in Indian art representing the abundance of springtime. } This too I remember as an incredible quality of the Buddha. 

I\marginnote{16.1} have learned this in the presence of the Buddha: ‘When the being intent on awakening emerges from his mother’s womb, gods receive him first, then humans.’ This too I remember as an incredible quality of the Buddha. 

I\marginnote{17.1} have learned this in the presence of the Buddha: ‘When the being intent on awakening emerges from his mother’s womb, before he reaches the ground, four gods receive him and place him before his mother, saying: “Rejoice, O Queen! An illustrious son is born to you.”’ This too I remember as an incredible quality of the Buddha. 

I\marginnote{18.1} have learned this in the presence of the Buddha: ‘When the being intent on awakening emerges from his mother’s womb, he emerges already clean, unsoiled by waters, mucus, blood, or any other kind of impurity, pure and clean. Suppose a jewel-treasure was placed on a cloth from \textsanskrit{Kāsi}. The jewel would not soil the cloth, nor would the cloth soil the jewel.\footnote{\textsanskrit{Kāsi} is the nation of which Varanasi is the capital. Their cloth was of exceptional quality. } Why is that? Because of the cleanliness of them both. 

In\marginnote{18.6} the same way, when the being intent on awakening emerges from his mother’s womb, he emerges already clean, unsoiled by waters, mucus, blood, or any other kind of impurity, pure and clean.’ This too I remember as an incredible quality of the Buddha. 

I\marginnote{19.1} have learned this in the presence of the Buddha: ‘When the being intent on awakening emerges from his mother’s womb, two showers of water appear from the sky, one cool, one warm, for bathing the being intent on awakening and his mother.’ This too I remember as an incredible quality of the Buddha. 

I\marginnote{20.1} have learned this in the presence of the Buddha: ‘As soon as he’s born, the being intent on awakening stands firm with his own feet on the ground. Facing north, he takes seven strides with a white parasol held above him, surveys all quarters, and makes this dramatic proclamation: “I am the foremost in the world! I am the eldest in the world! I am the first in the world! This is my last rebirth; now there’ll be no more future lives.”’\footnote{This passage implies that Buddhahood was destined from the time of birth, which stands in contrast to the rest of the suttas, where Buddhahood was hard-won by the Bodhisatta’s efforts while striving for awakening. | “Stands firm on his own feet” signifies that he will be awakened by his own efforts. | “North” is \textit{uttara}, which is also “the beyond”; this predicts his attaining Nibbana. | “Seven strides” signifies crossing over the vast cycles of birth and death, especially by developing the seven awakening factors. | The “white parasol” signifies purity and royalty. | “Surveying all quarters” signifies his universal knowledge. | The “dramatic proclamation” is \textit{\textsanskrit{āsabhiṁ} \textsanskrit{vācaṁ}}, literally the “voice of a bull”. Other contexts show that this is is an expression emphasizing speech that is dramatic and imposing (\href{https://suttacentral.net/sn52.9/en/sujato\#3.2}{SN 52.9:3.2}, \href{https://suttacentral.net/dn28/en/sujato\#1.5}{DN 28:1.5} = \href{https://suttacentral.net/sn47.12/en/sujato\#5.2}{SN 47.12:5.2}). | At \textsanskrit{Bṛhadāraṇyaka} \textsanskrit{Upaniṣad} 6.1.1 and \textsanskrit{Chāndogya} \textsanskrit{Upaniṣad} 5.1.1 the “vital breath” (\textit{\textsanskrit{prāṇa}}) is said to be “eldest and first” (\textit{\textsanskrit{jyeṣṭhaśca} \textsanskrit{śreṣṭhaśca}}). } This too I remember as an incredible quality of the Buddha. 

I\marginnote{21.1} have learned this in the presence of the Buddha: ‘When the being intent on awakening emerges from his mother’s womb, then—in this world with its gods, \textsanskrit{Māras}, and divinities, this population with its ascetics and brahmins, gods and humans—an immeasurable, magnificent light appears, surpassing the glory of the gods. Even in the boundless void of interstellar space—so utterly dark that even the light of the moon and the sun, so mighty and powerful, makes no impression—an immeasurable, magnificent light appears, surpassing the glory of the gods. And the sentient beings reborn there recognize each other by that light: “So, it seems other sentient beings have been reborn here!” And this ten-thousandfold galaxy shakes and rocks and trembles. And an immeasurable, magnificent light appears in the world, surpassing the glory of the gods.’ This too I remember as an incredible and amazing quality of the Buddha.”\footnote{Here the Chinese parallel at MA 32 adds a series of ten extra items, mostly details of Siddhattha’s life before awakening. } 

“Well\marginnote{22.1} then, Ānanda, you should also remember this as an incredible and amazing quality of the Realized One. It’s that the Realized One knows feelings as they arise, as they remain, and as they go away.\footnote{This practice is said to be the “awareness” part of “mindfulness and awareness” (\textit{\textsanskrit{satisampajañña}}) at \href{https://suttacentral.net/sn47.35/en/sujato\#3.1}{SN 47.35:3.1}, while at \href{https://suttacentral.net/dn33/en/sujato\#1.11.37}{DN 33:1.11.37} = \href{https://suttacentral.net/an4.41/en/sujato\#4.1}{AN 4.41:4.1} it is a further development of immersion leading to awareness. | The Buddha is gently reframing Ānanda’s devotionalism, pointing back to the practice of meditation, a quality that he shares with disciples (eg. \textsanskrit{Sāriputta} at \href{https://suttacentral.net/an7.39/en/sujato\#1.6}{AN 7.39:1.6} and Nanda at \href{https://suttacentral.net/an8.9/en/sujato\#5.1}{AN 8.9:5.1}). Subtly, he is also completing the circle of the sutta, as the first items speak of how the \textit{bodhisatta} was “mindful and aware” when reborn (\href{https://suttacentral.net/mn123/en/sujato\#3.2}{MN 123:3.2}). Thus while Ānanda spoke of results, the Buddha pointed to the cause of awakening, a rhetorical twist he also employed at \href{https://suttacentral.net/mn32/en/sujato\#17.5}{MN 32:17.5}. } He knows perceptions as they arise, as they remain, and as they go away. He knows thoughts as they arise, as they remain, and as they go away. This too you should remember as an incredible and amazing quality of the Realized One.” 

“Sir,\marginnote{23.1} the Buddha knows feelings as they arise, as they remain, and as they go away. He knows perceptions as they arise, as they remain, and as they go away. He knows thoughts as they arise, as they remain, and as they go away. This too I remember as an incredible and amazing quality of the Buddha.” 

That’s\marginnote{23.5} what Ānanda said, and the teacher approved. Satisfied, those mendicants approved what Venerable Ānanda said. 

%
\section*{{\suttatitleacronym MN 124}{\suttatitletranslation With Bakkula }{\suttatitleroot Bākulasutta}}
\addcontentsline{toc}{section}{\tocacronym{MN 124} \toctranslation{With Bakkula } \tocroot{Bākulasutta}}
\markboth{With Bakkula }{Bākulasutta}
\extramarks{MN 124}{MN 124}

\scevam{So\marginnote{1.1} I have heard.\footnote{Bakkula is depicted as an exponent of austerity who goes significantly beyond the norm in the early texts. According to the commentary, this sutta was added by the Elders at the Second Council, a century after the Buddha’s passing. This is supported by the lack of mention of where the Buddha is staying; by the advanced age of Bakkula; and by multiple textual indications. Bakkula apparently represents an ideal among a certain wing of the \textsanskrit{Saṅgha} at that time. His verses in the \textsanskrit{Theragāthā}, however, are simple and free from this severity (\href{https://suttacentral.net/thag3.3/en/sujato}{Thag 3.3}). } }At one time Venerable Bakkula was staying near \textsanskrit{Rājagaha}, in the Bamboo Grove, the squirrels’ feeding ground. 

Then\marginnote{2.1} the naked ascetic Kassapa, an old friend of Bakkula in the lay life, approached him, and exchanged greetings with him.\footnote{The phrase \textit{\textsanskrit{purāṇagihisahāya}} (“old friend from the lay life”) appears only here and at \href{https://suttacentral.net/sn41.9/en/sujato\#1.1}{SN 41.9:1.1}. There it is also said of a naked ascetic (\textit{acela}) called Kassapa, which cannot be a coincidence. In that case he, as the former friend of the householder Citta, is the one interrogated, in a discourse set while the Buddha was still alive. The events of the present discourse happened many decades after the Buddha’s death, so it seems likely Kassapa was introduced as a conventional foil for Bakkula. Several such individuals named Kassapa are found in Pali, and their stories are conflated (\href{https://suttacentral.net/dn8/en/sujato}{DN 8}, \href{https://suttacentral.net/sn12.17/en/sujato}{SN 12.17}, \href{https://suttacentral.net/sn41.9/en/sujato}{SN 41.9}). } When the greetings and polite conversation were over, he sat down to one side and said to Venerable Bakkula, “Reverend Bakkula, how long has it been since you went forth?” 

“It\marginnote{3.2} has been eighty years, reverend.”\footnote{One may “go forth” at fifteen or “when old enough to scare crows” (\href{https://suttacentral.net/pli-tv-kd1/en/sujato\#51.1.14}{Kd 1:51.1.14}), so he is probably over ninety here. } 

“But\marginnote{3.3} in these eighty years, how many times have you had sex?” 

“You\marginnote{3.4} shouldn’t ask me such a question. Rather, you should ask me this: ‘But in these eighty years, how many times have sensual perceptions ever arisen in you?’” 

“But\marginnote{3.8} in these eighty years, how many times have sensual perceptions ever arisen in you?”\footnote{This repetition is found in the PTS and BJT editions, but missing from MS and King of Siam editions. } 

“In\marginnote{3.9} these eighty years, I don’t recall that any sensual perception has ever arisen in me.” 

\textit{This\marginnote{3.10} we remember as an incredible and amazing quality of Venerable Bakkula.\footnote{According to the commentary, these responses were added by the Elders, representing praise of Bakkula’s austerities by the successful rigorist party at the Second Council. This is supported by several considerations. First, the commentators would rather claim something is early, so when they say it was added later it likely comes from a genuine tradition. Second, the refrains are in plural. Third, the refrains, and the statements of Bakkula that they answer, lack the close \textit{-ti} indicating a direct quote. Fourthly, they continue after the conversation with Kassapa has concluded. The refrains are also found in the Chinese parallel. This shows how comparing different versions of a sutta can take us back only to the point where the texts diverge, which was some time after the Second Council. Thus a shared feature might have been added a century or more after the Buddha. } }

“In\marginnote{4{-}5.1} these eighty years, I don’t recall that any perception of ill will … or cruelty has ever arisen in me.” 

\textit{This\marginnote{4{-}5.3} too we remember as an incredible and amazing quality of Venerable Bakkula. }

“In\marginnote{6.1} these eighty years, I don’t recall that any thought of sensuality … ill will … or cruelty has ever arisen in me.” 

\textit{This\marginnote{7{-}8.3} too we remember as an incredible and amazing quality of Venerable Bakkula. }

“In\marginnote{9.1} these eighty years, I don’t recall accepting a robe from a householder …\footnote{From a discussion of mental purity, Bakkula proceeds to list a series of details of conduct, which are not morally reprehensible, but which illustrate his extreme commitment to austerity. | Accepting a robe from a householder implies enjoying the spoils of wealthy donors. Austere monks wore robes made of cast-off rags. Compare the story of Ānanda at \href{https://suttacentral.net/mn52/en/sujato\#16.2}{MN 52:16.2} = \href{https://suttacentral.net/an11.16/en/sujato\#11.2}{AN 11.16:11.2}, or at \href{https://suttacentral.net/pli-tv-kd21/en/sujato\#1.13.1}{Kd 21:1.13.1}, where he explained that after accepting quality robes, they were shared and reused. | This sutta can be seen as representing one of the disputing factions at the Second Council (\href{https://suttacentral.net/pli-tv-kd22/en/sujato}{Kd 22}). The \textsanskrit{Saṅgha} was divided between “rigorists” who advocated a more austere monastic life, and “laxists” who wanted to take a more relaxed attitude to discipline. The Second Council decided ten points of Vinaya in the rigorists’ favor, the most important of which was whether monastics could use money. Bakkula represents a rigorist position so extreme that if it were followed generally it would undermine the very existence of the \textsanskrit{Saṅgha}. His severity, which even exceeds that of \textsanskrit{Mahākassapa}, is implicitly contrasted with the gentle and accommodating spirit of Ānanda. } cutting a robe with a knife …\footnote{The following items echo the Buddha’s concern at \href{https://suttacentral.net/mn122/en/sujato\#2.8}{MN 122:2.8}, where many monastics, including Ānanda, had gathered for making robes. | Note that most of these items are missing from the Chinese parallel at MA 34. } sewing a robe with a needle … dying a robe … sewing a robe during the robe-making ceremony … wandering to look for robe material for my companions in the spiritual life when they are making robes …\footnote{This case is discussed in \href{https://suttacentral.net/pli-tv-bu-vb-pc46/en/sujato}{Bu Pc 46}. It is normally forbidden to visit other families on the way to a meal invitation, but a special exemption is granted for the times of giving and making robes. } accepting an invitation …\footnote{Bakkula would eat only food collected on alms-round, refusing to eat a meal by invitation in a lay person’s home. } having such a thought: ‘If only someone would invite me!’ … sitting down inside a house …\footnote{This is allowed, but might sometimes get a monastic unwillingly involved in lay affairs (\href{https://suttacentral.net/pli-tv-bu-vb-pc43/en/sujato}{Bu Pc 43}). } eating inside a house … getting caught up in the details of female’s appearance …\footnote{This refers to the practice of sense restraint (\href{https://suttacentral.net/mn27/en/sujato\#15.1}{MN 27:15.1}). } teaching a female, even so much as a four line verse …\footnote{\href{https://suttacentral.net/pli-tv-bu-vb-pc7/en/sujato}{Bu Pc 7} forbids teaching up to five or six lines to someone of the opposite sex, unless there is a chaperone present. } going to the nuns’ quarters …\footnote{This is stricter than \href{https://suttacentral.net/pli-tv-bu-vb-pc23/en/sujato}{Bu Pc 23}, which forbids going to the nuns’s quarters for the purpose of giving the fortnightly instruction, with several exceptions noted. } teaching the nuns …\footnote{At \href{https://suttacentral.net/mn146/en/sujato\#3.3}{MN 146:3.3}, the Buddha encourages Nandaka to teach the nuns; indeed the Vinaya says that a monk should agree to teach the nuns (\href{https://suttacentral.net/pli-tv-kd20/en/sujato\#9.5.3}{Kd 20:9.5.3}), and in addition should go to the assistance of nuns (\href{https://suttacentral.net/pli-tv-kd3/en/sujato\#6.12.1}{Kd 3:6.12.1}). Compare \href{https://suttacentral.net/pli-tv-bu-vb-pc24/en/sujato}{Bu Pc 24}, which forbids teaching nuns for the sake of material profits. } teaching the trainee nuns … teaching the novice nuns … giving the going forth …\footnote{Presumably Bakkula wished to avoid the responsibility of taking on students. If this practice was universally adopted, the \textsanskrit{Saṅgha} could not survive. } giving the ordination … giving dependence … being looked after by a novice … bathing in the sauna …\footnote{Allowed for health at \href{https://suttacentral.net/pli-tv-kd15/en/sujato\#14.1.1}{Kd 15:14.1.1}. } bathing with bath powder …\footnote{“Bath powder” (\textit{\textsanskrit{cuṇṇa}}) was in regular use (\href{https://suttacentral.net/pli-tv-kd1/en/sujato\#25.12.4}{Kd 1:25.12.4}), but was sometimes considered sensual (\href{https://suttacentral.net/pli-tv-kd20/en/sujato\#27.4.1}{Kd 20:27.4.1}). } looking for a massage from my companions in the spiritual life …\footnote{Massaging was done for health and by respect (\href{https://suttacentral.net/pli-tv-kd12/en/sujato\#1.1.27}{Kd 12:1.1.27}), but was often problematic, as it could be sexually arousing (\href{https://suttacentral.net/pli-tv-bu-vb-ss1/en/sujato\#5.8.17}{Bu Ss 1:5.8.17}) or even dangerous (\href{https://suttacentral.net/pli-tv-bu-vb-pj3/en/sujato\#5.16.2}{Bu Pj 3:5.16.2}). At one point the Buddha forbade massages for monks (\href{https://suttacentral.net/pli-tv-kd15/en/sujato\#1.4.1}{Kd 15:1.4.1}) and nuns (\href{https://suttacentral.net/pli-tv-bi-vb-pc90/en/sujato}{Bi Pc 90}), although he allowed monks a “broad-handed” massage (\href{https://suttacentral.net/pli-tv-kd15/en/sujato\#1.5.6}{Kd 15:1.5.6}). } being ill, even for as long as it takes to pull a cow’s udder …\footnote{Bakkula was said to be the foremost monk in good health (\href{https://suttacentral.net/an1.226/en/sujato\#1.1}{AN 1.226:1.1}), an accolade that must have been added after his passing away. In later texts, he is often remembered for this (\href{https://suttacentral.net/tha-ap398/en/sujato}{Tha Ap 398}, \href{https://suttacentral.net/mil5.5.3/en/sujato}{Mil 5.5.3}; \textsanskrit{Mahāprajñāpāramitāśāstra} 8.2.3.3, \textsanskrit{Divyāvadāna} 253.020). It would seem he mentions it here because many of the other items are allowed for the sake of health. It may happen that something such as a massage, which has a genuine health purpose, can become an indulgence. } being presented with medicine, even as much as a bit of yellow myrobalan …\footnote{Allowed as medicine at \href{https://suttacentral.net/pli-tv-kd6/en/sujato\#6.1.4}{Kd 6:6.1.4}. } leaning on a leaning-board …\footnote{Allowed at \href{https://suttacentral.net/pli-tv-kd16/en/sujato\#20.2.15}{Kd 16:20.2.15}. This was a wooden plank that was leaned against the wall as a simple backrest. } preparing a cot …”\footnote{A simple bed was allowed (\href{https://suttacentral.net/pli-tv-bu-vb-pc16/en/sujato}{Bu Pc 16}), but Bakkula slept on the ground. } 

\textit{This\marginnote{30{-}36.8} too we remember as an incredible and amazing quality of Venerable Bakkula. }

“In\marginnote{37.1} these eighty years, I don’t recall commencing the rainy season residence within a village.” 

\textit{This\marginnote{37.2} too we remember as an incredible and amazing quality of Venerable Bakkula. }

“Reverend,\marginnote{38.1} for seven days I ate the nation’s almsfood as a debtor.\footnote{Said by \textsanskrit{Mahākassapa} at \href{https://suttacentral.net/sn16.11/en/sujato\#13.2}{SN 16.11:13.2}. The idea is that while still unenlightened, one is accruing “debt” by taking alms food offered in faith. But the arahant, as a “worthy one”, is cleared of all debts. } Then on the eighth day I became enlightened.” 

\textit{This\marginnote{38.3} too we remember as an incredible and amazing quality of Venerable Bakkula. }

“Reverend\marginnote{39.1} Bakkula, may I receive the going forth, the ordination in this teaching and training?” And the naked ascetic Kassapa received the going forth, the ordination in this teaching and training. 

Not\marginnote{39.3} long after his ordination, Venerable Kassapa, living alone, withdrawn, diligent, keen, and resolute, soon realized the supreme end of the spiritual path in this very life. He lived having achieved with his own insight the goal for which gentlemen rightly go forth from the lay life to homelessness. 

He\marginnote{39.4} understood: “Rebirth is ended; the spiritual journey has been completed; what had to be done has been done; there is nothing further for this place.” And Venerable Kassapa became one of the perfected. 

Then\marginnote{40.1} some time later Venerable Bakkula, taking a latchkey, went from dwelling to dwelling, saying,\footnote{The “latchkey” (\textit{\textsanskrit{avāpuraṇa}}) was a universal key inserted into outside doors to lift the bolt within (\href{https://suttacentral.net/sn22.90/en/sujato\#1.2}{SN 22.90:1.2}, \href{https://suttacentral.net/an9.11/en/sujato\#2.1}{AN 9.11:2.1}). He took it to access the inner dwellings in the monastery to make his announcement.  } “Come forth, venerables, come forth! Today will be my full extinguishment.” 

\textit{This\marginnote{40.3} too we remember as an incredible and amazing quality of Venerable Bakkula. }

And\marginnote{41.1} Venerable Bakkula became fully extinguished while seated right in the middle of the \textsanskrit{Saṅgha}. 

\textit{This\marginnote{41.2} too we remember as an incredible and amazing quality of Venerable Bakkula. }

%
\section*{{\suttatitleacronym MN 125}{\suttatitletranslation The Level of the Tamed }{\suttatitleroot Dantabhūmisutta}}
\addcontentsline{toc}{section}{\tocacronym{MN 125} \toctranslation{The Level of the Tamed } \tocroot{Dantabhūmisutta}}
\markboth{The Level of the Tamed }{Dantabhūmisutta}
\extramarks{MN 125}{MN 125}

\scevam{So\marginnote{1.1} I have heard.\footnote{This discourse uses the training of an elephant as an allegory for the Gradual Training. The Buddha contrasts the sensual lifestyle of Prince Jayasena, which clouds his vision of the truth, with the path of renunciation. Throughout, he lends support to the young novice Aciravata. } }At one time the Buddha was staying near \textsanskrit{Rājagaha}, in the Bamboo Grove, the squirrels’ feeding ground. 

Now\marginnote{2.1} at that time the novice Aciravata was staying in a wilderness hut. Then as Prince Jayasena was going for a walk he approached Aciravata, and exchanged greetings with him.\footnote{Jayasena features in this discourse and the next. The commentary says he was \textsanskrit{Bimbisāra}’s son. | Aciravata is otherwise unknown. } 

When\marginnote{2.3} the greetings and polite conversation were over, he sat down to one side and said to Aciravata, “Mister Aggivessana, I have heard that\footnote{Saccaka is also called Aggivessana (\href{https://suttacentral.net/mn35/en/sujato\#4.2}{MN 35:4.2}). } a mendicant who meditates diligently, keenly, and resolutely can experience unification of mind.” 

“That’s\marginnote{2.6} so true, Prince! That’s so true! A mendicant who meditates diligently, keenly, and resolutely can experience unification of mind.” 

“Mister\marginnote{3.1} Aggivessana, please teach me the Dhamma as you have learned and memorized it.” 

“I’m\marginnote{3.2} not competent to do so, Prince.\footnote{Aciravati’s lack of confidence is understandable. Novices were under twenty, so here we have a rather intimidating situation where a Prince is questioning a teenager. } For if I were to teach you the Dhamma as I have learned and memorized it, you might not understand the meaning, which would be wearying and troublesome for me.” 

“Mister\marginnote{4.1} Aggivessana, please teach me the Dhamma as you have learned and memorized it. Hopefully I will understand the meaning of what you say.” 

“I\marginnote{4.3} shall teach you. If you understand the meaning of what I say, that’s good. If not, then leave each to his own, and do not question me about it further.” 

“Mister\marginnote{4.6} Aggivessana, please teach me the Dhamma as you have learned and memorized it. If I understand the meaning of what you say, that’s good. If not, then I will leave each to his own, and not question you about it further.” 

Then\marginnote{5.1} the novice Aciravata taught Prince Jayasena the Dhamma as he had learned and memorized it. When he had spoken, Jayasena said to him, “It is impossible, Mister Aggivessana, it cannot happen that a mendicant who meditates diligently, keenly, and resolutely can experience unification of mind.” Having declared that this was impossible, Jayasena got up from his seat and left. 

Not\marginnote{6.1} long after he had left, Aciravata went to the Buddha, bowed, sat down to one side, and informed the Buddha of all they had discussed. 

When\marginnote{7.1} he had spoken, the Buddha said to him, 

“How\marginnote{7.2} could it possibly be otherwise, Aggivessana?\footnote{Knowing his student lacks confidence, the Buddha has encouraging words for him, not letting him blame himself for Jayasena’s rejection. } Prince Jayasena dwells in the midst of sensual pleasures, enjoying them, consumed by thoughts of them, burning with fever for them, and eagerly seeking more. It’s simply impossible for him to know or see or realize what can only be known, seen, and realized by renunciation.\footnote{A parallel passage occurs at \href{https://suttacentral.net/mn125/en/sujato\#7.3}{MN 125:7.3}, where the phrases “dwells in the midst of sensual pleasures, enjoying them” are reversed. The sequence there makes better sense: it starts with where they are, then what they enjoy, and gradually moves up to being consumed and seeking more. I take this as the correct sense and translate both passages accordingly. } 

Suppose\marginnote{8.1} there was a pair of elephants or horse or oxen in training who were well tamed and well trained. And there was a pair who were not tamed or trained. What do you think, Aggivessana? Wouldn’t the pair that was well tamed and well trained perform the tasks of the tamed and reach the level of the tamed?” 

“Yes,\marginnote{8.4} sir.” 

“But\marginnote{8.5} would the pair that was not tamed and trained perform the tasks of the tamed and reach the level of the tamed, just like the tamed pair?” 

“No,\marginnote{8.6} sir.” 

“In\marginnote{8.7} the same way, Prince Jayasena dwells in the midst of sensual pleasures, enjoying them, consumed by thoughts of them, burning with fever for them, and eagerly seeking more. It’s simply impossible for him to know or see or realize what can only be known, seen, and realized by renunciation. 

Suppose\marginnote{9.1} there was a big mountain not far from a town or village.\footnote{This simile is contextual, as \textsanskrit{Rājagaha} is indeed surrounded by mountains. } And two friends set out from that village or town, lending each other a hand up to the mountain. Once there, one friend would remain at the foot of the mountain, while the other would climb to the peak.\footnote{For \textit{\textsanskrit{hatthavilaṅghaka}} (“lending a hand up”) see \href{https://suttacentral.net/mn31/en/sujato\#9.7}{MN 31:9.7} = \href{https://suttacentral.net/mn128/en/sujato\#14.7}{MN 128:14.7}. } The one standing at the foot would say to the one at the peak, ‘My friend, what do you see, standing there at the peak?’ They’d reply, ‘Standing at the peak, I see delightful parks, woods, meadows, and lotus ponds!’ 

But\marginnote{9.7} the other would say, ‘It’s impossible, it cannot happen that, standing at the peak, you can see delightful parks, woods, meadows, and lotus ponds.’ So their friend would come down from the peak, take their friend by the arm, and make them climb to the peak. After giving them a while to catch their breath, they’d say, ‘My friend, what do you see, standing here at the peak?’ They’d reply, ‘Standing at the peak, I see delightful parks, woods, meadows, and lotus ponds!’ 

They’d\marginnote{9.13} say, ‘Just now I understood you to say: “It’s impossible, it cannot happen that, standing at the peak, you can see delightful parks, woods, meadows, and lotus ponds.” But now you say: “Standing at the peak, I see delightful parks, woods, meadows, and lotus ponds!”’ They’d say, ‘But my friend, it was because I was obstructed by this big mountain that I didn’t see what could be seen.’ 

But\marginnote{10.1} bigger than that is the mass of ignorance by which Prince Jayasena is veiled, shrouded, covered, and engulfed. Prince Jayasena dwells in the midst of sensual pleasures, enjoying them, consumed by thoughts of them, burning with fever for them, and eagerly seeking more. It is quite impossible for him to know or see or realize what can only be known, seen, and realized by renunciation. It would be no wonder if, had these two similes occurred to you, Prince Jayasena would have gained confidence in you and shown his confidence.” 

“But\marginnote{11.2} sir, how could these two similes have occurred to me as they did to the Buddha, since they were neither supernaturally inspired, nor learned before in the past?” 

“Suppose,\marginnote{12.1} Aggivessana, an anointed aristocratic king was to address his elephant tracker, ‘Please, my good elephant tracker, mount the royal bull elephant and enter the elephant wood. When you see a wild bull elephant, tether it by the neck to the royal elephant.’ ‘Yes, Your Majesty,’ replied the elephant tracker, and did as he was asked. The royal elephant leads the wild elephant out into the open; and it’s only then that it comes out into the open, for a wild bull elephant clings to the elephant wood. The elephant tracker informs the king, ‘Sire, the wild elephant has come out into the open.’ Then the king addresses his elephant trainer, ‘Please, my good elephant trainer, tame the wild bull elephant. Subdue its wild behaviors, its wild memories and thoughts, and its wild stress, weariness, and fever. Make it happy to be within a village, and instill behaviors congenial to humans.’ 

‘Yes,\marginnote{12.11} Your Majesty,’ replied the elephant trainer. He dug a large post into the earth and tethered the elephant to it by the neck, so as to subdue its wild behaviors, its wild memories and thoughts, and its wild stress, weariness, and fever, and to make it happy to be within a village, and instill behaviors congenial to humans. He spoke in a way that’s mellow, pleasing to the ear, lovely, going to the heart, polite, likable and agreeable to the people. Spoken to in such a way by the elephant trainer, the wild elephant wanted to listen. It actively listened and tried to understand. So the elephant trainer rewards it with grass, fodder, and water. 

When\marginnote{12.15} the wild elephant accepts the grass, fodder, and water, the trainer knows, ‘Now the wild elephant will survive!’ He sets it a further task: ‘Pick it up, sir! Put it down, sir!’ When the wild elephant picks up and puts down when the trainer says, following instructions, the trainer sets it a further task: ‘Forward, sir! Back, sir!’ When the wild elephant goes forward and back when the trainer says, following instructions, the trainer sets it a further task: ‘Stand, sir! Sit, sir!’ 

When\marginnote{12.23} the wild elephant stands and sits when the trainer says, following instructions, the trainer sets the task called imperturbability. He fastens a large plank to its trunk; a lancer sits on its neck; other lancers surround it on all sides; and the trainer himself stands in front with a long lance. While practicing this task, it doesn’t budge its fore-feet or hind-feet, its fore-quarters or hind-quarters, its head, ears, tusks, tail, or trunk. The wild bull elephant endures being struck by spears, swords, arrows, and axes; it endures the thunder of the drums, kettledrums, horns, and tom-toms. Rid of all crooks and flaws, and purged of defects, it is worthy of a king, fit to serve a king, and considered a factor of kingship. 

In\marginnote{13.1} the same way, Aggivessana, a Realized One arises in the world, perfected, a fully awakened Buddha, accomplished in knowledge and conduct, holy, knower of the world, supreme guide for those who wish to train, teacher of gods and humans, awakened, blessed.\footnote{The parallel at MA 198 is more straightforward, mentioning only purity of body, speech, and mind; undertaking mindfulness meditation; and entering the four absorptions. } He realizes with his own insight this world—with its gods, \textsanskrit{Māras}, and divinities, this population with its ascetics and brahmins, gods and humans—and he makes it known to others. He proclaims a teaching that is good in the beginning, good in the middle, and good in the end, meaningful and well-phrased. And he reveals a spiritual practice that’s entirely full and pure. 

A\marginnote{14.1} householder hears that teaching, or a householder’s child, or someone reborn in a good family. They gain faith in the Realized One and reflect, ‘Life at home is cramped and dirty, life gone forth is wide open. It’s not easy for someone living at home to lead the spiritual life utterly full and pure, like a polished shell. Why don’t I shave off my hair and beard, dress in ocher robes, and go forth from the lay life to homelessness?’ 

After\marginnote{14.7} some time they give up a large or small fortune, and a large or small family circle. They shave off hair and beard, dress in ocher robes, and go forth from the lay life to homelessness. And it’s only then that a noble disciple comes out into the open, for gods and humans cling to the five kinds of sensual stimulation. 

The\marginnote{15.1} Realized One guides them further: ‘Come, mendicant, be ethical and restrained in the monastic code, conducting yourself well and resorting for alms in suitable places. Seeing danger in the slightest fault, keep the rules you’ve undertaken.’ 

When\marginnote{16.1} they have ethical conduct, the Realized One guides them further: ‘Come, mendicant, guard your sense doors. When you see a sight with your eyes, don’t get caught up in the features and details. … 

\scexpansioninstructions{(Tell in full as in MN 107, the Discourse with \textsanskrit{Moggallāna} the Accountant.) }

They\marginnote{22.1} give up these five hindrances, corruptions of the heart that weaken wisdom. Then they meditate observing an aspect of the body—keen, aware, and mindful, rid of covetousness and displeasure for the world. 

They\marginnote{23.1} meditate observing an aspect of feelings … mind … principles—keen, aware, and mindful, rid of covetousness and displeasure for the world. It’s like when the elephant trainer dug a large post into the earth and tethered the elephant to it by the neck, so as to subdue its wild behaviors, its wild memories and thoughts, and its wild stress, weariness, and fever, and to make it happy to be within a village, and instill behaviors congenial to humans. In the same way, a noble disciple has these four kinds of mindfulness meditation as tethers for the mind so as to subdue behaviors tied to domestic life, memories and thoughts tied to domestic life, stress, weariness, and fever tied to domestic life, to discover the system, and to realize extinguishment. 

The\marginnote{24.1} Realized One guides them further: ‘Come, mendicant, meditate observing an aspect of the body, but don’t think thoughts connected with sensual pleasures.\footnote{Editions vary and it is not easy to decide between them. MS and BJT editions have \textit{\textsanskrit{kāmūpasaṁhita}} (“connected with sensual pleasures”), while PTS has \textit{\textsanskrit{kāyūpasaṁhita}} (“connected with the body”), followed by “connected with feelings, mind, and principles” respectively. The Chinese parallel has “sensuality” in first place, abbreviates the next two, then “against the dhamma” for the final \textit{\textsanskrit{satipaṭṭhāna}} (MA 198 at T i 758b15). Sanskrit texts \textsanskrit{Pañcaviṁśatisāhasrikā} \textsanskrit{Prajñāpāramitā} and \textsanskrit{Pratyutpannabuddhasaṁmukhāvasthitasamādhi}-\textsanskrit{sūtra} have \textit{\textsanskrit{kāya}}. So the balance of the non-Pali texts supports the reading \textit{\textsanskrit{kāyūpasaṁhita}}. On the other hand, \textit{\textsanskrit{kāmūpasaṁhita}} is a common term, while \textit{\textsanskrit{kāyūpasaṁhita}} (etc.) appear only here; but the passage is passed over in the commentary and subcommentary, which suggests that they thought it was self-evident, making it unlikely that unique terms were used. Finally, the overall topic of the discourse is overcoming sensuality through meditation, and it would seem odd for the Buddha to be teaching specialized meditation instructions, found nowhere else, to a novice. Overall, then, I accept the reading \textit{\textsanskrit{kāmūpasaṁhita}}; see too the next note. } Meditate observing an aspect of feelings … mind … principles, but don’t think thoughts connected with sensual pleasures.’ 

As\marginnote{25.1} the placing of the mind and keeping it connected are stilled, they enter and remain in the second absorption …\footnote{The Pali version, in a unique presentation, has the four \textit{\textsanskrit{satipaṭṭhānas}} in place of the first absorption, which offers further light on the problem discussed in the previous note. The first absorption is characterized by seclusion from sensual pleasures, while \textit{vitakka} is still present. Clearly one is not “thinking of sensual pleasures” at this point, so the reading \textit{\textsanskrit{kāmūpasaṁhita}} is unproblematic. But it is not clear that one is not having \textit{vitakka} for the body (and feeling, mind, and principles). For example, in mindfulness of breathing one is applying the mind or “thinking of” the breath, in other words “thinking of the body”. Thus the reading \textit{\textsanskrit{kāyūpasaṁhita}} would be difficult to fully explain. Any conclusions on this passage, however, are tenuous, and it seems likely there has been some textual corruption. Indeed, the Chinese parallel here has all four absorptions as usual. } third absorption … fourth absorption. 

When\marginnote{26.1} their mind has become immersed in \textsanskrit{samādhi} like this—purified, bright, flawless, rid of corruptions, pliable, workable, steady, and imperturbable—they extend it toward recollection of past lives. They recollect many kinds of past lives. That is: one, two, three, four, five, ten, twenty, thirty, forty, fifty, a hundred, a thousand, a hundred thousand rebirths; many eons of the world contracting, many eons of the world expanding, many eons of the world contracting and expanding. And so they recollect their many kinds of past lives, with features and details. 

When\marginnote{27{-}28.1} their mind has become immersed in \textsanskrit{samādhi} like this—purified, bright, flawless, rid of corruptions, pliable, workable, steady, and imperturbable—they extend it toward knowledge of the death and rebirth of sentient beings. With clairvoyance that is purified and superhuman, they see sentient beings passing away and being reborn—inferior and superior, beautiful and ugly, in a good place or a bad place. They understand how sentient beings are reborn according to their deeds. 

When\marginnote{29.1} their mind has become immersed in \textsanskrit{samādhi} like this—purified, bright, flawless, rid of corruptions, pliable, workable, steady, and imperturbable—they extend it toward knowledge of the ending of defilements. They truly understand: ‘This is suffering’ … ‘This is the origin of suffering’ … ‘This is the cessation of suffering’ … ‘This is the practice that leads to the cessation of suffering’. They truly understand: ‘These are defilements’ … ‘This is the origin of defilements’ … ‘This is the cessation of defilements’ … ‘This is the practice that leads to the cessation of defilements’. Knowing and seeing like this, their mind is freed from the defilements of sensuality, desire to be reborn, and ignorance. When they’re freed, they know they’re freed. 

They\marginnote{29.6} understand: ‘Rebirth is ended, the spiritual journey has been completed, what had to be done has been done, there is nothing further for this place.’ 

Such\marginnote{30.1} a mendicant endures cold, heat, hunger, and thirst; the touch of flies, mosquitoes, wind, sun, and reptiles; rude and unwelcome criticism; and puts up with physical pain—sharp, severe, acute, unpleasant, disagreeable, and life-threatening. Rid of all greed, hate, and delusion, and purged of defects, they are worthy of offerings dedicated to the gods, worthy of hospitality, worthy of a religious donation, worthy of greeting with joined palms, and are the supreme field of merit for the world. 

If\marginnote{31.1} a royal bull elephant passes away untamed and untrained—whether in their old age, middle age, or youth—they’re considered a royal bull elephant who passed away untamed. In the same way, if a mendicant passes away without having ended the defilements—whether as a senior, middle, or junior—they’re considered as a mendicant who passed away untamed. 

If\marginnote{32.1} a royal bull elephant passes away tamed and trained—whether in their old age, middle age, or youth—they’re considered a royal bull elephant who passed away tamed. In the same way, if a mendicant passes away having ended the defilements—whether as a senior, middle, or junior—they’re considered as a mendicant who passed away tamed.”\footnote{The Buddha is giving gentle encouragement to Aciravata, reminding him that his junior status does not limit his spiritual growth. } 

That\marginnote{32.7} is what the Buddha said. Satisfied, the novice Aciravata approved what the Buddha said. 

%
\section*{{\suttatitleacronym MN 126}{\suttatitletranslation With Bhūmija }{\suttatitleroot Bhūmijasutta}}
\addcontentsline{toc}{section}{\tocacronym{MN 126} \toctranslation{With Bhūmija } \tocroot{Bhūmijasutta}}
\markboth{With Bhūmija }{Bhūmijasutta}
\extramarks{MN 126}{MN 126}

\scevam{So\marginnote{1.1} I have heard. }At one time the Buddha was staying near \textsanskrit{Rājagaha}, in the Bamboo Grove, the squirrels’ feeding ground. 

Then\marginnote{2.1} Venerable \textsanskrit{Bhūmija} robed up in the morning and, taking his bowl and robe, went to the home of Prince Jayasena, where he sat on the seat spread out.\footnote{We know \textsanskrit{Bhūmija} from \href{https://suttacentral.net/sn12.25/en/sujato}{SN 12.25}, where he asks \textsanskrit{Sāriputta} about the efficacy of deeds. | This is the second discourse featuring Jayasena (after \href{https://suttacentral.net/mn125/en/sujato}{MN 125}), but despite his evident interest in the Dhamma, and his support for mendicants expressed by offering the meal here, we do not hear of his conversion or spiritual development. } 

Then\marginnote{3.1} Jayasena approached and exchanged greetings with him. When the greetings and polite conversation were over, he sat down to one side and said to \textsanskrit{Bhūmija}: 

“Mister\marginnote{3.3} \textsanskrit{Bhūmija}, there are some ascetics and brahmins who have this doctrine and view: ‘If you make a wish and lead the spiritual life, you can’t win the fruit.\footnote{“Wish” is \textit{\textsanskrit{āsa}}. } If you don’t make a wish and lead the spiritual life, you can’t win the fruit. If you both make a wish and don’t make a wish and lead the spiritual life, you can’t win the fruit. If you neither make a wish nor don’t make a wish and lead the spiritual life, you can’t win the fruit.’ What does Mister \textsanskrit{Bhūmija}’s Teacher say about this? How does he explain it?” 

“Prince,\marginnote{4.1} I haven’t heard and learned this in the presence of the Buddha.\footnote{Like Aciravata in \href{https://suttacentral.net/mn125/en/sujato\#3.2}{MN 125:3.2}, he is cautious in responding. } But it’s possible that he might explain it like this: ‘If you lead the spiritual life irrationally, you can’t win the fruit, regardless of whether you make a wish,\footnote{“Irrationally” (\textit{ayoniso}) here means, as the sutta will make clear, that the practices one does are not “rationally”(\textit{yoniso})  or causally connected to the results one expects. This sutta illuminates the meaning of the key term \textit{yoniso} through a series of similes. } you don’t make a wish, you both do and do not make a wish, or you neither do nor don’t make a wish. But if you lead the spiritual life rationally, you can win the fruit, regardless of whether you make a wish, you don’t make a wish, you both do and do not make a wish, or you neither do nor don’t make a wish.’ I haven’t heard and learned this in the presence of the Buddha. But it’s possible that he might explain it like that.” 

“If\marginnote{5.1} that’s what your teacher says, Mister \textsanskrit{Bhūmija}, he clearly stands head and shoulders above all the various other ascetics and brahmins.” Then Prince Jayasena served Venerable \textsanskrit{Bhūmija} from his own dish. 

Then\marginnote{7.1} after the meal, on his return from almsround, \textsanskrit{Bhūmija} went to the Buddha, bowed, sat down to one side, and told him all that had happened, adding: “Answering this way, I trust that I repeated what the Buddha has said, and didn’t misrepresent him with an untruth. I trust my explanation was in line with the teaching, and that there are no legitimate grounds for rebuttal or criticism.” 

“Indeed,\marginnote{8.1} \textsanskrit{Bhūmija}, in answering this way you repeated what I’ve said, and didn’t misrepresent me with an untruth. Your explanation was in line with the teaching, and there are no legitimate grounds for rebuttal or criticism. 

There\marginnote{9.1} are some ascetics and brahmins who have wrong view, wrong thought, wrong speech, wrong action, wrong livelihood, wrong effort, wrong mindfulness, and wrong immersion. If they lead the spiritual life, they can’t win the fruit, regardless of whether they make a wish, they don’t make a wish, they both do and do not make a wish, or they neither do nor don’t make a wish. Why is that? Because that’s an irrational way to win the fruit. 

Suppose\marginnote{10.1} there was a person in need of oil. While wandering in search of oil, they tried heaping sand in a bucket, sprinkling it thoroughly with water, and pressing it out. But by doing this, they couldn’t extract any oil, regardless of whether they made a wish, didn’t make a wish, both did and did not make a wish, or neither did nor did not make a wish. Why is that? Because that’s an irrational way to extract oil. 

And\marginnote{10.8} so it is for any ascetics and brahmins who have wrong view, wrong thought, wrong speech, wrong action, wrong livelihood, wrong effort, wrong mindfulness, and wrong immersion. If they lead the spiritual life, they can’t win the fruit, regardless of whether or not they make a wish. Why is that? Because that’s an irrational way to win the fruit. 

Suppose\marginnote{11.1} there was a person in need of milk. While wandering in search of milk, they tried pulling the horn of a newly-calved cow. But by doing this, they couldn’t get any milk, regardless of whether they made a wish, didn’t make a wish, both did and did not make a wish, or neither did nor did not make a wish. Why is that? Because that’s an irrational way to get milk. 

And\marginnote{11.8} so it is for any ascetics and brahmins who have wrong view … Because that’s an irrational way to win the fruit. 

Suppose\marginnote{12.1} there was a person in need of butter. While wandering in search of butter, they tried pouring water into a pot and churning it with a stick. But by doing this, they couldn’t produce any butter, regardless of whether they made a wish, didn’t make a wish, both did and did not make a wish, or neither did nor did not make a wish. Why is that? Because that’s an irrational way to produce butter. 

And\marginnote{12.8} so it is for any ascetics and brahmins who have wrong view … Because that’s an irrational way to win the fruit. 

Suppose\marginnote{13.1} there was a person in need of fire. While wandering in search of fire, they tried drilling a green, sappy log with a drill-stick. But by doing this, they couldn’t start a fire, regardless of whether they made a wish, didn’t make a wish, both did and did not make a wish, or neither did nor did not make a wish. Why is that? Because that’s an irrational way to start a fire. 

And\marginnote{13.8} so it is for any ascetics and brahmins who have wrong view … Because that’s an irrational way to win the fruit. 

There\marginnote{14.1} are some ascetics and brahmins who have right view, right thought, right speech, right action, right livelihood, right effort, right mindfulness, and right immersion. If they lead the spiritual life, they can win the fruit, regardless of whether they make a wish, they don’t make a wish, they both do and do not make a wish, or they neither do nor do not make a wish. Why is that? Because that’s a rational way to win the fruit. 

Suppose\marginnote{15.1} there was a person in need of oil. While wandering in search of oil, they tried heaping sesame flour in a bucket, sprinkling it thoroughly with water, and pressing it out. By doing this, they could extract oil, regardless of whether they made a wish, didn’t make a wish, both did and did not make a wish, or neither did nor did not make a wish. Why is that? Because that’s a rational way to extract oil. 

And\marginnote{15.8} so it is for any ascetics and brahmins who have right view … Because that’s a rational way to win the fruit. 

Suppose\marginnote{16.1} there was a person in need of milk. While wandering in search of milk, they tried pulling the udder of a newly-calved cow. By doing this, they could get milk, regardless of whether they made a wish, didn’t make a wish, both did and did not make a wish, or neither did nor did not make a wish. Why is that? Because that’s a rational way to get milk. 

And\marginnote{16.8} so it is for any ascetics and brahmins who have right view … Because that’s a rational way to win the fruit. 

Suppose\marginnote{17.1} there was a person in need of butter. While wandering in search of butter, they tried pouring curds into a pot and churning them with a stick. By doing this, they could produce butter, regardless of whether they made a wish, didn’t make a wish, both did and did not make a wish, or neither did nor did not make a wish. Why is that? Because that’s a rational way to produce butter. 

And\marginnote{17.8} so it is for any ascetics and brahmins who have right view … Because that’s a rational way to win the fruit. 

Suppose\marginnote{18.1} there was a person in need of fire. While wandering in search of fire, they tried drilling a dried up, withered log with a drill-stick. By doing this, they could start a fire, regardless of whether they made a wish, didn’t make a wish, both did and did not make a wish, or neither did nor did not make a wish. Why is that? Because that’s a rational way to start a fire. 

And\marginnote{18.8} so it is for any ascetics and brahmins who have right view … Because that’s a rational way to win the fruit. 

\textsanskrit{Bhūmija},\marginnote{19.1} it would be no wonder if, had these four similes occurred to you, Prince Jayasena would have gained confidence in you and shown his confidence.”\footnote{Above, Jayasena expresses his confidence in the Buddha, but not in \textsanskrit{Bhūmija} (\href{https://suttacentral.net/mn126/en/sujato\#5.1}{MN 126:5.1}). } 

“But\marginnote{19.2} sir, how could these four similes have occurred to me as they did to the Buddha, since they were neither supernaturally inspired, nor learned before in the past?”\footnote{As at \href{https://suttacentral.net/mn125/en/sujato\#11.2}{MN 125:11.2}. } 

That\marginnote{19.3} is what the Buddha said. Satisfied, Venerable \textsanskrit{Bhūmija} approved what the Buddha said. 

%
\section*{{\suttatitleacronym MN 127}{\suttatitletranslation With Anuruddha }{\suttatitleroot Anuruddhasutta}}
\addcontentsline{toc}{section}{\tocacronym{MN 127} \toctranslation{With Anuruddha } \tocroot{Anuruddhasutta}}
\markboth{With Anuruddha }{Anuruddhasutta}
\extramarks{MN 127}{MN 127}

\scevam{So\marginnote{1.1} I have heard. }At one time the Buddha was staying near \textsanskrit{Sāvatthī} in Jeta’s Grove, \textsanskrit{Anāthapiṇḍika}’s monastery. 

And\marginnote{2.1} then the chamberlain \textsanskrit{Pañcakaṅga} addressed a man,\footnote{For \textsanskrit{Pañcakaṅga} and his role as chamberlain, see note on \href{https://suttacentral.net/mn59/en/sujato\#1.3}{MN 59:1.3}. } “Please, mister, go to Venerable Anuruddha, and in my name bow with your head to his feet. Say to him, ‘Sir, the chamberlain \textsanskrit{Pañcakaṅga} bows with his head to your feet.’ And then ask him whether he might please accept tomorrow’s meal from \textsanskrit{Pañcakaṅga} together with the mendicant \textsanskrit{Saṅgha}. And ask whether he might please come earlier than usual, for \textsanskrit{Pañcakaṅga} has many duties, and much work to do for the king.” 

“Yes,\marginnote{2.8} sir,” that man replied. He did as \textsanskrit{Pañcakaṅga} asked, and Venerable Anuruddha consented with silence. 

Then\marginnote{3.1} when the night had passed, Anuruddha robed up in the morning and, taking his bowl and robe, went to \textsanskrit{Pañcakaṅga}’s home, where he sat on the seat spread out. Then \textsanskrit{Pañcakaṅga} served and satisfied Anuruddha with his own hands with delicious fresh and cooked foods. When Anuruddha had eaten and washed his hands and bowl, \textsanskrit{Pañcakaṅga} took a low seat, sat to one side, and said to him: 

“Sir,\marginnote{4.1} some senior mendicants have come to me and said, ‘Householder, develop the limitless release of heart.’ Others have said, ‘Householder, develop the expansive release of heart.’ Now, the limitless release of the heart and the expansive release of the heart: do these things differ in both meaning and phrasing? Or do they mean the same thing, and differ only in the phrasing?”\footnote{At \href{https://suttacentral.net/mn43/en/sujato\#29.5}{MN 43:29.5}, \textsanskrit{Mahākoṭṭhita} asks \textsanskrit{Sāriputta} a similar question about different meditative liberations. } 

“Well\marginnote{5.1} then, householder, let me know what you think about this. Afterwards you’ll get it without fail.”\footnote{“Without fail” renders \textit{\textsanskrit{apaṇṇaka}}, for which see note on \href{https://suttacentral.net/mn60/en/sujato\#4.3}{MN 60:4.3}. It’s an unusual response. As \textit{\textsanskrit{apaṇṇaka}} means a “set without a fifth”, Anuruddha might be punning on \textsanskrit{Pañcakaṅga}’s name, which means, “a set of five items”. } 

“Sir,\marginnote{5.2} this is what I think. The limitless release of the heart and the expansive release of the heart mean the same thing, and differ only in the phrasing.” 

“The\marginnote{6.1} limitless release of the heart and the expansive release of the heart differ in both meaning and phrasing. This is a way to understand how these things differ in both meaning and phrasing. 

And\marginnote{7.1} what is the limitless release of the heart?\footnote{“Limitless” is \textit{\textsanskrit{appamāṇa}}. This set of four meditations is known in later traditions as either the “divine meditations” (\textit{\textsanskrit{brahmavihārā}}) or the “immeasurables” (\textit{\textsanskrit{appamaññā}}). } It’s when a mendicant meditates spreading a heart full of love to one direction, and to the second, and to the third, and to the fourth. In the same way above, below, across, everywhere, all around, they spread a heart full of love to the whole world—abundant, expansive, limitless, free of enmity and ill will. They meditate spreading a heart full of compassion … They meditate spreading a heart full of rejoicing … They meditate spreading a heart full of equanimity to one direction, and to the second, and to the third, and to the fourth. In the same way above, below, across, everywhere, all around, they spread a heart full of equanimity to the whole world—abundant, expansive, limitless, free of enmity and ill will. This is called the limitless release of the heart. 

And\marginnote{8.1} what is the expansive release of the heart?\footnote{“Expansive” renders \textit{mahaggata}, literally “grown great”. This features as a state of mind that is developed in \textit{\textsanskrit{satipaṭṭhāna}} (\href{https://suttacentral.net/mn10/en/sujato\#34.10}{MN 10:34.10}), and which is known with psychic powers (\href{https://suttacentral.net/mn6/en/sujato\#16.10}{MN 6:16.10}). It also appears just above next to “limitless” in the description of the divine meditations, a detail that perhaps prompted \textsanskrit{Pañcakaṅga}’s question. } It’s when a mendicant meditates focused on pervading the extent of a single tree root as expansive.\footnote{This meditation practice is unique to this passage. It might be compared with the practice of emptiness at \href{https://suttacentral.net/mn121/en/sujato\#4.1}{MN 121:4.1}, which likewise moves from a perception of one’s limited surroundings towards something more universal. The meaning of “pervading” (\textit{\textsanskrit{pharitvā}}) might be understood in the same sense as it is found in the divine meditations, where I have translated it as “spread”. The sutta below, however, treats it as the spreading of light. In this interpretation, the “expansive” practice is a method of spreading love (etc.) by gradually expanding the perception of the field of love. The final extent is the land bordered by ocean, beyond which the mind become “limitless like the great ocean” (\href{https://suttacentral.net/mn72/en/sujato\#20.2}{MN 72:20.2}). In other words, the “expansive” liberation leads to the “limitless” liberation. } This is called the expansive release of the heart. Also, a mendicant meditates focused on pervading the extent of two or three tree roots … a single village district … two or three village districts … a single kingdom … two or three kingdoms … this land surrounded by ocean. This too is called the expansive release of the heart. This is a way to understand how these things differ in both meaning and phrasing. 

Householder,\marginnote{9.1} there are these four kinds of rebirth in a future life.\footnote{“Rebirth in a future life” renders \textit{\textsanskrit{bhavūpapatti}}. This section explains how the meditation one develops can shape one’s rebirth. } What four? Take someone who meditates focused on pervading ‘limited radiance’. When their body breaks up, after death, they’re reborn in the company of the gods of limited radiance.\footnote{The gods of limited radiance are mentioned at \href{https://suttacentral.net/mn41/en/sujato\#18-42.22}{MN 41:18–42.22}, where they are reborn due to a limited practice of the second absorption. } Next, take someone who meditates focused on pervading ‘limitless radiance’. When their body breaks up, after death, they’re reborn in the company of the gods of limitless radiance. Next, take someone who meditates focused on pervading ‘corrupted radiance’.\footnote{Normally the suttas maintain a clear distinction between the “corrupted” (\textit{\textsanskrit{saṅkiliṭṭha}}) mind that is purified before attaining the radiance of absorption. This passage, however, suggests that things might not always be so clear-cut. } When their body breaks up, after death, they’re reborn in the company of the gods of corrupted radiance. Next, take someone who meditates focused on pervading ‘pure radiance’. When their body breaks up, after death, they’re reborn in the company of the gods of pure radiance. These are the four kinds of rebirth in a future life. 

There\marginnote{10.1} comes a time, householder, when the deities gather together as one. When they do so, a difference in their color is evident, but not in their radiance. It’s like when a person brings several oil lamps into one house. You can detect a difference in their flames, but not in their radiance. In the same way, when the deities gather together as one, a difference in their color is evident, but not in their radiance. 

There\marginnote{11.1} comes a time when those deities go their separate ways. When they do so, a difference both in their color and also in their radiance is evident. It’s like when a person takes those several oil lamps out of that house. You can detect a difference both in their flames and also in their radiance. In the same way, when the deities go their separate ways, a difference both in their color and also in their radiance is evident. 

It’s\marginnote{12.1} not that those deities think, ‘What we have is permanent, lasting, and eternal.’ Rather, wherever those deities cling, that’s where they take pleasure. It’s like when flies are being carried along on a carrying-pole or basket. It’s not that they think, ‘What we have is permanent, lasting, and eternal.’ Rather, wherever those flies cling, that’s where they take pleasure. In the same way, it’s not that those deities think, ‘What we have is permanent, lasting, and eternal.’ Rather, wherever those deities cling, that’s where they take pleasure.”\footnote{This example shows that “craving for rebirth” need not be conscious. } 

When\marginnote{13.1} he had spoken, Venerable Sabhiya \textsanskrit{Kaccāna} said to Venerable Anuruddha:\footnote{We meet Sabhiya \textsanskrit{Kaccāna} also at \href{https://suttacentral.net/sn44.1/en/sujato}{SN 44.1}, where he is questioned by Vacchagotta. His verses are at \href{https://suttacentral.net/thag4.3/en/sujato}{Thag 4.3}. The conversion of a wanderer named Sabhiya is recorded in \href{https://suttacentral.net/snp3.6/en/sujato}{Snp 3.6}; this might be the same person. } 

“Good,\marginnote{13.2} Honorable Anuruddha! I have a further question about this. Do all the radiant deities have limited radiance, or do some there have limitless radiance?” 

“In\marginnote{13.5} that respect, Reverend \textsanskrit{Kaccāna}, some deities there have limited radiance, while some have limitless radiance.”\footnote{For \textit{\textsanskrit{tadaṅgena}} in the sense “in that respect”, see \href{https://suttacentral.net/an9.33/en/sujato\#1.3}{AN 9.33:1.3}. } 

“What\marginnote{14.1} is the cause, Honorable Anuruddha, what is the reason why, when those deities have been reborn in a single order of gods, some deities there have limited radiance, while some have limitless radiance?” 

“Well\marginnote{14.2} then, Reverend \textsanskrit{Kaccāna}, I’ll ask you about this in return, and you can answer as you like. What do you think, Reverend \textsanskrit{Kaccāna}? Which of these two kinds of mental development is more expansive: when a mendicant meditates focused on pervading as expansive the extent of a single tree root, or two or three tree roots?” 

“When\marginnote{14.6} a mendicant meditates on two or three tree roots.” 

“What\marginnote{14.8} do you think, Reverend \textsanskrit{Kaccāna}? Which of these two kinds of mental development is more expansive: when a mendicant meditates focused on pervading as expansive the extent of two or three tree roots, or a single village district … two or three village districts … a single kingdom … two or three kingdoms … this land surrounded by ocean?” 

“When\marginnote{14.27} a mendicant meditates on this land surrounded by ocean.” 

“This\marginnote{14.29} is the cause, Reverend \textsanskrit{Kaccāna}, this is the reason why, when those deities have been reborn in a single order of gods, some deities there have limited radiance, while some have limitless radiance.” 

“Good,\marginnote{15.1} Honorable Anuruddha! I have a further question about this. Do all the radiant deities have corrupted radiance, or do some there have pure radiance?” 

“In\marginnote{15.4} that respect, Reverend \textsanskrit{Kaccāna}, some deities there have corrupted radiance, while some have pure radiance.” 

“What\marginnote{16.1} is the cause, Venerable Anuruddha, what is the reason why, when those deities have been reborn in a single order of gods, some deities there have corrupted radiance, while some have pure radiance?” 

“Well\marginnote{16.2} then, Reverend \textsanskrit{Kaccāna}, I shall give you a simile. For by means of a simile some sensible people understand the meaning of what is said. Suppose an oil lamp was burning with impure oil and impure wick. Because of the impurity of the oil and the wick it burns dimly, as it were. 

In\marginnote{16.6} the same way, take some mendicant who meditates focused on pervading ‘corrupted radiance’. Their physical discomfort is not completely settled, their dullness and drowsiness is not completely eradicated, and their restlessness and remorse are not completely eliminated. Because of this they practice absorption dimly, as it were. When their body breaks up, after death, they’re reborn in the company of the gods of corrupted radiance. 

Suppose\marginnote{16.10} an oil lamp was burning with pure oil and pure wick. Because of the purity of the oil and the wick it doesn’t burn dimly, as it were. 

In\marginnote{16.12} the same way, take some mendicant who meditates focused on pervading ‘pure radiance’. Their physical discomfort is completely settled, their dullness and drowsiness is completely eradicated, and their restlessness and remorse is completely eliminated. Because of this they don’t practice absorption dimly, as it were. When their body breaks up, after death, they’re reborn in the company of the gods of pure radiance. 

“This\marginnote{16.16} is the cause, Reverend \textsanskrit{Kaccāna}, this is the reason why, when those deities have been reborn in a single order of gods, some deities there have corrupted radiance, while some have pure radiance.” 

When\marginnote{17.1} he had spoken, Venerable Sabhiya \textsanskrit{Kaccāna} said to Venerable Anuruddha, “Good, Honorable Anuruddha! 

Venerable\marginnote{17.3} Anuruddha, you don’t say, ‘So I have heard’ or ‘It ought to be like this.’\footnote{The phrase cited by Sabhiya here, “so I have heard” (\textit{\textsanskrit{evaṁ} me \textsanskrit{sutaṁ}}), is the standard opening for Buddhist suttas. This passage illustrates that this tag was used in order to indicate that the speaker was not present at the events, but “heard” about them. This is in contrast with the phrase “I heard and learned this in the presence” (\textit{\textsanskrit{sammukhā} \textsanskrit{sutaṁ}, \textsanskrit{sammukhā} \textsanskrit{paṭiggahitaṁ}}), which is used when reporting a teaching heard directly from the Buddha, eg. \href{https://suttacentral.net/sn55.52/en/sujato\#5.1}{SN 55.52:5.1}, \href{https://suttacentral.net/sn22.90/en/sujato\#9.1}{SN 22.90:9.1}, \href{https://suttacentral.net/mn47/en/sujato\#10.7}{MN 47:10.7}, etc. } Rather, you say: ‘These deities are like this, those deities are like that.’ Sir, it occurs to me, ‘Clearly, Venerable Anuruddha has previously lived together with those deities, conversed, and engaged in discussion.’” 

“Your\marginnote{17.8} words are clearly invasive and intrusive, Reverend \textsanskrit{Kaccāna}.\footnote{Sabhiya is bluntly asking about Anuruddha’s personal meditation attainments. It is a minor offence to declare these to lay people, but they may be discussed among monastics (\href{https://suttacentral.net/pli-tv-bu-vb-pc8/en/sujato}{Bu Pc 8}). In this case, Anuruddha is speaking to a monastic while at least one lay person can overhear, a situation not addressed in the Vinaya. } Nevertheless, I will answer you. For a long time I have previously lived together with those deities, conversed, and engaged in discussion.” 

When\marginnote{18.1} he had spoken, Venerable Sabhiya \textsanskrit{Kaccāna} said to \textsanskrit{Pañcakaṅga} the chamberlain, “You’re fortunate, householder, so very fortunate, to have given up your state of uncertainty, and to have got the chance to listen to this exposition of the teaching.” 

%
\section*{{\suttatitleacronym MN 128}{\suttatitletranslation Corruptions }{\suttatitleroot Upakkilesasutta}}
\addcontentsline{toc}{section}{\tocacronym{MN 128} \toctranslation{Corruptions } \tocroot{Upakkilesasutta}}
\markboth{Corruptions }{Upakkilesasutta}
\extramarks{MN 128}{MN 128}

\scevam{So\marginnote{1.1} I have heard.\footnote{This discourse offers a unique insight into the Buddha’s own meditative development. The “corruptions” of the title are the subtle mental flaws that undermine meditation when it is on the brink of absorption. | This discourse and \href{https://suttacentral.net/mn31/en/sujato}{MN 31} appear to be influenced by \textsanskrit{Bṛhadāraṇyaka} \textsanskrit{Upaniṣad} 4.3, a famous dialogue where an unusually reluctant \textsanskrit{Yājñavalkya} is repeatedly pressed by King Janaka to reveal the true nature of a person’s light. } }At one time the Buddha was staying near \textsanskrit{Kosambī}, in Ghosita’s Monastery. 

Now\marginnote{2.1} at that time the mendicants of \textsanskrit{Kosambī} were arguing, quarreling, and disputing, continually wounding each other with barbed words.\footnote{This is in reference to the notorious quarrel at \textsanskrit{Kosambī}, also referred to at \href{https://suttacentral.net/mn48/en/sujato\#2.1}{MN 48:2.1}. The events are detailed in the Vinaya at \href{https://suttacentral.net/pli-tv-kd10/en/sujato}{Kd 10}. } 

Then\marginnote{3.1} a mendicant went up to the Buddha, bowed, stood to one side, and told him what was happening, adding: “Please, sir go to those mendicants out of sympathy.” The Buddha consented with silence.\footnote{While both here and at \href{https://suttacentral.net/mn48/en/sujato}{MN 48} we see the Buddha intervening to settle the conflict, here he goes to them, while there he summons them to him; and the ensuing course of the sutta is quite different. } 

Then\marginnote{4.1} the Buddha went up to those mendicants and said, “Enough, mendicants! Stop arguing, quarreling, and disputing.” 

When\marginnote{4.3} he said this, one of the mendicants said to the Buddha, “Wait, sir! Let the Buddha, the Lord of the Dhamma, remain passive, dwelling in blissful meditation in this life. We will be known for this arguing, quarreling, and disputing.” 

For\marginnote{4.8} a second time … and a third time the Buddha said to those mendicants, “Enough, mendicants! Stop arguing, quarreling, and disputing.” 

For\marginnote{4.17} a third time that mendicant said to the Buddha, “Wait, sir! Let the Buddha, the Lord of the Dhamma, remain passive, dwelling in blissful meditation in this life. We will be known for this arguing, quarreling, and disputing.” 

Then\marginnote{5.1} the Buddha robed up in the morning and, taking his bowl and robe, entered \textsanskrit{Kosambī} for alms. After the meal, on his return from almsround, he set his lodgings in order. Taking his bowl and robe, he recited these verses while standing right there: 

\begin{verse}%
“Many\marginnote{6.1} voices shout at once,\footnote{These verses are also found in the Vinaya account at \href{https://suttacentral.net/pli-tv-kd10/en/sujato\#3.1.3}{Kd 10:3.1.3} and the Kosambiya \textsanskrit{Jātaka} at \href{https://suttacentral.net/ja428/en/sujato\#1.1}{Ja 428:1.1}. } \\
yet no-one thinks that they’re the fool. \\
Even as the \textsanskrit{Saṅgha} splits, \\
they think no better of the other. 

Dolts\marginnote{6.5} pretending to be astute, \\
they talk, their words out of bounds.\footnote{Read \textit{agocara} (“out of bounds”). } \\
They blab at will, their mouths agape, \\
and not one knows what leads them on. 

‘They\marginnote{6.9} abused me, they hit me!\footnote{These four verses are shared with \href{https://suttacentral.net/dhp3/en/sujato}{Dhp 3}–6. } \\
They beat me, they robbed me!’ \\
For those bearing such a grudge, \\
hatred is never laid to rest. 

‘They\marginnote{6.13} abused me, they hit me! \\
They beat me, they robbed me!’ \\
For those who bear no such grudge, \\
hatred is laid to rest. 

For\marginnote{6.17} never is hatred \\
laid to rest by hate, \\
it’s laid to rest by love: \\
this is an ancient teaching. 

When\marginnote{6.21} others do not understand,\footnote{This verse is found in the same context at \href{https://suttacentral.net/dhp6/en/sujato}{Dhp 6}, and in different contexts in \href{https://suttacentral.net/thag4.3/en/sujato\#1.1}{Thag 4.3:1.1} and \href{https://suttacentral.net/thag8.1/en/sujato\#5.1}{Thag 8.1:5.1}. } \\
let us, who do understand this,\footnote{\textit{\textsanskrit{Yamāmese}} is reflexive first person plural imperative of \textit{yamati}, “let us restrain ourselves”, cf. \href{https://suttacentral.net/sn10.6/en/sujato\#3.1}{SN 10.6:3.1}: \textit{\textsanskrit{pāṇesu} ca \textsanskrit{saṁyamāmase}} (“Let us restrain ourselves regarding living creatures”). } \\
restrain ourselves in this regard; \\
for that is how conflicts are laid to rest. 

Breakers\marginnote{6.25} of bones and takers of life, \\
thieves of cattle, horses, wealth, \\
those who plunder the nation: \\
even they can come together, \\
so why can’t you? 

If\marginnote{6.30} you find an alert companion, \\
an attentive friend to live happily together, \\
then, overcoming all adversities, \\
wander with them, joyful and mindful. 

If\marginnote{6.34} you find no alert companion, \\
no attentive friend to live happily together, \\
then, like a king who flees his conquered realm, \\
wander alone like a tusker in the wilds. 

It’s\marginnote{6.38} better to wander alone, \\
there’s no fellowship with fools. \\
Wander alone and do no wrong, \\
at ease like a tusker in the wilds.” 

%
\end{verse}

After\marginnote{7.1} speaking these verses while standing, the Buddha went to the village of the child salt-miners,\footnote{“Child salt-miners” is \textit{\textsanskrit{bālakaloṇakāra}} (cf. \textit{\textsanskrit{loṇakāradārako}} at \href{https://suttacentral.net/an4.188/en/sujato\#5.3}{AN 4.188:5.3}). Children are used to mine salt in India to this day. } where Venerable Bhagu was staying at the time.\footnote{Bhagu was a leading Sakyan who went forth along with Bhaddiya, Anuruddha, Ānanda, Kimbila, Devadatta, and \textsanskrit{Upāli} (\href{https://suttacentral.net/pli-tv-kd17/en/sujato\#1.4.1}{Kd 17:1.4.1}). He is known from his ardent striving described in his verses at \href{https://suttacentral.net/thag4.2/en/sujato}{Thag 4.2}, and is listed with other renowned mendicants at \href{https://suttacentral.net/mn68/en/sujato\#2.2}{MN 68:2.2}. | After the conflict at \textsanskrit{Kosambī}, the Buddha sought the company of his fellow Sakyans; first Bhagu, then the three Sakyans below. | Another Bhagu was one of the ancient Vedic seers, while yet another was a contemporary of \textsanskrit{Sāṇavāsī} around the time of the Second Council (\href{https://suttacentral.net/pli-tv-kd8/en/sujato\#24.6.6}{Kd 8:24.6.6}). } Bhagu saw the Buddha coming off in the distance, so he spread out a seat and placed water for washing the feet. The Buddha sat on the seat spread out, and washed his feet. Bhagu bowed to the Buddha and sat down to one side. 

The\marginnote{7.8} Buddha said to him, “I hope you’re keeping well, mendicant; I hope you’re all right. And I hope you’re having no trouble getting almsfood.” 

“I’m\marginnote{7.10} keeping well, Blessed One; I’m all right. And I’m having no trouble getting almsfood.” 

Then\marginnote{7.11} the Buddha educated, encouraged, fired up, and inspired Bhagu with a Dhamma talk, after which he got up from his seat and set out for the Eastern Bamboo Park. 

Now\marginnote{8.1} at that time the venerables Anuruddha, Nandiya, and Kimbila were staying in the Eastern Bamboo Park.\footnote{The Eastern Bamboo Park was in Ceti, not far from \textsanskrit{Kosambī}. | Apart from the setting, the following passage is identical with \href{https://suttacentral.net/mn31/en/sujato\#2.1}{MN 31:2.1}ff., down as far as the discussion on meditation; see that sutta for notes. The current sutta is the original narrative context (see note on \href{https://suttacentral.net/mn31/en/sujato\#6.1}{MN 31:6.1}). } The park keeper saw the Buddha coming off in the distance and said to the Buddha, “Don’t come into this park, ascetic.\footnote{MS and BJT read \textit{\textsanskrit{mahāsamaṇa}} here, whereas PTS and parallel passages at \href{https://suttacentral.net/mn31/en/sujato\#3.4}{MN 31:3.4} and \href{https://suttacentral.net/pli-tv-kd10/en/sujato\#4.2.3}{Kd 10:4.2.3} have simply \textit{\textsanskrit{samaṇa}}. \textit{\textsanskrit{Mahāsamaṇa}} (“great ascetic”) is a rare epithet of the Buddha found mainly in the Vinaya. It is unlikely the park keeper would have kept him out knowing who he really was. } There are three gentlemen staying here whose nature is to desire only the self. Don’t disturb them.” 

Anuruddha\marginnote{9.1} heard the park keeper conversing with the Buddha, and said to him, “Don’t keep the Buddha out, good park keeper! Our Teacher, the Blessed One, has arrived.” 

Then\marginnote{10.1} Anuruddha went to Nandiya and Kimbila, and said to them, “Come forth, venerables, come forth! Our Teacher, the Blessed One, has arrived!” 

Then\marginnote{10.3} Anuruddha, Nandiya, and Kimbila came out to greet the Buddha. One received his bowl and robe, one spread out a seat, and one set out water for washing his feet. The Buddha sat on the seat spread out and washed his feet. Those venerables bowed and sat down to one side. 

The\marginnote{10.8} Buddha said to Anuruddha, “I hope you’re keeping well, Anuruddha and friends; I hope you’re all right. And I hope you’re having no trouble getting almsfood.” 

“We’re\marginnote{10.10} keeping well, Blessed One; we’re all right. And we’re having no trouble getting almsfood.” 

“I\marginnote{11.1} hope you’re living in harmony, appreciating each other, without quarreling, blending like milk and water, and regarding each other with kindly eyes?” 

“Indeed,\marginnote{11.2} sir, we live in harmony as you say.” 

“But\marginnote{11.3} how do you live this way?” 

“In\marginnote{12.1} this case, sir, I think: ‘I’m fortunate, so very fortunate, to live together with spiritual companions such as these.’ I consistently treat these venerables with kindness by way of body, speech, and mind, both in public and in private. I think: ‘Why don’t I set aside my own ideas and just go along with these venerables’ ideas?’ And that’s what I do. Though we’re different in body, sir, we’re one in mind, it seems to me.” 

And\marginnote{12.11} the venerables Nandiya and Kimbila spoke likewise, and they added: “That’s how we live in harmony, appreciating each other, without quarreling, blending like milk and water, and regarding each other with kindly eyes.” 

“Good,\marginnote{13.1} good, Anuruddha and friends! But I hope you’re living diligently, keen, and resolute?” 

“Indeed,\marginnote{14.1} sir, we live diligently.” 

“But\marginnote{14.2} how do you live this way?” 

“In\marginnote{14.3} this case, sir, whoever returns first from almsround prepares the seats, and puts out the drinking water and the rubbish bin. If there’s anything left over, whoever returns last eats it if they like. Otherwise they throw it out where there is little that grows, or drop it into water that has no living creatures. Then they put away the seats, drinking water, and rubbish bin, and sweep the refectory. If someone sees that the pot of water for washing, drinking, or the toilet is empty they set it up. If he can’t do it, he summons another with a wave of the hand, and they set it up by lending each other a hand to lift. But we don’t break into speech for that reason.\footnote{For \textit{\textsanskrit{hatthavilaṅghaka}} (“lending a hand up”) see \href{https://suttacentral.net/mn125/en/sujato\#9.2}{MN 125:9.2}. } And every five days we sit together for the whole night and discuss the teachings. That’s how we live diligently, keen, and resolute.” 

“Good,\marginnote{15.1} good, Anuruddha and friends! But as you live diligently like this, have you achieved any superhuman distinction in knowledge and vision worthy of the noble ones, a comfortable meditation?”\footnote{As at \href{https://suttacentral.net/mn31/en/sujato\#10.3}{MN 31:10.3}ff., Anuruddha must be prodded to reveal the full extent of his meditation prowess. The form of both suttas parallels \textsanskrit{Yājñavalkya}’s reticence to answer Janaka. In addition, the theme also echoes the \textsanskrit{Upaniṣad}, for Janaka asks what is a person’s light (\textit{\textsanskrit{kiṁjyotirayaṁ} \textsanskrit{puruṣa}}, \textsanskrit{Bṛhadāraṇyaka} \textsanskrit{Upaniṣad} 4.3.2). \textsanskrit{Yājñavalkya} answers that it is simply the sun. When pressed further, he admits it might be the moon, then fire, or even speech, for you can make it home in the dark when someone is calling. All these things—sun, moon, fire, speech—are worshipped in the Vedic tradition as external manifestations of divinity. He only reluctantly admits that a person’s true light is their Self, made of consciousness, the light in the heart. This is the same Self referred to above in the phrase, “whose nature is to desire only the self”. | The Vedantic commentator \textsanskrit{Śaṅkara}, on a verse of the same chapter (4.3.7), discusses at length Buddhist objections to his doctrine. Thus the Buddhist text looks back to the \textsanskrit{Upaniṣad}, while the commentary to that same \textsanskrit{Upaniṣad} looks to Buddhism. } 

“Well,\marginnote{15.3} sir, while meditating diligent, keen, and resolute, we perceive both light and vision of forms.\footnote{The key terms here are “perceive” (\textit{\textsanskrit{sañjānāma}}), “light” (\textit{\textsanskrit{obhāsa}}), and “vision of forms” (\textit{\textsanskrit{dassanañca} \textsanskrit{rūpānaṁ}}). Variations of these are common terms for the visions seen in meditation (eg. \href{https://suttacentral.net/dn9/en/sujato\#14.2}{DN 9:14.2}, \href{https://suttacentral.net/mn77/en/sujato\#22.2}{MN 77:22.2}, \href{https://suttacentral.net/mn121/en/sujato\#5.4}{MN 121:5.4}). The “light” is the radiance that indicates freedom from defilements, while the “forms” may be the various phenomena that one in such a meditation might see. In particular this refers to the power of clairvoyance (\textit{dibbacakkhu}), in which Anuruddha was later declared foremost (\href{https://suttacentral.net/an1.192/en/sujato\#1.1}{AN 1.192:1.1}). See the discussion in the previous sutta (\href{https://suttacentral.net/mn127/en/sujato\#10.1}{MN 127:10.1}) and also eg. \href{https://suttacentral.net/mn32/en/sujato\#6.6}{MN 32:6.6}, \href{https://suttacentral.net/sn9.6/en/sujato\#1.2}{SN 9.6:1.2} \href{https://suttacentral.net/an3.130/en/sujato\#1.3}{AN 3.130:1.3}, and \href{https://suttacentral.net/an8.46/en/sujato\#1.3}{AN 8.46:1.3}. } But before long the light and the vision of forms vanish. We haven’t worked out the basis of that.”\footnote{Below, the Buddha investigates by way of “cause and reason” (\href{https://suttacentral.net/mn128/en/sujato\#16.5}{MN 128:16.5}), hence the commentary explains \textit{nimitta} here as “basis” (\textit{\textsanskrit{kāraṇa}}; for similar senses see \href{https://suttacentral.net/mn12/en/sujato\#23.1}{MN 12:23.1}, \href{https://suttacentral.net/sn48.40/en/sujato\#1.6}{SN 48.40:1.6}, \href{https://suttacentral.net/an3.102/en/sujato\#1.2}{AN 3.102:1.2}). | Confusingly, in the previous sentence we have “light and vision of forms”, but this is not \textit{nimitta}, which does not have this meaning in early texts. For related contexts, see \href{https://suttacentral.net/dn11/en/sujato\#80.12}{DN 11:80.12}, \href{https://suttacentral.net/dn33/en/sujato\#1.9.27}{DN 33:1.9.27}, \href{https://suttacentral.net/mn5/en/sujato\#6.1}{MN 5:6.1}, \href{https://suttacentral.net/mn20/en/sujato\#2.1}{MN 20:2.1}, \href{https://suttacentral.net/mn36/en/sujato\#45.6}{MN 36:45.6}, \href{https://suttacentral.net/mn44/en/sujato\#12.3}{MN 44:12.3}, \href{https://suttacentral.net/an3.19/en/sujato\#2.3}{AN 3.19:2.3}, \href{https://suttacentral.net/an4.14/en/sujato\#4.2}{AN 4.14:4.2}, \href{https://suttacentral.net/an6.27/en/sujato\#7.1}{AN 6.27:7.1}, and \href{https://suttacentral.net/an9.35/en/sujato\#1.11}{AN 9.35:1.11}. | Anuruddha learned his lesson, for at \href{https://suttacentral.net/mn68/en/sujato\#6.2}{MN 68:6.2} he explains how hindrances prevent absorption. } 

“Well,\marginnote{16.1} you should work out the basis of that. Before my awakening—when I was still unawakened but intent on awakening—I too perceived light and vision of forms. But before long my light and vision of forms vanished.\footnote{This refers to a stage in the development of meditation when immersion is not properly consolidated. It is a delicate phase, where familiar hindrances appear in subtle ways, which can nonetheless be an obstacle for absorption. Below I give my understanding of these meditation issues and how to deal with them. } It occurred to me: ‘What’s the cause, what’s the reason why my light and vision of forms vanish?’ It occurred to me: ‘Doubt arose in me, and because of that my immersion fell away.\footnote{When immersion is not consolidated, even the slightest uncertainty as to how to proceed can disturb the meditation. Doubt can also arise regarding whether the visions seen in meditation are real. This can be counteracted by determining one’s meditation at the beginning. | These three sentences use grammatical tense very precisely: an event is recalled (in past tense); a timeless principle is inferred (in present tense); and a resolution is made (in future tense). } When immersion falls away, the light and vision of forms vanish. I’ll make sure that doubt will not arise in me again.’ 

While\marginnote{17.1} meditating diligent, keen, and resolute, I perceived light and vision of forms. But before long my light and vision of forms vanished. It occurred to me: ‘What’s the cause, what’s the reason why my light and vision of forms vanish?’ It occurred to me: ‘Loss of focus arose in me, and because of that my immersion fell away.\footnote{The mind drifts from the meditation. At this subtle level, the mind may not be wandering or lost, but merely focusing imprecisely. } When immersion falls away, the light and vision of forms vanish. I’ll make sure that neither doubt nor loss of focus will arise in me again.’ 

While\marginnote{18.1} meditating … ‘Dullness and drowsiness arose in me …\footnote{Practicing to this level takes time, and any meditator will experience drowsiness at some point, even if it is merely a lack of fully clear energy. It is counteracted by rapture (\textit{\textsanskrit{pīti}}), which uplifts and inspires the mind. } I’ll make sure that neither doubt nor loss of focus nor dullness and drowsiness will arise in me again.’ 

While\marginnote{19.1} meditating … ‘Terror arose in me, and because of that my immersion fell away. When immersion falls away, the light and vision of forms vanish. Suppose a person was traveling along a road, and killers were to spring out at them from both sides. They’d feel terrified because of that.\footnote{While “terror” might feel like an overly-strong rendering for \textit{chambhitatta} in this context, this example shows what it means. At this point the mind is amplified, so that even small reactions can quickly get out of hand. The Buddha explains more about this point in his meditation at \href{https://suttacentral.net/mn4/en/sujato\#20.1}{MN 4:20.1}. } In the same way, terror arose in me … I’ll make sure that neither doubt nor loss of focus nor dullness and drowsiness nor terror will arise in me again.’ 

While\marginnote{20.1} meditating … ‘Elation arose in me, and because of that my immersion fell away.\footnote{This can happen when the mind is starting to coalesce, and one is excited that absorption is near. To address this, a meditator might reflect on the importance of patience, and focus more on developing tranquility in the fundamental stages of meditation. } When immersion falls away, the light and vision of forms vanish. Suppose a person was looking for an entrance to a hidden treasure. And all at once they’d come across five entrances! They’d feel excited because of that.\footnote{For “hidden treasure” in meditation, compare \href{https://suttacentral.net/mn52/en/sujato}{MN 52} = \href{https://suttacentral.net/an11.16/en/sujato}{AN 11.16}. } In the same way, elation arose in me … I’ll make sure that neither doubt nor loss of focus nor dullness and drowsiness nor terror nor elation will arise in me again.’ 

While\marginnote{21.1} meditating … ‘Discomfort arose in me …\footnote{All meditators experience discomfort when sitting for long periods of time. To deal with this, a meditator should first get as comfortable as they reasonably can, then develop pleasure in the body. Anuruddha overcame pain in the body through the four kinds of mindfulness meditation (\href{https://suttacentral.net/sn52.10/en/sujato\#2.1}{SN 52.10:2.1}). } I’ll make sure that neither doubt nor loss of focus nor dullness and drowsiness nor terror nor elation nor discomfort will arise in me again.’ 

While\marginnote{22.1} meditating … ‘Excessive energy arose in me, and because of that my immersion fell away.\footnote{Pushing too hard at the meditation, the body and mind can become over-wired with energy. To find balance, a meditator needs to relax and draw back from the meditation just a little. } When immersion falls away, the light and vision of forms vanish. Suppose a person was to grip a quail too tightly in their hands—it would die right there. I’ll make sure that neither doubt nor loss of focus nor dullness and drowsiness nor terror nor elation nor discomfort nor excessive energy will arise in me again.’ 

While\marginnote{23.1} meditating … ‘Overly lax energy arose in me, and because of that my immersion fell away.\footnote{The opposite of the previous, here the mind subtly falls back from the meditation too far. They are not exactly drowsy, but the mind lacks the keenness to go deeper. In response, a meditator needs to apply the mind a little more strongly to the meditation. } When immersion falls away, the light and vision of forms vanish. Suppose a person was to grip a quail too loosely—it would fly out of their hands. I’ll make sure that neither doubt nor loss of focus nor dullness and drowsiness nor terror nor elation nor discomfort nor excessive energy nor overly lax energy will arise in me again.’ 

While\marginnote{24.1} meditating … ‘Longing arose in me …\footnote{The meditator yearns for higher spiritual realizations, as at \href{https://suttacentral.net/thag19.1/en/sujato\#16.1}{Thag 19.1:16.1} or \href{https://suttacentral.net/an4.159/en/sujato\#6.7}{AN 4.159:6.7}. This is a wholesome motivation, but at this point it can disturb the meditation. } I’ll make sure that neither doubt nor loss of focus nor dullness and drowsiness nor terror nor elation nor discomfort nor excessive energy nor overly lax energy nor longing will arise in me again.’ 

While\marginnote{25.1} meditating … ‘Perceptions of diversity arose in me …\footnote{Diversity can arise through the activity of the senses, or through the variety of “forms” seen by the meditator. Either way, the mind must be settled to find true oneness. } I’ll make sure that neither doubt nor loss of focus nor dullness and drowsiness nor terror nor elation nor discomfort nor excessive energy nor overly lax energy nor longing nor perception of diversity will arise in me again.’ 

While\marginnote{26.1} meditating diligent, keen, and resolute, I perceived light and vision of forms. But before long my light and vision of forms vanished. It occurred to me: ‘What’s the cause, what’s the reason why my light and vision of forms vanish?’ It occurred to me: ‘Excessive concentration on forms arose in me, and because of that my immersion fell away.\footnote{The “forms” seen in meditation can be enticing and fascinating, appearing as glimpses of other realms, impressions of the presence of beings, or images of past lives. If the meditator pays too much attention to them, they lose focus on the meditation itself. } When immersion falls away, the light and vision of forms vanish. I’ll make sure that neither doubt nor loss of focus nor dullness and drowsiness nor terror nor elation nor discomfort nor excessive energy nor overly lax energy nor longing nor perception of diversity nor excessive concentration on forms will arise in me again.’ 

When\marginnote{27.1} I understood that doubt is a corruption of the mind, I gave it up. When I understood that loss of focus, dullness and drowsiness, terror, elation, discomfort, excessive energy, overly lax energy, longing, perception of diversity, and excessive concentration on forms are corruptions of the mind, I gave them up. 

While\marginnote{28.1} meditating diligent, keen, and resolute, I perceived light but did not see forms, or I saw forms, but did not see light. And this went on for a whole night, a whole day, even a whole night and day. I thought: ‘What is the cause, what is the reason for this?’ It occurred to me: ‘When I don’t focus on the basis of the forms, but focus on the basis of the light, then I perceive light and do not see forms.\footnote{This shows how \textit{nimitta} has a causal sense. \textit{\textsanskrit{Manasikāra}} (“focus”) is typically used of paying attention to causes. } But when I don’t focus on the basis of the light, but focus on the basis of the forms, then I see forms and do not perceive light. And this goes on for a whole night, a whole day, even a whole night and day.’ 

While\marginnote{29.1} meditating diligent, keen, and resolute, I perceived limited light and saw limited forms, or I perceived limitless light and saw limitless forms. And this went on for a whole night, a whole day, even a whole night and day. I thought: ‘What is the cause, what is the reason for this?’ It occurred to me: ‘When my immersion is limited, then my vision is limited, and with limited vision I perceive limited light and see limited forms. But when my immersion is limitless, then my vision is limitless, and with limitless vision I perceive limitless light and see limitless forms. And this goes on for a whole night, a whole day, even a whole night and day.’ 

After\marginnote{30.1} understanding that doubt, loss of focus, dullness and drowsiness, terror, excitement, discomfort, excessive energy, overly lax energy, longing, perception of diversity, and excessive concentration on forms are corruptions of the mind, I had given them up. 

I\marginnote{31.1} thought: ‘I’ve given up my mental corruptions. Now let me develop immersion in three ways.’ I developed immersion while placing the mind and keeping it connected; without placing the mind, merely keeping it connected; without placing the mind or keeping it connected; with rapture; without rapture; with pleasure; with equanimity.\footnote{This threefold presentation of the process of absorption focuses on \textit{vitakka} (“placing the mind”) and \textit{\textsanskrit{vicāra}} (“keeping it connected”), looking closely at how they cease (see also \href{https://suttacentral.net/dn33/en/sujato\#1.10.121}{DN 33:1.10.121}, \href{https://suttacentral.net/dn34/en/sujato\#1.4.7}{DN 34:1.4.7}, \href{https://suttacentral.net/sn43.3/en/sujato\#1.2}{SN 43.3:1.2}, and \href{https://suttacentral.net/an8.63/en/sujato\#3.1}{AN 8.63:3.1}). The standard \textit{\textsanskrit{jhāna}} formula focuses more on the refinement of feelings, from which perspective the first two stages of immersion here fall under the “rapture and bliss born of seclusion”, while only the third qualifies as “rapture and bliss born of immersion”. Thus Analayo describes the two descriptions as “complementary perspectives on the same process of deepening concentration” (\emph{Comparative Study}, vol. ii, pg. 739, note 263). } 

When\marginnote{32.1} I had developed immersion in these ways, the knowledge and vision arose in me: ‘My freedom is unshakable; this is my last rebirth; now there’ll be no more future lives.’” 

That\marginnote{32.4} is what the Buddha said. Satisfied, Venerable Anuruddha approved what the Buddha said. 

%
\section*{{\suttatitleacronym MN 129}{\suttatitletranslation The Foolish and the Astute }{\suttatitleroot Bālapaṇḍitasutta}}
\addcontentsline{toc}{section}{\tocacronym{MN 129} \toctranslation{The Foolish and the Astute } \tocroot{Bālapaṇḍitasutta}}
\markboth{The Foolish and the Astute }{Bālapaṇḍitasutta}
\extramarks{MN 129}{MN 129}

\scevam{So\marginnote{1.1} I have heard.\footnote{This discourse is a fundamental statement on the role of kamma. It identifies a fool with one who acts badly, and describes the suffering that awaits them in this life and the next. An astute person who acts well, by contrast, will experience countless joys. } }At one time the Buddha was staying near \textsanskrit{Sāvatthī} in Jeta’s Grove, \textsanskrit{Anāthapiṇḍika}’s monastery. There the Buddha addressed the mendicants, “Mendicants!” 

“Venerable\marginnote{1.5} sir,” they replied. The Buddha said this: 

“These\marginnote{2.1} are the three characteristics, signs, and manifestations of a fool.\footnote{“Fool” is \textit{\textsanskrit{bāla}}, originally “infant”. | A much simplified exposition of this theme is found at \href{https://suttacentral.net/an3.3/en/sujato}{AN 3.3}. } What three? A fool thinks poorly, speaks poorly, and acts poorly. If a fool didn’t think poorly, speak poorly, and act poorly, then how would the astute know of them, ‘This fellow is a fool, an untrue person’? But since a fool does think poorly, speak poorly, and act poorly, then the astute do know of them, ‘This fellow is a fool, an untrue person’. 

A\marginnote{3.1} fool experiences three kinds of suffering and sadness in this very life. 

Suppose\marginnote{3.2} a fool is sitting in a council hall, a street, or a crossroad, where people are discussing what is proper and fitting. And suppose that fool is someone who kills living creatures, steals, commits sexual misconduct, lies, and consumes beer, wine, and liquor intoxicants. Then that fool thinks, ‘These people are discussing what is proper and fitting. But those bad things are found in me and I exhibit them!’ This is the first kind of suffering and sadness that a fool experiences in this very life. 

Furthermore,\marginnote{4.1} a fool sees that the kings have arrested a bandit, a criminal, and subjected them to various punishments—whipping, caning, and clubbing; cutting off hands or feet, or both; cutting off ears or nose, or both; the ‘porridge pot’, the ‘shell-shave’, the ‘\textsanskrit{Rāhu}’s mouth’, the ‘garland of fire’, the ‘burning hand’, the ‘bulrush twist’, the ‘bark dress’, the ‘antelope’, the ‘meat hook’, the ‘coins’, the ‘caustic pickle’, the ‘twisting bar’, the ‘straw mat’; being splashed with hot oil, being fed to the dogs, being impaled alive, and being beheaded.\footnote{See note on \href{https://suttacentral.net/mn13/en/sujato\#14.3}{MN 13:14.3} for explanations of these. } Then that fool thinks, ‘The kinds of deeds for which the kings inflict such punishments—those things are found in me and I exhibit them! If the kings find out about me, they will inflict the same kinds of punishments on me!’ This is the second kind of suffering and sadness that a fool experiences in this very life. 

Furthermore,\marginnote{5.1} when a fool is resting on a chair or a bed or on the ground, their past bad deeds—misconduct of body, speech, and mind—settle down upon them, rest down upon them, and lay down upon them. It is like the shadow of a great mountain peak in the evening as it settles down, rests down, and lays down upon the earth. In the same way, when a fool is resting on a chair or a bed or on the ground, their past bad deeds—misconduct of body, speech, and mind—settle down upon them, rest down upon them, and lay down upon them. Then that fool thinks, ‘Alas, I haven’t done good and skillful things that keep me safe. And I have done bad things, violence and sin. When I depart, I’ll go to the place where people who’ve done such things go.’ They sorrow and wail and lament, beating their breasts and falling into confusion. This is the third kind of suffering and sadness that a fool experiences in this very life. 

Having\marginnote{6.1} done bad things by way of body, speech, and mind, when their body breaks up, after death, they’re reborn in a place of loss, a bad place, the underworld, hell. 

And\marginnote{7.1} if there’s anything of which it may be rightly said that it is utterly unlikable, undesirable, and disagreeable, it is of hell that this should be said.\footnote{“Hell” is \textit{niraya}, “bereft of good things”. } So much so that it’s not easy to give a simile for how painful hell is.” 

When\marginnote{7.5} he said this, one of the mendicants asked the Buddha, “But sir, is it possible to give a simile?” 

“It’s\marginnote{8.1} possible,” said the Buddha. 

“Suppose\marginnote{8.2} they arrest a bandit, a criminal and present him to the king, saying,\footnote{At \href{https://suttacentral.net/sn12.63/en/sujato\#6.5}{SN 12.63:6.5} this simile is applied to consciousness as a fuel or nutriment. } ‘Your Majesty, this is a bandit, a criminal. Punish him as you will.’ The king would say, ‘Go, my men, and strike this man in the morning with a hundred spears!’ The king’s men did as they were told. Then at midday the king would say, ‘My men, how is that man?’ ‘He’s still alive, Your Majesty.’ The king would say, ‘Go, my men, and strike this man in the midday with a hundred spears!’ The king’s men did as they were told. Then late in the afternoon the king would say, ‘My men, how is that man?’ ‘He’s still alive, Your Majesty.’ The king would say, ‘Go, my men, and strike this man in the late afternoon with a hundred spears!’ The king’s men did as they were told. 

What\marginnote{8.19} do you think, mendicants? Would that man experience pain and distress from being struck with three hundred spears?” 

“Sir,\marginnote{8.21} that man would experience pain and distress from being struck with one spear, let alone three hundred spears!” 

Then\marginnote{9.1} the Buddha, picking up a stone the size of his palm, addressed the mendicants,\footnote{Related similes are collected in \href{https://suttacentral.net/sn13.1/en/sujato}{SN 13.1}ff. and \href{https://suttacentral.net/sn56.49/en/sujato}{SN 56.49}ff., but this particular one is unique. } “What do you think, mendicants? Which is bigger: the stone the size of my palm that I’ve picked up, or the Himalayas, the king of mountains?” 

“Sir,\marginnote{9.4} the stone you’ve picked up is tiny. Compared to the Himalayas, it doesn’t count, it’s not worth a fraction, there’s no comparison.” 

“In\marginnote{9.5} the same way, compared to the suffering in hell, the pain and distress experienced by that man due to being struck with three hundred spears doesn’t count, it’s not worth a fraction, there’s no comparison. 

The\marginnote{10.1} wardens of hell punish them with the five-fold crucifixion.\footnote{The Buddhist traditions were uncomfortable with the idea that the “wardens of hell” (\textit{\textsanskrit{nirayapāla}}) are sentient beings, since this would imply that they are not merely working off the results of their own bad deeds, but actively creating new bad kamma by tormenting others. The commentary cites “certain elders” who argue that there are no wardens of hell because kamma works “like a machine” (\textit{\textsanskrit{yantarūpaṁ} viya}). These elders are probably the \textsanskrit{Mahāsaṅghikas} of Andhra who, according to the commentary on \href{https://suttacentral.net/kv20.3/en/sujato}{Kv 20.3}, say the torments are inflicted by the deeds themselves in the form of wardens of hell. The Theravadin commentator responds by analogy, since there are those who inflict punishment in the human realm. The situation was comparable in northern Indian Buddhism, for Vasubandhu at \textsanskrit{Abhidharmakośabhāṣya} 3.59a–c cites a number of opinions on both sides. In his \textsanskrit{Viṁśikā} vv. 4–7 and commentary he makes his own position clear, arguing that the wardens are mental projections, not sentient beings. } They drive red-hot stakes through the hands and feet, and another in the middle of the chest. And there they feel painful, sharp, severe, acute feelings—but they don’t die until that bad deed is eliminated. 

The\marginnote{11.1} wardens of hell throw them down and hack them with axes. … 

They\marginnote{12.1} hang them upside-down and hack them with hatchets. … 

They\marginnote{13.1} harness them to a chariot, and drive them back and forth across burning ground, blazing and glowing. … 

They\marginnote{14.1} make them climb up and down a huge mountain of burning coals, blazing and glowing. … 

The\marginnote{15.1} wardens of hell turn them upside down and throw them into a red-hot copper pot, burning, blazing, and glowing. There they’re seared in boiling scum, and they’re swept up and down and round and round. And there they feel painful, sharp, severe, acute feelings—but they don’t die until that bad deed is eliminated. 

The\marginnote{16.1} wardens of hell toss them in the Great Hell.\footnote{Identified by the commentary with \textsanskrit{Avīci} hell; cf. the reference to \textsanskrit{Avīci} with its four doors at \href{https://suttacentral.net/iti89/en/sujato\#5.3}{Iti 89:5.3}. } Now, about that Great Hell:\footnote{Here the text appears to quote a saying. } 

\begin{verse}%
‘Four\marginnote{16.3} are its corners, four its doors, \\
neatly divided in equal parts. \\
Surrounded by an iron wall, \\
of iron is its roof. 

The\marginnote{16.7} ground is even made of iron, \\
it burns with fierce fire. \\
The heat forever radiates \\
a hundred leagues around.’ 

%
\end{verse}

I\marginnote{17.1} could tell you many different things about hell.\footnote{From the “five-fold crucifixion” up to here is also found at \href{https://suttacentral.net/an3.36/en/sujato\#13.2}{AN 3.36:13.2}. In \href{https://suttacentral.net/mn130/en/sujato\#10.1}{MN 130:10.1}ff., the Buddha makes good on his claim here, and extends the description of the hells. See there for notes. } So much so that it’s not easy to completely describe the suffering in hell. 

There\marginnote{18.1} are, mendicants, animals that feed on grass. They eat by cropping fresh or dried grass with their teeth. And what animals feed on grass? Elephants, horses, cattle, donkeys, goats, deer, and various others. A fool who used to be a glutton here and did bad deeds here, when their body breaks up, after death, is reborn in the company of those sentient beings who feed on grass. 

There\marginnote{19.1} are animals that feed on dung. When they catch a whiff of dung they run to it, thinking, ‘There we’ll eat! There we’ll eat!’ It’s like when brahmins smell a burnt offering, they run to it, thinking, ‘There we’ll eat! There we’ll eat!’\footnote{Compare \textsanskrit{Chāndogya} \textsanskrit{Upaniṣad} 1:12.5, where white dogs sing hymns for food. The Brahmanical tradition does not treat that story as derogatory; rather, the dogs are sages in disguise, and the story illustrates how even dogs would avoid eating unclean food. } In the same way, there are animals that feed on dung. When they catch a whiff of dung they run to it, thinking, ‘There we’ll eat! There we’ll eat!’ And what animals feed on dung? Chickens, pigs, dogs, jackals, and various others. A fool who used to be a glutton here and did bad deeds here, after death is reborn in the company of those sentient beings who feed on dung. 

There\marginnote{20.1} are animals who are born, live, and die in darkness. And what animals are born, live, and die in darkness? Moths, maggots, earthworms, and various others. A fool who used to be a glutton here and did bad deeds here, after death is reborn in the company of those sentient beings who are born, live, and die in darkness. 

There\marginnote{21.1} are animals who are born, live, and die in water. And what animals are born, live, and die in water? Fish, turtles, crocodiles, and various others. A fool who used to be a glutton here and did bad deeds here, after death is reborn in the company of those sentient beings who are born, live, and die in water. 

There\marginnote{22.1} are animals who are born, live, and die in filth. And what animals are born, live, and die in filth? Those animals that are born, live, and die in a rotten fish, a rotten carcass, rotten porridge, or a sewer. A fool who used to be a glutton here and did bad deeds here, after death is reborn in the company of those sentient beings who are born, live, and die in filth. 

I\marginnote{23.1} could tell you many different things about the animal realm. So much so that it’s not easy to completely describe the suffering in the animal realm. 

Mendicants,\marginnote{24.1} suppose a person were to throw a yoke with a single hole into the ocean.\footnote{See \href{https://suttacentral.net/sn56.47/en/sujato}{SN 56.47} and \href{https://suttacentral.net/sn56.48/en/sujato}{SN 56.48}. } The east wind wafts it west; the west wind wafts it east; the north wind wafts it south; and the south wind wafts it north. And there was a one-eyed turtle who popped up once every hundred years. 

What\marginnote{24.4} do you think, mendicants? Would that one-eyed turtle still poke its neck through the hole in that yoke?” 

“No,\marginnote{24.6} sir. Only after a very long time, sir, if ever.” 

“That\marginnote{24.8} one-eyed turtle would poke its neck through the hole in that yoke sooner than a fool who has fallen to the underworld would be reborn as a human being, I say. Why is that? Because in that place there’s no principled or moral conduct, and no doing what is good and skillful. There they just prey on each other, preying on the weak. 

And\marginnote{25.1} suppose that fool, after a very long time, returned to the human realm. They’d be reborn in a low class family—a family of corpse-workers, hunters, bamboo-workers, chariot-makers, or scavengers. Such families are poor, with little to eat or drink, where life is tough, and food and shelter are hard to find. And they’d be ugly, unsightly, deformed, sickly—one-eyed, crippled, lame, or half-paralyzed. They don’t get to have food, drink, clothes, and vehicles; garlands, fragrance, and makeup; or bed, house, and lighting. And they do bad things by way of body, speech, and mind. When their body breaks up, after death, they’re reborn in a place of loss, a bad place, the underworld, hell. 

Suppose\marginnote{26.1} a gambler were to lose his wife and child, all his property, and then get thrown in jail with his first losing hand. Such a losing hand is trivial compared to the losing hand whereby a fool, having done bad things by way of body, speech, and mind, when their body breaks up, after death, is reborn in a place of loss, a bad place, the underworld, hell. This is the total fulfillment of the fool’s level. 

There\marginnote{27.1} are these three characteristics, signs, and manifestations of an astute person.\footnote{An “astute person” is \textit{\textsanskrit{paṇḍita}}, used for a scholar. } What three? An astute person thinks well, speaks well, and acts well. If an astute person didn’t think well, speak well, and act well, then how would the astute know of them, ‘This fellow is astute, a true person’? 

But\marginnote{28.1} since an astute person does think well, speak well, and act well, then the astute do know of them, ‘This fellow is astute, a true person’. An astute person experiences three kinds of pleasure and happiness in this very life. Suppose an astute person is sitting in a council hall, a street, or a crossroad, where people are discussing about what is proper and fitting. And suppose that astute person is someone who refrains from killing living creatures, stealing, committing sexual misconduct, lying, and beer, wine, and liquor intoxicants. Then that astute person thinks, ‘These people are discussing what is proper and fitting. And those good things are found in me and I exhibit them.’ This is the first kind of pleasure and happiness that an astute person experiences in this very life. 

Furthermore,\marginnote{29.1} an astute person sees that the kings have arrested a bandit, a criminal, and subjected them to various punishments—whipping, caning, and clubbing; cutting off hands or feet, or both; cutting off ears or nose, or both; the ‘porridge pot’, the ‘shell-shave’, the ‘\textsanskrit{Rāhu}’s mouth’, the ‘garland of fire’, the ‘burning hand’, the ‘bulrush twist’, the ‘bark dress’, the ‘antelope’, the ‘meat hook’, the ‘coins’, the ‘caustic pickle’, the ‘twisting bar’, the ‘straw mat’; being splashed with hot oil, being fed to the dogs, being impaled alive, and being beheaded. Then that astute person thinks, ‘The kinds of deeds for which the kings inflict such punishments—those things are not found in me and I do not exhibit them!’ This is the second kind of pleasure and happiness that an astute person experiences in this very life. 

Furthermore,\marginnote{30.1} when an astute person is resting on a chair or a bed or on the ground, their past good deeds—good conduct of body, speech, and mind—settle down upon them, rest down upon them, and lay down upon them. It is like the shadow of a great mountain peak in the evening as it settles down, rests down, and lays down upon the earth. In the same way, when an astute person is resting on a chair or a bed or on the ground, their past good deeds—good conduct of body, speech, and mind—settle down upon them, rest down upon them, and lay down upon them. Then that astute person thinks, ‘Well, I haven’t done bad things, violence and sin. And I have done good and skillful deeds that keep me safe. When I pass away, I’ll go to the place where people who’ve done such things go.’ So they don’t sorrow and wail and lament, beating their breast and falling into confusion. This is the third kind of pleasure and happiness that an astute person experiences in this very life. 

When\marginnote{31.1} their body breaks up, after death, they’re reborn in a good place, a heavenly realm. 

And\marginnote{32.1} if there’s anything of which it may be rightly said that it is utterly likable, desirable, and agreeable, it is of heaven that this should be said. So much so that it’s not easy to give a simile for how pleasurable heaven is.” 

When\marginnote{32.5} he said this, one of the mendicants asked the Buddha, “But sir, is it possible to give a simile?” 

“It’s\marginnote{33.1} possible,” said the Buddha. 

“Suppose\marginnote{33.2} there was a king, a wheel-turning monarch who possessed seven treasures and four blessings, and experienced pleasure and happiness because of them. 

What\marginnote{34.1} seven? It’s when, on the fifteenth day sabbath, an anointed aristocratic king has bathed his head and gone upstairs in the royal longhouse to observe the sabbath. And the heavenly wheel-treasure appears to him, with a thousand spokes, with rim and hub, complete in every detail.\footnote{The wheel is firstly the sun and secondly the wheel of the chariots that drove the Indo-Europeans in their conquests. It is the manifestation of unstoppable power. The whole story reflects the Indo-European dream of universal domination. | Compare \href{https://suttacentral.net/dn17/en/sujato\#1.7.3}{DN 17:1.7.3} and \href{https://suttacentral.net/dn26/en/sujato}{DN 26}. } Seeing this, the king thinks, ‘I have heard that when the heavenly wheel-treasure appears to a king in this way, he becomes a wheel-turning monarch. Am I then a wheel-turning monarch?’ 

Then\marginnote{35.1} the anointed aristocratic king, taking a ceremonial vase in his left hand, besprinkled the wheel-treasure with his right hand, saying,\footnote{Many of the details in this myth echo the Brahmanical horse sacrifice. Since the horse was the primary source of Indo-European dominion, its sacrifice served to authorize the power of a king. It was a costly and dangerous rite that was attempted only by the greatest of sovereigns. | \textit{\textsanskrit{Bhiṅkāra}} (“ceremonial vase”) and \textit{abbhukkirati} (“besprinkled”) are elevated terms. } ‘Roll forth, O wheel-treasure! Triumph, O wheel-treasure!’ Then the wheel-treasure rolls towards the east. And the king follows it together with his army of four divisions. In whatever place the wheel-treasure stands still, there the king comes to stay together with his army.\footnote{In the horse sacrifice, the horse is released for a year, while the king follows it with his army, claiming any land that it wanders on as his. } And any opposing rulers of the eastern quarter come to the wheel-turning monarch and say, ‘Come, great king! Welcome, great king! We are yours, great king, instruct us.’ The wheel-turning monarch says, ‘Do not kill living creatures. Do not steal. Do not commit sexual misconduct. Do not lie. Do not drink liquor. Maintain the current level of taxation.’\footnote{Read \textit{\textsanskrit{bhuñjati}} at \href{https://suttacentral.net/mn98/en/sujato\#10.30}{MN 98:10.30} with \textit{\textsanskrit{yathābhuttañca} \textsanskrit{bhuñjatha}} at \href{https://suttacentral.net/dn17/en/sujato\#1.9.4}{DN 17:1.9.4}, \href{https://suttacentral.net/dn26/en/sujato\#6.7}{DN 26:6.7}, and \href{https://suttacentral.net/mn129/en/sujato\#35.7}{MN 129:35.7}. These have sometimes been rendered “eat”, “enjoy”, or “govern”. But compare the archaic English “use” meaning “the benefit or profit of lands”. Thus \textit{\textsanskrit{yathābhuttañca} \textsanskrit{bhuñjatha}} means “use as has been used”, i.e. “maintain the current level of taxation”. } And so the opposing rulers of the eastern quarter become his vassals. 

Then\marginnote{35.9} the wheel-treasure, having plunged into the eastern ocean and emerged again, rolls towards the south. …\footnote{The Wheel plunges into the seas, while the sacrificial horse is born in the western and eastern seas. } Having plunged into the southern ocean and emerged again, it rolls towards the west. … Having plunged into the western ocean and emerged again, it rolls towards the north, followed by the king together with his army of four divisions. In whatever place the wheel-treasure stands still, there the king comes to stay together with his army. 

And\marginnote{35.12} any opposing rulers of the northern quarter come to the wheel-turning monarch and say, ‘Come, great king! Welcome, great king! We are yours, great king, instruct us.’ The wheel-turning monarch says, ‘Do not kill living creatures. Do not steal. Do not commit sexual misconduct. Do not lie. Do not drink liquor. Maintain the current level of taxation.’ And so the opposing rulers of the eastern quarter become his vassals.\footnote{Historically, India has usually been divided into squabbling realms, but from an early age there was a dream of a unified and peaceful continent. } 

And\marginnote{35.17} then the wheel-treasure, having triumphed over this land surrounded by ocean, returns to the royal capital. There it stands still at the gate to the royal compound as if fixed to an axle, illuminating the royal compound.\footnote{The phrase “fixed to an axle” is used of an ordinary wheel at \href{https://suttacentral.net/an3.15/en/sujato\#3.4}{AN 3.15:3.4}. | This passage is also found at \href{https://suttacentral.net/dn17/en/sujato\#1.11.1}{DN 17:1.11.1} and \href{https://suttacentral.net/dn26/en/sujato\#7.16}{DN 26:7.16} with the additional mention of the “High Court”. } Such is the wheel-treasure that appears to the wheel-turning monarch. 

Next,\marginnote{36.1} the elephant-treasure appears to the wheel-turning monarch. It was an all-white sky-walker with psychic power, touching the ground in seven places, a king of elephants named Sabbath.\footnote{The white elephant is a symbol of royalty to this day. The description recalls Indra’s elephant \textsanskrit{Airāvata}. | For \textit{\textsanskrit{sattappatiṭṭho}}, the commentary has \textit{\textsanskrit{susaṇṭhitaaṅgapaccaṅga}} (“well-grounded on each and every limb”), a sense confirmed by the \textsanskrit{Mūlasarvāstivāda} \textsanskrit{Bhaiṣajyavastu} which has \textit{\textsanskrit{saptāṅgaḥ} \textsanskrit{supratiṣṭhito}} (“well-established on seven limbs”). The subcommentary lists the four feet, trunk, tail, and penis (\textit{\textsanskrit{varaṅga}}). } Seeing him, the king was impressed, ‘This would truly be a fine elephant vehicle, if he would submit to taming.’ Then the elephant-treasure submitted to taming, as if he were a fine thoroughbred elephant that had been tamed for a long time. Once it so happened that the wheel-turning monarch, testing that same elephant-treasure, mounted him in the morning and traversed the land surrounded by ocean before returning to the royal capital in time for breakfast. Such is the elephant-treasure that appears to the wheel-turning monarch. 

Next,\marginnote{37.1} the horse-treasure appears to the wheel-turning monarch. It was an all-white sky-walker with psychic power, with head of black and mane like woven reeds, a royal steed named Thundercloud.\footnote{The sacrificial horse is likewise white with black head or forequarters. It is identified with the sun, thus being a “sky-walker”. “Thundercloud” (\textit{\textsanskrit{valāhaka}}; Sanskrit \textit{\textsanskrit{balāhaka}}) is the name of one of the four horses of \textsanskrit{Kṛṣṇa}’s chariot in the \textsanskrit{Mahābharata}. The description here also recalls the Vedic sacred horse \textit{\textsanskrit{uccaiḥśravas}}. } Seeing him, the king was impressed, ‘This would truly be a fine horse vehicle, if he would submit to taming.’ Then the horse-treasure submitted to taming, as if he were a fine thoroughbred horse that had been tamed for a long time. Once it so happened that the wheel-turning monarch, testing that same horse-treasure, mounted him in the morning and traversed the land surrounded by ocean before returning to the royal capital in time for breakfast. Such is the horse-treasure that appears to the wheel-turning monarch. 

Next,\marginnote{38.1} the jewel-treasure appears to the wheel-turning monarch. It is a beryl gem that’s naturally beautiful, eight-faceted, well-worked. And the radiance of that jewel spreads all-round for a league. Once it so happened that the wheel-turning monarch, testing that same jewel-treasure, mobilized his army of four divisions and, with the jewel hoisted on his banner, set out in the dark of the night. Then the villagers around them set off to work, thinking that it was day. Such is the jewel-treasure that appears to the wheel-turning monarch. 

Next,\marginnote{39.1} the woman-treasure appears to the wheel-turning monarch. She is attractive, good-looking, lovely, of surpassing beauty. She’s neither too tall nor too short; neither too thin nor too fat; neither too dark nor too light. She outdoes human beauty without reaching heavenly beauty. And her touch is like a tuft of cotton-wool or kapok. When it’s cool her limbs are warm, and when it’s warm her limbs are cool. The fragrance of sandal floats from her body, and lotus from her mouth. She gets up before the king and goes to bed after him, and is obliging, behaving nicely and speaking politely. The woman-treasure does not betray the wheel-turning monarch even in thought, still less in deed. Such is the woman-treasure who appears to the wheel-turning monarch. 

Next,\marginnote{40.1} the householder-treasure appears to the wheel-turning monarch. The power of clairvoyance manifests in him as a result of past deeds, by which he sees hidden treasure, both owned and ownerless. He approaches the wheel-turning monarch and says, ‘Relax, sire. I will take care of the treasury.’ Once it so happened that the wheel-turning monarch, testing that same householder-treasure, boarded a boat and sailed to the middle of the Ganges river. Then he said to the householder-treasure, ‘Householder, I need gold, both coined and uncoined.’ ‘Well then, great king, draw the boat up to one shore.’ ‘It’s right here, householder, that I need gold, both coined and uncoined.’ Then that householder-treasure, immersing both hands in the water, pulled up a pot full of gold, both coined and uncoined, and said to the king, ‘Is this sufficient, great king? Has enough been done, great king, enough offered?’ The wheel-turning monarch said, ‘That is sufficient, householder. Enough has been done, enough offered.’ Such is the householder-treasure that appears to the wheel-turning monarch. 

Next,\marginnote{41.1} the commander-treasure appears to the wheel-turning monarch. He is astute, competent, intelligent, and capable of getting the king to appoint who should be appointed, dismiss who should be dismissed, and retain who should be retained. He approaches the wheel-turning monarch and says, ‘Relax, sire. I shall issue instructions.’ Such is the commander-treasure that appears to the wheel-turning monarch. These are the seven treasures possessed by a wheel-turning monarch. 

And\marginnote{42.1} what are the four blessings?\footnote{“Blessing” is \textit{iddhi}, which normally means “psychic power”. } 

A\marginnote{42.2} wheel-turning monarch is attractive, good-looking, lovely, of surpassing beauty, more so than other people. This is the first blessing. 

Furthermore,\marginnote{43.1} he is long-lived, more so than other people. This is the second blessing. 

Furthermore,\marginnote{44.1} he is rarely ill or unwell, and his stomach digests well, being neither too hot nor too cold, more so than other people. This is the third blessing. 

Furthermore,\marginnote{45.1} a wheel-turning monarch is as dear and beloved to the brahmins and householders\footnote{“Householders” (\textit{gahapati}) is literal; it means land owners. Thus the “brahmins and householders” (not “brahmin householders”) were the wealthy class. } as a father is to his children. And the brahmins and householders are as dear to the wheel-turning monarch as children are to their father. 

Once\marginnote{45.7} it so happened that a wheel-turning monarch went with his army of four divisions to visit a park. Then the brahmins and householders went up to him and said, ‘Slow down, Your Majesty, so we may see you longer!’ And the king addressed his charioteer, ‘Drive slowly, charioteer, so I can see the brahmins and householders longer!’ This is the fourth blessing. 

These\marginnote{45.13} are the four blessings possessed by a wheel-turning monarch. 

What\marginnote{46.1} do you think, mendicants? Would a wheel-turning monarch who possessed these seven treasures and these four blessings experience pleasure and happiness because of them?” 

“Sir,\marginnote{46.3} a wheel-turning monarch who possessed even a single one of these treasures would experience pleasure and happiness because of that, let alone all seven treasures and four blessings!” 

Then\marginnote{47.1} the Buddha, picking up a stone the size of his palm, addressed the mendicants, “What do you think, mendicants? Which is bigger: the stone the size of my palm that I’ve picked up, or the Himalayas, the king of mountains?” 

“Sir,\marginnote{47.4} the stone you’ve picked up is tiny. Compared to the Himalayas, it doesn’t count, it’s not worth a fraction, there’s no comparison.” 

“In\marginnote{47.5} the same way, compared to the happiness of heaven, the pleasure and happiness experienced by a wheel-turning monarch due to those seven treasures and those four blessings doesn’t even count, it’s not even a fraction, there’s no comparison. 

And\marginnote{48.1} suppose that astute person, after a very long time, returned to the human realm. They’d be reborn in a well-to-do family of aristocrats, brahmins, or householders—rich, affluent, and wealthy, with lots of gold and silver, lots of property and assets, and lots of money and grain. And they’d be attractive, good-looking, lovely, of surpassing beauty. They’d get to have food, drink, clothes, and vehicles; garlands, fragrance, and makeup; and a bed, house, and lighting. And they do good things by way of body, speech, and mind. When their body breaks up, after death, they’re reborn in a good place, a heavenly realm. 

Suppose\marginnote{49.1} a gambler was to win a big pile of money with the first perfect hand. Such a perfect hand is trivial compared to the perfect hand whereby an astute person, when their body breaks up, after death, is reborn in a good place, a heavenly realm. This is the total fulfillment of the astute person’s level.” 

That\marginnote{49.5} is what the Buddha said. Satisfied, the mendicants approved what the Buddha said. 

%
\section*{{\suttatitleacronym MN 130}{\suttatitletranslation Messengers of the Gods }{\suttatitleroot Devadūtasutta}}
\addcontentsline{toc}{section}{\tocacronym{MN 130} \toctranslation{Messengers of the Gods } \tocroot{Devadūtasutta}}
\markboth{Messengers of the Gods }{Devadūtasutta}
\extramarks{MN 130}{MN 130}

\scevam{So\marginnote{1.1} I have heard.\footnote{Framed with the parable of the “messengers of the gods”, which are portents of mortality, this discourse expands the description of the power of clairvoyance or “divine eye” (\textit{dibbacakkhu}), with a special emphasis on the descriptions of the torments of hell (\textit{niraya}). This was a very popular discourse, with no fewer than six Chinese parallels and multiple references in Sanskrit literature. The Sinhala chronicle says it was taught as a primary text in \textsanskrit{Mahisamaṇḍala} by \textsanskrit{Mahādeva}, and in Sri Lanka by Mahinda, each time resulting in mass conversions (\textsanskrit{Mahāvaṁsa} 12:29, 14:63). } }At one time the Buddha was staying near \textsanskrit{Sāvatthī} in Jeta’s Grove, \textsanskrit{Anāthapiṇḍika}’s monastery. There the Buddha addressed the mendicants, “Mendicants!” 

“Venerable\marginnote{1.5} sir,” they replied. The Buddha said this: 

“Mendicants,\marginnote{2.1} suppose there were two houses with doors. A person with clear eyes standing in between them would see people entering and leaving a house and wandering to and fro. 

In\marginnote{2.2} the same way, with clairvoyance that is purified and superhuman, I see sentient beings passing away and being reborn—inferior and superior, beautiful and ugly, in a good place or a bad place. I understand how sentient beings are reborn according to their deeds: ‘These dear beings did good things by way of body, speech, and mind. They never denounced the noble ones; they had right view; and they chose to act out of that right view. When their body breaks up, after death, they’re reborn in a good place, a heavenly realm, or among humans. These dear beings did bad things by way of body, speech, and mind. They denounced the noble ones; they had wrong view; and they chose to act out of that wrong view. When their body breaks up, after death, they’re reborn in the ghost realm, the animal realm, or in a place of loss, a bad place, the underworld, hell.’ 

The\marginnote{3.1} wardens of hell take them by the arms and present them to King Yama, saying,\footnote{There is a set of beliefs about hell that is common to Buddhists, Hindus, and Jains. This “hell complex” includes such beliefs as: hell is long-lasting but impermanent; there are many named hells; punishments are terrible and varied; the punishment is determined by kamma; and tortures are inflicted by hell wardens. This sutta appears to be the earliest full account of this hell complex in Indian literature. Below I briefly outline the development of the hell complex in Brahmanical literature (\href{https://suttacentral.net/mn130/en/sujato\#10.1}{MN 130:10.1}). } ‘Your Majesty, this person did not pay due respect to their mother and father, ascetics and brahmins, or honor the elders in the family. May Your Majesty punish them!’ 

King\marginnote{3.4} Yama pursues, presses, and grills them about the first messenger of the gods. ‘Mister, did you not see the first messenger of the gods that appeared among human beings?’\footnote{Such “messengers of the gods” appear as reminders for the dangers of mortality. Five appear here—birth, old age, sickness, punishment, and death—while \href{https://suttacentral.net/an3.36/en/sujato}{AN 3.36} has three, namely old age, sickness, and death. At \href{https://suttacentral.net/mn83/en/sujato\#4.6}{MN 83:4.6} we meet just one, the grey hairs of old age. In non-Buddhist texts, such messengers can also be celestial interlocutors in a more literal sense (eg. \textsanskrit{Mahābhārata} 1.2.232b, 1.9.005d). | Compare the four signs that prompted \textsanskrit{Vipassī}’s going forth (\href{https://suttacentral.net/dn14/en/sujato\#2.1.1}{DN 14:2.1.1}). } 

He\marginnote{3.6} says, ‘I saw nothing, sir.’ 

King\marginnote{4.1} Yama says to them, ‘Mister, did you not see among human beings a little baby collapsed in their own urine and feces?’ 

He\marginnote{4.3} says, ‘I saw that, sir.’ 

King\marginnote{4.5} Yama says to them, ‘Mister, did it not occur to you—being sensible and mature—“I, too, am liable to be born. I’m not exempt from rebirth. I’d better do good by way of body, speech, and mind”?’ 

He\marginnote{4.8} says, ‘I couldn’t, sir. I was negligent.’ 

King\marginnote{4.10} Yama says to them, ‘Mister, because you were negligent, you didn’t do good by way of body, speech, and mind. Well, they’ll definitely punish you to fit your negligence. That bad deed wasn’t done by your mother, father, brother, or sister. It wasn’t done by friends and colleagues, by relatives and kin, by ascetics and brahmins, or by the deities. That bad deed was done by you alone, and you alone will experience the result.’ 

Then\marginnote{5.1} King Yama grills them about the second messenger of the gods. ‘Mister, did you not see the second messenger of the gods that appeared among human beings?’ 

He\marginnote{5.3} says, ‘I saw nothing, sir.’ 

King\marginnote{5.5} Yama says to them, ‘Mister, did you not see among human beings an elderly woman or a man—eighty, ninety, or a hundred years old—bent double, crooked, leaning on a staff, trembling as they walk, ailing, past their prime, with teeth broken, hair grey and scanty or bald, skin wrinkled, and limbs blotchy?’ 

He\marginnote{5.7} says, ‘I saw that, sir.’ 

King\marginnote{5.9} Yama says to them, ‘Mister, did it not occur to you—being sensible and mature—“I, too, am liable to grow old. I’m not exempt from old age. I’d better do good by way of body, speech, and mind”?’ 

He\marginnote{5.12} says, ‘I couldn’t, sir. I was negligent.’ 

King\marginnote{5.14} Yama says to them, ‘Mister, because you were negligent, you didn’t do good by way of body, speech, and mind. Well, they’ll definitely punish you to fit your negligence. That bad deed wasn’t done by your mother, father, brother, or sister. It wasn’t done by friends and colleagues, by relatives and kin, by ascetics and brahmins, or by the deities. That bad deed was done by you alone, and you alone will experience the result.’ 

Then\marginnote{6.1} King Yama grills them about the third messenger of the gods. ‘Mister, did you not see the third messenger of the gods that appeared among human beings?’ 

He\marginnote{6.3} says, ‘I saw nothing, sir.’ 

King\marginnote{6.5} Yama says to them, ‘Mister, did you not see among human beings a woman or a man, sick, suffering, gravely ill, collapsed in their own urine and feces, being picked up by some and put down by others?’ 

He\marginnote{6.7} says, ‘I saw that, sir.’ 

King\marginnote{6.9} Yama says to them, ‘Mister, did it not occur to you—being sensible and mature—“I, too, am liable to become sick. I’m not exempt from sickness. I’d better do good by way of body, speech, and mind”?’ He says, ‘I couldn’t, sir. I was negligent.’ 

King\marginnote{6.14} Yama says to them, ‘Mister, because you were negligent, you didn’t do good by way of body, speech, and mind. Well, they’ll definitely punish you to fit your negligence. That bad deed wasn’t done by your mother, father, brother, or sister. It wasn’t done by friends and colleagues, by relatives and kin, by ascetics and brahmins, or by the deities. That bad deed was done by you alone, and you alone will experience the result.’ 

Then\marginnote{7.1} King Yama grills them about the fourth messenger of the gods. ‘Mister, did you not see the fourth messenger of the gods that appeared among human beings?’ 

He\marginnote{7.3} says, ‘I saw nothing, sir.’ 

King\marginnote{7.5} Yama says to them, ‘Mister, did you not see among human beings when the rulers arrested a bandit, a criminal, and subjected them to various punishments—whipping, caning, and clubbing; cutting off hands or feet, or both; cutting off ears or nose, or both; the ‘porridge pot’, the ‘shell-shave’, the ‘demon’s mouth’, the ‘garland of fire’, the ‘burning hand’, the ‘bulrush twist’, the ‘bark dress’, the ‘antelope’, the ‘meat hook’, the ‘coins’, the ‘caustic pickle’, the ‘twisting bar’, the ‘straw mat’; being splashed with hot oil, being fed to the dogs, being impaled alive, and being beheaded?’ 

He\marginnote{7.8} says, ‘I saw that, sir.’ 

King\marginnote{7.10} Yama says to them, ‘Mister, did it not occur to you—being sensible and mature—that if someone who does bad deeds receives such punishment in this very life, what must happen to them in the next; I’d better do good by way of body, speech, and mind”?’ 

He\marginnote{7.13} says, ‘I couldn’t, sir. I was negligent.’ 

King\marginnote{7.15} Yama says to them, ‘Mister, because you were negligent, you didn’t do good by way of body, speech, and mind. Well, they’ll definitely punish you to fit your negligence. That bad deed wasn’t done by your mother, father, brother, or sister. It wasn’t done by friends and colleagues, by relatives and kin, by ascetics and brahmins, or by the deities. That bad deed was done by you alone, and you alone will experience the result.’ 

Then\marginnote{8.1} King Yama grills them about the fifth messenger of the gods. ‘Mister, did you not see the fifth messenger of the gods that appeared among human beings?’ 

He\marginnote{8.3} says, ‘I saw nothing, sir.’ 

King\marginnote{8.5} Yama says to them, ‘Mister, did you not see among human beings a woman or a man, dead for one, two, or three days, bloated, livid, and festering?’ 

He\marginnote{8.7} says, ‘I saw that, sir.’ 

King\marginnote{8.9} Yama says to them, ‘Mister, did it not occur to you—being sensible and mature—“I, too, am liable to die. I’m not exempt from death. I’d better do good by way of body, speech, and mind”?’ 

He\marginnote{8.12} says, ‘I couldn’t, sir. I was negligent.’ 

King\marginnote{8.14} Yama says to them, ‘Mister, because you were negligent, you didn’t do good by way of body, speech, and mind. Well, they’ll definitely punish you to fit your negligence. That bad deed wasn’t done by your mother, father, brother, or sister. It wasn’t done by friends and colleagues, by relatives and kin, by ascetics and brahmins, or by the deities. That bad deed was done by you alone, and you alone will experience the result.’ 

After\marginnote{9.1} grilling them about the fifth messenger of the gods, King Yama falls silent. 

The\marginnote{10.1} wardens of hell punish them with the five-fold crucifixion.\footnote{The history of hell in India begins with the Rig Veda, which speaks of evil-doers facing the “erasure of the wolf”, the “downfall” (2.29.6c), the “bottomless darkness” (7.104.3) “beneath all three earths” (7.104.11), the “endless abyss” (7.104.17), the “darkness below” (10.152.4). Atharva Veda similarly speaks of the “house below” (2.14.3a), the “deepest dark” (8.2.24). It also consigns a man stingy with cattle to the \textit{\textsanskrit{nārakaṁ} \textsanskrit{lokaṁ}}, which seems to mean “the world of the hell beings” (12.4.36c); there is an equally vague reference to a “hell being” at Śukla Yajur Veda 30.5. These seem to be the earliest uses of the word \textit{naraka} in the sense of “hell”; this sense is not found in early Pali (see note on \href{https://suttacentral.net/dn12/en/sujato\#78.2}{DN 12:78.2}). \textsanskrit{Jaiminīya} \textsanskrit{Brāhmaņa} 1.42 describes the “other world” in both hellish and heavenly terms (see too Śatapatha \textsanskrit{Brāhmaṇa} 11.6.1). It depicts the hellish torments taking place in regions of the “other world” rather than as a distinct hell realm; and it does not share many details in common with later accounts. Thus far the texts are pre-Buddhist. It is not until the much later \textsanskrit{Purāṇic} literature that lurid descriptions of hells, usually numbered 21 or 28, become a standard feature, along with the rest of the hell complex. Several of the places mentioned here correspond with the \textsanskrit{Purāṇic} hells. } They drive red-hot stakes through the hands and feet, and another in the middle of the chest. And there they suffer painful, sharp, severe, acute feelings—but they don’t die until that bad deed is eliminated.\footnote{I give a few examples of similar descriptions in the \textsanskrit{Purāṇas} and the Jain \textsanskrit{Sūtrakṛtāṅga}, without any attempt to be comprehensive. | The notion that one does not die despite the suffering is also found at \textsanskrit{Bhāgavata} \textsanskrit{Purāṇa} 5.26.28 and \textsanskrit{Sūtrakṛtāṅga} 1.5.1.16. | The lifespan of one in hell is illustrated in the story of \textsanskrit{Kokālika} (\href{https://suttacentral.net/sn6.10/en/sujato}{SN 6.10} = \href{https://suttacentral.net/an10.89/en/sujato}{AN 10.89}), which is repeated at \href{https://suttacentral.net/snp3.10/en/sujato}{Snp 3.10} with an expanded description of the hells that bears much in common with the present discourse. } 

The\marginnote{11.1} wardens of hell throw them down and hack them with axes. …\footnote{Brahma \textsanskrit{Purāṇa} 106.40–41. } 

They\marginnote{12.1} hang them upside-down and hack them with hatchets. …\footnote{Śiva \textsanskrit{Purāṇa} 5.9.26; \textsanskrit{Devī} \textsanskrit{Bhāgavata} \textsanskrit{Purāṇa} 8.23. There is a special “Upside-down hell” in \textsanskrit{Viṣṇu} \textsanskrit{Purāṇa} 2.63. } 

They\marginnote{13.1} harness them to a chariot, and drive them back and forth across burning ground, blazing and glowing. … 

They\marginnote{14.1} make them climb up and down a huge mountain of burning coals, blazing and glowing. … 

The\marginnote{15.1} wardens of hell turn them upside down and throw them in a red-hot copper pot, burning, blazing, and glowing.\footnote{Agni \textsanskrit{Purāṇa} 307.25; Śiva \textsanskrit{Purāṇa} 5.9.9, 34, 44. } There they’re seared in boiling scum, and they’re swept up and down and round and round. And there they suffer painful, sharp, severe, acute feelings—but they don’t die until that bad deed is eliminated. 

The\marginnote{16.1} wardens of hell toss them into the Great Hell. Now, about that Great Hell: 

\begin{verse}%
‘Four\marginnote{16.3} are its corners, four its doors, \\
neatly divided in equal parts. \\
Surrounded by an iron wall, \\
of iron is its roof. 

The\marginnote{16.7} ground is even made of iron, \\
it burns with fierce fire.\footnote{The \textsanskrit{Kālasūtra} hell is made of burning copper (\textsanskrit{Bhāgavata} \textsanskrit{Purāṇa} 5.26.14). } \\
The heat forever radiates \\
a hundred leagues around.’ 

%
\end{verse}

Now\marginnote{17.1} in the Great Hell, flames surge out of the walls and crash into the opposite wall: from east to west, from west to east, from north to south, from south to north, from bottom to top, from top to bottom. And there they suffer painful, sharp, severe, acute feelings—but they don’t die until that bad deed is eliminated. 

There\marginnote{18.1} comes a time when, after a very long period has passed, the eastern gate of the Great Hell is opened. So they run there as fast as they can.\footnote{\textsanskrit{Bhāgavata} \textsanskrit{Purāṇa} 5.26.15 describes how the sinner runs about in pain trying to escape. } And as they run, their outer skin, inner skin, flesh, and sinews burn and even their bones smoke. Such is their escape;\footnote{Śiva \textsanskrit{Purāṇa} 5.9.37. } but when they’ve managed to make it most of the way, the gate is slammed shut. And there they suffer painful, sharp, severe, acute feelings—but they don’t die until that bad deed is eliminated. 

There\marginnote{18.6} comes a time when, after a very long period has passed, the western gate … northern gate … southern gate of the Great Hell is opened. So they run there as fast as they can. And as they run, their outer skin, inner skin, flesh, and sinews burn and even their bones smoke. Such is their escape; but when they’ve managed to make it most of the way, the gate is slammed shut. And there they suffer painful, sharp, severe, acute feelings—but they don’t die until that bad deed is eliminated. 

There\marginnote{19.1} comes a time when, after a very long period has passed, the eastern gate of the Great Hell is opened. So they run there as fast as they can. And as they run, their outer skin, inner skin, flesh, and sinews burn and even their bones smoke. Such is their escape; and they make it out that door. 

Immediately\marginnote{20.1} adjacent to the Great Hell is the vast Dung Hell. And that’s where they fall. In that Dung Hell there are needle-mouthed creatures that bore through the outer skin, the inner skin, the flesh, sinews, and bones, until they reach the marrow and devour it.\footnote{“Needle-mouth” (\textit{\textsanskrit{sūcimukha}}) is one of the 28 \textsanskrit{Purāṇic} hells, the destiny of scrooges, where one is stitched as cloth by a tailor (\textsanskrit{Bhāgavata} \textsanskrit{Purāṇa} 526.36; \textsanskrit{Devī} \textsanskrit{Bhāgavata} \textsanskrit{Purāṇa} 8.23). Śiva \textsanskrit{Purāṇa} 5.9.42–3 describes how sinners are devoured by sharp-toothed worms while rotting in mounds of flesh; \textsanskrit{Bhāgavata} \textsanskrit{Purāṇa} 5.26.17 says one is devoured by the same creatures one killed while alive. } And there they suffer painful, sharp, severe, acute feelings—but they don’t die until that bad deed is eliminated. 

Immediately\marginnote{21.1} adjacent to the Dung Hell is the vast Hell of Burning Chaff.\footnote{Śiva \textsanskrit{Purāṇa} 5.9.18. } And that’s where they fall. And there they suffer painful, sharp, severe, acute feelings—but they don’t die until that bad deed is eliminated. 

Immediately\marginnote{22.1} adjacent to the Hell of Burning Chaff is the vast Red Silk-Cotton Forest. It’s a league high, full of sixteen-inch thorns, burning, blazing, and glowing.\footnote{The “Red Silk-Cotton Forest” (\textit{simbalivana}) is one of the 28 hells (\textsanskrit{Bhāgavata} \textsanskrit{Purāṇa} 5.26.7; \textsanskrit{Devī} \textsanskrit{Bhāgavatapurāṇa} 10.21.11–28; Śiva \textsanskrit{Purāṇa} 5.9.20). The red silk cotton tree (\textit{simbali}, Sanskrit \textit{\textsanskrit{śālmalī}}, \textit{bombax ceiba}) has a spiky trunk. Treading on the spikes could cause infections (Rig Veda 7.50.3). | This hell and the next reflect the miseries of struggling through thick, prickly jungles. } And there they make them climb up and down. And there they suffer painful, sharp, severe, acute feelings—but they don’t die until that bad deed is eliminated. 

Immediately\marginnote{23.1} adjacent to the Red Silk-Cotton Forest is the vast Sword-Leaf Forest.\footnote{The “Sword-Leaf Forest” (\textit{asipattavana}; also at \href{https://suttacentral.net/snp3.10/en/sujato\#25.1}{Snp 3.10:25.1}) is another of the 28 hells. This befalls those who destroy forests (\textsanskrit{Brahmāṇḍa} \textsanskrit{Purāṇa} 4.2.173; \textsanskrit{Viṣṇu} \textsanskrit{Purāṇa} 6.2), while Agni \textsanskrit{Purāṇa} 203.10 says it is for one who killed their mother, and \textsanskrit{Bhāgavata} \textsanskrit{Purāṇa} 5.26.15 says it is for one who strays from the Vedas. } They enter that. There the fallen leaves blown by the wind cut their hands, feet, both hands and feet; they cut their ears, nose, both ears and nose. And there they suffer painful, sharp, severe, acute feelings—but they don’t die until that bad deed is eliminated. 

Immediately\marginnote{24.1} adjacent to the Sword-Leaf Forest is the vast Acid River.\footnote{The “Acid River” (\textit{\textsanskrit{khārodakā} \textsanskrit{nadī}}) is yet another of the 28 hells (\textit{\textsanskrit{kṣārakardama}}, Śiva \textsanskrit{Purāṇa} 5.9.8; \textsanskrit{Bhāgavata} \textsanskrit{Purāṇa} 5.26.30; \textsanskrit{Devī} \textsanskrit{Bhāgavata} \textsanskrit{Purāṇa} 8.21; Skanda \textsanskrit{Purāṇa} 6.1.226.54). } And that’s where they fall. There they are swept upstream, swept downstream, and swept both up and down stream. And there they suffer painful, sharp, severe, acute feelings—but they don’t die until that bad deed is eliminated. 

The\marginnote{25.1} wardens of hell pull them out with a hook and place them on dry land, and say, ‘Mister, what do you want?’\footnote{Brahma \textsanskrit{Purāṇa} 106.25–32. } 

They\marginnote{25.3} say, ‘I’m hungry, sir.’ 

The\marginnote{25.5} wardens of hell force open their mouth with a hot iron spike—burning, blazing, glowing—and shove in a red-hot copper ball, burning, blazing, and glowing.\footnote{Śiva \textsanskrit{Purāṇa} 5.9.45. } It burns their lips, mouth, tongue, throat, and stomach before coming out below dragging their entrails. And there they feel painful, sharp, severe, acute feelings—but they don’t die until that bad deed is eliminated. 

The\marginnote{26.1} wardens of hell say, ‘Mister, what do you want?’ 

They\marginnote{26.3} say, ‘I’m thirsty, sir.’ 

The\marginnote{26.5} wardens of hell force open their mouth with a hot iron spike—burning, blazing, glowing—and pour in molten copper, burning, blazing, and glowing.\footnote{The “iron drink” (\textit{\textsanskrit{ayaḥpāna}}) is another \textsanskrit{Purāṇic} hell (Agni \textsanskrit{Purāṇa} 307.25–28; \textsanskrit{Bhāgavata} \textsanskrit{Purāṇa} 5.26.29; \textsanskrit{Devī} \textsanskrit{Bhāgavata} \textsanskrit{Purāṇa} 8.23). } It burns their lips, mouth, tongue, throat, and stomach before coming out below dragging their entrails. And there they feel painful, sharp, severe, acute feelings—but they don’t die until that bad deed is eliminated. 

The\marginnote{27.1} wardens of hell toss them back in the Great Hell. 

Once\marginnote{28.1} upon a time, King Yama thought: ‘Those who do such bad deeds in the world receive these many different punishments. Oh, I hope I may be reborn as a human being! And that a Realized One—a perfected one, a fully awakened Buddha—arises in the world! And that I may pay homage to the Buddha! Then the Buddha can teach me Dhamma, so that I may understand his teaching.’ 

Now,\marginnote{29.1} I don’t say this because I’ve heard it from some other ascetic or brahmin. I only say it because I’ve known, seen, and realized it for myself.”\footnote{This statement is found only in the Pali text, although Chinese texts are effectively similar, in that they all speak of the Buddha’s clairvoyance. } 

That\marginnote{30.1} is what the Buddha said. Then the Holy One, the Teacher, went on to say: 

\begin{verse}%
“When\marginnote{30.3} warned by the gods’ messengers, \\
those people who are negligent\footnote{\textit{\textsanskrit{Māṇava}}, which normally means “brahmin students”, here is equivalent to \textit{manussa} (“humans”); both words mean “descendant of the first man, Manu”. \textit{\textsanskrit{Māṇava}} is also used in the sense “human” at \href{https://suttacentral.net/snp3.4/en/sujato\#5.3}{Snp 3.4:5.3}. } \\
sorrow for a long time \\
when they go to that wretched place. 

When\marginnote{30.7} warned by the gods’ messengers, \\
the good and true persons here \\
never neglect \\
the teaching of the Noble One. 

Seeing\marginnote{30.11} the danger in grasping, \\
the origin of birth and death, \\
the unattached are freed \\
with the ending of birth and death. 

Happy,\marginnote{30.15} they’ve come to a safe place, \\
quenched in this very life. \\
They’ve gone beyond all threats and perils, \\
and risen above all suffering.” 

%
\end{verse}

%
\addtocontents{toc}{\let\protect\contentsline\protect\nopagecontentsline}
\chapter*{The Chapter on Analysis }
\addcontentsline{toc}{chapter}{\tocchapterline{The Chapter on Analysis }}
\addtocontents{toc}{\let\protect\contentsline\protect\oldcontentsline}

%
\section*{{\suttatitleacronym MN 131}{\suttatitletranslation One Fine Night }{\suttatitleroot Bhaddekarattasutta}}
\addcontentsline{toc}{section}{\tocacronym{MN 131} \toctranslation{One Fine Night } \tocroot{Bhaddekarattasutta}}
\markboth{One Fine Night }{Bhaddekarattasutta}
\extramarks{MN 131}{MN 131}

\scevam{So\marginnote{1.1} I have heard.\footnote{A favorite of meditators, this discourse takes a distinctive set of verses as a template for explaining the aspect of meditation called  \textit{\textsanskrit{vipassanā}}, commonly rendered “insight”; for why I use  “discernment” see \href{https://suttacentral.net/dn14/en/sujato\#1.37.5}{DN 14:1.37.5} and note. The next three discourses explain the same verses on different occasions, with \href{https://suttacentral.net/mn133/en/sujato}{MN 133} offering a complementary explanation. } }At one time the Buddha was staying near \textsanskrit{Sāvatthī} in Jeta’s Grove, \textsanskrit{Anāthapiṇḍika}’s monastery. There the Buddha addressed the mendicants, “Mendicants!” 

“Venerable\marginnote{1.5} sir,” they replied. The Buddha said this: 

“I\marginnote{2.1} shall teach you the summary recital and the analysis of the one who has one fine night. Listen and apply your mind well, I will speak.” 

“Yes,\marginnote{2.3} sir,” they replied. The Buddha said this: 

\begin{verse}%
“Don’t\marginnote{3.1} run back to the past, \\
don’t anticipate the future.\footnote{To “anticipate” (\textit{\textsanskrit{paṭikaṅkhe}}) is not always bad. A mendicant ought to “anticipate” dangers (\textit{\textsanskrit{pāṭikaṅkhitabba}}, \href{https://suttacentral.net/an4.122/en/sujato\#1.1}{AN 4.122:1.1} = \href{https://suttacentral.net/mn67/en/sujato\#14.2}{MN 67:14.2}); \textsanskrit{Nālaka} “anticipates” the arrival of the Buddha (\textit{\textsanskrit{patikkhaṁ}}, \href{https://suttacentral.net/snp3.11/en/sujato\#19.4}{Snp 3.11:19.4}); and an arahant “awaits their time” (\textit{\textsanskrit{kālaṁ} \textsanskrit{kaṅkhati}}, \href{https://suttacentral.net/sn2.29/en/sujato\#15.4}{SN 2.29:15.4}). The verse here has a specific application in meditation, which is explained below. } \\
What’s past is left behind, \\
the future has not arrived; 

and\marginnote{3.5} any present phenomenon\footnote{Here “phenomenon” (\textit{dhamma}) means anything that is directly knowable (see \href{https://suttacentral.net/mn133/en/sujato\#17.9}{MN 133:17.9}). | The two verses run on here, as indicated by the \textit{ca} linking this line with the previous. } \\
you clearly discern in every case.\footnote{For \textit{tattha tattha} as “in every case” see eg. \href{https://suttacentral.net/dn30/en/sujato\#2.1.3}{DN 30:2.1.3}, and compare \textit{tato tattha vipassati} at \href{https://suttacentral.net/snp5.15/en/sujato\#4.4}{Snp 5.15:4.4}. } \\
The unfaltering, the unshakable:\footnote{At \href{https://suttacentral.net/thag14.1/en/sujato\#5.1}{Thag 14.1:5.1}, this phrase refers to the mind developed in meditation on love and compassion, while at \href{https://suttacentral.net/snp5.19/en/sujato\#19.1}{Snp 5.19:19.1} it refers to Nibbana. In both cases, as here, it is the object of the verb, not an adverb. } \\
having known that, foster it. 

Today’s\marginnote{3.9} the day to keenly work—\\
who knows, tomorrow may bring death! \\
For there is no bargain to be struck \\
with Death and his mighty horde. 

One\marginnote{3.13} who keenly meditates like this, \\
tireless all night and day: \\
that’s who has one fine night—\\
so declares the peaceful sage.\footnote{The “peaceful sage” is the Buddha. } 

%
\end{verse}

And\marginnote{4.1} how do you run back to the past? You muster delight there, thinking: ‘I had such form in the past.’ … ‘I had such feeling … perception … choice … consciousness in the past.’\footnote{\textit{\textsanskrit{Samanvāneti}} is unique in early Pali, but the similar \textit{\textsanskrit{samānayi}} is found at \href{https://suttacentral.net/dn30/en/sujato\#1.33.4}{DN 30:1.33.4} and \href{https://suttacentral.net/thig12.1/en/sujato\#10.2}{Thig 12.1:10.2} in the sense of “reunite, bring together”. “Muster” means both to bring together and to arouse. } That’s how you run back to the past. 

And\marginnote{5.1} how do you not run back to the past? You don’t muster delight there, thinking: ‘I had such form in the past.’ … ‘I had such feeling … perception … choice … consciousness in the past.’ That’s how you don’t run back to the past. 

And\marginnote{6.1} how do you anticipate the future? You muster delight there, thinking: ‘May I have such form in the future.’ … ‘May I have such feeling … perception … choice … consciousness in the future.’ That’s how you anticipate the future. 

And\marginnote{7.1} how do you not anticipate the future? You don’t muster delight there, thinking: ‘May I have such form in the future.’ … ‘May I have such feeling … perception … choice … consciousness in the future.’ That’s how you don’t anticipate the future. 

And\marginnote{8.1} how do you falter amid presently arisen phenomena?\footnote{\textit{Dhammesu} (“phenomena”) is plural here, whereas in the verse it is singular. Sanskrit fragments, however, have plural in the verse as well. } It’s when an unlearned ordinary person has not seen the noble ones, and is neither skilled nor trained in the teaching of the noble ones. They’ve not seen true persons, and are neither skilled nor trained in the teaching of the true persons. They regard form as self, self as having form, form in self, or self in form. They regard feeling … perception … choices … consciousness as self, self as having consciousness, consciousness in self, or self in consciousness. That’s how you falter amid presently arisen phenomena. 

And\marginnote{9.1} how do you not falter amid presently arisen phenomena? It’s when a learned noble disciple has seen the noble ones, and is skilled and trained in the teaching of the noble ones. They’ve seen true persons, and are skilled and trained in the teaching of the true persons. They don’t regard form as self, self as having form, form in self, or self in form. They don’t regard feeling … perception … choices … consciousness as self, self as having consciousness, consciousness in self, or self in consciousness. That’s how you don’t falter amid presently arisen phenomena. 

\begin{verse}%
‘Don’t\marginnote{10.1} run back to the past, \\
don’t anticipate the future. \\
What’s past is left behind, \\
the future has not arrived; 

and\marginnote{10.5} any present phenomenon \\
you clearly discern in every case. \\
The unfaltering, the unshakable: \\
having known that, foster it. 

Today’s\marginnote{10.9} the day to keenly work—\\
who knows, tomorrow may bring death! \\
For there is no bargain to be struck \\
with Death and his mighty horde. 

One\marginnote{10.13} who keenly meditates like this, \\
tireless all night and day: \\
that’s who has one fine night—\\
so declares the peaceful sage.’ 

%
\end{verse}

And\marginnote{11.1} that’s what I meant when I said: ‘I shall teach you the summary recital and the analysis of the one who has one fine night.’” 

That\marginnote{11.3} is what the Buddha said. Satisfied, the mendicants approved what the Buddha said. 

%
\section*{{\suttatitleacronym MN 132}{\suttatitletranslation Ānanda and One Fine Night }{\suttatitleroot Ānandabhaddekarattasutta}}
\addcontentsline{toc}{section}{\tocacronym{MN 132} \toctranslation{Ānanda and One Fine Night } \tocroot{Ānandabhaddekarattasutta}}
\markboth{Ānanda and One Fine Night }{Ānandabhaddekarattasutta}
\extramarks{MN 132}{MN 132}

\scevam{So\marginnote{1.1} I have heard. }At one time the Buddha was staying near \textsanskrit{Sāvatthī} in Jeta’s Grove, \textsanskrit{Anāthapiṇḍika}’s monastery. 

Now\marginnote{2.1} at that time Venerable Ānanda was educating, encouraging, firing up, and inspiring the mendicants in the assembly hall with a Dhamma talk on the topic of the summary recital and the analysis of the one who has one fine night. 

Then\marginnote{2.2} in the late afternoon, the Buddha came out of retreat, went to the assembly hall, where he sat on the seat spread out, and addressed the mendicants, “Who was inspiring the mendicants with a talk on the summary recital and the analysis of the one who has one fine night?” 

“It\marginnote{2.5} was Venerable Ānanda, sir.” 

Then\marginnote{2.6} the Buddha said to Venerable Ānanda, “But in what way were you inspiring the mendicants with a talk on the summary recital and the analysis of the one who has one fine night?” 

“I\marginnote{3.1} was doing so in this way, sir,” replied Ānanda. 

\begin{verse}%
“Don’t\marginnote{3.2} run back to the past, \\
don’t anticipate the future. \\
What’s past is left behind, \\
the future has not arrived; 

and\marginnote{3.6} any present phenomenon \\
you clearly discern in every case. \\
The unfaltering, the unshakable: \\
having known that, foster it. 

Today’s\marginnote{3.10} the day to keenly work—\\
who knows, tomorrow may bring death! \\
For there is no bargain to be struck \\
with Death and his mighty horde. 

One\marginnote{3.14} who keenly meditates like this, \\
tireless all night and day: \\
that’s who has one fine night—\\
so declares the peaceful sage. 

%
\end{verse}

And\marginnote{4.1} how do you run back to the past? …\footnote{Ānanda goes on to repeat the analysis as in the previous discourse, \href{https://suttacentral.net/mn131/en/sujato\#4.1}{MN 131:4.1}ff. } 

And\marginnote{5.1} how do you not run back to the past? … 

And\marginnote{6.1} how do you anticipate the future? … 

And\marginnote{7.1} how do you not anticipate the future? … 

And\marginnote{8.1} how do you falter amid presently arisen phenomena? … 

And\marginnote{9.1} how do you not falter amid presently arisen phenomena? … That’s how you don’t falter amid presently arisen phenomena. 

\begin{verse}%
‘Don’t\marginnote{10.1} run back to the past, \\
don’t anticipate the future. … 

that’s\marginnote{10.13} who has one fine night—\\
so declares the peaceful sage.’ 

%
\end{verse}

That’s\marginnote{11.1} how I was inspiring the mendicants with a talk on the summary recital and the analysis of the one who has one fine night.” 

“Good,\marginnote{11.2} good, Ānanda. It’s good that you were inspiring the mendicants with a talk on the summary recital and the analysis of the one who has one fine night.” 

And\marginnote{13{-}19.1} the Buddha repeated the verses and analysis once more. 

That\marginnote{13{-}19.17} is what the Buddha said. Satisfied, Venerable Ānanda approved what the Buddha said. 

%
\section*{{\suttatitleacronym MN 133}{\suttatitletranslation Mahākaccāna and One Fine Night }{\suttatitleroot Mahākaccānabhaddekarattasutta}}
\addcontentsline{toc}{section}{\tocacronym{MN 133} \toctranslation{Mahākaccāna and One Fine Night } \tocroot{Mahākaccānabhaddekarattasutta}}
\markboth{Mahākaccāna and One Fine Night }{Mahākaccānabhaddekarattasutta}
\extramarks{MN 133}{MN 133}

\scevam{So\marginnote{1.1} I have heard. }At one time the Buddha was staying near \textsanskrit{Rājagaha} in the Hot Springs Monastery.\footnote{The hot springs near \textsanskrit{Rājagaha} were a popular place for monks to bathe, so much so that they prompted a rule ensuring that the monks did not monopolize the springs (\href{https://suttacentral.net/pli-tv-bu-vb-pc57/en/sujato}{Bu Pc 57}). They are still in use and just as popular as ever. } 

Then\marginnote{1.3} Venerable Samiddhi rose at the crack of dawn and went to the hot springs to bathe.\footnote{Soon after he went forth, a deity in the same location tempted Samiddhi to enjoy sensual delights (\href{https://suttacentral.net/sn1.20/en/sujato}{SN 1.20}). Had he succumbed, that would have been a case of “faltering amid presently arisen phenomena”. Samiddhi further appears in discourses at \href{https://suttacentral.net/mn136/en/sujato}{MN 136}, where he had gone forth only three years, and \href{https://suttacentral.net/an9.14 /en/sujato}{AN 9.14 }, where he seems more mature. In \href{https://suttacentral.net/sn4.22/en/sujato}{SN 4.22}, \textsanskrit{Māra} threatens to shake his faith, prompting him to ask the Buddha about \textsanskrit{Māra}’s true nature (\href{https://suttacentral.net/sn35.65/en/sujato}{SN 35.65}). This event is recalled in his verse at \href{https://suttacentral.net/thag1.46/en/sujato}{Thag 1.46}. } When he had bathed and emerged from the water he stood in one robe drying his limbs. 

Then,\marginnote{1.5} late at night, a glorious deity, lighting up the entire hot springs, went up to Samiddhi, stood to one side, and said to Samiddhi: 

“Mendicant,\marginnote{2.1} do you remember the summary recital and the analysis of the one who has one fine night?” 

“No,\marginnote{2.2} reverend, I do not. Do you?” 

“I\marginnote{2.4} also do not. But do you remember just the verses on the one who has one fine night?”\footnote{The “verses” are, in fact, the same thing as the “recitation passage” mentioned above. The Tibetan and Chinese parallels lack the doubled question. } 

“I\marginnote{2.6} do not. Do you?” 

“I\marginnote{2.8} also do not. Learn the summary recital and the analysis of the one who has one fine night, mendicant, memorize it, and remember it. It is beneficial and relates to the fundamentals of the spiritual life.” 

That’s\marginnote{2.13} what that deity said, before vanishing right there. 

Then,\marginnote{3.1} when the night had passed, Samiddhi went to the Buddha, bowed, sat down to one side, and told him what had happened. Then he added: 

“Sir,\marginnote{4.13} please teach me the summary recital and the analysis of the one who has one fine night.” 

“Well\marginnote{4.16} then, mendicant, listen and apply your mind well, I will speak.” 

“Yes,\marginnote{4.17} sir,” Samiddhi replied. The Buddha said this: 

\begin{verse}%
“Don’t\marginnote{5.1} run back to the past, \\
don’t anticipate the future. \\
What’s past is left behind, \\
the future has not arrived; 

and\marginnote{5.5} any present phenomenon \\
you clearly discern in every case. \\
The unfaltering, the unshakable: \\
having known that, foster it. 

Today’s\marginnote{5.9} the day to keenly work—\\
who knows, tomorrow may bring death! \\
For there is no bargain to be struck \\
with Death and his mighty horde. 

One\marginnote{5.13} who keenly meditates like this, \\
tireless all night and day: \\
that’s who has one fine night—\\
so declares the peaceful sage.” 

%
\end{verse}

That\marginnote{6.1} is what the Buddha said. When he had spoken, the Holy One got up from his seat and entered his dwelling.\footnote{Though he set to teach the “recitation passage and analysis”, the Buddha omits the analysis, so the monks ask \textsanskrit{Mahākaccāna} for the analysis. The same situation obtains in \href{https://suttacentral.net/mn138/en/sujato\#4.2}{MN 138:4.2}. In both cases the Chinese and Tibetan parallels avoid the problem, since the Buddha only says he will teach the verses. Contrast the situation at \href{https://suttacentral.net/mn137/en/sujato\#2.1}{MN 137:2.1} and \href{https://suttacentral.net/mn139/en/sujato\#2.1}{MN 139:2.1}. } 

Soon\marginnote{7.1} after the Buddha left, those mendicants considered, “The Buddha gave this brief summary recital, then entered his dwelling without explaining the meaning in detail. … 

Who\marginnote{7.19} can explain in detail the meaning of this brief summary given by the Buddha?” 

Then\marginnote{7.20} those mendicants thought: 

“This\marginnote{7.21} Venerable \textsanskrit{Mahākaccāna} is praised by the Buddha and esteemed by his sensible spiritual companions.\footnote{See note at \href{https://suttacentral.net/mn18/en/sujato\#10.9}{MN 18:10.9}. } He is capable of explaining in detail the meaning of this brief summary recital given by the Buddha. Let’s go to him, and ask him about this matter.” 

Then\marginnote{7.24} those mendicants went to \textsanskrit{Mahākaccāna}, and exchanged greetings with him. When the greetings and polite conversation were over, they sat down to one side. They told him what had happened, and said: 

“May\marginnote{8.1} Venerable \textsanskrit{Mahākaccāna} please explain this.” 

“Reverends,\marginnote{9.1} suppose there was a person in need of heartwood. And while wandering in search of heartwood he’d come across a large tree standing with heartwood. But he’d pass over the roots and trunk, imagining that the heartwood should be sought in the branches and leaves. Such is the consequence for the venerables. Though you were face to face with the Buddha, you overlooked him, imagining that you should ask me about this matter. For he is the Buddha, the one who knows and sees. He is vision, he is knowledge, he is the manifestation of principle, he is the manifestation of divinity. He is the teacher, the proclaimer, the elucidator of meaning, the bestower of freedom from death, the lord of truth, the Realized One. That was the time to approach the Buddha and ask about this matter. You should have remembered it in line with the Buddha’s answer.” 

“Certainly\marginnote{10.1} he is the Buddha, the one who knows and sees. He is vision, he is knowledge, he is the manifestation of principle, he is the manifestation of divinity. He is the teacher, the proclaimer, the elucidator of meaning, the bestower of freedom from death, the lord of truth, the Realized One. That was the time to approach the Buddha and ask about this matter. We should have remembered it in line with the Buddha’s answer. Still, Venerable \textsanskrit{Mahākaccāna} is praised by the Buddha and esteemed by his sensible spiritual companions. He is capable of explaining in detail the meaning of this brief summary recital given by the Buddha. Please explain this, if it’s no trouble.” 

“Well\marginnote{11.1} then, reverends, listen and apply your mind well, I will speak.” 

“Yes,\marginnote{11.2} reverend,” they replied. Venerable \textsanskrit{Mahākaccāna} said this: 

“Reverends,\marginnote{12.1} the Buddha gave this brief summary recital, then entered his dwelling without explaining the meaning in detail: 

\begin{verse}%
‘Don’t\marginnote{12.2} run back to the past … \\
that’s who has one fine night—\\
so declares the peaceful sage.’ 

%
\end{verse}

And\marginnote{12.6} this is how I understand the detailed meaning of this summary recital. 

And\marginnote{13.1} how do you run back to the past? Consciousness gets tied up there with desire and lust, thinking: ‘In the past I had such eyes and such sights.’\footnote{As in \href{https://suttacentral.net/mn18/en/sujato\#16.1}{MN 18:16.1} and \href{https://suttacentral.net/mn138/en/sujato\#10.2}{MN 138:10.2}, \textsanskrit{Kaccāna}’s favored mode of analysis is through the six senses. } So you take pleasure in that, and that’s when you run back to the past. 

Consciousness\marginnote{13.4} gets tied up there with desire and lust, thinking: ‘In the past I had such ears and such sounds … such a nose and such smells … such a tongue and such tastes … such a body and such touches … such a mind and such ideas.’ So you take pleasure in that, and that’s when you run back to the past. That’s how you run back to the past. 

And\marginnote{14.1} how do you not run back to the past? Consciousness doesn’t get tied up there with desire and lust, thinking: ‘In the past I had such eyes and such sights.’ So you don’t take pleasure in that, and that’s when you no longer run back to the past. 

Consciousness\marginnote{14.4} doesn’t get tied up there with desire and lust, thinking: ‘In the past I had such ears and such sounds … such a nose and such smells … such a tongue and such tastes … such a body and such touches … such a mind and such ideas.’ So you don’t take pleasure in that, and that’s when you no longer run back to the past. That’s how you don’t run back to the past. 

And\marginnote{15.1} how do you anticipate the future? The heart is set on getting what it does not have, thinking: ‘May I have such eyes and such sights in the future.’ So you take pleasure in that, and that’s when you anticipate the future. The heart is set on getting what it does not have, thinking: ‘May I have such ears and such sounds … such a nose and such smells … such a tongue and such tastes … such a body and such touches … such a mind and such ideas in the future.’ So you take pleasure in that, and that’s when you anticipate the future. That’s how you anticipate the future. 

And\marginnote{16.1} how do you not anticipate the future? The heart is not set on getting what it does not have, thinking: ‘May I have such eyes and such sights in the future.’ So you don’t take pleasure in that, and that’s when you no longer anticipate the future. The heart is not set on getting what it does not have, thinking: ‘May I have such ears and such sounds … such a nose and such smells … such a tongue and such tastes … such a body and such touches … such a mind and such ideas in the future.’ So you don’t take pleasure in that, and that’s when you no longer anticipate the future. That’s how you don’t anticipate the future. 

And\marginnote{17.1} how do you falter amid presently arisen phenomena? Both the eye and sights are presently arisen. If consciousness gets tied up there in the present with desire and lust, you take pleasure in that, and that’s when you falter amid presently arisen phenomena. Both the ear and sounds … nose and smells … tongue and tastes … body and touches … mind and ideas\footnote{\textit{Dhamma} is rendered here with “ideas”, namely those things which are known by the mind. Where it is rendered “phenomena”, it includes the “ideas” as well as all else that is known. } are presently arisen. If consciousness gets tied up there in the present with desire and lust, you take pleasure in that, and that’s when you falter amid presently arisen phenomena. That’s how you falter amid presently arisen phenomena. 

And\marginnote{18.1} how do you not falter amid presently arisen phenomena? Both the eye and sights are presently arisen. If consciousness doesn’t get tied up there in the present with desire and lust, you don’t take pleasure in that, and that’s when you no longer falter amid presently arisen phenomena. Both the ear and sounds … nose and smells … tongue and tastes … body and touches … mind and ideas are presently arisen. If consciousness doesn’t get tied up there in the present with desire and lust, you don’t take pleasure in that, and that’s when you no longer falter amid presently arisen phenomena. That’s how you don’t falter amid presently arisen phenomena. 

This\marginnote{19.1} is how I understand the detailed meaning of that brief summary recital given by the Buddha. 

If\marginnote{19.6} you wish, you may go to the Buddha and ask him about this. You should remember it in line with the Buddha’s answer.” 

Then\marginnote{20.1} those mendicants, approving and agreeing with what \textsanskrit{Mahākaccāna} said, rose from their seats and went to the Buddha, bowed, sat down to one side, and told him what had happened, adding: 

“\textsanskrit{Mahākaccāna}\marginnote{20.25} clearly explained the meaning to us in this manner, with these words and phrases.” 

“\textsanskrit{Mahākaccāna}\marginnote{21.1} is astute, mendicants, he has great wisdom. If you came to me and asked this question, I would answer it in exactly the same way as \textsanskrit{Mahākaccāna}. That is what it means, and that’s how you should remember it.” 

That\marginnote{21.4} is what the Buddha said. Satisfied, the mendicants approved what the Buddha said. 

%
\section*{{\suttatitleacronym MN 134}{\suttatitletranslation Lomasakaṅgiya and One Fine Night }{\suttatitleroot Lomasakaṅgiyabhaddekarattasutta}}
\addcontentsline{toc}{section}{\tocacronym{MN 134} \toctranslation{Lomasakaṅgiya and One Fine Night } \tocroot{Lomasakaṅgiyabhaddekarattasutta}}
\markboth{Lomasakaṅgiya and One Fine Night }{Lomasakaṅgiyabhaddekarattasutta}
\extramarks{MN 134}{MN 134}

\scevam{So\marginnote{1.1} I have heard. }At one time the Buddha was staying near \textsanskrit{Sāvatthī} in Jeta’s Grove, \textsanskrit{Anāthapiṇḍika}’s monastery. 

Now\marginnote{1.3} at that time Venerable \textsanskrit{Lomasakaṅgiya} was staying in the Sakyan country at Kapilavatthu in the Banyan Tree Monastery.\footnote{\textsanskrit{Lomasakaṅgiya}’s verse speaks of his determination to forge his way through the dense jungle in search of seclusion (\href{https://suttacentral.net/thag1.27/en/sujato}{Thag 1.27}). He is probably also the same monk who appears in \href{https://suttacentral.net/sn54.12/en/sujato}{SN 54.12}. His name means “hairy-limbs”. } 

Then,\marginnote{2.1} late at night, the glorious god Candana, lighting up the entire Banyan Tree Monastery, went up to the Venerable \textsanskrit{Lomasakaṅgiya}, and stood to one side. Standing to one side, he said to \textsanskrit{Lomasakaṅgiya}:\footnote{The god Candana, whose name means “sandalwood”, also appears at \href{https://suttacentral.net/dn20/en/sujato\#10.5}{DN 20:10.5}, \href{https://suttacentral.net/dn32/en/sujato\#10.4}{DN 32:10.4}, and \href{https://suttacentral.net/sn2/en/sujato\#2.15}{SN 2:2.15}. } 

“Mendicant,\marginnote{2.2} do you remember the summary recital and the analysis of the one who has one fine night?” 

“No,\marginnote{2.3} reverend, I do not. Do you?” 

“I\marginnote{2.5} also do not. But do you remember just the verses on the one who has one fine night?” 

“I\marginnote{2.7} do not. Do you?” 

“I\marginnote{2.9} do.” 

“How\marginnote{2.10} do you remember the verses on the one who has one fine night?” 

“This\marginnote{2.11} one time, the Buddha was staying among the gods of the thirty-three at the root of the Shady Orchid Tree on the stone spread with a cream rug.\footnote{This suggests that Candana was one of the Thirty-three. In the Chinese parallels, however, he heard this at \textsanskrit{Rājagaha}. | \textit{\textsanskrit{Paṇḍukambala}} is a kind of luxury upholstery. } There he taught the summary recital and the analysis of the one who has one fine night to the gods of the thirty-three: 

\begin{verse}%
‘Don’t\marginnote{3.1} run back to the past, \\
don’t anticipate the future. \\
What’s past is left behind, \\
the future has not arrived; 

and\marginnote{3.5} any present phenomenon \\
you clearly discern in every case. \\
The unfaltering, the unshakable: \\
having known that, foster it. 

Today’s\marginnote{3.9} the day to keenly work—\\
who knows, tomorrow may bring death! \\
For there is no bargain to be struck \\
with Death and his mighty horde. 

One\marginnote{3.13} who keenly meditates like this, \\
tireless all night and day: \\
that’s who has one fine night—\\
so declares the peaceful sage.’ 

%
\end{verse}

That’s\marginnote{4.1} how I remember the verses of the one who has one fine night. Learn the summary recital and the analysis of the one who has one fine night, mendicant, memorize it, and remember it. It is beneficial and relates to the fundamentals of the spiritual life.” 

That’s\marginnote{4.6} what the god Candana said before vanishing right there. 

Then\marginnote{5.1} \textsanskrit{Lomasakaṅgiya} set his lodgings in order and, taking his bowl and robe, set out for \textsanskrit{Sāvatthī}. Eventually he came to \textsanskrit{Sāvatthī} and Jeta’s Grove. He went up to the Buddha, bowed, sat down to one side, and told him what had happened. Then he added: 

“Sir,\marginnote{5.26} please teach me the summary recital and the analysis of the one who has one fine night.” 

“But\marginnote{6.1} mendicant, do you know that god?” 

“I\marginnote{6.2} do not, sir.” 

“That\marginnote{6.3} god was named Candana. Candana pays attention, applies the mind, concentrates wholeheartedly, and actively listens to the teaching.\footnote{Candana is here praised for his skill in learning, yet earlier he says that although he was present when the Buddha taught the “recitation passage and analysis”, he only remembered the verses (\href{https://suttacentral.net/mn134/en/sujato\#2.12}{MN 134:2.12}). As at \href{https://suttacentral.net/mn133/en/sujato\#6.2}{MN 133:6.2}, the parallels are more coherent, as they only mention him hearing the verses. } Well then, mendicant, listen and apply your mind well, I will speak.” 

“Yes,\marginnote{6.6} sir,” \textsanskrit{Lomasakaṅgiya} replied. The Buddha said this: 

\begin{verse}%
“Don’t\marginnote{7.1} run back to the past, \\
don’t anticipate the future. \\
What’s past is left behind, \\
the future has not arrived; 

and\marginnote{7.5} any present phenomenon \\
you clearly discern in every case. \\
The unfaltering, the unshakable: \\
having known that, foster it. 

Today’s\marginnote{7.9} the day to keenly work—\\
who knows, tomorrow may bring death! \\
For there is no bargain to be struck \\
with Death and his mighty horde. 

One\marginnote{7.13} who keenly meditates like this, \\
tireless all night and day: \\
that’s who has one fine night—\\
so declares the peaceful sage. 

%
\end{verse}

And\marginnote{8{-}13.1} how do you run back to the past? … ”\footnote{The Buddha goes on to repeat the analysis as in \href{https://suttacentral.net/mn131/en/sujato\#4.1}{MN 131:4.1}ff. } 

That\marginnote{14.17} is what the Buddha said. Satisfied, Venerable \textsanskrit{Lomasakaṅgiya} approved what the Buddha said. 

%
\section*{{\suttatitleacronym MN 135}{\suttatitletranslation The Shorter Analysis of Deeds }{\suttatitleroot Cūḷakammavibhaṅgasutta}}
\addcontentsline{toc}{section}{\tocacronym{MN 135} \toctranslation{The Shorter Analysis of Deeds } \tocroot{Cūḷakammavibhaṅgasutta}}
\markboth{The Shorter Analysis of Deeds }{Cūḷakammavibhaṅgasutta}
\extramarks{MN 135}{MN 135}

\scevam{So\marginnote{1.1} I have heard.\footnote{This discourse has more parallels than any other Majjhima sutta: six in Chinese, two in Tibetan, two in Sanskrit, and fragments in Khotanese, Sogdian, and Tocharian. Its popularity is due to the detailed yet straightforward way it connects deeds with results, a subject of perennial fascination. } }At one time the Buddha was staying near \textsanskrit{Sāvatthī} in Jeta’s Grove, \textsanskrit{Anāthapiṇḍika}’s monastery. 

Then\marginnote{2.1} the student Subha, Todeyya’s son, approached the Buddha, and exchanged greetings with him. When the greetings and polite conversation were over, he sat down to one side and said to the Buddha: 

“What\marginnote{3.1} is the cause, Mister Gotama, what is the reason why even among those who are human beings some are seen to be inferior and superior? For people are seen who are short-lived and long-lived, sickly and healthy, ugly and beautiful, insignificant and illustrious, poor and rich, from low and eminent families, witless and wise. What is the reason why even among those who are human beings some are seen to be inferior and superior?” 

“Student,\marginnote{4.1} sentient beings are the owners of their deeds and heir to their deeds. Deeds are their womb, their relative, and their refuge.\footnote{This phrase is also explained at \href{https://suttacentral.net/an10.216/en/sujato}{AN 10.216}. In \href{https://suttacentral.net/an5.57/en/sujato}{AN 5.57} it is one of five sayings that should be frequently recollected by “a woman or a man, a layperson or a renunciate”. } It is deeds that divide beings into inferior and superior.” 

“I\marginnote{4.4} don’t understand the meaning of what Mister Gotama has said in brief, without explaining the details. Mister Gotama, please teach me this matter in detail so I can understand the meaning.” 

“Well\marginnote{4.6} then, student, listen and apply your mind well, I will speak.” 

“Yes,\marginnote{4.7} sir,” replied Subha. The Buddha said this: 

“Take\marginnote{5.1} some woman or man who kills living creatures. They’re violent, bloody-handed, a hardened killer, merciless to living beings.\footnote{The phrase “some woman or man” is used only here and in a few suttas that also deal with kamma (\href{https://suttacentral.net/an3.70/en/sujato}{AN 3.70}, \href{https://suttacentral.net/an8.42/en/sujato}{AN 8.42}, \href{https://suttacentral.net/an8.43/en/sujato}{AN 8.43}, \href{https://suttacentral.net/an8.45/en/sujato}{AN 8.45}). The mention of “woman or man” undercuts assumptions that women lack agency, or that their rebirth is lesser. } Because of undertaking such deeds, when their body breaks up, after death, they’re reborn in a place of loss, a bad place, the underworld, hell. If they’re not reborn in a place of loss, but return to the human realm, then wherever they’re reborn they’re short-lived.\footnote{Rebirth in the human realm is always due to good deeds in the past, but while in this realm, bad kamma can take effect. Of course, not all afflictions are caused by past kamma, a doctrine of the Jains that the Buddha rejected. } For killing living creatures is the path leading to a short lifespan. 

But\marginnote{6.1} take some woman or man who gives up killing living creatures. They renounce the rod and the sword. They’re scrupulous and kind, living full of sympathy for all living beings. Because of undertaking such deeds, when their body breaks up, after death, they’re reborn in a good place, a heavenly realm. If they’re not reborn in a heavenly realm, but return to the human realm, then wherever they’re reborn they’re long-lived. For not killing living creatures is the path leading to a long lifespan. 

Take\marginnote{7.1} some woman or man who habitually hurts living creatures with a fist, stone, rod, or sword. Because of undertaking such deeds, after death they’re reborn in a place of loss … or if they return to the human realm, they’re sickly … 

But\marginnote{8.1} take some woman or man who would never hurt living creatures with a fist, stone, rod, or sword.\footnote{A reason for one of the Buddha’s marks (\href{https://suttacentral.net/dn30/en/sujato\#2.7.2}{DN 30:2.7.2}). } Because of undertaking such deeds, after death they’re reborn in a heavenly realm … or if they return to the human realm, they’re healthy … 

Take\marginnote{9.1} some woman or man who is irritable and bad-tempered. Even when lightly criticized they lose their temper, becoming annoyed, hostile, and hard-hearted, and displaying annoyance, hate, and bitterness.\footnote{A person with a “mind like an open sore” (\href{https://suttacentral.net/an3.25/en/sujato\#1.6}{AN 3.25:1.6}). } Because of undertaking such deeds, after death they’re reborn in a place of loss … or if they return to the human realm, they’re ugly … 

But\marginnote{10.1} take some woman or man who isn’t irritable and bad-tempered. Even when heavily criticized, they don’t lose their temper, become annoyed, hostile, and hard-hearted, or display annoyance, hate, and bitterness. Because of undertaking such deeds, after death they’re reborn in a heavenly realm … or if they return to the human realm, they’re lovely …\footnote{The text switches from \textit{\textsanskrit{vaṇṇavanta}} (“beautiful”, \href{https://suttacentral.net/mn135/en/sujato\#3.4}{MN 135:3.4}) to \textit{\textsanskrit{pāsādika}} (“lovely”). } 

Take\marginnote{11.1} some woman or man who is jealous. They envy, resent, and begrudge the possessions, honor, respect, reverence, homage, and veneration given to others.\footnote{Also at \href{https://suttacentral.net/an4.197/en/sujato\#6.5}{AN 4.197:6.5}. } Because of undertaking such deeds, after death they’re reborn in a place of loss … or if they return to the human realm, they’re insignificant … 

But\marginnote{12.1} take some woman or man who is not jealous … Because of undertaking such deeds, after death they’re reborn in a heavenly realm … or if they return to the human realm, they’re illustrious … 

Take\marginnote{13.1} some woman or man who doesn’t give to ascetics or brahmins such things as food, drink, clothing, vehicles; garlands, fragrance, and makeup; and bed, house, and lighting.\footnote{Also in \href{https://suttacentral.net/an4.197/en/sujato}{AN 4.197} and \href{https://suttacentral.net/an10.177/en/sujato}{AN 10.177}. } Because of undertaking such deeds, after death they’re reborn in a place of loss … or if they return to the human realm, they’re poor … 

But\marginnote{14.1} take some woman or man who does give to ascetics or brahmins … Because of undertaking such deeds, after death they’re reborn in a heavenly realm … or if they return to the human realm, they’re rich … 

Take\marginnote{15.1} some woman or man who is obstinate and arrogant. They don’t bow to those they should bow to. They don’t rise up for them, offer them a seat, make way for them, or honor, respect, esteem, or venerate those who are worthy of such.\footnote{The first three of these, phrased a little differently, are found at \href{https://suttacentral.net/pli-tv-pvr7/en/sujato\#51.1}{Pvr 7:51.1}. } Because of undertaking such deeds, after death they’re reborn in a place of loss … or if they return to the human realm, they’re reborn in a low class family … 

But\marginnote{16.1} take some woman or man who is not obstinate and arrogant … Because of undertaking such deeds, after death they’re reborn in a heavenly realm … or if they return to the human realm, they’re reborn in an eminent family … 

Take\marginnote{17.1} some woman or man who doesn’t approach an ascetic or brahmin to ask:\footnote{This is the duty of a wheel-turning monarch (\href{https://suttacentral.net/dn26/en/sujato\#5.6}{DN 26:5.6}), as well as being a reason for one of the Buddha’s marks (\href{https://suttacentral.net/dn30/en/sujato\#1.25.3}{DN 30:1.25.3}). } ‘Sir, what is skillful and what is unskillful? What is blameworthy and what is blameless? What should be cultivated and what should not be cultivated? What kind of action will lead to my lasting harm and suffering? Or what kind of action will lead to my lasting welfare and happiness?’ Because of undertaking such deeds, after death they’re reborn in a place of loss … or if they return to the human realm, they’re witless … 

But\marginnote{18.1} take some woman or man who does approach an ascetic or brahmin to ask: ‘Sir, what is skillful and what is unskillful? What is blameworthy and what is blameless? What should be cultivated and what should not be cultivated? What kind of action will lead to my lasting harm and suffering? Or what kind of action will lead to my lasting welfare and happiness?’ Because of undertaking such deeds, when their body breaks up, after death, they’re reborn in a good place, a heavenly realm. If they’re not reborn in a heavenly realm, but return to the human realm, then wherever they’re reborn they’re very wise. For asking questions of ascetics or brahmins is the path leading to wisdom. 

So\marginnote{19.1} it is the way people live that makes them how they are, whether short-lived or long-lived, sickly or healthy, ugly or lovely, insignificant or illustrious, poor or rich, in a low class or eminent family, or witless or wise. 

Sentient\marginnote{20.1} beings are the owners of their deeds and heir to their deeds. Deeds are their womb, their relative, and their refuge. It is deeds that divide beings into inferior and superior.” 

When\marginnote{21.1} he had spoken, Subha said to him, “Excellent, Mister Gotama! Excellent! As if he were righting the overturned, or revealing the hidden, or pointing out the path to the lost, or lighting a lamp in the dark so people with clear eyes can see what’s there, Mister Gotama has made the Teaching clear in many ways. I go for refuge to Mister Gotama, to the teaching, and to the mendicant \textsanskrit{Saṅgha}.\footnote{He went for refuge first in \href{https://suttacentral.net/mn99/en/sujato\#28.4}{MN 99:28.4} and subsequently in \href{https://suttacentral.net/dn10/en/sujato\#2.37.9}{DN 10:2.37.9}. } From this day forth, may Mister Gotama remember me as a lay follower who has gone for refuge for life.” 

%
\section*{{\suttatitleacronym MN 136}{\suttatitletranslation The Longer Analysis of Deeds }{\suttatitleroot Mahākammavibhaṅgasutta}}
\addcontentsline{toc}{section}{\tocacronym{MN 136} \toctranslation{The Longer Analysis of Deeds } \tocroot{Mahākammavibhaṅgasutta}}
\markboth{The Longer Analysis of Deeds }{Mahākammavibhaṅgasutta}
\extramarks{MN 136}{MN 136}

\scevam{So\marginnote{1.1} I have heard. }At one time the Buddha was staying near \textsanskrit{Rājagaha}, in the Bamboo Grove, the squirrels’ feeding ground. 

Now\marginnote{2.1} at that time Venerable Samiddhi was staying in a wilderness hut. Then as the wanderer Potaliputta was going for a walk he came up to Venerable Samiddhi and exchanged greetings with him.\footnote{This is the only time we meet Potaliputta. His name suggests he hails from the capital of Assaka, variously spelled Potali or Potana (\href{https://suttacentral.net/dn19/en/sujato\#36.4}{DN 19:36.4}). | As for Samiddhi, see note on \href{https://suttacentral.net/mn133/en/sujato\#1.3}{MN 133:1.3}. } When the greetings and polite conversation were over, he sat down to one side and said to him: 

“Reverend\marginnote{2.4} Samiddhi, I have heard and learned this in the presence of the ascetic Gotama: ‘Deeds by way of body and speech are done in vain. Only mental deeds are real.’ And: ‘There is such an attainment where the one who enters it does not feel anything at all.’” 

“Don’t\marginnote{2.7} say that, Reverend Potaliputta, don’t say that! Don’t misrepresent the Buddha, for misrepresentation of the Buddha is not good. And the Buddha would not say this: ‘Deeds by way of body and speech are done in vain. Only mental deeds are real.’\footnote{There is indeed no such statement in the suttas. } But, reverend, there is such an attainment where the one who enters it does not feel anything at all.”\footnote{This is the cessation of perception and feelings. } 

“Reverend\marginnote{2.10} Samiddhi, how long has it been since you went forth?” 

“Not\marginnote{2.11} long, reverend: three years.” 

“Well\marginnote{2.12} now, what are we to say to the senior mendicants, when even such a junior mendicant imagines their Teacher needs defending? After doing an intentional deed by way of body, speech, or mind, reverend, what does one feel?” 

“After\marginnote{2.14} doing an intentional deed by way of body, speech, or mind, reverend, one feels suffering.”\footnote{It is this answer of Samiddhi’s that is criticized below. } Then, neither approving nor dismissing Samiddhi’s statement, Potaliputta got up from his seat and left. 

Soon\marginnote{3.1} after he had left, Venerable Samiddhi went to Venerable Ānanda, and exchanged greetings with him. When the greetings and polite conversation were over, he sat down to one side, and informed Ānanda of all they had discussed. 

When\marginnote{3.4} he had spoken, Ānanda said to him, “Reverend Samiddhi, we should see the Buddha about this matter. Come, let’s go to the Buddha and inform him about this. As he answers, so we’ll remember it.” 

“Yes,\marginnote{3.8} reverend,” Samiddhi replied. 

Then\marginnote{4.1} Ānanda and Samiddhi went up to the Buddha, bowed, sat down to one side, and told him what had happened. 

When\marginnote{5.1} they had spoken, the Buddha said to Ānanda, “I don’t recall even seeing the wanderer Potaliputta, Ānanda, so how could we have had such a discussion? The wanderer Potaliputta’s question should have been answered after analyzing it, but this futile man answered categorically.” 

When\marginnote{6.1} he said this, Venerable \textsanskrit{Udāyī} said to him, “But perhaps, sir, Venerable Samiddhi spoke in reference to the statement: ‘Suffering includes whatever is felt.’” 

But\marginnote{6.4} the Buddha said to Venerable Ānanda, “See what this futile man \textsanskrit{Udāyī} comes up with?\footnote{There are several monks named \textsanskrit{Udāyī} and it is not easy to distinguish them. Of of then was known for his stupidity (\href{https://suttacentral.net/an6.29/en/sujato\#3.3}{AN 6.29:3.3}). | \textit{Ummagga} is from \textit{ummujjati}, just below, “uprising, emergence”. } I knew that he was coming up with something irrational. Right from the start Potaliputta asked about the three feelings. Suppose the futile man Samiddhi had answered the wanderer Potaliputta’s question like this: ‘After doing an intentional deed to be experienced as pleasant by way of body, speech, or mind, one feels pleasure. After doing an intentional deed to be experienced as painful by way of body, speech, or mind, one feels pain. After doing an intentional deed to be experienced as neutral by way of body, speech, or mind, one feels neutral.’ Answering in this way, Samiddhi would have rightly answered Potaliputta. 

Still,\marginnote{6.14} who are those foolish and incompetent wanderers of other religions to understand the Realized One’s great analysis of deeds? If you would all listen, Ānanda, I will explain the Realized One’s great analysis of deeds.”\footnote{Text here reads \textit{vibhajantassa}, which appears ungrammatical. This phrase is shared with \href{https://suttacentral.net/an6.62/en/sujato\#5.8}{AN 6.62:5.8}, which is also a case where a junior monk has made a silly statement. There the Pali has \textit{\textsanskrit{vibhajissāmi}} (“I will explain”). The commentary does not explicitly comment on this phrase, but it does say that the Buddha thought, “I will make it clear for the mendicant \textsanskrit{Saṅgha}” (\textit{\textsanskrit{bhikkhusaṅghassa} \textsanskrit{pākaṭaṁ} \textsanskrit{karissāmīti}}), which implies that it was reading the future tense. Compare the use of future tense in the stock phrase, “Listen and apply your mind well, I will speak” (\href{https://suttacentral.net/mn138/en/sujato\#2.2}{MN 138:2.2}). } 

“Now\marginnote{7.1} is the time, Blessed One! Now is the time, Holy One! Let the Buddha explain the great analysis of deeds. The mendicants will listen and remember it.” 

“Well\marginnote{7.3} then, Ānanda, listen and apply your mind well, I will speak.” 

“Yes,\marginnote{7.4} sir,” Ānanda replied. The Buddha said this: 

“Ānanda,\marginnote{8.1} these four people are found in the world. What four? It’s when a person here kills living creatures, steals, and commits sexual misconduct. They use speech that’s false, divisive, harsh, or nonsensical. And they’re covetous, malicious, and have wrong view. When their body breaks up, after death, they’re reborn in a place of loss, a bad place, the underworld, hell. 

But\marginnote{8.5} some other person here kills living creatures, steals, and commits sexual misconduct. They use speech that’s false, divisive, harsh, or nonsensical. And they’re covetous, malicious, and have wrong view. When their body breaks up, after death, they’re reborn in a good place, a heavenly realm. 

But\marginnote{8.7} some other person here refrains from killing living creatures, stealing, committing sexual misconduct, or using speech that’s false, divisive, harsh, or nonsensical. And they’re contented, kind-hearted, and have right view. When their body breaks up, after death, they’re reborn in a good place, a heavenly realm. 

But\marginnote{8.9} some other person here refrains from killing living creatures, stealing, committing sexual misconduct, or using speech that’s false, divisive, harsh, or nonsensical. And they’re contented, kind-hearted, and have right view. When their body breaks up, after death, they’re reborn in a place of loss, a bad place, the underworld, hell. 

Now,\marginnote{9.1} some ascetic or brahmin—by dint of keen, resolute, committed, and diligent effort, and right application of mind—experiences an immersion of the heart of such a kind that it gives rise to clairvoyance that is purified and superhuman. With that clairvoyance they see that person\footnote{Meditation is described in the same way at \href{https://suttacentral.net/dn1/en/sujato\#1.31.1}{DN 1:1.31.1}, which is also concerned with those who draw inaccurate inferences from deep meditation. } here who killed living creatures, stole, and committed sexual misconduct; who used speech that’s false, divisive, harsh, or nonsensical; and who was covetous, malicious, and had wrong view. And they see that, when their body breaks up, after death, that person is reborn in a place of loss, a bad place, the underworld, hell. They say: ‘It seems that there is such a thing as bad deeds, and the result of bad conduct.\footnote{The use of \textit{kira} (“seems”) conveys an appropriate degree of caution when framing a hypothesis. But as we shall see, the theorist does not maintain this. } For I saw a person here who killed living creatures … and had wrong view. And when their body broke up, after death, they were reborn in a place of loss, a bad place, the underworld, hell.’\footnote{Having framed their hypothesis, the theorist presents their supporting observations. } They say: ‘It seems that everyone who kills living creatures … and has wrong view is reborn in hell.\footnote{Now the theorist draws their inference, committing two logical fallacies. One is the fallacy of hasty generalization: the fact that \emph{some people} who do bad deeds go to hell does not imply that \emph{all people} who do bad deeds go to hell. The second is the post hoc fallacy: the fact that the bad deed \emph{preceded} rebirth in hell does not imply that the bad deed \emph{caused} rebirth in hell. The Buddha avoids the fallacy of hasty generalization by surveying a much larger, statistically significant scope, said to be ninety-one eons (\href{https://suttacentral.net/mn71/en/sujato\#14.2}{MN 71:14.2}). And he avoids the post hoc fallacy by flipping the causal arrow: if the \emph{presence} of kamma leads to rebirth, then the \emph{absence} of kamma should lead to the ending of rebirth. By practicing to realize the ending of kamma he confirmed this for himself. } Those whose notion is this have the right notion. Those whose notion is otherwise have a wrong notion.’\footnote{Leaving behind caution and going beyond observation, the theorist now asserts that the results of their reasoning are the truth. | Here words from the root \textit{√\textsanskrit{ñā}} are used for both right and wrong ideas. Normally I translate \textit{√\textsanskrit{ñā}} as “knowledge”, but in English one cannot “know” something false. One might \emph{believe} that it is nighttime when it is in fact the day, but one cannot \emph{know} this. Usually \textit{√\textsanskrit{ñā}} does have the positive sense of “true knowledge”, but when dealing with errant theorizers we also find it used for “false knowledge”, in which case I use “notion” (eg. \href{https://suttacentral.net/snp4.13/en/sujato\#17.2}{Snp 4.13:17.2}). } And so they obstinately stick to what they have known, seen, and understood for themselves, insisting that: ‘This is the only truth, anything else is futile.’\footnote{Contemplatives sometimes deprecate logic as inferior to meditative insight. By not exercising the critical skills to evaluate and improve their reasoning, they become especially prone to using poor logic. The brilliance of meditative experiences can be overwhelming and it is easy to be convinced of the truth of one’s inferences, without even recognizing them as inferences. sutta passages such as this show that the interpretation of meditative experiences must be tempered with clear logic. } 

But\marginnote{10.1} some other ascetic or brahmin—by dint of keen, resolute, committed, and diligent effort, and right application of mind—experiences an immersion of the heart of such a kind that it gives rise to clairvoyance that is purified and superhuman. With that clairvoyance they see that person here who killed living creatures … and had wrong view. And they see that that person is reborn in a heavenly realm. They say: ‘It seems that there is no such thing as bad deeds, and the result of bad conduct. For I have seen a person here who killed living creatures … and had wrong view. And I saw that that person was reborn in a heavenly realm.’ They say: ‘It seems that everyone who kills living creatures … and has wrong view is reborn in a heavenly realm. Those whose notion is this have the right notion. Those whose notion is otherwise have a wrong notion.’ And so they obstinately stick to what they have known, seen, and understood for themselves, insisting that: ‘This is the only truth, anything else is futile.’ 

Take\marginnote{11.1} some ascetic or brahmin who with clairvoyance sees a person here who refrained from killing living creatures … and had right view. And they see that that person is reborn in a heavenly realm. They say: ‘It seems that there is such a thing as good deeds, and the result of good conduct. For I have seen a person here who refrained from killing living creatures … and had right view. And I saw that that person was reborn in a heavenly realm.’ They say: ‘It seems that everyone who refrains from killing living creatures … and has right view is reborn in a heavenly realm. Those whose notion is this have the right notion. Those whose notion is otherwise have a wrong notion.’ And so they obstinately stick to what they have known, seen, and understood for themselves, insisting that: ‘This is the only truth, anything else is futile.’ 

Take\marginnote{12.1} some ascetic or brahmin who with clairvoyance sees a person here who refrained from killing living creatures … and had right view. And they see that that person is reborn in hell. They say: ‘It seems that there is no such thing as good deeds, and the result of good conduct. For I have seen a person here who refrained from killing living creatures … and had right view. And I saw that that person was reborn in hell.’ They say: ‘It seems that everyone who refrains from killing living creatures … and has right view is reborn in hell. Those whose notion is this have the right notion. Those whose notion is otherwise have a wrong notion.’ And so they obstinately stick to what they have known, seen, and understood for themselves, insisting that: ‘This is the only truth, anything else is futile.’ 

In\marginnote{13.1} this case, when an ascetic or brahmin says this: ‘It seems that there is such a thing as bad deeds, and the result of bad conduct,’ I grant them that. And when they say: ‘I have seen a person here who killed living creatures … and had wrong view. And after death, they were reborn in hell,’ I also grant them that. But when they say: ‘It seems that everyone who kills living creatures … and has wrong view is reborn in hell,’ I don’t grant them that. And when they say: ‘Those whose notion is this have the right notion. Those whose notion is otherwise have a wrong notion.’ I also don’t grant them that. And when they obstinately stick to what they have known, seen, and understood for themselves, insisting that: ‘This is the only truth, anything else is futile,’ I also don’t grant them that. Why is that? Because the Realized One’s knowledge of the great analysis of deeds is otherwise. 

In\marginnote{14.1} this case, when an ascetic or brahmin says this: ‘It seems that there is no such thing as bad deeds, and the result of bad conduct,’ I don’t grant them that. But when they say: ‘I have seen a person here who killed living creatures … and had wrong view. And I saw that that person was reborn in a heavenly realm,’ I grant them that. But when they say: ‘It seems that everyone who kills living creatures … and has wrong view is reborn in a heavenly realm,’ I don’t grant them that. … Because the Realized One’s knowledge of the great analysis of deeds is otherwise. 

In\marginnote{15.1} this case, when an ascetic or brahmin says this: ‘It seems that there is such a thing as good deeds, and the result of good conduct,’ I grant them that. And when they say: ‘I have seen a person here who refrained from killing living creatures … and had right view. And I saw that that person was reborn in a heavenly realm,’ I grant them that. But when they say: ‘It seems that everyone who refrains from killing living creatures … and has right view is reborn in a heavenly realm,’ I don’t grant them that. … Because the Realized One’s knowledge of the great analysis of deeds is otherwise. 

In\marginnote{16.1} this case, when an ascetic or brahmin says this: ‘It seems that there is no such thing as good deeds, and the result of good conduct,’ I don’t grant them that. But when they say: ‘I have seen a person here who refrained from killing living creatures … and had right view. And after death, they were reborn in hell,’ I grant them that. But when they say: ‘It seems that everyone who refrains from killing living creatures … and has right view is reborn in hell,’ I don’t grant them that. But when they say: ‘Those who know this are right. Those who know something else are wrong,’ I also don’t grant them that. And when they obstinately stick to what they have known, seen, and understood for themselves, insisting that: ‘This is the only truth, anything else is futile,’ I also don’t grant them that. Why is that? Because the Realized One’s knowledge of the great analysis of deeds is otherwise. 

Now,\marginnote{17.1} Ānanda, take the case of the person here who killed living creatures … and had wrong view, and who, when their body breaks up, after death, is reborn in a place of loss, a bad place, the underworld, hell. That bad deed of theirs that is to be experienced as painful was either done previously, or later, or else at the time of death they undertook wrong view.\footnote{The logic of this passage seems to be: “When a good result follows a bad deed, or bad follows good, then they must have done some other fitting deed at another time; and even if a bad result follows a bad deed, or good follows good, it is still possible that the result was caused by some other fitting deed.” But the phrasing of the Pali does not quite say this. I suspect that some nuance has been lost, perhaps due to over-zealous standardization. | For deathbed kamma see \href{https://suttacentral.net/mn97/en/sujato\#38.7}{MN 97:38.7} and note. } And that’s why, when their body breaks up, after death, they’re reborn in a place of loss, a bad place, the underworld, hell. But anyone here who kills living creatures … and has wrong view experiences the result of that in this very life, or in the next life, or in some subsequent period. 

Now,\marginnote{18.1} Ānanda, take the case of the person here who killed living creatures … and had wrong view, and who is reborn in a heavenly realm. That good deed of theirs that is to be experienced as pleasurable was either done previously, or later, or else at the time of death they undertook right view. And that’s why, when their body breaks up, after death, they’re reborn in a good place, a heavenly realm. But anyone here who kills living creatures … and has wrong view experiences the result of that in this very life, or in the next life, or in some subsequent period. 

Now,\marginnote{19.1} Ānanda, take the case of the person here who refrained from killing living creatures … and had right view, and who is reborn in a heavenly realm. That good deed of theirs that is to be experienced as pleasurable was either done previously, or later, or else at the time of death they undertook right view. And that’s why, when their body breaks up, after death, they’re reborn in a good place, a heavenly realm. But anyone here who refrains from killing living creatures … and has right view experiences the result of that in this very life, or in the next life, or in some subsequent period. 

Now,\marginnote{20.1} Ānanda, take the case of the person here who refrained from killing living creatures … and had right view, and who is reborn in hell. That bad deed of theirs that is to be experienced as painful was either done previously, or later, or else at the time of death they undertook wrong view. And that’s why, when their body breaks up, after death, they’re reborn in a place of loss, a bad place, the underworld, hell. But anyone here who refrains from killing living creatures … and has right view experiences the result of that in this very life, or in the next life, or in some subsequent period. 

So,\marginnote{21.1} Ānanda, there are deeds that are ineffective and appear ineffective. There are deeds that are ineffective but appear effective. There are deeds that are effective and appear effective. And there are deeds that are effective but appear ineffective.” 

That\marginnote{21.2} is what the Buddha said. Satisfied, Venerable Ānanda approved what the Buddha said. 

%
\section*{{\suttatitleacronym MN 137}{\suttatitletranslation The Analysis of the Six Sense Fields }{\suttatitleroot Saḷāyatanavibhaṅgasutta}}
\addcontentsline{toc}{section}{\tocacronym{MN 137} \toctranslation{The Analysis of the Six Sense Fields } \tocroot{Saḷāyatanavibhaṅgasutta}}
\markboth{The Analysis of the Six Sense Fields }{Saḷāyatanavibhaṅgasutta}
\extramarks{MN 137}{MN 137}

\scevam{So\marginnote{1.1} I have heard. }At one time the Buddha was staying near \textsanskrit{Sāvatthī} in Jeta’s Grove, \textsanskrit{Anāthapiṇḍika}’s monastery. There the Buddha addressed the mendicants, “Mendicants!” 

“Venerable\marginnote{1.5} sir,” they replied. The Buddha said this: 

“Mendicants,\marginnote{2.1} I shall teach you the analysis of the six sense fields.\footnote{Here and at \href{https://suttacentral.net/mn139/en/sujato\#2.1}{MN 139:2.1} the Buddha only announces that he will teach the analysis, but proceeds to give the recitation passage as well. Contrast with \href{https://suttacentral.net/mn133/en/sujato\#6.2}{MN 133:6.2} and \href{https://suttacentral.net/mn138/en/sujato\#4.2}{MN 138:4.2}. } Listen and apply your mind well, I will speak.” 

“Yes,\marginnote{2.3} sir,” they replied. The Buddha said this: 

“‘The\marginnote{3.1} six interior sense fields should be understood. The six exterior sense fields should be understood. The six classes of consciousness should be understood. The six classes of contact should be understood. The eighteen mental preoccupations should be understood. The thirty-six positions of sentient beings should be understood. Therein, relying on this, give up that. The Noble One cultivates the establishment of mindfulness in three cases, by virtue of which they are a Teacher worthy to instruct a group. Of all meditation tutors, it is he that is called the supreme guide for those who wish to train.’ This is the summary recital for the analysis of the six sense fields. 

‘The\marginnote{4.1} six interior sense fields should be understood.’ That’s what I said, but why did I say it? There are the sense fields of the eye, ear, nose, tongue, body, and mind. ‘The six interior sense fields should be understood.’ That’s what I said, and this is why I said it. 

‘The\marginnote{5.1} six exterior sense fields should be understood.’ That’s what I said, but why did I say it? There are the sense fields of sights, sounds, smells, tastes, touches, and ideas. ‘The six exterior sense fields should be understood.’ That’s what I said, and this is why I said it. 

‘The\marginnote{6.1} six classes of consciousness should be understood.’ That’s what I said, but why did I say it? There are eye, ear, nose, tongue, body, and mind consciousness. ‘The six classes of consciousness should be understood.’ That’s what I said, and this is why I said it. 

‘The\marginnote{7.1} six classes of contact should be understood.’ That’s what I said, but why did I say it? There is contact through the eye, ear, nose, tongue, body, and mind. ‘The six classes of contact should be understood.’ That’s what I said, and this is why I said it. 

‘The\marginnote{8.1} eighteen mental preoccupations should be understood.’\footnote{Also at \href{https://suttacentral.net/mn140/en/sujato\#10.1}{MN 140:10.1} and \href{https://suttacentral.net/an3.61/en/sujato\#12.1}{AN 3.61:12.1}. Outside of this context, \textit{\textsanskrit{upavicāra}} is only found in \href{https://suttacentral.net/an6.52/en/sujato}{AN 6.52}, where women are “preoccupied” with adornments and thieves are “preoccupied” with a hiding place. } That’s what I said, but why did I say it? Seeing a sight with the eye, one is preoccupied with a sight that’s a basis for happiness or sadness or equanimity. Hearing a sound with the ear … Smelling an odor with the nose … Tasting a flavor with the tongue … 

Feeling\marginnote{8.7} a touch with the body … Becoming conscious of an idea with the mind, one is preoccupied with an idea that’s a basis for happiness or sadness or equanimity. So there are six preoccupations with happiness, six preoccupations with sadness, and six preoccupations with equanimity. ‘The eighteen mental preoccupations should be understood.’ That’s what I said, and this is why I said it. 

‘The\marginnote{9.1} thirty-six positions of sentient beings should be understood.’ That’s what I said, but why did I say it? There are six kinds of domestic happiness and six kinds of renunciate happiness. There are six kinds of domestic sadness and six kinds of renunciate sadness. There are six kinds of domestic equanimity and six kinds of renunciate equanimity. 

And\marginnote{10.1} in this context what are the six kinds of domestic happiness? There are sights known by the eye, which are likable, desirable, agreeable, pleasing, connected with the worldly pleasures of the flesh. Happiness arises when you regard it as a gain to obtain such sights, or when you recollect sights you formerly obtained that have passed, ceased, and perished. Such happiness is called domestic happiness. There are sounds known by the ear … Smells known by the nose … Tastes known by the tongue … Touches known by the body … Ideas known by the mind, which are likable, desirable, agreeable, pleasing, connected with the world’s material delights. Happiness arises when you regard it as a gain to obtain such ideas, or when you recollect ideas you formerly obtained that have passed, ceased, and perished. Such happiness is called domestic happiness. These are the six kinds of domestic happiness. 

And\marginnote{11.1} in this context what are the six kinds of renunciate happiness? When you’ve understood the impermanence of sights—their perishing, fading away, and cessation—happiness arises as you truly understand through right understanding that both formerly and now all those sights are impermanent, suffering, and perishable. Such happiness is called renunciate happiness. When you’ve understood the impermanence of sounds … smells … tastes … touches … ideas—their perishing, fading away, and cessation—happiness arises as you truly understand through right understanding that both formerly and now all those ideas are impermanent, suffering, and perishable. Such happiness is called renunciate happiness. These are the six kinds of renunciate happiness. 

And\marginnote{12.1} in this context what are the six kinds of domestic sadness? There are sights known by the eye, which are likable, desirable, agreeable, pleasing, connected with the world’s material delights. Sadness arises when you regard it as a loss to lose such sights, or when you recollect sights you formerly lost that have passed, ceased, and perished. Such sadness is called lay sadness. There are sounds known by the ear … There are smells known by the nose … There are tastes known by the tongue … There are touches known by the body … There are ideas known by the mind, which are likable, desirable, agreeable, pleasing, connected with the worldly pleasures of the flesh. Sadness arises when you regard it as a loss to lose such ideas, or when you recollect ideas you formerly lost that have passed, ceased, and perished. Such sadness is called domestic sadness. These are the six kinds of domestic sadness. 

And\marginnote{13.1} in this context what are the six kinds of renunciate sadness? When you’ve understood the impermanence of sights—their perishing, fading away, and cessation—you truly understand through right understanding that both formerly and now all those sights are impermanent, suffering, and perishable. Upon seeing this, you give rise to yearning for the supreme liberations: ‘Oh, when will I enter and remain in the same dimension that the noble ones enter and remain in today?’ When you give rise to yearning for the supreme liberations like this, sadness arises because of the yearning.\footnote{Also at \href{https://suttacentral.net/mn44/en/sujato\#28.6}{MN 44:28.6}. That “dimension” (\textit{\textsanskrit{āyatana}}) is Nibbana (\href{https://suttacentral.net/ud8.1/en/sujato\#3.1}{Ud 8.1:3.1}, \href{https://suttacentral.net/sn35.117/en/sujato\#8.2}{SN 35.117:8.2}). } Such sadness is called renunciate sadness. When you’ve understood the impermanence of sounds … smells … tastes … touches … ideas—their perishing, fading away, and cessation—you truly understand through right understanding that both formerly and now all those ideas are impermanent, suffering, and perishable. Upon seeing this, you give rise to yearning for the supreme liberations: ‘Oh, when will I enter and remain in the same dimension that the noble ones enter and remain in today?’ When you give rise to yearning for the supreme liberations like this, sadness arises because of the yearning. Such sadness is called renunciate sadness. These are the six kinds of renunciate sadness. 

And\marginnote{14.1} in this context what are the six kinds of domestic equanimity? When seeing a sight with the eye, equanimity arises for the unlearned ordinary person—a foolish ordinary person who has not overcome their limitations and the results of deeds, and is blind to the drawbacks. Such equanimity does not transcend the sight. That’s why it’s called domestic equanimity. When hearing a sound with the ear … When smelling an odor with the nose … When tasting a flavor with the tongue … When feeling a touch with the body … When knowing an idea with the mind, equanimity arises for the unlearned ordinary person—a foolish ordinary person who has not overcome their limitations and the results of deeds, and is blind to the drawbacks. Such equanimity does not transcend the idea. That’s why it’s called domestic equanimity. These are the six kinds of domestic equanimity. 

And\marginnote{15.1} in this context what are the six kinds of renunciate equanimity? When you’ve understood the impermanence of sights—their perishing, fading away, and cessation—equanimity arises as you truly understand through right understanding that both formerly and now all those sights are impermanent, suffering, and perishable. Such equanimity transcends the sight. That’s why it’s called renunciate equanimity. When you’ve understood the impermanence of sounds … smells … tastes … touches … ideas—their perishing, fading away, and cessation—equanimity arises as you truly understand through right understanding that both formerly and now all those ideas are impermanent, suffering, and perishable. Such equanimity transcends the idea. That’s why it’s called renunciate equanimity. These are the six kinds of renunciate equanimity. ‘The thirty-six positions of sentient beings should be understood.’ That’s what I said, and this is why I said it. 

‘Therein,\marginnote{16.1} relying on this, give up that.’ That’s what I said, but why did I say it? 

Therein,\marginnote{16.3} by relying and depending on the six kinds of renunciate happiness, give up and go beyond the six kinds of domestic happiness. That’s how they are given up. 

Therein,\marginnote{16.5} by relying on the six kinds of renunciate sadness, give up the six kinds of domestic sadness. That’s how they are given up. 

Therein,\marginnote{16.7} by relying on the six kinds of renunciate equanimity, give up the six kinds of domestic equanimity. That’s how they are given up. 

Therein,\marginnote{16.9} by relying on the six kinds of renunciate happiness, give up the six kinds of renunciate sadness. That’s how they are given up. 

Therein,\marginnote{16.11} by relying on the six kinds of renunciate equanimity, give up the six kinds of renunciate happiness. That’s how they are given up. 

There\marginnote{17.1} is equanimity that is diversified, based on diversity, and equanimity that is unified, based on unity. 

And\marginnote{18.1} what is equanimity based on diversity? There is equanimity towards sights, sounds, smells, tastes, and touches. This is equanimity based on diversity. 

And\marginnote{19.1} what is equanimity based on unity? There is equanimity based on the dimensions of infinite space, infinite consciousness, nothingness, and neither perception nor non-perception. This is equanimity based on unity. 

Therein,\marginnote{20.1} relying on equanimity based on unity, give up equanimity based on diversity. That’s how it is given up. 

Relying\marginnote{20.3} on not being determined by that, give up equanimity based on unity. That’s how it is given up. ‘Therein, relying on this, give up that.’ That’s what I said, and this is why I said it. 

‘The\marginnote{21.1} Noble One cultivates the establishment of mindfulness in three cases, by virtue of which they are a Teacher worthy to instruct a group.’\footnote{The text has a unique teaching of three \textit{\textsanskrit{satipaṭṭhānas}} (“establishments of mindfulness”). } That’s what I said, but why did I say it? 

The\marginnote{22.1} first case is when the Teacher teaches Dhamma to his disciples out of kindness and sympathy: ‘This is for your welfare. This is for your happiness.’ But their disciples don’t want to listen. They don’t actively listen or try to understand. They proceed having turned away from the Teacher’s instruction. In this case the Realized One is not unhappy, he does not feel unhappiness. He remains unfestering, mindful and aware.\footnote{In the Pali manuscripts and commentary, the Burmese manuscripts appear to be alone in having the negative form \textit{anattamana} (“unhappy”). The Chinese parallel, however, supports the negative form (MA 163 at T i 693c29). | The commentarial derivation of \textit{attamana} as “self-mind” (= \textit{sakamana}, “own-mind”) is unsatisfactory. Read with Sanskrit \textit{\textsanskrit{āttamanas}} or \textit{\textsanskrit{ādattamanas}} (“uplifted mind”, “smiling mind”). } This is the first case in which the Noble One cultivates the establishment of mindfulness. 

The\marginnote{23.1} next case is when the Teacher teaches Dhamma to his disciples out of kindness and sympathy: ‘This is for your welfare. This is for your happiness.’ And some of their disciples don’t want to listen. They don’t actively listen or try to understand. They proceed having turned away from the Teacher’s instruction. But some of their disciples do want to listen. They actively listen and try to understand. They don’t proceed having turned away from the Teacher’s instruction. In this case the Realized One is not unhappy, nor is he happy. Rejecting both unhappiness and happiness, he remains equanimous, mindful and aware. This is the second case in which the Noble One cultivates the establishment of mindfulness. 

The\marginnote{24.1} next case is when the Teacher teaches Dhamma to his disciples out of kindness and sympathy: ‘This is for your welfare. This is for your happiness.’ And their disciples want to listen. They actively listen and try to understand. They don’t proceed having turned away from the Teacher’s instruction. In this case the Realized One is happy, he does feel happiness. He remains unfestering, mindful and aware. This is the third case in which the Noble One cultivates the establishment of mindfulness. ‘The Noble One cultivates the establishment of mindfulness in three cases, by virtue of which they are a Teacher worthy to instruct a group.’ That’s what I said, and this is why I said it. 

‘Of\marginnote{25.1} all meditation tutors, it is he that is called the supreme guide for those who wish to train.’ That’s what I said, but why did I say it? Driven by an elephant trainer, an elephant in training proceeds in just one direction:\footnote{It seems that the text is punning \textit{\textsanskrit{sārita}} (“driving”) with \textit{\textsanskrit{sārathi}} (“charioteer”). } east, west, north, or south. 

Driven\marginnote{26.1} by a horse trainer, a horse in training proceeds in just one direction: east, west, north, or south. Driven by an ox trainer, an ox in training proceeds in just one direction: east, west, north, or south. But driven by the Realized One, the perfected one, the fully awakened Buddha, a person in training proceeds in eight directions:\footnote{Here follow the eight liberations, which show the course of practice for realizing the highest form of “equanimity based on unity”. } 

Having\marginnote{27.1} physical form, they see forms. This is the first direction. Not perceiving physical form internally, they see forms externally. This is the second direction. They’re focused only on beauty. This is the third direction. Going totally beyond perceptions of form, with the ending of perceptions of impingement, not focusing on perceptions of diversity, aware that ‘space is infinite’, they enter and remain in the dimension of infinite space. This is the fourth direction. Going totally beyond the dimension of infinite space, aware that ‘consciousness is infinite’, they enter and remain in the dimension of infinite consciousness. This is the fifth direction. Going totally beyond the dimension of infinite consciousness, aware that ‘there is nothing at all’, they enter and remain in the dimension of nothingness. This is the sixth direction. Going totally beyond the dimension of nothingness, they enter and remain in the dimension of neither perception nor non-perception. This is the seventh direction. Going totally beyond the dimension of neither perception nor non-perception, they enter and remain in the cessation of perception and feeling. This is the eighth direction. Driven by the Realized One, the perfected one, the fully awakened Buddha, a person in training proceeds in these eight directions. 

‘Of\marginnote{28.1} all meditation tutors, it is he that is called the supreme guide for those who wish to train.’ That’s what I said, and this is why I said it.” 

That\marginnote{28.3} is what the Buddha said. Satisfied, the mendicants approved what the Buddha said. 

%
\section*{{\suttatitleacronym MN 138}{\suttatitletranslation A Summary Recital and its Analysis }{\suttatitleroot Uddesavibhaṅgasutta}}
\addcontentsline{toc}{section}{\tocacronym{MN 138} \toctranslation{A Summary Recital and its Analysis } \tocroot{Uddesavibhaṅgasutta}}
\markboth{A Summary Recital and its Analysis }{Uddesavibhaṅgasutta}
\extramarks{MN 138}{MN 138}

\scevam{So\marginnote{1.1} I have heard. }At one time the Buddha was staying near \textsanskrit{Sāvatthī} in Jeta’s Grove, \textsanskrit{Anāthapiṇḍika}’s monastery. There the Buddha addressed the mendicants, “Mendicants!” 

“Venerable\marginnote{1.5} sir,” they replied. The Buddha said this: 

“Mendicants,\marginnote{2.1} I shall teach you a summary recital and its analysis. Listen and apply your mind well, I will speak.” 

“Yes,\marginnote{2.3} sir,” they replied. The Buddha said this: 

“A\marginnote{3.1} mendicant should thoroughly examine such that for the examiner, if consciousness were not scattered and diffused externally, nor stuck internally, it would not be anxious because of grasping.\footnote{Also at \href{https://suttacentral.net/iti94/en/sujato\#2.1}{Iti 94:2.1}. | For \textit{\textsanskrit{vikkhittaṁ} \textsanskrit{visataṁ}}, see \href{https://suttacentral.net/an5.51/en/sujato\#3.4}{AN 5.51:3.4}, where a river is “dispersed and displaced”; \href{https://suttacentral.net/an4.246/en/sujato\#3.4}{AN 4.246:3.4}, where a lion ensures that his limbs are not “disordered or displaced”;  \href{https://suttacentral.net/sn51.20/en/sujato\#6.2}{SN 51.20:6.2}. At \href{https://suttacentral.net/sn51.20/en/sujato}{SN 51.20} the four bases of psychic power are “scattered and diffused” due to sensual pleasures. | \textit{\textsanskrit{Asaṇṭhita}} in the sense “not stuck” is also found in the phrase “not stuck in the formless” at \href{https://suttacentral.net/snp3.12/en/sujato\#49.2}{Snp 3.12:49.2} = \href{https://suttacentral.net/iti51/en/sujato\#3.2}{Iti 51:3.2} = \href{https://suttacentral.net/iti73/en/sujato\#4.2}{Iti 73:4.2}. See also \href{https://suttacentral.net/sn22.70/en/sujato\#1.4}{SN 22.70:1.4}, and the phrase “when I stood still, I went under” at \href{https://suttacentral.net/sn1.1/en/sujato\#1.7}{SN 1.1:1.7}. } When this is the case, there is no coming to be of the origin of suffering—of rebirth, old age, and death in the future.” 

That\marginnote{4.1} is what the Buddha said. When he had spoken, the Holy One got up from his seat and entered his dwelling.\footnote{See note at \href{https://suttacentral.net/mn133/en/sujato\#6.2}{MN 133:6.2}. } 

Soon\marginnote{5.1} after the Buddha left, those mendicants considered, “The Buddha gave this brief summary recital, then entered his dwelling without explaining the meaning in detail. Who can explain in detail the meaning of this brief summary recital given by the Buddha?” 

Then\marginnote{5.6} those mendicants thought, “This Venerable \textsanskrit{Mahākaccāna} is praised by the Buddha and esteemed by his sensible spiritual companions.\footnote{See note at \href{https://suttacentral.net/mn18/en/sujato\#10.9}{MN 18:10.9}. } He is capable of explaining in detail the meaning of this brief summary recital given by the Buddha. Let’s go to him, and ask him about this matter.” 

Then\marginnote{6.1} those mendicants went to \textsanskrit{Mahākaccāna}, and exchanged greetings with him. When the greetings and polite conversation were over, they sat down to one side. They told him what had happened, and said, “May Venerable \textsanskrit{Mahākaccāna} please explain this.” 

“Reverends,\marginnote{7.1} suppose there was a person in need of heartwood. And while wandering in search of heartwood he’d come across a large tree standing with heartwood. But he’d pass over the roots and trunk, imagining that the heartwood should be sought in the branches and leaves. Such is the consequence for the venerables. Though you were face to face with the Buddha, you overlooked him, imagining that you should ask me about this matter. For he is the Buddha, the one who knows and sees. He is vision, he is knowledge, he is the manifestation of principle, he is the manifestation of divinity. He is the teacher, the proclaimer, the elucidator of meaning, the bestower of freedom from death, the lord of truth, the Realized One. That was the time to approach the Buddha and ask about this matter. You should have remembered it in line with the Buddha’s answer.” 

“Certainly\marginnote{8.1} he is the Buddha, the one who knows and sees. He is vision, he is knowledge, he is the manifestation of principle, he is the manifestation of divinity. He is the teacher, the proclaimer, the elucidator of meaning, the bestower of freedom from death, the lord of truth, the Realized One. That was the time to approach the Buddha and ask about this matter. We should have remembered it in line with the Buddha’s answer. Still, Venerable \textsanskrit{Mahākaccāna} is praised by the Buddha and esteemed by his sensible spiritual companions. He is capable of explaining in detail the meaning of this brief summary recital given by the Buddha. Please explain this, if it’s no trouble.” 

“Well\marginnote{9.1} then, reverends, listen and apply your mind well, I will speak.” 

“Yes,\marginnote{9.2} reverend,” they replied. Venerable \textsanskrit{Mahākaccāna} said this: 

“Reverends,\marginnote{9.4} the Buddha gave this brief summary recital, then entered his dwelling without explaining the meaning in detail: ‘A mendicant should thoroughly examine such that for the examiner, if consciousness were not scattered and diffused externally, nor stuck internally, it would not be anxious because of grasping. When this is the case, there is no coming to be of the origin of suffering—of rebirth, old age, and death in the future.’ This is how I understand the detailed meaning of this summary recital. 

How\marginnote{10.1} is consciousness said to be scattered and diffused externally? In this case, when a mendicant sees a sight with their eyes, and consciousness follows after the features of that sight—tied, attached, and fettered to gratification in its features—consciousness is said to be scattered and diffused externally.\footnote{“Features” (\textit{nimitta}) are the aspects of sense experience that provoke a response. Compare the standard passage on sense restraint (eg. \href{https://suttacentral.net/mn27/en/sujato\#15.1}{MN 27:15.1}), and \href{https://suttacentral.net/mn137/en/sujato\#8.3}{MN 137:8.3}, where one is “preoccupied” with a sight giving rise to pleasure, etc. } When they hear a sound with their ears … When they smell an odor with their nose … When they taste a flavor with their tongue … When they feel a touch with their body … When they know an idea with their mind, and consciousness follows after the features of that idea—tied, attached, and fettered to gratification in its features—consciousness is said to be scattered and diffused externally. That’s how consciousness is said to be scattered and diffused externally. 

And\marginnote{11.1} how is consciousness said to be not scattered and diffused externally? In this case, when a mendicant sees a sight with their eyes, but consciousness does not follow after the features of that sight—not tied, attached, and fettered to gratification in its features—consciousness is said to be not scattered and diffused externally. When they hear a sound with their ears … When they smell an odor with their nose … When they taste a flavor with their tongue … When they feel a touch with their body … When they know an idea with their mind, but consciousness does not follow after the features of that idea—not tied, attached, and fettered to gratification in its features—consciousness is said to be not scattered and diffused externally. That’s how consciousness is said to be not scattered and diffused externally. 

And\marginnote{12.1} how is it said to be stuck internally?\footnote{In the main summary, consciousness is the subject of each of the three clauses. When the clauses are separated out, the subject is left implicit. } Take a mendicant who, quite secluded from sensual pleasures, secluded from unskillful qualities, enters and remains in the first absorption, which has the rapture and bliss born of seclusion, while placing the mind and keeping it connected. When consciousness follows after that rapture and bliss born of seclusion—tied, attached, and fettered to gratification in that rapture and bliss born of seclusion—the mind is said to be stuck internally.\footnote{\textit{\textsanskrit{Samādhi}} is the outcome of letting go (\href{https://suttacentral.net/sn48.9/en/sujato\#4.2}{SN 48.9:4.2}). This passage reminds us that letting go also applies to \textit{\textsanskrit{samādhi}} itself, so one should not be content to rest on one’s laurels. | Note that in this section, with its focus on states of absorption, \textsanskrit{Mahākaccāna} shifts from “consciousness” (\textit{\textsanskrit{viññāṇa}}) to “mind” (\textit{citta}). } 

Furthermore,\marginnote{13.1} as the placing of the mind and keeping it connected are stilled, a mendicant enters and remains in the second absorption, which has the rapture and bliss born of immersion, with internal clarity and mind at one, without placing the mind and keeping it connected. When consciousness follows after that rapture and bliss born of immersion—tied, attached, and fettered to gratification in that rapture and bliss born of immersion—the mind is said to be stuck internally. 

Furthermore,\marginnote{14.1} with the fading away of rapture, a mendicant enters and remains in the third absorption, where they meditate with equanimity, mindful and aware, personally experiencing the bliss of which the noble ones declare, ‘Equanimous and mindful, one meditates in bliss.’ When consciousness follows after that equanimity—tied, attached, and fettered to gratification in that bliss with equanimity—the mind is said to be stuck internally.\footnote{Text has “equanimity” in the first case, “bliss with equanimity” subsequently. } 

Furthermore,\marginnote{15.1} giving up pleasure and pain, and ending former happiness and sadness, a mendicant enters and remains in the fourth absorption, without pleasure or pain, with pure equanimity and mindfulness. When consciousness follows after that neutral feeling—tied, attached, and fettered to gratification in that neutral feeling—the mind is said to be stuck internally. That’s how it is said to be stuck internally. 

And\marginnote{16.1} how is it said to be not stuck internally? It’s when a mendicant, quite secluded from sensual pleasures, secluded from unskillful qualities, enters and remains in the first absorption, which has the rapture and bliss born of seclusion, while placing the mind and keeping it connected. When consciousness does not follow after that rapture and bliss born of seclusion—not tied, attached, and fettered to gratification in that rapture and bliss born of seclusion—the mind is said to be not stuck internally.\footnote{The solution to being “stuck internally” is to continually deepen \textit{\textsanskrit{samādhi}} by deepening letting go, not, as taught by some 20th century meditation teachers, to avoid it out of fear of becoming attached. } 

Furthermore,\marginnote{17.1} they enter the second absorption … When consciousness does not follow after that rapture and bliss born of immersion—not tied, attached, and fettered to gratification in that rapture and bliss born of immersion—the mind is said to be not stuck internally. 

Furthermore,\marginnote{18.1} they enter and remain in the third absorption … When consciousness does not follow after that equanimity—not tied, attached, and fettered to gratification in that bliss with equanimity—the mind is said to be not stuck internally. 

Furthermore,\marginnote{19.1} they enter and remain in the fourth absorption … When consciousness does not follow after that neutral feeling—not tied, attached, and fettered to gratification in that neutral feeling—the mind is said to be not stuck internally. That’s how it is said to be not stuck internally. 

And\marginnote{20.1} how is it anxious because of grasping?\footnote{All Pali editions read “anxious because of not grasping”. Implausible though it is, this is accepted by the commentary, which offers an explanation. The \textsanskrit{Saṁyuttanikāya}, however, has several similar passages that consistently have the expected “anxious because of grasping” (\href{https://suttacentral.net/sn22.7/en/sujato}{SN 22.7}, \href{https://suttacentral.net/sn22.8/en/sujato}{SN 22.8}, \href{https://suttacentral.net/22.90/en/sujato\#3.13}{22.90:3.13}). It seems the Majjhima preserves an ancient error and the \textsanskrit{Saṁyutta} reading should be preferred. } It’s when an unlearned ordinary person has not seen the noble ones, and is neither skilled nor trained in the teaching of the noble ones. They’ve not seen true persons, and are neither skilled nor trained in the teaching of the true persons. They regard form as self, self as having form, form in self, or self in form. But that form decays and perishes, and consciousness latches on to the perishing of form. Anxieties occupy the mind, born of latching on to the perishing of form, and originating in accordance with natural principles. So it becomes frightened, worried, concerned, and anxious because of grasping. They regard feeling … perception … choices … consciousness as self, self as having consciousness, consciousness in self, or self in consciousness. But that consciousness decays and perishes, and consciousness latches on to the perishing of consciousness. Anxieties occupy the mind, born of latching on to the perishing of consciousness, and originating in accordance with natural principles. So it becomes frightened, worried, concerned, and anxious because of grasping. That’s how it is anxious because of grasping. 

And\marginnote{21.1} how is it not anxious because of grasping? It’s when a learned noble disciple has seen the noble ones, and is skilled and trained in the teaching of the noble ones. They’ve seen true persons, and are skilled and trained in the teaching of the true persons. They don’t regard form as self, self as having form, form in self, or self in form. When that form decays and perishes, consciousness doesn’t latch on to the perishing of form. Anxieties—born of latching on to the perishing of form and originating in accordance with natural principles—don’t occupy the mind. So it does not become frightened, worried, concerned, or anxious because of grasping. They don’t regard feeling … perception … choices … consciousness as self, self as having consciousness, consciousness in self, or self in consciousness. When that consciousness decays and perishes, consciousness doesn’t latch on to the perishing of consciousness. Anxieties—born of latching on to the perishing of consciousness and originating in accordance with natural principles—don’t occupy the mind. So it does not become frightened, worried, concerned, or anxious because of grasping. That’s how it is not anxious because of grasping. 

The\marginnote{22.1} Buddha gave this brief summary recital, then entered his dwelling without explaining the meaning in detail: ‘A mendicant should thoroughly examine such that for the examiner, if consciousness were not scattered and diffused externally, nor stuck internally, it would not be anxious because of grasping. When this is the case, there is no coming to be of the origin of suffering—of rebirth, old age, and death in the future.’ And this is how I understand the detailed meaning of this summary recital. If you wish, you may go to the Buddha and ask him about this. You should remember it in line with the Buddha’s answer.” 

Then\marginnote{23.1} those mendicants, approving and agreeing with what \textsanskrit{Mahākaccāna} said, rose from their seats and went to the Buddha, bowed, sat down to one side, and told him what had happened, adding: 

“\textsanskrit{Mahākaccāna}\marginnote{23.14} clearly explained the meaning to us in this manner, with these words and phrases.” 

“\textsanskrit{Mahākaccāna}\marginnote{24.1} is astute, mendicants, he has great wisdom. If you came to me and asked this question, I would answer it in exactly the same way as \textsanskrit{Mahākaccāna}. That is what it means, and that’s how you should remember it.” 

That\marginnote{24.4} is what the Buddha said. Satisfied, the mendicants approved what the Buddha said. 

%
\section*{{\suttatitleacronym MN 139}{\suttatitletranslation The Analysis of Non-Conflict }{\suttatitleroot Araṇavibhaṅgasutta}}
\addcontentsline{toc}{section}{\tocacronym{MN 139} \toctranslation{The Analysis of Non-Conflict } \tocroot{Araṇavibhaṅgasutta}}
\markboth{The Analysis of Non-Conflict }{Araṇavibhaṅgasutta}
\extramarks{MN 139}{MN 139}

\scevam{So\marginnote{1.1} I have heard. }At one time the Buddha was staying near \textsanskrit{Sāvatthī} in Jeta’s Grove, \textsanskrit{Anāthapiṇḍika}’s monastery. There the Buddha addressed the mendicants, “Mendicants!” 

“Venerable\marginnote{1.5} sir,” they replied. The Buddha said this: 

“Mendicants,\marginnote{2.1} I shall teach you the analysis of non-conflict.\footnote{See note on \href{https://suttacentral.net/mn137/en/sujato\#2.1}{MN 137:2.1}. } Listen and apply your mind well, I will speak.” 

“Yes,\marginnote{2.3} sir,” they replied. The Buddha said this: 

“Don’t\marginnote{3.1} indulge in sensual pleasure, which is low, crude, ordinary, ignoble, and pointless. And don’t indulge in self-mortification, which is painful, ignoble, and pointless.\footnote{This is adapted from the first sermon; text has \textit{\textsanskrit{kāmasukhamanuyuñjeyya}} which is a briefer expression that the original \textit{\textsanskrit{kāmesu} \textsanskrit{kāmasukhallikānuyogo}} at \href{https://suttacentral.net/sn56.11/en/sujato\#2.3}{SN 56.11:2.3}. } Avoiding these two extremes, the Realized One understood the middle way of practice, which gives vision and knowledge, and leads to peace, direct knowledge, awakening, and extinguishment. Know what it means to flatter and to rebuke. Knowing these, avoid them, and just teach Dhamma. Know how to evaluate different kinds of pleasure. Knowing this, pursue inner pleasure. Don’t talk behind people’s backs, and don’t speak sharply in their presence. Don’t speak hurriedly. Don’t insist on popular terms and don’t overstep normal labels.\footnote{“Popular terms” is \textit{janapadanirutti}. | \textit{\textsanskrit{Samaññā}} (“label”, Sanskrit \textit{\textsanskrit{samājñā}}) ought not be confused with \textit{\textsanskrit{sāmañña}} (“agreement”, “equality”, Sanskrit \textit{\textsanskrit{sāmānya}}). } This is the summary recital for the analysis of non-conflict. 

‘Don’t\marginnote{4.1} indulge in sensual pleasure, which is low, crude, ordinary, ignoble, and pointless. And don’t indulge in self-mortification, which is painful, ignoble, and pointless.’ That’s what I said, but why did I say it? Indulging in the happiness of the pleasure linked to sensuality is low, crude, ordinary, ignoble, and pointless. It is a principle beset by pain, harm, stress, and fever, and it is the wrong way. Breaking off such indulgence is a principle free of pain, harm, stress, and fever, and it is the right way. Indulging in self-mortification is painful, ignoble, and pointless. It is a principle beset by pain, harm, stress, and fever, and it is the wrong way. Breaking off such indulgence is a principle free of pain, harm, stress, and fever, and it is the right way. ‘Don’t indulge in sensual pleasure, which is low, crude, ordinary, ignoble, and pointless. And don’t indulge in self-mortification, which is painful, ignoble, and pointless.’ That’s what I said, and this is why I said it. 

‘Avoiding\marginnote{5.1} these two extremes, the Realized One understood the middle way of practice, which gives vision and knowledge, and leads to peace, direct knowledge, awakening, and extinguishment.’ That’s what I said, but why did I say it? It is simply this noble eightfold path, that is: right view, right thought, right speech, right action, right livelihood, right effort, right mindfulness, and right immersion. ‘Avoiding these two extremes, the Realized One understood the middle way of practice, which gives vision and knowledge, and leads to peace, direct knowledge, awakening, and extinguishment.’ That’s what I said, and this is why I said it. 

‘Know\marginnote{6.1} what it means to flatter and to rebuke. Knowing these, avoid them, and just teach Dhamma.’ That’s what I said, but why did I say it? 

And\marginnote{7.1} how is there flattering and rebuking without teaching Dhamma? ‘All those who indulge in the happiness of the pleasure linked to sensuality—low, crude, ordinary, ignoble, and pointless—are beset by pain, harm, stress, and fever, and they are practicing the wrong way.’ In speaking like this, some here are rebuked. 

‘All\marginnote{7.4} those who have broken off indulging in the happiness of the pleasure linked to sensuality are free of pain, harm, stress, and fever, and they are practicing the right way.’ In speaking like this, some here are flattered. 

‘All\marginnote{7.6} those who indulge in self-mortification—painful, ignoble, and pointless—are beset by pain, harm, stress, and fever, and they are practicing the wrong way.’ In speaking like this, some here are rebuked. 

‘All\marginnote{7.8} those who have broken off indulging in self-mortification are free of pain, harm, stress, and fever, and they are practicing the right way.’ In speaking like this, some here are flattered. 

‘All\marginnote{7.10} those who have not given up the fetter of continued existence are beset by pain, harm, stress, and fever, and they are practicing the wrong way.’ In speaking like this, some here are rebuked. 

‘All\marginnote{7.12} those who have given up the fetter of continued existence are free of pain, harm, stress, and fever, and they are practicing the right way.’ In speaking like this, some here are flattered. That’s how there is flattering and rebuking without teaching Dhamma. 

And\marginnote{8.1} how is there neither flattering nor rebuking, and just teaching Dhamma? You don’t say: ‘All those who indulge in the happiness of the pleasure linked to sensuality—low, crude, ordinary, ignoble, and pointless—are beset by pain, harm, stress, and fever, and they are practicing the wrong way.’ Rather, by saying this you just teach Dhamma: ‘The indulgence is a principle beset by pain, harm, stress, and fever, and it is the wrong way.’ 

You\marginnote{8.7} don’t say: ‘All those who have broken off indulging in the happiness of the pleasure linked to sensuality are free of pain, harm, stress, and fever, and they are practicing the right way.’ Rather, by saying this you just teach Dhamma: ‘Breaking off the indulgence is a principle free of pain, harm, stress, and fever, and it is the right way.’ 

You\marginnote{8.12} don’t say: ‘All those who indulge in self-mortification—painful, ignoble, and pointless—are beset by pain, harm, stress, and fever, and they are practicing the wrong way.’ Rather, by saying this you just teach Dhamma: ‘The indulgence is a principle beset by pain, harm, stress, and fever, and it is the wrong way.’ 

You\marginnote{8.17} don’t say: ‘All those who have broken off indulging in self-mortification are free of pain, harm, stress, and fever, and they are practicing the right way.’ Rather, by saying this you just teach Dhamma: ‘Breaking off the indulgence is a principle free of pain, harm, stress, and fever, and it is the right way.’ 

You\marginnote{8.22} don’t say: ‘All those who have not given up the fetter of continued existence are beset by pain, harm, stress, and fever, and they are practicing the wrong way.’ Rather, by saying this you just teach Dhamma: ‘When the fetter of continued existence is not given up, continued existence is also not given up.’ 

You\marginnote{8.26} don’t say: ‘All those who have given up the fetter of continued existence are free of pain, harm, stress, and fever, and they are practicing the right way.’ Rather, by saying this you just teach Dhamma: ‘When the fetter of continued existence is given up, continued existence is also given up.’ That’s how there is neither flattering nor rebuking, and just teaching Dhamma. ‘Know what it means to flatter and to rebuke. Knowing these, avoid them, and just teach Dhamma.’ That’s what I said, and this is why I said it. 

‘Know\marginnote{9.1} how to evaluate different kinds of pleasure. Knowing this, pursue inner pleasure.’ That’s what I said, but why did I say it? There are these five kinds of sensual stimulation. What five? Sights known by the eye, which are likable, desirable, agreeable, pleasant, sensual, and arousing. Sounds known by the ear … Smells known by the nose … Tastes known by the tongue … Touches known by the body, which are likable, desirable, agreeable, pleasant, sensual, and arousing. These are the five kinds of sensual stimulation. The pleasure and happiness that arise from these five kinds of sensual stimulation is called sensual pleasure—a filthy, common, ignoble pleasure. Such pleasure should not be cultivated or developed, but should be feared, I say. Now, take a mendicant who, quite secluded from sensual pleasures, secluded from unskillful qualities, enters and remains in the first absorption, which has the rapture and bliss born of seclusion, while placing the mind and keeping it connected. As the placing of the mind and keeping it connected are stilled, they enter and remain in the second absorption … third absorption … fourth absorption. This is called the pleasure of renunciation, the pleasure of seclusion, the pleasure of peace, the pleasure of awakening. Such pleasure should be cultivated and developed, and should not be feared, I say. ‘Know how to evaluate different kinds of pleasure. Knowing this, pursue inner pleasure.’ That’s what I said, and this is why I said it. 

‘Don’t\marginnote{10.1} talk behind people’s backs, and don’t speak sharply in their presence.’ That’s what I said, but why did I say it? When you know that what you say behind someone’s back is untrue, false, and pointless, then to the best of your ability you should not speak. When you know that what you say behind someone’s back is true and correct, but pointless, then you should train yourself not to speak. When you know that what you say behind someone’s back is true, correct, and beneficial, then you should know the right time to speak. When you know that your sharp words in someone’s presence are untrue, false, and pointless, then to the best of your ability you should not speak. When you know that your sharp words in someone’s presence are true and correct, but pointless, then you should train yourself not to speak. When you know that your sharp words in someone’s presence are true, correct, and beneficial, then you should know the right time to speak. ‘Don’t talk behind people’s backs, and don’t speak sharply in their presence.’ That’s what I said, and this is why I said it. 

‘Don’t\marginnote{11.1} speak hurriedly.’ That’s what I said, but why did I say it? When speaking hurriedly, your body gets tired, your mind gets stressed, your voice gets stressed, your throat gets sore, and your words become unclear and hard to understand. When not speaking hurriedly, your body doesn’t get tired, your mind doesn’t get stressed, your voice doesn’t get stressed, your throat doesn’t get sore, and your words are clear and easy to understand. ‘Don’t speak hurriedly.’ That’s what I said, and this is why I said it. 

‘Don’t\marginnote{12.1} insist on popular terms and don’t overstep normal labels.’ That’s what I said, but why did I say it? And how do you insist on popular terms and overstep normal labels?\footnote{None of the terms below are regional or dialectical, so \textit{janapadanirutti} means “expressions current among the people”, “popular terms” rather than “regional dialects”. } It’s when among different populations they label the same thing as a ‘cup’, a ‘bowl’, a ‘jar’, a ‘scoop’, a ‘vessel’, a ‘dish’, or a ‘plate’.\footnote{This is not about “translating” the Dhamma wholesale, but about using different popular terms, as in the U.S. they say “soda” or “pop” in different regions. Compare \href{https://suttacentral.net/pli-tv-kd15/en/sujato\#33.1.5}{Kd 15:33.1.5}, which is about individuals giving their own glosses to terms (\textit{\textsanskrit{sakāya} niruttiya}). All these terms are either attested in Pali and/or Sanskrit or are readily explicable from Pali roots. | \textit{\textsanskrit{Pātī}} (“cup”) is from the root \textit{√\textsanskrit{pā}}, “to drink”, used in Pali for cups of bronze, gold, or silver. | \textit{Patta} (“bowl”), the masculine form of the same root, is the standard word for a mendicant’s alms bowl. | For \textit{vittha} (“jar”; variants \textit{vitta}, \textit{\textsanskrit{piṭṭha}}) see \textit{vitthaka}, a small jar for sewing gear (\href{https://suttacentral.net/pli-tv-kd15/en/sujato\#11.5.15}{Kd 15:11.5.15}). | \textit{\textsanskrit{Sarāva}} (“scoop”) is a common word for a scoop for water (eg. \href{https://suttacentral.net/an3.57/en/sujato\#3.2}{AN 3.57:3.2}, \href{https://suttacentral.net/pli-tv-kd15/en/sujato\#14.3.30}{Kd 15:14.3.30}, \href{https://suttacentral.net/kv1.1/en/sujato\#356.6}{Kv 1.1:356.6}; Sanskrit \textit{\textsanskrit{śarāva}}, eg. Śatapatha \textsanskrit{Brāhmaṇa} 5.1.4.12; \textsanskrit{Chāndogya} \textsanskrit{Upaniṣad} 8.8.1; Manu 6.56; also in Jain texts). | For \textit{\textsanskrit{dhāropa}} (“vessel”) accept variant \textit{harosa} by analogy with \textit{\textsanskrit{puṭosa}} (\href{https://suttacentral.net/mn118/en/sujato\#8.9}{MN 118:8.9}), yielding the sense “carrier for food”, “vessel”. | \textit{\textsanskrit{Poṇa}} (“dish”) is probably a concave dish with “sloped” sides (cp. English “hollowware”). | \textit{\textsanskrit{Pisīlava}} (“plate”) was used for the butter placed on the altar (Śatapatha \textsanskrit{Brāhmaṇa} 2.5.3.6: \textit{\textsanskrit{piśīle} \textsanskrit{vā} \textsanskrit{pātryau} \textsanskrit{vā}}). } And however it is known among those various populations, you speak accordingly, obstinately sticking to that and insisting: ‘This is the only truth, anything else is futile.’ That’s how you insist on popular terms and overstep normal labels. 

And\marginnote{12.8} how do you not insist on popular terms and overstep normal labels? It’s when among different populations they label the same thing as a ‘cup’, a ‘bowl’, a ‘jar’, a ‘scoop’, a ‘vessel’, a ‘dish’, or a ‘plate’. And however it is known among those various populations, you speak accordingly, thinking: ‘It seems that the venerables are referring to this.’ That’s how you don’t insist on popular terms and don’t overstep normal labels. ‘Don’t insist on popular terms and don’t overstep normal labels.’ That’s what I said, and this is why I said it. 

Now,\marginnote{13.1} mendicants, indulging in the happiness of the pleasure linked to sensuality is low, crude, ordinary, ignoble, and pointless. It is a principle beset by pain, harm, stress, and fever, and it is the wrong way. That’s why this is a principle beset by conflict. Breaking off such indulgence is a principle free of pain, harm, stress, and fever, and it is the right way. That’s why this is a principle free of conflict. 

Indulging\marginnote{13.7} in self-mortification is painful, ignoble, and pointless. It is a principle beset by pain, harm, stress, and fever, and it is the wrong way. That’s why this is a principle beset by conflict. Breaking off such indulgence is a principle free of pain, harm, stress, and fever, and it is the right way. That’s why this is a principle free of conflict. 

The\marginnote{13.13} middle way of practice that was understood by the Realized One gives vision and knowledge, and leads to peace, direct knowledge, awakening, and extinguishment. It is a principle free of pain, harm, stress, and fever, and it is the right way. That’s why this is a principle free of conflict. 

Flattering\marginnote{13.16} and rebuking without teaching Dhamma is a principle beset by pain, harm, stress, and fever, and it is the wrong way. That’s why this is a principle beset by conflict. Neither flattering nor rebuking, and just teaching Dhamma is a principle free of pain, harm, stress, and fever, and it is the right way. That’s why this is a principle free of conflict. 

Sensual\marginnote{13.22} pleasure—a filthy, common, ignoble pleasure—is a principle beset by pain, harm, stress, and fever, and it is the wrong way. That’s why this is a principle beset by conflict. The pleasure of renunciation, the pleasure of seclusion, the pleasure of peace, the pleasure of awakening is a principle free of pain, harm, stress, and fever, and it is the right way. That’s why this is a principle free of conflict. 

Saying\marginnote{13.28} untrue, false, and pointless things behind someone’s back is a principle beset by pain, harm, stress, and fever, and it is the wrong way. That’s why this is a principle beset by conflict. Saying true and correct, but pointless things behind someone’s back is a principle beset by pain, harm, stress, and fever, and it is the wrong way. That’s why this is a principle beset by conflict. Saying true, correct, and beneficial things behind someone’s back is a principle free of pain, harm, stress, and fever, and it is the right way. That’s why this is a principle free of conflict. 

Saying\marginnote{13.37} untrue, false, and pointless things in someone’s presence is a principle beset by pain, harm, stress, and fever, and it is the wrong way. That’s why this is a principle beset by conflict. Saying true and correct, but pointless things in someone’s presence is a principle beset by pain, harm, stress, and fever, and it is the wrong way. That’s why this is a principle beset by conflict. Saying true, correct, and beneficial things in someone’s presence is a principle free of pain, harm, stress, and fever, and it is the right way. That’s why this is a principle free of conflict. 

Speaking\marginnote{13.46} hurriedly is a principle beset by pain, harm, stress, and fever, and it is the wrong way. That’s why this is a principle beset by conflict. Speaking unhurriedly is a principle free of pain, harm, stress, and fever, and it is the right way. That’s why this is a principle free of conflict. 

Insisting\marginnote{13.52} on popular terms and overriding common usage is a principle beset by pain, harm, stress, and fever, and it is the wrong way. That’s why this is a principle beset by conflict. Not insisting on popular terms and not overriding common usage is a principle free of pain, harm, stress, and fever, and it is the right way. That’s why this is a principle free of conflict. 

So\marginnote{14.1} you should train like this: ‘We shall know the principles beset by conflict and the principles free of conflict. Knowing this, we will practice the way free of conflict.’ 

And,\marginnote{14.3} mendicants, \textsanskrit{Subhūti}, the gentleman, practices the way of non-conflict.”\footnote{This was apparently prompted by \textsanskrit{Subhūti}’s support of his fellow monk Saddha at \href{https://suttacentral.net/an11.14 /en/sujato}{AN 11.14 }, for which he was declared the foremost in living without conflict (\href{https://suttacentral.net/an1.201/en/sujato\#1.1}{AN 1.201:1.1}; he was additionally praised as worthy of donations (\href{https://suttacentral.net/an1.202/en/sujato\#1.1}{AN 1.202:1.1}. \textsanskrit{Subhūti}’s meditation is praised at \href{https://suttacentral.net/ud6.7/en/sujato}{Ud 6.7}, and his verse opens the \textsanskrit{Theragāthā} at \href{https://suttacentral.net/thag1.1/en/sujato}{Thag 1.1}. } 

That\marginnote{14.4} is what the Buddha said. Satisfied, the mendicants approved what the Buddha said. 

%
\section*{{\suttatitleacronym MN 140}{\suttatitletranslation The Analysis of the Elements }{\suttatitleroot Dhātuvibhaṅgasutta}}
\addcontentsline{toc}{section}{\tocacronym{MN 140} \toctranslation{The Analysis of the Elements } \tocroot{Dhātuvibhaṅgasutta}}
\markboth{The Analysis of the Elements }{Dhātuvibhaṅgasutta}
\extramarks{MN 140}{MN 140}

\scevam{So\marginnote{1.1} I have heard. }At one time the Buddha was wandering in the Magadhan lands when he arrived at \textsanskrit{Rājagaha}. He went to see Bhaggava the potter, and said,\footnote{The Bhaggava clan was descended from the ancient sage Bhagu (Sanskrit \textsanskrit{Bhṛgu}). They received the gift of fire conveyed by \textsanskrit{Mātariśvan} the wind from the god Agni (eg. Rig Veda 1.60.1). In Pali they appear as potters (\href{https://suttacentral.net/mn81/en/sujato\#19.3}{MN 81:19.3}, \href{https://suttacentral.net/sn1.50/en/sujato\#11.2}{SN 1.50:11.2}, \href{https://suttacentral.net/sn2.24/en/sujato\#12.2}{SN 2.24:12.2}), unless they have gone forth (\href{https://suttacentral.net/dn24/en/sujato\#1.1.6}{DN 24:1.1.6}). Archaeologists refer to the strata around the Buddha’s time as the Northern Black Polished Ware culture on account of the distinctive highly glazed polish that was achieved on the pottery of the time. This, together with the production of iron, marked a significant advance in the mastery of fire. Thus potters were no mere humble craftsmen, but leading technological innovators. } “Bhaggava, if it is no trouble, I’d like to spend a single night in your workshop.” 

“It’s\marginnote{2.2} no trouble, sir. But there’s a renunciate already staying there. If he allows it, sir, you may stay as long as you please.” 

Now\marginnote{3.1} at that time a gentleman named \textsanskrit{Pukkusāti} had gone forth out of faith from the lay life to homelessness in the Buddha’s name.\footnote{Buddhist texts of the middle period—starting a few centuries after the Buddha—share the story that \textsanskrit{Pukkusāti} had been the king of Taxila in \textsanskrit{Gandhāra}, who went forth out of faith upon reading texts of the Dhamma sent by his friend and ally, \textsanskrit{Bimbisāra}. This story is found in detail in the Pali commentary to this sutta, and more briefly in several canonical texts of the northern traditions (T 211 at T iv 580c19; T 511 at T xiv 779a; \textsanskrit{Mūlasarvāstivāda} Vinaya 3.2.26, which spells his name \textit{\textsanskrit{puṣkarasāri}}, the same name as the brahmin known in Pali as \textsanskrit{Pokkharasāti}). Texts of this period also know of a script called \textit{\textsanskrit{puṣkarasāri}} (Lalitavistara 10, Vaidya 87; \textsanskrit{Mahāvastu} 14, Senart 1.135). This would presumably have been the writing system in the city of \textsanskrit{Puṣkarāvati}, another city in \textsanskrit{Gandhāra}, the region where the oldest surviving Buddhist manuscripts have been found. The details of the \textsanskrit{Pukkusāti} legend are improbable and it is mostly likely an origin myth suggested by the similarity of his name with \textsanskrit{Puṣkarāvati}, authorizing the establishment of the Dhamma in \textsanskrit{Gandhāra} in the post-Ashokan period, as well as offering a precedent for the writing down of the Dhamma. } And it was he who had first taken up residence in the workshop. Then the Buddha approached Venerable \textsanskrit{Pukkusāti} and said, “Mendicant, if it is no trouble, I’d like to spend a single night in the workshop.”\footnote{The Buddha addresses \textsanskrit{Pukkusāti} as “mendicant” (\textit{bhikkhu}), implicitly recognizing him as a Buddhist monk (compare \href{https://suttacentral.net/mn26/en/sujato\#24.5}{MN 26:24.5} = \href{https://suttacentral.net/mn85/en/sujato\#48.5}{MN 85:48.5}), while \textsanskrit{Pukkusāti} uses the respectful familiar form “reverend” (\textit{\textsanskrit{āvuso}}). } 

“The\marginnote{3.5} potter’s workshop is spacious, reverend. Stay as long as you please.” 

Then\marginnote{4.1} the Buddha entered the workshop and spread out a grass mat to one side. He sat down cross-legged, set his body straight, and established mindfulness in his presence. He spent much of the night sitting in meditation, and so did \textsanskrit{Pukkusāti}. 

Then\marginnote{4.4} it occurred to the Buddha, “This gentleman’s behavior is impressive.\footnote{He had been sitting most of the night in meditation, indicating that he had already mastered immersion to a substantial degree. The commentary says that he had in fact attained the fourth absorption based on mindfulness of breathing. } Why don’t I question him?” 

So\marginnote{4.7} the Buddha said to \textsanskrit{Pukkusāti}, “In whose name have you gone forth, mendicant? Who is your Teacher? Whose teaching do you believe in?” 

“Reverend,\marginnote{5.2} there is the ascetic Gotama—a Sakyan, gone forth from a Sakyan family. He has this good reputation: ‘That Blessed One is perfected, a fully awakened Buddha, accomplished in knowledge and conduct, holy, knower of the world, supreme guide for those who wish to train, teacher of gods and humans, awakened, blessed.’ I’ve gone forth in his name. That Blessed One is my Teacher, and I believe in his teaching.” 

“But\marginnote{5.8} mendicant, where is the Blessed One at present, the perfected one, the fully awakened Buddha?” 

“In\marginnote{5.9} the northern lands there is a city called \textsanskrit{Sāvatthī}.\footnote{Compare \href{https://suttacentral.net/an2.37/en/sujato\#4.2}{AN 2.37:4.2}, where \textsanskrit{Sāvatthī} is said to be in the east (probably from \textsanskrit{Madhurā}). } There the Blessed One is now staying, the perfected one, the fully awakened Buddha.” 

“But\marginnote{5.11} have you ever seen that Buddha? Would you recognize him if you saw him?” 

“No,\marginnote{5.13} I’ve never seen him, and I wouldn’t recognize him if I did.”\footnote{Despite his wisdom, \textsanskrit{Pukkusāti} fails to recognize the Buddha. In other cases such failure is a narrative indication of spiritual blindness (\textsanskrit{Ajātasattu} at \href{https://suttacentral.net/dn2/en/sujato\#11.2}{DN 2:11.2} and the park keeper at \href{https://suttacentral.net/mn31/en/sujato\#3.4}{MN 31:3.4} = \href{https://suttacentral.net/mn128/en/sujato\#0.2}{MN 128:0.2} = \href{https://suttacentral.net/pli-tv-kd10/en/sujato\#4.2.3}{Kd 10:4.2.3}). } 

Then\marginnote{6.1} it occurred to the Buddha, “This gentleman has gone forth in my name. Why don’t I teach him the Dhamma?”\footnote{The Buddha leads by persuading with the Dhamma, rather than by revealing his identity. } 

So\marginnote{6.4} the Buddha said to \textsanskrit{Pukkusāti}, “Mendicant, I shall teach you the Dhamma. Listen and apply your mind well, I will speak.” 

“Yes,\marginnote{6.7} reverend,” replied \textsanskrit{Pukkusāti}. The Buddha said this: 

“‘This\marginnote{7.1} person has six elements, six fields of contact, eighteen mental preoccupations, and four foundations.\footnote{The six elements, six fields of contact, eighteen mental preoccupations are taught at \href{https://suttacentral.net/an3.61/en/sujato\#9.3}{AN 3.61:9.3}ff., where they are followed by the four noble truths. } Where they stand, the streams of conceiving do not flow. And where the streams of conceiving do not flow, they are called a sage at peace. Do not neglect wisdom; preserve truth; foster generosity; and train only for peace.’ This is the summary recital for the analysis of the elements. 

‘This\marginnote{8.1} person has six elements.’ That’s what I said, but why did I say it? There are these six elements: the elements of earth, water, fire, air, space, and consciousness.\footnote{First the Buddha establishes the fundamental scope of his teaching. The elements describe the world as it is, and on them the course of meditation is based. The extension of the four elements to six hints at the practice of formless attainments. } ‘This person has six elements.’ That’s what I said, and this is why I said it. 

‘This\marginnote{9.1} person has six fields of contact.’ That’s what I said, but why did I say it? The fields of contact of the eye, ear, nose, tongue, body, and mind. ‘This person has six fields of contact.’ That’s what I said, and this is why I said it. 

‘This\marginnote{10.1} person has eighteen mental preoccupations.’\footnote{The “fields of contact” are the scope within which experience occurs, in response to which one becomes “preoccupied” due to the power of feeling. See note on \href{https://suttacentral.net/mn137/en/sujato\#8.1}{MN 137:8.1}. } That’s what I said, but why did I say it? Seeing a sight with the eye, one is preoccupied with a sight that’s a basis for happiness or sadness or equanimity. Hearing a sound with the ear … Smelling an odor with the nose … Tasting a flavor with the tongue … 

Feeling\marginnote{10.7} a touch with the body … Becoming conscious of an idea with the mind, one is preoccupied with an idea that’s a basis for happiness or sadness or equanimity. So there are six preoccupations with happiness, six preoccupations with sadness, and six preoccupations with equanimity. ‘This person has eighteen mental preoccupations.’ That’s what I said, and this is why I said it. 

‘This\marginnote{11.1} person has four foundations.’\footnote{Thus far we have dealt with what is in the world (the elements), how we experience the world (contact), and how we react to that experience (preoccupation with feeling). Now the Buddha turns to the path; instead of uncritical reaction to feeling, we develop a reflective response. } That’s what I said, but why did I say it? The foundations of wisdom, truth, generosity, and peace.\footnote{“Foundation” is \textit{\textsanskrit{adhiṭṭhāna}}, one meaning of which is the “plinth” or “base” on which a building rests; the commentary glosses with \textit{\textsanskrit{patiṭṭhā}}. Also at \href{https://suttacentral.net/dn33/en/sujato\#1.11.150}{DN 33:1.11.150}. } ‘This person has four foundations.’ That’s what I said, and this is why I said it. 

‘Do\marginnote{12.1} not neglect wisdom; preserve truth; foster generosity; and train only for peace.’\footnote{Here the Buddha echoes the pattern of his first discourse, where he first identified what the four noble truths are, then identified the duty or task to be undertaken in relation to each of them (\href{https://suttacentral.net/sn56.11/en/sujato}{SN 56.11}; see also \href{https://suttacentral.net/sn56.12/en/sujato}{SN 56.12}, \href{https://suttacentral.net/sn56.29/en/sujato}{SN 56.29}). } That’s what I said, but why did I say it? 

And\marginnote{13.1} how does one not neglect wisdom?\footnote{Thus begins the longest section of the discourse, continuing up to \href{https://suttacentral.net/mn140/en/sujato\#25.3}{MN 140:25.3}. } There are these six elements:\footnote{The Buddha returns to the six elements, for which we have already been prepared in brief. The discussion of the “elements” (\textit{\textsanskrit{dhātu}}) now continues in similar fashion to the five elements at \href{https://suttacentral.net/mn62/en/sujato\#8.1}{MN 62:8.1}. \href{https://suttacentral.net/mn28/en/sujato\#5.1}{MN 28:5.1} teaches the four “principal states” (\textit{\textsanskrit{mahābhūta}}) in even more detail. } the elements of earth, water, fire, air, space, and consciousness. 

And\marginnote{14.1} what is the earth element? The earth element may be interior or exterior. And what is the interior earth element? Anything internal, pertaining to an individual, that’s hard, solid, and appropriated. This includes: head hair, body hair, nails, teeth, skin, flesh, sinews, bones, bone marrow, kidneys, heart, liver, diaphragm, spleen, lungs, intestines, mesentery, undigested food, feces, or anything else internal, pertaining to an individual, that’s hard, solid, and appropriated. This is called the interior earth element. The interior earth element and the exterior earth element are just the earth element. This should be truly seen with right understanding like this: ‘This is not mine, I am not this, this is not my self.’ When you truly see with right understanding, you grow disillusioned with the earth element, detaching the mind from the earth element. 

And\marginnote{15.1} what is the water element? The water element may be interior or exterior. And what is the interior water element? Anything internal, pertaining to an individual, that’s water, watery, and appropriated. This includes: bile, phlegm, pus, blood, sweat, fat, tears, grease, saliva, snot, synovial fluid, urine, or anything else internal, pertaining to an individual, that’s water, watery, and appropriated. This is called the interior water element. The interior water element and the exterior water element are just the water element. This should be truly seen with right understanding like this: ‘This is not mine, I am not this, this is not my self.’ When you truly see with right understanding, you grow disillusioned with the water element, detaching the mind from the water element. 

And\marginnote{16.1} what is the fire element? The fire element may be interior or exterior. And what is the interior fire element? Anything internal, pertaining to an individual, that’s fire, fiery, and appropriated. This includes: that which warms, that which ages, that which heats you up when feverish, that which properly digests food and drink; or anything else internal, pertaining to an individual, that’s fire, fiery, and appropriated. This is called the interior fire element. The interior fire element and the exterior fire element are just the fire element. This should be truly seen with right understanding like this: ‘This is not mine, I am not this, this is not my self.’ When you truly see with right understanding, you grow disillusioned with the fire element, detaching the mind from the fire element. 

And\marginnote{17.1} what is the air element? The air element may be interior or exterior. And what is the interior air element? Anything internal, pertaining to an individual, that’s air, airy, and appropriated. This includes: winds that go up or down, winds in the belly or the bowels, winds that flow through the limbs, in-breaths and out-breaths; or anything else internal, pertaining to an individual, that’s air, airy, and appropriated. This is called the interior air element. The interior air element and the exterior air element are just the air element. This should be truly seen with right understanding like this: ‘This is not mine, I am not this, this is not my self.’ When you truly see with right understanding, you grow disillusioned with the air element, detaching the mind from the air element. 

And\marginnote{18.1} what is the space element? The space element may be interior or exterior. And what is the interior space element? Anything internal, pertaining to an individual, that’s space, spacious, and appropriated. This includes: the ear canals, nostrils, and mouth; and the space for swallowing what is eaten and drunk, the space where it stays, and the space for excreting it from the nether regions; or anything else internal, pertaining to an individual, that’s space, spacious, and appropriated. This is called the interior space element. The interior space element and the exterior space element are just the space element. This should be truly seen with right understanding like this: ‘This is not mine, I am not this, this is not my self.’ When you truly see with right understanding, you grow disillusioned with the space element, detaching the mind from the space element.\footnote{From here, \href{https://suttacentral.net/mn62/en/sujato\#12.9}{MN 62:12.9} proceeds in a different direction, outlining a series of meditations based on the elements and other things. } 

What\marginnote{19.1} remains is sheer consciousness, pure and bright.\footnote{The phrase “sheer consciousness” (\textit{\textsanskrit{viññāṇaṁyeva}}) echoes \textsanskrit{Bṛhadāraṇyaka} \textsanskrit{Upaniṣad} 2.4.12, where the ultimate manifestation of the Self, the “principle reality”, is said to be “a sheer mass of consciousness” (\textit{\textsanskrit{vijñānaghana} eva}). Here it refers to the mind following absorption that develops insight into feelings. } And what does that consciousness know? It knows ‘pleasure’ and ‘pain’ and ‘neutral’. Pleasant feeling arises dependent on a contact to be experienced as pleasant.\footnote{Such as the pleasant feelings of the first three absorptions. } When they feel a pleasant feeling, they know: ‘I feel a pleasant feeling.’ They know: ‘With the cessation of that contact to be experienced as pleasant, the corresponding pleasant feeling ceases and stops.’ 

Painful\marginnote{19.7} feeling arises dependent on a contact to be experienced as painful.\footnote{Such as the painful feelings that arise in the body after emergence from absorption. } When they feel a painful feeling, they know: ‘I feel a painful feeling.’ They know: ‘With the cessation of that contact to be experienced as painful, the corresponding painful feeling ceases and stops.’ 

Neutral\marginnote{19.10} feeling arises dependent on a contact to be experienced as neutral.\footnote{Such as the neutral feeling of the fourth absorption. } When they feel a neutral feeling, they know: ‘I feel a neutral feeling.’ They know: ‘With the cessation of that contact to be experienced as neutral, the corresponding neutral feeling ceases and stops.’ 

When\marginnote{19.13} you rub two sticks together, heat is generated and fire is produced. But when you part the sticks and lay them aside, any corresponding heat ceases and stops.\footnote{Experience depends on the collision of disparate elements, whose friction or “resistance” (\textit{\textsanskrit{paṭigha}}, \href{https://suttacentral.net/dn15/en/sujato\#20.4}{DN 15:20.4}) sparks the light of consciousness. Hence consciousness, like fire, is beautiful but also burning (\href{https://suttacentral.net/sn35.235/en/sujato}{SN 35.235}). } In the same way, pleasant feeling arises dependent on a contact to be experienced as pleasant. … 

They\marginnote{19.20} know: ‘With the cessation of that contact to be experienced as neutral, the corresponding neutral feeling ceases and stops.’ 

There\marginnote{20.1} remains only equanimity, pure, bright, pliable, workable, and radiant.\footnote{This is the purified equanimity of the fourth absorption. Although the teaching is framed in terms of the development of discernment (\textit{\textsanskrit{vipassanā}}), nonetheless it follows the pattern of the absorptions, leading through the first four “form” absorptions to the radiant consciousness that can experience the formless. } It’s like when a deft goldsmith or a goldsmith’s apprentice prepares a forge, fires the crucible, picks up some native gold with tongs and puts it in the crucible. From time to time they fan it, from time to time they sprinkle water on it, and from time to time they just watch over it. That native gold becomes pliable, workable, and radiant, not brittle, and is ready to be worked. Then the goldsmith can successfully create any kind of ornament they want, whether a bracelet, earrings, a necklace, or a golden garland.\footnote{This simile is developed further at \href{https://suttacentral.net/an3.102/en/sujato\#2.1}{AN 3.102:2.1}. } In the same way, there remains only equanimity, pure, bright, pliable, workable, and radiant. 

They\marginnote{21.1} understand: ‘If I were to apply this equanimity, so pure and bright, to the dimension of infinite space, and develop my mind accordingly, this equanimity of mine, relying on that and grasping it, would remain for a very long time.\footnote{The lifespans of rebirth in the first three formless realms is given in \href{https://suttacentral.net/an3.116/en/sujato}{AN 3.116}. For the gods of infinite space, it is twenty thousand eons. } If I were to apply this equanimity, so pure and bright, to the dimension of infinite consciousness, and develop my mind accordingly, this equanimity of mine, relying on that and grasping it, would remain for a very long time.\footnote{Forty thousand eons. } If I were to apply this equanimity, so pure and bright, to the dimension of nothingness, and develop my mind accordingly, this equanimity of mine, relying on that and grasping it, would remain for a very long time.\footnote{Sixty thousand eons. } If I were to apply this equanimity, so pure and bright, to the dimension of neither perception nor non-perception, and develop my mind accordingly, this equanimity of mine, relying on that and grasping it, would remain for a very long time.’ 

They\marginnote{22.1} understand: ‘If I were to apply this equanimity, so pure and bright, to the dimension of infinite space, my mind would develop accordingly. But that is conditioned.\footnote{“Conditioned” is \textit{\textsanskrit{saṅkhata}}, the past participle whose active verbal form is translated below as “choice”. To be “conditioned”, is to be “created” by an act of will or “choice”. } If I were to apply this equanimity, so pure and bright, to the dimension of infinite consciousness … nothingness … neither perception nor non-perception, my mind would develop accordingly. But that is conditioned.’ 

They\marginnote{22.10} neither make a choice nor form an intention to continue existence or to end existence.\footnote{For the idiom “make a choice nor form an intention”, compare \href{https://suttacentral.net/mn52/en/sujato\#4.2}{MN 52:4.2}ff. and \href{https://suttacentral.net/mn121/en/sujato\#11.4}{MN 121:11.4}. | The craving to “end existence” (\textit{vibhava}; see \href{https://suttacentral.net/snp4.10/en/sujato\#9.3}{Snp 4.10:9.3}) is the urge to annihilation, for which see \href{https://suttacentral.net/mn102/en/sujato\#12.7}{MN 102:12.7}. } Because of this, they don’t grasp at anything in the world. Not grasping, they’re not anxious. Not being anxious, they personally become extinguished. 

They\marginnote{22.13} understand: ‘Rebirth is ended, the spiritual journey has been completed, what had to be done has been done, there is nothing further for this place.’ 

If\marginnote{23.1} they feel a pleasant feeling, they understand that it’s impermanent, that they’re not attached to it, and that they don’t take pleasure in it.\footnote{This describes the reflective experience of the arahant. } If they feel a painful feeling, they understand that it’s impermanent, that they’re not attached to it, and that they don’t take pleasure in it. If they feel a neutral feeling, they understand that it’s impermanent, that they’re not attached to it, and that they don’t take pleasure in it. 

If\marginnote{24.1} they feel a pleasant feeling, they feel it detached. If they feel a painful feeling, they feel it detached. If they feel a neutral feeling, they feel it detached. Feeling the end of the body approaching, they understand: ‘I feel the end of the body approaching.’ Feeling the end of life approaching, they understand: ‘I feel the end of life approaching.’ They understand: ‘When my body breaks up and my life has come to an end, everything that’s felt, since I no longer take pleasure in it, will become cool right here.’\footnote{To “become cool” is, as the simile makes clear, to realize extinguishment (\textit{\textsanskrit{nibbāna}}). } 

Suppose\marginnote{24.6} an oil lamp depended on oil and a wick to burn. As the oil and the wick are used up, it would be extinguished due to not being fed. In the same way, feeling the end of the body approaching, they understand: ‘I feel the end of the body approaching.’ Feeling the end of life approaching, they understand: ‘I feel the end of life approaching.’ They understand: ‘When my body breaks up and my life has come to an end, everything that’s felt, since I no longer take pleasure in it, will become cool right here.’ 

Therefore\marginnote{25.1} a mendicant thus endowed is endowed with the ultimate foundation of wisdom. For this is the ultimate noble wisdom, namely, the knowledge of the ending of all suffering.\footnote{Here ends the discussion of the first “foundation”, wisdom, which is perfected by an arahant. The remaining foundations, while each having a mundane meaning, are explained from this highest of perspectives. } 

Their\marginnote{26.1} freedom, being founded on truth, is unshakable. For that which is false has a deceptive nature, while that which is true has an undeceptive nature—extinguishment. Therefore a mendicant thus endowed is endowed with the ultimate foundation of truth. For this is the ultimate noble truth, namely, that which has an undeceptive nature—extinguishment. 

In\marginnote{27.1} their ignorance, they used to acquire attachments. Those have been cut off at the root, made like a palm stump, obliterated so they are unable to arise in the future. Therefore a mendicant thus endowed is endowed with the ultimate foundation of generosity. For this is the ultimate noble generosity, namely, letting go of all attachments. 

In\marginnote{28.1} their ignorance, they used to be covetous, full of desire and lust. That has been cut off at the root, made like a palm stump, obliterated so it’s unable to arise in the future. In their ignorance, they used to be contemptuous, full of ill will and malevolence. That has been cut off at the root, made like a palm stump, obliterated so it’s unable to arise in the future. In their ignorance, they used to be ignorant, full of delusion. That has been cut off at the root, made like a palm stump, obliterated so it’s unable to arise in the future. Therefore a mendicant thus endowed is endowed with the ultimate foundation of peace. For this is the ultimate noble peace, namely, the pacification of greed, hate, and delusion. 

‘Do\marginnote{29.1} not neglect wisdom; preserve truth; foster generosity; and train only for peace.’ That’s what I said, and this is why I said it.\footnote{This concludes the discussion of the four foundations. } 

‘Where\marginnote{30.1} they stand, the streams of conceiving do not flow. And where the streams of conceiving do not flow, they are called a sage at peace.’\footnote{“Streams of conceiving” (\textit{\textsanskrit{maññassavā}}) is a unique image, allied to the notion that defilements may “stream on to” a person (\textit{\textsanskrit{āsavā} \textsanskrit{assaveyyuṁ}}, \href{https://suttacentral.net/an4.195/en/sujato\#2.2}{AN 4.195:2.2}). } That’s what I said, but why did I say it? 

These\marginnote{31.1} are all forms of conceiving: ‘I am’, ‘I am this’, ‘I will be’, ‘I will not be’, ‘I will have form’, ‘I will be formless’, ‘I will be percipient’, ‘I will be non-percipient’, ‘I will be neither percipient nor non-percipient.’ Conceiving is a disease, a boil, a dart. Having gone beyond all conceiving, one is called a sage at peace. The sage at peace is not reborn, does not grow old, and does not die. They are not shaken, and do not yearn. For they have nothing which would cause them to be reborn. Not being reborn, how could they grow old? Not growing old, how could they die? Not dying, how could they be shaken? Not shaking, for what could they yearn? 

‘Where\marginnote{32.1} they stand, the streams of conceiving do not flow. And where the streams of conceiving do not flow, they are called a sage at peace.’ That’s what I said, and this is why I said it. Mendicant, you should remember this brief analysis of the six elements.”\footnote{The “brief analysis” here is the summary teaching around which the sutta is developed (\href{https://suttacentral.net/mn140/en/sujato\#7.4}{MN 140:7.4}). } 

Then\marginnote{33.1} Venerable \textsanskrit{Pukkusāti} thought, “It seems the Teacher has come to me! The Holy One has come to me! The fully awakened Buddha has come to me!” He got up from his seat, arranged his robe over one shoulder, bowed with his head at the Buddha’s feet, and said, “I have made a mistake, sir. It was foolish, stupid, and unskillful of me to presume to address the Buddha as ‘reverend’. Please, sir, accept my mistake for what it is, so I will restrain myself in future.” 

“Indeed,\marginnote{33.5} mendicant, you made a mistake. It was foolish, stupid, and unskillful of you to act in that way. But since you have recognized your mistake for what it is, and have dealt with it properly, I accept it. For it is growth in the training of the Noble One to recognize a mistake for what it is, deal with it properly, and commit to restraint in the future.” 

“Sir,\marginnote{34.1} may I receive the going forth, the ordination in the Buddha’s presence?” 

“But\marginnote{34.2} mendicant, do you have a full set of bowl and robes?”\footnote{This is one of the questions asked of an ordination candidate (\href{https://suttacentral.net/pli-tv-kd1/en/sujato\#76.1.14}{Kd 1:76.1.14}). A mendicant is expected to have a bowl—made of metal or clay, not too big or too small—and three robes—the lower robe or sabong (\textit{\textsanskrit{antaravāsaka}}), the upper robe (\textit{\textsanskrit{uttarasaṅga}}), and the double-layered outer cloak (\textit{\textsanskrit{saṅghāṭi}}). In the suttas, the Buddha does not usually ask this question, so he must have observed that \textsanskrit{Pukkusāti}, who had renounced of his own volition, lacked the proper requisites. } 

“No,\marginnote{34.3} sir, I do not.” 

“The\marginnote{34.4} Realized Ones do not ordain those without a full set of bowl and robes.” 

And\marginnote{35.1} then Venerable \textsanskrit{Pukkusāti} approved and agreed with what the Buddha said. He got up from his seat, bowed, and respectfully circled the Buddha, keeping him on his right, before leaving. 

But\marginnote{35.2} while he was wandering in search of a bowl and robes, a stray cow took his life.\footnote{Cows were normally regarded as gentle creatures (\href{https://suttacentral.net/snp2.7/en/sujato\#27.1}{Snp 2.7:27.1}). They sometimes become violent, however, especially when protecting young calves, resulting in the tragic fates of \textsanskrit{Bāhiya} (\href{https://suttacentral.net/ud1.10/en/sujato\#10.2}{Ud 1.10:10.2}) and Suppabuddha (\href{https://suttacentral.net/ud5.3/en/sujato\#7.2}{Ud 5.3:7.2}). Attacks by cows still occur today in India and wherever cows are raised. } 

Then\marginnote{36.1} several mendicants went up to the Buddha, bowed, sat down to one side, and said to him, “Sir, the gentleman named \textsanskrit{Pukkusāti}, who was advised in brief by the Buddha, has passed away. Where has he been reborn in his next life?” 

“Mendicants,\marginnote{36.4} \textsanskrit{Pukkusāti} was astute. He practiced in line with the teachings, and did not trouble me about the teachings. With the ending of the five lower fetters, he’s been reborn spontaneously and will become extinguished there, not liable to return from that world.”\footnote{\textsanskrit{Pukkusāti} is one of a group of seven who attained arahantship in the Pure Abode of Aviha (\href{https://suttacentral.net/sn1.50/en/sujato\#3.2}{SN 1.50:3.2}, \href{https://suttacentral.net/sn2.24/en/sujato\#4.2}{SN 2.24:4.2}). } 

That\marginnote{36.6} is what the Buddha said. Satisfied, the mendicants approved what the Buddha said. 

%
\section*{{\suttatitleacronym MN 141}{\suttatitletranslation The Analysis of the Truths }{\suttatitleroot Saccavibhaṅgasutta}}
\addcontentsline{toc}{section}{\tocacronym{MN 141} \toctranslation{The Analysis of the Truths } \tocroot{Saccavibhaṅgasutta}}
\markboth{The Analysis of the Truths }{Saccavibhaṅgasutta}
\extramarks{MN 141}{MN 141}

\scevam{So\marginnote{1.1} I have heard. }At one time the Buddha was staying near Varanasi, in the deer park at Isipatana. There the Buddha addressed the mendicants, “Mendicants!” 

“Venerable\marginnote{1.5} sir,” they replied. The Buddha said this: 

“Near\marginnote{2.1} Varanasi, in the deer park at Isipatana, the Realized One, the perfected one, the fully awakened Buddha rolled forth the supreme Wheel of Dhamma. And that wheel cannot be rolled back by any ascetic or brahmin or god or \textsanskrit{Māra} or divinity or by anyone in the world.\footnote{This is in reference to the Dhammacakkappavattanasutta, the so-called First Sermon (\href{https://suttacentral.net/sn56.11/en/sujato}{SN 56.11}), for which the current discourse serves as a canonical commentary. } It is the teaching, advocating, establishing, clarifying, analyzing, and revealing of the four noble truths. What four? 

The\marginnote{3.1} noble truths of suffering, the origin of suffering, the cessation of suffering, and the practice that leads to the cessation of suffering.\footnote{The four noble truths are sometimes said to be like a doctor who identifies the illness, discerns its cause, understands the recovery, and administers the cure until the patient is restored to health (SA 389 at T ii 105a25; SA-2 254 at T ii 462c10). This analogy is not found in the Pali canon. } 

Near\marginnote{4.1} Varanasi, in the deer park at Isipatana, the Realized One, the perfected one, the fully awakened Buddha rolled forth the supreme Wheel of Dhamma. And that wheel cannot be rolled back by any ascetic or brahmin or god or \textsanskrit{Māra} or divinity or by anyone in the world. It is the teaching, advocating, establishing, clarifying, analyzing, and revealing of the four noble truths. 

Mendicants,\marginnote{5.1} you should cultivate friendship with \textsanskrit{Sāriputta} and \textsanskrit{Moggallāna}. You should associate with \textsanskrit{Sāriputta} and \textsanskrit{Moggallāna}. They’re astute, and they support their spiritual companions. \textsanskrit{Sāriputta} is like one who gives birth,\footnote{To extol his leading students, the Buddha recalls his own upbringing, born of \textsanskrit{Māyā} and raised by \textsanskrit{Mahāpajāpatī} (who appears in the next discourse). Nonetheless, \textit{\textsanskrit{janetā}} is a masculine noun and it would be misleading to translate it as “mother”. Both mother and father “give birth” (\textsanskrit{Bṛhadāraṇyaka} \textsanskrit{Upaniṣad} 6.4.14; the creator is male at 1.4.4). } while \textsanskrit{Moggallāna} is like one who raises the child.\footnote{\textit{\textsanskrit{Āpādetā}} is a causative agent noun from \textit{\textsanskrit{ā}} + \textit{√pad} in the sense “one who raises”. Again, both mother and father are “raisers” (\textit{\textsanskrit{āpādakā}}) of the child (\href{https://suttacentral.net/an4.63/en/sujato\#2.6}{AN 4.63:2.6}). } \textsanskrit{Sāriputta} guides people to the fruit of stream-entry, \textsanskrit{Moggallāna} to the highest goal. \textsanskrit{Sāriputta} is able to explain, teach, assert, establish, clarify, analyze, and reveal the four noble truths in detail.” 

That\marginnote{6.1} is what the Buddha said. When he had spoken, the Holy One got up from his seat and entered his dwelling. 

Then\marginnote{7.1} soon after the Buddha left, Venerable \textsanskrit{Sāriputta} said to the mendicants, “Reverends, mendicants!” 

“Reverend,”\marginnote{7.3} they replied. \textsanskrit{Sāriputta} said this: 

“Near\marginnote{8.1} Varanasi, in the deer park at Isipatana, the Realized One, the perfected one, the fully awakened Buddha rolled forth the supreme Wheel of Dhamma. And that wheel cannot be rolled back by any ascetic or brahmin or god or \textsanskrit{Māra} or divinity or by anyone in the world. It is the teaching, advocating, establishing, clarifying, analyzing, and revealing of the four noble truths. What four? 

The\marginnote{9.1} noble truths of suffering, the origin of suffering, the cessation of suffering, and the practice that leads to the cessation of suffering. 

And\marginnote{10.1} what is the noble truth of suffering? Rebirth is suffering; old age is suffering; death is suffering; sorrow, lamentation, pain, sadness, and distress are suffering; not getting what you wish for is suffering. In brief, the five grasping aggregates are suffering. 

And\marginnote{11.1} what is rebirth?\footnote{Here \textsanskrit{Sāriputta} takes the analysis a step further than the Dhammacakkappavattanasutta. This extended analysis of the truths was adopted from here into \href{https://suttacentral.net/dn22/en/sujato\#18.4}{DN 22:18.4}, which further expanded the sections on the second and third noble truths in Abhidhamma style shared with \href{https://suttacentral.net/vb4/en/sujato}{Vb 4}. | The definitions of rebirth, old age, and death are standard in dependent origination (\href{https://suttacentral.net/sn12.2/en/sujato\#4.2}{SN 12.2:4.2}), and are taught by \textsanskrit{Sāriputta} also at \href{https://suttacentral.net/mn9/en/sujato\#22.2}{MN 9:22.2}. } The rebirth, inception, conception, reincarnation, manifestation of the aggregates, and acquisition of the sense fields of the various sentient beings in the various orders of sentient beings. This is called rebirth. 

And\marginnote{12.1} what is old age? The old age, decrepitude, broken teeth, grey hair, wrinkly skin, diminished vitality, and failing faculties of the various sentient beings in the various orders of sentient beings. This is called old age. 

And\marginnote{13.1} what is death? The passing away, passing on, disintegration, demise, mortality, death, decease, breaking up of the aggregates, laying to rest of the corpse, and cutting off of the life faculty of the various sentient beings in the various orders of sentient beings. This is called death. 

And\marginnote{14.1} what is sorrow? The sorrow, sorrowing, state of sorrow, inner sorrow, inner deep sorrow in someone who has undergone misfortune, who has experienced suffering. This is called sorrow. 

And\marginnote{15.1} what is lamentation? The wail, lament, wailing, lamenting, state of wailing and lamentation in someone who has undergone misfortune, who has experienced suffering. This is called lamentation. 

And\marginnote{16.1} what is pain? Physical pain, physical unpleasantness, the painful, unpleasant feeling that’s born from physical contact. This is called pain. 

And\marginnote{17.1} what is sadness? Mental pain, mental displeasure, the painful, unpleasant feeling that’s born from mind contact. This is called sadness. 

And\marginnote{18.1} what is distress? The stress, distress, state of stress and distress in someone who has undergone misfortune, who has experienced suffering. This is called distress. 

And\marginnote{19.1} what is ‘not getting what you wish for is suffering’? In sentient beings who are liable to be reborn, such a wish arises: ‘Oh, if only we were not liable to be reborn! If only rebirth would not come to us!’ But you can’t get that by wishing. This is: ‘not getting what you wish for is suffering.’ In sentient beings who are liable to grow old … fall ill … die … experience sorrow, lamentation, pain, sadness, and distress, such a wish arises: ‘Oh, if only we were not liable to experience sorrow, lamentation, pain, sadness, and distress! If only sorrow, lamentation, pain, sadness, and distress would not come to us!’ But you can’t get that by wishing. This is: ‘not getting what you wish for is suffering.’ 

And\marginnote{20.1} what is ‘in brief, the five grasping aggregates are suffering’?\footnote{The five aggregates sum up the entirety of suffering, since all suffering is experienced within and by means of the aggregates. } They are the grasping aggregates that consist of form, feeling, perception, choices, and consciousness. This is called ‘in brief, the five grasping aggregates are suffering.’ This is called the noble truth of suffering. 

And\marginnote{21.1} what is the noble truth of the origin of suffering?\footnote{The second and third noble truths are here analyzed to the same level of depth as the First Sermon, with no further definition of terms. See \href{https://suttacentral.net/mn9/en/sujato\#16.2}{MN 9:16.2} for notes. } It’s the craving that leads to future lives, mixed up with relishing and greed, taking pleasure wherever it lands. That is, craving for sensual pleasures, craving to continue existence, and craving to end existence. This is called the noble truth of the origin of suffering. 

And\marginnote{22.1} what is the noble truth of the cessation of suffering? It’s the fading away and cessation of that very same craving with nothing left over; giving it away, letting it go, releasing it, and not clinging to it. This is called the noble truth of the cessation of suffering. 

And\marginnote{23.1} what is the noble truth of the practice that leads to the cessation of suffering? It is simply this noble eightfold path, that is: right view, right thought, right speech, right action, right livelihood, right effort, right mindfulness, and right immersion. 

And\marginnote{24.1} what is right view? Knowing about suffering, the origin of suffering, the cessation of suffering, and the practice that leads to the cessation of suffering.\footnote{From here we have the standard definitions of the items of the noble eightfold path (\href{https://suttacentral.net/sn45.8/en/sujato\#3.2}{SN 45.8:3.2}; compare \href{https://suttacentral.net/mn117/en/sujato}{MN 117}). | Ignorance is described in the opposite way (\href{https://suttacentral.net/mn9/en/sujato\#64-66.7}{MN 9:64–66.7}, \href{https://suttacentral.net/sn12.2/en/sujato\#14.2}{SN 12.2:14.2}). } This is called right view. 

And\marginnote{25.1} what is right thought? Thoughts of renunciation, good will, and harmlessness. This is called right thought. 

And\marginnote{26.1} what is right speech?\footnote{The items of right speech and right action are covered in greater detail under the ten pathways of skilful deeds at \href{https://suttacentral.net/mn41/en/sujato\#12.4}{MN 41:12.4}. } Refraining from lying, divisive speech, harsh speech, and talking nonsense. This is called right speech. 

And\marginnote{27.1} what is right action? Refraining from killing living creatures, stealing, and sexual misconduct. This is called right action. 

And\marginnote{28.1} what is right livelihood? It’s when a noble disciple gives up wrong livelihood and earns a living by right livelihood. This is called right livelihood. 

And\marginnote{29.1} what is right effort? It’s when a mendicant generates enthusiasm, tries, makes an effort, exerts the mind, and strives so that bad, unskillful qualities don’t arise. They generate enthusiasm, try, make an effort, exert the mind, and strive so that bad, unskillful qualities that have arisen are given up. They generate enthusiasm, try, make an effort, exert the mind, and strive so that skillful qualities arise. They generate enthusiasm, try, make an effort, exert the mind, and strive so that skillful qualities that have arisen remain, are not lost, but increase, mature, and are completed by development. This is called right effort. 

And\marginnote{30.1} what is right mindfulness? It’s when a mendicant meditates by observing an aspect of the body—keen, aware, and mindful, rid of covetousness and displeasure for the world. They meditate observing an aspect of feelings … mind … principles—keen, aware, and mindful, rid of covetousness and displeasure for the world. This is called right mindfulness. 

And\marginnote{31.1} what is right immersion? It’s when a mendicant, quite secluded from sensual pleasures, secluded from unskillful qualities, enters and remains in the first absorption, which has the rapture and bliss born of seclusion, while placing the mind and keeping it connected. As the placing of the mind and keeping it connected are stilled, they enter and remain in the second absorption, which has the rapture and bliss born of immersion, with internal clarity and mind at one, without placing the mind and keeping it connected. And with the fading away of rapture, they enter and remain in the third absorption, where they meditate with equanimity, mindful and aware, personally experiencing the bliss of which the noble ones declare, ‘Equanimous and mindful, one meditates in bliss.’ Giving up pleasure and pain, and ending former happiness and sadness, they enter and remain in the fourth absorption, without pleasure or pain, with pure equanimity and mindfulness. This is called right immersion. This is called the noble truth of the practice that leads to the cessation of suffering. 

Near\marginnote{32.1} Varanasi, in the deer park at Isipatana, the Realized One, the perfected one, the fully awakened Buddha rolled forth the supreme Wheel of Dhamma. And that wheel cannot be rolled back by any ascetic or brahmin or god or \textsanskrit{Māra} or divinity or by anyone in the world. It is the teaching, advocating, establishing, clarifying, analyzing, and revealing of the four noble truths.” 

That’s\marginnote{32.3} what Venerable \textsanskrit{Sāriputta} said. Satisfied, the mendicants approved what \textsanskrit{Sāriputta} said. 

%
\section*{{\suttatitleacronym MN 142}{\suttatitletranslation The Analysis of Religious Donations }{\suttatitleroot Dakkhiṇāvibhaṅgasutta}}
\addcontentsline{toc}{section}{\tocacronym{MN 142} \toctranslation{The Analysis of Religious Donations } \tocroot{Dakkhiṇāvibhaṅgasutta}}
\markboth{The Analysis of Religious Donations }{Dakkhiṇāvibhaṅgasutta}
\extramarks{MN 142}{MN 142}

\scevam{So\marginnote{1.1} I have heard. }At one time the Buddha was staying in the land of the Sakyans, near Kapilavatthu in the Banyan Tree Monastery. 

Then\marginnote{2.1} \textsanskrit{Mahāpajāpati} \textsanskrit{Gotamī} approached the Buddha bringing a new pair of garments. She bowed, sat down to one side, and said to the Buddha,\footnote{\textsanskrit{Mahāpajāpati} was the Buddha’s foster-mother, sister of his birth mother \textsanskrit{Māyā}. She is honored in the tradition as the first nun, the founder of the \textsanskrit{bhikkhunī} order (\href{https://suttacentral.net/an8.51/en/sujato\#20.1}{AN 8.51:20.1}). She was honored as the most senior of the nuns (\href{https://suttacentral.net/an1.235/en/sujato\#1.1}{AN 1.235:1.1}). Though mostly remembered as the Buddha’s mother, her own verses deconstruct her motherhood, relating how she has previously been a “mother, a son, a father, a brother, and a grandmother” but now has transcended all such states (\href{https://suttacentral.net/thig6.6/en/sujato\#3.1}{Thig 6.6:3.1}). } “Sir, I have spun and woven this new pair of garments specially for the Buddha.\footnote{\textsanskrit{Mahāpajāpati} offers on behalf of her son alone, whereas elsewhere offerings are said to be “specially for the Buddha and the mendicant \textsanskrit{Saṅgha}” \href{https://suttacentral.net/an5.30/en/sujato\#2.4}{AN 5.30:2.4}, \href{https://suttacentral.net/an6.42/en/sujato\#2.4}{AN 6.42:2.4}, \href{https://suttacentral.net/an8.86/en/sujato\#2.5}{AN 8.86:2.5}. } May the Buddha please accept this from me out of sympathy.” 

When\marginnote{2.4} she said this, the Buddha said to her, “Give it to the \textsanskrit{Saṅgha}, \textsanskrit{Gotamī}. When you give to the \textsanskrit{Saṅgha}, both the \textsanskrit{Saṅgha} and I will be honored.”\footnote{The Buddha gently tries to deflect \textsanskrit{Mahāpajāpati}’s familial attachment. } 

For\marginnote{2.7} a second time … 

For\marginnote{2.13} a third time, \textsanskrit{Mahāpajāpati} \textsanskrit{Gotamī} said to the Buddha, “Sir, I have spun and woven this new pair of garments specially for the Buddha. May the Buddha please accept this from me out of sympathy.” 

And\marginnote{2.16} for a third time, the Buddha said to her, “Give it to the \textsanskrit{Saṅgha}, \textsanskrit{Gotamī}. When you give to the \textsanskrit{Saṅgha}, both the \textsanskrit{Saṅgha} and I will be honored.” 

When\marginnote{3.1} he said this, Venerable Ānanda said to the Buddha, “Sir, please accept the new pair of garments from \textsanskrit{Mahāpajāpati} \textsanskrit{Gotamī}. Sir, \textsanskrit{Mahāpajāpati} was very helpful to the Buddha. As his aunt, she raised him, nurtured him, and gave him her milk. When the Buddha’s birth mother passed away, she nurtured him at her own breast. 

And\marginnote{3.5} the Buddha has been very helpful to \textsanskrit{Mahāpajāpati}. It is owing to the Buddha that \textsanskrit{Mahāpajāpati} has gone for refuge to the Buddha, the teaching, and the \textsanskrit{Saṅgha}. It’s owing to the Buddha that she refrains from killing living creatures, stealing, committing sexual misconduct, lying, and consuming beer, wine, and liquor intoxicants. It’s owing to the Buddha that she has experiential confidence in the Buddha, the teaching, and the \textsanskrit{Saṅgha}, and has the ethics loved by the noble ones. It’s owing to the Buddha that she is free of doubt regarding suffering, its origin, its cessation, and the practice that leads to its cessation.\footnote{This description makes it sound like she is a virtuous laywoman who has attained stream-entry. Yet one of the parallels refers to her as a \textit{\textsanskrit{bhikkhunī}} (T 84 at T i 903c22). And while she was supposed to be the first \textit{\textsanskrit{bhikkhunī}}, the text below refers to the \textit{\textsanskrit{bhikkhunī}} order; see further discussion there (\href{https://suttacentral.net/mn142/en/sujato\#7.9}{MN 142:7.9}). } The Buddha has been very helpful to \textsanskrit{Mahāpajāpati}.” 

“That’s\marginnote{4.1} so true, Ānanda. When someone has enabled you to go for refuge, it’s not easy to repay them by bowing down to them, rising up for them, greeting them with joined palms, and observing proper etiquette for them; or by providing them with robes, almsfood, lodgings, and medicines and supplies for the sick. 

When\marginnote{4.4} someone has enabled you to refrain from killing, stealing, sexual misconduct, lying, and beer, wine, and liquor intoxicants, it’s not easy to repay them … 

When\marginnote{4.6} someone has enabled you to have experiential confidence in the Buddha, the teaching, and the \textsanskrit{Saṅgha}, and the ethics loved by the noble ones, it’s not easy to repay them … 

When\marginnote{4.8} someone has enabled you to be free of doubt regarding suffering, its origin, its cessation, and the practice that leads to its cessation, it’s not easy to repay them by bowing down to them, rising up for them, greeting them with joined palms, and observing proper etiquette for them; or by providing them with robes, almsfood, lodgings, and medicines and supplies for the sick. 

Ānanda,\marginnote{5.1} there are these fourteen religious donations to individuals.\footnote{“Religious donation” (or “honorarium”, \textit{\textsanskrit{dakkhiṇā}}, Sanskrit \textit{\textsanskrit{dakṣiṇā}}) is a Vedic term for the payment owed a brahmin priest for performing ritual services. In the Rig Veda this gift was of cattle offered at the dawn service; it yields benefits as the cow yields milk, and as Dawn sheds her bounteous light. Later it was found that gold or other precious things served just as well. The \textit{\textsanskrit{dakṣiṇā}} was said to absolve the sacrificer from the guilt of killing (Śatapatha \textsanskrit{Brāhmaṇa} 1.2.3.4). This is an overriding concern of the Śatapatha \textsanskrit{Brāhmaṇa}, which emphasizes how the dead of the sacrifice—a slain animal or pressed soma—is “enlivened” through the \textit{\textsanskrit{dakṣiṇā}} (2.2.2.2). In the Pali tradition, \textit{\textsanskrit{dakkhiṇa}} is etymologically linked with a gift given by the “right” or “capable” hand. In \href{https://suttacentral.net/snp3.4/en/sujato}{Snp 3.4}, a brahmin gives the rice-cake left over from a fire oblation as a \textit{\textsanskrit{dakkhiṇā}}. } What fourteen? One gives a gift to the Realized One, the perfected one, the fully awakened Buddha. This is the first religious donation to an individual. One gives a gift to an independent Buddha. This is the second religious donation to an individual. One gives a gift to a perfected one. This is the third religious donation to an individual. One gives a gift to someone practicing to realize the fruit of perfection. This is the fourth religious donation to an individual. One gives a gift to a non-returner. This is the fifth religious donation to an individual. One gives a gift to someone practicing to realize the fruit of non-return. This is the sixth religious donation to an individual. One gives a gift to a once-returner. This is the seventh religious donation to an individual. One gives a gift to someone practicing to realize the fruit of once-return. This is the eighth religious donation to an individual. One gives a gift to a stream-enterer. This is the ninth religious donation to an individual. One gives a gift to someone practicing to realize the fruit of stream-entry. This is the tenth religious donation to an individual. One gives a gift to an outsider who is free of desire for sensual pleasures. This is the eleventh religious donation to an individual. One gives a gift to an ordinary person who has good ethical conduct. This is the twelfth religious donation to an individual. One gives a gift to an ordinary person who has bad ethical conduct. This is the thirteenth religious donation to an individual. One gives a gift to an animal. This is the fourteenth religious donation to an individual.\footnote{Jamison and Brereton translate \textit{\textsanskrit{dakṣiṇā}} as “priestly gift”, which is correct for the Rig Veda. But whereas the Vedic tradition means the \textit{\textsanskrit{dakkhiṇā}} for brahmins alone, the Buddha redefines it to include all humans, including those of bad conduct, and even animals as well. All creatures are deserving of kindness, and such kindness is always a sacred act. } 

Now,\marginnote{6.1} Ānanda, gifts to the following persons may be expected to yield the following returns. To an animal, a hundred times. To an unethical ordinary person, a thousand. To an ethical ordinary person, a hundred thousand. To an outsider free of desire for sensual pleasures, 10,000,000,000. But a gift to someone practicing to realize the fruit of stream-entry may be expected to yield incalculable, immeasurable returns. How much more so a gift to a stream-enterer, someone practicing to realize the fruit of once-return, a once-returner, someone practicing to realize the fruit of non-return, a non-returner, someone practicing to realize the fruit of perfection, a perfected one, or an independent Buddha? How much more so a Realized One, a perfected one, a fully awakened Buddha?\footnote{The idea of such a multiplying scale is anticipated in \textsanskrit{Bṛhadāraṇyaka} \textsanskrit{Upaniṣad} 4.3.33, which speaks of the multiplying joy in human and divine realms. } 

But\marginnote{7.1} there are, Ānanda, seven religious donations bestowed on a \textsanskrit{Saṅgha}. What seven? One gives a gift to the communities of both monks and nuns headed by the Buddha. This is the first religious donation bestowed on a \textsanskrit{Saṅgha}. One gives a gift to the communities of both monks and nuns after the Buddha has finally become quenched.\footnote{Following the passing of the Buddha, a complete dispensation has both \textit{bhikkhus} and \textit{\textsanskrit{bhikkhunīs}}. } This is the second religious donation bestowed on a \textsanskrit{Saṅgha}. One gives a gift to the \textsanskrit{Saṅgha} of monks. This is the third religious donation bestowed on a \textsanskrit{Saṅgha}. One gives a gift to the \textsanskrit{Saṅgha} of nuns.\footnote{If \textsanskrit{Mahāpajāpati} was still a lay person at this point, this mention of the \textit{\textsanskrit{bhikkhunī}} order could be read in one of two ways. One possibility is that she was not, in fact, the first \textit{\textsanskrit{bhikkhunī}}, an argument that I have made previously. In brief, the passages that assert she was the first \textit{\textsanskrit{bhikkhunī}}, while widely shared in the tradition, are full of textual problems that indicate they were added at a somewhat late date, which I think was around the time of the Second Council (see \href{https://suttacentral.net/an8.51/en/sujato}{AN 8.51}). Further,  the \textsanskrit{Therīgāthā} is the most important early account of \textit{\textsanskrit{bhikkhunīs}} in their own voices, yet no nun mentions \textsanskrit{Mahāpajāpati} as teacher or founder of the order, while the verses of \textsanskrit{Bhaddā} \textsanskrit{Kuṇḍalakesā} appear to predate the formal ordination procedure ascribed to \textsanskrit{Mahāpajāpati} (\href{https://suttacentral.net/thig5.9/en/sujato}{Thig 5.9}). On the other hand, if this argument is incorrect, it might be the case that the order of nuns is mentioned here since the Buddha knew from the start of his dispensation that he would establish it (\href{https://suttacentral.net/dn16/en/sujato\#3.8.4}{DN 16:3.8.4}). Under this reading, the Buddha is perhaps dropping a hint for \textsanskrit{Mahāpajāpati} to ordain. Regardless of her ordination status at this point, the praise of the order of nuns acts as an encouragement, balancing out the mild rebuke with which the discourse started (compare the Buddha’s words to Ānanda at \href{https://suttacentral.net/dn16/en/sujato\#5.14.2}{DN 16:5.14.2}). } This is the fourth religious donation bestowed on a \textsanskrit{Saṅgha}. One gives a gift, thinking: ‘Appoint this many monks and nuns for me from the \textsanskrit{Saṅgha}.’ This is the fifth religious donation bestowed on a \textsanskrit{Saṅgha}. One gives a gift, thinking: ‘Appoint this many monks for me from the \textsanskrit{Saṅgha}.’ This is the sixth religious donation bestowed on a \textsanskrit{Saṅgha}. One gives a gift, thinking: ‘Appoint this many nuns for me from the \textsanskrit{Saṅgha}.’ This is the seventh religious donation bestowed on a \textsanskrit{Saṅgha}. 

In\marginnote{8.1} times to come there will be lambs of the flock wearing a scrap of ocher cloth, unethical and of bad character.\footnote{In early Pali, \textit{\textsanskrit{gotrabhū}} is always used for the very least of those who are worthy of gifts (\href{https://suttacentral.net/an9.10/en/sujato\#1.3}{AN 9.10:1.3}, \href{https://suttacentral.net/an10.16/en/sujato\#1.3}{AN 10.16:1.3}). It has a positive sense, this negative use being quite isolated. Later Buddhist traditions take \textit{\textsanskrit{gotrabhū}} as “one who is joining the clan”, the first entry on the path to awakening, but I doubt this is what it means in early texts. The Buddha recognized clan as a convention (\href{https://suttacentral.net/mn98/en/sujato\#12.1}{MN 98:12.1}), but it was abandoned by those gone forth (\href{https://suttacentral.net/an8.19/en/sujato\#14.2}{AN 8.19:14.2}), and he considered questions as to his own clan inappropriate (\href{https://suttacentral.net/snp3.4/en/sujato\#4.3}{Snp 3.4:4.3}) as it has nothing to do with purification (\href{https://suttacentral.net/sn2.20/en/sujato\#3.4}{SN 2.20:3.4}). Further, “clan” is always spelled \textit{gotta} in Pali. The older Vedic  spelling \textit{gotra} points to the Vedic meaning of \textit{gotra} as “cow-pen”, a safe place to gather the herd of cattle, and by extension a name for the herd itself. And while the \textsanskrit{Saṅgha} is never called a “clan” it is compared to a herd of cattle; \href{https://suttacentral.net/mn34/en/sujato}{MN 34} develops an extended simile where the different kinds of cattle in the herd are compared with different members of the \textsanskrit{Saṅgha}. Now, in that discourse the last to cross over are the “followers of teachings, followers by faith”. Like a newborn calf, they can only cross the ford with the assistance of their mother (\href{https://suttacentral.net/mn34/en/sujato\#10.1}{MN 34:10.1}); in \href{https://suttacentral.net/an10.16/en/sujato\#1.3}{AN 10.16:1.3} the \textit{\textsanskrit{gotrabhū}} follows even them. If we are on the right track, the suffix \textit{-\textsanskrit{bhū}} may be derived from \textit{\textsanskrit{bhūna}} (Sanskrit \textit{\textsanskrit{bhrūṇa}}) in the sense “baby”, or perhaps “embryo” (\href{https://suttacentral.net/mn75/en/sujato\#5.3}{MN 75:5.3}). In English we use “flock” for a religious group under the protection of a leader, so we can render \textit{\textsanskrit{gotrabhū}} as “babies of the flock” or “lambs of the flock”. | For \textit{\textsanskrit{kāsāvakaṇṭha}} as “scrap of ochre cloth”, see note on \href{https://suttacentral.net/iti48/en/sujato\#4.1}{Iti 48:4.1}. } People will give gifts to those unethical people in the name of the \textsanskrit{Saṅgha}. Even then, I say, a religious donation bestowed on the \textsanskrit{Saṅgha} is incalculable and immeasurable.\footnote{This rests uneasily with the statement below that a donation is purified by the recipient. This statement is not found in most parallels, so it is perhaps a later addition. But as it stands one argument might be that, since these offerings are made to the community as a whole, the merit that accrues from the community is not despoiled by the fact that the specific recipients are of bad character. } But I say that there is no way a personal offering can be more fruitful than one bestowed on a \textsanskrit{Saṅgha}. 

Ānanda,\marginnote{9.1} there are these four ways of purifying a religious donation.\footnote{This passage is at \href{https://suttacentral.net/an4.78/en/sujato}{AN 4.78} and included at \href{https://suttacentral.net/dn33/en/sujato\#1.11.185}{DN 33:1.11.185}. } What four? There’s a religious donation that’s purified by the giver, not the recipient. There’s a religious donation that’s purified by the recipient, not the giver. There’s a religious donation that’s purified by neither the giver nor the recipient. There’s a religious donation that’s purified by both the giver and the recipient. 

And\marginnote{10.1} how is a religious donation purified by the giver, not the recipient? It’s when the giver is ethical, of good character, but the recipient is unethical, of bad character. 

And\marginnote{11.1} how is a religious donation purified by the recipient, not the giver? It’s when the giver is unethical, of bad character, but the recipient is ethical, of good character. 

And\marginnote{12.1} how is a religious donation purified by neither the giver nor the recipient? It’s when both the giver and the recipient are unethical, of bad character. 

And\marginnote{13.1} how is a religious donation purified by both the giver and the recipient? It’s when both the giver and the recipient are ethical, of good character. These are the four ways of purifying a religious donation.” 

That\marginnote{14.1} is what the Buddha said. Then the Holy One, the Teacher, went on to say:\footnote{The discourse is wrapped up more satisfactorily in one of the parallels, as it omits these verses and instead concludes with \textsanskrit{Mahāpajāpati} offering the robes to the \textsanskrit{Saṅgha} (T 84 at T i 904b15). } 

\begin{verse}%
“When\marginnote{14.3} an ethical person with trusting heart \\
gives a proper gift to unethical persons, \\
trusting in the ample fruit of deeds, \\
that offering is purified by the giver. 

When\marginnote{14.7} an unethical and untrusting person, \\
gives an improper gift to ethical persons, \\
not trusting in the ample fruit of deeds, \\
that offering is purified by the receivers. 

When\marginnote{14.11} an unethical and untrusting person, \\
gives an improper gift to unethical persons, \\
not trusting in the ample fruit of deeds, \\
I declare that gift is not very fruitful. 

When\marginnote{14.15} an ethical person with trusting heart \\
gives a proper gift to ethical persons, \\
trusting in the ample fruit of deeds, \\
I declare that gift is abundantly fruitful. 

But\marginnote{14.19} when a passionless one gives to the passionless \\
a proper gift with trusting heart, \\
trusting in the ample fruit of deeds, \\
that’s truly the best of material gifts.” 

%
\end{verse}

%
\addtocontents{toc}{\let\protect\contentsline\protect\nopagecontentsline}
\chapter*{The Chapter on the Six Senses}
\addcontentsline{toc}{chapter}{\tocchapterline{The Chapter on the Six Senses}}
\addtocontents{toc}{\let\protect\contentsline\protect\oldcontentsline}

%
\section*{{\suttatitleacronym MN 143}{\suttatitletranslation Advice to Anāthapiṇḍika }{\suttatitleroot Anāthapiṇḍikovādasutta}}
\addcontentsline{toc}{section}{\tocacronym{MN 143} \toctranslation{Advice to Anāthapiṇḍika } \tocroot{Anāthapiṇḍikovādasutta}}
\markboth{Advice to Anāthapiṇḍika }{Anāthapiṇḍikovādasutta}
\extramarks{MN 143}{MN 143}

\scevam{So\marginnote{1.1} I have heard. }At one time the Buddha was staying near \textsanskrit{Sāvatthī} in Jeta’s Grove, \textsanskrit{Anāthapiṇḍika}’s monastery. 

Now\marginnote{2.1} at that time the householder \textsanskrit{Anāthapiṇḍika} was sick, suffering, gravely ill.\footnote{Renowned as the foremost of benefactors, \textsanskrit{Anāthapiṇḍika} appears in over twenty discourses in the \textsanskrit{Saṁyutta} and \textsanskrit{Aṅguttara} \textsanskrit{Nikāyas}, discussing matters such as generosity, guarding the mind, and the development of meditation; the story of his conversion is recorded in the Vinaya (\href{https://suttacentral.net/pli-tv-kd16/en/sujato\#4.1.1}{Kd 16:4.1.1}). He is well known for his offering of the monastery at which the Buddha most often stayed, but this discourse marks his only personal appearance in the Majjhima. } Then he addressed a man, “Please, mister, go to the Buddha, and in my name bow with your head to his feet. Say to him: ‘Sir, the householder \textsanskrit{Anāthapiṇḍika} is sick, suffering, gravely ill. He bows with his head to your feet.’ Then go to Venerable \textsanskrit{Sāriputta}, and in my name bow with your head to his feet. Say to him: ‘Sir, the householder \textsanskrit{Anāthapiṇḍika} is sick, suffering, gravely ill. He bows with his head to your feet.’ And then say: ‘Sir, please visit him at his home out of sympathy.’” 

“Yes,\marginnote{2.11} sir,” that man replied. He did as \textsanskrit{Anāthapiṇḍika} asked. \textsanskrit{Sāriputta} consented with silence. 

Then\marginnote{3.1} Venerable \textsanskrit{Sāriputta} robed up in the morning and, taking his bowl and robe, went with Venerable Ānanda as his second monk to \textsanskrit{Anāthapiṇḍika}’s home. He sat down on the seat spread out, and said to \textsanskrit{Anāthapiṇḍika}, “I hope you’re keeping well, householder; I hope you’re all right. And I hope the pain is fading, not growing, that its fading is evident, not its growing.”\footnote{The sick \textsanskrit{Anāthapiṇḍika} was consoled by Ānanda at \href{https://suttacentral.net/sn55.27/en/sujato\#3.2}{SN 55.27:3.2} and by \textsanskrit{Sāriputta} at \href{https://suttacentral.net/sn55.26/en/sujato\#4.2}{SN 55.26:4.2}. in other cases, we find the Buddha making such visits to sick monks (\href{https://suttacentral.net/sn22.87/en/sujato\#2.7}{SN 22.87:2.7}, \href{https://suttacentral.net/sn22.88/en/sujato\#2.8}{SN 22.88:2.8}, \href{https://suttacentral.net/sn46.14/en/sujato\#2.1}{SN 46.14:2.1}, \href{https://suttacentral.net/sn46.15/en/sujato\#2.1}{SN 46.15:2.1}, \href{https://suttacentral.net/an6.56/en/sujato\#2.1}{AN 6.56:2.1}), but for some reason visits to householders are only attested for \textsanskrit{Sāriputta} (\href{https://suttacentral.net/mn97/en/sujato\#28.2}{MN 97:28.2}) or Ānanda (\href{https://suttacentral.net/sn47.29/en/sujato\#2.2}{SN 47.29:2.2}). } 

“I’m\marginnote{4.1} not keeping well, Honorable \textsanskrit{Sāriputta}, I’m not getting by. The pain is terrible and growing, not fading, its growing, not its fading, is evident. The winds piercing my head are so severe, it feels like a strong man drilling into my head with a sharp point. The pain in my head is so severe, it feels like a strong man tightening a tough leather strap around my head. The winds slicing my belly are so severe, like a deft butcher or their apprentice were slicing open a cows’s belly with a sharp meat cleaver. The burning in my body is so severe, it feels like two strong men grabbing a weaker man by the arms to burn and scorch him on a pit of glowing coals. That’s how severe the burning is in my body. I’m not keeping well, Honorable \textsanskrit{Sāriputta}, I’m not getting by. The pain is terrible and growing, not fading, its growing, not its fading, is evident.” 

“That’s\marginnote{5.1} why, householder, you should train like this: ‘I shall not grasp the eye, and there shall be no consciousness of mine dependent on the eye.’\footnote{The teaching on the six senses typically begins by establishing awareness on the reality of the eye (etc.) as something that is grasped at (eg. \href{https://suttacentral.net/mn145/en/sujato\#3.1}{MN 145:3.1}). Here \textsanskrit{Sāriputta} skips over this straight to letting go, signfying that his teaching starts with an advanced level of insight. } That’s how you should train. 

You\marginnote{5.4} should train like this: ‘I shall not grasp the ear, and there shall be no consciousness of mine dependent on the ear.’ … 

‘I\marginnote{5.7} shall not grasp the nose, and there shall be no consciousness of mine dependent on the nose.’ … 

‘I\marginnote{5.10} shall not grasp the tongue, and there shall be no consciousness of mine dependent on the tongue.’ … 

‘I\marginnote{5.13} shall not grasp the body, and there shall be no consciousness of mine dependent on the body.’ … 

‘I\marginnote{5.16} shall not grasp the mind, and there shall be no consciousness of mine dependent on the mind.’ That’s how you should train. 

You\marginnote{6.1} should train like this: ‘I shall not grasp sight, and there shall be no consciousness of mine dependent on sight.’ … ‘I shall not grasp sound … smell … taste … touch … idea, and there shall be no consciousness of mine dependent on idea.’ That’s how you should train. 

You\marginnote{7.1} should train like this: ‘I shall not grasp eye consciousness, and there shall be no consciousness of mine dependent on eye consciousness.’ … ‘I shall not grasp ear consciousness … nose consciousness … tongue consciousness … body consciousness … mind consciousness, and there shall be no consciousness of mine dependent on mind consciousness.’ That’s how you should train. 

You\marginnote{8.1} should train like this: ‘I shall not grasp eye contact … ear contact … nose contact … tongue contact … body contact … mind contact, and there shall be no consciousness of mine dependent on mind contact.’ That’s how you should train. 

You\marginnote{9.1} should train like this: ‘I shall not grasp feeling born of eye contact … feeling born of ear contact … feeling born of nose contact … feeling born of tongue contact … feeling born of body contact … feeling born of mind contact, and there shall be no consciousness of mine dependent on the feeling born of mind contact.’ That’s how you should train. 

You\marginnote{10.1} should train like this: ‘I shall not grasp the earth element … water element … fire element … air element … space element … consciousness element, and there shall be no consciousness of mine dependent on the consciousness element.’ That’s how you should train. 

You\marginnote{11.1} should train like this: ‘I shall not grasp form … feeling … perception … choices … consciousness, and there shall be no consciousness of mine dependent on consciousness.’ That’s how you should train. 

You\marginnote{12.1} should train like this: ‘I shall not grasp the dimension of infinite space … the dimension of infinite consciousness … the dimension of nothingness … the dimension of neither perception nor non-perception, and there shall be no consciousness of mine dependent on the dimension of neither perception nor non-perception.’ That’s how you should train. 

You\marginnote{13.1} should train like this: ‘I shall not grasp this world, and there shall be no consciousness of mine dependent on this world.’ That’s how you should train. 

You\marginnote{14.1} should train like this: ‘I shall not grasp the other world, and there shall be no consciousness of mine dependent on the other world.’ That’s how you should train. You should train like this: ‘I shall not grasp whatever is seen, heard, thought, known, attained, sought, and explored by my mind, and my consciousness will not have that as support.’ That’s how you should train.” 

When\marginnote{15.1} he said this, \textsanskrit{Anāthapiṇḍika} cried and burst out in tears. Venerable Ānanda said to him, “Are you failing, householder? Are you fading, householder?” 

“No,\marginnote{15.4} Honorable Ānanda. But for a long time I have paid homage to the Buddha and the esteemed mendicants. Yet I have never before heard such a Dhamma talk.” 

“Householder,\marginnote{15.7} such Dhamma talk does not strike when teaching white-clothed laypeople.\footnote{The verb here is \textit{\textsanskrit{paṭibhāti}}, the spontaneous striking of inspiration. \textsanskrit{Sāriputta} is not saying such talks are never given, but that such topics are not what comes to mind when teaching laypeople. } Rather, it strikes when teaching those gone forth.” 

“Well\marginnote{15.9} then, Honorable \textsanskrit{Sāriputta}, let such Dhamma talk strike when teaching white-clothed laypeople as well! There are gentlemen with little dust in their eyes. They’re in decline because they haven’t heard the teaching. There will be those who understand the teaching!” 

And\marginnote{16.1} when the venerables \textsanskrit{Sāriputta} and Ānanda had given the householder \textsanskrit{Anāthapiṇḍika} this advice they got up from their seat and left. Not long after they had left, \textsanskrit{Anāthapiṇḍika} passed away and was reborn in the host of joyful gods. 

Then,\marginnote{17.1} late at night, the glorious god \textsanskrit{Anāthapiṇḍika}, lighting up the entire Jeta’s Grove, went up to the Buddha, bowed, stood to one side, and addressed the Buddha in verse: 

\begin{verse}%
“This\marginnote{17.3} is indeed that Jeta’s Grove, \\
frequented by the \textsanskrit{Saṅgha} of seers, \\
where the King of Dhamma stayed: \\
it brings me joy! 

Deeds,\marginnote{17.7} knowledge, and principle; \\
ethical conduct, an excellent livelihood; \\
by these are mortals purified, \\
not by clan or wealth. 

That’s\marginnote{17.11} why an astute person, \\
seeing what’s good for themselves, \\
would examine the teaching rationally, \\
and thus be purified in it. 

\textsanskrit{Sāriputta}\marginnote{17.15} is full of wisdom, \\
ethics, and peace. \\
Even a mendicant who has crossed over \\
might at best equal him.” 

%
\end{verse}

This\marginnote{18.1} is what the god \textsanskrit{Anāthapiṇḍika} said, and the teacher approved. Then the god \textsanskrit{Anāthapiṇḍika}, knowing that the teacher approved, bowed and respectfully circled the Buddha, keeping him on his right, before vanishing right there. 

Then,\marginnote{19.1} when the night had passed, the Buddha told the mendicants all that had happened. 

When\marginnote{20.1} he had spoken, Venerable Ānanda said to the Buddha: 

“Sir,\marginnote{20.2} that god must surely have been \textsanskrit{Anāthapiṇḍika}. For the householder \textsanskrit{Anāthapiṇḍika} was devoted to Venerable \textsanskrit{Sāriputta}.” 

“Good,\marginnote{20.4} good, Ānanda. You’ve reached the logical conclusion, as far as logic goes. For that was indeed the god \textsanskrit{Anāthapiṇḍika}.” 

That\marginnote{20.7} is what the Buddha said. Satisfied, Venerable Ānanda approved what the Buddha said. 

%
\section*{{\suttatitleacronym MN 144}{\suttatitletranslation Advice to Channa }{\suttatitleroot Channovādasutta}}
\addcontentsline{toc}{section}{\tocacronym{MN 144} \toctranslation{Advice to Channa } \tocroot{Channovādasutta}}
\markboth{Advice to Channa }{Channovādasutta}
\extramarks{MN 144}{MN 144}

\scevam{So\marginnote{1.1} I have heard.\footnote{This discourse recurs at \href{https://suttacentral.net/sn35.87/en/sujato}{SN 35.87}. In the \textsanskrit{Saṁyutta} \textsanskrit{Nikāya}, as here in the Majjhima, it is immediately followed by the \textsanskrit{Puṇṇovādasutta} (\href{https://suttacentral.net/mn145/en/sujato}{MN 145}, \href{https://suttacentral.net/sn35.88/en/sujato}{SN 35.88}). } }At one time the Buddha was staying near \textsanskrit{Rājagaha}, in the Bamboo Grove, the squirrels’ feeding ground. 

Now\marginnote{2.1} at that time the venerables \textsanskrit{Sāriputta}, \textsanskrit{Mahācunda}, and Channa were staying on the Vulture’s Peak Mountain.\footnote{For \textsanskrit{Mahācunda}, see note on \href{https://suttacentral.net/mn8/en/sujato\#2.1}{MN 8:2.1}. | This Channa only appears only in this discourse. He is evidently not the same person as the Budddha’s former charioteer who was famously admonished on the Buddha’s deathbed (\href{https://suttacentral.net/dn16/en/sujato\#6.4.1}{DN 16:6.4.1}). } 

Now\marginnote{3.1} at that time Venerable Channa was sick, suffering, gravely ill. 

Then\marginnote{3.2} in the late afternoon, Venerable \textsanskrit{Sāriputta} came out of retreat, went to Venerable \textsanskrit{Mahācunda} and said to him, “Come, Reverend Cunda, let’s go to see Venerable Channa and ask about his illness.” 

“Yes,\marginnote{3.4} reverend,” replied \textsanskrit{Mahācunda}. 

And\marginnote{4.1} then \textsanskrit{Sāriputta} and \textsanskrit{Mahācunda} went to see Channa and exchanged greetings with him. When the greetings and polite conversation were over, they sat down to one side. Then \textsanskrit{Sāriputta} said to Channa, “I hope you’re keeping well, Reverend Channa; I hope you’re all right. I hope that your pain is fading, not growing, that its fading is evident, not its growing.” 

“Reverend\marginnote{5.1} \textsanskrit{Sāriputta}, I’m not keeping well, I’m not getting by. The pain is terrible and growing, not fading; its growing is evident, not its fading. The winds piercing my head are so severe, it feels like a strong man drilling into my head with a sharp point. The pain in my head is so severe, it feels like a strong man tightening a tough leather strap around my head. The winds slicing my belly are so severe, like a deft butcher or their apprentice were slicing open a cows’s belly with a sharp meat cleaver. The burning in my body is so severe, it feels like two strong men grabbing a weaker man by the arms to burn and scorch him on a pit of glowing coals. I’m not keeping well, I’m not getting by. The pain is terrible and growing, not fading; its growing is evident, not its fading. Reverend \textsanskrit{Sāriputta}, I will take my life. I don’t wish to live.” 

“Please\marginnote{6.1} don’t take your life! Venerable Channa, keep going! We want you to keep going. If you don’t have any suitable food, we’ll find it for you. If you don’t have suitable medicine, we’ll find it for you. If you don’t have a capable carer, we’ll find one for you. Please don’t take your life! Venerable Channa, keep going! We want you to keep going.” 

“Reverend\marginnote{7.1} \textsanskrit{Sāriputta}, it’s not that I don’t have suitable food, or suitable medicine, or a capable carer. Moreover, for a long time now I have served the Teacher with love, not without love. For it is proper for a disciple to serve the Teacher with love, not without love. You should remember this: ‘The mendicant Channa will take his life blamelessly.’” 

“I’d\marginnote{8.1} like to ask you about a certain point, if you’d take the time to answer.” 

“Ask,\marginnote{8.2} Reverend \textsanskrit{Sāriputta}. When I’ve heard it I’ll know.” 

“Reverend\marginnote{9.1} Channa, do you regard the eye, eye consciousness, and things knowable by eye consciousness in this way: ‘This is mine, I am this, this is my self’? Do you regard the ear … nose … tongue … body … mind, mind consciousness, and things knowable by mind consciousness in this way: ‘This is mine, I am this, this is my self’?” 

“Reverend\marginnote{9.7} \textsanskrit{Sāriputta}, I regard the eye, eye consciousness, and things knowable by eye consciousness in this way: ‘This is not mine, I am not this, this is not my self.’ I regard the ear … nose … tongue … body … mind, mind consciousness, and things knowable by mind consciousness in this way: ‘This is not mine, I am not this, this is not my self’.” 

“Reverend\marginnote{10.1} Channa, what have you seen, what have you directly known in these things that you regard them in this way: ‘This is not mine, I am not this, this is not my self’?” 

“Reverend\marginnote{10.7} \textsanskrit{Sāriputta}, after seeing cessation, after directly knowing cessation in these things I regard them in this way: ‘This is not mine, I am not this, this is not my self’.” 

When\marginnote{11.1} he said this, Venerable \textsanskrit{Mahācunda} said to Venerable Channa: 

“So,\marginnote{11.2} Reverend Channa, you should regularly apply your mind to this instruction of the Buddha:\footnote{\textsanskrit{Mahācunda} is about to quote a phrase that appears at \href{https://suttacentral.net/ud8.4/en/sujato}{Ud 8.4}. } ‘For the dependent there is agitation. For the independent there’s no agitation.\footnote{The “dependent” is the world of conditions. The “independent” is Nibbana. | “Agitation” is \textit{calita}, otherwise “shaking, trembling”. } When there’s no agitation there is tranquility. When there is tranquility there’s no inclination.\footnote{“Inclination” is \textit{nati}, from a root meaning “to bend”. Sometimes used in the sense of a “inclination of mind” (\href{https://suttacentral.net/mn19/en/sujato\#6.1}{MN 19:6.1}), here, as the following lines make clear, it refers to the mind inclining to rebirth; explained by the commentary as “craving”. Compare the English idiom, “one is bent on destruction”. } When there’s no inclination there’s no coming and going. When there’s no coming and going there’s no passing away and reappearing. When there’s no passing away and reappearing there’s no this world or world beyond or between the two.\footnote{Vedic cosmology speaks of a tripartite world: the earth, the heavens, and the “midspace” (\textit{\textsanskrit{antarikṣa}}), namely the atmosphere. The gods journey through the midspace to earth for the sacrifice, bringing bounty for mortals (eg. Rig Veda 7.45). Later, the process of rebirth was conceived of as a path of cosmological ascent returning to the world of the gods or the fathers (eg. \textsanskrit{Bṛhadāraṇyaka} \textsanskrit{Upaniṣad} 5.10.1). For the suttas, rebirth is likewise seen as a process of moving between states, like a person leaving a house, walking down the street, and entering another house (\href{https://suttacentral.net/dn2/en/sujato\#96.1}{DN 2:96.1}). } Just this is the end of suffering.’” And when the venerables \textsanskrit{Sāriputta} and \textsanskrit{Mahācunda} had given Venerable Channa this advice they got up from their seat and left. 

Not\marginnote{12.1} long after those venerables had left, Venerable Channa took his life. 

Then\marginnote{13.1} \textsanskrit{Sāriputta} went up to the Buddha, bowed, sat down to one side, and said to him, “Sir, Venerable Channa has taken his life. Where has he been reborn in his next life?” 

“\textsanskrit{Sāriputta},\marginnote{13.4} didn’t the mendicant Channa declare his blamelessness to you personally?” 

“Sir,\marginnote{13.5} there is a Vajjian village named Pubbajira.\footnote{There are several variant spellings of this name, which does not seem to appear elsewhere. } There Channa had families who were friendly, intimate, and hospitable.”\footnote{The text plays with homonyms of \textit{upavajja} as “blameworthy” (Sanskrit \textit{upavadya}) and as “hospitable” (commentary: \textit{\textsanskrit{upasaṅkamitabba}}, Sanskrit \textit{upavrajya}). } 

“The\marginnote{13.7} mendicant Channa did indeed have such families. But this is not enough for me to call someone ‘blameworthy’. When someone lays down this body and takes up another body, I call them ‘blameworthy’. But the mendicant Channa did no such thing.\footnote{The Buddha confirms that Channa was an arahant. } You should remember this: ‘The mendicant Channa took his life blamelessly.’”\footnote{Suicide generally is condemned in Buddhism as the taking of a human life. But from this and similar incidents (\href{https://suttacentral.net/sn4.23/en/sujato}{SN 4.23}, \href{https://suttacentral.net/sn22.87/en/sujato}{SN 22.87}) it is widely accepted that suicide in the final throes of a terminal illness is allowable for arahants, as they are incapable of committing a blameworthy act. The suttas, however, can also be read to support the reverse reading: the reason it is allowable for arahants is because it is not blameworthy. The underlying logic would be that ending life is normally bad because life is valuable, but at such a point life no longer has any value. Obviously this is a complex issue that cannot be resolved by a few short scriptural passages, but a Buddhist would be motivated by compassion and understanding for those going through such terrible suffering. } 

That\marginnote{13.12} is what the Buddha said. Satisfied, Venerable \textsanskrit{Sāriputta} approved what the Buddha said. 

%
\section*{{\suttatitleacronym MN 145}{\suttatitletranslation Advice to Puṇṇa }{\suttatitleroot Puṇṇovādasutta}}
\addcontentsline{toc}{section}{\tocacronym{MN 145} \toctranslation{Advice to Puṇṇa } \tocroot{Puṇṇovādasutta}}
\markboth{Advice to Puṇṇa }{Puṇṇovādasutta}
\extramarks{MN 145}{MN 145}

\scevam{So\marginnote{1.1} I have heard.\footnote{This sutta recurs at \href{https://suttacentral.net/sn35.88/en/sujato}{SN 35.88}), with slight differences in the opening and closing. In addition to two discourse parallels in Chinese, the popularity of this narrative is attested by its appearance in later texts such as the \textsanskrit{Mūlasarvāstivāda} Vinaya and the \textsanskrit{Divyāvadāna}, as well as in artworks (see \textsanskrit{Anālayo}, \emph{Comparative Study}, vol. ii, p. 828). } }At one time the Buddha was staying near \textsanskrit{Sāvatthī} in Jeta’s Grove, \textsanskrit{Anāthapiṇḍika}’s monastery. 

Then\marginnote{1.3} in the late afternoon, Venerable \textsanskrit{Puṇṇa} came out of retreat and went to the Buddha. He bowed, sat down to one side, and said to the Buddha,\footnote{This \textsanskrit{Puṇṇa} is to be distinguished from the Koliyan ascetic of \href{https://suttacentral.net/mn57/en/sujato}{MN 57} and the son of \textsanskrit{Mantāṇī} of \href{https://suttacentral.net/mn23/en/sujato}{MN 23}. His only other appearance is his \textsanskrit{Theragāthā} verse at \href{https://suttacentral.net/thag1.70/en/sujato}{Thag 1.70}. } “Sir, may the Buddha please teach me Dhamma in brief. When I’ve heard it, I’ll live alone, withdrawn, diligent, keen, and resolute.” 

“Well\marginnote{2.2} then, \textsanskrit{Puṇṇa}, listen and apply your mind well, I will speak.” 

“Yes,\marginnote{2.3} sir,” replied \textsanskrit{Puṇṇa}. The Buddha said this: 

“\textsanskrit{Puṇṇa},\marginnote{3.1} there are sights known by the eye, which are likable, desirable, agreeable, pleasant, sensual, and arousing. If a mendicant approves, welcomes, and keeps clinging to them, this gives rise to relishing. Relishing is the origin of suffering, I say. 

There\marginnote{3.5} are sounds known by the ear … smells known by the nose … tastes known by the tongue … touches known by the body … ideas known by the mind, which are likable, desirable, agreeable, pleasant, sensual, and arousing. If a mendicant approves, welcomes, and keeps clinging to them, this gives rise to relishing. Relishing is the origin of suffering, I say. 

There\marginnote{4.1} are sights known by the eye, which are likable, desirable, agreeable, pleasant, sensual, and arousing. If a mendicant doesn’t approve, welcome, and keep clinging to them, relishing ceases. When relishing ceases, suffering ceases, I say. 

There\marginnote{4.5} are sounds known by the ear … smells known by the nose … tastes known by the tongue … touches known by the body … ideas known by the mind, which are likable, desirable, agreeable, pleasant, sensual, and arousing. If a mendicant doesn’t approve, welcome, and keep clinging to them, relishing ceases. When relishing ceases, suffering ceases, I say. 

\textsanskrit{Puṇṇa},\marginnote{5.1} now that I’ve given you this brief advice, what country will you live in?” 

“Sir,\marginnote{5.2} there’s a country named \textsanskrit{Sunāparanta}. I shall live there.”\footnote{The commentary says that \textsanskrit{Puṇṇa} was in fact born in \textsanskrit{Sunāparanta}, specifically the trading port of \textsanskrit{Suppāraka}, which is modern Nallasopara north of Mumbai on India’s west coast. \textsanskrit{Sunāparanta} is otherwise known as Aparanta (“Far West”, \href{https://suttacentral.net/bv29/en/sujato\#17.2}{Bv 29:17.2}). \textsanskrit{Puṇṇa}’s choice was far-sighted, as from the time of Ashoka, \textsanskrit{Suppāraka} became a major trading center with ports in India and lands west as far as Mesopotamia, Egypt, Eastern Africa, and Rome. A fabulous tale of such trading voyages is told in \href{https://suttacentral.net/ja463/en/sujato}{Ja 463}. } 

“The\marginnote{5.3} people of \textsanskrit{Sunāparanta} are wild and rough, \textsanskrit{Puṇṇa}. If they abuse and insult you, what will you think of them?” 

“If\marginnote{5.6} they abuse and insult me, I will think: ‘These people of \textsanskrit{Sunāparanta} are gracious, truly gracious, since they don’t hit me with their fists.’ That’s what I’ll think, Blessed One. That’s what I’ll think, Holy One.” 

“But\marginnote{5.10} if they do hit you with their fists, what will you think of them then?” 

“If\marginnote{5.11} they hit me with their fists, I’ll think: ‘These people of \textsanskrit{Sunāparanta} are gracious, truly gracious, since they don’t throw stones at me.’ That’s what I’ll think, Blessed One. That’s what I’ll think, Holy One.” 

“But\marginnote{5.15} if they do throw stones at you, what will you think of them then?” 

“If\marginnote{5.16} they throw stones at me, I’ll think: ‘These people of \textsanskrit{Sunāparanta} are gracious, truly gracious, since they don’t beat me with a club.’ That’s what I’ll think, Blessed One. That’s what I’ll think, Holy One.” 

“But\marginnote{5.20} if they do beat you with a club, what will you think of them then?” 

“If\marginnote{5.21} they beat me with a club, I’ll think: ‘These people of \textsanskrit{Sunāparanta} are gracious, truly gracious, since they don’t stab me with a knife.’ That’s what I’ll think, Blessed One. That’s what I’ll think, Holy One.” 

“But\marginnote{5.25} if they do stab you with a knife, what will you think of them then?” 

“If\marginnote{5.26} they stab me with a knife, I’ll think: ‘These people of \textsanskrit{Sunāparanta} are gracious, truly gracious, since they don’t take my life with a sharp knife.’ That’s what I’ll think, Blessed One. That’s what I’ll think, Holy One.” 

“But\marginnote{5.30} if they do take your life with a sharp knife, what will you think of them then?” 

“If\marginnote{5.31} they take my life with a sharp knife, I’ll think: ‘There are disciples of the Buddha who looked for something to take their life because they were horrified, repelled, and disgusted with the body and with life. And I have found this without looking!’\footnote{This is in reference to the notorious story of \textsanskrit{Migalaṇḍika} in \href{https://suttacentral.net/pli-tv-bu-vb-pj3/en/sujato}{Bu Pj 3}. Several monks, meditating improperly on the unattractiveness of the body, sought suicide as the way out. In response, the Buddha taught mindfulness of breathing, which is “peaceful and sublime, a deliciously pleasant meditation”. | The idiomatic phrase \textit{\textsanskrit{satthahāraka}} and its verbal form \textit{\textsanskrit{satthaṁ} \textsanskrit{āharitaṁ}} are difficult. The obvious reading is to take \textit{sattha} as “knife”, the meaning it has in the preceding sentence. However, that yields the sense “knife-bringer”, “assasin”, whereas the use of \textit{\textsanskrit{idaṁ}} in the current passage shows it must be neuter, i.e. it is a thing not a person. Richard Gombrich suggests we read \textit{sattha} here as equivalent to Sanskrit \textit{\textsanskrit{śvasita}}, “breathing, life” (see Brahmali’s note on \href{https://suttacentral.net/pli-tv-bu-vb-pj3/en/sujato\#2.49.1}{Bu Pj 3:2.49.1}). This agrees with the Vinaya commentary, which explains, “What does it take? Life.” (\textit{Kiṃ harati? \textsanskrit{Jīvitaṁ}}). } That’s what I’ll think, Blessed One. That’s what I’ll think, Holy One.” 

“Good,\marginnote{6.1} good \textsanskrit{Puṇṇa}! Having such self-control and peacefulness, you will be quite capable of living in \textsanskrit{Sunāparanta}. Now, \textsanskrit{Puṇṇa}, go at your convenience.” 

And\marginnote{7.1} then \textsanskrit{Puṇṇa} welcomed and agreed with the Buddha’s words. He got up from his seat, bowed, and respectfully circled the Buddha, keeping him on his right. Then he set his lodgings in order and, taking his bowl and robe, set out for \textsanskrit{Sunāparanta}. Traveling stage by stage, he arrived at \textsanskrit{Sunāparanta}, and stayed there. Within that rainy season he confirmed around five hundred male and five hundred female lay followers. And within that same rainy season he realized the three knowledges. Some time later he became fully extinguished.\footnote{The parallel passage at \href{https://suttacentral.net/sn35.88/en/sujato\#18.7}{SN 35.88:18.7}, says rather that he passed away in that same rainy season. } 

Then\marginnote{8.1} several mendicants went up to the Buddha, bowed, sat down to one side, and said to him, “Sir, the gentleman named \textsanskrit{Puṇṇa}, who was advised in brief by the Buddha, has passed away. Where has he been reborn in his next life?” 

“Mendicants,\marginnote{8.4} \textsanskrit{Puṇṇa} was astute. He practiced in line with the teachings, and did not trouble me about the teachings. \textsanskrit{Puṇṇa} has become fully quenched.” 

That\marginnote{8.6} is what the Buddha said. Satisfied, the mendicants approved what the Buddha said. 

%
\section*{{\suttatitleacronym MN 146}{\suttatitletranslation Advice from Nandaka }{\suttatitleroot Nandakovādasutta}}
\addcontentsline{toc}{section}{\tocacronym{MN 146} \toctranslation{Advice from Nandaka } \tocroot{Nandakovādasutta}}
\markboth{Advice from Nandaka }{Nandakovādasutta}
\extramarks{MN 146}{MN 146}

\scevam{So\marginnote{1.1} I have heard.\footnote{This discourse deals with the fortnightly teaching of the nuns, a practice ostensibly set up at \textsanskrit{Mahāpajāpati}’s ordination (\href{https://suttacentral.net/an8.51/en/sujato\#13.1}{AN 8.51:13.1}, \href{https://suttacentral.net/pli-tv-kd20/en/sujato\#1.4.7}{Kd 20:1.4.7}) and mandated in the Vinaya (\href{https://suttacentral.net/pli-tv-bu-vb-pc21/en/sujato}{Bu Pc 21}, \href{https://suttacentral.net/pli-tv-bi-vb-pc59/en/sujato}{Bi Pc 59}). This is the only mention of such a teaching in the discourses. However it does not follow the pattern laid out in the Vinaya, where the teaching consists of the eight \textit{garudhammas} (\href{https://suttacentral.net/pli-tv-bi-vb-pc58/en/sujato}{Bi Pc 58}), and the monks cannot go to the nuns’ monastery to advise them (\href{https://suttacentral.net/pli-tv-bu-vb-pc23/en/sujato}{Bu Pc 23}). } }At one time the Buddha was staying near \textsanskrit{Sāvatthī} in Jeta’s Grove, \textsanskrit{Anāthapiṇḍika}’s monastery. 

Then\marginnote{2.1} \textsanskrit{Mahāpajāpati} \textsanskrit{Gotamī} together with around five hundred nuns approached the Buddha, bowed, stood to one side, and said to him,\footnote{Here we see \textsanskrit{Mahāpajāpati} in the leadership role that would be expected were she the founder of the nuns. However, this is the only case where this occurs. } “Sir, may the Buddha please advise and instruct the nuns. Please give the nuns a Dhamma talk.” 

Now\marginnote{3.1} at that time the senior monks were taking turns to advise the nuns.\footnote{Monks must be appointed by the \textsanskrit{Saṅgha} to take on this role, and they must fulfill eight factors: they must be virtuous and learned, have memorized both Monastic Codes, be well-spoken, liked by the nuns, and capable, never having committed a heavy offense against a nun; and they must have been ordained at least twenty years (\href{https://suttacentral.net/pli-tv-bu-vb-pc21/en/sujato}{Bu Pc 21}). } But Venerable Nandaka didn’t want to take his turn.\footnote{Nandaka was praised for his teaching of monks (\href{https://suttacentral.net/an9.4/en/sujato}{AN 9.4}). At \href{https://suttacentral.net/an3.66/en/sujato}{AN 3.66} he taught the layman \textsanskrit{Sāḷha} how to discern the true teaching, using the method of questioning as he does in the present discourse. On the strength of this teaching, Nandaka was declared the foremost in advising nuns (\href{https://suttacentral.net/an1.229/en/sujato\#1.1}{AN 1.229:1.1}). His four \textsanskrit{Theragāthā} verses are at \href{https://suttacentral.net/thag4.4/en/sujato}{Thag 4.4} according to the commentary. A second Nandaka at \href{https://suttacentral.net/thag2.27/en/sujato}{Thag 2.27} was the brother of Bharata (\href{https://suttacentral.net/thag2.28/en/sujato}{Thag 2.28}). | While the text does not explain why he was reluctant to teach, the commentary says that those nuns were his wives in past lives and he wanted to avoid the potential for gossip. } 

Then\marginnote{3.3} the Buddha said to Venerable Ānanda, “Ānanda, whose turn is it to advise the nuns today?” 

“It’s\marginnote{3.5} Nandaka’s turn, sir, but he doesn’t want to do it.” 

Then\marginnote{4.1} the Buddha said to Nandaka, “Nandaka, please advise and instruct the nuns. Please, brahmin, give the nuns a Dhamma talk.” 

“Yes,\marginnote{4.5} sir,” replied Nandaka. Then, in the morning, he robed up and, taking his bowl and robe, entered \textsanskrit{Sāvatthī} for alms. He wandered for alms in \textsanskrit{Sāvatthī}. After the meal, on his return from almsround, he went to the Royal Monastery with a companion.\footnote{The Royal Monastery (\textit{\textsanskrit{rājakārāma}}) was a residence for nuns near \textsanskrit{Sāvatthī}, also featured in \href{https://suttacentral.net/sn55.11/en/sujato}{SN 55.11}. } Those nuns saw him coming off in the distance, so they spread out a seat and placed water for washing the feet. Nandaka sat down on the seat spread out, and washed his feet. Those nuns bowed, and sat down to one side. 

Nandaka\marginnote{4.12} said to them, “Sisters, this talk shall be in the form of questions. When you understand, say so. When you don’t understand, say so. If anyone has a doubt or uncertainty, ask me about it: ‘Why, sir, does it say this? What does that mean?’”\footnote{Nandaka does not lecture the nuns, but engages with them. He makes it clear that, against patriarchal expectations, the nuns should ask for clarifications. } 

“We’re\marginnote{5.5} already delighted and satisfied with Master Nandaka, since he invites us like this.”\footnote{Thus fulfilling one of the factors required for a teacher of nuns. } 

“What\marginnote{6.1} do you think, sisters? Is the eye permanent or impermanent?”\footnote{The discussion begins with the most obvious factors, the internal sense organs, and proceeds to analyze each in terms of the three characteristics: impermanence, suffering, and not-self. Nandaka ensures that the nuns are with him each step. } 

“Impermanent,\marginnote{6.3} sir.” 

“But\marginnote{6.4} if it’s impermanent, is it suffering or happiness?” 

“Suffering,\marginnote{6.5} sir.” 

“But\marginnote{6.6} if it’s impermanent, suffering, and perishable, is it fit to be regarded thus: ‘This is mine, I am this, this is my self’?” 

“No,\marginnote{6.8} sir.” 

“What\marginnote{6.9} do you think, sisters? Is the ear … nose … tongue … body … mind permanent or impermanent?” 

“Impermanent,\marginnote{6.19} sir.” 

“But\marginnote{6.20} if it’s impermanent, is it suffering or happiness?” 

“Suffering,\marginnote{6.21} sir.” 

“But\marginnote{6.22} if it’s impermanent, suffering, and perishable, is it fit to be regarded thus: ‘This is mine, I am this, this is my self’?” 

“No,\marginnote{6.24} sir. Why is that? Because we have already truly seen this with right wisdom: ‘So these six interior sense fields are impermanent.’”\footnote{This level of direct insight normally indicates stream-entry. See below at \href{https://suttacentral.net/mn146/en/sujato\#27.5}{MN 146:27.5}. } 

“Good,\marginnote{6.28} good, sisters! That’s how it is for a noble disciple who truly sees with right wisdom. 

What\marginnote{7.1} do you think, sisters? Are sights permanent or impermanent?” 

“Impermanent,\marginnote{7.3} sir.” 

“But\marginnote{7.4} if they're impermanent, are they suffering or happiness?” 

“Suffering,\marginnote{7.5} sir.” 

“But\marginnote{7.6} if they're impermanent, suffering, and perishable, are they fit to be regarded thus: ‘This is mine, I am this, this is my self’?” 

“No,\marginnote{7.8} sir.” 

“What\marginnote{7.9} do you think, sisters? Are sounds … smells … tastes … touches … ideas permanent or impermanent?” 

“Impermanent,\marginnote{7.19} sir.” 

“But\marginnote{7.20} if they're impermanent, are they suffering or happiness?” 

“Suffering,\marginnote{7.21} sir.” 

“But\marginnote{7.22} if they're impermanent, suffering, and perishable, are they fit to be regarded thus: ‘This is mine, I am this, this is my self’?” 

“No,\marginnote{7.24} sir. Why is that? Because we have already truly seen this with right wisdom: ‘So these six exterior sense fields are impermanent.’” 

“Good,\marginnote{7.28} good, sisters! That’s how it is for a noble disciple who truly sees with right wisdom. 

What\marginnote{8.1} do you think, sisters? Is eye consciousness … ear consciousness … nose consciousness … tongue consciousness … body consciousness … mind consciousness permanent or impermanent?” 

“Impermanent,\marginnote{8.18} sir.” 

“But\marginnote{8.19} if it’s impermanent, is it suffering or happiness?” 

“Suffering,\marginnote{8.20} sir.” 

“But\marginnote{8.21} if it’s impermanent, suffering, and perishable, is it fit to be regarded thus: ‘This is mine, I am this, this is my self’?” 

“No,\marginnote{8.23} sir. Why is that? Because we have already truly seen this with right wisdom: ‘So these six classes of consciousness are impermanent.’” 

“Good,\marginnote{8.27} good, sisters! That’s how it is for a noble disciple who truly sees with right wisdom. 

Suppose\marginnote{9.1} there was an oil lamp burning. The oil, wick, flame, and light were all impermanent and perishable.\footnote{While the imagery of the dependence of the flame on its fuel is common in the suttas, this exact metaphor is not found elsewhere. Normally the Buddha employs metaphors in an ABA structure: he explains the doctrine, illustrates it, and repeats the explanation. Here, however, the metaphor acts as a springboard for the next stage of the argument, the impermanence of feelings. } Now, suppose someone was to say: ‘While this oil lamp is burning, the oil, the wick, and the flame are all impermanent and perishable. But the light is permanent, lasting, eternal, and imperishable.’ Would they be speaking rightly?” 

“No,\marginnote{9.6} sir. Why is that? Because that oil lamp’s oil, wick, and flame are all impermanent and perishable, let alone the light.”\footnote{The Self is often said to be a light (eg. \textsanskrit{Bṛhadāraṇyaka} \textsanskrit{Upaniṣad} 4.3.7). } 

“In\marginnote{9.10} the same way, suppose someone was to say: ‘These six interior sense fields are impermanent. But the feeling—whether pleasant, painful, or neutral—that I experience due to these six interior sense fields is permanent, lasting, eternal, and imperishable.’ Would they be speaking rightly?” 

“No,\marginnote{9.14} sir. Why is that? Because each kind of feeling arises dependent on the corresponding condition. When the corresponding condition ceases, the appropriate feeling ceases.” 

“Good,\marginnote{9.18} good, sisters! That’s how it is for a noble disciple who truly sees with right wisdom. 

Suppose\marginnote{10.1} there was a large tree standing with heartwood. The roots, trunk, branches and leaves, and shadow were all impermanent and perishable.\footnote{This is another unique application of a familiar image, although a similar metaphor for the six senses as the shadow of a post is found at \href{https://suttacentral.net/an4.195/en/sujato\#11.1}{AN 4.195:11.1}. } Now, suppose someone was to say: ‘There’s a large tree standing with heartwood. The roots, trunk, and branches and leaves are all impermanent and perishable. But the shadow is permanent, lasting, eternal, and imperishable.’ Would they be speaking rightly?” 

“No,\marginnote{10.5} sir. Why is that? Because that large tree’s roots, trunk, and branches and leaves are all impermanent and perishable, let alone the shadow.”\footnote{Divinity is identified, albeit inadequately, with a shadow at \textsanskrit{Bṛhadāraṇyaka} \textsanskrit{Upaniṣad} 2.1.12. From Vedic times the shadow was both the place of relief from the hot sun (Rig Veda 6.16.38, 2.33.6) and the shadow of death (10.121.2). } 

“In\marginnote{10.9} the same way, suppose someone was to say: ‘These six exterior sense fields are impermanent.\footnote{The metaphor of shadow for the exterior sense fields is the inverse of that for the interior sense fields, light. This speaks to the high degree of care with which Nandaka structured his teaching. Further, the movement from light to shadow subconsciously prepares the audience for the darker metaphor to come. } But the feeling—whether pleasant, painful, or neutral—that I experience due to these six exterior sense fields is permanent, lasting, eternal, and imperishable.’ Would they be speaking rightly?” 

“No,\marginnote{10.13} sir. Why is that? Because each kind of feeling arises dependent on the corresponding condition. When the corresponding condition ceases, the appropriate feeling ceases.” 

“Good,\marginnote{10.17} good, sisters! That’s how it is for a noble disciple who truly sees with right wisdom. 

Suppose\marginnote{11.1} a deft butcher or their apprentice was to kill a cow and carve it with a sharp meat cleaver. Without damaging the flesh inside or the hide outside,\footnote{A similar image is applied to analysis by way of the elements (\href{https://suttacentral.net/mn10/en/sujato\#12.3}{MN 10:12.3} = \href{https://suttacentral.net/mn119/en/sujato\#8.3}{MN 119:8.3}). } they’d cut, carve, sever, and slice through the connecting tendons, sinews, and ligaments with a sharp meat cleaver,\footnote{\textit{\textsanskrit{Antarā}} is “in-between”, “connecting”, not “inner”. } and then peel off the outer hide. Then they’d wrap that cow up in that very same hide and say: ‘This cow is joined to its hide just like before.’ Would they be speaking rightly?” 

“No,\marginnote{11.6} sir. Why is that? Because even if they wrap that cow up in that very same hide and say: ‘This cow is joined to its hide just like before,’ still that cow is not joined to that hide.” 

“I’ve\marginnote{12.1} made up this simile to make a point. And this is the point. ‘The inner flesh’ is a term for the six interior sense fields. ‘The outer hide’ is a term for the six exterior sense fields. ‘The connecting tendons, sinews, and ligaments’ is a term for greed and relishing. ‘A sharp meat cleaver’ is a term for noble wisdom. And it is that noble wisdom which cuts, carves, severs, and slices the connecting corruption, fetter, and bond. 

Sisters,\marginnote{13.1} by developing and cultivating these seven awakening factors, a mendicant realizes the undefiled freedom of heart and freedom by wisdom in this very life. And they live having realized it with their own insight due to the ending of defilements.\footnote{Knowing that the nuns have good insight, Nandaka encourages them to continue deepening their meditation practice. } What seven? It’s when a mendicant develops the awakening factors of mindfulness, investigation of principles, energy, rapture, tranquility, immersion, and equanimity, which rely on seclusion, fading away, and cessation, and ripen as letting go. It is by developing and cultivating these seven awakening factors that a mendicant realizes the undefiled freedom of heart and freedom by wisdom in this very life. And they live having realized it with their own insight due to the ending of defilements.” 

Then\marginnote{14.1} after giving this advice to the nuns, Nandaka dismissed them, saying, “Go, sisters, it is time.”\footnote{This seems at odds with the detail that Nandaka had gone to the nuns’ monastery to teach. Parallels in Chinese and Tibetan say instead that Nandaka left ( SA 276 at T ii 75c2; T 1442 at T xxiii 793c12, and D (3) ’dul ba, ja 57b7). } 

And\marginnote{14.3} then those nuns approved and agreed with what Nandaka had said. They got up from their seat, bowed, and respectfully circled him, keeping him on their right. Then they went up to the Buddha, bowed, and stood to one side. The Buddha said to them, “Go, nuns, it is time.” 

Then\marginnote{14.5} those nuns bowed to the Buddha respectfully circled him, keeping him on their right, before departing. 

Soon\marginnote{15.1} after those nuns had left, the Buddha addressed the mendicants: “Suppose, mendicants, it was the sabbath of the fourteenth day. You wouldn’t get lots of people wondering whether the moon is full or not, since it is obviously not full. 

In\marginnote{15.4} the same way, those nuns were uplifted by Nandaka’s Dhamma teaching, but they still haven’t got all they wished for.” 

Then\marginnote{16.1} the Buddha said to Nandaka, “Well then, Nandaka, tomorrow you should give those nuns the same advice again.” 

“Yes,\marginnote{16.3} sir,” Nandaka replied. And the next day he went to those nuns, and all unfolded just like the previous day. 

Soon\marginnote{27.1} after those nuns had left, the Buddha addressed the mendicants: “Suppose, mendicants, it was the sabbath of the fifteenth day. You wouldn’t get lots of people wondering whether the moon is full or not, since it is obviously full. In the same way, those nuns were uplifted by Nandaka’s Dhamma teaching, and they got all they wished for. Even the last of these five hundred nuns is a stream-enterer, not liable to be reborn in the underworld, bound for awakening.”\footnote{As noted above, the nuns’ statements indicate that they had attained stream-entry even before being taught by Nandaka. Elsewhere, to have fulfilled one’s wishes like the moon on the fifteenth day unequivocally means arahantship (\href{https://suttacentral.net/thag10.2/en/sujato\#10.1}{Thag 10.2:10.1}). Indeed, parallels indicate that at this point they have all attained arahantship (SA 276 at T ii 75c16; T 1442 at T xxiii 794a14; D (3) ’dul ba, ja 59a1). Remarkably, this detail is also confirmed in the commentaries to the \textsanskrit{Aṅguttara} \textsanskrit{Nikāya}, \textsanskrit{Theragāthā}, and \textsanskrit{Therīgāthā}, which say they became arahants when listening to Nandaka’s second teaching. } 

That\marginnote{27.6} is what the Buddha said. Satisfied, the mendicants approved what the Buddha said. 

%
\section*{{\suttatitleacronym MN 147}{\suttatitletranslation The Shorter Advice to Rāhula }{\suttatitleroot Cūḷarāhulovādasutta}}
\addcontentsline{toc}{section}{\tocacronym{MN 147} \toctranslation{The Shorter Advice to Rāhula } \tocroot{Cūḷarāhulovādasutta}}
\markboth{The Shorter Advice to Rāhula }{Cūḷarāhulovādasutta}
\extramarks{MN 147}{MN 147}

\scevam{So\marginnote{1.1} I have heard.\footnote{This discourse recurs at \href{https://suttacentral.net/sn35.121/en/sujato}{SN 35.121}. } }At one time the Buddha was staying near \textsanskrit{Sāvatthī} in Jeta’s Grove, \textsanskrit{Anāthapiṇḍika}’s monastery. 

Then\marginnote{1.3} as he was in private retreat this thought came to his mind, “The qualities that ripen in freedom have ripened in \textsanskrit{Rāhula}.\footnote{Five perceptions that ripen in freedom are taught at \href{https://suttacentral.net/dn33/en/sujato\#2.1.139}{DN 33:2.1.139}, but apart from that we do not encounter this idea in the discourses. | In \href{https://suttacentral.net/sn18.1/en/sujato}{SN 18.1}–5, \textsanskrit{Rāhula} requests a teaching so he can go on retreat. The teachings given there can be compared with the parallel, which says the Buddha first told \textsanskrit{Rāhula} to teach the five aggregates, the six sense-spheres, and conditionality, and then to go on retreat and meditate on the same teachings (SA 200 at T ii 51a23). Only then did the Buddha consider him ready. } Why don’t I lead him further to the ending of defilements?” 

Then\marginnote{1.6} the Buddha robed up in the morning and, taking his bowl and robe, entered \textsanskrit{Sāvatthī} for alms. 

Then,\marginnote{1.7} after the meal, on his return from almsround, he addressed Venerable \textsanskrit{Rāhula}, “\textsanskrit{Rāhula}, get your sitting cloth.\footnote{The sitting cloth was a regular requisite of monastics (\href{https://suttacentral.net/pli-tv-bu-vb-pc89/en/sujato}{Bu Pc 89}). Regularly featured in the texts of the \textsanskrit{Sarvāstivāda} schools, it is rarely mentioned in the Pali suttas (but see \href{https://suttacentral.net/dn16/en/sujato\#3.1.3}{DN 16:3.1.3}). } Let’s go to the Dark Forest for the day’s meditation.” 

“Yes,\marginnote{1.10} sir,” replied \textsanskrit{Rāhula}. Taking his sitting cloth he followed behind the Buddha. 

Now\marginnote{2.1} at that time many thousands of deities followed the Buddha, thinking,\footnote{The presence of a thousand deities, which is unique to this discourse, is not mentioned in the parallel at all, and is obviously a later addition. } “Today the Buddha will lead \textsanskrit{Rāhula} further to the ending of defilements!” 

Then\marginnote{2.3} the Buddha plunged deep into the Dark Forest and sat at the root of a tree on the seat spread out. \textsanskrit{Rāhula} bowed to the Buddha and sat down to one side. The Buddha said to him: 

“What\marginnote{2.6} do you think, \textsanskrit{Rāhula}? Is the eye permanent or impermanent?” 

“Impermanent,\marginnote{2.8} sir.” 

“But\marginnote{2.9} if it’s impermanent, is it suffering or happiness?” 

“Suffering,\marginnote{2.10} sir.” 

“But\marginnote{2.11} if it’s impermanent, suffering, and perishable, is it fit to be regarded thus: ‘This is mine, I am this, this is my self’?” 

“No,\marginnote{2.13} sir.” 

“What\marginnote{3.1} do you think, \textsanskrit{Rāhula}? Are sights permanent or impermanent?” 

“Impermanent,\marginnote{3.3} sir.” 

“But\marginnote{3.4} if they're impermanent, are they suffering or happiness?” 

“Suffering,\marginnote{3.5} sir.” 

“But\marginnote{3.6} if they're impermanent, suffering, and perishable, are they fit to be regarded thus: ‘This is mine, I am this, this is my self’?” 

“No,\marginnote{3.8} sir.” 

“What\marginnote{3.9} do you think, \textsanskrit{Rāhula}? Is eye consciousness permanent or impermanent?” 

“Impermanent,\marginnote{3.11} sir.” 

“But\marginnote{3.12} if it’s impermanent, is it suffering or happiness?” 

“Suffering,\marginnote{3.13} sir.” 

“But\marginnote{3.14} if it’s impermanent, suffering, and perishable, is it fit to be regarded thus: ‘This is mine, I am this, this is my self’?” 

“No,\marginnote{3.16} sir.” 

“What\marginnote{3.17} do you think, \textsanskrit{Rāhula}? Is eye contact permanent or impermanent?” 

“Impermanent,\marginnote{3.19} sir.” 

“But\marginnote{3.20} if it’s impermanent, is it suffering or happiness?” 

“Suffering,\marginnote{3.21} sir.” 

“But\marginnote{3.22} if it’s impermanent, suffering, and perishable, is it fit to be regarded thus: ‘This is mine, I am this, this is my self’?” 

“No,\marginnote{3.24} sir.” 

“What\marginnote{3.25} do you think, \textsanskrit{Rāhula}? Anything included in feeling, perception, choices, and consciousness that arises conditioned by eye contact: is that permanent or impermanent?” 

“Impermanent,\marginnote{3.27} sir.” 

“But\marginnote{3.28} if it’s impermanent, is it suffering or happiness?” 

“Suffering,\marginnote{3.29} sir.” 

“But\marginnote{3.30} if it’s impermanent, suffering, and perishable, is it fit to be regarded thus: ‘This is mine, I am this, this is my self’?” 

“No,\marginnote{3.32} sir.” 

“What\marginnote{4{-}8.1} do you think, \textsanskrit{Rāhula}? Is the ear … nose … tongue … body … mind permanent or impermanent?” 

“Impermanent,\marginnote{4{-}8.10} sir.” 

“But\marginnote{4{-}8.11} if it’s impermanent, is it suffering or happiness?” 

“Suffering,\marginnote{4{-}8.12} sir.” 

“But\marginnote{4{-}8.13} if it’s impermanent, suffering, and perishable, is it fit to be regarded thus: ‘This is mine, I am this, this is my self’?” 

“No,\marginnote{4{-}8.15} sir.” 

“What\marginnote{4{-}8.16} do you think, \textsanskrit{Rāhula}? Are ideas permanent or impermanent?” 

“Impermanent,\marginnote{4{-}8.17} sir.” 

“But\marginnote{4{-}8.18} if they're impermanent, are they suffering or happiness?” 

“Suffering,\marginnote{4{-}8.19} sir.” 

“But\marginnote{4{-}8.20} if they're impermanent, suffering, and perishable, are they fit to be regarded thus: ‘This is mine, I am this, this is my self’?” 

“No,\marginnote{4{-}8.22} sir.” 

“What\marginnote{4{-}8.23} do you think, \textsanskrit{Rāhula}? Is mind consciousness permanent or impermanent?” 

“Impermanent,\marginnote{4{-}8.24} sir.” 

“But\marginnote{4{-}8.25} if it’s impermanent, is it suffering or happiness?” 

“Suffering,\marginnote{4{-}8.26} sir.” 

“But\marginnote{4{-}8.27} if it’s impermanent, suffering, and perishable, is it fit to be regarded thus: ‘This is mine, I am this, this is my self’?” 

“No,\marginnote{4{-}8.29} sir.” 

“What\marginnote{4{-}8.30} do you think, \textsanskrit{Rāhula}? Is mind contact permanent or impermanent?” 

“Impermanent,\marginnote{4{-}8.31} sir.” 

“But\marginnote{4{-}8.32} if it’s impermanent, is it suffering or happiness?” 

“Suffering,\marginnote{4{-}8.33} sir.” 

“But\marginnote{4{-}8.34} if it’s impermanent, suffering, and perishable, is it fit to be regarded thus: ‘This is mine, I am this, this is my self’?” 

“No,\marginnote{4{-}8.36} sir.” 

“What\marginnote{4{-}8.37} do you think, \textsanskrit{Rāhula}? Anything included in feeling, perception, choices, and consciousness that arises conditioned by mind contact: is that permanent or impermanent?” 

“Impermanent,\marginnote{4{-}8.39} sir.” 

“But\marginnote{4{-}8.40} if it’s impermanent, is it suffering or happiness?” 

“Suffering,\marginnote{4{-}8.41} sir.” 

“But\marginnote{4{-}8.42} if it’s impermanent, suffering, and perishable, is it fit to be regarded thus: ‘This is mine, I am this, this is my self’?” 

“No,\marginnote{4{-}8.44} sir.” 

“Seeing\marginnote{9.1} this, a learned noble disciple grows disillusioned with the eye, sights, eye consciousness, and eye contact. And they grow disillusioned with anything included in feeling, perception, choices, and consciousness that arises conditioned by eye contact. They grow disillusioned with the ear … nose … tongue … body … mind, ideas, mind consciousness, and mind contact. And they grow disillusioned with anything included in feeling, perception, choices, and consciousness that arises conditioned by mind contact. Being disillusioned, desire fades away. When desire fades away they’re freed. When they’re freed, they know they’re freed. 

They\marginnote{9.8} understand: ‘Rebirth is ended, the spiritual journey has been completed, what had to be done has been done, there is nothing further for this place.’” 

That\marginnote{9.9} is what the Buddha said. Satisfied, Venerable \textsanskrit{Rāhula} approved what the Buddha said. And while this discourse was being spoken, \textsanskrit{Rāhula}’s mind was freed from defilements by not grasping. 

And\marginnote{9.12} the stainless, immaculate vision of the Dhamma arose in those thousands of deities: “Everything that has a beginning has an end.” 

%
\section*{{\suttatitleacronym MN 148}{\suttatitletranslation Six By Six }{\suttatitleroot Chachakkasutta}}
\addcontentsline{toc}{section}{\tocacronym{MN 148} \toctranslation{Six By Six } \tocroot{Chachakkasutta}}
\markboth{Six By Six }{Chachakkasutta}
\extramarks{MN 148}{MN 148}

\scevam{So\marginnote{1.1} I have heard. }At one time the Buddha was staying near \textsanskrit{Sāvatthī} in Jeta’s Grove, \textsanskrit{Anāthapiṇḍika}’s monastery. There the Buddha addressed the mendicants, “Mendicants!” 

“Venerable\marginnote{1.5} sir,” they replied. The Buddha said this: 

“Mendicants,\marginnote{2.1} I shall teach you the Dhamma that’s good in the beginning, good in the middle, and good in the end, meaningful and well-phrased. And I shall reveal a spiritual practice that’s entirely full and pure, namely,\footnote{While this description of the Dhamma is frequently found, it is not elsewhere used as an introduction to a specific teaching. } the six sets of six. Listen and apply your mind well, I will speak.” 

“Yes,\marginnote{2.4} sir,” they replied. The Buddha said this: 

“The\marginnote{3.1} six interior sense fields should be understood. The six exterior sense fields should be understood. The six classes of consciousness should be understood. The six classes of contact should be understood. The six classes of feeling should be understood. The six classes of craving should be understood. 

‘The\marginnote{4.1} six interior sense fields should be understood.’ That’s what I said, but why did I say it? There are the sense fields of the eye, ear, nose, tongue, body, and mind. ‘The six interior sense fields should be understood.’ That’s what I said, and this is why I said it. This is the first set of six. 

‘The\marginnote{5.1} six exterior sense fields should be understood.’ That’s what I said, but why did I say it? There are the sense fields of sights, sounds, smells, tastes, touches, and ideas. ‘The six exterior sense fields should be understood.’ That’s what I said, and this is why I said it. This is the second set of six. 

‘The\marginnote{6.1} six classes of consciousness should be understood.’ That’s what I said, but why did I say it? Eye consciousness arises dependent on the eye and sights. Ear consciousness arises dependent on the ear and sounds. Nose consciousness arises dependent on the nose and smells. Tongue consciousness arises dependent on the tongue and tastes. Body consciousness arises dependent on the body and touches. Mind consciousness arises dependent on the mind and ideas. ‘The six classes of consciousness should be understood.’ That’s what I said, and this is why I said it. This is the third set of six. 

‘The\marginnote{7.1} six classes of contact should be understood.’ That’s what I said, but why did I say it? Eye consciousness arises dependent on the eye and sights. The meeting of the three is contact. Ear consciousness arises dependent on the ear and sounds. The meeting of the three is contact. Nose consciousness arises dependent on the nose and smells. The meeting of the three is contact. Tongue consciousness arises dependent on the tongue and tastes. The meeting of the three is contact. Body consciousness arises dependent on the body and touches. The meeting of the three is contact. Mind consciousness arises dependent on the mind and ideas. The meeting of the three is contact. ‘The six classes of contact should be understood.’ That’s what I said, and this is why I said it. This is the fourth set of six. 

‘The\marginnote{8.1} six classes of feeling should be understood.’ That’s what I said, but why did I say it? Eye consciousness arises dependent on the eye and sights. The meeting of the three is contact. Contact is a condition for feeling. Ear consciousness arises dependent on the ear and sounds. The meeting of the three is contact. Contact is a condition for feeling. Nose consciousness arises dependent on the nose and smells. The meeting of the three is contact. Contact is a condition for feeling. Tongue consciousness arises dependent on the tongue and tastes. The meeting of the three is contact. Contact is a condition for feeling. Body consciousness arises dependent on the body and touches. The meeting of the three is contact. Contact is a condition for feeling. Mind consciousness arises dependent on the mind and ideas. The meeting of the three is contact. Contact is a condition for feeling. ‘The six classes of feeling should be understood.’ That’s what I said, and this is why I said it. This is the fifth set of six. 

‘The\marginnote{9.1} six classes of craving should be understood.’ That’s what I said, but why did I say it? Eye consciousness arises dependent on the eye and sights. The meeting of the three is contact. Contact is a condition for feeling. Feeling is a condition for craving.\footnote{We have now included the six sense fields, contact, feeling, and craving, which are four of the items of dependent origination. } Ear consciousness … Nose consciousness … Tongue consciousness … Body consciousness … Mind consciousness arises dependent on the mind and ideas. The meeting of the three is contact. Contact is a condition for feeling. Feeling is a condition for craving. ‘The six classes of craving should be understood.’ That’s what I said, and this is why I said it. This is the sixth set of six. 

If\marginnote{10.1} anyone says, ‘the eye is self,’ that is not tenable.\footnote{This is an unusual use of \textit{na upapajjati} in the sense, “is not tenable”, “does not obtain”, or “is not justified”. } The arising and vanishing of the eye is evident, so it would follow that one’s self arises and vanishes. That’s why it’s not tenable to claim that the eye is self. So the eye is not self. 

If\marginnote{10.7} anyone says, ‘sights are self,’ that is not tenable. The arising and vanishing of sights is evident, so it would follow that one’s self arises and vanishes. That’s why it’s not tenable to claim that sights are self. So the eye is not self and sights are not self. 

If\marginnote{10.13} anyone says, ‘eye consciousness is self,’ that is not tenable. The arising and vanishing of eye consciousness is evident, so it would follow that one’s self arises and vanishes. That’s why it’s not tenable to claim that eye consciousness is self. So the eye, sights, and eye consciousness are not self. 

If\marginnote{10.19} anyone says, ‘eye contact is self,’ that is not tenable. The arising and vanishing of eye contact is evident, so it would follow that one’s self arises and vanishes. That’s why it’s not tenable to claim that eye contact is self. So the eye, sights, eye consciousness, and eye contact are not self. 

If\marginnote{10.25} anyone says, ‘feeling is self,’ that is not tenable. The arising and vanishing of feeling is evident, so it would follow that one’s self arises and vanishes. That’s why it’s not tenable to claim that feeling is self. So the eye, sights, eye consciousness, eye contact, and feeling are not self. 

If\marginnote{10.31} anyone says, ‘craving is self,’ that is not tenable. The arising and vanishing of craving is evident, so it would follow that one’s self arises and vanishes. That’s why it’s not tenable to claim that craving is self. So the eye, sights, eye consciousness, eye contact, feeling, and craving are not self. 

If\marginnote{11.1} anyone says, ‘the ear is self’ … ‘the nose is self’ … ‘the tongue is self’ … ‘the body is self’ … ‘the mind is self,’ that is not tenable. The arising and vanishing of the mind is evident, so it would follow that one’s self arises and vanishes. That’s why it’s not tenable to claim that the mind is self. So the mind is not self. 

If\marginnote{12.1} anyone says, ‘ideas are self’ … ‘mind consciousness is self’ … ‘mind contact is self’ … ‘feeling is self’ … ‘craving is self,’ that is not tenable. The arising and vanishing of craving is evident, so it would follow that one’s self arises and vanishes. That’s why it’s not tenable to claim that craving is self. So the mind, ideas, mind consciousness, mind contact, feeling, and craving are not self. 

Now,\marginnote{16.1} mendicants, this is the way that leads to the origin of substantial reality. You regard the eye like this: ‘This is mine, I am this, this is my self.’ 

You\marginnote{17{-}21.1} regard sights … eye consciousness … eye contact … feeling … craving like this: ‘This is mine, I am this, this is my self.’ You regard the ear … nose … tongue … body … mind … ideas … mind consciousness … mind contact … feeling … craving like this: ‘This is mine, I am this, this is my self.’ 

But\marginnote{22.1} this is the way that leads to the cessation of substantial reality. You regard the eye like this: ‘This is not mine, I am not this, this is not my self.’ 

You\marginnote{23{-}27.1} regard sights … eye consciousness … eye contact … feeling … craving like this: ‘This is not mine, I am not this, this is not my self.’ You regard the ear … nose … tongue … body … mind like this: ‘This is not mine, I am not this, this is not my self.’ You regard ideas … mind consciousness … mind contact … feeling … craving like this: ‘This is not mine, I am not this, this is not my self.’ 

Eye\marginnote{28.1} consciousness arises dependent on the eye and sights. The meeting of the three is contact. Contact is a condition for the arising of what is felt as pleasant, painful, or neutral. When you experience a pleasant feeling, if you approve, welcome, and keep clinging to it,\footnote{Compare the analysis at \href{https://suttacentral.net/sn36.6/en/sujato}{SN 36.6}. } the underlying tendency to greed underlies that. When you experience a painful feeling, if you sorrow and wail and lament, beating your breast and falling into confusion, the underlying tendency to repulsion underlies that. When you experience a neutral feeling, if you don’t truly understand that feeling’s origin, ending, gratification, drawback, and escape, the underlying tendency to ignorance underlies that. Mendicants, without giving up the underlying tendency to greed for pleasant feeling, without dispelling the underlying tendency to repulsion towards painful feeling, without eradicating ignorance in the case of neutral feeling, without giving up ignorance and without giving rise to knowledge, it is quite impossible to make an end of suffering in this very life. 

Ear\marginnote{29{-}33.1} consciousness … Nose consciousness … Tongue consciousness … Body consciousness … Mind consciousness arises dependent on the mind and ideas. The meeting of the three is contact. Contact is a condition for the arising of what is felt as pleasant, painful, or neutral. When you experience a pleasant feeling, if you approve, welcome, and keep clinging to it, the underlying tendency to greed underlies that. When you experience a painful feeling, if you sorrow and wail and lament, beating your breast and falling into confusion, the underlying tendency to repulsion underlies that. When you experience a neutral feeling, if you don’t truly understand that feeling’s origin, ending, gratification, drawback, and escape, the underlying tendency to ignorance underlies that. Mendicants, without giving up the underlying tendency to greed for pleasant feeling, without dispelling the underlying tendency to repulsion towards painful feeling, without eradicating ignorance in the case of neutral feeling, without giving up ignorance and without giving rise to knowledge, it is quite impossible to make an end of suffering in this very life. 

Eye\marginnote{34.1} consciousness arises dependent on the eye and sights. The meeting of the three is contact. Contact is a condition for the arising of what is felt as pleasant, painful, or neutral. When you experience a pleasant feeling, if you don’t approve, welcome, and keep clinging to it, the underlying tendency to greed does not underlie that. When you experience a painful feeling, if you don’t sorrow or wail or lament, beating your breast and falling into confusion, the underlying tendency to repulsion does not underlie that. When you experience a neutral feeling, if you truly understand that feeling’s origin, ending, gratification, drawback, and escape, the underlying tendency to ignorance does not underlie that. Mendicants, after giving up the underlying tendency to greed for pleasant feeling, after dispelling the underlying tendency to repulsion towards painful feeling, after eradicating ignorance in the case of neutral feeling, after giving up ignorance and giving rise to knowledge, it’s quite possible to make an end of suffering in this very life. 

Ear\marginnote{35.1} consciousness … 

Nose\marginnote{36.1} consciousness … 

Tongue\marginnote{37.1} consciousness … 

Body\marginnote{38.1} consciousness … 

Mind\marginnote{39.1} consciousness arises dependent on the mind and ideas. The meeting of the three is contact. Contact is a condition for what is felt as pleasant, painful, or neutral. When you experience a pleasant feeling, if you don’t approve, welcome, and keep clinging to it, the underlying tendency to greed does not underlie that. When you experience a painful feeling, if you don’t sorrow or wail or lament, beating your breast and falling into confusion, the underlying tendency to repulsion does not underlie that. When you experience a neutral feeling, if you truly understand that feeling’s origin, ending, gratification, drawback, and escape, the underlying tendency to ignorance does not underlie that. Mendicants, after giving up the underlying tendency to greed for pleasant feeling, after dispelling the underlying tendency to repulsion towards painful feeling, after eradicating ignorance in the case of neutral feeling, after giving up ignorance and giving rise to knowledge, it’s quite possible to make an end of suffering in this very life. 

Seeing\marginnote{40.1} this, a learned noble disciple grows disillusioned with the eye, sights, eye consciousness, eye contact, feeling, and craving. 

They\marginnote{41.1} grow disillusioned with the ear … nose … tongue … body … mind, ideas, mind consciousness, mind contact, feeling, and craving. Being disillusioned, desire fades away. When desire fades away they’re freed. When it is freed, they know it is freed. 

They\marginnote{41.7} understand: ‘Rebirth is ended, the spiritual journey has been completed, what had to be done has been done, there is nothing further for this place.’” 

That\marginnote{41.8} is what the Buddha said. Satisfied, the mendicants approved what the Buddha said. And while this discourse was being spoken, the minds of sixty mendicants were freed from defilements by not grasping.\footnote{This is not found in parallel passages and is probably a later addition. } 

%
\section*{{\suttatitleacronym MN 149}{\suttatitletranslation The Great Discourse on What Relates to the Six Sense Fields }{\suttatitleroot Mahāsaḷāyatanikasutta}}
\addcontentsline{toc}{section}{\tocacronym{MN 149} \toctranslation{The Great Discourse on What Relates to the Six Sense Fields } \tocroot{Mahāsaḷāyatanikasutta}}
\markboth{The Great Discourse on What Relates to the Six Sense Fields }{Mahāsaḷāyatanikasutta}
\extramarks{MN 149}{MN 149}

\scevam{So\marginnote{1.1} I have heard.\footnote{This discourse is “great” because it integrates multiple distinct teaching frameworks within the six sense fields. These include the majority of the \textit{\textsanskrit{saṁyutta}-\textsanskrit{mātikā}}, the list of topics making up the \textsanskrit{Saṁyutta} \textsanskrit{Nikāya}. } }At one time the Buddha was staying near \textsanskrit{Sāvatthī} in Jeta’s Grove, \textsanskrit{Anāthapiṇḍika}’s monastery. There the Buddha addressed the mendicants, “Mendicants!” 

“Venerable\marginnote{1.5} sir,” they replied. The Buddha said this: 

“Mendicants,\marginnote{2.1} I shall teach you the great discourse on what relates to the six sense fields.\footnote{For \textit{\textsanskrit{āyatanika}} see \href{https://suttacentral.net/mn121/en/sujato\#10.5}{MN 121:10.5}, where it refers to the life faculty of the body that is “related to the six sense fields”; there are also heavens and hells “related to the six fields of contact” (\href{https://suttacentral.net/mn50/en/sujato\#22.2}{MN 50:22.2}, \href{https://suttacentral.net/sb35/en/sujato\#35.135}{Sb 35:35.135}). In each case it refers to something that is connected to or related to the six sense fields, as indicated by the suffix \textit{ika}. Here this refers to the fact that this discourse treats multiple distinct teaching frameworks in relation to the six sense fields. These include the majority of the \textit{\textsanskrit{saṁyutta}-\textsanskrit{mātikā}}, the list of topics making up the \textsanskrit{Saṁyutta} \textsanskrit{Nikāya}. And this inclusive and integrative nature in turn explains why the teaching is “great”. } Listen and apply your mind well, I will speak.” 

“Yes,\marginnote{2.3} sir,” they replied. The Buddha said this: 

“Mendicants,\marginnote{3.1} when you don’t truly know and see the eye, sights, eye consciousness, eye contact, and what is felt as pleasant, painful, or neutral that arises conditioned by eye contact, you’re aroused by these things. 

Someone\marginnote{3.2} who lives aroused like this—fettered, confused, concentrating on gratification—accumulates the five grasping aggregates for themselves in the future.\footnote{In other words, one who becomes attached to sense experience creates kamma that generates a new life consisting of the five aggregates. This passage integrates the teachings on the six sense fields and the five aggregates under the first noble truth, as well as briefly implying the basic principle of dependent origination. } And their craving—which leads to future lives, mixed up with relishing and greed, taking pleasure wherever it lands—grows.\footnote{This is the second noble truth. } Their physical and mental stress, torment, and fever grow. And they experience physical and mental suffering. 

When\marginnote{4{-}7.1} you don’t truly know and see the ear … nose … tongue … body … mind, ideas, mind consciousness, mind contact, and what is felt as pleasant, painful, or neutral that arises conditioned by mind contact, you’re aroused by these things. 

Someone\marginnote{8.1} who lives aroused like this—fettered, confused, concentrating on gratification—accumulates the five grasping aggregates for themselves in the future. And their craving—which leads to future lives, mixed up with relishing and greed, taking pleasure wherever it lands—grows. Their physical and mental stress, torment, and fever grow. And they experience physical and mental suffering. 

When\marginnote{9.1} you do truly know and see the eye, sights, eye consciousness, eye contact, and what is felt as pleasant, painful, or neutral that arises conditioned by eye contact, you’re not aroused by these things. 

Someone\marginnote{9.2} who lives unaroused like this—unfettered, unconfused, concentrating on drawbacks—disperses the five grasping aggregates for themselves in the future. And their craving—which leads to future lives, mixed up with relishing and greed, taking pleasure wherever it lands—is given up. Their physical and mental stress, torment, and fever are given up. And they experience physical and mental pleasure. 

The\marginnote{10.1} view of such a person is right view. Their intention is right intention, their effort is right effort, their mindfulness is right mindfulness, and their immersion is right immersion. And their actions of body and speech, and their livelihood, have already been fully purified before. So this noble eightfold path is fully developed. 

When\marginnote{10.8} the noble eightfold path is developed, the following are fully developed: the four kinds of mindfulness meditation, the four right efforts, the four bases of psychic power, the five faculties, the five powers, and the seven awakening factors.\footnote{The eightfold path is brought in as well, along with the other main teachings of the path. } 

And\marginnote{10.9} these two qualities proceed in conjunction: serenity and discernment.\footnote{Discussed in more detail at \href{https://suttacentral.net/an4.170/en/sujato}{AN 4.170}. } They completely understand by direct knowledge those things that should be completely understood by direct knowledge.\footnote{This refers to the distinct functions to be exercised in relation to each of the four noble truths (\href{https://suttacentral.net/sn56.11/en/sujato\#5.1}{SN 56.11:5.1}). } They give up by direct knowledge those things that should be given up by direct knowledge. They develop by direct knowledge those things that should be developed by direct knowledge. They realize by direct knowledge those things that should be realized by direct knowledge. 

And\marginnote{11.1} what are the things that should be completely understood by direct knowledge? You should say: ‘The five grasping aggregates.’ That is: form, feeling, perception, choices, and consciousness. These are the things that should be completely understood by direct knowledge. 

And\marginnote{11.5} what are the things that should be given up by direct knowledge? Ignorance and craving for continued existence. These are the things that should be given up by direct knowledge. 

And\marginnote{11.8} what are the things that should be developed by direct knowledge? Serenity and discernment. These are the things that should be developed by direct knowledge. 

And\marginnote{11.11} what are the things that should be realized by direct knowledge? Knowledge and freedom. These are the things that should be realized by direct knowledge. 

When\marginnote{12{-}18.1} you truly know and see the ear … nose … tongue … body … mind, ideas, mind consciousness, mind contact, and what is felt as pleasant, painful, or neutral that arises conditioned by mind contact, you are not aroused by these things. … 

These\marginnote{26.1} are the things that should be realized by direct knowledge.” 

That\marginnote{26.4} is what the Buddha said. Satisfied, the mendicants approved what the Buddha said. 

%
\section*{{\suttatitleacronym MN 150}{\suttatitletranslation With the People of Nagaravinda }{\suttatitleroot Nagaravindeyyasutta}}
\addcontentsline{toc}{section}{\tocacronym{MN 150} \toctranslation{With the People of Nagaravinda } \tocroot{Nagaravindeyyasutta}}
\markboth{With the People of Nagaravinda }{Nagaravindeyyasutta}
\extramarks{MN 150}{MN 150}

\scevam{So\marginnote{1.1} I have heard. }At one time the Buddha was wandering in the land of the Kosalans together with a large \textsanskrit{Saṅgha} of mendicants when he arrived at a village of the Kosalan brahmins named Nagaravinda.\footnote{We travel with the Buddha to several “villages of the Kosalan brahmins”, but this is the only recorded visit to Nagaravinda. The name seems to mean “mountain-lotus” (\textit{naga} = “mountain”, \textit{aravinda} = “lotus”). \textsanskrit{Sāvatthī} lies only fifty kilometres south of the Himalayan foothills. } 

The\marginnote{2.1} brahmins and householders of Nagaravinda heard, “It seems the ascetic Gotama—a Sakyan, gone forth from a Sakyan family—while wandering in the land of the Kosalans has arrived at Nagaravinda, together with a large \textsanskrit{Saṅgha} of mendicants. He has this good reputation: ‘That Blessed One is perfected, a fully awakened Buddha, accomplished in knowledge and conduct, holy, knower of the world, supreme guide for those who wish to train, teacher of gods and humans, awakened, blessed.’ He has realized with his own insight this world—with its gods, \textsanskrit{Māras}, and divinities, this population with its ascetics and brahmins, gods and humans—and he makes it known to others. He proclaims a teaching that is good in the beginning, good in the middle, and good in the end, meaningful and well-phrased. And he reveals a spiritual practice that’s entirely full and pure. It’s good to see such perfected ones.” 

Then\marginnote{3.1} the brahmins and householders of Nagaravinda went up to the Buddha. Before sitting down to one side, some bowed, some exchanged greetings and polite conversation, some held up their joined palms toward the Buddha, some announced their name and clan, while some kept silent. The Buddha said to them: 

“Householders,\marginnote{4.1} if wanderers who follow another religion were to ask you:\footnote{This is an unusual framing, as usually the Buddha refers to wanderers of another religion when speaking with mendicants. The parallel simply says “other people” (SA 280 at T ii 76c8). Again, the fact that the villagers have diverse responses to the Buddha shows that they are not all Buddhists, and hence it would be out of character for the Buddha to assume otherwise, as the teaching here implies. Further, the Buddha’s statement here reads as if it were an answer to a question on how to discern which of the many different ascetics were worthy of respect. It seems to me likely that the narrative has become a little confused. } ‘What kind of ascetic or brahmin doesn’t deserve honor, respect, reverence, and veneration?’ You should answer them: ‘There are ascetics and brahmins who are not free of greed, hate, and delusion for sights known by the eye, who are not peaceful inside, and who conduct themselves badly among the good by way of body, speech, and mind. They don’t deserve honor, respect, reverence, and veneration.\footnote{The idiom \textit{\textsanskrit{samavisamaṁ} caranti} means “walk roughly in the smooth”, in other words, “conduct themselves badly among the good”, in contrast with those who “walk smoothly in the rough” (\textit{caranti visame \textsanskrit{samaṁ}}, \href{https://suttacentral.net/sn1.7/en/sujato\#3.4}{SN 1.7:3.4}, \href{https://suttacentral.net/sn2.6/en/sujato\#8.6}{SN 2.6:8.6}). } Why is that? Because we ourselves are not free of these things, so we do not see that they have any higher good conduct than us. That’s why they don’t deserve honor, respect, reverence, and veneration. There are ascetics and brahmins who are not free of greed, hate, and delusion for sounds known by the ear … smells known by the nose … tastes known by the tongue… touches known by the body … ideas known by the mind, who are not peaceful inside, and who conduct themselves badly among the good by way of body, speech, and mind. They don’t deserve honor, respect, reverence, and veneration. Why is that? Because we ourselves are not free of these things, so we do not see that they have any higher good conduct than us. That’s why they don’t deserve honor, respect, reverence, and veneration.’\footnote{This is a much higher standard than that implied at \href{https://suttacentral.net/mn142/en/sujato\#8.1}{MN 142:8.1}. However, the context is different: there, it is whether offerings will be fruitful, here it is whether they are truly deserving of respect. } When questioned by wanderers of other religions, that’s how you should answer them. 

If\marginnote{5.1} wanderers of other religions were to ask you: ‘What kind of ascetic or brahmin deserves honor, respect, reverence, and veneration?’ You should answer them: ‘There are ascetics and brahmins who are free of greed, hate, and delusion for sights known by the eye, who are peaceful inside, and who conduct themselves well by way of body, speech, and mind. They deserve honor, respect, reverence, and veneration. Why is that? Because we ourselves are not free of these things, but we see that they have a higher good conduct than us. That’s why they deserve honor, respect, reverence, and veneration. There are ascetics and brahmins who are free of greed, hate, and delusion for sounds known by the ear … smells known by the nose … tastes known by the tongue … touches known by the body … ideas known by the mind, who are peaceful inside, and who conduct themselves well by way of body, speech, and mind. They deserve honor, respect, reverence, and veneration. Why is that? Because we ourselves are not free of these things, but we see that they have a higher good conduct than us. That’s why they deserve honor, respect, reverence, and veneration. When questioned by wanderers of other religions, that’s how you should answer them. 

If\marginnote{6.1} wanderers of other religions were to ask you: ‘But what reasons and evidence do you have regarding those venerables that justifies saying, “Clearly those venerables are free of greed, hate, and delusion, or practicing to be free of them”?’ You should answer them: ‘It’s because those venerables frequent remote lodgings in the wilderness and the forest. In such places there are no sights known by the eye to see and enjoy, there are no sounds known by the ear to hear and enjoy, no odors known by the nose to smell and enjoy, no flavors known by the tongue to taste and enjoy, and no touches known by the body to feel and enjoy. These are the reasons and evidence that you have regarding those venerables that justifies saying, “Clearly those venerables are free of greed, hate, and delusion, or practicing to be free of them”.’ When questioned by wanderers of other religions, that’s how you should answer them.” 

When\marginnote{7.1} he had spoken, the brahmins and householders of Nagaravinda said to the Buddha, “Excellent, Mister Gotama! Excellent! As if he were righting the overturned, or revealing the hidden, or pointing out the path to the lost, or lighting a lamp in the dark so people with clear eyes can see what’s there, Mister Gotama has made the Teaching clear in many ways. We go for refuge to Mister Gotama, to the teaching, and to the mendicant \textsanskrit{Saṅgha}. From this day forth, may Mister Gotama remember us as lay followers who have gone for refuge for life.” 

%
\section*{{\suttatitleacronym MN 151}{\suttatitletranslation The Purification of Alms }{\suttatitleroot Piṇḍapātapārisuddhisutta}}
\addcontentsline{toc}{section}{\tocacronym{MN 151} \toctranslation{The Purification of Alms } \tocroot{Piṇḍapātapārisuddhisutta}}
\markboth{The Purification of Alms }{Piṇḍapātapārisuddhisutta}
\extramarks{MN 151}{MN 151}

\scevam{So\marginnote{1.1} I have heard. }At one time the Buddha was staying near \textsanskrit{Rājagaha}, in the Bamboo Grove, the squirrels’ feeding ground. 

Then\marginnote{1.3} in the late afternoon, \textsanskrit{Sāriputta} came out of retreat and went to the Buddha. He bowed and sat down to one side. The Buddha said to him, “\textsanskrit{Sāriputta}, your faculties are so very clear, and your complexion is pure and bright. What kind of meditation are you usually practicing these days?” 

“Sir,\marginnote{2.3} these days I usually practice the meditation on emptiness.”\footnote{See \href{https://suttacentral.net/mn121/en/sujato\#3.3}{MN 121:3.3} and note. This discourse offers a different perspective on this meditation. } 

“Good,\marginnote{2.4} good, \textsanskrit{Sāriputta}! It seems you usually practice the meditation of a great man.\footnote{While the concept of a “great man” is familiar, the “meditation of a great man” (\textit{\textsanskrit{mahāpurisavihāra}}) appears only here and at \href{https://suttacentral.net/pli-tv-kd22/en/sujato\#2.5.10}{Kd 22:2.5.10}, which seems to borrow from this passage. } For emptiness is the meditation of a great man. 

Now,\marginnote{3.1} a mendicant might wish: ‘May I usually practice the meditation on emptiness.’ So they should reflect: ‘Along the path that I went for alms, or in the place I wandered for alms, or along the path that I returned from alms, was there any desire or greed or hate or delusion or repulsion in my heart for sights known by the eye?’\footnote{Here the contemplation of the six sense fields is taught in the context of sense restraint through the active development of wisdom or discernment. } Suppose that, upon checking, a mendicant knows that there was such desire or greed or hate or delusion or repulsion in their heart, they should make an effort to give up those unskillful qualities. But suppose that, upon checking, a mendicant knows that there was no such desire or greed or hate or delusion or repulsion in their heart, they should meditate with rapture and joy, training day and night in skillful qualities. 

Furthermore,\marginnote{4{-}8.1} a mendicant should reflect: ‘Along the path that I went for alms, or in the place I wandered for alms, or along the path that I returned from alms, was there any desire or greed or hate or delusion or repulsion in my heart for sounds known by the ear … smells known by the nose … tastes known by the tongue … touches known by the body … ideas known by the mind?’ Suppose that, upon checking, a mendicant knows that there was such desire or greed or hate or delusion or repulsion in their heart, they should make an effort to give up those unskillful qualities. But suppose that, upon checking, a mendicant knows that there was no such desire or greed or hate or delusion or repulsion in their heart, they should meditate with rapture and joy, training day and night in skillful qualities. 

Furthermore,\marginnote{9.1} a mendicant should reflect:\footnote{The following sections may be a later addition, as they are absent from the parallel (SA 236 at T ii 57b24). } ‘Have I given up the five kinds of sensual stimulation?’ Suppose that, upon checking, a mendicant knows that they have not given them up, they should make an effort to do so. But suppose that, upon checking, a mendicant knows that they have given them up, they should meditate with rapture and joy, training day and night in skillful qualities. 

Furthermore,\marginnote{10.1} a mendicant should reflect: ‘Have I given up the five hindrances?’ Suppose that, upon checking, a mendicant knows that they have not given them up, they should make an effort to do so. But suppose that, upon checking, a mendicant knows that they have given them up, they should meditate with rapture and joy, training day and night in skillful qualities. 

Furthermore,\marginnote{11.1} a mendicant should reflect: ‘Have I completely understood the five grasping aggregates?’ Suppose that, upon checking, a mendicant knows that they have not completely understood them, they should make an effort to do so. But suppose that, upon checking, a mendicant knows that they have completely understood them, they should meditate with rapture and joy, training day and night in skillful qualities. 

Furthermore,\marginnote{12.1} a mendicant should reflect: ‘Have I developed the four kinds of mindfulness meditation?’ Suppose that, upon checking, a mendicant knows that they haven’t developed them, they should make an effort to do so. But suppose that, upon checking, a mendicant knows that they have developed them, they should meditate with rapture and joy, training day and night in skillful qualities. 

Furthermore,\marginnote{13.1} a mendicant should reflect: ‘Have I developed the four right efforts … the four bases of psychic power … the five faculties … the five powers … the seven awakening factors … the noble eightfold path?’ Suppose that, upon checking, a mendicant knows that they haven’t developed it, they should make an effort to do so. But suppose that, upon checking, a mendicant knows that they have developed it, they should meditate with rapture and joy, training day and night in skillful qualities. 

Furthermore,\marginnote{19.1} a mendicant should reflect: ‘Have I developed serenity and discernment?’ Suppose that, upon checking, a mendicant knows that they haven’t developed them, they should make an effort to do so. But suppose that, upon checking, a mendicant knows that they have developed them, they should meditate with rapture and joy, training day and night in skillful qualities. 

Furthermore,\marginnote{20.1} a mendicant should reflect: ‘Have I realized knowledge and freedom?’ Suppose that, upon checking, a mendicant knows that they haven’t realized them, they should make an effort to do so. But suppose that, upon checking, a mendicant knows that they have realized them, they should meditate with rapture and joy, training day and night in skillful qualities.\footnote{Even the arahant continues to meditate and practice since it is a happy way to live, and to set a good example. } 

Whether\marginnote{21.1} in the past, future, or present, all those who purify their almsfood do so by continually checking in this way. So, \textsanskrit{Sāriputta}, you should all train like this: ‘We shall purify our almsfood by continually checking.’”\footnote{This clarifies the overall purpose of the discourse, to set out a method for mendicants to reflect during their daily walk for almsfood, at which time they are likely to encounter distracting sights, sounds, and smells. } 

That\marginnote{21.6} is what the Buddha said. Satisfied, Venerable \textsanskrit{Sāriputta} approved what the Buddha said. 

%
\section*{{\suttatitleacronym MN 152}{\suttatitletranslation The Development of the Faculties }{\suttatitleroot Indriyabhāvanāsutta}}
\addcontentsline{toc}{section}{\tocacronym{MN 152} \toctranslation{The Development of the Faculties } \tocroot{Indriyabhāvanāsutta}}
\markboth{The Development of the Faculties }{Indriyabhāvanāsutta}
\extramarks{MN 152}{MN 152}

\scevam{So\marginnote{1.1} I have heard. }At one time the Buddha was staying near \textsanskrit{Kajaṅgalā} in a bamboo grove.\footnote{The names here are spelled \textit{\textsanskrit{gajaṅgalā}} and \textit{\textsanskrit{suveḷuvana}}, but I keep consistency with \href{https://suttacentral.net/an10.28/en/sujato}{AN 10.28}, where a nun residing at this town gives a penetrative discourse to the townsfolk. In the Vinaya it is mentioned as the easternmost point of the “middle country”, just beyond which lay \textsanskrit{Mahāsāla} (\href{https://suttacentral.net/pli-tv-kd5/en/sujato\#13.12.2}{Kd 5:13.12.2}). The \textsanskrit{Milindapañha} says it was a brahmin village on the slopes of the Himalayas where its main protagonist, the monk \textsanskrit{Nāgasena}, was born (\href{https://suttacentral.net/mil2/en/sujato\#22.3}{Mil 2:22.3}). Assuming both these are the same place, we might look somewhere near Darjeeling for this town. In any case, it represents the furthest east that the Buddha taught. It is sometimes identified with \langlzh{羯蠅揭羅} mentioned by Xuang Zang, but that has been located near Agra south of \textsanskrit{Sāvatthī}. } 

Then\marginnote{2.1} the student Uttara, a pupil of the brahmin \textsanskrit{Pārāsariya}, approached the Buddha, and exchanged greetings with him.\footnote{A brahmin teacher named \textsanskrit{Pārāsariya} (also spelled \textsanskrit{Pārāpariya} or \textsanskrit{Pārāsiviya}) features in \href{https://suttacentral.net/ja222/en/sujato}{Ja 222} and \href{https://suttacentral.net/ja353/en/sujato}{Ja 353} as one who warns that doing evil will lead to regret. The three similarly-named monks in the \textsanskrit{Theragāthā} (\href{https://suttacentral.net/thag1.116/en/sujato}{Thag 1.116}, \href{https://suttacentral.net/thag16.2/en/sujato}{Thag 16.2}, \href{https://suttacentral.net/thag16.10/en/sujato}{Thag 16.10}) share the same clan name (Sanskrit \textsanskrit{Pārāśarya}). These are descendants of the great sage \textsanskrit{Parāśara} \textsanskrit{Śāktya}, who is mentioned in Rig Veda 7.1.18 alongside \textsanskrit{Śatayātu}, \textsanskrit{Vasiṣṭha}; his prayers to Agni are recorded at Rig Veda 1.65–73, and to Soma at 9.97.31–44. } When the greetings and polite conversation were over, he sat down to one side. The Buddha said to him, “Uttara, does \textsanskrit{Pārāsariya} teach his disciples the development of the faculties?” 

“He\marginnote{2.4} does, Mister Gotama.” 

“But\marginnote{2.5} how does he teach it?” 

“Mister\marginnote{2.6} Gotama, it’s when the eye sees no sight and the ear hears no sound.\footnote{This is perhaps a distorted reference to certain \textsanskrit{Upaniṣadic} teachings that speak of the undifferentiated state of divinity that contains the potential for the faculties of seeing, etc. See \href{https://suttacentral.net/dn29/en/sujato\#16.13}{DN 29:16.13} and note there. Perhaps, too, it refers to the notion that in deep sleep the senses are withdrawn into the “space of the heart” (\textsanskrit{Bṛhadāraṇyaka} \textsanskrit{Upaniṣad} 2.1.17). } That’s how \textsanskrit{Pārāsariya} teaches his disciples the development of the faculties.” 

“In\marginnote{2.8} that case, Uttara, a blind person and a deaf person will have developed faculties according to what \textsanskrit{Pārāsariya} says. For a blind person sees no sight with the eye and a deaf person hears no sound with the ear.”\footnote{This teaching is not about deep meditation, where the senses can be in a state of abeyance, but about how we relate to the experience of the senses. } When he said this, Uttara sat silent, dismayed, shoulders drooping, downcast, depressed, with nothing to say. 

Knowing\marginnote{3.1} this, the Buddha addressed Venerable Ānanda, “Ānanda, the development of the faculties taught by \textsanskrit{Pārāsariya} is quite different from the supreme development of the faculties in the training of the Noble One.” 

“Now\marginnote{3.3} is the time, Blessed One! Now is the time, Holy One. Let the Buddha teach the supreme development of the faculties in the training of the Noble One. The mendicants will listen and remember it.” 

“Well\marginnote{3.5} then, Ānanda, listen and apply your mind well, I will speak.” 

“Yes,\marginnote{3.6} sir,” Ānanda replied. The Buddha said this: 

“And\marginnote{4.1} how, Ānanda, is there the supreme development of the faculties in the training of the Noble One? When a mendicant sees a sight with their eyes, in them arises what is agreeable, what is disagreeable, and what is both agreeable and disagreeable. They understand: ‘What is agreeable, what is disagreeable, and what is both agreeable and disagreeable have arisen in me. That’s conditioned, crude, and dependently originated. But this is peaceful and sublime, namely equanimity.’ Then the agreeable, the disagreeable, and the both agreeable and disagreeable that arose in them cease, and equanimity becomes stabilized. It’s like how a person with clear eyes might open their eyes then shut them; or might shut their eyes then open them.\footnote{The Buddha gives a series of similes, each one appropriate to the specific sense. } Such is the speed, the swiftness, the ease with which anything agreeable, disagreeable, and both agreeable and disagreeable that arose in them cease, and equanimity becomes stabilized. In the training of the Noble One this is called the supreme development of the faculties regarding sights known by the eye.\footnote{Bearing in mind that the word for “faculty”, \textit{indriya}, has the sense of “potential”, this phrase emphasizes that the “restraint” of the senses is also the way to “develop” them to their fullest. } 

Furthermore,\marginnote{5.1} when a mendicant hears a sound with their ears, the agreeable, the disagreeable, and the both agreeable and disagreeable arise in them. They understand: ‘What is agreeable, what is disagreeable, and what is both agreeable and disagreeable have arisen in me. That’s conditioned, crude, and dependently originated. But this is peaceful and sublime, namely equanimity.’ Then the agreeable, the disagreeable, and the both agreeable and disagreeable that arose in them cease, and equanimity becomes stabilized. It’s like how a strong person can effortlessly snap their fingers. Such is the speed, the swiftness, the ease with which anything agreeable, disagreeable, and both agreeable and disagreeable that arose in them cease, and equanimity becomes stabilized. In the training of the Noble One this is called the supreme development of the faculties regarding sounds known by the ear. 

Furthermore,\marginnote{6.1} when a mendicant smells an odor with their nose, the agreeable, the disagreeable, and the both agreeable and disagreeable arise in them. They understand: ‘What is agreeable, what is disagreeable, and what is both agreeable and disagreeable have arisen in me. That’s conditioned, crude, and dependently originated. But this is peaceful and sublime, namely equanimity.’ Then the agreeable, the disagreeable, and the both agreeable and disagreeable that arose in them cease, and equanimity becomes stabilized. It’s like how a drop of water would roll off a gently sloping lotus leaf, and would not stay there. Such is the speed, the swiftness, the ease with which anything agreeable, disagreeable, and both agreeable and disagreeable that arose in them cease, and equanimity becomes stabilized. In the training of the Noble One this is called the supreme development of the faculties regarding smells known by the nose. 

Furthermore,\marginnote{7.1} when a mendicant tastes a flavor with their tongue, the agreeable, the disagreeable, and the both agreeable and disagreeable arise in them. They understand: ‘What is agreeable, what is disagreeable, and what is both agreeable and disagreeable have arisen in me. That’s conditioned, crude, and dependently originated. But this is peaceful and sublime, namely equanimity.’ Then the agreeable, the disagreeable, and the both agreeable and disagreeable that arose in them cease, and equanimity becomes stabilized. It’s like how a strong person who’s formed a glob of spit on the tip of their tongue could easily spit it out. Such is the speed, the swiftness, the ease with which anything agreeable, disagreeable, and both agreeable and disagreeable that arose in them cease, and equanimity becomes stabilized. In the training of the Noble One this is called the supreme development of the faculties regarding tastes known by the tongue. 

Furthermore,\marginnote{8.1} when a mendicant feels a touch with their body, the agreeable, the disagreeable, and the both agreeable and disagreeable arise in them. They understand: ‘What is agreeable, what is disagreeable, and what is both agreeable and disagreeable have arisen in me. That’s conditioned, crude, and dependently originated. But this is peaceful and sublime, namely equanimity.’ Then the agreeable, the disagreeable, and the both agreeable and disagreeable that arose in them cease, and equanimity becomes stabilized. It’s like how a strong person can extend or contract their arm. Such is the speed, the swiftness, the ease with which anything agreeable, disagreeable, and both agreeable and disagreeable that arose in them cease, and equanimity becomes stabilized. In the training of the Noble One this is called the supreme development of the faculties regarding touches known by the body. 

Furthermore,\marginnote{9.1} when a mendicant knows a thought with their mind, the agreeable, the disagreeable, and the both agreeable and disagreeable arise in them. They understand: ‘What is agreeable, what is disagreeable, and what is both agreeable and disagreeable have arisen in me. That’s conditioned, crude, and dependently originated. But this is peaceful and sublime, namely equanimity.’ Then the agreeable, the disagreeable, and the both agreeable and disagreeable that arose in them cease, and equanimity becomes stabilized. It’s like how a strong person could let two or three drops of water fall onto an iron cauldron that had been heated all day. The drops would be slow to fall, but they’d quickly dry up and evaporate. Such is the speed, the swiftness, the ease with which anything agreeable, disagreeable, and both agreeable and disagreeable that arose in them cease, and equanimity becomes stabilized. In the training of the Noble One this is called the supreme development of the faculties regarding ideas known by the mind. That’s how there is the supreme development of the faculties in the training of the Noble One. 

And\marginnote{10.1} how are they a practicing trainee?\footnote{This is at least a stream-enterer; they still react to sense experiences that are agreeable or disagreeable, but they understand that they lead to suffering and train themselves to be freed.. } When a mendicant sees a sight with their eyes, the agreeable, the disagreeable, and the both agreeable and disagreeable arise in them. They are horrified, repelled, and disgusted by that. When they hear a sound with their ears … When they smell an odor with their nose … When they taste a flavor with their tongue … When they feel a touch with their body … When they know a thought with their mind, the agreeable, the disagreeable, and the both agreeable and disagreeable arise in them. They are horrified, repelled, and disgusted by that. That’s how they are a practicing trainee. 

And\marginnote{11{-}15.1} how are they a noble one with developed faculties?\footnote{This is the arahant. They still experience sense stimulation as pleasant, etc., but without reacting. } When a mendicant sees a sight with their eyes, the agreeable, the disagreeable, and the both agreeable and disagreeable arise in them. If they wish: ‘May I meditate perceiving the unrepulsive in the repulsive,’ that’s what they do.\footnote{Here this ability is fully mastered. It is taught to those in training as a practice at \href{https://suttacentral.net/an5.144/en/sujato}{AN 5.144}. } If they wish: ‘May I meditate perceiving the repulsive in the unrepulsive,’ that’s what they do. If they wish: ‘May I meditate perceiving the unrepulsive in the repulsive and the unrepulsive,’ that’s what they do. If they wish: ‘May I meditate perceiving the repulsive in the unrepulsive and the repulsive,’ that’s what they do. If they wish: ‘May I meditate staying equanimous, mindful and aware, rejecting both the repulsive and the unrepulsive,’ that’s what they do. 

When\marginnote{16.1} they hear a sound with their ear … When they smell an odor with their nose … When they taste a flavor with their tongue … When they feel a touch with their body … When they know a thought with their mind, the agreeable, the disagreeable, and the both agreeable and disagreeable arise in them. If they wish: ‘May I meditate perceiving the unrepulsive in the repulsive,’ that’s what they do. If they wish: ‘May I meditate perceiving the repulsive in the unrepulsive,’ that’s what they do. If they wish: ‘May I meditate perceiving the unrepulsive in the repulsive and the unrepulsive,’ that’s what they do. If they wish: ‘May I meditate perceiving the repulsive in the unrepulsive and the repulsive,’ that’s what they do. If they wish: ‘May I meditate staying equanimous, mindful and aware, rejecting both the repulsive and the unrepulsive,’ that’s what they do. That’s how they are a noble one with developed faculties. 

So,\marginnote{17.1} Ānanda, I have taught the supreme development of the faculties in the training of the Noble One, I have taught the practicing trainee, and I have taught the noble one with developed faculties. 

Out\marginnote{18.1} of sympathy, I’ve done what a teacher should do who wants what’s best for their disciples. Here are these roots of trees, and here are these empty huts. Practice absorption, Ānanda! Don’t be negligent! Don’t regret it later! This is my instruction to you.”\footnote{The Majjhima \textsanskrit{Nikāya} ends with the Buddha encouraging the archetypal student of the suttas, Ānanda, to develop meditation. } 

That\marginnote{18.3} is what the Buddha said. Satisfied, Venerable Ānanda approved what the Buddha said. 

\scendbook{The Middle Discourses are completed. }

%
\backmatter%
\chapter*{Colophon}
\addcontentsline{toc}{chapter}{Colophon}
\markboth{Colophon}{Colophon}

\section*{The Translator}

Bhikkhu Sujato was born as Anthony Aidan Best on 4/11/1966 in Perth, Western Australia. He grew up in the pleasant suburbs of Mt Lawley and Attadale alongside his sister Nicola, who was the good child. His mother, Margaret Lorraine Huntsman née Pinder, said “he’ll either be a priest or a poet”, while his father, Anthony Thomas Best, advised him to “never do anything for money”. He attended Aquinas College, a Catholic school, where he decided to become an atheist. At the University of WA he studied philosophy, aiming to learn what he wanted to do with his life. Finding that what he wanted to do was play guitar, he dropped out. His main band was named Martha’s Vineyard, which achieved modest success in the indie circuit. 

A seemingly random encounter with a roadside joey took him to Thailand, where he entered his first meditation retreat at Wat Ram Poeng, Chieng Mai in 1992. Feeling the call to the Buddha’s path, he took full ordination in Wat Pa Nanachat in 1994, where his teachers were Ajahn Pasanno and Ajahn Jayasaro. In 1997 he returned to Perth to study with Ajahn Brahm at Bodhinyana Monastery. 

He spent several years practicing in seclusion in Malaysia and Thailand before establishing Santi Forest Monastery in Bundanoon, NSW, in 2003. There he was instrumental in supporting the establishment of the Theravada bhikkhuni order in Australia and advocating for women’s rights. He continues to teach in Australia and globally, with a special concern for the moral implications of climate change and other forms of environmental destruction. He has published a series of books of original and groundbreaking research on early Buddhism. 

In 2005 he founded SuttaCentral together with Rod Bucknell and John Kelly. In 2015, seeing the need for a complete, accurate, plain English translation of the Pali texts, he undertook the task, spending nearly three years in isolation on the isle of Qi Mei off the coast of the nation of Taiwan. He completed the four main \textsanskrit{Nikāyas} in 2018, and the early books of the Khuddaka \textsanskrit{Nikāya} were complete by 2021. All this work is dedicated to the public domain and is entirely free of copyright encumbrance. 

In 2019 he returned to Sydney where he established Lokanta Vihara (The Monastery at the End of the World). 

\section*{Creation Process}

Primary source was the digital \textsanskrit{Mahāsaṅgīti} edition of the Pali \textsanskrit{Tipiṭaka}. Translated from the Pali, with reference to several English translations, especially those of Bhikkhu Bodhi.

\section*{The Translation}

This translation was part of a project to translate the four Pali \textsanskrit{Nikāyas} with the following aims: plain, approachable English; consistent terminology; accurate rendition of the Pali; free of copyright. It was made during 2016–2018 while Bhikkhu Sujato was staying in Qimei, Taiwan.

\section*{About SuttaCentral}

SuttaCentral publishes early Buddhist texts. Since 2005 we have provided root texts in Pali, Chinese, Sanskrit, Tibetan, and other languages, parallels between these texts, and translations in many modern languages. Building on the work of generations of scholars, we offer our contribution freely.

SuttaCentral is driven by volunteer contributions, and in addition we employ professional developers. We offer a sponsorship program for high quality translations from the original languages. Financial support for SuttaCentral is handled by the SuttaCentral Development Trust, a charitable trust registered in Australia.

\section*{About Bilara}

“Bilara” means “cat” in Pali, and it is the name of our Computer Assisted Translation (CAT) software. Bilara is a web app that enables translators to translate early Buddhist texts into their own language. These translations are published on SuttaCentral with the root text and translation side by side.

\section*{About SuttaCentral Editions}

The SuttaCentral Editions project makes high quality books from selected Bilara translations. These are published in formats including HTML, EPUB, PDF, and print.

You are welcome to print any of our Editions.

%
\end{document}