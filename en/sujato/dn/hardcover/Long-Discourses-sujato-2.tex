\documentclass[12pt,openany]{book}%
\usepackage{lastpage}%
%
\usepackage{ragged2e}
\usepackage{verse}
\usepackage[a-3u]{pdfx}
\usepackage[inner=1in, outer=1in, top=.7in, bottom=1in, papersize={6in,9in}, headheight=13pt]{geometry}
\usepackage{polyglossia}
\usepackage[12pt]{moresize}
\usepackage{soul}%
\usepackage{microtype}
\usepackage{tocbasic}
\usepackage{realscripts}
\usepackage{epigraph}%
\usepackage{setspace}%
\usepackage{sectsty}
\usepackage{fontspec}
\usepackage{marginnote}
\usepackage[bottom]{footmisc}
\usepackage{enumitem}
\usepackage{fancyhdr}
\usepackage{emptypage}
\usepackage{extramarks}
\usepackage{graphicx}
\usepackage{relsize}
\usepackage{etoolbox}

% improve ragged right headings by suppressing hyphenation and orphans. spaceskip plus and minus adjust interword spacing; increase rightskip stretch to make it want to push a word on the first line(s) to the next line; reduce parfillskip stretch to make line length more equal . spacefillskip and xspacefillskip can be deleted to use defaults.
\protected\def\BalancedRagged{
\leftskip     0pt
\rightskip    0pt plus 10em
\spaceskip=1\fontdimen2\font plus .5\fontdimen3\font minus 1.5\fontdimen4\font
\xspaceskip=1\fontdimen2\font plus 1\fontdimen3\font minus 1\fontdimen4\font
\parfillskip  0pt plus 15em
\relax
}

\hypersetup{
colorlinks=true,
urlcolor=black,
linkcolor=black,
citecolor=black,
allcolors=black
}

% use a small amount of tracking on small caps
\SetTracking[ spacing = {25*,166, } ]{ encoding = *, shape = sc }{ 25 }

% add a blank page
\newcommand{\blankpage}{
\newpage
\thispagestyle{empty}
\mbox{}
\newpage
}

% define languages
\setdefaultlanguage[]{english}
\setotherlanguage[script=Latin]{sanskrit}

%\usepackage{pagegrid}
%\pagegridsetup{top-left, step=.25in}

% define fonts
% use if arno sanskrit is unavailable
%\setmainfont{Gentium Plus}
%\newfontfamily\Marginalfont[]{Gentium Plus}
%\newfontfamily\Allsmallcapsfont[RawFeature=+c2sc]{Gentium Plus}
%\newfontfamily\Noligaturefont[Renderer=Basic]{Gentium Plus}
%\newfontfamily\Noligaturecaptionfont[Renderer=Basic]{Gentium Plus}
%\newfontfamily\Fleuronfont[Ornament=1]{Gentium Plus}

% use if arno sanskrit is available. display is applied to \chapter and \part, subhead to \section and \subsection.
\setmainfont[
  FontFace={sb}{n}{Font = {Arno Pro Semibold}},
  FontFace={sb}{it}{Font = {Arno  Pro Semibold Italic}}
]{Arno Pro}

% create commands for using semibold
\DeclareRobustCommand{\sbseries}{\fontseries{sb}\selectfont}
\DeclareTextFontCommand{\textsb}{\sbseries}

\newfontfamily\Marginalfont[RawFeature=+subs]{Arno Pro Regular}
\newfontfamily\Allsmallcapsfont[RawFeature=+c2sc]{Arno Pro}
\newfontfamily\Noligaturefont[Renderer=Basic]{Arno Pro}
\newfontfamily\Noligaturecaptionfont[Renderer=Basic]{Arno Pro Caption}

% chinese fonts
\newfontfamily\cjk{Noto Serif TC}
\newcommand*{\langlzh}[1]{\cjk{#1}\normalfont}%

% logo
\newfontfamily\Logofont{sclogo.ttf}
\newcommand*{\sclogo}[1]{\large\Logofont{#1}}

% use subscript numerals for margin notes
\renewcommand*{\marginfont}{\Marginalfont}

% ensure margin notes have consistent vertical alignment
\renewcommand*{\marginnotevadjust}{-.17em}

% use compact lists
\setitemize{noitemsep,leftmargin=1em}
\setenumerate{noitemsep,leftmargin=1em}
\setdescription{noitemsep, style=unboxed, leftmargin=1em}

% style ToC
\DeclareTOCStyleEntries[
  raggedentrytext,
  linefill=\hfill,
  pagenumberwidth=.5in,
  pagenumberformat=\normalfont,
  entryformat=\normalfont
]{tocline}{chapter,section}


  \setlength\topsep{0pt}%
  \setlength\parskip{0pt}%

% define new \centerpars command for use in ToC. This ensures centering, proper wrapping, and no page break after
\def\startcenter{%
  \par
  \begingroup
  \leftskip=0pt plus 1fil
  \rightskip=\leftskip
  \parindent=0pt
  \parfillskip=0pt
}
\def\stopcenter{%
  \par
  \endgroup
}
\long\def\centerpars#1{\startcenter#1\stopcenter}

% redefine part, so that it adds a toc entry without page number
\let\oldcontentsline\contentsline
\newcommand{\nopagecontentsline}[3]{\oldcontentsline{#1}{#2}{}}

    \makeatletter
\renewcommand*\l@part[2]{%
  \ifnum \c@tocdepth >-2\relax
    \addpenalty{-\@highpenalty}%
    \addvspace{0em \@plus\p@}%
    \setlength\@tempdima{3em}%
    \begingroup
      \parindent \z@ \rightskip \@pnumwidth
      \parfillskip -\@pnumwidth
      {\leavevmode
       \setstretch{.85}\large\scshape\centerpars{#1}\vspace*{-1em}\llap{#2}}\par
       \nobreak
         \global\@nobreaktrue
         \everypar{\global\@nobreakfalse\everypar{}}%
    \endgroup
  \fi}
\makeatother

\makeatletter
\def\@pnumwidth{2em}
\makeatother

% define new sectioning command, which is only used in volumes where the pannasa is found in some parts but not others, especially in an and sn

\newcommand*{\pannasa}[1]{\clearpage\thispagestyle{empty}\begin{center}\vspace*{14em}\setstretch{.85}\huge\itshape\scshape\MakeLowercase{#1}\end{center}}

    \makeatletter
\newcommand*\l@pannasa[2]{%
  \ifnum \c@tocdepth >-2\relax
    \addpenalty{-\@highpenalty}%
    \addvspace{.5em \@plus\p@}%
    \setlength\@tempdima{3em}%
    \begingroup
      \parindent \z@ \rightskip \@pnumwidth
      \parfillskip -\@pnumwidth
      {\leavevmode
       \setstretch{.85}\large\itshape\scshape\lowercase{\centerpars{#1}}\vspace*{-1em}\llap{#2}}\par
       \nobreak
         \global\@nobreaktrue
         \everypar{\global\@nobreakfalse\everypar{}}%
    \endgroup
  \fi}
\makeatother

% don't put page number on first page of toc (relies on etoolbox)
\patchcmd{\chapter}{plain}{empty}{}{}

% global line height
\setstretch{1.05}

% allow linebreak after em-dash
\catcode`\—=13
\protected\def—{\unskip\textemdash\allowbreak}

% style headings with secsty. chapter and section are defined per-edition
\partfont{\setstretch{.85}\normalfont\centering\textsc}
\subsectionfont{\setstretch{.95}\normalfont\BalancedRagged}%
\subsubsectionfont{\setstretch{1}\normalfont\itshape\BalancedRagged}

% style elements of suttatitle
\newcommand*{\suttatitleacronym}[1]{\smaller[2]{#1}\vspace*{.3em}}
\newcommand*{\suttatitletranslation}[1]{\linebreak{#1}}
\newcommand*{\suttatitleroot}[1]{\linebreak\smaller[2]\itshape{#1}}

\DeclareTOCStyleEntries[
  indent=3.3em,
  dynindent,
  beforeskip=.2em plus -2pt minus -1pt,
]{tocline}{section}

\DeclareTOCStyleEntries[
  indent=0em,
  dynindent,
  beforeskip=.4em plus -2pt minus -1pt,
]{tocline}{chapter}

\newcommand*{\tocacronym}[1]{\hspace*{-3.3em}{#1}\quad}
\newcommand*{\toctranslation}[1]{#1}
\newcommand*{\tocroot}[1]{(\textit{#1})}
\newcommand*{\tocchapterline}[1]{\bfseries\itshape{#1}}


% redefine paragraph and subparagraph headings to not be inline
\makeatletter
% Change the style of paragraph headings %
\renewcommand\paragraph{\@startsection{paragraph}{4}{\z@}%
            {-2.5ex\@plus -1ex \@minus -.25ex}%
            {1.25ex \@plus .25ex}%
            {\noindent\normalfont\itshape\small}}

% Change the style of subparagraph headings %
\renewcommand\subparagraph{\@startsection{subparagraph}{5}{\z@}%
            {-2.5ex\@plus -1ex \@minus -.25ex}%
            {1.25ex \@plus .25ex}%
            {\noindent\normalfont\itshape\footnotesize}}
\makeatother

% use etoolbox to suppress page numbers on \part
\patchcmd{\part}{\thispagestyle{plain}}{\thispagestyle{empty}}
  {}{\errmessage{Cannot patch \string\part}}

% and to reduce margins on quotation
\patchcmd{\quotation}{\rightmargin}{\leftmargin 1.2em \rightmargin}{}{}
\AtBeginEnvironment{quotation}{\small}

% titlepage
\newcommand*{\titlepageTranslationTitle}[1]{{\begin{center}\begin{large}{#1}\end{large}\end{center}}}
\newcommand*{\titlepageCreatorName}[1]{{\begin{center}\begin{normalsize}{#1}\end{normalsize}\end{center}}}

% halftitlepage
\newcommand*{\halftitlepageTranslationTitle}[1]{\setstretch{2.5}{\begin{Huge}\uppercase{\so{#1}}\end{Huge}}}
\newcommand*{\halftitlepageTranslationSubtitle}[1]{\setstretch{1.2}{\begin{large}{#1}\end{large}}}
\newcommand*{\halftitlepageFleuron}[1]{{\begin{large}\Fleuronfont{{#1}}\end{large}}}
\newcommand*{\halftitlepageByline}[1]{{\begin{normalsize}\textit{{#1}}\end{normalsize}}}
\newcommand*{\halftitlepageCreatorName}[1]{{\begin{LARGE}{\textsc{#1}}\end{LARGE}}}
\newcommand*{\halftitlepageVolumeNumber}[1]{{\begin{normalsize}{\Allsmallcapsfont{\textsc{#1}}}\end{normalsize}}}
\newcommand*{\halftitlepageVolumeAcronym}[1]{{\begin{normalsize}{#1}\end{normalsize}}}
\newcommand*{\halftitlepageVolumeTranslationTitle}[1]{{\begin{Large}{\textsc{#1}}\end{Large}}}
\newcommand*{\halftitlepageVolumeRootTitle}[1]{{\begin{normalsize}{\Allsmallcapsfont{\textsc{\itshape #1}}}\end{normalsize}}}
\newcommand*{\halftitlepagePublisher}[1]{{\begin{large}{\Noligaturecaptionfont\textsc{#1}}\end{large}}}

% epigraph
\renewcommand{\epigraphflush}{center}
\renewcommand*{\epigraphwidth}{.85\textwidth}
\newcommand*{\epigraphTranslatedTitle}[1]{\vspace*{.5em}\footnotesize\textsc{#1}\\}%
\newcommand*{\epigraphRootTitle}[1]{\footnotesize\textit{#1}\\}%
\newcommand*{\epigraphReference}[1]{\footnotesize{#1}}%

% map
\newsavebox\IBox

% custom commands for html styling classes
\newcommand*{\scnamo}[1]{\begin{Center}\textit{#1}\end{Center}\bigskip}
\newcommand*{\scendsection}[1]{\begin{Center}\begin{small}\textit{#1}\end{small}\end{Center}\addvspace{1em}}
\newcommand*{\scendsutta}[1]{\begin{Center}\textit{#1}\end{Center}\addvspace{1em}}
\newcommand*{\scendbook}[1]{\bigskip\begin{Center}\uppercase{#1}\end{Center}\addvspace{1em}}
\newcommand*{\scendkanda}[1]{\begin{Center}\textbf{#1}\end{Center}\addvspace{1em}} % use for ending vinaya rule sections and also samyuttas %
\newcommand*{\scend}[1]{\begin{Center}\begin{small}\textit{#1}\end{small}\end{Center}\addvspace{1em}}
\newcommand*{\scendvagga}[1]{\begin{Center}\textbf{#1}\end{Center}\addvspace{1em}}
\newcommand*{\scrule}[1]{\textsb{#1}}
\newcommand*{\scadd}[1]{\textit{#1}}
\newcommand*{\scevam}[1]{\textsc{#1}}
\newcommand*{\scspeaker}[1]{\hspace{2em}\textit{#1}}
\newcommand*{\scbyline}[1]{\begin{flushright}\textit{#1}\end{flushright}\bigskip}
\newcommand*{\scexpansioninstructions}[1]{\begin{small}\textit{#1}\end{small}}
\newcommand*{\scuddanaintro}[1]{\medskip\noindent\begin{footnotesize}\textit{#1}\end{footnotesize}\smallskip}

\newenvironment{scuddana}{%
\setlength{\stanzaskip}{.5\baselineskip}%
  \vspace{-1em}\begin{verse}\begin{footnotesize}%
}{%
\end{footnotesize}\end{verse}
}%

% custom command for thematic break = hr
\newcommand*{\thematicbreak}{\begin{center}\rule[.5ex]{6em}{.4pt}\begin{normalsize}\quad\Fleuronfont{•}\quad\end{normalsize}\rule[.5ex]{6em}{.4pt}\end{center}}

% manage and style page header and footer. "fancy" has header and footer, "plain" has footer only

\pagestyle{fancy}
\fancyhf{}
\fancyfoot[RE,LO]{\thepage}
\fancyfoot[LE,RO]{\footnotesize\lastleftxmark}
\fancyhead[CE]{\setstretch{.85}\Noligaturefont\MakeLowercase{\textsc{\firstrightmark}}}
\fancyhead[CO]{\setstretch{.85}\Noligaturefont\MakeLowercase{\textsc{\firstleftmark}}}
\renewcommand{\headrulewidth}{0pt}
\fancypagestyle{plain}{ %
\fancyhf{} % remove everything
\fancyfoot[RE,LO]{\thepage}
\fancyfoot[LE,RO]{\footnotesize\lastleftxmark}
\renewcommand{\headrulewidth}{0pt}
\renewcommand{\footrulewidth}{0pt}}
\fancypagestyle{plainer}{ %
\fancyhf{} % remove everything
\fancyfoot[RE,LO]{\thepage}
\renewcommand{\headrulewidth}{0pt}
\renewcommand{\footrulewidth}{0pt}}

% style footnotes
\setlength{\skip\footins}{1em}

\makeatletter
\newcommand{\@makefntextcustom}[1]{%
    \parindent 0em%
    \thefootnote.\enskip #1%
}
\renewcommand{\@makefntext}[1]{\@makefntextcustom{#1}}
\makeatother

% hang quotes (requires microtype)
\microtypesetup{
  protrusion = true,
  expansion  = true,
  tracking   = true,
  factor     = 1000,
  patch      = all,
  final
}

% Custom protrusion rules to allow hanging punctuation
\SetProtrusion
{ encoding = *}
{
% char   right left
  {-} = {    , 500 },
  % Double Quotes
  \textquotedblleft
      = {1000,     },
  \textquotedblright
      = {    , 1000},
  \quotedblbase
      = {1000,     },
  % Single Quotes
  \textquoteleft
      = {1000,     },
  \textquoteright
      = {    , 1000},
  \quotesinglbase
      = {1000,     }
}

% make latex use actual font em for parindent, not Computer Modern Roman
\AtBeginDocument{\setlength{\parindent}{1em}}%
%

% Default values; a bit sloppier than normal
\tolerance 1414
\hbadness 1414
\emergencystretch 1.5em
\hfuzz 0.3pt
\clubpenalty = 10000
\widowpenalty = 10000
\displaywidowpenalty = 10000
\hfuzz \vfuzz
 \raggedbottom%

\title{Long Discourses}
\author{Bhikkhu Sujato}
\date{}%
% define a different fleuron for each edition
\newfontfamily\Fleuronfont[Ornament=16]{Arno Pro}

% Define heading styles per edition for chapter and section. Suttatitle can be either of these, depending on the volume. 

\let\oldfrontmatter\frontmatter
\renewcommand{\frontmatter}{%
\chapterfont{\setstretch{.85}\normalfont\centering}%
\sectionfont{\setstretch{.85}\normalfont\BalancedRagged}%
\oldfrontmatter}

\let\oldmainmatter\mainmatter
\renewcommand{\mainmatter}{%
\chapterfont{\setstretch{.85}\normalfont\centering}%
\sectionfont{\setstretch{.85}\normalfont\BalancedRagged}%
\oldmainmatter}

\let\oldbackmatter\backmatter
\renewcommand{\backmatter}{%
\chapterfont{\setstretch{.85}\normalfont\centering}%
\sectionfont{\setstretch{.85}\normalfont\BalancedRagged}%
\pagestyle{plainer}%
\oldbackmatter}

% for reasons, flat texts align too far in the margin in ToC, this fixes it. 
\renewcommand*{\tocacronym}[1]{\hspace*{0em}{#1}\quad}%
%
\begin{document}%
\normalsize%
\frontmatter%
\setlength{\parindent}{0cm}

\pagestyle{empty}

\maketitle

\blankpage%
\begin{center}

\vspace*{2.2em}

\halftitlepageTranslationTitle{Long Discourses}

\vspace*{1em}

\halftitlepageTranslationSubtitle{A faithful translation of the Dīgha Nikāya}

\vspace*{2em}

\halftitlepageFleuron{•}

\vspace*{2em}

\halftitlepageByline{translated and introduced by}

\vspace*{.5em}

\halftitlepageCreatorName{Bhikkhu Sujato}

\vspace*{4em}

\halftitlepageVolumeNumber{Volume 2}

\smallskip

\halftitlepageVolumeAcronym{DN 14–23}

\smallskip

\halftitlepageVolumeTranslationTitle{The Great Chapter}

\smallskip

\halftitlepageVolumeRootTitle{Mahāvagga}

\vspace*{\fill}

\sclogo{0}
 \halftitlepagePublisher{SuttaCentral}

\end{center}

\newpage
%
\setstretch{1.05}

\begin{footnotesize}

\textit{Long Discourses} is a translation of the Dīghanikāya by Bhikkhu Sujato.

\medskip

Creative Commons Zero (CC0)

To the extent possible under law, Bhikkhu Sujato has waived all copyright and related or neighboring rights to \textit{Long Discourses}.

\medskip

This work is published from Australia.

\begin{center}
\textit{This translation is an expression of an ancient spiritual text that has been passed down by the Buddhist tradition for the benefit of all sentient beings. It is dedicated to the public domain via Creative Commons Zero (CC0). You are encouraged to copy, reproduce, adapt, alter, or otherwise make use of this translation. The translator respectfully requests that any use be in accordance with the values and principles of the Buddhist community.}
\end{center}

\medskip

\begin{description}
    \item[Web publication date] 2018
    \item[This edition] 2025-01-13 01:01:43
    \item[Publication type] hardcover
    \item[Edition] ed3
    \item[Number of volumes] 3
    \item[Publication ISBN] 978-1-76132-049-1
    \item[Volume ISBN] 978-1-76132-098-9
    \item[Publication URL] \href{https://suttacentral.net/editions/dn/en/sujato}{https://suttacentral.net/editions/dn/en/sujato}
    \item[Source URL] \href{https://github.com/suttacentral/bilara-data/tree/published/translation/en/sujato/sutta/dn}{https://github.com/suttacentral/bilara-data/tree/published/translation/en/sujato/sutta/dn}
    \item[Publication number] scpub2
\end{description}

\medskip

Map of Jambudīpa is by Jonas David Mitja Lang, and is released by him under Creative Commons Zero (CC0).

\medskip

Published by SuttaCentral

\medskip

\textit{SuttaCentral,\\
c/o Alwis \& Alwis Pty Ltd\\
Kaurna Country,\\
Suite 12,\\
198 Greenhill Road,\\
Eastwood,\\
SA 5063,\\
Australia}

\end{footnotesize}

\newpage

\setlength{\parindent}{1em}%%
\tableofcontents
\newpage
\pagestyle{fancy}
%
\chapter*{Summary of Contents}
\addcontentsline{toc}{chapter}{Summary of Contents}
\markboth{Summary of Contents}{Summary of Contents}

\begin{description}%
\item[The Great Chapter (\textit{\textsanskrit{Mahāvagga}})] This chapter contains a diverse range of discourses. Several focus on the events surrounding the Buddha’s death, while others range into fabulous scenarios set among the gods, and still others are grounded in detailed discussions of doctrine.%
\item[DN 14: The Great Discourse on Traces Left Behind (\textit{\textsanskrit{Mahāpadānasutta}})] The Buddha teaches about the six Buddhas of the past, and tells a lengthy account of one of those, \textsanskrit{Vipassī}.%
\item[DN 15: The Great Discourse on Causation (\textit{\textsanskrit{Mahānidānasutta}})] Rejecting Venerable Ānanda’s claim to easily understand dependent origination, the Buddha presents a complex and demanding analysis, revealing hidden nuances and implications of this central teaching.%
\item[DN 16: The Great Discourse on the Buddha’s Extinguishment (\textit{\textsanskrit{Mahāparinibbānasutta}})] The longest of all discourses, this extended narrative tells of the events surrounding the Buddha’s death. Full of vivid and moving details, it is an ideal entry point into knowing the Buddha as a person, and understanding how the Buddhist community coped with his passing.%
\item[DN 17: King \textsanskrit{Mahāsudassana} (\textit{\textsanskrit{Mahāsudassanasutta}})] An elaborate story of a past life of the Buddha as a legendary king who renounced all to practice meditation.%
\item[DN 18: With Janavasabha (\textit{\textsanskrit{Janavasabhasutta}})] Beginning with an account of the fates of disciples who had recently passed away, the scene shifts to a discussion of Dhamma held by the gods.%
\item[DN 19: The Great Steward (\textit{\textsanskrit{Mahāgovindasutta}})] A minor deity informs the Buddha of the conversations and business of the gods.%
\item[DN 20: The Great Congregation (\textit{\textsanskrit{Mahāsamayasutta}})] When deities from all realms gather in homage to the Buddha, he gives a series of verses describing them. These verses, which are commonly chanted in Theravadin countries, give one of the most detailed descriptions of the deities worshiped at the the time of the Buddha.%
\item[DN 21: Sakka’s Questions (\textit{\textsanskrit{Sakkapañhasutta}})] After hearing a love song from a god of music, the Buddha engages in a deep discussion with Sakka on the conditioned origin of attachment and suffering.%
\item[DN 22: The Longer Discourse on Mindfulness Meditation (\textit{\textsanskrit{Mahāsatipaṭṭhānasutta}})] The Buddha details the seventh factor of the noble eightfold path, mindfulness meditation. This discourse is essentially identical to MN 10, with the addition of an extended section on the four noble truths derived from MN 141.%
\item[DN 23: With \textsanskrit{Pāyāsi} (\textit{\textsanskrit{Pāyāsisutta}})] This is a long and entertaining debate between a monk and a skeptic, who went to elaborate and bizarre lengths to prove that there is no such thing as an afterlife. The discourse contains a colorful series of parables and examples.%
\end{description}

%
\mainmatter%
\pagestyle{fancy}%
\addtocontents{toc}{\let\protect\contentsline\protect\nopagecontentsline}
\part*{The Great Chapter }
\addcontentsline{toc}{part}{The Great Chapter }
\markboth{}{}
\addtocontents{toc}{\let\protect\contentsline\protect\oldcontentsline}

%
\chapter*{{\suttatitleacronym DN 14}{\suttatitletranslation The Great Discourse on Traces Left Behind }{\suttatitleroot Mahāpadānasutta}}
\addcontentsline{toc}{chapter}{\tocacronym{DN 14} \toctranslation{The Great Discourse on Traces Left Behind } \tocroot{Mahāpadānasutta}}
\markboth{The Great Discourse on Traces Left Behind }{Mahāpadānasutta}
\extramarks{DN 14}{DN 14}

\section*{1. On Past Lives }

\scevam{So\marginnote{1.1.1} I have heard.\footnote{In the suttas, \textit{\textsanskrit{apadāna}} means “evidence”, “traces left behind” (\href{https://suttacentral.net/dn27/en/sujato\#16.4}{DN 27:16.4}, \href{https://suttacentral.net/dn24/en/sujato\#2.7.3}{DN 24:2.7.3}, \href{https://suttacentral.net/mn15/en/sujato\#3.32}{MN 15:3.32}, \href{https://suttacentral.net/an3.2/en/sujato\#1.1}{AN 3.2:1.1}). The sense here seems to have been “(story about) the traces that deeds leave behind”. The sense “legend of past lives” probably grew out of this sutta, and later became the title of the collections of legends of past lives of the monks (Thera-\textsanskrit{apadāna}) and nuns (\textsanskrit{Therī}-\textsanskrit{apadāna}). } }At one time the Buddha was staying near \textsanskrit{Sāvatthī} in Jeta’s Grove, \textsanskrit{Anāthapiṇḍika}’s monastery, in the hut by the kareri tree.\footnote{\textit{Kareri} is evidently \textit{Miliusa tomentosa}, known as \textit{hoom} or \textit{kari} in Hindi. It is related to the custard apple, not the curry tree (\textit{Murraya koenigii}). } 

Then\marginnote{1.1.3} after the meal, on return from almsround, several mendicants sat together in the pavilion by the kareri tree and this Dhamma talk on the subject of past lives came up among them, “So it was in a past life; such it was in a past life.”\footnote{“Past life” is \textit{\textsanskrit{pubbenivāsa}}, literally “former abode”. } 

With\marginnote{1.2.1} clairaudience that is purified and superhuman, the Buddha heard that discussion among the mendicants. So he got up from his seat and went to the pavilion, where he sat on the seat spread out and addressed the mendicants, “Mendicants, what were you sitting talking about just now? What conversation was left unfinished?” 

The\marginnote{1.2.4} mendicants told him what had happened, adding, “This is the conversation that was unfinished when the Buddha arrived.” 

“Would\marginnote{1.3.1} you like to hear a Dhamma talk on the subject of past lives?” 

“Now\marginnote{1.3.2} is the time, Blessed One! Now is the time, Holy One! Let the Buddha give a Dhamma talk on the subject of past lives. The mendicants will listen and remember it.” 

“Well\marginnote{1.3.4} then, mendicants, listen and apply your mind well, I will speak.” 

“Yes,\marginnote{1.3.5} sir,” they replied. The Buddha said this: 

“Ninety-one\marginnote{1.4.1} eons ago, the Buddha \textsanskrit{Vipassī} arose in the world, perfected and fully awakened.\footnote{At \href{https://suttacentral.net/mn71/en/sujato\#14.2}{MN 71:14.2} and \href{https://suttacentral.net/sn42.9/en/sujato\#4.5}{SN 42.9:4.5} the Buddha says he recollects ninety-one eons, which must refer back to the life of \textsanskrit{Vipassī}. We shall see the origin of his name below. } Thirty-one eons ago, the Buddha \textsanskrit{Sikhī} arose in the world, perfected and fully awakened.\footnote{\textit{\textsanskrit{Sikhī}} means “crested one” and refers to a halo or crown. A story of him and his disciples is told at \href{https://suttacentral.net/sn6.14/en/sujato\#8.3}{SN 6.14:8.3}. } In the same thirty-first eon, the Buddha \textsanskrit{Vessabhū} arose in the world, perfected and fully awakened.\footnote{\textit{\textsanskrit{Vessabhū}} is Sanskrit \textsanskrit{Viśvabhṛt}. It means “all-bearing”, probably originating as a word for the earth. A legendary king of the same name once ruled over the city of \textsanskrit{Avantī} (\href{https://suttacentral.net/dn19/en/sujato\#36.14}{DN 19:36.14}). } In the present fortunate eon, the Buddhas Kakusandha,\footnote{Due to the appearance of four Buddhas in this cycle it is known as “fortunate” (\textit{bhadda}). | Stories from Kakusandha’s day appear at \href{https://suttacentral.net/mn50/en/sujato}{MN 50} and \href{https://suttacentral.net/sn15.20/en/sujato\#2.8}{SN 15.20:2.8}. There is no obvious etymology for his name, but perhaps it stems from \textit{kaku} (“peak”) and thus “the union of the peaks”, a valley at the junction of mountains. } \textsanskrit{Koṇāgamana},\footnote{The meaning of \textsanskrit{Koṇāgamana} is unclear. In Sanskrit it is often spelled Kanakamuni, “the golden sage”, while in the (dubious) Nigali Sagar Edict of Ashoka it is \textsanskrit{Konākamana}. At \href{https://suttacentral.net/thig16.1/en/sujato\#71.1}{Thig 16.1:71.1} we hear the past life of three nuns in the time of \textsanskrit{Koṇāgamana}. } and Kassapa arose in the world, perfected and fully awakened.\footnote{Kassapa (Sanskrit \textsanskrit{Kaśyapa}) means “tortoise”. It is a common Brahmanical clan name, stemming from an ancient figure reckoned as the eldest of the “seven sages”, to whom some Vedic verses are attributed. Details of the Buddha Kassapa’s time are found at \href{https://suttacentral.net/mn81/en/sujato}{MN 81}, \href{https://suttacentral.net/sn15.20/en/sujato\#4.1}{SN 15.20:4.1}, and \href{https://suttacentral.net/sn48.57/en/sujato\#3.1}{SN 48.57:3.1}. } And in the present fortunate eon, I have arisen in the world, perfected and fully awakened.\footnote{These numbers make up a quasi-logarithmic scale: the time gaps multiply by three, while the number of Buddhas divides by two. In eon one there are four Buddhas. Thirty eons ago the number is halved, so they had two Buddhas. Twice-thirty eons before that, the number is halved again, to one Buddha, thus ending the scale. } 

The\marginnote{1.5.1} Buddhas \textsanskrit{Vipassī}, \textsanskrit{Sikhī}, and \textsanskrit{Vessabhū} were born as aristocrats into aristocrat families. The Buddhas Kakusandha, \textsanskrit{Koṇāgamana}, and Kassapa were born as brahmins into brahmin families. I was born as an aristocrat into an aristocrat family.\footnote{In ancient India, the “aristocrats” (\textit{khattiya}) and the brahmins vied for the prestige of being the highest class. The traditional business of the aristocrats was land, politics, agriculture, and war, while the brahmins were a hereditary priestly caste who served as advisers and ritualists. It is said that the Buddha-to-be was born in whatever caste was most prestigious at the time so that his word would have the most impact. | In this sutta, the Buddha simply refers to himself as “I” with no personal name. When the seven Buddhas are listed at \href{https://suttacentral.net/dn32/en/sujato\#3.14}{DN 32:3.14}, however, he is called \textsanskrit{Aṅgīrasa}. } 

\textsanskrit{Koṇḍañña}\marginnote{1.6.1} was the clan of \textsanskrit{Vipassī},\footnote{The \textsanskrit{Koṇḍaññas} (Sanskrit \textsanskrit{Kauṇḍinya}) originated as the leading clan of the city of \textsanskrit{Kuṇḍina} the capital of Vidarbha, which is probably modern Kaundinyapura on the Wardha River. } \textsanskrit{Sikhī}, and \textsanskrit{Vessabhū}. Kassapa was the clan of Kakusandha, \textsanskrit{Koṇāgamana}, and Kassapa. Gotama is my clan.\footnote{Gotama is an ancient Brahmanical clan name, which like Kassapa stems from the Vedas and the seven sages. Why does the Buddha, an aristocrat, have a Brahmanical clan? During initiation by a Brahmanical family priest (\textit{purohita}), an aristocrat was ritually determined to be a brahmin for a short time, during which he would assume the lineage name of the priest. After being restored as an aristocrat, he and his family would still be referred to by that name. Thus the Sakyans’ \textit{purohita} must have been of the Gotama lineage. Other examples of this practice include the Mallas who are called \textsanskrit{Vāseṭṭha} (\href{https://suttacentral.net/dn16/en/sujato\#5.19.2}{DN 16:5.19.2}) and Saccaka who is called Aggivessana (\href{https://suttacentral.net/mn35/en/sujato\#4.2}{MN 35:4.2}). The Jain \textsanskrit{Ācārāṅgasūtra} 2.15.15 records a similar situation, for while \textsanskrit{Mahāvīra} was, like the Buddha, a \textit{khattiya}, he was of the \textsanskrit{Kāśyapa} \textit{gotta}, and various relatives were assigned to \textsanskrit{Vāsiṣṭha},  \textsanskrit{Kauṇḍinya}, and  \textsanskrit{Kauśika}. } 

For\marginnote{1.7.1} \textsanskrit{Vipassī}, the lifespan was 80,000 years. For \textsanskrit{Sikhī}, the lifespan was 70,000 years. For \textsanskrit{Vessabhū}, the lifespan was 60,000 years. For Kakusandha, the lifespan was 40,000 years. For \textsanskrit{Koṇāgamana}, the lifespan was 30,000 years. For Kassapa, the lifespan was 20,000 years. For me at this time the lifespan is short, brief, and fleeting. A long-lived person lives for a century or a little more. 

\textsanskrit{Vipassī}\marginnote{1.8.1} was awakened at the root of a patala tree.\footnote{\emph{Stereospermum chelonoides}. Dictionaries of Pali and Sanskrit call this the “trumpet-flower tree”, but that is more commonly used for \emph{Oroxylum indicum}. } \textsanskrit{Sikhī} was awakened at the root of a white-mango tree.\footnote{\textit{\textsanskrit{Puṇḍarīka}} is the white lotus, but here it is the name of a tree. The commentary says this is \textit{setamba} (“white-mango”). It is not, however, the mango variety known by that name today (\emph{Mangifera caesia}), which is not found in India. } \textsanskrit{Vessabhū} was awakened at the root of a sal tree.\footnote{\emph{Shorea robusta}. } Kakusandha was awakened at the root of a sirisa tree.\footnote{\emph{Albizia lebbeck}, sometimes still known by the old name of sirisa. Older sources sometimes call it an acacia. } \textsanskrit{Koṇāgamana} was awakened at the root of a cluster fig tree.\footnote{\emph{Ficus racemosa}. } Kassapa was awakened at the root of a banyan tree.\footnote{\emph{Ficus benghalensis}. } I was awakened at the root of a peepal tree.\footnote{\emph{Ficus religiosa}, sacred to Buddhists, Jains, and Hindus alike. Ancient sources referred to it both as \textit{assattha} (Sanskrit \textit{\textsanskrit{aśvattha}}) and \textit{pippala}. Buddhists today simply call it the Bodhi (or Bo) tree. } 

\textsanskrit{Vipassī}\marginnote{1.9.1} had a fine pair of chief disciples named \textsanskrit{Khaṇḍa} and Tissa. \textsanskrit{Sikhī} had a fine pair of chief disciples named \textsanskrit{Abhibhū} and Sambhava. \textsanskrit{Vessabhū} had a fine pair of chief disciples named \textsanskrit{Soṇa} and Uttara. Kakusandha had a fine pair of chief disciples named Vidhura and \textsanskrit{Sañjīva}. \textsanskrit{Koṇāgamana} had a fine pair of chief disciples named Bhiyyosa and Uttara. Kassapa had a fine pair of chief disciples named Tissa and \textsanskrit{Bhāradvāja}. I have a fine pair of chief disciples named \textsanskrit{Sāriputta} and \textsanskrit{Moggallāna}. 

\textsanskrit{Vipassī}\marginnote{1.10.1} had three gatherings of disciples—one of 6,800,000, one of 100,000, and one of 80,000—all of them mendicants who had ended their defilements.\footnote{The number of disciples diminishes as we approach historical time. } 

\textsanskrit{Sikhī}\marginnote{1.10.2} had three gatherings of disciples—one of 100,000, one of 80,000, and one of 70,000—all of them mendicants who had ended their defilements. 

\textsanskrit{Vessabhū}\marginnote{1.10.3} had three gatherings of disciples—one of 80,000, one of 70,000, and one of 60,000—all of them mendicants who had ended their defilements. 

Kakusandha\marginnote{1.10.4} had one gathering of disciples—40,000 mendicants who had ended their defilements. 

\textsanskrit{Koṇāgamana}\marginnote{1.10.5} had one gathering of disciples—30,000 mendicants who had ended their defilements. 

Kassapa\marginnote{1.10.6} had one gathering of disciples—20,000 mendicants who had ended their defilements. 

I\marginnote{1.10.7} have had one gathering of disciples—1,250 mendicants who had ended their defilements.\footnote{An extensive journey of the Buddha with 1,250 mendicants is documented in \href{https://suttacentral.net/pli-tv-kd6/en/sujato}{Kd 6}, at one point of which occurs the events of the Selasutta (\href{https://suttacentral.net/snp3.7/en/sujato}{Snp 3.7} and \href{https://suttacentral.net/mn91/en/sujato}{MN 91}). The Buddha is also accompanied by 1,250 mendicants in the \textsanskrit{Sāmaññaphalasutta} (\href{https://suttacentral.net/dn2/en/sujato}{DN 2}) and the Parosahassasutta (\href{https://suttacentral.net/sn8.8/en/sujato}{SN 8.8}). } 

\textsanskrit{Vipassī}\marginnote{1.11.1} had as chief attendant a mendicant named Asoka. \textsanskrit{Sikhī} had as chief attendant a mendicant named \textsanskrit{Khemaṅkara}. \textsanskrit{Vessabhū} had as chief attendant a mendicant named Upasanta. Kakusandha had as chief attendant a mendicant named Buddhija. \textsanskrit{Koṇāgamana} had as chief attendant a mendicant named Sotthija. Kassapa had as chief attendant a mendicant named Sabbamitta. I have as chief attendant a mendicant named Ānanda. 

\textsanskrit{Vipassī}’s\marginnote{1.12.1} father was King \textsanskrit{Bandhumā}, his birth mother was Queen \textsanskrit{Bandhumatī}, and their capital city was named \textsanskrit{Bandhumatī}. 

\textsanskrit{Sikhī}’s\marginnote{1.12.4} father was King \textsanskrit{Aruṇa},\footnote{The son of King Dawn and Queen Radiant at the City of the Dawn, \textsanskrit{Sikhī} is the streaming rays of sunrise. These names recall a solar myth. } his birth mother was Queen \textsanskrit{Pabhāvatī},\footnote{In \textsanskrit{Mārkaṇḍeya} \textsanskrit{Purāṇa} ch. 131, a \textsanskrit{Pabhāvatī}, daughter of the king of Vidarbha, is said to have married Marutta, king of \textsanskrit{Vajjī}. } and their capital city was named \textsanskrit{Aruṇavatī}.\footnote{There is an Arunawati River in Maharashtra, not far from the ancient homeland of the \textsanskrit{Koṇḍaññas} in \textsanskrit{Kuṇḍina}. } 

\textsanskrit{Vessabhū}’s\marginnote{1.12.7} father was King Suppatita,\footnote{\textsanskrit{Vessabhū}’s parents King Goodfall and Queen Rainy recall a seasonal fertility myth, where they gave life to their son the earth, “the nourisher and bearer of all”. Note, however the variants \textit{\textsanskrit{suppatīta}} (“well pleased”) and \textit{\textsanskrit{yasavatī}} (“celebrated”). } his birth mother was Queen \textsanskrit{Vassavatī}, and their capital city was named Anoma. 

Kakusandha’s\marginnote{1.12.10} father was the brahmin Aggidatta,\footnote{The names of the brahmin fathers all refer to ritual offerings. Aggidatta means “offered to the fire”. } and his birth mother was the brahmin lady \textsanskrit{Visākhā}.\footnote{The names of the brahmin mothers all recall Indian \textit{nakkhatta}. These are “constellations” or more properly “lunar mansions”; segments of the sky through which the Moon passes and which are associated with certain stars or clusters. In the Atharvaveda system, \textsanskrit{Visākhā} is the 16th lunar asterism (Libra). } At that time the king was Khema, whose capital city was named \textsanskrit{Khemavatī}.\footnote{Based on the Ashoka pillar there, this is identified with modern Gotihawa, southeast of Kapilavastu in Nepal. } 

\textsanskrit{Koṇāgamana}’s\marginnote{1.12.14} father was the brahmin \textsanskrit{Yaññadatta},\footnote{“Offered in sacrifice”. } and his birth mother was the brahmin lady \textsanskrit{Uttarā}.\footnote{Several constellations are distinguished as “former” (\textit{pubba}) and “later” (\textit{uttara}). Since it lies between \textsanskrit{Visākhā} and \textsanskrit{Dhanavatī}, this may be the 21st constellation, Uttara \textsanskrit{Āṣāḍhā} (Sagittarius/Capricorn). } At that time the king was Sobha, whose capital city was named \textsanskrit{Sobhavatī}.\footnote{If the dubious Ashokan edict there is to be believed, this was traditionally identified with the town known today as Nigali Sagar near Kapilavastu in Nepal. } 

Kassapa’s\marginnote{1.12.18} father was the brahmin Brahmadatta,\footnote{“Offered to \textsanskrit{Brahmā}”. Many of the \textsanskrit{Jātakas} feature a king of this name, usually said to reside in Benares. } and his birth mother was the brahmin lady \textsanskrit{Dhanavatī}.\footnote{\textsanskrit{Dhanavatī} means “wealthy”, but it is also an alternate name for the 23rd constellation, \textit{\textsanskrit{dhaniṣṭhā}} (or \textit{\textsanskrit{śraviṣṭā}}, Delphinus). } At that time the king was \textsanskrit{Kikī},\footnote{\textsanskrit{Kikī} (“blue jay”) appears in the story of the past Buddha Kassapa at \href{https://suttacentral.net/mn81/en/sujato}{MN 81}. } whose capital city was named Varanasi. 

In\marginnote{1.12.22} this life, my father was King Suddhodana,\footnote{Suddhodana means “white rice”. He is mentioned by name at \href{https://suttacentral.net/thag10.1/en/sujato\#8.1}{Thag 10.1:8.1}, \href{https://suttacentral.net/snp3.11/en/sujato\#7.2}{Snp 3.11:7.2}, and \href{https://suttacentral.net/pli-tv-kd1/en/sujato\#54.1.4}{Kd 1:54.1.4}. } my birth mother was Queen \textsanskrit{Māyā},\footnote{\textsanskrit{Māyā} means “illusion”. She is mentioned by name at \href{https://suttacentral.net/thag10.1/en/sujato\#8.2}{Thag 10.1:8.2} and \href{https://suttacentral.net/thig6.6/en/sujato\#6.2}{Thig 6.6:6.2}. } and our capital city was Kapilavatthu.” 

That\marginnote{1.12.25} is what the Buddha said. When he had spoken, the Holy One got up from his seat and entered his dwelling. 

Soon\marginnote{1.13.1} after the Buddha left, those mendicants discussed among themselves: 

“It’s\marginnote{1.13.2} incredible, reverends, it’s amazing! The Realized One has such psychic power and might! For he is able to recollect the birth, names, clans, lifespan, chief disciples, and gatherings of disciples of the Buddhas of the past who have become fully quenched, cut off proliferation, cut off the track, finished off the cycle, and transcended all suffering. He knows their birth, names, clans, conduct, qualities, wisdom, meditation, and freedom. 

Is\marginnote{1.13.5} it because the Realized One has clearly comprehended the principle of the teachings that he can recollect all these things?\footnote{“Principle of the teachings” is \textit{\textsanskrit{dhammadhātu}}. At \href{https://suttacentral.net/sn12.32/en/sujato\#18.1}{SN 12.32:18.1}, \textsanskrit{Sāriputta}’s mastery of the \textit{\textsanskrit{dhammadhātu}} gives him the ability to answer any questions on the Dhamma. } Or did deities tell him?” But this conversation among those mendicants was left unfinished. 

Then\marginnote{1.14.1} in the late afternoon, the Buddha came out of retreat and went to the pavilion by the kareri tree, where he sat on the seat spread out and addressed the mendicants, “Mendicants, what were you sitting talking about just now? What conversation was left unfinished?” 

The\marginnote{1.14.4} mendicants told him what had happened, adding, “This was our conversation that was unfinished when the Buddha arrived.” 

“It\marginnote{1.15.1} is because the Realized One has clearly comprehended the principle of the teachings that he can recollect all these things. And the deities also told me. 

Would\marginnote{1.15.5} you like to hear a further Dhamma talk on the subject of past lives?” 

“Now\marginnote{1.15.6} is the time, Blessed One! Now is the time, Holy One! Let the Buddha give a further Dhamma talk on the subject of past lives. The mendicants will listen and remember it.” 

“Well\marginnote{1.15.8} then, mendicants, listen and apply your mind well, I will speak.” 

“Yes,\marginnote{1.15.9} sir,” they replied. The Buddha said this: 

“Ninety-one\marginnote{1.16.1} eons ago, the Buddha \textsanskrit{Vipassī} arose in the world, perfected and fully awakened. He was born as an aristocrat into an aristocrat family. His clan was \textsanskrit{Koṇḍañña}. He lived for 80,000 years. He was awakened at the root of a patala tree. He had a fine pair of chief disciples named \textsanskrit{Khaṇḍa} and Tissa. He had three gatherings of disciples—one of 6,800,000, one of 100,000, and one of 80,000—all of them mendicants who had ended their defilements. He had as chief attendant a mendicant named Asoka. His father was King \textsanskrit{Bandhumā}, his birth mother was Queen \textsanskrit{Bandhumatī}, and their capital city was named \textsanskrit{Bandhumatī}. 

\section*{2. What’s Normal For One Intent on Awakening }

When\marginnote{1.17.1} \textsanskrit{Vipassī}, the being intent on awakening, passed away from the host of joyful gods, he was conceived in his mother’s womb, mindful and aware.\footnote{These characteristics of the birth of the Buddha-to-be are also found in \href{https://suttacentral.net/mn123/en/sujato}{MN 123}, where Ānanda quotes the Buddha, although the wording is a little different and three extra details are added at the start. In addition, this detail and that about emerging mindfully from the womb are found at \href{https://suttacentral.net/an4.127/en/sujato}{AN 4.127}. | Normally in early Pali, the word \textit{bodhisatta} is reserved for the Buddha-to-be once he has left home and is practicing “intent on awakening” (eg. \href{https://suttacentral.net/mn26/en/sujato\#13.1}{MN 26:13.1}). This passage extends the usage back as far as the end of the immediate past life. } This is normal in such a case.\footnote{“Normal” is \textit{\textsanskrit{dhammatā}}, a natural principle. This entire passage differs from the earliest concept of the bodhisatta (“one intent on awakening”), which in early texts is typically applied to Siddhattha after he left the home. } 

It’s\marginnote{1.17.3} normal that, when the being intent on awakening passes away from the host of joyful gods, he is conceived in his mother’s womb. And then—in this world with its gods, \textsanskrit{Māras}, and divinities, this population with its ascetics and brahmins, gods and humans—an immeasurable, magnificent light appears, surpassing the glory of the gods. Even in the boundless void of interstellar space—so utterly dark that even the light of the moon and the sun, so mighty and powerful, makes no impression—an immeasurable, magnificent light appears, surpassing the glory of the gods.\footnote{The commentary identifies this realm of “utter darkness” (\textit{\textsanskrit{andhakāratimisā}}) with a cold hell realm. There is a corresponding \textsanskrit{Purāṇic} hell called \textit{\textsanskrit{andhatāmisra}}. | \textit{\textsanskrit{Asaṁvutā}} was translated by \textsanskrit{Ñāṇamoḷī} as “abysmal”, but this relies on a commentarial cosmology that is not found in the suttas. The sense, rather, is “boundless”. The root harks back to the Vedic serpent \textsanskrit{Vṛtra} who wraps the world in darkness. | \textit{\textsanskrit{Nānubhonti}} (“makes no impression”) is glossed in the commentary to \href{https://suttacentral.net/an4.127/en/sujato}{AN 4.127} as \textit{nappahonti} “ineffective”. } And the sentient beings reborn there recognize each other by that light: ‘So, it seems other sentient beings have been reborn here!’\footnote{The light is a physical one, not just a metaphor. From this, it appears that sentient beings may be spontaneously reborn in interstellar space. Compare the problem of the “Boltzmann brain” in physics. } And this ten-thousandfold galaxy shakes and rocks and trembles. And an immeasurable, magnificent light appears in the world, surpassing the glory of the gods. This is normal in such a case. 

It’s\marginnote{1.17.9} normal that, when the being intent on awakening is conceived in his mother’s belly, four gods approach to guard the four quarters, so that no human or non-human or anyone at all shall harm the being intent on awakening or his mother.\footnote{These are the Four Great Kings, regarded as protector deities. } This is normal in such a case. 

It’s\marginnote{1.18.1} normal that, when the being intent on awakening is conceived in his mother’s belly, she becomes naturally ethical. She refrains from killing living creatures, stealing, sexual misconduct, lying, and beer, wine, and liquor intoxicants.\footnote{The five precepts. } This is normal in such a case. 

It’s\marginnote{1.19.1} normal that, when the being intent on awakening is conceived in his mother’s belly, she no longer feels sexual desire for men, and she cannot be violated by a man of lustful intent. This is normal in such a case. 

It’s\marginnote{1.20.1} normal that, when the being intent on awakening is conceived in his mother’s belly, she obtains the five kinds of sensual stimulation and amuses herself, supplied and provided with them.\footnote{While sensual pleasures provoke attachment, they are nonetheless a kind of pleasure and therefore a sign of virtue and good past kamma. } This is normal in such a case. 

It’s\marginnote{1.21.1} normal that, when the being intent on awakening is conceived in his mother’s belly, no afflictions beset her. She’s happy and free of bodily fatigue. And she sees the being intent on awakening in her womb, whole in his major and minor limbs, not deficient in any faculty. Suppose there was a beryl gem that was naturally beautiful, eight-faceted, well-worked, transparent, clear, and unclouded, endowed with all good qualities. And it was strung with a thread of blue, yellow, red, white, or golden brown. And someone with clear eyes were to take it in their hand and examine it: ‘This beryl gem is naturally beautiful, eight-faceted, well-worked, transparent, clear, and unclouded, endowed with all good qualities. And it’s strung with a thread of blue, yellow, red, white, or golden brown.’ 

In\marginnote{1.21.4} the same way, when the being intent on awakening is conceived in his mother’s belly, no afflictions beset her. She’s happy and free of bodily fatigue. And she sees the being intent on awakening in her womb, whole in his major and minor limbs, not deficient in any faculty. This is normal in such a case. 

It’s\marginnote{1.22.1} normal that, seven days after the being intent on awakening is born, his mother passes away and is reborn in the host of joyful gods.\footnote{This tragic detail is also mentioned in \href{https://suttacentral.net/ud5.2/en/sujato}{Ud 5.2}. The Buddha is raised by a step-mother, which in our Buddha’s case was \textsanskrit{Māyā}’s sister \textsanskrit{Mahāpajāpatī} (\href{https://suttacentral.net/an8.51/en/sujato\#9.8}{AN 8.51:9.8}, \href{https://suttacentral.net/mn142/en/sujato\#3.3}{MN 142:3.3}). } This is normal in such a case. 

It’s\marginnote{1.23.1} normal that, while other women carry the infant in the womb for nine or ten months before giving birth, not so the mother of the being intent on awakening. She gives birth after exactly ten months.\footnote{Ten signifies fullness and completion, as for example the “ten directions”. | The notion that the term of pregnancy was “nine or ten months” is also found at \textsanskrit{Chāndogya} \textsanskrit{Upaniṣad} 5.9.1. In the Rig Veda it is typically “in the tenth month” (5.78.7, 10.84.3). } This is normal in such a case. 

It’s\marginnote{1.24.1} normal that, while other women give birth while sitting or lying down, not so the mother of the being intent on awakening. She only gives birth standing up.\footnote{In illustrations she is depicted standing while holding a tree in the pose known as \textit{\textsanskrit{sālabhañjikā}}, a common motif in Indian art representing the abundance of springtime. } This is normal in such a case. 

It’s\marginnote{1.25.1} normal that, when the being intent on awakening emerges from his mother’s womb, gods receive him first, then humans. This is normal in such a case. 

It’s\marginnote{1.26.1} normal that, when the being intent on awakening emerges from his mother’s womb, before he reaches the ground, four gods receive him and place him before his mother, saying: ‘Rejoice, O Queen! An illustrious son is born to you.’ This is normal in such a case. 

It’s\marginnote{1.27.1} normal that, when the being intent on awakening emerges from his mother’s womb, he emerges already clean, unsoiled by waters, mucus, blood, or any other kind of impurity, pure and clean. Suppose a jewel-treasure was placed on a cloth from \textsanskrit{Kāsi}. The jewel would not soil the cloth, nor would the cloth soil the jewel.\footnote{\textsanskrit{Kāsi} is the nation of which Varanasi is the capital. Their cloth was of exceptional quality. } Why is that? Because of the cleanliness of them both. 

In\marginnote{1.27.5} the same way, when the being intent on awakening emerges from his mother’s womb, he emerges already clean, unsoiled by waters, mucus, blood, or any other kind of impurity, pure and clean. This is normal in such a case. 

It’s\marginnote{1.28.1} normal that, when the being intent on awakening emerges from his mother’s womb, two showers of water appear from the sky, one cool, one warm, for bathing the being intent on awakening and his mother. This is normal in such a case. 

It’s\marginnote{1.29.1} normal that, as soon as he’s born, the being intent on awakening stands firm with his own feet on the ground. Facing north, he takes seven strides with a white parasol held above him, surveys all quarters, and makes this dramatic proclamation: ‘I am the foremost in the world! I am the eldest in the world! I am the first in the world! This is my last rebirth; now there’ll be no more future lives.’\footnote{This passage implies that Buddhahood was destined from the time of birth, which stands in contrast to the rest of the suttas, where Buddhahood was hard-won by the Bodhisatta’s efforts while striving for awakening. | “Stands firm on his own feet” signifies that he will be awakened by his own efforts. | “North” is \textit{uttara}, which is also “the beyond”; this predicts his attaining Nibbana. | “Seven strides” signifies crossing over the vast cycles of birth and death, especially by developing the seven awakening factors. | The “white parasol” signifies purity and royalty. | “Surveying all quarters” signifies his universal knowledge. | The “dramatic proclamation” is \textit{\textsanskrit{āsabhiṁ} \textsanskrit{vācaṁ}}, literally the “voice of a bull”. Other contexts show that this is is an expression emphasizing speech that is dramatic and imposing (\href{https://suttacentral.net/sn52.9/en/sujato\#3.2}{SN 52.9:3.2}, \href{https://suttacentral.net/dn28/en/sujato\#1.5}{DN 28:1.5} = \href{https://suttacentral.net/sn47.12/en/sujato\#5.2}{SN 47.12:5.2}). | At \textsanskrit{Bṛhadāraṇyaka} \textsanskrit{Upaniṣad} 6.1.1 and \textsanskrit{Chāndogya} \textsanskrit{Upaniṣad} 5.1.1 the “vital breath” (\textit{\textsanskrit{prāṇa}}) is said to be “eldest and first” (\textit{\textsanskrit{jyeṣṭhaśca} \textsanskrit{śreṣṭhaśca}}). } This is normal in such a case. 

It’s\marginnote{1.30.1} normal that, when the being intent on awakening emerges from his mother’s womb, then—in this world with its gods, \textsanskrit{Māras}, and divinities, this population with its ascetics and brahmins, gods and humans—an immeasurable, magnificent light appears, surpassing the glory of the gods. Even in the boundless void of interstellar space—so utterly dark that even the light of the moon and the sun, so mighty and powerful, makes no impression—an immeasurable, magnificent light appears, surpassing the glory of the gods. And the sentient beings reborn there recognize each other by that light: ‘So, it seems other sentient beings have been reborn here!’ And this ten-thousandfold galaxy shakes and rocks and trembles. And an immeasurable, magnificent light appears in the world, surpassing the glory of the gods. This is normal in such a case. 

\section*{3. The Thirty-Two Marks of a Great Man }

When\marginnote{1.31.1} Prince \textsanskrit{Vipassī} was born, they announced it to King \textsanskrit{Bandhumā},\footnote{Many of the details of the following account were later incorporated into the life of Gotama, under the principle that the major events of the lives of Buddhas follow a natural order. Nonetheless, they are not always consistent with other accounts in early texts. For example, in the \textsanskrit{Attadaṇḍasutta} the Buddha says his going forth was prompted by disillusionment and fear due to chronic conflict and warfare (\href{https://suttacentral.net/snp4.15/en/sujato}{Snp 4.15}). } ‘Sire, your son is born! Let your majesty examine him!’ When the king had examined the prince, he had the brahmin soothsayers summoned and said to them,\footnote{In \href{https://suttacentral.net/snp3.11/en/sujato}{Snp 3.11}, the newborn Siddhattha is examined by the dark hermit Asita. These two versions are combined in later accounts. } ‘Gentlemen, please examine the prince.’ When they had examined him they said to the king, ‘Rejoice, O King! An illustrious son is born to you. You are fortunate, so very fortunate, to have a son such as this born in this family! For the prince has the thirty-two marks of a great man. A great man who possesses these has only two possible destinies, no other.\footnote{Asita did not look at the 32 marks, and he predicted only one destiny: that he would become a Buddha. } If he stays at home he becomes a king, a wheel-turning monarch, a just and principled king. His dominion extends to all four sides, he achieves stability in the country, and he possesses the seven treasures. He has the following seven treasures:\footnote{Various “treasures” (\textit{ratana}) or “gems” of a king are discussed in such texts as Śatapatha \textsanskrit{Brāhmaṇa} 5.3.1. } the wheel, the elephant, the horse, the jewel, the woman, the householder, and the commander as the seventh treasure. He has over a thousand sons who are valiant and heroic, crushing the armies of his enemies. After conquering this land girt by sea, he reigns by principle, without rod or sword. But if he goes forth from the lay life to homelessness, he becomes a perfected one, a fully awakened Buddha, who draws back the veil from the world. 

And\marginnote{1.32.1} what are the marks which he possesses?\footnote{The marks are elsewhere listed in \href{https://suttacentral.net/dn30/en/sujato\#1.2.1}{DN 30:1.2.1} and \href{https://suttacentral.net/mn91/en/sujato\#9.1}{MN 91:9.1}. Here I list the related marks in the \textsanskrit{Bṛhatsaṁhitā} as identified by Nathan McGovern (\emph{On the Origins of the 32 Marks of a Great Man}, Journal of the International Association of Buddhist Studies, 2016, vol. 39, pp. 207–247). } 

He\marginnote{1.32.7} has well-planted feet.\footnote{This echoes the posture of the newborn bodhisatta, and has the same meaning: that he will become awakened by “standing on his own two feet”. } 

On\marginnote{1.32.8} the soles of his feet there are thousand-spoked wheels, with rims and hubs, complete in every detail.\footnote{These leave marks that were seen by \textsanskrit{Doṇa} (\href{https://suttacentral.net/an4.36/en/sujato\#1.3}{AN 4.36:1.3}). They are often depicted in Buddhist art, signifying the perfection and completeness of the traces that the Buddha leaves behind in his teachings and practice. \textsanskrit{Bṛhatsaṁhitā} 69.17 lists several auspicious marks, including the wheel. } 

He\marginnote{1.32.9} has stretched heels.\footnote{Described as “abundantly long” at \href{https://suttacentral.net/dn30/en/sujato\#1.12.8}{DN 30:1.12.8}. } 

He\marginnote{1.32.10} has long fingers.\footnote{Same at \textsanskrit{Bṛhatsaṁhitā} 68.36. } 

His\marginnote{1.32.11} hands and feet are tender.\footnote{Tender feet at \textsanskrit{Bṛhatsaṁhitā} 68.2. } 

He\marginnote{1.32.12} has serried hands and feet.\footnote{\textsanskrit{Bṛhatsaṁhitā} 68.2 has \textit{\textsanskrit{śliṣtāṅgulī}} (“compact or sticky fingers”). The commentary denies that the Pali \textit{\textsanskrit{jāla}} means a physical web. I think it means that the fingers and toes were usually held together rather than splayed, hence not letting things slip through the fingers. } 

The\marginnote{1.32.13} tops of his feet are arched.\footnote{\textit{\textsanskrit{Ussaṅkha}} means “(curved) up like a shell”, while \textsanskrit{Bṛhatsaṁhitā} 68.2 says “curved up like a tortoise”. The descriptive verse at \href{https://suttacentral.net/dn30/en/sujato\#1.21.12}{DN 30:1.21.12} shows that it refers to the tops of the feet. } 

His\marginnote{1.32.14} calves are like those of an antelope.\footnote{These are presumably the long, elegant rear calves of the Indian Blackbuck. } 

When\marginnote{1.32.15} standing upright and not bending over, the palms of both hands touch the knees.\footnote{This agrees with \textsanskrit{Bṛhatsaṁhitā} 68.35. } 

His\marginnote{1.32.16} private parts are covered in a foreskin.\footnote{Same at \textsanskrit{Bṛhatsaṁhitā} 68.8. } 

He\marginnote{1.32.17} is golden colored; his skin shines like lustrous gold.\footnote{\textsanskrit{Bṛhatsaṁhitā} 68.102 says kings have a shining complexion. } 

He\marginnote{1.32.18} has delicate skin, so delicate that dust and dirt don’t stick to his body.\footnote{\textsanskrit{Bṛhatsaṁhitā} 68.102 mentions a “clean complexion” (\textit{\textsanskrit{śuddha}}). } 

His\marginnote{1.32.19} hairs grow one per pore.\footnote{Same at \textsanskrit{Bṛhatsaṁhitā} 68.5. } 

His\marginnote{1.32.20} hairs stand up; they’re blue-black and curl clockwise.\footnote{\textsanskrit{Bṛhatsaṁhitā} 68.26 says those with hairs turning right become kings. } 

His\marginnote{1.32.21} body is tall and straight-limbed.\footnote{Here \textit{brahm-} is an adjective from √\textit{brah} + \textit{ma}, equivalent to the Sanskrit \textit{\textsanskrit{bṛṁh}}, having the sense “grown, extended”. The Sanskrit form here is \textit{\textsanskrit{bṛhadṛjugātra}}. } 

He\marginnote{1.32.22} has bulging muscles in seven places.\footnote{Hands, feet, shoulders, and chest (\href{https://suttacentral.net/dn30/en/sujato\#1.13.5}{DN 30:1.13.5}). } 

His\marginnote{1.32.23} chest is like that of a lion.\footnote{\textsanskrit{Bṛhatsaṁhitā} 68.18 compares not the chest but the hips with a lion. } 

He\marginnote{1.32.24} is filled out between the shoulders.\footnote{\textsanskrit{Bṛhatsaṁhitā} 68.27 says the heart is raised and muscular. } 

He\marginnote{1.32.25} has the proportional circumference of a banyan tree: the span of his arms equals the height of his body.\footnote{\textsanskrit{Bṛhatsaṁhitā} 69.13 has the same proportions without the simile. These are the normal human proportions, yet we cannot touch our knees without bending. The only way these marks could be reconciled is if the arms were extra long and the length of the legs below the knees was extra long as well. And this is exactly what we are told: the ankles are stretched and long, and the calves are like those of an antelope, whose rear calves are long proportionate to the thigh. Thus in this regard the marks appear to be internally consistent, though not describing normal human anatomy. } 

His\marginnote{1.32.26} torso is cylindrical. 

He\marginnote{1.32.27} has ridged taste buds.\footnote{“Ridged taste buds” is \textit{\textsanskrit{rasaggasaggī}}. \textit{Rasa} can mean either “taste” or “nutrition”, but the use of \textit{\textsanskrit{ojā}} in \href{https://suttacentral.net/dn30/en/sujato\#2.9.8}{DN 30:2.9.8} confirms the latter. \textit{Gasa} is “swallow” and per \href{https://suttacentral.net/dn30/en/sujato\#2.7.4}{DN 30:2.7.4} it is the “conveyance of savor” (\textit{\textsanskrit{rasaharaṇīyo}}). \textit{Agga} often means “best”, but this is derived from the primary sense of “peak”. The descriptors \textit{uddhagga} (“raised”) at \href{https://suttacentral.net/dn30/en/sujato\#2.7.4}{DN 30:2.7.4} and \textit{\textsanskrit{susaṇṭhitā}} (“prominent”) at \href{https://suttacentral.net/dn30/en/sujato\#2.9.8}{DN 30:2.9.8} confirm that the latter is meant. The mark refers to taste buds raised in noticeable ridges on the tongue that absorb nutrition and aid digestion. It has often been interpreted as “excellent (\textit{\textsanskrit{aggī}}) sense (\textit{gasa}) of taste (\textit{rasa})”, but this, being imperceptible to others, is rather a secondary consequence of the mark. } 

His\marginnote{1.32.28} jaw is like that of a lion. 

He\marginnote{1.32.29} has forty teeth. 

His\marginnote{1.32.30} teeth are even.\footnote{Even, gapless, and white teeth are at \textsanskrit{Bṛhatsaṁhitā} 68.52. } 

His\marginnote{1.32.31} teeth have no gaps. 

His\marginnote{1.32.32} teeth are perfectly white. 

He\marginnote{1.32.33} has a large tongue.\footnote{Same at \textsanskrit{Bṛhatsaṁhitā} 68.53. } 

He\marginnote{1.32.34} has the voice of the Divinity, like a cuckoo’s call. 

His\marginnote{1.32.35} eyes are indigo.\footnote{At \href{https://suttacentral.net/thig13.1/en/sujato\#6.2}{Thig 13.1:6.2} \textsanskrit{Ambapālī} describes her eyes as \textit{\textsanskrit{abhinīla}}. While some Indians do indeed have blue eyes, this probably describes a black so deep it appears blue. } 

He\marginnote{1.32.36} has eyelashes like a cow’s.\footnote{Cows have long and elegant eyelashes. } 

Between\marginnote{1.32.37} his eyebrows there grows a tuft, soft and white like cotton-wool. 

The\marginnote{1.32.38} crown of his head is like a turban.\footnote{The \textit{\textsanskrit{uṇhīsa}} is depicted as a bulge on the Buddha’s crown. } 

These\marginnote{1.33.1} are the thirty-two marks of a great man that the prince has. A great man who possesses these has only two possible destinies, no other. If he stays at home he becomes a king, a wheel-turning monarch. But if he goes forth from the lay life to homelessness, he becomes a perfected one, a fully awakened Buddha, who draws back the veil from the world.’ 

\section*{4. How He Came to be Known as \textsanskrit{Vipassī} }

Then\marginnote{1.33.9} King \textsanskrit{Bandhumā} had the brahmin soothsayers dressed in unworn clothes and satisfied all their needs.\footnote{This is the \textit{\textsanskrit{dakkhiṇā}}, the religious offering given in gratitude and respect for the services. } Then the king appointed nursemaids for Prince \textsanskrit{Vipassī}.\footnote{His birth mother has passed away and there is no mention of a step-mother. } Some suckled him, some bathed him, some held him, and some carried him on their hip. From when he was born, a white parasol was held over him night and day, with the thought, ‘Don’t let cold, heat, grass, dust, or damp bother him.’ He was dear and beloved by many people, like a blue water lily, or a pink or white lotus. He was always passed from hip to hip. 

From\marginnote{1.35.1} when he was born, his voice was charming, graceful, sweet, and lovely. It was as sweet as the song of a cuckoo-bird found in the Himalayas. 

From\marginnote{1.36.1} when he was born, Prince \textsanskrit{Vipassī} had the power of clairvoyance which manifested as a result of past deeds, by which he could see for a league all around both by day and by night.\footnote{Normally clairvoyance and related abilities are said to arise due to the power of the fourth \textit{\textsanskrit{jhāna}}, whereas here it comes naturally due to past kamma. After \textit{\textsanskrit{jhāna}} this ability is empowered by the radiant mind clear of hindrances, whereas here it seems to be a more limited ability to see clearly and in the dark. } 

And\marginnote{1.37.1} he was unblinkingly watchful, like the gods of the thirty-three. And because it was said that he was unblinkingly watchful, he came to be known as ‘\textsanskrit{Vipassī}’.\footnote{\textsanskrit{Vipassī}’s name is simply the personal form of the word made famous in Buddhist meditation, \textit{\textsanskrit{vipassanā}}. This is usually rendered as “insight”, but the sense here is more like “clear seeing”. } 

Then\marginnote{1.37.3} while King \textsanskrit{Bandhumā} was sitting in judgment, he’d sit Prince \textsanskrit{Vipassī} in his lap and explain the case to him. And sitting there in his father’s lap, \textsanskrit{Vipassī} would thoroughly consider the case and draw a conclusion using a logical procedure. So this was all the more reason for him to be known as ‘\textsanskrit{Vipassī}’.\footnote{Indian epistemology acknowledges two fundamental sources of knowledge: direct perception (\textit{paccakkha}) and inference (\textit{\textsanskrit{anumāna}}). This passage shows that \textit{\textsanskrit{vipassanā}} includes both. I render \textit{\textsanskrit{vipassanā}} with “discernment” in an attempt to capture both nuances, rather than the standard “insight”. } 

Then\marginnote{1.38.1} King \textsanskrit{Bandhumā} had three stilt longhouses built for him—one for the winter, one for the summer, and one for the rainy season, and provided him with the five kinds of sensual stimulation. Prince \textsanskrit{Vipassī} stayed in a stilt longhouse without coming downstairs for the four months of the rainy season, where he was entertained by musicians—none of them men. 

\scendsection{The first recitation section. }

\section*{5. The Old Man }

Then,\marginnote{2.1.1} after many years, many hundred years, many thousand years had passed, Prince \textsanskrit{Vipassī} addressed his charioteer,\footnote{Here begins the story of the four signs that led to \textsanskrit{Vipassī}’s going forth—an old man, a sick man, a dead man, and a renunciate. Later texts apply the same story to Siddhattha, indeed to all bodhisattvas. | These four “signs” (\textit{nimitta}) are also called “messengers of the gods”, reckoned as five (\href{https://suttacentral.net/mn130/en/sujato}{MN 130}) or three (\href{https://suttacentral.net/an3.36/en/sujato}{AN 3.36}). } ‘My dear charioteer, harness the finest chariots. We will go to a park and see the scenery.’ 

‘Yes,\marginnote{2.1.3} sir,’ replied the charioteer. He harnessed the chariots and informed the prince, ‘Sire, the finest chariots are harnessed. Please go at your convenience.’ Then Prince \textsanskrit{Vipassī} mounted a fine carriage and, along with other fine carriages, set out for the park. 

Along\marginnote{2.2.1} the way he saw an elderly man, bent double, crooked, leaning on a staff, trembling as he walked, ailing, past his prime. He addressed his charioteer, ‘My dear charioteer, what has that man done? For his hair and his body are unlike those of other men.’ 

‘That,\marginnote{2.2.5} Your Majesty, is called an old man.’ 

‘But\marginnote{2.2.6} why is he called an old man?’ 

‘He’s\marginnote{2.2.7} called an old man because now he has not long to live.’ 

‘But\marginnote{2.2.8} my dear charioteer, am I liable to grow old? Am I not exempt from old age?’ 

‘Everyone\marginnote{2.2.9} is liable to grow old, Your Majesty, including you. No-one is exempt from old age.’ 

‘Well\marginnote{2.2.10} then, my dear charioteer, that’s enough of the park for today. Let’s return to the royal compound.’\footnote{\textit{\textsanskrit{Antepuraṁ}} (“royal compound”) was the inner sanctum of the royal residence. Maximally it referred to the area enclosed by walls within which the ruling families and staff lived. It later became used in the more restricted sense of “harem”. } 

‘Yes,\marginnote{2.2.11} Your Majesty,’ replied the charioteer and did so. 

Back\marginnote{2.2.12} at the royal compound, the prince brooded, miserable and sad:\footnote{“Brood” is \textit{\textsanskrit{pajjhāyati}}. He is having an existential crisis. } ‘Curse this thing called rebirth, since old age will come to anyone who’s born.’ 

Then\marginnote{2.3.1} King \textsanskrit{Bandhumā} summoned the charioteer and said, ‘My dear charioteer, I hope the prince enjoyed himself at the park? I hope he was happy there?’ 

‘No,\marginnote{2.3.3} Your Majesty, the prince didn’t enjoy himself at the park.’ 

‘But\marginnote{2.3.4} what did he see on the way to the park?’ And the charioteer told the king about seeing the old man and the prince’s reaction. 

\section*{6. The Sick Man }

Then\marginnote{2.4.1} King \textsanskrit{Bandhumā} thought, ‘Prince \textsanskrit{Vipassī} must not renounce the throne. He must not go forth from the lay life to homelessness. And the words of the brahmin soothsayers must not come true.’\footnote{In any story of prophecy, efforts are made to avert it and they invariably fail. This is a recursive property of prophetic myth. If the prophecy were averted, the myth would not exist and we would not know of it; but because the myth does exist, we know how it ends and the prophecy must come true. } To this end he provided the prince with even more of the five kinds of sensual stimulation, with which the prince amused himself. 

Then,\marginnote{2.5.1} after many thousand years had passed, Prince \textsanskrit{Vipassī} had his charioteer drive him to the park once more.\footnote{An existential crisis takes its own time; it cannot be rushed. } 

Along\marginnote{2.6.1} the way he saw a man who was sick, suffering, gravely ill, collapsed in his own urine and feces, being picked up by some and put down by others. He addressed his charioteer, ‘My dear charioteer, what has that man done? For his eyes and his voice are unlike those of other men.’ 

‘That,\marginnote{2.6.5} Your Majesty, is called a sick man.’ 

‘But\marginnote{2.6.6} why is he called a sick man?’ 

‘He’s\marginnote{2.6.7} called a sick man; hopefully he will recover from that illness.’ 

‘But\marginnote{2.6.8} my dear charioteer, am I liable to fall sick? Am I not exempt from sickness?’ 

‘Everyone\marginnote{2.6.9} is liable to fall sick, Your Majesty, including you. No-one is exempt from sickness.’ 

‘Well\marginnote{2.6.10} then, my dear charioteer, that’s enough of the park for today. Let’s return to the royal compound.’ 

‘Yes,\marginnote{2.6.11} Your Majesty,’ replied the charioteer and did so. 

Back\marginnote{2.6.12} at the royal compound, the prince brooded, miserable and sad: ‘Curse this thing called rebirth, since old age and sickness will come to anyone who’s born.’ 

Then\marginnote{2.7.1} King \textsanskrit{Bandhumā} summoned the charioteer and said, ‘My dear charioteer, I hope the prince enjoyed himself at the park? I hope he was happy there?’ 

‘No,\marginnote{2.7.3} Your Majesty, the prince didn’t enjoy himself at the park.’ 

‘But\marginnote{2.7.4} what did he see on the way to the park?’ And the charioteer told the king about seeing the sick man and the prince’s reaction. 

\section*{7. The Dead Man }

Then\marginnote{2.8.1} King \textsanskrit{Bandhumā} thought, ‘Prince \textsanskrit{Vipassī} must not renounce the throne. He must not go forth from the lay life to homelessness. And the words of the brahmin soothsayers must not come true.’ To this end he provided the prince with even more of the five kinds of sensual stimulation, with which the prince amused himself. 

Then,\marginnote{2.9.2} after many thousand years had passed, Prince \textsanskrit{Vipassī} had his charioteer drive him to the park once more. 

Along\marginnote{2.10.1} the way he saw a large crowd gathered making a bier out of garments of different colors.\footnote{Neither reading \textit{\textsanskrit{vilāta}} or \textit{\textsanskrit{milāta}} appears to occur elsewhere in this sense. The commentary says it is a bier (\textit{sivika}). } He addressed his charioteer, ‘My dear charioteer, why is that crowd making a bier?’ 

‘That,\marginnote{2.10.4} Your Majesty, is for someone who’s departed.’ 

‘Well\marginnote{2.10.5} then, drive the chariot up to the departed.’ 

‘Yes,\marginnote{2.10.6} Your Majesty,’ replied the charioteer, and did so. 

When\marginnote{2.10.7} the prince saw the corpse of the departed, he addressed the charioteer, ‘But why is he called departed?’ 

‘He’s\marginnote{2.10.9} called departed because now his mother and father, his relatives and kin shall see him no more, and he shall never again see them.’ 

‘But\marginnote{2.10.10} my dear charioteer, am I liable to die? Am I not exempt from death? Will the king and queen and my other relatives and kin see me no more? And shall I never again see them?’ 

‘Everyone\marginnote{2.10.13} is liable to die, Your Majesty, including you. No-one is exempt from death. The king and queen and your other relatives and kin shall see you no more, and you shall never again see them.’ 

‘Well\marginnote{2.10.16} then, my dear charioteer, that’s enough of the park for today. Let’s return to the royal compound.’ 

‘Yes,\marginnote{2.10.17} Your Majesty,’ replied the charioteer and did so. 

Back\marginnote{2.10.18} at the royal compound, the prince brooded, miserable and sad: ‘Curse this thing called rebirth, since old age, sickness, and death will come to anyone who’s born.’ 

Then\marginnote{2.11.1} King \textsanskrit{Bandhumā} summoned the charioteer and said, ‘My dear charioteer, I hope the prince enjoyed himself at the park? I hope he was happy there?’ 

‘No,\marginnote{2.11.3} Your Majesty, the prince didn’t enjoy himself at the park.’ 

‘But\marginnote{2.11.4} what did he see on the way to the park?’ And the charioteer told the king about seeing the dead man and the prince’s reaction. 

\section*{8. The Renunciate }

Then\marginnote{2.12.1} King \textsanskrit{Bandhumā} thought, ‘Prince \textsanskrit{Vipassī} must not renounce the throne. He must not go forth from the lay life to homelessness. And the words of the brahmin soothsayers must not come true.’ To this end he provided the prince with even more of the five kinds of sensual stimulation, with which the prince amused himself. 

Then,\marginnote{2.13.2} after many thousand years had passed, Prince \textsanskrit{Vipassī} had his charioteer drive him to the park once more. 

Along\marginnote{2.14.1} the way he saw a man, a renunciate with shaven head, wearing an ocher robe.\footnote{“Renunciate” is \textit{pabbajita} (“one who has gone forth”), one of the many words for religious ascetics. It is a general term and does not specify his affiliation. } He addressed his charioteer, ‘My dear charioteer, what has that man done? For his head and his clothes are unlike those of other men.’ 

‘That,\marginnote{2.14.5} Your Majesty, is called a renunciate.’ 

‘But\marginnote{2.14.6} why is he called a renunciate?’ 

‘He\marginnote{2.14.7} is called a renunciate because he celebrates principled and fair conduct, skillful actions, good deeds, harmlessness, and sympathy for living creatures.’\footnote{“Celebrate” is \textit{\textsanskrit{sādhu}}, the famous Buddhist words of approval and rejoicing still heard every day in \textsanskrit{Theravāda} Buddhist cultures. It later acquired the meaning “renunciate” but does not have that sense in early Pali. The virtues that he celebrates are common values of Indian religions. } 

‘Then\marginnote{2.14.8} I celebrate the one called a renunciate, who celebrates principled and fair conduct, skillful actions, good deeds, harmlessness, and sympathy for living creatures! Well then, drive the chariot up to that renunciate.’ 

‘Yes,\marginnote{2.14.10} Your Majesty,’ replied the charioteer, and did so. 

Then\marginnote{2.14.11} Prince \textsanskrit{Vipassī} said to that renunciate, ‘My good man, what have you done? For your head and your clothes are unlike those of other men.’ 

‘Sire,\marginnote{2.14.14} I am what is called a renunciate.’ 

‘But\marginnote{2.14.15} why are you called a renunciate?’ 

‘I\marginnote{2.14.16} am called a renunciate because I celebrate principled and fair conduct, skillful actions, good deeds, harmlessness, and sympathy for living creatures.’ 

‘Then\marginnote{2.14.17} I celebrate the one called a renunciate, who celebrates principled and fair conduct, skillful actions, good deeds, harmlessness, and sympathy for living creatures!’ 

\section*{9. The Going Forth }

Then\marginnote{2.15.1} the prince addressed the charioteer, ‘Well then, my dear charioteer, take the chariot and return to the royal compound. I shall shave off my hair and beard right here, dress in ocher robes, and go forth from the lay life to homelessness.’\footnote{His apparently sudden decision to go forth arises only after an extensive period of crisis and contemplation. } 

‘Yes,\marginnote{2.15.4} Your Majesty,’ replied the charioteer and did so. 

Then\marginnote{2.15.5} Prince \textsanskrit{Vipassī} shaved off his hair and beard, dressed in ocher robes, and went forth from the lay life to homelessness. 

\section*{10. A Great Crowd Goes Forth }

A\marginnote{2.16.1} large crowd of 84,000 people in the capital of \textsanskrit{Bandhumatī} heard that \textsanskrit{Vipassī} had gone forth. It occurred to them, ‘This must be no ordinary teaching and training, no ordinary going forth in which Prince \textsanskrit{Vipassī} has gone forth. If even the prince goes forth, why don’t we do the same?’ 

Then\marginnote{2.16.6} that great crowd of 84,000 people shaved off their hair and beard, dressed in ocher robes, and followed the one intent on awakening, \textsanskrit{Vipassī}, by going forth from the lay life to homelessness.\footnote{The idea that a whole populace would follow the bodhisatta on his renunciate path occurs several times in the \textsanskrit{Jātakas}. } Escorted by that assembly, \textsanskrit{Vipassī} wandered on tour among the villages, towns, and capital cities.\footnote{Following the PTS edition in omitting \textit{janapada}, which is absent from the commentary and the parallel passage at \href{https://suttacentral.net/dn19/en/sujato\#58.3}{DN 19:58.3}. } 

Then\marginnote{2.17.1} as he was in private retreat this thought came to his mind, ‘It’s not appropriate for me to live in a crowd. Why don’t I live alone, withdrawn from the group?’ After some time he withdrew from the group to live alone. The 84,000 went one way, but \textsanskrit{Vipassī} went another. 

\section*{11. \textsanskrit{Vipassī}’s Reflections }

Then\marginnote{2.18.1} as \textsanskrit{Vipassī}, the one intent on awakening, was in private retreat in his dwelling, this thought came to his mind,\footnote{This is the only place \textit{\textsanskrit{vāsūpagata}} (“entered his dwelling”) is added to this stock phrase. } ‘Alas, this world has fallen into trouble. It’s born, grows old, dies, passes away, and is reborn, yet it doesn’t understand how to escape from this suffering, from old age and death.\footnote{At \href{https://suttacentral.net/sn12.4/en/sujato}{SN 12.4}–10 this same reflection is attributed to each of the seven past Buddhas, kicking off an investigation into dependent origination in reverse order, starting with the outcome: suffering. Here this is treated as a meditative contemplation, whereas the next sutta, \href{https://suttacentral.net/dn15/en/sujato}{DN 15} \textsanskrit{Mahānidānasutta}, delves into the philosophical implications. } Oh, when will an escape be found from this suffering, from old age and death?’ 

Then\marginnote{2.18.4} \textsanskrit{Vipassī} thought, ‘When what exists is there old age and death? What is a condition for old age and death?’\footnote{The reflection shows how the bodhisatta is still digging into the trauma of discovering the reality of old age and death. } Then, through rational application of mind, \textsanskrit{Vipassī} comprehended with wisdom,\footnote{\textit{Yoniso \textsanskrit{maniskāra}} (“rational application of mind”) is a distinctively Buddhist term that literally means “applying the mind by way of source”. It is mostly used in investigating causality, although over time it came to have a more general sense of “reflection, inquiry, attention”. } ‘When rebirth exists there’s old age and death. Rebirth is a condition for old age and death.’\footnote{Here begins the sequence of dependent origination. I give basic definitions of terms here, and more details in \href{https://suttacentral.net/dn15/en/sujato}{DN 15}. | Rebirth is a necessary antecedent condition for old age and death. Note that it is not a \emph{sufficient} condition for old age, for many die when young. } 

Then\marginnote{2.18.8} \textsanskrit{Vipassī} thought, ‘When what exists is there rebirth? What is a condition for rebirth?’\footnote{Since there cannot be an end to the “birth” that starts this life, \textit{\textsanskrit{jāti}} means “rebirth”, the next stage in the ongoing cycle. } Then, through rational application of mind, \textsanskrit{Vipassī} comprehended with wisdom, ‘When continued existence exists there’s rebirth. Continued existence is a condition for rebirth.’\footnote{\textit{Bhava} means “being, existence, life” in the sense of “past and future lives”. It refers to the ongoing process of continued existence, transmigrating through life after life. By itself, \textit{bhava} has a positive connotation, and represents the longing that many people have to continue to exist after death in a permanent and happy state. The Buddha, however, situates it as just one more dimension of the flow of conditions. } 

Then\marginnote{2.18.12} \textsanskrit{Vipassī} thought, ‘When what exists is there continued existence? What is a condition for continued existence?’ Then, through rational application of mind, \textsanskrit{Vipassī} comprehended with wisdom, ‘When grasping exists there’s continued existence. Grasping is a condition for continued existence.’\footnote{“Grasping” (\textit{\textsanskrit{upādāna}}) at sensual pleasures, views, precepts and observances, and theories of a self (\href{https://suttacentral.net/dn15/en/sujato\#6.3}{DN 15:6.3}). Grasping has the active sense of “taking up” a new life, not just “clinging” to what one has. It has a dual sense, because it also means the “fuel” that sustains the fire of existence. } 

Then\marginnote{2.18.16} \textsanskrit{Vipassī} thought, ‘When what exists is there grasping? What is a condition for grasping?’ Then, through rational application of mind, \textsanskrit{Vipassī} comprehended with wisdom, ‘When craving exists there’s grasping. Craving is a condition for grasping.’\footnote{Craving (\textit{\textsanskrit{taṇhā}}, literally “thirst”) and grasping have a similar meaning, but craving is primal desire while grasping is more complex, involving doctrines and behaviors. } 

Then\marginnote{2.18.20} \textsanskrit{Vipassī} thought, ‘When what exists is there craving? What is a condition for craving?’ Then, through rational application of mind, \textsanskrit{Vipassī} comprehended with wisdom, ‘When feeling exists there’s craving. Feeling is a condition for craving.’\footnote{“Feeling” (\textit{\textsanskrit{vedanā}}) is more fundamental than the complex concept of “emotion”. It refers to the hedonic tone of experience as pleasant, painful, or neutral. } 

Then\marginnote{2.18.24} \textsanskrit{Vipassī} thought, ‘When what exists is there feeling? What is a condition for feeling?’ Then, through rational application of mind, \textsanskrit{Vipassī} comprehended with wisdom, ‘When contact exists there’s feeling. Contact is a condition for feeling.’\footnote{“Contact” is literally “touch” (\textit{phassa}), and refers to the stimulation that occurs when sense object meets sense organ in experience. } 

Then\marginnote{2.18.28} \textsanskrit{Vipassī} thought, ‘When what exists is there contact? What is a condition for contact?’ Then, through rational application of mind, \textsanskrit{Vipassī} comprehended with wisdom, ‘When the six sense fields exist there’s contact. The six sense fields are a condition for contact.’\footnote{The five senses with the mind as sixth. This topic is treated extensively throughout early Buddhism, with a special focus on understanding and restraining the pull of sense stimulation. \textit{Āyatana} has a root sense “stretch”, from which derived senses include “dimension”, “field”, etc. } 

Then\marginnote{2.18.32} \textsanskrit{Vipassī} thought, ‘When what exists are there the six sense fields? What is a condition for the six sense fields?’ Then, through rational application of mind, \textsanskrit{Vipassī} comprehended with wisdom, ‘When name and form exist there are the six sense fields. Name and form are a condition for the six sense fields.’\footnote{“Name and form” (\textit{\textsanskrit{nāmarūpa}}) is a Vedic concept referring to the multiplicity of material forms (\textit{\textsanskrit{rūpa}}) and associated names (\textit{\textsanskrit{nāma}}), especially the individual “sentient organisms” such as gods and humans (Rig Veda 5.43.10, \textsanskrit{Bṛhadāraṇyaka} \textsanskrit{Upaniṣad} 1.6.1), which are ultimately absorbed into the divine, like rivers in the ocean (\textsanskrit{Muṇḍaka} \textsanskrit{Upaniṣad} 3.2.8, \textsanskrit{Praśna} \textsanskrit{Upaniṣad} 6.5). The Buddha treated “name” analytically as feeling, perception, intention, contact, and application of mind, and “form” as the four elements and derived matter (\href{https://suttacentral.net/dn15/en/sujato\#20.8}{DN 15:20.8}, \href{https://suttacentral.net/mn9/en/sujato\#52-54.7}{MN 9:52–54.7}, and \href{https://suttacentral.net/sn12.2/en/sujato\#11.1}{SN 12.2:11.1}). } 

Then\marginnote{2.18.36} \textsanskrit{Vipassī} thought, ‘When what exists are there name and form? What is a condition for name and form?’ Then, through rational application of mind, \textsanskrit{Vipassī} comprehended with wisdom, ‘When consciousness exists there are name and form. Consciousness is a condition for name and form.’\footnote{“Consciousness” (\textit{\textsanskrit{viññāṇa}}) is simple subjective awareness, the sense of knowing. It arises stimulated by either an external sense impression or a mental phenomena such as thought, memory, etc. It is the subjective awareness that makes the entire multiform world of concepts and appearances possible. Thus far the analysis agrees with \textsanskrit{Yājñavalkya}, who says that the manifold appearances in the world arise from consciousness (\textit{etebhyo \textsanskrit{bhūtebhyaḥ} \textsanskrit{samutthāya}}, \textsanskrit{Bṛhadāraṇyaka} \textsanskrit{Upaniṣad} 2.4.12). } 

Then\marginnote{2.18.40} \textsanskrit{Vipassī} thought, ‘When what exists is there consciousness? What is a condition for consciousness?’ Then, through rational application of mind, \textsanskrit{Vipassī} comprehended with wisdom, ‘When name and form exist there’s consciousness. Name and form are a condition for consciousness.’\footnote{Here the Buddha decisively departs from \textsanskrit{Yājñavalkya}’s view that individuated awareness (\textit{\textsanskrit{saññā}}) returns into “this great reality, infinite, unbounded, a sheer mass of consciousness” (\textit{\textsanskrit{idaṁ} \textsanskrit{mahadbhūtam} anantam \textsanskrit{apāraṁ} \textsanskrit{vijñānaghana} eva}, \textsanskrit{Bṛhadāraṇyaka} \textsanskrit{Upaniṣad} 2.4.12). Consciousness (\textit{\textsanskrit{viññāṇa}}) is not a fundamental reality (\textit{\textsanskrit{mahadbhūta}}) underlying multiplicity, but rather a conditioned process that exists only together with name and form. } 

Then\marginnote{2.19.1} \textsanskrit{Vipassī} thought, ‘This consciousness turns back from name and form, and doesn’t go beyond that.’\footnote{Dependent origination normally continues with two further factors: choices and ignorance. The full series does not appear in the \textsanskrit{Dīghanikāya}. This truncated series emphasizes the mutuality of name and form with consciousness, but does not preclude the normal linear series. Each presentation of dependent origination reveals a different aspect of a complex, ramified process. } It is to this extent that one may be reborn, grow old, die, pass away, or reappear. That is: Name and form are conditions for consciousness. Consciousness is a condition for name and form. Name and form are conditions for the six sense fields. The six sense fields are conditions for contact. Contact is a condition for feeling. Feeling is a condition for craving. Craving is a condition for grasping. Grasping is a condition for continued existence. Continued existence is a condition for rebirth. Rebirth is a condition for old age and death, sorrow, lamentation, pain, sadness, and distress to come to be.\footnote{Note the use of repetition to sum up the main doctrinal teachings. This serves to lock the sequence in memory and ensure no items are missing or displaced, while for one who is reciting the text from memory it provides an opportunity to reflect and apply the meaning in their own experience. } That is how this entire mass of suffering originates.’ 

‘Origination,\marginnote{2.19.6} origination.’ Such was the vision, knowledge, wisdom, realization, and light that arose in \textsanskrit{Vipassī}, the one intent on awakening, regarding teachings not learned before from another.\footnote{This phrasing recalls the Buddha’s first sermon (\href{https://suttacentral.net/sn56.11/en/sujato\#5.1}{SN 56.11:5.1}), an insight that is said to be common to all Buddhas (\href{https://suttacentral.net/sn56.12/en/sujato\#1.1}{SN 56.12:1.1}). \textit{Pubbe ananussutesu} (“not learned before from another”) means that this is a fresh insight not passed down in an oral tradition. } 

Then\marginnote{2.20.1} \textsanskrit{Vipassī} thought, ‘When what doesn’t exist is there no old age and death? When what ceases do old age and death cease?’ Then, through rational application of mind, \textsanskrit{Vipassī} comprehended with wisdom, ‘When rebirth doesn’t exist there’s no old age and death. When rebirth ceases, old age and death cease.’ 

Then\marginnote{2.20.5} \textsanskrit{Vipassī} thought, ‘When what doesn’t exist is there no rebirth? When what ceases does rebirth cease?’ Then, through rational application of mind, \textsanskrit{Vipassī} comprehended with wisdom, ‘When continued existence doesn’t exist there’s no rebirth. When continued existence ceases, rebirth ceases.’ 

Then\marginnote{2.20.9} \textsanskrit{Vipassī} thought, ‘When what doesn’t exist is there no continued existence? When what ceases does continued existence cease?’ Then, through rational application of mind, \textsanskrit{Vipassī} comprehended with wisdom, ‘When grasping doesn’t exist there’s no continued existence. When grasping ceases, continued existence ceases.’ 

Then\marginnote{2.20.13} \textsanskrit{Vipassī} thought, ‘When what doesn’t exist is there no grasping? When what ceases does grasping cease?’ Then, through rational application of mind, \textsanskrit{Vipassī} comprehended with wisdom, ‘When craving doesn’t exist there’s no grasping. When craving ceases, grasping ceases.’ 

Then\marginnote{2.20.17} \textsanskrit{Vipassī} thought, ‘When what doesn’t exist is there no craving? When what ceases does craving cease?’ Then, through rational application of mind, \textsanskrit{Vipassī} comprehended with wisdom, ‘When feeling doesn’t exist there’s no craving. When feeling ceases, craving ceases.’ 

Then\marginnote{2.20.21} \textsanskrit{Vipassī} thought, ‘When what doesn’t exist is there no feeling? When what ceases does feeling cease?’ Then, through rational application of mind, \textsanskrit{Vipassī} comprehended with wisdom, ‘When contact doesn’t exist there’s no feeling. When contact ceases, feeling ceases.’ 

Then\marginnote{2.20.25} \textsanskrit{Vipassī} thought, ‘When what doesn’t exist is there no contact? When what ceases does contact cease?’ Then, through rational application of mind, \textsanskrit{Vipassī} comprehended with wisdom, ‘When the six sense fields don’t exist there’s no contact. When the six sense fields cease, contact ceases.’ 

Then\marginnote{2.20.29} \textsanskrit{Vipassī} thought, ‘When what doesn’t exist are there no six sense fields? When what ceases do the six sense fields cease?’ Then, through rational application of mind, \textsanskrit{Vipassī} comprehended with wisdom, ‘When name and form don’t exist there are no six sense fields. When name and form cease, the six sense fields cease.’ 

Then\marginnote{2.20.33} \textsanskrit{Vipassī} thought, ‘When what doesn’t exist are there no name and form? When what ceases do name and form cease?’ Then, through rational application of mind, \textsanskrit{Vipassī} comprehended with wisdom, ‘When consciousness doesn’t exist there are no name and form. When consciousness ceases, name and form cease.’ 

Then\marginnote{2.20.37} \textsanskrit{Vipassī} thought, ‘When what doesn’t exist is there no consciousness? When what ceases does consciousness cease?’ Then, through rational application of mind, \textsanskrit{Vipassī} comprehended with wisdom, ‘When name and form don’t exist there’s no consciousness. When name and form cease, consciousness ceases.’ 

Then\marginnote{2.21.1} \textsanskrit{Vipassī} thought, ‘I have discovered the path to awakening. That is:\footnote{The PTS edition has \textit{\textsanskrit{vipassanā}-maggo} here, despite admitting the term is found in no manuscripts and is taken from the commentary, where it is clearly an explanation not a reading (\textit{maggoti \textsanskrit{vipassanāmaggo}}). This error is followed by Rhys Davids and Walshe in their translations. } When name and form cease, consciousness ceases. When consciousness ceases, name and form cease. When name and form cease, the six sense fields cease. When the six sense fields cease, contact ceases. When contact ceases, feeling ceases. When feeling ceases, craving ceases. When craving ceases, grasping ceases. When grasping ceases, continued existence ceases. When continued existence ceases, rebirth ceases. When rebirth ceases, old age and death, sorrow, lamentation, pain, sadness, and distress cease. That is how this entire mass of suffering ceases.’ 

‘Cessation,\marginnote{2.21.5} cessation.’ Such was the vision, knowledge, wisdom, realization, and light that arose in \textsanskrit{Vipassī}, the one intent on awakening, regarding teachings not learned before from another.\footnote{Insight into dependent origination here indicates the attaining of stream entry. } 

After\marginnote{2.22.1} some time he meditated observing rise and fall in the five grasping aggregates.\footnote{The five grasping aggregates (\textit{\textsanskrit{pañcūpādānakkhandhā}}) are mentioned as a summary of suffering in the Buddha’s first sermon (\href{https://suttacentral.net/sn56.11/en/sujato\#4.2}{SN 56.11:4.2}). Most of the teachings on this topic are collected in the \textsanskrit{Khandhasaṁyutta} at SN 22, but they are also found in the \textsanskrit{Dīghanikāya} at \href{https://suttacentral.net/dn22/en/sujato\#14.1}{DN 22:14.1}, \href{https://suttacentral.net/dn33/en/sujato\#1.11.45}{DN 33:1.11.45}, \href{https://suttacentral.net/dn33/en/sujato\#2.1.4}{DN 33:2.1.4}, and \href{https://suttacentral.net/dn34/en/sujato\#1.6.16}{DN 34:1.6.16}. The contemplation on the aggregates dispels the mistaken assumption of a self. Many of the theorists in \href{https://suttacentral.net/dn1/en/sujato}{DN 1} identify the self with one or other of the aggregates. } ‘Such is form, such is the origin of form, such is the ending of form. Such is feeling, such is the origin of feeling, such is the ending of feeling. Such is perception, such is the origin of perception, such is the ending of perception. Such are choices, such is the origin of choices, such is the ending of choices.\footnote{\textit{\textsanskrit{Saṅkhāra}} in early Buddhism has three main doctrinal senses. (1) The broadest sense is “conditioned phenomena”, which we find in the \textsanskrit{Dīghanikāya} at \href{https://suttacentral.net/dn16/en/sujato\#6.10.10}{DN 16:6.10.10}, \href{https://suttacentral.net/dn17/en/sujato\#2.16.1}{DN 17:2.16.1}, and \href{https://suttacentral.net/dn34/en/sujato\#1.8.59}{DN 34:1.8.59}. This essentially means “everything except Nibbana”. (2) Sometimes it is a physical or mental “process” or “activity” as at \href{https://suttacentral.net/dn18/en/sujato\#24.1}{DN 18:24.1}, where it refers to the gradual stilling of energies in the development of meditation. (3) In the five aggregates and dependent origination it has the sense of “morally potent volitions or choices” and is a synonym for \textit{\textsanskrit{cetanā}} (“intention”). It is defined as good, bad, and imperturbable choices (\href{https://suttacentral.net/dn33/en/sujato\#1.10.77}{DN 33:1.10.77}), the latter of which refers to the kamma of the fourth \textit{\textsanskrit{jhāna}} and above. In this sense it is the moral “choices” for good or ill that propel consciousness into a new rebirth. } Such is consciousness, such is the origin of consciousness, such is the ending of consciousness.’\footnote{The radical thesis of the Buddha’s teaching is the idea that consciousness is merely another empirical phenomena that comes to an end, as is also emphasized in the concluding verses of \href{https://suttacentral.net/dn11/en/sujato}{DN 11}. } Meditating like this his mind was soon freed from defilements by not grasping.\footnote{This indicates the attainment of arahantship, the complete release from all attachments leading to rebirth. } 

\scendsection{The second recitation section. }

\section*{12. The Appeal of the Divinity }

Then\marginnote{3.1.1} the Blessed One \textsanskrit{Vipassī}, the perfected one, the fully awakened Buddha, thought,\footnote{No longer a bodhisatta, he is now called a Buddha for the first time. } ‘Why don’t I teach the Dhamma?’\footnote{In early Buddhism, the idea of teaching the Dhamma only arose after awakening. } 

Then\marginnote{3.1.3} he thought, ‘This principle I have discovered is deep, hard to see, hard to understand, peaceful, sublime, beyond the scope of logic, subtle, comprehensible to the astute.\footnote{A similar account is told of Gotama Buddha at \href{https://suttacentral.net/sn6.1/en/sujato\#1.4}{SN 6.1:1.4}, \href{https://suttacentral.net/mn26/en/sujato\#19.2}{MN 26:19.2}, and \href{https://suttacentral.net/mn85/en/sujato\#43.2}{MN 85:43.2}. } But people like clinging, they love it and enjoy it.\footnote{Here “clinging” is \textit{\textsanskrit{ālaya}}, from a root meaning “to stick”. } It’s hard for them to see this topic; that is, specific conditionality, dependent origination.\footnote{“Specific conditionality” (\textit{ \textsanskrit{idappaccayatā}}) refers to the fact that dependent origination traces the specific conditions for other specific things: this conditions that. It is not a general principle of universal conditionality (“everything is interconnected”). } It’s also hard for them to see this topic; that is, the stilling of all activities, the letting go of all attachments, the ending of craving, fading away, cessation, extinguishment.\footnote{“Stilling of all activities” (\textit{\textsanskrit{sabbasaṅkhārasamatho}}) is the cessation of all conditioned energies or phenomena. | “Attachments” here is \textit{upadhi}, the things of the world to which we cling and which bolster our complacency. } And if I were to teach the Dhamma, others might not understand me, which would be wearying and troublesome for me.’\footnote{The commentary is careful to specify that the Buddha means physical exhaustion only. } 

And\marginnote{3.2.1} then these verses, which were neither supernaturally inspired, nor learned before in the past, occurred to him:\footnote{“Not supernaturally inspired” (\textit{\textsanskrit{anacchariyā}}) rejects the Vedic “channeling” of scripture from the Divinity, while “not learned before in the past” (\textit{pubbe \textsanskrit{assutapubbā}}), echoing the Dhammacakkappavattanasutta (\href{https://suttacentral.net/sn56.11/en/sujato\#5.1}{SN 56.11:5.1}), rejects the oral tradition. } 

\begin{verse}%
‘I’ve\marginnote{3.2.2} struggled hard to realize this, \\
enough with trying to explain it! \\
Those mired in greed and hate \\
can’t really understand this teaching. 

It\marginnote{3.2.6} goes against the stream, subtle, \\
deep, obscure, and very fine. \\
Those besotted by greed cannot see, \\
for they’re shrouded in a mass of darkness.’\footnote{“Shrouded” is \textit{\textsanskrit{āvuṭā}}, which is from the same root as \textit{\textsanskrit{asaṁvutā}} in \href{https://suttacentral.net/dn14/en/sujato\#1.17.5}{DN 14:1.17.5} above, as well as \textit{\textsanskrit{nīvaraṇa}} (“hindrance”). All these terms ultimately stem from the Vedic serpent \textsanskrit{Vṛtra} (“the constrictor”) who wraps the world in darkness. } 

%
\end{verse}

So,\marginnote{3.2.10} as the Buddha \textsanskrit{Vipassī} reflected like this, his mind inclined to remaining passive, not to teaching the Dhamma.\footnote{Had he followed this inclination he would have been a \textit{paccekabuddha}, a Buddha “awakened for himself”. } 

Then\marginnote{3.2.11} a certain Great Divinity, knowing the Buddha \textsanskrit{Vipassī}’s train of thought, thought,\footnote{In the accounts of Gotama’s life this is specified as \textsanskrit{Brahmā} Sahampati. This whole passage is a moment of high cosmic solemnity and drama. } ‘Alas! The world will be lost, the world will perish! For the mind of the Blessed One \textsanskrit{Vipassī}, the perfected one, the fully awakened Buddha, inclines to remaining passive, not to teaching the Dhamma.’ Then, as easily as a strong person would extend or contract their arm, he vanished from the realm of divinity and reappeared in front of the Buddha \textsanskrit{Vipassī}. He arranged his robe over one shoulder, knelt on his right knee, raised his joined palms toward the Buddha \textsanskrit{Vipassī}, and said, ‘Sir, let the Blessed One teach the Dhamma! Let the Holy One teach the Dhamma!\footnote{The Buddha teaches on the invitation of the highest divinity. This sets a precedent for Buddhists to refrain from proselytizing, but rather teach by invitation. These passages are still recited in some places to invite a teaching. } There are beings with little dust in their eyes. They’re in decline because they haven’t heard the teaching. There will be those who understand the teaching!’\footnote{\textit{\textsanskrit{Aññātāro}} is an agent noun in plural, literally “understanders”. } 

When\marginnote{3.4.1} he said this, the Buddha \textsanskrit{Vipassī} said to him, ‘I too thought this, Divinity, “Why don’t I teach the Dhamma?” Then it occurred to me, “If I were to teach the Dhamma, others might not understand me, which would be wearying and troublesome for me.” 

So,\marginnote{3.4.19} as I reflected like this, my mind inclined to remaining passive, not to teaching the Dhamma.’ 

For\marginnote{3.5.1} a second time, and a third time that Great Divinity begged the Buddha to teach. 

Then,\marginnote{3.6.1} understanding the Divinity’s invitation, the Buddha \textsanskrit{Vipassī} surveyed the world with the eye of a Buddha, out of his compassion for sentient beings.\footnote{Previously he simply reflected to himself, now he uses his psychic abilities to ascertain people’s spiritual potential. } And he saw sentient beings with little dust in their eyes, and some with much dust in their eyes; with keen faculties and with weak faculties, with good qualities and with bad qualities, easy to teach and hard to teach. And some of them lived seeing the danger in the fault to do with the next world, while others did not.\footnote{\textit{Indriya} (“faculty”) is an abstract noun from \textit{indra}, the name of the potent Vedic god of war. In the Vedas, Indra manifests his \textit{indriya} by drinking soma. The drug enables him to release his full potential and power; originally this probably referred to drinking an amphetamine-like substance before battle. Here we see that it means something like “spiritual potential” which is unleashed by the practice of the eightfold path. } It’s like a pool with blue water lilies, or pink or white lotuses. Some of them sprout and grow in the water without rising above it, thriving underwater. Some of them sprout and grow in the water reaching the water’s surface. And some of them sprout and grow in the water but rise up above the water and stand with no water clinging to them. 

In\marginnote{3.6.4} the same way, the Buddha \textsanskrit{Vipassī} saw sentient beings with little dust in their eyes, and some with much dust in their eyes. 

Then\marginnote{3.7.1} that Great Divinity, knowing the Buddha \textsanskrit{Vipassī}’s train of thought, addressed him in verse: 

\begin{verse}%
‘Standing\marginnote{3.7.2} high on a rocky mountain, \\
you can see the people all around. \\
In just the same way, All-seer, so intelligent, \\
having ascended the Temple of Truth, \\
rid of sorrow, look upon the people \\
swamped with sorrow, \\>oppressed by rebirth and old age. 

Rise,\marginnote{3.7.8} hero! Victor in battle, leader of the caravan, \\
wander the world free of debt. \\
Let the Blessed One teach the Dhamma! \\
There will be those who understand!’ 

%
\end{verse}

Then\marginnote{3.7.12} the Buddha \textsanskrit{Vipassī} addressed that Great Divinity in verse: 

\begin{verse}%
‘Flung\marginnote{3.7.13} open are the doors to freedom from death! \\
Let those with ears to hear commit to faith.\footnote{\textit{\textsanskrit{Pamuñcantu} \textsanskrit{saddhaṁ}} has long troubled translators, as the basic sense of \textit{\textsanskrit{pamuñcantu}} is “release”. The problem is a long-standing one, for Sanskrit variants include \textit{pramodantu} (“celebrate”) or \textit{\textsanskrit{praṇudantu} \textsanskrit{kāṅkṣāḥ}} (“dispel doubts”). I think it is a poetic variant of \textit{\textsanskrit{adhimuñcantu}}, to “decide” or “commit” to faith. Pali commonly uses a synonymous verb to reinforce the sense of the noun. In \href{https://suttacentral.net/snp5.19/en/sujato}{Snp 5.19}, \textit{muttasaddho}, \textit{\textsanskrit{pamuñcassu} \textsanskrit{saddhaṁ}}, and \textit{\textsanskrit{adhimuttacittaṁ}} are all used in this sense. } \\
Thinking it would be troublesome, Divinity, \\>I did not teach \\
the sophisticated, sublime Dhamma among humans.’ 

%
\end{verse}

Then\marginnote{3.7.17} the Great Divinity, knowing that his request for the Buddha \textsanskrit{Vipassī} to teach the Dhamma had been granted, bowed and respectfully circled him, keeping him on his right, before vanishing right there. 

\section*{13. The Chief Disciples }

Then\marginnote{3.8.1} the Blessed One \textsanskrit{Vipassī}, the perfected one, the fully awakened Buddha, thought, ‘Who should I teach first of all? Who will quickly understand this teaching?’ Then he thought, ‘That \textsanskrit{Khaṇḍa}, the king’s son, and Tissa, the high priest’s son, are astute, competent, clever, and have long had little dust in their eyes.\footnote{In the account of Gotama, he first thinks to teach his former colleagues under whom he practiced the formless attainments. Here we see the start of a tendency in the legends of past Buddhas to erase the education among other spiritual teachers before awakening. Note that these two, who will become \textsanskrit{Vipassī}’s chief disciples, are leading \textit{khattiya} and brahmin sons of the royal household. \textsanskrit{Khaṇḍa} was \textsanskrit{Vipassī}’s brother, and the priest’s son was virtually family. } Why don’t I teach them first of all? They will quickly understand this teaching.’ 

Then,\marginnote{3.9.1} as easily as a strong person would extend or contract their arm, he vanished from the tree of awakening and reappeared near the capital city of \textsanskrit{Bandhumatī}, in the deer park named Sanctuary.\footnote{\textit{Khema} (“sanctuary”) originally meant “oasis”. It is common name for lakes and nature parks in the \textsanskrit{Jātakas}. } 

Then\marginnote{3.9.2} the Buddha \textsanskrit{Vipassī} addressed the park keeper,\footnote{There is a clear distinction between such managed “parks” and wilderness regions (\textit{\textsanskrit{arañña}}). } ‘My dear park keeper, please enter the city and say this to the king’s son \textsanskrit{Khaṇḍa} and the high priest’s son Tissa: “Sirs, the Blessed One \textsanskrit{Vipassī}, the perfected one, the fully awakened Buddha, has arrived at \textsanskrit{Bandhumatī} and is staying in the deer park named Sanctuary. He wishes to see you.”’ 

‘Yes,\marginnote{3.9.5} sir,’ replied the park keeper, and did as he was asked. 

Then\marginnote{3.10.1} the king’s son \textsanskrit{Khaṇḍa} and the high priest’s son Tissa had the finest carriages harnessed. Then they mounted a fine carriage and, along with other fine carriages, set out from \textsanskrit{Bandhumatī} for the Sanctuary. They went by carriage as far as the terrain allowed, then descended and approached the Buddha \textsanskrit{Vipassī} on foot. They bowed and sat down to one side. 

The\marginnote{3.11.1} Buddha \textsanskrit{Vipassī} taught them step by step, with a talk on giving, ethical conduct, and heaven. He explained the drawbacks of sensual pleasures, so sordid and corrupt, and the benefit of renunciation. And when he knew that their minds were ready, pliable, rid of hindrances, elated, and confident he explained the special teaching of the Buddhas: suffering, its origin, its cessation, and the path. Just as a clean cloth rid of stains would properly absorb dye, in that very seat the stainless, immaculate vision of the Dhamma arose in the king’s son \textsanskrit{Khaṇḍa} and the high priest’s son Tissa: ‘Everything that has a beginning has an end.’ 

They\marginnote{3.12.1} saw, attained, understood, and fathomed the Dhamma. They went beyond doubt, got rid of indecision, and became self-assured and independent of others regarding the Teacher’s instructions. They said to the Buddha \textsanskrit{Vipassī}, ‘Excellent, sir! Excellent! As if he were righting the overturned, or revealing the hidden, or pointing out the path to the lost, or lighting a lamp in the dark so people with clear eyes can see what’s there, the Buddha has made the teaching clear in many ways. We go for refuge to the Blessed One and to the teaching. Sir, may we receive the going forth and ordination in the Buddha’s presence?’ 

And\marginnote{3.13.1} they received the going forth, the ordination in the Buddha \textsanskrit{Vipassī}’s presence.\footnote{In the early period, there was no distinction between “going forth” (\textit{\textsanskrit{pabbajjā}}) and “ordination” (\textit{\textsanskrit{upasampadā}}). They refer to two sides of the same coin: leaving the home life and entering the ascetic life. Ordination was originally granted with the simple call, “Come mendicant!” (\textit{ehi bhikkhu}). } Then the Buddha \textsanskrit{Vipassī} educated, encouraged, fired up, and inspired them with a Dhamma talk. He explained the drawbacks of conditioned phenomena, so sordid and corrupt, and the benefit of extinguishment. Being taught like this their minds were soon freed from defilements by not grasping.\footnote{Thus they became arahants, realizing the same truth that \textsanskrit{Vipassī} had. } 

\section*{14. The Going Forth of the Large Crowd }

A\marginnote{3.14.1} large crowd of 84,000 people in the capital of \textsanskrit{Bandhumatī} heard that the Blessed One \textsanskrit{Vipassī}, the perfected one, the fully awakened Buddha, had arrived at \textsanskrit{Bandhumatī} and was staying in the deer park named Sanctuary. And they heard that the king’s son \textsanskrit{Khaṇḍa} and the high priest’s son Tissa had shaved off their hair and beard, dressed in ocher robes, and gone forth from the lay life to homelessness in the Buddha’s presence. It occurred to them, ‘This must be no ordinary teaching and training, no ordinary going forth in which the king’s son \textsanskrit{Khaṇḍa} and the high priest’s son Tissa have gone forth. If even they go forth, why don’t we do the same?’ Then those 84,000 people left \textsanskrit{Bandhumatī} for the deer park named Sanctuary, where they approached the Buddha \textsanskrit{Vipassī}, bowed and sat down to one side. 

The\marginnote{3.15.1} Buddha \textsanskrit{Vipassī} taught them step by step, with a talk on giving, ethical conduct, and heaven. He explained the drawbacks of sensual pleasures, so sordid and corrupt, and the benefit of renunciation. And when he knew that their minds were ready, pliable, rid of hindrances, elated, and confident he explained the special teaching of the Buddhas: suffering, its origin, its cessation, and the path. Just as a clean cloth rid of stains would properly absorb dye, in that very seat the stainless, immaculate vision of the Dhamma arose in those 84,000 people: ‘Everything that has a beginning has an end.’ 

They\marginnote{3.16.1} saw, attained, understood, and fathomed the Dhamma. They went beyond doubt, got rid of indecision, and became self-assured and independent of others regarding the Teacher’s instructions. They said to the Buddha \textsanskrit{Vipassī}, ‘Excellent, sir! Excellent!’ And just like \textsanskrit{Khaṇḍa} and Tissa they asked for and received ordination. Then the Buddha taught them further. 

Being\marginnote{3.17.1} taught like this their minds were soon freed from defilements by not grasping. 

\section*{15. The 84,000 Who Had Gone Forth Previously }

The\marginnote{3.18.1} 84,000 people who had gone forth previously also heard: ‘It seems the Blessed One \textsanskrit{Vipassī}, the perfected one, the fully awakened Buddha, has arrived at \textsanskrit{Bandhumatī} and is staying in the deer park named Sanctuary. And he is teaching the Dhamma!’ Then they too went to see the Buddha \textsanskrit{Vipassī}, realized the Dhamma, went forth, and became freed from defilements. 

\section*{16. The Allowance to Wander }

Now\marginnote{3.22.1} at that time a large \textsanskrit{Saṅgha} of 6,800,000 mendicants were residing at \textsanskrit{Bandhumatī}. As the Buddha \textsanskrit{Vipassī} was in private retreat this thought came to his mind, ‘The \textsanskrit{Saṅgha} residing at \textsanskrit{Bandhumatī} now is large. What if I was to urge them: 

“Wander\marginnote{3.22.4} forth, mendicants, for the welfare and happiness of the people, out of sympathy for the world, for the benefit, welfare, and happiness of gods and humans.\footnote{As at \href{https://suttacentral.net/pli-tv-kd1/en/sujato\#11.1.4}{Kd 1:11.1.4} and \href{https://suttacentral.net/sn4.5/en/sujato\#2.3}{SN 4.5:2.3}. One of the Buddha’s first acts is to empower his students. } Let not two go by one road. Teach the Dhamma that’s good in the beginning, good in the middle, and good in the end, meaningful and well-phrased. And reveal a spiritual practice that’s entirely full and pure. There are beings with little dust in their eyes. They’re in decline because they haven’t heard the teaching. There will be those who understand the teaching! But when six years have passed, you must all come to \textsanskrit{Bandhumatī} to recite the monastic code.”’\footnote{In the Vinaya of the current Buddha, the recitation occurs every fortnight on the \textit{uposatha} (“sabbath”), and it includes only the mendicants resident in a specific monastery. Here they gathered from all over India. } 

Then\marginnote{3.23.1} a certain Great Divinity, knowing the Buddha \textsanskrit{Vipassī}’s train of thought, as easily as a strong person would extend or contract their arm, vanished from the realm of divinity and reappeared in front of the Buddha \textsanskrit{Vipassī}. He arranged his robe over one shoulder, raised his joined palms toward the Buddha \textsanskrit{Vipassī}, and said, ‘That’s so true, Blessed One! That’s so true, Holy One! The \textsanskrit{Saṅgha} residing at \textsanskrit{Bandhumatī} now is large. Please urge them to wander, as you thought. And sir, I’ll make sure that when six years have passed the mendicants will return to \textsanskrit{Bandhumatī} to recite the monastic code.’ 

That’s\marginnote{3.23.10} what that Great Divinity said. Then he bowed and respectfully circled the Buddha \textsanskrit{Vipassī}, keeping him on his right side, before vanishing right there. 

Then\marginnote{3.24.1} in the late afternoon, the Buddha \textsanskrit{Vipassī} came out of retreat and addressed the mendicants, telling them all that had happened. Then he said, 

‘Wander\marginnote{3.26.1} forth, mendicants, for the welfare and happiness of the people, out of sympathy for the world, for the benefit, welfare, and happiness of gods and humans. Let not two go by one road. Teach the Dhamma that’s good in the beginning, good in the middle, and good in the end, meaningful and well-phrased. And reveal a spiritual practice that’s entirely full and pure. There are beings with little dust in their eyes. They’re in decline because they haven’t heard the teaching. There will be those who understand the teaching! But when six years have passed, you must all come to \textsanskrit{Bandhumatī} to recite the monastic code.’ 

Then\marginnote{3.26.7} most of the mendicants departed to wander the country that very day. 

Now\marginnote{3.27.1} at that time there were 84,000 monasteries in the Black Plum Tree Land.\footnote{“Black Plum Tree Land” is \textit{\textsanskrit{jambudīpa}}, the South Asian subcontinent, including India, Pakistan, Nepal, Bangladesh, and Afghanistan. The number 84,000 does not seem excessive, as the Chinese pilgrims recorded many thousands of monasteries throughout India. In modern Thailand there are around 40,000 monasteries. } And when the first year came to an end the deities raised the cry: ‘Good sirs, the first year has ended. Now five years remain. When five years have passed, you must all go to \textsanskrit{Bandhumatī} to recite the monastic code.’ 

And\marginnote{3.27.6} when the second year … the third year … the fourth year … the fifth year came to an end, the deities raised the cry: ‘Good sirs, the fifth year has ended. Now one year remains. When one year has passed, you must all go to \textsanskrit{Bandhumatī} to recite the monastic code.’ 

And\marginnote{3.27.13} when the sixth year came to an end the deities raised the cry: ‘Good sirs, the sixth year has ended. Now is the time that you must go to \textsanskrit{Bandhumatī} to recite the monastic code.’ Then that very day the mendicants went to \textsanskrit{Bandhumatī} to recite the monastic code. Some went by their own psychic power, and some by the psychic power of the deities. 

And\marginnote{3.28.1} there the Blessed One \textsanskrit{Vipassī}, the perfected one, the fully awakened Buddha, recited the monastic code thus:\footnote{The verses that follow form the climax of the whole discourse. They are known as the \textsanskrit{Ovāda} \textsanskrit{Pātimokkha}, the “monastic code for exhortation”. It seems that they were recited on the \textit{uposatha} in the days before the establishment of the list of rules that is also called \textit{\textsanskrit{pātimokkha}}. Most of the extant Vinayas still include these verses in the \textit{\textsanskrit{pātimokkha}} texts, the Pali being a notable exception. } 

\begin{verse}%
‘Patient\marginnote{3.28.2} acceptance is the ultimate fervor.\footnote{The Buddha redefines \textit{tapas} from painful mortification to gentle acceptance. } \\
Extinguishment is the ultimate, say the Buddhas.\footnote{These verses speak of “Buddhas” in plural. } \\
No true renunciate injures another, \\
nor does an ascetic hurt another. 

Not\marginnote{3.28.6} to do any evil;\footnote{These famous lines serve as a succinct summary of the entire Buddhist path. } \\
to embrace the good;\footnote{“Embrace” is \textit{\textsanskrit{upasampadā}}, to “enter into”. It is the same word used for taking ordination as well as entering \textit{\textsanskrit{jhāna}}. } \\
to purify one’s mind:\footnote{Through meditation. } \\
this is the instruction of the Buddhas.\footnote{The \textit{\textsanskrit{buddhasāsana}} (“instruction of the Buddha” or “dispensation of the Buddha”) is the normal word used by Buddhists to refer to their own religion. } 

Not\marginnote{3.28.10} speaking ill nor doing harm; \\
restraint in the monastic code;\footnote{Here “monastic code” means the principles outlined in these verses. They are spelled out in more detail in the Gradual Training (\href{https://suttacentral.net/dn2/en/sujato}{DN 2}, etc.), and ultimately in the detailed legal code of the \textsanskrit{Vinayapiṭaka}. } \\
moderation in eating; \\
staying in remote lodgings; \\
commitment to the higher mind—\footnote{The “higher mind” is the four \textit{\textsanskrit{jhānas}} (\href{https://suttacentral.net/an3.90/en/sujato\#3.2}{AN 3.90:3.2}). } \\
this is the instruction of the Buddhas.’ 

%
\end{verse}

\section*{17. Being Informed by Deities }

This\marginnote{3.29.1} one time, mendicants, I was staying near \textsanskrit{Ukkaṭṭhā}, in the Subhaga Forest at the root of a magnificent sal tree.\footnote{The Buddha now returns to the present life. \textsanskrit{Ukkaṭṭhā} is the home of \textsanskrit{Pokkharasāti} (\href{https://suttacentral.net/dn3/en/sujato\#2.13.1}{DN 3:2.13.1}) and the site of the astonishing and cosmically significant discourses \href{https://suttacentral.net/mn1/en/sujato}{MN 1} and \href{https://suttacentral.net/mn49/en/sujato}{MN 49}. } As I was in private retreat this thought came to mind, ‘It’s not easy to find an abode of sentient beings where I haven’t previously abided in all this long time, except for the gods of the pure abodes.\footnote{Only non-returners are born in the pure abodes (\textit{\textsanskrit{suddhāvāsā}}), from where they realize full awakening. } Why don’t I go to see them?’ 

Then,\marginnote{3.29.5} as easily as a strong person would extend or contract their arm, I vanished from the Subhaga Forest and reappeared with the gods of Aviha. 

In\marginnote{3.29.6} that order of gods, many thousands, many hundreds of thousands of deities approached me, bowed, stood to one side, and said to me,\footnote{“Order of gods” is \textit{\textsanskrit{devanikāya}}. I use “deity” to distinguish \textit{\textsanskrit{devatā}} from \textit{deva} but there is no substantial difference. } ‘Ninety-one eons ago, good sir, the Buddha \textsanskrit{Vipassī} arose in the world, perfected and fully awakened. He was born as an aristocrat into an aristocrat family. \textsanskrit{Koṇḍañña} was his clan. He lived for 80,000 years. He was awakened at the root of a patala tree. He had a fine pair of chief disciples named \textsanskrit{Khaṇḍa} and Tissa. He had three gatherings of disciples—one of 6,800,000, one of 100,000, and one of 80,000—all of them mendicants who had ended their defilements. He had as chief attendant a mendicant named Asoka. His father was King \textsanskrit{Bandhumā}, his birth mother was Queen \textsanskrit{Bandhumatī}, and their capital city was named \textsanskrit{Bandhumatī}. And such was his renunciation, such his going forth, such his striving, such his awakening, and such his rolling forth of the wheel of Dhamma.\footnote{This is a list of the major events in any Buddha biography. } And good sir, after leading the spiritual life under that Buddha \textsanskrit{Vipassī} we lost our desire for sensual pleasures and were reborn here.’\footnote{The lifespan in such a realm can be many hundreds of eons. } 

And\marginnote{3.29.20} other deities came and similarly recounted the details of the Buddhas \textsanskrit{Sikhī}, \textsanskrit{Vessabhū}, Kakusandha, \textsanskrit{Koṇāgamana}, and Kassapa.\footnote{These are absent in the Pali text at this point, but they are mentioned below, and the commentary says they should be included here. } 

In\marginnote{3.30.1} that order of gods, many thousands, many hundreds of thousands of deities approached me, bowed, stood to one side, and said to me, ‘In the present fortunate eon, good sir, you have arisen in the world, perfected and fully awakened. You were born as an aristocrat into an aristocrat family. Gotama is your clan. For you the lifespan is short, brief, and fleeting. A long-lived person lives for a century or a little more. You were awakened at the root of a peepal tree. You have a fine pair of chief disciples named \textsanskrit{Sāriputta} and \textsanskrit{Moggallāna}. You have had one gathering of disciples—1,250 mendicants who had ended their defilements. You have as chief attendant a mendicant named Ānanda. Your father was King Suddhodana, your birth mother was Queen \textsanskrit{Māyā}, and your capital city was Kapilavatthu. And such was your renunciation, such your going forth, such your striving, such your awakening, and such your rolling forth of the wheel of Dhamma. And good sir, after leading the spiritual life under you we lost our desire for sensual pleasures and were reborn here.’ 

Then\marginnote{3.31.1} together with the gods of Aviha I went to see the gods of Atappa … the gods fair to see … and the fair seeing gods. Then together with all these gods I went to see the gods of \textsanskrit{Akaniṭṭha}, where we had a similar conversation. 

And\marginnote{3.33.1} that is how the Realized One is able to recollect the birth, names, clans, lifespan, chief disciples, and gatherings of disciples of the Buddhas of the past who have become fully quenched, cut off proliferation, cut off the track, finished off the cycle, and transcended all suffering. It is both because I have clearly comprehended the principle of the teachings, and also because the deities told me.”\footnote{This is the detailed answer to the question asked by the mendicants at \href{https://suttacentral.net/dn14/en/sujato\#1.13.6}{DN 14:1.13.6}. } 

That\marginnote{3.33.5} is what the Buddha said. Satisfied, the mendicants approved what the Buddha said. 

%
\chapter*{{\suttatitleacronym DN 15}{\suttatitletranslation The Great Discourse on Causation }{\suttatitleroot Mahānidānasutta}}
\addcontentsline{toc}{chapter}{\tocacronym{DN 15} \toctranslation{The Great Discourse on Causation } \tocroot{Mahānidānasutta}}
\markboth{The Great Discourse on Causation }{Mahānidānasutta}
\extramarks{DN 15}{DN 15}

\section*{1. Dependent Origination }

\scevam{So\marginnote{1.1} I have heard.\footnote{This sutta with its commentary was translated by Bhikkhu Bodhi in his \emph{The Great Discourse on Causation}. It is the longest and most complex discourse on dependent origination in early Buddhism. } }At one time the Buddha was staying in the land of the Kurus, near the Kuru town named \textsanskrit{Kammāsadamma}.\footnote{The \textsanskrit{Kurukṣetra} was an ancient Brahmanical kingdom situated in the region around modern Delhi, bordered by the Ganges in the east, the \textsanskrit{Sarasvatī} in the west, the Himalayas in the north, and the hills of the Aravalli Range in the south. The \textsanskrit{Mahābharata} climaxes with the battle fought there between the Kurus and the \textsanskrit{Pāṇḍavas}. These events, to the extent that they can be established historically, took place several centuries before the Buddha. Kuru marks the north-western extent of the Buddha’s travels. \textsanskrit{Kammāsadamma} is identified with modern Kumashpur in Haryana, about 40 km north of Delhi. } 

Then\marginnote{1.3} Venerable Ānanda went up to the Buddha, bowed, sat down to one side, and said to him,\footnote{While we cannot date this text, the fact that Ānanda has evidently been studying deep matters for a long time, that it takes place outside the Buddha’s accustomed locales, and that it builds on teachings found elsewhere, suggests that it happened rather late in the Buddha’s life. At \href{https://suttacentral.net/sn12.60/en/sujato}{SN 12.60} we find the same introduction to a much shorter discourse. } “It’s incredible, sir, it’s amazing, in that this dependent origination is deep and appears deep, yet to me it seems as plain as can be.”\footnote{At \href{https://suttacentral.net/sn6.1/en/sujato\#1.6}{SN 6.1:1.6} (and \href{https://suttacentral.net/dn14/en/sujato\#3.1.6}{DN 14:3.1.6} in the case of \textsanskrit{Vipassī}), the Buddha hesitated to teach the Dhamma because dependent origination is so hard to see. Ānanda was not only learned and wise, he was a stream enterer who had directly experienced dependent origination (\href{https://suttacentral.net/an10.92/en/sujato\#6.1}{AN 10.92:6.1}), yet he still underestimates it. } 

“Don’t\marginnote{1.6} say that, Ānanda, don’t say that! This dependent origination is deep and appears deep.\footnote{The sutta introduce subtle variations in the standard formula as a means to illuminate hidden implications and dimensions. } It is because of not understanding and not penetrating this teaching that this population has become tangled like string, knotted like a ball of thread, and matted like rushes and reeds, and it doesn’t escape the places of loss, the bad places, the underworld, transmigration.\footnote{The Buddha establishes the primary purpose of dependent origination: to understand transmigration so as to be free from it. } 

When\marginnote{2.1} asked, ‘Is there a specific condition for old age and death?’ you should answer, ‘There is.’\footnote{Dependent origination is here taught in “reverse order” (\textit{\textsanskrit{paṭiloma}}, \href{https://suttacentral.net/ud1.2/en/sujato\#1.4}{Ud 1.2:1.4}), starting with the existential problem: we are all going to die. This factor and the next are resultant, so we cannot solve them directly. } If they say, ‘What is a condition for old age and death?’ you should answer, ‘Rebirth is a condition for old age and death.’\footnote{With this the Buddha denies the promise of immortality in this or any other realm. } 

When\marginnote{2.3} asked, ‘Is there a specific condition for rebirth?’ you should answer, ‘There is.’ If they say, ‘What is a condition for rebirth?’ you should answer, ‘Continued existence is a condition for rebirth.’\footnote{“Continued existence” (or “life”) is an encompassing term, including both resultant and causal dimensions. } 

When\marginnote{2.5} asked, ‘Is there a specific condition for continued existence?’ you should answer, ‘There is.’ If they say, ‘What is a condition for continued existence?’ you should answer, ‘Grasping is a condition for continued existence.’\footnote{“Grasping” and “craving” (together with “ignorance” in the full sequence) are the defilements that drive the process on. It is here that the practice of the path takes effect, uprooting them entirely. } 

When\marginnote{2.7} asked, ‘Is there a specific condition for grasping?’ you should answer, ‘There is.’ If they say, ‘What is a condition for grasping?’ you should answer, ‘Craving is a condition for grasping.’ 

When\marginnote{2.9} asked, ‘Is there a specific condition for craving?’ you should answer, ‘There is.’ If they say, ‘What is a condition for craving?’ you should answer, ‘Feeling is a condition for craving.’\footnote{The next three factors spell out the process of sense experience that unfolds automatically. Meditation slows it down so it can be seen clearly, but the process remains even for the perfected ones. } 

When\marginnote{2.11} asked, ‘Is there a specific condition for feeling?’ you should answer, ‘There is.’ If they say, ‘What is a condition for feeling?’ you should answer, ‘Contact is a condition for feeling.’ 

When\marginnote{2.13} asked, ‘Is there a specific condition for contact?’ you should answer, ‘There is.’ If they say, ‘What is a condition for contact?’ you should answer, ‘Name and form are conditions for contact.’\footnote{Here we encounter the first unique feature of this sequence, as normally the six sense fields appear as the condition for contact. The reason for this special presentation becomes clear later on. } 

When\marginnote{2.15} asked, ‘Is there a specific condition for name and form?’ you should answer, ‘There is.’ If they say, ‘What is a condition for name and form?’ you should answer, ‘Consciousness is a condition for name and form.’ 

When\marginnote{2.17} asked, ‘Is there a specific condition for consciousness?’ you should answer, ‘There is.’ If they say, ‘What is a condition for consciousness?’ you should answer, ‘Name and form are conditions for consciousness.’\footnote{The mutual conditioning of consciousness with name and form is a further subtlety of this presentation. We have met this idea before in \href{https://suttacentral.net/dn14/en/sujato\#2.18.43}{DN 14:2.18.43}; it also occurs in \href{https://suttacentral.net/sn12.65/en/sujato}{SN 12.65} and \href{https://suttacentral.net/sn12.67/en/sujato}{SN 12.67}. Again, implications of this are explored below. } 

So:\marginnote{3.1} name and form are conditions for consciousness. Consciousness is a condition for name and form. Name and form are conditions for contact. Contact is a condition for feeling. Feeling is a condition for craving. Craving is a condition for grasping. Grasping is a condition for continued existence. Continued existence is a condition for rebirth. Rebirth is a condition for old age and death, sorrow, lamentation, pain, sadness, and distress to come to be.\footnote{The sequence is recapped in forward order (\textit{anuloma}, \href{https://suttacentral.net/ud1.1/en/sujato\#1.4}{Ud 1.1:1.4}). } That is how this entire mass of suffering originates. 

‘Rebirth\marginnote{4.1} is a condition for old age and death’—that’s what I said. And this is a way to understand how this is so.\footnote{The Buddha now takes up each of the terms in more depth. } Suppose there were totally and utterly no rebirth for anyone anywhere.\footnote{This emphatic phrasing drives home that “cessation” (\textit{nirodha}) is not simply temporary suppression or non-arising, but permanent and complete absence. } That is, there were no rebirth of sentient beings into their various realms—of gods, centaurs, spirits, creatures, humans, quadrupeds, birds, or reptiles, each into their own realm. When there’s no rebirth at all, with the cessation of rebirth, would old age and death still be found?”\footnote{“Rebirth” (\textit{\textsanskrit{jāti}}) is defined as the birth of a new life, not as simple arising. } 

“No,\marginnote{4.4} sir.” 

“That’s\marginnote{4.5} why this is the cause, source, origin, and reason of old age and death, namely rebirth.\footnote{The various terms for “cause” are used as synonyms (\textit{hetu}, \textit{\textsanskrit{nidāna}}, \textit{samudaya}, \textit{paccaya}). The purpose of using different terms is not to add nuances, but to reinforce the central meaning and guard against the corruption of meaning; if one word is lost or misunderstood, the meaning of the sentence as a whole is not changed. } 

‘Continued\marginnote{5.1} existence is a condition for rebirth’—that’s what I said. And this is a way to understand how this is so. Suppose there were totally and utterly no continued existence for anyone anywhere. That is, continued existence in the sensual realm, the realm of luminous form, or the formless realm. When there’s no continued existence at all, with the cessation of continued existence, would rebirth still be found?”\footnote{These realms relate to the kamma that projects consciousness into them. The realms of “luminous form” (\textit{\textsanskrit{rūpabhava}}) and the “formless” (\textit{\textsanskrit{arūpabhava}}) are generated by the form and formless attainments respectively. Any other kamma, good or bad, pertains to the sensual realm (\textit{\textsanskrit{kāmabhava}}). All rebirth takes place in one or other of these realms. } 

“No,\marginnote{5.4} sir.” 

“That’s\marginnote{5.5} why this is the cause, source, origin, and reason of rebirth, namely continued existence. 

‘Grasping\marginnote{6.1} is a condition for continued existence’—that’s what I said. And this is a way to understand how this is so. Suppose there were totally and utterly no grasping for anyone anywhere. That is, grasping at sensual pleasures, views, precepts and observances, and theories of a self. When there’s no grasping at all, with the cessation of grasping, would continued existence still be found?”\footnote{“Grasping” (\textit{\textsanskrit{upādāna}}) begins with the primal desire of the senses, but the three other graspings are rather intellectual and sophisticated. Only a grown human being with a developed linguistic ability is able to formulate a view to become attached to, and likewise with attachment to religious observances and vows, and to theories of a self. This is why the bulk of kamma is produced by adult humans, rather than by, say, animals or children, for whom these dimensions of grasping are nascent. } 

“No,\marginnote{6.4} sir.” 

“That’s\marginnote{6.5} why this is the cause, source, origin, and reason of continued existence, namely grasping. 

‘Craving\marginnote{7.1} is a condition for grasping’—that’s what I said. And this is a way to understand how this is so. Suppose there were totally and utterly no craving for anyone anywhere. That is, craving for sights, sounds, smells, tastes, touches, and ideas. When there’s no craving at all, with the cessation of craving, would grasping still be found?”\footnote{“Craving” (\textit{\textsanskrit{taṇhā}}) is a fundamental desire or urge. Unlike grasping, it is fully active in children and animals. Often it has a threefold definition, which we find below, but in dependent origination it is usually defined in terms of the six senses, which relates it to the items to come. } 

“No,\marginnote{7.4} sir.” 

“That’s\marginnote{7.5} why this is the cause, source, origin, and reason of grasping, namely craving. 

‘Feeling\marginnote{8.1} is a condition for craving’—that’s what I said. And this is a way to understand how this is so. Suppose there were totally and utterly no feeling for anyone anywhere. That is, feeling born of contact through the eye, ear, nose, tongue, body, and mind. When there’s no feeling at all, with the cessation of feeling, would craving still be found?”\footnote{The usual threefold definition of feeling (pleasant, painful, neutral) is reframed in terms of the six senses. } 

“No,\marginnote{8.4} sir.” 

“That’s\marginnote{8.5} why this is the cause, source, origin, and reason of craving, namely feeling. 

So\marginnote{9.1} it is, Ānanda, that feeling is a cause of craving. Craving is a cause of seeking. Seeking is a cause of gaining material things. Gaining material things is a cause of evaluation. Evaluation is a cause of desire and lust. Desire and lust is a cause of attachment. Attachment is a cause of ownership. Ownership is a cause of stinginess. Stinginess is a cause of safeguarding.\footnote{Here the Buddha introduces another, even more radical, departure from the typical sequence. Rather than continue back to contact and name and form, he branches out in an entirely new direction. These “nine things rooted in craving” are found independently at \href{https://suttacentral.net/an9.23/en/sujato\#1.2}{AN 9.23:1.2} and \href{https://suttacentral.net/dn34/en/sujato\#2.2.31}{DN 34:2.2.31}, but only here are they integrated with the standard dependent origination. } Owing to safeguarding, many bad, unskillful things come to be: taking up the rod and the sword, quarrels, arguments, and disputes, accusations, divisive speech, and lies.\footnote{While the purpose of the main dependent origination is to disclose the web of conditions that generates the suffering of transmigration, here the Buddha looks at the arising of social conflict and disorder. } 

‘Owing\marginnote{10.1} to safeguarding, many bad, unskillful things come to be: taking up the rod and the sword, quarrels, arguments, and disputes, accusations, divisive speech, and lies’—that’s what I said. And this is a way to understand how this is so.\footnote{In \href{https://suttacentral.net/dn27/en/sujato}{DN 27} \textsanskrit{Aggaññasutta} the Buddha narrates a legend showing how these things arise. The point is not that we should not safeguard (\textit{\textsanskrit{ārakkha}}) our possessions. It is, rather, that so long as we live in a world where safeguarding possessions is necessary, there will be conflict and violence. } Suppose there were totally and utterly no safeguarding for anyone anywhere. When there’s no safeguarding at all, with the cessation of safeguarding, would those many bad, unskillful things still come to be?” 

“No,\marginnote{10.3} sir.” 

“That’s\marginnote{10.4} why this is the cause, source, origin, and reason for the origination of those many bad, unskillful things, namely safeguarding. 

‘Stinginess\marginnote{11.1} is a cause of safeguarding’—that’s what I said. And this is a way to understand how this is so.\footnote{“Stinginess” is \textit{macchariya}. } Suppose there were totally and utterly no stinginess for anyone anywhere. When there’s no stinginess at all, with the cessation of stinginess, would safeguarding still be found?” 

“No,\marginnote{11.3} sir.” 

“That’s\marginnote{11.4} why this is the cause, source, origin, and reason of safeguarding, namely stinginess. 

‘Ownership\marginnote{12.1} is a cause of stinginess’—that’s what I said. And this is a way to understand how this is so. Suppose there were totally and utterly no ownership for anyone anywhere. When there’s no ownership at all, with the cessation of ownership, would stinginess still be found?” 

“No,\marginnote{12.3} sir.” 

“That’s\marginnote{12.4} why this is the cause, source, origin, and reason of stinginess, namely ownership. 

‘Attachment\marginnote{13.1} is a cause of ownership’—that’s what I said. And this is a way to understand how this is so.\footnote{There are many words in Pali that approximate the English word “attachment”. Here it is \textit{\textsanskrit{ajjhosāna}}. } Suppose there were totally and utterly no attachment for anyone anywhere. When there’s no attachment at all, with the cessation of attachment, would ownership still be found?” 

“No,\marginnote{13.3} sir.” 

“That’s\marginnote{13.4} why this is the cause, source, origin, and reason of ownership, namely attachment. 

‘Desire\marginnote{14.1} and lust is a cause of attachment’—that’s what I said. And this is a way to understand how this is so. Suppose there were totally and utterly no desire and lust for anyone anywhere. When there’s no desire and lust at all, with the cessation of desire and lust, would attachment still be found?” 

“No,\marginnote{14.3} sir.” 

“That’s\marginnote{14.4} why this is the cause, source, origin, and reason of attachment, namely desire and lust. 

Evaluation\marginnote{15.1} is a cause of desire and lust’—that’s what I said. And this is a way to understand how this is so.\footnote{“Evaluation” is \textit{vinicchaya}. We like to weigh up and consider the pros and cons of different objects of desire. } Suppose there were totally and utterly no evaluation for anyone anywhere. When there’s no evaluation at all, with the cessation of evaluation, would desire and lust still be found?” 

“No,\marginnote{15.3} sir.” 

“That’s\marginnote{15.4} why this is the cause, source, origin, and reason of desire and lust, namely evaluation. 

‘Gaining\marginnote{16.1} material things is a cause of evaluation’—that’s what I said. And this is a way to understand how this is so.\footnote{Those who have nothing are grateful for any small thing, and do not indulge in picking and choosing. } Suppose there were totally and utterly no gaining of material things for anyone anywhere. When there’s no gaining of material things at all, with the cessation of gaining material things, would evaluation still be found?” 

“No,\marginnote{16.3} sir.” 

“That’s\marginnote{16.4} why this is the cause, source, origin, and reason of evaluation, namely the gaining of material things. 

‘Seeking\marginnote{17.1} is a cause of gaining material things’—that’s what I said. And this is a way to understand how this is so.\footnote{Our senses are tuned to hunt out and acquire pleasure. } Suppose there were totally and utterly no seeking for anyone anywhere. When there’s no seeking at all, with the cessation of seeking, would the gaining of material things still be found?” 

“No,\marginnote{17.3} sir.” 

“That’s\marginnote{17.4} why this is the cause, source, origin, and reason of gaining material things, namely seeking. 

‘Craving\marginnote{18.1} is a cause of seeking’—that’s what I said. And this is a way to understand how this is so. Suppose there were totally and utterly no craving for anyone anywhere. That is, craving for sensual pleasures, craving for continued existence, and craving to end existence. When there’s no craving at all, with the cessation of craving, would seeking still be found?”\footnote{This is the normal definition of craving in the four noble truths, supplementing the previous definition in terms of the six senses. Both are included in this sutta to show that they do not contradict, but rather reveal different aspects of the same thing. It is not just sensual desire that drives acquisition. For example, religious people fight over sacred ground or holy objects to gain a place in heaven; or else those driven by nihilism go to any lengths for alcohol or drugs to erase existence. } 

“No,\marginnote{18.4} sir.” 

“That’s\marginnote{18.5} why this is the cause, source, origin, and reason of seeking, namely craving. And so, Ānanda, these two things are united by the two aspects of feeling.\footnote{The threefold analysis of feeling leads to the process of acquisition, while the sixfold analysis of feeling leads to dependent origination. } 

‘Contact\marginnote{19.1} is a condition for feeling’—that’s what I said. And this is a way to understand how this is so.\footnote{And now we rejoin the main sequence. } Suppose there were totally and utterly no contact for anyone anywhere. That is, contact through the eye, ear, nose, tongue, body, and mind. When there’s no contact at all, with the cessation of contact, would feeling still be found?”\footnote{People mostly want to enjoy pleasant sensations, in this life and the next, but those sensations depend on a constant supply of the appropriate stimuli. } 

“No,\marginnote{19.4} sir.” 

“That’s\marginnote{19.5} why this is the cause, source, origin, and reason of feeling, namely contact. 

‘Name\marginnote{20.1} and form are conditions for contact’—that’s what I said. And this is a way to understand how this is so.\footnote{Name and form are said to be conditions for contact also at \href{https://suttacentral.net/snp4.11/en/sujato\#11.1}{Snp 4.11:11.1}. } Suppose there were none of the features, attributes, signs, and details by which the set of mental phenomena known as name is found. Would labeling contact still be found in the set of physical phenomena?”\footnote{The “set of mental phenomena known as name” is \textit{\textsanskrit{nāmakāya}}. Its function depends not any underlying essence, but on the “features” by which it is “made known”; this is a phenomenological analysis. | “Labeling contact” is \textit{adhivacanasamphassa}; it is the active process by which the mind makes sense of the world by attaching labels to experience. This passage reinforces the linguistic significance of \textit{\textsanskrit{nāma}}. } 

“No,\marginnote{20.3} sir.” 

“Suppose\marginnote{20.4} there were none of the features, attributes, signs, and details by which the set of physical phenomena known as form is found. Would impingement contact still be found in the set of mental phenomena?”\footnote{The “set of physical phenomena known as form” is \textit{\textsanskrit{rūpakāya}}. | “Impingement contact” is \textit{\textsanskrit{paṭighasamphassa}}. Here \textit{\textsanskrit{paṭigha}} refers to the “striking” of physical phenomena against each other, such as light “hitting” the eye. It most commonly appears in this sense in the formula that begins the formless attainments. } 

“No,\marginnote{20.5} sir.” 

“Suppose\marginnote{20.6} there were none of the features, attributes, signs, and details by which the set of phenomena known as name and the set of phenomena known as form are found. Would either labeling contact or impingement contact still be found?”\footnote{Labeling moves from the mind to the world; impingement moves from the world to the mind. Together they create a dynamic two-way process by which we learn about the world and how to make sense of it. } 

“No,\marginnote{20.7} sir.” 

“Suppose\marginnote{20.8} there were none of the features, attributes, signs, and details by which name and form are found. Would contact still be found?”\footnote{Contact is fundamentally a meeting, normally expressed as the coming together of the sense stimulus (light), the sense organ (eye), and sense consciousness. By skipping the direct mention of the six senses, the Buddha opens another perspective on this process: mental labeling meets sense impingement, each essential to the other, and together making contact possible. The analysis itself exemplifies this process, as it starts by looking at the process from each side, and moves towards integration, seeing them both together. } 

“No,\marginnote{20.9} sir.” 

“That’s\marginnote{20.10} why this is the cause, source, origin, and reason of contact, namely name and form. 

‘Consciousness\marginnote{21.1} is a condition for name and form’—that’s what I said. And this is a way to understand how this is so.\footnote{Consciousness in dependent origination is normally defined as the six kinds of sense consciousness (\href{https://suttacentral.net/sn12/en/sujato\#6.3}{SN 12:6.3}). The purpose of this is to emphasize that the process of rebirth and transmigration is an empirical process, which depends on the same ordinary consciousness we are experiencing now. Here, once again, by skipping the six senses, a new mode of analysis opens up, which emphasizes the organic growth of the individual. } If consciousness were not conceived in the mother’s womb, would name and form coagulate there?”\footnote{“Conceived” is \textit{okkamissatha}, literally “descend” or “arrive”. | “Coagulate” assumes the PTS reading \textit{samucchissatha} (Sanskrit \textit{sammurch}). I believe this echoes the belief that the embryo is “coagulated” from the mix of blood and semen. | Linguistically, this passage through to \href{https://suttacentral.net/dn15/en/sujato\#22.2}{DN 15:22.2} is marked with the extremely rare verbal ending \textit{-issatha}, which is the middle form of the third person singular conditional. } 

“No,\marginnote{21.3} sir.” 

“If\marginnote{21.4} consciousness, after being conceived in the mother’s womb, were to be miscarried, would name and form be born into this place?”\footnote{“Miscarried” is \textit{vokkamissatha}. | “State of existence” is \textit{\textsanskrit{itthattā}}, which is most commonly found in the declaration of the arahant that they will no longer be reborn into “this state of existence”. | Here “born” is \textit{abhinibbatti}, which is listed along with \textit{\textsanskrit{jāti}}, \textit{okkanti}, and other terms as a synonym in the standard definition of rebirth (\href{https://suttacentral.net/sn12.2/en/sujato\#4.2}{SN 12.2:4.2}, \href{https://suttacentral.net/mn9/en/sujato\#24-26.7}{MN 9:24–26.7}, \href{https://suttacentral.net/dn22/en/sujato\#18.4}{DN 22:18.4}). } 

“No,\marginnote{21.5} sir.” 

“If\marginnote{21.6} the consciousness of a young boy or girl were to be cut off, would name and form achieve growth, increase, and maturity?”\footnote{The connection between dependent origination and childhood development is further explored in \href{https://suttacentral.net/mn38/en/sujato\#28.1}{MN 38:28.1}. } 

“No,\marginnote{21.7} sir.” 

“That’s\marginnote{21.8} why this is the cause, source, origin, and reason of name and form, namely consciousness. 

‘Name\marginnote{22.1} and form are conditions for consciousness’—that’s what I said. And this is a way to understand how this is so.\footnote{Now we turn to the mirror side of the pair of conditions. } If consciousness were not to gain a footing in name and form, would the coming to be of the origin of suffering—of rebirth, old age, and death in the future—be found?”\footnote{Just as name and form—the organic, sensual, and sense-making body—cannot grow without consciousness, so too consciousness must acquire a landing or grounding place to be “planted” in name and form. | \textit{Dukkhasamudayasambhavo} (“the coming to be of the origin of suffering”) might be rendered “the coming to be and origin of suffering”. However, \textit{dukkhasamudaya} occurs some hundreds of times in the sense “origin of suffering” so I take it in the same way here. This is supported by the PTS variant reading \textit{dukkhasamudayo sambhavo}. } 

“No,\marginnote{22.3} sir.” 

“That’s\marginnote{22.4} why this is the cause, source, origin, and reason of consciousness, namely name and form. This is the extent to which one may be reborn, grow old, die, pass away, or reappear.\footnote{This passage continues to employ rare middle forms, this time \textit{-etha}, the third person singular optative. } This is how far the scope of labeling, terminology, and description extends; how far the sphere of wisdom extends; how far the cycle of rebirths proceeds so that this state of being may be found; namely, name and form together with consciousness.\footnote{This passage explains why the sequence ends here rather than proceeding in the usual way to choices and ignorance. Any state of being ultimately depends on the codependency of name and form with consciousness. Within this key relationship is the extent not only of language, but also of wisdom, and the secret to the undoing of transmigration itself. One of the many profound implications of this is that there is no such thing as a state of pure consciousness independent of concepts. | \textit{\textsanskrit{Ettāvatā} \textsanskrit{vaṭṭaṁ} vattati \textsanskrit{itthattaṁ} \textsanskrit{paññāpanāya}} should be read with such passages as \href{https://suttacentral.net/sn22.56/en/sujato\#5.3}{SN 22.56:5.3}: \textit{ye kevalino \textsanskrit{vaṭṭaṁ} \textsanskrit{tesaṁ} natthi \textsanskrit{paññāpanāya}} (“For consummate ones, there is no cycle of rebirths to be found”). | The \textsanskrit{Mahāsaṅgīti} reading \textit{\textsanskrit{aññamaññapaccayatā} pavattati} is spurious, since it inserts an Abhidhamma concept from the commentary. } 

\section*{2. Describing the Self }

How\marginnote{23.1} do those who describe the self describe it?\footnote{The text now turns to an analysis of theories of “self” (\textit{\textsanskrit{attā}}), which is comparable to some of the passages of \href{https://suttacentral.net/dn1/en/sujato}{DN 1}. The Buddha began his discourse by stating that it is the failure to understand dependent origination that keeps beings trapped in transmigration. Dependent origination explains transmigration in a purely empirical way by inferring from the mental and physical phenomena we experience here and now. Self theorists, on the other hand, explain transmigration by introducing a new metaphysical principle, the “self” or “soul”, by which they assume that the individual has an eternal underlying essence. } They describe it as formed and limited:\footnote{“Formed” is \textit{\textsanskrit{rūpī}} (“possessing form”), identifying the self with the first of the five aggregates. If something were really the core essence of a person, you would think it is readily knowable. But the Buddha shows that theorists describe the self in multiple different and incompatible ways. Each of these draws on some more-or-less arbitrary aspect of empirical reality, such as “form”, to describe an unknowable metaphysical entity that is in fact just pure supposition. } ‘My self is formed and limited.’\footnote{An example of a self that is “physical and limited” would be the body. } Or they describe it as formed and infinite: ‘My self is formed and infinite.’\footnote{Such as the cosmos. } Or they describe it as formless and limited: ‘My self is formless and limited.’\footnote{Perhaps this is the self of “limited perception” (\href{https://suttacentral.net/dn1/en/sujato\#2.38.13}{DN 1:2.38.13}). This would be where the mind is aware of something limited, and the self is identified with the mental dimension of that awareness. } Or they describe it as formless and infinite: ‘My self is formless and infinite.’\footnote{Such as the formless dimensions. } 

Now,\marginnote{24.1} take those who describe the self as formed and limited. They describe the self in the present as formed and limited; or they describe it as sure to be in some other place formed and limited; or else they think: ‘Though it is not like that, I will ensure it is provided with what it needs to become like that.’\footnote{The three options (“it is”, “it will be so”, and “I will make it be so”) illustrate how the theorist resorts to ever more convoluted means to justify the lack of empirical support for their pet theory. | \textit{\textsanskrit{Bhāviṁ}} (“sure to become”) is the root \textit{\textsanskrit{bhū}} with the primary affix \textit{-\textsanskrit{ī}}, which connotes an inevitable future state. | \textit{Tattha} (“in some other place”, literally “there”) is explained by the commentary as \textit{paraloke} (“in the next world”). } This being so, it’s appropriate to say that a view of self as formed and limited underlies them.\footnote{Their surface differences rest on the same underlying assumption, so if the assumption is disproved there is no need to refute each individual theory. | \textit{\textsanskrit{Iccālaṁ}} resolves to \textit{iti \textsanskrit{alaṁ}}. } 

Now,\marginnote{24.4} take those who describe the self as formed and infinite … formless and limited … formless and infinite. They describe the self as formless and infinite in the present; or as sure to become formless and infinite in some other place; or else they think: ‘Though it is not like that, I will ensure it is provided with what it needs to become like that.’ This being so, it’s appropriate to say that a view of self as formless and infinite underlies them. That’s how those who describe the self describe it. 

\section*{3. Not Describing the Self }

How\marginnote{25.1} do those who don’t describe the self not describe it?\footnote{This is the Buddha, who does not theorize a metaphysical self. Implicit in this argument is Occam’s razor, “entities must not be multiplied beyond necessity”. Since the self theorists want to prove the existence of the “self”, it is up to them to supply the grounds to support their suppositions. Since they fail to do so, the rational position is that there is no self. The Buddha is not under a similar obligation to prove the non-existence of the “self”, since it is reasonable to assume that things do not exist until the evidence says otherwise. } They don’t describe it as formed and limited … formed and infinite … formless and limited … formless and infinite: ‘My self is formless and infinite.’ 

Now,\marginnote{26.1} take those who don’t describe the self as formed and limited … formed and infinite … formless and limited … formless and infinite. They don’t describe the self in the present as formless and infinite; or as sure to become in some other place formless and infinite; and they don’t think: ‘Though it is not like that, I will ensure it is provided with what it needs to become like that.’ This being so, it’s appropriate to say that a view of self as formless and infinite doesn’t underlie them. That’s how those who don’t describe the self don’t describe it. 

\section*{4. Regarding a Self }

How\marginnote{27.1} do those who regard the self regard it?\footnote{Having asserted a metaphysical “self”, the theorists go on to make certain observations and interpretations regarding it. } They regard feeling as self:\footnote{The Buddha moves from theories of the self as form to feeling, the second of the five aggregates. } ‘Feeling is my self.’\footnote{For example, identifying the self with the supposed eternal bliss of heaven. As with the description of self in physical terms, the theorist proceeds from a simple assertion of identity to more complicated hypotheses. } Or they regard it like this: ‘Feeling is definitely not my self. My self does not experience feeling.’\footnote{This is the inverse of the previous. The self is still defined \emph{in relation to} feeling, but it is a negative relation. Such theories are commonly found in the \textsanskrit{Upaniṣads}, where a prominent thread of analysis systematically rejects all the things that are \emph{not} the self (\textit{neti}), before finally arriving at what \emph{is} the self (eg. \textsanskrit{Bṛhadāraṇyaka} \textsanskrit{Upaniṣad} 4.5.15). } Or they regard it like this: ‘Feeling is definitely not my self. But it’s not that my self does not experience feeling. My self feels, for my self is liable to feel.’\footnote{Here the theorist describes feeling as a function of the self: it is not what the self \emph{is}, but what the self \emph{does}. At \href{https://suttacentral.net/mn38/en/sujato\#5.11}{MN 38:5.11} (= \href{https://suttacentral.net/mn2/en/sujato\#8.8}{MN 2:8.8}), \textsanskrit{Sāti} describes the self as consciousness, “the speaker and feeler who experiences the results of good and bad deeds in all the different realms”. } 

Now,\marginnote{28.1} as to those who say:\footnote{The Buddha goes on to develop specific arguments addressing each position. } ‘Feeling is my self.’ You should say this to them: ‘Reverend, there are three feelings: pleasant, painful, and neutral.\footnote{The Buddha points out the universal experience of feeling, thus establishing his argument on common ground. This same argument is found at \href{https://suttacentral.net/mn74/en/sujato\#10.1}{MN 74:10.1}. } Which one of these do you regard as self?’ Ānanda, at a time when you feel a pleasant feeling, you don’t feel a painful or neutral feeling;\footnote{The idea that only one kind of feeling can be experienced at a time became an adage of Buddhist psychology, but it is not obvious to me that it is the case. Here and at MN 74 it is assumed to be a shared belief with non-Buddhists. } you only feel a pleasant feeling. At a time when you feel a painful feeling, you don’t feel a pleasant or neutral feeling; you only feel a painful feeling. At a time when you feel a neutral feeling, you don’t feel a pleasant or painful feeling; you only feel a neutral feeling. 

Pleasant\marginnote{29.1} feelings, painful feelings, and neutral feelings are all impermanent, conditioned, dependently originated, liable to end, vanish, fade away, and cease. When feeling a pleasant feeling they think: ‘This is my self.’ When their pleasant feeling ceases they think: ‘My self has disappeared.’ When feeling a painful feeling they think: ‘This is my self.’ When their painful feeling ceases they think: ‘My self has disappeared.’ When feeling a neutral feeling they think: ‘This is my self.’ When their neutral feeling ceases they think: ‘My self has disappeared.’ So those who say ‘feeling is my self’ regard as self that which is evidently impermanent, mixed with pleasure and pain, and liable to rise and fall. That’s why it’s not acceptable to regard feeling as self. 

Now,\marginnote{30.1} as to those who say: ‘Feeling is definitely not my self. My self does not experience feeling.’ You should say this to them, ‘But reverend, where there is nothing felt at all, would the thought “I am” occur there?’”\footnote{Feeling is part of the fundamental structure of consciousness. This argument comes through more clearly in Pali, for the word for “feeling” (\textit{\textsanskrit{vedanā}}) is derived from and still lies close to the sense of “knowing, experiencing”. Thus the question is, “If there was no mind, would there be the thought ‘I am this’?” The commentary explains that this refers to the bare material realm which is devoid of consciousness. The commentary appears to support the variant reading \textit{\textsanskrit{ahamasmī}} here. This makes sense in context, for “I am” is the first and most primordial assertion of a self, while “I am this” is a more sophisticated identification of the self in relation to the aggregates. } 

“No,\marginnote{30.4} sir.” 

“That’s\marginnote{30.5} why it’s not acceptable to regard self as that which does not experience feeling. 

Now,\marginnote{31.1} as to those who say: ‘Feeling is definitely not my self. But it’s not that my self does not experience feeling. My self feels, for my self is liable to feel.’ You should say this to them, ‘Suppose feelings were to totally and utterly cease without anything left over. When there’s no feeling at all, with the cessation of feeling, would the thought “I am this” occur there?’”\footnote{The theorist avoids identifying feeling as the self, but they must identify \emph{something} as the self (as for example, \textsanskrit{Sāti} said the self was \textit{\textsanskrit{viññāṇa}}). Feeling, however, is deeply wound into the structure of consciousness, so if feeling were to be utterly absent, no other mental phenomena could continue, and there would therefore be no possibility of forming a theory of self. } 

“No,\marginnote{31.6} sir.” 

“That’s\marginnote{31.7} why it’s not acceptable to regard self as that which is liable to feel. 

Not\marginnote{32.1} regarding anything in this way, they don’t grasp at anything in the world. Not grasping, they’re not anxious. Not being anxious, they personally become extinguished.\footnote{Letting go is not just a conceptual matter, it has immediate emotional consequences. \textit{Paritassati} conveys the twin senses of desire and agitation, for which “anxiety” seems the best fit. } They understand: ‘Rebirth is ended, the spiritual journey has been completed, what had to be done has been done, there is nothing further for this place.’ 

It\marginnote{32.5} wouldn’t be appropriate to say that a mendicant whose mind is freed like this holds the following views: ‘A realized one still exists after death’; ‘A realized one no longer exists after death’; ‘A realized one both still exists and no longer exists after death’; ‘A realized one neither still exists nor no longer exists after death’. 

Why\marginnote{32.10} is that? A mendicant is freed by directly knowing this: how far labeling and the scope of labeling extend; how far terminology and the scope of terminology extend; how far description and the scope of description extend; how far wisdom and the sphere of wisdom extend; how far the cycle of rebirths and its continuation extend. It wouldn’t be appropriate to say that a mendicant freed by directly knowing this holds the view: ‘There is no such thing as knowing and seeing.’\footnote{This recalls the similar statement at \href{https://suttacentral.net/dn15/en/sujato\#22.6}{DN 15:22.6}. Whereas there it was a statement about name and form with consciousness, here it is a description of the arahant who has fully realized it. Unlike the theorists whose views do not withstand empirical scrutiny, the arahant’s liberation is based on a direct understanding of reality. } 

\section*{5. Planes of Consciousness }

Ānanda,\marginnote{33.1} there are seven planes of consciousness and two dimensions.\footnote{The Buddha returns once more to the question of rebirth, describing various states of rebirth in terms of consciousness. The seven planes are also mentioned at \href{https://suttacentral.net/dn33/en/sujato\#2.3.28}{DN 33:2.3.28}, \href{https://suttacentral.net/dn34/en/sujato\#1.8.11}{DN 34:1.8.11}, and \href{https://suttacentral.net/an7.44/en/sujato\#1.1}{AN 7.44:1.1}. } What seven? 

There\marginnote{33.3} are sentient beings that are diverse in body and diverse in perception, such as human beings, some gods, and some beings in the underworld. This is the first plane of consciousness.\footnote{“Plane of consciousness” is \textit{\textsanskrit{viññāṇaṭṭhiti}}, which could also be rendered “station”. } 

There\marginnote{33.5} are sentient beings that are diverse in body and unified in perception, such as the gods reborn in the Divinity’s host through the first absorption. This is the second plane of consciousness. 

There\marginnote{33.7} are sentient beings that are unified in body and diverse in perception, such as the gods of streaming radiance. This is the third plane of consciousness. 

There\marginnote{33.9} are sentient beings that are unified in body and unified in perception, such as the gods of universal beauty. This is the fourth plane of consciousness. 

There\marginnote{33.11} are sentient beings that have gone totally beyond perceptions of form. With the ending of perceptions of impingement, not focusing on perceptions of diversity, aware that ‘space is infinite’, they have been reborn in the dimension of infinite space. This is the fifth plane of consciousness. 

There\marginnote{33.13} are sentient beings that have gone totally beyond the dimension of infinite space. Aware that ‘consciousness is infinite’, they have been reborn in the dimension of infinite consciousness. This is the sixth plane of consciousness. 

There\marginnote{33.15} are sentient beings that have gone totally beyond the dimension of infinite consciousness. Aware that ‘there is nothing at all’, they have been reborn in the dimension of nothingness. This is the seventh plane of consciousness. 

Then\marginnote{33.17} there is the dimension of non-percipient beings, and secondly, the dimension of neither perception nor non-perception.\footnote{In the first of these dimensions there is no consciousness at all, and in the second there is no consciousness in the normal sense, which is why they cannot be called “planes of consciousness”. } 

Now,\marginnote{34.1} regarding these seven planes of consciousness and two dimensions, is it appropriate for someone who understands them—and their origin, ending, gratification, drawback, and escape—to take pleasure in them?”\footnote{\textit{\textsanskrit{Abhinandituṁ}}, to “take pleasure in”, to “relish”, or to “delight in” appears in the standard formula for the second noble truth, where craving “takes pleasure in various realms” (\textit{\textsanskrit{tatratatrābhinandinī}}). } 

“No,\marginnote{34.3} sir.” 

“When\marginnote{34.10} a mendicant, having truly understood the origin, ending, gratification, drawback, and escape regarding these seven planes of consciousness and these two dimensions, is freed by not grasping, they’re called a mendicant who is freed by wisdom.\footnote{One “freed by wisdom” has wisdom as the dominant faculty. } 

\section*{6. The Eight Liberations }

Ānanda,\marginnote{35.1} there are these eight liberations.\footnote{The eight liberations (\textit{\textsanskrit{vimokkhā}}) are an alternative way of describing the meditative experiences of \textit{\textsanskrit{jhāna}}. Elsewhere they are listed at \href{https://suttacentral.net/dn16/en/sujato\#3.33.1}{DN 16:3.33.1}, \href{https://suttacentral.net/dn33/en/sujato\#3.1.168}{DN 33:3.1.168}, \href{https://suttacentral.net/dn34/en/sujato\#2.1.191}{DN 34:2.1.191}, \href{https://suttacentral.net/an/en/sujato\#8.66}{AN:8.66}, \href{https://suttacentral.net/mn77/en/sujato\#22.1}{MN 77:22.1}, and referred to at \href{https://suttacentral.net/an4.189/en/sujato\#1.8}{AN 4.189:1.8} and \href{https://suttacentral.net/thag20.1/en/sujato\#33.1}{Thag 20.1:33.1}. At \href{https://suttacentral.net/an8.120/en/sujato}{AN 8.120} and \href{https://suttacentral.net/mn137/en/sujato\#27.1}{MN 137:27.1} they are listed but not called the eight liberations. } What eight? 

Having\marginnote{35.3} physical form, they see forms.\footnote{Someone sees a meditative vision based on the perception of their own body, such as through mindfulness of breathing or one’s own body parts. The first three liberations all cover the four \textit{\textsanskrit{jhānas}}. } This is the first liberation. 

Not\marginnote{35.5} perceiving form internally, they see forms externally.\footnote{A meditator grounds their practice on some external focus, such as a light, the sight of a corpse, or an external element such as earth. } This is the second liberation. 

They’re\marginnote{35.7} focused only on beauty.\footnote{This is a practice based on wholly pure and exalted meditation, such as the meditation on love, or the sight of a pure brilliant color like the sky. } This is the third liberation. 

Going\marginnote{35.9} totally beyond perceptions of form, with the ending of perceptions of impingement, not focusing on perceptions of diversity, aware that ‘space is infinite’, they enter and remain in the dimension of infinite space. This is the fourth liberation. 

Going\marginnote{35.11} totally beyond the dimension of infinite space, aware that ‘consciousness is infinite’, they enter and remain in the dimension of infinite consciousness. This is the fifth liberation. 

Going\marginnote{35.13} totally beyond the dimension of infinite consciousness, aware that ‘there is nothing at all’, they enter and remain in the dimension of nothingness. This is the sixth liberation. 

Going\marginnote{35.15} totally beyond the dimension of nothingness, they enter and remain in the dimension of neither perception nor non-perception. This is the seventh liberation. 

Going\marginnote{35.17} totally beyond the dimension of neither perception nor non-perception, they enter and remain in the cessation of perception and feeling.\footnote{The “cessation of perception and feeling” (\textit{\textsanskrit{saññāvedayitanirodha}}) is a culminating meditation state of supreme subtlety that often leads directly to awakening (but see \href{https://suttacentral.net/an5.166/en/sujato}{AN 5.166}). The state itself, like all meditation states, is temporary, but afterwards the defilements can be eliminated forever. This liberating insight is the consequence of the balanced development of all eight factors of the path. } This is the eighth liberation. 

These\marginnote{35.19} are the eight liberations. 

When\marginnote{36.1} a mendicant enters into and withdraws from these eight liberations—in forward order, in reverse order, and in forward and reverse order—wherever they wish, whenever they wish, and for as long as they wish;\footnote{This passage emphasizes that this person is fully adept and has mastered all the states of meditation. The Buddha claimed such mastery (\href{https://suttacentral.net/an9.41/en/sujato\#16.1}{AN 9.41:16.1}), and retained the ability even on his deathbed (\href{https://suttacentral.net/dn16/en/sujato\#6.8.1}{DN 16:6.8.1}). } and when they realize the undefiled freedom of heart and freedom by wisdom in this very life, and live having realized it with their own insight due to the ending of defilements, they’re called a mendicant who is freed both ways.\footnote{Here we see the terms “one who is freed” used in two ways. All arahants have “freedom of heart” (by means of \textit{\textsanskrit{samādhi}}) and “freedom by wisdom” (the realization of the Dhamma). At the same time, one who emphasizes \textit{\textsanskrit{samādhi}} is said to have “freedom of heart” in contrast with one who emphasizes wisdom, who has “freedom by wisdom”. One who has consummate mastery of both \textit{\textsanskrit{samādhi}} and wisdom is said to be “freed both ways”. } And, Ānanda, there is no other freedom both ways that is better or finer than this.” 

That\marginnote{36.4} is what the Buddha said. Satisfied, Venerable Ānanda approved what the Buddha said. 

%
\chapter*{{\suttatitleacronym DN 16}{\suttatitletranslation The Great Discourse on the Buddha’s Extinguishment }{\suttatitleroot Mahāparinibbānasutta}}
\addcontentsline{toc}{chapter}{\tocacronym{DN 16} \toctranslation{The Great Discourse on the Buddha’s Extinguishment } \tocroot{Mahāparinibbānasutta}}
\markboth{The Great Discourse on the Buddha’s Extinguishment }{Mahāparinibbānasutta}
\extramarks{DN 16}{DN 16}

\scevam{So\marginnote{1.1.1} I have heard.\footnote{The longest of all early discourses, this dramatic and moving narrative tells the story of the Buddha’s slow journey towards his final passing. } }At one time the Buddha was staying near \textsanskrit{Rājagaha}, on the Vulture’s Peak Mountain. Now at that time King \textsanskrit{Ajātasattu} of Magadha, son of the princess of Videha, wanted to invade the Vajjis.\footnote{After the events of \href{https://suttacentral.net/dn2/en/sujato}{DN 2}, \textsanskrit{Ajātasattu} retained and consolidated his power. Magadha ultimately conquered the Vajji Federation and continued to expand until almost all of India was under its sway. } He declared: “I shall wipe out these Vajjis, so mighty and powerful! I shall destroy them, and lay ruin and devastation upon them!”\footnote{According to the commentary, \textsanskrit{Ajātasattu}’s anger was rooted in a dispute on trade routes. Control of shipping on the Ganges was essential for establishing international trade. There was a port on the Ganges extending over a league, split half and half between Magadha and Vajji. Valuable products were sourced from a mountain and brought down for trade, but the Vajjis kept absconding with the whole lot. The place is not identified, but Munger, a strategic port east of Patna, fits the description. The nearby hills have been mined since paleolithic times. It is also likely, as maintained in Jain tradition, that the Vajjis disputed \textsanskrit{Ajātasattu}’s accession after committing regicide. His threatening posture towards the Vajjis is also mentioned in \href{https://suttacentral.net/sn20.8/en/sujato\#2.2}{SN 20.8:2.2}. } 

And\marginnote{1.2.1} then King \textsanskrit{Ajātasattu} addressed \textsanskrit{Vassakāra} the brahmin minister of Magadha,\footnote{\textsanskrit{Vassakāra} appears in the suttas as a devoted follower of the Buddha. } “Please, brahmin, go to the Buddha, and in my name bow with your head to his feet. Ask him if he is healthy and well, nimble, strong, and living comfortably. And then say: ‘Sir, King \textsanskrit{Ajātasattu} of Magadha, son of the princess of Videha, wants to invade the Vajjis. He says, “I shall wipe out these Vajjis, so mighty and powerful! I shall destroy them, and lay ruin and devastation upon them!”’\footnote{The Vajji Federation harks back to an early settlement founded by the legendary \textsanskrit{Nābhānediṣṭha} in the Vedic period. It built its wealth on its extensive fertile plains and the trading possibilities opened up by the Gandak and Ganges rivers. } Remember well how the Buddha answers and tell it to me. For Realized Ones say nothing that is not so.”\footnote{While it may seem strange to consult the Buddha on such a violent plan, \textsanskrit{Ajātasattu} knows from his experience in DN 2 that the Buddha will not hesitate to tell him the truth, even if it is bad news. It seems he is trying to avoid the downfall of tyrants who are surrounded only with yes men. } 

\section*{1. The Brahmin \textsanskrit{Vassakāra} }

“Yes,\marginnote{1.3.1} sir,” \textsanskrit{Vassakāra} replied. He had the finest carriages harnessed. Then he mounted a fine carriage and, along with other fine carriages, set out from \textsanskrit{Rājagaha} for the Vulture’s Peak Mountain.\footnote{Reading \textit{\textsanskrit{yojapetvā}} (“had them harnessed”), which is found in the \textsanskrit{Mahāsaṅgīti} in similar passages, and in this passage in the PTS edition. } He went by carriage as far as the terrain allowed, then descended and approached the Buddha on foot, and exchanged greetings with him. 

When\marginnote{1.3.3} the greetings and polite conversation were over, he sat down to one side and said to the Buddha, “Mister Gotama, King \textsanskrit{Ajātasattu} of Magadha, son of the princess of Videha, bows with his head to your feet. He asks if you are healthy and well, nimble, strong, and living comfortably. Mister Gotama, King \textsanskrit{Ajātasattu} wants to invade the Vajjis. He has declared: ‘I shall wipe out these Vajjis, so mighty and powerful! I shall destroy them, and lay ruin and devastation upon them!’” 

\section*{2. Principles That Prevent Decline }

Now\marginnote{1.4.1} at that time Venerable Ānanda was standing behind the Buddha fanning him.\footnote{Ānanda cares for the Buddha in his old age. Shortly after these events, Ānanda was entrusted with reciting the suttas at the First Council. This discourse would have been composed by him in the years following his Master’s death. More than a simple interlocutor, Ānanda shapes the story as its second lead, a relatable character with an empathetic point of view. He imbues the discourse with his emotional struggles as he deals with the Buddha’s passing and helps ensure the future survival of the Dhamma, while still developing his own meditation practice. } Then the Buddha said to him, “Ānanda, have you heard that the Vajjis meet frequently and have many meetings?”\footnote{As for example at \href{https://suttacentral.net/an8.12/en/sujato\#1.2}{AN 8.12:1.2}. } 

“I\marginnote{1.4.4} have heard that, sir.” 

“As\marginnote{1.4.5} long as the Vajjis meet frequently and have many meetings, they can expect growth, not decline. 

Ānanda,\marginnote{1.4.6} have you heard that the Vajjis meet in harmony, leave in harmony, and carry on their business in harmony?”\footnote{Harmony is especially important as the Vajji Federation was comprised of several different clans, among them the Licchavis of \textsanskrit{Vesālī}, the \textsanskrit{Ñātikas} just south of \textsanskrit{Vesālī}, the Uggas of \textsanskrit{Hatthigāma} (Elephant Village), and the Bhogas of Bhoganagara. Some sources say the Vajjis proper were another tribe within the alliance. The Videhans of \textsanskrit{Mithilā} are also sometimes included, but several sources indicate that they remained an independent, if reduced, kingdom until they were conquered by Magadha some years later. The Mallas of \textsanskrit{Pāvā} and \textsanskrit{Kusinārā} formed a closely allied independent republic (\href{https://suttacentral.net/mn35/en/sujato\#12.8}{MN 35:12.8}). } 

“I\marginnote{1.4.7} have heard that, sir.” 

“As\marginnote{1.4.8} long as the Vajjis meet in harmony, leave in harmony, and carry on their business in harmony, they can expect growth, not decline. 

Ānanda,\marginnote{1.4.9} have you heard that the Vajjis don’t make new decrees or abolish existing decrees, but proceed having undertaken the ancient Vajjian traditions as they have been decreed?”\footnote{The “ancient Vajjian traditions” (\textit{\textsanskrit{porāṇe} vajjidhamme}) would have been established with the founding of the Vajji Federation some centuries earlier, which ensured that the members of the federation would have a voice in the Licchavi-dominated union. Compare with the “ancient traditions of the brahmins” (\textit{\textsanskrit{porāṇā} \textsanskrit{brāhmaṇadhammā}} at \href{https://suttacentral.net/an5.191/en/sujato}{AN 5.191} and \href{https://suttacentral.net/snp2.7/en/sujato}{Snp 2.7}), the falling away from which is said to be the cause of Brahmanical decline. Similarly, the Buddha elsewhere scolds the Vajjis for their indulgence (\href{https://suttacentral.net/an5.143/en/sujato}{AN 5.143}), suggesting that the decline may have already set in. } 

“I\marginnote{1.4.10} have heard that, sir.” 

“As\marginnote{1.4.11} long as the Vajjis don’t make new decrees or abolish existing decrees, but proceed having undertaken the ancient Vajjian traditions as they have been decreed, they can expect growth, not decline. 

Ānanda,\marginnote{1.4.12} have you heard that the Vajjis honor, respect, esteem, and venerate Vajjian elders, and think them worth listening to?” 

“I\marginnote{1.4.13} have heard that, sir.” 

“As\marginnote{1.4.14} long as the Vajjis honor, respect, esteem, and venerate Vajjian elders, and think them worth listening to, they can expect growth, not decline. 

Ānanda,\marginnote{1.4.15} have you heard that the Vajjis don’t forcibly abduct the women or girls of the clans and make them live with them?”\footnote{This advocates for legal protection for women from sexual violence. | For various legal regulations regarding women and sex, see \textsanskrit{Kauṭilya}’s \textsanskrit{Arthaśāstra} 12. } 

“I\marginnote{1.4.16} have heard that, sir.” 

“As\marginnote{1.4.17} long as the Vajjis don’t forcibly abduct the women or girls of the clans and make them live with them, they can expect growth, not decline. 

Ānanda,\marginnote{1.4.18} have you heard that the Vajjis honor, respect, esteem, and venerate the Vajjian shrines, whether inner or outer, not neglecting the proper spirit-offerings that were given and made in the past?”\footnote{The Buddha will stay at several of these shrines later in this discourse. They were sacred groves, maintained by the people in reverence for the powerful spirits of nature. | “Spirit-offerings” is \textit{bali}. | The commentary explains “inner or outer” as inside or outside the town. } 

“I\marginnote{1.4.19} have heard that, sir.” 

“As\marginnote{1.4.20} long as the Vajjis honor, respect, esteem, and venerate the Vajjian shrines, whether inner or outer, not neglecting the proper spirit-offerings that were given and made in the past, they can expect growth, not decline. 

Ānanda,\marginnote{1.4.21} have you heard that the Vajjis organize proper protection, shelter, and security for perfected ones, so that more perfected ones might come to the realm and those already here may live in comfort?”\footnote{It is an old Indian belief that the presence of holy persons provides a kind of umbrella effect that protects the realm. } 

“I\marginnote{1.4.22} have heard that, sir.” 

“As\marginnote{1.4.23} long as the Vajjis organize proper protection, shelter, and security for perfected ones, so that more perfected ones might come to the realm and those already here may live in comfort, they can expect growth, not decline.” 

Then\marginnote{1.5.1} the Buddha said to \textsanskrit{Vassakāra}, “Brahmin, this one time I was staying near \textsanskrit{Vesālī} at the \textsanskrit{Sārandada} woodland shrine.\footnote{This event is recorded at \href{https://suttacentral.net/an7.21/en/sujato}{AN 7.21}, where the Buddha addresses the Licchavis. That chapter of the \textsanskrit{Aṅguttara} mostly consists of the teachings found here presented as separate suttas. | The \textsanskrit{Sārandada} shrine was short walk from \textsanskrit{Vesālī} (\href{https://suttacentral.net/an5.143/en/sujato}{AN 5.143}). } There I taught the Vajjis these seven principles that prevent decline.\footnote{There is a certain tension here: the Buddha taught these principles to the Vajjis, one of which is that the Vajjis should not adopt new decrees. It seems that a “decree” (\textit{\textsanskrit{paññatta}}) is more like constitutional law, whereas a “principle” (\textit{dhamma}) is more like a behavioral guideline. } As long as these seven principles that prevent decline last among the Vajjis, and as long as the Vajjis are seen following them, they can expect growth, not decline.” 

When\marginnote{1.5.5} the Buddha had spoken, \textsanskrit{Vassakāra} said to him, “Mister Gotama, if the Vajjis follow even a single one of these principles they can expect growth, not decline. How much more so all seven! King \textsanskrit{Ajātasattu} cannot defeat the Vajjis in war, unless by bribery or by sowing dissension.\footnote{\textit{\textsanskrit{Upalāpana}} is used a number of times in the Vinaya, where it always has the sense of giving someone something to get them to do what you want. It has the same sense at \href{https://suttacentral.net/sn3.25/en/sujato\#4.13}{SN 3.25:4.13}, where an enemy king may be bribed with gold from the royal treasury. | The commentary says that \textsanskrit{Vassakāra} himself was assigned with the task of weakening the Vajjis in this way. } Well, now, Mister Gotama, I must go. I have many duties, and much to do.” 

“Please,\marginnote{1.5.10} brahmin, go at your convenience.” Then \textsanskrit{Vassakāra} the brahmin, having approved and agreed with what the Buddha said, got up from his seat and left. 

\section*{3. Principles That Prevent Decline Among the Mendicants }

Soon\marginnote{1.6.1} after he had left, the Buddha said to Ānanda, “Go, Ānanda, gather all the mendicants staying in the vicinity of \textsanskrit{Rājagaha} together in the assembly hall.”\footnote{There were several monasteries and hermitages around \textsanskrit{Rājagaha} (\href{https://suttacentral.net/pli-tv-kd2/en/sujato\#11.1.1}{Kd 2:11.1.1}). } 

“Yes,\marginnote{1.6.3} sir,” replied Ānanda. He did what the Buddha asked. Then he went back, bowed, stood to one side, and said to him, “Sir, the mendicant \textsanskrit{Saṅgha} has assembled. Please, sir, go at your convenience.” 

Then\marginnote{1.6.5} the Buddha went to the assembly hall, where he sat on the seat spread out and addressed the mendicants: “Mendicants, I will teach you these seven principles that prevent decline.\footnote{Also at \href{https://suttacentral.net/an7.23/en/sujato}{AN 7.23}. } Listen and apply your mind well, I will speak.” 

“Yes,\marginnote{1.6.9} sir,” they replied. The Buddha said this: 

“As\marginnote{1.6.11} long as the mendicants meet frequently and have many meetings, they can expect growth, not decline.\footnote{This especially refers to the fortnightly \textit{uposatha}. Despite the dangers faced in the rugged hills around \textsanskrit{Rājagaha} (\href{https://suttacentral.net/pli-tv-kd2/en/sujato\#12.1.1}{Kd 2:12.1.1}), the Buddha insisted that all monks of the locality attended the \textit{uposatha} (\href{https://suttacentral.net/pli-tv-kd2/en/sujato\#5.3.1}{Kd 2:5.3.1}). } 

As\marginnote{1.6.12} long as the mendicants meet in harmony, leave in harmony, and carry on their business in harmony, they can expect growth, not decline.\footnote{The Buddha encouraged the mendicants to recite the Dhamma in harmony (\href{https://suttacentral.net/mn103/en/sujato\#3.2}{MN 103:3.2}), to resolve issues in harmony (\href{https://suttacentral.net/mn104/en/sujato\#14.3}{MN 104:14.3}), and on the \textit{uposatha} to recite and train in the \textit{\textsanskrit{pātimokkha}} in harmony (\href{https://suttacentral.net/pli-tv-bu-vb-as7/en/sujato\#4.11}{Bu As 7:4.11}). } 

As\marginnote{1.6.13} long as the mendicants don’t make new decrees or abolish existing decrees, but undertake and follow the training rules as they have been decreed, they can expect growth, not decline.\footnote{Picking up from the similar injunction to the Vajjis, this kicks off a long narrative arc that binds together the Buddha’s invitation to abolish the minor rules (\href{https://suttacentral.net/dn16/en/sujato\#6.3.1}{DN 16:6.3.1}) with the bad monk Subhadda’s seizing on the Buddha’s death as an excuse to give up the rules, which was the direct motivation for the First Council (\href{https://suttacentral.net/dn16/en/sujato\#6.20.3}{DN 16:6.20.3} = \href{https://suttacentral.net/pli-tv-kd21/en/sujato\#1.1.24}{Kd 21:1.1.24}), at the end of which the Sangha agreed not to abolish any rules (\href{https://suttacentral.net/pli-tv-kd21/en/sujato\#1.9.20}{Kd 21:1.9.20}), a decision that was affirmed a century later at the Second Council (\href{https://suttacentral.net/pli-tv-kd22/en/sujato}{Kd 22}). } 

As\marginnote{1.6.14} long as the mendicants honor, respect, esteem, and venerate the senior mendicants—of long standing, long gone forth, fathers and leaders of the \textsanskrit{Saṅgha}—and think them worth listening to, they can expect growth, not decline.\footnote{The nature of seniority in the \textsanskrit{Saṅgha} is often misunderstood. There is no “hierarchy” (literally “rule of priests”) in the sense of power-based relationships: no monastic has the authority to command another monk or nun. Seniority is owed respect, not obedience. } 

As\marginnote{1.6.15} long as the mendicants don’t fall under the sway of arisen craving for future lives, they can expect growth, not decline.\footnote{This stands in place of the injunction against sexual violence, both principles being concerned with the harmful effects of desire. } 

As\marginnote{1.6.16} long as the mendicants take care to live in wilderness lodgings, they can expect growth, not decline.\footnote{This stands in place of the injunction to maintain the shrines, where mendicants would frequently stay. } 

As\marginnote{1.6.17} long as the mendicants individually establish mindfulness, so that more good-hearted spiritual companions might come, and those that have already come may live comfortably, they can expect growth, not decline.\footnote{This stands in place of the injunction to look after arahants. } 

As\marginnote{1.6.18} long as these seven principles that prevent decline last among the mendicants, and as long as the mendicants are seen following them, they can expect growth, not decline. 

I\marginnote{1.7.1} will teach you seven more principles that prevent decline. …\footnote{Also at \href{https://suttacentral.net/an7.24/en/sujato}{AN 7.24}. } 

As\marginnote{1.7.4} long as the mendicants don’t relish work, loving it and liking to relish it, they can expect growth, not decline.\footnote{“Work” is \textit{kamma}, which especially means “building work”. Of course it is essential to do work, but one should not get too caught up in it. The Buddha did not believe that working hard was ennobling in and of itself. } 

As\marginnote{1.7.5} long as they don’t relish talk … 

sleep\marginnote{1.7.6} … 

company\marginnote{1.7.7} … 

they\marginnote{1.7.8} don’t have corrupt wishes, falling under the sway of corrupt wishes …\footnote{“Corrupt wishes” (\textit{\textsanskrit{pāpicchā}}) is defined as when a faithless person wishes to be known as faithful, or person otherwise lacking good qualities wishes to be known as having them (\href{https://suttacentral.net/an10.23/en/sujato\#6.2}{AN 10.23:6.2}). } 

they\marginnote{1.7.9} don’t have bad friends, companions, and associates … 

they\marginnote{1.7.10} don’t stop half-way after achieving some insignificant distinction, they can expect growth, not decline.\footnote{This item and the preceding two were the downfall of Devadatta (\href{https://suttacentral.net/iti89/en/sujato}{Iti 89}). } 

As\marginnote{1.8.1} long as these seven principles that prevent decline last among the mendicants, and as long as the mendicants are seen following them, they can expect growth, not decline. 

I\marginnote{1.8.2} will teach you seven more principles that prevent decline. …\footnote{Also at \href{https://suttacentral.net/an7.25/en/sujato}{AN 7.25}. } As long as the mendicants are faithful … conscientious … prudent … learned … energetic … mindful … wise, they can expect growth, not decline. As long as these seven principles that prevent decline last among the mendicants, and as long as the mendicants are seen following them, they can expect growth, not decline. 

I\marginnote{1.8.11} will teach you seven more principles that prevent decline. …\footnote{Also at \href{https://suttacentral.net/an7.26/en/sujato}{AN 7.26}. } 

As\marginnote{1.9.1} long as the mendicants develop the awakening factors of mindfulness … investigation of principles … energy … rapture … tranquility … immersion … equanimity, they can expect growth, not decline. 

As\marginnote{1.10.1} long as these seven principles that prevent decline last among the mendicants, and as long as the mendicants are seen following them, they can expect growth, not decline. 

I\marginnote{1.10.2} will teach you seven more principles that prevent decline. …\footnote{As at \href{https://suttacentral.net/an7.27/en/sujato}{AN 7.27}. These “perceptions” are all meditation practices which are described at \href{https://suttacentral.net/an10.60/en/sujato}{AN 10.60}. } 

As\marginnote{1.10.5} long as the mendicants develop the perceptions of impermanence … not-self … ugliness … drawbacks … giving up … fading away … cessation, they can expect growth, not decline. 

As\marginnote{1.10.12} long as these seven principles that prevent decline last among the mendicants, and as long as the mendicants are seen following them, they can expect growth, not decline. 

I\marginnote{1.11.1} will teach you six principles that prevent decline. …\footnote{These six principles are found in several places, but they are not elsewhere called “principles that prevent decline”; for example at \href{https://suttacentral.net/an6.11/en/sujato}{AN 6.11} they are called “warm-hearted qualities” (\textit{\textsanskrit{dhammā} \textsanskrit{sāraṇīyā}}). At \href{https://suttacentral.net/an6.22/en/sujato}{AN 6.22}, however, a different six qualities are called “principles that prevent decline”: not relishing work, talk, sleep, and company, being easy to admonish, and having good friends. } 

As\marginnote{1.11.4} long as the mendicants consistently treat their spiritual companions with bodily kindness … verbal kindness … and mental kindness both in public and in private, they can expect growth, not decline. 

As\marginnote{1.11.7} long as the mendicants share without reservation any material things they have gained by legitimate means, even the food placed in the alms-bowl, using them in common with their ethical spiritual companions, they can expect growth, not decline. 

As\marginnote{1.11.8} long as the mendicants live according to the precepts shared with their spiritual companions, both in public and in private—such precepts as are intact, impeccable, spotless, and unmarred, liberating, praised by sensible people, not mistaken, and leading to immersion—they can expect growth, not decline. 

As\marginnote{1.11.9} long as the mendicants live according to the view shared with their spiritual companions, both in public and in private—the view that is noble and emancipating, and delivers one who practices it to the complete end of suffering—they can expect growth, not decline.\footnote{This is the right view of the noble eightfold path, in other words, the four noble truths. } 

As\marginnote{1.11.10} long as these six principles that prevent decline last among the mendicants, and as long as the mendicants are seen following them, they can expect growth, not decline.” 

And\marginnote{1.12.1} while staying there at the Vulture’s Peak the Buddha often gave this Dhamma talk to the mendicants:\footnote{This epitome of the Dhamma is repeated eight times in this discourse, summarizing the teachings in the Gradual Training. It is not found in this exact form elsewhere. } 

“Such\marginnote{1.12.2} is ethics, such is immersion, such is wisdom. When immersion is imbued with ethics it’s very fruitful and beneficial.\footnote{“Imbued” (\textit{\textsanskrit{paribhāvita}}) as a mother hen imbues her eggs with warmth by sitting on them (\href{https://suttacentral.net/an7.71/en/sujato\#2.2}{AN 7.71:2.2}). See also \href{https://suttacentral.net/sn55.21/en/sujato\#2.3}{SN 55.21:2.3}. } When wisdom is imbued with immersion it’s very fruitful and beneficial. When the mind is imbued with wisdom it is rightly freed from the defilements, namely, the defilements of sensuality, desire to be reborn, and ignorance.”\footnote{Some editions add \textit{\textsanskrit{diṭṭhāsava}} (“defilement of views”), but since that appears to be a late interpolation I follow the \textsanskrit{Mahāsaṅgīti} in omitting it. } 

When\marginnote{1.13.1} the Buddha had stayed in \textsanskrit{Rājagaha} as long as he pleased, he addressed Venerable Ānanda, “Come, Ānanda, let’s go to \textsanskrit{Ambalaṭṭhikā}.”\footnote{This is the same rest-house at which the Buddha sojourned on his way from \textsanskrit{Rājagaha} to \textsanskrit{Nāḷandā} in the \textsanskrit{Brahmajālasutta} (\href{https://suttacentral.net/dn1/en/sujato}{DN 1}). } 

“Yes,\marginnote{1.13.3} sir,” Ānanda replied. Then the Buddha together with a large \textsanskrit{Saṅgha} of mendicants arrived at \textsanskrit{Ambalaṭṭhikā}, where he stayed in the royal rest-house. And while staying there, too, he often gave this Dhamma talk to the mendicants: 

“Such\marginnote{1.14.3} is ethics, such is immersion, such is wisdom. When immersion is imbued with ethics it’s very fruitful and beneficial. When wisdom is imbued with immersion it’s very fruitful and beneficial. When the mind is imbued with wisdom it is rightly freed from the defilements, namely, the defilements of sensuality, desire to be reborn, and ignorance.” 

When\marginnote{1.15.1} the Buddha had stayed in \textsanskrit{Ambalaṭṭhikā} as long as he pleased, he addressed Venerable Ānanda, “Come, Ānanda, let’s go to \textsanskrit{Nāḷandā}.” 

“Yes,\marginnote{1.15.3} sir,” Ānanda replied. Then the Buddha together with a large \textsanskrit{Saṅgha} of mendicants arrived at \textsanskrit{Nāḷandā}, where he stayed in \textsanskrit{Pāvārika}’s mango grove. 

\section*{4. \textsanskrit{Sāriputta}’s Lion’s Roar }

Then\marginnote{1.16.1} \textsanskrit{Sāriputta} went up to the Buddha, bowed, sat down to one side, and said to him,\footnote{This was apparently the last time that the Buddha met \textsanskrit{Sāriputta}. This encounter is recorded in an independent sutta at \href{https://suttacentral.net/sn47.12/en/sujato}{SN 47.12} and expanded into a long discourse at \href{https://suttacentral.net/dn28/en/sujato}{DN 28}. The \textsanskrit{Saṁyutta} follows this meeting with the record of \textsanskrit{Sāriputta}’s passing at \href{https://suttacentral.net/sn47.13/en/sujato}{SN 47.13}, and then the lament for the absence of both \textsanskrit{Sāriputta} and \textsanskrit{Moggallāna} at \href{https://suttacentral.net/sn47.14 /en/sujato}{SN 47.14 }. For some reason these events were omitted from the present discourse, even though they fit well thematically. } “Sir, I have such confidence in the Buddha that I believe there’s no other ascetic or brahmin—whether past, future, or present—whose direct knowledge is superior to the Buddha when it comes to awakening.” 

“That’s\marginnote{1.16.4} a grand and dramatic statement, \textsanskrit{Sāriputta}. You’ve roared a definitive, categorical lion’s roar, saying: ‘I have such confidence in the Buddha that I believe there’s no other ascetic or brahmin—whether past, future, or present—whose direct knowledge is superior to the Buddha when it comes to awakening.’ 

What\marginnote{1.16.7} about all the perfected ones, the fully awakened Buddhas who lived in the past? Have you comprehended their minds to know that those Buddhas had such ethics, or such qualities, or such wisdom, or such meditation, or such freedom?” 

“No,\marginnote{1.16.9} sir.” 

“And\marginnote{1.16.10} what about all the perfected ones, the fully awakened Buddhas who will live in the future? Have you comprehended their minds to know that those Buddhas will have such ethics, or such qualities, or such wisdom, or such meditation, or such freedom?” 

“No,\marginnote{1.16.12} sir.” 

“And\marginnote{1.16.13} what about me, the perfected one, the fully awakened Buddha at present? Have you comprehended my mind to know that I have such ethics, or such teachings, or such wisdom, or such meditation, or such freedom?” 

“No,\marginnote{1.16.15} sir.” 

“Well\marginnote{1.16.16} then, \textsanskrit{Sāriputta}, given that you don’t comprehend the minds of Buddhas past, future, or present, what exactly are you doing, making such a grand and dramatic statement, roaring such a definitive, categorical lion’s roar?” 

“Sir,\marginnote{1.17.1} though I don’t comprehend the minds of Buddhas past, future, and present, still I understand this by inference from the teaching.\footnote{As at \href{https://suttacentral.net/dn14/en/sujato\#1.37.4}{DN 14:1.37.4}, inference (\textit{anvaya}) is regarded as a valid form of knowledge. The Buddha himself is said to rely on inference using the same simile at \href{https://suttacentral.net/an10.95/en/sujato\#10.4}{AN 10.95:10.4}. } Suppose there was a king’s frontier citadel with fortified embankments, ramparts, and arches, and a single gate. And it has a gatekeeper who is astute, competent, and intelligent. He keeps strangers out and lets known people in. As he walks around the patrol path, he doesn’t see a hole or cleft in the wall, not even one big enough for a cat to slip out. He thinks: ‘Whatever sizable creatures enter or leave the citadel, all of them do so via this gate.’ 

In\marginnote{1.17.8} the same way, I understand this by inference from the teaching: ‘All the perfected ones, fully awakened Buddhas—whether past, future, or present—give up the five hindrances, corruptions of the heart that weaken wisdom. Their mind is firmly established in the four kinds of mindfulness meditation. They correctly develop the seven awakening factors. And they wake up to the supreme perfect awakening.’” 

And\marginnote{1.18.1} while staying at \textsanskrit{Nāḷandā}, too, the Buddha often gave this Dhamma talk to the mendicants: 

“Such\marginnote{1.18.2} is ethics, such is immersion, such is wisdom. When immersion is imbued with ethics it’s very fruitful and beneficial. When wisdom is imbued with immersion it’s very fruitful and beneficial. When the mind is imbued with wisdom it is rightly freed from the defilements, namely, the defilements of sensuality, desire to be reborn, and ignorance.” 

\section*{5. The Drawbacks of Unethical Conduct }

When\marginnote{1.19.1} the Buddha had stayed in \textsanskrit{Nāḷandā} as long as he pleased, he addressed Venerable Ānanda, “Come, Ānanda, let’s go to \textsanskrit{Pāṭali} Village.”\footnote{This is modern Patna. In this account we see how it was developed from the simple riverside village of \textsanskrit{Pāṭaligāma} to the citadel of \textsanskrit{Pāṭaliputta}. \textsanskrit{Ajātasattu} ultimately moved the capital of Magadha from the defensively-postured \textsanskrit{Rājagaha} surrounded by hills to this trading center on open waters. Under Ashoka it became one of the greatest cities of the ancient world. It was named after the \textit{\textsanskrit{pāṭalī}} tree (\emph{Stereospermum chelonoides}), known as \emph{patala} in Hindi. } 

“Yes,\marginnote{1.19.3} sir,” Ānanda replied. Then the Buddha together with a large \textsanskrit{Saṅgha} of mendicants arrived at \textsanskrit{Pāṭali} Village. 

The\marginnote{1.20.1} lay followers of \textsanskrit{Pāṭali} Village heard that he had arrived. So they went to see him, bowed, sat down to one side, and said to him, “Sir, please consent to come to our guest house.” The Buddha consented with silence. 

Then,\marginnote{1.21.1} knowing that the Buddha had consented, the lay followers of \textsanskrit{Pāṭali} Village got up from their seat, bowed, and respectfully circled the Buddha, keeping him on their right. Then they went to the guest house, where they spread carpets all over, prepared seats, set up a water jar, and placed an oil lamp. Then they went back to the Buddha, bowed, stood to one side, and told him of their preparations, saying: “Please, sir, come at your convenience.” 

In\marginnote{1.22.1} the morning, the Buddha robed up and, taking his bowl and robe, went to the guest house together with the \textsanskrit{Saṅgha} of mendicants. Having washed his feet he entered the guest house and sat against the central column facing east. The \textsanskrit{Saṅgha} of mendicants also washed their feet, entered the guest house, and sat against the west wall facing east, with the Buddha right in front of them. The lay followers of \textsanskrit{Pāṭali} Village also washed their feet, entered the guest house, and sat against the east wall facing west, with the Buddha right in front of them. 

Then\marginnote{1.23.1} the Buddha addressed them: 

“Householders,\marginnote{1.23.2} there are these five drawbacks for an unethical person because of their failure in ethics.\footnote{Also at \href{https://suttacentral.net/an5.213/en/sujato}{AN 5.213}, \href{https://suttacentral.net/ud8.6/en/sujato\#4.1}{Ud 8.6:4.1}, \href{https://suttacentral.net/dn33/en/sujato\#2.1.36}{DN 33:2.1.36}, and \href{https://suttacentral.net/pli-tv-kd1/en/sujato\#28.4.1}{Kd 1:28.4.1}. } What five? 

Firstly,\marginnote{1.23.4} an unethical person loses great wealth on account of negligence.\footnote{The Buddha begins with the things most obvious and pertinent to the audience. } This is the first drawback for an unethical person because of their failure in ethics. 

Furthermore,\marginnote{1.23.6} an unethical person gets a bad reputation. This is the second drawback. 

Furthermore,\marginnote{1.23.8} an unethical person enters any kind of assembly timid and embarrassed, whether it’s an assembly of aristocrats, brahmins, householders, or ascetics. This is the third drawback. 

Furthermore,\marginnote{1.23.10} an unethical person feels lost when they die.\footnote{Bewildered by fear and regret. } This is the fourth drawback. 

Furthermore,\marginnote{1.23.12} an unethical person, when their body breaks up, after death, is reborn in a place of loss, a bad place, the underworld, hell. This is the fifth drawback. 

These\marginnote{1.23.14} are the five drawbacks for an unethical person because of their failure in ethics. 

\section*{6. The Benefits of Ethical Conduct }

There\marginnote{1.24.1} are these five benefits for an ethical person because of their accomplishment in ethics. What five? 

Firstly,\marginnote{1.24.3} an ethical person gains great wealth on account of diligence. This is the first benefit. 

Furthermore,\marginnote{1.24.5} an ethical person gets a good reputation. This is the second benefit. 

Furthermore,\marginnote{1.24.7} an ethical person enters any kind of assembly bold and self-assured, whether it’s an assembly of aristocrats, brahmins, householders, or ascetics. This is the third benefit. 

Furthermore,\marginnote{1.24.9} an ethical person dies not feeling lost. This is the fourth benefit. 

Furthermore,\marginnote{1.24.11} when an ethical person’s body breaks up, after death, they’re reborn in a good place, a heavenly realm. This is the fifth benefit. 

These\marginnote{1.24.13} are the five benefits for an ethical person because of their accomplishment in ethics.” 

The\marginnote{1.25.1} Buddha spent much of the night educating, encouraging, firing up, and inspiring the lay followers of \textsanskrit{Pāṭali} Village with a Dhamma talk. Then he dismissed them, “The night is getting late, householders. Please go at your convenience.” 

“Yes,\marginnote{1.25.3} sir,” replied the lay followers of \textsanskrit{Pāṭali} Village. They got up from their seat, bowed, and respectfully circled the Buddha, keeping him on their right, before leaving. Soon after they left the Buddha entered a private cubicle.\footnote{\textit{\textsanskrit{Suññāgāra}} normally means an “empty dwelling”, but here the commentary describes it as a curtained-off cubicle. } 

\section*{7. Building a Citadel }

Now\marginnote{1.26.1} at that time the Magadhan ministers Sunidha and \textsanskrit{Vassakāra} were building a citadel at \textsanskrit{Pāṭali} Village to keep the Vajjis out.\footnote{\textsanskrit{Vassakāra} was no dawdler, as he preceded the Buddha to \textsanskrit{Pāṭaligāma}. And not long after, he returned to \textsanskrit{Rājagaha} to build more fortifications there (\href{https://suttacentral.net/mn108/en/sujato\#6.2}{MN 108:6.2}). | \textit{Nagara} here means “citadel, fortress”, not “city”. It has the same sense as \textit{pura}. The construction is defensive and may well have begun before \textsanskrit{Ajātasattu} began thinking of invading. } At that time thousands of deities were taking possession of building sites in \textsanskrit{Pāṭali} Village.\footnote{The relation between deities and building sites (\textit{vatthu}) appears only in this passage in the early texts (repeated at \href{https://suttacentral.net/ud8.6/en/sujato\#15.2}{Ud 8.6:15.2} and \href{https://suttacentral.net/pli-tv-kd1/en/sujato\#28.7.3}{Kd 1:28.7.3}). Divination and geomancy (\textit{\textsanskrit{vatthuvijjā}}) for building sites is said to be a wrong livelihood at \href{https://suttacentral.net/dn1/en/sujato\#1.21.2}{DN 1:1.21.2}. } Illustrious rulers or royal ministers inclined to build houses at sites possessed by illustrious deities. Middling rulers or royal ministers inclined to build houses at sites possessed by middling deities. Lesser rulers or royal ministers inclined to build houses at sites possessed by lesser deities. 

With\marginnote{1.27.3} clairvoyance that is purified and superhuman, the Buddha saw those deities taking possession of building sites in \textsanskrit{Pāṭali} Village. The Buddha rose at the crack of dawn and addressed Ānanda, “Ānanda, who is building a citadel at \textsanskrit{Pāṭali} Village?” 

“Sir,\marginnote{1.27.6} the Magadhan ministers Sunidha and \textsanskrit{Vassakāra} are building a citadel to keep the Vajjis out.” 

“It’s\marginnote{1.28.1} as if they were building the citadel in consultation with the gods of the thirty-three. With clairvoyance that is purified and superhuman, I saw those deities taking possession of building sites. Illustrious rulers or royal ministers inclined to build houses at sites possessed by illustrious deities. Middling rulers or royal ministers inclined to build houses at sites possessed by middling deities. Lesser rulers or royal ministers inclined to build houses at sites possessed by lesser deities. As far as the civilized region extends, as far as the trading zone extends, this will be the chief city: the \textsanskrit{Pāṭaliputta} trade center.\footnote{From this point, later texts always refer to \textsanskrit{Pāṭaliputta}, but the reason for the change of name from \textsanskrit{Pāṭaligāma} is not explained in the commentary. \textit{Putta} is a suffix indicating the people of a certain clan or place, so \textit{\textsanskrit{pāṭaliputta}} means “a person from \textsanskrit{Pāṭalī} (village)” or as we might say, “\textsanskrit{Pāṭalian}”. As the city became well known and its identity as a village receded, it must have become known as the “city of the \textsanskrit{Pāṭaliputtas}” and hence simply \textsanskrit{Pāṭaliputta}. Compare the uses of \textit{\textsanskrit{ñātika}} below. | The commentary explains that \textit{\textsanskrit{puṭabhedanaṁ}} refers to the “opening of packages”, signifying that it was a center of trade. This usage is attested as late as the 13th century Jain \textsanskrit{Vividhatīrthakalpa} of \textsanskrit{Jinaprabhasūri}. } But \textsanskrit{Pāṭaliputta} will face three threats: from fire, flood, and dissension.”\footnote{Excavations reveal that \textsanskrit{Pāṭaliputta} was constructed of wood, so the danger of fire was real. It is situated on a low-lying flood plain of the Ganges, so flooding would also have been an obvious danger. As for dissension (\textit{mithubheda}), \textsanskrit{Vassakāra} had indicated that this was a means of bringing down the Vajjis, so it would seem that kamma was lying in wait for them. } 

Then\marginnote{1.29.1} the Magadhan ministers Sunidha and \textsanskrit{Vassakāra} approached the Buddha, and exchanged greetings with him. When the greetings and polite conversation were over, they stood to one side and said, “Would Mister Gotama together with the mendicant \textsanskrit{Saṅgha} please accept today’s meal from me?” The Buddha consented with silence. 

Then,\marginnote{1.30.1} knowing that the Buddha had consented, they went to their own guest house, where they had delicious fresh and cooked foods prepared. Then they had the Buddha informed of the time, saying, “It’s time, Mister Gotama, the meal is ready.” 

Then\marginnote{1.30.3} the Buddha robed up in the morning and, taking his bowl and robe, went to their guest house together with the mendicant \textsanskrit{Saṅgha}, where he sat on the seat spread out. Then Sunidha and \textsanskrit{Vassakāra} served and satisfied the mendicant \textsanskrit{Saṅgha} headed by the Buddha with their own hands with delicious fresh and cooked foods. When the Buddha had eaten and washed his hand and bowl, Sunidha and \textsanskrit{Vassakāra} took a low seat and sat to one side. 

The\marginnote{1.31.1} Buddha expressed his appreciation with these verses:\footnote{This is the \textit{\textsanskrit{anumodanā}} recited for the meal offering. Other examples are found at \href{https://suttacentral.net/snp3.7/en/sujato\#34.3}{Snp 3.7:34.3} = \href{https://suttacentral.net/mn92/en/sujato\#25.6}{MN 92:25.6}, \href{https://suttacentral.net/sn55.26/en/sujato\#20.1}{SN 55.26:20.1}, \href{https://suttacentral.net/pli-tv-kd1/en/sujato\#15.14.4}{Kd 1:15.14.4}, and \href{https://suttacentral.net/pli-tv-kd1/en/sujato\#1.5.1}{Kd 1:1.5.1}. | It is worth noting that no early \textit{\textsanskrit{anomodanā}} uses the imperative verb form \textit{-tu} signifying giving a blessing (eg. \textit{bhavatu \textsanskrit{sabbamaṅgalaṁ}}, “may all blessings be”). They strictly use the indicative \textit{-ti} to teach cause and effect: if you do this, that happens. } 

\begin{verse}%
“In\marginnote{1.31.2} the place he makes his dwelling, \\
having fed the astute \\
and the virtuous here, \\
the restrained spiritual practitioners, 

he\marginnote{1.31.6} should dedicate an offering\footnote{Buddhism promotes good neighborliness with all beings, seen and unseen. Making an offering to the local spirits helps create a positive and healthy sense of place. } \\
to the deities there. \\
Venerated, they venerate him; \\
honored, they honor him. 

After\marginnote{1.31.10} that they have sympathy for him, \\
like a mother for the child at her breast. \\
A man beloved of the deities \\
always sees nice things.”\footnote{At \href{https://suttacentral.net/an4.67/en/sujato\#8.3}{AN 4.67:8.3} a similar sentiment is expressed for animals. } 

%
\end{verse}

When\marginnote{1.31.14} the Buddha had expressed his appreciation to Sunidha and \textsanskrit{Vassakāra} with these verses, he got up from his seat and left. 

Sunidha\marginnote{1.32.1} and \textsanskrit{Vassakāra} followed behind the Buddha, thinking, “The gate through which the ascetic Gotama departs today shall be named the Gotama Gate. The ford at which he crosses the Ganges River shall be named the Gotama Ford.”\footnote{The “ford” (\textit{tittha}) is symbolic; those who forge a path to salvation are called \textit{titthakara}, such as the six leading ascetics of \href{https://suttacentral.net/dn2/en/sujato}{DN 2}. Today there is a 5.7 km bridge at this crossing, which is appropriately called the Mahatma Gandhi Setu. } 

Then\marginnote{1.32.4} the gate through which the Buddha departed was named the Gotama Gate. 

Then\marginnote{1.32.5} the Buddha came to the Ganges River. 

Now\marginnote{1.33.1} at that time the Ganges was full to the brim so a crow could drink from it. Wanting to cross from the near to the far shore, some people were seeking a boat, some a dinghy, while some were tying up a raft.\footnote{“Dinghy” translates \textit{\textsanskrit{uḷumpa}}. } But, as easily as a strong person would extend or contract their arm, the Buddha, together with the mendicant \textsanskrit{Saṅgha}, vanished from the near shore and landed on the far shore.\footnote{In this stock phrase, we sometimes find \textit{\textsanskrit{paccuṭṭhāsi}} (“landed”) and sometimes \textit{\textsanskrit{pāturahosi}} (“reappeared”). } 

He\marginnote{1.34.1} saw all those people wanting to cross over. Knowing the meaning of this, on that occasion the Buddha expressed this heartfelt sentiment: 

\begin{verse}%
“Those\marginnote{1.34.3} who cross a deluge or stream\footnote{\textit{\textsanskrit{Aṇṇavaṁ}} cannot mean “sea” here, since it is to be crossed with a bridge. } \\
have built a bridge and left the marshes behind. \\
While some people are still tying a raft, \\
intelligent people have crossed over.” 

%
\end{verse}

\scendsection{The first recitation section. }

\section*{8. Talk on the Noble Truths }

Then\marginnote{2.1.1} the Buddha said to Venerable Ānanda, “Come, Ānanda, let’s go to the village of \textsanskrit{Koṭi}.”\footnote{\textsanskrit{Koṭigāma} lay a short distance from the Ganges. The Buddha’s stay here must therefore have followed the events of \href{https://suttacentral.net/sn47.14/en/sujato}{SN 47.14} at \textsanskrit{Ukkacelā} on the Vajjian bank of the Ganges, where he laments the passing of \textsanskrit{Sāriputta} and \textsanskrit{Moggallāna}. He would have heard of \textsanskrit{Sāriputta}’s passing before then, while on the road from \textsanskrit{Nāḷandā} to \textsanskrit{Pāṭaligāma}, even though that is implausibly said to have happened at \textsanskrit{Sāvatthī} (\href{https://suttacentral.net/sn47.13/en/sujato}{SN 47.13}). } 

“Yes,\marginnote{2.1.3} sir,” Ānanda replied. Then the Buddha together with a large \textsanskrit{Saṅgha} of mendicants arrived at the village of \textsanskrit{Koṭi}, and stayed there. 

There\marginnote{2.2.1} he addressed the mendicants:\footnote{Also at \href{https://suttacentral.net/sn56.21/en/sujato}{SN 56.21}. } 

“Mendicants,\marginnote{2.2.2} due to not understanding and not penetrating four noble truths, both you and I have wandered and transmigrated for such a very long time. What four? The noble truths of suffering, the origin of suffering, the cessation of suffering, and the practice that leads to the cessation of suffering. These noble truths of suffering, origin, cessation, and the path have been understood and comprehended. Craving for continued existence has been cut off; the conduit to rebirth is ended; now there’ll be no more future lives.” 

That\marginnote{2.3.1} is what the Buddha said. Then the Holy One, the Teacher, went on to say: 

\begin{verse}%
“Because\marginnote{2.3.3} of not truly seeing \\
the four noble truths, \\
we have transmigrated for a long time \\
from one rebirth to the next. 

But\marginnote{2.3.7} now that these truths have been seen, \\
the conduit to rebirth is eradicated. \\
The root of suffering is cut off, \\
now there’ll be no more future lives.” 

%
\end{verse}

And\marginnote{2.4.1} while staying at the village of \textsanskrit{Koṭi}, too, the Buddha often gave this Dhamma talk to the mendicants: 

“Such\marginnote{2.4.2} is ethics, such is immersion, such is wisdom. When immersion is imbued with ethics it’s very fruitful and beneficial. When wisdom is imbued with immersion it’s very fruitful and beneficial. When the mind is imbued with wisdom it is rightly freed from the defilements, namely, the defilements of sensuality, desire to be reborn, and ignorance.” 

\section*{9. The Deaths in \textsanskrit{Ñātika} }

When\marginnote{2.5.1} the Buddha had stayed in the village of \textsanskrit{Koṭi} as long as he pleased, he said to Ānanda, “Come, Ānanda, let’s go to the land of the \textsanskrit{Ñātikas}.”\footnote{\textsanskrit{Ñātika} (also spelled \textit{\textsanskrit{nātika}}, or \textit{\textsanskrit{nādika}}; Sanskrit \textit{\textsanskrit{jñātṛka}}; \textsanskrit{Prākrit} \textit{\textsanskrit{nāyika}}) was the clan to which the Jain leader \textsanskrit{Mahāvīra} (called \textsanskrit{Nāṭaputta}, i.e. \textsanskrit{Ñātiputta}) belonged. Here it is unusually spelled in plural, which means “the land of the \textsanskrit{Ñātika} clan”. } 

“Yes,\marginnote{2.5.3} sir,” Ānanda replied. Then the Buddha together with a large \textsanskrit{Saṅgha} of mendicants arrived in the land of the \textsanskrit{Ñātikas}, where he stayed in the brick house at \textsanskrit{Ñātika}.\footnote{Here \textsanskrit{Ñātika} is in singular and appears to be the name of a town. Thus \textsanskrit{Ñātika} is first of all a name of a clan, then the land they live in, then their chief town. | Over a millennium previously, the Indus Valley Civilization had built cities of fired brick with standardized size and construction methods, but in the Buddha’s day most buildings were wood. This is the only brick building mentioned in the suttas; brick construction is also discussed in the Vinaya. } 

Then\marginnote{2.6.1} Venerable Ānanda went up to the Buddha, bowed, sat down to one side, and said to him, “Sir, the monk named \textsanskrit{Sāḷha} has passed away in \textsanskrit{Ñātika}. Where has he been reborn in his next life?\footnote{It is unprecedented to discuss the spiritual destinies of an entire town like this. Normally this question is only asked when specific individuals known to the Buddha have passed away. Likely the passage was created in the wake of the passing of \textsanskrit{Mahāvīra} to show that even his own people were enthusiastic Buddhists. This portion of the text is also found at \href{https://suttacentral.net/sn55.8/en/sujato}{SN 55.8}. } The nun named \textsanskrit{Nandā}, the layman named Sudatta, and the laywoman named \textsanskrit{Sujātā} have passed away in \textsanskrit{Ñātika}. Where have they been reborn in the next life?\footnote{\href{https://suttacentral.net/sn55.9/en/sujato}{SN 55.9} inserts another series on a monk, nun, layman, and laywoman all called Asoka. } The laymen named \textsanskrit{Kakkaṭa},\footnote{This portion of the text is at \href{https://suttacentral.net/sn55.10/en/sujato}{SN 55.10}. I spell names consistently with there. } \textsanskrit{Kaḷibha}, Nikata, \textsanskrit{Kaṭissaha}, \textsanskrit{Tuṭṭha}, \textsanskrit{Santuṭṭha}, Bhadda, and Subhadda have passed away in \textsanskrit{Ñātika}. Where have they been reborn in the next life?” 

“Ānanda,\marginnote{2.7.1} the monk \textsanskrit{Sāḷha} had realized the undefiled freedom of heart and freedom by wisdom in this very life, having realized it with his own insight due to the ending of defilements. 

The\marginnote{2.7.2} nun \textsanskrit{Nandā} had ended the five lower fetters. She’s been reborn spontaneously, and will be extinguished there, not liable to return from that world. 

The\marginnote{2.7.3} layman Sudatta had ended three fetters, and weakened greed, hate, and delusion. He’s a once-returner; he will come back to this world once only, then make an end of suffering. 

The\marginnote{2.7.4} laywoman \textsanskrit{Sujātā} had ended three fetters. She’s a stream-enterer, not liable to be reborn in the underworld, bound for awakening. 

The\marginnote{2.7.5} laymen \textsanskrit{Kakkaṭa}, \textsanskrit{Kaḷibha}, Nikata, \textsanskrit{Kaṭissaha}, \textsanskrit{Tuṭṭha}, \textsanskrit{Santuṭṭha}, Bhadda, and Subhadda had ended the five lower fetters. They’ve been reborn spontaneously, and will be extinguished there, not liable to return from that world.\footnote{The suttas consistently depict lay folk as attaining the third stage of awakening, non-return, but not the highest stage of arahantship, except in rare cases on the deathbed. The tradition extended this by saying that if a lay person becomes an arahant, they must ordain or die that day, but this is not supported in the early texts. The point is not whether you receive an ordination, but whether you let go of attachments. } 

Over\marginnote{2.7.13} fifty laymen in \textsanskrit{Ñātika} have passed away having ended the five lower fetters. They’ve been reborn spontaneously, and will be extinguished there, not liable to return from that world. 

More\marginnote{2.7.14} than ninety laymen in \textsanskrit{Ñātika} have passed away having ended three fetters, and weakened greed, hate, and delusion. They’re once-returners, who will come back to this world once only, then make an end of suffering. 

More\marginnote{2.7.15} than five hundred laymen in \textsanskrit{Ñātika} have passed away having ended three fetters. They’re stream-enterers, not liable to be reborn in the underworld, bound for awakening.\footnote{The commentary to the \textsanskrit{Saṁyutta} attributes the large number of deaths to a recent plague (\textit{\textsanskrit{ahivātaroga}}, “viper’s breath disease”). This deadly contagion was known to kill off entire families (\href{https://suttacentral.net/pli-tv-kd1/en/sujato\#50.1.1}{Kd 1:50.1.1}). According to the commentaries, the same disease once afflicted \textsanskrit{Vesālī}, prompting the teaching of the Ratanasutta (\href{https://suttacentral.net/snp2.1/en/sujato}{Snp 2.1}, \href{https://suttacentral.net/kp6/en/sujato}{Kp 6}). } 

\section*{10. The Mirror of the Teaching }

It’s\marginnote{2.8.1} no wonder that a human being should pass away. But if you should come and ask me about it each and every time someone passes away, that would be a bother for me. 

So\marginnote{2.8.3} Ānanda, I will teach you the explanation of the Dhamma called ‘the mirror of the teaching’. A noble disciple who has this may declare of themselves:\footnote{Mirrors are for doing makeup (\href{https://suttacentral.net/dn1/en/sujato\#1.16.2}{DN 1:1.16.2}) or admiring oneself (\href{https://suttacentral.net/mn77/en/sujato\#33.18}{MN 77:33.18}), which a mendicant should not do (\href{https://suttacentral.net/pli-tv-kd15/en/sujato\#2.4.1}{Kd 15:2.4.1}). In the Dhamma a mirror is for self-reflection (\href{https://suttacentral.net/mn61/en/sujato\#8.2}{MN 61:8.2}), while the mind is purified like polishing a mirror (\href{https://suttacentral.net/an3.70/en/sujato\#13.4}{AN 3.70:13.4}). } ‘I’ve finished with rebirth in hell, the animal realm, and the ghost realm. I’ve finished with all places of loss, bad places, the underworld. I am a stream-enterer! I’m not liable to be reborn in the underworld, and am bound for awakening.’ 

And\marginnote{2.9.1} what is that mirror of the teaching?\footnote{The four principles that follow are elsewhere identified as four factors of stream-entry (\href{https://suttacentral.net/an9.27/en/sujato}{AN 9.27}). At \href{https://suttacentral.net/sn12.41/en/sujato}{SN 12.41}, understanding of dependent origination is included. } 

It’s\marginnote{2.9.3} when a noble disciple has experiential confidence in the Buddha:\footnote{“Experiential” is \textit{avecca}, literally “having undergone”. “Experiential confidence” is the faith of a stream-enterer, who has seen for themselves. } ‘That Blessed One is perfected, a fully awakened Buddha, accomplished in knowledge and conduct, holy, knower of the world, supreme guide for those who wish to train, teacher of gods and humans, awakened, blessed.’\footnote{They have experiential confidence in the Buddha as a teacher because they have followed his path and realized the results that he speaks of. } 

They\marginnote{2.9.5} have experiential confidence in the teaching: ‘The teaching is well explained by the Buddha—apparent in the present life, immediately effective, inviting inspection, relevant, so that sensible people can know it for themselves.’\footnote{A stream-enterer has direct experience of the four noble truths, so they have confirmed that the teaching is indeed realizable in this very life. } 

They\marginnote{2.9.7} have experiential confidence in the \textsanskrit{Saṅgha}: ‘The \textsanskrit{Saṅgha} of the Buddha’s disciples is practicing the way that’s good, sincere, systematic, and proper. It consists of the four pairs, the eight individuals. This is the \textsanskrit{Saṅgha} of the Buddha’s disciples that is worthy of offerings dedicated to the gods, worthy of hospitality, worthy of a religious donation, worthy of greeting with joined palms, and is the supreme field of merit for the world.’\footnote{The suttas distinguish between two senses of \textsanskrit{Saṅgha}. The “mendicant \textsanskrit{Saṅgha}” (\textit{\textsanskrit{bhikkhusaṅgha}}) is the conventional community of monks and nuns. The “\textsanskrit{Saṅgha} of disciples” (\textit{\textsanskrit{sāvakasaṅgha}}) is classified as fourfold according to the stages of awakening: stream-entry, once-return, non-return, and perfection. Each of these stages is further subdivided into those of the path who are practicing for realization and those of the fruit who have realized. These are referred to as “noble disciples”, four of the path and four of the fruit, making eight individuals in total. } 

And\marginnote{2.9.9} a noble disciple’s ethical conduct is loved by the noble ones, unbroken, impeccable, spotless, and unmarred, liberating, praised by sensible people, not mistaken, and leading to immersion.\footnote{This entails keeping the five precepts at a minimum. } 

This\marginnote{2.9.10} is that mirror of the teaching.” 

And\marginnote{2.10.1} while staying there in \textsanskrit{Ñātika} the Buddha often gave this Dhamma talk to the mendicants: 

“Such\marginnote{2.10.2} is ethics, such is immersion, such is wisdom. When immersion is imbued with ethics it’s very fruitful and beneficial. When wisdom is imbued with immersion it’s very fruitful and beneficial. When the mind is imbued with wisdom it is rightly freed from the defilements, namely, the defilements of sensuality, desire to be reborn, and ignorance.” 

When\marginnote{2.11.1} the Buddha had stayed in \textsanskrit{Ñātika} as long as he pleased, he addressed Venerable Ānanda, “Come, Ānanda, let’s go to \textsanskrit{Vesālī}.” 

“Yes,\marginnote{2.11.3} sir,” Ānanda replied. Then the Buddha together with a large \textsanskrit{Saṅgha} of mendicants arrived at \textsanskrit{Vesālī}, where he stayed in \textsanskrit{Ambapālī}’s mango grove.\footnote{\textsanskrit{Ambapālī} means “protector of mangoes”. Here the text just says she had a “grove”, but it is identified as a mango grove below (\href{https://suttacentral.net/dn16/en/sujato\#2.14.1}{DN 16:2.14.1}). } 

There\marginnote{2.12.1} the Buddha addressed the mendicants: 

“Mendicants,\marginnote{2.12.2} a mendicant should live mindful and aware.\footnote{In some versions, the Buddha is said to have taught the monks mindfulness in anticipation of the arrival of the beautiful courtesan. } This is my instruction to you. 

And\marginnote{2.12.4} how is a mendicant mindful?\footnote{In the Gradual Training the mendicant “establishes mindfulness” to begin meditation. Here this practice is expanded, and in \href{https://suttacentral.net/dn22/en/sujato}{DN 22} it is expanded even further. } It’s when a mendicant meditates by observing an aspect of the body—keen, aware, and mindful, rid of covetousness and displeasure for the world.\footnote{The locative idiom \textit{\textsanskrit{kāye} \textsanskrit{kāyānupassī}} indicates that the meditator focuses on a particular aspect of body contemplation, such as observing the breath, awareness of the body parts, or reflection on the four elements. | “Rid of desire and aversion for the world” refers to clearing the mind through the previous practice of sense restraint. } They meditate observing an aspect of feelings … mind … principles—keen, aware, and mindful, rid of covetousness and displeasure for the world.\footnote{\textit{\textsanskrit{Dhammā}} here refers to the “principles” of cause and effect and the four noble truths which are understood by reflecting on the psychology of meditation itself. In this context \textit{\textsanskrit{dhammā}} does not mean “mental objects” or “phenomena” or “mental qualities”. } That’s how a mendicant is mindful. 

And\marginnote{2.13.1} how is a mendicant aware? It’s when a mendicant acts with situational awareness when going out and coming back; when looking ahead and aside; when bending and extending the limbs; when bearing the outer robe, bowl and robes; when eating, drinking, chewing, and tasting; when urinating and defecating; when walking, standing, sitting, sleeping, waking, speaking, and keeping silent. That’s how a mendicant is aware. A mendicant should live mindful and aware. This is my instruction to you.” 

\section*{11. \textsanskrit{Ambapālī} the Courtesan }

\textsanskrit{Ambapālī}\marginnote{2.14.1} the courtesan heard that the Buddha had arrived and was staying in her mango grove.\footnote{A \textit{\textsanskrit{gaṇikā}} was a trained entertainer and escort whose company commanded a high fee (\href{https://suttacentral.net/pli-tv-kd8/en/sujato\#1.1.7}{Kd 8:1.1.7}). It was a respected position, as we see the city guild of \textsanskrit{Rājagaha}, with King \textsanskrit{Bimbisāra}’s blessing, appoint the young woman \textsanskrit{Sālavatī} in a similar position (\href{https://suttacentral.net/pli-tv-kd8/en/sujato\#1.2.13}{Kd 8:1.2.13}). } She had the finest carriages harnessed. Then she mounted a fine carriage and, along with other fine carriages, set out from \textsanskrit{Vesālī} for her own park. She went by carriage as far as the terrain allowed, then descended and approached the Buddha on foot. She bowed and sat down to one side. The Buddha educated, encouraged, fired up, and inspired her with a Dhamma talk. 

Then\marginnote{2.14.5} she said to the Buddha, “Sir, may the Buddha together with the mendicant \textsanskrit{Saṅgha} please accept tomorrow’s meal from me.” The Buddha consented with silence. Then, knowing that the Buddha had consented, \textsanskrit{Ambapālī} got up from her seat, bowed, and respectfully circled the Buddha, keeping him on her right, before leaving. 

The\marginnote{2.15.1} Licchavis of \textsanskrit{Vesālī} also heard that the Buddha had arrived and was staying in \textsanskrit{Ambapālī}’s mango grove. They had the finest carriages harnessed. Then they mounted a fine carriage and, along with other fine carriages, set out from \textsanskrit{Vesālī}. Some of the Licchavis were in blue, of blue color, clad in blue, adorned with blue. And some were similarly colored in yellow, red, or white.\footnote{The text spends more time on the Vajjis’ appearance than on \textsanskrit{Ambapālī}’s. At \href{https://suttacentral.net/thig15.1/en/sujato\#12.1}{Thig 15.1:12.1} the nun \textsanskrit{Isidāsī} recalls how she used to do the makeup for her husband every day, like his own personal beautician. } 

Then\marginnote{2.16.1} \textsanskrit{Ambapālī} the courtesan collided with those Licchavi youths, axle to axle, wheel to wheel, yoke to yoke.\footnote{There are various reading for \textit{\textsanskrit{paṭivaṭṭesi}}. The commentary glosses with \textit{pahari} (“struck, collided”). } The Licchavis said to her, “What, you wench \textsanskrit{Ambapālī}, are you doing colliding with us axle to axle, wheel to wheel, yoke to yoke?”\footnote{The derogatory indeclinable vocative \textit{je} is otherwise used of the “naughty maid” \textsanskrit{Kāḷī} at \href{https://suttacentral.net/mn21/en/sujato\#9.13}{MN 21:9.13}. “Wench” is a term meant to insult on the basis of servitude or promiscuity, so it seems appropriate. The Licchavis are not distinguishing themselves here; given their youth, wealth, licentiousness, and evident familiarity with \textsanskrit{Ambapālī}, it seems safe to conclude that they had been her clients. } 

“Well,\marginnote{2.16.4} masters, it’s because I’ve invited the Buddha for tomorrow’s meal together with the mendicant \textsanskrit{Saṅgha}.” 

“Wench,\marginnote{2.16.5} give us that meal for a hundred thousand!” 

“Masters,\marginnote{2.16.6} even if you were to give me \textsanskrit{Vesālī} with her provinces, I still wouldn’t give that meal to you.”\footnote{\textit{Dassatha} is second future plural of \textit{\textsanskrit{dadāti}}. The Vinaya parallel at \href{https://suttacentral.net/pli-tv-kd6/en/sujato\#30.4.5}{Kd 6:30.4.5} uses optative forms instead. | “With provinces” is \textit{\textsanskrit{sāhāraṁ}} (literally “with food”, commentary: \textit{\textsanskrit{sajanapadaṁ}}) refers to the holdings around the city that supplied food and other necessities. } 

Then\marginnote{2.16.7} the Licchavis snapped their fingers, saying, “We’ve been beaten by the aunty! We’ve been beaten by the aunty!”\footnote{Despite the context, \textit{\textsanskrit{ambakā}} has no etymological relation to \textit{amba} (“mango”) but is a term for “woman, mother”, used here as a play on words. } Then they continued on to \textsanskrit{Ambapālī}’s mango grove. 

The\marginnote{2.17.1} Buddha saw them coming off in the distance, and addressed the mendicants: “Any of the mendicants who’ve never seen the gods of the thirty-three, just have a look at the assembly of Licchavis. See the assembly of Licchavis, check them out: they’re just like the thirty-three!” 

The\marginnote{2.18.1} Licchavis went by carriage as far as the terrain allowed, then descended and approached the Buddha on foot. They bowed to the Buddha, sat down to one side, and the Buddha educated, encouraged, fired up, and inspired them with a Dhamma talk. 

Then\marginnote{2.18.3} they said to the Buddha, “Sir, may the Buddha together with the mendicant \textsanskrit{Saṅgha} please accept tomorrow’s meal from us.” 

Then\marginnote{2.18.5} the Buddha said to the Licchavis, “I have already accepted tomorrow’s meal from \textsanskrit{Ambapālī} the courtesan.” 

Then\marginnote{2.18.7} the Licchavis snapped their fingers, saying, “We’ve been beaten by the aunty! We’ve been beaten by the aunty!” 

And\marginnote{2.18.9} then those Licchavis approved and agreed with what the Buddha said. They got up from their seat, bowed, and respectfully circled the Buddha, keeping him on their right, before leaving. 

And\marginnote{2.19.1} when the night had passed \textsanskrit{Ambapālī} had delicious fresh and cooked foods prepared in her own park. Then she had the Buddha informed of the time, saying, “Sir, it’s time. The meal is ready.” 

Then\marginnote{2.19.3} the Buddha robed up in the morning and, taking his bowl and robe, went to the home of \textsanskrit{Ambapālī} together with the mendicant \textsanskrit{Saṅgha}, where he sat on the seat spread out. Then \textsanskrit{Ambapālī} served and satisfied the mendicant \textsanskrit{Saṅgha} headed by the Buddha with her own hands with delicious fresh and cooked foods. 

When\marginnote{2.19.5} the Buddha had eaten and washed his hands and bowl, \textsanskrit{Ambapālī} took a low seat, sat to one side, and said to the Buddha, “Sir, I present this park to the mendicant \textsanskrit{Saṅgha} headed by the Buddha.”\footnote{This demonstrates that a woman of \textsanskrit{Ambapālī}’s profession could own land and control significant resources. } 

The\marginnote{2.19.8} Buddha accepted the park.\footnote{An \textit{\textsanskrit{ārāma}} is a tended and pleasant ground, a “park”. Here it has previously been referred to as “grove” (\textit{vana}) and “mango grove”. Usually in the suttas, however, it is a name for a place where monastics live. In English, monastics don’t live in parks, they live in monasteries. So once a place has been dedicated for the \textsanskrit{Saṅgha}, I translate \textit{\textsanskrit{ārāma}} as “monastery”. } 

Then\marginnote{2.19.9} the Buddha educated, encouraged, fired up, and inspired her with a Dhamma talk, after which he got up from his seat and left. 

And\marginnote{2.20.1} while staying at \textsanskrit{Vesālī}, too, the Buddha often gave this Dhamma talk to the mendicants: 

“Such\marginnote{2.20.2} is ethics, such is immersion, such is wisdom. When immersion is imbued with ethics it’s very fruitful and beneficial. When wisdom is imbued with immersion it’s very fruitful and beneficial. When the mind is imbued with wisdom it is rightly freed from the defilements, namely, the defilements of sensuality, desire to be reborn, and ignorance.” 

\section*{12. Commencing the Rains at Beluva }

When\marginnote{2.21.1} the Buddha had stayed in \textsanskrit{Ambapālī}’s mango grove as long as he pleased, he addressed Venerable Ānanda, “Come, Ānanda, let’s go to the little village of Beluva.”\footnote{Pali texts waver between \textit{beluva} (“wood apple”) and \textit{\textsanskrit{veḷuva}} (from \textit{\textsanskrit{veḷuvant}}, “full of bamboo”). } 

“Yes,\marginnote{2.21.3} sir,” Ānanda replied. Then the Buddha together with a large \textsanskrit{Saṅgha} of mendicants arrived at the little village of Beluva, and stayed there. 

There\marginnote{2.22.1} the Buddha addressed the mendicants: “Mendicants, please enter the rainy season residence with whatever friends or acquaintances you have around \textsanskrit{Vesālī}.\footnote{The Buddha was travelling with a large retinue, which could become burdensome on a small village if they were to stay the three months of the rains residence. The texts note several monasteries and places to stay around \textsanskrit{Vesālī}. } I’ll commence the rainy season residence right here in the little village of Beluva.” 

“Yes,\marginnote{2.22.4} sir,” those mendicants replied. They did as the Buddha said, while the Buddha commenced the rainy season residence right there in the little village of Beluva. 

After\marginnote{2.23.1} the Buddha had commenced the rainy season residence, he fell severely ill, struck by dreadful pains, close to death. But he endured unbothered, with mindfulness and situational awareness. Then it occurred to the Buddha, “It would not be appropriate for me to be fully extinguished before informing my supporters and taking leave of the mendicant \textsanskrit{Saṅgha}.\footnote{Here \textit{\textsanskrit{upaṭṭhāka}} refers to the lay devotees. | \textit{Apaloketi} (“take leave”) also has a literal sense of “glance back”; both senses are found in the sutta (see \href{https://suttacentral.net/dn16/en/sujato\#4.1.2}{DN 16:4.1.2}). } Why don’t I forcefully suppress this illness, stabilize the life force, and live on?”\footnote{“Life force” is \textit{\textsanskrit{jīvitasaṅkhāra}} whereas below we find \textit{\textsanskrit{āyusaṅkhāra}} (\href{https://suttacentral.net/dn16/en/sujato\#3.10.1}{DN 16:3.10.1}). They are evidently synonyms; \href{https://suttacentral.net/ps1.6/en/sujato\#4.3}{Ps 1.6:4.3} refers to the three \textit{\textsanskrit{jīvitasaṅkhāras}} from \href{https://suttacentral.net/mn43/en/sujato\#24.2}{MN 43:24.2}, but there the text has \textit{\textsanskrit{āyusaṅkhāra}}. \textit{\textsanskrit{Saṅkhāra}} could be interpreted here either as “volition” (“will to live”) or as “life force” (i.e. the vital energy that sustains life). Elsewhere, however, \textit{\textsanskrit{āyusaṅkhāra}} clearly means “life force” (\href{https://suttacentral.net/sn20.6/en/sujato\#3.2}{SN 20.6:3.2}, \href{https://suttacentral.net/mn43/en/sujato\#23.1}{MN 43:23.1}). } 

So\marginnote{2.24.1} that is what he did. Then the Buddha’s illness died down. 

Soon\marginnote{2.24.3} after the Buddha had recovered from that sickness, he came out from his dwelling and sat in the shade of the porch on the seat spread out. Then Venerable Ānanda went up to the Buddha, bowed, sat down to one side, and said to him, “Sir, it’s fantastic that the Buddha is comfortable and well. Because when the Buddha was sick, my body felt like it was drugged. I was disorientated, and the teachings weren’t clear to me.\footnote{This is the first indication of Ānanda’s fragile emotional state in the days to come. | Read \textit{\textsanskrit{diṭṭhā}} (= Sanskrit \textit{\textsanskrit{diṣṭyā}}), “fantastic, how fortunate”, per \href{https://suttacentral.net/dn26/en/sujato\#21.9}{DN 26:21.9} and \href{https://suttacentral.net/ja81/en/sujato\#1.4}{Ja 81:1.4}. } Still, at least I was consoled by the thought that the Buddha won’t be fully extinguished without bringing something up regarding the \textsanskrit{Saṅgha} of mendicants.”\footnote{Previously the Buddha had spoken of “taking leave” of the \textsanskrit{Saṅgha}, but here something more specific is meant. \textit{\textsanskrit{Udāharati}} is typically used when a previously-mentioned matter (often unpleasant) is “brought up”, as at the monastic procedures for confession or invitation to admonish (\href{https://suttacentral.net/pli-tv-kd1/en/sujato\#16.2.5}{Kd 1:16.2.5}, \href{https://suttacentral.net/pli-tv-kd1/en/sujato\#2.1.9}{Kd 1:2.1.9}). Ānanda is implying that there is some unfinished disciplinary business that needs attention. Despite the Buddha’s response, he did in fact go on to make a number of statements and rulings. } 

“But\marginnote{2.25.1} what could the mendicant \textsanskrit{Saṅgha} expect from me, Ānanda?\footnote{For forty-five years he had been teaching and supporting the community. } I’ve taught the Dhamma without making any distinction between secret and public teachings.\footnote{A principle not followed by some contemporary Buddhist schools that harbor “secret teachings”. } The Realized One doesn’t have the closed fist of a tutor when it comes to the teachings.\footnote{“Closed fist of a teacher” is \textit{\textsanskrit{ācariyamuṭṭhi}}. } If there’s anyone who thinks: ‘I shall lead the mendicant \textsanskrit{Saṅgha},’ or ‘the \textsanskrit{Saṅgha} of mendicants is meant for me,’ let them bring something up regarding the \textsanskrit{Saṅgha}. But the Realized One doesn’t think like this, so why should he bring something up regarding the \textsanskrit{Saṅgha}?\footnote{\textit{\textsanskrit{Pariharissāmi}} needs careful parsing with regard to tenses. The general meaning is to “carry about” or “maintain, nurture”. The Buddha elsewhere said that he “leads” the \textsanskrit{Saṅgha} (\href{https://suttacentral.net/dn26/en/sujato\#25.6}{DN 26:25.6}: \textit{\textsanskrit{pariharāmi}}). When the Buddha was about to go on retreat, \textsanskrit{Moggallāna} said that he and \textsanskrit{Sāriputta} “shall lead” (\textit{\textsanskrit{pariharissāmi}}), to which the Buddha said they “should lead” (\href{https://suttacentral.net/mn67/en/sujato\#13.7}{MN 67:13.7}: \textit{\textsanskrit{parihareyyaṁ}}). Devadatta, on the other hand, ensured his downfall when he determined that he “shall lead” the \textsanskrit{Saṅgha}, having taken over from the Buddha (\href{https://suttacentral.net/an5.100/en/sujato\#2.3}{AN 5.100:2.3}, \href{https://suttacentral.net/pli-tv-kd1/en/sujato\#2.1.23}{Kd 1:2.1.23}). Thus the future tense indicates the determination to lead. But at this point, the Buddha’s mind is already set on letting go. He is not saying that no-one should lead the \textsanskrit{Saṅgha}, but that it is up to whoever leads it to address the issues. } 

I’m\marginnote{2.25.9} now old, elderly and senior. I’m advanced in years and have reached the final stage of life. I’m currently eighty years old. Just as a decrepit old cart is kept going by a rope,\footnote{The reading and derivation of \textit{\textsanskrit{veṭhamissakena}} are unclear. As a form of violence we find \textit{\textsanskrit{veṭhamissena}} at \href{https://suttacentral.net/thag2.12/en/sujato\#1.1}{Thag 2.12:1.1}, notably also in instrumental. \textit{\textsanskrit{Veṭha}} means “twist, strap, turban”. \textit{Missa} means “mixed” or “plaited” (\href{https://suttacentral.net/pli-tv-bu-vb-ss2/en/sujato\#2.1.21}{Bu Ss 2:2.1.21}). Thus it probably refers to a kind of strong twisted material used to tie or bind, i.e. rope. } in the same way, the Realized One’s body is kept going as if by a rope.\footnote{Compare Rig Veda 8.48.5: “As leather binds a chariot, soma knits my joints together.” \textsanskrit{Bṛhadāraṇyaka} \textsanskrit{Upaniṣad} 4.3.35 compares the labored breathing of one near death with the creaking of a heavily-laden cart. } Sometimes the Realized One, not focusing on any signs, and with the cessation of certain feelings, enters and remains in the signless immersion of the heart. Only then does the Realized One’s body become more comfortable.\footnote{The suttas say little about this signless (\textit{animitta}) meditation. Its defining characteristic is that consciousness does not “follow after signs” (\textit{\textsanskrit{nimittānusāri} \textsanskrit{viññāṇaṁ}}, eg. \href{https://suttacentral.net/an6.13/en/sujato\#5.3}{AN 6.13:5.3}). This is explained in \href{https://suttacentral.net/mn138/en/sujato\#10.2}{MN 138:10.2} as not being distracted or affected by the features of sense impressions, as the “signs” are created by greed, hate, and delusion (\href{https://suttacentral.net/mn43/en/sujato\#37.1}{MN 43:37.1}). The mental unification (\textit{\textsanskrit{ekattaṁ}})  based on this practice is listed after the form (and formless) \textit{\textsanskrit{jhānas}}, so it is very advanced. It is nonetheless a conditioned state (\href{https://suttacentral.net/mn121/en/sujato\#11.4}{MN 121:11.4}), so it is possible that a mendicant might fall from it and disrobe (\href{https://suttacentral.net/an6.60/en/sujato\#8.10}{AN 6.60:8.10}). However it may also be used to describe the meditation of an arahant (\href{https://suttacentral.net/sn41.7/en/sujato\#6.12}{SN 41.7:6.12}). It is clear from \href{https://suttacentral.net/mn121/en/sujato\#10.5}{MN 121:10.5} that the six senses are still functioning, unlike in deep serenity meditations. It seems that in such a state, the Buddha was able to function normally while seeing through the pain in his body. } 

So\marginnote{2.26.1} Ānanda, live as your own island, your own refuge, with no other refuge. Let the teaching be your island and your refuge, with no other refuge.\footnote{This central theme of the Buddha’s teaching—that each of us is responsible for our own salvation—becomes even more important as the Buddha’s days grow short. } And how does a mendicant do this? It’s when a mendicant meditates by observing an aspect of the body—keen, aware, and mindful, rid of covetousness and displeasure for the world. They meditate observing an aspect of feelings … mind … principles—keen, aware, and mindful, rid of covetousness and displeasure for the world. That’s how a mendicant is their own island, their own refuge, with no other refuge. That’s how the teaching is their island and their refuge, with no other refuge. 

Whether\marginnote{2.26.8} now or after I have passed, any who shall live as their own island, their own refuge, with no other refuge; with the teaching as their island and their refuge, with no other refuge—those mendicants of mine who want to train shall be among the best of the best.”\footnote{Read \textit{tama(t)agge} with \textit{tama} as superlative, literally “at the peak of the best”. } 

\scendsection{The second recitation section. }

\section*{13. An Obvious Hint }

Then\marginnote{3.1.1} the Buddha robed up in the morning and, taking his bowl and robe, entered \textsanskrit{Vesālī} for alms. Then, after the meal, on his return from almsround, he addressed Venerable Ānanda: “Ānanda, get your sitting cloth.\footnote{This detail is often mentioned in texts of the \textsanskrit{Sarvāstivāda} school, but rarely in Pali. } Let’s go to the \textsanskrit{Cāpāla} shrine for the day’s meditation.” 

“Yes,\marginnote{3.1.5} sir,” replied Ānanda. Taking his sitting cloth he followed behind the Buddha. 

Then\marginnote{3.2.1} the Buddha went up to the \textsanskrit{Cāpāla} shrine, where he sat on the seat spread out. Ānanda bowed to the Buddha and sat down to one side. 

The\marginnote{3.2.3} Buddha said to him: “Ānanda, \textsanskrit{Vesālī} is lovely. And the Udena, Gotamaka, Seven Maidens, Many Sons, \textsanskrit{Sārandada}, and \textsanskrit{Cāpāla} Tree-shrines are all lovely. 

Whoever\marginnote{3.3.1} has developed and cultivated the four bases of psychic power—made them a vehicle and a basis, kept them up, consolidated them, and properly implemented them—may, if they wish, live for the proper lifespan or what’s left of it.\footnote{Normally \textit{kappa} as a period of time means “eon”, but the late canonical texts \textsanskrit{Kathāvatthu} (\href{https://suttacentral.net/kv11.5}{Kv 11.5}) and \textsanskrit{Milindapañha} (\href{https://suttacentral.net/mil5.1.10}{Mil 5.1.10}) argue that it means the “lifespan”, an interpretation followed by the commentaries. Support for this comes from \href{https://suttacentral.net/dn26/en/sujato\#28.3}{DN 26:28.3}, which says that “long life” for a mendicant is the four bases of psychic power which enable you to remain for the \textit{kappa}. Here it is surely talking about a full lifespan. Underlying this is the idea that in different epochs the “proper lifespan” varies; in the Buddha’s day it was one hundred years (\href{https://suttacentral.net/dn14/en/sujato\#1.7.7}{DN 14:1.7.7}). We can resolve the problem if we read \textit{kappa} here, not as a period of time, but as “proper, fitting”, i.e. the “proper” lifespan of a hundred years. } The Realized One has developed and cultivated the four bases of psychic power, made them a vehicle and a basis, kept them up, consolidated them, and properly implemented them. If he wished, the Realized One could live for the proper lifespan or what’s left of it.” 

But\marginnote{3.4.1} Ānanda didn’t get it, even though the Buddha dropped such an obvious hint, such a clear sign. He didn’t beg the Buddha: “Sir, may the Blessed One please remain for the eon! May the Holy One please remain for the eon! That would be for the welfare and happiness of the people, out of sympathy for the world, for the benefit, welfare, and happiness of gods and humans.” For his mind was as if possessed by \textsanskrit{Māra}.\footnote{Here Ānanda’s mind is “as if” possessed (\textit{\textsanskrit{yathā}}), but in the Vinaya \textit{\textsanskrit{yathā}} is missing: he is possessed (\href{https://suttacentral.net/pli-tv-kd1/en/sujato\#1.10.17}{Kd 1:1.10.17}). Remember, Ānanda is telling this story about himself. } 

For\marginnote{3.5.1} a second time … And for a third time, the Buddha said to Ānanda: “Ānanda, \textsanskrit{Vesālī} is lovely. And the Udena, Gotamaka, Seven Maidens, Many Sons, \textsanskrit{Sārandada}, and \textsanskrit{Cāpāla} Tree-shrines are all lovely. Whoever has developed and cultivated the four bases of psychic power—made them a vehicle and a basis, kept them up, consolidated them, and properly implemented them—may, if they wish, live for the proper lifespan or what’s left of it. The Realized One has developed and cultivated the four bases of psychic power, made them a vehicle and a basis, kept them up, consolidated them, and properly implemented them. If he wished, the Realized One could live for the proper lifespan or what’s left of it.” 

But\marginnote{3.5.6} Ānanda didn’t get it, even though the Buddha dropped such an obvious hint, such a clear sign. He didn’t beg the Buddha: “Sir, may the Blessed One please remain for the eon! May the Holy One please remain for the eon! That would be for the welfare and happiness of the people, out of sympathy for the world, for the benefit, welfare, and happiness of gods and humans.” For his mind was as if possessed by \textsanskrit{Māra}. 

Then\marginnote{3.6.1} the Buddha got up and said to Venerable Ānanda, “Go now, Ānanda, at your convenience.” 

“Yes,\marginnote{3.6.4} sir,” replied Ānanda. He rose from his seat, bowed, and respectfully circled the Buddha, keeping him on his right, before sitting at the root of a tree close by. 

\section*{14. The Appeal of \textsanskrit{Māra} }

And\marginnote{3.7.1} then, not long after Ānanda had left, \textsanskrit{Māra} the Wicked went up to the Buddha, stood to one side, and said to him:\footnote{\textsanskrit{Māra} is the Buddhist deity of death, sex, and delusion; his aim is to trap beings in transmigration. He appears in many guises, both real and metaphorical, throughout the canon, but this sequence is his only direct appearance in the \textsanskrit{Dīghanikāya}. Here he feigns compassion, urging the Buddha to find his final peace; but he has an ulterior motive, for with the Buddha’s passing \textsanskrit{Māra}’s job will be much easier. } 

“Sir,\marginnote{3.7.2} may the Blessed One now be fully extinguished! May the Holy One now be fully extinguished! Now is the time for the full extinguishment of the Buddha.\footnote{\textit{\textsanskrit{Parinibbāna}} means “extinguishment”, as of a flame. Here it appears as both noun (“extinguishment”) and verb (“become extinguished”). It is not a particularly difficult term to translate. Linguistically it has nothing to do with “attachment”, so renderings such as “unbinding” are untenable. It should be translated rather than just keeping “\textsanskrit{Nibbāna}”, not least because English resists verbifying adopted words, resulting in such constructions as “enter \textsanskrit{Nibbāna}”, which reifies it in a way that the Pali does not. } Sir, you once made this statement: ‘Wicked One, I shall not be fully extinguished until I have monk disciples who are competent, educated, assured, learned, have memorized the teachings, and practice in line with the teachings. Not until they practice properly, living in line with the teaching. Not until they’ve learned their tradition, and explain, teach, assert, establish, disclose, analyze, and make it clear. Not until they can legitimately and completely refute the doctrines of others that come up, and teach with a demonstrable basis.’\footnote{\textsanskrit{Māra} is citing the Buddha’s words for his own purpose. The Pali tradition does not say when this encounter took place, but the Sanskrit \textsanskrit{Sarvāstivāda} \textsanskrit{Catuṣparisatsūtra} places it shortly after the Buddha’s awakening. } 

Today\marginnote{3.8.1} you do have such monk disciples. May the Blessed One now be fully extinguished! May the Holy One now be fully extinguished! Now is the time for the full extinguishment of the Buddha. 

Sir,\marginnote{3.8.3} you once made this statement: ‘Wicked One, I shall not be fully extinguished until I have nun disciples who are competent, educated, assured, learned …’\footnote{This passage makes it clear that it was the Buddha’s intention from the beginning to establish an order of nuns (\textit{\textsanskrit{bhikkhunī}}). } 

Today\marginnote{3.8.5} you do have such nun disciples. May the Blessed One now be fully extinguished! May the Holy One now be fully extinguished! Now is the time for the full extinguishment of the Buddha. 

Sir,\marginnote{3.8.7} you once made this statement: ‘Wicked One, I shall not be fully extinguished until I have layman disciples who are competent, educated, assured, learned …’ 

Today\marginnote{3.8.9} you do have such layman disciples. May the Blessed One now be fully extinguished! May the Holy One now be fully extinguished! Now is the time for the full extinguishment of the Buddha. 

Sir,\marginnote{3.8.11} you once made this statement: ‘Wicked One, I shall not be fully extinguished until I have laywoman disciples who are competent, educated, assured, learned …’ 

Today\marginnote{3.8.13} you do have such laywoman disciples. May the Blessed One now be fully extinguished! May the Holy One now be fully extinguished! Now is the time for the full extinguishment of the Buddha. 

Sir,\marginnote{3.8.15} you once made this statement: ‘Wicked One, I will not be fully extinguished until my spiritual path is successful and prosperous, extensive, popular, widespread, and well proclaimed wherever there are gods and humans.’ 

Today\marginnote{3.8.17} your spiritual path is successful and prosperous, extensive, popular, widespread, and well proclaimed wherever there are gods and humans. May the Blessed One now be fully extinguished! May the Holy One now be fully extinguished! Now is the time for the full extinguishment of the Buddha.” 

When\marginnote{3.9.1} this was said, the Buddha said to \textsanskrit{Māra}, “Relax, Wicked One. The full extinguishment of the Realized One will be soon.\footnote{This passage is the narrative inverse of the occasion when \textsanskrit{Brahmā} begged the Buddha to teach (\href{https://suttacentral.net/sn6.1/en/sujato\#5.5}{SN 6.1:5.5}, etc.). Throughout, the sutta artfully preserves a degree of narrative ambiguity. Here it almost appears as if the Buddha assents to passing away because of \textsanskrit{Māra}’s request, although his intent was clear earlier. \textsanskrit{Māra}’s reasoning, moreover, is based on the Buddha’s own words. } Three months from now the Realized One will be fully extinguished.”\footnote{It seems that at this point, the Buddha is still spending the rains in Beluva village, from where he would sometimes go to nearby \textsanskrit{Vesālī} for alms, or to a local shrine for meditation. He left \textsanskrit{Vesālī} only after holding a meeting for all the \textsanskrit{Saṅgha}, which probably signified the completion of the rains (\href{https://suttacentral.net/dn16/en/sujato\#4.1.2}{DN 16:4.1.2}). If this reasoning is correct—and the text is not explicit—then he made this statement during the rains retreat. If, however, this reasoning is incorrect and we are already after the rains, it could not have been long after. Thus it is probably September/October, meaning that the final extinguishment was to take place in December/January. This conflicts with the tradition of ascribing his final extinguishment to the full moon of Vesak, in May. } 

\section*{15. Surrendering the Life Force }

So\marginnote{3.10.1} at the \textsanskrit{Cāpāla} Tree-shrine the Buddha, mindful and aware, surrendered the life force.\footnote{The \textsanskrit{Cāpāla} shrine is unknown outside of this context. The \textsanskrit{Udāna} commentary says that it was named after the \textit{yakkha} who formerly lived there. Now, \textit{\textsanskrit{cāpalla}} means “fickleness, propensity to tremble” and is from the same root as earth-“quake” (\textit{\textsanskrit{bhumicāla}}; \textit{capala} = \textit{pacala} via metathesis). It is no great leap to \textit{\textsanskrit{cāpāla}}. If this is correct, it suggests that the shrine was dedicated to an earth spirit who commanded earthquakes; a fitting setting for the events to follow. } When he did so there was a great earthquake, awe-inspiring and hair-raising, and thunder cracked the sky.\footnote{Northern India lies in the shadow of the seismically active Alpide Belt, so the occurrence of earthquakes is realistic even if the causes are not scientific. } Then, understanding this matter, on that occasion the Buddha expressed this heartfelt sentiment: 

\begin{verse}%
“Comparing\marginnote{3.10.4} the incomparable \\>with the creation of prolonged life,\footnote{For this difficult verse, I generally follow Bhikkhu Bodhi’s long discussion in \emph{Connected Discourses of the Buddha}, note 255 on the \textsanskrit{Mahāvagga}. | \textit{Sambhava} means “production, creation”, and here I think it refers to the prolonging of life which the Buddha had just rejected. } \\
the sage surrendered the life force.\footnote{\textit{\textsanskrit{Bhavasaṅkhāra}} (“life force”) here is equivalent to \textit{\textsanskrit{āyusaṅkhāra}} and \textit{\textsanskrit{jīvitasaṅkhāra}}. } \\
Happy inside, serene, \\
he shattered self-creation like a suit of armor.”\footnote{\textit{Attasambhava} (“self-creation”) refers back to \textit{sambhava} in the first line. } 

%
\end{verse}

\section*{16. The Causes of Earthquakes }

Then\marginnote{3.11.1} Venerable Ānanda thought, “How incredible, how amazing! That was a really big earthquake! That was really a very big earthquake; awe-inspiring and hair-raising, and thunder cracked the sky! What’s the cause, what’s the reason for a great earthquake?” 

Then\marginnote{3.12.1} Venerable Ānanda went up to the Buddha, bowed, sat down to one side, and said to him, “How incredible, sir, how amazing! That was a really big earthquake! That was really a very big earthquake; awe-inspiring and hair-raising, and thunder cracked the sky! What’s the cause, what’s the reason for a great earthquake?” 

“Ānanda,\marginnote{3.13.1} there are these eight causes and reasons for a great earthquake. What eight? 

This\marginnote{3.13.3} great earth is grounded on water, the water is grounded on air, and the air stands in space. At a time when a great wind blows, it stirs the water, and the water stirs the earth.\footnote{So far as it goes, this is a naturalistic explanation. The “water element” has the quality of softening, while the “air element” is traditionally understood as “movement”. Thus in modern terms this means, “When underground forces disturb a region of instability.” } This is the first cause and reason for a great earthquake. 

Furthermore,\marginnote{3.14.1} there is an ascetic or brahmin with psychic power who has achieved mastery of the mind, or a god who is mighty and powerful. They’ve developed a limited perception of earth and a limitless perception of water. They make the earth shake and rock and tremble.\footnote{The might of ascetics was legendary in ancient India (eg. \href{https://suttacentral.net/mn56/en/sujato\#14.2}{MN 56:14.2}). The meditation described here might be compared with the “dimensions of mastery” below (\href{https://suttacentral.net/dn16/en/sujato\#3.25.1}{DN 16:3.25.1}). } This is the second cause and reason for a great earthquake. 

Furthermore,\marginnote{3.15.1} when the being intent on awakening passes away from the host of joyful gods, he’s conceived in his mother’s belly, mindful and aware. Then the earth shakes and rocks and trembles.\footnote{At \href{https://suttacentral.net/dn14/en/sujato\#1.17.7}{DN 14:1.17.7} and \href{https://suttacentral.net/mn123/en/sujato\#7.6}{MN 123:7.6} it is, rather, the entire galaxy that trembles, perhaps indicating that the enhanced miracle is a later development. } This is the third cause and reason for a great earthquake. 

Furthermore,\marginnote{3.16.1} when the being intent on awakening comes out of his mother’s belly mindful and aware, the earth shakes and rocks and trembles. This is the fourth cause and reason for a great earthquake. 

Furthermore,\marginnote{3.17.1} when the Realized One realizes the supreme perfect awakening, the earth shakes and rocks and trembles.\footnote{There are many accounts of the Buddha’s awakening in the suttas, but none, so far as I know, that mention an earthquake. } This is the fifth cause and reason for a great earthquake. 

Furthermore,\marginnote{3.18.1} when the Realized One rolls forth the supreme Wheel of Dhamma, the earth shakes and rocks and trembles.\footnote{See \href{https://suttacentral.net/sn56.11/en/sujato\#13.2}{SN 56.11:13.2}. } This is the sixth cause and reason for a great earthquake. 

Furthermore,\marginnote{3.19.1} when the Realized One, mindful and aware, surrenders the life force, the earth shakes and rocks and trembles. This is the seventh cause and reason for a great earthquake. 

Furthermore,\marginnote{3.20.1} when the Realized One becomes fully extinguished in the element of extinguishment with no residue, the earth shakes and rocks and trembles.\footnote{This refers to the Buddha’s death. The Pali here uses both \textit{\textsanskrit{nibbāna}} and \textit{\textsanskrit{parinibbāna}}. Sometimes it is said that \textit{\textsanskrit{nibbāna}} is the attainment of arahantship, while \textit{\textsanskrit{parinibbāna}} is the death of an arahant, but this distinction is not consistently applied in the suttas. | Regarding \textit{\textsanskrit{nibbānadhātuyā}} (“the element of extinguishment”), in Pali the case is ambiguous. It is sometimes translated “by means of” or “through”, which assumes the instrumental; but the Sanskrit is \textit{\textsanskrit{nirvāṇadhātau}}, which must be locative. } This is the eighth cause and reason for a great earthquake. 

These\marginnote{3.20.3} are the eight causes and reasons for a great earthquake. 

\section*{17. Eight Assemblies }

There\marginnote{3.21.1} are, Ānanda, these eight assemblies.\footnote{As at \href{https://suttacentral.net/an8.69/en/sujato}{AN 8.69}. The following series of “eight things” seems arbitrarily inserted here and breaks the flow of the narrative. They are absent from the Sanskrit \textsanskrit{Sarvāstivāda} text edited by Waldschmidt, which goes directly to the conversation where Ānanda realizes that the Buddha is going to die. } What eight? The assemblies of aristocrats, brahmins, householders, and ascetics. An assembly of the gods of the four great kings. An assembly of the gods of the thirty-three. An assembly of \textsanskrit{Māras}. An assembly of divinities.\footnote{These “assemblies” were formal deliberative meetings, not just gatherings. Thus “householders” here does not mean “lay folk” in apposition to “ascetics”; rather it means “home owners”. Some of these assemblies are depicted or alluded to in the \textsanskrit{Dīghanikāya}. At \href{https://suttacentral.net/dn6/en/sujato\#1.3}{DN 6:1.3}, brahmins assemble in \textsanskrit{Vesālī}; while at \href{https://suttacentral.net/dn3/en/sujato\#1.13.3}{DN 3:1.13.3}, \textsanskrit{Ambaṭṭha} speaks of an assembly of Sakyans, who were aristocrats. At \href{https://suttacentral.net/dn18/en/sujato\#12.1}{DN 18:12.1}, we get a glimpse of the proceedings at a meeting of the gods of the thirty-three. } 

I\marginnote{3.22.1} recall having approached an assembly of hundreds of aristocrats. There I used to sit with them, converse, and engage in discussion. And my appearance and voice became just like theirs. I educated, encouraged, fired up, and inspired them with a Dhamma talk. But when I spoke they didn’t know: ‘Who is this that speaks? Is it a god or a human?’ And when my Dhamma talk was finished I vanished. But when I vanished they didn’t know: ‘Who was that who vanished? Was it a god or a human?’\footnote{There do not appear to be any records of this happening in the early texts, although  that is perhaps to be expected. It does feel out of character for the Buddha, as normally he is very up front. The commentary says that this ruse was adopted so the Buddha could plant subtle seeds (\textit{\textsanskrit{vāsanā}}) for the future. } 

I\marginnote{3.23.1} recall having approached an assembly of hundreds of brahmins … householders … ascetics … the gods of the four great kings … the gods of the thirty-three … \textsanskrit{Māras} … divinities. There too I used to sit with them, converse, and engage in discussion. And my appearance and voice became just like theirs. I educated, encouraged, fired up, and inspired them with a Dhamma talk. But when I spoke they didn’t know: ‘Who is this that speaks? Is it a god or a human?’ And when my Dhamma talk was finished I vanished. But when I vanished they didn’t know: ‘Who was that who vanished? Was it a god or a human?’ 

These\marginnote{3.23.17} are the eight assemblies. 

\section*{18. Eight Dimensions of Mastery }

Ānanda,\marginnote{3.24.1} there are these eight dimensions of mastery.\footnote{These are another way of describing the different experiences of \textit{\textsanskrit{jhāna}}. Also at \href{https://suttacentral.net/an8.65/en/sujato}{AN 8.65}, \href{https://suttacentral.net/an10.29/en/sujato\#6.1}{AN 10.29:6.1}, \href{https://suttacentral.net/dn33/en/sujato\#3.1.142}{DN 33:3.1.142}, \href{https://suttacentral.net/dn34/en/sujato\#2.1.160}{DN 34:2.1.160}, and \href{https://suttacentral.net/mn77/en/sujato\#23.1}{MN 77:23.1}. } What eight? 

Perceiving\marginnote{3.25.1} form internally, someone sees forms externally, limited, both pretty and ugly.\footnote{“Perceiving form (\textit{\textsanskrit{rūpa}}) internally” refers to someone developing meditation based on an aspect of their own body, such as the breath or the parts of the body. The “forms” (\textit{\textsanskrit{rūpā}}) seen externally are the lights or other meditation phenomena that today are usually called \textit{nimitta}. An “ugly” form is the mental image that arises in such contemplations as the parts of the body. A “beautiful” image arises from practices such as mindfulness of breathing. } Mastering them, they perceive: ‘I know and see.’ This is the first dimension of mastery. 

Perceiving\marginnote{3.26.1} form internally, someone sees forms externally, limitless, both pretty and ugly. Mastering them, they perceive: ‘I know and see.’ This is the second dimension of mastery. 

Not\marginnote{3.27.1} perceiving form internally, someone sees forms externally, limited, both pretty and ugly.\footnote{“Not perceiving form internally” refers to meditations such as the external elements, or the decaying of another’s body. } Mastering them, they perceive: ‘I know and see.’ This is the third dimension of mastery. 

Not\marginnote{3.28.1} perceiving form internally, someone sees forms externally, limitless, both pretty and ugly. Mastering them, they perceive: ‘I know and see.’ This is the fourth dimension of mastery. 

Not\marginnote{3.29.1} perceiving form internally, someone sees forms externally that are blue, with blue color and blue appearance.\footnote{This is the meditation where one contemplates an external color, either a prepared disk or a natural phenomena such as the sky or a flower, which eventually gives rise to a “counterpart” image. Today such meditations are called \textit{\textsanskrit{kasiṇa}} following the Visuddhimagga, but in the early texts \textit{\textsanskrit{kasiṇa}} means “totality” and refers rather to the state of \textit{\textsanskrit{samādhi}} that results. } They’re like a flax flower that’s blue, with blue color and blue appearance. Or a cloth from Varanasi that’s smoothed on both sides, blue, with blue color and blue appearance. In the same way, not perceiving form internally, someone sees forms externally, blue, with blue color and blue appearance. Mastering them, they perceive: ‘I know and see.’ This is the fifth dimension of mastery. 

Not\marginnote{3.30.1} perceiving form internally, someone sees forms externally that are yellow, with yellow color and yellow appearance. They’re like a champak flower that’s yellow, with yellow color and yellow appearance. Or a cloth from Varanasi that’s smoothed on both sides, yellow, with yellow color and yellow appearance. In the same way, not perceiving form internally, someone sees forms externally that are yellow, with yellow color and yellow appearance. Mastering them, they perceive: ‘I know and see.’ This is the sixth dimension of mastery. 

Not\marginnote{3.31.1} perceiving form internally, someone sees forms externally that are red, with red color and red appearance. They’re like a scarlet mallow flower that’s red, with red color and red appearance. Or a cloth from Varanasi that’s smoothed on both sides, red, with red color and red appearance. In the same way, not perceiving form internally, someone sees forms externally that are red, with red color and red appearance. Mastering them, they perceive: ‘I know and see.’ This is the seventh dimension of mastery. 

Not\marginnote{3.32.1} perceiving form internally, someone sees forms externally that are white, with white color and white appearance. They’re like the morning star that’s white, with white color and white appearance. Or a cloth from Varanasi that’s smoothed on both sides, white, with white color and white appearance. In the same way, not perceiving form internally, someone sees forms externally that are white, with white color and white appearance. Mastering them, they perceive: ‘I know and see.’ This is the eighth dimension of mastery. 

These\marginnote{3.32.6} are the eight dimensions of mastery. 

\section*{19. The Eight Liberations }

Ānanda,\marginnote{3.33.1} there are these eight liberations.\footnote{Already encountered at \href{https://suttacentral.net/dn15/en/sujato\#35.1}{DN 15:35.1}. } What eight? 

Having\marginnote{3.33.3} physical form, they see forms. This is the first liberation. 

Not\marginnote{3.33.5} perceiving form internally, they see forms externally. This is the second liberation. 

They’re\marginnote{3.33.7} focused only on beauty. This is the third liberation. 

Going\marginnote{3.33.9} totally beyond perceptions of form, with the ending of perceptions of impingement, not focusing on perceptions of diversity, aware that ‘space is infinite’, they enter and remain in the dimension of infinite space. This is the fourth liberation. 

Going\marginnote{3.33.11} totally beyond the dimension of infinite space, aware that ‘consciousness is infinite’, they enter and remain in the dimension of infinite consciousness. This is the fifth liberation. 

Going\marginnote{3.33.13} totally beyond the dimension of infinite consciousness, aware that ‘there is nothing at all’, they enter and remain in the dimension of nothingness. This is the sixth liberation. 

Going\marginnote{3.33.15} totally beyond the dimension of nothingness, they enter and remain in the dimension of neither perception nor non-perception. This is the seventh liberation. 

Going\marginnote{3.33.17} totally beyond the dimension of neither perception nor non-perception, they enter and remain in the cessation of perception and feeling. This is the eighth liberation. 

These\marginnote{3.33.19} are the eight liberations. 

Ānanda,\marginnote{3.34.1} this one time, when I was first awakened, I was staying in \textsanskrit{Uruvelā} at the goatherd’s banyan tree on the bank of the \textsanskrit{Nerañjarā} River. Then \textsanskrit{Māra} the wicked approached me, stood to one side, and said: ‘Sir, may the Blessed One now be fully extinguished! May the Holy One now be fully extinguished! Now is the time for the full extinguishment of the Buddha.’ When he had spoken, I said to \textsanskrit{Māra}: 

‘Wicked\marginnote{3.35.1} One, I shall not be fully extinguished until I have monk disciples … nun disciples … layman disciples … laywoman disciples who are competent, educated, assured, learned. 

I\marginnote{3.35.5} shall not be fully extinguished until my spiritual path is successful and prosperous, extensive, popular, widespread, and well proclaimed wherever there are gods and humans.’ 

Today,\marginnote{3.36.1} just now at the \textsanskrit{Cāpāla} shrine \textsanskrit{Māra} the Wicked approached me once more with the same request, reminding me of my former statement, and saying that those conditions had been fulfilled. 

When\marginnote{3.37.1} he had spoken, I said to \textsanskrit{Māra}: ‘Relax, Wicked One. The full extinguishment of the Realized One will be soon. Three months from now the Realized One will be fully extinguished.’ So today, just now at the \textsanskrit{Cāpāla} Tree-shrine, mindful and aware, I surrendered the life force.” 

\section*{20. The Appeal of Ānanda }

When\marginnote{3.38.1} he said this, Venerable Ānanda said to the Buddha, “Sir, may the Blessed One please remain for the eon! May the Holy One please remain for the eon! That would be for the welfare and happiness of the people, out of sympathy for the world, for the benefit, welfare, and happiness of gods and humans.” 

“Enough\marginnote{3.38.3} now, Ānanda. Do not beg the Realized One. Now is not the time to beg the Realized One.” 

For\marginnote{3.39.1} a second time … For a third time, Ānanda said to the Buddha, “Sir, may the Blessed One please remain for the eon! May the Holy One please remain for the eon! That would be for the welfare and happiness of the people, out of sympathy for the world, for the benefit, welfare, and happiness of gods and humans.” 

“Ānanda,\marginnote{3.39.4} do you have faith in the Realized One’s awakening?” 

“Yes,\marginnote{3.39.5} sir.” 

“Then\marginnote{3.39.6} why do you keep pressing me up to the third time?” 

“Sir,\marginnote{3.40.1} I have heard and learned this in the presence of the Buddha:\footnote{Note that this is the idiom that Ānanda uses when directly quoting the Buddha. } ‘Whoever has developed and cultivated the four bases of psychic power—made them a vehicle and a basis, kept them up, consolidated them, and properly implemented them—may, if they wish, live for the proper lifespan or what’s left of it. The Realized One has developed and cultivated the four bases of psychic power, made them a vehicle and a basis, kept them up, consolidated them, and properly implemented them. If he wished, the Realized One could live for the proper lifespan or what’s left of it.’” 

“Do\marginnote{3.40.4} you have faith, Ānanda?” 

“Yes,\marginnote{3.40.5} sir.” 

“Therefore,\marginnote{3.40.6} Ānanda, the misdeed is yours alone, the mistake is yours alone. For even though the Realized One dropped such an obvious hint, such a clear sign, you didn’t beg me to remain for the eon, or what’s left of it.\footnote{“Misdeed” is \textit{\textsanskrit{dukkaṭaṁ}}. This term is familiar as the most minor class of offences in the Vinaya. Here we see an informal use of the term as something that has been wrongly done, rather than a legal violation. It is used in a similar way by \textsanskrit{Mākassapa} at the First Council, who accuses Ānanda of several misdeeds. As a legal term, \textit{\textsanskrit{dukkaṭa}} is late; the category of offences is not found in the Vinayas of the \textsanskrit{Mahāsaṅghika} group of schools, which use \textit{vinayatikkrama} for a similar idea. } If you had begged me, I would have refused you twice, but consented on the third time. Therefore, Ānanda, the misdeed is yours alone, the mistake is yours alone. 

Ānanda,\marginnote{3.41.1} this one time I was staying near \textsanskrit{Rājagaha}, on the Vulture’s Peak Mountain.\footnote{The Buddha goes on to list multiple places where he hinted to Ānanda, but the only record of such a conversation is at the \textsanskrit{Cāpāla} Shrine. } There I said to you: ‘Ānanda, \textsanskrit{Rājagaha} is lovely, and so is the Vulture’s Peak. Whoever has developed and cultivated the four bases of psychic power—made them a vehicle and a basis, kept them up, consolidated them, and properly implemented them—may, if they wish, live for the proper lifespan or what’s left of it. The Realized One has developed and cultivated the four bases of psychic power, made them a vehicle and a basis, kept them up, consolidated them, and properly implemented them. If he wished, the Realized One could live for the proper lifespan or what’s left of it.’ But you didn’t get it, even though I dropped such an obvious hint, such a clear sign. You didn’t beg me to remain for the eon, or what’s left of it. If you had begged me, I would have refused you twice, but consented on the third time. Therefore, Ānanda, the misdeed is yours alone, the mistake is yours alone. 

Ānanda,\marginnote{3.42.1} this one time I was staying right there near \textsanskrit{Rājagaha}, at the Gotama banyan tree …\footnote{The following group of places are all near \textsanskrit{Rājagaha}, and collectively illustrate the variety of dwellings and environs enjoyed by the mendicants practicing there. | The Gotama banyan tree shrine is mentioned only here. Gotama is a common name, and this shrine does not appear to have any connection with the Buddha. } at Bandit’s Cliff …\footnote{A cliff from which bandits were tossed as a means of execution. } in the \textsanskrit{Sattapaṇṇi} cave on the slopes of Vebhara …\footnote{The Pali tradition says the First Council was held near the entrance to this cave  (\textsanskrit{Dīpavaṁsa} 4.19, \textsanskrit{Mahāvaṁsa} 3.19, \textsanskrit{Samantapāsādikā} \textsanskrit{Paṭhamamahāsaṅgītikathā}). The Vinayas all agree that it was at \textsanskrit{Rājagaha}, but traditions vary as to the exact location. } at the Black Rock on the slopes of Isigili …\footnote{A large open area where Jains did their penances (\href{https://suttacentral.net/mn14/en/sujato\#15.2}{MN 14:15.2}) and the Buddha taught occasionally (\href{https://suttacentral.net/sn8.10/en/sujato}{SN 8.10}), but it is most famous as the place the monks Godhika (\href{https://suttacentral.net/sn4.23/en/sujato}{SN 4.23}) and Vakkali took their lives (\href{https://suttacentral.net/sn22.87/en/sujato}{SN 22.87}). } in the Cool Grove, under the Snake’s Hood Grotto …\footnote{Mentioned several times in the suttas as a pleasant place to meditate, it is notable as the site of the conversion of the Buddha’s chief layman disciple \textsanskrit{Anāthapiṇḍika} (\href{https://suttacentral.net/sn10.8/en/sujato}{SN 10.8}). But the most famous event there was when the Buddha moderated the monk \textsanskrit{Soṇa}’s excessive striving (\href{https://suttacentral.net/an6.55/en/sujato}{AN 6.55}). } in the Hot Springs Monastery …\footnote{The hot springs near \textsanskrit{Rājagaha} were a popular place for monks to bathe, so much so that they prompted a rule ensuring that the monks did not monopolize the springs (\href{https://suttacentral.net/pli-tv-bu-vb-pc57/en/sujato}{Bu Pc 57}). They are still in use and just as popular as ever. } in the Bamboo Grove, the squirrels’ feeding ground …\footnote{A personal gift of King \textsanskrit{Bimbisāra}, this was the first permanent monastery offered to the Buddha and his \textsanskrit{Saṅgha} (\href{https://suttacentral.net/pli-tv-kd1/en/sujato\#22.17.3}{Kd 1:22.17.3}). } in \textsanskrit{Jīvaka}’s mango grove … in the Maddakucchi deer park …\footnote{\textsanskrit{Mahākappina} stayed here (\href{https://suttacentral.net/pli-tv-kd1/en/sujato\#5.3.1}{Kd 1:5.3.1}), and the Buddha rested there when his foot was first injured, apparently by Devadatta, before \textsanskrit{Jīvaka} offered his mango grove (\href{https://suttacentral.net/sn1.38/en/sujato}{SN 1.38}, \href{https://suttacentral.net/sn4.13/en/sujato}{SN 4.13}). } 

And\marginnote{3.43.1} in each place I said to you: ‘Ānanda, \textsanskrit{Rājagaha} is lovely, and so are all these places. … If he wished, the Realized One could live for the proper lifespan or what’s left of it.’ But you didn’t get it, even though I dropped such an obvious hint, such a clear sign. You didn’t beg me to remain for the eon, or what’s left of it. 

Ānanda,\marginnote{3.45.1} this one time I was staying right here near \textsanskrit{Vesālī}, at the Udena shrine …\footnote{The following group of places are all in the Vajjian country near \textsanskrit{Vesālī}. Whereas the varied sites around \textsanskrit{Rājagaha} reflect the rugged terrain there, at \textsanskrit{Vesālī} all the sites are tree-shrines, as the surrounding land is flat. \href{https://suttacentral.net/dn24/en/sujato\#1.11.5}{DN 24:1.11.5} indicates where they lay relative to \textsanskrit{Vesālī}; the Udena shrine was to the east, and the next three to the south, west, and north respectively. | \textit{Udena} is an epithet of the “arising” sun (Rig Veda 1.48.7), although here it may have the sense “prosperity”. } at the Gotamaka shrine …\footnote{Named after the local \textit{yakkha}, to whom offerings would have been made. It could get cold enough to snow, prompting the Buddha to lay down a rule permitting three robes (\href{https://suttacentral.net/pli-tv-kd1/en/sujato\#13.2.1}{Kd 1:13.2.1}). A short but galaxy-shaking discourse was once taught there (\href{https://suttacentral.net/an3.125/en/sujato\#0.3}{AN 3.125:0.3}). } at the Seven Maidens shrine …\footnote{The Pali has \textit{amba} (“mango”) and \textit{ambaka} (“maiden”) as variants. The commentary supports the latter, saying it was named for seven legendary princesses of Varanasi in the time of Buddha Kassapa, all of whom went on to become great disciples of our Buddha. I made this translation on the small island of Qimei off the south coast of Taiwan, where there is a shrine to the “seven maidens” who gave the island its name. Stories of “seven maidens” are widespread in myth and folklore all over the world, including Aboriginal Dreamtime stories. The archetype descends from the constellation known in English as Pleiades and in Sanskrit as \textsanskrit{Kṛttikā}. Since there are six main visible stars in the cluster today, the stories often tell of how the youngest of the seven sisters was lost. In the Buddhist telling this is \textsanskrit{Visākhā}, the only sister not to become an arahant in this life. } at the Many Sons shrine …\footnote{This was, obviously, a fertility shrine. } at the \textsanskrit{Sārandada} shrine …\footnote{Previously mentioned as the place where the seven principles of non-decline were taught to the Licchavis. The meaning of \textsanskrit{Sārandada} is obscure and variants profligate, but it might be derived from the Munda word \textit{sara}, “funeral pyre”. The commentary says it was taken over from an old shrine to a \textit{yakkha} of that name. } and just now, today at the \textsanskrit{Cāpāla} shrine. There I said to you: ‘Ānanda, \textsanskrit{Vesālī} is lovely. And the Udena, Gotamaka, Seven Maidens, Many Sons, \textsanskrit{Sārandada}, and \textsanskrit{Cāpāla} Tree-shrines are all lovely. Whoever has developed and cultivated the four bases of psychic power—made them a vehicle and a basis, kept them up, consolidated them, and properly implemented them—may, if they wish, live for the proper lifespan or what’s left of it. The Realized One has developed and cultivated the four bases of psychic power, made them a vehicle and a basis, kept them up, consolidated them, and properly implemented them. If he wished, the Realized One could live for the proper lifespan or what’s left of it.’ But you didn’t get it, even though I dropped such an obvious hint, such a clear sign. You didn’t beg me to remain for the eon, or what’s left of it, saying: ‘Sir, may the Blessed One please remain for the eon! May the Holy One please remain for the eon! That would be for the welfare and happiness of the people, out of sympathy for the world, for the benefit, welfare, and happiness of gods and humans.’ 

If\marginnote{3.47.7} you had begged me, I would have refused you twice, but consented on the third time. Therefore, Ānanda, the misdeed is yours alone, the mistake is yours alone. 

Did\marginnote{3.48.1} I not prepare for this when I explained that we must be parted and separated from all we hold dear and beloved? How could it possibly be so that what is born, created, conditioned, and liable to wear out should not wear out? The Realized One has discarded, eliminated, released, given up, relinquished, and surrendered the life force. He has categorically stated: ‘The full extinguishment of the Realized One will be soon. Three months from now the Realized One will be fully extinguished.’ It’s not possible for the Realized One, for the sake of life, to take back the life force once it has been given up like that.\footnote{“Take back” is \textit{\textsanskrit{paccāvamissati}}, which has several variants. It occurs in \href{https://suttacentral.net/ja69/en/sujato}{Ja 69} and \href{https://suttacentral.net/ja509/en/sujato}{Ja 509}, where it is the opposite of \textit{vamati}, to “expel, vomit”. } 

Come,\marginnote{3.48.8} Ānanda, let’s go to the Great Wood, the hall with the peaked roof.”\footnote{This was the major \textsanskrit{Saṅgha} residence near \textsanskrit{Vesālī}. The Great Wood, according to the commentary, stretched as far as the Himalayas. Later tradition says that a town should have three woods: a “great wood” for wilderness (\textit{\textsanskrit{mahāvana}}); a “prosperity wood” for resources (\textit{sirivana}); and an “ascetic wood” for spiritual practice (\textit{tapovana}). } 

“Yes,\marginnote{3.48.9} sir,” Ānanda replied. 

So\marginnote{3.49.1} the Buddha went with Ānanda to the hall with the peaked roof, and said to him, “Go, Ānanda, gather all the mendicants staying in the vicinity of \textsanskrit{Vesālī} together in the assembly hall.”\footnote{Compare the earlier request to gather all the mendicants around \textsanskrit{Rājagaha} (\href{https://suttacentral.net/dn16/en/sujato\#1.6.2}{DN 16:1.6.2}). } 

“Yes,\marginnote{3.49.3} sir,” replied Ānanda. He did what the Buddha asked, went up to him, bowed, stood to one side, and said to him, “Sir, the mendicant \textsanskrit{Saṅgha} has assembled. Please, sir, go at your convenience.” 

Then\marginnote{3.50.1} the Buddha went to the assembly hall, where he sat on the seat spread out and addressed the mendicants: 

“So,\marginnote{3.50.3} mendicants, having carefully memorized those things I have taught you from my direct knowledge, you should cultivate, develop, and make much of them so that this spiritual practice may last for a long time. That would be for the welfare and happiness of the people, out of sympathy for the world, for the benefit, welfare, and happiness of gods and humans.\footnote{Memorization was considered an essential basis for learning. In similar passages, we also find the injunction to “recite” the teachings, thus authorizing the recitation of the Buddha’s teachings after his death, which was later formalized in the Councils (\textit{\textsanskrit{saṅgīti}}). } And what are those things I have taught from my direct knowledge? They are: the four kinds of mindfulness meditation, the four right efforts, the four bases of psychic power, the five faculties, the five powers, the seven awakening factors, and the noble eightfold path.\footnote{These are the sets of practices later called the 37 \textit{\textsanskrit{bodhiyapakkhiyā} \textsanskrit{dhammā}}, the “wings to awakening”. Here they are presented as an essential summary of the Buddha’s teachings. They form the backbone of the final book of the \textsanskrit{Saṁyuttanikāya}, the Maggavagga (or \textsanskrit{Mahāvagga}). It is likely that the Buddha was referring to the earliest recension of this text. They are found as a summary of the Buddha’s teachings at \href{https://suttacentral.net/dn29/en/sujato\#17.3}{DN 29:17.3}, \href{https://suttacentral.net/mn103/en/sujato\#3.2}{MN 103:3.2}, \href{https://suttacentral.net/mn104/en/sujato\#5.3}{MN 104:5.3}, and \href{https://suttacentral.net/an8.19/en/sujato\#17.2}{AN 8.19:17.2} = \href{https://suttacentral.net/ud5.5/en/sujato\#25.2}{Ud 5.5:25.2}. } 

These\marginnote{3.51.1} are the things I have taught from my direct knowledge. Having carefully memorized them, you should cultivate, develop, and make much of them so that this spiritual practice may last for a long time. That would be for the welfare and happiness of the people, out of sympathy for the world, for the benefit, welfare, and happiness of gods and humans.” 

Then\marginnote{3.51.2} the Buddha said to the mendicants: 

“Come\marginnote{3.51.3} now, mendicants, I say to you all: ‘Conditions fall apart. Persist with diligence.’\footnote{This is perhaps the briefest summary possible of the Buddha’s teaching. The world as it is, and all the beings in it, are created and sustained through conditions. Since the conditions that sustain them are finite, the conditioned phenomena are also finite and must come to an end. Reflecting on this gives rise to gratitude for the life we have and the opportunities it grants us, and determination to live and practice dedicated to the realization of the unconditioned, Nibbana. } The full extinguishment of the Realized One will be soon. Three months from now the Realized One will be fully extinguished.” 

That\marginnote{3.51.7} is what the Buddha said. Then the Holy One, the Teacher, went on to say: 

\begin{verse}%
“I’ve\marginnote{3.51.9} reached a ripe old age,\footnote{“Ripe old age” is \textit{paripakko vayo}. } \\
and little of my life is left. \\
Having given it up, I’ll depart; \\
I’ve made a refuge for myself. 

Diligent\marginnote{3.51.13} and mindful, \\
be of good virtues, mendicants! \\
With well-settled thoughts,\footnote{“Well-settled thoughts” is \textit{\textsanskrit{susamāhitasaṅkappā}}. Compare \textit{\textsanskrit{asamāhitasaṅkappo}} at \href{https://suttacentral.net/an4.22/en/sujato\#5.3}{AN 4.22:5.3}. } \\
take good care of your minds. 

Whoever\marginnote{3.51.17} meditates diligently \\
in this teaching and training, \\
giving up transmigration through rebirths, \\
will make an end to suffering.” 

%
\end{verse}

\scendsection{The third recitation section. }

\section*{21. The Elephant Look }

Then\marginnote{4.1.1} the Buddha robed up in the morning and, taking his bowl and robe, entered \textsanskrit{Vesālī} for alms. Then, after the meal, on his return from almsround, he turned to gaze back at \textsanskrit{Vesālī}, the way that elephants do. He said to Venerable Ānanda:\footnote{\textit{\textsanskrit{Nāgāpalokitaṁ}} is the “elephant look”. There is a similar Sanskrit term \textit{\textsanskrit{siṁhāvalokana}}, the “lion look”, said to be the slow glance back that a lion makes as he leaves his kill. There is a nuance in meaning here, because while \textit{apalokana} literally means to “look back”, it is used in the sense to “take leave” before setting out on a journey. } “Ānanda, this will be the last time the Realized One sees \textsanskrit{Vesālī}. Come, Ānanda, let’s go to Wares Village.”\footnote{\textit{\textsanskrit{Bhaṇḍagāma}}, where \textit{\textsanskrit{bhaṇḍa}} means “wares”; it must have been a trading post. It seems the villages in this passage were named after their chief economic activity, so I translate them to highlight this pragmatic system. } 

“Yes,\marginnote{4.1.5} sir,” Ānanda replied. 

Then\marginnote{4.2.1} the Buddha together with a large \textsanskrit{Saṅgha} of mendicants arrived at Wares Village, and stayed there. There the Buddha addressed the mendicants: 

“Mendicants,\marginnote{4.2.4} due to not understanding and not penetrating four things, both you and I have wandered and transmigrated for such a very long time. What four? Noble ethics,\footnote{This four also found at \href{https://suttacentral.net/an4.1/en/sujato\#2.3}{AN 4.1:2.3} and \href{https://suttacentral.net/an7.66/en/sujato\#14.4}{AN 7.66:14.4}. } immersion, wisdom, and freedom. These noble ethics, immersion, wisdom, and freedom have been understood and comprehended. Craving for continued existence has been cut off; the conduit to rebirth is ended; now there’ll be no more future lives.” 

That\marginnote{4.3.1} is what the Buddha said. Then the Holy One, the Teacher, went on to say: 

\begin{verse}%
“Ethics,\marginnote{4.3.3} immersion, and wisdom, \\
and the supreme freedom: \\
these things have been understood \\
by Gotama the renowned. 

And\marginnote{4.3.7} so the Buddha, having insight,\footnote{Here the text refers to the Buddha in the third person; compare the previous set of verses in first person. Although the text states that these verses were spoken by the Buddha, it is possible that, as with many similar cases especially in the \textsanskrit{Aṅguttaranikāya}, the verses were added by redactors. } \\
explained this teaching to the mendicants. \\
The teacher made an end of suffering, \\
seeing clearly, he is fully quenched.”\footnote{Note the use of “quenched” (\textit{parinibbuto}) here while the Buddha is still alive. } 

%
\end{verse}

And\marginnote{4.4.1} while staying there, too, he often gave this Dhamma talk to the mendicants: 

“Such\marginnote{4.4.2} is ethics, such is immersion, such is wisdom. When immersion is imbued with ethics it’s very fruitful and beneficial. When wisdom is imbued with immersion it’s very fruitful and beneficial. When the mind is imbued with wisdom it is rightly freed from the defilements, namely, the defilements of sensuality, desire to be reborn, and ignorance.” 

\section*{22. The Four Great References }

When\marginnote{4.5.1} the Buddha had stayed in Wares Village as long as he pleased, he addressed Ānanda, “Come, Ānanda, let’s go to Elephant Village.”…\footnote{\textit{\textsanskrit{Hatthigāma}}, probably a village which specialized in training elephants. } 

“Let’s\marginnote{4.5.3} go to Mango Village.”…\footnote{\textit{\textsanskrit{Ambagāma}}. } 

“Let’s\marginnote{4.5.4} go to Black Plum Village.”…\footnote{\textit{\textsanskrit{Jambugāma}}. } 

“Let’s\marginnote{4.5.5} go to Bhoga City.”\footnote{Departing from the economic naming scheme, here \textit{bhoga} does not mean “wealth”. Rather, the Bhogas were one of the clans of the Vajji Federation. } 

“Yes,\marginnote{4.6.1} sir,” Ānanda replied. Then the Buddha together with a large \textsanskrit{Saṅgha} of mendicants arrived at Bhoga City, where he stayed at the Ānanda shrine. 

There\marginnote{4.7.2} the Buddha addressed the mendicants: “Mendicants, I will teach you the four great references.\footnote{“Great references” is \textit{\textsanskrit{mahāpadesa}}; also at \href{https://suttacentral.net/an4.180/en/sujato}{AN 4.180}. A different set of four \textit{\textsanskrit{mahāpadesas}} are \href{https://suttacentral.net/pli-tv-kd1/en/sujato\#40.1.1}{Kd 1:40.1.1}, which provide criteria by which mendicants may judge what is and is not allowable. } Listen and apply your mind well, I will speak.” 

“Yes,\marginnote{4.7.5} sir,” they replied. The Buddha said this: 

“Take\marginnote{4.8.1} a mendicant who says: ‘Reverend, I have heard and learned this in the presence of the Buddha:\footnote{The phrase used in the suttas when reporting a teaching heard directly from the Buddha, eg. \href{https://suttacentral.net/sn55.52/en/sujato\#5.1}{SN 55.52:5.1}, \href{https://suttacentral.net/sn22.90/en/sujato\#9.1}{SN 22.90:9.1}, \href{https://suttacentral.net/mn47/en/sujato\#10.7}{MN 47:10.7}, etc. } this is the teaching, this is the training, this is the Teacher’s instruction.’\footnote{\textit{Dhamma}, \textit{vinaya}, and \textit{\textsanskrit{satthusāsana}} are used in the suttas in the general sense of the entirety of the Buddha’s teachings. In his advice to \textsanskrit{Mahāpajāpatī} (\href{https://suttacentral.net/an8.53/en/sujato}{AN 8.53}) and \textsanskrit{Upāli} (\href{https://suttacentral.net/an7.83/en/sujato}{AN 7.83}), they are said to be whatever conforms with letting go. A stream-enterer who is a lay person is grounded in them (\href{https://suttacentral.net/an6.16/en/sujato\#7.2}{AN 6.16:7.2}). They are more frequent in the Vinaya, being used as a general qualifier of a legitimate legal procedure (\href{https://suttacentral.net/pli-tv-bu-vb-pc63/en/sujato\#2.1.8}{Bu Pc 63:2.1.8}). Due to this, the fact that there are three items, and the fact that \textit{dhamma} and \textit{sutta} are different words, we should not take \textit{dhamma} and \textit{vinaya} here as equivalent to \textit{sutta} and \textit{vinaya} below. Rather, they are a general claim to authenticity and do not specify particular texts. } You should neither approve nor dismiss that mendicant’s statement.\footnote{As at \href{https://suttacentral.net/mn112/en/sujato\#3.1}{MN 112:3.1}, \href{https://suttacentral.net/dn29/en/sujato\#18.4}{DN 29:18.4}, and \href{https://suttacentral.net/an4.180/en/sujato\#2.5}{AN 4.180:2.5}. } Instead, having carefully memorized those words and phrases, you should make sure they fit in the discourse and are exhibited in the training.\footnote{It is tempting to assume that \textit{sutta} here means the \textsanskrit{Suttapiṭaka} and \textit{vinaya} means the \textsanskrit{Vinayapiṭaka}, or at least an earlier version of these texts; but this is not supported by tradition. The late canonical Netti explains \textit{sutta} as “the four noble truths” and \textit{vinaya} as “the removal of greed, hate, and delusion” (\href{https://suttacentral.net/ne6/en/sujato}{Ne 6}). The commentaries discuss many interpretations; ultimately they agree with the Netti for \textit{vinaya}, but say \textit{sutta} means the entire \textsanskrit{Tipiṭaka}. I think the Netti is correct: teachings should “fit in” with the four noble truths (like other footprints fit in to an elephant’s footprint, \href{https://suttacentral.net/mn28/en/sujato}{MN 28}), and they should “be exhibited” in that the results of practice should be apparent (\textit{\textsanskrit{sandiṭṭhika}}). | “Memorize” (\textit{\textsanskrit{uggahetvā}}), because in an oral tradition a text is not learned until it is known by heart; such claims must be taken seriously. | “Fit into” is \textit{\textsanskrit{osāreti}} or \textit{\textsanskrit{otāreti}} per \href{https://suttacentral.net/an4.180/en/sujato}{AN 4.180}; the two words have a similar meaning, “to flow down into”. } If they do not fit in the discourse and are not exhibited in the training, you should draw the conclusion:\footnote{The text has \textit{ca} (“and”) rather than \textit{\textsanskrit{vā}} (“or”), which would be expected if they were to be found in one or other textual collection. This is urging that any teaching fits in with the fundamental principles of the four noble truths, and is effective in removing defilements. } ‘Clearly this is not the word of the Buddha. It has been incorrectly memorized by that mendicant.’ And so you should reject it. If they do fit in the discourse and are exhibited in the training, you should draw the conclusion: ‘Clearly this is the word of the Buddha. It has been correctly memorized by that mendicant.’ You should remember it. This is the first great reference.\footnote{The Pali text appears to say that one should remember the \textit{\textsanskrit{mahāpadesa}}, whereas the Sanskrit clearly says one should remember the passage (\textit{\textsanskrit{ayaṁ} dharmo’\textsanskrit{yaṁ} vinaya \textsanskrit{idaṁ} \textsanskrit{śāstuḥ} \textsanskrit{śāsanam} iti \textsanskrit{viditvā} \textsanskrit{dhārayitavyāḥ}}). It seems as if this pattern should underlie the Pali too, for if the passage is false, one “should reject it” (\textit{\textsanskrit{chaḍḍeyyātha}}). In the case of a correct passage we should expect a parallel injunction that one “should remember it” (\textit{\textsanskrit{dhāreyyātha}}). But if this applies to the \textit{\textsanskrit{mahāpadesa}} there is no such injunction. Thus it seems as if the Sanskrit is correct here and the Pali corrupt. I restore it by reversing the order of the phrases. } 

Take\marginnote{4.9.1} another mendicant who says: ‘In such-and-such monastery lives a \textsanskrit{Saṅgha} with seniors and leaders.\footnote{Each of the four references is a little less solid than the previous. Here one hasn’t heard from the Buddha, but from the \textsanskrit{Saṅgha}. The fact that one hears it from the \textsanskrit{Saṅgha} implies a kind of formal group recitation such as a \textit{\textsanskrit{saṅgīti}} or “council”. } I’ve heard and learned this in the presence of that \textsanskrit{Saṅgha}: this is the teaching, this is the training, this is the Teacher’s instruction.’ You should neither approve nor dismiss that mendicant’s statement. Instead, having carefully memorized those words and phrases, you should make sure they fit in the discourse and are exhibited in the training. If they do not fit in the discourse and are not exhibited in the training, you should draw the conclusion: ‘Clearly this is not the word of the Buddha. It has been incorrectly memorized by that \textsanskrit{Saṅgha}.’ And so you should reject it. If they do fit in the discourse and are exhibited in the training, you should draw the conclusion: ‘Clearly this is the word of the Buddha. It has been correctly memorized by that \textsanskrit{Saṅgha}.’ You should remember it. This is the second great reference. 

Take\marginnote{4.10.1} another mendicant who says: ‘In such-and-such monastery there are several senior mendicants who are very learned, inheritors of the heritage, who have memorized the teachings, the monastic law, and the outlines.\footnote{Now one learns not from a unified \textsanskrit{Saṅgha}, but from several learned elders. This is a stock description of learned mendicants (eg. \href{https://suttacentral.net/an3.20/en/sujato\#6.2}{AN 3.20:6.2}). | “Inheritors of the heritage” is \textit{\textsanskrit{āgatāgamā}}, where \textit{\textsanskrit{āgama}} means “what has come down”, namely the scriptural heritage. \textit{Āgama} is a synonym for \textit{\textsanskrit{nikāya}} in the sense of “collection of scripture”. | The “outlines” (\textit{\textsanskrit{mātikā}}, literally “matrix”) are the summary outlines of topics that served as seeds for the development of Abhidhamma. \href{https://suttacentral.net/dn16/en/sujato\#3.50.5}{DN 16:3.50.5} features one of the earliest of such lists, the 37 path factors that the Buddha “taught from his own direct knowledge”. These serve as outline for the section on the path in the \textsanskrit{Saṁyutta}, from where they were adopted in various Abhidhamma texts such as the \textsanskrit{Vibhaṅga}. } I’ve heard and learned this in the presence of those senior mendicants: this is the teaching, this is the training, this is the Teacher’s instruction.’ You should neither approve nor dismiss that mendicant’s statement. Instead, having carefully memorized those words and phrases, you should make sure they fit in the discourse and are exhibited in the training. If they do not fit in the discourse and are not exhibited in the training, you should draw the conclusion: ‘Clearly this is not the word of the Buddha. It has not been correctly memorized by those senior mendicants.’ And so you should reject it. If they do fit in the discourse and are exhibited in the training, you should draw the conclusion: ‘Clearly this is the word of the Buddha. It has been correctly memorized by those senior mendicants.’ You should remember it. This is the third great reference. 

Take\marginnote{4.11.1} another mendicant who says: ‘In such-and-such monastery there is a single senior mendicant who is very learned, an inheritor of the heritage, who has memorized the teachings, the monastic law, and the outlines.\footnote{Finally the testimony of a single mendicant, which is the weakest of all. Nonetheless, the procedure is the same. } I’ve heard and learned this in the presence of that senior mendicant: this is the teaching, this is the training, this is the Teacher’s instruction.’ You should neither approve nor dismiss that mendicant’s statement. Instead, having carefully memorized those words and phrases, you should make sure they fit in the discourse and are exhibited in the training. If they do not fit in the discourse and are not exhibited in the training, you should draw the conclusion: ‘Clearly this is not the word of the Buddha. It has been incorrectly memorized by that senior mendicant.’ And so you should reject it. If they do fit in the discourse and are exhibited in the training, you should draw the conclusion: ‘Clearly this is the word of the Buddha. It has been correctly memorized by that senior mendicant.’ You should remember it. This is the fourth great reference. 

These\marginnote{4.11.15} are the four great references.”\footnote{The parallel at \href{https://suttacentral.net/an4.180/en/sujato\#9.11}{AN 4.180:9.11} omits \textit{\textsanskrit{dhāreyyāthā}}, which I take to be the correct reading. See my note on \textit{\textsanskrit{dhāreyyāthā}} at \href{https://suttacentral.net/dn16/en/sujato\#4.8.13}{DN 16:4.8.13}. } 

And\marginnote{4.12.1} while staying at the Ānanda shrine, too, the Buddha often gave this Dhamma talk to the mendicants: 

“Such\marginnote{4.12.2} is ethics, such is immersion, such is wisdom. When immersion is imbued with ethics it’s very fruitful and beneficial. When wisdom is imbued with immersion it’s very fruitful and beneficial. When the mind is imbued with wisdom it is rightly freed from the defilements, namely, the defilements of sensuality, desire to be reborn, and ignorance.” 

\section*{23. On Cunda the Smith }

When\marginnote{4.13.1} the Buddha had stayed in Bhoga City as long as he pleased, he addressed Ānanda, “Come, Ānanda, let’s go to \textsanskrit{Pāvā}.”\footnote{\textsanskrit{Pāvā}, a town of the Mallas, has more significance for this narrative than appears at first sight. It was, according to Buddhist texts, the place where \textsanskrit{Mahāvīra} had recently died, plunging the Jains into chaos. (The Jains, however, say this was another \textsanskrit{Pāvā}, east of \textsanskrit{Nāḷandā}.) \textsanskrit{Pāvā} became associated with especially ascetic monks: thirty \textsanskrit{Pāvā} monks became awakened on hearing a particularly strong discourse (\href{https://suttacentral.net/sn15.13/en/sujato}{SN 15.13}); \textsanskrit{Mahākassapa} heard the news of the Buddha’s passing at \textsanskrit{Pāvā}; and monks from \textsanskrit{Pāvā} allied with monks of “Avanti and the south” arguing for strict Vinaya in the Second Council (\href{https://suttacentral.net/pli-tv-kd1/en/sujato\#1.7.11}{Kd 1:1.7.11}). } 

“Yes,\marginnote{4.13.3} sir,” Ānanda replied. Then the Buddha together with a large \textsanskrit{Saṅgha} of mendicants arrived at \textsanskrit{Pāvā},\footnote{This passage also at \href{https://suttacentral.net/ud8.5/en/sujato}{Ud 8.5}. } where he stayed in Cunda the smith’s mango grove.\footnote{Cunda was apparently a metal-worker, which was an advanced form of technological craftsmanship. } 

Cunda\marginnote{4.14.1} heard that the Buddha had arrived and was staying in his mango grove. Then he went to the Buddha, bowed, and sat down to one side. The Buddha educated, encouraged, fired up, and inspired him with a Dhamma talk.\footnote{This is, according to the commentary, preserved as the other discourse taught to Cunda, which deals with proper asceticism (\href{https://suttacentral.net/snp1.5/en/sujato}{Snp 1.5}). } Then Cunda said to the Buddha, “Sir, may the Buddha together with the mendicant \textsanskrit{Saṅgha} please accept tomorrow’s meal from me.” The Buddha consented with silence. 

Then,\marginnote{4.16.1} knowing that the Buddha had consented, Cunda got up from his seat, bowed, and respectfully circled the Buddha, keeping him on his right, before leaving. 

And\marginnote{4.17.1} when the night had passed Cunda had delicious fresh and cooked foods prepared in his own home, and plenty of pork on the turn. Then he had the Buddha informed of the time, saying,\footnote{The exact meaning of \textit{\textsanskrit{sūkaramaddava}} is unclear. \textit{\textsanskrit{Sūkara}} is “pig”, but some Chinese translations suggest the sense “mushroom”; and it is true that there are several names of plants or plant dishes that begin with \textit{\textsanskrit{sūkara}} or other animals. At \href{https://suttacentral.net/dhp377/en/sujato}{Dhp 377}, flowers are described as \textit{maddava}, meaning “overripe, withering on the vine”. It is a common practice to allow meat to sit for a while to become tender and “high” for extra flavor when cooked. But this can lead to a proliferation of dangerous bacteria unless properly cooked, and such seems to have been the case here. In any case, this disputed term is marginal in a consideration of meat-eating in early Buddhism, which is discussed more fully elsewhere. } “Sir, it’s time. The meal is ready.” 

Then\marginnote{4.18.1} the Buddha robed up in the morning and, taking his bowl and robe, went to the home of Cunda together with the mendicant \textsanskrit{Saṅgha}, where he sat on the seat spread out and addressed Cunda, “Cunda, please serve me with the pork on the turn that you’ve prepared.\footnote{If the Buddha knew the meal was dangerous, why ask to be served it? This is even more striking in light of the fact that it is a Vinaya offence to request fine food, including meat, and nowhere else is such a request recorded (\href{https://suttacentral.net/pli-tv-bu-vb-pc39/en/sujato\#2.10.1}{Bu Pc 39:2.10.1}). Such narrative ambiguities serve to escalate a sense of wonder. } And serve the mendicant \textsanskrit{Saṅgha} with the other foods.” 

“Yes,\marginnote{4.18.5} sir,” replied Cunda, and did as he was asked. 

Then\marginnote{4.19.1} the Buddha addressed Cunda, “Cunda, any pork on the turn that’s left over, you should bury it in a pit. I don’t see anyone in this world—with its gods, \textsanskrit{Māras}, and Divinities, this population with its ascetics and brahmins, its gods and humans—who could properly digest it except for the Realized One.” 

“Yes,\marginnote{4.19.4} sir,” replied Cunda. He did as he was asked, then came back to the Buddha, bowed, and sat down to one side. Then the Buddha educated, encouraged, fired up, and inspired him with a Dhamma talk, after which he got up from his seat and left. 

After\marginnote{4.20.1} the Buddha had eaten Cunda’s meal, he fell severely ill with bloody dysentery, struck by dreadful pains, close to death.\footnote{The Buddha’s illness is sometimes diagnosed as mesenteric infarction. The Buddha’s body rejects the world of conditioned existence entirely. } But he endured unbothered, with mindfulness and situational awareness. Then he addressed Ānanda, “Come, Ānanda, let’s go to \textsanskrit{Kusinārā}.” 

“Yes,\marginnote{4.20.5} sir,” Ānanda replied. 

\begin{verse}%
I’ve\marginnote{4.20.6} heard that after eating\footnote{The commentary says that these verses were added by the elders at the Council. } \\
the meal of Cunda the smith, \\
the attentive one fell severely ill, \\
with pains, close to death. 

A\marginnote{4.20.10} severe sickness struck the Teacher \\
who had eaten the pork on the turn. \\
While still purging the Buddha said: \\
“I’ll go to the citadel of \textsanskrit{Kusinārā}.” 

%
\end{verse}

\section*{24. Bringing a Drink }

Then\marginnote{4.21.1} the Buddha left the road and went to the root of a certain tree, where he addressed Ānanda, “Please, Ānanda, fold my outer robe in four and spread it out for me. I am tired and will sit down.”\footnote{The \textit{\textsanskrit{saṅghāṭi}} (“outer robe” or “cloak”) is double-layered, and served both for warmth and as bedding. The Buddha has not recovered fully from his illness. } 

“Yes,\marginnote{4.21.3} sir,” replied Ānanda, and did as he was asked. The Buddha sat on the seat spread out. 

When\marginnote{4.22.1} he was seated he said to Venerable Ānanda, “Please, Ānanda, fetch me some water. I am thirsty and will drink.”\footnote{The Buddha was an exemplary patient. He did not complain, but made his needs known clearly to his carer. } 

When\marginnote{4.22.3} he said this, Venerable Ānanda said to the Buddha, “Sir, just now around five hundred carts have passed by. The shallow water has been churned up by their wheels, and it flows cloudy and murky. The \textsanskrit{Kakutthā} river is not far away, with clear, sweet, cool water, clean, with smooth banks, delightful. There the Buddha can drink and cool his limbs.”\footnote{It would seem that, since the Buddha was enduring his illness without complaint and with dignity, Ānanda did not fully realize how weak he has become. } 

For\marginnote{4.23.1} a second time, the Buddha asked Ānanda for a drink, and for a second time Ānanda suggested going to the \textsanskrit{Kakutthā} river. 

And\marginnote{4.24.1} for a third time, the Buddha said to Ānanda, “Please, Ānanda, fetch me some water. I am thirsty and will drink.” 

“Yes,\marginnote{4.24.3} sir,” replied Ānanda. Taking his bowl he went to the river.\footnote{The bowl (\textit{patta}) was used both for eating and drinking. } Now, though the shallow water in that creek had been churned up by wheels, and flowed cloudy and murky, when Ānanda approached it flowed transparent, clear, and unclouded. 

Then\marginnote{4.25.1} Ānanda thought, “Oh, how incredible, how amazing! The Realized One has such psychic power and might! For though the shallow water in that creek had been churned up by wheels, and flowed cloudy and murky, when I approached it flowed transparent, clear, and unclouded.” Gathering a bowl of drinking water he went back to the Buddha, and said to him, “It’s incredible, sir, it’s amazing! The Realized One has such psychic power and might! Just now, though the shallow water in that creek had been churned up by wheels, and flowed cloudy and murky, when I approached it flowed transparent, clear, and unclouded. Drink the water, Blessed One! Drink the water, Holy One!” So the Buddha drank the water. 

\section*{25. On Pukkusa the Malla }

Now\marginnote{4.26.1} at that time Pukkusa the Malla, a disciple of \textsanskrit{Āḷāra} \textsanskrit{Kālāma}, was traveling along the road from \textsanskrit{Kusinārā} to \textsanskrit{Pāvā}.\footnote{Pukkusa is not elsewhere mentioned. The reference to \textsanskrit{Āḷāra} \textsanskrit{Kālāma} recalls the narrative of the bodhisatta’s practices before awakening. There are several details that indicate the stories of the first part and the last part of the Buddha’s life were unified by such callbacks. We meet the Malla clan later on. } He saw the Buddha sitting at the root of a certain tree. He went up to him, bowed, sat down to one side, and said, “It’s incredible, sir, it’s amazing! Those who have gone forth remain in such peaceful meditations. 

Once\marginnote{4.27.1} it so happened that \textsanskrit{Āḷāra} \textsanskrit{Kālāma}, while traveling along a road, left the road and sat at the root of a nearby tree for the day’s meditation. Then around five hundred carts passed by right beside \textsanskrit{Āḷāra} \textsanskrit{Kālāma}.\footnote{\textit{\textsanskrit{Niśraya}} in the sense of “near” occurs at \textsanskrit{Śatapathabrāhmaṇa} 3.1.2.17. } Then a certain person coming behind those carts went up to \textsanskrit{Āḷāra} \textsanskrit{Kālāma} and said to him: ‘Sir, didn’t you see the five hundred carts pass by?’\footnote{In deep meditation the senses cease to function. } 

‘No,\marginnote{4.27.5} friend, I didn’t see them.’ 

‘But\marginnote{4.27.6} sir, didn’t you hear a sound?’ 

‘No,\marginnote{4.27.7} friend, I didn’t hear a sound.’ 

‘But\marginnote{4.27.8} sir, were you asleep?’ 

‘No,\marginnote{4.27.9} friend, I wasn’t asleep.’ 

‘But\marginnote{4.27.10} sir, were you conscious?’ 

‘Yes,\marginnote{4.27.11} friend.’ ‘So, sir, while conscious and awake you neither saw nor heard a sound as five hundred carts passed by right beside you? Why sir, even your outer robe is covered with dust!’\footnote{As were \textsanskrit{Mahāpajāpatī}’s limbs (\href{https://suttacentral.net/an8.51/en/sujato}{AN 8.51}). } 

‘Yes,\marginnote{4.27.14} friend.’ 

Then\marginnote{4.27.15} that person thought: ‘Oh, how incredible, how amazing! Those who have gone forth remain in such peaceful meditations, in that, while conscious and awake he neither saw nor heard a sound as five hundred carts passed by right next to him.’ And after declaring his lofty confidence in \textsanskrit{Āḷāra} \textsanskrit{Kālāma}, he left.” 

“What\marginnote{4.28.1} do you think, Pukkusa? Which is harder and more challenging to do while conscious and awake: to neither see nor hear a sound as five hundred carts pass by right next to you? Or to neither see nor hear a sound as the heavens are raining and pouring, lightning’s flashing, and thunder’s cracking?” 

“What\marginnote{4.29.1} do five hundred carts matter, or six hundred, or seven hundred, or eight hundred, or nine hundred, or a thousand, or even a hundred thousand carts? It’s far harder and more challenging to neither see nor hear a sound as the heavens are raining and pouring, lightning’s flashing, and thunder’s cracking!” 

“This\marginnote{4.30.1} one time, Pukkusa, I was staying near \textsanskrit{Ātumā} in a threshing-hut.\footnote{A nearby town where the Buddha had stayed previously (\href{https://suttacentral.net/pli-tv-kd1/en/sujato\#37.4.2}{Kd 1:37.4.2}). } At that time the heavens were raining and pouring, lightning was flashing, and thunder was cracking. And not far from the threshing-hut two farmers who were brothers were killed, as well as four oxen. Then a large crowd came from \textsanskrit{Ātumā} to the place where that happened. 

Now\marginnote{4.31.1} at that time I came out of the threshing-hut and was walking mindfully in the open near the door of the hut. Then having left that crowd, a certain person approached me, bowed, and stood to one side. I said to them, ‘Why, friend, has this crowd gathered?’ 

‘Just\marginnote{4.32.2} now, sir, the heavens were raining and pouring, lightning was flashing, and thunder was cracking. And two farmers who were brothers were killed, as well as four oxen. Then this crowd gathered here. But sir, where were you?’ 

‘I\marginnote{4.32.5} was right here, friend.’ 

‘But\marginnote{4.32.6} sir, did you see?’ 

‘No,\marginnote{4.32.7} friend, I didn’t see anything.’ 

‘But\marginnote{4.32.8} sir, didn’t you hear a sound?’ 

‘No,\marginnote{4.32.9} friend, I didn’t hear a sound.’ 

‘But\marginnote{4.32.10} sir, were you asleep?’ 

‘No,\marginnote{4.32.11} friend, I wasn’t asleep.’ 

‘But\marginnote{4.32.12} sir, were you conscious?’ 

‘Yes,\marginnote{4.32.13} friend.’ 

‘So,\marginnote{4.32.14} sir, while conscious and awake you neither saw nor heard a sound as the heavens were raining and pouring, lightning was flashing, and thunder was cracking?’ 

‘Yes,\marginnote{4.32.15} friend.’ 

Then\marginnote{4.33.1} that person thought: ‘Oh, how incredible, how amazing! Those who have gone forth remain in such peaceful meditations, in that, while conscious and awake he neither saw nor heard a sound as the heavens were raining and pouring, lightning was flashing, and thunder was cracking.’ And after declaring their lofty confidence in me, they bowed and respectfully circled me, keeping me on their right, before leaving.” 

When\marginnote{4.34.1} he said this, Pukkusa said to him, “Any confidence I had in \textsanskrit{Āḷāra} \textsanskrit{Kālāma} I sweep away as in a strong wind, or float away as down a swift stream. Excellent, sir! Excellent! As if he were righting the overturned, or revealing the hidden, or pointing out the path to the lost, or lighting a lamp in the dark so people with clear eyes can see what’s there, the Buddha has made the teaching clear in many ways. I go for refuge to the Buddha, to the teaching, and to the mendicant \textsanskrit{Saṅgha}. From this day forth, may the Buddha remember me as a lay follower who has gone for refuge for life.” 

Then\marginnote{4.35.1} Pukkusa addressed a certain man, “Please, my man, fetch a pair of ready to wear garments the color of mountain gold.”\footnote{\textit{\textsanskrit{Siṅgīvaṇṇa}}, or else \textit{\textsanskrit{siṅgīsuvaṇṇa}} (\href{https://suttacentral.net/an3.70/en/sujato\#38.3}{AN 3.70:38.3}), is gold from a mountain peak (\textit{\textsanskrit{siṅgī}}). \textsanskrit{Kauṭilya} mentions a form of gold that is \textit{\textsanskrit{śṛṅgaśūktija}}, “occurring on mount \textsanskrit{Śūkti}” (\textsanskrit{Arthaśāstra} 2.13.3). The commentary to that passage says it had the color of red arsenic, so it would have been a rose gold color. } 

“Yes,\marginnote{4.35.3} sir,” replied that man, and did as he was asked. Then Pukkusa brought the garments to the Buddha, “Sir, please accept this pair of ready to wear garments the color of mountain gold from me out of sympathy.” 

“Well\marginnote{4.35.6} then, Pukkusa, clothe me in one, and Ānanda in the other.” 

“Yes,\marginnote{4.35.7} sir,” replied Pukkusa, and did so. 

Then\marginnote{4.36.1} the Buddha educated, encouraged, fired up, and inspired Pukkusa the Malla with a Dhamma talk, after which he got up from his seat, bowed, and respectfully circled the Buddha before leaving. 

Then,\marginnote{4.37.1} not long after Pukkusa had left, Ānanda placed the pair of garments the color of mountain gold by the Buddha’s body. But when placed by the Buddha’s body they seemed to lose their shine. Then Ānanda said to the Buddha, “It’s incredible, sir, it’s amazing, how pure and bright is the color of the Realized One’s skin. When this pair of ready to wear garments  the color of mountain gold is placed by the Buddha’s body they seem to lose their lustre.” 

“That’s\marginnote{4.37.6} so true, Ānanda, that’s so true! There are two times when the color of the Realized One’s skin becomes extra pure and bright. What two? The night when a Realized One understands the supreme perfect awakening; and the night he becomes fully extinguished in the element of extinguishment with no residue.\footnote{Closing the narrative circle with the story of awakening. } These are the are two times when the color of the Realized One’s skin becomes extra pure and bright. 

Today,\marginnote{4.38.4} Ānanda, in the last watch of the night, between a pair of sal trees in the sal forest of the Mallas at Upavattana near \textsanskrit{Kusinārā}, shall be the Realized One’s full extinguishment. Come, Ānanda, let’s go to the \textsanskrit{Kakutthā} River.” 

“Yes,\marginnote{4.38.6} sir,” Ānanda replied. 

\begin{verse}%
A\marginnote{4.38.7} pair of garments the color of mountain gold\footnote{The commentary says this verse was added by the senior monks at the Council. } \\
was presented by Pukkusa; \\
when the teacher was clothed with them, \\
his snow gold skin glowed bright. 

%
\end{verse}

Then\marginnote{4.39.1} the Buddha together with a large \textsanskrit{Saṅgha} of mendicants went to the \textsanskrit{Kakutthā} River. He plunged into the river and bathed and drank. And when he had emerged, he went to the mango grove, where he addressed Venerable Cundaka,\footnote{The monk Cundaka makes an abrupt appearance here; this is the only mention of this name. The commentary to this passage calls him \textit{cunda}, while the gloss to the parallel passage at \href{https://suttacentral.net/ud8.5/en/sujato\#14.3}{Ud 8.5:14.3} has \textit{cundaka} throughout. The diminutive ending \textit{-ka} is perhaps meant to differentiate him from the smith Cunda. There are various Cundas whose connection is unclear. \textsanskrit{Mahācunda} was one of the great disciples (\href{https://suttacentral.net/mn118/en/sujato}{MN 118:}, \href{https://suttacentral.net/an6.17/en/sujato}{AN 6.17}, \href{https://suttacentral.net/mn8/en/sujato}{MN 8}), who later brought the Dhamma to the land of the \textsanskrit{Cetīs} (\href{https://suttacentral.net/an6.46/en/sujato}{AN 6.46}, \href{https://suttacentral.net/an10.24/en/sujato}{AN 10.24}). He once stayed with Channa and \textsanskrit{Sāriputta} (\href{https://suttacentral.net/mn144/en/sujato\#4.1}{MN 144:4.1}, \href{https://suttacentral.net/sn35.87/en/sujato\#1.2}{SN 35.87:1.2}); the commentaries say he was in fact \textsanskrit{Sāriputta}’s younger brother. They also identify him with the “novice Cunda” who reported the deaths of \textsanskrit{Sāriputta} (\href{https://suttacentral.net/sn47.13/en/sujato\#1.3}{SN 47.13:1.3}) and \textsanskrit{Mahāvīra} (\href{https://suttacentral.net/dn29/en/sujato\#2.1}{DN 29:2.1}, \href{https://suttacentral.net/mn104/en/sujato\#3.1}{MN 104:3.1}), explaining that the title “novice” was a nickname that persisted from the time he ordained as a young novice. It is unclear whether the commentaries take the Cundaka of this passage to be the same person, but he is here performing a similar role as carer adjacent to the Buddha’s death. Later there appears a \textsanskrit{Cūḷacunda} (\href{https://suttacentral.net/tha-ap52/en/sujato\#13.3}{Tha Ap 52:13.3}). Thus there may have been one person known by different names, or several people whose stories have become conflated. } “Please, Cundaka, fold my outer robe in four and spread it out for me. I am tired and will lie down.” 

“Yes,\marginnote{4.40.1} sir,” replied Cundaka, and did as he was asked. And then the Buddha laid down in the lion’s posture—on the right side, placing one foot on top of the other—mindful and aware, and focused on the time of getting up. But Cundaka sat down right there in front of the Buddha. 

\begin{verse}%
Having\marginnote{4.41.1} gone to \textsanskrit{Kakutthā} Creek,\footnote{These verses were also added at the Council according to the commentary. } \\
whose water was transparent, sweet, and clear, \\
the Teacher, being tired, plunged in, \\
the Realized One, without compare in the world. 

And\marginnote{4.41.5} after bathing and drinking the Teacher emerged. \\
Before the group of mendicants, in the middle, \\>the Buddha, \\
the Teacher \\>who rolled forth the present dispensation,\footnote{Adopt PTS and BJT reading \textit{\textsanskrit{satthā} \textsanskrit{pavattā}}. } \\
the great seer went to the mango grove. 

He\marginnote{4.41.9} addressed the mendicant named Cundaka: \\
“Spread out my folded robe so I can lie down.” \\
The evolved one urged Cunda, \\
who quickly spread the folded robe. \\
The Teacher lay down so tired, \\
while Cunda sat there before him. 

%
\end{verse}

Then\marginnote{4.42.1} the Buddha said to Venerable Ānanda: 

“Now\marginnote{4.42.2} it may happen, Ānanda, that someone may give rise to regret in Cunda the smith: ‘It’s your loss, respected Cunda, it’s your misfortune, in that the Realized One was fully quenched after eating his last almsmeal from you.’\footnote{The text addresses Cunda with the respectful \textit{\textsanskrit{āvuso}}. } You should dispel remorse in Cunda the smith like this: ‘You’re fortunate, respected Cunda, you’re so very fortunate, in that the Realized One was fully quenched after eating his last almsmeal from you. I have heard and learned this in the presence of the Buddha. 

There\marginnote{4.42.8} are two almsmeal offerings that have identical fruit and result, and are more fruitful and beneficial than other almsmeal offerings.\footnote{Again, the narrative circle. } What two? The almsmeal after eating which a Realized One understands the supreme perfect awakening; and the almsmeal after eating which he becomes fully extinguished in the element of extinguishment with no residue. These two almsmeal offerings have identical fruit and result, and are more fruitful and beneficial than other almsmeal offerings. 

You’ve\marginnote{4.42.12} accumulated a deed that leads to long life, beauty, happiness, fame, heaven, and sovereignty.’\footnote{The text uses \textit{\textsanskrit{āyasmā}} here for Cunda, as does the Sanskrit, whereas it is normally reserved for mendicants (see below, \href{https://suttacentral.net/dn16/en/sujato\#6.2.1}{DN 16:6.2.1}). } That’s how you should dispel remorse in Cunda the smith.” 

Then,\marginnote{4.43.1} understanding this matter, on that occasion the Buddha expressed this heartfelt sentiment: 

\begin{verse}%
“A\marginnote{4.43.2} giver’s merit grows;\footnote{This became a point of discussion in later Buddhism: can it be that the merit of a gift grows after it is given? } \\
enmity doesn’t build up when you have self-control. \\
A skillful person gives up bad things—\\
with the end of greed, hate, and delusion, \\>they’re quenched.” 

%
\end{verse}

\scendsection{The fourth recitation section. }

\section*{26. The Pair of Sal Trees }

Then\marginnote{5.1.1} the Buddha said to Ānanda, “Come, Ānanda, let’s go to the far shore of the Golden River, and on to the sal forest of the Mallas at Upavattana near \textsanskrit{Kusinārā}.”\footnote{Known today as Kushinagar, it is a popular site for pilgrims, with many ancient stupas, Buddha images, and a peaceful park for meditation. } 

“Yes,\marginnote{5.1.3} sir,” Ānanda replied. And that’s where they went. Then the Buddha addressed Ānanda, “Please, Ānanda, set up a cot for me between the twin sal trees, with my head to the north. I am tired and will lie down.” 

“Yes,\marginnote{5.1.6} sir,” replied Ānanda, and did as he was asked. And then the Buddha laid down in the lion’s posture—on the right side, placing one foot on top of the other—mindful and aware.\footnote{Normally when the Buddha lies down, his mind is focused on getting up. But now he knows that he will not rise again. } 

Now\marginnote{5.2.1} at that time the twin sal trees were in full blossom with flowers out of season.\footnote{Sal trees blossom in April/May. This detail agrees with the Buddha’s final extinguishment in December/January, rather than in May (Vesak) as is currently celebrated (see note to \href{https://suttacentral.net/dn16/en/sujato\#3.9.3}{DN 16:3.9.3}). } They sprinkled and bestrewed the Realized One’s body in honor of the Realized One. And the flowers of the heavenly Flame Tree fell from the sky, and they too sprinkled and bestrewed the Realized One’s body in honor of the Realized One. And heavenly sandalwood powder fell from the sky, and it too sprinkled and bestrewed the Realized One’s body in honor of the Realized One. And heavenly music played in midair in honor of the Realized One. And heavenly choirs sang in midair in honor of the Realized One. 

Then\marginnote{5.3.1} the Buddha pointed out to Ānanda what was happening, adding: “That’s not the full extent of how the Realized One is honored, respected, revered, venerated, and esteemed. Any monk or nun or male or female lay follower who practices in line with the teachings, practicing properly, living in line with the teachings—they honor, respect, revere, venerate, and esteem the Realized One with the highest honor.\footnote{This calls back to the description of the fourfold assembly in the \textsanskrit{Māra} section above (\href{https://suttacentral.net/dn16/en/sujato\#3.7.4}{DN 16:3.7.4}). } So Ānanda, you should train like this: ‘We shall practice in line with the teachings, practicing properly, living in line with the teaching.’”\footnote{This kind of narrative elevation is characteristic of the Buddha’s teaching. He did not try to deny or eliminate any belief in the miraculous, or in the power of devotion, but rather to show that such things were of limited worth compared with practice. } 

\section*{27. The Monk \textsanskrit{Upavāna} }

Now\marginnote{5.4.1} at that time Venerable \textsanskrit{Upavāna} was standing in front of the Buddha fanning him.\footnote{Like Cundaka above, \textsanskrit{Upavāna} shares the duties of an attendant with Ānanda; at \href{https://suttacentral.net/dn29/en/sujato\#41.1}{DN 29:41.1} he is also fanning the Buddha. At \href{https://suttacentral.net/an5.166/en/sujato}{AN 5.166}, Ānanda goes to \textsanskrit{Upavāna} for support when he feels he has disappointed the Buddha. } Then the Buddha made him move, “Move over, mendicant, don’t stand in front of me.” 

Ānanda\marginnote{5.4.4} thought, “This Venerable \textsanskrit{Upavāna} has been the Buddha’s attendant for a long time, close to him, living in his presence.\footnote{In \href{https://suttacentral.net/sn7.13/en/sujato}{SN 7.13} \textsanskrit{Upavāna} fetched hot water and molasses for the Buddha, an event he remembered in his own verses at \href{https://suttacentral.net/thag2.33/en/sujato}{Thag 2.33}. } Yet in his final hour the Buddha makes him move, saying: ‘Move over, mendicant, don’t stand in front of me.’ What is the cause, what is the reason for this?” 

Then\marginnote{5.5.1} Ānanda said to the Buddha, “This Venerable \textsanskrit{Upavāna} has been the Buddha’s attendant for a long time, close to him, living in his presence. Yet in his final hour the Buddha makes him move, saying: ‘Move over, mendicant, don’t stand in front of me.’ What is the cause, sir, what is the reason for this?” 

“Most\marginnote{5.5.7} of the deities from ten solar systems have gathered to see the Realized One. For twelve leagues all around this sal grove there’s no spot, not even a fraction of a hair’s tip, that’s not crowded full of illustrious deities.\footnote{For this usage of \textit{\textsanskrit{phuṭo}}, see \href{https://suttacentral.net/an3.56/en/sujato\#1.3}{AN 3.56:1.3}. } The deities are complaining: ‘We’ve come such a long way to see the Realized One! Only rarely do Realized Ones arise in the world, perfected ones, fully awakened Buddhas. This very day, in the last watch of the night, the Realized One will be fully extinguished. And this illustrious mendicant is standing in front of the Buddha blocking the view. We won’t get to see the Realized One in his final hour!’” 

“But\marginnote{5.6.1} sir, what kind of deities are you thinking of?” 

“There\marginnote{5.6.2} are, Ānanda, deities—both in space and on the earth—who are aware of the earth. With hair disheveled and arms raised, they fall down like their feet were chopped off, rolling back and forth, lamenting:\footnote{Elsewhere, \textit{\textsanskrit{pathavīsaññī}} refers to those who develop a form of meditation where they are “percipient of earth” (\href{https://suttacentral.net/an10.6/en/sujato\#1.2}{AN 10.6:1.2}). Here, however, it simply means those deities who are aware of events on the ground. | Read \textit{\textsanskrit{chinnaṁpādaṁ} viya papatanti}. } ‘Too soon the Blessed One will be fully extinguished! Too soon the Holy One will be fully extinguished! Too soon the Eye of the World will vanish!’\footnote{The Buddha as “eye” evokes the common (eg. Rig Veda 1.164.14, 5.40.8, 5.59.5, 10.10.9) Vedic image of the Sun as the “eye of all” (\textit{\textsanskrit{viśvacakṣāḥ}}, 7.63.1), the “eye” for “eyes to see” (10.158.4), moving as an unaging wheel through the sky (1.164.14). See \href{https://suttacentral.net/snp3.9/en/sujato\#11.1}{Snp 3.9:11.1} = \href{https://suttacentral.net/mn98/en/sujato\#7.23}{MN 98:7.23}. | Pali has \textit{\textsanskrit{cakkhuṁ} loke} (“eye in the world”), Sanskrit has \textit{\textsanskrit{cakṣur} lokasya} (“eye of the world”). } 

But\marginnote{5.6.6} the deities who are free of desire endure, mindful and aware, thinking: ‘Conditions are impermanent. How could it possibly be otherwise?’”\footnote{These two reactions—grief and equanimity—are depicted often in Buddhist art and narrative. They stimulated the two poles of development of the Buddhist community. The devotional tradition, feeling the need for an emotional connection with the Teacher, developed art, story, and doctrines to, as it were, keep him alive, resulting in the \textsanskrit{Jātakas} and the Bodhisattva doctrine. A cooler, rational tradition developed a comprehensive system analyzing the impermanence of “conditions”, leading to the Abhidhamma. } 

\section*{28. The Four Inspiring Places }

“Previously,\marginnote{5.7.1} sir, when mendicants had completed the rainy season residence in various districts they came to see the Realized One.\footnote{As for example at \href{https://suttacentral.net/mn24/en/sujato\#2.1}{MN 24:2.1}. } We got to see the esteemed mendicants, and to pay homage to them.\footnote{\textit{\textsanskrit{Manobhāvanīya}} is explained in the commentaries as “those who, when seen, cause the mind to grow in what is skillful.” } But when the Buddha has passed, we won’t get to see the esteemed mendicants or to pay homage to them.” 

“Ānanda,\marginnote{5.8.1} a faithful gentleman should go to see these four inspiring places.\footnote{These four sites are the primary destinations of modern Buddhist pilgrims in India. | \textit{\textsanskrit{Saṁvejanīya}} is “stirring, provoking inspiration or urgency” such as when seeing an astonishing, disturbing, or amazing sight. } What four? Thinking: ‘Here the Realized One was born!’—that is an inspiring place.\footnote{\textsanskrit{Lumbinī} in modern Nepal, which today is a well-maintained and quiet place for devotion and meditation. The site is marked with an Ashokan pillar dated perhaps 150 years after this time. On the pillar is inscribed in \textsanskrit{Brahmī} characters \textit{hida \textsanskrit{bhagavaṁ} \textsanskrit{jāte} ti}, which is a direct quote in \textsanskrit{Māgadhī} of the Pali phrase here, \textit{idha \textsanskrit{tathāgato} \textsanskrit{jāto} ti}. (\textit{\textsanskrit{Tathāgata}} “realized one” is how the Buddha referred to himself, while \textit{\textsanskrit{bhagavā}} “blessed one” is how his followers referred to him.) This is the oldest direct quote from the early canon preserved in the archaeological record. } Thinking: ‘Here the Realized One became awakened as a supreme fully awakened Buddha!’—that is an inspiring place.\footnote{Bodhgaya in modern Bihar, which is called \textsanskrit{Uruvelā} in the Pali texts. It is a bustling center for pilgrims from all over the Buddhist world, at the center of which is the great stupa next to the Bodhi tree. } Thinking: ‘Here the supreme Wheel of Dhamma was rolled forth by the Realized One!’—that is an inspiring place.\footnote{Isipatana is modern Sarnath, in the northern part of Varanasi on the Ganges. In the temple next to the park, the first sermon is recited every night. | Note the passive voice, whereas the parallel at \href{https://suttacentral.net/an4.118/en/sujato}{AN 4.118} is active. } Thinking: ‘Here the Realized One was fully quenched in the element of extinguishment with no residue!’—that is an inspiring place.\footnote{The phrase \textit{\textsanskrit{nibbānadhātuyā} parinibbuto}, “fully quenched in the element of extinguishment” shows how the Pali uses two roots simultaneously for Nibbana. The better-known term, \textit{\textsanskrit{nibbāna}} (Sanskrit \textit{\textsanskrit{nirvāṇa}}), is from the root \textit{\textsanskrit{vā}} (“to blow”), with the basic sense of going out through being deprived of air. The past participle form \textit{nibbuta} (Sanskrit \textit{\textsanskrit{nirvṛta}}), from the root \textit{var} (“to check or restrain”), has the basic sense of freedom from obstacles and constrictions. Generally, the Pali treats \textit{nibbuta} as the past participle form of \textit{\textsanskrit{nibbāna}}, and for the most part the difference is purely verbal. Yet I believe the ambiguity is deliberate, as each captures a slightly different nuance of Nibbana. “Quenched” means both to go out like a flame and to be satisfied like a thirst. This positive feel appears in Sanskrit passages using \textit{\textsanskrit{nirvṛta}}, such as \textsanskrit{Manusmṛti} 1.54, where it describes the blissful absorption of all beings into the great divinity. By using “extinguishment” for \textit{\textsanskrit{nibbāna}} and “quenched” for \textit{nibbuta} I hope to express this subtle distinction, or at least to alert the reader that it exists. } These are the four inspiring places that a faithful gentleman should go to see. 

Faithful\marginnote{5.8.8} monks, nuns, laymen, and laywomen will come, and think: ‘Here the Realized One was born!’ and ‘Here the Realized One became awakened as a supreme fully awakened Buddha!’ and ‘Here the supreme Wheel of Dhamma was rolled forth by the Realized One!’ and ‘Here the Realized One was fully quenched in the element of extinguishment with no residue!’ Anyone who passes away while on pilgrimage to these shrines will, when their body breaks up, after death, be reborn in a good place, a heavenly realm.” 

\section*{29. Ānanda’s Questions }

“Sir,\marginnote{5.9.1} how do we proceed when it comes to females?”\footnote{This passage is absent from the Sanskrit parallel and it is probably a late interpolation. Ānanda was handsome, and many stories of his encounters with women are preserved, for example at \href{https://suttacentral.net/pli-tv-bu-vb-pc41/en/sujato\#1.1.5}{Bu Pc 41:1.1.5}. } 

“Without\marginnote{5.9.2} looking, Ānanda.”\footnote{As it stands, this appears to contradict \href{https://suttacentral.net/mn152/en/sujato\#2.10}{MN 152:2.10}, where the Buddha ridicules the idea that sense restraint implies not seeing. The commentary, however, explains it as not looking at a woman who is standing in the doorway of a monk’s hut, so as not to give rise to lust. Thus it restricts this apparently general rule to an unusually intimate encounter. } 

“But\marginnote{5.9.3} when looking, how to proceed?”\footnote{The verb changes from \textit{\textsanskrit{paṭipajjāma}} (first person plural) to \textit{\textsanskrit{paṭipajjitabbaṁ}} (future passive participle), a shift that mirrors the following passage regarding the funeral proceedings. There, the shift to future passive participle indicates that the subject is the lay folk who carry out the funeral, whereas here that does not apply. This suggests that this passage has been derived from that later passage. } 

“Without\marginnote{5.9.4} chatting, Ānanda.”\footnote{The commentary refers to \href{https://suttacentral.net/an5.55/en/sujato\#6.5}{AN 5.55:6.5}, which speaks of chatting alone in private with a woman, a circumstance also dealt with in \href{https://suttacentral.net/pli-tv-bu-vb-pc45/en/sujato}{Bu Pc 45}. } 

“But\marginnote{5.9.5} when chatting, how to proceed?”\footnote{Ānanda does not hesitate to let the Buddha know he has no intention of following his advice. } 

“Be\marginnote{5.9.6} mindful, Ānanda.”\footnote{The commentary says, quoting \href{https://suttacentral.net/sn35.127/en/sujato\#1.6}{SN 35.127:1.6}, that when a woman has sincere motivations, one should speak while thinking of them as a mother, a sister, or a daughter. } 

“Sir,\marginnote{5.10.1} how do we proceed when it comes to the Realized One’s corpse?”\footnote{“Corpse” is \textit{\textsanskrit{sarīra}}. } 

“Don’t\marginnote{5.10.2} get involved in the rites for venerating the Realized One’s corpse, Ānanda.\footnote{For \textit{\textsanskrit{abyāvaṭā}} (“don’t get involved”), compare \textit{\textsanskrit{samaṇena} \textsanskrit{bhavitabbaṁ} \textsanskrit{abyāvaṭena}} (“a monastic shouldn’t get involved” (in domestic matters)) at \href{https://suttacentral.net/pli-tv-bu-vb-ss5/en/sujato\#1.3.34}{Bu Ss 5:1.3.34}. } Please, Ānanda, you must all strive and practice for your own goal! Meditate diligent, keen, and resolute for your own goal!\footnote{Read \textit{sadatthe} (“own goal”) rather than \textit{\textsanskrit{sāratthe}} (“essential goal”). } There are astute aristocrats, brahmins, and householders who are devoted to the Realized One. They will perform the rites for venerating the Realized One’s corpse.” 

“But\marginnote{5.11.1} sir, how to proceed when it comes to the Realized One’s corpse?” 

“Proceed\marginnote{5.11.2} in the same way as they do for the corpse of a wheel-turning monarch.” 

“But\marginnote{5.11.3} how do they proceed with a wheel-turning monarch’s corpse?” 

“They\marginnote{5.11.4} wrap a wheel-turning monarch’s corpse with unworn cloth, then with uncarded cotton, then again with unworn cloth. In this way they wrap the corpse with five hundred double-layers. Then they place it in an iron case filled with oil and close it up with another case. Then, having built a funeral pyre out of all kinds of aromatics, they cremate the corpse.\footnote{An iron case was also used for Queen \textsanskrit{Bhaddā}’s body at \href{https://suttacentral.net/an5.50/en/sujato\#1.6}{AN 5.50:1.6}. The second iron case does not “enclose” the first; rather it “crooks” (\textit{\textsanskrit{paṭikujjati}}) like a lid to “close it up”. } They build a monument for the wheel-turning monarch at the crossroads.\footnote{This is a rare mention of the \textit{\textsanskrit{thūpa}} (“sepulchral momunent”) that became a major feature of the Buddhist landscape. Buddhist monuments were round in shape, which was apparently a characteristic of the lands around Magadha. \textsanskrit{Śatapathabrāhmaṇa} 13.8.1.5 says that the correct form of such a monument was square, and that round monuments were built by godless “easterners”. } That’s how they proceed with a wheel-turning monarch’s corpse. Proceed in the same way with the Realized One’s corpse.\footnote{Here the Buddha is said to be explicitly instructing his followers to adopt the building of monuments from the local customs. } A monument for the Realized One is to be built at the crossroads. When someone there lifts up garlands or fragrance or powder, or bows, or inspires confidence in their heart, that will be for their lasting welfare and happiness. 

\section*{30. Persons Worthy of Monument }

Ānanda,\marginnote{5.12.1} these four are worthy of a monument. What four? A Realized One, a perfected one, a fully awakened Buddha; an independent Buddha; a disciple of a Realized One; and a wheel-turning monarch. 

And\marginnote{5.12.4} for what reason is a Realized One worthy of a monument? So that many people will inspire confidence in their hearts, thinking: ‘This is the monument for that Blessed One, perfected and fully awakened!’ And having done so, when their body breaks up, after death, they are reborn in a good place, a heavenly realm. It is for this reason that a Realized One is worthy of a monument. 

And\marginnote{5.12.8} for what reason is an independent Buddha worthy of a monument?\footnote{“Independent Buddhas” are sages who discover the Dhamma and are awakened independently, but who do not themselves go on to found a dispensation or establish a monastic order. Text has \textit{paccekasambuddho} rather than the usual \textit{paccekabuddho}. } So that many people will inspire confidence in their hearts, thinking: ‘This is the monument for that independent Buddha!’ And having done so, when their body breaks up, after death, they are reborn in a good place, a heavenly realm. It is for this reason that an independent Buddha is worthy of a monument. 

And\marginnote{5.12.12} for what reason is a Realized One’s disciple worthy of a monument?\footnote{This refers to the “eight individuals” who make up the “\textsanskrit{Saṅgha} of disciples”, namely those who have achieved the four stages of awakening and those on the path. } So that many people will inspire confidence in their hearts, thinking: ‘This is the monument for that Blessed One’s disciple!’ And having done so, when their body breaks up, after death, they are reborn in a good place, a heavenly realm. It is for this reason that a Realized One’s disciple is worthy of a monument. 

And\marginnote{5.12.16} for what reason is a wheel-turning monarch worthy of a monument? So that many people will inspire confidence in their hearts, thinking: ‘This is the monument for that just and principled king!’\footnote{In contrast with \textsanskrit{Ajātasattu}. } And having done so, when their body breaks up, after death, they are reborn in a good place, a heavenly realm. It is for this reason that a wheel-turning monarch is worthy of a monument. 

These\marginnote{5.12.20} four are worthy of a monument.” 

\section*{31. Ānanda’s Incredible Qualities }

Then\marginnote{5.13.1} Venerable Ānanda entered a building, and stood there leaning against the door-jamb and crying,\footnote{Like Queen \textsanskrit{Subhaddā}, wife of \textsanskrit{Mahāsudassana}, at \href{https://suttacentral.net/dn17/en/sujato\#2.8.7}{DN 17:2.8.7} and \href{https://suttacentral.net/dn17/en/sujato\#2.12.1}{DN 17:2.12.1}. | \textit{\textsanskrit{Kapisīsa}} is door-jamb, not lintel. | For \textit{\textsanskrit{vihāra}} (“building”) the commentary has “pavilion” (\textit{\textsanskrit{maṇḍalamāla}}), which may have been a temporary construction for the occasion (compare \href{https://suttacentral.net/mn92/en/sujato\#4.6}{MN 92:4.6}). } “Oh! I’m still only a trainee with work left to do; and my Teacher is about to be fully extinguished, he who is so kind to me!”\footnote{At this point Ānanda was a stream-enterer. } 

Then\marginnote{5.13.3} the Buddha said to the mendicants, “Mendicants, where is Ānanda?” 

“Sir,\marginnote{5.13.5} Ānanda has entered a dwelling, and stands there leaning against the door-jamb and crying: ‘Oh! I’m still only a trainee with work left to do; and my Teacher is about to be fully extinguished, he who is so kind to me!’” 

So\marginnote{5.13.7} the Buddha addressed one of the monks, “Please, monk, in my name tell Ānanda that the teacher summons him.” 

“Yes,\marginnote{5.13.10} sir,” that monk replied. He went to Ānanda and said to him, “Reverend Ānanda, the teacher summons you.” 

“Yes,\marginnote{5.14.1} reverend,” Ānanda replied. He went to the Buddha, bowed, and sat down to one side. The Buddha said to him: 

“Enough,\marginnote{5.14.2} Ānanda! Do not grieve, do not lament. Did I not prepare for this when I explained that\footnote{When admonishing Ānanda, the Buddha first gently but firmly restrains him, then gives words of support and encouragement. } we must be parted and separated from all we hold dear and beloved? How could it possibly be so that what is born, created, conditioned, and liable to wear out should not wear out, even the Realized One’s body? For a long time, Ānanda, you’ve treated the Realized One with deeds of body, speech, and mind that are loving, beneficial, pleasant, undivided, and limitless.\footnote{The phrase “undivided and limitless” (\textit{advayena \textsanskrit{appamāṇena}}) normally describes \textit{\textsanskrit{kasiṇa}} meditation (\href{https://suttacentral.net/an10.25/en/sujato\#1.3}{AN 10.25:1.3}); here it is also found in the Sanskrit: \textit{\textsanskrit{ānanda} \textsanskrit{maitreṇa} \textsanskrit{kāyakarmaṇā} hitena \textsanskrit{sukhenādvayenāpramāṇena}}. } You have done good deeds, Ānanda. Devote yourself to meditation, and you will soon be free of defilements.”\footnote{On the eve of the First Council—in about six months time—Ānanda devoted himself to meditation and achieved arahantship. } 

Then\marginnote{5.15.1} the Buddha said to the mendicants: 

“The\marginnote{5.15.2} Buddhas of the past or the future have attendants who are no better than Ānanda is for me.\footnote{The same is said regarding the chief disciples \textsanskrit{Sāriputta} and \textsanskrit{Moggallāna} at \href{https://suttacentral.net/sn47.14/en/sujato\#2.4}{SN 47.14:2.4} and regarding assemblies of deities at \href{https://suttacentral.net/dn20/en/sujato\#4.3}{DN 20:4.3}. } Ānanda is astute, he is intelligent. He knows the time for monks, nuns, laymen, laywomen, king’s ministers, monastics of other religions and their disciples to visit the Realized One.\footnote{\textit{Tittha}, literally “ford”, is a path to salvation, used as a term for a non-Buddhist “religion”. \textit{Titthakara} is a “religious founder” (literally “ford-maker”); \textit{titthiya} is a “monastic of (another) religion” (for example at \href{https://suttacentral.net/pli-tv-bu-vb-np22/en/sujato\#1.2.5}{Bu NP 22:1.2.5}); \textit{\textsanskrit{titthiyasāvaka}} is a “disciple of a monastic of (another) religion”. } 

There\marginnote{5.16.1} are these four incredible and amazing things about Ānanda.\footnote{As at \href{https://suttacentral.net/an4.130/en/sujato\#6.4}{AN 4.130:6.4}. Ānanda was in awe of the “incredible and amazing” qualities of the Buddha, and spoke of them often. Here, when Ānanda is at his most vulnerable, the Buddha turns the teaching around, pointing out that Ānanda is incredible and amazing too. } What four? If an assembly of monks goes to see Ānanda, they’re uplifted by seeing him and uplifted by hearing him speak. And when he falls silent, they’ve never had enough. If an assembly of nuns … laymen … or laywomen goes to see Ānanda, they’re uplifted by seeing him and uplifted by hearing him speak. And when he falls silent, they’ve never had enough. These are the four incredible and amazing things about Ānanda. 

There\marginnote{5.16.16} are these four incredible and amazing things about a wheel-turning monarch.\footnote{This anticipates the story to follow. } What four? If an assembly of aristocrats goes to see a wheel-turning monarch, they’re uplifted by seeing him and uplifted by hearing him speak. And when he falls silent, they’ve never had enough. If an assembly of brahmins … householders … or ascetics goes to see a wheel-turning monarch, they’re uplifted by seeing him and uplifted by hearing him speak. And when he falls silent, they’ve never had enough. 

In\marginnote{5.16.26} the same way, there are those four incredible and amazing things about Ānanda.” 

\section*{32. Teaching the Discourse on \textsanskrit{Mahāsudassana} }

When\marginnote{5.17.1} he said this, Venerable Ānanda said to the Buddha: 

“Sir,\marginnote{5.17.2} please don’t be fully extinguished in this little hamlet, this jungle hamlet, this branch hamlet. There are other great cities such as \textsanskrit{Campā}, \textsanskrit{Rājagaha}, \textsanskrit{Sāvatthī}, \textsanskrit{Sāketa}, \textsanskrit{Kosambī}, and Varanasi.\footnote{\textsanskrit{Campā}, \textsanskrit{Rājagaha}, \textsanskrit{Sāvatthī}, and \textsanskrit{Kosambī} were the capitals of \textsanskrit{Aṅga}, Magadha, Kosala, and Vaccha respectively. \textsanskrit{Sāketa} was the former capital of Kosala. Varanasi was formerly the capital of \textsanskrit{Kāsi}, but at this time was contested by Kosala and Magadha, and had recently been won from \textsanskrit{Ajātasattu} (\href{https://suttacentral.net/sn3.14/en/sujato}{SN 3.14}, \href{https://suttacentral.net/sn3.15/en/sujato}{SN 3.15}). } Let the Buddha be fully extinguished there. There are many well-to-do aristocrats, brahmins, and householders there who are devoted to the Buddha. They will perform the rites of venerating the Realized One’s corpse.” 

“Don’t\marginnote{5.17.8} say that Ānanda! Don’t say that this is a little hamlet, a jungle hamlet, a branch hamlet. 

Once\marginnote{5.18.1} upon a time there was a king named \textsanskrit{Mahāsudassana} who was a wheel-turning monarch, a just and principled king. His dominion extended to all four sides, he achieved stability in the country, and he possessed the seven treasures.\footnote{This story is also found in \href{https://suttacentral.net/dn17/en/sujato}{DN 17}. It seems that the Pali tradition extracted the story and greatly expanded it in an independent long discourse, whereas the Sanskrit tradition kept it at a more moderate length within the \textsanskrit{Mahāparinirvāṇasūtra} itself. } His capital was this \textsanskrit{Kusinārā}, which at the time was named \textsanskrit{Kusāvatī}.\footnote{\textsanskrit{Kusāvatī} features in the \textsanskrit{Kusajātaka} (\href{https://suttacentral.net/ja531/en/sujato}{Ja 531}), where the ugly but wise prince Kusa, son of the legendary \textsanskrit{Okkāka}, wins the hand of the radiant \textsanskrit{Pabhāvatī}. The \textsanskrit{Rāmāyaṇa} also tells the story of a Kusa, son of \textsanskrit{Rāma}, who ruled the city of \textsanskrit{Kusāvatī}, although this city was located far to the south in the Vindhya ranges. Both stories are united by the detail that \textit{kusa} grass, a prominent feature of Vedic ritual, ensured the kingly lineage. } It stretched for twelve leagues from east to west, and seven leagues from north to south.\footnote{A “league” (\textit{yojana}) is usually estimated at between seven and twelve kilometers. By way of comparison, even at its greatest extent under Ashoka, \textsanskrit{Pāṭaliputta} was less than a league per side, so the dimensions of \textsanskrit{Kusāvatī} here are strictly legendary. } The royal capital of \textsanskrit{Kusāvatī} was successful, prosperous, populous, full of people, with plenty of food. It was just like \textsanskrit{Āḷakamandā}, the royal capital of the gods, which is successful, prosperous, populous, full of spirits, with plenty of food.\footnote{According to \href{https://suttacentral.net/dn32/en/sujato}{DN 32}, \textsanskrit{Āḷakamandā} was one of the many cities of Kuvera in Uttarakuru. } \textsanskrit{Kusāvatī} was never free of ten sounds by day or night, namely: the sound of elephants, horses, chariots, drums, clay drums, arched harps, singing, horns, gongs, and handbells; and the cry: ‘Eat, drink, be merry!’ as the tenth. 

Go,\marginnote{5.19.1} Ānanda, into \textsanskrit{Kusinārā} and inform the Mallas: ‘This very day, \textsanskrit{Vāseṭṭhas}, in the last watch of the night, the Realized One will be fully extinguished.\footnote{\textsanskrit{Vāseṭṭha} is a Vedic priestly clan. The Mallas adopted the name of their priest’s lineage, which was the normal custom for initiated \textit{khattiyas}. Other examples in the Pali are Saccaka who is called Aggivessana (\href{https://suttacentral.net/mn35/en/sujato\#4.2}{MN 35:4.2}), and the Buddha and his family who are called Gotama. } Come forth, \textsanskrit{Vāseṭṭhas}! Come forth, \textsanskrit{Vāseṭṭhas}! Don’t regret it later, thinking: ‘The Realized One became fully extinguished in our own village district, but we didn’t get a chance to see him in his final hour.’” 

“Yes,\marginnote{5.19.6} sir,” replied Ānanda. Then he robed up and, taking his bowl and robe, entered \textsanskrit{Kusinārā} with a companion.\footnote{\textit{Attadutiyo} “with self as second” is also at \href{https://suttacentral.net/mn146/en/sujato\#4.6}{MN 146:4.6} where Nandaka visits the nuns. Sanskrit has \textit{\textsanskrit{bhikṣuṇā} \textsanskrit{paścācchramaṇena}} (“with a mendicant as accompanying ascetic”). It was apparently evening (the “wrong time”, \textit{\textsanskrit{vikāla}}), when it is inappropriate for a monk to be wandering the town alone. } 

\section*{33. The Mallas Pay Homage }

Now\marginnote{5.20.1} at that time the Mallas of \textsanskrit{Kusinārā} were sitting together at the town hall on some business.\footnote{Evidently the Mallas, like the Vajjis, met frequently. } Ānanda went up to them, and announced: “This very day, \textsanskrit{Vāseṭṭhas}, in the last watch of the night, the Realized One will be fully extinguished. Come forth, \textsanskrit{Vāseṭṭhas}! Come forth, \textsanskrit{Vāseṭṭhas}! Don’t regret it later, thinking: ‘The Realized One became fully extinguished in our own village district, but we didn’t get a chance to see him in his final hour.’” 

When\marginnote{5.21.1} they heard what Ānanda had to say, the Mallas, their sons, daughters-in-law, and wives became distraught, saddened, and grief-stricken. And some, with hair disheveled and arms raised, falling down like their feet were chopped off, rolling back and forth, lamented, “Too soon the Blessed One will be fully extinguished! Too soon the Holy One will be fully extinguished! Too soon the Eye of the World will vanish!” 

Then\marginnote{5.21.3} the Mallas, their sons, daughters-in-law, and wives, distraught, saddened, and grief-stricken went to the Mallian sal grove at Upavattana and approached Ānanda. 

Then\marginnote{5.22.1} Ānanda thought, “If I have the Mallas pay homage to the Buddha one by one, they won’t be finished before first light. I’d better separate them family by family and then have them pay homage, saying: ‘Sir, the Malla named so-and-so with children, wives, retinue, and ministers bows with his head at your feet.’” And so that’s what he did. So by this means Ānanda got the Mallas to finish paying homage to the Buddha in the first watch of the night. 

\section*{34. On Subhadda the Wanderer }

Now\marginnote{5.23.1} at that time a wanderer named Subhadda was residing near \textsanskrit{Kusinārā}.\footnote{This is the second Subhadda we have met in this discourse, the first being a deceased devotee of \textsanskrit{Ñātika} (\href{https://suttacentral.net/dn16/en/sujato\#2.6.13}{DN 16:2.6.13}). The corrupt monk who, after the Buddha’s death, urges the rejection of the Vinaya rules is another person of the same name (\href{https://suttacentral.net/dn16/en/sujato\#6.20.1}{DN 16:6.20.1}). The \textsanskrit{Mahāsudassanasutta} also features a Queen \textsanskrit{Subhaddā} (\href{https://suttacentral.net/dn17/en/sujato\#2.5.9}{DN 17:2.5.9}). } He heard that on that very day, in the last watch of the night, will be the full extinguishment of the ascetic Gotama. He thought: “I have heard that brahmins of the past who were elderly and senior, the tutors of tutors, said: ‘Only rarely do Realized Ones arise in the world, perfected ones, fully awakened Buddhas.’ And this very day, in the last watch of the night, will be the full extinguishment of the ascetic Gotama. This state of uncertainty has come up in me. I am quite confident that the Buddha is capable of teaching me so that I can give up this state of uncertainty.” 

Then\marginnote{5.24.1} Subhadda went to the Mallian sal grove at Upavattana, approached Ānanda, and said to him, “Master Ānanda, I have heard that brahmins of the past who were elderly and senior, the tutors of tutors, said: ‘Only rarely do Realized Ones arise in the world, perfected ones, fully awakened Buddhas.’ And this very day, in the last watch of the night, will be the full extinguishment of the ascetic Gotama. This state of uncertainty has come up in me. I am quite confident that the Buddha is capable of teaching me so that I can give up this state of uncertainty. Mister Ānanda, please let me see the ascetic Gotama.” 

When\marginnote{5.24.8} he had spoken, Ānanda said, “Enough, Reverend Subhadda, do not trouble the Realized One. He is tired.” 

For\marginnote{5.24.10} a second time, and a third time, Subhadda asked Ānanda, and a third time Ānanda refused. 

The\marginnote{5.25.1} Buddha heard that discussion between Ānanda and Subhadda. He said to Ānanda, “Enough, Ānanda, don’t obstruct Subhadda; let him see the Realized One. For whatever he asks me, he will only be looking to understand, not to trouble me. And he will quickly understand any answer I give to his question.” 

So\marginnote{5.25.6} Ānanda said to the wanderer Subhadda, “Go, Reverend Subhadda, the Buddha is making time for you.” 

Then\marginnote{5.26.1} the wanderer Subhadda went up to the Buddha, and exchanged greetings with him. When the greetings and polite conversation were over, he sat down to one side and said to the Buddha: 

“Mister\marginnote{5.26.2} Gotama, there are those ascetics and brahmins who lead an order and a community, and tutor a community. They’re well-known and famous religious founders, deemed holy by many people. Namely: \textsanskrit{Pūraṇa} Kassapa, the bamboo-staffed ascetic \textsanskrit{Gosāla}, Ajita of the hair blanket, Pakudha \textsanskrit{Kaccāyana}, \textsanskrit{Sañjaya} \textsanskrit{Belaṭṭhiputta}, and the Jain ascetic of the \textsanskrit{Ñātika} clan. According to their own claims, did all of them have direct knowledge, or none of them, or only some?”\footnote{While some such as \textsanskrit{Mahāvīra} the \textsanskrit{Ñātika} (\href{https://suttacentral.net/mn14/en/sujato\#17.2}{MN 14:17.2}) and \textsanskrit{Pūraṇa} Kassapa (\href{https://suttacentral.net/an9.38/en/sujato\#2.1}{AN 9.38:2.1}) claimed to have direct knowledge, others such as Ajita Kesakambala denied that such knowledge was possible (\href{https://suttacentral.net/dn2/en/sujato\#23.2}{DN 2:23.2}). } 

“Enough,\marginnote{5.26.5} Subhadda, let that be.\footnote{The Buddha responded the same way when asked this question by the brahmin \textsanskrit{Piṅgalakoccha} (\href{https://suttacentral.net/mn30/en/sujato\#2.6}{MN 30:2.6}), and to a similar question at \href{https://suttacentral.net/an9.38/en/sujato\#3.2}{AN 9.38:3.2}. } I shall teach you the Dhamma. Listen and apply your mind well, I will speak.” 

“Yes,\marginnote{5.26.9} sir,” Subhadda replied. The Buddha said this: 

“Subhadda,\marginnote{5.27.1} in whatever teaching and training the noble eightfold path is not found, there is no ascetic found, no second ascetic, no third ascetic, and no fourth ascetic.\footnote{The four ascetics are defined at \href{https://suttacentral.net/an4.241/en/sujato\#1.1}{AN 4.241:1.1} as those on the four paths. } In whatever teaching and training the noble eightfold path is found, there is an ascetic found, a second ascetic, a third ascetic, and a fourth ascetic.\footnote{While many aspects of the eightfold path are shared with others, some details may be missing (such as not-self or \textsanskrit{Nibbāna}), while others are added (such as the belief in the efficacy of rituals, an eternal soul, or a creator god). } In this teaching and training the noble eightfold path is found. Only here is there an ascetic, here a second ascetic, here a third ascetic, and here a fourth ascetic. Other sects are empty of ascetics. 

Were\marginnote{5.27.4} these mendicants to practice well, the world would not be empty of perfected ones.\footnote{The Buddha points to the mendicants who have gathered there. } 

\begin{verse}%
I\marginnote{5.27.5} was twenty-nine years of age, Subhadda,\footnote{This is the only place in the early texts where the Buddha identifies his age when going forth. } \\
when I went forth to discover what is skillful.\footnote{Detailed in such suttas as \href{https://suttacentral.net/mn36/en/sujato}{MN 36}. } \\
It’s been over fifty years \\
since I went forth. \\
Teacher of the references \\>for the systematic teaching:\footnote{This verse appears to be corrupt. The sense can be restored through two extra lines in the Sanskrit: “Ethics, immersion, conduct, and knowledge, and unification of mind have been developed by me, teacher of the references for the noble teaching.” | For \textit{\textsanskrit{padesavattī}}, the Sanskrit has \textit{\textsanskrit{pradeśavaktā}}, where \textit{\textsanskrit{vaktā}} means “speaker” and \textit{\textsanskrit{pradeśa}} has the sense “pointing out”. I think it means, “I am the one who taught the four great references (\textit{\textsanskrit{mahāpadesa}})”. } \\
outside of here there is no ascetic, 

%
\end{verse}

no\marginnote{5.27.11} second ascetic, no third ascetic, and no fourth ascetic. Other sects are empty of ascetics. Were these mendicants to practice well, the world would not be empty of perfected ones.” 

When\marginnote{5.28.1} he had spoken, Subhadda said to the Buddha, “Excellent, sir! Excellent! As if he were righting the overturned, or revealing the hidden, or pointing out the path to the lost, or lighting a lamp in the dark so people with clear eyes can see what’s there, the Buddha has made the teaching clear in many ways. I go for refuge to the Buddha, to the teaching, and to the mendicant \textsanskrit{Saṅgha}. Sir, may I receive the going forth, the ordination in the Buddha’s presence?” 

“Subhadda,\marginnote{5.29.1} if someone formerly ordained in another sect wishes to take the going forth, the ordination in this teaching and training, they must spend four months on probation. When four months have passed, if the mendicants are satisfied, they’ll give the going forth, the ordination into monkhood. However, I have recognized individual differences in this matter.” 

“Sir,\marginnote{5.29.3} if four months probation are required in such a case, I’ll spend four years on probation. When four years have passed, if the mendicants are satisfied, let them give me the going forth, the ordination into monkhood.” 

Then\marginnote{5.29.4} the Buddha said to Ānanda, “Well then, Ānanda, give Subhadda the going forth.” 

“Yes,\marginnote{5.29.6} sir,” Ānanda replied. 

Then\marginnote{5.30.1} Subhadda said to Ānanda, “You’re so fortunate, Reverand Ānanda, so very fortunate, to be anointed here in the Teacher’s presence as his pupil!” And the wanderer Subhadda received the going forth, the ordination in the Buddha’s presence. Not long after his ordination, Venerable Subhadda, living alone, withdrawn, diligent, keen, and resolute, soon realized the supreme end of the spiritual path in this very life. He lived having achieved with his own insight the goal for which gentlemen rightly go forth from the lay life to homelessness. 

He\marginnote{5.30.6} understood: “Rebirth is ended; the spiritual journey has been completed; what had to be done has been done; there is nothing further for this place.” And Venerable Subhadda became one of the perfected. He was the last personal disciple of the Buddha.\footnote{The commentary says this line was added at the Council. } 

\scendsection{The fifth recitation section. }

\section*{35. The Buddha’s Last Words }

Then\marginnote{6.1.1} the Buddha addressed Venerable Ānanda: 

“Now,\marginnote{6.1.2} Ānanda, some of you might think: ‘The teacher’s dispensation has passed. Now we have no Teacher.’ But you should not see it like this. The teaching and training that I have taught and pointed out for you shall be your Teacher after my passing.\footnote{This is similar to the idea of the Four Great references, and sets the scene for the First Council at which the teachings were recited. | \textit{\textsanskrit{Paññatto}} here means “pointed out” rather than “laid down”, as can be seen from \href{https://suttacentral.net/dn9/en/sujato\#33.21}{DN 9:33.21}, where the same phrase refers to the four noble truths. } 

After\marginnote{6.2.1} my passing, mendicants ought not address each other as ‘reverend’, as they do today.\footnote{\textit{Āvuso} is from the root \textit{\textsanskrit{āyu}} (“age”) and thus has a respectful sense and does not mean “friend” as it is often translated. Like \textit{bhante}, it is an indeclinable vocative of address that may be used with or without the name (see eg. \href{https://suttacentral.net/mn5/en/sujato\#31.2}{MN 5:31.2}). } A more senior mendicant ought to address a more junior mendicant by name or clan, or by saying ‘reverend’. A more junior mendicant ought to address a more senior mendicant using ‘sir’ or ‘venerable’.\footnote{I render \textit{bhante} as “sir” when it stands alone and “honorable” when it prefixes a name. | Unlike \textit{bhante}, \textit{\textsanskrit{āyasmā}} (“venerable”) is declinable, so it is used in parts of speech other than direct address. It is from the same root as \textit{\textsanskrit{āvuso}} but with a slightly more respectful tone, perhaps because it sounds more Sanskritic. } 

If\marginnote{6.3.1} it wishes, after my passing the \textsanskrit{Saṅgha} may abolish the lesser and minor training rules.\footnote{These are not defined here, and the senior monks at the First Council were unable to agree on them (\href{https://suttacentral.net/pli-tv-kd21/en/sujato\#1.9.3}{Kd 21:1.9.3}). Nonetheless, the Pali Vinaya consistently labels the \textsanskrit{Pācittiya} rules as “lesser” (\textit{khuddaka}; \href{https://suttacentral.net/pli-tv-bu-vb-pc92/en/sujato\#2.2.22}{Bu Pc 92:2.2.22}, \href{https://suttacentral.net/pli-tv-bi-vb-pc96/en/sujato\#2.2.22}{Bi Pc 96:2.2.22}, \href{https://suttacentral.net/pli-tv-pvr1.1/en/sujato\#219.3}{Pvr 1.1:219.3}), which would make the \textsanskrit{Pātidesanīyas} “minor” (\textit{anukhuddaka}). The Sekhiya rules are also “minor”, but they were not at this point reckoned among the training rules for recitation. } 

After\marginnote{6.4.1} my passing, give the divine punishment to the mendicant Channa.”\footnote{“Divine punishment” is \textit{\textsanskrit{brahmadaṇḍa}}. Channa features often in the Vinaya as a monk who refuses correction and acts disrespectfully. The Sangha had already imposed an act of “ejection” (\textit{\textsanskrit{ukkhepanīyakamma}}) on him due to his persistent bad behavior, but that was still not enough (\href{https://suttacentral.net/pli-tv-kd1/en/sujato\#25.1.1}{Kd 1:25.1.1}). The \textit{\textsanskrit{brahmadaṇḍa}} was imposed at the First Council (\href{https://suttacentral.net/pli-tv-kd1/en/sujato\#1.12.1}{Kd 1:1.12.1}), upon which Channa finally saw the error of his ways. \textit{\textsanskrit{Brahmadaṇḍa}} is encountered in a different sense at \href{https://suttacentral.net/dn3/en/sujato\#1.23.21}{DN 3:1.23.21}. } 

“But\marginnote{6.4.2} sir, what is the divine punishment?” 

“Channa\marginnote{6.4.3} may say what he likes, but the mendicants should not correct, advise, or instruct him.”\footnote{\textit{Vattabba} in such contexts means “advise, correct” rather than more generally “speak to”. Thus the \textit{\textsanskrit{brahmadaṇḍa}} is not the “silent treatment”. } 

Then\marginnote{6.5.1} the Buddha said to the mendicants, “Perhaps even a single mendicant has doubt or uncertainty regarding the Buddha, the teaching, the \textsanskrit{Saṅgha}, the path, or the practice. So ask, mendicants! Don’t regret it later, thinking: ‘We were in the Teacher’s presence and we weren’t able to ask the Buddha a question.’” 

When\marginnote{6.5.4} this was said, the mendicants kept silent. 

For\marginnote{6.6.1} a second time, and a third time the Buddha addressed the mendicants: “Perhaps even a single mendicant has doubt or uncertainty regarding the Buddha, the teaching, the \textsanskrit{Saṅgha}, the path, or the practice. So ask, mendicants! Don’t regret it later, thinking: ‘We were in the Teacher’s presence and we weren’t able to ask the Buddha a question.’” 

For\marginnote{6.6.5} a third time, the mendicants kept silent. Then the Buddha said to the mendicants, 

“Mendicants,\marginnote{6.6.7} perhaps you don’t ask out of respect for the Teacher. So let a friend tell a friend.” 

When\marginnote{6.6.8} this was said, the mendicants kept silent. 

Then\marginnote{6.6.9} Venerable Ānanda said to the Buddha, “It’s incredible, sir, it’s amazing! I am quite confident that there is not even a single mendicant in this \textsanskrit{Saṅgha} who has doubt or uncertainty regarding the Buddha, the teaching, the \textsanskrit{Saṅgha}, the path, or the practice.” 

“Ānanda,\marginnote{6.6.11} you speak out of faith. But the Realized One knows that there is not even a single mendicant in this \textsanskrit{Saṅgha} who has doubt or uncertainty regarding the Buddha, the teaching, the \textsanskrit{Saṅgha}, the path, or the practice. Even the last of these five hundred mendicants is a stream-enterer, not liable to be reborn in the underworld, bound for awakening.” 

Then\marginnote{6.7.1} the Buddha said to the mendicants: “Come now, mendicants, I say to you all: ‘Conditions fall apart. Persist with diligence.’” 

These\marginnote{6.7.4} were the Realized One’s last words.\footnote{The commentary says this line was added at the Council. } 

\section*{36. Fully Quenched }

Then\marginnote{6.8.1} the Buddha entered the first absorption. Emerging from that, he entered the second absorption. Emerging from that, he successively entered into and emerged from the third absorption, the fourth absorption, the dimension of infinite space, the dimension of infinite consciousness, the dimension of nothingness, and the dimension of neither perception nor non-perception. Then he entered the cessation of perception and feeling.\footnote{Even on his deathbed, the Buddha retains mastery over his mind. } 

Then\marginnote{6.8.2} Venerable Ānanda said to Venerable Anuruddha, “Honorable Anuruddha, has the Buddha become fully quenched?”\footnote{Following the commentary, which reads this as a question. Anuruddha was renowned for his psychic powers. Note that Ānanda and Anuruddha have immediately adopted the forms of address recommended by the Buddha above. } 

“No,\marginnote{6.8.4} Reverend Ānanda. He has entered the cessation of perception and feeling.” 

Then\marginnote{6.9.1} the Buddha emerged from the cessation of perception and feeling and entered the dimension of neither perception nor non-perception. Emerging from that, he successively entered into and emerged from the dimension of nothingness, the dimension of infinite consciousness, the dimension of infinite space, the fourth absorption, the third absorption, the second absorption, and the first absorption. Emerging from that, he successively entered into and emerged from the second absorption and the third absorption. Then he entered the fourth absorption. Emerging from that the Buddha immediately became fully extinguished. 

When\marginnote{6.10.1} the Buddha was fully quenched, along with the full extinguishment there was a great earthquake, awe-inspiring and hair-raising, and thunder cracked the sky. When the Buddha was fully quenched, the divinity Sahampati recited this verse: 

\begin{verse}%
“All\marginnote{6.10.3} creatures in this world\footnote{Each of these characters reveal something of themselves in their verses. Sahampati, as a royal deity, emphasizes the universal nature of the teaching and the grandeur of the Buddha. } \\
must lay down this bag of bones.\footnote{“Bag of bones” is a loose rendering of \textit{samussaya}. } \\
For even a Teacher such as this, \\
unrivaled in the world, \\
the Realized One, attained to power, \\
the Buddha was fully quenched.” 

%
\end{verse}

When\marginnote{6.10.9} the Buddha was fully quenched, Sakka, lord of gods, recited this verse: 

\begin{verse}%
“Oh!\marginnote{6.10.10} Conditions are impermanent,\footnote{Less creative than \textsanskrit{Brahmā}, Sakka repeats a famous verse spoken by the Buddha at \href{https://suttacentral.net/sn15.20/en/sujato\#8.1}{SN 15.20:8.1} and \href{https://suttacentral.net/dn17/en/sujato\#2.17.5}{DN 17:2.17.5}. } \\
their nature is to rise and fall; \\
having arisen, they cease; \\
their stilling is blissful.” 

%
\end{verse}

When\marginnote{6.10.14} the Buddha was fully quenched, Venerable Anuruddha recited this verse: 

\begin{verse}%
“There\marginnote{6.10.15} was no more breathing\footnote{Anuruddha was a reclusive meditator who specialized in mindfulness of breathing. } \\
for the unaffected one of steady heart. \\
Imperturbable, committed to peace, \\
the sage has done his time. 

He\marginnote{6.10.19} put up with painful feelings \\
without flinching. \\
The liberation of his heart \\
was like the extinguishing of a lamp.” 

%
\end{verse}

When\marginnote{6.10.23} the Buddha was fully quenched, Venerable Ānanda recited this verse: 

\begin{verse}%
“Then\marginnote{6.10.24} there was terror!\footnote{Ānanda has the most emotional reaction. While Anuruddha speaks only of peace, Ānanda empathizes with those who were distraught. } \\
Then they had goosebumps! \\
When the Buddha, endowed with all fine qualities, \\
became fully quenched.” 

%
\end{verse}

When\marginnote{6.10.28} the Buddha was fully quenched, some of the mendicants there who were not free of desire, with arms raised, falling down like their feet were chopped off, rolling back and forth, lamented: “Too soon the Blessed One has become fully quenched! Too soon the Holy One has become fully quenched! Too soon the Eye of the World has vanished!”\footnote{Text omits “with hair disheveled” (\textit{kese pakiriya}) when describing the shaven-headed monks. } But the mendicants who were free of desire endured, mindful and aware, thinking, “Conditions are impermanent. How could it possibly be otherwise?” 

Then\marginnote{6.11.1} Anuruddha addressed the mendicants: “Enough, reverends, do not grieve or lament. Did the Buddha not prepare us for this when he explained that we must be parted and separated from all we hold dear and beloved? How could it possibly be so that what is born, created, conditioned, and liable to wear out should not wear out? The deities are complaining.” 

“But\marginnote{6.11.7} sir, what kind of deities are you thinking of?” 

“There\marginnote{6.11.8} are, Ānanda, deities—both in space and on the earth—who are aware of the earth. With hair disheveled and arms raised, they fall down like their feet were chopped off, rolling back and forth, lamenting: ‘Too soon the Blessed One has become fully quenched! Too soon the Holy One has become fully quenched! Too soon the Eye of the World has vanished!’ But the deities who are free of desire endure, mindful and aware, thinking: ‘Conditions are impermanent. How could it possibly be otherwise?’” 

Ānanda\marginnote{6.11.14} and Anuruddha spent the rest of the night talking about Dhamma. 

Then\marginnote{6.12.1} Anuruddha said to Ānanda, “Go, Ānanda, into \textsanskrit{Kusinārā} and inform the Mallas: ‘\textsanskrit{Vāseṭṭhas}, the Buddha has become fully quenched. Please come at your convenience.’” 

“Yes,\marginnote{6.12.5} sir,” replied Ānanda. Then, in the morning, he robed up and, taking his bowl and robe, entered \textsanskrit{Kusinārā} with a companion. 

Now\marginnote{6.12.6} at that time the Mallas of \textsanskrit{Kusinārā} were sitting together at the town hall still on the same business.\footnote{“Still on the same business” (\textit{teneva \textsanskrit{karaṇīyena}}) calls back to \href{https://suttacentral.net/dn16/en/sujato\#5.20.1}{DN 16:5.20.1}, where they were said to be “on some business” (\textit{kenacideva \textsanskrit{karaṇīyena}}). They had been discussing all night. } Ānanda went up to them, and announced, “\textsanskrit{Vāseṭṭhas}, the Buddha has become fully quenched. Please come at your convenience.” 

When\marginnote{6.12.10} they heard what Ānanda had to say, the Mallas, their sons, daughters-in-law, and wives became distraught, saddened, and grief-stricken. And some, with hair disheveled and arms raised, falling down like their feet were chopped off, rolling back and forth, lamented, “Too soon the Blessed One has become fully quenched! Too soon the Holy One has become fully quenched! Too soon the Eye of the World has vanished!” 

\section*{37. The Rites of Venerating the Buddha’s Corpse }

Then\marginnote{6.13.1} the Mallas ordered their men, “So then, my men, collect fragrances and garlands, and all the musical instruments in \textsanskrit{Kusinārā}.” 

Then—taking\marginnote{6.13.3} those fragrances and garlands, all the musical instruments, and five hundred pairs of garments—they went to the Mallian sal grove at Upavattana and approached the Buddha’s corpse. They spent the day honoring, respecting, revering, and venerating the Buddha’s corpse with dance and song and music and garlands and fragrances, and making awnings and setting up pavilions.\footnote{An uplifting response to tragedy, full of beauty and celebration. } 

Then\marginnote{6.13.4} they thought, “It’s too late to cremate the Buddha’s corpse today. Let’s do it tomorrow.” But they spent the next day the same way, and so too the third, fourth, fifth, and sixth days.\footnote{It seem unlikely that everyone simply forgot. It was probably the custom to wait seven days before the cremation. } 

Then\marginnote{6.14.1} on the seventh day they thought, “Honoring, respecting, revering, and venerating the Buddha’s corpse with dance and song and music and garlands and fragrances, let us carry it to the south of the town, and cremate it there outside the town.” 

Now\marginnote{6.14.3} at that time eight of the leading Mallas, having bathed their heads and dressed in unworn clothes, said,\footnote{“Unworn” is \textit{ahata}, as at \href{https://suttacentral.net/dn14/en/sujato\#1.33.9}{DN 14:1.33.9} and \href{https://suttacentral.net/pli-tv-kd1/en/sujato\#1.6.2}{Kd 1:1.6.2}. } “We shall lift the Buddha’s corpse.” But they were unable to do so. 

The\marginnote{6.14.5} Mallas said to Anuruddha, “What is the cause, Honorable Anuruddha, what is the reason why these eight Mallian chiefs are unable to lift the Buddha’s corpse?” 

“\textsanskrit{Vāseṭṭhas},\marginnote{6.14.8} you have one plan, but the deities have a different one.” 

“But\marginnote{6.15.1} sir, what is the deities’ plan?” 

“You\marginnote{6.15.2} plan to carry the Buddha’s corpse to the south of the town while venerating it with dance and song and music and garlands and fragrances, and cremate it there outside the town. The deities plan to carry the Buddha’s corpse to the north of the town while venerating it with heavenly dance and song and music and garlands and fragrances. Then they plan to enter the town by the northern gate, carry it through the center of the town, leave by the eastern gate, and cremate it there at the Mallian shrine named Coronation.”\footnote{“Coronation” is \textit{\textsanskrit{makuṭabandhana}}, “the binding of the crown”. The commentary says there was, fittingly, an auspicious decorative royal hall there; perhaps too the name was felt to pun with \textit{muktabandhana}, “freedom from ties”. } 

“Sir,\marginnote{6.15.6} let it be as the deities plan.” 

Now\marginnote{6.16.1} at that time the whole of \textsanskrit{Kusinārā} was covered knee-deep with the flowers of the Flame Tree, without gaps even on the filth and rubbish heaps.\footnote{I think \textit{sandhi} here means “covered without gaps”. } Then the deities and the Mallas of \textsanskrit{Kusinārā} carried the Buddha’s corpse to the north of the town while venerating it with heavenly and human dance and song and music and garlands and fragrances. Then they entered the town by the northern gate, carried it through the center of the town, left by the eastern gate, and deposited the corpse there at the Mallian shrine named Coronation. 

Then\marginnote{6.17.1} the Mallas said to Ānanda, “Honorable Ānanda, how do we proceed when it comes to the Realized One’s corpse?” 

“Proceed\marginnote{6.17.3} in the same way as they do for the corpse of a wheel-turning monarch.” 

“But\marginnote{6.17.4} how do they proceed with a wheel-turning monarch’s corpse?” 

“They\marginnote{6.17.5} wrap a wheel-turning monarch’s corpse with unworn cloth, then with uncarded cotton, then again with unworn cloth. In this way they wrap the corpse with five hundred double-layers. Then they place it in an iron case filled with oil and close it up with another case. Then, having built a funeral pyre out of all kinds of aromatics, they cremate the corpse. They build a monument for the wheel-turning monarch at the crossroads. That’s how they proceed with a wheel-turning monarch’s corpse. Proceed in the same way with the Realized One’s corpse. A monument for the Realized One is to be built at the crossroads. When someone there lifts up garlands or fragrance or powder, or bows, or inspires confidence in their heart, that will be for their lasting welfare and happiness.” 

Then\marginnote{6.18.1} the Mallas ordered their men, “So then, my men, collect uncarded cotton.” 

So\marginnote{6.18.3} the Mallas wrapped the Buddha’s corpse, and placed it in an iron case filled with oil. Then, having built a funeral pyre out of all kinds of aromatics, they lifted the corpse on to the pyre. 

\section*{38. \textsanskrit{Mahākassapa}’s Arrival }

Now\marginnote{6.19.1} at that time Venerable \textsanskrit{Mahākassapa} was traveling along the road from \textsanskrit{Pāvā} to \textsanskrit{Kusinārā} together with a large \textsanskrit{Saṅgha} of five hundred mendicants.\footnote{With the passing of \textsanskrit{Sāriputta} and \textsanskrit{Moggallāna}, \textsanskrit{Mahākassapa} was the most senior of the remaining leading mendicants. As a solitary recluse, it was unusual for him to be travelling with such a large group, or with anyone at all really. He was to cite the events depicted here at the start of the First Council (\href{https://suttacentral.net/pli-tv-kd1/en/sujato\#1.1.1}{Kd 1:1.1.1}). } Then he left the road and sat at the root of a tree. 

Now\marginnote{6.19.3} at that time a certain \textsanskrit{Ājīvaka} ascetic had picked up a Flame Tree flower in \textsanskrit{Kusinārā} and was traveling along the road to \textsanskrit{Pāvā}.\footnote{A follower of the Bamboo-staffed Ascetic \textsanskrit{Gosāla} (\href{https://suttacentral.net/dn2/en/sujato\#19.1}{DN 2:19.1}). | This may be an inspiration for the so-called “Flower Sermon”, a medieval Zen story that depicts the Buddha holding up a flower and Mahakassapa smiling. } \textsanskrit{Mahākassapa} saw him coming off in the distance and said to him, “Reverend, might you know about our Teacher?” 

“Yes,\marginnote{6.19.6} reverend. Seven days ago the ascetic Gotama was fully quenched. From there I picked up this Flame Tree flower.” Some of the mendicants there who were not free of desire, with arms raised, falling down like their feet were chopped off, rolling back and forth, lamented, “Too soon the Blessed One has become fully quenched! Too soon the Holy One has become fully quenched! Too soon the Eye of the World has vanished!” But the mendicants who were free of desire endured, mindful and aware, thinking, “Conditions are impermanent. How could it possibly be otherwise?” 

Now\marginnote{6.20.1} at that time a monk named Subhadda, who had gone forth when old, was sitting in that assembly. He said to those mendicants, “Enough, reverends, do not grieve or lament. We’re well rid of that Great Ascetic. And we are oppressed:\footnote{Compare \href{https://suttacentral.net/pli-tv-bu-vb-pc72/en/sujato}{Bu Pc 72}. The syntax is clumsy here, perhaps deliberately so. } ‘This is allowable for you; this is not allowable for you.’ Well, now we shall do what we want and not do what we don’t want.” 

Then\marginnote{6.20.6} Venerable \textsanskrit{Mahākassapa} addressed the mendicants, “Enough, reverends, do not grieve or lament. Did the Buddha not prepare us for this when he explained that we must be parted and separated from all we hold dear and beloved? How could it possibly be so that what is born, created, conditioned, and liable to wear out should not wear out, even the Realized One’s body?” 

Now\marginnote{6.21.1} at that time four of the leading Mallas, having bathed their heads and dressed in unworn clothes, said, “We shall light the Buddha’s funeral pyre.” But they were unable to do so. 

The\marginnote{6.21.3} Mallas said to Anuruddha, “What is the cause, Venerable Anuruddha, what is the reason why these four Mallian chiefs are unable to light the Buddha’s funeral pyre?” 

“\textsanskrit{Vāseṭṭhas},\marginnote{6.21.6} the deities have a different plan.” 

“But\marginnote{6.21.7} sir, what is the deities’ plan?” 

“The\marginnote{6.21.8} deities’ plan is this: Venerable \textsanskrit{Mahākassapa} is traveling along the road from \textsanskrit{Pāvā} to \textsanskrit{Kusinārā} together with a large \textsanskrit{Saṅgha} of five hundred mendicants. The Buddha’s funeral pyre shall not burn until he bows with his head at the Buddha’s feet.” 

“Sir,\marginnote{6.21.11} let it be as the deities plan.” 

Then\marginnote{6.22.1} Venerable \textsanskrit{Mahākassapa} arrived at the Mallian shrine named Coronation at \textsanskrit{Kusinārā} and approached the Buddha’s funeral pyre. Arranging his robe over one shoulder and raising his joined palms, he respectfully circled the Buddha three times, keeping him on his right, and bowed with his head at the Buddha’s feet. And the five hundred mendicants did likewise. And when \textsanskrit{Mahākassapa} and the five hundred mendicants bowed the Buddha’s funeral pyre burst into flames all by itself. 

And\marginnote{6.23.1} when the Buddha’s corpse was cremated no ash or soot was found from outer or inner skin, flesh, sinews, or synovial fluid. Only the relics remained.\footnote{Here \textit{\textsanskrit{sarīrāneva}} is plural and so must mean “relics”, whereas previously it was singular, “corpse”. } It’s like when ghee or oil blaze and burn, and neither ashes nor soot are found. In the same way, when the Buddha’s corpse was cremated no ash or soot was found from outer or inner skin, flesh, sinews, or synovial fluid. Only the relics remained. And of those five hundred pairs of garments only two were not burnt: the innermost and the outermost. But when the Buddha’s corpse was consumed the funeral pyre was extinguished by a stream of water that appeared in the sky,\footnote{As when he was born (\href{https://suttacentral.net/dn14/en/sujato\#1.28.1}{DN 14:1.28.1}). } by water dripping from the sal trees, and by the Mallas’ fragrant water. 

Then\marginnote{6.23.10} the Mallas made a cage of spears for the Buddha’s relics in the town hall and surrounded it with a buttress of bows. For seven days they honored, respected, revered, and venerated them with dance and song and music and garlands and fragrances. 

\section*{39. Distributing the Relics }

King\marginnote{6.24.1} \textsanskrit{Ajātasattu} of Magadha, son of the princess of Videha, heard\footnote{\textsanskrit{Ajātasattu} would have learned of the news from spies. It is a 600 km round trip to \textsanskrit{Rājagaha} and back, which a mounted messenger could make in fourteen days. } that the Buddha had become fully quenched at \textsanskrit{Kusinārā}. He sent an envoy to the Mallas of \textsanskrit{Kusinārā}: “The Buddha was an aristocrat and so am I. I too deserve a share of the Buddha’s relics. I will build a monument for them and conduct a memorial service.”\footnote{A bold move, considering that his designs on the Vajjis were no secret. Perhaps he was seeking a pretext for war. The justification for taking a share of relics is caste, rather than practice of the Dhamma. } 

The\marginnote{6.24.5} Licchavis of \textsanskrit{Vesālī} also heard that the Buddha had become fully quenched at \textsanskrit{Kusinārā}. They sent an envoy to the Mallas of \textsanskrit{Kusinārā}: “The Buddha was an aristocrat and so are we. We too deserve a share of the Buddha’s relics. We will build a monument for them and conduct a memorial service.” 

The\marginnote{6.24.9} Sakyans of Kapilavatthu also heard that the Buddha had become fully quenched at \textsanskrit{Kusinārā}. They sent an envoy to the Mallas of \textsanskrit{Kusinārā}: “The Buddha was our foremost relative. We too deserve a share of the Buddha’s relics. We will build a monument for them and conduct a memorial service.” 

The\marginnote{6.24.13} Bulis of Allakappa also heard\footnote{Both the tribe and the town are exceedingly obscure, mentioned nowhere else in early texts. They must have been a small clan nearby. } that the Buddha had become fully quenched at \textsanskrit{Kusinārā}. They sent an envoy to the Mallas of \textsanskrit{Kusinārā}: “The Buddha was an aristocrat and so are we. We too deserve a share of the Buddha’s relics. We will build a monument for them and conduct a memorial service.” 

The\marginnote{6.24.17} Koliyans of \textsanskrit{Rāmagāma} also heard\footnote{The Koliyans were south-eastern neighbors of the Sakyans, and several of their towns and people feature in the early texts. } that the Buddha had become fully quenched at \textsanskrit{Kusinārā}. They sent an envoy to the Mallas of \textsanskrit{Kusinārā}: “The Buddha was an aristocrat and so are we. We too deserve a share of the Buddha’s relics. We will build a monument for them and conduct a memorial service.” 

The\marginnote{6.24.21} brahmin of \textsanskrit{Veṭhadīpa} also heard\footnote{This brahmin is mentioned nowhere else. } that the Buddha had become fully quenched at \textsanskrit{Kusinārā}. He sent an envoy to the Mallas of \textsanskrit{Kusinārā}: “The Buddha was an aristocrat and I am a brahmin. I too deserve a share of the Buddha’s relics. I will build a monument for them and conduct a memorial service.” 

The\marginnote{6.24.25} Mallas of \textsanskrit{Pāvā} also heard that the Buddha had become fully quenched at \textsanskrit{Kusinārā}. They sent an envoy to the Mallas of \textsanskrit{Kusinārā}: “The Buddha was an aristocrat and so are we. We too deserve a share of the Buddha’s relics. We will build a monument for them and conduct a memorial service.” 

When\marginnote{6.25.1} they had spoken, the Mallas of \textsanskrit{Kusinārā} said to those various groups: “The Buddha was fully quenched in our village district. We will not give away a share of his relics.”\footnote{\textit{\textsanskrit{Dassāma}} (“we shall give”) is the future second plural of \textit{dadati}. } 

Then\marginnote{6.25.3} \textsanskrit{Doṇa} the brahmin said to those various groups:\footnote{The brahmin \textsanskrit{Doṇa} appears suddenly in the narrative, a reminder that there were many more people than the ones who are mentioned. The suttas record two encounters with a brahmin of this name: one is the wondrous story of seeing the Buddhas footprints (\href{https://suttacentral.net/an4.36/en/sujato}{AN 4.36}), while the other discusses the five kinds of brahmin (\href{https://suttacentral.net/an5.192/en/sujato}{AN 5.192}). } 

\begin{verse}%
“Hear,\marginnote{6.25.4} sirs, a single word from me. \\
Our Buddha’s teaching was acceptance. \\
It would not be good to fight over\footnote{The fear of war was justified and the resolution achieved by \textsanskrit{Doṇa} probably marks the last time these parties achieved a diplomatic outcome. The relative peace that had lasted most of the Buddha’s life was crumbling. We hear of war or threats of war between the Kosalans and the Magadhans, the Magadhans and the Vajjis, the Sakyans and the Koliyans, the Kosalans and the Mallas, and the Kosalans and the Sakyans. It is probably because of the latter two conflicts that \textsanskrit{Viḍūḍabha}—Pasenadi’s son and the newly crowned king of Kosala—did not send an emissary to the funeral. By sparking conflicts with former allies the Sakyans and Mallas, \textsanskrit{Viḍūḍabha} undid the successes of his father and fatally weakened the Kosalan Empire. When the dust cleared a few decades later, all these lands had been consumed by Magadha. } \\
a share of the supreme person’s relics. 

Let\marginnote{6.25.8} us make eight portions, good sirs, \\
rejoicing in unity and harmony. \\
Let there be monuments far and wide,\footnote{Thus begins the practice of establishing Buddhism by interring relics in a stupa. } \\
so many folk may gain faith in the Clear-eyed One!” 

%
\end{verse}

“Well\marginnote{6.25.12} then, brahmin, you yourself should fairly divide the Buddha’s relics in eight portions.” 

“Yes,\marginnote{6.25.13} sirs,” replied \textsanskrit{Doṇa} to those various groups. He divided the relics as asked and said to them, “Sirs, please give me the urn, and I shall build a monument for it and conduct a memorial service.” So they gave \textsanskrit{Doṇa} the urn. 

The\marginnote{6.26.1} Moriyas of Pippalivana heard\footnote{The Moriyas were a minor clan of the region, unmentioned outside of this passage, but their obscurity was not to last long. About a century after these events, Chandragupta the Moriyan, having won the Magadhan crown from the Nandas, proceeded to route the Greeks in the west. His empire, which covered most of northern India, was further expanded to the south by his son Bindusara and grandson Ashoka, under whom the Mauryan Empire became the greatest of all Indian empires. Thus \textsanskrit{Ajātasattu}’s expansionist dreams were ultimately fulfilled beyond his imagining. } that the Buddha had become fully quenched at \textsanskrit{Kusinārā}. They sent an envoy to the Mallas of \textsanskrit{Kusinārā}: “The Buddha was an aristocrat, and so are we. We too deserve a share of the Buddha’s relics. We will build a monument for them and conduct a memorial service.” 

“There\marginnote{6.26.5} is no portion of the Buddha’s relics left, they have already been portioned out. Here, take the embers.” So they took the embers. 

\section*{40. Venerating the Relics }

Then\marginnote{6.27.1} King \textsanskrit{Ajātasattu} of Magadha,\footnote{The commentary says this summary was added at the Council. } the Licchavis of \textsanskrit{Vesālī}, the Sakyans of Kapilavatthu, the Bulis of Allakappa, the Koliyans of \textsanskrit{Rāmagāma}, the brahmin of \textsanskrit{Veṭhadīpa}, the Mallas of \textsanskrit{Pāvā}, the Mallas of \textsanskrit{Kusinārā}, the brahmin \textsanskrit{Doṇa}, and the Moriyas of Pippalivana built monuments for them and conducted memorial services. Thus there were eight monuments for the relics, a ninth for the urn, and a tenth for the embers. That is how it was in the old days.\footnote{The commentary says this line was added at the Third Council, which was held about 150 years after the Buddha under Ashoka. The Buddha’s life was already fading into legend. } 

\begin{verse}%
There\marginnote{6.28.1} were eight shares \\>of the Clear-eyed One’s relics.\footnote{According to the commentary, the remainder of the text was added by the monks of Sri Lanka. Note, however, that this verse and the next are fairly similar to those in the Sanskrit text. Since that is a northern text, it seems unlikely these verses were composed in Sri Lanka. } \\
Seven were worshipped in the Black Plum Tree Land. \\
But one share of the most excellent of men \\
was worshipped in \textsanskrit{Rāmagāma} by a dragon king. 

One\marginnote{6.28.5} tooth is venerated \\>by the gods of the Three and Thirty, \\
and one is worshipped in the city of \textsanskrit{Gandhāra}; \\
another one in the realm of the \textsanskrit{Kaliṅga} King, \\
and one is worshipped by a dragon king. 

Through\marginnote{6.28.9} their glory this rich earth \\
is adorned with the best of offerings.\footnote{“Offering” is \textit{\textsanskrit{āyāga}}.  } \\
Thus the Clear-eyed One’s corpse \\
is well honored by the honorable. 

It’s\marginnote{6.28.13} venerated by lords of gods, dragons, and spirits; \\
and likewise venerated by the finest lords of men. \\
Honor it with joined palms when you get the chance, \\
for a Buddha is rare even in a hundred eons. 

Altogether\marginnote{6.28.17} forty even teeth, \\
and the body hair and head hair, \\
were carried off individually by gods \\
across the universe. 

%
\end{verse}

%
\chapter*{{\suttatitleacronym DN 17}{\suttatitletranslation King Mahāsudassana }{\suttatitleroot Mahāsudassanasutta}}
\addcontentsline{toc}{chapter}{\tocacronym{DN 17} \toctranslation{King Mahāsudassana } \tocroot{Mahāsudassanasutta}}
\markboth{King Mahāsudassana }{Mahāsudassanasutta}
\extramarks{DN 17}{DN 17}

\scevam{So\marginnote{1.1.1} I have heard. }At one time the Buddha was staying between a pair of sal trees in the sal forest of the Mallas at Upavattana near \textsanskrit{Kusinārā} at the time of his full extinguishment.\footnote{Picking up the story of the Buddha’s past life as King \textsanskrit{Mahāsudassana} from \href{https://suttacentral.net/dn16/en/sujato\#5.17.1}{DN 16:5.17.1}. } 

Then\marginnote{1.2.1} Venerable Ānanda went up to the Buddha, bowed, sat down to one side, and said to him, “Sir, please don’t be fully extinguished in this little hamlet, this jungle hamlet, this branch hamlet. There are other great cities such as \textsanskrit{Campā}, \textsanskrit{Rājagaha}, \textsanskrit{Sāvatthī}, \textsanskrit{Sāketa}, \textsanskrit{Kosambī}, and Varanasi. Let the Buddha be fully extinguished there. There are many well-to-do aristocrats, brahmins, and householders there who are devoted to the Buddha. They will perform the rites of venerating the Realized One’s corpse.” 

“Don’t\marginnote{1.3.1} say that, Ānanda! Don’t say that this is a little hamlet, a jungle hamlet, a branch hamlet. 

\section*{1. The Capital City of \textsanskrit{Kusāvatī} }

Once\marginnote{1.3.4} upon a time there was a king named \textsanskrit{Mahāsudassana} whose dominion extended to all four sides, and who achieved stability in the country. His capital was this \textsanskrit{Kusinārā}, which at the time was named \textsanskrit{Kusāvatī}. It stretched for twelve leagues from east to west, and seven leagues from north to south. The royal capital of \textsanskrit{Kusāvatī} was successful, prosperous, populous, full of people, with plenty of food. It was just like \textsanskrit{Āḷakamandā}, the royal capital of the gods, which is successful, prosperous, populous, full of spirits, with plenty of food. 

\textsanskrit{Kusāvatī}\marginnote{1.3.10} was never free of ten sounds by day or night, namely: the sound of elephants, horses, chariots, drums, clay drums, arched harps, singing, horns, gongs, and handbells; and the cry, ‘Eat, drink, be merry!’ as the tenth. 

\textsanskrit{Kusāvatī}\marginnote{1.4.1} was encircled by seven ramparts:\footnote{From here we leave \href{https://suttacentral.net/dn16/en/sujato}{DN 16} behind and with it any semblance of realism. Some of these details are shared with \href{https://suttacentral.net/sn22.96/en/sujato}{SN 22.96}, which however does not mention the name \textsanskrit{Mahāsudassana}. } one made of gold, one made of silver, one made of beryl, one made of crystal, one made of ruby, one made of emerald, and one made of all precious things. 

It\marginnote{1.5.1} had four gates, made of gold, silver, beryl, and crystal. At each gate there were seven pillars, three fathoms deep and four fathoms high,\footnote{Readings vary. The Sanskrit text has \textit{\textsanskrit{saptapauruṣā} \textsanskrit{ardhacaturthapauruṣāś} ca \textsanskrit{nikhātā}}. } made of gold, silver, beryl, crystal, ruby, emerald, and all precious things. 

It\marginnote{1.6.1} was surrounded by seven rows of palm trees, made of gold, silver, beryl, crystal, ruby, emerald, and all precious things. The golden palms had trunks of gold, and leaves and fruits of silver. The silver palms had trunks of silver, and leaves and fruits of gold. The beryl palms had trunks of beryl, and leaves and fruits of crystal. The crystal palms had trunks of crystal, and leaves and fruits of beryl. The ruby palms had trunks of ruby, and leaves and fruits of emerald. The emerald palms had trunks of emerald, and leaves and fruits of ruby. The palms of all precious things had trunks of all precious things, and leaves and fruits of all precious things. When those rows of palm trees were blown by the wind they sounded graceful, tantalizing, sensuous, lovely, and intoxicating, like a quintet made up of skilled musicians who had practiced well and kept excellent rhythm. And any addicts, carousers, or drunkards in \textsanskrit{Kusāvatī} at that time were entertained by that sound.\footnote{I think the point is that they listened to the music rather than doing bad things. } 

\section*{2. The Seven Treasures }

\subsection*{2.1. The Wheel-Treasure }

King\marginnote{1.7.1} \textsanskrit{Mahāsudassana} possessed seven treasures and four blessings. What seven? 

On\marginnote{1.7.3} a fifteenth day sabbath, King \textsanskrit{Mahāsudassana} had bathed his head and gone upstairs in the royal longhouse to observe the sabbath.\footnote{This is specific instance, so I use past tense. Where it is an abstract description of a legendary past, I use the present tense to denote an eternal or mythic present. } And the heavenly wheel-treasure appeared to him, with a thousand spokes, with rim and hub, complete in every detail.\footnote{The wheel is firstly the sun and secondly the wheel of the chariots that drove the Indo-Europeans in their conquests. It is the manifestation of unstoppable power. The solar imagery is reflected in the name \textsanskrit{Mahāsudassana} (“Great Splendor”). The whole story reflects the Indo-European dream of universal domination. } Seeing this, the king thought, ‘I have heard that when the heavenly wheel-treasure appears to a king in this way, he becomes a wheel-turning monarch. Am I then a wheel-turning monarch?’ 

Then\marginnote{1.8.1} King \textsanskrit{Mahāsudassana}, rising from his seat and arranging his robe over one shoulder, took a ceremonial vase in his left hand and besprinkled the wheel-treasure with his right hand, saying:\footnote{Many of the details in this myth echo the Brahmanical horse sacrifice. Since the horse was the primary source of Indo-European dominion, its sacrifice served to authorize the power of a king. It was a costly and dangerous rite that was attempted only by the greatest of sovereigns. | \textit{\textsanskrit{Bhiṅkāra}} (“ceremonial vase”) and \textit{abbhukkirati} (“besprinkled”) are elevated terms. } ‘Roll forth, O wheel-treasure! Triumph, O wheel-treasure!’ 

Then\marginnote{1.8.3} the wheel-treasure rolled towards the east. And the king followed it together with his army of four divisions. In whatever place the wheel-treasure stood still, there the king came to stay together with his army.\footnote{In the horse sacrifice, the horse is released for a year, while the king follows it with his army, claiming any land that it wanders on as his. } 

And\marginnote{1.9.1} any opposing rulers of the eastern quarter came to him and said, ‘Come, great king! Welcome, great king! We are yours, great king, instruct us.’ 

The\marginnote{1.9.3} king said, ‘Do not kill living creatures. Do not steal. Do not commit sexual misconduct. Do not lie. Do not drink liquor. Maintain the current level of taxation.’\footnote{Read \textit{\textsanskrit{bhuñjati}} at \href{https://suttacentral.net/mn98/en/sujato\#10.30}{MN 98:10.30} with \textit{\textsanskrit{yathābhuttañca} \textsanskrit{bhuñjatha}} at \href{https://suttacentral.net/dn17/en/sujato\#1.9.4}{DN 17:1.9.4}, \href{https://suttacentral.net/dn26/en/sujato\#6.7}{DN 26:6.7}, and \href{https://suttacentral.net/mn129/en/sujato\#35.7}{MN 129:35.7}. These have sometimes been rendered “eat”, “enjoy”, or “govern”. But compare the archaic English “use” meaning “the benefit or profit of lands”. Thus \textit{\textsanskrit{yathābhuttañca} \textsanskrit{bhuñjatha}} means “use as has been used”, i.e. “maintain the current level of taxation”. } And so the opposing rulers of the eastern quarter became his vassals. 

Then\marginnote{1.10.1} the wheel-treasure, having plunged into the eastern ocean and emerged again, rolled towards the south. …\footnote{The Wheel plunges into the seas, while the sacrificial horse is born in the western and eastern seas. } 

Having\marginnote{1.10.2} plunged into the southern ocean and emerged again, it rolled towards the west. … 

Having\marginnote{1.10.3} plunged into the western ocean and emerged again, it rolled towards the north, followed by the king together with his army of four divisions. In whatever place the wheel-treasure stood still, there the king came to stay together with his army. 

And\marginnote{1.10.5} any opposing rulers of the northern quarter came to him and said, ‘Come, great king! Welcome, great king! We are yours, great king, instruct us.’ 

The\marginnote{1.10.7} king said, ‘Do not kill living creatures. Do not steal. Do not commit sexual misconduct. Do not lie. Do not drink liquor. Maintain the current level of taxation.’ And so the opposing rulers of the northern quarter became his vassals.\footnote{Historically, India has usually been divided into squabbling realms, but from an early age there was a dream of a unified and peaceful continent. } 

And\marginnote{1.11.1} then the wheel-treasure, having triumphed over this land surrounded by ocean, returned to the royal capital. There it stood still by the gate to the royal compound at the High Court as if fixed to an axle, illuminating the royal compound.\footnote{“At the High Court” (\textit{\textsanskrit{atthakaraṇapamukhe}}) is uncertain. \textit{Attha-} has many variants, including \textit{\textsanskrit{aḍḍa}-} and \textit{\textsanskrit{aṭṭa}-}. \textit{\textsanskrit{Atthakaraṇa}} normally refers to a king “sitting in judgment” as at \href{https://suttacentral.net/sn3.7/en/sujato\#1.5}{SN 3.7:1.5} or \href{https://suttacentral.net/mn89/en/sujato\#13.3}{MN 89:13.3}. Here, however it is a place. \textit{-Pamukha} is a standard descriptor of the attributes of a wheel-turning monarch in the sense of “finest” (eg. \href{https://suttacentral.net/dn17/en/sujato\#2.5.1}{DN 17:2.5.1}) rather than “entrance”. Thus I take it as meaning “the supreme place of judgment” i.e. “High Court”. } Such is the wheel-treasure that appeared to King \textsanskrit{Mahāsudassana}. 

\subsection*{2.2. The Elephant-Treasure }

Next,\marginnote{1.12.1} the elephant-treasure appeared to King \textsanskrit{Mahāsudassana}. It was an all-white sky-walker with psychic power, touching the ground in seven places, a king of elephants named Sabbath.\footnote{The white elephant is a symbol of royalty to this day. The description recalls Indra’s elephant \textsanskrit{Airāvata}. | For \textit{\textsanskrit{sattappatiṭṭho}}, the commentary has \textit{\textsanskrit{susaṇṭhitaaṅgapaccaṅga}} (“well-grounded on each and every limb”), a sense confirmed by the \textsanskrit{Mūlasarvāstivāda} \textsanskrit{Bhaiṣajyavastu} which has \textit{\textsanskrit{saptāṅgaḥ} \textsanskrit{supratiṣṭhito}} (“well-established on seven limbs”). The subcommentary lists the four feet, trunk, tail, and penis (\textit{\textsanskrit{varaṅga}}). } Seeing him, the king was impressed, ‘This would truly be a fine elephant vehicle, if he would submit to taming.’ Then the elephant-treasure submitted to taming, as if he was a fine thoroughbred elephant that had been tamed for a long time. 

Once\marginnote{1.12.6} it so happened that King \textsanskrit{Mahāsudassana}, testing that same elephant-treasure, mounted him in the morning and traversed the land surrounded by ocean before returning to the royal capital in time for breakfast. Such is the elephant-treasure that appeared to King \textsanskrit{Mahāsudassana}. 

\subsection*{2.3. The Horse-Treasure }

Next,\marginnote{1.13.1} the horse-treasure appeared to King \textsanskrit{Mahāsudassana}. It was an all-white sky-walker with psychic power, with head of black and mane like woven reeds, a royal steed named Thundercloud.\footnote{The sacrificial horse is likewise white with black head or forequarters. It is identified with the sun, thus being a “sky-walker”. “Thundercloud” (\textit{\textsanskrit{valāhaka}}; Sanskrit \textit{\textsanskrit{balāhaka}}) is the name of one of the four horses of \textsanskrit{Kṛṣṇa}’s chariot in the \textsanskrit{Mahābharata}. The description here also recalls the Vedic sacred horse \textit{\textsanskrit{uccaiḥśravas}}. } Seeing him, the king was impressed, ‘This would truly be a fine horse vehicle, if he would submit to taming.’ Then the horse-treasure submitted to taming, as if he was a fine thoroughbred horse that had been tamed for a long time. 

Once\marginnote{1.13.6} it so happened that King \textsanskrit{Mahāsudassana}, testing that same horse-treasure, mounted him in the morning and traversed the land surrounded by ocean before returning to the royal capital in time for breakfast. Such is the horse-treasure that appeared to King \textsanskrit{Mahāsudassana}. 

\subsection*{2.4. The Jewel-Treasure }

Next,\marginnote{1.14.1} the jewel-treasure appeared to King \textsanskrit{Mahāsudassana}. It was a beryl gem that was naturally beautiful, eight-faceted, well-worked, transparent, clear, and unclouded, endowed with all good qualities. And the radiance of that jewel spread all-round for a league. 

Once\marginnote{1.14.4} it so happened that King \textsanskrit{Mahāsudassana}, testing that same jewel-treasure, mobilized his army of four divisions and, with the jewel hoisted on his banner, set out in the dark of the night. Then the villagers around them set off to work, thinking that it was day. Such is the jewel-treasure that appeared to King \textsanskrit{Mahāsudassana}. 

\subsection*{2.5. The Woman-Treasure }

Next,\marginnote{1.15.1} the woman-treasure appeared to King \textsanskrit{Mahāsudassana}. She was attractive, good-looking, lovely, of surpassing beauty. She was neither too tall nor too short; neither too thin nor too fat; neither too dark nor too light. She outdid human beauty without reaching heavenly beauty. And her touch was like a tuft of cotton-wool or kapok. When it was cool her limbs were warm, and when it was warm her limbs were cool. The fragrance of sandal floated from her body, and lotus from her mouth. She got up before the king and went to bed after him, and was obliging, behaving nicely and speaking politely. The woman-treasure did not betray the wheel-turning monarch even in thought, still less in deed. Such is the woman-treasure that appeared to King \textsanskrit{Mahāsudassana}. 

\subsection*{2.6. The Householder-Treasure }

Next,\marginnote{1.16.1} the householder-treasure appeared to King \textsanskrit{Mahāsudassana}. The power of clairvoyance manifested in him as a result of past deeds, by which he sees hidden treasure, both owned and ownerless. 

He\marginnote{1.16.3} approached the king and said, ‘Relax, sire. I will take care of the treasury.’ 

Once\marginnote{1.16.5} it so happened that the wheel-turning monarch, testing that same householder-treasure, boarded a boat and sailed to the middle of the Ganges river. Then he said to the householder-treasure, ‘Householder, I need gold, both coined and uncoined.’ 

‘Well\marginnote{1.16.7} then, great king, draw the boat up to one shore.’ 

‘It’s\marginnote{1.16.8} right here, householder, that I need gold, both coined and uncoined.’ 

Then\marginnote{1.16.9} that householder-treasure, immersing both hands in the water, pulled up a pot full of gold, both coined and uncoined, and said to the king, ‘Is this sufficient, great king? Has enough been done, great king, enough offered?’ 

The\marginnote{1.16.11} king said, ‘That is sufficient, householder. Enough has been done, enough offered.’ 

Such\marginnote{1.16.13} is the householder-treasure that appeared to King \textsanskrit{Mahāsudassana}. 

\subsection*{2.7. The Commander-Treasure }

Next,\marginnote{1.18.1} the commander-treasure appeared to King \textsanskrit{Mahāsudassana}. He was astute, competent, intelligent, and capable of getting the king to appoint who should be appointed, dismiss who should be dismissed, and retain who should be retained.\footnote{The “commander” (\textit{\textsanskrit{pariṇāyaka}}) is mostly found in Buddhist texts of this context, and is not a regular term of governance. This passage shows that he was responsible for management of the realm, while below he appears as commander of the army (\href{https://suttacentral.net/dn17/en/sujato\#2.8.2}{DN 17:2.8.2}). Elsewhere he is said to excel in strategy (\href{https://suttacentral.net/an5.134/en/sujato\#2.5}{AN 5.134:2.5}). } 

He\marginnote{1.18.3} approached the king and said, ‘Relax, sire. I shall issue instructions.’ 

Such\marginnote{1.18.5} is the commander-treasure that appeared to King \textsanskrit{Mahāsudassana}. 

These\marginnote{1.18.6} are the seven treasures possessed by King \textsanskrit{Mahāsudassana}. 

\section*{3. The Four Blessings }

King\marginnote{1.18.8} \textsanskrit{Mahāsudassana} possessed four blessings. And what are the four blessings? 

He\marginnote{1.18.10} was attractive, good-looking, lovely, of surpassing beauty, more so than other people. This is the first blessing. 

Furthermore,\marginnote{1.19.1} he was long-lived, more so than other people. This is the second blessing. 

Furthermore,\marginnote{1.20.1} he was rarely ill or unwell, and his stomach digested well, being neither too hot nor too cold, more so than other people. This is the third blessing. 

Furthermore,\marginnote{1.21.1} he was as dear and beloved to the brahmins and householders as a father is to his children. And the brahmins and householders were as dear to the king as children are to their father. 

Once\marginnote{1.21.7} it so happened that King \textsanskrit{Mahāsudassana} went with his army of four divisions to visit a park. Then the brahmins and householders went up to him and said, ‘Slow down, Your Majesty, so we may see you longer!’ And the king addressed his charioteer, ‘Drive slowly, charioteer, so I can see the brahmins and householders longer!’ This is the fourth blessing. 

These\marginnote{1.21.13} are the four blessings possessed by King \textsanskrit{Mahāsudassana}. 

\section*{4. Lotus Ponds in the Palace of Principle }

Then\marginnote{1.22.1} King \textsanskrit{Mahāsudassana} thought, ‘Why don’t I have lotus ponds built between the palms, at intervals of a hundred bow lengths?’ 

So\marginnote{1.22.3} that’s what he did. The lotus ponds were lined with tiles of four colors, made of gold, silver, beryl, and crystal. 

And\marginnote{1.22.6} four flights of stairs of four colors descended into each lotus pond, made of gold, silver, beryl, and crystal. The golden stairs had posts of gold, and banisters and finials of silver. The silver stairs had posts of silver, and banisters and finials of gold. The beryl stairs had posts of beryl, and banisters and finials of crystal. The crystal stairs had posts of crystal, and banisters and finials of beryl. Those lotus ponds were surrounded by two balustrades, made of gold and silver. The golden balustrades had posts of gold, and banisters and finials of silver. The silver balustrades had posts of silver, and banisters and finials of gold. 

Then\marginnote{1.23.1} King \textsanskrit{Mahāsudassana} thought, ‘Why don’t I plant flowers in the lotus ponds such as blue water lilies, and lotuses of pink, yellow, and white, blooming all year round, and accessible to the public?’ So that’s what he did. 

Then\marginnote{1.23.4} King \textsanskrit{Mahāsudassana} thought, ‘Why don’t I appoint bath attendants to help bathe the people who come to bathe in the lotus ponds?’ So that’s what he did. 

Then\marginnote{1.23.7} King \textsanskrit{Mahāsudassana} thought, ‘Why don’t I set up charities on the banks of the lotus ponds, so that those in need of food, drink, clothes, vehicles, beds, women, coined gold, or uncoined gold can get what they need?’\footnote{The jarring inclusion of “women” (\textit{\textsanskrit{itthī}}) in this list probably refers to state-sponsored prostitution. } So that’s what he did. 

Then\marginnote{1.24.1} the brahmins and householders came to the king bringing abundant wealth and said,\footnote{As at \href{https://suttacentral.net/dn5/en/sujato\#19.1}{DN 5:19.1}. } ‘Sire, this abundant wealth is specially for you alone; may Your Highness accept it!’ 

‘There’s\marginnote{1.24.3} enough raised for me through regular taxes. Let this be for you; and here, take even more!’ 

When\marginnote{1.24.4} the king turned them down, they withdrew to one side to think up a plan, ‘It wouldn’t be proper for us to take this abundant wealth back to our own homes. Why don’t we build a home for King \textsanskrit{Mahāsudassana}?’ 

They\marginnote{1.24.7} went up to the king and said, ‘We shall have a home built for you, sire!’ King \textsanskrit{Mahāsudassana} consented with silence. 

And\marginnote{1.25.1} then Sakka, lord of gods, knowing the king’s train of thought, addressed the god Vissakamma, ‘Come, dear Vissakamma, build a palace named Principle as a home for King \textsanskrit{Mahāsudassana}.’\footnote{The palace is named “Principle” (\textit{dhamma}) in recognition of the king being subject to a higher law. Normally I translate \textit{\textsanskrit{pāsāda}} as “stilt longhouse” but here something grander is meant. } 

‘Yes,\marginnote{1.25.3} lord,’ replied Vissakamma. Then, as easily as a strong person would extend or contract their arm, he vanished from the gods of the thirty-three and appeared in front of King \textsanskrit{Mahāsudassana}. 

Vissakamma\marginnote{1.25.4} said to the king, ‘I shall build a palace named Principle as a home for you, sire.’ King \textsanskrit{Mahāsudassana} consented with silence. 

And\marginnote{1.25.7} so that’s what Vissakamma did. 

The\marginnote{1.25.8} Palace of Principle stretched for a league from east to west, and half a league from north to south. It was lined with tiles of four colors, three fathoms high, made of gold, silver, beryl, and crystal. 

It\marginnote{1.26.1} had 84,000 pillars of four colors, made of gold, silver, beryl, and crystal. It was covered with panels of four colors, made of gold, silver, beryl, and crystal. 

It\marginnote{1.26.5} had twenty-four staircases of four colors, made of gold, silver, beryl, and crystal. The golden stairs had posts of gold, and banisters and finials of silver. The silver stairs had posts of silver, and banisters and finials of gold. The beryl stairs had posts of beryl, and banisters and finials of crystal. The crystal stairs had posts of crystal, and banisters and finials of beryl. 

It\marginnote{1.26.11} had 84,000 chambers of four colors,\footnote{For \textit{\textsanskrit{kūṭāgāra}} as “chamber” see \href{https://suttacentral.net/mn37/en/sujato\#8.10}{MN 37:8.10}. } made of gold, silver, beryl, and crystal. In each chamber a couch was spread: in the golden chamber a couch of silver; in the silver chamber a couch of gold; in the beryl chamber a couch of ivory; in the crystal chamber a couch of hardwood. At the door of the golden chamber stood a palm tree of silver, with trunk of silver, and leaves and fruits of gold. At the door of the silver chamber stood a palm tree of gold, with trunk of gold, and leaves and fruits of silver. At the door of the beryl chamber stood a palm tree of crystal, with trunk of crystal, and leaves and fruits of beryl. At the door of the crystal chamber stood a palm tree of beryl, with trunk of beryl, and leaves and fruits of crystal. 

Then\marginnote{1.27.1} King \textsanskrit{Mahāsudassana} thought, ‘Why don’t I build a grove of golden palm trees at the door to the great foyer, where I can sit for the day?’ So that’s what he did. 

The\marginnote{1.28.1} Palace of Principle was surrounded by two balustrades, made of gold and silver. The golden balustrades had posts of gold, and banisters and finials of silver. The silver balustrades had posts of silver, and banisters and finials of gold. 

The\marginnote{1.29.1} Palace of Principle was surrounded by two nets of bells, made of gold and silver. The golden net had bells of silver, and the silver net had bells of gold. When those nets of bells were blown by the wind they sounded graceful, tantalizing, sensuous, lovely, and intoxicating, like a quintet made up of skilled musicians who had practiced well and kept excellent rhythm. And any addicts, carousers, or drunkards in \textsanskrit{Kusāvatī} at that time were entertained by that sound. When it was finished, the palace was hard to look at, dazzling to the eyes. It was like how in the last month of the rainy season, in autumn, when the heavens are clear and cloudless, as the sun is rising to the firmament, it is hard to look at, dazzling to the eyes. 

Then\marginnote{1.30.1} King \textsanskrit{Mahāsudassana} thought, ‘Why don’t I build a lotus pond named Principle in front of the palace?’ So that’s what he did. The Lotus Pond of Principle stretched for a league from east to west, and half a league from north to south. It was lined with tiles of four colors, made of gold, silver, beryl, and crystal. 

It\marginnote{1.31.1} had twenty-four staircases of four colors, made of gold, silver, beryl, and crystal. The golden stairs had posts of gold, and banisters and finials of silver. The silver stairs had posts of silver, and banisters and finials of gold. The beryl stairs had posts of beryl, and banisters and finials of crystal. The crystal stairs had posts of crystal, and banisters and finials of beryl. 

It\marginnote{1.31.7} was surrounded by two balustrades, made of gold and silver. The golden balustrades had posts of gold, and banisters and finials of silver. The silver balustrades had posts of silver, and banisters and finials of gold. 

It\marginnote{1.32.1} was surrounded by seven rows of palm trees, made of gold, silver, beryl, crystal, ruby, emerald, and all precious things. The golden palms had trunks of gold, and leaves and fruits of silver. The silver palms had trunks of silver, and leaves and fruits of gold. The beryl palms had trunks of beryl, and leaves and fruits of crystal. The crystal palms had trunks of crystal, and leaves and fruits of beryl. The ruby palms had trunks of ruby, and leaves and fruits of emerald. The emerald palms had trunks of emerald, and leaves and fruits of ruby. The palms of all precious things had trunks of all precious things, and leaves and fruits of all precious things. When those rows of palm trees were blown by the wind they sounded graceful, tantalizing, sensuous, lovely, and intoxicating, like a quintet made up of skilled musicians who had practiced well and kept excellent rhythm. And any addicts, carousers, or drunkards in \textsanskrit{Kusāvatī} at that time were entertained by that sound. 

When\marginnote{1.33.1} the palace and its lotus pond were finished, King \textsanskrit{Mahāsudassana} served those who were deemed true ascetics and brahmins with all they desired. Then he ascended the Palace of Principle. 

\scendsection{The first recitation section. }

\section*{5. Attaining Absorption }

Then\marginnote{2.1.1} King \textsanskrit{Mahāsudassana} thought, ‘Of what deed of mine is this the fruit and result, that I am now so mighty and powerful?’ 

Then\marginnote{2.1.3} King \textsanskrit{Mahāsudassana} thought, ‘It is the fruit and result of three kinds of deeds:\footnote{Thus denying the doctrine of \textsanskrit{Pūraṇa} Kassapa at \href{https://suttacentral.net/dn2/en/sujato\#17.5}{DN 2:17.5}. This is the recognition, at least partially, of right view. } giving, self-control, and restraint.’ 

Then\marginnote{2.2.1} he went to the great foyer, stood at the door, and expressed this heartfelt sentiment:\footnote{The \textit{\textsanskrit{mahāviyūha}} must have been some kind of structure at the entrance to the palace, a “foyer”. } ‘Stop here, sensual, malicious, and cruel thoughts—\footnote{These are the three factors of right thought. The king skillfully uses a transition in physical space to set up his intention to meditate. } no further!’ 

Then\marginnote{2.3.1} he entered the great foyer and sat on the golden couch. Quite secluded from sensual pleasures, secluded from unskillful qualities, he entered and remained in the first absorption, which has the rapture and bliss born of seclusion, while placing the mind and keeping it connected.\footnote{Here as in \href{https://suttacentral.net/dn1/en/sujato}{DN 1}, \textit{\textsanskrit{jhāna}} is not a uniquely Buddhist practice. } As the placing of the mind and keeping it connected were stilled, he entered and remained in the second absorption, which has the rapture and bliss born of immersion, with internal clarity and mind at one, without placing the mind and keeping it connected. And with the fading away of rapture, he entered and remained in the third absorption, where he meditated with equanimity, mindful and aware, personally experiencing the bliss of which the noble ones declare, ‘Equanimous and mindful, one meditates in bliss.’ With the giving up of pleasure and pain, and the ending of former happiness and sadness, he entered and remained in the fourth absorption, without pleasure or pain, with pure equanimity and mindfulness. 

Then\marginnote{2.4.1} King \textsanskrit{Mahāsudassana} left the great foyer and entered the golden chamber, where he sat on the golden couch. He meditated spreading a heart full of love to one direction, and to the second, and to the third, and to the fourth. In the same way he spread a heart full of love above, below, across, everywhere, all around, to everyone in the world—abundant, expansive, limitless, free of enmity and ill will. He meditated spreading a heart full of compassion … He meditated spreading a heart full of rejoicing … He meditated spreading a heart full of equanimity to one direction, and to the second, and to the third, and to the fourth. In the same way above, below, across, everywhere, all around, he spread a heart full of equanimity to the whole world—abundant, expansive, limitless, free of enmity and ill will. 

\section*{6. Of All Cities }

King\marginnote{2.5.1} \textsanskrit{Mahāsudassana} had 84,000 cities, with the royal capital of \textsanskrit{Kusāvatī} foremost. He had 84,000 palaces, with the Palace of Principle foremost. He had 84,000 chambers, with the great foyer foremost. He had 84,000 couches made of gold, silver, ivory, and hardwood. They were spread with woollen covers—shag-piled, pure white, or embroidered with flowers—and spread with a fine deer hide, with a canopy above and red pillows at both ends.\footnote{This segment breaks the expected pattern of “foremost” things. } He had 84,000 bull elephants with gold adornments and banners, covered with snow gold netting, with the royal bull elephant named Sabbath foremost. He had 84,000 horses with gold adornments and banners, covered with snow gold netting, with the royal steed named Thundercloud foremost. He had 84,000 chariots upholstered with the hide of lions, tigers, and leopards, and cream rugs, with gold adornments and banners, covered with snow gold netting, with the chariot named Triumph foremost.\footnote{Not the British car of the same name. The chariot is the source of victory in battle. } He had 84,000 jewels, with the jewel-treasure foremost. He had 84,000 women, with Queen \textsanskrit{Subhaddā} foremost. He had 84,000 householders, with the householder-treasure foremost. He had 84,000 aristocrat vassals, with the commander-treasure foremost. He had 84,000 milk-cows with silken reins and bronze pails.\footnote{Read \textit{\textsanskrit{dukūlasandanāni}}, where \textit{\textsanskrit{dukūla}} is “fine cloth, silk” and \textit{sandana} is “cord, tether”. } He had 8,400,000,000 fine cloths of linen, cotton, silk, and wool. He had 84,000 servings of food, which were presented to him as offerings in the morning and evening. 

Now\marginnote{2.6.1} at that time his 84,000 royal elephants came to attend on him in the morning and evening. Then King \textsanskrit{Mahāsudassana} thought, ‘What if instead half of the elephants took turns to attend on me at the end of each century?’\footnote{An odd detail. Maybe the sound of the elephants disturbed his meditation? } He instructed the commander-treasure to do this, and so it was done. 

\section*{7. The Visit of Queen \textsanskrit{Subhaddā} }

Then,\marginnote{2.7.1} after many years, many hundred years, many thousand years had passed, Queen \textsanskrit{Subhaddā} thought, ‘It is long since I have seen the king. Why don’t I go to see him?’ 

So\marginnote{2.7.3} the queen addressed the ladies of the harem, ‘Come, bathe your heads and dress in yellow. It is long since we saw the king, and we shall go to see him.’ 

‘Yes,\marginnote{2.7.6} ma’am,’ replied the ladies of the harem. They did as she asked and returned to the queen. 

Then\marginnote{2.8.1} the queen addressed the commander-treasure, ‘Dear commander-treasure, please ready the army of four divisions. It is long since we saw the king, and we shall go to see him.’ 

‘Yes,\marginnote{2.8.3} my queen,’ he replied, and did as he was asked. He informed the queen, ‘My queen, the army of four divisions is ready, please go at your convenience.’ 

Then\marginnote{2.8.6} Queen \textsanskrit{Subhaddā} together with the ladies of the harem went with the army to the Palace of Principle. She ascended the palace and went to the great foyer, where she stood leaning against a door-post.\footnote{Like Ānanda at \href{https://suttacentral.net/dn16/en/sujato\#5.13.1}{DN 16:5.13.1}. } 

Hearing\marginnote{2.8.8} them, the king thought, ‘What’s that, it sounds like a big crowd!’ Coming out of the foyer he saw Queen \textsanskrit{Subhaddā} leaning against a door-post and said to her, ‘Please stay there, my queen, don’t enter in here.’\footnote{He makes it clear that this will not be an intimate visit. } 

Then\marginnote{2.9.1} he addressed a certain man, ‘Come, mister, bring the golden couch from the great foyer and set it up in the golden palm grove.’ 

‘Yes,\marginnote{2.9.3} Your Majesty,’ that man replied, and did as he was asked. The king laid down in the lion’s posture—on the right side, placing one foot on top of the other—mindful and aware. 

Then\marginnote{2.10.1} Queen \textsanskrit{Subhaddā} thought, ‘The king’s faculties are so very clear, and the complexion of his skin is pure and bright. Let him not pass away!’ She said to him,\footnote{The aorist is not past tense, as it is governed by \textit{\textsanskrit{mā}}. } ‘Sire, you have 84,000 cities, with the royal capital of \textsanskrit{Kusāvatī} foremost. Arouse desire for these! Take an interest in life!’\footnote{“Desire” is \textit{chanda}, the first of the four bases of psychic power, which in \href{https://suttacentral.net/dn16/en/sujato\#3.3.1}{DN 16:3.3.1} are said to lead to long life. By urging him to live long, she inverts the \textsanskrit{Mahāparinibbānasutta} where Ānanda fails to do the same. } 

And\marginnote{2.10.5} she likewise urged the king to live on by taking an interest in all his possessions as described above. 

When\marginnote{2.11.1} the queen had spoken, the king said to her, ‘For a long time, my queen, you have spoken to me with words that are welcome, desirable, agreeable, and pleasant. And yet in my final hour, your words are unwelcome, undesirable, disagreeable, and unpleasant!’\footnote{Implying that Ānanda was right to not beg the Buddha to live long. } 

‘Then\marginnote{2.11.4} how exactly, Your Majesty, am I to speak to you?’ 

‘Like\marginnote{2.11.5} this, my queen: “Sire, we must be parted and separated from all we hold dear and beloved. Don’t pass away with concerns. Such concern is suffering, and it’s criticized.\footnote{At \href{https://suttacentral.net/an6.16/en/sujato\#2.2}{AN 6.16:2.2} this advice is given by Nakula’s mother. } Sire, you have 84,000 cities, with the royal capital of \textsanskrit{Kusāvatī} foremost. Give up desire for these! Take no interest in life!”’ And so on for all the king’s possessions. 

When\marginnote{2.12.1} the king had spoken, Queen \textsanskrit{Subhaddā} cried and burst out in tears.\footnote{Like Ānanda at \href{https://suttacentral.net/dn16/en/sujato\#5.13.1}{DN 16:5.13.1}. } Wiping away her tears, the queen said to the king: ‘Sire, we must be parted and separated from all we hold dear and beloved. Don’t pass away with concerns. Such concern is suffering, and it’s criticized. Sire, you have 84,000 cities, with the royal capital of \textsanskrit{Kusāvatī} foremost. Give up desire for these! Take no interest in life!’ And she continued, listing all the king’s possessions. 

\section*{8. Rebirth in the Realm of Divinity}

Not\marginnote{2.13.1} long after that, King \textsanskrit{Mahāsudassana} passed away. And the feeling he had close to death was like a householder or their child falling asleep after eating a delectable meal.\footnote{This contrasts with the Buddha’s last meal at \href{https://suttacentral.net/dn16/en/sujato\#4.20.1}{DN 16:4.20.1}, which caused sickness and distress. The Buddha was rejecting existence entirely, whereas \textsanskrit{Mahāsudassana} was continuing in a pleasant form of conditioned existence. } 

When\marginnote{2.13.3} he passed away King \textsanskrit{Mahāsudassana} was reborn in a good place, a realm of divinity. Ānanda, King \textsanskrit{Mahāsudassana} played children’s games for 84,000 years. He ruled as viceroy for 84,000 years. He ruled as king for 84,000 years. He led the spiritual life as a layman in the Palace of Principle for 84,000 years. And having developed the four divine meditations, when his body broke up, after death, he was reborn in a good place, a realm of divinity. 

Now,\marginnote{2.14.1} Ānanda, you might think: ‘Surely King \textsanskrit{Mahāsudassana} must have been someone else at that time?’ But you should not see it like that. I myself was King \textsanskrit{Mahāsudassana} at that time.\footnote{After \href{https://suttacentral.net/dn5/en/sujato\#21.16}{DN 5:21.16}, this is the second \textsanskrit{Jātaka} in the \textsanskrit{Dīghanikāya}. } 

Mine\marginnote{2.14.4} were the 84,000 cities, with the royal capital of \textsanskrit{Kusāvatī} foremost. And mine were all the other possessions. 

Of\marginnote{2.15.1} those 84,000 cities, I only stayed in one, the capital \textsanskrit{Kusāvatī}. Of those 84,000 mansions, I only dwelt in one, the Palace of Principle. Of those 84,000 chambers, I only dwelt in the great foyer. Of those 84,000 couches, I only used one, made of gold or silver or ivory or heartwood. Of those 84,000 bull elephants, I only rode one, the royal bull elephant named Sabbath. Of those 84,000 horses, I only rode one, the royal horse named Thundercloud. Of those 84,000 chariots, I only rode one, the chariot named Triumph. Of those 84,000 women, I was only served by one, a maiden of the aristocratic or peasant classes.\footnote{Accepting the \textsanskrit{Mahāsaṅgīti}’s reading of \textit{\textsanskrit{vessinī}}. I believe the variant \textit{\textsanskrit{velāmikā}} is a ghost word contaminated from \href{https://suttacentral.net/an9.20/en/sujato\#4.1}{AN 9.20:4.1}. } Of those 8,400,000,000 cloths, I only wore one pair, made of fine linen, cotton, silk, or wool. Of those 84,000 servings of food, I only had one, eating at most a cup of rice with suitable sauce. 

See,\marginnote{2.16.1} Ānanda! All those conditioned phenomena have passed, ceased, and perished. So impermanent are conditions, so unstable are conditions, so unreliable are conditions. This is quite enough for you to become disillusioned, dispassionate, and freed regarding all conditions. 

Six\marginnote{2.17.1} times, Ānanda, I recall having laid down my body at this place. And the seventh time was as a wheel-turning monarch, a just and principled king, at which time my dominion extended to all four sides, I achieved stability in the country, and I possessed the seven treasures. But Ānanda, I do not see any place in this world with its gods, \textsanskrit{Māras}, and Divinities, this population with its ascetics and brahmins, its gods and humans where the Realized One would lay down his body for the eighth time.” 

That\marginnote{2.17.3} is what the Buddha said. Then the Holy One, the Teacher, went on to say: 

\begin{verse}%
“Oh!\marginnote{2.17.5} Conditions are impermanent, \\
their nature is to rise and fall; \\
having arisen, they cease; \\
their stilling is blissful.” 

%
\end{verse}

%
\chapter*{{\suttatitleacronym DN 18}{\suttatitletranslation With Janavasabha }{\suttatitleroot Janavasabhasutta}}
\addcontentsline{toc}{chapter}{\tocacronym{DN 18} \toctranslation{With Janavasabha } \tocroot{Janavasabhasutta}}
\markboth{With Janavasabha }{Janavasabhasutta}
\extramarks{DN 18}{DN 18}

\section*{1. Declaring the Rebirths of People From \textsanskrit{Ñātika} and Elsewhere }

\scevam{So\marginnote{1.1} I have heard. }At one time the Buddha was staying at \textsanskrit{Ñātika} in the brick house.\footnote{This picks up from the events of \href{https://suttacentral.net/dn16/en/sujato\#2.5.1}{DN 16:2.5.1}, which however discusses only the \textsanskrit{Ñātikans} (who are reckoned among the Vajjis). Like \href{https://suttacentral.net/dn17/en/sujato}{DN 17}, it bears many signs of a late sutta. The overall theme is the presence of a cosmic order where the liberating teaching of the Buddha is reconciled with the requirements of worldly power. } 

Now\marginnote{1.3} at that time the Buddha was explaining the rebirths of devotees all over the nations; the \textsanskrit{Kāsis} and Kosalans, Vajjis and Mallas, \textsanskrit{Cetīs} and Vacchas, Kurus and \textsanskrit{Pañcālas}, Macchas and \textsanskrit{Sūrasenas}:\footnote{This list of ten nations is unique. The omission of the \textsanskrit{Aṅgas} and Magadhans is noted below. The more famous list of “sixteen nations” adds Avanti and Assaka in the south, and \textsanskrit{Gandhāra} and Kamboja in the north-west (eg. \href{https://suttacentral.net/an3.70/en/sujato\#28.3}{AN 3.70:28.3}). The Buddha did not visit these lands, which explains why they are omitted. } 

“This\marginnote{1.4} one was reborn here, while that one was reborn there. 

Over\marginnote{1.5} fifty devotees in \textsanskrit{Ñātika} have passed away having ended the five lower fetters. They’ve been reborn spontaneously, and will be extinguished there, not liable to return from that world.\footnote{“Devotees” is \textit{\textsanskrit{paricāraka}}, which normally means “servant”. It is not in the relevant passages of DN 16. This sense is found  in only a couple of other places, both late (\href{https://suttacentral.net/snp5.18/en/sujato\#1.2}{Snp 5.18:1.2}, \href{https://suttacentral.net/pli-tv-kd1/en/sujato\#22.14.8}{Kd 1:22.14.8}). } 

More\marginnote{1.6} than ninety devotees in \textsanskrit{Ñātika} have passed away having ended three fetters, and weakened greed, hate, and delusion. They’re once-returners, who will come back to this world once only, then make an end of suffering. 

More\marginnote{1.7} than five hundred devotees in \textsanskrit{Ñātika} have passed away having ended three fetters. They’re stream-enterers, not liable to be reborn in the underworld, bound for awakening.” 

When\marginnote{2.1} the devotees of \textsanskrit{Ñātika} heard about the Buddha’s answers to those questions, they became uplifted and overjoyed, full of rapture and happiness. 

Venerable\marginnote{3.1} Ānanda heard of the Buddha’s statements and the \textsanskrit{Ñātikans}’ happiness. 

\section*{2. Ānanda’s Suggestion }

Then\marginnote{4.1} Venerable Ānanda thought, “But there were also Magadhan devotees—many, and of long standing too—who have passed away.\footnote{Bearing in mind that none of the nations were actually mentioned in the relevant \textsanskrit{Mahāparinibbānasutta} passage, this whole introduction is framed to emphasize the importance of Magadha to Buddhism, especially given the problematic character of \textsanskrit{Ajātasattu}. In the decades following the Buddha’s death, the Buddhist community adapted to a new political landscape which for a time saw the entirety of Buddhism contained within the sprawling Magadhan empire. } You’d think that \textsanskrit{Aṅga} and Magadha were empty of devotees who have passed away! But they too had confidence in the Buddha, the teaching, and the \textsanskrit{Saṅgha}, and had fulfilled their ethics. The Buddha hasn’t declared their passing. It would be good to do so, for many people would gain confidence, and so be reborn in a good place. 

That\marginnote{4.7} King Seniya \textsanskrit{Bimbisāra} of Magadha was a just and principled king who benefited the brahmins and householders, and people of town and country.\footnote{As we known from \href{https://suttacentral.net/dn2/en/sujato}{DN 2}, \textsanskrit{Bimbisāra} had recently been murdered by \textsanskrit{Ajātasattu}. } People still sing his praises: ‘That just and principled king, who made us so happy, has passed away. Life was good under his dominion.’\footnote{A not-so-subtle hint of the changes under \textsanskrit{Ajātasattu}. The passing of \textsanskrit{Bimbisāra} signifies the crumbling of the social order that prevailed in the Buddha’s lifetime, bringing with it the immediate threat of war and chaos. This sutta establishes a timeless order that persists while worldly conditions fluctuate. } He too had confidence in the Buddha, the teaching, and the \textsanskrit{Saṅgha}, and had fulfilled his ethics. People say: ‘Until his dying day, King \textsanskrit{Bimbisāra} sang the Buddha’s praises!’\footnote{Elsewhere the suttas say that \textsanskrit{Bimbisāra} went for refuge (\href{https://suttacentral.net/dn4/en/sujato\#6.33}{DN 4:6.33}), while the Vinaya says he was in fact a stream-enterer (\href{https://suttacentral.net/pli-tv-kd1/en/sujato\#22.9.1}{Kd 1:22.9.1}), which is confirmed below. The Jains, however, claim him as one of their own but say he went to hell for committing suicide. Like his son \textsanskrit{Ajātasattu}, it is likely that he frequented several teachers in his realm. } The Buddha hasn’t declared his passing. It would be good to do so, for many people would gain confidence, and so be reborn in a good place. 

Besides,\marginnote{4.15} the Buddha was awakened in Magadha;\footnote{Near the town of  \textsanskrit{Uruvelā} on the bank of the \textsanskrit{Nerañjarā} River at the place known today as Bodhgaya. } so why hasn’t he declared the rebirth of the Magadhan devotees? If he fails to do so, they will be dejected.” 

After\marginnote{5.1} pondering the fate of the Magadhan devotees alone in private, Ānanda rose at the crack of dawn and went to see the Buddha. He bowed, sat down to one side, and told the Buddha of his concerns, finishing by saying, “Why hasn’t the Buddha declared the rebirth of the Magadhan devotees? If he fails to do so, they will be dejected.” Then Ānanda, after making this suggestion regarding the Magadhan devotees, got up from his seat, bowed, and respectfully circled the Buddha, keeping him on his right, before leaving.\footnote{“Suggestion” is \textit{\textsanskrit{parikathaṁ}}, which elsewhere occurs in the Vinaya (\href{https://suttacentral.net/pli-tv-kd7/en/sujato\#1.5.16}{Kd 7:1.5.16}) and the Abhidhamma \textsanskrit{Vibhaṅga} (\href{https://suttacentral.net/vb17/en/sujato\#43.2}{Vb 17:43.2}) in the sense of “hint”. } 

Soon\marginnote{7.1} after Ānanda had left, the Buddha robed up in the morning and, taking his bowl and robe, entered \textsanskrit{Ñātika} for alms. He wandered for alms in \textsanskrit{Ñātika}. After the meal, on his return from almsround, he washed his feet and entered the brick house. He paid attention, applied the mind, and concentrated wholeheartedly on the fate of Magadhan devotees, and sat on the seat spread out, thinking,\footnote{Normally \textit{\textsanskrit{aṭṭhiṁ} \textsanskrit{katvā} \textsanskrit{manasikatvā} \textsanskrit{sabbaṁ} \textsanskrit{cetasā} \textsanskrit{samannāharitvā}} describes listening to Dhamma. This whole process seems unusually laborious. } “I shall know their destiny, where they are reborn in the next life.” And he saw where they had been reborn. 

Then\marginnote{7.6} in the late afternoon, the Buddha came out of retreat. Emerging from the brick house, he sat on the seat spread out in the shade of the porch. 

Then\marginnote{8.1} Venerable Ānanda went up to the Buddha, bowed, sat down to one side, and said to him, “Sir, you look so serene; your face seems to shine owing to the clarity of your faculties. Have you been abiding in a peaceful meditation today, sir?” 

The\marginnote{9.1} Buddha then recounted what had happened since speaking to Ānanda, revealing that he had seen the destiny of the Magadhan devotees. He continued: 

\section*{3. Janavasabha the Spirit }

“Then,\marginnote{9.5} Ānanda, a vanished spirit called out: ‘I am Janavasabha, Blessed One! I am Janavasabha, Holy One!’ Ānanda, do you recall having previously heard such a name as Janavasabha?”\footnote{\textit{\textsanskrit{Nāmadheyyaṁ}} means “name (borne by someone or something)”, not “one who bears the name” (\href{https://suttacentral.net/mn50/en/sujato\#22.1}{MN 50:22.1}). } 

“No,\marginnote{9.9} sir. But when I heard the word, I got goosebumps! I thought, ‘This must be no ordinary spirit to bear such an exalted name as Janavasabha.’”\footnote{\textit{Janavasabha} is “chief of men”, spelled \textit{janesabha} at \href{https://suttacentral.net/dn20/en/sujato\#10.10}{DN 20:10.10} and \href{https://suttacentral.net/dn32/en/sujato\#10.9}{DN 32:10.9}. The synonym \textit{\textsanskrit{narāsabha}} is an occasional poetic epithet of the Buddha (\href{https://suttacentral.net/snp3.11/en/sujato\#6.2}{Snp 3.11:6.2}, \href{https://suttacentral.net/snp5.1/en/sujato\#21.3}{Snp 5.1:21.3}, \href{https://suttacentral.net/sn11.3/en/sujato\#14.2}{SN 11.3:14.2}). In Sanskrit we find \textit{\textsanskrit{puruṣaṛṣabha}} in the same sense. } 

“After\marginnote{10.1} making himself heard while vanished, Ānanda, a very beautiful spirit appeared in front of me.\footnote{\textit{\textsanskrit{Uḷāravaṇṇa}} describes beautiful people at \href{https://suttacentral.net/mn96/en/sujato\#7.8}{MN 96:7.8}. } And for a second time he called out: ‘I am \textsanskrit{Bimbisāra}, Blessed One! I am \textsanskrit{Bimbisāra}, Holy One! This is the seventh time I am reborn in the company of the Great King \textsanskrit{Vessavaṇa}. When I pass away from here, I can become a king of men.\footnote{Read \textit{ito}. I think this implies what is stated more explicitly in Anuruddha’s partly parallel verses at \href{https://suttacentral.net/thag16.9/en/sujato\#23.1}{Thag 16.9:23.1}: after each of seven rebirths under \textsanskrit{Vessavaṇa} he can become a king of men. This explains the name Janavasabha. } 

\begin{verse}%
Seven\marginnote{10.6} from here, seven from there—\\
fourteen transmigrations in all.\footnote{It is rare to see \textit{\textsanskrit{saṁsāra}} used as a countable noun, but see \href{https://suttacentral.net/thag2.48/en/sujato\#2.2}{Thag 2.48:2.2}. } \\
I remember these lives \\
where I lived before. 

%
\end{verse}

For\marginnote{10.10} a long time I’ve known that I won’t be reborn in the underworld, but that I still hope to become a once-returner.’\footnote{As a stream-enterer he is freed from any lower rebirths, yet he still aspires to a higher realization. } 

‘It’s\marginnote{10.11} incredible and amazing that you, the venerable spirit Janavasabha, should say: 

“For\marginnote{10.12} a long time I’ve been aware that I won’t be reborn in the underworld” and also “But I still hope to become a once-returner.” But from what source do you know that you’ve achieved such a high distinction?’\footnote{In this idiom, the verb for “aware” varies between \textit{\textsanskrit{sañjānāti}}, \textit{\textsanskrit{jānāti}}, \textit{\textsanskrit{pajānāti}}, and \textit{\textsanskrit{sampajānāti}}. } 

‘None\marginnote{11.1} other than the Blessed One’s instruction! None other than the Holy One’s instruction!\footnote{Compare \href{https://suttacentral.net/sn1.50/en/sujato\#5.1}{SN 1.50:5.1} and \href{https://suttacentral.net/sn2.24/en/sujato\#6.1}{SN 2.24:6.1}. } From the day I had absolute devotion to the Buddha I have known that I won’t be reborn in the underworld, but that I still hope to become a once-returner. Just now, sir, I had been sent out by the great king \textsanskrit{Vessavaṇa} to the great king \textsanskrit{Virūḷhaka}’s presence on some business, and on the way I saw the Buddha giving his attention to the fate of the Magadhan devotees. But it comes as no surprise that I have heard and learned the fate of the Magadhan devotees in the presence of the great king \textsanskrit{Vessavaṇa} as he was speaking to his assembly. It occurred to me, “I shall see the Buddha and inform him of this.” These are the two reasons I’ve come to see the Buddha. 

\section*{4. The Council of the Gods }

Sir,\marginnote{12.1} it was more than a few days ago—on the fifteenth day sabbath on the full moon day at the entry to the rainy season—when all the gods of the thirty-three were sitting together in the Hall of Justice.\footnote{It is not just the Vajjis and the Buddhist \textsanskrit{Saṅgha} who meet frequently in a hall to discuss business, but the gods as well. Here we get a rare glimpse into how the heavens work, or more to the point, how depictions of heavenly proceedings act as a template for how things should be on earth. } A large assembly of gods was sitting all around, and the four great kings were seated at the four quarters.\footnote{The thirty-three and the Four Great Kings are both present; the heavenly realms are not shut off from one another. } 

The\marginnote{12.3} Great King \textsanskrit{Dhataraṭṭha} was seated to the east, facing west, in front of his gods. The Great King \textsanskrit{Virūḷhaka} was seated to the south, facing north, in front of his gods. The Great King \textsanskrit{Virūpakkha} was seated to the west, facing east, in front of his gods. The Great King \textsanskrit{Vessavaṇa} was seated to the north, facing south, in front of his gods. When the gods of the thirty-three have a gathering like this, that is how they are seated. After that come our seats. 

Sir,\marginnote{12.9} those gods who had been recently reborn in the company of the thirty-three after leading the spiritual life under the Buddha outshone the other gods in beauty and glory. The gods of the thirty-three became uplifted and overjoyed at that, full of rapture and happiness, saying, “The heavenly hosts swell, while the titan hosts dwindle!” 

Seeing\marginnote{13.1} the joy of those gods, Sakka, lord of gods, celebrated with these verses: 

\begin{verse}%
“The\marginnote{13.2} gods rejoice—\\
the thirty-three with their Lord—\\
revering the Realized One, \\
and the natural excellence of the teaching; 

and\marginnote{13.6} seeing the new gods, \\
so beautiful and glorious, \\
who have come here after leading \\
the spiritual life under the Buddha! 

They\marginnote{13.10} outshine the others \\
in beauty, glory, and lifespan. \\
Here are the distinguished disciples \\
of he whose wisdom is vast. 

Seeing\marginnote{13.14} this, they delight—\\
the thirty-three with their Lord—\\
revering the Realized One, \\
and the natural excellence of the teaching.” 

%
\end{verse}

The\marginnote{13.18} gods of the thirty-three became even more uplifted and overjoyed at that, saying: “The heavenly hosts swell, while the titan hosts dwindle!” 

Then\marginnote{14.1} the gods of the thirty-three, having considered and deliberated on the matter for which they were seated together in the Hall of Justice, advised and instructed the four great kings on the subject. Each one, having been advised, stood at his own seat without departing.\footnote{For \textit{vipakkamati} compare \href{https://suttacentral.net/mn127/en/sujato\#11.1}{MN 127:11.1}. } 

\begin{verse}%
The\marginnote{14.3} Kings were instructed, \\
and heeded good advice. \\
With clear and peaceful minds, \\
they stood by their own seats. 

%
\end{verse}

Then\marginnote{15.1} in the northern quarter a magnificent light arose and radiance appeared, surpassing the glory of the gods.\footnote{Compare \href{https://suttacentral.net/dn11/en/sujato\#80.12}{DN 11:80.12}. } Then Sakka, lord of gods, addressed the gods of the thirty-three, “As indicated by the signs—light arising and radiance appearing—The Divinity will appear. For this is the precursor for the appearance of the Divinity, namely light arising and radiance appearing.” 

\begin{verse}%
As\marginnote{15.4} indicated by the signs, \\
The Divinity will appear. \\
For this is the sign of the Divinity: \\
a light vast and great. 

%
\end{verse}

\section*{5. On \textsanskrit{Sanaṅkumāra} }

Then\marginnote{16.1} the gods of the thirty-three sat in their own seats, saying, “We shall find out what has caused that light, and only when we have realized it shall we go to it.”\footnote{For \textit{\textsanskrit{vipāko} bhavissati} in discerning the results of signs, compare \href{https://suttacentral.net/dn1/en/sujato\#1.24.2}{DN 1:1.24.2}. } And the four great kings did likewise. 

Hearing\marginnote{16.5} that, the gods of the thirty-three agreed in unison, “We shall find out what has caused that light, and only when we have realized it shall we go to it.” 

When\marginnote{17.1} the divinity \textsanskrit{Sanaṅkumāra} appears to the gods of the thirty-three, he does so after manifesting in a solid life-form.\footnote{See \href{https://suttacentral.net/an3.127/en/sujato\#2.3}{AN 3.127:2.3}. } For a Divinity’s normal appearance is imperceptible in the visual range of the gods of the thirty-three.\footnote{This phrase is also at \href{https://suttacentral.net/dn19/en/sujato\#16.8}{DN 19:16.8}, but apart from that the words \textit{\textsanskrit{pakativaṇṇa}} (“normal appearance”), \textit{\textsanskrit{anabhisambhavanīya}} (“imperceptible”), and \textit{cakkhupatha} (“visual range”) are all unique in the early texts. } When the divinity \textsanskrit{Sanaṅkumāra} appears to the gods of the thirty-three, he outshines the other gods in beauty and glory, as a golden statue outshines the human form.\footnote{For the unique term “human form” (\textit{\textsanskrit{mānusaṁ} \textsanskrit{viggahaṁ}}) compare the Vinaya phrase \textit{\textsanskrit{manussaviggahaṁ}} (\href{https://suttacentral.net/pli-tv-bu-vb-pj3/en/sujato\#2.49.1}{Bu Pj 3:2.49.1}). } 

When\marginnote{17.6} the divinity \textsanskrit{Sanaṅkumāra} appears to the gods of the thirty-three, not a single god in that assembly greets him by bowing down or rising up or inviting him to a seat. They all sit silently on their couches with their joined palms raised, thinking, “Now the divinity \textsanskrit{Sanaṅkumāra} will sit on the couch of whatever god he chooses.” And the god on whose couch the Divinity sits is overjoyed and brimming with happiness, like a king on the day of his coronation. 

Then\marginnote{18.3} the divinity \textsanskrit{Sanaṅkumāra} manifested in a solid life-form, taking on the appearance of the youth \textsanskrit{Pañcasikha}, and appeared to the gods of the thirty-three.\footnote{Also appearing in \href{https://suttacentral.net/dn21/en/sujato\#1.8.3}{DN 21:1.8.3}, \href{https://suttacentral.net/dn20/en/sujato\#10.11}{DN 20:10.11}, and \href{https://suttacentral.net/sn35.119/en/sujato}{SN 35.119}, \textsanskrit{Pañcasikha} (“Five-Crest”) was a handsome and virile deity of the \textit{gandhabbas}. } Rising into the air, he sat cross-legged in the sky, like a strong man might sit cross-legged on a well-appointed couch or on level ground. Seeing the joy of those gods, the divinity \textsanskrit{Sanaṅkumāra} celebrated with these verses: 

\begin{verse}%
“The\marginnote{18.7} gods rejoice—\\
the thirty-three with their Lord—\\
revering the Realized One, \\
and the natural excellence of the teaching; 

and\marginnote{18.11} seeing the new gods, \\
so beautiful and glorious, \\
who have come here after leading \\
the spiritual life under the Buddha! 

They\marginnote{18.15} outshine the others \\
in beauty, glory, and lifespan. \\
Here are the distinguished disciples \\
of he whose wisdom is vast. 

Seeing\marginnote{18.19} this, they delight—\\
the thirty-three with their Lord—\\
revering the Realized One, \\
and the natural excellence of the teaching!” 

%
\end{verse}

That\marginnote{19.1} is the topic on which the divinity \textsanskrit{Sanaṅkumāra} spoke.\footnote{\textit{\textsanskrit{Bhāsittha}} is 3rd singular aorist middle voice. } And while he was speaking on that topic, his voice had eight qualities: it was clear, comprehensible, charming, audible, lucid, undistorted, deep, and resonant.\footnote{Compare \href{https://suttacentral.net/mn91/en/sujato\#21.4}{MN 91:21.4}. } He makes sure his voice is intelligible as far as the assembly goes, but the sound doesn’t extend outside the assembly. When someone has a voice like this, they’re said to have the voice of the Divinity. 

Then\marginnote{20.1} the divinity \textsanskrit{Sanaṅkumāra}, having manifested thirty-three life-forms, sat down on the couches of each of the gods of the thirty-three and addressed them, “What do the good gods of the thirty-three think? How the Buddha has acted for the welfare and happiness of the people, out of sympathy for the world, for the benefit, welfare, and happiness of gods and humans!\footnote{Everything \textsanskrit{Sanaṅkumāra} says is just a bit off. Here he adopts an idiom commonly used by the Buddha, but in third person rather than the Buddha’s second person; and he asks only rhetorically, where the Buddha engages with his audience. If it were only this one example it would mean nothing, but similar changes happen throughout. I believe this is a subtle literary device that tells two narratives to two audiences. To potential converts it sounds like \textsanskrit{Sanaṅkumāra} is giving a ringing and learned endorsement of Buddhism, while to knowledgeable Buddhists he appears as less than well versed in the teachings. I note these eccentric wordings as we proceed. } For consider those who have gone for refuge to the Buddha, the teaching, and the \textsanskrit{Saṅgha}, and have fulfilled their ethics. When their bodies break up, after death, some are reborn in the company of the gods who control what is imagined by others, some with the gods who love to imagine, some with the joyful gods, some with the gods of Yama, some with the gods of the thirty-three, and some with the gods of the four great kings.\footnote{The sutta began by recounting those who have achieved various stages of awakening. But escape from rebirth threatens the gods; this potentially fraught relationship is played out in \href{https://suttacentral.net/mn49/en/sujato}{MN 49}. \textsanskrit{Sanaṅkumāra} is preempting such arguments by pointing out that many of the Buddha’s followers are reborn among the gods and do not escape transmigration, at least not for now. This is no trivial metaphysical argument. In order for Buddhism to prevail, it must show that its radical soteriology is compatible with worldly prosperity, lest it face opposition from kings and other temporal powers. The gods act as proxies to demonstrate the appropriate behavior for terrestrial kings. This is why the leading character is \textsanskrit{Bimbisāra}/Janavasabha, who straddles the two realms. } And at the very least they swell the hosts of the centaurs.” 

That\marginnote{21.1} is the topic on which the divinity \textsanskrit{Sanaṅkumāra} spoke. And hearing the sound of the Divinity speaking on that topic, the gods fancied, “The one sitting on my couch is the only one speaking.” 

\begin{verse}%
When\marginnote{21.4} one is speaking, \\
all the forms speak. \\
When one sits in silence, \\
they all remain silent. 

But\marginnote{21.8} those gods imagine—\\
the thirty-three with their Lord—\\
that the one on their seat \\
is the only one to speak. 

%
\end{verse}

Next\marginnote{22.1} the divinity \textsanskrit{Sanaṅkumāra} merged into one corporeal form. Then he sat on the couch of Sakka, lord of gods, and addressed the gods of the thirty-three:\footnote{\textsanskrit{Sanaṅkumāra} begins sharing the Buddha’s teaching with the gods, starting with the bases of psychic power that featured prominently in DN 16. Here, however, he focuses on the worldly dimensions of psychic powers, ignoring the liberating dimension that was central to the Buddha. This sets the pattern for the teachings to follow, except for the very last. } 

\section*{6. Developing the Bases of Psychic Power }

“What\marginnote{22.3} do the good gods of the thirty-three think? How well described by the Blessed One—who knows and sees, the perfected one, the fully awakened Buddha—are the four bases of psychic power! They are taught for the amplification, burgeoning, and transformation of psychic power.\footnote{\textit{\textsanskrit{Yāva} \textsanskrit{supaññattā}}, which is also at \href{https://suttacentral.net/mn51/en/sujato\#4.3}{MN 51:4.3}, is a variation of the common exclamation \textit{\textsanskrit{yāva} \textsanskrit{subhāsita}}. The phrase \textit{\textsanskrit{iddhipahutāya} \textsanskrit{iddhivisavitāya} \textsanskrit{iddhivikubbanatāya}} is unique. The \textsanskrit{Paṭisambhidāmagga} draws on this passage to explain \textit{vikubbana} both as a general term for development of psychic powers (\href{https://suttacentral.net/ps3.2/en/sujato\#4.3}{Ps 3.2:4.3}) and as specific kind of psychic power, namely the transformation of one’s apparent form as demonstrated by \textsanskrit{Sanaṅkumāra} (\href{https://suttacentral.net/ps3.2/en/sujato\#15.1}{Ps 3.2:15.1}). } What four? It’s when a mendicant develops the basis of psychic power that has immersion due to enthusiasm, and active effort. They develop the basis of psychic power that has immersion due to energy, and active effort. They develop the basis of psychic power that has immersion due to mental development, and active effort. They develop the basis of psychic power that has immersion due to inquiry, and active effort. These are the four bases of psychic power taught by the Buddha for the amplification, burgeoning, and transformation of psychic power. 

All\marginnote{22.10} the ascetics and brahmins in the past, future, or present who wield the many kinds of psychic power do so by developing and cultivating these four bases of psychic power. Gentlemen, do you see such psychic might and power in me?” 

“Yes,\marginnote{22.14} Great Divinity.” 

“I\marginnote{22.15} too became so mighty and powerful by developing and cultivating these four bases of psychic power.” 

That\marginnote{23.1} is the topic on which the divinity \textsanskrit{Sanaṅkumāra} spoke. And having spoken about that, he addressed the gods of the thirty-three: 

\section*{7. The Three Openings }

“What\marginnote{23.4} do the good gods of the thirty-three think? How well understood by the Buddha are the three opportunities for achieving happiness!\footnote{These “three opportunities” are not found elsewhere. } What three? 

First,\marginnote{23.6} take someone who lives mixed up with sensual pleasures and unskillful qualities.\footnote{This unique term is the reverse of the phrase that begins the first \textit{\textsanskrit{jhāna}}. The passage, however, speaks only of lifestyle rather than deep meditation. } After some time they hear the teaching of the noble ones, rationally apply the mind to how it applies to them, and practice accordingly. They live aloof from sensual pleasures and unskillful qualities.\footnote{“Aloof” (\textit{\textsanskrit{asaṁsaṭṭha}}) is a synonym of “secluded” (\textit{vivicca}). The \textit{\textsanskrit{jhāna}} formulas are extremely stable and it is rare to find them played with like this. } That gives rise to pleasure, and more than pleasure, happiness,\footnote{Since \textit{\textsanskrit{jhāna}} arises from pleasure, it cannot be meant here. } like the joy that’s born from gladness. This is the first opportunity for achieving happiness. 

Next,\marginnote{24.1} take someone whose coarse physical, verbal, and mental processes have not died down.\footnote{“Coarse physical, verbal, and mental processes” (\textit{\textsanskrit{oḷārikā} \textsanskrit{kāyasaṅkhārā} …\textsanskrit{vacīsaṅkhārā} … \textsanskrit{cittasaṅkhārā}}) is another unique term. They are probably to be identified with the three “processes” of \href{https://suttacentral.net/mn44/en/sujato\#14.2}{MN 44:14.2}. } After some time they hear the teaching of the noble ones, rationally apply the mind to how it applies to them, and practice accordingly. Their coarse physical, verbal, and mental processes die down. That gives rise to pleasure, and more than pleasure, happiness, like the joy that’s born from gladness. This is the second opportunity for achieving happiness. 

Next,\marginnote{25.1} take someone who doesn’t truly understand what is skillful and what is unskillful,\footnote{This builds off \href{https://suttacentral.net/dn1/en/sujato\#2.24.1}{DN 1:2.24.1}, but the full phrase is only here and \href{https://suttacentral.net/dn19/en/sujato\#7.2}{DN 19:7.2}. } what is blameworthy and what is blameless, what should be cultivated and what should not be cultivated, what is inferior and what is superior, and what is on the side of dark and the side of bright. After some time they hear the teaching of the noble ones, rationally apply the mind to how it applies to them, and practice accordingly. They truly understand what is skillful and what is unskillful, and so on. Knowing and seeing like this, ignorance is given up and knowledge arises. That gives rise to pleasure, and more than pleasure, happiness, like the joy that’s born from gladness. This is the third opportunity for achieving happiness. 

These\marginnote{25.11} are the three opportunities for achieving happiness that have been understood by the Buddha.” 

That\marginnote{26.1} is the topic on which the divinity \textsanskrit{Sanaṅkumāra} spoke. And having spoken about that, he addressed the gods of the thirty-three: 

\section*{8. Mindfulness Meditation }

“What\marginnote{26.4} do the good gods of the thirty-three think? How well described by the Buddha are the four kinds of mindfulness meditation! They are taught for achieving what is skillful.\footnote{Normally \textit{\textsanskrit{satipaṭṭhāna}} is taught “in order to purify sentient beings, to get past sorrow and crying, to make an end of pain and sadness, to end the cycle of suffering, and to realize extinguishment” (eg. \href{https://suttacentral.net/dn22/en/sujato\#1.7}{DN 22:1.7}). Here, in yet another unique phrasing, a more humble goal is sought. } What four? 

It’s\marginnote{26.6} when a mendicant meditates by observing an aspect of the body internally—keen, aware, and mindful, rid of covetousness and displeasure for the world. As they meditate in this way, they become rightly immersed in that, and rightly serene. Then they give rise to knowledge and vision of other people’s bodies externally.\footnote{Again the liberating dimension is ignored in favor of worldly psychic abilities. This is another unique formulation. } 

They\marginnote{26.9} meditate observing an aspect of feelings internally … Then they give rise to knowledge and vision of other people’s feelings externally. 

They\marginnote{26.11} meditate observing an aspect of the mind internally … Then they give rise to knowledge and vision of other people’s minds externally. 

They\marginnote{26.13} meditate observing an aspect of principles internally—keen, aware, and mindful, rid of covetousness and displeasure for the world. As they meditate in this way, they become rightly immersed in that, and rightly serene. Then they give rise to knowledge and vision of other people’s principles externally. 

These\marginnote{26.16} are the four kinds of mindfulness meditation taught by the Buddha for achieving what is skillful.” 

That\marginnote{27.1} is the topic on which the divinity \textsanskrit{Sanaṅkumāra} spoke. And having spoken about that, he addressed the gods of the thirty-three: 

\section*{9. Seven Prerequisites of Immersion }

“What\marginnote{27.4} do the good gods of the thirty-three think? How well described by the Buddha are the seven prerequisites of immersion for the development and fulfillment of right immersion!\footnote{Also found at \href{https://suttacentral.net/an7.45/en/sujato\#1.1}{AN 7.45:1.1}. At \href{https://suttacentral.net/mn44/en/sujato\#12.4}{MN 44:12.4} the four right efforts are said to be the “prerequisites of immersion”. Here at last \textsanskrit{Sanaṅkumāra} introduces the liberating dimension of the path as a whole. } What seven? Right view, right thought, right speech, right action, right livelihood, right effort, and right mindfulness. Unification of mind with these seven factors as prerequisites is what is called noble right immersion ‘with its vital conditions’ and also ‘with its prerequisites’. 

Right\marginnote{27.8} view gives rise to right thought. Right thought gives rise to right speech. Right speech gives rise to right action. Right action gives rise to right livelihood. Right livelihood gives rise to right effort. Right effort gives rise to right mindfulness. Right mindfulness gives rise to right immersion. Right immersion gives rise to right knowledge. Right knowledge gives rise to right freedom.\footnote{Also at \href{https://suttacentral.net/sn45.1/en/sujato}{SN 45.1}, etc. } 

If\marginnote{27.9} anything should be rightly described as ‘a teaching that’s well explained by the Buddha, apparent in the present life, immediately effective, inviting inspection, relevant, so that sensible people can know it for themselves; and the doors to freedom from death are flung open,’ it’s this.\footnote{To the standard passage on the qualities of the Dhamma, \textsanskrit{Sanaṅkumāra} clumsily tacks on the line spoken by the Buddha when he was persuaded by \textsanskrit{Brahmā} to teach (\href{https://suttacentral.net/sn6.1/en/sujato\#10.1}{SN 6.1:10.1}, \href{https://suttacentral.net/dn14/en/sujato\#3.7.13}{DN 14:3.7.13}, \href{https://suttacentral.net/mn26/en/sujato\#21.6}{MN 26:21.6}, \href{https://suttacentral.net/mn85/en/sujato\#45.6}{MN 85:45.6}). } For the teaching is well explained by the Buddha—apparent in the present life, immediately effective, inviting inspection, relevant, so that sensible people can know it for themselves—and the doors to freedom from death are flung open. 

Whoever\marginnote{27.12} has experiential confidence in the Buddha, the teaching, and the \textsanskrit{Saṅgha}, and has the ethical conduct loved by the noble ones; and whoever is spontaneously reborn, and is trained in the teaching; more than 2,400,000 such Magadhan devotees have passed away having ended three fetters. They’re stream-enterers, not liable to be reborn in the underworld, bound for awakening.\footnote{Finally the question is answered, although the number of deaths is obviously exaggerated. We cannot estimate the population size at the time with any confidence. But we know that \textsanskrit{Pāṭaliputta} under Ashoka covered about 25 km², which suggests a population somewhere around 250,000. Probably there were a few million people in the whole of Magadha. } And there are once-returners here, too. 

\begin{verse}%
And\marginnote{27.14} as for the rest of folk,\footnote{The commentary identifies these as non-returners, but this is unlikely as \textit{\textsanskrit{puññābhāga}} (“share of merit”) indicates those who do good for the sake of a good rebirth rather than liberation (\href{https://suttacentral.net/an6.63/en/sujato\#30.2}{AN 6.63:30.2}). There would have been countless more of such good folk than stream-enterers, whereas the number of non-returners, as indicated by the count of the \textsanskrit{Ñātikans}, would have been small, not “countless”. This verse is also at \href{https://suttacentral.net/sn6.13/en/sujato\#7.1}{SN 6.13:7.1}, where the commentary says nothing of non-returners. } \\
who I think have shared in merit—\\
I couldn’t even number them, \\
for fear of speaking falsely.” 

%
\end{verse}

That\marginnote{28.1} is the topic on which the divinity \textsanskrit{Sanaṅkumāra} spoke. And while he was speaking on that topic, this thought came to the great king \textsanskrit{Vessavaṇa}, “Oh, how incredible, how amazing! That there should be such a magnificent Teacher, and such a magnificent exposition of the teaching! And that such achievements of high distinction should be made known!” 

And\marginnote{28.3} then the divinity \textsanskrit{Sanaṅkumāra}, knowing the great king \textsanskrit{Vessavaṇa}’s train of thought, said to him, “What does Great King \textsanskrit{Vessavaṇa} think? In the past, too, there was such a magnificent Teacher, and such a magnificent exposition of the teaching! And such achievements of high distinction were made known!\footnote{\textsanskrit{Sanaṅkumāra} lives longer than the lesser gods so he has a broader perspective. He emphasizes the long-term stability of cosmic order, which prevails through the crises that afflict the short-lived kingdoms of men. } In the future, too, there will be such a magnificent Teacher, and such a magnificent exposition of the teaching! And such achievements of high distinction will be made known!” 

That,\marginnote{29.1} sir, is the topic on which the divinity \textsanskrit{Sanaṅkumāra} spoke to the gods of the thirty-three. And the great king \textsanskrit{Vessavaṇa}, having heard and learned it in the presence of the Divinity as he was speaking on that topic, informed his own assembly.’”\footnote{Astonishingly, this sutta traces its textual lineage directly to \textsanskrit{Brahmā}, exactly like the Brahmanical texts (eg. \textsanskrit{Bṛhadāraṇyaka} \textsanskrit{Upaniṣad} 6.5.4). Of course, \textsanskrit{Brahmā} gets his teaching from the Buddha, but this is clearly copying the Brahmanical model. } 

And\marginnote{29.2} the spirit Janavasabha, having heard and learned it in the presence of the great king \textsanskrit{Vessavaṇa} as he was speaking on that topic to his own assembly, informed the Buddha. And the Buddha, having heard and learned it in the presence of the spirit Janavasabha, and also from his own direct knowledge, informed Venerable Ānanda. And Venerable Ānanda, having heard and learned it in the presence of the Buddha, informed the monks, nuns, laymen, and laywomen.\footnote{Here Ānanda plays a key role as the linchpin of the oral tradition. It is likely that not just the \textsanskrit{Mahāparinibbānasutta} itself, but all these suttas of this cycle were composed by Ānanda or his students. } And that’s how this spiritual life has become successful and prosperous, extensive, popular, widespread, and well proclaimed wherever there are gods and humans.\footnote{With this unique ending, there is no claim that this sutta was heard by the mendicants in the usual way. Rather it makes its purpose explicit, to act as a means to widely propagate the Dhamma. The \textsanskrit{Mahāparinibbānasutta} also speaks of the spread of the Dhamma, and the Janavasabhasutta indicates some of the means by which that was achieved in the decades following the Buddha’s death; in particular, by allying itself with the ascendant power of Magadha. } 

%
\chapter*{{\suttatitleacronym DN 19}{\suttatitletranslation The Great Steward }{\suttatitleroot Mahāgovindasutta}}
\addcontentsline{toc}{chapter}{\tocacronym{DN 19} \toctranslation{The Great Steward } \tocroot{Mahāgovindasutta}}
\markboth{The Great Steward }{Mahāgovindasutta}
\extramarks{DN 19}{DN 19}

\scevam{So\marginnote{1.1} I have heard. }At one time the Buddha was staying near \textsanskrit{Rājagaha}, on the Vulture’s Peak Mountain. 

Then,\marginnote{1.3} late at night, the centaur \textsanskrit{Pañcasikha}, lighting up the entire Vulture’s Peak, went up to the Buddha, bowed, stood to one side, and said to him,\footnote{We were introduced to \textsanskrit{Pañcasikha} in \href{https://suttacentral.net/dn18/en/sujato}{DN 18}, which is similar to the current sutta in many other details as well. Thus this sutta can be considered as a distant cousin to the \textsanskrit{Mahāparinibbāna} cycle. } “Sir, I would tell you of what I heard and learned directly from the gods of the thirty-three.” 

“Tell\marginnote{1.5} me, \textsanskrit{Pañcasikha},” said the Buddha. 

\section*{1. The Council of the Gods }

“Sir,\marginnote{2.1} it was more than a few days ago—on the fifteenth day sabbath on the full moon day at the invitation to admonish held at the end of the rainy season—when all the gods of the thirty-three were sitting together in the Hall of Justice.\footnote{The council of gods described by Janavasabha at \href{https://suttacentral.net/dn18/en/sujato\#12.1}{DN 18:12.1}, while otherwise similar, took place at the entry to a rainy season, whereas this one is at the end of a rainy season. This cannot have been the Buddha’s last rains, for he was already in Vajji by then. } A large assembly of gods was sitting all around, and the four great kings were seated at the four quarters. 

The\marginnote{2.3} Great King \textsanskrit{Dhataraṭṭha} was seated to the east, facing west, in front of his gods. The Great King \textsanskrit{Virūḷhaka} was seated to the south, facing north, in front of his gods. The Great King \textsanskrit{Virūpakkha} was seated to the west, facing east, in front of his gods. The Great King \textsanskrit{Vessavaṇa} was seated to the north, facing south, in front of his gods. 

When\marginnote{2.7} the gods of the thirty-three have a gathering like this, that is how they are seated. After that come our seats. 

Sir,\marginnote{3.1} those gods who had been recently reborn in the company of the thirty-three after leading the spiritual life under the Buddha outshine the other gods in beauty and glory. The gods of the thirty-three became uplifted and overjoyed at that, full of rapture and happiness, saying, ‘The heavenly hosts swell, while the titan hosts dwindle!’ 

Seeing\marginnote{3.4} the joy of those gods, Sakka, lord of gods, celebrated with these verses: 

\begin{verse}%
‘The\marginnote{3.5} gods rejoice—\\
the thirty-three with their Lord—\\
revering the Realized One, \\
and the natural excellence of the teaching; 

and\marginnote{3.9} seeing the new gods, \\
so beautiful and glorious, \\
who have come here after leading \\
the spiritual life under the Buddha! 

They\marginnote{3.13} outshine the others \\
in beauty, glory, and lifespan. \\
Here are the distinguished disciples \\
of he whose wisdom is vast. 

Seeing\marginnote{3.17} this, they delight—\\
the thirty-three with their Lord—\\
revering the Realized One, \\
and the natural excellence of the teaching!’ 

%
\end{verse}

The\marginnote{3.21} gods of the thirty-three became even more uplifted and overjoyed at that, full of rapture and happiness, saying, ‘The heavenly hosts swell, while the titan hosts dwindle!’ 

\section*{2. Eight Genuine Praises }

Seeing\marginnote{4.1} the joy of those gods, Sakka, lord of gods, addressed them, ‘Gentlemen, would you like to hear eight genuine praises of the Buddha?’\footnote{For “genuine praise” (\textit{\textsanskrit{yathābhucce} \textsanskrit{vaṇṇe}}), see \href{https://suttacentral.net/dn1/en/sujato\#1.28.1}{DN 1:1.28.1}. } 

‘Indeed\marginnote{4.3} we would, sir.’ 

Then\marginnote{4.4} Sakka proffered these eight genuine praises of the Buddha:\footnote{This set of eight is not found elsewhere. | “Proffered” is \textit{\textsanskrit{payirudāhāsi}}, a unique term in early texts. } 

‘What\marginnote{5.1} do the good gods of the thirty-three think?\footnote{Sakka adopts the same rhetorical style as \textsanskrit{Sanaṅkumāra} (\href{https://suttacentral.net/dn18/en/sujato\#20.2}{DN 18:20.2}). } How the Buddha has acted for the welfare and happiness of the people, out of sympathy for the world, for the benefit, welfare, and happiness of gods and humans! I don’t see any Teacher, past or present, who has such sympathy for the world, apart from the Buddha. 

Also,\marginnote{6.1} the Buddha has explained the teaching well—apparent in the present life, immediately effective, inviting inspection, relevant, so that sensible people can know it for themselves. I don’t see any Teacher, past or present, who explains such a relevant teaching, apart from the Buddha. 

Also,\marginnote{7.1} the Buddha has well described what is skillful and what is unskillful,\footnote{Compare \href{https://suttacentral.net/dn18/en/sujato\#25.1}{DN 18:25.1}. } what is blameworthy and what is blameless, what should be cultivated and what should not be cultivated, what is inferior and what is superior, and what is on the side of dark and the side of bright. I don’t see any Teacher, past or present, who so clearly describes all these things, apart from the Buddha. 

Also,\marginnote{8.1} the Buddha has well described the practice that leads to extinguishment for his disciples. And extinguishment converges with the practice,\footnote{\textit{\textsanskrit{Paṭipadā}} is instrumental. } as the waters of the Ganges come together and converge with the waters of the Yamuna.\footnote{I don’t think this idea is found elsewhere in quite this way. Normally it is said that the path leads to \textsanskrit{Nibbāna} like the rivers lead to the ocean (eg. \href{https://suttacentral.net/sn45.114/en/sujato\#1.1}{SN 45.114:1.1}). } I don’t see any Teacher, past or present, who so clearly describes the practice that leads to extinguishment for his disciples, apart from the Buddha. 

Also,\marginnote{9.1} possessions and popularity have accrued to the Buddha, so much that you’d think it would thrill even the aristocrats. But he takes his food free of vanity.\footnote{This item is unique. } I don’t see any Teacher, past or present, who takes their food so free of vanity, apart from the Buddha. 

Also,\marginnote{10.1} the Buddha has gained companions, both trainees who are practicing, and those with defilements ended who have completed their journey.\footnote{Another unique item. } The Buddha is committed to the joy of solitude, but doesn’t send them away.\footnote{This is normally true, but in some cases the Buddha left a badly-behaved community (\href{https://suttacentral.net/mn48/en/sujato}{MN 48}), dismissed misbehaving monks (eg. \href{https://suttacentral.net/an8.10/en/sujato\#2.2}{AN 8.10:2.2}), or even dismissed a large community (\href{https://suttacentral.net/mn67/en/sujato}{MN 67}). | The “joy of solitude” is \textit{\textsanskrit{ekārāmataṁ} }. } I don’t see any Teacher, past or present, so committed to the joy of solitude, apart from the Buddha. 

Also,\marginnote{11.1} the Buddha does as he says, and says as he does, thus: he does as he says, and says as he does.\footnote{Also at \href{https://suttacentral.net/dn29/en/sujato\#29.5}{DN 29:29.5}, \href{https://suttacentral.net/an4.23/en/sujato\#4.1}{AN 4.23:4.1}, and \href{https://suttacentral.net/iti112/en/sujato\#5.1}{Iti 112:5.1}. } I don’t see any Teacher, past or present, who so practices in line with the teaching, apart from the Buddha. 

Also,\marginnote{12.1} the Buddha has gone beyond doubt and got rid of indecision. He has achieved all he wished for regarding the fundamental purpose of the spiritual life.\footnote{\textit{\textsanskrit{Pariyositasaṅkappo}} (“achieved all he wished for”) is a synonym of the more common \textit{\textsanskrit{paripuṇṇasaṅkappo}} (eg. \href{https://suttacentral.net/mn29/en/sujato\#2.5}{MN 29:2.5}). } I don’t see any Teacher, past or present, who has achieved these things, apart from the Buddha.’ 

These\marginnote{12.3} are the eight genuine praises of the Buddha proffered by Sakka. Hearing them, the gods of the thirty-three became even more uplifted and overjoyed. 

Then\marginnote{13.1} some gods thought, ‘If only four fully awakened Buddhas might arise in the world and teach the Dhamma, just like the Blessed One! That would be for the welfare and happiness of the people, out of sympathy for the world, for the benefit, welfare, and happiness of gods and humans!’ 

Other\marginnote{13.4} gods thought, ‘Let alone four fully awakened Buddhas; if only three fully awakened Buddhas, or two fully awakened Buddhas might arise in the world and teach the Dhamma, just like the Blessed One! That would be for the welfare and happiness of the people, out of sympathy for the world, for the benefit, welfare, and happiness of gods and humans!’ 

When\marginnote{14.1} they said this, Sakka said, ‘It is impossible, gentlemen, for two perfected ones, fully awakened Buddhas to arise in the same solar system at the same time.\footnote{This impossibility is also at \href{https://suttacentral.net/an1.277/en/sujato\#1.1}{AN 1.277:1.1} and \href{https://suttacentral.net/mn115/en/sujato\#14.1}{MN 115:14.1}. | “In one solar system” (\textit{\textsanskrit{ekissā} \textsanskrit{lokadhātuyā}}): a single \textit{\textsanskrit{lokadhātu}} encompasses a single terrestrial world with a single moon and sun. } May that Blessed One be healthy and well, and remain with us for a long time!\footnote{It is not common to wish for the Buddha’s long life, but we do find this sentiment expressed in a conversation between \textsanskrit{Sāriputta} and Ānanda (\href{https://suttacentral.net/sn21.2/en/sujato\#3.5}{SN 21.2:3.5}). } That would be for the welfare and happiness of the people, out of sympathy for the world, for the benefit, welfare, and happiness of gods and humans!’ 

Then\marginnote{14.5} the gods of the thirty-three, having considered and deliberated on the matter for which they were seated together in the Hall of Justice, advised and instructed the four great kings on the subject. Each one, having been advised, stood at his own seat without departing. 

\begin{verse}%
The\marginnote{14.7} Kings were instructed, \\
and heeded good advice. \\
With clear and peaceful minds, \\
they stood by their own seats. 

%
\end{verse}

Then\marginnote{15.1} in the northern quarter a magnificent light arose and radiance appeared, surpassing the glory of the gods. Then Sakka, lord of gods, addressed the gods of the thirty-three, ‘As indicated by the signs—light arising and radiance appearing—the Divinity will appear. For this is the precursor for the appearance of the Divinity, namely light arising and radiance appearing.’ 

\begin{verse}%
As\marginnote{15.5} indicated by the signs, \\
the Divinity will appear. \\
For this is the sign of the Divinity: \\
a light vast and great. 

%
\end{verse}

\section*{3. On \textsanskrit{Sanaṅkumāra} }

Then\marginnote{16.1} the gods of the thirty-three sat in their own seats, saying, ‘We shall find out what has caused that light, and only when we have realized it shall we go to it.’ And the four great kings did likewise. Hearing that, the gods of the thirty-three agreed in unison, ‘We shall find out what has caused that light, and only when we have realized it shall we go to it.’ 

When\marginnote{16.7} the divinity \textsanskrit{Sanaṅkumāra} appears to the gods of the thirty-three, he does so after manifesting in a solid corporeal form. For a Divinity’s normal appearance is imperceptible in the visual range of the gods of the thirty-three. When the divinity \textsanskrit{Sanaṅkumāra} appears to the gods of the thirty-three, he outshines the other gods in beauty and glory, as a golden statue outshines the human form. When the divinity \textsanskrit{Sanaṅkumāra} appears to the gods of the thirty-three, not a single god in that assembly greets him by bowing down or rising up or inviting him to a seat. They all sit silently on their couches with their joined palms raised, thinking, ‘Now the divinity \textsanskrit{Sanaṅkumāra} will sit on the couch of whatever god he chooses.’ And the god on whose couch the Divinity sits is overjoyed and brimming with happiness, like a king on the day of his coronation. 

Seeing\marginnote{17.3} the joy of those gods, the divinity \textsanskrit{Sanaṅkumāra} celebrated with these verses: 

\begin{verse}%
‘The\marginnote{17.4} gods rejoice—\\
the thirty-three with their Lord—\\
revering the Realized One, \\
and the natural excellence of the teaching; 

and\marginnote{17.8} seeing the new gods, \\
so beautiful and glorious, \\
who have come here after leading \\
the spiritual life under the Buddha! 

They\marginnote{17.12} outshine the others \\
in beauty, glory, and lifespan. \\
Here are the distinguished disciples \\
of he whose wisdom is vast. 

Seeing\marginnote{17.16} this, they delight—\\
the thirty-three with their Lord—\\
revering the Realized One, \\
and the natural excellence of the teaching!’ 

%
\end{verse}

That\marginnote{18.1} is the topic on which the divinity \textsanskrit{Sanaṅkumāra} spoke. And while he was speaking on that topic, his voice had eight qualities: it was clear, comprehensible, charming, audible, lucid, undistorted, deep, and resonant. He makes sure his voice is intelligible as far as the assembly goes, but the sound doesn’t extend outside the assembly. When someone has a voice like this, they’re said to have the voice of the Divinity. 

Then\marginnote{19.1} the gods of the thirty-three said to the divinity \textsanskrit{Sanaṅkumāra}, ‘Good, Great Divinity! Having appraised this, we rejoice.\footnote{“Having appraised” (\textit{\textsanskrit{saṅkhāya}}) is glossed by the commentary with “having known” (\textit{\textsanskrit{jānitvā}}). Compare the discussion on “judgmentalism” at \href{https://suttacentral.net/dn1/en/sujato\#1.3.1}{DN 1:1.3.1}. } And there are the eight genuine praises of the Buddha spoken by Sakka—having appraised them, too, we rejoice.’ 

\section*{4. Eight Genuine Praises }

Then\marginnote{19.6} the Divinity said to Sakka, ‘It would be good, lord of gods, if I could also hear the eight genuine praises of the Buddha.’ 

Saying,\marginnote{19.8} ‘Yes, Great Divinity,’ Sakka repeated the eight genuine praises for him. 

Hearing\marginnote{28.1} them, the divinity \textsanskrit{Sanaṅkumāra} was uplifted and overjoyed, full of rapture and happiness. Then the divinity \textsanskrit{Sanaṅkumāra} manifested in a solid corporeal form, taking on the appearance of the youth \textsanskrit{Pañcasikha}, and appeared to the gods of the thirty-three. Rising into the air, he sat cross-legged in the sky, like a strong man might sit cross-legged on a well-appointed couch or on level ground. There he addressed the gods of the thirty-three: 

\section*{5. The Story of the Steward }

‘What\marginnote{28.8} do the gods of the thirty-three think about the extent of the Buddha’s great wisdom? 

Once\marginnote{29.1} upon a time, there was a king named Disampati.\footnote{Disampati means “head of the directions”, i.e. king of all the land. He and his story are found nowhere else in the early texts. } He had a brahmin high priest named the Steward.\footnote{“Steward” is Govinda, literally “lord of cows”. The “high priest” (\textit{purohita}) was a hereditary office, a learned ritualist and adviser attached to a specific family. For royal families the post could be highly contested. The closeness of the role is shown in that aristocratic families could be referred to by the lineage of the high priest. } Disampati’s son was the prince named \textsanskrit{Reṇu},\footnote{While the current sutta does not explicitly identify their realm, \textsanskrit{Dīpavaṁsa} iii 40 records Disampati and \textsanskrit{Reṇu} as kings of \textsanskrit{Kāsi}. This fits with the geographical layout depicted in this sutta, with \textsanskrit{Kāsi} at the center. The story, then, depicts the establishment of Brahmanical kings across India from \textsanskrit{Kāsi}. } while the Steward’s son was the student named \textsanskrit{Jotipāla}.\footnote{\textsanskrit{Jotipāla} means “guardian of the sacred flame”, i.e. someone who maintains the Vedic fire ritual (see below at \href{https://suttacentral.net/dn19/en/sujato\#47.26}{DN 19:47.26}). There was another \textsanskrit{Jotipāla} in the time of Buddha Kassapa at \href{https://suttacentral.net/mn81/en/sujato\#6.2}{MN 81:6.2}, and another who was a religious founder of the past (\href{https://suttacentral.net/an6.54/en/sujato\#18.1}{AN 6.54:18.1}, \href{https://suttacentral.net/an7.73/en/sujato\#2.5}{AN 7.73:2.5}). Despite the popularity of the name in Pali, it does not seem to appear in Sanskrit sources. Given that it appears in Pali only in legendary contexts, it is probably a vocational epithet. } There were \textsanskrit{Reṇu} the prince, \textsanskrit{Jotipāla} the student, and six other aristocrats; these eight became friends. 

In\marginnote{29.6} due course the brahmin Steward passed away. At his passing, King Disampati lamented, “At a time when I have relinquished all my duties to the brahmin Steward and amuse myself, supplied and provided with the five kinds of sensual stimulation, he passes away!”\footnote{The king’s only care, it seems, is not for the passing of his friend, but that he no longer gets to indulge in whatever he wants. } 

When\marginnote{29.9} he said this, Prince \textsanskrit{Reṇu} said to him, “Sire, don’t lament too much at the Steward’s passing. He has a son named \textsanskrit{Jotipāla}, who is even more astute and expert than his father. He should manage the affairs that were managed by his father.” 

“Is\marginnote{29.13} that so, my prince?” 

“That\marginnote{29.14} is so, sire.” 

\section*{6. The Story of the Great Steward }

So\marginnote{30.1} King Disampati addressed one of his men, “Please, mister, go to the student \textsanskrit{Jotipāla}, and say to him, ‘Best wishes, \textsanskrit{Jotipāla}! You are summoned by King Disampati; he wants to see you.’” 

“Yes,\marginnote{30.4} Your Majesty,” replied that man, and did as he was asked. Then \textsanskrit{Jotipāla} went to the king and exchanged greetings with him. 

When\marginnote{30.7} the greetings and polite conversation were over, he sat down to one side, and the king said to him, “May you, \textsanskrit{Jotipāla}, manage my affairs—please don’t turn me down! I shall appoint you to your father’s position, and anoint you as Steward.”\footnote{This shows that Govinda is an office rather than a personal name. } 

“Yes,\marginnote{30.10} sir,” replied \textsanskrit{Jotipāla}. 

So\marginnote{31.1} the king anointed him as Steward and appointed him to his father’s position. After his appointment, the Steward \textsanskrit{Jotipāla} managed both the affairs that his father had managed, and other affairs that his father had not managed.\footnote{Following the \textsanskrit{Mahāsaṅgīti} reading rather than the PTS (\textit{\textsanskrit{nānusāsi} … \textsanskrit{nānusāsati}}). It seems required by the context that he does more than his father. } He organized both the works that his father had organized, and other works that his father had not organized. When people noticed this they said, “The brahmin is indeed a Steward, a Great Steward!” And that’s how the student \textsanskrit{Jotipāla} came to be known as the Great Steward. 

\subsection*{6.1. Dividing the Realm }

Then\marginnote{32.1} the Great Steward went to the six aristocrats and said, “King Disampati is old, elderly and senior, advanced in years, and has reached the final stage of life. Who knows how long he has to live? It’s likely that when he passes away the king-makers will anoint Prince \textsanskrit{Reṇu} as king. Come, sirs, go to Prince \textsanskrit{Reṇu} and say, ‘Prince \textsanskrit{Reṇu}, we are your friends, dear, beloved, and cherished. We have shared your joys and sorrows. King Disampati is old, elderly and senior, advanced in years, and has reached the final stage of life. Who knows how long he has to live? It’s likely that when he passes away the king-makers will anoint you as king. If you should gain kingship, share it with us.’” 

“Yes,\marginnote{33.1} sir,” replied the six aristocrats. They went to Prince \textsanskrit{Reṇu} and put the proposal to him. 

The\marginnote{33.7} prince replied, “Who else, sirs, in my realm ought to prosper if not you?\footnote{Following PTS reading \textit{\textsanskrit{sukhaṁ} \textsanskrit{edheyyātha}}. } If I gain kingship, I will share it with you all.” 

In\marginnote{34.1} due course King Disampati passed away. At his passing, the king-makers anointed Prince \textsanskrit{Reṇu} as king. But after being anointed, King \textsanskrit{Reṇu} amused himself, supplied and provided with the five kinds of sensual stimulation. 

Then\marginnote{34.4} the Great Steward went to the six aristocrats and said, “King Disampati has passed away. But after being anointed, King \textsanskrit{Reṇu} amused himself, supplied and provided with the five kinds of sensual stimulation. Who knows the intoxicating power of sensual pleasures? Come, sirs, go to Prince \textsanskrit{Reṇu} and say, ‘Sir, King Disampati has passed away, and you have been anointed as king. Do you remember what you said?’” 

“Yes,\marginnote{34.10} sir,” replied the six aristocrats. They went to King \textsanskrit{Reṇu} and said, “Sir, King Disampati has passed away, and you have been anointed as king. Do you remember what you said?” 

“I\marginnote{34.12} remember, sirs. Who is able to neatly divide into seven equal parts this great land, so broad in the north and narrow as the front of a cart in the south?”\footnote{The meaning of \textit{\textsanskrit{sakaṭamukhaṁ}} (“front of a cart”) is clarified by comparison with the parallels. T 8 is identical (\langlzh{其界廣闊。正南南隅,其界狹略,猶如車形}), while DA 3 expresses a similar idea more briefly (\langlzh{此閻浮提地, 內廣外狹}). The \textsanskrit{Mahāvastu} (Mvu 85.17) confirms this sense with \textit{\textsanskrit{dakṣiṇena} \textsanskrit{saṁkṣiptā} \textsanskrit{śakaṭamukhasaṁsthitaṁ}}. These appear to demonstrate a knowledge of the shape of the Indian subcontinent. Below it says that each of the kingdoms is shaped like the front of a cart, just as India is as a whole. } 

“Who\marginnote{34.14} else, sir, if not the Great Steward?” 

So\marginnote{35.1} King \textsanskrit{Reṇu} addressed one of his men, “Please, mister, go to the brahmin Great Steward and say that King \textsanskrit{Reṇu} summons him.” 

“Yes,\marginnote{35.4} Your Majesty,” replied that man, and did as he was asked. Then the Great Steward went to the king and exchanged greetings with him. 

When\marginnote{35.7} the greetings and polite conversation were over, he sat down to one side, and the king said to him, “Come, let the good Steward neatly divide into seven equal parts this great land, so broad in the north and narrow as the front of a cart in the south.” 

“Yes,\marginnote{35.9} sir,” replied the Great Steward, and did as he was asked. All were arranged like the fronts of carts,\footnote{Neither Rhys Davids nor Walshe translate this line, but it receives an extensive discussion in the commentary. It says that six kingdoms were arranged around \textsanskrit{Reṇu}’s kingdom in the middle, like an umbrella (\textit{\textsanskrit{vitānasadisaṁ}}). } and right in the middle was King \textsanskrit{Reṇu}’s nation.\footnote{The verses tell the realms starting with the rising sun at \textsanskrit{Kaliṅga} in the east and proceeding clockwise (\textit{\textsanskrit{padakkhiṇā}}) until the circle is complete with \textsanskrit{Aṅga} on the \textsanskrit{Kaliṅga} border. Thus the original center was probably the last place on the list, \textsanskrit{Kāsi}, which is indeed geographically central. } 

\begin{verse}%
Dantapura\marginnote{36.3} for the \textsanskrit{Kaliṅgas};\footnote{\textsanskrit{Kaliṅga} was a coastal realm in modern Odisha and Andhra Pradesh. There is no consensus on the exact location of Dantapura. } \\
Potana for the Assakas;\footnote{Assaka stretched from \textsanskrit{Kaliṅga}’s western border across the interior. Potana is modern Bodhan in Telangana state. } \\
\textsanskrit{Māhissatī} for the Avantis;\footnote{\textit{Mahesaya} of the Pali editions is a variant spelling for \textit{\textsanskrit{māhissatī}}; \textsanskrit{Mahāvastu} has \textit{\textsanskrit{māhiṣmatī}}. It is in present-day Madhya Pradesh, on the banks of Narmada River, and is perhaps to be identified with modern Maheshwar. } \\
Roruka for the \textsanskrit{Sovīras};\footnote{\textsanskrit{Sovīra} was on the lower Indus, and Roruka is identified with modern Rohri in Sindh province, Pakistan. It is implausible that they received Aryan culture from \textsanskrit{Kāsi}. } 

\textsanskrit{Mithilā}\marginnote{36.7} for the Videhas;\footnote{\textsanskrit{Mithilā} was the capital of Videha, to the north east of the Vajjian federation, nestled against the Himalayas. \textsanskrit{Mithilā} was a dominant kingdom before the Buddha, its king Janaka featuring prominently in early \textsanskrit{Upaniṣads}. It features rarely in the suttas (\href{https://suttacentral.net/mn91/en/sujato}{MN 91}, \href{https://suttacentral.net/thig6.2/en/sujato}{Thig 6.2}, \href{https://suttacentral.net/thig13.4/en/sujato}{Thig 13.4}) and had apparently declined in importance. Śatapatha \textsanskrit{Brāhmaṇa} 1.4.1.10–19 depicts its origins in terms of the spread of fire-worship from the west by its founding king \textsanskrit{Māthava} Videgha with his priest Gotama \textsanskrit{Rāhūgaṇa}. } \\
\textsanskrit{Campā} was laid out for the \textsanskrit{Aṅgas};\footnote{\textsanskrit{Campā} is modern Champapuri near Bhagalpur in Bihar state, not far from West Bengal. \textsanskrit{Aṅga} lay between the Chandan river to the west and the Rajmahal hills to the east. In the centuries before the Buddha it expanded its dominion to the sea on the south-east and Magadha in the west. But with the \textsanskrit{Kaliṅgas} pushing back from the coast and the rise of \textsanskrit{Bimbisāra} in Magadha, they were pushed back to their ancestral lands and fell under Magadhan dominion. } \\
and Varanasi for the \textsanskrit{Kāsis}:\footnote{Varanasi, one of the oldest cities in the world, was the capital of the \textsanskrit{Kāsī} kingdom. It lost its status as an independent kingdom shortly before the Buddha, when it was taken over by Kosala. It appears in countless Buddhist stories of the past as the dominant city of the region in what appears to be a timeless and ageless past. However, despite its great antiquity, it is a historical settlement, of which the discovered remains date back to perhaps 1200 BCE; it was a capital city from perhaps 800 BCE. } \\
these were laid out by the Steward. 

%
\end{verse}

Then\marginnote{36.11} those six aristocrats were delighted with their respective gains, having achieved all they wished for, “We have received exactly what we wanted, what we wished for, what we desired, what we yearned for.” 

\begin{verse}%
\textsanskrit{Sattabhū}\marginnote{36.13} and Brahmadatta,\footnote{The names of the aristocrat kings—most of which are difficult to identify from other sources—are listed in an abruptly-inserted verse. The names are missing from the parallels in \textsanskrit{Mahāvastu} and DA 3, but T 8 appears to have the same list of names in the same order, except with \textsanskrit{Reṇu} at the start. Rhys Davids proposes to assign each one to a kingdom as listed in the same order. This seems to work for the first four names, as \href{https://suttacentral.net/ja424/en/sujato}{Ja 424}, invoking this sutta, identifies Bharata as king of \textsanskrit{Sovīra}. But it falls apart with \textsanskrit{Reṇu}, who as we have seen, belongs in \textsanskrit{Kāsi} rather than Videha. } \\
\textsanskrit{Vessabhū} and Bharata,\footnote{“Bharata” is the name of an individual king, while they are collectively known as “\textsanskrit{Bhāratas}” (with a long initial \textit{\textsanskrit{ā}} due to secondary derivation). \href{https://suttacentral.net/ja424/en/sujato}{Ja 424}, invoking this sutta, identifies Bharata as king of \textsanskrit{Sovīra}. } \\
\textsanskrit{Reṇu} and the two \textsanskrit{Dhataraṭṭhas}: \\
these were the seven \textsanskrit{Bhāratas}.\footnote{Here \textit{\textsanskrit{bhāratā}} means “kings of India” or perhaps, if read together with the preceding verses, “kingdoms of India”. India was called \textsanskrit{Bhārata} after the legendary King Bharata of \textsanskrit{Hastināpura} (Delhi), founder of the Lunar dynasty, conqueror of India, and sire of the warring tribes of the \textsanskrit{Mahābharata}. His story is told in the \textsanskrit{Saṁbhavapara} of the \textsanskrit{Mahābhārata}. The name represents the success of Vedic culture across the subcontinent. } 

%
\end{verse}

\scendsection{The first recitation section is finished. }

\subsection*{6.2. A Good Reputation }

Then\marginnote{37.1} the six aristocrats approached the Great Steward and said, “Steward, just as you are King \textsanskrit{Reṇu}’s friend, dear, beloved, and cherished, you are also our friend. Would you manage our affairs? Please don’t turn us down!” 

“Yes,\marginnote{37.5} sirs,” replied the Great Steward. Then the Great Steward managed the realms of the seven kings. And he taught seven well-to-do brahmins, and seven hundred bathed initiates to recite the hymns. 

After\marginnote{38.1} some time he got this good reputation, “The Great Steward sees the Divinity in person! The Great Steward discusses, converses, and consults with the Divinity in person!”\footnote{In spiritual circles, such rumors spread like wildfire in an Australian summer. } 

The\marginnote{38.3} Great Steward thought, “I have the reputation of seeing the Divinity in person, and discussing with him in person. But I don’t. I have heard that brahmins of the past who were elderly and senior, the tutors of tutors, said: ‘Whoever goes on retreat for the four months of the rainy season and practices the absorption on compassion sees the Divinity and discusses with him.’\footnote{The ancient Brahmanical teachers are invoked to justify the Buddhist rains retreat. | Compassion is one of the “meditations of \textsanskrit{Brahmā}” (\textit{\textsanskrit{brahmavihāra}}). Normally the suttas speak of the absorptions (\textit{\textsanskrit{jhāna}}) and of the meditation on compassion (\textit{\textsanskrit{karuṇā}}), but the idea of an “absorption on compassion” is unique to this sutta. } Why don’t I do that?” 

So\marginnote{39.1} the Great Steward went to King \textsanskrit{Reṇu} and told him of the situation, saying, “Sir, I wish to go on retreat for the four months of the rainy season and practice the absorption on compassion. No one should approach me, except for the one who brings my meal.” 

“Please\marginnote{39.9} do so, Steward, at your convenience.” 

Then\marginnote{40.1} the Great Steward went to the six aristocrats to put the same proposal, and received the same reply. 

He\marginnote{41.1} also went to the seven well-to-do brahmins and seven hundred bathed initiates and put to them the same proposal, adding, “Sirs, recite the hymns in detail as you have learned and memorized them, and teach each other how to recite.” 

And\marginnote{41.9} they too said, “Please do so, Steward, at your convenience.” 

Then\marginnote{42.1} the Great Steward went to his forty equal wives to put the same proposal to them, and received the same reply. 

Then\marginnote{43.1} the Great Steward had a new ceremonial hall built to the east of his citadel, where he went on retreat for the four months of the rainy season and practiced the absorption on compassion.\footnote{Compare \href{https://suttacentral.net/mn51/en/sujato\#10.3}{MN 51:10.3}, where the same construction is a site for the sacrifice. } And no one approached him except the one who brought him meals. 

But\marginnote{43.3} then, when the four months had passed, the Great Steward became dissatisfied and anxious, “I have heard that brahmins of the past said that whoever goes on retreat for the four months of the rainy season and practices the absorption on compassion sees the Divinity and discusses with him. But I neither see the Divinity nor discuss with him.” 

\subsection*{6.3. A Discussion With the Divinity }

And\marginnote{44.1} then the divinity \textsanskrit{Sanaṅkumāra}, knowing the Great Steward’s train of thought, as easily as a strong person would extend or contract their arm, vanished from the realm of divinity and reappeared in the Great Steward’s presence. At that, the Great Steward became frightened, scared, his hair standing on end, as he had never seen such a sight before. So he addressed the divinity \textsanskrit{Sanaṅkumāra} in verse: 

\begin{verse}%
“Who\marginnote{44.4} might you be, sir, \\
so beautiful, glorious, majestic? \\
Not knowing, I ask—\\
how am I to know who you are?” 

“In\marginnote{44.8} the realm of divinity they know me \\
as ‘The Eternal Youth’. \\
All the gods know me thus, \\
and so you should know me, Steward.” 

“A\marginnote{44.12} Divinity deserves a seat and water, \\
foot-salve, and sweet cakes. \\
Sir, I ask you to please accept \\
these gifts of hospitality.” 

“I\marginnote{44.16} accept the gifts of hospitality \\
of which you speak. \\
I grant you the opportunity \\
to ask whatever you desire—\\
about welfare and benefit in this life, \\
or happiness in lives to come.” 

%
\end{verse}

Then\marginnote{45.1} the Great Steward thought, “the divinity \textsanskrit{Sanaṅkumāra} has granted me an opportunity. Should I ask him about what is beneficial for this life or lives to come?” 

Then\marginnote{45.4} he thought, “I’m skilled in what is beneficial for this life, and others even ask me about it. Why don’t I ask the Divinity about the benefit that specifically applies to lives to come?” So he addressed the divinity \textsanskrit{Sanaṅkumāra} in verse: 

\begin{verse}%
“I’m\marginnote{45.8} in doubt, so I ask the Divinity—\\>who is free of doubt—\\
about things one may learn from another. \\
Standing on what, training in what \\
may a mortal reach the deathless realm of divinity?”\footnote{“Deathless” from a Brahmanical perspective, but very much within the realm of \textit{\textsanskrit{saṁsāra}} from a Buddhist perspective. } 

“He\marginnote{45.12} among men, O brahmin, \\>has given up possessiveness,\footnote{\textit{Brahme} is the normal vocative for \textit{\textsanskrit{brahmā}}. However in verse it is sometimes used as vocative for \textit{\textsanskrit{brāhmaṇa}} (eg. \href{https://suttacentral.net/snp5.1/en/sujato\#7.3}{Snp 5.1:7.3}, \href{https://suttacentral.net/snp5.19/en/sujato\#3.4}{Snp 5.19:3.4}). } \\
at one, compassionate, \\
free of putrefaction, and refraining from sex. \\
Standing on that, training in that \\
a mortal may reach the deathless realm of divinity.” 

%
\end{verse}

“Sir,\marginnote{46.1} I understand what ‘giving up possessiveness’ means.\footnote{These factors are explained as a summary of the Gradual Training. } It’s when someone gives up a large or small fortune, and a large or small family circle. They shave off hair and beard, dress in ocher robes, and go forth from the lay life to homelessness. That’s how I understand ‘giving up possessiveness’. 

Sir,\marginnote{46.4} I understand what ‘at one’ means.\footnote{“At one” (\textit{\textsanskrit{ekodibhūta}} or more commonly \textit{\textsanskrit{ekodibhāva}}) normally describes deep meditation, and is part of the formula for the second \textit{\textsanskrit{jhāna}}. Here it is explained as equivalent to “seclusion” (\textit{viveka}). } It’s when someone frequents a secluded lodging—a wilderness, the root of a tree, a hill, a ravine, a mountain cave, a charnel ground, a forest, the open air, a heap of straw. That’s how I understand ‘at one’. 

Sir,\marginnote{46.7} I understand what ‘compassionate’ means. It’s when someone meditates spreading a heart full of compassion to one direction, and to the second, and to the third, and to the fourth. In the same way above, below, across, everywhere, all around, they spread a heart full of compassion to the whole world—abundant, expansive, limitless, free of enmity and ill will. That’s how I understand ‘compassionate’. 

But\marginnote{46.10} I don’t understand what you say about putrefaction.\footnote{“Putrefaction” (\textit{\textsanskrit{āmagandha}}) is the smell of (moral) decay or corruption (\href{https://suttacentral.net/an3.128/en/sujato}{AN 3.128}, \href{https://suttacentral.net/snp2.2/en/sujato}{Snp 2.2}). } 

\begin{verse}%
What\marginnote{46.11} among men, O the Divinity, is putrefaction? \\
I don’t understand, so tell me, attentive one: \\
wrapped in what do people stink, \\
headed for hell, shut out of the realm of divinity?” 

“Anger,\marginnote{46.15} lies, fraud, and deceit, \\
miserliness, vanity, jealousy, \\
desire, stinginess, harassing others,\footnote{“Desire” is (\textit{\textsanskrit{icchā}}). PTS reads \textit{\textsanskrit{vicikicchā}} (“doubt”), but this must be incorrect as the commentary explains it as “craving” (\textit{\textsanskrit{taṇhā}}). } \\
greed, hate, vanity, and delusion—\\
those bound to such things \\>are not devoid of putrefaction; \\
they’re headed for hell, \\>shut out of the realm of divinity.” 

%
\end{verse}

“As\marginnote{46.21} I understand what you say about putrefaction, it’s not easy to quell while living at home.\footnote{“Easy to quell” (\textit{sunimmadaya}) is unique to this passage. } I shall go forth from the lay life to homelessness!” 

“Please\marginnote{46.23} do so, Steward, at your convenience.” 

\subsection*{6.4. Informing King \textsanskrit{Reṇu} }

So\marginnote{47.1} the Great Steward went to King \textsanskrit{Reṇu} and said, “Sir, please now find another high priest to manage the affairs of state for you. I wish to go forth from the lay life to homelessness. As I understand what the Divinity says about putrefaction, it’s not easy to quell while living at home. I shall go forth from the lay life to homelessness. 

\begin{verse}%
I\marginnote{47.6} announce to King \textsanskrit{Reṇu}, \\
the lord of the land: \\
you must learn how to rule, \\
for I no longer care for my ministry.” 

“If\marginnote{47.10} you’re lacking any pleasures, \\
I’ll supply them for you. \\
I’ll protect you from any harm, \\
for I command the nation’s army. \\
You are my father, I am your son!\footnote{As noted in the comment to \href{https://suttacentral.net/dn16/en/sujato\#5.19.2}{DN 16:5.19.2}, kingly clans adopted the lineage name of the high priest (\textit{purohita}) during initiation. } \\
O Steward, please don’t leave!” 

“I’m\marginnote{47.16} lacking no pleasures, \\
and no-one is harming me. \\
I’ve heard a non-human voice, \\
so I no longer care for lay life.” 

“What\marginnote{47.20} was that non-human like? \\
What did he say to you, \\
hearing which you would abandon \\
our house and all our people?” 

“Before\marginnote{47.24} entering this retreat, \\
I only liked to sacrifice. \\
I kindled the sacred flame, \\
strewn about with kusa grass. 

But\marginnote{47.28} then the Divinity the Eternal Youth \\
appeared to me from the realm of divinity. \\
He answered my question, \\
hearing which I no longer care for lay life.” 

“I\marginnote{47.32} have faith, O Steward, \\
in that of which you speak. \\
Having heard a non-human voice, \\
what else could you do? 

We\marginnote{47.36} will follow your example, \\
Steward, be my Teacher! \\
Like a gem of beryl—\\
flawless, immaculate, beautiful—\\
that’s how pure we shall live, \\
in the Steward’s dispensation. 

%
\end{verse}

If\marginnote{47.42} the Steward is going forth from the lay life to homelessness, we shall do so too. Your destiny shall be ours.” 

\subsection*{6.5. Informing the Six Aristocrats }

Then\marginnote{48.1} the Great Steward went to the six aristocrats and said, “Good sirs, please now find another high priest to manage the affairs of state for you. I wish to go forth from the lay life to homelessness. As I understand what the Divinity says about putrefaction, it’s not easy to quell while living at home. I shall go forth from the lay life to homelessness!” 

Then\marginnote{49.1} the six aristocrats withdrew to one side and thought up a plan, “These brahmins are greedy for wealth. Why don’t we try to persuade him with wealth?” 

They\marginnote{49.4} returned to the Great Steward and said, “In these seven kingdoms there is abundant wealth. We’ll get you as much as you want.” 

“Enough,\marginnote{49.6} sirs. I already have abundant wealth, owing to my lords. Giving up all that, I shall go forth.” 

Then\marginnote{49.10} the six aristocrats withdrew to one side and thought up a plan, “These brahmins are greedy for women. Why don’t we try to persuade him with women?” 

They\marginnote{49.13} returned to the Great Steward and said, “In these seven kingdoms there are many women. We’ll get you as many as you want.” 

“Enough,\marginnote{49.15} sirs. I already have forty equal wives. Giving up all them, I shall go forth.” 

“If\marginnote{50.1} the Steward is going forth from the lay life to homelessness, we shall do so too. Your destiny shall be ours.” 

\begin{verse}%
“If\marginnote{50.2} you all give up sensual pleasures, \\
to which ordinary people are attached, \\
exert yourselves, being strong, \\
and possessing the power of patience. 

This\marginnote{50.6} path is the straight path, \\
this path is supreme. \\
Guarded by the good, the true teaching\footnote{“True teaching” (\textit{saddhamma}) normally describes the Buddha’s teaching. } \\
leads to rebirth in the realm of divinity.”\footnote{“Leads to rebirth in the \textsanskrit{Brahmā} realm” (\textit{\textsanskrit{brahmalokūpapattiyā}}), just as the teaching of \textsanskrit{Āḷāra} \textsanskrit{Kālāma}, based on the even more refined formless meditations, leads to rebirth in the dimension of nothingness (\href{https://suttacentral.net/mn36/en/sujato\#14.14}{MN 36:14.14}). } 

%
\end{verse}

“Well\marginnote{51.1} then, sir, please wait for seven years. When seven years have passed, we shall go forth with you. Your destiny shall be ours.” 

“Seven\marginnote{51.3} years is too long, sirs. I cannot wait that long. Who knows what will happen to the living? We are heading to the next life. We must be thoughtful and wake up! We must do what’s good and lead the spiritual life, for no-one born can escape death.\footnote{Echoing the ancient hermit Araka at \href{https://suttacentral.net/an7.74/en/sujato\#2.2}{AN 7.74:2.2}, who said life is evanescent like a dewdrop. } I shall go forth.” 

“Well\marginnote{52.1} then, sir, please wait for six years, five years, four years, three years, two years, one year, seven months, six months, five months, four months, three months, two months, one month, or even a fortnight. When a fortnight has passed, we shall go forth. Your destiny shall be ours.” 

“A\marginnote{55.1} fortnight is too long, sirs. I cannot wait that long. Who knows what will happen to the living? We are heading to the next life. We must be thoughtful and wake up! We must do what’s good and lead the spiritual life, for no-one born can escape death. As I understand what the Divinity says about putrefaction, it’s not easy to quell while living at home. I shall go forth from the lay life to homelessness.” 

“Well\marginnote{55.6} then, sir, please wait for a week, so that we can instruct our sons and brothers in kingship. When a week has passed, we shall go forth. Your destiny shall be ours.” 

“A\marginnote{55.7} week is not too long, sirs. I will wait that long.” 

\subsection*{6.6. Informing the Brahmins }

Then\marginnote{56.1} the Great Steward also went to the seven well-to-do brahmins and seven hundred bathed initiates and said, “Good sirs, please now find another tutor to teach you to recite the hymns. I wish to go forth from the lay life to homelessness. As I understand what the Divinity says about putrefaction, it’s not easy to quell while living at home. I shall go forth from the lay life to homelessness.” 

“Please\marginnote{56.6} don’t go forth from the lay life to homelessness! The life of one gone forth is of little influence or profit, whereas the life of a brahmin is of great influence and profit.” 

“Please,\marginnote{56.9} good sirs, don’t say that. Who has greater influence and profit than myself? For now I am like a king to kings, like the Divinity to brahmins, like a deity to householders. Giving up all that, I shall go forth. As I understand what the Divinity says about putrefaction, it’s not easy to quell while living at home. I shall go forth from the lay life to homelessness.” 

“If\marginnote{56.15} the Steward is going forth from the lay life to homelessness, we shall do so too. Your destiny shall be ours.” 

\subsection*{6.7. Informing the Wives }

Then\marginnote{57.1} the Great Steward went to his forty equal wives and said, “Ladies, please do whatever you wish, whether returning to your own families, or finding another husband. I wish to go forth from the lay life to homelessness. As I understand what the Divinity says about putrefaction, it’s not easy to quell while living at home. I shall go forth from the lay life to homelessness.” 

“You\marginnote{57.6} are the only family we want! You are the only husband we want! If you are going forth from the lay life to homelessness, we shall do so too. Your destiny shall be ours.”\footnote{It is regarded as normal that woman should renounce. } 

\subsection*{6.8. The Great Steward Goes Forth }

When\marginnote{58.1} a week had passed, the Great Steward shaved off his hair and beard, dressed in ocher robes, and went forth from the lay life to homelessness. And when he had gone forth, the seven anointed aristocrat kings, the seven brahmins with seven hundred initiates, the forty equal wives, and many thousands of aristocrats, brahmins, householders, and many harem women shaved off their hair and beards, dressed in ocher robes, and went forth from the lay life to homelessness.\footnote{Compare with the mass renunciation under \textsanskrit{Vipassī} (\href{https://suttacentral.net/dn14/en/sujato\#2.16.6}{DN 14:2.16.6}). } 

Escorted\marginnote{58.3} by that assembly, the Great Steward wandered on tour among the villages, towns, and capital cities. And at that time, whenever he arrived at a village or town, he was like a king to kings, like the Divinity to brahmins, like a deity to householders. And whenever people sneezed or tripped over they’d say: “Homage to the Great Steward! Homage to the high priest for the seven!” 

And\marginnote{59.1} the Great Steward meditated spreading a heart full of love to one direction, and to the second, and to the third, and to the fourth. In the same way above, below, across, everywhere, all around, he spread a heart full of love to the whole world—abundant, expansive, limitless, free of enmity and ill will.\footnote{His meditation expands from compassion to include all four of the \textit{\textsanskrit{brahmavihāras}}. } He meditated spreading a heart full of compassion … rejoicing … equanimity to one direction, and to the second, and to the third, and to the fourth. In the same way above, below, across, everywhere, all around, he spread a heart full of equanimity to the whole world—abundant, expansive, limitless, free of enmity and ill will. And he taught his disciples the path to rebirth in the company of Divinity. 

Those\marginnote{60.1} of his disciples who completely understood the Great Steward’s instructions, at the breaking up of the body, after death, were reborn in the realm of divinity. Of those disciples who only partly understood the Great Steward’s instructions, some were reborn in the company of the gods who control what is imagined by others, while some were reborn in the company of the gods who love to imagine, or the joyful gods, or the gods of Yama, or the gods of the thirty-three, or the gods of the four great kings. And at the very least they swelled the hosts of the centaurs. 

And\marginnote{60.10} so the going forth of all those gentlemen was not in vain, was not wasted, but was fruitful and fertile.’ 

Do\marginnote{60.11} you remember this, Blessed One?” 

“I\marginnote{61.1} remember, \textsanskrit{Pañcasikha}. I myself was the brahmin Great Steward at that time.\footnote{This is the third and final \textsanskrit{Jātaka} in the \textsanskrit{Dīghanikāya}, after \href{https://suttacentral.net/dn5/en/sujato}{DN 5} and \href{https://suttacentral.net/dn17/en/sujato}{DN 17}. } And I taught those disciples the path to rebirth in the company of Divinity. But that spiritual path of mine doesn’t lead to disillusionment, dispassion, cessation, peace, insight, awakening, and extinguishment. It only leads as far as rebirth in the realm of divinity.\footnote{According to the doctrine of the “perfections” (\textit{\textsanskrit{pāramī}}), which emerged around two to four centuries after the Buddha’s passing, the practices he undertook in past lives laid the foundation for awakening in this life. Here, however, the Buddha states that his former practices did not lead to awakening. Rather, since they were based on the wrong view of eternal bliss in the \textsanskrit{Brahmā} realm, they only led to a good rebirth so long as that kamma lasted. It is the eightfold path, which the Buddha discovered in his final life, which leads to awakening. The same saying in a similar context is found at \href{https://suttacentral.net/mn83/en/sujato\#21.7}{MN 83:21.7}. } 

But\marginnote{61.5} this spiritual path does lead to disillusionment, dispassion, cessation, peace, insight, awakening, and extinguishment. And what is the spiritual path that leads to extinguishment? It is simply this noble eightfold path, that is: right view, right thought, right speech, right action, right livelihood, right effort, right mindfulness, and right immersion. This is the spiritual path that leads to disillusionment, dispassion, cessation, peace, insight, awakening, and extinguishment. 

Those\marginnote{62.1} of my disciples who completely understand my instructions realize the undefiled freedom of heart and freedom by wisdom in this very life. And they live having realized it with their own insight due to the ending of defilements. 

Of\marginnote{62.2} those disciples who only partly understand my instructions, some, with the ending of the five lower fetters, become reborn spontaneously. They are extinguished there, and are not liable to return from that world. 

Some,\marginnote{62.3} with the ending of three fetters, and the weakening of greed, hate, and delusion, become once-returners. They come back to this world once only, then make an end of suffering. 

And\marginnote{62.4} some, with the ending of three fetters, become stream-enterers, not liable to be reborn in the underworld, bound for awakening. 

And\marginnote{62.5} so the going forth of all those gentlemen was not in vain, was not wasted, but was fruitful and fertile.” 

That\marginnote{62.6} is what the Buddha said. Delighted, the centaur \textsanskrit{Pañcasikha} approved and agreed with what the Buddha said. He bowed and respectfully circled the Buddha, keeping him on his right, before vanishing right there. 

%
\chapter*{{\suttatitleacronym DN 20}{\suttatitletranslation The Great Congregation }{\suttatitleroot Mahāsamayasutta}}
\addcontentsline{toc}{chapter}{\tocacronym{DN 20} \toctranslation{The Great Congregation } \tocroot{Mahāsamayasutta}}
\markboth{The Great Congregation }{Mahāsamayasutta}
\extramarks{DN 20}{DN 20}

\scevam{So\marginnote{1.1} I have heard. }At one time the Buddha was staying in the land of the Sakyans, in the Great Wood near Kapilavatthu, together with a large \textsanskrit{Saṅgha} of five hundred mendicants, all of whom were perfected ones. And most of the deities from ten solar systems had gathered to see the Buddha and the \textsanskrit{Saṅgha} of mendicants.\footnote{This discourse gives an extensive account of divine beings. Some are familiar from elsewhere in the Pali texts, others found in various Brahmanical sources, while still others occur only here. For the names we can compare the edition of the Sanskrit text by Waldschmidt and Sander and available on SuttaCentral as \href{https://suttacentral.net/sf140/san/waldschmidt-sander}{SF 140}. } 

Then\marginnote{2.1} four deities of the Pure Abodes, aware of what was happening, thought:\footnote{The Pure Abodes are inhabited entirely by non-returners. } “Why don’t we go to the Buddha and each recite a verse in his presence?” 

Then,\marginnote{3.1} as easily as a strong person would extend or contract their arm, they vanished from the Pure Abodes and reappeared in front of the Buddha. They bowed to the Buddha and stood to one side. Standing to one side, one deity recited this verse in the Buddha’s presence: 

\begin{verse}%
“There’s\marginnote{3.4} a great congregation in the woods,\footnote{“Great congregation” is \textit{\textsanskrit{mahāsamaya}}. } \\
where heavenly hosts have assembled. \\
We’ve come to this righteous congregation \\
to see the invincible \textsanskrit{Saṅgha}!” 

%
\end{verse}

Then\marginnote{3.8} another deity recited this verse in the Buddha’s presence: 

\begin{verse}%
“The\marginnote{3.9} mendicants there are immersed in \textsanskrit{samādhi}, \\
they’ve straightened their own minds. \\
Like a charioteer holding the reins, \\
the astute ones protect their senses.” 

%
\end{verse}

Then\marginnote{3.13} another deity recited this verse in the Buddha’s presence: 

\begin{verse}%
“They\marginnote{3.14} snapped the post and snapped the cross-bar, \\
unstirred, they tore out the boundary post.\footnote{A village (\href{https://suttacentral.net/pli-tv-bu-vb-pj2/en/sujato\#3.8}{Bu Pj 2:3.8}) or royal compound (\href{https://suttacentral.net/pli-tv-bu-vb-pc83/en/brahmali\#1.3.56.1}{Bu Pc 83:1.3.56.1}) was marked with a “boundary post” (\textit{\textsanskrit{indakhīla}}). It symbolized a fixed and immovable point, either in a good sense (\href{https://suttacentral.net/sn56.39/en/sujato\#4.1}{SN 56.39:4.1}) or, as here, an obstacle. | For \textsanskrit{Mahāsaṅgīti}’s \textit{\textsanskrit{ūhacca} \textsanskrit{manejā}} read \textit{\textsanskrit{ūhacca}-m-\textsanskrit{anejā}}. } \\
They live pure and immaculate, \\
the young giants tamed by the Clear-eyed One.”\footnote{\textit{\textsanskrit{Nāga}} can refer to a class of semi-divine beings in a powerful serpentine form (“dragon”); a large and powerful snake, especially a king cobra; a bull elephant; or any powerful and mighty being (“giant”). } 

%
\end{verse}

Then\marginnote{3.18} another deity recited this verse in the Buddha’s presence: 

\begin{verse}%
“Anyone\marginnote{3.19} who has gone to the Buddha for refuge \\
won’t go to a plane of loss. \\
After giving up this human body, \\
they swell the hosts of gods.”\footnote{The discourse up to here is also found at \href{https://suttacentral.net/sn1.37/en/sujato}{SN 1.37}. } 

%
\end{verse}

\section*{1. The Gathering of Deities }

Then\marginnote{4.1} the Buddha said to the mendicants: 

“Mendicants,\marginnote{4.2} most of the deities from ten solar systems have gathered to see the Realized One and the mendicant \textsanskrit{Saṅgha}. The Buddhas of the past had, and the Buddhas of the future will have, gatherings of deities that are at most like the gathering for me now.\footnote{This recalls \href{https://suttacentral.net/dn14/en/sujato\#1.10.1}{DN 14:1.10.1}, where the relative sizes of the mendicant congregations of different Buddhas are tallied. That sutta, in another point of similarity, also featured a conversation with deities of the Pure Abodes. } I shall declare the names of the heavenly hosts; I shall extol the names of the heavenly hosts; I shall teach the names of the heavenly hosts. Listen and apply your mind well, I will speak.” 

“Yes,\marginnote{4.9} sir,” they replied. 

The\marginnote{4.10} Buddha said this: 

\begin{verse}%
“I\marginnote{5.1} invoke a paean of praise!\footnote{“Paean of praise” is \textit{siloka}, a rare case where this means “verse (of praise)” rather than “fame, renown”. | \textit{\textsanskrit{Anukassāmi}} is present tense from the root \textit{kass}, “to draw (up), to drag”. Compare Sanskrit \textit{\textsanskrit{anukarṣa}} in the sense “invoking, summoning by incantation”. } \\
Where the earth-gods dwell, \\
there, in mountain caves, \\
resolute and composed, 

dwell\marginnote{5.5} many like lonely lions, \\
who have mastered their fears.\footnote{This gives a hint as to one of the purposes of this sutta. Living alone in remote forests can be terrifying. Such places were widely believed to be haunted by all manner of supernatural creatures, not all of them friendly. While for an arahant this posed no threat, there would have been then, as there are today, many young or aspiring meditators who faced such challenges with trepidation. These verses offer succor, surrounding them with powerful, albeit invisible, allies. } \\
Their minds are bright and pure, \\
clear and undisturbed.” 

The\marginnote{5.9} teacher knew that over five hundred \\
were in the wood at Kapilavatthu. \\
Therefore he addressed \\
the disciples who love the teaching: 

“The\marginnote{5.13} heavenly hosts have come forth; \\
mendicants, you should be aware of them.” \\
Those monks grew keen, \\
hearing the Buddha’s instruction. 

Knowledge\marginnote{6.1} manifested in them, \\
seeing those non-human beings. \\
Some saw a hundred, \\
a thousand, even seventy thousand, 

while\marginnote{6.5} some saw a hundred thousand \\
non-human beings. \\
But some saw an endless number \\
spread out in every direction. 

And\marginnote{6.9} all that was known \\
and distinguished by the Clear-eyed One. \\
Therefore he addressed \\
the disciples who love the teaching: 

“The\marginnote{6.13} heavenly hosts have come forth; \\
mendicants, you should be aware of them. \\
I shall extol them for you, \\
with lyrics in proper order. 

There\marginnote{7.1} are seven thousand spirits,\footnote{“Spirit” is \textit{yakkha}. In later legend they appear as monstrous figures, but in the suttas they are ambiguous and may often be friendly to the Dhamma. Early statues of \textit{yakkhas} at \textsanskrit{Madhurā} (circa 100 BCE) depict noble and powerful kings, not ogres. \textit{Yakkha} is also sometimes used more generally in the sense of an individual or deity. } \\
earth-gods of Kapilavatthu. \\
They’re powerful and brilliant, \\
so beautiful and glorious. \\
Rejoicing, they’ve come forth \\
to the meeting of mendicants in the wood. 

From\marginnote{7.7} the Himalayas there are six thousand \\
spirits of different colors.\footnote{Colorful like the nymphs of \href{https://suttacentral.net/mn50/en/sujato\#25.6}{MN 50:25.6} or \textsanskrit{Susīma}’s gods at \href{https://suttacentral.net/sn2.29/en/sujato\#7.1}{SN 2.29:7.1}. } \\
They’re powerful and brilliant, \\
so beautiful and glorious. \\
Rejoicing, they’ve come forth \\
to the meeting of mendicants in the wood. 

From\marginnote{7.13} Mount \textsanskrit{Sātā} there are three thousand\footnote{Mount \textsanskrit{Sātā} is unidentified, but according to the commentary it was in the middle region. } \\
spirits of different colors. \\
They’re powerful and brilliant, \\
so beautiful and glorious. \\
Rejoicing, they’ve come forth \\
to the meeting of mendicants in the wood. 

And\marginnote{7.19} thus there are sixteen thousand \\
spirits of different colors. \\
They’re powerful and brilliant, \\
so beautiful and glorious. \\
Rejoicing, they’ve come forth \\
to the meeting of mendicants in the wood. 

From\marginnote{8.1} \textsanskrit{Vessāmitta}’s mountain there are five hundred\footnote{\textsanskrit{Vessāmitta} (“friend of all”) is the name of a Vedic hermit. Legend has it that he was a \textit{khattiya} who earned brahminhood due to his intense austerities in the remote Himalayas. Perhaps this is a mountain named for him, although I can find no trace of it in Sanskrit. } \\
spirits of different colors. \\
They’re powerful and brilliant, \\
so beautiful and glorious. \\
Rejoicing, they’ve come forth \\
to the meeting of mendicants in the wood. 

And\marginnote{8.7} there’s \textsanskrit{Kumbhīra} of \textsanskrit{Rājagaha},\footnote{\textit{\textsanskrit{Kumbhīra}} means “crocodile”. } \\
whose home is on Mount Vepulla. \\
Attended by more than \\
a hundred thousand spirits, \\
\textsanskrit{Kumbhīra} of \textsanskrit{Rājagaha} \\
also came to the meeting in the wood. 

King\marginnote{9.1} \textsanskrit{Dhataraṭṭha} rules\footnote{We met two human King \textsanskrit{Dhataraṭṭhas} (“Strongrealm”) in the \textsanskrit{Mahāgovindasutta}. } \\
the eastern quarter. \\
Lord of the centaurs,\footnote{“Centaur” is \textit{gandhabba}. \textit{Gandhabbas} are wild, sexual beings who, being first to yoke the horse (Śatapatha \textsanskrit{Brāhmaṇa} 5.1.4.8) and riding the steed called “Racer” (\textit{\textsanskrit{vājin}}, 10.6.4.1), “take the reins” from Indra (Rig Veda 1.163.2) as his charioteer \textsanskrit{Mātali}. It is likely the Sanskrit \textit{gandharva} stems from the same root as the Greek \textit{kentauro}, and the ultimate origin of the idea comes from the proto-Indo-Europeans, who rode the horse, creating a potent entity that was invincible in battle. } \\
he’s a great king, glorious. 

And\marginnote{9.5} he has many mighty sons \\
all of them named Indra.\footnote{By implication, they are renowned as being powerful as Indra, the king of gods who in Pali is more commonly called Sakka. It is also a nod to the multiplicity of divinities in the Vedic system, where one becomes many and many become one. } \\
They’re powerful and brilliant, \\
so beautiful and glorious. \\
Rejoicing, they’ve come forth \\
to the meeting of mendicants in the wood. 

King\marginnote{9.11} \textsanskrit{Virūḷhaka} rules\footnote{\textit{\textsanskrit{Virūḷhaka}} is “growth”; he was probably a god of the fertile crops. In the Pali it is spelled \textsanskrit{Virūḷha} here due to the meter. } \\
the southern quarter. \\
Lord of the goblins,\footnote{“Goblins” is \textit{\textsanskrit{kumbhaṇḍa}}, a race of lesser deities often depicted as ugly and misshapen. On the face of it, \textit{\textsanskrit{kumbhaṇḍa}} means “potballs”, i.e. deities whose testicles are as big as pots. Perhaps more likely they are related to the beings known as \textit{\textsanskrit{kuṣmāṇḍa}} in Sanskrit, whose bellies are round as “pumpkins”. } \\
he’s a great king, glorious. 

And\marginnote{9.15} he has many mighty sons \\
all of them named Indra. \\
They’re powerful and brilliant, \\
so beautiful and glorious. \\
Rejoicing, they’ve come forth \\
to the meeting of mendicants in the wood. 

King\marginnote{9.21} \textsanskrit{Virūpakkha} rules\footnote{One of the “royal snake families” at \href{https://suttacentral.net/an4.67/en/sujato\#3.2}{AN 4.67:3.2}. His name means “multiple eyes” (\textit{\textsanskrit{virūpa}-akkha}) for the false eyes on a cobra’s hood. } \\
the western quarter. \\
Lord of the dragons, \\
he’s a great king, glorious. 

And\marginnote{9.25} he has many mighty sons \\
all of them named Indra. \\
They’re powerful and brilliant, \\
so beautiful and glorious. \\
Rejoicing, they’ve come forth \\
to the meeting of mendicants in the wood. 

King\marginnote{9.31} Kuvera rules\footnote{Also known as \textsanskrit{Vessavaṇa} (\href{https://suttacentral.net/dn32/en/sujato\#7.43}{DN 32:7.43}, \href{https://suttacentral.net/snp2.14/en/sujato\#6.1}{Snp 2.14:6.1}), which means “Son of the Renowned” from his father \textsanskrit{Viśrava}. These two names appear together elsewhere in contemporary literature (Atharvaveda 8,10.28c, Śatapatha \textsanskrit{Brāhmaṇa} 13.4.3.10), where, although the passages are obscure, he appears to be associated with wickedness, concealment, and theft. Although a god of wealth (\textit{dhanada}), the name Kuvera is explained by lexicographers as “deformed”. Probably he was originally a god of the underground, blessed with the earth’s riches, yet deformed by its great pressure. } \\
the northern quarter. \\
Lord of spirits, \\
he’s a great king, glorious. 

And\marginnote{9.35} he has many mighty sons \\
all of them named Indra. \\
They’re powerful and brilliant, \\
so beautiful and glorious. \\
Rejoicing, they’ve come forth \\
to the meeting of mendicants in the wood. 

\textsanskrit{Dhataraṭṭha}\marginnote{9.41} in the east, \\
\textsanskrit{Virūḷhaka} to the south, \\
\textsanskrit{Virūpakkha} to the west, \\
and Kuvera in the north. 

These\marginnote{9.45} four great kings,\footnote{The following verses are a treasure-trove of ancient Indian mythology, recording the names of deities otherwise lost to history. Many of the names are obscure and variant readings are recorded in manuscripts and noted in the commentary. My spellings follow Ānandajoti’s translation, \emph{\href{https://suttacentral.nethttp://www.ancient-buddhist-texts.net/Texts-and-Translations/Safeguard/02x-Atirekani-Sattasuttani-23.htm}{The Discourse on the Great Convention}}, unless there is a reason to change. I try to identify the deities as best as I can, but many of them remain speculative. } \\
all around in the four quarters, \\
stood there dazzling \\
in the wood at Kapilavatthu. 

Their\marginnote{10.1} deceitful heathens came,\footnote{These deities are called \textit{\textsanskrit{dāsa}}, which normally means “slave” in Pali. But here, it seems, we have a singular instance of the old Vedic meaning, an uncivilized foe. The \textit{\textsanskrit{dāsas}} \textsanskrit{Vṛtra}, Namuci, and \textsanskrit{Vṛṣaśipra} are described as “deceivers” (\textit{\textsanskrit{māyin}}, Rig Veda 1.53.7; 2.11.10; 7.99.4; 10.73.7). This epithet is also used of the closely related \textit{dasyu} (Rig Veda 1.33.10; 4.16.9; 8.14.14; 10.73.5; see note on \href{https://suttacentral.net/dn5/en/sujato\#11.5}{DN 5:11.5}). Such beings are said to be “godless” since they oppose the Vedic deities (Rig Veda 2.19.7; 3.31.19; 7.1.10; 10.11.6; 10.138.4, etc.), so we can translate as “heathen”. } \\
so treacherous and crafty—\\
the deceivers \textsanskrit{Kuṭeṇḍu}, \textsanskrit{Viṭeṇḍu},\footnote{I am not able to identify any of these names with confidence, but perhaps \textit{\textsanskrit{kuṭeṇḍu}} and \textit{\textsanskrit{viṭeṇḍu}} could be traced to \textit{indu} (“moon”), thus “crooked moon” and “defective moon”. Compare \textit{\textsanskrit{khaṇḍendu}}, a later term for Śiva as the crescent moon. } \\
with \textsanskrit{Viṭucca} and \textsanskrit{Viṭuṭa}.\footnote{The Sanskrit names are \textit{\textsanskrit{kiṭi}}, \textit{\textsanskrit{vikiṭi}}, \textit{\textsanskrit{bhṛgu}}, and \textit{\textsanskrit{bhṛkuṭi}}. Only \textit{\textsanskrit{bhṛgu}} is attested in the Vedas, but he was a sage, not a “crafty heathen”. } 

And\marginnote{10.5} Candana and \textsanskrit{Kāmaseṭṭha},\footnote{Candana appears in \href{https://suttacentral.net/mn134/en/sujato\#6.4}{MN 134:6.4} and \href{https://suttacentral.net/sn40.11/en/sujato\#1.1}{SN 40.11:1.1}, and together with \textsanskrit{Kāmaseṭṭha} in \href{https://suttacentral.net/dn32/en/sujato\#10.4}{DN 32:10.4}. \textit{Candana} means “sandalwood”, although the name of the god might also relate to its root “shining”. | \textit{\textsanskrit{Kāmaseṭṭha}} means “Chief of Sex”, i.e. “Eros”; he does not seem to appear in a Brahmanical context. } \\
\textsanskrit{Kinnughaṇḍu} and \textsanskrit{Nighaṇḍu}, \\
\textsanskrit{Panāda} and \textsanskrit{Opamañña},\footnote{\textsanskrit{Panāda} (“roarer”) was the name of an ancient king who performed the horse sacrifice (\href{https://suttacentral.net/thag2.22/en/sujato\#1.1}{Thag 2.22:1.1}, \href{https://suttacentral.net/dn26/en/sujato\#26.1}{DN 26:26.1}). In \href{https://suttacentral.net/ja265/en/sujato}{Ja 265} he is said to have been the son of \textsanskrit{Vessavaṇa} (Kuvera). | \textsanskrit{Opamañña} is a descendant of the ascended sage Upamanyu (“zealous one”; see note on \href{https://suttacentral.net/mn99/en/sujato\#10.3}{MN 99:10.3}). } \\
and \textsanskrit{Mātali}, the god’s charioteer.\footnote{\textsanskrit{Mātali} is the charioteer of Sakka (i.e. Indra; \href{https://suttacentral.net/sn11.6/en/sujato\#1.5}{SN 11.6:1.5}, \href{https://suttacentral.net/mn83/en/sujato\#14.1}{MN 83:14.1}, etc.), a role he plays throughout Brahmanical literature as well. He appears here as one of the centaur lords, and is father to the centaur \textsanskrit{Sikhaṇḍī}, \textsanskrit{Pañcasikha}’s rival in love \href{https://suttacentral.net/dn21/en/sujato\#1.6.8}{DN 21:1.6.8}. } 

Cittasena\marginnote{10.9} the centaur came too,\footnote{The \textit{gandharva} Citrasena (“Brightspear”) appears in the Sanskrit \textsanskrit{Purāṇas} as a friend of Arjuna in various adventures. } \\
and the kings Nala and Janesabha.\footnote{\textit{Nala} means “reed”. There was a King Nala of Vidarbha whose love for his Queen \textsanskrit{Damayantī} is celebrated in the \textsanskrit{Mahābharata}. | Janesabha is an alternate spelling of Janavasabha (\href{https://suttacentral.net/dn18/en/sujato}{DN 18}). } \\
\textsanskrit{Pañcasikha} came too, with\footnote{In \href{https://suttacentral.net/dn21/en/sujato}{DN 21} we shall learn of how the centaur \textsanskrit{Pañcasikha} (“Fivecrest”) wooed Timbaru’s daughter \textsanskrit{Suriyavaccasā} (“Sunshine”). } \\
Timbaru and \textsanskrit{Suriyavaccasā}. 

These\marginnote{10.13} and other kings there were, \\
the centaurs with their kings. \\
Rejoicing, they’ve come forth \\
to the meeting of mendicants in the wood. 

Then\marginnote{11.1} came the dragons of \textsanskrit{Nābhasa} lake,\footnote{Read \textit{\textsanskrit{nābhasa}}, which means “celestial, heavenly”. The commentary says this was the name of a lake, which agrees with the legend that a \textsanskrit{Nābhasa} was a son of Nala (“reed”) and father of \textsanskrit{Puṇḍarīka} (“lotus”). The Sanskrit here, however, has \textit{\textsanskrit{sahabhuṁ} \textsanskrit{nāgo}}. } \\
and those from \textsanskrit{Vesālī} with those from \textsanskrit{Takkasilā}.\footnote{\textit{\textsanskrit{Vesālā}} is from the adjectival form \textit{\textsanskrit{vesāla}} (“of \textsanskrit{Vesālī}”). | Pali \textit{taccha} can represent Sanskrit \textit{\textsanskrit{takṣa}}, and the Sanskrit does indeed have \textit{\textsanskrit{takṣakaḥ}} here, so Tacchaka is probably “of \textsanskrit{Takṣasilā}”. This famous city, normally spelled \textsanskrit{Takkasilā} in Pali, is Taxila in West Pakistan, an ancient center of learning. } \\
The Kambalas and Assataras came\footnote{These \textit{\textsanskrit{nāgas}} are frequently mentioned together in Sanskrit literature such as the \textsanskrit{Mahābhārata} (1.31.1a, 2.9.9a, 5.101.9c), where they are also said to come from \textsanskrit{Pāyāga} (3.83.72a: \textit{\textsanskrit{prayāgaṁ} \textsanskrit{sapratiṣṭhānaṁ} \textsanskrit{kambalāśvatarau}}). } \\
and those from \textsanskrit{Payāga} with their kin.\footnote{\textsanskrit{Payāga} (modern Prayagraj, formerly Allahabad) is the sacred ford at the confluence of the Ganges and the \textsanskrit{Yamunā} beside \textsanskrit{Kosambī} (see \href{https://suttacentral.net/pli-tv-bu-vb-pj1/en/sujato\#4.18}{Bu Pj 1:4.18}). } 

Those\marginnote{11.5} from \textsanskrit{Yamunā}, and the \textsanskrit{Dhataraṭṭha}\footnote{The \textsanskrit{Dhataraṭṭha} and \textsanskrit{Erāvaṇa} \textit{\textsanskrit{nāgas}} also appear frequently in Sanskrit literature, and often in association with other deities in our text as part of a loose cluster of divinities (eg. \textsanskrit{Harivamśa} 3.112–117). \textsanskrit{Erāvaṇa} also appears at \href{https://suttacentral.net/snp2.14/en/sujato\#5.1}{Snp 2.14:5.1}. Here he is a \textit{\textsanskrit{nāga}} as in “dragon”, but later tradition saw him as a \textit{\textsanskrit{nāga}} as in “elephant”, in which role he became the mighty mount of Indra/Sakka. } \\
dragons came, so glorious. \\
And \textsanskrit{Erāvaṇa} the great dragon \\
also came to the meeting in the wood. 

Those\marginnote{11.9} who seize the dragon kings by force—\footnote{Here we see the eternal mythic struggle between birds and snakes, the creatures of the sky and the underworld. } \\
Heavenly, twice-born birds with piercing vision—\footnote{Birds are called “twice-born”, once from the mother, once from the egg. “Twice-born” is also an epithet of brahmins (\href{https://suttacentral.net/thig15.1/en/sujato\#31.3}{Thig 15.1:31.3}). } \\
swoop down to the wood from the sky; \\
their name is ‘Rainbow Phoenix’.\footnote{The \textit{\textsanskrit{supaṇṇā}} (Sanskrit \textit{\textsanskrit{suparṇa}}, “brightwing”) or \textit{\textsanskrit{garuḷā}} (Sanskrit \textit{\textsanskrit{garuḍā}}, “devourer”) is the great eagle, king of birds, and later the vehicle of \textsanskrit{Viṣṇu}. He has been falsely compared with the loathsome harpy; but his golden wings, closeness to the sun, and role as bearer of the nectar of immortality show that he is the Indian phoenix, the golden eagle of the sun. Unlike the western phoenix, he is not said to combust himself and be reborn anew. However this motif is merely the surface expression of the sun’s fiery immortality. As the brother of the Dawn (\textsanskrit{Aruṇa}), his golden wings at sunset “devour” the sun, and he disappears only to be reborn the next day. } 

But\marginnote{11.13} the dragon kings remained fearless, \\
for the Buddha kept them safe from the phoenixes. \\
Introducing each other with gentle words, \\
the dragons and phoenixes \\>took the Buddha as refuge.\footnote{The Buddha reconciles even such inveterate enemies. In the Buddhist view, all of these beings have their place in the great pattern of nature. } 

Defeated\marginnote{12.1} by Vajirahattha,\footnote{\textit{Vajirahattha} (“thunderbolt-in-hand”) is a synonym of \textit{\textsanskrit{vajirapāṇi}} (\href{https://suttacentral.net/dn3/en/sujato\#1.21.3}{DN 3:1.21.3}, \href{https://suttacentral.net/mn35/en/sujato\#14.1}{MN 35:14.1}). It is a frequent epithet of Sakka in the Vedas (eg. Rig Veda 1.173.10a \textit{indro vajrahastaḥ}), an identification confirmed by the commentary here. } \\
the titans live in the ocean. \\
They’re brothers of \textsanskrit{Vāsava},\footnote{\textsanskrit{Vāsava} is another epithet of Sakka (see below, \href{https://suttacentral.net/dn20/en/sujato\#14.9}{DN 20:14.9}), meaning “endowed with wealth” (\textit{vasu}; see for example Rig Veda 1.9.9a \textit{\textsanskrit{indraṁ} \textsanskrit{vasupatiṁ}} “Indra, lord of wealth”). The Buddhist explanation of his name \href{https://suttacentral.net/sn11.13/en/sujato\#9.1}{SN 11.13:9.1}, rather, plays on the word \textit{\textsanskrit{āvasatha}} and calls him the “giver of a guesthouse”. Later Brahmanical texts enumerated eight Vasus, with Indra as their lord, who are reckoned among the thirty-three. | Sakka's wife is the demon (\textit{asura}) princess \textsanskrit{Sujā}, daughter of Vepacitti, hence they are his “brothers”, technically in-laws (\href{https://suttacentral.net/sn11.12/en/sujato\#6.1}{SN 11.12:6.1}). } \\
powerful and glorious. 

There’s\marginnote{12.5} the terrifying \textsanskrit{Kālakañjas},\footnote{The \textsanskrit{Kālakañja} (“Blue-lotus”) is said to be the lowest class of demons (\href{https://suttacentral.net/dn24/en/sujato\#1.7.19}{DN 24:1.7.19}). } \\
the \textsanskrit{Dānava} and Ghasa titans,\footnote{The \textsanskrit{Dānavas} are a prominent group of \textit{asuras} otherwise unattested in early Pali. The name stems from “rivers, waters” after their mother \textsanskrit{Danū}. | Ghasa (“devourer”) is occasionally mentioned as the name of a \textit{\textsanskrit{rakṣasa}} (\textsanskrit{Rāmāyaṇa} 5.22.36a \textit{\textsanskrit{praghasā} \textsanskrit{nāma} \textsanskrit{rākṣasī}}). The commentary treats them as a single class of arrow-wielding \textit{asuras}. } \\
Vepacitti and Sucitti,\footnote{Vepacitti means “wise thinker”, Sucitti means “good thinker”. Vepacitti was the lord of the \textit{asuras} and Sakka’s counterpart. Their relationship was complicated (\href{https://suttacentral.net/sn11.23/en/sujato}{SN 11.23}). Brahmanical literature regards him as the eldest son of \textsanskrit{Danū} and hence chief of the \textsanskrit{Dānavas}. The Sanskrit form is \textit{vipracitti}, but in Buddhist texts it is incorrectly Sanskritized to \textit{vemacitra}. Vepacitti’s fame matches Sucitti’s obscurity, for he does not seem to appear elsewhere. } \\
\textsanskrit{Pahārāda} with Namuci,\footnote{\textsanskrit{Pahārāda} the ocean-loving \textit{asura} lord appears in \href{https://suttacentral.net/an8.19/en/sujato}{AN 8.19}. His name (Sanskrit \textit{\textsanskrit{prahlāda}}) means “mirth”. Later Brahmanical legends say he was a righteous son of the wicked \textsanskrit{Hiraṇyakaśipu}, who hated him for his devotion to \textsanskrit{Viṣṇu}. | The story goes that when Namuci stole the soma from Indra, Indra’s vengeance was frustrated by his vow not to harm him with anything wet or dry, in the dark or the light (Śatapatha \textsanskrit{Brāhmaṇa} 12.7.3.1). Indra’s ingenious solution was to dismember him with foam (Rig Veda 8.14.13) at dawn. Namuci’s name is explained as \textit{na-muci}, “not letting go” (the waters). In Buddhism, Namuci is a name of \textsanskrit{Māra}, although here \textsanskrit{Māra} appears separately below.. } 

and\marginnote{12.9} Bali’s hundred sons,\footnote{According to legend, Bali was the grandson of \textsanskrit{Pahārāda} and son of Virocana. His hundred sons were dread warriors (\textsanskrit{Bhāgavata} \textsanskrit{Purāṇa} 8.10.30). The word \textit{bali} also refers to a kind of sacrificial offering, but the roots of the term are obscure. } \\
all named after Virocana.\footnote{\textit{Virocana} was their grandfather. He was another adversary of Indra, despite which they were both said to have sought knowledge from \textsanskrit{Prajāpatī}, but Virocana misunderstood since he saw only the surface meaning (\textsanskrit{Chāndogya} \textsanskrit{Upaniṣad} 8.7.2–8.5). Here he is associated with \textsanskrit{Rāhu}, while \textsanskrit{Gaṇeśa} \textsanskrit{Purāṇa} 2.29 says he was granted a crown by the sun which he then lost. These details hint at a connection with the eclipse, in which case \textit{virocana} would be the rays that emanate during a solar eclipse. } \\
Bali’s army armed themselves \\
and went to the auspicious \textsanskrit{Rāhu}, saying:\footnote{\textsanskrit{Rāhu} is the mighty \textit{asura} who “seizes” the moon (\href{https://suttacentral.net/sn2.9/en/sujato}{SN 2.9}) or the sun (\href{https://suttacentral.net/sn2.10/en/sujato}{SN 2.10}) to create an eclipse. He is the son of Vepacitti. } \\
‘Now is the time, sir, \\
for the meeting of mendicants in the wood.’ 

The\marginnote{13.1} gods of water and earth, \\
and fire and wind came there.\footnote{These four are normally treated in the suttas as physical properties. Each of them is worshiped in the Rig Veda as a deity, and the Upanishads list them together along with other elements (eg. \textsanskrit{Bṛhadāraṇyaka} \textsanskrit{Upaniṣad} 4.4.5). I am not aware of any earlier contexts that group the four together systematically as deities in this way. } \\
The gods of \textsanskrit{Varuṇa} and \textsanskrit{Varuṇa}’s offsping,\footnote{\textsanskrit{Varuṇa} is invoked frequently in the Vedas, and in the suttas he is associated with other leading deities such as Indra and Soma (\href{https://suttacentral.net/dn13/en/sujato\#25.2}{DN 13:25.2}, \href{https://suttacentral.net/dn32/en/sujato\#10.2}{DN 32:10.2}, \href{https://suttacentral.net/sn11.3/en/sujato\#5.1}{SN 11.3:5.1}). As one of the twelve children of Aditi he stood for a calendar month, and grew a wide and sometimes baffling array of associations—the oceans, water, the sky (at night), and justice. Vedic \textsanskrit{Varuṇa} was the god of command, the king of tough rule. He was identified with the aristocrats, while his partner Mitra was the brahmins (Śatapatha \textsanskrit{Brāhmaṇa} 4.1.4). } \\
and Soma together with Yasa.\footnote{\textit{Soma} is the divine nectar that filled the ancient proto-Indo-Europeans with vitality. Its biological identity is disputed, but may have been ephedra. It is associated with the moon, hence the connection with \textsanskrit{Varuṇa} as the night sky. | \textit{Yasa} means “fame, glory”; so far as I know it is not personified in early Brahmanical texts, and rarely later. The connection between these two is illuminated by such passages as Rig Veda 7.85.3. To uphold Indra and \textsanskrit{Varuṇa} in battle, there is an offering of Soma, which is described as \textit{\textsanskrit{svayaśasaḥ}}, “Self-Glorious”, i.e. the state of being high on speed, if speed were a god. } 

A\marginnote{13.5} host of the gods of love\footnote{In the Vedas, Mitra (“friend”) is almost always paired with \textsanskrit{Varuṇa}, and the appearance of \textsanskrit{Mettā} (“friendliness”) here echoes that closeness. He was the god of alliances. | \textit{\textsanskrit{Karuṇā}} in the sense of “compassion” does not occur at all in pre-Buddhist Sanskrit. It is probably introduced here as companion to \textsanskrit{Mettā} by association with Mitra. The Sanskrit here, however, is \textit{\textsanskrit{maitrī} \textsanskrit{varuṇikā}}. From a Buddhist point of view, these are deities reborn due to the development of \textit{\textsanskrit{jhāna}} based on love and compassion. } \\
and compassion came, so glorious. \\
These ten hosts of gods \\
shone in all different colors. 

They’re\marginnote{13.9} powerful and brilliant, \\
so beautiful and glorious. \\
Rejoicing, they’ve come forth \\
to the meeting of mendicants in the wood. 

The\marginnote{14.1} Vishnu and \textsanskrit{Sahalī} gods,\footnote{\textsanskrit{Veṇhu} (variant \textit{\textsanskrit{veṇḍu}}) is the Pali spelling of Sanskrit \textit{\textsanskrit{viṣṇu}}, who appears only here and at \href{https://suttacentral.net/sn2.12/en/sujato}{SN 2.12}. In the Rig Veda he was a solar god who made three great strides (said to encompass the earth, the sky, and the heavens). | A deity named \textsanskrit{Sahalī} (“with plow”) appears at \href{https://suttacentral.net/sn2.30/en/sujato\#3.1}{SN 2.30:3.1} where he praises Makkhali \textsanskrit{Gosāla}, but I cannot trace him anywhere else. } \\
and the unequaled pair of twins.\footnote{\textit{Yama} often refers to a god who guards the paths to the land of the dead (\href{https://suttacentral.net/sn1.33/en/sujato\#10.3}{SN 1.33:10.3}, \href{https://suttacentral.net/mn130/en/sujato\#5.1}{MN 130:5.1}), lord of the \textsanskrit{Yāma} gods. Here, however, it refers to “twins” who in the Vedas are often identified with the \textsanskrit{Aśvins} (the twin horses yoked to the chariot). In the Vedas, it seems \textit{asama} (“unequalled”) is used as an epithet only, although the commentary says it is a name. | In Pali these lines lack a verb, while the Sanskrit supplies \textit{\textsanskrit{āgataś}} (“come”) instead of \textit{\textsanskrit{asamā}}, so perhaps we should read “the pair of Twins came”. } \\
The gods living on the moon came, \\
with the Moon before them. 

The\marginnote{14.5} gods living on the sun came, \\
with the Sun before them. \\
And with the stars before them \\
came the languid gods of clouds. 

And\marginnote{14.9} \textsanskrit{Vāsava} came, the greatest of the Vasus,\footnote{\textsanskrit{Vāsava} is above at \href{https://suttacentral.net/dn20/en/sujato\#12.3}{DN 20:12.3}. } \\
who is Sakka the Able, Purindada the Firstgiver:\footnote{\textit{Purandara} (“Fortbreaker”) is another epithet of Sakka or Indra (eg. Rig Veda 1.102.7). The Buddha reforms it to \textit{Purindada} (“Firstgiver”) at \href{https://suttacentral.net/sn11.12/en/sujato\#2.1}{SN 11.12:2.1}. But Indra’s generosity is long renowned, eg. Rig Veda 1.10.6c: “He is the able one, and he will be able for us—Indra who distributes the goods” (\textit{sa \textsanskrit{śakra} uta naḥ \textsanskrit{śakad} indro vasu \textsanskrit{dayamānaḥ}}). } \\
These ten hosts of gods \\
shone in all different colors. 

They’re\marginnote{14.13} powerful and brilliant, \\
so beautiful and glorious. \\
Rejoicing, they’ve come forth \\
to the meeting of mendicants in the wood. 

Then\marginnote{15.1} came the \textsanskrit{Sahabhū} gods,\footnote{Sanskrit has \textit{\textsanskrit{sabhikā}}. Most of the deities in the following verses do not seem to be met with elsewhere. The Sanskrit forms are often quite different but equally untraceable. } \\
blazing like a crested flame; \\
and the \textsanskrit{Ariṭṭhakas} and Rojas too,\footnote{Their names might mean “Unhurt” and “Hurter”. The Sanskrit implausibly has \textit{\textsanskrit{romā}} (“Romans”). } \\
and the gods hued blue as flax. 

The\marginnote{15.5} \textsanskrit{Varuṇas} and Sahadhammas,\footnote{For \textsanskrit{Varuṇa} see above (\href{https://suttacentral.net/dn20/en/sujato\#13.3}{DN 20:13.3}). | Sahadhamma means “who share the same duty”. } \\
the Accutas and Anejakas,\footnote{The “Unfallen” and the “Unshakeable”. } \\
the \textsanskrit{Sūleyyas} and Ruciras all came,\footnote{The \textsanskrit{Sūleyyas} might be “spear (or trident) bearers”; \textit{\textsanskrit{śūli}} is a name of Shiva. | Rucira means “brilliant, beautiful”. } \\
as did the \textsanskrit{Vāsavanesi} gods.\footnote{\textit{\textsanskrit{Vāsavanesi}} means “seekers of \textsanskrit{Vāsava}”. } \\
These ten hosts of gods \\
shone in all different colors. 

They’re\marginnote{15.11} powerful and brilliant, \\
so beautiful and glorious. \\
Rejoicing, they’ve come forth \\
to the meeting of mendicants in the wood. 

The\marginnote{16.1} \textsanskrit{Samānas}, \textsanskrit{Mahāsamānas},\footnote{“Equals” and “Great Equals”. } \\
\textsanskrit{Mānusas}, and \textsanskrit{Mānusuttamas} all came,\footnote{“Humans” and “Superhumans”, but here perhaps “Descendants of Manu”. } \\
and the gods depraved by play,\footnote{As per \href{https://suttacentral.net/dn1/en/sujato\#2.7.2}{DN 1:2.7.2}. } \\
and those who are malevolent.\footnote{\href{https://suttacentral.net/dn12.10.2/en/sujato}{DN 12.10.2}. } 

Then\marginnote{16.5} came the gods of Mercury,\footnote{\textit{Hari} means “yellow, green” and is the astrological color of the planet \textit{budha} (“Mercury”). It came to have the general meaning of “sacred, holy” and was an epithet for various deities, which in the Rig Veda included Indra and Vishnu. However the astrological connection is suggested by the references to Mars and Venus below. } \\
and those who live on Mars.\footnote{“Dwellers in the Red Place (= planet)”, i.e. “Martians”. } \\
The \textsanskrit{Pāragas} and \textsanskrit{Mahāpāragas} came,\footnote{\textit{\textsanskrit{Pāraga}} means “one who has crossed over”. } \\
such glorious gods. \\
These ten hosts of gods \\
shone in all different colors. 

They’re\marginnote{16.11} powerful and brilliant, \\
so beautiful and glorious. \\
Rejoicing, they’ve come forth \\
to the meeting of mendicants in the wood. 

The\marginnote{17.1} gods of Venus, the newborn Sun, \\>and the Dawn\footnote{The deities seem to represent the three main lights that announce a new day: the morning star, the newborn sun, and the aura of the dawn. | \textit{Sukka} (“Bright”) is the name of the planet Venus, the morning star. | \textit{\textsanskrit{Aruṇa}} is the dawn, a major Vedic deity signifying the arising of consciousness and order. | \textit{Karambha} means “porridge”, which was offered to the sun under the name \textsanskrit{Pūṣan} (Rig Veda 3.52.7, 6.56.1; Śatapatha \textsanskrit{Brāhmaṇa} 4.2.5.22; \textsanskrit{Kauṣītaki} \textsanskrit{Brāhmaṇa} 6.8.18; \textsanskrit{Maitrāyaṇī} \textsanskrit{Saṁhitā} 3.10.6). Hence he became known as \textit{\textsanskrit{karambhād}}, “porridge-eater”. Apparently the sun lost its teeth and could only eat porridge (Śatapatha \textsanskrit{Brāhmaṇa} 1.7.4.8). Being toothless like a baby, and lacking the risen sun’s “teeth” (i.e. “rays”), he probably represents the “newborn sun”. Compare for example Rig Veda 10.35, where \textsanskrit{Pūṣan}  is invoked, along with many other deities, in a hymn to the dawn. } \\
came along with those from Saturn.\footnote{\textsanskrit{Mahāsaṅgīti} has \textit{\textsanskrit{veghanasā}}, for which the Buddha Jayanthi variant \textit{vekhanasa} is preferable. We meet a wanderer of that name in \href{https://suttacentral.net/mn80/en/sujato}{MN 80}, and in Sanskrit \textit{\textsanskrit{vaikhānasa}} is a general term for ascetics; but this meaning seems unlikely in context. The astrologer \textsanskrit{Varāhamitra} names it a constellation (\textsanskrit{Bṛhat}-\textsanskrit{saṁhitā} 47.62), but that is much later. The Sanskrit here is \textit{\textsanskrit{nīlakavāsinī}} (“Dwellers in the Blue”), which pairs well with \textit{\textsanskrit{lohitavāsino}} above. \textit{\textsanskrit{Nīlavāsa}} is a name for Saturn, whose color is blue. } \\
And the gods of the white globe came\footnote{\textit{\textsanskrit{Odātagayha}} is probably “white planet”, i.e. the sun. Sanskrit has \textit{\textsanskrit{avadātakeśā}} (“white-haired”). Compare the description of solar gods \textsanskrit{Sūrya} at Rig Veda 1.50.8 as \textit{\textsanskrit{śociṣkeśaṁ} \textsanskrit{vicakṣaṇa}} (“flame-haired, brilliant”) and \textsanskrit{Varuṇa} at Rig Veda 8.41.9 as \textit{\textsanskrit{śvetā} \textsanskrit{vicakṣaṇā}} (“white, brilliant”). } \\
leading the brilliant gods.\footnote{Sanskrit has \textit{\textsanskrit{pītakavāsinī}}, “dwellers in the yellow”, either Mercury or Jupiter. } 

The\marginnote{17.5} \textsanskrit{Sadāmattas} and \textsanskrit{Hāragajas},\footnote{The \textit{\textsanskrit{sadāmatta}} (“everdrunk”) gods appear in some later Buddhist texts, alongside the \textit{\textsanskrit{māyādharas}} (or \textit{\textsanskrit{mālādharas}}), but they don’t seem to appear in a Vedic context. | I cannot find any reference to the \textit{\textsanskrit{hāragaja}} gods.  Sanskrit has \textit{\textsanskrit{hāritakā}}, perhaps “sons of \textsanskrit{Harītī}” (the \textit{\textsanskrit{yakkhinī}}; but see below, \href{https://suttacentral.net/dn20/en/sujato\#20.12}{DN 20:20.12}). } \\
and assorted glorious ones;\footnote{“Assorted” is \textit{missaka}, not elsewhere attested as a name of gods. } \\
Pajjuna came thundering,\footnote{Pajjuna (Sanskrit \textit{parjanya}) is a Vedic god of thunderstorms closely associated with Indra/Sakka. His daughters appear in \href{https://suttacentral.net/sn1.39/en/sujato}{SN 1.39} and \href{https://suttacentral.net/sn1.40/en/sujato}{SN 1.40}. Sanskrit has instead \textit{\textsanskrit{śuddhakā} \textsanskrit{rucikā}} (“pure, bright”). } \\
he who rains on all quarters. 

These\marginnote{17.9} ten hosts of gods \\
shone in all different colors. \\
They’re powerful and brilliant, \\
so beautiful and glorious. \\
Rejoicing, they’ve come forth \\
to the meeting of mendicants in the wood. 

The\marginnote{18.1} Khemiyas from the realms of Tusita and Yama,\footnote{The gods of Tusita and Yama are normal parts of the Buddhist cosmology. Neither \textit{khemiya} nor the Sanskrit \textit{\textsanskrit{kṣemaka}} appear to be the names of gods elsewhere, and the commentary explains that they dwell in both the Tusita and Yama realms. } \\
and the glorious \textsanskrit{Kaṭṭhakas} came;\footnote{The commentary acknowledges the variant readings \textit{\textsanskrit{kaṭṭhakā}} (from “stick” or “plowed”) and \textit{\textsanskrit{kathakā}} (“reciters”), while Sanskrit has \textit{\textsanskrit{kṛṣṇuktāś}} (“reciters of the black”). Perhaps related to Sanskrit \textit{\textsanskrit{kāṭhaka}}, a recension of the Black Yajur Veda taught by the sage \textsanskrit{Kaṭha}. } \\
the \textsanskrit{Lambītakas}, \textsanskrit{Lāmaseṭṭhas},\footnote{Perhaps from \textit{lamba} (“hang, droop”) in reference to deities of the sunset (cp. \textsanskrit{Mahābhārata} 4.31.4c \textit{\textsanskrit{sūrye} vilambati}). Sanskrit has \textit{\textsanskrit{lumbinī} \textsanskrit{lumbinīśreṣṭhā}}, which is perhaps normalized to mean “of \textsanskrit{Lumbinī}”. } \\
those called the Shining, \\>and the gods of Granted Wishes.\footnote{“Shining” (\textit{joti}) is used widely of many gods, including the Sun and Indra. | The commentary derives \textit{\textsanskrit{āsavā}} from \textit{\textsanskrit{āsa}} (“wish”) not \textit{\textsanskrit{āsava}} (“defilement”), explaining that they achieved their status due to desire (\textit{chandavasena}). This agrees with the Sanskrit \textit{\textsanskrit{svāśiṣā}} (“well-wish, prayer”), i.e. words of wishing or blessing uttered over the sacrifice (see Rig Veda 10.44.5; cp. 8.44.23). This is a case where using the Pali in a translation would be actively misleading. } \\
The gods who love to imagine came too, \\
and those who control what is imagined by others. 

These\marginnote{18.7} ten hosts of gods \\
shone in all different colors. \\
They’re powerful and brilliant, \\
so beautiful and glorious. \\
Rejoicing, they’ve come forth \\
to the meeting of mendicants in the wood. 

These\marginnote{19.1} sixty hosts of gods \\
shone in all different colors. \\
They came organized by name, \\
these and others likewise, thinking: 

‘They\marginnote{19.5} who have shed rebirth, the kindly ones,\footnote{\textit{\textsanskrit{Pavuṭṭha}} (“shed”), from \textit{pavasati} (“stay away”), seems to be unique in the Pali in this sense. Commentary glosses \textit{vigata}, “disappeared”. | The first part of this verse, according to the commentary, addresses the \textsanskrit{Saṅgha} as a group in the singular, but I translate in plural for clarity. } \\
the undefiled ones who have crossed the flood—\\
let us see them, with the dragon who brought them across, \\
who like the Moon has overcome darkness.’ 

\textsanskrit{Subrahmā}\marginnote{20.1} and Paramatta came,\footnote{\textsanskrit{Subrahmā} appears at \href{https://suttacentral.net/sn2.17/en/sujato}{SN 2.17} and \href{https://suttacentral.net/sn6.6/en/sujato}{SN 6.6}. | Paramatta appears only here in Pali. In Brahmanical texts it is a common term for the highest conception of \textsanskrit{Brahmā} as the “supreme soul” (\textit{\textsanskrit{paramātman}}) of the cosmos. This term came into common usage later, but is found occasionally in earlier texts such as \textsanskrit{Maitrāyaṇī} \textsanskrit{Saṁhitā} 2.9.1. } \\
with sons of those powerful ones. \\
\textsanskrit{Sanaṅkumāra} and Tissa\footnote{\textsanskrit{Sanaṅkumāra} makes regular guest appearances in the suttas (\href{https://suttacentral.net/dn3/en/sujato}{DN 3}, \href{https://suttacentral.net/dn18/en/sujato}{DN 18}, \href{https://suttacentral.net/dn19/en/sujato}{DN 19}, \href{https://suttacentral.net/dn27/en/sujato}{DN 27}, \href{https://suttacentral.net/sn6.11/en/sujato}{SN 6.11}, \href{https://suttacentral.net/an11.10/en/sujato}{AN 11.10}). | Tissa is a common name, meaning “born under the star Sirius”. Perhaps this is the former monk of that name who was reborn in the \textsanskrit{Brahmā} realm (\href{https://suttacentral.net/an6.34/en/sujato}{AN 6.34}). } \\
also came to the meeting in the wood. 

Of\marginnote{20.5} a thousand realms of divinity, \\
the Great Divinity stands forth.\footnote{The moniker “Great \textsanskrit{Brahmā}” (\textit{\textsanskrit{mahābrahmā}}) is here used of an individual, but sometimes it is a class of leading \textsanskrit{Brahmās}. } \\
He has arisen, resplendent, \\
his formidable body so glorious. 

Ten\marginnote{20.9} Gods Almighty came there,\footnote{“God Almighty” (\textit{issara}) is a \textsanskrit{Brahmā} god to whom the creation of the world is falsely attributed (\href{https://suttacentral.net/dn24/en/sujato\#2.14.3}{DN 24:2.14.3}, \href{https://suttacentral.net/mn101/en/sujato\#22.5}{MN 101:22.5}, \href{https://suttacentral.net/an3.61/en/sujato\#1.6}{AN 3.61:1.6}). } \\
each one of them wielding power, \\
and in the middle of them came \\
\textsanskrit{Hārita} with his following.”\footnote{Sanskrit has \textit{\textsanskrit{hāritī}}, the name of the fabled \textit{\textsanskrit{yakkhiṇī}} of \textsanskrit{Madhurā} converted by the Buddha. } 

When\marginnote{21.1} they had all come forth—\\
the gods with their Lord, and the divinities—\\
\textsanskrit{Māra}’s army came forth too: \\
see the stupidity of the Dark Lord! 

“Come,\marginnote{21.5} seize them and bind them,” he said, \\
“let them be bound by desire! \\
Surround them on all sides, \\
don’t let any escape!” 

And\marginnote{21.9} so there the great general, \\
the Dark Lord sent forth his army. \\
He struck the ground with his fist \\
to make a horrifying sound 

like\marginnote{21.13} a storm cloud shedding rain, \\
thundering and flashing. \\
But then he retreated, \\
furious, out of control. 

And\marginnote{22.1} all that was known \\
and distinguished by the Clear-eyed One. \\
Thereupon the Teacher addressed \\
the disciples who love the teaching: 

“\textsanskrit{Māra}’s\marginnote{22.5} army has arrived; \\
mendicants, you should be aware of them.” \\
Those monks grew keen, \\
hearing the Buddha’s instruction. \\
The army fled from those free of passion, \\
and not a single hair was stirred! 

“All\marginnote{22.11} triumphant in battle, \\
so fearless and glorious. \\
They rejoice with all the spirits, \\
the disciples well-known among men.” 

%
\end{verse}

%
\chapter*{{\suttatitleacronym DN 21}{\suttatitletranslation Sakka’s Questions }{\suttatitleroot Sakkapañhasutta}}
\addcontentsline{toc}{chapter}{\tocacronym{DN 21} \toctranslation{Sakka’s Questions } \tocroot{Sakkapañhasutta}}
\markboth{Sakka’s Questions }{Sakkapañhasutta}
\extramarks{DN 21}{DN 21}

\scevam{So\marginnote{1.1.1} I have heard.\footnote{This late sutta exhibits considerable literary sophistication. The A-plot concerns Sakka’s rejection of his violent past and embrace of the Dhamma, while the B-plot concerns \textsanskrit{Pañcasikkha}’s more modest growth from a libertine to a married man. | This discourse is quoted by name and a passage discussed at \href{https://suttacentral.net/sn22.4/en/sujato}{SN 22.4}. In addition, the dialogue at \href{https://suttacentral.net/mn37/en/sujato}{MN 37} picks up directly where this leaves off and may be considered a sequel. A Sanskrit parallel published by Ernst Waldschmidt in 1979 is available on SuttaCentral as \href{https://suttacentral.net/sf241/san/waldschmidt}{SF 241}. A different discourse of the same name is found at \href{https://suttacentral.net/sn35.118/en/sujato}{SN 35.118}. } }At one time the Buddha was staying in the land of the Magadhans, where east of \textsanskrit{Rājagaha} there is a brahmin village named \textsanskrit{Ambasaṇḍā}, north of which, on Mount Vediyaka, is Indra’s hill cave.\footnote{Rhys Davids follows the commentary in translating this “the cave of Indra’s \textsanskrit{Sāl} Tree”. However the Sanskrit  \textit{\textsanskrit{yāvacchailaguhāyām}} supports “hill cave”. } 

Now\marginnote{1.1.3} at that time Sakka, the lord of gods, grew eager to see the Buddha. He thought, “Where is the Blessed One at present, the perfected one, the fully awakened Buddha?” 

Seeing\marginnote{1.1.6} that the Buddha was at Indra’s hill cave, he addressed the gods of the thirty-three, “Good sirs, the Buddha is staying in the land of the Magadhans at Indra’s hill cave. What if we were to go and see that Blessed One, the perfected one, the fully awakened Buddha?” 

“Yes,\marginnote{1.1.10} lord,” replied the gods. 

Then\marginnote{1.2.1} Sakka addressed the centaur \textsanskrit{Pañcasikha},\footnote{Introduced in a minor role in \href{https://suttacentral.net/dn18/en/sujato}{DN 18}, \href{https://suttacentral.net/dn19/en/sujato}{DN 19}, and \href{https://suttacentral.net/dn20/en/sujato}{DN 20}, \textsanskrit{Pañcasikha} gets his moment to shine in this sutta. Outside of the \textsanskrit{Dīghanikāya}, he appears only in \href{https://suttacentral.net/sn35.119/en/sujato}{SN 35.119}. His name means “Fivecrest”, evidently in reference to his impressive hairdo; a nymph or group of nymphs bore the synonymous name \textsanskrit{Pañcacūḍā}, “Five Topknots”. The centaur (\textit{gandhabba}) was a wild rogue of music, dancing, and sex. He must be tamed, and here this is achieved through the power of love and filial respect. A less diplomatic approach is taken in Atharvaveda 4.37.7, where the \textit{gandharva} with his “egg-crest” (\textit{\textsanskrit{śikhaṇḍino}}) is threatened with castration. } “Dear \textsanskrit{Pañcasikha}, the Buddha is staying in the land of the Magadhans at Indra’s hill cave. What if we were to go and see that Blessed One, the perfected one, the fully awakened Buddha?” 

“Yes,\marginnote{1.2.4} lord,” replied the centaur \textsanskrit{Pañcasikha}. Taking his arched harp made from the pale timber of wood-apple, he went as Sakka’s attendant.\footnote{The “arched harp” (\textit{\textsanskrit{vīṇa}}) was a multi-stringed instrument held in the lap, with the strings attached across a curved open arch or bow. It is not the modern Indian instrument called veena, which is a lute or stick zither. \textit{Gandhabbas} are closely associated with music. } 

Then\marginnote{1.2.5} Sakka went at the head of a retinue consisting of the gods of the thirty-three and the centaur \textsanskrit{Pañcasikha}.\footnote{In the Buddhist hierarchy of divinity, \textit{gandhabbas} occupy a more humble place than the gods of the thirty-three, yet here \textsanskrit{Pañcasikha} appears with them in a position of honor. In \href{https://suttacentral.net/dn19/en/sujato\#28.3}{DN 19:28.3} we saw that the even more exalted \textsanskrit{Brahmā} takes the form of \textsanskrit{Pañcasikha}. This fluidity is characteristic of old Vedic cosmology, where there is no clear hierarchy of divinity. The formalized hierarchies in Buddhism and Hinduism are a later conception. } As easily as a strong person would extend or contract their arm, he vanished from the heaven of the gods of the thirty-three and landed on Mount Vediyaka north of \textsanskrit{Ambasaṇḍā}. 

Now\marginnote{1.3.1} at that time a dazzling light appeared over Mount Vediyaka and \textsanskrit{Ambasaṇḍā}, as happens through the glory of the gods. People in the villages round about, terrified, shocked, and awestruck, said, “Mount Vediyaka must be on fire today, blazing and burning! Oh why has such a dazzling light appeared over Mount Vediyaka and \textsanskrit{Ambasaṇḍā}?” 

Then\marginnote{1.4.1} Sakka addressed the centaur \textsanskrit{Pañcasikha}, “My dear \textsanskrit{Pañcasikha}, it is hard for one like me to get near the Realized Ones while they are on retreat practicing absorption, enjoying absorption.\footnote{We will learn later of how Sakka’s appearance could be too disturbing for the Buddha while on retreat (\href{https://suttacentral.net/dn21/en/sujato\#1.10.13}{DN 21:1.10.13}). \textsanskrit{Pañcasikha}, on the other hand, would appear alone, without a retinue. Note, however, that on subsequent occasions Sakka showed no such reticence, either because he was already a stream-enterer or because the Buddha was not in solitary retreat. } But if you were to charm the Buddha first, then I could go to see him.” 

“Yes,\marginnote{1.4.4} lord,” replied the centaur \textsanskrit{Pañcasikha}. Taking his arched harp made from the pale timber of wood-apple, he went to Indra’s hill cave. When he had drawn near, he stood to one side, thinking, “This is neither too far nor too near; and he’ll hear my voice.” 

\section*{1. \textsanskrit{Pañcasikha}’s Song }

Standing\marginnote{1.4.8} to one side, \textsanskrit{Pañcasikha} played his arched harp, and sang these verses on the Buddha, the teaching, the \textsanskrit{Saṅgha}, the perfected ones, and sensual love.\footnote{The idea that the Buddha would be wooed by a love song seems like an absurd conceit, until you take into account the centaurs’ dangerous reputation as inveterate lechers. \textsanskrit{Pañcasikha} wants to show that he is maturing and finally seeking a commitment in a respectable marriage. } 

\begin{verse}%
“O\marginnote{1.5.1} \textsanskrit{Bhaddā} \textsanskrit{Suriyavaccasā}, my Darling Sunshine,\footnote{\textsanskrit{Suriyavaccasā} (“Sunshine”) accords with the Vedic conception of the wife as the Sun (\textsanskrit{Sūryā}, Rig Veda 10.85) who takes as husband Soma, the moon. She appears in Atharvaveda 8.10.27, which invokes the \textit{gandharvas} with their female counterpart the “nymphs” (Pali \textit{\textsanskrit{accharā}}, Sanskrit \textit{\textsanskrit{āpsarásā}}). } \\
I pay homage to your father Timbaru,\footnote{In Sanskrit literature, Timbaru (Sanskrit \textit{tumburu}) was an elder \textit{gandharva} musician in the courts of Indra and Kuvera. According to \textsanskrit{Rāmāyaṇa} 3.4, he was cursed with a monstrous form by Kuvera due to an inappropriate lust for the nymph \textsanskrit{Rambhā}. Upon his defeat at the hands of \textsanskrit{Rāma} he was restored to his former status. \textsanskrit{Pañcasikha} would have wanted to avoid a similar fate. Rather than \textsanskrit{Suriyavaccasā}, the Sanskrit literature mentions his daughters \textsanskrit{Manuvantī} and \textsanskrit{Sukeśī} (\textsanskrit{Vāyupurāṇa} 69.49 = \textsanskrit{Brahmāṇḍapurāṇa} 2.7.13). } \\
through whom was born a lady so fine, \\
to fill me with a joy I never knew. 

As\marginnote{1.5.5} sweet as a breeze to one who’s sweating, \\
or when thirsty, a sweet and cooling drink, \\
so dear are you, \textsanskrit{Aṅgīrasi}, to me—\footnote{\textit{\textsanskrit{Aṅgīrasi}} means “shining one” and is etymologically linked with \textit{agni} (“fire”). \textsanskrit{Suriyavaccasā} earned the name due to both her radiant nature and her family connections: her father Timbaru was a son of \textsanskrit{Kaśyapa}, who was (according to certain lineages) brother of \textsanskrit{Aṅgīras}, from whom the \textsanskrit{Aṅgīrasas} were descended. In the Arthavaveda—the only early non-Buddhist source for a nymph named \textsanskrit{Suriyavaccasā}—\textit{\textsanskrit{aṅgīrasi}} is associated with witchcraft (Atharvaveda 12.5.52). \textit{\textsanskrit{Aṅgīrasa}} is also a patronymic of the Buddha (\href{https://suttacentral.net/dn32/en/sujato\#3.14}{DN 32:3.14}). } \\
just like the teaching is to all the saints!\footnote{“Saints” is \textit{\textsanskrit{arahantā}}, which I normally translate as “perfected ones”. } 

Like\marginnote{1.5.9} a cure when you’re struck by fever dire, \\
or food to ease the hunger pain, \\
come on, \textsanskrit{Bhaddā}, please put out my fire,\footnote{\textit{\textsanskrit{Parinibbāpaya}}, “please quench me”. \textsanskrit{Pañcasikha} subverts imagery employed by the Buddha—quenching thirst, cooling breezes, extinguished flames—to erotic effect. } \\
quench me like water on a flame. 

As\marginnote{1.5.13} elephants burning in the heat of summer, \\
sink down in a lotus pond to rest, \\
so cool, full of petals and of pollen—\\
that’s how I would plunge into your breast. 

Like\marginnote{1.5.17} elephants bursting bonds in rutting season,\footnote{\textsanskrit{Pañcasikha} is showing off his literary skill. The same image, the elephant, illustrates opposing qualities, peaceful and violent. In these two sides of sensual desire, he reveals, unconsciously perhaps, the rapacious side of his own nature. For \textsanskrit{Pañcasikha}, the elephant bursting its bonds illustrates his crazed devotion, whereas later it illustrates breaking free of desire (\href{https://suttacentral.net/dn21/en/sujato\#1.12.45}{DN 21:1.12.45}). } \\
beating off the pricks of lance and pikes—\\
I just don’t understand what is the reason \\
I’m so crazy for your shapely thighs! 

For\marginnote{1.5.21} you, my heart is full of passion, \\
I’m in a besotted state of mind. \\
There is no going back, I’m just not able, \\
I’m like a fish that’s hooked up on the line. 

Come\marginnote{1.5.25} on, my \textsanskrit{Bhaddā}, hold me, fair of thighs! \\
Embrace me, maid of captivating eyes! \\
Take me in your arms, my lovely lady, \\
that’s all I’d ever want or could desire. 

Ah,\marginnote{1.5.29} then my desire was such a small thing, \\
my sweet, with your curling wavy hair; \\
now, like to arahants an offering, \\
it’s grown so very much from there. 

Whatever\marginnote{1.5.33} the merit I have forged \\
by giving to such perfected beings—\\
may that, my altogether gorgeous, \\
ripen in togetherness with you.\footnote{For how a married couple may stay together in this life and the next, see \href{https://suttacentral.net/an4.55/en/sujato}{AN 4.55}. } 

Whatever\marginnote{1.5.37} the merit I have forged \\
in this vast territory, \\
may that, my altogether gorgeous, \\
ripen in togetherness with you. 

As\marginnote{1.5.41} the Sakyan, absorbed, did meditate \\
at one, alert, and mindful too, \\
the sage seeking the state free of death—\footnote{\textsanskrit{Pañcasikha} is using the present participle \textit{\textsanskrit{jigīsāno}} in the historical present. } \\
so I, oh my Sunshine, seek for you! 

And\marginnote{1.5.45} just like the sage would delight, \\
once he had awakened to the truth, \\
so I would delight, my fine lady, \\
were I to become one with you. 

If\marginnote{1.5.49} Sakka were to grant me one wish,\footnote{In Indian culture, a \textit{gandharva} wedding is a pure love-match, without the blessing of parents or priests. Here, however, \textsanskrit{Pañcasikha} wants to do the right thing and get the father’s blessing. } \\
as Lord of the Thirty and Three, \\
my \textsanskrit{Bhaddā}, you’re all I would wish for, \\
so strong is the love in me. 

Like\marginnote{1.5.53} a freshly blossoming sal tree \\
is your father, my lady so wise. \\
I pay homage to he, bowing humbly, \\
whose daughter is of such a kind.” 

%
\end{verse}

When\marginnote{1.6.1} \textsanskrit{Pañcasikha} had spoken, the Buddha said to him, “\textsanskrit{Pañcasikha}, the sound of your strings blends well with the sound of your singing,\footnote{If \textsanskrit{Pañcasikha} was seeking the Buddha’s approval for his love-match, he was doomed to disappointment, as the Buddha and his mendicants do not involve themselves in arranging marriages. The Buddha sidesteps the issue by politely complementing \textsanskrit{Pañcasikha} on his musicianship. } so that neither overpowers the other. But when did you compose these verses on the Buddha, the teaching, the \textsanskrit{Saṅgha}, the perfected ones, and sensual love?”\footnote{“When did you compose” is also at \href{https://suttacentral.net/mn56/en/sujato\#30.1}{MN 56:30.1}. } 

“This\marginnote{1.6.5} one time, sir, when you were first awakened, you were staying in \textsanskrit{Uruvelā} at the goatherd’s banyan tree on the bank of the \textsanskrit{Nerañjarā} River. And at that time I was in love with the lady named \textsanskrit{Bhaddā} \textsanskrit{Suriyavaccasā}, ‘Darling Sunshine’, the daughter of the centaur king Timbaru. But the sister desired another.\footnote{“Sister” (\textit{\textsanskrit{bhaginī}}) is a respectful term of address, but is normally used in a chaste sense. When \textsanskrit{Raṭṭhapāla} calls his wives “sister”, they faint in shock (\href{https://suttacentral.net/mn82/en/sujato\#23.4}{MN 82:23.4}). } It was the one named \textsanskrit{Sikhaṇḍī}, the son of \textsanskrit{Mātali} the charioteer, who she loved.\footnote{\textsanskrit{Sikhaṇḍī} (“egg-crested”) is the term for the centaur as seductive threat to women in Atharvaveda 4.37.7. It is also a poetic term for a peacock. } Since I couldn’t win that sister by any means, I took my arched harp to Timbaru’s home, where I played those verses. 

When\marginnote{1.7.10} I finished, \textsanskrit{Suriyavaccasā} said to me, ‘Dear sir, I have not personally seen the Buddha. But I did hear about him when I went to dance for the gods of the thirty-three in the Hall of Justice. Since you extol the Buddha, let us meet up today.’ And that’s when I met up with that sister. But we have not met since.” 

\section*{2. The Approach of Sakka }

Then\marginnote{1.8.1} Sakka, lord of gods, thought, “\textsanskrit{Pañcasikha} is exchanging pleasantries with the Buddha.” 

So\marginnote{1.8.3} he addressed \textsanskrit{Pañcasikha}, “My dear \textsanskrit{Pañcasikha}, please bow to the Buddha for me, saying: ‘Sir, Sakka, lord of gods, with his ministers and retinue, bows with his head at your feet.’” 

“Yes,\marginnote{1.8.6} lord,” replied \textsanskrit{Pañcasikha}. He bowed to the Buddha and said, “Sir, Sakka, lord of gods, with his ministers and retinue, bows with his head at your feet.” 

“So\marginnote{1.8.8} may Sakka with his ministers and retinue be happy, \textsanskrit{Pañcasikha},” said the Buddha, “for all want to be happy—whether gods, humans, titans, dragons, centaurs, or any of the other diverse creatures there may be.” 

For\marginnote{1.8.10} that is how the Realized Ones salute such illustrious spirits. And being saluted by the Buddha, Sakka entered Indra’s hill cave, bowed to the Buddha, and stood to one side. And the gods of the thirty-three did likewise, as did \textsanskrit{Pañcasikha}. 

And\marginnote{1.9.1} at that time the uneven places were evened out, the cramped places were opened up, the darkness vanished in the cave and light appeared, as happens through the glory of the gods. 

Then\marginnote{1.9.2} the Buddha said to Sakka, “It’s incredible and amazing that you, the venerable Kosiya, who has so many duties and so much to do, should come here.”\footnote{It is rare for the Buddha to address a deity as “venerable” (\textit{\textsanskrit{āyasmā}}), a term usually reserved for monks. \textsanskrit{Moggallāna} addresses Sakka as “the venerable Kosiya” at \href{https://suttacentral.net/mn37/en/sujato\#10.6}{MN 37:10.6}; \textsanskrit{Mahākassapa} calls him Kosiya without honorific at \href{https://suttacentral.net/ud3.7/en/sujato\#3.12}{Ud 3.7:3.12}. | The word \textit{kosiya} is explained by the commentaries as “owl”, which, if correct, would have been the totem for a clan of that name. It is, however, a patronymic: Rig Veda 1.10.11 has \textit{indra \textsanskrit{kauśika}} which means “Indra, son of \textsanskrit{Kuśika} (or \textsanskrit{Kuśa})”. \textit{\textsanskrit{Kuśa}} grass is critical to the performance of Vedic rites, and the label probably initially implied “Brahmanized”, i.e. a king whose reign was authorized according to Vedic ritual. Kosiya is said to be a low class family name (\href{https://suttacentral.net/pli-tv-bu-vb-pc2/en/sujato\#2.1.18}{Bu Pc 2:2.1.18}). } 

“For\marginnote{1.9.4} a long time I’ve wanted to come and see the Buddha, but I wasn’t able, being prevented by my many duties and responsibilities for the gods of the thirty-three. This one time, sir, the Buddha was staying near \textsanskrit{Sāvatthī} in the frankincense-tree hut. Then I went to \textsanskrit{Sāvatthī} to see the Buddha. But at that time the Buddha was sitting immersed in some kind of meditation. And a divine maiden of Great King \textsanskrit{Vessavaṇa} named \textsanskrit{Bhūjati} was attending on the Buddha, standing there paying homage to him with joined palms.\footnote{A \textit{\textsanskrit{paricārikā}} is a maid attending on a nymph (\href{https://suttacentral.net/mn37/en/sujato\#8.12}{MN 37:8.12}). | \textsanskrit{Bhūjati} (variant \textit{\textsanskrit{bhuñjatī}}) does not seem to appear elsewhere. } 

So\marginnote{1.10.2} I said to her, ‘Sister, please bow to the Buddha for me, saying: “Sir, Sakka, lord of gods, with his ministers and retinue, bows with his head at your feet.”’ 

When\marginnote{1.10.5} I said this, she said to me, ‘It’s the wrong time to see the Buddha, as he’s in retreat.’ 

‘Well\marginnote{1.10.8} then, sister, please convey my message when the Buddha emerges from that immersion.’ I hope that sister bowed to you? Do you remember what she said?” 

“She\marginnote{1.10.12} did bow, lord of gods, and I remember what she said. I also remember that it was the sound of your chariot wheels that pulled me out of that immersion.”\footnote{Sakka pointedly ignores this, but perhaps his shame at disturbing the Buddha’s meditation explains his oblique method of gaining an audience. } 

“Sir,\marginnote{1.11.1} I have heard and learned this in the presence of the gods who were reborn in the host of the thirty-three before me: ‘When a Realized One arises in the world, perfected and fully awakened, the heavenly hosts swell, while the titan hosts dwindle.’ And I have seen this with my own eyes. 

\subsection*{2.1. The Story of \textsanskrit{Gopikā} }

Right\marginnote{1.11.5} here in Kapilavatthu there was a Sakyan lady named \textsanskrit{Gopikā} who had confidence in the Buddha, the teaching, and the \textsanskrit{Saṅgha}, and had fulfilled her ethics.\footnote{This implies she was a stream-enterer. } Losing her attachment to femininity, she developed masculinity. When her body broke up, after death, she was reborn in a good place, a heavenly realm.\footnote{In Buddhism, sex is mutable between lives and even within one life (\href{https://suttacentral.net/pli-tv-bu-vb-pj1/en/sujato\#10.6.1}{Bu Pj 1:10.6.1}). Later Buddhists sometimes held that the status of a male was preferable. Here this is not stated outright, but the narrative implies that she thought her station was improved when becoming a male. Notably, it is not the Buddha who says this, but Sakka, relaying the story of another deity. That the proponent was a woman, at least formerly, is a trope of Indian literature: a woman is the ideal misogynist (see eg. \href{https://suttacentral.net/ja61/en/sujato}{Ja 61} or \textsanskrit{Śivapurāna} chapter 4). Here, though, the situation is nuanced, because despite her (presumed) inferior status, she has better rebirth than the monks she served. } In the company of the gods of the thirty-three she became one of my sons. There they knew him as the god Gopaka. 

Meanwhile\marginnote{1.11.10} three others, mendicants who had led the spiritual life under the Buddha, were reborn in the inferior centaur realm.\footnote{The realm of centaurs is “inferior” because the centaurs are known for their ribald sexuality. } There they amused themselves, supplied and provided with the five kinds of sensual stimulation, and became my servants and attendants. 

At\marginnote{1.11.12} that, Gopaka scolded them, ‘Where on earth were you at, good sirs, when you heard the Buddha’s teaching!\footnote{\textit{\textsanskrit{Kutomukhā}}, which occurs only in this sutta, contrasts with \textit{\textsanskrit{sammukhā}} (“in the presence of”, “from the mouth of”). The commentary explains it as a rhetorical question, “Were you distracted or falling asleep?” Alternatively, it might mean “From whom did you learn the Buddha’s teaching?” It recurs later in the verses. This is not the only such case: \textit{\textsanskrit{duddiṭṭharūpaṁ}} and \textit{\textsanskrit{kāyaṁ} \textsanskrit{brahmapurohitaṁ}} are other examples of terms found nowhere else in the Pali canon, yet found here both in prose and in verse. Clearly one has been copied from the other; it seems to me that the rarity of these terms and the greater coherence of the verses suggests that the prose is a (somewhat altered) summary of the verses. } For while I was still a woman I had confidence in the Buddha, the teaching, and the \textsanskrit{Saṅgha}, and had fulfilled my ethics. I lost my attachment to femininity and developed masculinity. When my body broke up, after death, I was reborn in a good place, a heavenly realm. In the company of the gods of the thirty-three I became one of Sakka’s sons. Here they know me as the god Gopaka. But you, having led the spiritual life under the Buddha, were reborn in the inferior centaur realm. It is a sad sight indeed to see fellow practitioners reborn in the inferior centaur realm.’ 

When\marginnote{1.11.19} scolded by Gopaka, two of those gods regained their memory right away. They went to the host of the priests of Divinity, but one god remained attached to sensual pleasures.\footnote{The commentary explains the verb “gained” (\textit{\textsanskrit{paṭilabhiṁsu}}) as serving dual roles here: “gaining” of “absorption mindfulness” (\textit{\textsanskrit{jhānasatiṁ}}), upon which they “gain” rebirth in the \textsanskrit{Brahmā} realm. However, the verses below speak of “recollecting” (\textit{\textsanskrit{anussaraṁ}}) the Buddha’s teachings at this juncture. Compare \href{https://suttacentral.net/pli-tv-bu-vb-np23/en/sujato\#1.1.20}{Bu NP 23:1.1.20}, where the king “remembered” his long-forgotten promise (\textit{\textsanskrit{vissaritvā} cirena \textsanskrit{satiṁ} \textsanskrit{paṭilabhitvā}}). \href{https://suttacentral.net/an4.191/en/sujato\#2.24}{AN 4.191:2.24} speaks of the “arising of memory” (\textit{\textsanskrit{satuppādo}}) in a deity when reminded of Dhamma passages they once recited as a monk. | The phrase \textit{\textsanskrit{kāyaṁ} \textsanskrit{brahmapurohitaṁ}} is syntactically incongruous; I think it was clumsily copied from the verses below, where the required verb (\textit{\textsanskrit{ajjhagaṁsu}}) is present. The commentary explains it as “a Minister of \textsanskrit{Brahmā}’s body” (\textit{\textsanskrit{brahmapurohitasarīraṁ}}), but in such contexts \textit{\textsanskrit{kāya}} usually means “host”. } 

\begin{verse}%
‘“I\marginnote{1.12.1} was a laywoman disciple of the Clear-eyed One, \\
and my name was \textsanskrit{Gopikā}. \\
I was devoted to the Buddha and the teaching, \\
and I faithfully served the \textsanskrit{Saṅgha}. 

Because\marginnote{1.12.5} of the excellence of the Buddha’s teaching, \\
I’m now a mighty, splendid son of Sakka, \\
reborn among the Three and Thirty. \\
And here they know me as Gopaka. 

Then\marginnote{1.12.9} I saw some mendicants who I’d seen before, \\
dwelling in the host of centaurs. \\
When I used to be a human, \\
they were disciples of Gotama. 

I\marginnote{1.12.13} served them with food and drink, \\
and clasped their feet in my own home.\footnote{When bowing, it is a special sign of devotion to touch the feet of the one revered (eg. \href{https://suttacentral.net/mn89/en/sujato\#9.1}{MN 89:9.1}). } \\
Where on earth were these good sirs at \\
when they learned the Buddha’s teachings? 

For\marginnote{1.12.17} each must know for themselves the teaching \\
so well-taught, realized by the Clear-eyed One. \\
I was one who followed you,\footnote{Here Gopaka shifts from speaking about them in third person to speaking to them in second person. } \\
having heard the fine words of the noble ones. 

I’m\marginnote{1.12.21} now a mighty, splendid son of Sakka, \\
reborn among the Three and Thirty.\footnote{\textit{Tidiva} is an abbreviated reference to the thirty-three gods. } \\
But you followed the best of men, \\
and led the supreme spiritual life, 

but\marginnote{1.12.25} still you’re born in this lesser realm, \\
a rebirth unbefitting. \\
It’s a sad sight I see, good sirs, \\
fellow practitioners in a lesser realm. 

Reborn\marginnote{1.12.29} in the host of centaurs, \\
only to wait upon the gods. \\
Meanwhile, I dwelt in a house—\footnote{I.e. she was a lay person. } \\
but see my distinction now! 

Having\marginnote{1.12.33} been a woman now I’m a male god, \\
blessed with heavenly sensual pleasures.” \\
Scolded by that disciple of Gotama,\footnote{Up until now, the verses have been spoken by Gopaka, but here a narrator appears. } \\
comprehending Gopaka, \\>they were struck with urgency. 

“Let’s\marginnote{1.12.37} strive, let’s try hard—\\
we won’t serve others any more!” \\
Two of them roused up energy, \\
recalling the Buddha’s instructions. 

Right\marginnote{1.12.41} away they became dispassionate, \\
seeing the drawbacks in sensual pleasures. \\
The fetters and bonds of sensual pleasures—\\
the ties of the Wicked One so hard to break—

they\marginnote{1.12.45} burst them like a bull elephant his ropes, \\
and passed right over the thirty-three. \\
The gods with Indra and the Progenitor \\
were all gathered in the Hall of Justice. 

As\marginnote{1.12.49} they sat there, they passed over them, \\
the heroes desireless, practicing purity. \\
Seeing them, \textsanskrit{Vāsava} was struck with urgency;\footnote{Normally rebirth is described as passing away in one realm then being reborn in another. Yet here the gods of the thirty-three—in their traditional abode on Mount Meru—were watching them literally fly from one realm to another. } \\
the master of gods in the midst of the group said, 

“These\marginnote{1.12.53} were born in the lesser centaur realm, \\
but now they pass us by!” \\
Heeding the speech of one so moved, \\
Gopaka addressed \textsanskrit{Vāsava}, 

“There\marginnote{1.12.57} is a Buddha, a lord of men, in the world.\footnote{The truncated \textit{janind’} is not vocative \textit{janinda} for Sakka, but nominative \textit{janindo} for the Buddha. } \\
Known as the Sakyan Sage, \\>he’s mastered the senses.\footnote{“Sakyan Sage” (\textit{sakyamuni}) became a favorite epithet of the Buddha, but in early texts it appears only here and the Ratanasutta (\href{https://suttacentral.net/snp2.1/en/sujato\#4.2}{Snp 2.1:4.2}). } \\
Those sons of his lost their memory;\footnote{“Sons” in a spiritual sense. } \\
but when scolded by me they gained it back. 

Of\marginnote{1.12.61} the three, there is one who remains \\
dwelling in the host of centaurs. \\
But two, recollecting the path to awakening, \\
serene, spurn even the gods. 

Such\marginnote{1.12.65} is the explanation of the teaching here:\footnote{The commentary says, “In this dispensation the explanation of the teaching is such that because of it, disciples become endowed with such qualities” (\textit{ettha \textsanskrit{sāsane} \textsanskrit{evarūpā} \textsanskrit{dhammappakāsanā}, \textsanskrit{yāya} \textsanskrit{sāvakā} etehi \textsanskrit{guṇehi} \textsanskrit{samannāgatā} honti}). This explains the feminine \textit{\textsanskrit{etādisī}}; \textit{\textsanskrit{pakasanā}}, which is normally neuter, is feminine here. This use is fairly common in later texts (it occurs at the conclusion of the \textsanskrit{Ṭīkā} for this sutta). } \\
not a single disciple doubts that.\footnote{\textit{\textsanskrit{Kiṅkaṅkhati}} is a unique term for “doubt”. } \\
We venerate the Buddha, the victor, lord of men, \\
who has crossed the flood and cut off doubt. 

They\marginnote{1.12.69} attained distinction to the extent \\
they understood the teaching here; \\
two of them distinguished \\
in the host of the priests of Divinity.”’\footnote{The end point of Gopaka’s speech is hard to determine, as there is no \textit{-ti} to mark it. Various editions and translations either ignore the issue or end it in different places. The next line marks the beginning of Sakka’s direct speech to the Buddha, and I think it makes the best sense to assume that up to here he has been relating the story he was told. } 

We\marginnote{1.12.73} have come here, dear sir, \\
to realize this same teaching. \\
If the Buddha would give me a chance, \\
I would ask a question, dear sir.” 

%
\end{verse}

Then\marginnote{1.13.1} the Buddha thought, “For a long time now this spirit has led a pure life. Any question he asks me will be beneficial, not unbeneficial. And he will quickly understand any answer I give to his question.” 

So\marginnote{1.13.4} the Buddha addressed Sakka in verse: 

\begin{verse}%
“Ask\marginnote{1.13.5} me your question, \textsanskrit{Vāsava}, \\
whatever you want. \\
I’ll solve each and every \\
question you have.” 

%
\end{verse}

\scendsection{The first recitation section is finished. }

Having\marginnote{2.1.1} been granted an opportunity by the Buddha, Sakka asked the first question. 

“Dear\marginnote{2.1.2} sir, what fetters bind the gods, humans, titans, dragons, centaurs—and any of the other diverse creatures—so that, though they wish to be free of enmity, violence, hostility, and hate, they still have enmity, violence, hostility, and hate?”\footnote{This question hints at Sakka’s own evolution from the battlegod of the Vedas to an acolyte of peace. } 

Such\marginnote{2.1.4} was Sakka’s question to the Buddha. And the Buddha answered him: 

“Lord\marginnote{2.1.6} of gods, the fetters of jealousy and stinginess bind the gods, humans, titans, dragons, centaurs—and any of the other diverse creatures—\footnote{Jealousy is wanting what others have, while stinginess is not wanting to share what you have. } so that, though they wish to be free of enmity, violence, hostility, and hate, they still have enmity, violence, hostility, and hate.” 

Such\marginnote{2.1.8} was the Buddha’s answer to Sakka. Delighted, Sakka approved and agreed with what the Buddha said, saying, “That’s so true, Blessed One! That’s so true, Holy One! Hearing the Buddha’s answer, I’ve gone beyond doubt and got rid of indecision.” 

And\marginnote{2.2.1} then, having approved and agreed with what the Buddha said, Sakka asked another question: 

“But\marginnote{2.2.2} dear sir, what is the source, origin, birthplace, and inception of jealousy and stinginess?\footnote{Compare with such contexts as the side branch of dependent origination at \href{https://suttacentral.net/dn15/en/sujato\#9.1}{DN 15:9.1}, the origins of disputes at \href{https://suttacentral.net/snp4.11/en/sujato}{Snp 4.11}, and the analysis of proliferation at \href{https://suttacentral.net/mn18/en/sujato}{MN 18}. Note the constructive use of the “yes, and” method of questioning. Sakka finds wisdom due to his curiosity, neither being too-easily sated with a simple answer, nor quibbling that the answer is inadequate, but building on the foundations of understanding. } When what exists is there jealousy and stinginess? When what doesn’t exist is there no jealousy and stinginess?” 

“The\marginnote{2.2.5} liked and the disliked, lord of gods, are the source of jealousy and stinginess. When the liked and the disliked exist there is jealousy and stinginess. When the liked and the disliked don’t exist there is no jealousy and stinginess.” 

“But\marginnote{2.2.8} dear sir, what is the source of what is liked and disliked?” 

“Desire\marginnote{2.2.11} is the source of what is liked and disliked.”\footnote{Compare \href{https://suttacentral.net/snp4.11/en/sujato\#4.1}{Snp 4.11:4.1}. } 

“But\marginnote{2.2.14} what is the source of desire?” 

“Thought\marginnote{2.2.17} is the source of desire.”\footnote{Here we depart from the normal sequence, which is that feelings give rise to craving (\textit{\textsanskrit{taṇhā}}). In \href{https://suttacentral.net/snp4.11/en/sujato\#6.1}{Snp 4.11:6.1} it is said that pleasure and pain give rise to desire, not thought. } 

“But\marginnote{2.2.20} what is the source of thought?” 

“Judgments\marginnote{2.2.23} driven by the proliferation of perceptions are the source of thoughts.”\footnote{“Judgments that emerge from the proliferation of perceptions” renders \textit{\textsanskrit{papañcasaññāsaṅkhā}}. Again the sequence departs from \href{https://suttacentral.net/mn18/en/sujato\#16.1}{MN 18:16.1}, which says that thoughts—which are normal and morally neutral psychological processes—give rise to “proliferation” (\textit{\textsanskrit{papañca}}), which is when craving and delusion cause thought to spin out of control. “Proliferation” then solidifies into “judgments” that fuel an individual’s delusion of “self” persisting through time. } 

“But\marginnote{2.3.1} how does a mendicant fittingly practice for the cessation of judgments driven by the proliferation of perceptions?” 

\subsection*{2.2. Meditation on Feelings }

“Lord\marginnote{2.3.3} of gods, there are two kinds of happiness, I say: that which you should cultivate, and that which you should not cultivate. There are two kinds of sadness, I say: that which you should cultivate, and that which you should not cultivate. There are two kinds of equanimity, I say: that which you should cultivate, and that which you should not cultivate. 

Why\marginnote{2.3.9} did I say that there are two kinds of happiness? Well, should you know of a happiness:\footnote{\textit{\textsanskrit{Jaññā}} is 3rd singular optative (cp. \href{https://suttacentral.net/an9.6/en/sujato\#3.4}{AN 9.6:3.4}). } ‘When I cultivate this kind of happiness, unskillful qualities grow, and skillful qualities decline.’ You should not cultivate that kind of happiness. Whereas, should you know of a happiness: ‘When I cultivate this kind of happiness, unskillful qualities decline, and skillful qualities grow.’ You should cultivate that kind of happiness. And that which is free of placing the mind and keeping it connected is better than that which still involves placing the mind and keeping it connected.\footnote{That is, second \textit{\textsanskrit{jhāna}} is better than first \textit{\textsanskrit{jhāna}}. } That’s why I said there are two kinds of happiness. 

Why\marginnote{2.3.17} did I say that there are two kinds of sadness? Well, should you know of a sadness: ‘When I cultivate this kind of sadness, unskillful qualities grow, and skillful qualities decline.’ You should not cultivate that kind of sadness. Whereas, should you know of a sadness: ‘When I cultivate this kind of sadness, unskillful qualities decline, and skillful qualities grow.’ You should cultivate that kind of sadness.\footnote{Overwhelmingly, the suttas speak of happiness in the path to liberation. They do, however, also acknowledge that sometimes temporary states of sadness leading to disillusionment can spur a person on the path. } And that which is free of placing the mind and keeping it connected is better than that which still involves placing the mind and keeping it connected.\footnote{It is hard to know what the text is getting at here, as there is no “sadness” in any \textit{\textsanskrit{jhāna}}. } That’s why I said there are two kinds of sadness. 

Why\marginnote{2.3.25} did I say that there are two kinds of equanimity? Well, should you know of an equanimity: ‘When I cultivate this kind of equanimity, unskillful qualities grow, and skillful qualities decline.’ You should not cultivate that kind of equanimity. Whereas, should you know of an equanimity: ‘When I cultivate this kind of equanimity, unskillful qualities decline, and skillful qualities grow.’ You should cultivate that kind of equanimity. And that which is free of placing the mind and keeping it connected is better than that which still involves placing the mind and keeping it connected. That’s why I said there are two kinds of equanimity. 

That’s\marginnote{2.3.33} how a mendicant fittingly practices for the cessation of judgments driven by the proliferation of perceptions.” 

Such\marginnote{2.3.34} was the Buddha’s answer to Sakka. Delighted, Sakka approved and agreed with what the Buddha said, saying, “That’s so true, Blessed One! That’s so true, Holy One! Hearing the Buddha’s answer, I’ve gone beyond doubt and got rid of indecision.” 

\subsection*{2.3. Restraint in the Monastic Code }

And\marginnote{2.4.1} then Sakka asked another question: 

“But\marginnote{2.4.2} dear sir, how does a mendicant practice for restraint in the monastic code?” 

“Lord\marginnote{2.4.3} of gods, I say that there are two kinds of bodily behavior:\footnote{Here the Buddha emphasizes the psychological and spiritual underpinnings of the “monastic code” (\textit{\textsanskrit{pātimokkha}}). } that which you should cultivate, and that which you should not cultivate. I say that there are two kinds of verbal behavior: that which you should cultivate, and that which you should not cultivate. There are two kinds of search, I say: that which you should cultivate, and that which you should not cultivate. 

Why\marginnote{2.4.9} did I say that there are two kinds of bodily behavior? Well, should you know of a bodily conduct: ‘When I cultivate this kind of bodily conduct, unskillful qualities grow, and skillful qualities decline.’ You should not cultivate that kind of bodily conduct. Whereas, should you know of a bodily conduct: ‘When I cultivate this kind of bodily conduct, unskillful qualities decline, and skillful qualities grow.’ You should cultivate that kind of bodily conduct. That’s why I said there are two kinds of bodily behavior. 

Why\marginnote{2.4.17} did I say that there are two kinds of verbal behavior? Well, should you know of a kind of verbal behavior that it causes unskillful qualities to grow while skillful qualities decline, you should not cultivate it. Whereas, should you know of a kind of verbal behavior that it causes unskillful qualities to decline while skillful qualities grow, you should cultivate it. That’s why I said there are two kinds of verbal behavior. 

Why\marginnote{2.4.24} did I say that there are two kinds of search? Well, should you know of a kind of search that it causes unskillful qualities to grow while skillful qualities decline, you should not cultivate it.\footnote{The suttas frequently describe three kinds of search: for sensual pleasures, for continued existence, and for a spiritual path (eg. \href{https://suttacentral.net/sn45.161/en/sujato\#7.3}{SN 45.161:7.3}). The last search lead the Buddha to awakening (\href{https://suttacentral.net/mn26/en/sujato\#15.1}{MN 26:15.1}). } Whereas, should you know of a kind of search that it causes unskillful qualities to decline while skillful qualities grow, you should cultivate it. That’s why I said there are two kinds of search. 

That’s\marginnote{2.4.31} how a mendicant practices for restraint in the monastic code.” 

Such\marginnote{2.4.32} was the Buddha’s answer to Sakka. Delighted, Sakka approved and agreed with what the Buddha said, saying, “That’s so true, Blessed One! That’s so true, Holy One! Hearing the Buddha’s answer, I’ve gone beyond doubt and got rid of indecision.” 

\subsection*{2.4. Sense Restraint }

And\marginnote{2.5.1} then Sakka asked another question: 

“But\marginnote{2.5.2} dear sir, how does a mendicant practice for restraint of the sense faculties?” 

“Lord\marginnote{2.5.3} of gods, I say that there are two kinds of sight known by the eye: that which you should cultivate, and that which you should not cultivate. There are two kinds of sound known by the ear … smells known by the nose … tastes known by the tongue … touches known by the body … ideas known by the mind: that which you should cultivate, and that which you should not cultivate.” 

When\marginnote{2.5.15} the Buddha said this, Sakka said to him: 

“Sir,\marginnote{2.5.16} this is how I understand the detailed meaning of the Buddha’s brief statement:\footnote{Sakka switches his term of address for the Buddha from \textit{\textsanskrit{mārisa}} to the more respectful \textit{bhante} here, then back to \textit{\textsanskrit{mārisa}} below. This passage is found at \href{https://suttacentral.net/mn114/en/sujato\#24.5}{MN 114:24.5}, where \textit{bhante} is used, suggesting it was imported from there. } You should not cultivate the kind of sight known by the eye which causes unskillful qualities to grow while skillful qualities decline.\footnote{For a more detailed analysis of this, see \href{https://suttacentral.net/mn152/en/sujato}{MN 152}. } And you should cultivate the kind of sight known by the eye which causes unskillful qualities to decline while skillful qualities grow. You should not cultivate the kind of sound, smell, taste, touch, or idea known by the mind which causes unskillful qualities to grow while skillful qualities decline. And you should cultivate the kind of idea known by the mind which causes unskillful qualities to decline while skillful qualities grow. 

Sir,\marginnote{2.5.25} that’s how I understand the detailed meaning of the Buddha’s brief statement. Hearing the Buddha’s answer, I’ve gone beyond doubt and got rid of indecision.” 

And\marginnote{2.6.1} then Sakka asked another question: 

“Dear\marginnote{2.6.2} sir, do all ascetics and brahmins have a single doctrine, ethics, desire, and attachment?”\footnote{The Pali \textit{ekanta} (“single”) here is a little tricky; it could mean either “the same” or “of one goal”. The Sanskrit has just \textit{eka} (\textit{\textsanskrit{ekakāmā} \textsanskrit{ekacchandāḥ}}, “one desire, one wish”), while the commentary explains the first and last terms as “one goal” and the middle terms as “one”. } 

“No,\marginnote{2.6.3} lord of gods, they do not.” 

“Why\marginnote{2.6.4} not?” 

“The\marginnote{2.6.5} world has many and diverse elements. Whatever element sentient beings insist on in this world of many and diverse elements, they obstinately stick to it, insisting that:\footnote{The Buddha points out that spiritual teachers, when faced with a world of experiential diversity, tend to prioritize their own experience or doctrine, dismissing others as less real or meaningful. } ‘This is the only truth, anything else is futile.’ That’s why not all ascetics and brahmins have a single doctrine, ethics, desire, and attachment.” 

“Dear\marginnote{2.6.9} sir, have all ascetics and brahmins reached the ultimate end, the ultimate sanctuary from the yoke, the ultimate spiritual life, the ultimate goal?” 

“No,\marginnote{2.6.10} lord of gods, they have not.” 

“Why\marginnote{2.6.11} not?” 

“Those\marginnote{2.6.12} mendicants who are freed through the ending of craving have reached the ultimate end, the ultimate sanctuary from the yoke, the ultimate spiritual life, the ultimate goal. That’s why not all ascetics and brahmins have reached the ultimate end, the ultimate sanctuary from the yoke, the ultimate spiritual life, the ultimate goal.” 

Such\marginnote{2.6.14} was the Buddha’s answer to Sakka. Delighted, Sakka approved and agreed with what the Buddha said, saying, “That’s so true, Blessed One! That’s so true, Holy One! Hearing the Buddha’s answer, I’ve gone beyond doubt and got rid of indecision.”\footnote{This is the final substantive question of the series. It is picked up in \href{https://suttacentral.net/mn37/en/sujato\#2.2}{MN 37:2.2} where Sakka, in a subsequent visit, inquires further as to the nature of those who have realized the ultimate goal. } 

And\marginnote{2.6.18} then Sakka, having approved and agreed with what the Buddha said, said to him, 

“Turbulence,\marginnote{2.7.1} sir, is a disease, a boil, a dart. Turbulence drags a person to be reborn in life after life.\footnote{“Turbulence” is \textit{\textsanskrit{ejā}}, from a root meaning “motion, agitation”, and hence the opposite of the “imperturbable” (\textit{aneja}) peace of the Buddha (\href{https://suttacentral.net/sn35.90/en/sujato}{SN 35.90}). } That’s why a person finds themselves in states high and low. Elsewhere, among other ascetics and brahmins, I wasn’t even given a chance to ask these questions that the Buddha has answered. The dart of doubt and indecision has lain within me for a long time, but the Buddha has plucked it out.” 

“Lord\marginnote{2.7.5} of gods, do you recall having asked this question of other ascetics and brahmins?”\footnote{Compare \textsanskrit{Ajātasattu}’s experience at \href{https://suttacentral.net/dn2/en/sujato\#15.1}{DN 2:15.1}. } 

“I\marginnote{2.7.6} do, sir.” 

“If\marginnote{2.7.7} you wouldn’t mind, lord of gods, tell me how they answered.” 

“It’s\marginnote{2.7.8} no trouble when someone such as the Blessed One is sitting here.” 

“Well,\marginnote{2.7.9} speak then, lord of gods.” 

“Sir,\marginnote{2.7.10} I approached those who I imagined were ascetics and brahmins living in the wilderness, in remote lodgings. But they were stumped by my question, and they even questioned me in return: ‘What is the venerable’s name?’ So I answered them: ‘Dear sir, I am Sakka, lord of gods.’ So they asked me another question: ‘But lord of gods, what deed brought you to this position?’ So I taught them the Dhamma as I had learned and memorized it. And they were pleased with just that much: ‘We have seen Sakka, lord of gods! And he answered our questions!’ Invariably, they become my disciples, I don’t become theirs. But sir, I am the Buddha’s disciple, a stream-enterer, not liable to be reborn in the underworld, bound for awakening.” 

\subsection*{2.5. On Feeling Happy }

“Lord\marginnote{2.7.22} of gods, do you recall ever feeling such joy and happiness before?” 

“I\marginnote{2.7.23} do, sir.” 

“But\marginnote{2.7.24} how?” 

“Once\marginnote{2.7.25} upon a time, sir, a battle was fought between the gods and the titans. In that battle the gods won and the titans lost. It occurred to me as victor, ‘Now the gods shall enjoy both the nectar of the gods and the nectar of the titans.’\footnote{“Nectar” is \textit{\textsanskrit{ojā}}, in Vedic texts called \textit{soma} or \textit{\textsanskrit{amṛta}} (“ambrosia” of immortality). The battle over nectar by gods (\textit{deva}) and demons (\textit{asura}) is a very ancient element of Indo-European mythology. } But sir, that joy and happiness is in the sphere of the rod and the sword. It doesn’t lead to disillusionment, dispassion, cessation, peace, insight, awakening, and extinguishment. But the joy and happiness I feel listening to the Buddha’s teaching is not in the sphere of the rod and the sword. It does lead to disillusionment, dispassion, cessation, peace, insight, awakening, and extinguishment.”\footnote{Here Sakka makes a clean break from the violent delights he enjoyed in former (Vedic) times. } 

“But\marginnote{2.8.1} lord of gods, what reason do you see for speaking of such joy and happiness?” 

“I\marginnote{2.8.2} see six reasons to speak of such joy and happiness, sir. 

\begin{verse}%
While\marginnote{2.8.3} staying right here, \\
remaining in the godly form, \\
I have gained an extended life:\footnote{\textit{\textsanskrit{Punarāyu}} is a unique term. It probably means that Sakka can extend his current fortunate birth due to the merit of Dhamma. } \\
know this, dear sir. 

%
\end{verse}

This\marginnote{2.8.7} is the first reason. 

\begin{verse}%
When\marginnote{2.8.8} I fall from the heavenly host, \\
leaving behind the non-human life, \\
I shall consciously go to a new womb,\footnote{To die “unconfused” or “consciously” is said to be a benefit of Dhamma practice (eg. \href{https://suttacentral.net/an5.215/en/sujato\#2.3}{AN 5.215:2.3}). } \\
wherever my mind delights. 

%
\end{verse}

This\marginnote{2.8.12} is the second reason. 

\begin{verse}%
Living\marginnote{2.8.13} happily under the guidance \\
of the one of unclouded wisdom, \\
I shall practice systematically,\footnote{“According to method” (\textit{\textsanskrit{ñāyena}}) recalls \textit{\textsanskrit{ñāyapaṭipanno}} as a description of the noble ones. It means practicing according with the noble eightfold path, a “method” that yields definite results. } \\
aware and mindful. 

%
\end{verse}

This\marginnote{2.8.17} is the third reason. 

\begin{verse}%
And\marginnote{2.8.18} if awakening should arise \\
as I practice systematically, \\
I shall live as one who understands, \\
and my end shall come right there.\footnote{This refers to the possibility of achieving perfection (\textit{\textsanskrit{arahattā}}). } 

%
\end{verse}

This\marginnote{2.8.22} is the fourth reason. 

\begin{verse}%
When\marginnote{2.8.23} I fall from the human realm, \\
leaving behind the human life, \\
I shall become a god again, \\
supreme in the heaven realm.\footnote{That is, after passing away as Sakka and being reborn as a human, he will subsequently return to the heaven realm. } 

%
\end{verse}

This\marginnote{2.8.27} is the fifth reason. 

\begin{verse}%
They\marginnote{2.8.28} are the finest of gods, \\
the glorious \textsanskrit{Akaniṭṭhas}. \\
So long as my final life goes on,\footnote{He predicts that his final life will be in this high realm that is inhabited by non-returners (\href{https://suttacentral.net/an9.12/en/sujato\#6.7}{AN 9.12:6.7}). } \\
there my home will be. 

%
\end{verse}

This\marginnote{2.8.32} is the sixth reason. 

Seeing\marginnote{2.8.33} these six reasons I speak of such joy and happiness. 

\begin{verse}%
My\marginnote{2.9.1} wishes unfulfilled, \\
doubting and indecisive, \\
I wandered for such a long time, \\
in search of the Realized One. 

I\marginnote{2.9.5} imagined that ascetics \\
living in seclusion \\
must surely be awakened, \\
so I went to sit near them. 

‘How\marginnote{2.9.9} is there success? \\
How is there failure?’ \\
But they were stumped by such questions \\
about the path and practice. 

And\marginnote{2.9.13} when they found out that I \\
was Sakka, come from the gods, \\
they questioned me instead about \\
the deed that brought me to this state. 

I\marginnote{2.9.17} taught them the Dhamma \\
as I had learned it among men. \\
They were delighted with that, saying: \\
‘We’ve seen \textsanskrit{Vāsava}!’ 

Now\marginnote{2.9.21} since I’ve seen the Buddha, \\
who helps us overcome doubt, \\
today, free of fear, \\
I pay homage to the awakened one. 

Destroyer\marginnote{2.9.25} of the dart of craving, \\
the Buddha is unrivaled. \\
I bow to the great hero, \\
the Buddha, kinsman of the Sun. 

In\marginnote{2.9.29} the same way that Divinity ought be revered\footnote{\textit{Karomasi} is middle imperative 1st plural, “we should do”, glossed by the commentary as “we should pay homage” (\textit{\textsanskrit{namakkāraṁ} karoma}). } \\
by we gods, dear sir,\footnote{\textit{\textsanskrit{Samaṁ}} is an indeclinable in the sense “likewise”. } \\
today we shall revere you—\\
come, let us revere you ourselves!\footnote{\textit{\textsanskrit{Sāmaṁ}} is an indeclinable in the sense “oneself, personally, one’s own”. } 

You\marginnote{2.9.33} alone are the Awakened! \\
You are the Teacher supreme! \\
In the world with its gods, \\
you have no rival.” 

%
\end{verse}

Then\marginnote{2.10.1} Sakka addressed the centaur \textsanskrit{Pañcasikha}, “Dear \textsanskrit{Pañcasikha}, you were very helpful to me, since you first charmed the Buddha, after which I went to see him. I shall appoint you to your father’s position—you shall be king of the centaurs. And I give you \textsanskrit{Bhaddā} \textsanskrit{Suriyavaccasā}, for she loves you very much.”\footnote{A \textit{gandharva} wedding must be a mutual love match. } 

Then\marginnote{2.10.5} Sakka, touching the ground with his hand, expressed this heartfelt sentiment three times:\footnote{“Touching the ground with his hand” is a unique gesture in the early Pali, immortalized in later legend when the Buddha touches the ground to invoke the Earth Goddess as witness of his striving. } 

“Homage\marginnote{2.10.6} to that Blessed One, the perfected one, the fully awakened Buddha! 

Homage\marginnote{2.10.7} to that Blessed One, the perfected one, the fully awakened Buddha! 

Homage\marginnote{2.10.8} to that Blessed One, the perfected one, the fully awakened Buddha!” 

And\marginnote{2.10.9} while this discourse was being spoken, the stainless, immaculate vision of the Dhamma arose in Sakka, lord of gods: “Everything that has a beginning has an end.” And also for another 80,000 deities.\footnote{Sakka is also said to have been accompanied by 80,000 deities at \href{https://suttacentral.net/sn40.10/en/sujato\#4.4}{SN 40.10:4.4}. This is one of many indications that this is a late discourse, several of which have been touched on above: late or unique terminology; fancy literary styles; adoption of doctrinal passages from elsewhere in a sometimes clumsy fashion; and so on. In determining whether a sutta is late or early, we do not rely on a single definitive reason, but on a cluster of independent criteria which taken together are most easily explained in terms of historical development. } 

Such\marginnote{2.10.12} were the questions Sakka was invited to ask, and which were answered by the Buddha. And that’s why the name of this discussion is “Sakka’s Questions”. 

%
\chapter*{{\suttatitleacronym DN 22}{\suttatitletranslation The Longer Discourse on Mindfulness Meditation }{\suttatitleroot Mahāsatipaṭṭhānasutta}}
\addcontentsline{toc}{chapter}{\tocacronym{DN 22} \toctranslation{The Longer Discourse on Mindfulness Meditation } \tocroot{Mahāsatipaṭṭhānasutta}}
\markboth{The Longer Discourse on Mindfulness Meditation }{Mahāsatipaṭṭhānasutta}
\extramarks{DN 22}{DN 22}

\scevam{So\marginnote{1.1} I have heard.\footnote{This discourse is copied from \href{https://suttacentral.net/mn10/en/sujato}{MN 10}. The section on the four noble truths has been expanded with material mostly drawn from \href{https://suttacentral.net/mn141/en/sujato}{MN 141}. These discourses are the most influential texts for modern Theravada meditation, prompting countless modern commentaries. Comparative study of the several parallel versions reveals that this discourse, while comprised almost entirely of early material, was compiled in this form as one of the latest texts in the Pali suttas. | While mindfulness is always useful (\href{https://suttacentral.net/sn46.53/en/sujato\#15.4}{SN 46.53:15.4}), the “establishment of mindfulness” (\textit{\textsanskrit{satipaṭṭhāna}}) refers especially to a conscious development of contemplative practices based on mindfulness, i.e. “mindfulness meditation”. } }At one time the Buddha was staying in the land of the Kurus, near the Kuru town named \textsanskrit{Kammāsadamma}. There the Buddha addressed the mendicants, “Mendicants!” 

“Venerable\marginnote{1.5} sir,” they replied. The Buddha said this: 

“Mendicants,\marginnote{1.7} the four kinds of mindfulness meditation are the path to convergence. They are in order to purify sentient beings, to get past sorrow and crying, to make an end of pain and sadness, to discover the system, and to realize extinguishment.\footnote{The phrase \textit{\textsanskrit{ekāyano} maggo} (“path to convergence”) is given multiple meanings in commentaries and ancient translations. Outside of \textit{\textsanskrit{satipaṭṭhāna}}, it is used in only one context in Pali, where it means to “come together with” (\href{https://suttacentral.net/mn12/en/sujato\#37.5}{MN 12:37.5}). At \href{https://suttacentral.net/sn47.18/en/sujato\#3.4}{SN 47.18:3.4} the phrase is spoken by \textsanskrit{Brahmā}, which suggests it was a Brahmanical term. At \textsanskrit{Bṛhadāraṇyaka} \textsanskrit{Upaniṣad} 2.4.11—a passage full of details shared with the suttas—it means a place where things unite or converge. Thus \textit{\textsanskrit{satipaṭṭhāna}} leads to everything “coming together as one”. In other words, as seventh factor of the noble eightfold path, it leads to \textit{\textsanskrit{samādhi}}, the eighth factor (\href{https://suttacentral.net/sn45.1/en/sujato\#3.9}{SN 45.1:3.9}; see also \href{https://suttacentral.net/mn44/en/sujato\#12.3}{MN 44:12.3}). } 

What\marginnote{1.8} four? It’s when a mendicant meditates by observing an aspect of the body—keen, aware, and mindful, rid of covetousness and displeasure for the world.\footnote{The idiom \textit{\textsanskrit{kāye} \textsanskrit{kāyānupassī}}, literally “observes a body in the body” refers to focusing on a specific aspect of embodied experience, such as the breath, the postures, etc. | “Keen” (or “ardent”, \textit{\textsanskrit{ātāpī}}) implies effort, while “aware” (\textit{\textsanskrit{sampajāno}}) is the wisdom of understanding situation and context. | “Covetousness and displeasure” (\textit{\textsanskrit{abhijjhādomanassaṁ}}) are the strong forms of desire and aversion that are overcome by sense restraint in preparation for meditation. } They meditate observing an aspect of feelings—keen, aware, and mindful, rid of covetousness and displeasure for the world.\footnote{“Feelings” (\textit{\textsanskrit{vedanā}}) are the basic tones of pleasant, painful, or neutral, not the complexes we call “emotions”. } They meditate observing an aspect of the mind—keen, aware, and mindful, rid of covetousness and displeasure for the world.\footnote{“Mind” (\textit{citta}) is simple awareness. In meditation contexts, “mind” is often similar in meaning to \textit{\textsanskrit{samādhi}}. } They meditate observing an aspect of principles—keen, aware, and mindful, rid of covetousness and displeasure for the world.\footnote{“Principles” (\textit{\textsanskrit{dhammā}}) are the natural “systems” of cause and effect that underlie the “teachings”. The renderings “mind objects” or “mental qualities” are incorrect, as many of the things spoken of in this section are neither mind objects nor mental qualities. “Phenomena” is a possible translation, but the emphasis is not on the “appearance” of things, but on the “principles” governing their conditional relations. } 

\section*{1. Observing the Body }

\subsection*{1.1. Mindfulness of Breathing }

And\marginnote{2.1} how does a mendicant meditate observing an aspect of the body? 

It’s\marginnote{2.2} when a mendicant—gone to a wilderness, or to the root of a tree, or to an empty hut—sits down cross-legged, sets their body straight, and establishes mindfulness in their presence.\footnote{The context here—a mendicant gone to the forest—establishes that this practice takes place in the wider context of the Gradual Training. Indeed, this whole sutta can be understood as an expansion of this phrase, mentioned briefly at \href{https://suttacentral.net/dn2/en/sujato\#67.3}{DN 2:67.3}. } Just mindful, they breathe in. Mindful, they breathe out.\footnote{The most fundamental meditation instruction in Buddhism. Notice how the Buddha phrases it: not “concentrate on the breath” as an object, but rather “breathing” as an activity to which one brings mindfulness. } 

Breathing\marginnote{2.4} in heavily they know: ‘I’m breathing in heavily.’ Breathing out heavily they know: ‘I’m breathing out heavily.’\footnote{The stages of breath meditation are not meant to be done deliberately, but to be observed and understood as the natural process of deepening meditation. When starting out, the breath is somewhat rough and coarse. } 

When\marginnote{2.5} breathing in lightly they know: ‘I’m breathing in lightly.’ Breathing out lightly they know: ‘I’m breathing out lightly.’\footnote{Over time, the breath becomes more subtle and soft. } 

They\marginnote{2.6} practice like this: ‘I’ll breathe in experiencing the whole body.’ They practice like this: ‘I’ll breathe out experiencing the whole body.’\footnote{Contextually the idiom “whole body” (\textit{\textsanskrit{sabbakāya}}) here refers to the breath, marking the fuller and more continuous awareness that arises with tranquility. Some practitioners, however, interpret it as the “whole physical body”, broadening awareness to encompass the movement and settling of energies throughout the body. } 

They\marginnote{2.7} practice like this: ‘I’ll breathe in stilling the physical process.’ They practice like this: ‘I’ll breathe out stilling the physical process.’\footnote{The “physical process” (\textit{\textsanskrit{kāyasaṅkhāraṁ}}) is the breath (\href{https://suttacentral.net/sn41.6/en/sujato\#1.8}{SN 41.6:1.8}). } 

It’s\marginnote{2.8} like a deft carpenter or carpenter’s apprentice. When making a deep cut they know: ‘I’m making a deep cut,’ and when making a shallow cut they know: ‘I’m making a shallow cut.’\footnote{Text has “long” and “short”, but “deep” and “shallow” or “heavy” and “light” are more idiomatic for describing the breath in English. } 

And\marginnote{2.11} so they meditate observing an aspect of the body internally, externally, and both internally and externally.\footnote{“Internally” is one’s own body, “externally” the bodies of others, or external physical phenomena. This distinction is applied broadly in Buddhist meditation, but it is more relevant in some contexts than others. In the case of the breath, one is focusing on one’s own breath, but when contemplating, say, a dead body, or the material elements, there is more of an external dimension. Starting with “me” in here and the “world” out there, this practice dissolves this distinction so that we see we are of the same nature as everything else. } They meditate observing the body as liable to originate, as liable to vanish, and as liable to both originate and vanish.\footnote{This is the \textit{\textsanskrit{vipassanā}} (“insight” or “discernment”) dimension of meditation, observing not just the rise and fall of phenomena, but also their conditioned “nature” as being “liable” (\textit{-dhamma}) to impermanence. The meaning of this passage is explained at (\href{https://suttacentral.net/sn47.42/en/sujato}{SN 47.42}). Apart from these passages, \textit{\textsanskrit{vipassanā}} in \textit{\textsanskrit{satipaṭṭhāna}} pertains specially to the observation of principles. } Or mindfulness is established that the body exists, to the extent necessary for knowledge and mindfulness. They meditate independent, not grasping at anything in the world.\footnote{Mindfulness meditation leads to a range of knowledges as detailed by Anuruddha at \href{https://suttacentral.net/sn52.6/en/sujato}{SN 52.6} and \href{https://suttacentral.net/sn52.11/en/sujato}{SN 52.11}–24. An arahant is “independent” of any attachment (eg. \href{https://suttacentral.net/mn143/en/sujato}{MN 143}), but \textit{\textsanskrit{satipaṭṭhāna}} is also taught to give up dependency on views of the past and future (\href{https://suttacentral.net/dn29/en/sujato\#40.1}{DN 29:40.1}). } 

That’s\marginnote{2.14} how a mendicant meditates by observing an aspect of the body. 

\subsection*{1.2. The Postures }

Furthermore,\marginnote{3.1} when a mendicant is walking they know: ‘I am walking.’ When standing they know: ‘I am standing.’ When sitting they know: ‘I am sitting.’ And when lying down they know: ‘I am lying down.’\footnote{In early Pali, this practice is found only in the two \textsanskrit{Satipaṭṭhānasuttas} and the closely related \textsanskrit{Kāyagatāsatisutta} (\href{https://suttacentral.net/mn119/en/sujato}{MN 119}). It is practiced by developing a reflexive awareness of one’s posture and activity as it proceeds, often assisted by moving slowly and carefully. } Whatever posture their body is in, they know it. 

And\marginnote{3.3} so they meditate observing an aspect of the body internally, externally, and both internally and externally. They meditate observing the body as liable to originate, as liable to vanish, and as liable to both originate and vanish. Or mindfulness is established that the body exists, to the extent necessary for knowledge and mindfulness. They meditate independent, not grasping at anything in the world. 

That\marginnote{3.6} too is how a mendicant meditates by observing an aspect of the body. 

\subsection*{1.3. Situational Awareness }

Furthermore,\marginnote{4.1} a mendicant acts with situational awareness when going out and coming back; when looking ahead and aside; when bending and extending the limbs; when bearing the outer robe, bowl, and robes; when eating, drinking, chewing, and tasting; when urinating and defecating; when walking, standing, sitting, sleeping, waking, speaking, and keeping silent.\footnote{“Situational awareness” (\textit{\textsanskrit{sampajañña}}) understands the context and purpose of activities. The main examples here illustrate the activities of daily monastic life: leaving the monastery on almsround, restraint while in the town, care wearing the robes, then mindfully eating and going to the toilet. } 

And\marginnote{4.2} so they meditate observing an aspect of the body internally … 

That\marginnote{4.3} too is how a mendicant meditates by observing an aspect of the body. 

\subsection*{1.4. Focusing on the Repulsive }

Furthermore,\marginnote{5.1} a mendicant examines their own body, up from the soles of the feet and down from the tips of the hairs, wrapped in skin and full of many kinds of filth.\footnote{This practice is intended to counter sexual desire and obsession. The primary focus is on one’s own body, rather than another’s body, although that can be brought in also. By focusing on aspects of our body that we normally prefer to ignore, we move towards a healthy sense of acceptance and neutrality towards our body. } ‘In this body there is head hair, body hair, nails, teeth, skin, flesh, sinews, bones, bone marrow, kidneys, heart, liver, diaphragm, spleen, lungs, intestines, mesentery, undigested food, feces, bile, phlegm, pus, blood, sweat, fat, tears, grease, saliva, snot, synovial fluid, urine.’\footnote{Thirty-one parts are mentioned in early texts, later expanded to thirty-two with the addition of the “brain” (\textit{\textsanskrit{matthaluṅga}}). } 

It’s\marginnote{5.3} as if there were a bag with openings at both ends, filled with various kinds of grains, such as fine rice, wheat, mung beans, peas, sesame, and ordinary rice. And someone with clear eyes were to open it and examine the contents: ‘These grains are fine rice, these are wheat, these are mung beans, these are peas, these are sesame, and these are ordinary rice.’\footnote{The “bag with openings at both ends” is the body. Not all the varieties of grains and beans can be positively identified. } 

And\marginnote{5.6} so they meditate observing an aspect of the body internally … 

That\marginnote{5.7} too is how a mendicant meditates by observing an aspect of the body. 

\subsection*{1.5. Focusing on the Elements }

Furthermore,\marginnote{6.1} a mendicant examines their own body, whatever its placement or posture, according to the elements:\footnote{While meditation on the elements is commonly taught in early texts, this phrase is found only in the two \textsanskrit{Satipaṭṭhānasuttas} and the \textsanskrit{Kāyagatāsatisutta}. Detailed instructions are found in such suttas as \href{https://suttacentral.net/mn28/en/sujato}{MN 28} and \href{https://suttacentral.net/mn140/en/sujato}{MN 140}. This meditation works in any posture, whereas breath meditation is best done sitting, to allow the breath to become still. } ‘In this body there is the earth element, the water element, the fire element, and the air element.’\footnote{The “elements” are the four states of matter as represented by their primary material example and their dominant property: earth as a solid with the property of resisting or upholding; water as a liquid with the property of binding; air as a gas with the property of movement; and fire (“heat” or “energy”) as plasma with the property of transformation. } 

It’s\marginnote{6.3} as if a deft butcher or butcher’s apprentice were to kill a cow and sit down at the crossroads with the meat cut into chops.\footnote{This gruesome image shows that butchery of cows was a normal feature of ancient Indian life. } 

And\marginnote{6.6} so they meditate observing an aspect of the body internally … 

That\marginnote{6.7} too is how a mendicant meditates by observing an aspect of the body. 

\subsection*{1.6. The Charnel Ground Contemplations }

Furthermore,\marginnote{7.1} suppose a mendicant were to see a corpse discarded in a charnel ground. And it had been dead for one, two, or three days, bloated, livid, and festering.\footnote{Cremation was expensive and not available to everyone. Bodies might be left in the charnel ground for a variety of reasons, such as local customs, lack of funds, or in cases of inauspicious death such as murder or execution. This is still seen in some places today, and monastics occasionally take the opportunity to practice meditation beside a corpse. However the wording of the Pali sounds like an imaginative exercise. } They’d compare it with their own body:\footnote{The observed corpse is not gendered. The purpose is not to become repulsed by an objectified other, but to understand the mortality of one’s own body. } ‘This body is also of that same nature, that same kind, and cannot go beyond that.’ And so they meditate observing an aspect of the body internally … 

That\marginnote{7.5} too is how a mendicant meditates by observing an aspect of the body. 

Furthermore,\marginnote{8.1} suppose they were to see a corpse discarded in a charnel ground being devoured by crows, hawks, vultures, herons, dogs, tigers, leopards, jackals, and many kinds of little creatures. They’d compare it with their own body: ‘This body is also of that same nature, that same kind, and cannot go beyond that.’ And so they meditate observing an aspect of the body internally … 

That\marginnote{8.5} too is how a mendicant meditates by observing an aspect of the body. 

Furthermore,\marginnote{9.1} suppose they were to see a corpse discarded in a charnel ground, a skeleton with flesh and blood, held together by sinews … 

A\marginnote{9.2} skeleton without flesh but smeared with blood, and held together by sinews … 

A\marginnote{9.3} skeleton rid of flesh and blood, held together by sinews … 

Bones\marginnote{9.4} rid of sinews, scattered in every direction. Here a hand-bone, there a foot-bone, here an ankle bone, there a shin-bone, here a thigh-bone, there a hip-bone, here a rib-bone, there a back-bone, here an arm-bone, there a neck-bone, here a jaw-bone, there a tooth, here the skull. … 

White\marginnote{10.1} bones, the color of shells … 

Decrepit\marginnote{10.2} bones, heaped in a pile … 

Bones\marginnote{10.3} rotted and crumbled to powder.\footnote{It takes decades for bones to rot to powder, again suggesting it is an imaginative contemplation. } They’d compare it with their own body: ‘This body is also of that same nature, that same kind, and cannot go beyond that.’ And so they meditate observing an aspect of the body internally, externally, and both internally and externally. They meditate observing the body as liable to originate, as liable to vanish, and as liable to both originate and vanish. Or mindfulness is established that the body exists, to the extent necessary for knowledge and mindfulness. They meditate independent, not grasping at anything in the world. 

That\marginnote{10.9} too is how a mendicant meditates by observing an aspect of the body. 

\section*{2. Observing the Feelings }

And\marginnote{11.1} how does a mendicant meditate observing an aspect of feelings?\footnote{Literally “a feeling among the feelings”; the practice shows that the meditator contemplates specific feelings as they occur. } 

It’s\marginnote{11.2} when a mendicant who feels a pleasant feeling knows: ‘I feel a pleasant feeling.’\footnote{Pali employs direct quotes to indicate reflexive awareness: you feel the feeling and you know that you feel the feeling. It does not mean that you have to literally say “I feel a pleasant feeling”, although some adopt that as a meditation method. } 

When\marginnote{11.3} they feel a painful feeling, they know: ‘I feel a painful feeling.’ 

When\marginnote{11.4} they feel a neutral feeling, they know: ‘I feel a neutral feeling.’ 

When\marginnote{11.5} they feel a pleasant feeling of the flesh, they know: ‘I feel a pleasant feeling of the flesh.’\footnote{Feelings “of the flesh” (\textit{\textsanskrit{sāmisa}})  are associated with the body and sensual desires (\href{https://suttacentral.net/sn36.31/en/sujato\#4.1}{SN 36.31:4.1}). } 

When\marginnote{11.6} they feel a pleasant feeling not of the flesh, they know: ‘I feel a pleasant feeling not of the flesh.’\footnote{Feelings “not of the flesh” (\textit{\textsanskrit{nirāmisa}}) are associated with renunciation and especially with the \textit{\textsanskrit{jhānas}} and liberation (\href{https://suttacentral.net/sn36.31/en/sujato\#5.1}{SN 36.31:5.1}). } 

When\marginnote{11.7} they feel a painful feeling of the flesh, they know: ‘I feel a painful feeling of the flesh.’ 

When\marginnote{11.8} they feel a painful feeling not of the flesh, they know: ‘I feel a painful feeling not of the flesh.’\footnote{This would include the feelings of loss, doubt, and dejection that can occur during the spiritual path (see \href{https://suttacentral.net/mn44/en/sujato\#28.6}{MN 44:28.6}). } 

When\marginnote{11.9} they feel a neutral feeling of the flesh, they know: ‘I feel a neutral feeling of the flesh.’ 

When\marginnote{11.10} they feel a neutral feeling not of the flesh, they know: ‘I feel a neutral feeling not of the flesh.’\footnote{The feeling of the fourth \textit{\textsanskrit{jhāna}} and higher liberations (\href{https://suttacentral.net/sn36.31/en/sujato\#8.2}{SN 36.31:8.2}). } 

And\marginnote{11.11} so they meditate observing an aspect of feelings internally, externally, and both internally and externally. They meditate observing feelings as liable to originate, as liable to vanish, and as liable to both originate and vanish. Or mindfulness is established that feelings exist, to the extent necessary for knowledge and mindfulness. They meditate independent, not grasping at anything in the world. 

That’s\marginnote{11.14} how a mendicant meditates by observing an aspect of feelings. 

\section*{3. Observing the Mind }

And\marginnote{12.1} how does a mendicant meditate observing an aspect of the mind? 

It’s\marginnote{12.2} when a mendicant understands mind with greed as ‘mind with greed,’\footnote{In Buddhist theory, awareness of the presence or absence of qualities such as greed is explained on three levels. There is the simple happenstance of whether greed is present at that time or not. Then there is the mind freed of greed through the power of absorption. Finally there is the liberation from greed which comes with full awakening. } and mind without greed as ‘mind without greed.’ They understand mind with hate as ‘mind with hate,’ and mind without hate as ‘mind without hate.’ They understand mind with delusion as ‘mind with delusion,’ and mind without delusion as ‘mind without delusion.’ They know constricted mind as ‘constricted mind,’\footnote{The mind is “constricted internally” due to dullness and “scattered externally” due to the distractions of desire (\href{https://suttacentral.net/sn51.20/en/sujato\#18.1}{SN 51.20:18.1}). } and scattered mind as ‘scattered mind.’ They know expansive mind as ‘expansive mind,’\footnote{The following terms “expansive” (\textit{mahaggata}), “supreme” (\textit{anuttara}), “immersed” (\textit{\textsanskrit{samāhita}}), and “freed” (\textit{vimutta}) all refer to states of absorption and/or awakening. } and unexpansive mind as ‘unexpansive mind.’ They know mind that is not supreme as ‘mind that is not supreme,’ and mind that is supreme as ‘mind that is supreme.’ They know mind immersed in \textsanskrit{samādhi} as ‘mind immersed in \textsanskrit{samādhi},’ and mind not immersed in \textsanskrit{samādhi} as ‘mind not immersed in \textsanskrit{samādhi}.’ They know freed mind as ‘freed mind,’ and unfreed mind as ‘unfreed mind.’ 

And\marginnote{12.18} so they meditate observing an aspect of the mind internally, externally, and both internally and externally. They meditate observing the mind as liable to originate, as liable to vanish, and as liable to both originate and vanish. Or mindfulness is established that the mind exists, to the extent necessary for knowledge and mindfulness. They meditate independent, not grasping at anything in the world. 

That’s\marginnote{12.21} how a mendicant meditates by observing an aspect of the mind. 

\section*{4. Observing Principles }

\subsection*{4.1. The Hindrances }

And\marginnote{13.1} how does a mendicant meditate observing an aspect of principles? 

It’s\marginnote{13.2} when a mendicant meditates by observing an aspect of principles with respect to the five hindrances.\footnote{The \textsanskrit{Satipaṭṭhānavibhaṅga} of the Pali Abhidhamma only mentions the hindrances and awakening factors in this section (\href{https://suttacentral.net/vb7}{Vb 7}). This, together with a range of other evidence, suggests that this was the original content of the observation of principles. } And how does a mendicant meditate observing an aspect of principles with respect to the five hindrances? 

It’s\marginnote{13.4} when a mendicant who has sensual desire in them understands: ‘I have sensual desire in me.’ When they don’t have sensual desire in them, they understand: ‘I don’t have sensual desire in me.’ They understand how sensual desire arises; how, when it’s already arisen, it’s given up; and how, once it’s given up, it doesn’t arise again in the future.\footnote{Here causality is introduced. In the contemplation of mind, the meditator was aware of the presence or absence of desire in the mind. Now they look deeper, investigating the cause of desire and understanding how to be free of it forever. This contemplation of the “principles” of cause and effect is the distinctive feature of this section. } 

When\marginnote{13.5} they have ill will in them, they understand: ‘I have ill will in me.’ When they don’t have ill will in them, they understand: ‘I don’t have ill will in me.’ They understand how ill will arises; how, when it’s already arisen, it’s given up; and how, once it’s given up, it doesn’t arise again in the future. 

When\marginnote{13.6} they have dullness and drowsiness in them, they understand: ‘I have dullness and drowsiness in me.’ When they don’t have dullness and drowsiness in them, they understand: ‘I don’t have dullness and drowsiness in me.’ They understand how dullness and drowsiness arise; how, when they’ve already arisen, they’re given up; and how, once they’re given up, they don’t arise again in the future.\footnote{The Buddhist schools debated whether this included physical tiredness or not. The Theravada argued that it was purely a mental laziness, as even the Buddha got sleepy. } 

When\marginnote{13.7} they have restlessness and remorse in them, they understand: ‘I have restlessness and remorse in me.’ When they don’t have restlessness and remorse in them, they understand: ‘I don’t have restlessness and remorse in me.’ They understand how restlessness and remorse arise; how, when they’ve already arisen, they’re given up; and how, once they’re given up, they don’t arise again in the future. 

When\marginnote{13.8} they have doubt in them, they understand: ‘I have doubt in me.’ When they don’t have doubt in them, they understand: ‘I don’t have doubt in me.’ They understand how doubt arises; how, when it’s already arisen, it’s given up; and how, once it’s given up, it doesn’t arise again in the future. 

And\marginnote{13.9} so they meditate observing an aspect of principles internally, externally, and both internally and externally. They meditate observing the principles as liable to originate, as liable to vanish, and as liable to both originate and vanish. Or mindfulness is established that principles exist, to the extent necessary for knowledge and mindfulness. They meditate independent, not grasping at anything in the world. 

That’s\marginnote{13.12} how a mendicant meditates by observing an aspect of principles with respect to the five hindrances. 

\subsection*{4.2. The Aggregates }

Furthermore,\marginnote{14.1} a mendicant meditates by observing an aspect of principles with respect to the five grasping aggregates. And how does a mendicant meditate observing an aspect of principles with respect to the five grasping aggregates? 

It’s\marginnote{14.3} when a mendicant contemplates: Such is form, such is the origin of form, such is the ending of form.\footnote{“Form” (\textit{\textsanskrit{rūpa}}) is one’s own body and the external material world experienced through the senses. More subtly, it represents the “appearance” of physical phenomena, even when experienced solely in the mind as color, visions, etc. } Such is feeling, such is the origin of feeling, such is the ending of feeling. Such is perception, such is the origin of perception, such is the ending of perception.\footnote{“Perception” (\textit{\textsanskrit{saññā}}) is the recognition or interpretation of experience in terms of meaningful wholes. We see, for example, “color” yet we perceive a “person”. In the Vinaya we find many examples where a person perceived things in one way, yet they turned out to be something else. } Such are choices, such is the origin of choices, such is the ending of choices.\footnote{In the five aggregates, \textit{\textsanskrit{saṅkhārā}} is a synonym for “volition” (\textit{\textsanskrit{cetanā}}). The traditions later used it as a catch-all category for everything that does not fit in the other aggregates. In the suttas, however, the purpose of the aggregates is not to classify everything that exists, but to contemplate aspects of experience that we tend to identify as a “self”. } Such is consciousness, such is the origin of consciousness, such is the ending of consciousness.’ And so they meditate observing an aspect of principles internally … 

That’s\marginnote{14.12} how a mendicant meditates by observing an aspect of principles with respect to the five grasping aggregates. 

\subsection*{4.3. The Sense Fields }

Furthermore,\marginnote{15.1} a mendicant meditates by observing an aspect of principles with respect to the six interior and exterior sense fields. And how does a mendicant meditate observing an aspect of principles with respect to the six interior and exterior sense fields? 

It’s\marginnote{15.3} when a mendicant understands the eye, sights, and the fetter that arises dependent on both of these. They understand how the fetter that has not arisen comes to arise; how the arisen fetter comes to be abandoned; and how the abandoned fetter comes to not rise again in the future.\footnote{At \href{https://suttacentral.net/sn35.232/en/sujato\#3.2}{SN 35.232:3.2} the “fetter that arises dependent on both” is identified as “desire and lust” (\textit{\textsanskrit{chandarāga}}). } 

They\marginnote{15.4} understand the ear, sounds, and the fetter … 

They\marginnote{15.5} understand the nose, smells, and the fetter … 

They\marginnote{15.6} understand the tongue, tastes, and the fetter … 

They\marginnote{15.7} understand the body, touches, and the fetter … 

They\marginnote{15.8} understand the mind, ideas, and the fetter that arises dependent on both of these. They understand how the fetter that has not arisen comes to arise; how the arisen fetter comes to be abandoned; and how the abandoned fetter comes to not rise again in the future. 

And\marginnote{15.9} so they meditate observing an aspect of principles internally … 

That’s\marginnote{15.12} how a mendicant meditates by observing an aspect of principles with respect to the six internal and external sense fields. 

\subsection*{4.4. The Awakening Factors }

Furthermore,\marginnote{16.1} a mendicant meditates by observing an aspect of principles with respect to the seven awakening factors.\footnote{These seven factors that lead to awakening (\textit{\textsanskrit{bojjhaṅgā}}, \href{https://suttacentral.net/sn46.5/en/sujato}{SN 46.5}) are commonly presented in opposition to the five hindrances (eg. \href{https://suttacentral.net/sn46.2/en/sujato}{SN 46.2}, \href{https://suttacentral.net/sn46.23/en/sujato}{SN 46.23}, \href{https://suttacentral.net/sn46.55/en/sujato}{SN 46.55}). } And how does a mendicant meditate observing an aspect of principles with respect to the seven awakening factors? 

It’s\marginnote{16.3} when a mendicant who has the awakening factor of mindfulness in them understands: ‘I have the awakening factor of mindfulness in me.’ When they don’t have the awakening factor of mindfulness in them, they understand: ‘I don’t have the awakening factor of mindfulness in me.’ They understand how the awakening factor of mindfulness that has not arisen comes to arise; and how the awakening factor of mindfulness that has arisen becomes fulfilled by development.\footnote{“Mindfulness” includes the recollection of the teachings (\href{https://suttacentral.net/sn46.3/en/sujato\#1.8}{SN 46.3:1.8}) as well as mindfulness meditation. } 

When\marginnote{16.4} they have the awakening factor of investigation of principles …\footnote{Likewise, this includes the inquiry into \textit{dhammas} as “teachings” as well as “phenomena” or “principles”. } energy … rapture … tranquility … immersion … equanimity in them, they understand: ‘I have the awakening factor of equanimity in me.’ When they don’t have the awakening factor of equanimity in them, they understand: ‘I don’t have the awakening factor of equanimity in me.’ They understand how the awakening factor of equanimity that has not arisen comes to arise; and how the awakening factor of equanimity that has arisen becomes fulfilled by development. 

And\marginnote{16.10} so they meditate observing an aspect of principles internally, externally, and both internally and externally. They meditate observing the principles as liable to originate, as liable to vanish, and as liable to both originate and vanish. Or mindfulness is established that principles exist, to the extent necessary for knowledge and mindfulness. They meditate independent, not grasping at anything in the world. 

That’s\marginnote{16.13} how a mendicant meditates by observing an aspect of principles with respect to the seven awakening factors. 

\subsection*{4.5. The Truths }

Furthermore,\marginnote{17.1} a mendicant meditates by observing an aspect of principles with respect to the four noble truths.\footnote{Due to their development of the two wings of \textit{samatha} and \textit{\textsanskrit{vipassanā}} meditation as described in this sutta, practised in the context of the teaching and training as a whole, the meditator realizes the four noble truths at the moment of stream-entry. } And how does a mendicant meditate observing an aspect of principles with respect to the four noble truths? 

It’s\marginnote{17.3} when a mendicant truly understands: ‘This is suffering’ … ‘This is the origin of suffering’ … ‘This is the cessation of suffering’ … ‘This is the practice that leads to the cessation of suffering.’ 

\scendsection{The first recitation section is finished. }

\subsubsection*{4.5.1. The Truth of Suffering }

And\marginnote{18.1} what is the noble truth of suffering?\footnote{The sutta now proceeds in analytical fashion, digging deeper into the details of the four noble truths. The fundamental definitions were taught in the Buddha’s first sermon (\href{https://suttacentral.net/sn56.11/en/sujato\#4.1}{SN 56.11:4.1}). } 

Rebirth\marginnote{18.2} is suffering; old age is suffering; death is suffering; sorrow, lamentation, pain, sadness, and distress are suffering; association with the disliked is suffering; separation from the liked is suffering; not getting what you wish for is suffering. In brief, the five grasping aggregates are suffering. 

And\marginnote{18.3} what is rebirth? The rebirth, inception, conception, reincarnation, manifestation of the sets of phenomena, and acquisition of the sense fields of the various sentient beings in the various orders of sentient beings.\footnote{As at \href{https://suttacentral.net/dn15.4.3/en/sujato}{DN 15.4.3}, \textit{\textsanskrit{jāti}} is invariably defined as the rebirth of beings, not as simple arising. The same applies to old age and death. } This is called rebirth. 

And\marginnote{18.6} what is old age? The old age, decrepitude, broken teeth, grey hair, wrinkly skin, diminished vitality, and failing faculties of the various sentient beings in the various orders of sentient beings. This is called old age. 

And\marginnote{18.9} what is death? The passing away, passing on, disintegration, demise, mortality, death, decease, breaking up of the aggregates, laying to rest of the corpse, and cutting off of the life faculty of the various sentient beings in the various orders of sentient beings. This is called death. 

And\marginnote{18.12} what is sorrow? The sorrow, sorrowing, state of sorrow, inner sorrow, inner deep sorrow in someone who has undergone misfortune, who has experienced suffering.\footnote{In this and following definitions we mainly find mere verbal variations of the basic term. } This is called sorrow. 

And\marginnote{18.15} what is lamentation? The wail, lament, wailing, lamenting, state of wailing and lamentation in someone who has undergone misfortune, who has experienced suffering. This is called lamentation. 

And\marginnote{18.18} what is pain? Physical pain, physical unpleasantness, the painful, unpleasant feeling that’s born from physical contact.\footnote{\textit{Dukkha} (“pain”, “suffering”) is here restricted to physical pain. Elsewhere it may be any kind of painful feeling, while in the four noble truths it is suffering of any sort, including subtle forms of existential suffering. } This is called pain. 

And\marginnote{18.21} what is sadness?\footnote{\textit{Domanassa} normally means “sadness”, but sometimes it contrasts with desire, in which case it is a form of aversion. } Mental pain, mental displeasure, the painful, unpleasant feeling that’s born from mental contact. This is called sadness. 

And\marginnote{18.24} what is distress? The stress, distress, state of stress and distress in someone who has undergone misfortune, who has experienced suffering. This is called distress. 

And\marginnote{18.27} what is meant by ‘association with the disliked is suffering’? There are sights, sounds, smells, tastes, touches, and ideas, which are unlikable, undesirable, and disagreeable. And there are those who want to harm, injure, disturb, and threaten you. The coming together with these, the joining, inclusion, mixing with them: this is what is meant by ‘association with the disliked is suffering’. 

And\marginnote{18.30} what is meant by ‘separation from the liked is suffering’? There are sights, sounds, smells, tastes, touches, and ideas, which are likable, desirable, and agreeable. And there are those who want to benefit, help, comfort, and protect you: mother and father, brother and sister, friends and colleagues, relatives and kin. The division from these, the disconnection, segregation, and parting from them:\footnote{\textit{Piya} often refers to those who are dear and beloved, but as shown here it can also mean simply anything that is liked. } this is what is meant by ‘separation from the liked is suffering’. 

And\marginnote{18.33} what is meant by ‘not getting what you wish for is suffering’? In sentient beings who are liable to be reborn, such a wish arises: ‘Oh, if only we were not liable to be reborn! If only rebirth would not come to us!’\footnote{This is an example of “painful feeling not of the flesh”. The Buddha is here denying the efficacy of prayer, invocation, or magic spells. } But you can’t get that by wishing. This is what is meant by ‘not getting what you wish for is suffering.’ In sentient beings who are liable to grow old … fall ill … die … experience sorrow, lamentation, pain, sadness, and distress, such a wish arises: ‘Oh, if only we were not liable to experience sorrow, lamentation, pain, sadness, and distress! If only sorrow, lamentation, pain, sadness, and distress would not come to us!’ But you can’t get that by wishing. This is what is meant by ‘not getting what you wish for is suffering.’ 

And\marginnote{18.48} what is meant by ‘in brief, the five grasping aggregates are suffering’? They are the grasping aggregates that consist of form, feeling, perception, choices, and consciousness. This is what is meant by ‘in brief, the five grasping aggregates are suffering’. 

This\marginnote{18.51} is called the noble truth of suffering. 

\subsubsection*{4.5.2. The Origin of Suffering }

And\marginnote{19.1} what is the noble truth of the origin of suffering? 

It’s\marginnote{19.2} the craving that leads to future lives, mixed up with relishing and greed, taking pleasure wherever it lands. That is, craving for sensual pleasures, craving for continued existence, and craving to end existence. 

But\marginnote{19.4} where does that craving arise and where does it settle? Whatever in the world seems nice and pleasant, it is there that craving arises and settles. 

And\marginnote{19.6} what in the world seems nice and pleasant? The eye in the world seems nice and pleasant, and it is there that craving arises and settles.\footnote{The following list of properties that relate to the senses gradually moves from the more basic to the more sophisticated. } The ear … nose … tongue … body … mind in the world seems nice and pleasant, and it is there that craving arises and settles. 

Sights\marginnote{19.13} … sounds … smells … tastes … touches … ideas in the world seem nice and pleasant, and it is there that craving arises and settles. 

Eye\marginnote{19.19} consciousness …\footnote{“Eye consciousness” is aware only of “light”; it sees colors but does not interpret them. } ear consciousness … nose consciousness … tongue consciousness … body consciousness … mind consciousness in the world seems nice and pleasant, and it is there that craving arises and settles. 

Eye\marginnote{19.25} contact …\footnote{“Contact” or “stimulus” happens when the sense base, the sense object, and the sense consciousness all occur together. } ear contact … nose contact … tongue contact … body contact … mind contact in the world seems nice and pleasant, and it is there that craving arises and settles. 

Feeling\marginnote{19.31} born of eye contact … feeling born of ear contact … feeling born of nose contact … feeling born of tongue contact … feeling born of body contact … feeling born of mind contact in the world seems nice and pleasant, and it is there that craving arises and settles. 

Perception\marginnote{19.37} of sights …\footnote{“Perception” interprets the “light” that is seen, organizing it in meaningful wholes. For example, the eye sees the color white, while perception recognizes that it is “white”, and further, that that white color is in fact a “wall”. } perception of sounds … perception of smells … perception of tastes … perception of touches … perception of ideas in the world seems nice and pleasant, and it is there that craving arises and settles. 

Intention\marginnote{19.43} regarding sights …\footnote{To continue the example, once the light has been interpreted by perception as a “wall”, we then make the choice to walk around it rather than through it. Choices therefore depend on perceptions. } intention regarding sounds … intention regarding smells … intention regarding tastes … intention regarding touches … intention regarding ideas in the world seems nice and pleasant, and it is there that craving arises and settles. 

Craving\marginnote{19.49} for sights … craving for sounds … craving for smells … craving for tastes … craving for touches … craving for ideas in the world seems nice and pleasant, and it is there that craving arises and settles. 

Thoughts\marginnote{19.55} about sights …\footnote{“Thought” is \textit{vitakka}. } thoughts about sounds … thoughts about smells … thoughts about tastes … thoughts about touches … thoughts about ideas in the world seem nice and pleasant, and it is there that craving arises and settles. 

Considerations\marginnote{19.61} regarding sights …\footnote{“Consideration” is \textit{\textsanskrit{vicāra}}, a more sustained exercize of thought. } considerations regarding sounds … considerations regarding smells … considerations regarding tastes … considerations regarding touches … considerations regarding ideas in the world seem nice and pleasant, and it is there that craving arises and settles. 

This\marginnote{19.67} is called the noble truth of the origin of suffering. 

\subsubsection*{4.5.3. The Cessation of Suffering }

And\marginnote{20.1} what is the noble truth of the cessation of suffering? 

It’s\marginnote{20.2} the fading away and cessation of that very same craving with nothing left over; giving it away, letting it go, releasing it, and not clinging to it. 

Whatever\marginnote{20.3} in the world seems nice and pleasant, it is there that craving is given up and ceases. 

And\marginnote{20.5} what in the world seems nice and pleasant? The eye in the world seems nice and pleasant, and it is there that craving is given up and ceases. … 

Considerations\marginnote{20.60} regarding ideas in the world seem nice and pleasant, and it is there that craving is given up and ceases. 

This\marginnote{20.66} is called the noble truth of the cessation of suffering. 

\subsubsection*{4.5.4. The Path }

And\marginnote{21.1} what is the noble truth of the practice that leads to the cessation of suffering? 

It\marginnote{21.2} is simply this noble eightfold path, that is: right view, right thought, right speech, right action, right livelihood, right effort, right mindfulness, and right immersion.\footnote{Mindfulness is not a path in and of itself, but rather is the seventh factor of the eightfold path. } 

And\marginnote{21.4} what is right view? Knowing about suffering, the origin of suffering, the cessation of suffering, and the practice that leads to the cessation of suffering.\footnote{The fourth noble truth is the path, while the first path factor is the noble truths. These two teachings are different perspectives on the same \textit{dhamma}. } This is called right view. 

And\marginnote{21.7} what is right thought? Thoughts of renunciation, good will, and harmlessness.\footnote{\textit{\textsanskrit{Saṅkappa}} is normally a synonym of \textit{vitakka} in the suttas, hence the rendering “right thought”. It is, however, not just verbalized thought, but the direction in which one applies the mind. This factor is the emotional counterpart of right view, ensuring that the path is motivated by love and compassion. } This is called right thought. 

And\marginnote{21.10} what is right speech? The refraining from lying, divisive speech, harsh speech, and talking nonsense. This is called right speech. 

And\marginnote{21.13} what is right action? Refraining from killing living creatures, stealing, and sexual misconduct.\footnote{The first three of the five precepts. “Sexual misconduct” is the betrayal of trust in a sexual relationship. } This is called right action. 

And\marginnote{21.16} what is right livelihood? It’s when a noble disciple gives up wrong livelihood and earns a living by right livelihood.\footnote{Defined for a monastic in the long section on ethics in the \textsanskrit{Sāmaññaphalasutta} (\href{https://suttacentral.net/dn2/en/sujato\#56.1}{DN 2:56.1}) and for a lay person as trade in weapons, living creatures, meat, intoxicants, and poisons (\href{https://suttacentral.net/an5.177/en/sujato\#1.3}{AN 5.177:1.3}). } This is called right livelihood. 

And\marginnote{21.19} what is right effort? It’s when a mendicant generates enthusiasm, tries, makes an effort, exerts the mind, and strives so that bad, unskillful qualities don’t arise. They generate enthusiasm, try, make an effort, exert the mind, and strive so that bad, unskillful qualities that have arisen are given up. They generate enthusiasm, try, make an effort, exert the mind, and strive so that skillful qualities arise. They generate enthusiasm, try, make an effort, exert the mind, and strive so that skillful qualities that have arisen remain, are not lost, but increase, mature, and are completed by development. This is called right effort. 

And\marginnote{21.25} what is right mindfulness? It’s when a mendicant meditates by observing an aspect of the body—keen, aware, and mindful, rid of covetousness and displeasure for the world. They meditate observing an aspect of feelings—keen, aware, and mindful, rid of covetousness and displeasure for the world. They meditate observing an aspect of the mind—keen, aware, and mindful, rid of covetousness and displeasure for the world. They meditate observing an aspect of principles—keen, aware, and mindful, rid of covetousness and displeasure for the world. This is called right mindfulness. 

And\marginnote{21.31} what is right immersion? It’s when a mendicant, quite secluded from sensual pleasures, secluded from unskillful qualities, enters and remains in the first absorption, which has the rapture and bliss born of seclusion, while placing the mind and keeping it connected. As the placing of the mind and keeping it connected are stilled, they enter and remain in the second absorption, which has the rapture and bliss born of immersion, with internal clarity and mind at one, without placing the mind and keeping it connected. And with the fading away of rapture, they enter and remain in the third absorption, where they meditate with equanimity, mindful and aware, personally experiencing the bliss of which the noble ones declare, ‘Equanimous and mindful, one meditates in bliss.’ Giving up pleasure and pain, and ending former happiness and sadness, they enter and remain in the fourth absorption, without pleasure or pain, with pure equanimity and mindfulness. This is called right immersion. 

This\marginnote{21.37} is called the noble truth of the practice that leads to the cessation of suffering. 

And\marginnote{21.38} so they meditate observing an aspect of principles internally, externally, and both internally and externally. They meditate observing the principles as liable to originate, as liable to vanish, and as liable to both originate and vanish. Or mindfulness is established that principles exist, to the extent necessary for knowledge and mindfulness. They meditate independent, not grasping at anything in the world. 

That’s\marginnote{21.41} how a mendicant meditates by observing an aspect of principles with respect to the four noble truths. 

Anyone\marginnote{22.1} who develops these four kinds of mindfulness meditation in this way for seven years can expect one of two results:\footnote{The emphasis is on “develop in this way” (\textit{\textsanskrit{evaṁ} \textsanskrit{bhāveyya}}), that is, with the full practice including deep absorption as the culmination of the path as a whole. } enlightenment in this very life, or if there’s something left over, non-return. 

Let\marginnote{22.3} alone seven years,\footnote{A similar promise of results in at most seven years is found at \href{https://suttacentral.net/dn25/en/sujato\#22.9}{DN 25:22.9}, \href{https://suttacentral.net/mn10/en/sujato\#46.3}{MN 10:46.3}, and \href{https://suttacentral.net/mn85/en/sujato\#59.3}{MN 85:59.3}; and at most ten years at \href{https://suttacentral.net/an10.46/en/sujato\#7.3}{AN 10.46:7.3}. } anyone who develops these four kinds of mindfulness meditation in this way for six years … five years … four years … three years … two years … one year … seven months … six months … five months … four months … three months … two months … one month … a fortnight … Let alone a fortnight, anyone who develops these four kinds of mindfulness meditation in this way for seven days can expect one of two results: enlightenment in this very life, or if there’s something left over, non-return. 

‘The\marginnote{22.24} four kinds of mindfulness meditation are the path to convergence. They are in order to purify sentient beings, to get past sorrow and crying, to make an end of pain and sadness, to discover the system, and to realize extinguishment.’ That’s what I said, and this is why I said it.” 

That\marginnote{22.26} is what the Buddha said. Satisfied, the mendicants approved what the Buddha said. 

%
\chapter*{{\suttatitleacronym DN 23}{\suttatitletranslation With Pāyāsi }{\suttatitleroot Pāyāsisutta}}
\addcontentsline{toc}{chapter}{\tocacronym{DN 23} \toctranslation{With Pāyāsi } \tocroot{Pāyāsisutta}}
\markboth{With Pāyāsi }{Pāyāsisutta}
\extramarks{DN 23}{DN 23}

\scevam{So\marginnote{1.1} I have heard. }At one time Venerable Kassapa the Prince was wandering in the land of the Kosalans together with a large \textsanskrit{Saṅgha} of five hundred mendicants when he arrived at a Kosalan citadel named \textsanskrit{Setavyā}.\footnote{\textsanskrit{Kumārakassapa} was ordained at twenty (\href{https://suttacentral.net/pli-tv-kd1/en/sujato\#75.1.1}{Kd 1:75.1.1}). He features in the Vammikasutta (\href{https://suttacentral.net/mn23/en/sujato}{MN 23}), and his verses are collected in the \textsanskrit{Theragāthā} (\href{https://suttacentral.net/thag2.41/en/sujato}{Thag 2.41}). He was declared the foremost of those with brilliant speech (\href{https://suttacentral.net/an1.217/en/sujato}{AN 1.217}), apparently on the basis of this discourse. | The \textsanskrit{Pāyāsisuta} is the only major Buddhist text that has a Jain parallel, Paesi-\textsanskrit{kahāṇayaṁ}, a similar dialogue between Paesi and \textsanskrit{Keśin} in \textsanskrit{Setavyā}. } He stayed in the grove of Indian Rosewood to the north of \textsanskrit{Setavyā}.\footnote{\textsanskrit{Setavyā} was north-east of \textsanskrit{Sāvatthī}. } 

Now\marginnote{1.4} at that time the chieftain \textsanskrit{Pāyāsi} was living in \textsanskrit{Setavyā}. It was a crown property given by King Pasenadi of Kosala, teeming with living creatures, full of hay, wood, water, and grain, a royal park endowed to a brahmin.\footnote{\textsanskrit{Pāyāsi}  was a \textit{khattiya} yet he receives a \textit{brahmadeyya}. The sense of \textit{brahmadeyya} as a donation to brahmins is well attested in inscriptions, so I think this is likely a mistake in the text, rather than evidence that the practice was not for brahmins only. } 

\section*{1. On \textsanskrit{Pāyāsi} }

Now\marginnote{2.1} at that time \textsanskrit{Pāyāsi} had the following harmful misconception: “There is no afterlife. No beings are reborn spontaneously. There’s no fruit or result of good and bad deeds.”\footnote{This is wrong view per \href{https://suttacentral.net/mn117/en/sujato\#5.1}{MN 117:5.1}. } 

The\marginnote{2.3} brahmins and householders of \textsanskrit{Setavyā} heard, “It seems the ascetic Kassapa the Prince—a disciple of the ascetic Gotama—is staying in the grove of Indian Rosewood to the north of \textsanskrit{Setavyā}. He has this good reputation: ‘He is astute, competent, intelligent, learned, a brilliant speaker, eloquent, mature, a perfected one.’ It’s good to see such perfected ones.” Then, having departed \textsanskrit{Setavyā}, they formed into companies and headed north to the grove. 

Now\marginnote{3.1} at that time the chieftain \textsanskrit{Pāyāsi} had retired to the upper floor of his stilt longhouse for his midday nap. He saw the brahmins and householders heading north towards the grove, and addressed his butler, “My butler, why are the brahmins and householders heading north towards the grove?” 

“The\marginnote{3.5} ascetic Kassapa the Prince—a disciple of the ascetic Gotama—is staying in the grove of Indian Rosewood to the north of \textsanskrit{Setavyā}. He has this good reputation: ‘He is astute, competent, intelligent, learned, a brilliant speaker, eloquent, mature, a perfected one.’ They’re going to see that Kassapa the Prince.” 

“Well\marginnote{3.9} then, go to the brahmins and householders and say to them: ‘Sirs, the chieftain \textsanskrit{Pāyāsi} asks you to wait, as he will also go to see the ascetic Kassapa the Prince.’ Before Kassapa the Prince convinces those foolish and incompetent brahmins and householders that there is an afterlife, there are beings reborn spontaneously, and there is a fruit or result of good and bad deeds—for none of these things are true!” 

“Yes,\marginnote{3.15} sir,” replied the butler, and did as he was asked. 

Then\marginnote{4.1} \textsanskrit{Pāyāsi} escorted by the brahmins and householders, went up to Kassapa the Prince, and exchanged greetings with him. When the greetings and polite conversation were over, he sat down to one side. Before sitting down to one side, some of the brahmins and householders of \textsanskrit{Setavyā} bowed, some exchanged greetings and polite conversation, some held up their joined palms toward Kassapa the Prince, some announced their name and clan, while some kept silent. 

\section*{2. Nihilism }

Seated\marginnote{5.1} to one side, the chieftain \textsanskrit{Pāyāsi} said to Venerable Kassapa the Prince, “Mister Kassapa, this is my doctrine and view: ‘There is no afterlife. No beings are reborn spontaneously. There’s no fruit or result of good and bad deeds.’” 

“Chieftain,\marginnote{5.4} may I never see or hear of anyone holding such a doctrine or view!\footnote{This idiom recurs at \href{https://suttacentral.net/an6.38/en/sujato\#1.5}{AN 6.38:1.5} where, as here, it refers to a well-known view and cannot mean that they have never heard of it. There the text and commentary read \textit{\textsanskrit{māhaṁ}}, which with the aorist is prohibitive not negatory. It is an idiom expressing dislike. } For how on earth can anyone say such a thing? 

\subsection*{2.1. The Simile of the Moon and Sun }

Well\marginnote{5.8} then, chieftain, I’ll ask you about this in return, and you can answer as you like. What do you think, chieftain? Are the moon and sun in this world or the other world? Are they gods or humans?” 

“They\marginnote{5.11} are in the other world, Mister Kassapa, and they are gods, not humans.”\footnote{While the world “down here” is bogged down in its messy and ephemeral issues, the heavenly bodies proceed in their serene, glorious, stately indifference. Ignoring gravity, self-luminous, and apparently eternal, they operate according to what appears to be an entirely different set of rules, a “meta-physics”. What Newton did in physics the Buddha did in spirituality: show that the heavens, despite appearances, operate with the same set of conditioned rules as apply down here. } 

“By\marginnote{5.12} this method it ought to be proven that\footnote{Kassapa’s argument here doesn’t directly establish his conclusion. It is possible that gods exist on an entirely separate plane that has nothing to do with kamma and rebirth. } there is an afterlife, there are beings reborn spontaneously, and there is a fruit or result of good and bad deeds.” 

“Even\marginnote{6.1} though Mister Kassapa says this, still I think that there is no afterlife, no beings are reborn spontaneously, and there’s no fruit or result of good and bad deeds.” 

“Is\marginnote{6.3} there a method by which you can prove what you say?” 

“There\marginnote{6.5} is, Mister Kassapa.” 

“How,\marginnote{6.7} exactly, chieftain?” 

“Well,\marginnote{6.8} I have friends and colleagues, relatives and kin who kill living creatures, steal, and commit sexual misconduct. They use speech that’s false, divisive, harsh, or nonsensical. And they’re covetous, malicious, with wrong view. Some time later they become sick, suffering, gravely ill. When I know that they will not recover from their illness, I go to them and say, ‘Sirs, there are some ascetics and brahmins who have this doctrine and view: “Those who kill living creatures, steal, and commit sexual misconduct; use speech that’s false, divisive, harsh, or nonsensical; and are covetous, malicious, and have wrong view—when their body breaks up, after death, are reborn in a place of loss, a bad place, the underworld, hell.” You do all these things. If what those ascetics and brahmins say is true, when your body breaks up, after death, you’ll be reborn in a place of loss, a bad place, the underworld, hell. If that happens, sirs, come and tell me that there is an afterlife, there are beings reborn spontaneously, and there is a fruit or result of good and bad deeds. I trust you and believe you. Anything you see will be just as if I’ve seen it for myself.’\footnote{This is still a major factor in shaping belief. People will reject the opinions of experts and believe people that they know. } They agree to this. But they don’t come back to tell me, nor do they send a messenger. This is the method by which I prove that there is no afterlife, no beings are reborn spontaneously, and there’s no fruit or result of good and bad deeds.”\footnote{Compare \textsanskrit{Jaiminīya} \textsanskrit{Brāhmaņa} 1.42, where \textsanskrit{Varuṇa} asphyxiated his son \textsanskrit{Bhṛgu} in order to send him on a journey to the “other world”, inducing a near-death experience. \textsanskrit{Bhṛgu} saw men cutting other men to pieces and eating them, and other sights both horrifying and beautiful, all the while wondering if what he saw was real, before his breath was returned to him. } 

\subsection*{2.2. The Simile of the Bandit }

“Well\marginnote{7.1} then, chieftain, I’ll ask you about this in return, and you can answer as you like. What do you think, chieftain? Suppose they were to arrest a bandit, a criminal and present him to you, saying, ‘Sir, this is a bandit, a criminal. Punish him as you will.’ Then you’d say to them, ‘Well then, my men, tie this man’s arms tightly behind his back with a strong rope. Shave his head and march him from street to street and square to square to the beating of a harsh drum. Then take him out the south gate and there, at the place of execution to the south of the city, chop off his head.’ Saying, ‘Good,’ they’d do as they were told, sitting him down at the place of execution. Could that bandit get the executioners to wait, saying, ‘Please, good executioners! I have friends and colleagues, relatives and kin in such and such village or town. Wait until I’ve visited them, then I’ll come back’? Or would they just chop off his head as he prattled on?”\footnote{Follow PTS reading \textit{\textsanskrit{uddassetvā}}, which at \href{https://suttacentral.net/mn82/en/sujato\#11.9}{MN 82:11.9} has the sense “visit”. } 

“They’d\marginnote{7.13} just chop off his head.” 

“So\marginnote{7.14} even a human bandit couldn’t get his human executioners to stay his execution. What then of your friends and colleagues, relatives and kin who are reborn in a lower realm after doing bad things? Could they get the wardens of hell to wait, saying, ‘Please, good wardens of hell! Wait until I’ve gone to the chieftain \textsanskrit{Pāyāsi} to tell him that there is an afterlife, there are beings reborn spontaneously, and there is a fruit or result of good and bad deeds’? By this method, too, it ought to be proven that\footnote{Kassapa, however, has not established the existence of an afterlife, he has merely refuted \textsanskrit{Pāyāsi}’s argument. Per Occam’s razor, the burden of proof lies on the one who wishes to establish the existence of the afterlife, not on the one who denies it. His arguments, however, become more persuasive if they are understood as building on the initial agreement on the divinity of the sun and moon. He knows that \textsanskrit{Pāyāsi} accepts some form of other world, even if he says otherwise, so the argument hinges on whether \textsanskrit{Pāyāsi}’s methods are sufficient to \emph{disprove} the kind of other world that Kassapa proposes, i.e. one driven by kamma. } there is an afterlife, there are beings reborn spontaneously, and there is a fruit or result of good and bad deeds.” 

“Even\marginnote{8.1} though Mister Kassapa says this, still I think that there is no afterlife.” 

“Is\marginnote{8.3} there a method by which you can prove what you say?” 

“There\marginnote{8.5} is, Mister Kassapa.” 

“How,\marginnote{8.7} exactly, chieftain?” 

“Well,\marginnote{8.8} I have friends and colleagues, relatives and kin who refrain from killing living creatures, stealing, and committing sexual misconduct. They refrain from speech that’s false, divisive, harsh, or nonsensical. And they’re content, kind-hearted, with right view. Some time later they become sick, suffering, gravely ill. When I know that they will not recover from their illness, I go to them and say, ‘Sirs, there are some ascetics and brahmins who have this doctrine and view: “Those who refrain from killing living creatures, stealing, and committing sexual misconduct; who refrain from speech that’s false, divisive, harsh, or nonsensical; and are content, kind-hearted, with right view—when their body breaks up, after death, are reborn in a good place, a heavenly realm.” You do all these things. If what those ascetics and brahmins say is true, when your body breaks up, after death, you’ll be reborn in a good place, a heavenly realm. If that happens, sirs, come and tell me that there is an afterlife. I trust you and believe you. Anything you see will be just as if I’ve seen it for myself.’ They agree to this. But they don’t come back to tell me, nor do they send a messenger. This is the method by which I prove that there is no afterlife.” 

\subsection*{2.3. The Simile of the Sewer }

“Well\marginnote{9.1} then, chieftain, I shall give you a simile. For by means of a simile some sensible people understand the meaning of what is said. Suppose there were a man sunk over his head in a sewer. Then you were to order someone to pull him out of the sewer, and they’d agree to do so. Then you’d tell them to carefully scrape the dung off that man’s body with bamboo scrapers, and they’d agree to do so. Then you’d tell them to carefully scrub that man’s body down with pale clay three times, and they’d do so. Then you’d tell them to smear that man’s body with oil, and carefully wash him down with fine paste three times, and they’d do so. Then you’d tell them to dress that man’s hair and beard, and they’d do so. Then you’d tell them to provide that man with costly garlands, makeup, and clothes, and they’d do so. Then you’d tell them to bring that man up to the stilt longhouse and set him up with the five kinds of sensual stimulation, and they’d do so. 

What\marginnote{9.25} do you think, chieftain? Now that man is nicely bathed and anointed, with hair and beard dressed, bedecked with garlands and bracelets, dressed in white, supplied and provided with the five kinds of sensual stimulation upstairs in the royal longhouse. Would he want to dive back into that sewer again?” 

“No,\marginnote{9.27} Mister Kassapa. Why is that? Because that sewer is filthy, stinking, disgusting, and repulsive, and it’s regarded as such.” 

“In\marginnote{9.30} the same way, chieftain, to the gods, human beings are filthy, stinking, disgusting, and repulsive, and are regarded as such. The smell of humans reaches the gods even a hundred leagues away.\footnote{Follow PTS reading \textit{\textsanskrit{ubbāhati}}. Cf. \href{https://suttacentral.net/an3.93/en/sujato\#6.4}{AN 3.93:6.4} for the sense “transport”. } What then of your friends and colleagues, relatives and kin who are reborn in a higher realm after doing good things? Will they come back to tell you that there is an afterlife? By this method, too, it ought to be proven that there is an afterlife.” 

“Even\marginnote{10.1} though Mister Kassapa says this, still I think that there is no afterlife.” 

“Can\marginnote{10.3} you prove it?” 

“I\marginnote{10.4} can.” 

“How,\marginnote{10.5} exactly, chieftain?” 

“Well,\marginnote{10.6} I have friends and colleagues, relatives and kin who refrain from killing living creatures and so on. Some time later they become sick, suffering, gravely ill. When I know that they will not recover from their illness, I go to them and say, ‘Sirs, there are some ascetics and brahmins who have this doctrine and view: “Those who refrain from killing living creatures and so on are reborn in a good place, a heavenly realm, in the company of the gods of the thirty-three.” You do all these things. If what those ascetics and brahmins say is true, when your body breaks up, after death, you’ll be reborn in the company of the gods of the thirty-three. If that happens, sirs, come and tell me that there is an afterlife. I trust you and believe you. Anything you see will be just as if I’ve seen it for myself.’ They agree to this. But they don’t come back to tell me, nor do they send a messenger. This is how I prove that there is no afterlife.” 

\subsection*{2.4. The Simile of the gods of the thirty-three }

“Well\marginnote{11.1} then, chieftain, I’ll ask you about this in return, and you can answer as you like. A hundred human years are equivalent to one day and night for the gods of the thirty-three. Thirty such days make a month, and twelve months make a year. The gods of the thirty-three have a lifespan of a thousand such years.\footnote{36,000,000 years. } Now, as to your friends who are reborn in the company of the gods of the thirty-three after doing good things. If they think, ‘First I’ll amuse myself for two or three days, supplied and provided with the five kinds of heavenly sensual stimulation. Then I’ll go back to \textsanskrit{Pāyāsi} and tell him that there is an afterlife.’ Would they come back to tell you that there is an afterlife?” 

“No,\marginnote{11.9} Mister Kassapa. For I would be long dead by then.\footnote{This recognizes the relativity of time. } But Mister Kassapa, who has told you that the gods of the thirty-three exist, or that they have such a long lifespan? I don’t believe you.”\footnote{\textsanskrit{Pāyāsi} makes a good point; he only relies on sources that he knows he can trust. } 

\subsection*{2.5. Blind From Birth }

“Chieftain,\marginnote{11.16} suppose there was a person blind from birth. They couldn’t see sights that are dark or bright, or blue, yellow, red, or magenta. They couldn’t see even and uneven ground, or the stars, or the moon and sun. They’d say, ‘There’s no such thing as dark and bright sights, and no-one who sees them. There’s no such thing as blue, yellow, red, magenta, even and uneven ground, stars, moon and sun, and no-one who sees these things. I don’t know it or see it, therefore it doesn’t exist.’ Would they be speaking rightly?” 

“No,\marginnote{11.28} Mister Kassapa. There are such things as dark and bright sights, and one who sees them. And those other things are real, too, as is the one who sees them. So it’s not right to say this: ‘I don’t know it or see it, therefore it doesn’t exist.’” 

“In\marginnote{11.36} the same way, chieftain, when you tell me you don’t believe me you seem like the blind man in the simile. You can’t see the other world the way you think, with the eye of the flesh.\footnote{Kassapa establishes the empirical method by which these truths are known. Science extends knowledge by means of external instruments, while meditation extends the scope of consciousness. A non-scientist cannot understand how a scientist establishes their conclusions, and can only rely on trust in the scientific establishment. Likewise a non-meditator cannot understand the capacity of expanded consciousness. } There are ascetics and brahmins who live in the wilderness, frequenting remote lodgings in the wilderness and the forest. Meditating diligent, keen, and resolute, they purify the heavenly eye, the power of clairvoyance. With clairvoyance that is purified and superhuman, they see this world and the other world, and sentient beings who are spontaneously reborn. That’s how to see the other world, not how you think, with the eye of the flesh. By this method, too, it ought to be proven that there is an afterlife.” 

“Even\marginnote{12.1} though Mister Kassapa says this, still I think that there is no afterlife.” 

“Can\marginnote{12.3} you prove it?” 

“I\marginnote{12.4} can.” 

“How,\marginnote{12.5} exactly, chieftain?” 

“Well,\marginnote{12.6} I see ascetics and brahmins who are ethical, of good character, who want to live and don’t want to die, who want to be happy and recoil from pain. I think to myself, ‘If those ascetics and brahmins knew that things were going to be better for them after death, they’d drink poison, take their lives, hang themselves, or throw themselves off a cliff. They mustn’t know that things are going to be better for them after death. That’s why they are ethical, of good character, wanting to live and not wanting to die, wanting to be happy and recoiling from pain.’ This is the method by which I prove that there is no afterlife.” 

\subsection*{2.6. The Simile of the Pregnant Woman }

“Well\marginnote{13.1} then, chieftain, I shall give you a simile. For by means of a simile some sensible people understand the meaning of what is said. 

Once\marginnote{13.3} upon a time, a certain brahmin had two wives. One had a son ten or twelve years of age, while the other was pregnant and about to give birth. Then the brahmin passed away. 

So\marginnote{13.6} the youth said to his mother’s co-wife, ‘Madam, all the money, grain, silver, and gold is mine, and you get nothing. Transfer to me my father’s inheritance.’ 

But\marginnote{13.10} the brahmin lady said, ‘Wait, my dear, until I give birth. If it’s a boy, one portion shall be his. If it’s a girl, she will be your reward.’\footnote{The commentary explains \textit{\textsanskrit{opabhoggā}} (“reward”) as \textit{\textsanskrit{pādaparicārikā}} (“wife”). } 

But\marginnote{13.14} for a second time, and a third time, the youth insisted that the entire inheritance must be his. 

So\marginnote{13.26} the brahmin lady took a knife, went to her bedroom, and sliced open her belly, thinking,\footnote{Read \textit{\textsanskrit{opāṭesi}}. } ‘Until I give birth—whether it’s a boy or a girl!’\footnote{Prefer \textsanskrit{Mahāsaṅgīti} reading \textit{\textsanskrit{yāva} \textsanskrit{vijāyāmi}} over PTS \textit{\textsanskrit{yāva} \textsanskrit{jānāmi}} (“until I know”); it echoes \textit{\textsanskrit{yāva} \textsanskrit{vijāyāmi}} above. The phrase is not fully coherent, which is understandable given the circumstances. } She destroyed her own life and that of the fetus, as well as any wealth. 

Being\marginnote{13.29} foolish and incompetent, she sought an inheritance irrationally and fell to ruin and disaster. In the same way, chieftain, being foolish and incompetent, you’re seeking the other world irrationally and will fall to ruin and disaster,\footnote{“Irrationally” is \textit{ayoniso}, literally “not sourcewise”. This passage gives a nice real world example of what it means: the means employed are unrelated to the end sought. } just like that brahmin lady. Good ascetics and brahmins don’t force what is unripe to ripen; rather, they wait for it to ripen. For the life of clever ascetics and brahmins is beneficial. So long as they remain, good ascetics and brahmins produce much merit, and act for the welfare and happiness of the people, out of sympathy for the world, for the benefit, welfare, and happiness of gods and humans. By this method, too, it ought to be proven that there is an afterlife.” 

“Even\marginnote{14.1} though Mister Kassapa says this, still I think that there is no afterlife.” 

“Can\marginnote{14.3} you prove it?” 

“I\marginnote{14.4} can.” 

“How,\marginnote{14.5} exactly, chieftain?” 

“Suppose\marginnote{14.6} they were to arrest a bandit, a criminal and present him to me, saying, ‘Sir, this is a bandit, a criminal. Punish him as you will.’ I say to them, ‘Well then, sirs, place this man in a pot while he’s still alive. Close up the mouth, bind it up with damp leather, and seal it with a thick coat of damp clay. Then lift it up on a stove and light the fire.’\footnote{\textsanskrit{Pāyāsi}’s experiments were cruel, but no more so than many recorded in recent history. } They agree, and do what I ask. When we know that that man has passed away, we lift down the pot and break it open, uncover the mouth, and slowly peek inside, thinking, ‘Hopefully we’ll see his soul escaping.’\footnote{We assume that a soul must be immaterial and invisible, but clearly this was not always the case at the time. From \href{https://suttacentral.net/dn1/en/sujato}{DN 1} we know that there was an almost inexhaustible variety of views about the self or soul. } But we don’t see his soul escaping. This is how I prove that there is no afterlife.” 

\subsection*{2.7. The Simile of the Dream }

“Well\marginnote{15.1} then, chieftain, I’ll ask you about this in return, and you can answer as you like. Do you recall ever having a midday nap and seeing delightful parks, woods, meadows, and lotus ponds in a dream?”\footnote{This echoes \textsanskrit{Upaniṣadic} discussions of the nature of the dream state and its relation to death. See for example \textsanskrit{Bṛhadāraṇyaka} \textsanskrit{Upaniṣad} 2.1.18, which says one might become a brahmin or king in a dream. } 

“I\marginnote{15.3} do, sir.” 

“At\marginnote{15.4} that time were you guarded by hunchbacks, dwarves, midgets, and younglings?” 

“I\marginnote{15.5} was.” 

“But\marginnote{15.6} did they see your soul entering or leaving?”\footnote{\textsanskrit{Bṛhadāraṇyaka} \textsanskrit{Upaniṣad} 4.3.19 describes the soul returning to the body like a tired hawk returning to the nest. } 

“No\marginnote{15.7} they did not.” 

“So\marginnote{15.8} if they couldn’t even see your soul entering or leaving while you were still alive, how could you see the soul of a dead man? By this method, too, it ought to be proven that there is an afterlife, there are beings reborn spontaneously, and there is a fruit or result of good and bad deeds.” 

“Even\marginnote{16.1} though Mister Kassapa says this, still I think that there is no afterlife.” 

“Can\marginnote{16.3} you prove it?” 

“I\marginnote{16.4} can.” 

“How,\marginnote{16.5} exactly, chieftain?” 

“Suppose\marginnote{16.6} they were to arrest a bandit, a criminal and present him to me, saying, ‘Sir, this is a bandit, a criminal. Punish him as you will.’ I say to them, ‘Well then, sirs, weigh this man with scales while he’s still alive. Then strangle him with a bowstring, and when he’s dead, weigh him again.’\footnote{Aside from the cruelty, this echoes the dictum attributed to Galileo: “Measure what is measurable, and make measurable what is not so.” Given sufficient precision, this method could be effective in testing for the existence for a physical soul that has mass. } They agree, and do what I ask. So long as they are alive, they’re lighter, softer, more flexible. But when they die they become heavier, stiffer, less flexible.\footnote{A dead body will, if anything, weigh less due to excretion and dehydration. } This is how I prove that there is no afterlife.” 

\subsection*{2.8. The Simile of the Hot Iron Ball }

“Well\marginnote{17.1} then, chieftain, I shall give you a simile. For by means of a simile some sensible people understand the meaning of what is said. Suppose a person was to heat an iron ball all day until it was burning, blazing, and glowing, and then they weigh it with scales. After some time, when it had cooled and become quenched, they’d weigh it again. When would that iron ball be lighter, softer, and more workable—when it’s burning or when it’s cool?”\footnote{The misconception that iron is lighter when heated is repeated elsewhere (eg. \href{https://suttacentral.net/sn51.22/en/sujato\#4.1}{SN 51.22:4.1}). In fact, assuming there are no chemical reactions, it will be very slightly heavier due to relativistic effects, yet less dense and hence more buoyant. These changes are too small to be detected by \textsanskrit{Pāyāsi}’s methods, though. } 

“So\marginnote{17.6} long as the iron ball is full of heat and air—burning, blazing, and glowing—it’s lighter, softer, and more workable. But when it lacks heat and air—cooled and quenched—it’s heavier, stiffer, and less workable.” 

“In\marginnote{17.8} the same way, so long as this body is full of life and warmth and consciousness it’s lighter, softer, and more flexible. But when it lacks life and warmth and consciousness it’s heavier, stiffer, and less flexible. By this method, too, it ought to be proven that there is an afterlife.” 

“Even\marginnote{18.1} though Mister Kassapa says this, still I think that there is no afterlife.” 

“Can\marginnote{18.3} you prove it?” 

“I\marginnote{18.4} can.” 

“How,\marginnote{18.5} exactly, chieftain?” 

“Suppose\marginnote{18.6} they were to arrest a bandit, a criminal and present him to me, saying, ‘Sir, this is a bandit, a criminal. Punish him as you will.’ I say to them, ‘Well then, sirs, take this man’s life without injuring his outer skin, inner skin, flesh, sinews, bones, or marrow. Hopefully we’ll see his soul escaping.’ They agree, and do what I ask. When he’s nearly dead, I tell them to\footnote{The commentary explains \textit{\textsanskrit{āmato}} as \textit{addhamato} (“half-dead”). } lay him on his back in hope of seeing his soul escape. They do so. But we don’t see his soul escaping. I tell them to lay him bent over, to lay him on his side, to lay him on the other side; to stand him upright, to stand him upside down; to strike him with fists, stones, rods, and swords; and to give him a good shaking in hope of seeing his soul escape. They do all these things. But we don’t see his soul escaping. For him the eye itself is present, and so are those sights. Yet he does not experience that sense-field.\footnote{At \href{https://suttacentral.net/an9.37/en/sujato}{AN 9.37} this rather abrupt insertion more aptly describes a deep meditation. } The ear itself is present, and so are those sounds. Yet he does not experience that sense-field. The nose itself is present, and so are those smells. Yet he does not experience that sense-field. The tongue itself is present, and so are those tastes. Yet he does not experience that sense-field. The body itself is present, and so are those touches. Yet he does not experience that sense-field. This is how I prove that there is no afterlife.” 

\subsection*{2.9. The Simile of the Horn Blower }

“Well\marginnote{19.1} then, chieftain, I shall give you a simile. For by means of a simile some sensible people understand the meaning of what is said. 

Once\marginnote{19.3} upon a time, a certain horn blower took his horn and traveled to a borderland,\footnote{\textsanskrit{Bṛhadāraṇyaka} \textsanskrit{Upaniṣad} 2.4.8 employs the same metaphor in the search for the soul. } where he went to a certain village. Standing in the middle of the village, he sounded his horn three times, then placed it on the ground and sat down to one side. 

Then\marginnote{19.5} the people of the borderland thought, ‘What is making this sound, so arousing, sensuous, intoxicating, infatuating, and captivating?’ They gathered around the horn blower and said, ‘Mister, what is making this sound, so arousing, sensuous, intoxicating, infatuating, and captivating?’ 

‘The\marginnote{19.9} sound is made by this, which is called a horn.’ 

They\marginnote{19.10} laid that horn on its back, saying, ‘Speak, good horn! Speak, good horn!’ But still the horn made no sound. 

Then\marginnote{19.13} they lay the horn bent over, they lay it on its side, they lay it on its other side; they stood it upright, they stood it upside down; they struck it with fists, stones, rods, and swords; and they gave it a good shake, saying, ‘Speak, good horn! Speak, good horn!’ But still the horn made no sound. 

So\marginnote{19.16} the horn blower thought, ‘How foolish are these borderland folk! For how can they seek the sound of a horn so irrationally?’ And as they looked on, he picked up the horn, sounded it three times, and took it away with him. 

Then\marginnote{19.19} the people of the borderland thought, ‘So, it seems, when what is called a horn is accompanied by a person, effort, and wind, it makes a sound. But when these things are absent it makes no sound.’ 

In\marginnote{19.21} the same way, so long as this body is full of life and warmth and consciousness it walks back and forth, stands, sits, and lies down. It sees sights with the eye, hears sounds with the ear, smells odors with the nose, tastes flavors with the tongue, feels touches with the body, and knows ideas with the mind. But when it lacks life and warmth and consciousness it does none of these things. By this method, too, it ought to be proven that there is an afterlife.” 

“Even\marginnote{20.1} though Mister Kassapa says this, still I think that there is no afterlife.” 

“Can\marginnote{20.3} you prove it?” 

“I\marginnote{20.4} can.” 

“How,\marginnote{20.5} exactly, chieftain?” 

“Suppose\marginnote{20.6} they were to arrest a bandit, a criminal and present him to me, saying, ‘Sir, this is a bandit, a criminal. Punish him as you will.’ I say to them, ‘Well then, sirs, cut open this man’s outer skin. Hopefully we might see his soul.’ They cut open his outer skin, but we see no soul. I say to them, ‘Well then, sirs, cut open his inner skin, flesh, sinews, bones, or marrow. Hopefully we’ll see his soul.’ They do so, but we see no soul. This is how I prove that there is no afterlife.” 

\subsection*{2.10. The Simile of the Fire-Worshiping Matted-Hair Ascetic }

“Well\marginnote{21.1} then, chieftain, I shall give you a simile. For by means of a simile some sensible people understand the meaning of what is said. 

Once\marginnote{21.3} upon a time, a certain fire-worshiping matted-hair ascetic settled in a leaf hut in a wilderness region. Then a caravan came out from a certain country. It stayed for one night not far from that ascetic’s hermitage, and then moved on. The ascetic thought, ‘Why don’t I go to that caravan’s campsite? Hopefully I’ll find something useful there.’ 

So\marginnote{21.8} he went, and he saw a little baby boy abandoned there. When he saw this he thought, ‘It’s not proper for me to look on while a human being dies. Why don’t I bring this boy back to my hermitage, nurse him, provide for him, and raise him?’ So that’s what he did. 

When\marginnote{21.13} the boy was ten or twelve years old, the ascetic had some business come up in the country. So he said to the boy, ‘My dear, I wish to go to the country. Serve the sacred flame. Do not extinguish it. But if you should extinguish it, here is the hatchet, the firewood, and the bundle of drill-sticks. Light the fire and serve it.’ And having instructed the boy, the ascetic went to the country. 

But\marginnote{21.20} the boy was so engrossed in his play, the fire went out. He thought, ‘My father told me to serve the sacred flame. Why don’t I light it again and serve it?’ 

So\marginnote{21.27} he chopped the bundle of drill-sticks with the hatchet, thinking, ‘Hopefully I’ll get a fire!’ But he still got no fire. 

He\marginnote{21.30} split the bundle of drill-sticks into two, three, four, five, ten, or a hundred parts. He chopped them into splinters, pounded them in a mortar, and swept them away in a strong wind, thinking, ‘Hopefully I’ll get a fire!’ But he still got no fire. 

Then\marginnote{21.33} the matted-hair ascetic, having concluded his business in the country, returned to his own hermitage, and said to the boy, ‘I trust, my dear, that the fire didn’t go out?’ And the boy told him what had happened. Then the ascetic thought, ‘How foolish is this boy, how incompetent! For how can he seek a fire so irrationally?’ 

So\marginnote{21.49} while the boy looked on, he took a bundle of fire-sticks, lit the fire, and said, ‘Dear boy, this is how to light a fire. Not the foolish and incompetent way you sought it so irrationally.’ In the same way, chieftain, being foolish and incompetent, you seek the other world irrationally.\footnote{Accept PTS reading \textit{gavesasi}. } Let go of this harmful misconception, chieftain, let go of it! Don’t create lasting harm and suffering for yourself!” 

“Even\marginnote{22.1} though Mister Kassapa says this, still I’m not able to let go of that harmful misconception. King Pasenadi of Kosala knows my views, and so do foreign kings. If I let go of this harmful misconception, people will say, ‘How foolish is the chieftain \textsanskrit{Pāyāsi}, how incompetent, that he should hold on to a mistake!’ I shall carry on with this view out of anger, contempt, and spite!”\footnote{\textsanskrit{Pāyāsi} honestly acknowledges the role of social conformity and shame in shaping views. } 

\subsection*{2.11. The Simile of the Two Caravan Leaders }

“Well\marginnote{23.1} then, chieftain, I shall give you a simile. For by means of a simile some sensible people understand the meaning of what is said. 

Once\marginnote{23.3} upon a time, a large caravan of a thousand wagons traveled from a country in the east to the west.\footnote{Although this became the first story of the \textsanskrit{Jātaka} collection (\href{https://suttacentral.net/ja1/en/sujato}{Ja 1}), here it is not a \textsanskrit{Jātaka}, as it is not framed as a past life of the Buddha. } Wherever they went they quickly used up the grass, wood, water, and the green foliage. Now, that caravan had two leaders, each in charge of five hundred wagons. They thought, ‘This is a large caravan of a thousand wagons. Wherever we go we quickly use up the grass, wood, water, and the green foliage.\footnote{While the places are not specified, the journey from “east to west” across desolate lands sounds suggests they may have been venturing from the rich plains of the Ganges west across the Thar Desert of Rajasthan. } Why don’t we split the caravan in two halves?’ So that’s what they did. 

One\marginnote{23.12} caravan leader, having prepared much grass, wood, and water, started the caravan. After two or three days’ journey he saw a dark man with red eyes coming the other way in a donkey cart with muddy wheels. He was armored with a quiver and wreathed with yellow lotus, his clothes and hair all wet. Seeing him, he said,\footnote{Read with PTS \textit{gadrabharathena} (“donkey cart”) for \textit{bhadrena rathena} (“fine cart”). } ‘Sir, where do you come from?’ 

‘From\marginnote{23.15} such and such a country.’ 

‘And\marginnote{23.16} where are you going?’ 

‘To\marginnote{23.17} the country named so and so.’ 

‘But\marginnote{23.18} has there been much rain in the desert up ahead?’ 

‘Indeed\marginnote{23.19} there has, sir. The paths are sprinkled with water, and there is much grass, wood, and water. Toss out your grass, wood, and water. Your wagons will move swiftly when lightly-laden, so don’t tire your draught teams.’ 

So\marginnote{23.21} the caravan leader addressed his drivers, ‘This man says that there has been much rain in the desert up ahead. He advises us to toss out the grass, wood, and water. The wagons will move swiftly when lightly-laden, and won’t tire our draught teams. So let’s toss out the grass, wood, and water and restart the caravan with lightly-laden wagons.’ 

‘Yes,\marginnote{23.26} sir,’ the drivers replied, and that’s what they did. 

But\marginnote{23.27} in the caravan’s first campsite they saw no grass, wood, or water. And in the second, third, fourth, fifth, sixth, and seventh campsites they saw no grass, wood, or water. And all fell to ruin and disaster. And the men and beasts in that caravan were all devoured by that non-human spirit.\footnote{The obviously suspicious stranger depicts the genuine dangers of trade in unknown regions, warning the traveler of raiding tribes as well as supernatural creatures. } Only their bones remained. 

Now,\marginnote{23.37} when the second caravan leader knew that the first caravan was well underway, he prepared much grass, wood, and water and started the caravan. After two or three days’ journey he saw a dark man with red eyes coming the other way in a donkey cart with muddy wheels. He was armored with a quiver and wreathed with yellow lotus, his clothes and hair all wet. Seeing him, he said, ‘Sir, where do you come from?’ 

‘From\marginnote{23.41} such and such a country.’ 

‘And\marginnote{23.42} where are you going?’ 

‘To\marginnote{23.43} the country named so and so.’ 

‘But\marginnote{23.44} has there been much rain in the desert up ahead?’ 

‘Indeed\marginnote{23.45} there has, sir. The paths are sprinkled with water, and there is much grass, wood, and water. Toss out your grass, wood, and water. Your wagons will move swiftly when lightly-laden, so don’t tire your draught teams.’ 

So\marginnote{23.47} the caravan leader addressed his drivers, ‘This man says that there has been much rain in the desert up ahead. He advises us to toss out the grass, wood, and water. The wagons will move swiftly when lightly-laden, and won’t tire our draught teams. But this person is neither our friend nor relative. How can we proceed out of trust in him?\footnote{Again the criteria that only those who are known may be trusted. } We shouldn’t toss out any grass, wood, or water, but continue with our goods laden as before. We shall not toss out any old stock.’ 

‘Yes,\marginnote{23.54} sir,’ the drivers replied, and they restarted the caravan with the goods laden as before. 

And\marginnote{23.55} in the caravan’s first campsite they saw no grass, wood, or water. And in the second, third, fourth, fifth, sixth, and seventh campsites they saw no grass, wood, or water. And they saw the other caravan that had come to ruin. And they saw the bones of the men and beasts who had been devoured by that non-human spirit. 

So\marginnote{23.64} the caravan leader addressed his drivers, ‘This caravan came to ruin, as happens when guided by a foolish caravan leader. Well then, sirs, toss out any of our merchandise that’s of little value, and take what’s valuable from this caravan.’ 

‘Yes,\marginnote{23.67} sir’ replied the drivers, and that’s what they did. They crossed over the desert safely, as happens when guided by an astute caravan leader. 

In\marginnote{23.68} the same way, chieftain, being foolish and incompetent, you will come to ruin seeking the other world irrationally, like the first caravan leader. And those who think you’re worth listening to and trusting will also come to ruin, like the drivers. Let go of this harmful misconception, chieftain, let go of it! Don’t create lasting harm and suffering for yourself!” 

“Even\marginnote{24.1} though Mister Kassapa says this, still I’m not able to let go of that harmful misconception. King Pasenadi of Kosala knows my views, and so do foreign kings. I shall carry on with this view out of anger, contempt, and spite!” 

\subsection*{2.12. The Simile of the Dung-Carrier }

“Well\marginnote{25.1} then, chieftain, I shall give you a simile. For by means of a simile some sensible people understand the meaning of what is said. 

Once\marginnote{25.3} upon a time, a certain swineherd went from his own village to another village. There he saw a large pile of dry dung abandoned. He thought, ‘This pile of dry dung can serve as food for my pigs. Why don’t I carry it off?’ So he spread out his upper robe, shoveled the dry dung onto it, tied it up into a bundle, lifted it on to his head, and went on his way. While on his way a large sudden storm poured down. Smeared with leaking, oozing dung down to his fingernails, he kept on carrying the load of dung. 

When\marginnote{25.11} people saw him they said, ‘Have you gone mad, sir? Have you lost your mind? For how can you, smeared with leaking, oozing dung down to your fingernails, keep on carrying that load of dung?’ 

‘You’re\marginnote{25.13} the mad ones, sirs! You’re the ones who’ve lost your minds! For this will serve as food for my pigs.’ 

In\marginnote{25.14} the same way, chieftain, you seem like the dung carrier in the simile. Let go of this harmful misconception, chieftain, let go of it! Don’t create lasting harm and suffering for yourself!” 

“Even\marginnote{26.1} though Mister Kassapa says this, still I’m not able to let go of that harmful misconception. King Pasenadi of Kosala knows my views, and so do foreign kings. I shall carry on with this view out of anger, contempt, and spite!” 

\subsection*{2.13. The Simile of the Gamblers }

“Well\marginnote{27.1} then, chieftain, I shall give you a simile. For by means of a simile some sensible people understand the meaning of what is said. 

Once\marginnote{27.3} upon a time, two gamblers were playing with seed dice. One gambler, every time they made a bad throw, swallowed the losing seed.\footnote{The ancient Indian game of dice involved casting an handful of \textit{\textsanskrit{vibhītaka}} seeds (also known as “bedda nuts”, from Terminalia bellirica). A number divisible by four was “perfect” (\textit{\textsanskrit{kaṭa}}), so the fifth seed meant a “losing” throw (\textit{kali}; cf. \textit{\textsanskrit{apaṇṇaka}} in \href{https://suttacentral.net/mn60/en/sujato}{MN 60} and notes). } 

The\marginnote{27.5} second gambler saw him, and said, ‘Well, my friend, you’ve won it all! Give me the seed dice, I will roll them.’\footnote{\textit{\textsanskrit{Pajohissāmi}} is related to Sanskrit \textit{juhoti}, usually used in the sense “to offer as libation”, a meaning accepted by the commentary here. More likely it simply means to “roll out” (like pouring a sacrifice) by analogy. } 

‘Yes,\marginnote{27.7} my friend,’ the gambler replied, and gave them. 

Then\marginnote{27.8} the gambler soaked the seed dice in poison and said to the other, ‘Come, my friend, let’s play seed dice.’ 

‘Yes,\marginnote{27.10} my friend,’ the other gambler replied. 

And\marginnote{27.11} for a second time the gamblers played with seed dice. And for the second time, every time they made a bad throw, that gambler swallowed the losing seed. 

The\marginnote{27.13} second gambler saw him, and said, 

\begin{verse}%
‘The\marginnote{27.14} man swallows the dice without realizing \\
they’re smeared with burning poison. \\
Swallow, you damn cheat, swallow! \\
Soon you’ll know the bitter fruit!’ 

%
\end{verse}

In\marginnote{27.18} the same way, chieftain, you seem like the gambler in the simile. Let go of this harmful misconception, chieftain, let go of it! Don’t create lasting harm and suffering for yourself!” 

“Even\marginnote{28.1} though Mister Kassapa says this, still I’m not able to let go of that harmful misconception. King Pasenadi of Kosala knows my views, and so do foreign kings. I shall carry on with this view out of anger, contempt, and spite!” 

\subsection*{2.14. The Simile of the Man Who Carried Hemp }

“Well\marginnote{29.1} then, chieftain, I shall give you a simile. For by means of a simile some sensible people understand the meaning of what is said. 

Once\marginnote{29.3} upon a time, the inhabitants of a certain country emigrated. Then one friend said to another, ‘Come, my friend, let’s go to that country. Hopefully we’ll get some riches there!’ 

‘Yes,\marginnote{29.6} my friend,’ the other replied. 

They\marginnote{29.7} went to that country, and to a certain deserted village. There they saw a pile of abandoned sunn hemp. Seeing it, one friend said to the other,\footnote{\textit{\textsanskrit{Gāmapaṭṭaṁ}} (variants \textit{-\textsanskrit{padaṁ}}, \textit{-\textsanskrit{paddhanaṁ}}, \textit{-\textsanskrit{patthaṁ}}, \textit{\textsanskrit{pajjaṁ}}) is explained by the commentary as an abandoned village site. } ‘This is a pile of abandoned sunn hemp. Well then, my friend, you make up a bundle of hemp, and I’ll make one too. Let’s both take a bundle of hemp and go on.’ 

‘Yes,\marginnote{29.9} my friend,’ he said. Carrying their bundles of hemp they went to another deserted village. 

There\marginnote{29.10} they saw much sunn hemp thread abandoned. Seeing it, one friend said to the other, ‘This pile of abandoned sunn hemp thread is just what we wanted the hemp for! Well then, my friend, let’s abandon our bundles of hemp, and both take a bundle of hemp thread and go on.’ 

‘I’ve\marginnote{29.13} already carried this bundle of hemp a long way, and it’s well tied up. It’s good enough for me, you understand.’\footnote{The sunk cost fallacy. } So one friend abandoned their bundle of hemp and picked up a bundle of hemp thread. 

They\marginnote{29.15} went to another deserted village. There they saw much sunn hemp cloth abandoned. Seeing it, one friend said to the other, ‘This pile of abandoned sunn hemp cloth is just what we wanted the hemp and hemp thread for! Well then, my friend, let’s abandon our bundles, and both take a bundle of hemp cloth and go on.’ 

‘I’ve\marginnote{29.19} already carried this bundle of hemp a long way, and it’s well tied up. It’s good enough for me, you understand.’ So one friend abandoned their bundle of hemp thread and picked up a bundle of hemp cloth. 

They\marginnote{29.21} went to another deserted village. There they saw a pile of flax, and by turn, linen thread, linen cloth, silk, silk thread, silk cloth, iron, copper, tin, lead, silver, and gold abandoned. Seeing it, one friend said to the other, ‘This pile of gold is just what we wanted all those other things for! Well then, my friend, let’s abandon our bundles, and both take a bundle of gold and go on.’ 

‘I’ve\marginnote{29.36} already carried this bundle of hemp a long way, and it’s well tied up. It’s good enough for me, you understand.’ So one friend abandoned their bundle of silver and picked up a bundle of gold. 

Then\marginnote{29.38} they returned to their own village. When one friend returned with a bundle of sunn hemp, they didn’t please their parents, their partners and children, or their friends and colleagues. And they got no pleasure and happiness on that account. But when the other friend returned with a bundle of gold, they pleased their parents, their partners and children, and their friends and colleagues. And they got much pleasure and happiness on that account. 

In\marginnote{29.41} the same way, chieftain, you seem like the hemp-carrier in the simile. Let go of this harmful misconception, chieftain, let go of it! Don’t create lasting harm and suffering for yourself!” 

\section*{3. Going for Refuge }

“I\marginnote{30.1} was delighted and satisfied with your very first simile, Mister Kassapa!\footnote{Say what you will about \textsanskrit{Pāyāsi}, he had character. } Nevertheless, I wanted to hear your various solutions to the problem, so I thought I’d oppose you in this way. Excellent, Mister Kassapa! Excellent! As if he were righting the overturned, or revealing the hidden, or pointing out the path to the lost, or lighting a lamp in the dark so people with clear eyes can see what’s there, Mister Kassapa has made the teaching clear in many ways. I go for refuge to Mister Gotama, to the teaching, and to the mendicant \textsanskrit{Saṅgha}. From this day forth, may Mister Kassapa remember me as a lay follower who has gone for refuge for life. 

Mister\marginnote{30.7} Kassapa, I wish to perform a great sacrifice. Please instruct me so it will be for my lasting welfare and happiness.”\footnote{Compare \href{https://suttacentral.net/dn5/en/sujato\#4.5}{DN 5:4.5}. } 

\section*{4. On Sacrifice }

“Chieftain,\marginnote{31.1} take the kind of sacrifice where cattle, goats and sheep, chickens and pigs, and various kinds of creatures are slaughtered. And the recipients have wrong view, wrong thought, wrong speech, wrong action, wrong livelihood, wrong effort, wrong mindfulness, and wrong immersion. That kind of sacrifice is not very fruitful or beneficial or splendid or bountiful. 

Suppose\marginnote{31.2} a farmer was to enter a wood taking seed and plough. And on that barren field, that barren ground, with uncleared stumps he sowed seeds that were broken, spoiled, weather-damaged, infertile, and ill kept. And the heavens don’t provide enough rain when needed. Would those seeds grow, increase, and mature, and would the farmer get abundant fruit?” 

“No,\marginnote{31.6} Mister Kassapa.” 

“In\marginnote{31.7} the same way, chieftain, take the kind of sacrifice where cattle, goats and sheep, chickens and pigs, and various kinds of creatures are slaughtered. And the recipients have wrong view, wrong thought, wrong speech, wrong action, wrong livelihood, wrong effort, wrong mindfulness, and wrong immersion. That kind of sacrifice is not very fruitful or beneficial or splendid or bountiful. 

But\marginnote{31.8} take the kind of sacrifice where cattle, goats and sheep, chickens and pigs, and various kinds of creatures are not slaughtered. And the recipients have right view, right thought, right speech, right action, right livelihood, right effort, right mindfulness, and right immersion. That kind of sacrifice is very fruitful and beneficial and splendid and bountiful. 

Suppose\marginnote{31.9} a farmer was to enter a wood taking seed and plough. And on that fertile field, that fertile ground, with well-cleared stumps he sowed seeds that were intact, unspoiled, not weather-damaged, fertile, and well kept. And the heavens provide plenty of rain when needed. Would those seeds grow, increase, and mature, and would the farmer get abundant fruit?” 

“Yes,\marginnote{31.13} Mister Kassapa.” 

“In\marginnote{31.14} the same way, chieftain, take the kind of sacrifice where cattle, goats and sheep, chickens and pigs, and various kinds of creatures are not slaughtered. And the recipients have right view, right thought, right speech, right action, right livelihood, right effort, right mindfulness, and right immersion. That kind of sacrifice is very fruitful and beneficial and splendid and bountiful.” 

\section*{5. On the Student Uttara }

Then\marginnote{32.1} the chieftain \textsanskrit{Pāyāsi} set up an offering for ascetics and brahmins, for paupers, vagrants, supplicants, and beggars. At that offering such food as rough gruel with pickles was given, and rough clothes with knotted fringes.\footnote{For \textit{\textsanskrit{guḷavālakāni}}, \textit{\textsanskrit{guḷa}} is “ball”, \textit{\textsanskrit{vāla}} is “tail”; compare \textit{\textsanskrit{macchavāḷaka}} (“fish-tailed”) at \href{https://suttacentral.net/pli-tv-kd15/en/sujato\#29.4.2}{Kd 15:29.4.2}. } Now, it was a student named Uttara who organized that offering. 

When\marginnote{32.4} the offering was over he referred to it like this, “Through this offering may I be together with the chieftain \textsanskrit{Pāyāsi} in this world, but not in the next.” 

\textsanskrit{Pāyāsi}\marginnote{32.6} heard of this, so he summoned Uttara and said, “Is it really true, dear Uttara, that you referred to the offering in this way?” 

“Yes,\marginnote{32.12} sir.” 

“But\marginnote{32.13} why? Don’t we who seek merit expect some result from the offering?” 

“At\marginnote{32.16} your offering such food as rough gruel with pickles was given, which you wouldn’t even want to touch with your foot, much less eat. And also rough clothes with knotted fringes, which you also wouldn’t want to touch with your foot, much less wear. Sir, you’re dear and beloved to me. But how can I reconcile one so dear with something so disagreeable?” 

“Well\marginnote{32.18} then, dear Uttara, set up an offering with the same kind of food that I eat, and the same kind of clothes that I wear.” 

“Yes,\marginnote{32.20} sir,” replied Uttara, and did so. 

So\marginnote{32.22} the chieftain \textsanskrit{Pāyāsi} gave gifts carelessly, thoughtlessly, not with his own hands, giving the dregs. When his body broke up, after death, he was reborn in company with the gods of the four great kings, in an empty palace of sirisa wood.\footnote{Cp. \href{https://suttacentral.net/an5.147/en/sujato}{AN 5.147}, \href{https://suttacentral.net/an9.20/en/sujato}{AN 9.20}. } But the student Uttara who organized the offering gave gifts carefully, thoughtfully, with his own hands, not giving the dregs. When his body broke up, after death, he was reborn in company with the gods of the thirty-three. 

\section*{6. The God \textsanskrit{Pāyāsi} }

Now\marginnote{33.1} at that time Venerable Gavampati would often go to that empty sirisa palace for the day’s meditation.\footnote{\textit{\textsanskrit{Divāvihāra}} is the “day’s meditation”, while \textit{\textsanskrit{divāseyya}} is “siesta”. } Then the god \textsanskrit{Pāyāsi} went up to him, bowed, and stood to one side. Gavampati said to him, “Who are you, reverend?” 

“Sir,\marginnote{33.4} I am the chieftain \textsanskrit{Pāyāsi}.” 

“Didn’t\marginnote{33.5} you have the view that there is no afterlife, no beings are reborn spontaneously, and there’s no fruit or result of good and bad deeds?” 

“It’s\marginnote{33.7} true, sir, I did have such a view. But Master Kassapa the Prince dissuaded me from that harmful misconception.” 

“But\marginnote{33.10} the student named Uttara who organized that offering for you—where has he been reborn?” 

“Sir,\marginnote{33.11} Uttara gave gifts carefully, thoughtfully, with his own hands, not giving the dregs. When his body broke up, after death, he was reborn in company with the gods of the thirty-three. But I gave gifts carelessly, thoughtlessly, not with my own hands, giving the dregs. When my body broke up, after death, I was reborn in company with the gods of the four great kings, in an empty sirisa palace. 

So,\marginnote{33.13} sir, when you’ve returned to the human realm, please announce this: ‘Give gifts carefully, thoughtfully, with your own hands, not giving the dregs. The chieftain \textsanskrit{Pāyāsi} gave gifts carelessly, thoughtlessly, not with his own hands, giving the dregs. When his body broke up, after death, he was reborn in company with the gods of the four great kings, in an empty palace of sirisa. But the student Uttara who organized the offering gave gifts carefully, thoughtfully, with his own hands, not giving the dregs. When his body broke up, after death, he was reborn in company with the gods of the Thirty-Three.’” 

So\marginnote{34.1} when Venerable Gavampati returned to the human realm he made that announcement. 

%
\backmatter%
%
\chapter*{Colophon}
\addcontentsline{toc}{chapter}{Colophon}
\markboth{Colophon}{Colophon}

\section*{The Translator}

Bhikkhu Sujato was born as Anthony Aidan Best on 4/11/1966 in Perth, Western Australia. He grew up in the pleasant suburbs of Mt Lawley and Attadale alongside his sister Nicola, who was the good child. His mother, Margaret Lorraine Huntsman née Pinder, said “he’ll either be a priest or a poet”, while his father, Anthony Thomas Best, advised him to “never do anything for money”. He attended Aquinas College, a Catholic school, where he decided to become an atheist. At the University of WA he studied philosophy, aiming to learn what he wanted to do with his life. Finding that what he wanted to do was play guitar, he dropped out. His main band was named Martha’s Vineyard, which achieved modest success in the indie circuit. 

A seemingly random encounter with a roadside joey took him to Thailand, where he entered his first meditation retreat at Wat Ram Poeng, Chieng Mai in 1992. Feeling the call to the Buddha’s path, he took full ordination in Wat Pa Nanachat in 1994, where his teachers were Ajahn Pasanno and Ajahn Jayasaro. In 1997 he returned to Perth to study with Ajahn Brahm at Bodhinyana Monastery. 

He spent several years practicing in seclusion in Malaysia and Thailand before establishing Santi Forest Monastery in Bundanoon, NSW, in 2003. There he was instrumental in supporting the establishment of the Theravada bhikkhuni order in Australia and advocating for women’s rights. He continues to teach in Australia and globally, with a special concern for the moral implications of climate change and other forms of environmental destruction. He has published a series of books of original and groundbreaking research on early Buddhism. 

In 2005 he founded SuttaCentral together with Rod Bucknell and John Kelly. In 2015, seeing the need for a complete, accurate, plain English translation of the Pali texts, he undertook the task, spending nearly three years in isolation on the isle of Qi Mei off the coast of the nation of Taiwan. He completed the four main \textsanskrit{Nikāyas} in 2018, and the early books of the Khuddaka \textsanskrit{Nikāya} were complete by 2021. All this work is dedicated to the public domain and is entirely free of copyright encumbrance. 

In 2019 he returned to Sydney where he established Lokanta Vihara (The Monastery at the End of the World). 

\section*{Creation Process}

Primary source was the digital \textsanskrit{Mahāsaṅgīti} edition of the Pali \textsanskrit{Tipiṭaka}. Translated from the Pali, with reference to several English translations, especially those of Bhikkhu Bodhi. Older translations by Maurice Walshe and T.W. and C.A.F. Rhys Davids were also consulted.

\section*{The Translation}

This translation was part of a project to translate the four Pali \textsanskrit{Nikāyas} with the following aims: plain, approachable English; consistent terminology; accurate rendition of the Pali; free of copyright. It was made during 2016–2018 while Bhikkhu Sujato was staying in Qimei, Taiwan.

\section*{About SuttaCentral}

SuttaCentral publishes early Buddhist texts. Since 2005 we have provided root texts in Pali, Chinese, Sanskrit, Tibetan, and other languages, parallels between these texts, and translations in many modern languages. Building on the work of generations of scholars, we offer our contribution freely.

SuttaCentral is driven by volunteer contributions, and in addition we employ professional developers. We offer a sponsorship program for high quality translations from the original languages. Financial support for SuttaCentral is handled by the SuttaCentral Development Trust, a charitable trust registered in Australia.

\section*{About Bilara}

“Bilara” means “cat” in Pali, and it is the name of our Computer Assisted Translation (CAT) software. Bilara is a web app that enables translators to translate early Buddhist texts into their own language. These translations are published on SuttaCentral with the root text and translation side by side.

\section*{About SuttaCentral Editions}

The SuttaCentral Editions project makes high quality books from selected Bilara translations. These are published in formats including HTML, EPUB, PDF, and print.

You are welcome to print any of our Editions.

%
\end{document}