\documentclass[12pt,openany]{book}%
\usepackage{lastpage}%
%
\usepackage{ragged2e}
\usepackage{verse}
\usepackage[a-3u]{pdfx}
\usepackage[inner=1in, outer=1in, top=.7in, bottom=1in, papersize={6in,9in}, headheight=13pt]{geometry}
\usepackage{polyglossia}
\usepackage[12pt]{moresize}
\usepackage{soul}%
\usepackage{microtype}
\usepackage{tocbasic}
\usepackage{realscripts}
\usepackage{epigraph}%
\usepackage{setspace}%
\usepackage{sectsty}
\usepackage{fontspec}
\usepackage{marginnote}
\usepackage[bottom]{footmisc}
\usepackage{enumitem}
\usepackage{fancyhdr}
\usepackage{emptypage}
\usepackage{extramarks}
\usepackage{graphicx}
\usepackage{relsize}
\usepackage{etoolbox}

% improve ragged right headings by suppressing hyphenation and orphans. spaceskip plus and minus adjust interword spacing; increase rightskip stretch to make it want to push a word on the first line(s) to the next line; reduce parfillskip stretch to make line length more equal . spacefillskip and xspacefillskip can be deleted to use defaults.
\protected\def\BalancedRagged{
\leftskip     0pt
\rightskip    0pt plus 10em
\spaceskip=1\fontdimen2\font plus .5\fontdimen3\font minus 1.5\fontdimen4\font
\xspaceskip=1\fontdimen2\font plus 1\fontdimen3\font minus 1\fontdimen4\font
\parfillskip  0pt plus 15em
\relax
}

\hypersetup{
colorlinks=true,
urlcolor=black,
linkcolor=black,
citecolor=black,
allcolors=black
}

% use a small amount of tracking on small caps
\SetTracking[ spacing = {25*,166, } ]{ encoding = *, shape = sc }{ 25 }

% add a blank page
\newcommand{\blankpage}{
\newpage
\thispagestyle{empty}
\mbox{}
\newpage
}

% define languages
\setdefaultlanguage[]{english}
\setotherlanguage[script=Latin]{sanskrit}

%\usepackage{pagegrid}
%\pagegridsetup{top-left, step=.25in}

% define fonts
% use if arno sanskrit is unavailable
%\setmainfont{Gentium Plus}
%\newfontfamily\Marginalfont[]{Gentium Plus}
%\newfontfamily\Allsmallcapsfont[RawFeature=+c2sc]{Gentium Plus}
%\newfontfamily\Noligaturefont[Renderer=Basic]{Gentium Plus}
%\newfontfamily\Noligaturecaptionfont[Renderer=Basic]{Gentium Plus}
%\newfontfamily\Fleuronfont[Ornament=1]{Gentium Plus}

% use if arno sanskrit is available. display is applied to \chapter and \part, subhead to \section and \subsection.
\setmainfont[
  FontFace={sb}{n}{Font = {Arno Pro Semibold}},
  FontFace={sb}{it}{Font = {Arno  Pro Semibold Italic}}
]{Arno Pro}

% create commands for using semibold
\DeclareRobustCommand{\sbseries}{\fontseries{sb}\selectfont}
\DeclareTextFontCommand{\textsb}{\sbseries}

\newfontfamily\Marginalfont[RawFeature=+subs]{Arno Pro Regular}
\newfontfamily\Allsmallcapsfont[RawFeature=+c2sc]{Arno Pro}
\newfontfamily\Noligaturefont[Renderer=Basic]{Arno Pro}
\newfontfamily\Noligaturecaptionfont[Renderer=Basic]{Arno Pro Caption}

% chinese fonts
\newfontfamily\cjk{Noto Serif TC}
\newcommand*{\langlzh}[1]{\cjk{#1}\normalfont}%

% logo
\newfontfamily\Logofont{sclogo.ttf}
\newcommand*{\sclogo}[1]{\large\Logofont{#1}}

% use subscript numerals for margin notes
\renewcommand*{\marginfont}{\Marginalfont}

% ensure margin notes have consistent vertical alignment
\renewcommand*{\marginnotevadjust}{-.17em}

% use compact lists
\setitemize{noitemsep,leftmargin=1em}
\setenumerate{noitemsep,leftmargin=1em}
\setdescription{noitemsep, style=unboxed, leftmargin=1em}

% style ToC
\DeclareTOCStyleEntries[
  raggedentrytext,
  linefill=\hfill,
  pagenumberwidth=.5in,
  pagenumberformat=\normalfont,
  entryformat=\normalfont
]{tocline}{chapter,section}


  \setlength\topsep{0pt}%
  \setlength\parskip{0pt}%

% define new \centerpars command for use in ToC. This ensures centering, proper wrapping, and no page break after
\def\startcenter{%
  \par
  \begingroup
  \leftskip=0pt plus 1fil
  \rightskip=\leftskip
  \parindent=0pt
  \parfillskip=0pt
}
\def\stopcenter{%
  \par
  \endgroup
}
\long\def\centerpars#1{\startcenter#1\stopcenter}

% redefine part, so that it adds a toc entry without page number
\let\oldcontentsline\contentsline
\newcommand{\nopagecontentsline}[3]{\oldcontentsline{#1}{#2}{}}

    \makeatletter
\renewcommand*\l@part[2]{%
  \ifnum \c@tocdepth >-2\relax
    \addpenalty{-\@highpenalty}%
    \addvspace{0em \@plus\p@}%
    \setlength\@tempdima{3em}%
    \begingroup
      \parindent \z@ \rightskip \@pnumwidth
      \parfillskip -\@pnumwidth
      {\leavevmode
       \setstretch{.85}\large\scshape\centerpars{#1}\vspace*{-1em}\llap{#2}}\par
       \nobreak
         \global\@nobreaktrue
         \everypar{\global\@nobreakfalse\everypar{}}%
    \endgroup
  \fi}
\makeatother

\makeatletter
\def\@pnumwidth{2em}
\makeatother

% define new sectioning command, which is only used in volumes where the pannasa is found in some parts but not others, especially in an and sn

\newcommand*{\pannasa}[1]{\clearpage\thispagestyle{empty}\begin{center}\vspace*{14em}\setstretch{.85}\huge\itshape\scshape\MakeLowercase{#1}\end{center}}

    \makeatletter
\newcommand*\l@pannasa[2]{%
  \ifnum \c@tocdepth >-2\relax
    \addpenalty{-\@highpenalty}%
    \addvspace{.5em \@plus\p@}%
    \setlength\@tempdima{3em}%
    \begingroup
      \parindent \z@ \rightskip \@pnumwidth
      \parfillskip -\@pnumwidth
      {\leavevmode
       \setstretch{.85}\large\itshape\scshape\lowercase{\centerpars{#1}}\vspace*{-1em}\llap{#2}}\par
       \nobreak
         \global\@nobreaktrue
         \everypar{\global\@nobreakfalse\everypar{}}%
    \endgroup
  \fi}
\makeatother

% don't put page number on first page of toc (relies on etoolbox)
\patchcmd{\chapter}{plain}{empty}{}{}

% global line height
\setstretch{1.05}

% allow linebreak after em-dash
\catcode`\—=13
\protected\def—{\unskip\textemdash\allowbreak}

% style headings with secsty. chapter and section are defined per-edition
\partfont{\setstretch{.85}\normalfont\centering\textsc}
\subsectionfont{\setstretch{.95}\normalfont\BalancedRagged}%
\subsubsectionfont{\setstretch{1}\normalfont\itshape\BalancedRagged}

% style elements of suttatitle
\newcommand*{\suttatitleacronym}[1]{\smaller[2]{#1}\vspace*{.3em}}
\newcommand*{\suttatitletranslation}[1]{\linebreak{#1}}
\newcommand*{\suttatitleroot}[1]{\linebreak\smaller[2]\itshape{#1}}

\DeclareTOCStyleEntries[
  indent=3.3em,
  dynindent,
  beforeskip=.2em plus -2pt minus -1pt,
]{tocline}{section}

\DeclareTOCStyleEntries[
  indent=0em,
  dynindent,
  beforeskip=.4em plus -2pt minus -1pt,
]{tocline}{chapter}

\newcommand*{\tocacronym}[1]{\hspace*{-3.3em}{#1}\quad}
\newcommand*{\toctranslation}[1]{#1}
\newcommand*{\tocroot}[1]{(\textit{#1})}
\newcommand*{\tocchapterline}[1]{\bfseries\itshape{#1}}


% redefine paragraph and subparagraph headings to not be inline
\makeatletter
% Change the style of paragraph headings %
\renewcommand\paragraph{\@startsection{paragraph}{4}{\z@}%
            {-2.5ex\@plus -1ex \@minus -.25ex}%
            {1.25ex \@plus .25ex}%
            {\noindent\normalfont\itshape\small}}

% Change the style of subparagraph headings %
\renewcommand\subparagraph{\@startsection{subparagraph}{5}{\z@}%
            {-2.5ex\@plus -1ex \@minus -.25ex}%
            {1.25ex \@plus .25ex}%
            {\noindent\normalfont\itshape\footnotesize}}
\makeatother

% use etoolbox to suppress page numbers on \part
\patchcmd{\part}{\thispagestyle{plain}}{\thispagestyle{empty}}
  {}{\errmessage{Cannot patch \string\part}}

% and to reduce margins on quotation
\patchcmd{\quotation}{\rightmargin}{\leftmargin 1.2em \rightmargin}{}{}
\AtBeginEnvironment{quotation}{\small}

% titlepage
\newcommand*{\titlepageTranslationTitle}[1]{{\begin{center}\begin{large}{#1}\end{large}\end{center}}}
\newcommand*{\titlepageCreatorName}[1]{{\begin{center}\begin{normalsize}{#1}\end{normalsize}\end{center}}}

% halftitlepage
\newcommand*{\halftitlepageTranslationTitle}[1]{\setstretch{2.5}{\begin{Huge}\uppercase{\so{#1}}\end{Huge}}}
\newcommand*{\halftitlepageTranslationSubtitle}[1]{\setstretch{1.2}{\begin{large}{#1}\end{large}}}
\newcommand*{\halftitlepageFleuron}[1]{{\begin{large}\Fleuronfont{{#1}}\end{large}}}
\newcommand*{\halftitlepageByline}[1]{{\begin{normalsize}\textit{{#1}}\end{normalsize}}}
\newcommand*{\halftitlepageCreatorName}[1]{{\begin{LARGE}{\textsc{#1}}\end{LARGE}}}
\newcommand*{\halftitlepageVolumeNumber}[1]{{\begin{normalsize}{\Allsmallcapsfont{\textsc{#1}}}\end{normalsize}}}
\newcommand*{\halftitlepageVolumeAcronym}[1]{{\begin{normalsize}{#1}\end{normalsize}}}
\newcommand*{\halftitlepageVolumeTranslationTitle}[1]{{\begin{Large}{\textsc{#1}}\end{Large}}}
\newcommand*{\halftitlepageVolumeRootTitle}[1]{{\begin{normalsize}{\Allsmallcapsfont{\textsc{\itshape #1}}}\end{normalsize}}}
\newcommand*{\halftitlepagePublisher}[1]{{\begin{large}{\Noligaturecaptionfont\textsc{#1}}\end{large}}}

% epigraph
\renewcommand{\epigraphflush}{center}
\renewcommand*{\epigraphwidth}{.85\textwidth}
\newcommand*{\epigraphTranslatedTitle}[1]{\vspace*{.5em}\footnotesize\textsc{#1}\\}%
\newcommand*{\epigraphRootTitle}[1]{\footnotesize\textit{#1}\\}%
\newcommand*{\epigraphReference}[1]{\footnotesize{#1}}%

% map
\newsavebox\IBox

% custom commands for html styling classes
\newcommand*{\scnamo}[1]{\begin{Center}\textit{#1}\end{Center}\bigskip}
\newcommand*{\scendsection}[1]{\begin{Center}\begin{small}\textit{#1}\end{small}\end{Center}\addvspace{1em}}
\newcommand*{\scendsutta}[1]{\begin{Center}\textit{#1}\end{Center}\addvspace{1em}}
\newcommand*{\scendbook}[1]{\bigskip\begin{Center}\uppercase{#1}\end{Center}\addvspace{1em}}
\newcommand*{\scendkanda}[1]{\begin{Center}\textbf{#1}\end{Center}\addvspace{1em}} % use for ending vinaya rule sections and also samyuttas %
\newcommand*{\scend}[1]{\begin{Center}\begin{small}\textit{#1}\end{small}\end{Center}\addvspace{1em}}
\newcommand*{\scendvagga}[1]{\begin{Center}\textbf{#1}\end{Center}\addvspace{1em}}
\newcommand*{\scrule}[1]{\textsb{#1}}
\newcommand*{\scadd}[1]{\textit{#1}}
\newcommand*{\scevam}[1]{\textsc{#1}}
\newcommand*{\scspeaker}[1]{\hspace{2em}\textit{#1}}
\newcommand*{\scbyline}[1]{\begin{flushright}\textit{#1}\end{flushright}\bigskip}
\newcommand*{\scexpansioninstructions}[1]{\begin{small}\textit{#1}\end{small}}
\newcommand*{\scuddanaintro}[1]{\medskip\noindent\begin{footnotesize}\textit{#1}\end{footnotesize}\smallskip}

\newenvironment{scuddana}{%
\setlength{\stanzaskip}{.5\baselineskip}%
  \vspace{-1em}\begin{verse}\begin{footnotesize}%
}{%
\end{footnotesize}\end{verse}
}%

% custom command for thematic break = hr
\newcommand*{\thematicbreak}{\begin{center}\rule[.5ex]{6em}{.4pt}\begin{normalsize}\quad\Fleuronfont{•}\quad\end{normalsize}\rule[.5ex]{6em}{.4pt}\end{center}}

% manage and style page header and footer. "fancy" has header and footer, "plain" has footer only

\pagestyle{fancy}
\fancyhf{}
\fancyfoot[RE,LO]{\thepage}
\fancyfoot[LE,RO]{\footnotesize\lastleftxmark}
\fancyhead[CE]{\setstretch{.85}\Noligaturefont\MakeLowercase{\textsc{\firstrightmark}}}
\fancyhead[CO]{\setstretch{.85}\Noligaturefont\MakeLowercase{\textsc{\firstleftmark}}}
\renewcommand{\headrulewidth}{0pt}
\fancypagestyle{plain}{ %
\fancyhf{} % remove everything
\fancyfoot[RE,LO]{\thepage}
\fancyfoot[LE,RO]{\footnotesize\lastleftxmark}
\renewcommand{\headrulewidth}{0pt}
\renewcommand{\footrulewidth}{0pt}}
\fancypagestyle{plainer}{ %
\fancyhf{} % remove everything
\fancyfoot[RE,LO]{\thepage}
\renewcommand{\headrulewidth}{0pt}
\renewcommand{\footrulewidth}{0pt}}

% style footnotes
\setlength{\skip\footins}{1em}

\makeatletter
\newcommand{\@makefntextcustom}[1]{%
    \parindent 0em%
    \thefootnote.\enskip #1%
}
\renewcommand{\@makefntext}[1]{\@makefntextcustom{#1}}
\makeatother

% hang quotes (requires microtype)
\microtypesetup{
  protrusion = true,
  expansion  = true,
  tracking   = true,
  factor     = 1000,
  patch      = all,
  final
}

% Custom protrusion rules to allow hanging punctuation
\SetProtrusion
{ encoding = *}
{
% char   right left
  {-} = {    , 500 },
  % Double Quotes
  \textquotedblleft
      = {1000,     },
  \textquotedblright
      = {    , 1000},
  \quotedblbase
      = {1000,     },
  % Single Quotes
  \textquoteleft
      = {1000,     },
  \textquoteright
      = {    , 1000},
  \quotesinglbase
      = {1000,     }
}

% make latex use actual font em for parindent, not Computer Modern Roman
\AtBeginDocument{\setlength{\parindent}{1em}}%
%

% Default values; a bit sloppier than normal
\tolerance 1414
\hbadness 1414
\emergencystretch 1.5em
\hfuzz 0.3pt
\clubpenalty = 10000
\widowpenalty = 10000
\displaywidowpenalty = 10000
\hfuzz \vfuzz
 \raggedbottom%

\title{Long Discourses}
\author{Bhikkhu Sujato}
\date{}%
% define a different fleuron for each edition
\newfontfamily\Fleuronfont[Ornament=16]{Arno Pro}

% Define heading styles per edition for chapter and section. Suttatitle can be either of these, depending on the volume. 

\let\oldfrontmatter\frontmatter
\renewcommand{\frontmatter}{%
\chapterfont{\setstretch{.85}\normalfont\centering}%
\sectionfont{\setstretch{.85}\normalfont\BalancedRagged}%
\oldfrontmatter}

\let\oldmainmatter\mainmatter
\renewcommand{\mainmatter}{%
\chapterfont{\setstretch{.85}\normalfont\centering}%
\sectionfont{\setstretch{.85}\normalfont\BalancedRagged}%
\oldmainmatter}

\let\oldbackmatter\backmatter
\renewcommand{\backmatter}{%
\chapterfont{\setstretch{.85}\normalfont\centering}%
\sectionfont{\setstretch{.85}\normalfont\BalancedRagged}%
\pagestyle{plainer}%
\oldbackmatter}

% for reasons, flat texts align too far in the margin in ToC, this fixes it. 
\renewcommand*{\tocacronym}[1]{\hspace*{0em}{#1}\quad}%
%
\begin{document}%
\normalsize%
\frontmatter%
\setlength{\parindent}{0cm}

\pagestyle{empty}

\maketitle

\blankpage%
\begin{center}

\vspace*{2.2em}

\halftitlepageTranslationTitle{Long Discourses}

\vspace*{1em}

\halftitlepageTranslationSubtitle{A faithful translation of the Dīgha Nikāya}

\vspace*{2em}

\halftitlepageFleuron{•}

\vspace*{2em}

\halftitlepageByline{translated and introduced by}

\vspace*{.5em}

\halftitlepageCreatorName{Bhikkhu Sujato}

\vspace*{4em}

\halftitlepageVolumeNumber{Volume 1}

\smallskip

\halftitlepageVolumeAcronym{DN 1–13}

\smallskip

\halftitlepageVolumeTranslationTitle{The Chapter on the Entire Spectrum of Ethics}

\smallskip

\halftitlepageVolumeRootTitle{Sīlakkhandhavagga}

\vspace*{\fill}

\sclogo{0}
 \halftitlepagePublisher{SuttaCentral}

\end{center}

\newpage
%
\setstretch{1.05}

\begin{footnotesize}

\textit{Long Discourses} is a translation of the Dīghanikāya by Bhikkhu Sujato.

\medskip

Creative Commons Zero (CC0)

To the extent possible under law, Bhikkhu Sujato has waived all copyright and related or neighboring rights to \textit{Long Discourses}.

\medskip

This work is published from Australia.

\begin{center}
\textit{This translation is an expression of an ancient spiritual text that has been passed down by the Buddhist tradition for the benefit of all sentient beings. It is dedicated to the public domain via Creative Commons Zero (CC0). You are encouraged to copy, reproduce, adapt, alter, or otherwise make use of this translation. The translator respectfully requests that any use be in accordance with the values and principles of the Buddhist community.}
\end{center}

\medskip

\begin{description}
    \item[Web publication date] 2018
    \item[This edition] 2025-01-13 01:01:43
    \item[Publication type] hardcover
    \item[Edition] ed3
    \item[Number of volumes] 3
    \item[Publication ISBN] 978-1-76132-049-1
    \item[Volume ISBN] 978-1-76132-097-2
    \item[Publication URL] \href{https://suttacentral.net/editions/dn/en/sujato}{https://suttacentral.net/editions/dn/en/sujato}
    \item[Source URL] \href{https://github.com/suttacentral/bilara-data/tree/published/translation/en/sujato/sutta/dn}{https://github.com/suttacentral/bilara-data/tree/published/translation/en/sujato/sutta/dn}
    \item[Publication number] scpub2
\end{description}

\medskip

Map of Jambudīpa is by Jonas David Mitja Lang, and is released by him under Creative Commons Zero (CC0).

\medskip

Published by SuttaCentral

\medskip

\textit{SuttaCentral,\\
c/o Alwis \& Alwis Pty Ltd\\
Kaurna Country,\\
Suite 12,\\
198 Greenhill Road,\\
Eastwood,\\
SA 5063,\\
Australia}

\end{footnotesize}

\newpage

\setlength{\parindent}{1em}%%
\newpage

\vspace*{\fill}

\begin{center}
\epigraph{These are the principles—deep, hard to see, hard to understand, peaceful, sublime, beyond the scope of logic, subtle, comprehensible to the astute—which the Realized One makes known after realizing them with his own insight.}
{
\epigraphTranslatedTitle{“The Divine Net”}
\epigraphRootTitle{\textsanskrit{Brahmajālasutta}}
\epigraphReference{\textsanskrit{Dīgha} \textsanskrit{Nikāya} 1}
}
\end{center}

\vspace*{2in}

\vspace*{\fill}

\newgeometry{inner=0mm, outer=.5in, top=.6in, bottom=0mm}
\setlength{\parindent}{0em}
\sbox\IBox{\includegraphics{/app/sutta_publisher/images/jambudipa_map.png}}%
\includegraphics[trim=0 0 \dimexpr\wd\IBox-\textwidth{} 0,clip]{/app/sutta_publisher/images/jambudipa_map.png}
\newpage
\includegraphics[trim=\textwidth{} 0 0 0,clip]{/app/sutta_publisher/images/jambudipa_map.png}
\newpage
\restoregeometry

\blankpage%

\setlength{\parindent}{1em}
%
\tableofcontents
\newpage
\pagestyle{fancy}
%
\chapter*{The SuttaCentral Editions Series}
\addcontentsline{toc}{chapter}{The SuttaCentral Editions Series}
\markboth{The SuttaCentral Editions Series}{The SuttaCentral Editions Series}

Since 2005 SuttaCentral has provided access to the texts, translations, and parallels of early Buddhist texts. In 2018 we started creating and publishing our translations of these seminal spiritual classics. The “Editions” series now makes selected translations available as books in various forms, including print, PDF, and EPUB.

Editions are selected from our most complete, well-crafted, and reliable translations. They aim to bring these texts to a wider audience in forms that reward mindful reading. Care is taken with every detail of the production, and we aim to meet or exceed professional best standards in every way. These are the core scriptures underlying the entire Buddhist tradition, and we believe that they deserve to be preserved and made available in the highest quality without compromise.

SuttaCentral is a charitable organization. Our work is accomplished by volunteers and through the generosity of our donors. Everything we create is offered to all of humanity free of any copyright or licensing restrictions. 

%
\chapter*{Preface to \emph{Long Discourses}}
\addcontentsline{toc}{chapter}{Preface to \emph{Long Discourses}}
\markboth{Preface to \emph{Long Discourses}}{Preface to \emph{Long Discourses}}

I grew up in Perth, Western Australia. It’s a city that is often described as “nice”, a somewhat backhanded compliment. The weather is bright and sunny, it’s safe and prosperous, life is good. But it’s not a place where anything particularly happens. Certainly not anything meaningful or interesting to anyone outside of Perth.

As a musician, I would sing songs about New York, about Paris, about Memphis or Singapore or even Darlinghurst. I didn’t know those places, but I knew that they were meaningful places, places deserving of a song. My own life, by contrast, seemed entirely on the surface. The bright sun and clear skies of Perth had no poetry, it banished all the shadows, everything was just so bland. There was nothing to sing about.

You’re sensing a plot twist coming up, and you’re right. In those days—the early 80s—the Perth indie music scene produced its finest band, the Triffids. The singer Dave McComb wrote about things that had happened to me: “he swam out to the edge of the reef, there were cuts along his skin.” I knew what that was like, not because someone had told me, but because I’d done it myself. Suddenly I was living in a world of meaning. I realized that my place, and therefore my life, was just as real and just as meaningful as anything else. The bleaching light of Perth was its meaning, the lack of shadows was its shadow.

When I came to Buddhism, it all seemed so exotic, so distant. I was made to chant in this strange language “Pali”, which I’d never even heard of. It took me a while to even realize that Pali was an actual language, not just a mystical invocation. The monks I met were strange and incomprehensible: who would choose such a life? It had a depth that made my own paltry life pale in comparison.

As I began to study Buddhism in depth, grappling with deep matters, I discovered a range of other scholars and practitioners to learn from. There were the meditation masters of the Thai forest tradition in which I had ordained—Ajahn Chah, Ajahn Mun, Ajahn Thate, Ajahn Lee Dhammadharo. I grew to find sustenance also in the great scholar-monks of the modern Theravada—Venerables \textsanskrit{Ñāṇatiloka}, \textsanskrit{Ñāṇapoṇika}, \textsanskrit{Ñāṇamoḷī}, \textsanskrit{Guṇaratana}, Bodhi, Narada, \textsanskrit{Kaṭukurunde} \textsanskrit{Ñāṇananda}, Buddhadasa, and many others. I struggled to learn the broader history and nature of the Buddhist schools and traditions from scholars such as I.B. Horner, T.W. Rhys Davids, A.K. Warder and Étienne Lamotte. The knowledge and understanding of all these people seemed so lofty, so confident and capable. I devoured everything I could get my hands on.

It never really occurred to me that I might have something to add. I could hardly even manage to master the basics. The masters of the Buddhist tradition appeared as peerless savants, holders of an ancient and impenetrable wisdom.

If you’re sensing another plot twist, you’re right again. Around 1994 I was still a young resident at Wat Nanachat in north-east Thailand when we received a guest, an elderly English gentleman who introduced himself as Maurice Walshe. Of course, I knew that name very well: he had translated the \textsanskrit{Dīgha} \textsanskrit{Nikāya}. I was so excited to meet one of my heroes. He was a charming and witty man, and it was an honor for me to meet him and spend some time together. I am always grateful to him because he made me realize that the Buddhist tradition was created and formed by ordinary people. He had studied Pali but did not regard himself as an accomplished scholar. He undertook the translation at the behest of Venerable Ānandamaitreya—another figure of legend for me. Maurice was very humble about his abilities and his achievement. And it was no false modesty; while his translation was eminently readable, it was not especially accurate. But he did it. And in doing so, helped the Dhamma take one more step forward.

It was after meeting Rod Bucknell and John Kelly, the co-founders of SuttaCentral, in 2004, that I started making my own contributions to the Dhamma through SuttaCentral. Modest as they were, I realized that my talents and skills could help others, as I had been helped. It took a long while, much learning and many trials, but eventually I dared imagine that maybe I could make my contribution to the corpus of Pali translations. It would surely be imperfect and inadequate, but perhaps I had something to give.

Those of us who have enjoyed the sweet taste of the Dhamma owe a debt of gratitude to all those who have made it possible; to all the teachers, the supporters, the donors, the monks and nuns and layfolk, the scholars, the meditators, the builders and cooks and plumbers and weavers, the artists and storytellers, the repairers of leaky roofs and the kindlers of lamps. There is not a single one who can hold the whole tradition. But I believe that there is not a single one who has nothing to offer.

Allow me to indulge in a further recollection of my days in the indie music scene. One song that has stuck with me is \textit{Song of the Siren}, written by Tim Buckley, but known from the version by This Mortal Coil. In three short verses it tells the story of the protagonist lost on “shipless oceans”, who was drawn in and given shelter by one they came to love. Just as they thought they were safe, the beloved seemed to turn away, leaving them “broken lovelorn on your rocks”. Despairing and confused, they considered ending it all. Until at last, they realized: now it was their turn. They could not live forever relying on the other to offer shelter and protection. When they were lost, they had been saved, and now they called to the other, “swim to me, let me enfold you”.

As a person of faith, I believe that the Buddha was a perfected human being. The Buddhist tradition, on the other hand, is made up of people who are usually notably imperfect. Sometimes we feel inspiration and uplift, other times disappointment or disillusionment. I reached a point of frustration when I knew that, for all the efforts of many people, we were still not able to make all the Suttas available in translation for free. It seemed wrong, and I didn’t know what to do. It was then that I realized that it was my turn to offer shelter.

%
\chapter*{A Reader’s Guide to the Pali Suttas}
\addcontentsline{toc}{chapter}{A Reader’s Guide to the Pali Suttas}
\markboth{A Reader’s Guide to the Pali Suttas}{A Reader’s Guide to the Pali Suttas}

\scbyline{Bhikkhu Sujato, 2019}

The suttas of the Pali Canon (\textsanskrit{Tipiṭaka}), especially the four main \textit{\textsanskrit{nikāyas}}, are essential reading for anyone who wishes to understand the Buddha and his teaching. They have been preserved and passed down in the Pali language by the \textsanskrit{Theravāda} tradition of Buddhism as the word of the Buddha.

These texts were originally passed down orally, by generations of monks and nuns who memorized them and recited them together. Around 30 BCE they were written down in the \textsanskrit{Āluvihāra} in Sri Lanka, and subsequently were transmitted in manuscripts of palm leaves.

From the 19th century, the manuscripts were edited and published as modern editions in sets of books. In addition, the Pali text was translated into a number of modern languages, including Thai, Burmese, Sinhalese, and English.

The word \textsanskrit{Tipiṭaka} means “Three Baskets”. The Basket of Discourses is traditionally listed as the second of the three. The four \textit{\textsanskrit{nikāyas}} make up the bulk of the Basket of Discourses. Here is how they are situated within the canon as a whole.

\begin{itemize}%
\item Vinaya \textsanskrit{Piṭaka} (Basket of Monastic Law)%
\item Sutta \textsanskrit{Piṭaka} (Basket of Discourses)
\begin{itemize}%
\item \textbf{\textsanskrit{Dīgha} \textsanskrit{Nikāya}} (Long Discourses)%
\item \textbf{Majjhima \textsanskrit{Nikāya}} (Middle Discourses)%
\item \textbf{\textsanskrit{Saṁyutta} \textsanskrit{Nikāya}} (Linked Discourses)%
\item \textbf{\textsanskrit{Aṅguttara} \textsanskrit{Nikāya}} (Numbered Discourses)%
\item Khuddaka \textsanskrit{Nikāya} (Minor Discourses)%
\end{itemize}

%
\item Abhidhamma \textsanskrit{Piṭaka} (Basket of Systematic Treatises)%
\end{itemize}

Similar collections are found in ancient Chinese translations, and substantial portions of them are also in Sanskrit and Tibetan. The diverse collections of scriptures arose among the Buddhist communities who spread across greater India in the centuries following the Buddha, especially under the Buddhist emperor, Ashoka. These missions are documented in the ancient chronicles of Sri Lanka as well as the Vinaya commentaries in Pali and Chinese, and have been partially corroborated by modern archaeology.

SuttaCentral hosts almost all of these texts and provides comprehensive parallels showing the relations between them. A comparative understanding based on the full spectrum of these texts is essential for any study of early Buddhism. The Chinese Buddhist canon, in particular, contains a vast amount of translations of early texts, and in terms of quantity it outweighs the Pali texts by some margin.

For many reasons, though, the Pali texts will always retain a special place for those who wish to understand what the Buddha taught.

\begin{itemize}%
\item They are the only complete set of scriptures of an early school of Buddhism.%
\item They are by far the largest body of texts to survive in an early Indic dialect.%
\item They are accompanied by an extensive and detailed set of ancient commentaries (\textit{\textsanskrit{aṭṭhakathā}}).%
\item They are, for the most part, linguistically clear, well-edited, and readily comprehensible.%
\end{itemize}

Moreover, the Pali texts are the core scriptures of a living tradition, the \textsanskrit{Theravāda} school found in Sri Lanka, Thailand, Myanmar, Bangladesh, Cambodia, Laos, India, China, and Vietnam. To this day they are recited, taught, studied, and practiced daily, and are regarded in the traditions as being a reliable witness to the teachings of the Buddha himself.

Among the Pali texts, it is the four \textit{\textsanskrit{nikāyas}} that command the most attention. It is here that we find extensive and definitive explanations of Buddhist teachings, as well as the living personality of the Buddha and his immediate disciples.

By comparison, the Vinaya \textsanskrit{Piṭaka} details the life of the monastic communities, and in addition it reveals much about the history and social background; but it contains only a few teaching passages. The Abhidhamma \textsanskrit{Piṭaka} is made up of systematic treatises that were composed in the centuries following the Buddha’s passing. And the books of the Khuddaka \textsanskrit{Nikāya} are very mixed. There are six fairly short books that are supplements to the main four \textit{\textsanskrit{nikāyas}}, mostly in verse: the \href{https://suttacentral.net/dhp}{Dhammapada}, \href{https://suttacentral.net/ud}{Udāna}, \href{https://suttacentral.net/iti}{Itivuttaka}, \href{https://suttacentral.net/snp}{Sutta Nipāta}, \href{https://suttacentral.net/thag}{Theragāthā}, and \href{https://suttacentral.net/thig}{Therīgāthā}. However, most of the other books in the Khuddaka are later, and represent a phase of Buddhism a few centuries after the Buddha.

\section*{About These Guides}

I have prepared these guides in order to support a student who wishes to develop a deeper understanding of the \textit{\textsanskrit{nikāyas}}. They accompany my translations of the four \textit{\textsanskrit{nikāyas}} as found on SuttaCentral. This general guide is meant to be read first, as it covers a variety of issues that are common to all the \textit{\textsanskrit{nikāyas}}. The four \textit{\textsanskrit{nikāyas}} are a highly unified body of texts, sharing most of the significant doctrinal passages. The general information presented here is fleshed out in individual essays on each of the four \textit{\textsanskrit{nikāyas}}, which highlight the shifts in emphasis and orientation from one collection to the next. These may be read in any order. While the guides for the specific \textit{\textsanskrit{nikāyas}} naturally focus on the texts in that \textit{\textsanskrit{nikāya}}, this is not adhered to rigidly, and they may refer also to passages found elsewhere.

Summaries of major doctrinal themes may be found mostly in the \textit{\textsanskrit{nikāya}} guides, especially that for the \textsanskrit{Saṁyutta} \textsanskrit{Nikāya}, rather than here. However, I would urge a degree of caution when it comes to summaries, including my own. The true joy of the suttas is in the undigested teachings, in that raw moment when the Buddha encounters a person in suffering and helps them, not by giving them a digested abstract, but by reaching out to them as people. Summaries and surveys are best treated as starting points for discovery, rather than as definitive treatises.

I almost completely avoid sideways glances at the various Chinese and other parallels. Understanding these relations is critical, and the entire basis of SuttaCentral is founded on this fact. But the number of texts is very large, and the complexity of the subject is daunting. I fear that if I were to deal with parallels in any kind of depth, these essays would never be completed; and if they were, they would have become unreadably complex. Hence I have set myself a more manageable scope, sticking to the Pali texts, on the understanding that most things probably apply to the parallels as well. The reader can easily check the parallels on SuttaCentral if they wish.

Among students of the suttas, the names of these collections are often abbreviated to “\textsanskrit{Dīgha}”, “Majjhima”, and so on, just as the word “Sutta” is often omitted from sutta titles. Strictly speaking, it’d be best to use the Pali title when referring to the original text, and the translated title when referring to the translation; but this distinction is often overlooked.

\section*{An Approachable Translation}

In 2015 I determined to create freely available translations of the main Pali discourses, so that all of these teachings might be made freely available in a clear, consistent, and accurate rendition. My aim was to translate the four main \textit{\textsanskrit{nikāyas}} as well as the 6 early books of the Khuddaka \textsanskrit{Nikāya}: \textsanskrit{Theragāthā}, \textsanskrit{Therīgāthā}, \textsanskrit{Udāna}, Itivuttaka, Dhammapada, and \textsanskrit{Suttanipāta}. I did this so that these astonishing works of ancient spiritual insight might enjoy the wider audience they so richly deserve.

In considering my translation style, I reflected on the standard trope that introduces the prose suttas: a person “approaches” the Buddha to ask a question or hear a teaching. It’s one of those passages that became so standard that we usually just pass it by. But it is no small thing to “approach” a spiritual teacher. It takes time, effort, curiosity, and courage; many of those people would have been more than a little nervous.

How, then, would the Buddha respond when approached? Would he have been archaic and obscure? Would he use words in odd, alienating ways? Would you need to have another monk by your side, whispering notes into your ear every second sentence—“He said this; but what he really meant was…”?

I think not. I think that the Buddha would have spoken clearly, kindly, and with no more complication than was necessary. I think that he would have respected the effort that people made to “approach” his teachings, and he would have tried the best he could, given the limitations of language and comprehension, to explain the Dhamma so that people could understand it.

An approachable translation expresses the meaning of the text in a manner that is simple, friendly, and idiomatic. It should not just be technically correct, it should sound like something someone might actually say.

Which means that it should strive to dispense entirely with the formalisms, technicalities, and Indic idioms that has dominated Buddhist translations, into which English has been coerced by translators who were writing for Indologists, linguists, and Buddhist philosophers. Such translations are a death by a thousand papercuts; with each obscurity the reader is distanced, taken out of the text, pushed into a mode of acting on the text, rather than being drawn into it.

That is not how those who listened to the Buddha would have experienced it. They were not being annoyed by the grit of dubious diction, nor were they being constantly nagged to check the footnotes. They were drawn inwards and upwards, fully experiencing the transformative power of the Dhamma as it came to life in the words of the Awakened One. We cannot hope to recapture this experience fully; but at least we can try to not make things worse than they need to be.

At each step of the way I asked myself, “Would an ordinary person, with little or no understanding of Buddhism, be able to read this and understand what it is actually saying?” To this end, I have favored the simpler word over the more complex; the direct phrasing rather than the oblique; the active voice rather than the passive; the informal rather than the formal; and the explicit rather than the implicit.

Still, it should not be thought that these are loose adaptations or simplifications. There is a place for re-imaginings of ancient texts, and for versions that strip the complexity to focus on the main point. But my work is intended as a full and accurate translation, one that omits nothing of substance. It is just that I try to express this without undue complexity.

I still feel I am a long way from achieving my goal. No-one is more aware than the translator of the compromises and losses along the way. Consistency, clarity, correctness, and beauty all make their competing claims, and only rarely, it seems, can all be met. It is a work in progress, and I will probably be making corrections and adjustments for many years to come.

I have been especially influenced in this approach by my fellow monks, Ajahn Brahm and Ajahn Brahmali. It is from Ajahn Brahm that I have learned the virtue of plain English; of the kindness of speaking such that people actually understand. And with Ajahn Brahmali, who has been working on Vinaya translations at the same time, I have had many illuminating discussions about the meaning of various words and phrases. He said one thing that stuck in my mind: a translation should mean \emph{something}. Even if you’re not sure what the text means, we can be sure that it had some meaning, so to translate it based purely on lexical correspondences is to not really translate it at all. Say what you think the text means, and if you make a mistake, fix it.

\section*{Plan Your Route or Wander in the Garden}

The Buddha’s teaching is a graduated one, leading from simple principles to profound realizations. This pattern is found within almost all of the discourses in one way or another. However it does \emph{not} apply to the collections of discourses. From collection to collection or discourse to discourse there is no graduation in difficulty, no build up of assumed knowledge in the student.

On the contrary, the \textsanskrit{Dīgha} \textsanskrit{Nikāya} begins with the \textsanskrit{Brahmajāla} Sutta, while the Majjhima \textsanskrit{Nikāya} begins with the \textsanskrit{Mūlapariyāya} Sutta, both of which are among the most profound and difficult discourses in the whole canon. A student who dives in unwarily will suddenly find themselves in very deep waters indeed.

If we wish to build up knowledge step by step, we can’t rely on simply reading the suttas in order. Students often find it helpful to use a structured reading guide such as that offered here. On SuttaCentral, we offer several other approaches.

Having said which, there’s nothing wrong with simply plunging in at random, so long as you don’t expect everything to make sense at first. Take your time and enjoy wandering about. Don’t worry too much about things that seem odd or unexpected. Usually you’ll find that obscure or difficult ideas are explained somewhere else; discovering those unexpected connections is one of the joys of reading the suttas.

In these introductory essays, you will find many references to the suttas. You don’t need to look up each reference to understand the essays. But if you do, you will get a reasonable survey of many important texts, and learn how to find the passage that you need. I suggest reading each essay on its own first, and then a second time, looking up and reading the sutta references as you go.

\section*{Looking Up References}

When you delve into sutta reading, you’ll notice that texts and passages are referenced in sometimes confusing ways. On SuttaCentral we employ a simple and widely adopted form of semantic referencing. By “semantic” references, we mean that the reference numbers are based on meaningful divisions in the texts themselves.

For the four \textit{\textsanskrit{nikāyas}}, this means:

\begin{itemize}%
\item In the \textsanskrit{Dīgha} and Majjhima, texts are referenced by the ID for the collection (DN and MN respectively) and the sutta number, counted in a simple sequence from the beginning.%
\item In the \textsanskrit{Saṁyutta}, the collection ID (SN), \textit{\textsanskrit{saṁyutta}} (thematic group), and sutta.%
\item In the \textsanskrit{Aṅguttara}, the collection ID (AN), \textit{\textsanskrit{nipāta}} (numerical group), and sutta.%
\end{itemize}

More granular referencing is provided by section numbers. These follow pre-existing conventions:

\begin{itemize}%
\item In the \textsanskrit{Dīgha}, the section numbers of the PTS Pali edition, which have been widely adopted in translations.%
\item In the Majjhima, the section numbers introduced by Bhikkhu \textsanskrit{Ñāṇamoḷi}, and used in Bhikkhu Bodhi’s edition.%
\item In the \textsanskrit{Saṁyutta} and \textsanskrit{Aṅguttara}, the paragraph numbers as found in the \textsanskrit{Mahāsaṅgīti} Pali text.%
\end{itemize}

Each of these is further subdivided so that each section contains a number of “segments”, a short piece of text usually about a sentence or so long. In my translations, the segments are matched with the underlying Pali text.

In our system, the numbers following a colon represent the section and segment numbers, that is, the subdivisions within a sutta. So, for example:

\begin{itemize}%
\item DN 3 means “the third discourse in the collection of Long Discourses (\textsanskrit{Dīgha} \textsanskrit{Nikāya})”.%
\item MN 43:3 means “discourse 43, section 3 in the collection of Middle Discourses (Majjhima \textsanskrit{Nikāya})”.%
\item MN 43:3.7 means “discourse 43, section 3, segment 7 in the collection of Middle Discourses (Majjhima \textsanskrit{Nikāya})”.%
\item SN 12.2:6.2 means “discourse 2, section 6, segment 2 in the 12th \textit{\textsanskrit{saṁyutta}} of the collection of Linked Discourses (\textsanskrit{Saṁyutta} \textsanskrit{Nikāya})”.%
\end{itemize}

You may encounter various other referencing systems. In academic works, texts are often referenced by volume and page of the Pali Text Society (PTS) edition of the original Pali. This is a regrettable and clumsy convention, since it binds references to a specific paper edition. I hope it is swiftly abandoned in favor of proper semantic references. However, the PTS volume/page numbers are displayed on SuttaCentral in case you need to look up a legacy reference.

Traditionally, the texts were—and often still are—referenced the long way: by \textit{\textsanskrit{nikāya}}, then \textit{\textsanskrit{saṁyutta}} or \textit{\textsanskrit{nipāta}} and/or \textit{\textsanskrit{paṇṇasaka}} (as applicable), then \textit{vagga} and \textit{sutta}. This system is helpful when using manuscripts, as you can narrow your search step by step through the manuscript to find what you need. On the web, or even in books, however, it is unnecessary. Nevertheless, you can use the traditional navigation structure in our sidebar if you wish.

\section*{Elements of Structure}

As students of Buddhist texts we are interested in the content, in learning what the Buddha and his disciples had to say and how they lived. However, due to the manner in which the texts are arranged, we quickly discover that it’s not easy to know how different texts relate to each other. So while it may seem dry, it is worth spending a little time to consider the \emph{structure} of the texts.

Early Buddhist texts were organized, not for reading, but for oral recitation and memorization. The overriding concern was to divide the texts into chunks that could be memorized and recited together. Since the texts were preserved in memory, they were largely “random access”: a skilled student could instantly recall a passage from anywhere in the texts, without having to flip through the pages or look up an index. In this way, the earliest system of organization is a little similar to how we find information today through a search engine.

It follows from this that we cannot expect early Buddhist texts to be structured sequentially like a modern book. But this does not mean that the collections are random or chaotic. They follow their own logic, which can be discerned if the texts are approached sympathetically.

Here are some of the structural or formal elements you will encounter in the early Buddhist texts.

\subsection*{Imagery and Narrative}

The suttas frequently employ an ABA pattern. A statement is made; a simile is given; and the statement is repeated.

This formal pattern is highly effective in reinforcing learning. First we get the basic idea. But abstract philosophical or psychological statements are hard to understand and remember without any context, so the Buddha illuminates his teaching with a simile. He ends by driving the message home once again.

The range of similes in the suttas is truly astounding. The Buddha had an uncanny ability to effortlessly summon an apt comparison from anything that he saw around him. The similes also convey a great deal of incidental detail regarding life and culture in the Buddha’s day, and, more importantly, they show how the Dhamma teaching makes sense in its context. Most of the classic Buddhist images that are familiar today trace their roots to similes used by the Buddha in the early texts.

Sometimes the similes are extended to a brief parable or fable. Curiously enough, however, we rarely see the Buddha engage in story-telling of any length.

Where narratives are developed in some detail, they are typically as part of the background story (\textit{\textsanskrit{nidāna}}) rather than in the Buddha’s teaching as such. It is an elementary principle of historical scholarship that the background story is of a somewhat later date than the main doctrinal material. Such stories vary considerably in the parallels, showing that the traditions treated narrative more flexibly than doctrine.

\subsection*{Repetition}

It won’t take long before you notice that the suttas tend to be repetitious. \emph{Very} repetitious. This can be a major hurdle for a new reader, so let’s take a little time to consider what is happening.

Like so many patterns found in nature, the repetitions are \emph{fractal}. That is, they occur at every level: the word, the phrase, the sentence, the paragraph, the passage, the whole text, even the group of texts. This shows that the repetition is not something alien to these texts, not something forced on them by an over-zealous editor, but is intrinsic from the beginning.

But why? The thing to remember is that the texts were formed in an oral tradition. And in an oral tradition, repetition works very differently than it does in writing. When you \emph{read} a repetition, it can be annoying; it feels like a waste of time, and you want to skip over it. But when \emph{reciting}, repetition has exactly the opposite effect. It is soothing and relaxing. The parts that are different take more work, you have to exercise your memory; but when the repeated passage comes around—like the chorus of a song—you relax into the flow of the chanting. Repetitions give the reciter space to be at ease and contemplate. Reciting a highly repetitive text becomes a form of meditation, where you reflect the meaning and apply it to your experience as you recite.

But in addition to this spiritual aspect, repetition has a definite practical purpose: preservation. By saying the same thing again and again, identically or with small variations, the reciters were constantly checking their memories, ensuring the accuracy of the texts. And if a text was lost, there is always another similar passage somewhere else. Thus the repetitions ensured the long-term survival of the Dhamma by creating backups of important information in multiple places, retained in the minds of Buddhist practitioners.

Understanding the historical role of repetitions, however, doesn’t help us when we just want to read a sutta. What are we to do? Well, there is no single way to read a sutta. Some people prefer to read them in full, contemplating each repetition. Others read them more briefly, getting to the important point. You’ll figure out a way that suits you. But when you understand the role of repetitions, hopefully you will not find them such an obstruction.

\subsection*{Abbreviation}

The flipside of repetition is abbreviation. Since the repetitions are so abundant, they are often abbreviated. Such abbreviation is not a modern invention; it is found throughout the manuscripts, and indeed there is no edition that fully spells out all the repetitions. The Pali texts have their own convention for indicating abbreviations, marked with the syllable \textit{pe}, itself an abbreviation of \textit{\textsanskrit{peyyāla}}.

Generally speaking, the abbreviations in the Pali editions, and the occasional instructions on how to spell out the full text, are sensible and fairly consistent between editions. Modern translations follow the manuscript tradition, but not slavishly. Sometimes the translation will spell out abbreviated passages, or else abbreviate passages spelled out in the original.

Abbreviations are both “internal” or “external”. By internal abbreviation, I mean that there is enough information in the text itself to fully reconstruct it. Typically only the first and last items in a list are spelled out in full, and for the rest, only the key terms are mentioned. Here is an example from \href{https://suttacentral.net/sn22.137}{SN 22.137}:

\begin{quotation}%
Form is impermanent; you should give up desire for it. Feeling … Perception … Choices … Consciousness is impermanent; you should give up desire for it.

%
\end{quotation}

In external abbreviations, an abbreviated passage cannot be fully reconstructed from the context, but requires looking up another text to fill in the blanks. This is another example of how the oral tradition differs from written texts. A reciter would obviously know, say, the formula for the four noble truths, so there is no need to write it every time; just enough to jog the memory. But in modern editions, especially on the web, a reader can access a specific text from anywhere, and may never have encountered the abbreviated passage before. For this reason I tried to reduce the number of external abbreviations in my translations.

\subsection*{Titles}

Buddhist manuscripts rarely have titles at the start like modern texts. Rather, the title is recorded at the end. In modern editions, these titles have been added at the start for clarity.

In many cases, especially in the titles of suttas and \textit{vaggas}, what we actually have in the manuscript is not really a title as such, but a key word found in the summary verse (\textit{\textsanskrit{uddāna}}) found at the end of a \textit{vagga} or other division. These verses were inserted by the redactors of the canon in order to help keep the contents organized, much like a Table of Contents. However, the summary verses do vary to some extent between editions, so the titles of suttas are not always consistent. In addition, some suttas are assigned more than one title in the text itself—for example \href{https://suttacentral.net/dn1}{DN 1} \textit{The Prime Net} (\textit{\textsanskrit{Brahmajālasutta}})—or there are spelling variations. So take care, for it is quite common to find different titles for the same text.

\subsection*{Textual Divisions}

\subsubsection*{Vagga (“Chapter”)}

The \textit{vagga} is the most widespread and distinctive structural unit in early Buddhist texts. It usually consists of ten texts, which may be ten discourses, ten verses, ten rules, and so on. The number ten is adhered to fairly consistently, although on occasion a \textit{vagga} may contain more or less than ten.

The \textit{vagga} is often little more than a convenient grouping to help organize the discourses neatly. In such cases, a \textit{vagga} is usually just named after its first discourse.

However, it is also common to find that a \textit{vagga} collects discourses with a loose thematic unity. For example, in the \href{https://suttacentral.net/dn-silakkhandhavagga}{Chapter on the Full Spectrum of Ethics} (\textit{\textsanskrit{Sīlakkhandhavagga}}) of the \textsanskrit{Dīgha} \textsanskrit{Nikāya} (thirteen discourses in this case), almost all the texts deal with the “gradual training” of ethics, meditation, and wisdom.

In some cases, a \textit{vagga} in Pali may parallel a similar \textit{vagga} in another language. For example, the famous \href{https://suttacentral.net/snp-atthakavagga}{Chapter of the Eights} (\textit{\textsanskrit{Aṭṭhakavagga}}) of the Sutta \textsanskrit{Nipāta} exists in Chinese translation, though the Sutta \textsanskrit{Nipāta} as a whole does not. Similarly, the \textit{\textsanskrit{Sīlakkhandhavagga}} of the \textsanskrit{Dīgha} \textsanskrit{Nikāya} has parallels in both the Dharmaguptaka (Chinese) and \textsanskrit{Sarvāstivāda} (Sanskrit) texts of the \textsanskrit{Dīgha}.

Occasionally the word \textit{vagga} is applied to a larger textual unit, one that includes a number of sections, each of which composed of “little” \textit{vaggas}. Examples of such nested \textit{vagga} structures include the \textsanskrit{Saṁyutta} \textsanskrit{Nikāya} and the Khandhakas of the Pali Vinaya.

\subsubsection*{\textsanskrit{Paṇṇāsaka} (“Fifty”)}

The word \textit{\textsanskrit{paṇṇāsa}} means “fifty”, and a \textit{\textsanskrit{paṇṇāsaka}} is a group of approximately fifty suttas, or five \textit{vaggas}. It is used to organize collections that contain many \textit{vaggas}. Most of the collections with large numbers of discourses use this structure, for example the “\href{https://suttacentral.net/mn-mulapannasa}{Root Fifty}” of the Majjhima \textsanskrit{Nikāya} (\textit{\textsanskrit{Mūlapaṇṇāsa}}). The \textit{\textsanskrit{paṇṇāsaka}} is purely for convenience and does not correspond to any meaningful division of the text.

\subsubsection*{\textsanskrit{Nipāta} (“Group”)}

The literal meaning of \textit{\textsanskrit{nipāta}} is “fallen down”, and it is a generic term for texts that have been placed together. In the \textsanskrit{Aṅguttara}, it is used for each division of texts gathered together by number: the \href{https://suttacentral.net/an1}{group of discourses consisting of one item}, and so on. Elsewhere it is used, for example, in the title of the \href{https://suttacentral.net/snp" class="text-root}{Sutta Nipāta}, the “Group of Discourses”.

\subsubsection*{\textsanskrit{Saṁyutta} (“Collection of Linked Discourses”)}

Whereas the \textit{\textsanskrit{nipāta}} is quite generic, the \textit{\textsanskrit{saṁyutta}} has a more specific meaning: texts collected according to a similar theme or subject matter. The \textsanskrit{Saṁyutta} \textsanskrit{Nikāya} consists of 56 such collections. For example, the fourteenth \textit{\textsanskrit{saṁyutta}} contains \href{https://suttacentral.net/sn14}{39 discourses on the topic of the elements}.

\subsubsection*{\textsanskrit{Nikāya} or Āgama (“Division”)}

The largest structural unit, usually known as \textit{\textsanskrit{nikāya}} in the Pali tradition of the \textsanskrit{Theravāda}, and as \textit{\textsanskrit{āgama}} in the northern traditions. However, the term \textit{\textsanskrit{āgama}}, while it has fallen into disuse in modern \textsanskrit{Theravāda}, is found quite commonly in the Pali commentaries.

Collections similar to the four \textit{\textsanskrit{nikāyas}} as found in the Pali are found in all the other schools. However, while the overall nature of the collections is similar, and they are organized in similar ways, the detailed content varies considerably. It frequently happens that a sutta found in the Majjhima of one school, for example, may be found in the \textsanskrit{Saṁyutta} or \textsanskrit{Dīgha} of another school. In addition, the internal sequence of texts is quite different. Thus it seems that the \textit{\textsanskrit{nikāyas}} or \textit{\textsanskrit{āgamas}} functioned more as organizational guidelines than as fixed units.

The fifth Pali \textit{\textsanskrit{nikāya}}, the Khuddaka \textsanskrit{Nikāya}, is more flexible and varies more between traditions. It seems it originated as a place for collecting verses and minor texts not gathered elsewhere. However the Pali collection became a handy place to include later texts, so it has now become the biggest of all the \textit{\textsanskrit{nikāyas}}. While there are occasional references to a similar collection in the northern schools, none exist in that form today. Nevertheless, many of the individual texts of the Khuddaka have parallels, especially the Dhammapada, which survives in many different editions.

\section*{In the Buddha’s Day: A Time of Change}

Each discourse begins with a brief statement saying that at “one time” the Buddha was staying at a particular place. In this way the redactors of the texts were concerned to locate the Buddha and his teachings in a specific historical and cultural context. Modern scholars have been able to reconstruct a fairly reliable picture of the Buddha’s life and times, relying on the early Buddhist texts, as well as what information may be gleaned from Brahmanical and Jaina scriptures.

Archeology is, unfortunately, of limited use, as there are few archeological remnants from the Buddha’s day. In fact, before the time of Ashoka—perhaps 150 years after the Buddha—there are very few remains at all of ancient India, until the time of the Indic Valley civilization, many centuries earlier. For the period we are interested in, what has been found consists of some pottery and similar small implements, as well as a few remnants of fortifications around \textsanskrit{Kosambī}. The paucity of evidence is due to two main reasons. The first is that buildings at the time were mostly of wood or other perishable materials. And the second is that archeological work in India has been very spotty and incomplete.

The Buddha lived in the 5th century BCE in the Ganges plain in northern India. The exact dates of his birth and death are uncertain, but modern scholarly opinion tends to place his birth around 480 BCE and his death 80 years later at around 400 BCE. He was born in Lumbini and grew up in Kapilavatthu, both of which belong to the Sakyan republic, straddling the modern border of India and Nepal. His family name was Gotama; the earliest texts do not mention his personal name, but tradition says it was Siddhattha.

After his Awakening, the Buddha traveled about the Gangetic plain. The area he traversed was part of the cultural/political region known as the “sixteen nations” (\textit{janapada}). This spanned from modern Delhi to the north-west, the Bangladesh border to the east, the Himalayan foothills to the north, the Deccan to the south, and Ujjain to the south-west. Most of his time was spent around the cities of \textsanskrit{Sāvatthī} in the kingdom of Kosala and \textsanskrit{Rājagaha} in the kingdom of Magadha. Despite the proliferation of local legends in most Buddhist countries, the Buddha never ventured outside this area.

It was not just the Buddha who was restricted to this region. It seems that trade and other close cultural contacts normally took place within this region, too. Occasional references to places further afield—southern India or the Greeks—were vague and often legendary. It was in the century after the Buddha passed away that the kingdoms of northern India were unified and regular international trade routes were opened, first to Europe, and, a couple of centuries later, to south-east Asia and China.

\section*{Economics and Politics}

Though cities and urban life feature prominently in the texts, they are still on a small scale. The economy was largely rural, with farming playing a prominent role.

However there are lists of occupations in the Pali canon that show a diverse range of employment—accountants, jewelers, builders, soldiers, doctors, government officials, and many more.

The rise in diversity of employment was linked to the growth of cities, which in turn is associated with the appearance of new technologies. The archeological record, though thin, has furnished us with records of two significant innovations: forges for iron, and a kind of fine pottery known as Northern Black Polished Ware. These new developments attest to a growing mastery in the industrial use of fire, something that the suttas mention in several memorable similes.

Technological innovation drove economic growth. We frequently hear of wealthy individuals, employing large staffs and managing properties or businesses. There was enough economic surplus to support a large class of spiritual seekers. Such ascetics made no material contributions to society; their value lay in spiritual and ethical development.

These technological and economic shifts were mirrored in the political sphere. There were two major forms of governance. Traditional clans such as the Sakyans or the Vajjians followed an ancient restricted form of democracy, where decisions were made in a town council, and the clan elected a leading member as temporary ruler. Other nations, like Kosala and Magadha, had formed a more familiar kind of kingdom, with an absolute hereditary monarch. While the Buddha evidently favored the democratic ideals under which he grew up, and after which he modeled the governance of the Sangha, it was the kingdoms that were growing in economic and military dominance. During the Buddha’s lifetime, there were repeated skirmishes between Kosala and Magadha, vying for dominion over the ancient sacred city of Varanasi.

Of even greater significance, towards the end of his life, Magadha was announcing its intentions to invade the Vajjian republic. History attests to the success of this policy: in the decades following the Buddha’s death, Magadha conquered virtually all of the sixteen nations, establishing an unquestioned supremacy over the region, and establishing pan-Indian trade networks. So powerful was the resulting kingdom that Alexander the Great’s troops rebelled at the mere rumor of Magadhan military might.

\section*{Social Life}

The growing complexity of economic and political life required corresponding changes in social roles and responsibilities. Like any society in a time of change, people in the time of the Buddha were trying to balance their traditional values and structures with the new realities. It seems that people were for the most part reasonably well off. Still, poverty and famine, along with injustice, banditry, and economic uncertainty, were present and posed a very real threat. We hear frequent laments about how unpredictable wealth is, whether the older forms of wealth in cattle and land, or the newer forms in money and career.

The Buddha was not a social revolutionary and did not overtly agitate for an overthrow of social systems, even those he recognized as unjust. Typically he would argue for a more just and fair implementation of existing norms. For example, rather than saying all societies should be democratic, he spoke of the moral duty of a king to look after his people.

A man was expected to earn a living so as to maintain and protect his family. He should use his earnings to provide his family with both essentials and luxuries, and to treat workers with kindness and decency, while not neglecting to assign some funds for savings, and some for donations to charity.

A woman’s traditional role was to marry and bear children. Aside from this, her options were limited. We rarely hear of professional women aside from sex workers. In this context, the opportunity to become a nun allowed women to pursue their own spiritual and intellectual development, to act as leaders and teachers, and to be respected and supported in that role.

Many of the more extreme manifestations of sexual discrimination are not found in the early texts. We find no mention of child brides, immolation of widows, or the essential subjugation of women to men.

India had not yet developed a full-fledged caste system. But there was a much simpler fourfold division of society:

\begin{description}%
\item[Aristocrats (\textit{khattiya})] Owners of land (\textit{khetta}), typically wealthy and powerful, engaged in statecraft, war, and production. The Buddha was from an aristocratic clan. The aristocrats placed themselves at the highest tier of society.%
\item[Brahmins (\textit{\textsanskrit{brāhmaṇa}})] Members of a hereditary priestly class. The brahmins were custodians of sacred texts called “Vedas”, and performed rituals and ceremonies to their various deities. However by the time of the Buddha many brahmins were simply engaged in ordinary worldly livelihoods and their religious role was secondary. They believed themselves to be the children of God (\textsanskrit{Brahmā}).%
\item[Merchants (\textit{vessa}):] Engaged in trade and commerce.%
\item[Workers (\textit{sudda})] Performed physical labor.%
\end{description}

Not everyone fit into this neat scheme. We hear reference to outcastes and various tribal peoples. In addition, there were slaves or bonded servants. Finally, the ascetics (\textit{\textsanskrit{samaṇa}}) such as the Buddha saw themselves as having left behind all such notions of caste.

\section*{The Many Spiritual Paths of Ancient India}

Change in the Buddha’s day was not limited to the worldly sphere. The religious life of ancient India was equally dynamic. For this reason it would be a mistake to assume that India in the time of the Buddha was primarily a Hindu society. Some of the elements that make up modern Hinduism may be discerned, but Indian religion, like spiritual and religious practice everywhere, has always been in a state of flux and evolution.

In the time of the Buddha, and indeed even to this day, the ancient pre-Buddhist Vedas formed the basis for the spiritual life of the brahmins and those who followed them. Rituals such as the \textit{agnihotra}, the daily pouring of ghee onto the fire as an offering to the fire-god Agni—originated long before the Aryan people even came to India, and continue to be performed today.

Nevertheless, many of the old gods featured in the Vedas had vanished by the time of the Buddha, and many of the famed deities of later Hinduism had not yet appeared. Those who do appear take on a markedly different aspect; prominent gods such as Vishnu (Pali: \textit{\textsanskrit{Veṇhu}}) or Shiva (Pali: \textit{Siva}) appear in minor roles, and a warrior like Sakka (AKA Indra) appears as an apostle of peace. There were no temples, no images, and no cult of devotion (\textit{bhakti}). There is no mention of a system of \textit{avatars}, nor of familiar concepts from modern Hindu-inspired spirituality such as \textit{\textsanskrit{śakti}}, \textit{\textsanskrit{kuṇḍalinī}}, chakras, or yoga exercises.

Moreover, when we look at the aspects of modern Hinduism that were present at the time, many of them are completely separate from each other. No-one considered, for example, the worship of a local dragon (\textit{\textsanskrit{nāga}}) to have anything to do with the rites of the brahmins. The outstanding feature of Hinduism—the great synthesis of religious and philosophical ideas and practices, from simple animism to profound non-dualism—had not yet been undertaken. Different strands of religious life were quite distinct and were not considered to be part of the same path.

Thus historians do not refer to the brahmanical religion of the time as Hinduism, but rather as Vedism or Brahmanism. It was nearly a thousand years later that the movement recognizable as modern Hinduism became prominent in India. To be sure, much of Hinduism is drawn from the Vedas, in the same way that much of Catholicism is drawn from the Hebrew scriptures that Christians call the Old Testament. But were you to meet Abraham or Noah and address them as “Catholics”, they would not know what you are talking about. And the Indians of the Buddha’s time would have been equally unfamiliar with the very idea of “Hinduism”.

All this notwithstanding, there is an oft-repeated claim to the effect that the Buddha “was born, lived, and died a Hindu”, attributed to the great pioneer of Indology, Thomas Rhys Davids. While it is true that he did write this, it was in an early work, page 116 of \textit{Buddhism: its history and literature}, a lecture series published in 1896. But by 1912 his views had changed, for on page 83 of \textit{Buddhism: Being a sketch of the life and teachings of Gautama, the Buddha}, he said:

\begin{quotation}%
Gautama was born, and brought up, and lived, and died a typical Indian. Hinduism had not yet, in his time, arisen.

%
\end{quotation}

Rhys Davids emphasizes that the Buddha did not have an antagonistic relation to the Brahmanical religion. His purpose was not religious reform, but freedom from suffering. However, on page 85 of the same work he comments:

\begin{quotation}%
In the long run the two systems were quite incompatible. … Gautama’s whole training lay indeed outside of the ritualistic lore of the Brahmanas and the brahmins. The local deities of his clan were not Vedic.

%
\end{quotation}

The lesson here is that we must avoid reading modern conditions back into ancient times. The peoples of ancient India had their own rich, complex, and many-faceted spiritual lives. We can only begin to understand them, and to understand how the Buddha related to them, when we set aside our modern preconceptions and preoccupations and listen to what they had to say for themselves.

An outstanding feature of early Buddhist texts is interreligious dialogue. The Buddha did not live in a Buddhist culture. We frequently encounter the Buddha and his disciples discussing various aspects of spiritual philosophy and practice with followers of other spiritual paths, or with people of no particular path. Sometimes they would come to the Buddha seeking to learn or even to attack. And it is not uncommon to find the Buddha and his disciples actively seeking out followers of other spiritual paths simply to engage in conversation. In this, the early texts are quite different from later Buddhist literature, which almost always consists of Buddhists speaking with other Buddhists.

While many of these people ended up declaring themselves followers of the Buddha, this was not the purpose of the dialogue. The Buddha did not debate simply to win an argument, but out of compassion, to help alleviate suffering.

Amid the complex sets of religious practices, we may discern three major domains.

\subsection*{Animism}

In the villages and towns of ancient India, the most widespread folk religion was a belief in the omnipresent reality of spirits in nature. Such deities might embody aspects of the weather, or live in plants or rivers or caves; they might promote abundance, or take ferocious and threatening forms. They were unreliable, but could be wooed through simple offerings of rice, flowers, or ghee.

Animist beliefs were derived from local legends and rituals, not from religious philosophy. However, the higher religious paths such as Buddhism or Jainism, rather than repressing such beliefs, were happy to assign them a minor role in the scheme of things, so long as they eliminated harmful practices like human or animal sacrifice.

Throughout the Buddhist texts, we hear of \textit{yakkhas} (spirits), \textit{\textsanskrit{nāgas}} (dragons), \textit{gandhabbas} (fairies), \textit{garudas} (phoenixes), and many more. The Buddhist attitude towards such beings might best be described as “good neighborliness”. Neither they, nor any higher beings, are worshiped or looked to for salvation. Rather, they are treated with respect and dignity, for who knows? If they are real, it would be good to have them on your side.

\subsection*{Brahmanism}

The caste who called themselves “brahmins” inherited an ancient body of sacred lore known as the Vedas. This consisted of sets of oral scriptures, among which the Ṛg Veda was primary. In the early Buddhist texts there are three Vedas: Ṛg, \textsanskrit{Sāma}, and Yajur; the Atharva is mentioned, but was not yet considered to be a Veda.

The Ṛg Veda grew out of the cultural and religious milieu of the ancient Indo-European peoples. It shares a common heritage with the Avestan texts of Iranian Zoroastrianism, and more distantly, the mythologies of Europe.

It seems that Indo-European peoples moved into India around a millennium before the Buddha, with distinct clans maintaining sets of sacred lore. In the early centuries of the first millennium BCE, in the area known as the Kuru country (modern Delhi), the clans were unified into the classical brahmanical kingdom whose story is echoed in the \textsanskrit{Mahābharata}. The Ṛg Veda was forged from the books of the clans, wrapped in opening and closing chapters emphasizing unity. By the time of the Buddha, the brahmanical culture and language had already become strongly established in the regions further south and east where the Buddha lived.

The brahmins insisted on the holiness of their caste, the efficacy of their rituals, the truth of their scriptures, and the omnipotence of their deity. The Buddha rejected all these claims out of hand.

However, Brahmanical traditions were far from a unified monolith. We see a strong strand of questioning of tradition, of seeking out new ways, of earnest seeking of the truth; and such attitudes are just as strong in the Brahmanical texts as the Buddhist.

Brahmins were typically family men, living a settled life, and expecting a degree of social respect and standing due to their learning and caste. But some brahmins had adopted an ascetic lifestyle, apparently influenced by the \textit{\textsanskrit{samaṇas}}.

In the generations preceding the Buddha, brahmanical philosophy had reached a peak in the \textsanskrit{Upaniṣads}, with their sophisticated debates and mystic philosophy of the essential unity of self and cosmos. These texts form the immediate dialectical context of the Buddha’s philosophy. \textsanskrit{Yājñavalkya}, a key \textsanskrit{Upaniṣadic} philosopher, lived around \textsanskrit{Mithilā}, in the same region traversed by the Buddha, no more than a century or two before him. Some early \textsanskrit{Upaniṣads} are apparently referred to in \href{https://suttacentral.net/dn13}{DN 13} \textit{The Three Knowledges} (\textit{Tevijjasutta}), and the \textsanskrit{Upaniṣadic} doctrine of “self” (\textit{\textsanskrit{ātman}}) was prominently rejected by the Buddha in his most distinctive teaching: not-self (\textit{\textsanskrit{anattā}}).

\subsection*{The \textsanskrit{Samaṇas}}

Quite distinct from the brahmins, and often in opposition to them, was a complex set of religious movements known as the \textit{\textsanskrit{samaṇas}} or “ascetics”. Six prominent ascetic schools were acknowledged in the time of the Buddha. The Buddha counted himself as an ascetic, too, in view of the many similarities between his own movement and theirs.

Like the Buddhist mendicants, the other \textit{\textsanskrit{samaṇas}} were typically celibate renunciates, living either in solitude or in monastic communities, and relying on alms for food. The most famous movement—and the only one to survive until today—was Jainism, which flourished under their leader \textsanskrit{Mahāvīra}, known as \textsanskrit{Nigaṇṭha} \textsanskrit{Nātaputta} in the Buddhist texts.

The ascetics shared an iconoclastic attitude, and all rejected the brahmanical system \textit{in toto}. However they varied amongst each other, as shown in their teachings attested at \href{https://suttacentral.net/dn2}{DN 2} \textit{The Fruits of the Ascetic Life} (\textit{\textsanskrit{Sāmaññaphalasutta}}) as well as \href{https://suttacentral.net/mn60}{MN 60} and \href{https://suttacentral.net/mn76}{MN 76}. Some emphasized austerities and self-mortification, others rationality and debate. Some advocated ardent effort, others a resigned fatalism. Some taught rebirth, while others asserted that this material world was the only reality.

While their doctrines may appear florid and obscure, and their practices sometimes bizarre and pointless, the ascetic movements are a lasting testament to the diversity, vigor, and innovation of religious life in ancient India.

\section*{Cosmology}

A recurring theme in many of the religious strands of India is a concern for cosmology. A religious philosophy was expected to paint a picture of the world on a large scale and indicate humanity’s role within it. Like all aspects of religious life, such cosmologies were partly shared across traditions and in part were unique to each tradition.

Some traditions asserted a materialist cosmology, rejecting the notion that one would be reborn in any other state, and asserting that only this life was real.

Most cosmologies, however, posited multiple realms of existence. Beings would come and go from these different stations. Some were pleasant and desirable, while others were not. As to why this was so, different reasons were given.

\begin{itemize}%
\item Some ascetics argued that beings transmigrated due to destiny or chance.%
\item Mainstream Brahmanical traditions said it was due to the performance of rituals and sacrifices to the gods.%
\item Some said that rebirth was determined by intentional actions, whether moral or immoral.%
\end{itemize}

The latter view was held by some ascetic schools, such as Buddhism and Jainism, and some of the more advanced and innovative threads of Brahmanism. These traditions shared a conception of transmigration that in many ways is quite similar. Three common elements can be discerned:

\begin{enumerate}%
\item All sentient beings are reborn countless times in process called \textit{\textsanskrit{saṁsāra}} (“transmigration”).%
\item This process is driven by ethical choices (\textit{kamma}). Good deeds lead to a pleasant rebirth; bad deeds lead to a painful rebirth.%
\item True salvation is not found in any realm of existence, but only in liberation from transmigration itself.%
\end{enumerate}

While these aspects of the cosmology were shared, the details differed in both theory and practice.

Jain and Brahmanical theory proposed that transmigration was undergone by a soul or self which could attain freedom. For the Jains, the individual soul (\textit{\textsanskrit{jīva}}) attains eternal purity and bliss. For the most sophisticated among the brahmins, the individual self (\textit{\textsanskrit{ātman}}) realizes its true nature as identical with the divinity that is the cosmos itself (Sanskrit: \textit{tat tvam asi}; Pali: \textit{eso hasmasmi}; \textit{so \textsanskrit{attā} so loko}).

The Buddha rejected all such metaphysical notions of self or soul. Instead, he explained transmigration as an ongoing process of changing conditions, formulated as the famous twelve links of dependent origination (\textit{\textsanskrit{paṭicca} \textsanskrit{samuppāda}}).

In the practical application of their theory, Jains believed that the way to salvation was to firstly avoid harming any sentient beings, even unintentionally, and then to burn off past \textit{kamma} through painful self-mortification. Such practices are described frequently and in detail, attesting to their prominence in early Indian spiritual life.

The brahmins, as seen in the suttas, did not have such a clear and unambiguous path to a highest goal, and indeed are depicted as arguing among themselves as to the correct path. This reflects the historical situation, where the earlier, simpler, and more worldly goals of Vedic Brahmanism were growing into a more sophisticated \textsanskrit{Upaniṣadic} form.

For the \textsanskrit{Upaniṣads}, the key to salvation was understanding. It is only one who understands the rituals and philosophies correctly (\textit{ya evam veda}) who will experience their full benefit. In the centuries prior to the Buddha, this path of wisdom had developed into a profound contemplative culture, expressed in the ecstatic and mystical passages of the \textsanskrit{Upaniṣads}.

The Buddha shared the Jain concern for avoiding harm, but rejected the practice of extreme austerities. Rather than bodily torment, he emphasized mental development.

Certain Brahmanical lineages had developed meditation to a high degree, but meditative states were still conceived in metaphysical terms. The Buddha adopted such meditations for their value in purifying the mind, but interpreted them in purely psychological terms, rejecting metaphysics entirely.

One of the benefits of advanced meditation was that a practitioner would develop the ability to perceive many past lives and many realms into which beings may be reborn. In this, we may distinguish between the core doctrinal texts, which typically speak of rebirth in general terms as good or bad destinies, and the narrative portions, which depicted the realms of rebirth in terms familiar from popular Indian cosmology.

The early texts do not attempt to systematize these realms in any great detail. Indeed, the various deities and realms mentioned defy any simple categorization. Later Buddhism developed a theory of various realms, sometimes called the 31 planes of existence, but this does not fully represent the situation as found in the early texts.

Here is a general overview of the most important realms found in the suttas. It is crucial to remember that, in the Buddhist view, all of these, even the most high, are impermanent and do not constitute true freedom. They are not separate metaphysical planes, but mere stations in which consciousness may spend some time during its long journey.

\begin{description}%
\item[\textsanskrit{Brahmā} realms] The highest heavens, which correspond to attainments of absorption meditation (\textit{\textsanskrit{jhāna}}), and may only be attained by \textit{\textsanskrit{jhāna}} practitioners. The \textsanskrit{Brahmā} realms include the realms of luminous form (\textit{\textsanskrit{rūpaloka}}) and the formless realms (\textit{\textsanskrit{arūpaloka}}). The former are attained by means of the four primary absorptions. In this context, the word “form” refers to the refined and radiant echo or reflection of the original meditation subject upon which these states are based. The formless states lie beyond this, and are realized when even that subtle luminosity disappears.%
\item[Heavens of sensual pleasures] Many of these are mentioned, most commonly the realm of the Thirty-Three, governed by Sakka. Various beings from Indian animist beliefs are said to inhabit the lower tiers of these realms.%
\item[Human realm] The most important realm, where Buddhas appear and the spiritual path is taught.%
\item[Lower realms] These include the animal realm, the ghost realm, and the hells. The realm of the \textit{asuras}—titans or demons—is usually placed here in the later cosmologies, but the early texts seem to treat it as one of the heavens.%
\end{description}

The Buddha taught that doing good and avoiding bad was the path to rebirth in one of the fortunate realms, which include the human realm and all higher realms. However, the course of transmigration is long and unpredictable, so no heaven realm provides a sure refuge.

Far from teaching rebirth as a solace for naive followers unable to face the inevitability of death, rebirth is depicted in traumatic and terrifying terms: the tears that one has shed in the endless course of transmigration are greater than all the waters in all the oceans of the world. Thus the true significance of doing good deeds is not merely to get a better rebirth, but to lay the foundations for higher spiritual development, primarily through meditation.

In the core teaching of the four noble truths, the origin of all suffering is traced to the craving that is connected with rebirth (\textit{\textsanskrit{yāyaṁ} \textsanskrit{taṇhā} \textsanskrit{ponobbhavikā}}). The practice of the noble eightfold path is the only thing that enables one to let go of that craving and be free of suffering. This is what the Buddha called “extinguishment” or “quenching” (Pali: \textit{\textsanskrit{nibbāna}}; Sanskrit \textit{\textsanskrit{nirvāṇa}}).

\section*{On the Pali Language}

\textsanskrit{Theravāda} Buddhist texts are written in Pali. This ancient Indic language can be thought of as a simplified cousin of Sanskrit. Here are some well-known words in both languages.

\begin{itemize}%
\item Sanskrit \textit{karma} is Pali \textit{kamma}.%
\item Sanskrit \textit{dharma} is Pali \textit{dhamma}.%
\item Sanskrit \textit{bodhisattva} is Pali \textit{bodhisatta}.%
\item Sanskrit \textit{\textsanskrit{nirvāṇa}} is Pali \textit{\textsanskrit{nibbāna}}.%
\item Sanskrit \textit{\textsanskrit{bhikṣuṇī}} is Pali \textit{\textsanskrit{bhikkhunī}}.%
\end{itemize}

Pali belongs to the Indo-European language family, which was introduced to India perhaps a millennium or so before the Buddha with the Indo-European (or “Āryan”) incursions. It was subsequently enriched by a large number of words adopted from local languages, primarily Dravidian and Munda. Pali or closely related languages were used across the greater Gangetic plain in the cultural and economic region identified by the use of Northern Black Polished Ware. 

Despite its great antiquity, Pali is a well studied and clearly defined language. This is because the ancient Indians developed an advanced science of linguistics, which is already mentioned in the earliest Pali texts. Within a few generations of the Buddha, these were synthesized and standardized by \textsanskrit{Pāṇini} in his Sanskrit grammar, the \textsanskrit{Aṣṭādhyāyī}, a comprehensive technical treatise which is perhaps the greatest intellectual achievement in the secular sciences of ancient times. \textsanskrit{Pāṇini}’s methods were later adapted to create Pali grammars, which remain the basis for the study of Pali to this day. The \textsanskrit{Theravāda} tradition has always prided itself on its commitment to precision and care in matters of language and scripture, with the result that the canonical Pali texts are, for the most part, well edited, clear, and consistent.

Pali is a highly inflected language, which means that the forms of words are adapted to indicate such things as gender, number, case, and the like. This makes the study of Pali grammar quite intensive, but it also helps in parsing out the structure of sentences.

The name of the language can be a little confusing. The \textsanskrit{Theravāda} commentaries say they are written in \textsanskrit{Māgadhī}, “the language of Magadha”. For them, \textit{\textsanskrit{pāḷi}} means “text”, i.e. the canonical scripture as opposed to commentary. In modern times, however, the language of the \textsanskrit{Theravāda} scriptures has come to be called Pali, while the dialect of the Ashokan pillars is called \textsanskrit{Māgadhī}. I follow these modern conventions.

The historical origins of Pali are a matter of ongoing academic research, which I will not go into here. Tradition holds that Pali was spoken by the Buddha, and this is surely true in the sense that what he spoke would have been closer to Pali than any other known language, with the possible exception of Ashokan \textsanskrit{Māgadhī}.

In my view, the least persuasive theory is that the Buddha spoke a single consistent language all the time. This is so implausible as to be virtually impossible; it’s just not how language works. In ancient times language was highly diverse, with different accents and dialects sometimes from one village to the next. The process of language standardization is slow and uncertain, and it is still common to adapt languages per context. In Thailand, for example, it is normal for monks from the north-east to speak to the villagers in their local Isaan language—which itself varies regionally—and to visitors from Bangkok in central Thai. In Malaysia, monks might switch between English, Hokkien, and Mandarin in a single talk. Even those who know only one language use context-switching all the time, adapting vocabulary, accent, and grammar to accommodate speaking with people of different classes, generations, or language communities.

This problem was addressed by the Buddha himself in the \textsanskrit{Araṇavibhaṅgasutta} (\href{https://suttacentral.net/mn139}{MN 139:12.8}), and he could not have made his policy more clear.

\begin{quotation}%
“How do you not insist on local terminology and not override normal usage?  It’s when in different localities the same thing is known as a ‘plate’, a ‘bowl’, a ‘cup’, a ‘dish’, a ‘basin’, a ‘tureen’, or a ‘porringer’.  And however it is known in those various localities, you speak accordingly, thinking: ‘It seems that the venerables are referring to this.’”

%
\end{quotation}

Unless the Buddha failed to follow his own advice, we can assume that he adopted local dialects in his journeys, just as anyone else would. In his travels from Nepal to the Deccan, from Bengal to Delhi, and in his conversations with folks from all walks of life, from outcastes to brahmins and kings, he would have used language that fitted his context. This is more apparent in literary forms than in linguistic variation. Verse forms in particular are strongly correlated with educated and elite classes: deities, brahmins, and kings. 

The language we call Pali has been standardized and streamlined to some degree, while not eliminating all traces of its dialectical origins. I think the specific features of Pali are likely to be influenced by the language of Avanti, the home town of Mahinda son of Ashoka, who first brought the texts to Sri Lanka. This historical problem is really of most interest to linguists, since the differences between these dialects are minor. Understanding of the dialectical situation can help to clear up some tricky problems, but does not affect the vast majority of Pali passages which are perfectly clear. 

In modern times the study of Pali remains popular in \textsanskrit{Theravāda} Buddhism, and any serious student will pick up some technical vocabulary. However, as my father used to warn me, “A little knowledge is a dangerous thing.” There are people who learn a few terms and ideas and think this grants them the expertise to discard 2,500 years of sophisticated linguistic learning. Some even start their own schools and teachings based on nothing more than silly linguistic mistakes. I’ve been studying, memorizing, reciting, translating, and teaching Pali for half my life, and I still learn something new most every day. If you want to set out on the journey of exploring Pali, make sure to stash plenty of patience and humility along with the curiosity in your backpack.

Pali is written phonetically, which means that the way it is spelled matches how it sounds. This has been the case since the beginning, as even the first Indic script, \textsanskrit{Brāhmī}, is phonetic. \textsanskrit{Brāhmī}, which is first attested in the Ashokan edicts, became the source of all later Indic scripts, as well as those of south and south-east Asia such as Thai, Sinhala, Lanna, Khmer, and Myanmar. Over time, Pali came to be written in all these scripts.

These phonetic scripts were made possible by the linguistic science of the ancient pandits, which precisely defined the value of each sound. Despite this, the popular pronunciation of Pali varies from place to place. This is because local Buddhist traditions use different transliteration systems influenced by the national languages, and the sound of letters in Pali does not always match the sound of the same letter in the local language. Thus \textit{buddha} in Pali is pronounced \textit{budda} in Sinhala, \textit{phuttha} in Thai, and \textit{booda} in American English. Nonetheless, scholars of every nation are well aware of the correct pronunciation. 

In the nineteenth century, various schemes for representing Pali and Sanskrit in Roman characters were developed. Scholars such as Sir Muthu Coomaraswamy urged the adoption of a standardized Romanization so that international scholars could easily read texts from the different regions. Such a unified standard for transliteration of Sanskrit and Pali was formalized in 1894 in Geneva at the International Congress of Orientalists. This became widely used by international scholars, and with minor emendations was adopted as ISO 15919 in 2001. Due to the strictly mechanical nature of such phonetic representations, SuttaCentral is able to show Pali texts in dozens of scripts, powered by the Aksaramukha conversion utility.

The pronunciation of Pali is quite simple. The ancient grammars clearly define the places and manner of articulation in the mouth. Vowels are pure monophthongs, not dipthongs; that is, the sound remains fixed throughout and does not glide while being articulated. Most vowels in many varieties of English are pronounced with some degree of dipthongalization, so native English speakers should train to remove this by pronouncing each vowel with the mouth held still. With this qualification, here are approximate vowel equivalents in English.

\begin{description}%
\item[a] short: \textit{lal} = “lull”%
\item[\textsanskrit{ā}] long: \textit{\textsanskrit{lā}} = “lah”%
\item[i] short: \textit{thil} = “till”%
\item[\textsanskrit{ī}] long: \textit{\textsanskrit{thī}} = “tea”%
\item[u] short: \textit{phul} = “pull”%
\item[\textsanskrit{ū}] long: \textit{\textsanskrit{cū}} = “chew”%
\item[o] long: \textit{so} = “so” (short before double consonants)%
\item[e] long: \textit{jel} = “jell” (short before double consonants)%
\end{description}

Consonants are all pronounced distinctly and not combined. The twenty-five main consonants follow a systematic pattern: there are five points of articulation in the mouth, and at each point there are five possible letters varied by aspiration and voicing, as well as a liquid consonant. In addition there are a few miscellaneous consonants. 

\begin{description}%
\item[guttural: back of the mouth] \textit{k}, \textit{kh}, \textit{g}, \textit{gh}, \textit{\textsanskrit{ṅ}}%
\item[palatal: middle of the tongue on the roof of the mouth] \textit{c}, \textit{ch}, \textit{j}, \textit{jh}, \textit{\textsanskrit{ñ}}%
\item[retroflex: tongue-tip curled to touch the roof] \textit{\textsanskrit{ṭ}}, \textit{\textsanskrit{ṭh}}, \textit{\textsanskrit{ḍ}}, \textit{\textsanskrit{ḍh}}, \textit{\textsanskrit{ṇ}}%
\item[dental: tongue touches the base of the teeth] \textit{t}, \textit{th}, \textit{d}, \textit{dh}, \textit{n}%
\item[labial: at the lips] \textit{p}, \textit{ph}, \textit{b}, \textit{bh}, \textit{m}%
\item[miscellaneous] \textit{y}, \textit{r}, \textit{l}, \textit{\textsanskrit{ḷ}}, \textit{v}, \textit{s}, \textit{h}, \textit{\textsanskrit{ḷh}}, \textit{\textsanskrit{ṁ}}%
\end{description}

The only ambiguous sign is \textit{h}. By itself or when following most soft consonants it is equivalent to the English \textit{h}. Thus \textit{\textsanskrit{haṁ}} is English “hung”, while \textit{\textsanskrit{amhā}} would be “umm, hah”. When following a hard consonant or the letter \textit{\textsanskrit{ḷ}}, \textit{h} is a sign of aspiration, not a separate letter. It is realized as a puff of air following the letter.

Most unvoiced consonants in English are aspirated, whereas Pali has both aspirated and unaspirated variants. We can emulate this by taking advantage of the fact that in English we drop the aspiration when following \textit{s}. Thus \textit{t} is pronounced like \textit{t} in “still”, while \textit{th} is like \textit{t} in “till”. Likewise for \textit{p} in “spill” versus \textit{ph} in “pill”, and \textit{k} in “skill” versus \textit{kh} in “kill”. \textit{C} is like \textit{ch} in “beach”, while \textit{ch} is like \textit{chh} in “beachhouse”. Note that Pali does not have the fricative sounds spelled in English with \textit{f} or \textit{ph}, as well as \textit{th} as in “thing” or “there”. \textit{Th} is an aspirated \textit{t}, and \textit{ph} is an aspirated \textit{p}. 

Voiced aspirated consonants (\textit{gh}, \textit{jh}, \textit{\textsanskrit{ḍh}}, \textit{dh}, \textit{bh}) are not found in English, but they can be approximated by combining words such as “bigheart” or “madhouse”.

Doubled consonants are always pronounced, such “big gun”, or “sad dad”. Before a double consonant, a long vowel is shortened.

Retroflex consonants are not found in English, but the effect is similar to the American \textit{r}. Thus Pali \textit{\textsanskrit{pāṭi}} is similar to the American pronunciation of “party”.

Finally, \textit{v} is soft like \textit{w}, \textit{r} is like \textit{r} in American English, and \textit{s} is pure sibilant (\textit{\textsanskrit{sū}} is “sue” not “zoo”). \textit{Ṁ}, known as \textit{\textsanskrit{anusvāra}}, nasalizes the preceding vowel, but in practice it is usually pronounced as the \textit{ng} in “sing”.

\section*{On the Pali Commentaries}

The Pali canonical texts are accompanied by an extensive and detailed set of commentaries (\textit{\textsanskrit{aṭṭhakathā}}) and subcommentaries (\textit{\textsanskrit{ṭīkā}}). These texts are, for most people, even more mysterious than the canon itself, so let me say a few words on them.

The main commentaries were compiled by the monk Buddhaghosa at the \textsanskrit{Mahāvihāra} monastery in Anuradhapura, then the capital of Sri Lanka, in the 5th century. Buddhaghosa inherited a series of older commentaries in the old Sinhalese language, now lost. These had been compiled over the centuries in Sri Lanka, mostly between around 200 BCE–200 CE; that is to say, the main content of the commentaries was closed several centuries before Buddhaghosa.

It was all a bit messy, with text in Pali and commentaries in Sinhala, and a variety of different commentarial texts. Buddhaghosa aimed to streamline the situation by combining all the old commentaries into a single system, translated into Pali.

Buddhaghosa’s work remains as an extraordinary accomplishment of traditional scholarship. He had an almost preternatural mastery of his materials, and the clarity and rigor of his writings make light work of what must have been an exceedingly difficult task. It is crucial to remember that he saw his work as that of an editor, compiler, and translator. That is what he claimed to be doing, and from everything we know about his work, he was a scholar of integrity who did exactly what he said. When he felt a need to express his own opinion he said so; but such interventions were rare and hesitant. The commentaries are the record of discussions and explanations of the Pali texts handed down in the \textsanskrit{Mahāvihāra} tradition, not the opinions of Buddhaghosa.

While Buddhaghosa compiled commentaries on the major texts, he left some incomplete. It is not always certain which commentaries were by him; but in any case later scholars completed his work. Subsequently, subcommentaries were written to clarify obscure points in the commentaries.

In modern \textsanskrit{Theravāda}, the commentaries have become a sadly and unnecessarily divisive issue. Some people take the entire tradition uncritically and regard the commentaries as essentially infallible. Others flip to an extreme of suspecting anything in the commentaries, rewriting \textsanskrit{Theravādin} history as a conspiracy of the commentaries. But any serious scholar knows that the commentaries are often helpful, even indispensable, on countless difficult and obscure points. Without them, there is no way we would have been able to create the accurate dictionaries and translations that we have today. Yet they cannot be relied on blindly, for, like any resource, they are fallible, and must be read with a careful and critical eye. On some doctrinal issues, the position of the commentaries had shifted considerably from the stance in the suttas, and not in illuminating ways.

I once read some advice from a Burmese Sayadaw—I am afraid this was many years ago and I have forgotten who it was—on how to use the commentaries. He said—and I paraphrase—something like this. First read the sutta. Try to understand it. Read it and meditate on it again and again. If there’s anything you don’t understand, see if it can be explained elsewhere in the suttas. If, at the end of the day, you still cannot understand it, check the commentary. If it answers the question, good. But if, after equally careful study, the commentary is still unclear, then check the subcommentary.

This has always seemed like sound advice to me, and I have tried to follow it. The purpose of the commentary is to help explain the suttas. Where the suttas are clear—and mostly they are—there is no need to refer to the commentary. The only extra thing I would add is that, in addition to the commentaries and subcommentaries, we now also have Chinese and Sanskrit parallels to help us understand difficult passages.

In these guides, I almost completely leave aside the commentarial explanations. In several places the explanations I have given differ from those in the commentaries. I am aware of this, and have written on most of these things elsewhere, but I do not want to burden the guides by re-litigating every controversy. I don’t contradict the commentaries out of ignorance or stubbornness, but because after many years of study, contemplation, discussion, and practice, I have come to see some things differently.

\section*{A Brief and Incomplete Textual History}

The significance of the \textit{\textsanskrit{nikāyas}} was recognized by European scholars early on. I will discuss specifics of the editions and translations in the essays on the individual \textit{\textsanskrit{nikāyas}}, and here offer some general remarks.

During the 19th century, European scholars became aware of the Pali tradition, seeing in it a reliable source of information for the Buddha, his times and his teachings. An English civil servant in Sri Lanka, Thomas Rhys Davids, learned Pali from the monks, initially to help him better understand Sri Lankan legal practices. Recognizing the significance of these texts, he returned to England and established the Pali Text Society (PTS), largely funded by Asian donors. They obtained palm-leaf manuscripts, on the basis of which the PTS prepared print editions of the main Pali texts.

The PTS editions introduced a number of ideas from European scholarship. Most obviously, they used a set of conventions for presenting Indic scripts with European letters. This system is lossless, so texts may be automatically changed from one script to another. It enables easy comparison between the editions of the Pali canon from different countries, which traditionally had been written in diverse local scripts. They also introduced titles at the start of texts, punctuation and capitalization, page numbers, footnotes, variant readings, and various other modern innovations.

One innovation that was not pursued consistently was the use of chapter and section numbers. These were added to the PTS Pali editions of the \textsanskrit{Dīgha} \textsanskrit{Nikāya} and the Vinaya, and are used in subsequent translations. However most of the PTS editions lack such sections, with the unfortunate consequence that academic referencing of Pali texts is still based on the volume and page of the PTS edition, a system that is neither practical nor precise.

The PTS editions were ground-breaking and have exerted an unparalleled influence on modern Buddhism, both east and west. Asian scholars have been well aware of them, with the consequence that it is probably hard to find any printed edition from the 20th century that is completely free of their influence. Nevertheless, the PTS texts are not particularly reliable. They were put together over a considerable period of time, with scant resources and few workers. The editors used whatever manuscripts they had to hand, and, apart from a general preference for Sri Lankan readings, it is hard to discern a consistent or clear methodology in their choices of readings. The limitations of these editions are well known among experts in the field, and in some cases updated and improved editions have been published.

For my translation of the \textit{\textsanskrit{nikāyas}}, I preferred to use the \textsanskrit{Mahāsaṅgīti} edition. This is essentially a digital representation of the Burmese textual tradition of the 6th Council, itself based on the 19th century 5th Council text. It is based on the digital edition prepared by the Vipassana Research Institute, with extensive proofreading and corrections by the Dhamma Society of Bangkok. The \textsanskrit{Mahāsaṅgīti} is a consistent and carefully edited digital text, and for that reason was chosen as the root Pali text for SuttaCentral. But it should not be assumed that it is the most authentic. On the contrary, it preserves the Burmese readings, which tend to correct the text in conformity with the Pali grammars. Nevertheless, in almost all such cases there is no difference in the meaning, just minor differences in spelling.

Like most translators, when editions vary I did not adhere to one edition, but simply selected what seems to be the best reading in each case. I referred to the PTS editions fairly often. More rarely, I consulted the romanized Buddha Jayanthi edition found on GRETIL; note, however, that the digital edition is widely regarded as being inferior to the original in Sinhalese script. Occasionally I also consulted the Rama 5 edition in Thai script. I also consulted previous translations, especially those of Bhikkhu Bodhi.

In problematic cases I cross-checked the Pali against the Sanskrit and Chinese parallels; I did not make use of Tibetan sources. However in every case the overriding intention was to accurately represent the Pali text. Only in a very few exceptional cases did I rely on the Sanskrit or Chinese parallels to correct the Pali.

%
\chapter*{The Long Discourses: Dhamma as literature and compilation}
\addcontentsline{toc}{chapter}{The Long Discourses: Dhamma as literature and compilation}
\markboth{The Long Discourses: Dhamma as literature and compilation}{The Long Discourses: Dhamma as literature and compilation}

\scbyline{Bhikkhu Sujato, 2019}

The \textsanskrit{Dīgha} \textsanskrit{Nikāya} is the first of the four main divisions in the Sutta \textsanskrit{Piṭaka} of the Pali Canon (\textit{\textsanskrit{tipiṭaka}}). It is translated here as \textit{Long Discourses}. As the title suggests, its discourses are somewhat longer than those of other \textit{\textsanskrit{nikāyas}}. There are, however, only 34 discourses in the collection, so despite the length of the individual discourses, the collection as a whole is the shortest of the \textit{\textsanskrit{nikāyas}}.

It is distinguished from the other \textit{\textsanskrit{nikāyas}} by its more developed and elaborate literary forms. Outgrowing the bare and direct style of most of the early texts, here the extra length offers space for narratives and doctrinal expositions to find a fuller expression. This is an early hint at how the literary form of Buddhist texts was to develop in later years, moving towards expansiveness and abundance.

It is no coincidence that these elaborate texts are often addressed to the brahmins, who were the self-proclaimed spiritual leaders of the time. The brahmins were the custodians of the most sophisticated texts in ancient India up to this time, the Vedic literature. It seems that one aim of the \textsanskrit{Dīgha} was to impress such learned men. These discourses offer a wide range of examples of how the Buddha related to those of other religious paths.

Another overriding theme of the \textsanskrit{Dīgha} is the passing away of the Buddha. The centerpiece of the collection is \href{https://suttacentral.net/dn16}{DN 16}, \textit{The Great Discourse on the Buddha’s Extinguishment} (\textit{\textsanskrit{Mahāparinibbānasutta}}), a discourse of unrivaled importance. This presents the last journey of the Buddha, wandering in unhurried stages from town to town, each step bringing him closer to his passing. In the very length of the text, recording so many details of the journey, we can sense a longing to draw out those last precious days as far as possible.

\section*{How the \textsanskrit{Dīgha} is Organized}

The 34 discourses are grouped in three \textit{vaggas}. The first \textit{vagga} consists of thirteen discourses, each of which includes a lengthy passage on the spiritual practice of a monastic, known as the Gradual Training (\textit{\textsanskrit{anupubbasikkhā}}).

In the second \textit{vagga} we find several discourses of a more biographical nature. \href{https://suttacentral.net/dn14}{DN 14} \textit{The Great Discourse on Traces Left Behind} (\textit{\textsanskrit{Mahāpadānasutta}}) tells of past Buddhas, while \href{https://suttacentral.net/dn16}{DN 16} \textit{\textsanskrit{Mahāparinibbānasutta}} tells of Gotama’s last days. In addition, some other discourses in this section are closely related to the \textit{\textsanskrit{Mahāparinibbānasutta}}. I will discuss this cycle further below.

The final \textit{vagga} is more miscellaneous. It includes long poetic sections, doctrinal compilations—some of which are precursors to the Abhidhamma—and narratives that are often humorous and occasionally border on farce.

As usual in the \textit{\textsanskrit{nikāyas}}, there is no overall sequence of the teaching and many details of organization appear quite arbitrary. Still, we can discern a purpose in the arrangement of a few of the major discourses. These details are unique to the Theravadin tradition, so should be seen as reflecting their concerns, rather than the fundamental principles of the \textsanskrit{Dīgha}.

The first discourse, \href{https://suttacentral.net/dn1}{DN 1} \textit{The Prime Net} (\textit{\textsanskrit{Brahmajālasutta}}), sets out a scheme of wrong views, and thus acts as a filter for the Dhamma, screening out possible misinterpretations. It seems that this arrangement was connected with the events of the so-called “Third Council” under King Ashoka, at a time when the \textsanskrit{Saṅgha} was overrun with imposters who were not genuine Buddhists. The second discourse, \href{https://suttacentral.net/dn2}{DN 2} \textit{The Fruits of the Ascetic Life} (\textit{\textsanskrit{Sāmaññaphalasutta}}), addresses a fundamental question: why do people follow a life of renunciation? In answering this, it sets forth the Gradual Training, a distinctively Buddhist path to peace.

The middle of the collection is dominated by discourses that deal in one way or another with the cosmic significance of the Buddha (\href{https://suttacentral.net/dn14}{DN 14}, \href{https://suttacentral.net/dn16}{DN 16}, \href{https://suttacentral.net/dn17}{DN 17}, \href{https://suttacentral.net/dn18}{DN 18}, \href{https://suttacentral.net/dn19}{DN 19}, \href{https://suttacentral.net/dn20}{DN 20}, \href{https://suttacentral.net/dn21}{DN 21}; to these may be added \href{https://suttacentral.net/dn26}{DN 26}, \href{https://suttacentral.net/dn27}{DN 27}, \href{https://suttacentral.net/dn30}{DN 30}, and \href{https://suttacentral.net/dn32}{DN 32}). Where the biographical texts of the Majjhima emphasize the practical and the personal, the specifics of how \emph{our} Buddha lived, these discourses exist in an arena of mythic grandeur. Time and space are expanded as the poignant and personal details of the \textit{\textsanskrit{Mahāparinibbānasutta}} are set among a series of mythological texts that show the potency of the Buddha and his teachings in the deep past, in the apocalyptic future, and in the present among the orders of gods.

The central event in all this is the death of the Buddha. Historically this was a traumatic crisis for the Buddhist community, and many feared that the Dhamma would not survive. By lifting attention from the present trauma and pointing to a longer meaning, these suttas show that the Dhamma need not die with the Buddha. The events of the \textit{\textsanskrit{Mahāparinibbānasutta}} spurred the \textsanskrit{Saṅgha} to hold the First Council, where the discourses were collected and organized to ensure their preservation. And these are, of course, the very scriptures that we are reading. In this way, these narratives tell the story of their own origin.

The \textsanskrit{Dīgha} finishes with mostly doctrinal compilations (\href{https://suttacentral.net/dn28}{DN 28}, \href{https://suttacentral.net/dn29}{DN 29}, \href{https://suttacentral.net/dn33}{DN 33}, \href{https://suttacentral.net/dn34}{DN 34}). If the beginning of the \textsanskrit{Dīgha} tells us \emph{why} the teachings matter and the middle tells us \emph{how} they came to be, the ending tells us \emph{what} they are. It is a rather curious thing that in the \textsanskrit{Dīgha}, many of the doctrines that we think of as fundamental to the Buddha’s teachings occur only rarely. These discourses rectify this situation, ensuring that the students of the \textsanskrit{Dīgha} had access to a wide range of teachings. The last two discourses, in particular, are compiled as handy mnemonics for memorizing sets of doctrinal teachings.

\section*{The Gradual Training}

The Gradual Training sets out the steps taken by a Buddhist renunciate on their path. It begins with the arising of a Buddha in the world. Hearing the Buddha’s teaching, a person reflects on how best it can be applied to their own life. Realizing that “the household life is cramped and dirty, but the life of one gone forth is wide open”, they give up their worldly possessions and attachments, don the ochre robe of a Buddhist mendicant, and undertake a life of morality, simplicity, and meditation. Proceeding step by step to ever more advanced practices, they eventually enter into deep meditative absorption (\textit{\textsanskrit{jhāna}}) before realizing the four noble truths and finding true freedom.

The Gradual Training is an expansion of the threefold training (\textit{tisso \textsanskrit{sikkhā}}): ethics (\textit{\textsanskrit{sīla}}), meditative immersion (\textit{\textsanskrit{samādhi}}), and wisdom (\textit{\textsanskrit{paññā}}). At \href{https://suttacentral.net/an3.89}{AN 3.89} the three trainings are defined:

\begin{itemize}%
\item ethics (in a monastic context) requires keeping the monastic rules;%
\item meditative immersion is the four absorptions;%
\item wisdom is the understanding of the four noble truths.%
\end{itemize}

This teaching is distributed widely throughout the early Buddhist texts. In the \textsanskrit{Dīgha}, for example, it’s found in the \textit{\textsanskrit{Mahāparinibbānasutta}} as a standard teaching repeated by the Buddha at many of the stops on his journey. A series of shorter discourses on this subject may be found in the \textsanskrit{Samaṇa} Vagga of the \textsanskrit{Aṅguttara} (\href{https://suttacentral.net/an3-samanavagga}{AN 3.81–91}).

This brief overview of the path is explained more fully in the Gradual Training, which explains each of the three trainings in considerable detail. This longer exposition appears to have been the original teaching on the overall lifestyle, practices, and aims of the Buddha’s mendicant followers. It seems that the Buddha preferred to encourage his monastics by exhorting them to follow the highest ideals of conduct and meditation. Only reluctantly did he set up the legal system of the Vinaya texts, with its procedures and punishments.

The Gradual Training is found, in somewhat varying forms, in the Majjhima (\href{https://suttacentral.net/mn27}{MN 27}, \href{https://suttacentral.net/mn51}{MN 51}, \href{https://suttacentral.net/mn38}{MN 38}, \href{https://suttacentral.net/mn39}{MN 39}, \href{https://suttacentral.net/mn53}{MN 53}, \href{https://suttacentral.net/mn107}{MN 107}, \href{https://suttacentral.net/mn125}{MN 125}), the \textsanskrit{Aṅguttara} (\href{https://suttacentral.net/an4.198}{AN 4.198}, \href{https://suttacentral.net/an10.99}{AN 10.99}), and even the Abhidhamma (\href{https://suttacentral.net/vb12}{Vb 12}, \href{https://suttacentral.net/pp2.4\#114}{Pp 2.4:114}). Curiously enough, however, it is not found among the collected discourses on the path found in the last book of the \textsanskrit{Saṁyutta}. While virtually all of the practices of the Gradual Training are found in the \textsanskrit{Saṁyutta}, the overall framework is not.

The \textsanskrit{Dīgha} makes up for this lack by placing a \textit{vagga} of thirteen discourses right at the start featuring the Gradual Training. This is called the \textsanskrit{Sīlakkhandhavagga}, the “Chapter on the Aggregate of Ethics”. Despite the title, however, these texts treat the full training on ethics, meditative immersion, and wisdom.

While the content is similar in each place that the Gradual Training appears, the \textsanskrit{Dīgha} versions feature a pronounced emphasis on beauty and pleasure. The stages of the path are illustrated by similes that are as lovely as they are apt, while each step of the path is said to be accompanied by a deepening sense of pleasure and happiness. The Gradual Training is not a path of suffering, but one of grace and joy and freedom.

Due to the repetition, the texts invariably abbreviate all the expositions except for the first two discourses. It should be remembered, however, that this is merely a consequence of how the Pali tradition arranged these texts. In the Sanskrit and Chinese \textsanskrit{Dīrghāgamas}, the texts in this section are arranged differently, and different suttas are either expanded or abbreviated accordingly.

While the focus is firmly on monastic life, the general principles hold good for everyone, and indeed at \href{https://suttacentral.net/mn53}{MN 53} \textit{A Trainee} (\textit{Sekhasutta}), Ānanda teaches essentially the same path to a lay audience. In the \textsanskrit{Sīlakkhandhavagga}, many discourses are addressed to lay people, most of whom are brahmins.

The question of King \textsanskrit{Ajātasattu} in \href{https://suttacentral.net/dn2}{DN 2} provides the key to understanding why this is so. He points out that in worldly life, each profession can be seen to have its own benefit. But what is the benefit of the renunciate life? While other ascetics falter before this question, the Buddha presents the Gradual Training. He shows how the life gone forth is not one of pain and distress, nor one of delayed gratification, but one that shows real benefits in this life. It is about the power and transformative potential of inner development and meditation. In contrast, the household path offers only limited happiness, with much uncertainty and stress, while the paths of other ascetics are unclear, ineffective, or painful, and the brahmins can only offer rituals and prayers of dubious efficacy. Thus the Gradual Training explains why there is a need for the \textsanskrit{Saṅgha} at all.

Just as the Gradual Training is built from the kernel of the threefold training, the code of monastic ethics is built from the core principles of basic precepts. It is divided into three sections. The first section begins with the most fundamental precept for everyone in Buddhism: non-violence, to refrain from killing any creature, however small. It continues with items found in such common teachings as the five precepts and the ten paths of skillful action. But it adds items that especially pertain to monastic life, such as avoiding luxuries and ownership of property. The second section on ethics expands these specifically monastic and renunciate precepts in much greater detail, while the final section deals with right livelihood. A Buddhist monastic, who relies on alms food given in faith, should not make a living by other means, especially through superstitious and magical practices.

The Gradual Training builds on these ethical foundations as the mendicant undertakes a series of practices designed to quell the busyness and activity of the mind. They rein in their senses, avoiding overly stimulating things. They focus on remaining mindful and aware throughout all their activities. They aim at contentment, being satisfied with a few simple possessions.

Only when all these have been developed does the mendicant resort to seclusion for meditation. Going to the forest, they undertake mindfulness meditation and give up the five hindrances that prevent peace of mind. These hindrances are one of the core meditation teachings in the suttas, regarded as the key obstacle to absorption. They are:

\begin{description}%
\item[Sensual desire] Includes any kind of craving, greed, or desire for sensual experience. It includes powerful forms such as sexual desire as well as more subtle kinds of attachment.%
\item[Ill will] Anything from outright hatred to subtle forms of annoyance and aversion come under this hindrance.%
\item[Dullness and drowsiness] When the mind begins to settle down in meditation, it commonly becomes sleepy or dull.%
\item[Restlessness and remorse] Restlessness is always looking for some future experience, while remorse keeps digging up the past, especially moments of regret.%
\item[Doubt] It is normal and healthy to doubt when it comes to things that we do not know. But if we do not understand the elements of what is right and what is wrong, doubt will subtly undermine our meditation.%
\end{description}

Experiencing an ever-deepening peace and bliss, they ultimately enter a series of profoundly still states of meditative immersion known as the four absorptions (\textit{\textsanskrit{jhānas}}).

The absorptions are the fundamental meditation practice in early Buddhism and are essential to all stages of Awakening. They occur in many contexts, but it is here, in the Gradual Training, that they emerge most naturally from the life and practice undertaken by the mendicants. This context was so central to early Buddhists that when they compiled the early Abhidhamma text, the \textsanskrit{Vibhaṅga}, the chapter on Absorption begins with the Gradual Training. It is true, there are lay followers in the early texts who were said to have practiced absorption. But it is equally true that when the Buddha taught how to attain such profound peace, he emphasized the power of deep renunciation.

It has become common in certain modern forms of Buddhism to assert that absorptions are not an essential part of the path. Others say that, while important, the absorptions are relatively shallow states of concentration that may be easily attained on a short retreat. Suffice to say, neither of these views finds support in the early texts. The absorptions are essential, profound, and difficult to attain. Even with the full strength of renunciation, many mendicants in the Buddha’s day struggled to realize them. Nevertheless, it is a special quality of the Dhamma that each step along the path is accompanied by deepening peace and joy, and letting go gets easier the further one travels. This is what makes the realization of even such profound and subtle states possible.

Emerging from the absorptions, the mendicant harnesses the power of a deeply purified mind to realize a series of special forms of knowledge or insight. These culminate in the realization of the four noble truths:

\begin{enumerate}%
\item Suffering (\textit{dukkha}).%
\item The origin of suffering, i.e. craving (\textit{samudaya}).%
\item The cessation of suffering, i.e. \textsanskrit{Nibbāna} (\textit{nirodha}).%
\item The practice that leads to the end of suffering (\textit{magga}).%
\end{enumerate}

Suffering is the spur that drives us to undertake spiritual practice. Only when we have some experience of suffering will we look for an escape. And when encountering the Buddha’s teaching, a seeker recognizes that the Dhamma speaks to that which matters in their own life, offering a powerful and pragmatic solution. But wallowing in suffering gets you nowhere. When you understand that this suffering is real, but has causes and conditions that you can do something about, it sparks faith and the resolve to act. The path itself is one of unfolding happiness and receding pain; the truth of the ending of suffering is experienced at every step. This culminates in the experience of profound meditative stillness, called absorption (\textit{\textsanskrit{jhāna}}) or immersion (\textit{\textsanskrit{samādhi}}). In such states, having let go of sensual desire, the five external senses cease (\textit{vivicc’eva \textsanskrit{kāmehi}}) and the mind feels a sense of peace and happiness unlike anything it has known before. Empowered by the clarity and brilliance of absorption, the reality of suffering and its cause becomes apparent. This signifies that one has realized the first stage of awakening, that of the stream-enterer (\textit{\textsanskrit{sotāpanna}}).

Stream-entry occurs when all the factors of the path—from the arousing of faith to the practice of absorption and deep insight—have been developed to a sufficient degree. At this point one has a profound insight into the nature of reality, letting go of three of the ten fetters that bind a person to rebirth. In the Gradual Training, the understanding of the four noble truths is usually followed by the understanding of the end of the defilements (\textit{\textsanskrit{āsava}}), which signifies the attainment of full perfection (\textit{\textsanskrit{arahattā}}). The remaining fetters are given up at this point, which is the final stage of the path: full awakening and freedom.

\section*{How to Build a Long Discourse}

There are over a thousand discourses recorded in each of the \textsanskrit{Aṅguttara} and the \textsanskrit{Saṁyutta} \textsanskrit{Nikāyas}, but only 34 long texts are recorded in the \textsanskrit{Dīgha}. The relatively short texts of the \textsanskrit{Aṅguttara} and \textsanskrit{Saṁyutta} are reminiscent of the pre-Buddhist \textsanskrit{Upaniṣads}, especially the \textsanskrit{Bṛhadāraṇyaka} and Chandogya. These consist of a series of mostly independent passages, each episode covering no more than a few pages, and assembled into a much larger text. They are recollections of concise and focused teachings at certain times and places by certain people. It would seem, then, from the overwhelming majority of contemporary texts both Buddhist and Brahmanical, that the short discourse or dialogue was the standard format.

How, then, were these long texts constructed? Why? And for whom? Let us approach these questions by briefly considering a few different forms employed in the \textsanskrit{Dīgha}.

\subsection*{Inherently Complex Subjects}

Some discourses are long because the subject matter is inherently complex and demands a lengthy explanation. Of course, the Buddha was a master of presenting subjects in both pithy and detailed forms. Nevertheless, there are a few discourses whose subject matter requires extensive treatment.

The most prominent example of this is the Gradual Training. In some cases—for example \href{https://suttacentral.net/dn6}{DN 6} \textit{With \textsanskrit{Mahāli}} and \href{https://suttacentral.net/dn7}{DN 7} \textit{With \textsanskrit{Jāliya}}—the discourse consists of little more than this passage, with a simple narrative background and some short extra teachings. So it seems that the presence of the long Gradual Training section was itself enough to qualify a discourse as “long”. Since this passage aims to provide a detailed guide to the whole of the renunciate spiritual life, from hearing the teaching to full awakening, the length is inherent in the subject matter. True, it is taught more briefly elsewhere, but even in those cases it tends to be somewhat long, and there was a tendency to make it more inclusive.

In other cases the Gradual Training is taught in the middle of a discourse that is already quite extensive. Such is the case with \href{https://suttacentral.net/dn1}{DN 1} \textit{\textsanskrit{Brahmajāla}}, although here, uniquely, it is only the first section on Ethics that is taught. But the bulk of the text sets forth a network of 62 kinds of wrong view. Here, the nature of the subject matter is so extensive and complex that a shorter exposition would not do it justice. Indeed, when this teaching is mentioned in shorter discourses (\href{https://suttacentral.net/sn41.3}{SN 41.3}), it is not summarized, but the reader is referred rather to the full text.

\subsection*{Compilations}

Far more common than inherently lengthy teachings are the compilations. In such cases, a long text provides an occasion or background framework within which a series of short passages are collected. Such passages usually occur in identical or near-identical form in the \textsanskrit{Aṅguttara} or the \textsanskrit{Saṁyutta}, and occasionally the Majjhima. Compiling them here enables the reciters of the \textsanskrit{Dīgha} to learn a wide range of doctrines, and provides an essential backup, preserving the texts in case the shorter discourses are lost.

In a few instances, such short passages are not found elsewhere in the same form. Whether that is because they were unique to the \textsanskrit{Dīgha}, or because the parallel passages have become lost, is hard to say.

How do we end up with parallel passages in so many different places?

 the Buddha taught very often and, like all teachers, repeated his message many times. Such repeated teachings would have been collected in various places. This would be the case with important and generic teachings found throughout the Buddhist literature, like the four noble truths or the four absorptions.

In some cases, though, this is unlikely or impossible. For example, we sometimes find the same event on the same occasion—with the same teaching, the same location, and the same people—occurring in more than one text. In such instances, it is clear that there is, in fact, just one passage, and it has been copied into two or more places.

Generally speaking, it is prudent to assume that such passages existed as short discourses before being collected into larger forms. This is because, as noted above, the short discourse is the dominant form, and rests closest to the oral tradition. It is a principle observed everywhere through early Buddhist texts that the redactors preferred to add rather than subtract. Thus texts commonly become longer over time, and rarely shorter.

Examples of compilation are very common, and almost every discourse in the \textsanskrit{Dīgha} does this to some extent. Here are just a few examples.

\href{https://suttacentral.net/dn16}{DN 16} \textit{\textsanskrit{Mahāparinibbānasutta}} includes a wide range of collected passages. In some cases, events pertinent to the narrative may have occurred there originally and been extracted later, while in other cases the included passages seem strangely extraneous to the context and were no doubt added in at some point.

Venerable \textsanskrit{Sāriputta} is said to be the main author of several such long compilations. He is the teacher in \href{https://suttacentral.net/dn33}{DN 33} \textit{Reciting in Concert} (\textit{\textsanskrit{Saṅgītisutta}}) and \href{https://suttacentral.net/dn34}{DN 34} \textit{Up to Ten} (\textit{Dasuttarasutta}), which consist almost entirely of short passages collected from elsewhere in the suttas and arranged by number. In \href{https://suttacentral.net/dn28}{DN 28} \textit{Inspiring Confidence} (\textit{\textsanskrit{Sampasādanīyasutta}}) he expresses his great faith in the Buddha and cites a long series of passages to display the Buddha’s glory.

A more sophisticated form of compilation is found in \href{https://suttacentral.net/dn22}{DN 22} \textit{The Longer Discourse on Mindfulness Meditation} (\textit{\textsanskrit{Mahāsatipaṭṭhānasutta}}), the most important meditation discourse in 20th century \textsanskrit{Theravāda}. It gives a detailed account of the four kinds of mindfulness meditation. These are taught in brief in many places, but the details are found only here and at the mostly identical \href{https://suttacentral.net/mn10}{MN 10} \textit{Mindfulness Meditation} (\textit{\textsanskrit{Satipaṭṭhānasutta}}). Whereas many compilations simply list a series of different teachings, here the text is very systematic, organizing the compiled passages under the four heads. These meditation passages are mostly not found elsewhere in the \textsanskrit{Dīgha}, and were no doubt added to ensure the \textsanskrit{Dīgha} reciters preserved the full range of meditation teachings.

To the already lengthy discourse at \href{https://suttacentral.net/mn10}{MN 10} is added a full exposition on the four noble truths, sourced from \href{https://suttacentral.net/mn141}{MN 141} \textit{The Analysis of the Truths} (\textit{\textsanskrit{Saccavibhaṅgasutta}}). In Burmese editions, this extended section later made its way back into the text of \href{https://suttacentral.net/mn10}{MN 10}. Since SuttaCentral’s text is a Burmese one, we include this in our Pali, but mark it as an addition.

\subsection*{Narratives: backgrounds, parables, and myths}

Unusually for early Buddhist texts, the \textsanskrit{Dīgha} includes several lengthy narratives. Most obviously this includes \href{https://suttacentral.net/dn16}{DN 16} \textit{\textsanskrit{Mahāparinibbānasutta}}. But it also includes several other narratives.

In common with the discourses of other collections, we often find a simple narrative background that gives context to the teaching. However, in some cases, this is developed in much greater detail as the narratives come to play a more sophisticated literary role than a mere setting.

\href{https://suttacentral.net/dn2}{DN 2} \textit{\textsanskrit{Sāmaññaphala}} opens with King \textsanskrit{Ajātasattu} of Magadha exclaiming over the beauty of the moonlit night and asking his ministers for advice as to which ascetic teacher he should visit. From the \textit{\textsanskrit{Mahāparinibbānasutta}} and discourses elsewhere we know that \textsanskrit{Ajātasattu} was a warlike king, so this setting immediately establishes a sense of wonder. The narrative unfolds gracefully, avoiding the excess of ornament so typical of later Indian narratives, and holding the key to its mystery close to its chest. Only at the end of the text do we learn the dreadful secret that plagues the king’s heart. Thus the narrative portions imbue the teachings—on the doctrines of other teachers as contrasted with the Buddha’s Gradual Training—with a tragic pathos.

In addition to backgrounds, we also find narratives that are told as stories in the discourses themselves. These include short parables like the tale of the monk who mistakenly sought among the gods for an answer to his question (\href{https://suttacentral.net/dn11}{DN 11}). In \href{https://suttacentral.net/dn23}{DN 23} the monk \textsanskrit{Kumāra} Kassapa debates with the skeptic \textsanskrit{Pāyāsi}, illustrating his arguments with a series of tales alternatively humorous and gruesome. Such parables are found not infrequently elsewhere in the suttas, but in the \textsanskrit{Dīgha} certain stories expand beyond this and approach the stature of myth. This includes some of the texts in the \textit{\textsanskrit{Mahāparinibbānasutta}} cycle, such as \href{https://suttacentral.net/dn17}{DN 17} \textit{\textsanskrit{Mahāsudassanasutta}} and \href{https://suttacentral.net/dn14}{DN 14} \textit{\textsanskrit{Mahāpadānasutta}}.

To forestall a common misunderstanding, in the study of religion, “myth” does not mean “something believed to be true that is actually false”, as it does in popular culture. Rather, a myth is a sacred story. Some sacred stories are true, some are inventions. But this is a matter for historians and is irrelevant to the mythology itself. The purpose of myth is to tell a story that creates meaning for those who participate in it, so they can understand their own lives in the context of the story being expressed.

The \textsanskrit{Dīgha} contains truly mythic texts in \href{https://suttacentral.net/dn26}{DN 26} \textit{The Wheel-Turning Monarch} (\textit{\textsanskrit{Cakkavattisīhanādasutta}}) and \href{https://suttacentral.net/dn27}{DN 27} \textit{The Origin of the World} (\textit{\textsanskrit{Aggaññasutta}}). These set forth a myth of origins, replacing conventional creation mythology with an evolutionary account of how the world came to be the way it is. In these stories, human choices play a critical role in how the environment evolves, and in how it will all fall apart. The \textit{\textsanskrit{Aggañña}} depicts anthropogenic climate change quite explicitly, showing how human activity affects the plants, the weather, and the natural ecosystem of which we are a part (see also \href{https://suttacentral.net/an3.56}{AN 3.56}).

The mythology is essentially cyclic. There is no absolute beginning, just another turning of the wheel. Thus even when the world falls apart and civilization collapses, there will be a new renaissance, far in the future, and ultimately another Buddha will arise. He is named Metteyya (Sanskrit: \textit{Maitreya}), who in the early texts appears only in \href{https://suttacentral.net/dn26}{DN 26} \textit{\textsanskrit{Cakkavattisīhanāda}}. He went on to become one of the most important figures in \textsanskrit{Mahāyāna} Buddhism, and many Buddhists even today still await his coming with hope. Yet \href{https://suttacentral.net/dn26}{DN 26} is not taught in order to encourage devotees to dedicate themselves to Metteyya, but to illustrate the impermanence and uncertainty of our lives. The Buddha always taught that we should practice as best we can to understand the Dhamma in this life.

\section*{The \textsanskrit{Mahāparinibbāna} Cycle}

In several instances, episodes mentioned in brief in the \textit{\textsanskrit{Mahāparinibbānasutta}} have been spun off and expanded to become individual discourses in their own right. Thus the \textit{\textsanskrit{Mahāparinibbānasutta}} dominates much of the \textsanskrit{Dīgha}, not just through its length and thematic weight, but through its influence and connections with other discourses.

In my view, this cycle of suttas was likely composed by Ānanda and his students, beginning this great literary work with the \textit{\textsanskrit{Mahāparinibbānasutta}} itself, and gradually branching off into related works. The cycle as a whole shows not only Ānanda’s characteristic personal love and devotion for the Buddha, but also reveals a concern for what is to come, for the fate of the Dhamma in the years after the Buddha’s passing. One distinctive unifying detail of these discourses is that they do not end with the standard phrase saying that the listeners rejoiced in the teachings, but instead finish directly with a teaching or a verse on the subject of impermanence or the long-lasting of the dispensation. Ānanda survived the Buddha for several decades, and his legacy was the establishment of the texts, thus preserving the memory of his beloved Teacher for future generations.

\begin{description}%
\item[\href{https://suttacentral.net/dn16}{DN 16} \textit{The Great Discourse on the Buddha’s Extinguishment} (\textit{\textsanskrit{Mahāparinibbānasutta}})] Beginning with King \textsanskrit{Ajātasattu} of Magadha declaring his intent to invade the Vajjis, and ending with the peaceful distribution of the Buddha’s relics to the potentially warring nations and clans, the story of the Buddha’s last journeys is as politically revealing as it is spiritually moving. Throughout, the theme of impermanence unifies the diverse events and teachings. The weight of constructing such an epic shows, however, in the considerable differences between extant versions of the text. Many of the extra repetitious sections—such as the superfluous sets of eight that follow the eight causes of earthquakes—are not found in all parallels. It seems that over time, more and more material was added, and at certain points, portions of the text were split off to form other discourses in the cycle.%
\item[\href{https://suttacentral.net/dn17}{DN 17} \textit{King \textsanskrit{Mahāsudassana}}] In a small scene of the \textit{\textsanskrit{Mahāparinibbānasutta}}, Ānanda encourages the Buddha to pass away in a well-known city, not in the obscure village of \textsanskrit{Kusinārā}. The Buddha rebukes him, saying that in the past it had been a great city. The Sanskrit (\textsanskrit{Sarvāstivāda}) versions of the \textit{\textsanskrit{Mahāparinibbānasutta}} include a shorter account of the story of King \textsanskrit{Mahāsudassana} in their \textit{\textsanskrit{Mahāparinibbānasutta}} itself, but in the Pali, it has become greatly extended and formed into its own long discourse. The discourse itself is fabulous, full of extended passages on the crystal balustrades and other wonders of \textsanskrit{Mahāsudassana}’s palace. But at its heart is a very human story: the love of the queen for her king, and the pain of letting go. The struggle that the queen undergoes to fully understand that her king must pass mirrors the struggles of Ānanda in the \textit{\textsanskrit{Mahāparinibbānasutta}} as he comes to terms with the passing of his beloved Teacher.%
\item[\href{https://suttacentral.net/dn18}{DN 18} \textit{With Janavasabha}] Like \href{https://suttacentral.net/dn17}{DN 17}, this begins with a short passage extracted from the \textit{\textsanskrit{Mahāparinibbānasutta}}, to which has been added an extended narrative. During the journey in the \textit{\textsanskrit{Mahāparinibbānasutta}}, Ānanda asks the Buddha to reveal the fate after the death of devotees in the town of \textsanskrit{Ñātika}. This otherwise obscure town was the main city of the \textsanskrit{Ñātika} clan, to which the Jain leader \textsanskrit{Mahāvīra} belonged, and it appears to come to prominence to show that even his own people became Buddhists. Characteristically, it ends with the Buddha showing how people may know for themselves their own spiritual progress. This short passage is preserved as an independent discourse also in \href{https://suttacentral.net/sn55.10}{SN 55.10}. In \href{https://suttacentral.net/dn18}{DN 18}, however, the discourse continues with a long story of the doings of the gods, as told by the spirit Janavasabha. It culminates by saying that this discourse was learned by the Buddha from Janavasabha, and from there was taught to Ānanda, and he informed the assemblies of monks, nuns, laymen, and laywomen, resulting in the Buddha’s dispensation being famous and successful among gods and men. This corroborates the idea that these discourses, shaped by Ānanda, were aimed at ensuring the long-lasting of Buddhism.%
\item[\href{https://suttacentral.net/dn28}{DN 28} \textit{Inspiring Confidence} (\textit{\textsanskrit{Sampasādanīyasuttasutta}})] The \textit{\textsanskrit{Mahāparinibbānasutta}} records an incident where \textsanskrit{Sāriputta}, the Buddha’s foremost disciple, comes to him and makes a “lion’s roar” of his faith in the Buddha, based on his understanding of Dhamma. This is recorded as an independent discourse at \href{https://suttacentral.net/sn47.12}{SN 47.12}. We also have a short discourse at \href{https://suttacentral.net/sn47.13}{SN 47.13} that tells of \textsanskrit{Sāriputta}’s death. This echoes the themes of the \textit{\textsanskrit{Mahāparinibbānasutta}}, even including the famous saying that one should be one’s own refuge. This must have happened during the journey recorded in the \textit{\textsanskrit{Mahāparinibbānasutta}}. Oddly, however, it is not included in \href{https://suttacentral.net/dn16}{DN 16}, and in addition, it situates the Buddha in \textsanskrit{Sāvatthī}, far from the track of his journey. Regardless, in \href{https://suttacentral.net/dn28}{DN 28} the passage on the lion’s roar was expanded into its extensive discourse, with \textsanskrit{Sāriputta} expounding at length on various inspiring qualities of the Buddha. This gives an opportunity to list many standard doctrinal teachings. Like \href{https://suttacentral.net/dn18}{DN 18}, the sutta ends with an exhortation to share the teaching.%
\end{description}

In addition to texts that have a direct literary and narrative connection with the \textit{\textsanskrit{Mahāparinibbānasutta}}, there is a further series of discourses that share a more indirect or thematic connection.

\begin{description}%
\item[\href{https://suttacentral.net/dn14}{DN 14} \textit{The Great Discourse on Traces Left Behind} (\textit{\textsanskrit{Mahāpadānasuttasutta}})] The Buddha gives biographical details of six past Buddhas, as well as a lengthy account of the life of one of them, \textsanskrit{Vipassī}. This discourse establishes the historical Buddha Gotama as one of a series of world teachers that stretches back into the deep past, and whose dispensations all follow similar patterns.%
\item[\href{https://suttacentral.net/dn29}{DN 29} \textit{An Impressive Discourse} (\textit{\textsanskrit{Pāsādikasutta}})] This begins with the story of the passing away of the Jain leader \textsanskrit{Mahāvīra}. The Pali texts call him \textit{\textsanskrit{nigaṇṭha} \textsanskrit{nātaputta}}, which is often misunderstood as a proper name. \textit{\textsanskrit{Nigaṇṭha}}, rather, means “knotless” and is a term for a Jain ascetic, while \textit{\textsanskrit{nāta}} is a misspelling of his clan, the \textsanskrit{Ñātikas}. \textit{\textsanskrit{Nigaṇṭha} \textsanskrit{nātaputta}} therefore means “the Jain ascetic of the \textsanskrit{Ñātika} clan”, just as \textit{\textsanskrit{samaṇa} gotama} means “the ascetic of the Gotama clan”, or \textit{acela kassapa} means “the naked ascetic of the Kassapa clan”. In the Buddhist texts, his death is depicted as a disaster for the Jains, as they fell apart in conflict right away. Whether this is historically accurate or not, the text shows the Buddha taking the opportunity to teach the qualities that make a religious movement last long after the passing of the founder. Discourses in response to this are found at \href{https://suttacentral.net/dn16}{DN 16}, \href{https://suttacentral.net/dn29}{DN 29}, \href{https://suttacentral.net/dn33}{DN 33}, and \href{https://suttacentral.net/mn104}{MN 104}. In the current sutta, contrasting his dispensation with what he claims was the inadequacy of Jain teachings, the Buddha declares that the faith and practice of his followers are well-grounded since they are based on genuine Awakening.%
\item[\href{https://suttacentral.net/dn30}{DN 30} \textit{The Marks of a Great Man} (\textit{\textsanskrit{Lakkhaṇasutta}})] The early texts refer several times to a mysterious set of bodily characteristics known as the “marks of a great man”. These are said to fulfill a Brahmanical prophecy that one who possesses such marks will either become a universal emperor or a fully awakened Buddha. This prophecy and the list of thirty-two marks have not been exactly identified in extant Brahmanical texts, but recent research has uncovered a complex system of similar marks in old Brahmanical texts, many of which invite comparison with the Buddhist list. The story of the two paths is a classic mythological theme, found in the oldest known myth, the story of Gilgamesh. The marks of a great man exist as a curious counterpoint to the rational teachings found in most of the suttas. In this particular sutta, the Buddha is said to have explained each mark as a consequence of a specific kind of kammic deed. The literary and verse styles betray this as a late composition, and it has no real parallel in other collections. Nevertheless, it remains as a testament to the evolution of the idea of the Buddha, relating his spiritual qualities to his physical presence.%
\item[\href{https://suttacentral.net/dn33}{DN 33} \textit{Reciting in Concert} (\textit{\textsanskrit{Saṅgītisutta}})] Like \href{https://suttacentral.net/dn29}{DN 29}, this discourse is set after the death of \textsanskrit{Mahāvīra}. Speaking to the Mallians of \textsanskrit{Pāvā}—who appear also in the \textit{\textsanskrit{Mahāparinibbānasutta}}—the Buddha asks \textsanskrit{Sāriputta} to speak on his behalf. This echoes the theme of \href{https://suttacentral.net/dn28}{DN 28} and \href{https://suttacentral.net/dn29}{DN 29}, that it is the disciples who will be responsible for the continuation of the teachings. \textsanskrit{Sāriputta} gives an extensive systematic presentation of doctrines, using the \textsanskrit{Aṅguttara} principle of organizing teachings by number. Indeed, a study of this discourse can serve as an introduction to the teachings found in the \textsanskrit{Aṅguttara} \textsanskrit{Nikāya}. The monastics are encouraged to recite these teachings in concert so that they may be preserved and the dispensation continued for a long time. This discourse anticipates the systematic tendencies of the Abhidhamma, and indeed one of the \textsanskrit{Sarvāstivādin} Abhidhamma texts (\textit{\textsanskrit{Saṅgītiparyāya}}) consists of an expansion and commentary on this discourse.%
\item[\href{https://suttacentral.net/dn34}{DN 34} \textit{Up to Ten} (\textit{Dasuttarasutta})] This is similar to the \textit{\textsanskrit{Saṅgīti}}, but with a briefer narrative context and an even more systematic style. Here the Buddha no longer appears, and the discourse is simply spoken by \textsanskrit{Sāriputta}.%
\end{description}

This does not exhaust the scope of the \textit{\textsanskrit{Mahāparinibbānasutta}} cycle, for it is not confined to the \textsanskrit{Dīgha}. We have already mentioned that several shorter suttas contain episodes either found in the \textit{\textsanskrit{Mahāparinibbānasutta}} or related to it. And the story does not end with the Buddha’s death. The \textit{\textsanskrit{Mahāparinibbānasutta}} tells of the funeral arrangements and events following the Buddha’s passing. In several versions apart from the Pali, this story continues directly into the account of the First Council. This narrative is the 21st chapter of the Vinaya Khandhakas, and indeed the \textit{\textsanskrit{Mahāparinibbānasutta}} is found in the Vinaya of several schools. It is, in fact, one continuous narrative, and one of the many purposes of the \textit{\textsanskrit{Mahāparinibbānasutta}} is to authorize the actions of the \textsanskrit{Saṅgha} at the First Council, establishing the fundamental Buddhist scriptures in an organized and definitive manner. The First Council narrative was then extended to the Second Council, which echoes many of the same themes and ideas.

These stories of the end of the Buddha’s life and teaching are also echoed in the first chapter of the Vinaya Khandhakas, which tells the story of the Buddha’s awakening, first teaching, and establishing of his community of followers. These are not just separate episodes in the Buddha’s life. The texts as we have them frequently echo ideas, turns of phrase, events, and people, all of which show that they were edited and composed as a coherent whole. Taken together, they make up a framework of a magnificent mythology: the life and death of the greatest spiritual teacher that the world has ever known.

\section*{A Brief Textual History}

The \textsanskrit{Dīgha} \textsanskrit{Nikāya} was edited by T.W. Rhys Davids and J.E. Carpenter based on manuscripts in Sinhalese, Burmese, and Thai scripts, and published in three volumes in Latin script by the Pali Text Society from 1890 to 1910.

The first translation followed in 1899–1921 by T.W. and C.A.F. Rhys Davids, and was published under the “Sacred Books of the Buddhists” series under the title \textit{Dialogues of the Buddha}. This was a milestone in the publication of Buddhist texts, and marked the first occasion a full \textit{\textsanskrit{nikāya}} was available in English. The translation endeavored to retain something of the literary flavor of the texts and is accompanied by introductory essays and notes that are often useful and sometimes brilliant. But it is far from perfect and contains many errors in both reading and interpretation. Today the insights of Rhys Davids remain valuable especially in the area of history and society.

An updated translation by Maurice Walshe was published by Wisdom Publications in 1987 under the title \textit{Thus Have I Heard: The Long Discourses of the Buddha}, a title that in later editions was changed to \textit{The Long Discourses of the Buddha}. The Walshe edition benefited from many decades of study and practice of Dhamma in the west. Avoiding the archaic stylings of the older translations, it remains a clear and approachable translation, with a far more accurate handling of doctrinal terms and passages. But it is far from perfect. It leans heavily on the Rhys Davids translation, and while it corrects many errors, it sometimes repeats errors found in the older translation. Worse, it not infrequently introduces new errors.

In addition, there have been many translations of individual discourses and passages. Of these, the following were especially useful for my work:

\begin{itemize}%
\item For \href{https://suttacentral.net/dn1}{DN 1}, \href{https://suttacentral.net/dn2}{DN 2}, and \href{https://suttacentral.net/dn15}{DN 15}, the translations of text and commentary by Bhikkhu Bodhi.%
\item For \href{https://suttacentral.net/dn16}{DN 16}, the translation by Bhikkhu Ānandajoti.%
\item For the verses of \href{https://suttacentral.net/dn30}{DN 30}, the translations and studies by K.R. Norman.%
\item For \href{https://suttacentral.net/dn31}{DN 31}, the translation by John Kelly, Sue Sawyer, and Victoria Yareham.%
\end{itemize}

%
\chapter*{Acknowledgements}
\addcontentsline{toc}{chapter}{Acknowledgements}
\markboth{Acknowledgements}{Acknowledgements}

I remember with gratitude all those from whom I have learned the Dhamma, especially Ajahn Brahm and Bhikkhu Bodhi, the two monks who more than anyone else showed me the depth, meaning, and practical value of the Suttas.

Special thanks to Dustin and Keiko Cheah and family, who sponsored my stay in Qi Mei while I made this translation.

Thanks also for Blake Walshe, who provided essential software support for my translation work.

Throughout the process of translation, I have frequently sought feedback and suggestions from the SuttaCentral community on our forum, “Discuss and Discover”. I want to thank all those who have made suggestions and contributed to my understanding, as well as to the moderators who have made the forum possible. These translations were significantly improved due to the careful work of my proofreaders: \textsanskrit{Ayyā} \textsanskrit{Pāsādā}, John and Lynn Kelly, and Derek Sola. Special thanks are due to \textsanskrit{Sabbamittā}, a true friend of all, who has tirelessly and precisely checked my work.

Finally my everlasting thanks to all those people, far too many to mention, who have supported SuttaCentral, and those who have supported my life as a monastic. None of this would be possible without you.

%
\chapter*{Summary of Contents}
\addcontentsline{toc}{chapter}{Summary of Contents}
\markboth{Summary of Contents}{Summary of Contents}

\begin{description}%
\item[The Chapter on the Entire Spectrum of Ethics (\textit{\textsanskrit{Sīlakkhandhavagga}})] The Chapter Containing the Section on Ethics (\textsanskrit{Sīlakkhandhavagga}) is a chapter of 13 discourses. Each of these contains a long passage on the Gradual Training in ethics, meditation, and wisdom. The chapter is named after the first of these sections. The two other known versions of the \textsanskrit{Dīrghāgama} (in Chinese and Sanskrit) also contain a similar chapter. Despite the monastic nature of the central teaching, most of these discourses are presented in dialog with lay people, with a strong emphasis on the relation between the Buddha’s teachings and other contemporary movements.%
\item[DN 1: The Prime Net (\textit{\textsanskrit{Brahmajālasutta}})] While others may praise or criticize the Buddha, they tend to focus on trivial details. The Buddha presents an analysis of 62 kinds of wrong view, seeing through which one becomes detached from meaningless speculations.%
\item[DN 2: The Fruits of the Ascetic Life (\textit{\textsanskrit{Sāmaññaphalasutta}})] The newly crowned King \textsanskrit{Ajātasattu} is disturbed by the violent means by which he achieved the crown. He visits the Buddha to find peace of mind, and asks him about the benefits of spiritual practice. This is one of the greatest literary and spiritual texts of early Buddhism.%
\item[DN 3: With \textsanskrit{Ambaṭṭha} (\textit{\textsanskrit{Ambaṭṭhasutta}})] A young brahmin student attacks the Buddha’s family, but is put in his place.%
\item[DN 4: With \textsanskrit{Soṇadaṇḍa} (\textit{\textsanskrit{Soṇadaṇḍasutta}})] A reputed brahmin visits the Buddha, despite the reservations of other brahmins. They discuss the true meaning of a brahmin, and the Buddha skillfully draws him around to his own point of view.%
\item[DN 5: With \textsanskrit{Kūṭadanta} (\textit{\textsanskrit{Kūṭadantasutta}})] A brahmin wishes to undertake a great sacrifice, and asks for the Buddha’s advice. The Buddha tells a legend of the past, in which a king is persuaded to give up violent sacrifice, and instead to devote his resources to supporting the needy citizens of his realm. However, even such a beneficial and non-violent sacrifice pales in comparison to the spiritual sacrifice of giving up attachments.%
\item[DN 6: With \textsanskrit{Mahāli} (\textit{\textsanskrit{Mahālisutta}})] The Buddha explains to a diverse group of lay people how the results of meditation depend on the manner of development.%
\item[DN 7: With \textsanskrit{Jāliya} (\textit{\textsanskrit{Jāliyasutta}})] This discourse is mostly quoted by the Buddha in the previous.%
\item[DN 8: The Longer Discourse on the Lion’s Roar (\textit{\textsanskrit{Mahāsīhanādasutta}})] The Buddha is challenged by a naked ascetic on the topic of spiritual austerities. He points out that it is quite possible to perform all kinds of austere practices without having any inner purity of mind.%
\item[DN 9: With \textsanskrit{Poṭṭhapāda} (\textit{\textsanskrit{Poṭṭhapādasutta}})] The Buddha discusses with a wanderer the nature of perception and how it evolves through deeper states of meditation. None of these, however, should be identified with a self or soul.%
\item[DN 10: With Subha (\textit{\textsanskrit{Subhasutta}})] Shortly after the Buddha’s death, Venerable Ānanda is invited to explain the core teachings.%
\item[DN 11: With Kevaddha (\textit{\textsanskrit{Kevaṭṭasutta}})] The Buddha refuses to perform miracles, explaining that this is not the right way to inspire faith. He goes on to tell the story of a monk whose misguided quest for answers led him as far as \textsanskrit{Brahmā}.%
\item[DN 12: With Lohicca (\textit{\textsanskrit{Lohiccasutta}})] A brahmin has fallen into the idea that there is no point in trying to offer spiritual help to others. The Buddha goes to see him, and persuades him of the genuine benefits of spiritual teaching.%
\item[DN 13: The Three Knowledges (\textit{\textsanskrit{Tevijjasutta}})] A number of brahmins are discussing the true path to \textsanskrit{Brahmā}. Contesting the claims to authority based on the Vedas, the Buddha insists that only personal experience can lead to the truth.%
\end{description}

%
\mainmatter%
\pagestyle{fancy}%
\addtocontents{toc}{\let\protect\contentsline\protect\nopagecontentsline}
\part*{The Chapter on the Entire Spectrum of Ethics }
\addcontentsline{toc}{part}{The Chapter on the Entire Spectrum of Ethics }
\markboth{}{}
\addtocontents{toc}{\let\protect\contentsline\protect\oldcontentsline}

%
\chapter*{{\suttatitleacronym DN 1}{\suttatitletranslation The Divine Net }{\suttatitleroot Brahmajālasutta}}
\addcontentsline{toc}{chapter}{\tocacronym{DN 1} \toctranslation{The Divine Net } \tocroot{Brahmajālasutta}}
\markboth{The Divine Net }{Brahmajālasutta}
\extramarks{DN 1}{DN 1}

\section*{1. Talk on Wanderers }

\scevam{So\marginnote{1.1.1} I have heard.\footnote{Tradition holds that these were the words spoken by Ānanda when reciting the \textsanskrit{Suttapiṭaka} at the First Council following the Buddha’s death. In fact it is a tag signifying that the text has been passed down through oral tradition and the speaker was not present at the events (\href{https://suttacentral.net/dn5/en/sujato\#21.10}{DN 5:21.10}, \href{https://suttacentral.net/mn127/en/sujato\#17.4}{MN 127:17.4}). | This sutta with its commentary was translated by Bhikkhu Bodhi in his \emph{The All-Embracing Net of Views}. } }At one time the Buddha was traveling along the road between \textsanskrit{Rājagaha} and \textsanskrit{Nāḷandā} together with a large \textsanskrit{Saṅgha} of five hundred mendicants.\footnote{By convention, suttas do not specify the date, so we have scant internal chronology. | Walk about fifteen kilometers north from \textsanskrit{Rājagaha} (modern Rajgir) to reach \textsanskrit{Nāḷandā}. | “Mendicant” is a literal translation of \textit{bhikkhu}, one who goes for alms. } The wanderer Suppiya was also traveling along the same road, together with his resident pupil, the student Brahmadatta.\footnote{A “wanderer” (\textit{\textsanskrit{paribbājaka}}) was a homeless religious renunciate—male or female, Brahmanical or other—who wandered seeking alms. | “Resident pupil” (or elsewhere just “pupil”) is \textit{\textsanskrit{antevāsi}}, a live-in apprentice of a master. | “Student” (\textit{\textsanskrit{māṇava}}) is a young man who was learning the Vedas from a master. } Meanwhile, Suppiya criticized the Buddha, the teaching, and the \textsanskrit{Saṅgha} in many ways,\footnote{Today these are called the Triple Gem that makes up the Buddhist religion; however they are not known by that term in the early texts. } but his pupil Brahmadatta praised them in many ways.\footnote{While the disagreement of student and teacher signifies their confusion, it also represents the diversity of views within the brahmanical caste and the openness with which a student could disagree with their teacher. } And so both tutor and pupil followed behind the Buddha and the \textsanskrit{Saṅgha} of mendicants directly contradicting each other. 

Then\marginnote{1.2.1} the Buddha took up residence for the night in the royal rest-house in \textsanskrit{Ambalaṭṭhikā} together with the \textsanskrit{Saṅgha} of mendicants.\footnote{\textsanskrit{Ambalaṭṭhikā} means “place of mango saplings”. It was a rest-house set up by the king of \textsanskrit{Rājagaha} about a day’s journey from the capital. It must have been sizable. } And Suppiya and Brahmadatta did likewise. There too, Suppiya criticized the Buddha, the teaching, and the \textsanskrit{Saṅgha} in many ways, but his pupil Brahmadatta praised them in many ways. And so both tutor and pupil kept on directly contradicting each other. 

Then\marginnote{1.3.1} several mendicants rose at the crack of dawn and sat together in the pavilion, where the topic of judgmentalism came up:\footnote{\textit{\textsanskrit{Saṅkhiyadhamma}} is a unique term. The commentary glosses it as \textit{\textsanskrit{kathādhamma}}, following which it has been translated as “conversation” or “trend of conversation”. But \textit{\textsanskrit{saṅkhā}} means “appraisal, assessment, evaluation, measuring, calculating”, and here the subject of discussion is the different ways the two parties assess or judge the Buddha, Dhamma, and \textsanskrit{Saṅgha}. Compare the “appraisal” of the Buddha at \href{https://suttacentral.net/dn19/en/sujato\#19.2}{DN 19:19.2}. } 

“It’s\marginnote{1.3.2} incredible, reverends, it’s amazing how the diverse convictions of sentient beings have been clearly comprehended by the Blessed One, who knows and sees, the perfected one, the fully awakened Buddha.\footnote{\textit{Adhimutti} is something that has been decided, a conviction or belief. } For this Suppiya criticizes the Buddha, the teaching, and the \textsanskrit{Saṅgha} in many ways, while his pupil Brahmadatta praises them in many ways. And so both tutor and pupil followed behind the Buddha and the \textsanskrit{Saṅgha} of mendicants directly contradicting each other.” 

When\marginnote{1.4.1} the Buddha found out about this discussion on judgmentalism among the mendicants, he went to the pavilion, where he sat on the seat spread out and addressed the mendicants,\footnote{This would have been an open air pavilion in the rest-house. By convention, when a teacher or other respected person is to sit, a sitting mat is spread out or made ready for them. } “Mendicants, what were you sitting talking about just now? What conversation was left unfinished?”\footnote{The very first words of the Buddha in the \textsanskrit{Suttapiṭaka}: he asks to hear what others are saying. } 

The\marginnote{1.4.3} mendicants told him what had happened, adding,\footnote{Here and in similar passages the Pali repeats all and I abbreviate. } “This was our conversation that was unfinished when the Buddha arrived.” 

“Mendicants,\marginnote{1.5.1} if others criticize me, the teaching, or the \textsanskrit{Saṅgha}, don’t make yourselves resentful, bitter, and exasperated.\footnote{The phrasing here is somewhat unusual and specific. They “should not do” what creates bitterness (i.e. judging others). Compare \href{https://suttacentral.net/mn22/en/sujato}{MN 22}, where the same phrases are used. In the Buddha’s case, it has the neutral \textit{hoti}, while for the mendicants it uses \textit{\textsanskrit{karaṇīya}}, as here. } You’ll get angry and upset, which would be an obstacle for you alone.\footnote{Complaining about others does not hurt them, only the one who gets upset. } If others were to criticize me, the teaching, or the \textsanskrit{Saṅgha}, and you got angry and upset, would you be able to understand whether they spoke well or poorly?”\footnote{Equanimity is a prerequisite for evaluating facts. } 

“No,\marginnote{1.5.4} sir.” 

“If\marginnote{1.6.1} others criticize me, the teaching, or the \textsanskrit{Saṅgha}, you should explain that what is untrue is in fact untrue: ‘This is why that’s untrue, this is why that’s false. There’s no such thing in us, it’s not found among us.’ 

If\marginnote{1.6.3} others praise me, the teaching, or the \textsanskrit{Saṅgha}, don’t make yourselves thrilled, elated, and excited. You’ll get thrilled, elated, and excited, which would be an obstacle for you alone. If others praise me, the teaching, or the \textsanskrit{Saṅgha}, you should acknowledge that what is true is in fact true: ‘This is why that’s true, this is why that’s correct. There is such a thing in us, it is found among us.’ 

\section*{2. Ethics }

\subsection*{2.1. The Shorter Section on Ethics }

When\marginnote{1.7.1} an ordinary person speaks praise of the Realized One, they speak only of trivial, insignificant details of mere ethics.\footnote{Ethics (or morality or virtue, \textit{\textsanskrit{sīla}}) is important, but it pales in comparison with the higher dimensions of the Buddha’s path. For an example of this kind of praise see \href{https://suttacentral.net/mn77/en/sujato\#8.1}{MN 77:8.1}. | The Buddha often referred to himself as “the Realized One” (\textit{\textsanskrit{tathāgata}}). } And what are the trivial, insignificant details of mere ethics that an ordinary person speaks of?\footnote{Here the Buddha lays out in detail the ethical conduct for mendicant followers. } 

‘The\marginnote{1.8.1} ascetic Gotama has given up killing living creatures. He has renounced the rod and the sword. He’s scrupulous and kind, living full of sympathy for all living beings.’\footnote{The first and most important precept. It is not just the negative injunction to avoid killing, but also the positive injunction to have compassion for all creatures. | The Buddha is called “the ascetic Gotama” by non-Buddhists. } Such is an ordinary person’s praise of the Realized One.\footnote{An “ordinary person” (\textit{puthujjana}) is anyone who has not, at minimum, entered the path to stream-entry. } 

‘The\marginnote{1.8.3} ascetic Gotama has given up stealing. He takes only what’s given, and expects only what’s given. He keeps himself clean by not thieving.’ Such is an ordinary person’s praise of the Realized One. 

‘The\marginnote{1.8.5} ascetic Gotama has given up unchastity. He is celibate, set apart, avoiding the vulgar act of sex.’\footnote{“Chastity” is \textit{brahmacariya}, literally “divine conduct”. Here it is used in the narrow sense of refraining from sex, but more commonly it has a broader sense of “spiritual life”. } Such is an ordinary person’s praise of the Realized One. 

‘The\marginnote{1.9.1} ascetic Gotama has given up lying. He speaks the truth and sticks to the truth. He’s honest and dependable, and doesn’t trick the world with his words.’\footnote{This is the first of the four kinds of right speech. Just as the precept of not killing implies the positive injunction to live with compassion, the precepts on speech enjoin a positive and constructive use of speech. } Such is an ordinary person’s praise of the Realized One. 

‘The\marginnote{1.9.3} ascetic Gotama has given up divisive speech. He doesn’t repeat in one place what he heard in another so as to divide people against each other. Instead, he reconciles those who are divided, supporting unity, delighting in harmony, loving harmony, speaking words that promote harmony.’\footnote{“Harmony” (or “unanimity”, \textit{samagga}) does not excuse untrue, bigoted, or otherwise harmful speech. True harmony is only achieved in the presence of the Dhamma. } Such is an ordinary person’s praise of the Realized One. 

‘The\marginnote{1.9.5} ascetic Gotama has given up harsh speech. He speaks in a way that’s mellow, pleasing to the ear, lovely, going to the heart, polite, likable and agreeable to the people.’ Such is an ordinary person’s praise of the Realized One. 

‘The\marginnote{1.9.7} ascetic Gotama has given up talking nonsense. His words are timely, true, and meaningful, in line with the teaching and training. He says things at the right time which are valuable, reasonable, succinct, and beneficial.’\footnote{\textit{Attha} is a polyvalent term, here taking the senses  “meaningful” and “beneficial”. Elsewhere it means “goal”, “need”, “purpose”, “lawsuit”, or “ending”, and the senses are not always easy to untangle. } Such is an ordinary person’s praise of the Realized One. 

‘The\marginnote{1.10.1} ascetic Gotama refrains from injuring plants and seeds.’\footnote{Buddhists generally do not regard plants as sentient, but value them as part of the ecosystem that supports all life. } 

‘He\marginnote{1.10.3} eats in one part of the day, abstaining from eating at night and food at the wrong time.’\footnote{From \href{https://suttacentral.net/mn66/en/sujato\#6.4}{MN 66:6.4} and \href{https://suttacentral.net/mn70/en/sujato\#4.8}{MN 70:4.8} we can see that “at night” means after dark, while “at the wrong time” means in the afternoon. More explicitly, these are the “wrong time at night” and the “wrong time in the day”, in which case they are both the “wrong time”. } 

‘He\marginnote{1.10.4} refrains from seeing shows of dancing, singing, and music .’\footnote{Such sensual entertainments distract and excite the mind. This and the next three precepts encourage peace of mind for meditation. } 

‘He\marginnote{1.10.5} refrains from beautifying and adorning himself with garlands, fragrance, and makeup.’\footnote{This was ignored by the Buddha’s cousin, Nanda (\href{https://suttacentral.net/sn21.8/en/sujato\#1.2}{SN 21.8:1.2}). } 

‘He\marginnote{1.10.6} refrains from high and luxurious beds.’\footnote{To avoid sleeping too much. } 

‘He\marginnote{1.10.7} refrains from receiving gold and currency,\footnote{Literally “gold and silver” (\textit{\textsanskrit{jātarūparajata}}), but \textit{rajata} is explained in \href{https://suttacentral.net/pli-tv-bu-vb-np18/en/sujato\#2.8}{Bu NP 18:2.8} as currency of any kind. } raw grains,\footnote{Mendicants receive only the day’s meal and do not store or cook food. } raw meat, women and girls, male and female bondservants,\footnote{According to ancient Indian law (\textsanskrit{Arthaśāstra} 3.13), a person in a time of trouble may bind themselves in service for a fee. Such bondservants were protected against cruelty, sexual abuse, and unfair work. After earning back the fee of their indenture they were freed, retaining their original inheritance and status. } goats and sheep,\footnote{These are animals raised for food. } chickens and pigs, elephants, cows, horses, and mares, and fields and land.’\footnote{Land for a monastery may be accepted by the \textsanskrit{Saṅgha} as a community, but not by individual mendicants. } 

‘He\marginnote{1.10.16} refrains from running errands and messages;\footnote{These items are discussed in detail below. | Acting as a go-between for lay business was tempting due to the mendicants’ wandering lifestyle. However, it exposes them to risk if the message is not delivered or if it is bad news. } buying and selling;\footnote{For example, trading in monastery property. } falsifying weights, metals, or measures; bribery, fraud, cheating, and duplicity; mutilation, murder, abduction, banditry, plunder, and violence.’ Such is an ordinary person’s praise of the Realized One. 

\scendsection{The shorter section on ethics is finished. }

\subsection*{2.2. The Middle Section on Ethics }

‘There\marginnote{1.11.1} are some ascetics and brahmins who, while enjoying food given in faith, still engage in injuring plants and seeds.\footnote{This section expands some of the former section in further detail. | The “middle” and “large” sections on ethics are not found in briefer presentations such as \href{https://suttacentral.net/mn27/en/sujato\#14.1}{MN 27:14.1}. } These include plants propagated from roots, stems, cuttings, or joints; and those from regular seeds as the fifth.\footnote{That these are not five “kinds of seeds” but five kinds of “plants grown from seeds” is clear from the Vinaya and its commentary (\href{https://suttacentral.net/pli-tv-bu-vb-pc11/en/sujato}{Bu Pc 11}:  \textit{\textsanskrit{Bhūtagāmo} \textsanskrit{nāma} \textsanskrit{pañca} \textsanskrit{bījajātāni}}). } The ascetic Gotama refrains from such injury to plants and seeds.’ Such is an ordinary person’s praise of the Realized One. 

‘There\marginnote{1.12.1} are some ascetics and brahmins who, while enjoying food given in faith, still engage in storing up goods for their own use.\footnote{For storing up food as a sign of decline, see \href{https://suttacentral.net/dn27/en/sujato\#17.5}{DN 27:17.5}. } This includes such things as food, drink, clothes, vehicles, bedding, fragrance, and things of the flesh. The ascetic Gotama refrains from storing up such goods.’ Such is an ordinary person’s praise of the Realized One. 

‘There\marginnote{1.13.1} are some ascetics and brahmins who, while enjoying food given in faith, still engage in seeing shows. This includes such things as dancing, singing, music, performances, and storytelling; clapping, gongs, and kettledrums; beauty pageants; pole-acrobatics and bone-washing displays of the corpse-workers; battles of elephants, horses, buffaloes, bulls, goats, rams, chickens, and quails; staff-fights, boxing, and wrestling; combat, roll calls of the armed forces, battle-formations, and regimental reviews.\footnote{\textit{Sobhanaka} (“beauty pageant”) is explained by the commentary as the movement (or “sprinkling”) of dancers, or their beautification and painting. The PTS reading \textit{sobha-\textsanskrit{nagarakaṁ}}, supported by an unrelated reference to a \textit{gandhabba} city of that name, is spurious. | \textit{\textsanskrit{Caṇḍālaṁ} \textsanskrit{vaṁsaṁ} \textsanskrit{dhovanaṁ}} should be a compound, as shown by the prose to \href{https://suttacentral.net/ja498/en/sujato}{Ja 498}, where it is a performance by corpse-workers (\textit{\textsanskrit{caṇḍāla}}) in \textsanskrit{Ujjenī}. \textit{\textsanskrit{Vaṁsa}} is the bamboo used by \textit{\textsanskrit{caṇḍāla}} acrobats (\href{https://suttacentral.net/sn47.19/en/sujato}{SN 47.19}). \textit{Dhovana} is referred by the commentary to \href{https://suttacentral.net/an10.107/en/sujato}{AN 10.107}, where it is a southern ceremony accompanied by drink and dance. The commentaries to AN and DN say it was the ritual washing of the bones of the buried dead after the decomposition of the flesh. Such “second funeral” rites have been observed world-wide. From the \textsanskrit{Jātaka} it appears that the tradition had declined to a mere display for passers-by, perhaps featuring naked tribal girls. | \textit{Uyodhika} is sometimes said to be “sham fights”, but at \href{https://suttacentral.net/an10.30/en/sujato}{AN 10.30} it is not a sham. And the definition at \href{https://suttacentral.net/pli-tv-bu-vb-pc50/en/sujato}{Bu Pc 50} says “where strife is seen”. } The ascetic Gotama refrains from such shows.’ Such is an ordinary person’s praise of the Realized One. 

‘There\marginnote{1.14.1} are some ascetics and brahmins who, while enjoying food given in faith, still engage in gambling that causes negligence.\footnote{See too \href{https://suttacentral.net/dn31/en/sujato\#11.1}{DN 31:11.1}. } This includes such things as checkers with eight or ten rows, checkers in the air, hopscotch, spillikins, board-games, tip-cat, drawing straws, dice, leaf-flutes, toy plows, somersaults, pinwheels, toy measures, toy carts, toy bows, guessing words from syllables, guessing another’s thoughts, and imitating musical instruments.\footnote{“Checkers” (\textit{\textsanskrit{aṭṭhapada}}) was presumably the ancestor of the Gupta period \textit{\textsanskrit{caturaṅga}} and hence modern chess. | \textit{\textsanskrit{Yathāvajja}} is explained in the commentary as “mimicking deformities”, but I cannot find support elsewhere in Pali or Sanskrit for \textit{vajja} in this sense. More likely it refers to musical instruments (Sanskrit \textit{\textsanskrit{vādya}}). } The ascetic Gotama refrains from such gambling.’ Such is an ordinary person’s praise of the Realized One. 

‘There\marginnote{1.15.1} are some ascetics and brahmins who, while enjoying food given in faith, still make use of high and luxurious bedding. This includes such things as sofas, couches, woolen covers—shag-piled, colorful, white, embroidered with flowers, quilted, embroidered with animals, double- or single-fringed—and silk covers studded with gems, as well as silken sheets, woven carpets, rugs for elephants, horses, or chariots, antelope hide rugs, and spreads of fine deer hide, with a canopy above and red cushions at both ends. The ascetic Gotama refrains from such bedding.’ Such is an ordinary person’s praise of the Realized One. 

‘There\marginnote{1.16.1} are some ascetics and brahmins who, while enjoying food given in faith, still engage in beautifying and adorning themselves with garlands, fragrance, and makeup. This includes such things as applying beauty products by anointing, massaging, bathing, and rubbing; mirrors, ointments, garlands, fragrances, and makeup; face-powder, foundation, bracelets, headbands, fancy walking-sticks or containers, rapiers, parasols, fancy sandals, turbans, jewelry, chowries, and long-fringed white robes. The ascetic Gotama refrains from such beautification and adornment.’ Such is an ordinary person’s praise of the Realized One. 

‘There\marginnote{1.17.1} are some ascetics and brahmins who, while enjoying food given in faith, still engage in low talk. This includes such topics as\footnote{\textit{\textsanskrit{Tiracchānakathā}} literally means “animal talk”. The Pali word for animal, \textit{\textsanskrit{tiracchāna}} has the sense of “moving horizontally”, and “low talk” is that which does not elevate. } talk about kings, bandits, and ministers; talk about armies, threats, and wars; talk about food, drink, clothes, and beds; talk about garlands and fragrances; talk about family, vehicles, villages, towns, cities, and countries; talk about women and heroes; street talk and well talk; talk about the departed; motley talk; tales of land and sea; and talk about being reborn in this or that place.\footnote{\textit{\textsanskrit{Bhavābhava}} does not mean “existence and non-existence” but is a distributive compound, “this or that state of existence”. Indian religious texts are full of discussions about different heavens and hells. } The ascetic Gotama refrains from such low talk.’ Such is an ordinary person’s praise of the Realized One. 

‘There\marginnote{1.18.1} are some ascetics and brahmins who, while enjoying food given in faith, still engage in arguments. They say such things as: “You don’t understand this teaching and training. I understand this teaching and training. What, you understand this teaching and training? You’re practicing wrong. I’m practicing right. I stay on topic, you don’t. You said last what you should have said first. You said first what you should have said last. What you’ve thought so much about has been disproved. Your doctrine is refuted. Go on, save your doctrine! You’re trapped; get yourself out of this—if you can!”\footnote{The folly of disputatiousness is a consistent theme in the suttas, but is a special focus of the \textsanskrit{Aṭṭhakavagga} of the \textsanskrit{Suttanipāta}. } The ascetic Gotama refrains from such argumentative talk.’ Such is an ordinary person’s praise of the Realized One. 

‘There\marginnote{1.19.1} are some ascetics and brahmins who, while enjoying food given in faith, still engage in running errands and messages. This includes running errands for rulers, ministers, aristocrats, brahmins, householders, or princes who say: “Go here, go there. Take this, bring that from there.”\footnote{“Rulers” (\textit{\textsanskrit{raññaṁ}}, genitive plural) include hereditary kings as well as the elected joint leaders of republican states such as the Sakyans or Vajjis. } The ascetic Gotama refrains from such errands.’ Such is an ordinary person’s praise of the Realized One. 

‘There\marginnote{1.20.1} are some ascetics and brahmins who, while enjoying food given in faith, still engage in deceit, flattery, hinting, and belittling, and using material things to chase after other material things.\footnote{Some renunciants like to butter up potential donors, or make ostentatious displays to prompt further donations. | “Using material possessions to chase after other material possessions” includes trading monastery property for profit. } The ascetic Gotama refrains from such deceit and flattery.’ Such is an ordinary person’s praise of the Realized One. 

\scendsection{The middle section on ethics is finished. }

\subsection*{2.3. The Large Section on Ethics }

‘There\marginnote{1.21.1} are some ascetics and brahmins who, while enjoying food given in faith, still earn a living by low lore, by wrong livelihood.\footnote{This section focuses on practices that are wrong livelihood for a mendicant, though not for lay people. The Vinaya explains “low lore” as whatever is non-Buddhist or useless (\href{https://suttacentral.net/pli-tv-bi-vb-pc49/en/sujato}{Bi Pc 49}), while the commentary says it leads not to emancipation but to heaven. } This includes such fields as limb-reading, omenology, divining celestial portents, interpreting dreams, divining bodily marks, divining holes in cloth gnawed by mice, fire offerings, ladle offerings, offerings of husks, rice powder, rice, ghee, or oil; offerings from the mouth, blood sacrifices, palmistry; geomancy for building sites, fields, and cemeteries; exorcisms, earth magic, snake charming, poisons; the lore of the scorpion, the rat, the bird, and the crow; prophesying lifespan, chanting for protection, and divining omens from wild animals.\footnote{Reading \textit{\textsanskrit{khattavijjā}} (“political science”) per variant as \textit{\textsanskrit{khettavijjā}} (“geomancy”). | \textit{Sara} in \textit{saraparitta} means “sound” not “arrow”; compare with \textit{\textsanskrit{sarabhañña}} “chanting”. | \textit{Migacakka} is explained in the commentary, supported by the astrological text \textsanskrit{Bṛhatsaṁhitā}, as interpretation of the cries and behaviors of wild animals. Here the suffix \textit{-cakka} refers to the field of study. See also \href{https://suttacentral.net/mil5.3.7/en/sujato\#6.1}{Mil 5.3.7:6.1} \textit{\textsanskrit{sācakkaṁ} \textsanskrit{migacakkaṁ} \textsanskrit{antaracakkaṁ}} (“divining omens from dogs, wild animals, and the directions around”), terms which are also found at \textsanskrit{Bṛhatsaṁhitā} 2. } The ascetic Gotama refrains from such low lore, such wrong livelihood.’ Such is an ordinary person’s praise of the Realized One. 

‘There\marginnote{1.22.1} are some ascetics and brahmins who, while enjoying food given in faith, still earn a living by low lore, by wrong livelihood. This includes reading the marks of gems, cloth, clubs, swords, spears, arrows, bows, weapons, women, men, boys, girls, male and female bondservants, elephants, horses, buffaloes, bulls, cows, goats, rams, chickens, quails, monitor lizards, rabbits, tortoises, or deer.\footnote{The commentary oddly has “earrings or house-gables” for \textit{\textsanskrit{kaṇṇika}} (“eared one”), but it must be “rabbit”, for which see \textit{\textsanskrit{sasakaṇṇikā}} at \href{https://suttacentral.net/ja535/en/sujato\#76}{Ja 535:76}. } The ascetic Gotama refrains from such low lore, such wrong livelihood.’ Such is an ordinary person’s praise of the Realized One. 

‘There\marginnote{1.23.1} are some ascetics and brahmins who, while enjoying food given in faith, still earn a living by low lore, by wrong livelihood. This includes making predictions that the king will march forth or march back; or that our king will attack and the enemy king will retreat, or vice versa; or that our king will triumph and the enemy king will be defeated, or vice versa; and so there will be victory for one and defeat for the other. The ascetic Gotama refrains from such low lore, such wrong livelihood.’ Such is an ordinary person’s praise of the Realized One. 

‘There\marginnote{1.24.1} are some ascetics and brahmins who, while enjoying food given in faith, still earn a living by low lore, by wrong livelihood. This includes making predictions that there will be an eclipse of the moon, or sun, or stars; that the sun, moon, and stars will be in conjunction or in opposition; that there will be a meteor shower, a fiery sky, an earthquake, or thunder in the heavens; that there will be a rising, a setting, a darkening, a brightening of the moon, sun, and stars. And it also includes making predictions about the results of all such phenomena.\footnote{Despite this, astrology is commonly practiced today among Buddhist mendicants. | \textit{(Up)pathagamana} can hardly mean that the sun, moon, and stars will “go astray”. Rather, \textit{patha} here has the sense of “range”, so it means “come within range”, which describes an astrological conjunction. | For “fiery sky” (\textit{\textsanskrit{disāḍāha}}) as an ill omen, see \textit{\textsanskrit{diśāṁ} \textsanskrit{dāhe}} at \textsanskrit{Manusmṛti} 4.115. } The ascetic Gotama refrains from such low lore, such wrong livelihood.’ Such is an ordinary person’s praise of the Realized One. 

‘There\marginnote{1.25.1} are some ascetics and brahmins who, while enjoying food given in faith, still earn a living by low lore, by wrong livelihood. This includes predicting whether there will be plenty of rain or drought; plenty to eat or famine; an abundant harvest or a bad harvest; security or peril; sickness or health. It also includes such occupations as arithmetic, accounting, calculating, poetry, and cosmology.\footnote{“Cosmology” (\textit{\textsanskrit{lokāyata}}) in early Buddhist texts is not, as it later became known, the heterodox school of materialism. Rather, it was a branch of worldly knowledge within regular Vedic studies concerned with the nature and extent of the world and how this may be known (\href{https://suttacentral.net/an9.38/en/sujato}{AN 9.38}, \href{https://suttacentral.net/sn12.48/en/sujato}{SN 12.48}). } The ascetic Gotama refrains from such low lore, such wrong livelihood.’ Such is an ordinary person’s praise of the Realized One. 

‘There\marginnote{1.26.1} are some ascetics and brahmins who, while enjoying food given in faith, still earn a living by low lore, by wrong livelihood. This includes making arrangements for giving and taking in marriage; for engagement and divorce; and for scattering rice inwards or outwards at the wedding ceremony. It also includes casting spells for good or bad luck, treating impacted fetuses, binding the tongue, or locking the jaws; charms for the hands and ears; questioning a mirror, a girl, or a god as an oracle; worshiping the sun, worshiping the Great One, breathing fire, and invoking Siri, the goddess of luck.\footnote{The commentary has \textit{\textsanskrit{saṅkiraṇa}}/\textit{\textsanskrit{vikiraṇa}} as “saving and spending” (cp. \href{https://suttacentral.net/snp1.6/en/sujato\#23.1}{Snp 1.6:23.1}), but it seems unlikely. \textit{\textsanskrit{Vikiraṇa}} means “scattering” food or sand, while Sanskrit  \textit{vikira} is the ritual scattering of rice. Given the context, I think it refers to the custom of scattering rice at a wedding. | For \textit{\textsanskrit{viruddhagabbhakaraṇa}}, \textit{viruddha} means “obstructed”. The commentary here, in general agreement with the Niddesa on \textit{\textsanskrit{gabbhakaraṇa}} at \href{https://suttacentral.net/snp4.14/en/sujato}{Snp 4.14}, explains as giving treatments for the survival of the fetus. | I omit \textit{hanujappana} as it is absent from the commentary and seems to have just arisen by confusion. } The ascetic Gotama refrains from such low lore, such wrong livelihood.’ Such is an ordinary person’s praise of the Realized One. 

‘There\marginnote{1.27.1} are some ascetics and brahmins who, while enjoying food given in faith, still earn a living by low lore, by wrong livelihood. This includes rites for propitiation, for granting wishes, for ghosts, for the earth, for rain, for property settlement, and for preparing and consecrating house sites, and rites involving rinsing and bathing, and oblations. It also includes administering emetics, purgatives, expectorants, and phlegmagogues; administering ear-oils, eye restoratives, nasal medicine, ointments, and counter-ointments; surgery with needle and scalpel, treating children, prescribing root medicines, and binding on herbs.\footnote{Medicine is right livelihood, but a mendicant should not make a living from it. They may treat fellow mendicants, family members, or those close to the monastery. | \textit{Santikamma} is the Sanskrit \textit{\textsanskrit{śāntikakarman}}, a rite for averting evil. | For \textit{vassakamma} and \textit{vossakamma}, the commentary has “fertile and infertile men” (\textit{vassoti puriso, vossoti \textsanskrit{paṇḍako}}), taking “rain” as a metaphor for semen. Such usages do have precedent elsewhere. But in context I take \textit{vassa} simply as “rain” and \textit{vossa} as equivalent to Sanskrit \textit{vyavasya} in the sense of making a settlement for land. | I take \textit{\textsanskrit{paṭimokkho}} in the sense of “binding” (cf. \textit{\textsanskrit{paṭimukka}} at \href{https://suttacentral.net/mn38/en/sujato\#41.11}{MN 38:41.11} etc.) rather than the commentary’s “release” (from the effects of caustic medicines; cf. \href{https://suttacentral.net/tha-ap25/en/sujato\#5.4}{Tha Ap 25:5.4}). } The ascetic Gotama refrains from such low lore, such wrong livelihood.’ Such is an ordinary person’s praise of the Realized One. 

These\marginnote{1.27.5} are the trivial, insignificant details of mere ethics that an ordinary person speaks of when they speak praise of the Realized One. 

\scendsection{The longer section on ethics is finished. }

\section*{3. Views }

\subsection*{3.1. Theories About the Past }

There\marginnote{1.28.1} are other principles—deep, hard to see, hard to understand, peaceful, sublime, beyond the scope of logic, subtle, comprehensible to the astute—which the Realized One makes known after realizing them with his own insight. Those who genuinely praise the Realized One would rightly speak of these things.\footnote{One meaning of \textit{dhamma} is “principle” in the sense of a natural law as well as a moral value. | Here begins the famous exposition of the sixty-two views. The subtlety of the analysis lies in how, rather than refuting the details of the views, the Buddha traces them all back to their fundamental psychology. } And what are these principles? 

There\marginnote{1.29.1} are some ascetics and brahmins who theorize about the past, and assert various hypotheses concerning the past on eighteen grounds. And what are the eighteen grounds on which they rely? 

\subsubsection*{3.1.1. Eternalism }

There\marginnote{1.30.1} are some ascetics and brahmins who are eternalists, who assert that the self and the cosmos are eternal on four grounds.\footnote{In such contexts, the “self” (\textit{\textsanskrit{attā}}) is a postulated metaphysical entity rather than a simple psychological sense of personal identity. The nature of this “self” or “soul” was endlessly debated. The Buddha rejected all theories of a “self”, and elsewhere it is said that “identity view” underlies all sixty-two views of the \textsanskrit{Brahmajāla} (\href{https://suttacentral.net/sn41.3/en/sujato\#4.13}{SN 41.3:4.13}). | The “cosmos” is the \textit{loka}, otherwise translated as “world”. This sometimes refers to the simple physical realm, sometimes to the world of experience, or else, as here, the vast universe as conceived in ancient Indian thought. } And what are the four grounds on which they rely? 

It’s\marginnote{1.31.1} when some ascetic or brahmin—by dint of keen, resolute, committed, and diligent effort, and right application of mind—experiences an immersion of the heart of such a kind that they recollect their many kinds of past lives.\footnote{“Immersion” (\textit{\textsanskrit{samādhi}}) is deep meditative stillness. The word conveys the sense of “gathered”, “collected”, with a secondary sense of “ignited”, “illuminated”. The practice of \textit{\textsanskrit{samādhi}} (or \textit{\textsanskrit{jhāna}}, “absorption”) has never been regarded as uniquely Buddhist. However, right meditation begins with right view. Since these meditators begin with wrong view, their \textit{\textsanskrit{samādhi}} is “wrong” because it merely reinforces their error. } That is: one, two, three, four, five, ten, twenty, thirty, forty, fifty, a hundred, a thousand, a hundred thousand rebirths. They remember: ‘There, I was named this, my clan was that, I looked like this, and that was my food. This was how I felt pleasure and pain, and that was how my life ended. When I passed away from that place I was reborn somewhere else. There, too, I was named this, my clan was that, I looked like this, and that was my food. This was how I felt pleasure and pain, and that was how my life ended. When I passed away from that place I was reborn here.’ And so they recollect their many kinds of past lives, with features and details.\footnote{The recollection of past lives is specific, detailed, and confident as it is based on the clear mind of deep immersion. } 

They\marginnote{1.31.3} say: ‘The self and the cosmos are eternal, barren, steady as a mountain peak, standing firm like a pillar.\footnote{This is the \textsanskrit{Upaniṣadic} view of the eternal \textit{\textsanskrit{ātman}} that is the immanent soul of the world or cosmos, \textit{loka}. Elsewhere in the suttas such theorists assert that the self and the cosmos are identical (\href{https://suttacentral.net/sn24.3/en/sujato\#1.3}{SN 24.3:1.3}: \textit{so \textsanskrit{attā} so loko}). } They remain the same for all eternity, while these sentient beings wander and transmigrate and pass away and rearise.\footnote{The eternal “self” is contrasted with the ephemeral lives of beings. | The famous word \textit{\textsanskrit{saṁsara}} is often understood as a “cycle” of rebirths, but the meaning is, rather, to “wander on” or “transmigrate”. | For the phrase \textit{\textsanskrit{sassatisamaṁ}} (“lasting forever and ever”), compare \textsanskrit{Bṛhadāraṇyaka} \textsanskrit{Upaniṣad} 5.10.1: “He reaches that world free of sorrow and snow, where he lives forever and ever” (\textit{sa lokam \textsanskrit{āgacchaty} \textsanskrit{aśokam} ahimam | tasmin vasati \textsanskrit{śāśvatīḥ} \textsanskrit{samāḥ}}). } Why is that? Because by dint of keen, resolute, committed, and diligent effort, and right application of mind I experience an immersion of the heart of such a kind that I recollect my many kinds of past lives, with features and details. 

Because\marginnote{1.31.9} of this I know:\footnote{Their meditative experience revealed a process of transient and changing lives, yet from that they infer that there must be an eternal self. } 

“The\marginnote{1.31.10} self and the cosmos are eternal, barren, steady as a mountain peak, standing firm like a pillar. They remain the same for all eternity, while these sentient beings wander and transmigrate and pass away and rearise.”’ This is the first ground on which some ascetics and brahmins rely to assert that the self and the cosmos are eternal. 

And\marginnote{1.32.1} what is the second ground on which they rely? It’s when some ascetic or brahmin—by dint of keen, resolute, committed, and diligent effort, and right application of mind—experiences an immersion of the heart of such a kind that they recollect their many kinds of past lives. That is: one eon of the cosmos contracting and expanding; two, three, four, five, or ten eons of the cosmos contracting and expanding. They remember: ‘There, I was named this, my clan was that, I looked like this, and that was my food. This was how I felt pleasure and pain, and that was how my life ended. When I passed away from that place I was reborn somewhere else. There, too, I was named this, my clan was that, I looked like this, and that was my food. This was how I felt pleasure and pain, and that was how my life ended. When I passed away from that place I was reborn here.’ And so they recollect their many kinds of past lives, with features and details.\footnote{This differs only in the length of time, which is now up to ten eons. A single eon (\textit{kappa}) lasts longer than it would take to wear away a huge mountain by stroking it with a cloth once a century \href{https://suttacentral.net/sn15.5/en/sujato}{SN 15.5}, while the number of eons is greater than the sands in the Ganges river \href{https://suttacentral.net/sn15.8/en/sujato}{SN 15.8}. The vast time periods envisaged in early Buddhist texts are comparable with those of modern cosmology in physics. } 

They\marginnote{1.32.4} say: ‘The self and the cosmos are eternal, barren, steady as a mountain peak, standing firm like a pillar. They remain the same for all eternity, while these sentient beings wander and transmigrate and pass away and rearise. Why is that? Because by dint of keen, resolute, committed, and diligent effort, and right application of mind I experience an immersion of the heart of such a kind that I recollect my many kinds of past lives, with features and details. 

Because\marginnote{1.32.10} of this I know: 

“The\marginnote{1.32.11} self and the cosmos are eternal, barren, steady as a mountain peak, standing firm like a pillar. They remain the same for all eternity, while these sentient beings wander and transmigrate and pass away and rearise.”’ This is the second ground on which some ascetics and brahmins rely to assert that the self and the cosmos are eternal. 

And\marginnote{1.33.1} what is the third ground on which they rely? It’s when some ascetic or brahmin—by dint of keen, resolute, committed, and diligent effort, and right application of mind—experiences an immersion of the heart of such a kind that they recollect their many kinds of past lives. That is: ten eons of the cosmos contracting and expanding; twenty, thirty, or forty eons of the cosmos contracting and expanding. They remember: ‘There, I was named this, my clan was that, I looked like this, and that was my food. This was how I felt pleasure and pain, and that was how my life ended. When I passed away from that place I was reborn somewhere else. There, too, I was named this, my clan was that, I looked like this, and that was my food. This was how I felt pleasure and pain, and that was how my life ended. When I passed away from that place I was reborn here.’ And so they recollect their many kinds of past lives, with features and details. 

They\marginnote{1.33.4} say: ‘The self and the cosmos are eternal, barren, steady as a mountain peak, standing firm like a pillar. They remain the same for all eternity, while these sentient beings wander and transmigrate and pass away and rearise. Why is that? Because by dint of keen, resolute, committed, and diligent effort, and right application of mind I experience an immersion of the heart of such a kind that I recollect my many kinds of past lives, with features and details. 

Because\marginnote{1.33.10} of this I know: 

“The\marginnote{1.33.11} self and the cosmos are eternal, barren, steady as a mountain peak, standing firm like a pillar. They remain the same for all eternity, while these sentient beings wander and transmigrate and pass away and rearise.”’ This is the third ground on which some ascetics and brahmins rely to assert that the self and the cosmos are eternal. 

And\marginnote{1.34.1} what is the fourth ground on which they rely? It’s when some ascetic or brahmin relies on logic and inquiry. They speak of what they have worked out by logic, following a line of inquiry, expressing their own perspective:\footnote{These theorists used a process of logic to arrive at the same conclusion as the meditators. Different groups of ascetic philosophers emphasized contemplation or rational inquiry as the means to the truth. The Buddha acknowledged that both are useful but limited because, as here, they can sometimes lead to mistaken conclusions. } ‘The self and the cosmos are eternal, barren, steady as a mountain peak, standing firm like a pillar. They remain the same for all eternity, while these sentient beings wander and transmigrate and pass away and rearise.’ This is the fourth ground on which some ascetics and brahmins rely to assert that the self and the cosmos are eternal. 

These\marginnote{1.35.1} are the four grounds on which those ascetics and brahmins assert that the self and the cosmos are eternal. Any ascetics and brahmins who assert that the self and the cosmos are eternal do so on one or other of these four grounds. Outside of this there is none.\footnote{I have my doubts about this phrase. Everywhere else, \textit{ito \textsanskrit{bahiddhā}} means “outside of the Buddhist community”, not “outside of the cases just considered”. Still, the commentary and the Chinese parallel at T 21 agree on this sense. } 

The\marginnote{1.36.1} Realized One understands this: ‘If you hold on to and attach to these grounds for views it leads to such and such a destiny in the next life.’\footnote{A “view” (\textit{\textsanskrit{diṭṭhi}}) is a relatively fixed framework for understanding the world; a “theory”. The “grounds for views” (\textit{\textsanskrit{diṭṭhiṭṭhānāni}}) are the bases from which the views are derived. In this case these are the meditative experiences or the logical reasoning. } He understands this, and what goes beyond this. And since he does not misapprehend that understanding, he has realized quenching within himself.\footnote{The word \textit{\textsanskrit{parāmasati}} means “to take hold” and is often used in the sense “to misapprehend”. } Having truly understood the origin, ending, gratification, drawback, and escape from feelings, the Realized One is freed through not grasping.\footnote{\textit{\textsanskrit{Yathābhūtaṁ}} is often translated as “as it really is”, while I usually render it simply as “truly”. It often has a technical sense of seeing “how things came to be (\textit{bhuta})” as a process of conditionality (\href{https://suttacentral.net/sn12.31/en/sujato\#7.1}{SN 12.31:7.1}). Such direct vision of the truth is an attribute of the stream-enterer, who has realized the first of the four stages of awakening, in contrast with those on the path who still rely on faith or inference (\href{https://suttacentral.net/sn25.1/en/sujato}{SN 25.1}). Here it refers to the understanding of feelings from a fivefold perspective. Feelings underlie intellectual theories and arguments, which serve to sate cravings and fears. } 

These\marginnote{1.37.1} are the principles—deep, hard to see, hard to understand, peaceful, sublime, beyond the scope of logic, subtle, comprehensible to the astute—which the Realized One makes known after realizing them with his own insight. And those who genuinely praise the Realized One would rightly speak of these things. 

\scendsection{The first recitation section.\footnote{Long texts are sometimes marked by their “recitation sections” (\textit{\textsanskrit{bhāṇavāra}}), which was the length that would be recited in one session. } }

\subsubsection*{3.1.2. Partial Eternalism }

There\marginnote{2.1.1} are some ascetics and brahmins who are partial eternalists, who assert that the self and the cosmos are partially eternal and partially not eternal on four grounds.\footnote{Despite being views of the “self and the cosmos”, the main focus in the next four views is the self. } And what are the four grounds on which they rely? 

There\marginnote{2.2.1} comes a time when, after a very long period has passed, this cosmos contracts.\footnote{This is the end of an eon. It might be compared with what the physicists call the “big crunch”. } As the cosmos contracts, sentient beings are mostly headed for the realm of streaming radiance.\footnote{The human and similar realms are destroyed in the conflagration at the end of the universe, but sentient beings are sustained by the power of their past kamma. | The “realm of streaming radiance” is a \textsanskrit{Brahmā} heaven corresponding to the second \textit{\textsanskrit{jhāna}}. } There they are mind-made, feeding on rapture, self-luminous, wandering in midair, steadily glorious, and they remain like that for a very long time.\footnote{“Mind-made” (\textit{manomaya}) beings are spontaneously born due to past kamma, not by sex. | “Rapture” (\textit{\textsanskrit{pīti}}) is a joyous emotional response to pleasure, usually a spiritual sense of elevation or uplift in meditation. } 

There\marginnote{2.3.1} comes a time when, after a very long period has passed, this cosmos expands.\footnote{This might be compared with the “big bang” of a cyclic universe. } As it expands an empty mansion of divinity appears.\footnote{The realms into which beings are reborn exist interdependently with the beings themselves. The different dimensions correspond with different kinds of kamma. } Then a certain sentient being—due to the running out of their lifespan or merit—passes away from that host of radiant deities and is reborn in that empty mansion of divinity.\footnote{That is, they pass from a world corresponding to the second \textit{\textsanskrit{jhāna}} to one corresponding to the first \textit{\textsanskrit{jhāna}}. } There they are mind-made, feeding on rapture, self-luminous, wandering in midair, steadily glorious, and they remain like that for a very long time. 

But\marginnote{2.4.1} after staying there all alone for a long time, they become dissatisfied and anxious:\footnote{This passage echoes the creation myth in \textsanskrit{Bṛhadāraṇyaka} \textsanskrit{Upaniṣad} 1.4. At 1.4.2, the self-created Divinity feels fearful and alone, and at 1.4.3 and 1.4.17 wishes for a partner. From a Buddhist point of view, this simply shows how even God is trapped by emotional attachments in the cycle of transmigration. } ‘Oh, if only other beings would come to this place.’ Then other sentient beings—due to the running out of their lifespan or merit—pass away from that host of radiant deities and are reborn in that mansion of divinity in company with that being.\footnote{These beings are reborn according to their own kamma, and it is just a coincidence that they appear after the first being made their wish. } There they too are mind-made, feeding on rapture, self-luminous, wandering in midair, steadily glorious, and they remain like that for a very long time. 

Now,\marginnote{2.5.1} the being who was reborn there first thinks: ‘I am the Divinity, the Great Divinity, the Vanquisher, the Unvanquished, the Universal Seer, the Wielder of Power, God Almighty, the Maker, the Creator, the First, the Begetter, the Controller, the Father of those who have been born and those yet to be born.\footnote{His first words \textit{ahamasmi \textsanskrit{brahmā}} (“I am \textsanskrit{Brahmā}”) are equivalent to \textit{aham \textsanskrit{brahmāsmīti}} at 1.4.10. See too \textit{so’\textsanskrit{hamasmīti}} (“I am that”) at \textsanskrit{Bṛhadāraṇyaka} \textsanskrit{Upaniṣad} 1.4.1, which parallels the Pali \textit{eso’hamasmi} at \href{https://suttacentral.net/mn28/en/sujato\#6.8}{MN 28:6.8} etc. } These beings were created by me!\footnote{At \textsanskrit{Bṛhadāraṇyaka} \textsanskrit{Upaniṣad} 1.4.5 the Divinity thinks, “I created all this”. } Why is that? Because first I thought: 

“Oh,\marginnote{2.5.6} if only other beings would come to this place.” Such was my heart’s wish, and then these creatures came to this place.’\footnote{The other creatures appeared after his wish, not because of it. God confuses correlation with causation, a mistake perpetuated by no small number of his followers. } 

And\marginnote{2.5.8} the beings who were reborn there later also think: ‘This must be the Divinity, the Great Divinity, the Vanquisher, the Unvanquished, the Universal Seer, the Wielder of Power, God Almighty, the Maker, the Creator, the First, the Begetter, the Controller, the Father of those who have been born and those yet to be born. And we were created by him. Why is that? Because we see that he was reborn here first, and we arrived later.’\footnote{At \textsanskrit{Bṛhadāraṇyaka} \textsanskrit{Upaniṣad} 1.4.9–10 the created humans also think first about the Divinity who preceded them, from whose knowledge of self all was created. } 

And\marginnote{2.6.1} the being who was reborn first is more long-lived, beautiful, and illustrious than those who arrived later. 

It’s\marginnote{2.6.3} possible that one of those beings passes away from that host and is reborn in this place. Having done so, they go forth from the lay life to homelessness. By dint of keen, resolute, committed, and diligent effort, and right application of mind, they experience an immersion of the heart of such a kind that they recollect that past life, but no further.\footnote{Again, their meditation experience is genuine, but what they infer from it goes beyond the facts. } 

They\marginnote{2.6.6} say: ‘He who is the Divinity—the Great Divinity, the Vanquisher, the Unvanquished, the Universal Seer, the Wielder of Power, God Almighty, the Maker, the Creator, the First, the Begetter, the Controller, the Father of those who have been born and those yet to be born—by he we were created. He is permanent, everlasting, eternal, imperishable, remaining the same for all eternity. We who were created by that Divinity are impermanent, not lasting, short-lived, liable to pass away, and have come to this place.’\footnote{The surviving forms of Indic religion (Buddhism, Hinduism, Jainism) typically hold that all creatures ultimately share the same nature and hence can find liberation. Here we see this was not always the case, for these theorists believed that there are inherently different orders of beings in the cosmos. This is not due to their conduct but to the circumstances of their creation. } This is the first ground on which some ascetics and brahmins rely to assert that the self and the cosmos are partially eternal. 

And\marginnote{2.7.1} what is the second ground on which they rely? There are gods named ‘depraved by play.’ They spend too much time laughing, playing, and making merry. And in doing so, they lose their mindfulness, and they pass away from that host of gods.\footnote{Delightful as the life of the gods is, even they are supposed to retain a sense of moderation (\href{https://suttacentral.net/mn37/en/sujato\#11.2}{MN 37:11.2}), a lesson forgotten by those “depraved by play” (\textit{\textsanskrit{khiḍḍāpadosikā}}). | Note that mindfulness (\textit{sati}) is not held to be a specifically Buddhist virtue. Here it refers to a sense of moral compass and self-awareness, rather than a meditation practice. } 

It’s\marginnote{2.8.1} possible that one of those beings passes away from that host and is reborn in this place. Having done so, they go forth from the lay life to homelessness. By dint of keen, resolute, committed, and diligent effort, and right application of mind, they experience an immersion of the heart of such a kind that they recollect that past life, but no further. 

They\marginnote{2.9.1} say: ‘The gods not depraved by play don’t spend too much time laughing, playing, and making merry. So they don’t lose their mindfulness, and don’t pass away from that host of gods. They are permanent, everlasting, eternal, imperishable, remaining the same for all eternity. But we who were depraved by play spent too much time laughing, playing, and making merry. In doing so, we lost our mindfulness, and passed away from that host of gods. We are impermanent, not lasting, short-lived, liable to pass away, and have come to this place.’\footnote{Here the difference in beings is attributed not to the circumstances of their creation but to their behavior. } This is the second ground on which some ascetics and brahmins rely to assert that the self and the cosmos are partially eternal. 

And\marginnote{2.10.1} what is the third ground on which they rely? There are gods named ‘malevolent’. They spend too much time gazing at each other, so they grow angry with each other, and their bodies and minds get tired. They pass away from that host of gods.\footnote{The parallel between \textit{manopadosika} (“malevolent”) and \textit{khiddapadosika} (“depraved by play”) suggests a rendering “depraved in mind” for \textit{manopadosika}. However, elsewhere in the suttas \textit{manopadosa} consistently means “malicious intent” (\href{https://suttacentral.net/mn56/en/sujato\#13.15}{MN 56:13.15}, \href{https://suttacentral.net/mn93/en/sujato\#18.30}{MN 93:18.30}, \href{https://suttacentral.net/dn26/en/sujato\#20.3}{DN 26:20.3}. Also see \textit{mano \textsanskrit{padūseyya}} at \href{https://suttacentral.net/mn21/en/sujato\#20.1}{MN 21:20.1} and \href{https://suttacentral.net/mn28/en/sujato\#9.6}{MN 28:9.6}. Thus the contrast is between greed and hate. | Here, as usual, \textit{mano} and \textit{citta} are synonyms for “mind”. } 

It’s\marginnote{2.11.1} possible that one of those beings passes away from that host and is reborn in this place. Having done so, they go forth from the lay life to homelessness. By dint of keen, resolute, committed, and diligent effort, and right application of mind, they experience an immersion of the heart of such a kind that they recollect that past life, but no further. 

They\marginnote{2.12.1} say: ‘The gods who are not malevolent don’t spend too much time gazing at each other, so they don’t grow angry with each other, their bodies and minds don’t get tired, and they don’t pass away from that host of gods. They are permanent, everlasting, eternal, imperishable, remaining the same for all eternity. But we who were malevolent spent too much time gazing at each other, we grew angry with each other, our bodies and minds got tired, and we passed away from that host of gods. We are impermanent, not lasting, short-lived, liable to pass away, and have come to this place.’ This is the third ground on which some ascetics and brahmins rely to assert that the self and the cosmos are partially eternal. 

And\marginnote{2.13.1} what is the fourth ground on which they rely? It’s when some ascetic or brahmin relies on logic and inquiry. They speak of what they have worked out by logic, following a line of inquiry, expressing their own perspective: ‘That which is called “the eye”, “the ear”, “the nose”, “the tongue”, and also “the body”: that self is impermanent, not lasting, transient, perishable. That which is called “mind” or “sentience” or “consciousness”: that self is permanent, everlasting, eternal, imperishable, remaining the same for all eternity.’\footnote{This is mind-body dualism, the idea that the mind is made of a fundamentally different stuff than the body. } This is the fourth ground on which some ascetics and brahmins rely to assert that the self and the cosmos are partially eternal. 

These\marginnote{2.14.1} are the four grounds on which those ascetics and brahmins assert that the self and the cosmos are partially eternal and partially not eternal. Any ascetics and brahmins who assert that the self and the cosmos are partially eternal and partially not eternal do so on one or other of these four grounds. Outside of this there is none. 

The\marginnote{2.15.1} Realized One understands this: ‘If you hold on to and attach to these grounds for views it leads to such and such a destiny in the next life.’ He understands this, and what goes beyond this. And since he does not misapprehend that understanding, he has realized quenching within himself. Having truly understood the origin, ending, gratification, drawback, and escape from feelings, the Realized One is freed through not grasping. 

These\marginnote{2.15.5} are the principles—deep, hard to see, hard to understand, peaceful, sublime, beyond the scope of logic, subtle, comprehensible to the astute—which the Realized One makes known after realizing them with his own insight. And those who genuinely praise the Realized One would rightly speak of these things. 

\subsubsection*{3.1.3. The Cosmos is Finite or Infinite }

There\marginnote{2.16.1} are some ascetics and brahmins who theorize about size, and assert that the cosmos is finite or infinite on four grounds.\footnote{Here we move from views that conceive of both the self and the cosmos together to those that focus only on the physical extent of the cosmos. It is not  clear why these are classified as “views of the past”. } And what are the four grounds on which they rely? 

It’s\marginnote{2.17.1} when some ascetic or brahmin—by dint of keen, resolute, committed, and diligent effort, and right application of mind—experiences an immersion of the heart of such a kind that they meditate perceiving the cosmos as finite.\footnote{Once again the view is inferred from meditation, showing that meditative experience was regarded by some as revealing genuine truths about the physical realm. } 

They\marginnote{2.17.2} say: ‘The cosmos is finite and bounded. Why is that? Because by dint of keen, resolute, committed, and diligent effort, and right application of mind I experience an immersion of the heart of such a kind that I meditate perceiving the cosmos as finite.\footnote{The nature of their meditation is assumed to be the nature of the world itself. } Because of this I know: 

“The\marginnote{2.17.7} cosmos is finite and bounded.”’ This is the first ground on which some ascetics and brahmins rely to assert that the cosmos is finite or infinite. 

And\marginnote{2.18.1} what is the second ground on which they rely? It’s when some ascetic or brahmin—by dint of keen, resolute, committed, and diligent effort, and right application of mind—experiences an immersion of the heart of such a kind that they meditate perceiving the cosmos as infinite.\footnote{In each of the two previous sets of four views, the views themselves were the same, only the means of knowing them differed. Here the views themselves differ. The differences take the form of a tetralemma: A, not-A, both A and not-A, neither A nor not-A. This pattern is commonly found in early Buddhism, as well as Indian thought more generally. The final two items are not meant to be obscure or mysterious, but to express genuine possibilities that cannot be captured by a simple duality. } 

They\marginnote{2.18.3} say: ‘The cosmos is infinite and unbounded. The ascetics and brahmins who say that the cosmos is finite are wrong.\footnote{It is common today to say that one’s own experience is valid for oneself. Clearly that is not how these philosophers thought. } The cosmos is infinite and unbounded. Why is that? Because by dint of keen, resolute, committed, and diligent effort, and right application of mind I experience an immersion of the heart of such a kind that I meditate perceiving the cosmos as infinite. Because of this I know: 

“The\marginnote{2.18.11} cosmos is infinite and unbounded.”’ This is the second ground on which some ascetics and brahmins rely to assert that the cosmos is finite or infinite. 

And\marginnote{2.19.1} what is the third ground on which they rely? It’s when some ascetic or brahmin—by dint of keen, resolute, committed, and diligent effort, and right application of mind—experiences an immersion of the heart of such a kind that they meditate perceiving the cosmos as finite vertically but infinite horizontally.\footnote{They perceive the universe as spread out like a disc. One might call it a “discworld”. } 

They\marginnote{2.19.3} say: ‘The cosmos is both finite and infinite. The ascetics and brahmins who say that the cosmos is finite are wrong, and so are those who say that the cosmos is infinite. The cosmos is both finite and infinite. Why is that? Because by dint of keen, resolute, committed, and diligent effort, and right application of mind I experience an immersion of the heart of such a kind that I meditate perceiving the cosmos as finite vertically but infinite horizontally. Because of this I know: 

“The\marginnote{2.19.13} cosmos is both finite and infinite.”’ This is the third ground on which some ascetics and brahmins rely to assert that the cosmos is finite or infinite. 

And\marginnote{2.20.1} what is the fourth ground on which they rely? It’s when some ascetic or brahmin relies on logic and inquiry. They speak of what they have worked out by logic, following a line of inquiry, expressing their own perspective: ‘The cosmos is neither finite nor infinite.\footnote{The text doesn’t specify what this is, but it might include the view that the ideas “finite” and “infinite” are inadequate to describe the universe. Consider a universe expanding at the speed of light. At any point in time it is not infinite, but as it is impossible to reach its end it is not finite either. } The ascetics and brahmins who say that the cosmos is finite are wrong, as are those who say that the cosmos is infinite, and also those who say that the cosmos is both finite and infinite. The cosmos is neither finite nor infinite.’ This is the fourth ground on which some ascetics and brahmins rely to assert that the cosmos is finite or infinite. 

These\marginnote{2.21.1} are the four grounds on which those ascetics and brahmins assert that the cosmos is finite or infinite. Any ascetics and brahmins who assert that the cosmos is finite or infinite do so on one or other of these four grounds. Outside of this there is none. 

The\marginnote{2.22.1} Realized One understands this: ‘If you hold on to and attach to these grounds for views it leads to such and such a destiny in the next life.’ He understands this, and what goes beyond this. And since he does not misapprehend that understanding, he has realized quenching within himself. Having truly understood the origin, ending, gratification, drawback, and escape from feelings, the Realized One is freed through not grasping. 

These\marginnote{2.22.5} are the principles—deep, hard to see, hard to understand, peaceful, sublime, beyond the scope of logic, subtle, comprehensible to the astute—which the Realized One makes known after realizing them with his own insight. And those who genuinely praise the Realized One would rightly speak of these things. 

\subsubsection*{3.1.4. Endless Flip-floppers }

There\marginnote{2.23.1} are some ascetics and brahmins who are endless flip-floppers. Whenever they’re asked a question, they resort to verbal flip-flops and endless flip-flops on four grounds.\footnote{\textit{Vikkhepa} is “flip-flopping”. | \textit{\textsanskrit{Amarā}} is explained in the commentary as either “undying” or “eel-like”. However, \textit{\textsanskrit{amarā}} in the sense of “eel” is found only in the commentary to this term so is probably spurious. } And what are the four grounds on which they rely? 

It’s\marginnote{2.24.1} when some ascetic or brahmin doesn’t truly understand what is skillful and what is unskillful.\footnote{This is a basic requirement for any spiritual teacher. } They think: ‘I don’t truly understand what is skillful and what is unskillful. If I were to declare that something was skillful or unskillful I might be wrong. That would be stressful for me, and that stress would be an obstacle.’ So from fear and disgust with false speech they avoid stating whether something is skillful or unskillful. Whenever they’re asked a question, they resort to verbal flip-flops and endless flip-flops:\footnote{Despite their dullness, they have a genuine sense of conscience and wish to avoid breaking precepts. } ‘I don’t say it’s like this. I don’t say it’s like that. I don’t say it’s otherwise. I don’t say it’s not so. And I don’t deny it’s not so.’\footnote{A wise teacher avoids making pronouncements about what they do not understand, but these teachers use this as a cover to hide the fact that they do not understand anything. } This is the first ground on which some ascetics and brahmins rely when resorting to verbal flip-flops and endless flip-flops. 

And\marginnote{2.25.1} what is the second ground on which they rely? It’s when some ascetic or brahmin doesn’t truly understand what is skillful and what is unskillful. They think: ‘I don’t truly understand what is skillful and what is unskillful. If I were to declare that something was skillful or unskillful I might feel desire or greed or hate or repulsion.\footnote{Here too they show a certain sincerity to avoid giving rise to unwholesome qualities. } That would be grasping on my part. That would be stressful for me, and that stress would be an obstacle.’ So from fear and disgust with grasping they avoid stating whether something is skillful or unskillful. Whenever they’re asked a question, they resort to verbal flip-flops and endless flip-flops: ‘I don’t say it’s like this. I don’t say it’s like that. I don’t say it’s otherwise. I don’t say it’s not so. And I don’t deny it’s not so.’ This is the second ground on which some ascetics and brahmins rely when resorting to verbal flip-flops and endless flip-flops. 

And\marginnote{2.26.1} what is the third ground on which they rely? It’s when some ascetic or brahmin doesn’t truly understand what is skillful and what is unskillful. They think: ‘I don’t truly understand what is skillful and what is unskillful. Suppose I were to declare that something was skillful or unskillful. There are clever ascetics and brahmins who are subtle, accomplished in the doctrines of others, hair-splitters. You’d think they live to demolish convictions with their intellect.\footnote{They avoid making statements, not from a sense of conscience, but for fear of public shaming. } They might pursue, press, and grill me about that. I’d be stumped by such a grilling. That would be stressful for me, and that stress would be an obstacle.’ So from fear and disgust with examination they avoid stating whether something is skillful or unskillful. Whenever they’re asked a question, they resort to verbal flip-flops and endless flip-flops: ‘I don’t say it’s like this. I don’t say it’s like that. I don’t say it’s otherwise. I don’t say it’s not so. And I don’t deny it’s not so.’ This is the third ground on which some ascetics and brahmins rely when resorting to verbal flip-flops and endless flip-flops. 

And\marginnote{2.27.1} what is the fourth ground on which they rely? It’s when some ascetic or brahmin is dull and stupid.\footnote{Also at \href{https://suttacentral.net/mn76/en/sujato\#30.2}{MN 76:30.2}. } Because of that, whenever they’re asked a question, they resort to verbal flip-flops and endless flip-flops: ‘Suppose you were to ask me whether there is another world. If I believed that to be the case, I would say so.\footnote{Here begins a series of four tetrads that are commonly encountered in the suttas. The first is the belief in an afterlife. } But I don’t say it’s like this. I don’t say it’s like that. I don’t say it’s otherwise. I don’t say it’s not so. And I don’t deny it’s not so. Suppose you were to ask me whether there is no other world …\footnote{The denial of an afterlife. } whether there both is and is not another world …\footnote{This could include a belief that eternal life is offered only to adherents of a certain religion. } whether there neither is nor is not another world …\footnote{This could include the idea that our intrinsic nature is one with the cosmos, and our separation from that infinitude in this life is only a veil of delusion. Thus there is no other world, because all worlds are this world, but it is also not the case that there is nothing after death. } whether there are beings who are reborn spontaneously …\footnote{This is beings such as the gods or various ghosts and spirits, which are not born organically. } whether there are not beings who are reborn spontaneously …\footnote{This denies the existence of such beings. Not everyone in ancient India believed in the various orders of beings. } whether there both are and are not beings who are reborn spontaneously …\footnote{The belief that beings are both spontaneously born and organically born. Perhaps this denies that such a distinction can be made clearly, because both kinds of birth take place within the same order of beings. } whether there neither are nor are not beings who are reborn spontaneously …\footnote{Beings are reborn in other ways. } whether there is fruit and result of good and bad deeds …\footnote{This is the standard Buddhist view of kamma, shared with some, but not all, of the other Indian religions of the time. } whether there is no fruit and result of good and bad deeds …\footnote{Doing good or bad has no result; moral nihilism. } whether there both is and is not fruit and result of good and bad deeds …\footnote{Sometimes good and bad deeds have results, other times not. } whether there neither is nor is not fruit and result of good and bad deeds …\footnote{The results of actions are too subtle to be described as good or bad. } whether a realized one still exists after death …\footnote{An awakened one, whether the Buddha or anyone else, exists after death, for example in an eternal state of Nirvana. } whether A realized one no longer exists after death …\footnote{A sage ceases to exist at the time of death. From a Buddhist point of view, this is incoherent since it assumes the underlying attachment to a “self”, which the Realized One has done away with. } whether a realized one both still exists and no longer exists after death …\footnote{For example, their body does not exist but their mind does. } whether a Realized One neither exists nor doesn’t exist after death. If I believed that to be the case, I would say so.\footnote{A sage is in a subtle state that cannot be characterized in terms of existence or non-existence. } But I don’t say it’s like this. I don’t say it’s like that. I don’t say it’s otherwise. I don’t say it’s not so. And I don’t deny it’s not so.’ This is the fourth ground on which some ascetics and brahmins rely when resorting to verbal flip-flops and endless flip-flops. 

These\marginnote{2.28.1} are the four grounds on which those ascetics and brahmins who are flip-floppers resort to verbal flip-flops and endless flip-flops whenever they’re asked a question. Any ascetics and brahmins who resort to verbal flip-flops and endless flip-flops do so on one or other of these four grounds. Outside of this there is none. The Realized One understands this … And those who genuinely praise the Realized One would rightly speak of these things. 

\subsubsection*{3.1.5. Doctrines of Origination by Chance }

There\marginnote{2.30.1} are some ascetics and brahmins who theorize about chance. They assert that the self and the cosmos arose by chance on two grounds. And what are the two grounds on which they rely? 

There\marginnote{2.31.1} are gods named ‘non-percipient beings’.\footnote{This is an obscure realm of existence where the operations of consciousness are suspended. } When perception arises they pass away from that host of gods. It’s possible that one of those beings passes away from that host and is reborn in this place. Having done so, they go forth from the lay life to homelessness. By dint of keen, resolute, committed, and diligent effort, and right application of mind, they experience an immersion of the heart of such a kind that they recollect the arising of perception, but no further. They say: ‘The self and the cosmos arose by chance. Why is that? Because formerly I didn’t exist, whereas now, having not existed, I’ve transformed into the state of existing.” This is the first ground on which some ascetics and brahmins rely to assert that the self and the cosmos arose by chance. 

And\marginnote{2.32.1} what is the second ground on which they rely? It’s when some ascetic or brahmin relies on logic and inquiry. They speak of what they have worked out by logic, following a line of inquiry, expressing their own perspective: ‘The self and the cosmos arose by chance.’ This is the second ground on which some ascetics and brahmins rely to assert that the self and the cosmos arose by chance. 

These\marginnote{2.33.1} are the two grounds on which those ascetics and brahmins who theorize about chance assert that the self and the cosmos arose by chance. Any ascetics and brahmins who theorize about chance do so on one or other of these two grounds. Outside of this there is none. The Realized One understands this … And those who genuinely praise the Realized One would rightly speak of these things. 

These\marginnote{2.35.1} are the eighteen grounds on which those ascetics and brahmins who theorize about the past assert various hypotheses concerning the past.\footnote{First each section is concluded, then the whole first part is concluded. This formalism is a characteristic of oral tradition. It creates a nested hierarchy of content, clarifying the structure and helping to preserve the text in memory. } Any ascetics and brahmins who theorize about the past do so on one or other of these eighteen grounds. Outside of this there is none. 

The\marginnote{2.36.1} Realized One understands this: ‘If you hold on to and attach to these grounds for views it leads to such and such a destiny in the next life.’ He understands this, and what goes beyond this. And since he does not misapprehend that understanding, he has realized quenching within himself. Having truly understood the origin, ending, gratification, drawback, and escape from feelings, the Realized One is freed through not grasping. 

These\marginnote{2.36.5} are the principles—deep, hard to see, hard to understand, peaceful, sublime, beyond the scope of logic, subtle, comprehensible to the astute—which the Realized One makes known after realizing them with his own insight. And those who genuinely praise the Realized One would rightly speak of these things. 

\scendsection{The second recitation section. }

\subsection*{3.2. Theories About the Future }

There\marginnote{2.37.1} are some ascetics and brahmins who theorize about the future, and assert various hypotheses concerning the future on forty-four grounds.\footnote{This section introduces more tetralemmas. Many of the views describe the self in terms of the five aggregates—form, feeling, perception, choices, and consciousness. } And what are the forty-four grounds on which they rely? 

\subsubsection*{3.2.1. Percipient Life After Death }

There\marginnote{2.38.1} are some ascetics and brahmins who say there is life after death, and assert that the self lives on after death in a percipient form on sixteen grounds. And what are the sixteen grounds on which they rely? 

They\marginnote{2.38.3} assert: ‘The self is healthy and percipient after death, and it is formed …\footnote{Usually a self is conceived of as percipient, so that the subject experiences a continuity. | The term \textit{aroga} (“free of disease”) is explained by the commentary as “permanent” (\textit{nicca}), drawing on the root sense of the word, “unbroken”. However, \textit{aroga} is always used in the sense “free of disease, well, healthy” (eg. \href{https://suttacentral.net/mn97/en/sujato\#2.4}{MN 97:2.4}), and this applies to the Brahmanical tradition as well as the Buddhist. Chandogya \textsanskrit{Upaniṣad} 7.26.2 says that one who sees (the self) does not see death, they have no disease or pain. \textsanskrit{Bṛhadāraṇyaka} \textsanskrit{Upaniṣad} 4.4.12 similarly says that one who sees the self will not suffer in the wake of the body, which \textsanskrit{Śaṅkāra} explains, “Struggling with desires for himself, for his son, for his wife, and so on, he is born and dies again and again, and is diseased when his body is diseased.” } 

formless\marginnote{2.38.4} … 

both\marginnote{2.38.5} formed and formless … 

neither\marginnote{2.38.6} formed nor formless … 

finite\marginnote{2.38.7} … 

infinite\marginnote{2.38.8} … 

both\marginnote{2.38.9} finite and infinite … 

neither\marginnote{2.38.10} finite nor infinite … 

of\marginnote{2.38.11} unified perception … 

of\marginnote{2.38.12} diverse perception … 

of\marginnote{2.38.13} limited perception … 

of\marginnote{2.38.14} limitless perception … 

experiences\marginnote{2.38.15} nothing but happiness … 

experiences\marginnote{2.38.16} nothing but suffering … 

experiences\marginnote{2.38.17} both happiness and suffering … 

experiences\marginnote{2.38.18} neither happiness nor suffering.’ 

These\marginnote{2.39.1} are the sixteen grounds on which those ascetics and brahmins assert that the self lives on after death in a percipient form. Any ascetics and brahmins who assert that the self lives on after death in a percipient form do so on one or other of these sixteen grounds. Outside of this there is none. The Realized One understands this … And those who genuinely praise the Realized One would rightly speak of these things. 

\subsubsection*{3.2.2. Non-Percipient Life After Death }

There\marginnote{3.1.1} are some ascetics and brahmins who say there is life after death, and assert that the self lives on after death in a non-percipient form on eight grounds. And what are the eight grounds on which they rely? 

They\marginnote{3.2.1} assert: ‘The self is healthy and non-percipient after death, and it is formed …\footnote{Here the self has a physical dimension but no perception. This might include rebirth as a plant or inanimate object. } 

formless\marginnote{3.2.2} … 

both\marginnote{3.2.3} formed and formless … 

neither\marginnote{3.2.4} formed nor formless … 

finite\marginnote{3.2.5} … 

infinite\marginnote{3.2.6} … 

both\marginnote{3.2.7} finite and infinite … 

neither\marginnote{3.2.8} finite nor infinite.’ 

These\marginnote{3.3.1} are the eight grounds on which those ascetics and brahmins assert that the self lives on after death in a non-percipient form. Any ascetics and brahmins who assert that the self lives on after death in a non-percipient form do so on one or other of these eight grounds. Outside of this there is none. The Realized One understands this … And those who genuinely praise the Realized One would rightly speak of these things. 

\subsubsection*{3.2.3. Neither Percipient Nor Non-Percipient Life After Death }

There\marginnote{3.5.1} are some ascetics and brahmins who say there is life after death, and assert that the self lives on after death in a neither percipient nor non-percipient form on eight grounds.\footnote{Buddhism acknowledges a formless realm of neither perception nor non-perception, which is attained through advanced meditation. } And what are the eight grounds on which they rely? 

They\marginnote{3.6.1} assert: ‘The self is healthy and neither percipient nor non-percipient after death, and it is formed … 

formless\marginnote{3.6.2} … 

both\marginnote{3.6.3} formed and formless … 

neither\marginnote{3.6.4} formed nor formless … 

finite\marginnote{3.6.5} … 

infinite\marginnote{3.6.6} … 

both\marginnote{3.6.7} finite and infinite … 

neither\marginnote{3.6.8} finite nor infinite.’ 

These\marginnote{3.7.1} are the eight grounds on which those ascetics and brahmins assert that the self lives on after death in a neither percipient nor non-percipient form. Any ascetics and brahmins who assert that the self lives on after death in a neither percipient nor non-percipient form do so on one or other of these eight grounds. Outside of this there is none. The Realized One understands this … And those who genuinely praise the Realized One would rightly speak of these things. 

\subsubsection*{3.2.4. Annihilationism }

There\marginnote{3.9.1} are some ascetics and brahmins who are annihilationists. They assert the annihilation, eradication, and obliteration of an existing being on seven grounds.\footnote{These theorists assert the true existence of a being, thus falling into the fallacy of identity view. For the Buddha, the words “being” or a “self” describe an ongoing process that is conditioned and impermanent, and do not correspond to a genuine metaphysical reality. The distinction between contingent, empirical reality and metaphysical, absolute existence is essential to understanding early Buddhism. } And what are the seven grounds on which they rely? 

There\marginnote{3.10.1} are some ascetics and brahmins who have this doctrine and view: ‘This self is formed, made up of the four principal states, and produced by mother and father. Since it’s annihilated and destroyed when the body breaks up, and doesn’t exist after death, that’s how this self becomes rightly annihilated.’\footnote{This is the materialist view, which accepts only the coarse physical realm. This view is common today, but was also well known in the Buddha’s time. | The “four principal states” are earth, water, fire, and air, i.e. the states of matter: solid, liquid, plasma, and gas. } That is how some assert the annihilation of an existing being. 

But\marginnote{3.11.1} someone else says to them: ‘\emph{That} self of which you speak does exist, I don’t deny it.\footnote{The theorist accepts multiple selves. As self theories evolve, they typically move from more coarse materialist theories towards more subtle conceptions. Sometimes the former view is rejected as being false. Sometimes, as here, the former view is seen not as false, but as incomplete and shallow. } But that’s not how \emph{this} self becomes rightly annihilated. There is another self that is heavenly, formed, sensual, consuming solid food.\footnote{“Form” (\textit{\textsanskrit{rūpa}}) includes not just the physical realm of the elements, but various kinds of subtle form (\textit{\textsanskrit{sukhumarūpa}}). These include the energetic or mind-made bodies of beings in various dimensions. \textit{\textsanskrit{Rūpa}}  ultimately refers to the appearance or manifestation of physical properties, and can even include the perception of colors, lights, and shapes in the mind. Here the bodies of the divine beings are not very distant from our own, as they still consume solid food. This probably refers to various nature deities or entities that were believed to consume the food offered to them by humans. } You don’t know or see that. But I know it and see it. Since this self is annihilated and destroyed when the body breaks up, and doesn’t exist after death, that’s how this self becomes rightly annihilated.’\footnote{Whereas the eternalists believe that their heavenly rebirth will last forever, annihilationists believe that even heaven is limited, and it ends in final annihilation. } That is how some assert the annihilation of an existing being. 

But\marginnote{3.12.1} someone else says to them: ‘\emph{That} self of which you speak does exist, I don’t deny it. But that’s not how \emph{this} self becomes rightly annihilated. There is another self that is heavenly, formed, mind-made, whole in its major and minor limbs, not deficient in any faculty.\footnote{This is a more subtle kind of divine rebirth, no longer dependent on physical food. The subtle body still takes on a humanoid form, however, appearing complete in all its limbs. It includes realms produced through the practice of the four \textit{\textsanskrit{jhānas}}. | \textit{\textsanskrit{Sabbaṅga}} has the sense “whole and healthy of limb” (Rig Veda 10.161.5c, Atharva Veda 8.2.8c, 11.3.32 ff.). One is reborn with “whole body” (\textit{sarvatanu}, Atharva Veda 5.6.11c, Śatapatha \textsanskrit{Brāhmaṇa} 11.1.8.60, 12.8.3.31). | \textit{\textsanskrit{Paccaṅga}} means “minor limb”, for example the fingers or internal organs. } You don’t know or see that. But I know it and see it. Since this self is annihilated and destroyed when the body breaks up, and doesn’t exist after death, that’s how this self becomes rightly annihilated.’ That is how some assert the annihilation of an existing being. 

But\marginnote{3.13.1} someone else says to them: ‘\emph{That} self of which you speak does exist, I don’t deny it. But that’s not how \emph{this} self becomes rightly annihilated. There is another self which has gone totally beyond perceptions of form. With the ending of perceptions of impingement, not focusing on perceptions of diversity, aware that “space is infinite”, it’s reborn in the dimension of infinite space.\footnote{Such a rebirth has left even the subtle body behind, becoming sheer consciousness. | The word \textit{\textsanskrit{āyatana}} is from a root meaning “expanse”. It is prominently used in this context, where it refers to a realm or “dimension” of rebirth, and in the analysis of sense experience, where it refers to a “field” of sense experience. } You don’t know or see that. But I know it and see it. Since this self is annihilated and destroyed when the body breaks up, and doesn’t exist after death, that’s how this self becomes rightly annihilated.’\footnote{This phrase appears incongruous as formless beings do not have a body. However the Chinese parallel at DA 21 does not mention \textit{\textsanskrit{kāya}} here, so it is likely to have arisen as an error in transmission where an earlier phrase was mistakenly copied. } That is how some assert the annihilation of an existing being. 

But\marginnote{3.14.1} someone else says to them: ‘\emph{That} self of which you speak does exist, I don’t deny it. But that’s not how \emph{this} self becomes rightly annihilated. There is another self which has gone totally beyond the dimension of infinite space. Aware that “consciousness is infinite”, it’s reborn in the dimension of infinite consciousness. You don’t know or see that. But I know it and see it. Since this self is annihilated and destroyed when the body breaks up, and doesn’t exist after death, that’s how this self becomes rightly annihilated.’ That is how some assert the annihilation of an existing being. 

But\marginnote{3.15.1} someone else says to them: ‘\emph{That} self of which you speak does exist, I don’t deny it. But that’s not how \emph{this} self becomes rightly annihilated. There is another self that has gone totally beyond the dimension of infinite consciousness. Aware that “there is nothing at all”, it’s been reborn in the dimension of nothingness. You don’t know or see that. But I know it and see it. Since this self is annihilated and destroyed when the body breaks up, and doesn’t exist after death, that’s how this self becomes rightly annihilated.’ That is how some assert the annihilation of an existing being. 

But\marginnote{3.16.1} someone else says to them: ‘\emph{That} self of which you speak does exist, I don’t deny it. But that’s not how \emph{this} self becomes rightly annihilated. There is another self that has gone totally beyond the dimension of nothingness. Aware that “this is peaceful, this is sublime”, it’s been reborn in the dimension of neither perception nor non-perception.\footnote{The extension of the normal description of this state with the phrase “this is peaceful, this is sublime” is found only here and at \href{https://suttacentral.net/an10.99/en/sujato\#35.1}{AN 10.99:35.1}. } You don’t know or see that. But I know it and see it. Since this self is annihilated and destroyed when the body breaks up, and doesn’t exist after death, that’s how this self becomes rightly annihilated.’ That is how some assert the annihilation of an existing being. 

These\marginnote{3.17.1} are the seven grounds on which those ascetics and brahmins assert the annihilation, eradication, and obliteration of an existing being. Any ascetics and brahmins who assert the annihilation, eradication, and obliteration of an existing being do so on one or other of these seven grounds. Outside of this there is none. The Realized One understands this … And those who genuinely praise the Realized One would rightly speak of these things. 

\subsubsection*{3.2.5. Extinguishment of Suffering in This Life }

There\marginnote{3.19.1} are some ascetics and brahmins who speak of extinguishment in this life. They assert ultimate extinguishment for an existing being in this life on five grounds.\footnote{These five theories argue for the extinguishment of suffering through the experience of pleasure in the present life. The Buddha taught extinguishment (\textit{\textsanskrit{nibbāna}}) in this very life, but not “of an existing being” (\textit{sato sattassa}) or “self” (\textit{\textsanskrit{attā}}). The Buddha denied that there is such a thing, pointing out that we are a stream of ever-changing conditions, fueled by desire and attachment, and liable to suffering. With the end of craving there is no fuel to sustain the stream, so suffering comes to an end. | It is unclear why these views of the “present life” (\textit{\textsanskrit{diṭṭhadhamma}}) are classified under views of the future; see \href{https://suttacentral.net/mn102/en/sujato\#2.8}{MN 102:2.8}. } And what are the five grounds on which they rely? 

There\marginnote{3.20.1} are some ascetics and brahmins who have this doctrine and view: ‘When this self amuses itself, supplied and provided with the five kinds of sensual stimulation, that’s how this self attains ultimate extinguishment in this life.’\footnote{The hedonist. } That is how some assert ultimate extinguishment for an existing being in this life. 

But\marginnote{3.21.1} someone else says to them: ‘\emph{That} self of which you speak does exist, I don’t deny it. But that’s not how \emph{this} self attains ultimate extinguishment in this life. Why is that? Because sensual pleasures are impermanent, suffering, and perishable. Their decay and perishing give rise to sorrow, lamentation, pain, sadness, and distress.\footnote{Here we see the philosophical reasoning that prompts the evolution of more refined conceptions of self. } Quite secluded from sensual pleasures, secluded from unskillful qualities, this self enters and remains in the first absorption, which has the rapture and bliss born of seclusion, while placing the mind and keeping it connected. That’s how this self attains ultimate extinguishment in this life.’\footnote{The “absorptions” (\textit{\textsanskrit{jhāna}}) are central to Buddhist meditation. The Buddha did say that they can be considered “extinguishment in the present life” in a qualified sense (\href{https://suttacentral.net/an9.51/en/sujato}{AN 9.51}). As we have seen above, however, if they are undertaken with wrong view, the experience itself will tend to reinforce the attachment to a self. } That is how some assert ultimate extinguishment for an existing being in this life. 

But\marginnote{3.22.1} someone else says to them: ‘\emph{That} self of which you speak does exist, I don’t deny it. But that’s not how \emph{this} self attains ultimate extinguishment in this life.\footnote{The theorist has an experience of a deeper state of meditation, so they know that the first \textit{\textsanskrit{jhāna}} cannot be the ultimate. } Why is that? Because the placing of the mind and the keeping it connected there are coarse.\footnote{\textit{\textsanskrit{Jhānas}} are subtle states of refined consciousness in which nothing is coarse when compared to ordinary consciousness. Within each state, however, certain mental factors are coarse relative to others. A meditator proceeds through the \textit{\textsanskrit{jhānas}} with the successive stilling of the relatively coarser factors in each state. } But when the placing of the mind and keeping it connected are stilled, this self enters and remains in the second absorption, which has the rapture and bliss born of immersion, with internal clarity and mind at one, without placing the mind and keeping it connected. That’s how this self attains ultimate extinguishment in this life.’ That is how some assert ultimate extinguishment for an existing being in this life. 

But\marginnote{3.23.1} someone else says to them: ‘\emph{That} self of which you speak does exist, I don’t deny it. But that’s not how \emph{this} self attains ultimate extinguishment in this life. Why is that? Because the rapture and emotional elation there are coarse. But with the fading away of rapture, this self enters and remains in the third absorption, where it meditates with equanimity, mindful and aware, personally experiencing the bliss of which the noble ones declare, “Equanimous and mindful, one meditates in bliss”. That’s how this self attains ultimate extinguishment in this life.’ That is how some assert ultimate extinguishment for an existing being in this life. 

But\marginnote{3.24.1} someone else says to them: ‘\emph{That} self of which you speak does exist, I don’t deny it. But that’s not how \emph{this} self attains ultimate extinguishment in this life. Why is that? Because the mental partaking of that as ‘blissful’ is said to be coarse.\footnote{\textit{Ābhoga} (“partaking”) is unique in the early texts in this sense. The commentary says that after emerging from \textit{\textsanskrit{jhāna}}, one repeatedly partakes and attends to that bliss (for \textit{\textsanskrit{ābhoga}} with \textit{\textsanskrit{manasikāra}}, see \href{https://suttacentral.net/mil5.1.1/en/sujato\#8.3}{Mil 5.1.1:8.3}). Compare \textsanskrit{Patañjalī}’s commentary on \textsanskrit{Yogasūtra} 1.17: “\textit{vitarka} is the gross (\textit{sthula}) partaking in the mental object, \textit{\textsanskrit{vicāra}} is subtle”. } But giving up pleasure and pain, and ending former happiness and sadness, this self enters and remains in the fourth absorption, without pleasure or pain, with pure equanimity and mindfulness. That’s how this self attains ultimate extinguishment in this life.’ That is how some assert ultimate extinguishment for an existing being in this life. 

These\marginnote{3.25.1} are the five grounds on which those ascetics and brahmins assert ultimate extinguishment for an existing being in this life. Any ascetics and brahmins who assert ultimate extinguishment for an existing being in this life do so on one or other of these five grounds. Outside of this there is none.\footnote{It is not clear why the still more subtle states of the formless attainments are not included here. } The Realized One understands this … And those who genuinely praise the Realized One would rightly speak of these things. 

These\marginnote{3.27.1} are the forty-four grounds on which those ascetics and brahmins who theorize about the future assert various hypotheses concerning the future. Any ascetics and brahmins who theorize about the future do so on one or other of these forty-four grounds. Outside of this there is none. The Realized One understands this … And those who genuinely praise the Realized One would rightly speak of these things. 

These\marginnote{3.29.1} are the sixty-two grounds on which those ascetics and brahmins who theorize about the past and the future assert various hypotheses concerning the past and the future. 

Any\marginnote{3.29.2} ascetics and brahmins who theorize about the past or the future do so on one or other of these sixty-two grounds. Outside of this there is none. 

The\marginnote{3.30.1} Realized One understands this: ‘If you hold on to and attach to these grounds for views it leads to such and such a destiny in the next life.’\footnote{Even the views of annihilation or extinguishment lead to rebirth, contrary to the beliefs of those who hold them. } He understands this, and what goes beyond this. And since he does not misapprehend that understanding, he has realized quenching within himself. Having truly understood the origin, ending, gratification, drawback, and escape from feelings, the Realized One is freed through not grasping. 

These\marginnote{3.31.1} are the principles—deep, hard to see, hard to understand, peaceful, sublime, beyond the scope of logic, subtle, comprehensible to the astute—which the Realized One makes known after realizing them with his own insight. And those who genuinely praise the Realized One would rightly speak of these things. 

\section*{4. The Grounds For Assertions About the Self and the Cosmos }

\subsection*{4.1. Anxiety and Evasiveness }

Now,\marginnote{3.32.1} these things are only the feeling of those who do not know or see, the anxiety and evasiveness of those under the sway of craving. Namely, when those ascetics and brahmins assert that the self and the cosmos are eternal on four grounds …\footnote{Here the Buddha brings to the fore the notion of feelings which has been briefly mentioned throughout the text. Views are not objective descriptions of the world, but responses to our innermost needs. The word \textit{paritassita} (“anxiety”) conveys both fear and desire, while \textit{vipphandita} (“evasiveness”) captures how attachment to theorizing serves as an avoidance strategy. } 

partially\marginnote{3.33.1} eternal on four grounds … 

finite\marginnote{3.34.1} or infinite on four grounds … 

or\marginnote{3.35.1} they resort to flip-flops on four grounds … 

or\marginnote{3.36.1} they assert that the self and the cosmos arose by chance on two grounds … 

they\marginnote{3.37.1} theorize about the past on these eighteen grounds … 

or\marginnote{3.38.1} they assert that the self lives on after death in a percipient form on sixteen grounds … 

or\marginnote{3.39.1} that the self lives on after death in a non-percipient form on eight grounds … 

or\marginnote{3.40.1} that the self lives on after death in a neither percipient nor non-percipient form on eight grounds … 

or\marginnote{3.41.1} they assert the annihilation of an existing being on seven grounds … 

or\marginnote{3.42.1} they assert ultimate extinguishment for an existing being in this life on five grounds … 

they\marginnote{3.43.1} theorize about the future on these forty-four grounds … 

When\marginnote{3.44.1} those ascetics and brahmins theorize about the past and the future on these sixty-two grounds, these things are only the feeling of those who do not know or see, the anxiety and evasiveness of those under the sway of craving. 

\subsection*{4.2. Conditioned by Contact }

Now,\marginnote{3.45.1} these things are conditioned by contact. Namely, when those ascetics and brahmins assert that the self and the cosmos are eternal on four grounds …\footnote{The analysis is introducing more elements of dependent origination. The famous twelve links say that contact is the condition for feeling, which in turn causes craving. } 

partially\marginnote{3.46.1} eternal on four grounds … 

finite\marginnote{3.47.1} or infinite on four grounds … 

or\marginnote{3.48.1} they resort to flip-flops on four grounds … 

or\marginnote{3.49.1} they assert that the self and the cosmos arose by chance on two grounds … 

they\marginnote{3.50.1} theorize about the past on these eighteen grounds … 

or\marginnote{3.51.1} they assert that the self lives on after death in a percipient form on sixteen grounds … 

or\marginnote{3.52.1} that the self lives on after death in a non-percipient form on eight grounds … 

or\marginnote{3.53.1} that the self lives on after death in a neither percipient nor non-percipient form on eight grounds … 

or\marginnote{3.54.1} they assert the annihilation of an existing being on seven grounds … 

or\marginnote{3.55.1} they assert ultimate extinguishment for an existing being in this life on five grounds … 

they\marginnote{3.56.1} theorize about the future on these forty-four grounds … 

When\marginnote{3.57.1} those ascetics and brahmins theorize about the past and the future on these sixty-two grounds, that too is conditioned by contact. 

\subsection*{4.3. Not Possible }

Now,\marginnote{3.70.1} when those ascetics and brahmins theorize about the past and the future on these sixty-two grounds, it is not possible that they should experience these things without contact.\footnote{The text repeats all, but I abbreviate for legibility. In the oral tradition, extensive repetitions serve to reinforce the learning and ensure reliability of transmission. More subtly, they also help deepen understanding and contemplation. After reciting the extensive and complex treatment of the sixty-two views, the reciter takes the time to go over them again and again, letting them settle and consolidate. True learning takes time, but the repetitions that are reflective and reassuring in recitation become irksome and ponderous in reading. } 

\subsection*{4.4. Dependent Origination }

Now,\marginnote{3.71.1} when those ascetics and brahmins theorize about the past and the future on these sixty-two grounds, all of them experience this by repeated contact through the six fields of contact. Their feeling is a condition for craving. Craving is a condition for grasping. Grasping is a condition for continued existence. Continued existence is a condition for rebirth. Rebirth is a condition for old age and death, sorrow, lamentation, pain, sadness, and distress to come to be.\footnote{Finally the process of dependent origination, which has been foreshadowed little by little, is brought to its ultimate conclusion. It still does not include the full twelve factors, but the process beginning with ignorance is implied throughout. } 

\section*{5. The End of the Round }

When\marginnote{3.72.1} a mendicant truly understands the six fields of contact’s origin, ending, gratification, drawback, and escape, they understand what lies beyond all these things.\footnote{“Contact” (\textit{phassa}) is the conjunction of the sense stimulus, the sense organ, and the associated consciousness. } 

All\marginnote{3.72.2} of these ascetics and brahmins who theorize about the past or the future are trapped in the net of these sixty-two grounds, so that wherever they emerge they are caught and trapped in this very net.\footnote{They cannot see a way past attachment so long as they theorize in terms of an existing self. } 

Suppose\marginnote{3.72.3} a deft fisherman or his apprentice were to cast a fine-meshed net over a small pond. They’d think: ‘Any sizable creatures in this pond will be trapped in the net. Wherever they emerge they are caught and trapped in this very net.’\footnote{The title is explained with a vivid simile. The Buddha was a master of observation, and constantly drew from everyday experience to illustrate his teachings. The metaphor works on a surface level to illustrate how theorists are trapped. But it also conveys something deeper, a sense of pathos and empathy with the helpless creatures who have no idea why they suffer. } In the same way, all of these ascetics and brahmins who theorize about the past or the future are trapped in the net of these sixty-two grounds, so that wherever they emerge they are caught and trapped in this very net. 

The\marginnote{3.73.1} Realized One’s body remains, but his conduit to rebirth has been cut off.\footnote{The Buddha is not his body, which is merely the remnant of past kamma. The phrase \textit{bhavanetti} (“conduit to rebirth”) invokes a channel that leads to a future life. } As long as his body remains he will be seen by gods and humans. But when his body breaks up, after life has ended, gods and humans will see him no more. 

When\marginnote{3.73.4} the stalk of a bunch of mangoes is cut, all the mangoes attached to the stalk will follow along.\footnote{This metaphor is found at \textsanskrit{Bṛhadāraṇyaka} \textsanskrit{Upaniṣad} 4.3.36, where it refers to the separation of the self from the body at death. } In the same way, the Realized One’s body remains, but his conduit to rebirth has been cut off. As long as his body remains he will be seen by gods and humans. But when his body breaks up, after life has ended, gods and humans will see him no more.” 

When\marginnote{3.74.1} he had spoken, Venerable Ānanda said to the Buddha, “It’s incredible, sir, it’s amazing! What is the name of this exposition of the teaching?” 

“Well\marginnote{3.74.3} then, Ānanda, you may remember this exposition of the teaching as ‘The Net of Meaning’, or else ‘The Net of the Teaching’, or else ‘The Divine Net’, or else ‘The Net of Views’, or else ‘The Supreme Victory in Battle’.”\footnote{It is not uncommon to find multiple names for the same sutta, and here we see this practice originated with the Buddha himself. When is referred to by name at \href{https://suttacentral.net/sn41.3/en/sujato\#2.4}{SN 41.3:2.4} and \href{https://suttacentral.net/pli-tv-kd1/en/sujato\#1.8.8}{Kd 1:1.8.8}, however, it is always called the \textsanskrit{Brahmajāla}. } 

That\marginnote{3.74.4} is what the Buddha said. Satisfied, the mendicants approved what the Buddha said. And while this discourse was being spoken, the ten-thousandfold galaxy shook. 

%
\chapter*{{\suttatitleacronym DN 2}{\suttatitletranslation The Fruits of the Ascetic Life }{\suttatitleroot Sāmaññaphalasutta}}
\addcontentsline{toc}{chapter}{\tocacronym{DN 2} \toctranslation{The Fruits of the Ascetic Life } \tocroot{Sāmaññaphalasutta}}
\markboth{The Fruits of the Ascetic Life }{Sāmaññaphalasutta}
\extramarks{DN 2}{DN 2}

\section*{1. A Discussion With the King’s Ministers }

\scevam{So\marginnote{1.1} I have heard.\footnote{This sutta with its commentary was translated by Bhikkhu Bodhi in his \emph{The Fruits of Recluseship}. } }At one time the Buddha was staying near \textsanskrit{Rājagaha} in the Mango Grove of \textsanskrit{Jīvaka} \textsanskrit{Komārabhacca}, together with a large \textsanskrit{Saṅgha} of 1,250 mendicants.\footnote{This monastery belonged to the Buddha’s doctor, \textsanskrit{Jīvaka}, who appears later in the sutta. His story is told in \href{https://suttacentral.net/pli-tv-kd8/en/sujato}{Kd 8}, where we learn that he was raised by Prince Abhaya of Magadha, a Jain (\href{https://suttacentral.net/mn58/en/sujato}{MN 58}) who was also interested in the teachings of  \textsanskrit{Pūraṇa} Kassapa (\href{https://suttacentral.net/sn46.56/en/sujato}{SN 46.56}). \href{https://suttacentral.net/mn55/en/sujato}{MN 55} on eating meat is addressed to him. } 

Now,\marginnote{1.3} at that time it was the sabbath—the \textsanskrit{Komudī} full moon on the fifteenth day of the fourth month—and King \textsanskrit{Ajātasattu} of Magadha, son of the princess of Videha, was sitting upstairs in the royal longhouse surrounded by his ministers.\footnote{The \textsanskrit{Komudī} was an especially celebrated full moon on the last month of the rainy season (\textit{\textsanskrit{kattikā}}, October/November), when the skies were clear, the lotuses (\textit{\textsanskrit{kumudā}}) were in bloom, and the moon was in conjunction with the Pleiades, which gave the month its name. } 

Then\marginnote{1.4} \textsanskrit{Ajātasattu} of Magadha, son of the princess of Videha, expressed this heartfelt sentiment,\footnote{\textit{\textsanskrit{Ajātasattu}} (“one against whom no foe is born”) was the son of \textsanskrit{Bimbisāra} and heir to the \textsanskrit{Haryaṅka} dynasty of Magadha. In inscription and Jain tradition he is also called \textsanskrit{Kūṇika}. Jain tradition holds that his mother was \textsanskrit{Cellaṇā} daughter of \textsanskrit{Ceṭaka}, a \textsanskrit{Licchavī} ruler from \textsanskrit{Vesālī}. This begs the question as to why \textsanskrit{Cellaṇā} was said to be “of Videha”; perhaps he married a Videhan princess to forge an alliance with the \textsanskrit{Licchavī}’s northern neighbors. In any case, this is more plausible than the Buddhist tradition that his mother was Kosalan. } “Oh, sirs, this moonlit night is so very delightful, so beautiful, so glorious, so lovely, so striking.\footnote{\textit{\textsanskrit{Pāsādikā}} here does not mean “tranquil”; it is part of a stock list of terms meaning “beautiful, attractive” (eg. \href{https://suttacentral.net/dn4/en/sujato\#13.7}{DN 4:13.7}). | \textit{\textsanskrit{Lakkhaññā}} is unique in early Pali. It is probably a synonym in the sense of “possessing remarkable features, striking”, rather than “auspicious”. } Now, what ascetic or brahmin might I pay homage to today, paying homage to whom my mind might find peace?”\footnote{The king seeks redemption through his own actions; it is not that the ascetic has any special power to bring peace to his mind. | The sutta is deliberately holding back the reason for the king’s distress. Compare \href{https://suttacentral.net/an5.50/en/sujato}{AN 5.50}, where the reason for King \textsanskrit{Muṇḍa} seeking solace is stated up front. } 

When\marginnote{2.1} he had spoken, one of the king’s ministers said to him,\footnote{Though the king mentioned “ascetics and brahmins”, his advisers only recommend famous teachers of the ascetics (\textit{\textsanskrit{samaṇa}}). For a shorter survey of their doctrines, see \href{https://suttacentral.net/sn2.30/en/sujato}{SN 2.30}. } “Sire, \textsanskrit{Pūraṇa} Kassapa leads an order and a community, and tutors a community. He’s a well-known and famous religious founder, deemed holy by many people. He is of long standing, long gone forth; he is advanced in years and has reached the final stage of life.\footnote{A little-known teacher of the inefficacy of action and consequence, \textsanskrit{Pūraṇa} was poorly regarded even by his own students (\href{https://suttacentral.net/mn77/en/sujato\#6.19}{MN 77:6.19}). He is said to have advocated a doctrine of six classes of rebirth (\href{https://suttacentral.net/an6.57/en/sujato}{AN 6.57}; the same text reverentially mentions Makkhali \textsanskrit{Gosāla}). | \textit{\textsanskrit{Pūraṇa}} means “filling” (not \textit{\textsanskrit{purāṇa}}, “ancient”). } Let Your Majesty pay homage to him. Hopefully in so doing your mind will find peace.” But when he had spoken, the king kept silent.\footnote{The reason for the king’s silence is revealed later. The narrative is full of foreshadowing. } 

Another\marginnote{3.1} of the king’s ministers said to him, “Sire, the bamboo-staffed ascetic \textsanskrit{Gosāla} leads an order and a community, and tutors a community. He’s a well-known and famous religious founder, deemed holy by many people. He is of long standing, long gone forth; he is advanced in years and has reached the final stage of life.\footnote{Founder of the \textsanskrit{Ājīvakas}, who became the third largest ascetic movement after Buddhism and Jainism. None of their texts survive, but their teachings can be partially reconstructed from Buddhist and Jain sources. He practiced with \textsanskrit{Mahāvīra} for six years before an acrimonious split, following which he developed his doctrine of hard determinism. \textit{Makkhali} denotes a kind of ascetic bearing a bamboo staff, so (like the similar appellations \textit{\textsanskrit{nigaṇṭha}} and \textit{\textsanskrit{samaṇa}}) it indicates his affiliation. This being so, and noting that Buddhist Sanskrit texts spell his name as \textit{\textsanskrit{gośālīputra}} etc., the second name (like \textit{\textsanskrit{nātaputta}} and \textit{gotama}) might indicate his clan. However, I can find no trace of such a clan, and both Jain and Buddhist tradition, albeit unreliably, say the name arose because he was born in a cowshed. } Let Your Majesty pay homage to him. Hopefully in so doing your mind will find peace.” But when he had spoken, the king kept silent. 

Another\marginnote{4.1} of the king’s ministers said to him, “Sire, Ajita of the hair blanket leads an order and a community, and tutors a community. He’s a well-known and famous religious founder, deemed holy by many people. He is of long standing, long gone forth; he is advanced in years and has reached the final stage of life.\footnote{A materialist, he was an early proponent of the ideas later known as \textsanskrit{Cārvāka}. \textit{Kesakambala} means “hair-blanket”, which was worn as an ascetic practice (\href{https://suttacentral.net/an3.137/en/sujato}{AN 3.137}). } Let Your Majesty pay homage to him. Hopefully in so doing your mind will find peace.” But when he had spoken, the king kept silent. 

Another\marginnote{5.1} of the king’s ministers said to him, “Sire, Pakudha \textsanskrit{Kaccāyana} leads an order and a community, and tutors a community. He’s a well-known and famous religious founder, deemed holy by many people. He is of long standing, long gone forth; he is advanced in years and has reached the final stage of life.\footnote{Another obscure teacher, he taught a reductive atomism which negated the possibility of action with consequences. His first name is sometimes spelled Kakudha; both words signify a hump or crest. } Let Your Majesty pay homage to him. Hopefully in so doing your mind will find peace.” But when he had spoken, the king kept silent. 

Another\marginnote{6.1} of the king’s ministers said to him, “Sire, \textsanskrit{Sañjaya} \textsanskrit{Belaṭṭhiputta} leads an order and a community, and tutors a community. He’s a well-known and famous religious founder, deemed holy by many people. He is of long standing, long gone forth; he is advanced in years and has reached the final stage of life.\footnote{An agnostic, he is evidently the “wanderer \textsanskrit{Sañjaya}” who was the first teacher of \textsanskrit{Sāriputta} and \textsanskrit{Moggallāna} before they left him to follow the Buddha (\href{https://suttacentral.net/pli-tv-kd1/en/sujato\#23.1.1}{Kd 1:23.1.1}). His name is obscure. Sanskrit spells it \textit{\textsanskrit{vairaṭṭīputra}}, with several variations, but always with \textit{\textsanskrit{ṭi}}. The commentary says he was the “son of \textsanskrit{Belaṭṭha}”; a \textsanskrit{Belaṭṭha} \textsanskrit{Kaccāna} is found selling sugar at \href{https://suttacentral.net/pli-tv-kd1/en/sujato\#26.1.2}{Kd 1:26.1.2}, supporting the idea that \textsanskrit{Belaṭṭha} was a personal rather than clan name. } Let Your Majesty pay homage to him. Hopefully in so doing your mind will find peace.” But when he had spoken, the king kept silent. 

Another\marginnote{7.1} of the king’s ministers said to him, “Sire, the Jain ascetic of the \textsanskrit{Ñātika} clan leads an order and a community, and tutors a community. He’s a well-known and famous religious founder, deemed holy by many people. He is of long standing, long gone forth; he is advanced in years and has reached the final stage of life.\footnote{The Jain leader \textsanskrit{Mahāvīra} \textsanskrit{Vardhamāna} is known as \textsanskrit{Nigaṇṭha} \textsanskrit{Nātaputta} in Pali texts. He is regarded as the 24th supreme leader of the Jains, although only he and his predecessor \textsanskrit{Pārśvanātha} (not mentioned in the Pali) are historical. \textit{\textsanskrit{Nigaṇṭha}} means “knotless” (i.e. without attachments). As a term for Jain ascetics it is also found in Jain literature. \textsanskrit{Nātaputta} indicates his clan the \textsanskrit{Ñātikas} (Sanskrit \textit{\textsanskrit{jñātiputra}}; \textsanskrit{Prākrit} \textit{\textsanskrit{nāyaputta}}). The Pali tradition has confused \textit{\textsanskrit{ñāti}}  (“family”) with \textit{\textsanskrit{nāṭa}} (“dancer”). Thus \textsanskrit{Nigaṇṭha} \textsanskrit{Nātaputta} means “the Jain monk of the \textsanskrit{Ñātika} clan”. | Jainism and Buddhism are the only ancient \textit{\textsanskrit{samaṇa}} movements to survive to the present day. The primary Jain teaching is the practice of non-violence while burning off past kamma by fervent austerities in order to reach omniscient liberation. } Let Your Majesty pay homage to him. Hopefully in so doing your mind will find peace.” But when he had spoken, the king kept silent. 

\section*{2. A Discussion With \textsanskrit{Jīvaka} \textsanskrit{Komārabhacca} }

Now\marginnote{8.1} at that time \textsanskrit{Jīvaka} \textsanskrit{Komārabhacca} was sitting silently not far from the king.\footnote{His absence of speech signifies his wisdom. The narrative creates a dramatic expectation through his stillness, an exquisitely Buddhist aesthetic choice. } Then the king said to him, “But my dear \textsanskrit{Jīvaka}, why are you silent?” 

“Sire,\marginnote{8.4} the Blessed One, the perfected one, the fully awakened Buddha is staying in my mango grove together with a large \textsanskrit{Saṅgha} of 1,250 mendicants.\footnote{According to the Vinaya, a monastery is normally offered to the “\textsanskrit{Saṅgha} of the four quarters” and becomes their inalienable property. In the suttas this is not so clear, and it seems that \textsanskrit{Jīvaka} still regarded the property as his. In practice there would have been a variety of arrangements, as there are today. } He has this good reputation: ‘That Blessed One is perfected, a fully awakened Buddha, accomplished in knowledge and conduct, holy, knower of the world, supreme guide for those who wish to train, teacher of gods and humans, awakened, blessed.’\footnote{The first appearance of the famous \textit{iti pi so} formula. It is still recited in praise of the Buddha in Theravada communities. } Let Your Majesty pay homage to him. Hopefully in so doing your mind will find peace.” 

“Well\marginnote{8.9} then, my dear \textsanskrit{Jīvaka}, have the elephants readied.” 

“Yes,\marginnote{9.1} Your Majesty,” replied \textsanskrit{Jīvaka}. He had around five hundred female elephants readied, in addition to the king’s bull elephant for riding. Then he informed the king, “The elephants are ready, sire. Please go at your convenience.” 

Then\marginnote{10.1} King \textsanskrit{Ajātasattu} had women mounted on each of the five hundred female elephants, while he mounted his bull elephant. With attendants carrying torches, he set out in full royal pomp from \textsanskrit{Rājagaha} to \textsanskrit{Jīvaka}’s mango grove.\footnote{Indian kings were guarded by armed women inside the harem (\textsanskrit{Kauṭilya}’s \textsanskrit{Arthaśāstra} 1.21.1) and also while on hunt (Megasthenes’s Indica, via Strabo XV. i. 53–56). This passage may be the earliest evidence for this long-lasting practice. } 

But\marginnote{10.2} as he drew near the mango grove, the king became frightened, scared, his hair standing on end. He said to \textsanskrit{Jīvaka}, “My dear \textsanskrit{Jīvaka}, I hope you’re not deceiving me! I hope you’re not betraying me! I hope you’re not turning me over to my enemies! For how on earth can there be no sound of coughing or clearing throats or any noise in such a large \textsanskrit{Saṅgha} of 1,250 mendicants?”\footnote{The silence of the Buddha’s assembly is often contrasted with the rowdy gatherings of other ascetics, for example that of \textsanskrit{Poṭṭhapāda} at \href{https://suttacentral.net/dn9/en/sujato\#3.1}{DN 9:3.1}. } 

“Do\marginnote{10.8} not fear, great king, do not fear! I am not deceiving you, or betraying you, or turning you over to your enemies. Go forward, great king, go forward! Those are lamps shining in the pavilion.”\footnote{This is a double pun. \textit{\textsanskrit{Dīpa}} means “lamp” or “island, refuge”, while \textit{\textsanskrit{jhāyati}} means “burning” or “meditating”. So it could be rendered, “those are saviors meditating in the pavilion”. \textit{\textsanskrit{Jhāyati}} is the verb form of \textit{\textsanskrit{jhāna}} (“absorption”), which is the central practice of meditation described below. } 

\section*{3. The Question About the Fruits of the Ascetic Life }

Then\marginnote{11.1} King \textsanskrit{Ajātasattu} rode on the elephant as far as the terrain allowed, then descended and approached the pavilion door on foot, where he asked \textsanskrit{Jīvaka}, “But my dear \textsanskrit{Jīvaka}, where is the Buddha?”\footnote{The Buddha looked like any other monk. But this also reveals \textsanskrit{Ajātasattu}’s spiritual blindness. } 

“That\marginnote{11.3} is the Buddha, great king, that is the Buddha! He’s sitting against the central column facing east, in front of the \textsanskrit{Saṅgha} of mendicants.” 

Then\marginnote{12.1} the king went up to the Buddha and stood to one side.\footnote{He has not yet gained faith, so does not bow. } He looked around the \textsanskrit{Saṅgha} of mendicants, who were so very silent, like a still, clear lake, and expressed this heartfelt sentiment, “May my son, Prince \textsanskrit{Udāyibhadda}, be blessed with such peace as the \textsanskrit{Saṅgha} of mendicants now enjoys!” 

“Has\marginnote{12.4} your mind gone to one you love, great king?”\footnote{The Buddha, though fully aware of \textsanskrit{Ajātasattu}’s crimes, responds to him with compassion. } 

“I\marginnote{12.5} love my son, sir, Prince \textsanskrit{Udāyibhadda}. May he be blessed with such peace as the \textsanskrit{Saṅgha} of mendicants now enjoys!” 

Then\marginnote{13.1} the king bowed to the Buddha, raised his joined palms toward the \textsanskrit{Saṅgha}, and sat down to one side. He said to the Buddha, “Sir, I’d like to ask you about a certain point, if you’d take the time to answer.” 

“Ask\marginnote{13.5} what you wish, great king.” 

“Sir,\marginnote{14.1} there are many different professional fields.\footnote{Most translators render \textit{sippa} as “craft”. However, the basic meaning of “craft” is skill in doing or making things. What is meant here is a paid occupation regardless of whether it involves making things, i.e. “profession”. } These include elephant riders, cavalry, charioteers, archers, bannermen, adjutants, food servers, warrior-chiefs, princes, chargers, great warriors, heroes, leather-clad soldiers, and sons of bondservants.\footnote{These are the professions on \textsanskrit{Ajātasattu}’s mind. The first set of these is defined as branches of the military at \href{https://suttacentral.net/an7.67/en/sujato}{AN 7.67}. } They also include bakers, barbers, bathroom attendants, cooks, garland-makers, dyers, weavers, basket-makers, potters, accountants, finger-talliers, or those following any similar professions. All these live off the fruits of their profession which are apparent in this very life.\footnote{The question pertains to right livelihood, the fifth factor of the noble eightfold path. } With that they make themselves happy and pleased. They make their parents, their children and partners, and their friends and colleagues happy and pleased. And they establish an uplifting religious donation for ascetics and brahmins that’s conducive to heaven, ripens in happiness, and leads to heaven.\footnote{The purpose of right livelihood is to bring happiness in this life and the next. } Sir, can you point out a fruit of the ascetic life that’s likewise apparent in this very life?”\footnote{\textsanskrit{Ajātasattu}’s question only pertains to happiness in this life. He would have seen ascetics living hard and austere lives for the sake of future happiness. } 

“Great\marginnote{15.1} king, do you recall having asked this question of other ascetics and brahmins?”\footnote{The term “Great King” (\textit{\textsanskrit{mahārāja}}) identifies \textsanskrit{Ajātasattu} as the hereditary monarch of a large realm, in contrast with the multiple elected “rulers” of the aristocratic republics such as Vajji and Sakya. } 

“I\marginnote{15.2} do, sir.” 

“If\marginnote{15.3} you wouldn’t mind, great king, tell me how they answered.”\footnote{As in \href{https://suttacentral.net/dn1/en/sujato\#1.4.2}{DN 1:1.4.2}, the Buddha begins by asking to hear what others have said. } 

“It’s\marginnote{15.4} no trouble when someone such as the Blessed One is sitting here.” 

“Well,\marginnote{15.5} speak then, great king.” 

\subsection*{3.1. The Doctrine of \textsanskrit{Pūraṇa} Kassapa }

“This\marginnote{16.1} one time, sir, I approached \textsanskrit{Pūraṇa} Kassapa and exchanged greetings with him. When the greetings and polite conversation were over, I sat down to one side, and asked him the same question.\footnote{As with his meeting with the Buddha, \textsanskrit{Ajātasattu} is respectful but not reverential. } 

He\marginnote{17.1} said to me: ‘Great king, the one who acts does nothing wrong when they punish, mutilate, torture, aggrieve, oppress, intimidate, or when they encourage others to do the same. They do nothing wrong when they kill, steal, break into houses, plunder wealth, steal from isolated buildings, commit highway robbery, commit adultery, and lie. If you were to reduce all the living creatures of this earth to one heap and mass of flesh with a razor-edged chakram, no evil comes of that, and no outcome of evil.\footnote{This is a denial of the doctrine of kamma. While his doctrine appears to be morally nihilistic, it seems unlikely this was \textsanskrit{Pūraṇa} Kassapa’s full teaching. He may have subscribed to hard determinism, so that we have no choice in what we do. He may also have believed that we should keep moral rules as a social contract, but that this had no effect on the afterlife. | In such contexts, \textit{kar-} means “punish, inflict” (\href{https://suttacentral.net/mn129/en/sujato\#29.2}{MN 129:29.2}). } If you were to go along the south bank of the Ganges killing, mutilating, and torturing, and encouraging others to do the same, no evil comes of that, and no outcome of evil. If you were to go along the north bank of the Ganges giving and sacrificing and encouraging others to do the same, no merit comes of that, and no outcome of merit. In giving, self-control, restraint, and truthfulness there is no merit or outcome of merit.’ 

And\marginnote{18.1} so, when I asked \textsanskrit{Pūraṇa} Kassapa about the fruits of the ascetic life apparent in the present life, he answered with the doctrine of inaction.\footnote{The unsatisfying nature of the answers given by these teachers is also emphasized at \href{https://suttacentral.net/mn36/en/sujato\#48.4}{MN 36:48.4}. } It was like someone who, when asked about a mango, answered with a breadfruit, or when asked about a breadfruit, answered with a mango.\footnote{Breadfruit is a starchy, fibrous fruit that is, needless to say, very different from a mango. } I thought: ‘How could one such as I presume to rebuke an ascetic or brahmin living in my realm?’\footnote{Kings had a duty to protect all religions in their realm, even those with such extreme views. } So I neither approved nor dismissed that statement of \textsanskrit{Pūraṇa} Kassapa. I was displeased, but did not express my displeasure. Neither accepting what he said nor contradicting it, I got up from my seat and left.\footnote{The commentary takes \textit{uggahita} and \textit{nikkujjita} as synonyms. Elsewhere, however, \textit{nikkujjati} always means “overturns”. } 

\subsection*{3.2. The Doctrine of the Bamboo-staffed Ascetic \textsanskrit{Gosāla} }

This\marginnote{19.1} one time, sir, I approached the bamboo-staffed ascetic \textsanskrit{Gosāla} and exchanged greetings with him. When the greetings and polite conversation were over, I sat down to one side, and asked him the same question. 

He\marginnote{20.1} said: ‘Great king, there is no cause or reason for the corruption of sentient beings. Sentient beings are corrupted without cause or reason.\footnote{This denies the principle of causality and the efficacy of action. The fatalistic teachings of the \textsanskrit{Ājīvikas} led to them becoming popular as prognosticators. } There’s no cause or reason for the purification of sentient beings. Sentient beings are purified without cause or reason. One does not act of one’s own volition, one does not act of another’s volition, one does not act from a person’s volition. There is no power, no energy, no human strength or vigor.\footnote{The first three phrases, with the Magadhan nominative singular in \textit{-e}, are unique to this passage. In \href{https://suttacentral.net/an6.38/en/sujato\#1.4}{AN 6.38:1.4} we find the regular nominative form in \textit{-o}. They are omitted in the otherwise parallel passage at \href{https://suttacentral.net/mn76/en/sujato\#13.6}{MN 76:13.6}. } All sentient beings, all living creatures, all beings, all souls lack control, power, and energy. Molded by destiny, circumstance, and nature, they experience pleasure and pain in the six classes of rebirth.\footnote{Everything is destined by circumstances beyond our control. } There are 1.4 million main wombs, and 6,000, and 600. There are 500 deeds, and five, and three. There are deeds and half-deeds. There are 62 paths, 62 sub-eons, six classes of rebirth, and eight stages in a person’s life. There are 4,900 \textsanskrit{Ājīvaka} ascetics, 4,900 wanderers, and 4,900 naked ascetics. There are 2,000 faculties, 3,000 hells, and 36 realms of dust. There are seven percipient embryos, seven non-percipient embryos, and seven knotless embryos. There are seven gods, seven humans, and seven goblins. There are seven lakes, seven rivers, 700 rivers, seven cliffs, and 700 cliffs. There are seven dreams and 700 dreams. There are 8.4 million great eons through which the foolish and the astute transmigrate before making an end of suffering.\footnote{This strange cosmology lays out the course through which souls must proceed before their final liberation. | Since \textit{\textsanskrit{nāgāvāsa}} (“abode of dragons”) occurs in a list of kinds of ascetics, I think it is a corruption of \textit{\textsanskrit{naggāvāsa}} (“abode of naked ascetics”) and translate accordingly. Each of these refers to the number of times one will be reborn in each of these states. | \textit{\textsanskrit{Nigaṇṭhigabbha}} could mean “rebirth as a Jain ascetic” but here we have moved on from listing ascetics, and I think it refers to one who is born free of attachments. Compare the Buddhist idea of “four kinds of conception” (\href{https://suttacentral.net/dn33/en/sujato\#1.11.175}{DN 33:1.11.175}). | \textit{\textsanskrit{Pavuṭā}} is explained by the commentary as \textit{\textsanskrit{gaṇṭhikā}} (“knot”). However, Rig Veda 9.54.2 mentions “seven rivers” (\textit{sapta pravata}) that flow from heaven, preceded in the same verse by the mention of \textit{\textsanskrit{sarā}} (“lakes”). This detail suggests a Vedic influence. The “seven rivers” are normally called \textit{sapta sindhu} (Rig Veda 1.35.8) in reference to the river systems of north-west India and Pakistan (cf. Punjab, “five rivers”). } And here there is no such thing as this: “By this precept or observance or fervent austerity or spiritual life I shall force unripened deeds to bear their fruit, or eliminate old deeds by experiencing their results little by little,” for that cannot be.\footnote{To “force unripened deeds to bear their fruit” by means of “fervent austerity” (\textit{tapas}) is a Jain practice, whereas to “eliminate old deeds by experiencing their results little by little” is distinguished from the Jain view at \href{https://suttacentral.net/an3.74/en/sujato}{AN 3.74}. } Pleasure and pain are allotted. Transmigration lasts only for a limited period, so there’s no increase or decrease, no getting better or worse. It’s like how, when you toss a ball of string, it rolls away unraveling. In the same way, after transmigrating the foolish and the astute will make an end of suffering.’ 

And\marginnote{21.1} so, when I asked the bamboo-staffed ascetic \textsanskrit{Gosāla} about the fruits of the ascetic life apparent in the present life, he answered with the doctrine of purification through transmigration.\footnote{“Purification through transmigration” is \textit{\textsanskrit{saṁsārasuddhi}}. } It was like someone who, when asked about a mango, answered with a breadfruit, or when asked about a breadfruit, answered with a mango. I thought: ‘How could one such as I presume to rebuke an ascetic or brahmin living in my realm?’ So I neither approved nor dismissed that statement of the bamboo-staffed ascetic \textsanskrit{Gosāla}. I was displeased, but did not express my displeasure. Neither accepting what he said nor contradicting it, I got up from my seat and left. 

\subsection*{3.3. The Doctrine of Ajita of the Hair Blanket }

This\marginnote{22.1} one time, sir, I approached Ajita of the hair blanket and exchanged greetings with him. When the greetings and polite conversation were over, I sat down to one side, and asked him the same question. 

He\marginnote{23.1} said: ‘Great king, there is no meaning in giving, sacrifice, or offerings. There’s no fruit or result of good and bad deeds. There’s no afterlife. There’s no such thing as mother and father, or beings that are reborn spontaneously. And there’s no ascetic or brahmin who is rightly comported and rightly practiced, and who describes the afterlife after realizing it with their own insight.\footnote{The denial of “mother and father” is usually interpreted as the denial of moral duty towards one’s parents. However, I think it is a doctrine of conception which denies that a child is created by the mother and father. Rather, the child is produced by the four elements, with parents as mere instigators and incubators. } This person is made up of the four principal states. When they die, the earth in their body merges and coalesces with the substance of earth. The water in their body merges and coalesces with the substance of water. The fire in their body merges and coalesces with the substance of fire. The air in their body merges and coalesces with the substance of air. The faculties are transferred to space.\footnote{This is a materialist analysis of the person. | The word \textit{\textsanskrit{kāya}} (“substance”) is central to Jainism. \textsanskrit{Ācārāṅgasūtra} 8.1.11 speaks of the “substances” of earth, water, fire, and air as being imbued with life so one should avoid damaging them. | The Buddha’s use of \textit{\textsanskrit{mahābhūtā}} (“principal states”) responds to \textsanskrit{Yājñavalkya}’s core teaching at \textsanskrit{Bṛhadāraṇyaka} \textsanskrit{Upaniṣad} 2.4.12, where the several “states” or “real entities” (\textit{\textsanskrit{bhūtā}})—namely the diverse manifestations of creation—arise from and dissolve into the “principal state” (\textit{\textsanskrit{mahābhūta}}) of the Self, singular and infinite. For the Buddha, the “principal states” are themselves plural, as there is no underlying singular reality. Later Sanskrit literature lists the “five states” (\textit{\textsanskrit{pañcabhūta}}) as earth, water, fire, air, and space. } Four men with a bier carry away the corpse. Their footprints show the way to the cemetery. The bones become bleached. Offerings dedicated to the gods end in ashes. Giving is a doctrine of morons. When anyone affirms a positive teaching it’s just hollow, false nonsense. Both the foolish and the astute are annihilated and destroyed when their body breaks up, and don’t exist after death.’ 

And\marginnote{24.1} so, when I asked Ajita of the hair blanket about the fruits of the ascetic life apparent in the present life, he answered with the doctrine of annihilationism. It was like someone who, when asked about a mango, answered with a breadfruit, or when asked about a breadfruit, answered with a mango. I thought: ‘How could one such as I presume to rebuke an ascetic or brahmin living in my realm?’ So I neither approved nor dismissed that statement of Ajita of the hair blanket. I was displeased, but did not express my displeasure. Neither accepting what he said nor contradicting it, I got up from my seat and left. 

\subsection*{3.4. The Doctrine of Pakudha \textsanskrit{Kaccāyana} }

This\marginnote{25.1} one time, sir, I approached Pakudha \textsanskrit{Kaccāyana} and exchanged greetings with him. When the greetings and polite conversation were over, I sat down to one side, and asked him the same question. 

He\marginnote{26.1} said: ‘Great king, these seven substances are not made, not derived, not created, without a creator, barren, steady as a mountain peak, standing firm like a pillar.\footnote{This is a reductive atomism. It argues that since all things are made of the seven fundamental substances (\textit{\textsanskrit{kāya}}), higher-order entities have no significance. } They don’t move or deteriorate or obstruct each other. They’re unable to cause pleasure, pain, or both pleasure and pain to each other. What seven? The substances of earth, water, fire, air; pleasure, pain, and the soul is the seventh.\footnote{Unlike the materialism of Ajita Kesakambala, one of the basic substances is the soul. He uses \textit{\textsanskrit{jīva}}, the same term used by the Jains, rather than \textit{\textsanskrit{attā}} as preferred by the brahmins. Likewise, the Jains held a similar doctrine of six uncreated and eternal “substances” (\textit{\textsanskrit{kāya}} or \textit{dravya}): soul, the media for motion and rest, matter, space, and time. } These seven substances are not made, not derived, not created, without a creator, barren, steady as a mountain peak, standing firm like a pillar. They don’t move or deteriorate or obstruct each other. They’re unable to cause pleasure, pain, or both pleasure and pain to each other. And here there is no-one who kills or who makes others kill; no-one who learns or who educates others; no-one who understands or who helps others understand.\footnote{Compare \href{https://suttacentral.net/an8.16/en/sujato\#1.3}{AN 8.16:1.3}. } If you chop off someone’s head with a sharp sword, you don’t take anyone’s life. The sword simply passes through the gap between the seven substances.’ 

And\marginnote{27.1} so, when I asked Pakudha \textsanskrit{Kaccāyana} about the fruits of the ascetic life apparent in the present life, he answered with something else entirely. It was like someone who, when asked about a mango, answered with a breadfruit, or when asked about a breadfruit, answered with a mango. I thought: ‘How could one such as I presume to rebuke an ascetic or brahmin living in my realm?’ So I neither approved nor dismissed that statement of Pakudha \textsanskrit{Kaccāyana}. I was displeased, but did not express my displeasure. Neither accepting what he said nor contradicting it, I got up from my seat and left. 

\subsection*{3.5. The Doctrine of the Jain Ascetic of the \textsanskrit{Ñātika} Clan }

This\marginnote{28.1} one time, sir, I approached the Jain \textsanskrit{Ñātika} and exchanged greetings with him. When the greetings and polite conversation were over, I sat down to one side, and asked him the same question. 

He\marginnote{29.1} said: ‘Great king, consider a Jain ascetic who is restrained in the fourfold constraint.\footnote{While this is a genuine Jain teaching, it has not been identified as a “fourfold restraint”. \href{https://suttacentral.net/dn25/en/sujato\#16.3}{DN 25:16.3} preserves another “fourfold restraint” that is closer to that found in Jainism. At \href{https://suttacentral.net/mn12/en/sujato\#44.1}{MN 12:44.1} the Buddha says he once practiced a “four-factored spiritual path” that consisted of Jain-like austerities. } And how is a Jain ascetic restrained in the fourfold constraint? It’s when a Jain ascetic is restrained in all that is to be restrained, is bridled in all that is to be restrained, has shaken off evil in all that is to be restrained, and is curbed in all that is to be restrained.\footnote{At \textsanskrit{Isibhāsiyāiṁ} 29.19, \textsanskrit{Vardhamāna} (\textsanskrit{Mahāvīra}) teaches that a sage is \textit{savva-\textsanskrit{vārīhiṁ} \textsanskrit{vārie}}, “restrained in all restraints”, which clearly parallels our current passage. In that passage, “restraint” refers to stopping the influx of defilements through the five senses, neither delighting in the pleasant nor loathing the unpleasant. Similarly we find \textit{\textsanskrit{vāriya}-savva-\textsanskrit{vāri}} in the commentary to \textsanskrit{Sūyagaḍa} 1.6.28. | Read \textit{\textsanskrit{vāri}} as future passive participle (cf. Sanskrit \textit{\textsanskrit{vārya}}). | \textit{Dhuta} in the sense “shaken off (evil by means of ascetic practices)” is a characteristic Jain term. | For \textit{\textsanskrit{sabbavāriphuṭo}} compare \textit{\textsanskrit{ophuṭo}} at \href{https://suttacentral.net/mn99/en/sujato\#15.5}{MN 99:15.5}. In both cases \textit{\textsanskrit{phuṭ}} appears in a string of terms from the root \textit{var}, and is possibly a corrupted form, or at least has the same meaning. } That’s how a Jain ascetic is restrained in the fourfold constraint. When a Jain ascetic is restrained in the fourfold constraint, they’re called a knotless one who is self-realized, self-controlled, and steadfast.’ 

And\marginnote{30.1} so, when I asked the Jain \textsanskrit{Ñātika} about the fruits of the ascetic life apparent in the present life, he answered with the fourfold constraint. It was like someone who, when asked about a mango, answered with a breadfruit, or when asked about a breadfruit, answered with a mango. I thought: ‘How could one such as I presume to rebuke an ascetic or brahmin living in my realm?’ So I neither approved nor dismissed that statement of the Jain \textsanskrit{Ñātika}. I was displeased, but did not express my displeasure. Neither accepting what he said nor contradicting it, I got up from my seat and left. 

\subsection*{3.6. The Doctrine of \textsanskrit{Sañjaya} \textsanskrit{Belaṭṭhiputta} }

This\marginnote{31.1} one time, sir, I approached \textsanskrit{Sañjaya} \textsanskrit{Belaṭṭhiputta} and exchanged greetings with him. When the greetings and polite conversation were over, I sat down to one side, and asked him the same question. 

He\marginnote{32.1} said: ‘Suppose you were to ask me whether there is another world. If I believed that to be the case, I would say so. But I don’t say it’s like this. I don’t say it’s like that. I don’t say it’s otherwise. I don’t say it’s not so. And I don’t deny it’s not so.\footnote{This places him among the “endless flip-floppers” of \href{https://suttacentral.net/dn1/en/sujato\#2.23.1}{DN 1:2.23.1}. However, we do not know on which of the four grounds he justified his evasiveness. } Suppose you were to ask me whether there is no other world … whether there both is and is not another world … whether there neither is nor is not another world … whether there are beings who are reborn spontaneously … whether there are no beings who are reborn spontaneously … whether there both are and are not beings who are reborn spontaneously … whether there neither are nor are not beings who are reborn spontaneously … whether there is fruit and result of good and bad deeds … whether there is no fruit and result of good and bad deeds … whether there both is and is not fruit and result of good and bad deeds … whether there neither is nor is not fruit and result of good and bad deeds … whether a realized one still exists after death … whether A realized one no longer exists after death … whether a realized one both still exists and no longer exists after death … whether a Realized One neither exists nor doesn’t exist after death. If I believed that to be the case, I would say so. But I don’t say it’s like this. I don’t say it’s like that. I don’t say it’s otherwise. I don’t say it’s not so. And I don’t deny it’s not so.’ 

And\marginnote{33.1} so, when I asked \textsanskrit{Sañjaya} \textsanskrit{Belaṭṭhiputta} about the fruits of the ascetic life apparent in the present life, he answered with flip-flopping. It was like someone who, when asked about a mango, answered with a breadfruit, or when asked about a breadfruit, answered with a mango. I thought: ‘This is the most foolish and stupid of all these ascetics and brahmins! How on earth can he answer with flip-flopping when asked about the fruits of the ascetic life apparent in the present life?’ I thought: ‘How could one such as I presume to rebuke an ascetic or brahmin living in my realm?’ So I neither approved nor dismissed that statement of \textsanskrit{Sañjaya} \textsanskrit{Belaṭṭhiputta}. I was displeased, but did not express my displeasure. Neither accepting what he said nor contradicting it, I got up from my seat and left. 

\section*{4. The Fruits of the Ascetic Life }

\subsection*{4.1. The First Fruit of the Ascetic Life }

And\marginnote{34.1} so I ask the Buddha: Sir, there are many different professional fields. These include elephant riders, cavalry, charioteers, archers, bannermen, adjutants, food servers, warrior-chiefs, princes, chargers, great warriors, heroes, leather-clad soldiers, and sons of bondservants. They also include bakers, barbers, bathroom attendants, cooks, garland-makers, dyers, weavers, basket-makers, potters, accountants, finger-talliers, or those following any similar professions. All these live off the fruits of their profession which are apparent in this very life. With that they make themselves happy and pleased. They make their parents, their children and partners, and their friends and colleagues happy and pleased. And they establish an uplifting religious donation for ascetics and brahmins that’s conducive to heaven, ripens in happiness, and leads to heaven. Sir, can you point out a fruit of the ascetic life that’s likewise apparent in this very life?” 

“I\marginnote{34.7} can, great king.\footnote{The Buddha answers directly, with confidence. This whole passage is a masterclass in effective dialogue. } Well then, I’ll ask you about this in return, and you can answer as you like.\footnote{He engages \textsanskrit{Ajātasattu} rather than lecturing him. } What do you think, great king? Suppose you had a person who was a bondservant, a worker. They get up before you and go to bed after you, and are obliging, behaving nicely and speaking politely, and gazing up at your face.\footnote{See \textit{\textsanskrit{mukhaṁ} \textsanskrit{ullokentī}} at \href{https://suttacentral.net/mn79/en/sujato}{MN 79} and \href{https://suttacentral.net/sn56.39/en/sujato}{SN 56.39}. } They’d think: ‘The outcome and result of good deeds is just so incredible, so amazing!\footnote{Even a servant believed in the doctrine of kamma. } For this King \textsanskrit{Ajātasattu} is a human being, and so am I.\footnote{There is no question of the divinity of kings. } Yet he amuses himself, supplied and provided with the five kinds of sensual stimulation as if he were a god. Whereas I’m his bondservant, his worker. I get up before him and go to bed after him, and am obliging, behaving nicely and speaking politely, and gazing up at his face. I really should do good deeds.\footnote{The doctrine of kamma leads to living a better life, not stewing in resentment. } Why don’t I shave off my hair and beard, dress in ocher robes, and go forth from the lay life to homelessness?’\footnote{\href{https://suttacentral.net/pli-tv-kd1/en/sujato\#47.1.1}{Kd 1:47.1.1} penalizes the ordination of bondservants or slaves, despite the fact that \textsanskrit{Ajātasattu}’s father, \textsanskrit{Bimbisāra}, had ordered that no action was to be taken against any bondservant who ordained under the Buddha. } 

After\marginnote{35.10} some time, that is what they do. Having gone forth they’d live restrained in body, speech, and mind, living content with nothing more than food and clothes, delighting in seclusion.\footnote{Here the Buddha foreshadows the larger themes detailed later. } And suppose your men were to report all this to you. Would you say to them: ‘Bring that person to me! Let them once more be my bondservant, my worker’?” 

“No,\marginnote{36.1} sir. Rather, I would bow to them, rise in their presence, and offer them a seat. I’d invite them to accept robes, almsfood, lodgings, and medicines and supplies for the sick. And I’d organize their lawful guarding and protection.”\footnote{Even under a king so compromised as \textsanskrit{Ajātasattu},  a runaway slave who has ordained is rewarded not punished. } 

“What\marginnote{36.3} do you think, great king? If this is so, is there a fruit of the ascetic life apparent in the present life or not?”\footnote{In contrast with the former teachers, the Buddha gives a clear answer in terms that \textsanskrit{Ajātasattu} would understand. } 

“Clearly,\marginnote{36.5} sir, there is.”\footnote{The Buddha establishes common ground with the king before venturing into deeper waters. } 

“This\marginnote{36.6} is the first fruit of the ascetic life that’s apparent in this very life, which I point out to you.” 

\subsection*{4.2. The Second Fruit of the Ascetic Life }

“But\marginnote{37.1} sir, can you point out another fruit of the ascetic life that’s likewise apparent in this very life?”\footnote{By starting with a very basic and obvious fruit, the Buddha stimulates \textsanskrit{Ajātasattu} to seek a deeper answer. } 

“I\marginnote{37.2} can, great king. Well then, I’ll ask you about this in return, and you can answer as you like. What do you think, great king? Suppose you had a person who was a farmer, a householder, a hard worker, someone who builds up their capital.\footnote{\textit{\textsanskrit{Karakārako} \textsanskrit{rāsivaḍḍhako}} is a unique phrase. For \textit{\textsanskrit{karakāraka}}, compare \href{https://suttacentral.net/mn57/en/sujato\#2.3}{MN 57:2.3}, where a naked ascetic “does a hard thing”. \textit{\textsanskrit{Rāsi}} means “heap” (of grain or wealth according to the commentary). } They’d think: ‘The outcome and result of good deeds is just so incredible, so amazing! For this King \textsanskrit{Ajātasattu} is a human being, and so am I. Yet he amuses himself, supplied and provided with the five kinds of sensual stimulation as if he were a god. Whereas I’m a farmer, a householder, a hard worker, someone who builds up their capital. I really should do good deeds. Why don’t I shave off my hair and beard, dress in ocher robes, and go forth from the lay life to homelessness?’ 

After\marginnote{37.13} some time they give up a large or small fortune, and a large or small family circle. They’d shave off hair and beard, dress in ocher robes, and go forth from the lay life to homelessness.\footnote{The bonded servant had no wealth or family to renounce, but the worker does. } Having gone forth they’d live restrained in body, speech, and mind, living content with nothing more than food and clothes, delighting in seclusion. And suppose your men were to report all this to you. Would you say to them: ‘Bring that person to me! Let them once more be a farmer, a householder, a hard worker, someone who builds up their capital’?” 

“No,\marginnote{38.1} sir. Rather, I would bow to them, rise in their presence, and offer them a seat. I’d invite them to accept robes, almsfood, lodgings, and medicines and supplies for the sick. And I’d organize their lawful guarding and protection.” 

“What\marginnote{38.3} do you think, great king? If this is so, is there a fruit of the ascetic life apparent in the present life or not?” 

“Clearly,\marginnote{38.5} sir, there is.” 

“This\marginnote{38.6} is the second fruit of the ascetic life that’s apparent in this very life, which I point out to you.” 

\subsection*{4.3. The Finer Fruits of the Ascetic Life }

“But\marginnote{39.1} sir, can you point out a fruit of the ascetic life that’s apparent in this very life which is better and finer than these?” 

“I\marginnote{39.2} can, great king. Well then, listen and apply your mind well, I will speak.”\footnote{Having established the king’s genuine interest and understanding, the Buddha prepares him for the long discourse to follow. } 

“Yes,\marginnote{39.4} sir,” replied the king. 

The\marginnote{39.5} Buddha said this: 

“Consider\marginnote{40.1} when a Realized One arises in the world, perfected, a fully awakened Buddha, accomplished in knowledge and conduct, holy, knower of the world, supreme guide for those who wish to train, teacher of gods and humans, awakened, blessed.\footnote{This is the start of the teaching on the Gradual Training, encompassing ethics (\textit{\textsanskrit{sīla}}), meditation (\textit{\textsanskrit{samādhi}}), and wisdom (\textit{\textsanskrit{paññā}}). Only the ethics portion appeared in the \textsanskrit{Brahmajālasutta}, while all three are restated in all the remaining suttas of this chapter, although in truncated form. | It is exceedingly rare for a Buddha to appear. } He has realized with his own insight this world—with its gods, \textsanskrit{Māras}, and divinities, this population with its ascetics and brahmins, gods and humans—and he makes it known to others.\footnote{The Buddha realizes the truth by his own understanding, not through divine intervention or other metaphysical means. } He proclaims a teaching that is good in the beginning, good in the middle, and good in the end, meaningful and well-phrased. And he reveals a spiritual practice that’s entirely full and pure.\footnote{It is good when first heard, when practicing, and when one has realized the fruits. } 

A\marginnote{41.1} householder hears that teaching, or a householder’s child, or someone reborn in a good family.\footnote{The word “householder” (\textit{gahapati}) informally refers to any lay person, but more specifically indicates someone who owns a house, i.e. a person of standing. The renunciate life is not just for slaves or workers wishing to escape their station. } They gain faith in the Realized One and reflect: ‘Life at home is cramped and dirty, life gone forth is wide open. It’s not easy for someone living at home to lead the spiritual life utterly full and pure, like a polished shell. Why don’t I shave off my hair and beard, dress in ocher robes, and go forth from the lay life to homelessness?’ 

After\marginnote{41.7} some time they give up a large or small fortune, and a large or small family circle. They shave off hair and beard, dress in ocher robes, and go forth from the lay life to homelessness. 

Once\marginnote{42.1} they’ve gone forth, they live restrained in the monastic code, conducting themselves well and resorting for alms in suitable places. Seeing danger in the slightest fault, they keep the rules they’ve undertaken. They act skillfully by body and speech. They’re purified in livelihood and accomplished in ethical conduct. They guard the sense doors, have mindfulness and situational awareness, and are content.\footnote{This serves as a table of contents for the teachings to come. | Nowadays, the “monastic code” (\textit{\textsanskrit{pātimokkha}}) means the list of rules for monks and nuns found in the \textsanskrit{Vinayapiṭaka}. In the early texts, however, it has three main meanings. Sometimes it does refer to the list of rules, as at \href{https://suttacentral.net/an10.36/en/sujato\#1.6}{AN 10.36:1.6}. Here it refers to the code of conduct that follows, which is a non-legalistic set of guidelines that preceded the \textsanskrit{Vinayapiṭaka}. At \href{https://suttacentral.net/dn14/en/sujato\#3.28.1}{DN 14:3.28.1} it refers to the verses summarizing monastic conduct known as the “\textsanskrit{Ovāda} \textsanskrit{Pātimokkha}”. } 

\subsubsection*{4.3.1. Ethics }

\paragraph*{4.3.1.1. The Shorter Section on Ethics }

And\marginnote{43.1} how, great king, is a mendicant accomplished in ethics? It’s when a mendicant gives up killing living creatures, renouncing the rod and the sword. They’re scrupulous and kind, living full of sympathy for all living beings.\footnote{While the precept includes any living creature, if a monastic murders a human being they are immediately and permanently expelled. } This pertains to their ethics. 

They\marginnote{43.4} give up stealing. They take only what’s given, and expect only what’s given. They keep themselves clean by not thieving.\footnote{To steal anything of substantial value is an expulsion offence. } This pertains to their ethics. 

They\marginnote{43.6} give up unchastity. They are celibate, set apart, avoiding the vulgar act of sex.\footnote{Buddhist monastics are forbidden from any form of sexual activity. To engage in penetrative intercourse is an expulsion offence. } This pertains to their ethics. 

They\marginnote{44.1} give up lying. They speak the truth and stick to the truth. They’re honest and dependable, and don’t trick the world with their words.\footnote{While any form of lying is forbidden, if a monastic falsely claims states of enlightenment or deep meditation they are expelled. } This pertains to their ethics. 

They\marginnote{44.3} give up divisive speech. They don’t repeat in one place what they heard in another so as to divide people against each other. Instead, they reconcile those who are divided, supporting unity, delighting in harmony, loving harmony, speaking words that promote harmony. This pertains to their ethics. 

They\marginnote{44.5} give up harsh speech. They speak in a way that’s mellow, pleasing to the ear, lovely, going to the heart, polite, likable and agreeable to the people. This pertains to their ethics. 

They\marginnote{44.7} give up talking nonsense. Their words are timely, true, and meaningful, in line with the teaching and training. They say things at the right time which are valuable, reasonable, succinct, and beneficial. This pertains to their ethics. 

They\marginnote{45.1} refrain from injuring plants and seeds. They eat in one part of the day, abstaining from eating at night and food at the wrong time. They refrain from seeing shows of dancing, singing, and music . They refrain from beautifying and adorning themselves with garlands, fragrance, and makeup. They refrain from high and luxurious beds.\footnote{To avoid sleeping too much. } They refrain from receiving gold and currency, raw grains, raw meat, women and girls, male and female bondservants, goats and sheep, chickens and pigs, elephants, cows, horses, and mares, and fields and land. They refrain from running errands and messages; buying and selling; falsifying weights, metals, or measures; bribery, fraud, cheating, and duplicity; mutilation, murder, abduction, banditry, plunder, and violence. This pertains to their ethics. 

\scendsection{The shorter section on ethics is finished. }

\paragraph*{4.3.1.2. The Middle Section on Ethics }

There\marginnote{46.1} are some ascetics and brahmins who, while enjoying food given in faith, still engage in injuring plants and seeds. These include plants propagated from roots, stems, cuttings, or joints; and those from regular seeds as the fifth. They refrain from such injury to plants and seeds. This pertains to their ethics. 

There\marginnote{47.1} are some ascetics and brahmins who, while enjoying food given in faith, still engage in storing up goods for their own use. This includes such things as food, drink, clothes, vehicles, bedding, fragrance, and things of the flesh. They refrain from storing up such goods. This pertains to their ethics. 

There\marginnote{48.1} are some ascetics and brahmins who, while enjoying food given in faith, still engage in seeing shows. This includes such things as dancing, singing, music, performances, and storytelling; clapping, gongs, and kettledrums; beauty pageants; pole-acrobatics and bone-washing displays of the corpse-workers; battles of elephants, horses, buffaloes, bulls, goats, rams, chickens, and quails; staff-fights, boxing, and wrestling; combat, roll calls of the armed forces, battle-formations, and regimental reviews. They refrain from such shows. This pertains to their ethics. 

There\marginnote{49.1} are some ascetics and brahmins who, while enjoying food given in faith, still engage in gambling that causes negligence. This includes such things as checkers with eight or ten rows, checkers in the air, hopscotch, spillikins, board-games, tip-cat, drawing straws, dice, leaf-flutes, toy plows, somersaults, pinwheels, toy measures, toy carts, toy bows, guessing words from syllables, guessing another’s thoughts, and imitating musical instruments. They refrain from such gambling. This pertains to their ethics. 

There\marginnote{50.1} are some ascetics and brahmins who, while enjoying food given in faith, still make use of high and luxurious bedding. This includes such things as sofas, couches, woolen covers—shag-piled, colorful, white, embroidered with flowers, quilted, embroidered with animals, double-or single-fringed—and silk covers studded with gems, as well as silken sheets, woven carpets, rugs for elephants, horses, or chariots, antelope hide rugs, and spreads of fine deer hide, with a canopy above and red cushions at both ends. They refrain from such bedding. This pertains to their ethics. 

There\marginnote{51.1} are some ascetics and brahmins who, while enjoying food given in faith, still engage in beautifying and adorning themselves with garlands, fragrance, and makeup. This includes such things as applying beauty products by anointing, massaging, bathing, and rubbing; mirrors, ointments, garlands, fragrances, and makeup; face-powder, foundation, bracelets, headbands, fancy walking-sticks or containers, rapiers, parasols, fancy sandals, turbans, jewelry, chowries, and long-fringed white robes. They refrain from such beautification and adornment. This pertains to their ethics. 

There\marginnote{52.1} are some ascetics and brahmins who, while enjoying food given in faith, still engage in low talk. This includes such topics as talk about kings, bandits, and ministers; talk about armies, threats, and wars; talk about food, drink, clothes, and beds; talk about garlands and fragrances; talk about family, vehicles, villages, towns, cities, and countries; talk about women and heroes; street talk and well talk; talk about the departed; motley talk; tales of land and sea; and talk about being reborn in this or that place. They refrain from such low talk. This pertains to their ethics. 

There\marginnote{53.1} are some ascetics and brahmins who, while enjoying food given in faith, still engage in arguments. They say such things as: ‘You don’t understand this teaching and training. I understand this teaching and training. What, you understand this teaching and training? You’re practicing wrong. I’m practicing right. I stay on topic, you don’t. You said last what you should have said first. You said first what you should have said last. What you’ve thought so much about has been disproved. Your doctrine is refuted. Go on, save your doctrine! You’re trapped; get yourself out of this—if you can!’ They refrain from such argumentative talk. This pertains to their ethics. 

There\marginnote{54.1} are some ascetics and brahmins who, while enjoying food given in faith, still engage in running errands and messages. This includes running errands for rulers, ministers, aristocrats, brahmins, householders, or princes who say: ‘Go here, go there. Take this, bring that from there.’ They refrain from such errands. This pertains to their ethics. 

There\marginnote{55.1} are some ascetics and brahmins who, while enjoying food given in faith, still engage in deceit, flattery, hinting, and belittling, and using material things to chase after other material things. They refrain from such deceit and flattery. This pertains to their ethics. 

\scendsection{The middle section on ethics is finished. }

\paragraph*{4.3.1.3. The Long Section on Ethics }

There\marginnote{56.1} are some ascetics and brahmins who, while enjoying food given in faith, still earn a living by low lore, by wrong livelihood. This includes such fields as limb-reading, omenology, divining celestial portents, interpreting dreams, divining bodily marks, divining holes in cloth gnawed by mice, fire offerings, ladle offerings, offerings of husks, rice powder, rice, ghee, or oil; offerings from the mouth, blood sacrifices, palmistry; geomancy for building sites, fields, and cemeteries; exorcisms, earth magic, snake charming, poisons; the lore of the scorpion, the rat, the bird, and the crow; prophesying lifespan, chanting for protection, and divining omens from wild animals. They refrain from such low lore, such wrong livelihood. This pertains to their ethics. 

There\marginnote{57.1} are some ascetics and brahmins who, while enjoying food given in faith, still earn a living by low lore, by wrong livelihood. This includes reading the marks of gems, cloth, clubs, swords, spears, arrows, bows, weapons, women, men, boys, girls, male and female bondservants, elephants, horses, buffaloes, bulls, cows, goats, rams, chickens, quails, monitor lizards, rabbits, tortoises, or deer. They refrain from such low lore, such wrong livelihood. This pertains to their ethics. 

There\marginnote{58.1} are some ascetics and brahmins who, while enjoying food given in faith, still earn a living by low lore, by wrong livelihood. This includes making predictions that the king will march forth or march back; or that our king will attack and the enemy king will retreat, or vice versa; or that our king will triumph and the enemy king will be defeated, or vice versa; and so there will be victory for one and defeat for the other. They refrain from such low lore, such wrong livelihood. This pertains to their ethics. 

There\marginnote{59.1} are some ascetics and brahmins who, while enjoying food given in faith, still earn a living by low lore, by wrong livelihood. This includes making predictions that there will be an eclipse of the moon, or sun, or stars; that the sun, moon, and stars will be in conjunction or in opposition; that there will be a meteor shower, a fiery sky, an earthquake, or thunder in the heavens; that there will be a rising, a setting, a darkening, a brightening of the moon, sun, and stars. And it also includes making predictions about the results of all such phenomena. They refrain from such low lore, such wrong livelihood. This pertains to their ethics. 

There\marginnote{60.1} are some ascetics and brahmins who, while enjoying food given in faith, still earn a living by low lore, by wrong livelihood. This includes predicting whether there will be plenty of rain or drought; plenty to eat or famine; an abundant harvest or a bad harvest; security or peril; sickness or health. It also includes such occupations as arithmetic, accounting, calculating, poetry, and cosmology. They refrain from such low lore, such wrong livelihood. This pertains to their ethics. 

There\marginnote{61.1} are some ascetics and brahmins who, while enjoying food given in faith, still earn a living by low lore, by wrong livelihood. This includes making arrangements for giving and taking in marriage; for engagement and divorce; and for scattering rice inwards or outwards at the wedding ceremony. It also includes casting spells for good or bad luck, treating impacted fetuses, binding the tongue, or locking the jaws; charms for the hands and ears; questioning a mirror, a girl, or a god as an oracle; worshiping the sun, worshiping the Great One, breathing fire, and invoking Siri, the goddess of luck. They refrain from such low lore, such wrong livelihood. This pertains to their ethics. 

There\marginnote{62.1} are some ascetics and brahmins who, while enjoying food given in faith, still earn a living by low lore, by wrong livelihood. This includes rites for propitiation, for granting wishes, for ghosts, for the earth, for rain, for property settlement, and for preparing and consecrating house sites, and rites involving rinsing and bathing, and oblations. It also includes administering emetics, purgatives, expectorants, and phlegmagogues; administering ear-oils, eye restoratives, nasal medicine, ointments, and counter-ointments; surgery with needle and scalpel, treating children, prescribing root medicines, and binding on herbs. They refrain from such low lore, such wrong livelihood. This pertains to their ethics. 

A\marginnote{63.1} mendicant thus accomplished in ethics sees no danger in any quarter in regards to their ethical restraint. It’s like a king who has defeated his enemies. He sees no danger from his foes in any quarter. In the same way, a mendicant thus accomplished in ethics sees no danger in any quarter in regards to their ethical restraint. When they have this entire spectrum of noble ethics, they experience a blameless happiness inside themselves.\footnote{This is the first step in the Buddha’s answer to \textsanskrit{Ajātasattu}. This is the sense of happiness and well-being that you have when you know you have done nothing wrong for which anyone might blame you. It is the psychological foundation for meditation. } That’s how a mendicant is accomplished in ethics. 

\scendsection{The longer section on ethics is finished. }

\subsubsection*{4.3.2. Immersion }

\paragraph*{4.3.2.1. Sense Restraint }

And\marginnote{64.1} how does a mendicant guard the sense doors?\footnote{Here begins the series of practices that build on moral fundamentals to lay the groundwork for meditation. } When a mendicant sees a sight with their eyes, they don’t get caught up in the features and details. If the faculty of sight were left unrestrained, bad unskillful qualities of covetousness and displeasure would become overwhelming. For this reason, they practice restraint, protecting the faculty of sight, and achieving its restraint.\footnote{It is not that one cannot see things, but that, mindful of its effect, one avoids unnecessary stimulation. | “Covetousness and bitterness” (\textit{\textsanskrit{abhijjhā} \textsanskrit{domanassā}}) are the strong forms of desire and aversion caused by lack of restraint. } When they hear a sound with their ears … When they smell an odor with their nose … When they taste a flavor with their tongue … When they feel a touch with their body … When they know an idea with their mind, they don’t get caught up in the features and details. If the faculty of mind were left unrestrained, bad unskillful qualities of covetousness and displeasure would become overwhelming. For this reason, they practice restraint, protecting the faculty of mind, and achieving its restraint. When they have this noble sense restraint, they experience an unsullied bliss inside themselves.\footnote{Their happiness deepens, as they see that not only their actions but also their mind is becoming free of anything unwholesome. } That’s how a mendicant guards the sense doors. 

\paragraph*{4.3.2.2. Mindfulness and Situational Awareness }

And\marginnote{65.1} how does a mendicant have mindfulness and situational awareness?\footnote{Situational awareness is a psychological term popularized in the 1990s. It has to do with the perception of environmental phenomena and the comprehension of their meaning, which is very close to the sense of the Pali term \textit{\textsanskrit{sampajañña}}. } It’s when a mendicant acts with situational awareness when going out and coming back; when looking ahead and aside; when bending and extending the limbs; when bearing the outer robe, bowl and robes; when eating, drinking, chewing, and tasting; when urinating and defecating; when walking, standing, sitting, sleeping, waking, speaking, and keeping silent.\footnote{These acts describe the daily life of  a mendicant: going into the village for alms, at which time there are many distracting sights. Then they return, eat their meal, and spend their day in meditation. } That’s how a mendicant has mindfulness and situational awareness. 

\paragraph*{4.3.2.3. Contentment }

And\marginnote{66.1} how is a mendicant content? It’s when a mendicant is content with robes to look after the body and almsfood to look after the belly. Wherever they go, they set out taking only these things.\footnote{A Buddhist monk has three robes: a lower robe (sabong or sarong), an upper robe, and an outer cloak. } They’re like a bird: wherever it flies, wings are its only burden. In the same way, a mendicant is content with robes to look after the body and almsfood to look after the belly. Wherever they go, they set out taking only these things. That’s how a mendicant is content. 

\paragraph*{4.3.2.4. Giving Up the Hindrances }

When\marginnote{67.1} they have this entire spectrum of noble ethics, this noble sense restraint, this noble mindfulness and situational awareness, and this noble contentment,\footnote{These are the prerequisite conditions for embarking on deep meditation. } they frequent a secluded lodging—a wilderness, the root of a tree, a hill, a ravine, a mountain cave, a charnel ground, a forest, the open air, a heap of straw. After the meal, they return from almsround, sit down cross-legged, set their body straight, and establish mindfulness in their presence.\footnote{For \textit{parimukha} (“in their presence”) we find \textit{pratimukha} in Sanskrit, which can mean “presence” or the reflection of the face. Late canonical Pali explains \textit{parimukha} as “the tip of the nose or the reflection of the face (\textit{mukhanimitta})”. \textit{Parimukha} in Sanskrit is rare, but it appears in \textsanskrit{Pāṇini} 4.4.29, which the commentary illustrates with the example of a servant “in the presence” of their master (cp. \href{https://suttacentral.net/sn47.8/en/sujato}{SN 47.8}). So it seems the sense is “before the face” or more generally “in the presence”. | To “establish mindfulness” (\textit{\textsanskrit{satiṁ} \textsanskrit{upaṭṭhapetvā}}) is literally to “do \textsanskrit{satipaṭṭhāna}”. } 

Giving\marginnote{68.1} up covetousness for the world, they meditate with a heart rid of covetousness, cleansing the mind of covetousness.\footnote{Covetousness (\textit{abhijjha}) has been curbed by sense restraint, and now is fully abandoned. } Giving up ill will and malevolence, they meditate with a mind rid of ill will, full of sympathy for all living beings, cleansing the mind of ill will.\footnote{Likewise ill will (\textit{\textsanskrit{byāpādapadosa}}), which was called \textit{domanassa} in the formula for sense restraint. } Giving up dullness and drowsiness, they meditate with a mind rid of dullness and drowsiness, perceiving light, mindful and aware, cleansing the mind of dullness and drowsiness.\footnote{“Mindfulness and situational awareness” has a prominent role in abandoning dullness. } Giving up restlessness and remorse, they meditate without restlessness, their mind peaceful inside, cleansing the mind of restlessness and remorse.\footnote{Restlessness hankers for the future and is countered by contentment. Remorse digs up the past and is countered by ethical purity. } Giving up doubt, they meditate having gone beyond doubt, not undecided about skillful qualities, cleansing the mind of doubt.\footnote{The meditator set out on their path after gaining faith in the Buddha. } 

Suppose\marginnote{69.1} a man who has gotten into debt were to apply himself to work,\footnote{The happiness of meditation is hard to understand without practicing, so the Buddha gives a series of five similes to illustrate in terms \textsanskrit{Ajātasattu} would understand. } and his efforts proved successful. He would pay off the original loan and have enough left over to support his partner. Thinking about this, he’d be filled with joy and happiness. 

Suppose\marginnote{70.1} there was a person who was sick, suffering, gravely ill. They’d lose their appetite and get physically weak. But after some time they’d recover from that illness, and regain their appetite and their strength. Thinking about this, they’d be filled with joy and happiness. 

Suppose\marginnote{71.1} a person was imprisoned in a jail. But after some time they were released from jail, safe and sound, with no loss of wealth. Thinking about this, they’d be filled with joy and happiness. 

Suppose\marginnote{72.1} a person was a bondservant. They would not be their own master, but indentured to another, unable to go where they wish. But after some time they’d be freed from servitude. They would be their own master, not indentured to another, an emancipated individual able to go where they wish. Thinking about this, they’d be filled with joy and happiness. 

Suppose\marginnote{73.1} there was a person with wealth and property who was traveling along a desert road, which was perilous, with nothing to eat. But after some time they crossed over the desert safely, arriving within a village, a sanctuary free of peril. Thinking about this, they’d be filled with joy and happiness. 

In\marginnote{74.1} the same way, as long as these five hindrances are not given up inside themselves, a mendicant regards them thus as a debt, a disease, a prison, slavery, and a desert crossing.\footnote{The five hindrances remain a pillar of meditation teaching. The root sense means to “obstruct” but also to “obscure, darken, veil”. } 

But\marginnote{74.2} when these five hindrances are given up inside themselves, a mendicant regards this as freedom from debt, good health, release from prison, emancipation, and a place of sanctuary at last.\footnote{Each simile illustrates not the happiness of acquisition, but of letting go. } 

Seeing\marginnote{74.4} that the hindrances have been given up in them, joy springs up. Being joyful, rapture springs up. When the mind is full of rapture, the body becomes tranquil. When the body is tranquil, they feel bliss. And when blissful, the mind becomes immersed.\footnote{The Buddha did not emphasize technical details of technique, but the emotional wholeness and joy that leads to deep meditation. } 

\paragraph*{4.3.2.5. First Absorption }

Quite\marginnote{75.1} secluded from sensual pleasures, secluded from unskillful qualities, they enter and remain in the first absorption, which has the rapture and bliss born of seclusion, while placing the mind and keeping it connected.\footnote{\textit{\textsanskrit{Jhāna}} is a state of “elevated consciousness” (\textit{adhicitta}), so all the terms have an elevated sense. | The plural form indicates that “sensual pleasures” includes sense experience, which the meditator can turn away from since they no longer have any desire for it. | The “unskillful qualities” are the five hindrances. | The “rapture and bliss born of seclusion” is the happiness of abandoning the hindrances and freedom from sense impingement. | “Placing the mind and keeping it connected” (\textit{vitakka}, \textit{\textsanskrit{vicāra}}) uses terms that mean “thought” in coarse consciousness, but which in “elevated consciousness” refer to the subtle function of applying the mind to the meditation. } They drench, steep, fill, and spread their body with rapture and bliss born of seclusion. There’s no part of the body that’s not spread with rapture and bliss born of seclusion.\footnote{As a meditator proceeds, their subjective experience of the “body” evolves from tactile sense impressions (\textit{\textsanskrit{phoṭṭhabba}}), to the interior mental experience of bliss and light (\textit{\textsanskrit{manomayakāya}}), to the direct personal realization of highest truth (\href{https://suttacentral.net/mn70/en/sujato\#23.2}{MN 70:23.2}: \textit{\textsanskrit{kāyena} ceva \textsanskrit{paramasaccaṁ} sacchikaroti}). } 

It’s\marginnote{76.1} like when a deft bathroom attendant or their apprentice pours bath powder into a bronze dish, sprinkling it little by little with water. They knead it until the ball of bath powder is soaked and saturated with moisture, spread through inside and out; yet no moisture oozes out.\footnote{The kneading is the “placing the mind and keeping it connected”, the water is bliss, while the lack of leaking speaks to the contained interiority of the experience. | Here as elsewhere, water is used as a metaphor for the mind in absorption. Compare \textsanskrit{Bṛhadāraṇyaka} \textsanskrit{Upaniṣad} 4.3.32: “He becomes like water, one, the seer without duality; this is the world of \textsanskrit{Brahmā}.” } In the same way, a mendicant drenches, steeps, fills, and spreads their body with rapture and bliss born of seclusion. There’s no part of the body that’s not spread with rapture and bliss born of seclusion. This, great king, is a fruit of the ascetic life that’s apparent in the present life which is better and finer than the former ones.\footnote{The Buddha has answered \textsanskrit{Ajātasattu}’s question. But he is far from finished. } 

\paragraph*{4.3.2.6. Second Absorption }

Furthermore,\marginnote{77.1} as the placing of the mind and keeping it connected are stilled, a mendicant enters and remains in the second absorption, which has the rapture and bliss born of immersion, with internal clarity and mind at one, without applying the mind and keeping it connected.\footnote{Each \textit{\textsanskrit{jhāna}} begins as the least refined aspect of the previous \textit{\textsanskrit{jhāna}} ends. This is not consciously directed, but describes the natural process of settling. The meditator is now fully confident and no longer needs to apply their mind: it is simply still and fully unified. } They drench, steep, fill, and spread their body with rapture and bliss born of immersion. There’s no part of the body that’s not spread with rapture and bliss born of immersion. 

It’s\marginnote{78.1} like a deep lake fed by spring water. There’s no inlet to the east, west, north, or south, and the heavens would not properly bestow showers from time to time.\footnote{The simile emphasizes the water as bliss, while the lack of inflow expresses containment and unification. } But the stream of cool water welling up in the lake drenches, steeps, fills, and spreads throughout the lake. There’s no part of the lake that’s not spread through with cool water.\footnote{The water welling up is the rapture, which is the uplifting emotional response to the experience of bliss. } 

In\marginnote{78.3} the same way, a mendicant drenches, steeps, fills, and spreads their body with rapture and bliss born of immersion. There’s no part of the body that’s not spread with rapture and bliss born of immersion. This too, great king, is a fruit of the ascetic life that’s apparent in the present life which is better and finer than the former ones. 

\paragraph*{4.3.2.7. Third Absorption }

Furthermore,\marginnote{79.1} with the fading away of rapture, a mendicant enters and remains in the third absorption, where they meditate with equanimity, mindful and aware, personally experiencing the bliss of which the noble ones declare, ‘Equanimous and mindful, one meditates in bliss.’\footnote{The emotional response to bliss matures from the subtle thrill of rapture to the poise of equanimity. Mindfulness is present in all states of deep meditation, but with equanimity it becomes prominent. } They drench, steep, fill, and spread their body with bliss free of rapture. There’s no part of the body that’s not spread with bliss free of rapture. 

It’s\marginnote{80.1} like a pool with blue water lilies, or pink or white lotuses. Some of them sprout and grow in the water without rising above it, thriving underwater. From the tip to the root they’re drenched, steeped, filled, and soaked with cool water. There’s no part of them that’s not soaked with cool water.\footnote{The meditator is utterly immersed in stillness and bliss. } In the same way, a mendicant drenches, steeps, fills, and spreads their body with bliss free of rapture. There’s no part of the body that’s not spread with bliss free of rapture. This too, great king, is a fruit of the ascetic life that’s apparent in the present life which is better and finer than the former ones. 

\paragraph*{4.3.2.8. Fourth Absorption }

Furthermore,\marginnote{81.1} giving up pleasure and pain, and ending former happiness and sadness, a mendicant enters and remains in the fourth absorption, without pleasure or pain, with pure equanimity and mindfulness.\footnote{The emotional poise of equanimity leads to the feeling of pleasure settling into the more subtle neutral feeling. Pain and sadness have been abandoned long before, but are emphasized here as they are subtle counterpart of pleasure. } They sit spreading their body through with pure bright mind. There’s no part of the body that’s not spread with pure bright mind.\footnote{The equanimity of the fourth \textit{\textsanskrit{jhāna}} is not dullness and indifference, but a brilliant and radiant awareness. } 

It’s\marginnote{82.1} like someone sitting wrapped from head to foot with white cloth. There’s no part of the body that’s not spread over with white cloth.\footnote{The white cloth is the purity and brightness of equanimity. The commentary explains this as a person who has just got out of a bath and sits perfectly dry and content. } In the same way, they sit spreading their body through with pure bright mind. There’s no part of the body that’s not spread with pure bright mind. This too, great king, is a fruit of the ascetic life that’s apparent in the present life which is better and finer than the former ones. 

\subsubsection*{4.3.3. The Eight Knowledges }

\paragraph*{4.3.3.1. Knowledge and Vision }

When\marginnote{83.1} their mind has become immersed in \textsanskrit{samādhi} like this—purified, bright, flawless, rid of corruptions, pliable, workable, steady, and imperturbable—they project it and extend it toward knowledge and vision.\footnote{Of the eight kinds of knowledge and vision, only the last is considered indispensable. The fourth \textit{\textsanskrit{jhāna}} is the ideal basis for developing higher knowledges, although elsewhere the canon shows that even the first \textit{\textsanskrit{jhāna}} can be a basis for liberating insight. Without \textit{\textsanskrit{jhāna}}, however, the eightfold path is incomplete and liberating insight is impossible. | The verb \textit{\textsanskrit{abhininnāmeti}} (“extend”) indicates that the meditator comes out of full immersion like a tortoise sticking out its limbs (\href{https://suttacentral.net/sn35.240/en/sujato\#1.7}{SN 35.240:1.7}). } They understand: ‘This body of mine is formed. It’s made up of the four principal states, produced by mother and father, built up from rice and porridge, liable to impermanence, to wearing away and erosion, to breaking up and destruction.\footnote{This is the “coarse” (\textit{\textsanskrit{olārika}}) body. Note that its generation by mother and father contradicts the doctrine of Ajita Kesakambala. The obvious impermanence of the body invites the tempting but fallacious notion that the mind or soul is permanent, which is dispelled by deeper insight. } And this consciousness of mine is attached to it, tied to it.’\footnote{This distinction should not be mistaken for mind-body dualism. These are not fundamental substances but experiences of a meditator. } 

Suppose\marginnote{84.1} there was a beryl gem that was naturally beautiful, eight-faceted, well-worked, transparent, clear, and unclouded, endowed with all good qualities. And it was strung with a thread of blue, yellow, red, white, or golden brown.\footnote{Strung gems were loved in India from the time in the Harappan civilization, millennia before the Buddha. } And someone with clear eyes were to take it in their hand and check it: ‘This beryl gem is naturally beautiful, eight-faceted, well-worked, transparent, clear, and unclouded, endowed with all good qualities. And it’s strung with a thread of blue, yellow, red, white, or golden brown.’ 

In\marginnote{84.6} the same way, when their mind has become immersed in \textsanskrit{samādhi} like this—purified, bright, flawless, rid of corruptions, pliable, workable, steady, and imperturbable—they project it and extend it toward knowledge and vision.\footnote{This form of “knowledge and vision” is rarely mentioned, being found only here, at \href{https://suttacentral.net/dn10/en/sujato\#2.21.3}{DN 10:2.21.3}, and at \href{https://suttacentral.net/mn77/en/sujato\#29.2}{MN 77:29.2}. The next realization, the “mind-made body” is also only found in these three suttas. | The \textsanskrit{Mahāsaṅgīti} edition adds the spurious title \textit{\textsanskrit{vipassanāñāṇa}} (“insight knowledge”) to this section. This term does not appear anywhere in the Pali canon. } This too, great king, is a fruit of the ascetic life that’s apparent in the present life which is better and finer than the former ones. 

\paragraph*{4.3.3.2. Mind-Made Body }

When\marginnote{85.1} their mind has become immersed in \textsanskrit{samādhi} like this—purified, bright, flawless, rid of corruptions, pliable, workable, steady, and imperturbable—they project it and extend it toward the creation of a mind-made body.\footnote{The “mind-made body” is the interior mental representation of the physical body. In ordinary consciousness it is proprioception, which here is enhanced by the power of meditation. The higher powers in Buddhism are regarded as extensions and evolutions of aspects of ordinary experience, not as metaphysical realities separate from the world of mundane experience. } From this body they create another body—formed, mind-made, whole in its major and minor limbs, not deficient in any faculty.\footnote{This is similar to the experience of the “astral body” described by modern spiritualists. Note that it is still “physical” (\textit{\textsanskrit{rūpī}}) even though it is mind-made. This is the subtle (\textit{sukhuma}) body, which is an energetic experience of physical properties by the mind. } 

Suppose\marginnote{86.1} a person was to draw a reed out from its sheath. They’d think: ‘This is the reed, this is the sheath. The reed and the sheath are different things. The reed has been drawn out from the sheath.’ Or suppose a person was to draw a sword out from its scabbard. They’d think: ‘This is the sword, this is the scabbard. The sword and the scabbard are different things. The sword has been drawn out from the scabbard.’ Or suppose a person was to draw a snake out from its slough. They’d think: ‘This is the snake, this is the slough. The snake and the slough are different things. The snake has been drawn out from the slough.’ 

In\marginnote{86.10} the same way, when their mind has become immersed in \textsanskrit{samādhi} like this—purified, bright, flawless, rid of corruptions, pliable, workable, steady, and imperturbable—they project it and extend it toward the creation of a mind-made body. From this body they create another body—formed, mind-made, whole in its major and minor limbs, not deficient in any faculty. This too, great king, is a fruit of the ascetic life that’s apparent in the present life which is better and finer than the former ones. 

\paragraph*{4.3.3.3. Psychic Powers }

When\marginnote{87.1} their mind has become immersed in \textsanskrit{samādhi} like this—purified, bright, flawless, rid of corruptions, pliable, workable, steady, and imperturbable—they project it and extend it toward psychic power.\footnote{Here begin the “six direct knowledges” (\textit{\textsanskrit{chaḷabhiññā}}), which are found commonly throughout the early texts. | “Psychic powers” (\textit{iddhi}) were much cultivated in the Buddha’s day, but the means to acquire them varied: devotion to a god, brutal penances, or magic rituals. The Buddha taught that the mind developed in \textit{\textsanskrit{samādhi}} was capable of things that are normally incomprehensible. } They wield the many kinds of psychic power: multiplying themselves and becoming one again; appearing and disappearing; going unobstructed through a wall, a rampart, or a mountain as if through space; diving in and out of the earth as if it were water; walking on water as if it were earth; flying cross-legged through the sky like a bird; touching and stroking with the hand the sun and moon, so mighty and powerful; controlling the body as far as the realm of divinity.\footnote{Only a few of these are attested as events in the early texts. The most common is the ability to disappear and reappear, exhibited by the Buddha (\href{https://suttacentral.net/an8.30/en/sujato\#2.1}{AN 8.30:2.1}), some disciples (\href{https://suttacentral.net/mn37/en/sujato\#6.1}{MN 37:6.1}), and deities (\href{https://suttacentral.net/mn67/en/sujato\#8.1}{MN 67:8.1}). } 

Suppose\marginnote{88.1} a deft potter or their apprentice had some well-prepared clay. They could produce any kind of pot that they like.\footnote{These similes hark back to the descriptions of the purified mind as pliable and workable. } Or suppose a deft ivory-carver or their apprentice had some well-prepared ivory. They could produce any kind of ivory item that they like. Or suppose a deft goldsmith or their apprentice had some well-prepared gold. They could produce any kind of gold item that they like.\footnote{This simile is extended in detail at \href{https://suttacentral.net/an3.101/en/sujato}{AN 3.101}. } 

In\marginnote{88.4} the same way, when their mind has become immersed in \textsanskrit{samādhi} like this—purified, bright, flawless, rid of corruptions, pliable, workable, steady, and imperturbable—they project it and extend it toward psychic power. This too, great king, is a fruit of the ascetic life that’s apparent in the present life which is better and finer than the former ones. 

\paragraph*{4.3.3.4. Clairaudience }

When\marginnote{89.1} their mind has become immersed in \textsanskrit{samādhi} like this—purified, bright, flawless, rid of corruptions, pliable, workable, steady, and imperturbable—they project it and extend it toward clairaudience.\footnote{“Clairaudience” is a literal rendition of \textit{dibbasota}. The root sense of \textit{dibba} is to “shine” like the bright sky or a divine being. The senses of clarity and divinity are both present. } With clairaudience that is purified and superhuman, they hear both kinds of sounds, human and heavenly, whether near or far.\footnote{The Buddha occasionally used this ability for teaching, as at \href{https://suttacentral.net/mn75/en/sujato\#6.1}{MN 75:6.1}. } 

Suppose\marginnote{90.1} there was a person traveling along the road. They’d hear the sound of drums, clay drums, horns, kettledrums, and tom-toms. They’d think: ‘That’s the sound of drums,’ and ‘that’s the sound of clay drums,’ and ‘that’s the sound of horns, kettledrums, and tom-toms.’\footnote{The simile emphasizes the clarity and distinctness of the sounds. Compare \href{https://suttacentral.net/an4.114/en/sujato}{AN 4.114}: \textit{\textsanskrit{bheripaṇavasaṅkhatiṇavaninnādasaddānaṁ}}. } 

In\marginnote{90.2} the same way, when their mind has become immersed in \textsanskrit{samādhi} like this—purified, bright, flawless, rid of corruptions, pliable, workable, steady, and imperturbable—they project it and extend it toward clairaudience. With clairaudience that is purified and superhuman, they hear both kinds of sounds, human and heavenly, whether near or far. This too, great king, is a fruit of the ascetic life that’s apparent in the present life which is better and finer than the former ones. 

\paragraph*{4.3.3.5. Comprehending the Minds of Others }

When\marginnote{91.1} their mind has become immersed in \textsanskrit{samādhi} like this—purified, bright, flawless, rid of corruptions, pliable, workable, steady, and imperturbable—they project it and extend it toward comprehending the minds of others.\footnote{Note that the Indic idiom is not the “reading” of minds, which suggests hearing the words spoken in inner dialogue. While this is exhibited by the Buddha (eg. \href{https://suttacentral.net/an8.30/en/sujato\#2.1}{AN 8.30:2.1}), the main emphasis is on the comprehension of the overall state of mind. } They understand the minds of other beings and individuals, having comprehended them with their own mind. They understand mind with greed as ‘mind with greed’, and mind without greed as ‘mind without greed’. They understand mind with hate … mind without hate … mind with delusion … mind without delusion … constricted mind … scattered mind … expansive mind … unexpansive mind … mind that is not supreme … mind that is supreme … immersed mind … unimmersed mind … freed mind … They understand unfreed mind as ‘unfreed mind’. 

Suppose\marginnote{92.1} there was a woman or man who was young, youthful, and fond of adornments, and they check their own reflection in a clean bright mirror or a clear bowl of water. If they had a spot they’d know ‘I have a spot,’ and if they had no spots they’d know ‘I have no spots.’\footnote{Again the simile emphasizes how clear and direct the experience is. Without deep meditation, we have some intuitive sense for the minds of others, but it is far from clear. } In the same way, when their mind has become immersed in \textsanskrit{samādhi} like this—purified, bright, flawless, rid of corruptions, pliable, workable, steady, and imperturbable—they project it and extend it toward comprehending the minds of others. They understand the minds of other beings and individuals, having comprehended them with their own mind. This too, great king, is a fruit of the ascetic life that’s apparent in the present life which is better and finer than the former ones. 

\paragraph*{4.3.3.6. Recollection of Past Lives }

When\marginnote{93.1} their mind has become immersed in \textsanskrit{samādhi} like this—purified, bright, flawless, rid of corruptions, pliable, workable, steady, and imperturbable—they project it and extend it toward recollection of past lives.\footnote{Here begins the “three knowledges” (\textit{\textsanskrit{tevijjā}}), a subset of the six direct knowledges. The first two of these play an important role in deepening understanding of the nature of suffering in \textit{\textsanskrit{saṁsāra}}. While they are not necessary for those whose wisdom is keen, they are helpful. } They recollect many kinds of past lives, that is, one, two, three, four, five, ten, twenty, thirty, forty, fifty, a hundred, a thousand, a hundred thousand rebirths; many eons of the world contracting, many eons of the world expanding, many eons of the world contracting and expanding. They remember: ‘There, I was named this, my clan was that, I looked like this, and that was my food. This was how I felt pleasure and pain, and that was how my life ended. When I passed away from that place I was reborn somewhere else. There, too, I was named this, my clan was that, I looked like this, and that was my food. This was how I felt pleasure and pain, and that was how my life ended. When I passed away from that place I was reborn here.’ And so they recollect their many kinds of past lives, with features and details.\footnote{Empowered by the fourth \textit{\textsanskrit{jhāna}}, memory breaks through the veil of birth and death, revealing the vast expanse of time and dispelling the illusion that there is any place of eternal rest or sanctuary in the cycle of transmigration. The knowledge of these events is not hazy or murky, but clear and precise, illuminated by the brilliance of purified consciousness. } 

Suppose\marginnote{94.1} a person was to leave their home village and go to another village. From that village they’d go to yet another village. And from that village they’d return to their home village. They’d think: ‘I went from my home village to another village. There I stood like this, sat like that, spoke like this, or kept silent like that. From that village I went to yet another village. There too I stood like this, sat like that, spoke like this, or kept silent like that. And from that village I returned to my home village.’\footnote{The word for “past life” is \textit{\textsanskrit{pubbanivāsa}}, literally “former home”, and the imagery of houses is found in the second of the three knowledges as well. Recollection of past lives is as fresh and clear as the memory of a recent journey. } 

In\marginnote{94.2} the same way, when their mind has become immersed in \textsanskrit{samādhi} like this—purified, bright, flawless, rid of corruptions, pliable, workable, steady, and imperturbable—they project it and extend it toward recollection of past lives. This too, great king, is a fruit of the ascetic life that’s apparent in the present life which is better and finer than the former ones. 

\paragraph*{4.3.3.7. Clairvoyance }

When\marginnote{95.1} their mind has become immersed in \textsanskrit{samādhi} like this—purified, bright, flawless, rid of corruptions, pliable, workable, steady, and imperturbable—they project it and extend it toward knowledge of the death and rebirth of sentient beings. With clairvoyance that is purified and superhuman, they see sentient beings passing away and being reborn—inferior and superior, beautiful and ugly, in a good place or a bad place. They understand how sentient beings are reborn according to their deeds: ‘These dear beings did bad things by way of body, speech, and mind. They denounced the noble ones; they had wrong view; and they chose to act out of that wrong view. When their body breaks up, after death, they’re reborn in a place of loss, a bad place, the underworld, hell. These dear beings, however, did good things by way of body, speech, and mind. They never denounced the noble ones; they had right view; and they chose to act out of that right view. When their body breaks up, after death, they’re reborn in a good place, a heavenly realm.’ And so, with clairvoyance that is purified and superhuman, they see sentient beings passing away and being reborn—inferior and superior, beautiful and ugly, in a good place or a bad place. They understand how sentient beings are reborn according to their deeds.\footnote{Here knowledge extends to the rebirths of others as well as oneself. Even more significant, it brings in the understanding of cause and effect; \emph{why} rebirth happens the way it does. Such knowledge, however, is not infallible, as the Buddha warns in \href{https://suttacentral.net/dn1/en/sujato\#2.5.3}{DN 1:2.5.3} and \href{https://suttacentral.net/mn136/en/sujato}{MN 136}. The experience is one thing; the inferences drawn from it are another. One should draw conclusions only tentatively, after long experience. | “Clairvoyance” renders \textit{dibbacakkhu} (“celestial eye”), for which see \textsanskrit{Chāndogya} \textsanskrit{Upaniṣad} 8.12.5, “the mind is (the self’s) celestial eye” (\textit{mano’sya \textsanskrit{daivaṁ} \textsanskrit{cakṣuḥ}}). } 

Suppose\marginnote{96.1} there was a stilt longhouse at the central square. A person with clear eyes standing there might see people entering and leaving a house, walking along the streets and paths, and sitting at the central square. They’d think: ‘These are people entering and leaving a house, walking along the streets and paths, and sitting at the central square.’\footnote{This simile is also found at \href{https://suttacentral.net/dn10/en/sujato\#2.33.1}{DN 10:2.33.1}. The Majjhima employs a slightly different simile (\href{https://suttacentral.net/mn39/en/sujato\#20.3}{MN 39:20.3}, \href{https://suttacentral.net/mn77/en/sujato\#35.2}{MN 77:35.2}, \href{https://suttacentral.net/mn130/en/sujato\#2.1}{MN 130:2.1}). | \textit{\textsanskrit{Pāsāda}} is often translated as “palace” or “mansion”, but in early Pali it meant a “stilt longhouse”. As here, it is an elevated place from which one can observe the street below. } 

In\marginnote{96.2} the same way, when their mind has become immersed in \textsanskrit{samādhi} like this—purified, bright, flawless, rid of corruptions, pliable, workable, steady, and imperturbable—they project it and extend it toward knowledge of the death and rebirth of sentient beings. This too, great king, is a fruit of the ascetic life that’s apparent in the present life which is better and finer than the former ones. 

\paragraph*{4.3.3.8. Ending of Defilements }

When\marginnote{97.1} their mind has become immersed in \textsanskrit{samādhi} like this—purified, bright, flawless, rid of corruptions, pliable, workable, steady, and imperturbable—they project it and extend it toward knowledge of the ending of defilements.\footnote{This is the experience of awakening that is the true goal of the Buddhist path. The defilements—properties of the mind that create suffering—have been curbed by the practice of ethics and suppressed by the power of \textit{\textsanskrit{jhāna}}. Here they are eliminated forever. } They truly understand: ‘This is suffering’ … ‘This is the origin of suffering’ … ‘This is the cessation of suffering’ … ‘This is the practice that leads to the cessation of suffering’.\footnote{These are the four noble truths, which form the main content of the Buddha’s first sermon. They are the overarching principle into which all other teachings fall. The initial realization of the four noble truths indicates the first stage of awakening, stream-entry. } They truly understand: ‘These are defilements’ … ‘This is the origin of defilements’ … ‘This is the cessation of defilements’ … ‘This is the practice that leads to the cessation of defilements’.\footnote{The application of the four noble truths to defilements indicates that this is the final stage of awakening, perfection (or “arahantship”, \textit{arahatta}). | Many translators use “defilement” to render \textit{kilesa}, but since \textit{kilesa} appears only rarely in the early texts, I use “defilement” for \textit{\textsanskrit{āsava}}. Both terms refer to a stain, corruption, or pollution in the mind. } Knowing and seeing like this, their mind is freed from the defilements of sensuality, desire to be reborn, and ignorance.\footnote{\textit{\textsanskrit{Bhavāsava}} is the defilement that craves to continue life in a new birth. } When they’re freed, they know they’re freed.\footnote{This is a reflective awareness of the fact of awakening. The meditator reviews their mind and sees that it is free from all forces that lead to suffering. } They understand: ‘Rebirth is ended, the spiritual journey has been completed, what had to be done has been done, there is nothing further for this place.’\footnote{This is a standard declaration of full awakening in the suttas, said both of the Buddha and of any arahant (“perfected one”). Each of the four phrases illustrates a cardinal principle of awakening. (1) Further transmigration through rebirths has come to an end due to the exhaustion (\textit{\textsanskrit{khīṇa}}) of that which propels rebirth, namely deeds motivated by craving. (2) The eightfold path has been developed fully in all respects. (3) All functions relating to the four noble truths have been completed, namely: understanding suffering, letting go craving, witnessing extinguishment, and developing the path. (4) Extinguishment is final, with no falling back to this or any other state of existence. | For “state of existence” (literally “thusness”, \textit{itthatta}), see \href{https://suttacentral.net/dn15/en/sujato\#21.4}{DN 15:21.4}. } 

Suppose\marginnote{98.1} that in a mountain glen there was a lake that was transparent, clear, and unclouded. A person with clear eyes standing on the bank would see the clams and mussels, and pebbles and gravel, and schools of fish swimming about or staying still. They’d think: ‘This lake is transparent, clear, and unclouded. And here are the clams and mussels, and pebbles and gravel, and schools of fish swimming about or staying still.’\footnote{Once again the pool of water represents the mind, but now the meditator is not immersed in the experience, but looks back and reviews it objectively. } 

In\marginnote{98.2} the same way, when their mind has become immersed in \textsanskrit{samādhi} like this—purified, bright, flawless, rid of corruptions, pliable, workable, steady, and imperturbable—they project it and extend it toward knowledge of the ending of defilements. This too, great king, is a fruit of the ascetic life that’s apparent in the present life which is better and finer than the former ones. And, great king, there is no other fruit of the ascetic life apparent in the present life which is better and finer than this.”\footnote{The Buddha roars his lion’s roar. His teaching leads not just to some benefits, but to the highest benefits that are possible. } 

\section*{5. \textsanskrit{Ajātasattu} Declares Himself a Lay Follower }

When\marginnote{99.1} the Buddha had spoken, King \textsanskrit{Ajātasattu} said to him, “Excellent, sir! Excellent!\footnote{The king’s distress has been alleviated by the Buddha’s uplifting words. } As if he were righting the overturned, or revealing the hidden, or pointing out the path to the lost, or lighting a lamp in the dark so people with clear eyes can see what’s there, the Buddha has made the teaching clear in many ways. I go for refuge to the Buddha, to the teaching, and to the mendicant \textsanskrit{Saṅgha}.\footnote{This is the standard form in which lay people went for refuge. It is not something that the Buddha required, but a spontaneous act of inspiration. Conventionally, it indicates that someone is a “Buddhist”. Today Theravadins recite the going for refuge thrice, but in the early texts this is found only as the ordination for novices (\href{https://suttacentral.net/pli-tv-kd1/en/sujato\#12.4.1}{Kd 1:12.4.1}). } From this day forth, may the Buddha remember me as a lay follower who has gone for refuge for life. 

I\marginnote{99.6} have made a mistake, sir. It was foolish, stupid, and unskillful of me to take the life of my father, a just and principled king, for the sake of authority.\footnote{The king, unprompted, makes an astonishing confession. To say it in such a public forum, before a spiritual community and his own retinue, displays courage and integrity. } Please, sir, accept my mistake for what it is, so I will restrain myself in future.”\footnote{The Buddha hears his confession, but it is \textsanskrit{Ajātasattu}’s responsibility to do better. } 

“Indeed,\marginnote{100.1} great king, you made a mistake. It was foolish, stupid, and unskillful of you to take the life of your father, a just and principled king, for the sake of sovereignty.\footnote{Even before such a dangerous and emotionally volatile king, the Buddha does not mince words. The Buddha makes a point to acknowledge what the king had done, without dismissing it and thereby discounting the inner work he had achieved to get to this point. } But since you have recognized your mistake for what it is, and have dealt with it properly, I accept it. For it is growth in the training of the Noble One to recognize a mistake for what it is, deal with it properly, and commit to restraint in the future.”\footnote{Confession does not erase the past, but it does set a better course for the future. This confession is similar to that done by monastics when they have broken Vinaya rules. } 

When\marginnote{101.1} the Buddha had spoken, King \textsanskrit{Ajātasattu} said to him, “Well, now, sir, I must go. I have many duties, and much to do.” 

“Please,\marginnote{101.3} great king, go at your convenience.” 

Then\marginnote{101.4} the king, having approved and agreed with what the Buddha said, got up from his seat, bowed, and respectfully circled him, keeping him on his right, before leaving. 

Soon\marginnote{102.1} after the king had left, the Buddha addressed the mendicants, “The king is broken, mendicants,\footnote{These terms are commonly used in the context of keeping precepts (eg. \href{https://suttacentral.net/an3.50/en/sujato\#4.1}{AN 3.50:4.1}). The Buddha is supportive when he speaks with \textsanskrit{Ajātasattu}, but does not hide the severity of his crime. } he is ruined. If he had not taken the life of his father, a just and principled king, the stainless, immaculate vision of the Dhamma would have arisen in him in that very seat.”\footnote{The killing of one’s father is one of five “incurable” acts that doom a person to hell in the next life (\href{https://suttacentral.net/an5.129/en/sujato\#1.3}{AN 5.129:1.3}). If he had not done so, he would have become a stream-enterer. } 

That\marginnote{102.5} is what the Buddha said. Satisfied, the mendicants approved what the Buddha said. 

%
\chapter*{{\suttatitleacronym DN 3}{\suttatitletranslation With Ambaṭṭha }{\suttatitleroot Ambaṭṭhasutta}}
\addcontentsline{toc}{chapter}{\tocacronym{DN 3} \toctranslation{With Ambaṭṭha } \tocroot{Ambaṭṭhasutta}}
\markboth{With Ambaṭṭha }{Ambaṭṭhasutta}
\extramarks{DN 3}{DN 3}

\scevam{So\marginnote{1.1.1} I have heard.\footnote{This sutta marks a turning point where the Buddha’s teachings were embraced by the leading brahmin \textsanskrit{Pokkharasāti}. The suttas that follow reverberate with the consequences of this encounter. He was one of the most influential brahmins of his time, although the Buddha elsewhere denied that he had any special knowledge (\href{https://suttacentral.net/mn99/en/sujato\#15.5}{MN 99:15.5}). Brahmanical texts confirm that he was a real person, an influential teacher around the time of the Buddha known as \textsanskrit{Pauṣkarasādi} in Sanskrit. He is cited on grammar by \textsanskrit{Kātyāyana} and \textsanskrit{Patañjali}, and in the \textsanskrit{Taittirīya}-\textsanskrit{prātiśākhya}; on allowable food and theft in the Āpastamba Dharmasūtra; and on Vedic ritual in the \textsanskrit{Śāṅkhāyana}-\textsanskrit{Āraṇyaka}. His name identifies him as descended from a man of \textsanskrit{Puṣkarāvati}, capital of \textsanskrit{Gandhāra}. \href{https://suttacentral.net/mn99/en/sujato\#10.3}{MN 99:10.3} clarifies that he is of the \textsanskrit{Upamañña} lineage. } }At one time the Buddha was wandering in the land of the Kosalans together with a large \textsanskrit{Saṅgha} of five hundred mendicants when he arrived at a village of the Kosalan brahmins named \textsanskrit{Icchānaṅgala}.\footnote{\textsanskrit{Icchānaṅgala} was a center east of \textsanskrit{Sāvatthī} for the innovative brahmins of the Kosala region. } He stayed in a forest near \textsanskrit{Icchānaṅgala}. 

\section*{1. The Section on \textsanskrit{Pokkharasāti} }

Now\marginnote{1.2.1} at that time the brahmin \textsanskrit{Pokkharasāti} was living in \textsanskrit{Ukkaṭṭhā}. It was a crown property given by King Pasenadi of Kosala, teeming with living creatures, full of hay, wood, water, and grain, a royal park endowed to a brahmin.\footnote{\textsanskrit{Ukkaṭṭhā} is mentioned only rarely, and always in the context of extraordinary teachings and events that emphasize the cosmic grandeur of the Buddha against the brahmins (\href{https://suttacentral.net/dn14/en/sujato\#3.29.1}{DN 14:3.29.1}, \href{https://suttacentral.net/mn1/en/sujato\#1.2}{MN 1:1.2}, \href{https://suttacentral.net/mn49/en/sujato\#2.1}{MN 49:2.1}). Sanskrit sources call it a \textit{\textsanskrit{droṇamukha}}, a leading market town accessible by land and water (\textsanskrit{Divyāvadāna} 319.010). At \href{https://suttacentral.net/mn99/en/sujato\#10.3}{MN 99:10.3} \textsanskrit{Pokkharasāti} is said to be “of the Subhaga Forest”. | “Royal park” is \textit{\textsanskrit{rājadāya}} (cp. \textit{\textsanskrit{migadāya}}, “deer park”). | A \textit{brahmadeyya} is a gift of land by a king to a brahmin, which was an outstanding feature of Indian feudalism. } \textsanskrit{Pokkharasāti} heard: 

“It\marginnote{1.2.3} seems the ascetic Gotama—a Sakyan, gone forth from a Sakyan family—has arrived at \textsanskrit{Icchānaṅgala} and is staying in a forest nearby. He has this good reputation: ‘That Blessed One is perfected, a fully awakened Buddha, accomplished in knowledge and conduct, holy, knower of the world, supreme guide for those who wish to train, teacher of gods and humans, awakened, blessed.’ He has realized with his own insight this world—with its gods, \textsanskrit{Māras}, and divinities, this population with its ascetics and brahmins, gods and humans—and he makes it known to others.\footnote{Contrast with his rejection of this possibility at \href{https://suttacentral.net/mn99/en/sujato\#10.7}{MN 99:10.7}. } He proclaims a teaching that is good in the beginning, good in the middle, and good in the end, meaningful and well-phrased. And he reveals a spiritual practice that’s entirely full and pure. It’s good to see such perfected ones.”\footnote{\textsanskrit{Pokkharasāti} does not care whether the Buddha identified as a follower of the Vedas. The wise do not concern themselves with religious identity. } 

\section*{2. The Student \textsanskrit{Ambaṭṭha} }

Now\marginnote{1.3.1} at that time \textsanskrit{Pokkharasāti} had a student named \textsanskrit{Ambaṭṭha} as his resident pupil. He was one who recited and remembered the hymns, and had mastered in the three Vedas, together with their vocabularies and ritual performance, their phonology and word classification, and the testaments as fifth. He knew them word-by-word, and their grammar. He was well versed in cosmology and the marks of a great man. He had been authorized as a master in his own tutor’s scriptural heritage of the three Vedas with the words:\footnote{“Vocabularies” is \textit{\textsanskrit{nighaṇḍu}} (Sanskrit \textit{\textsanskrit{nighaṇṭu}}), known from the Nirukta of \textsanskrit{Yāska}. | \textit{\textsanskrit{Keṭubha}} lacks an obvious Sanskrit form. The commentary explains, “The study of proper and improper actions for the assistance of poets.” This suggests a connection with ritual performance, which is the special area of the Śatapatha \textsanskrit{Brāhmaṇa}. There we often find phrases such as \textit{\textsanskrit{kṛtam} bhavati}, “it is performed”, of which \textit{\textsanskrit{keṭubha}} is perhaps a contraction. | \textit{Akkhara} (literally “syllable”) is explained by the commentary as \textit{\textsanskrit{sikkhā}} (Sanskrit \textit{\textsanskrit{śikṣā}}), which is the study of pronunciation. This can be traced back to \textsanskrit{Pāṇinī}, and is sometimes referred to as \textit{\textsanskrit{akṣara}-\textsanskrit{samāmnāya}}, “collation of syllables”. | \textit{Pabheda} is found in Buddhist Sanskrit texts as \textit{padaprabheda}, “classification of words”, such as into the different parts of speech. The commentary identifies it with \textit{nirutti}. | \textit{Padaka} is one skilled in the \textit{\textsanskrit{padapāṭha}} recitation of Vedas, which separates the individual words. | For “testaments” (\textit{\textsanskrit{itihāsa}}) see \textit{\textsanskrit{itihāsa}-\textsanskrit{purāṇa}} in Śatapatha \textsanskrit{Brāhmaṇa} 11.5.6.8, explained by the commentator there as legends of creation and olden times. | For “cosmology” (\textit{\textsanskrit{lokāyata}}), see note on \href{https://suttacentral.net/dn1/en/sujato\#1.25.2}{DN 1:1.25.2}. | For “authorized as a master” (\textit{\textsanskrit{anuññātapaṭiññāta}}) see \href{https://suttacentral.net/mn98/en/sujato\#7.1}{MN 98:7.1} and \href{https://suttacentral.net/snp3.9/en/sujato\#6.1}{Snp 3.9:6.1}. | For “scriptural heritage of the three Vedas” (\textit{tevijjake \textsanskrit{pāvacane}}) see \href{https://suttacentral.net/mn95/en/sujato\#12.2}{MN 95:12.2}. } “What I know, you know.\footnote{Almost the same words are spoken to the bodhisatta by his first teachers, \textsanskrit{Āḷāra} \textsanskrit{Kālāma} and Uddaka \textsanskrit{Rāmaputta} (\href{https://suttacentral.net/mn36/en/sujato\#14.9}{MN 36:14.9}). This connects \textsanskrit{Pokkharasāti} with \textsanskrit{Āḷāra} \textsanskrit{Kālāma} and Uddaka \textsanskrit{Rāmaputta}, and suggests that the anointing of a talented student in this way was a regular practice of wise brahmins. } And what you know, I know.” 

Then\marginnote{1.4.1} \textsanskrit{Pokkharasāti} addressed \textsanskrit{Ambaṭṭha}, “Dear \textsanskrit{Ambaṭṭha}, the ascetic Gotama—a Sakyan, gone forth from a Sakyan family—has arrived at \textsanskrit{Icchānaṅgala} and is staying in a forest nearby. He has this good reputation … It’s good to see such perfected ones. Please, dear \textsanskrit{Ambaṭṭha}, go to the ascetic Gotama and find out whether or not he lives up to his reputation. Through you I shall learn about Mister Gotama.”\footnote{Following PTS and BJT editions of the parallel phrase at \href{https://suttacentral.net/mn91/en/sujato\#4.9}{MN 91:4.9}, which read \textit{\textsanskrit{tayā}} for \textit{\textsanskrit{tathā}}. } 

“But\marginnote{1.5.1} sir, how shall I find out whether or not the ascetic Gotama lives up to his reputation?” 

“Dear\marginnote{1.5.2} \textsanskrit{Ambaṭṭha}, the thirty-two marks of a great man have been handed down in our hymns. A great man who possesses these has only two possible destinies, no other.\footnote{The thirty-two marks are detailed in \href{https://suttacentral.net/dn14/en/sujato\#1.32.7}{DN 14:1.32.7}, \href{https://suttacentral.net/dn30/en/sujato\#1.2.4}{DN 30:1.2.4}, and \href{https://suttacentral.net/mn91/en/sujato\#9.1}{MN 91:9.1}. In Buddhist texts they are presented as the fulfillment of Brahmanical prophecy, but they are not found in any Brahmanical texts of the Buddha’s time. However, later astrological texts such as the \textsanskrit{Gārgīyajyotiṣa} (1st century BCE?) and \textsanskrit{Bṛhatsaṁhitā} (6th century CE?) contain references to many of these marks, albeit in a different context, so it seems likely the Buddhist texts are drawing on now-lost Brahmanical scriptures. | The notion of a two-fold course for a great hero—worldly success or spiritual—can be traced back as far as the epic of Gilgamesh. } If he stays at home he becomes a king, a wheel-turning monarch, a just and principled king. His dominion extends to all four sides, he achieves stability in the country, and he possesses the seven treasures.\footnote{The idea of the wheel-turning monarch draws from the Vedic horse sacrifice, which establishes the authority of a king from sea to sea. The Buddhist telling is divested of all coarse and violent elements. The wheeled chariot gave military supremacy to the ancient Indo-Europeans, allowing them to spread from their ancient homeland north of the Black Sea. In Buddhism, the wheel, which also has solar connotations, symbolizes unstoppable power. For a legendary account of such a king, see the \textsanskrit{Mahāsudassanasutta} (\href{https://suttacentral.net/dn17/en/sujato}{DN 17}). } He has the following seven treasures: the wheel, the elephant, the horse, the jewel, the woman, the householder, and the commander as the seventh treasure. He has over a thousand sons who are valiant and heroic, crushing the armies of his enemies.\footnote{The sacrificial horse on its journey across the land is protected by a hundred sons. } After conquering this land girt by sea, he reigns by principle, without rod or sword. But if he goes forth from the lay life to homelessness, he becomes a perfected one, a fully awakened Buddha, who draws back the veil from the world. But, dear \textsanskrit{Ambaṭṭha}, I am the one who gives the hymns,\footnote{The relation between \textsanskrit{Pokkharasāti} and \textsanskrit{Ambaṭṭha} is similar to that between the Buddha and his followers. They share the same understanding, but the Buddha is distinguished as the teacher. } and you are the one who receives them.” 

“Yes,\marginnote{1.6.1} sir,” replied \textsanskrit{Ambaṭṭha}. He got up from his seat, bowed, and respectfully circled \textsanskrit{Pokkharasāti}, keeping him to his right. He mounted a chariot drawn by mares and, together with several young students, set out for the forest near \textsanskrit{Icchānaṅgala}.\footnote{In this sutta, \textit{\textsanskrit{māṇava}} is always applied to \textsanskrit{Ambaṭṭha} and \textit{\textsanskrit{māṇavaka}} to the rest. It seems that the diminutive \textit{\textsanskrit{māṇavaka}} means “young student”. | There are said to be \textit{sambahula} students, a word that is often translated as “many”. But later we see that they all fit inside the Buddha’s hut, so the sense must be “several”. } He went by carriage as far as the terrain allowed, then descended and entered the monastery on foot. 

At\marginnote{1.7.1} that time several mendicants were walking mindfully in the open air.\footnote{This is the practice of walking meditation. Meditators pace mindfully up and down a smooth path, keeping attention on their body. } Then the student \textsanskrit{Ambaṭṭha} went up to those mendicants and said, “Good sirs, where is Mister Gotama at present?\footnote{\textit{Bho} is a respectful term of address used by brahmins. The forms of address used in Pali are complex, and it is rarely possible to map them to modern English with any precision. } For we have come here to see him.”\footnote{The parallel passage at \href{https://suttacentral.net/mn35/en/sujato\#7.4}{MN 35:7.4} has a different phrase here. } 

Then\marginnote{1.8.1} those mendicants thought, “This \textsanskrit{Ambaṭṭha} is from a well-known family, and he is the pupil of the well-known brahmin \textsanskrit{Pokkharasāti}. The Buddha won’t mind having a discussion together with such gentlemen.”\footnote{The term \textit{kulaputta} (literally, “son of a family”) typically refers to someone from a well-to-do or respected family, a “gentleman”. It is a gendered term which assumes the social status of men. } 

They\marginnote{1.8.4} said to \textsanskrit{Ambaṭṭha}, “\textsanskrit{Ambaṭṭha}, that’s his dwelling, with the door closed. Approach it quietly, without hurrying; go onto the porch, clear your throat, and knock on the door-panel. The Buddha will open the door.”\footnote{The introduction has told us that the Buddha was staying in a forest at this time. Nonetheless, this was not a wilderness, but was developed enough to have huts with latched doors. } 

So\marginnote{1.9.1} he approached the Buddha’s dwelling, cleared his throat and knocked on the door-panel, and the Buddha opened the door. \textsanskrit{Ambaṭṭha} and the young students entered the dwelling. The young students exchanged greetings with the Buddha, and when the greetings and polite conversation were over, sat down to one side. But while the Buddha was sitting, \textsanskrit{Ambaṭṭha} spoke some polite words or other while walking around or standing. 

So\marginnote{1.9.4} the Buddha said to him, “\textsanskrit{Ambaṭṭha}, is this how you hold a discussion with elderly and senior brahmins, the tutors of tutors: walking around or standing while I’m sitting, speaking some polite words or other?”\footnote{The Buddha draws attention to \textsanskrit{Ambaṭṭha}’s rude behavior. Throughout the suttas, the manner in which people greet the Buddha gives us a hint as to their attitudes and qualities. } 

\subsection*{2.1. The First Use of the Word “Primitives” }

“No,\marginnote{1.10.1} Mister Gotama. For it is proper for one brahmin to converse with another while both are walking, standing, sitting, or lying down. But as to these shavelings, fake ascetics, primitives, black spawn from the feet of our kinsman, I converse with them as I do with Mister Gotama.”\footnote{Note the racial connotations of using \textit{\textsanskrit{kaṇha}} (“black”) as a slur. The brahmin caste hailed from the (relatively) fair-skinned Indo-Europeans who entered India from the north. Vedic texts indicate that there was Brahmanical prejudice against dark-skinned natives, but also that they were assimilated and raised to positions of honor. } 

“But\marginnote{1.11.1} \textsanskrit{Ambaṭṭha}, you must have come here for some purpose. You should focus on that. Though this \textsanskrit{Ambaṭṭha} is unqualified, he thinks he’s qualified. What is that but lack of qualifications?”\footnote{\textsanskrit{Ambaṭṭha} is “qualified” (\textit{vusita}) in scripture, but far from “qualified” in spiritual development. \textit{Vusita} is normally an expression of arahantship: \textit{\textsanskrit{vusitaṁ} \textsanskrit{brahmacariyaṁ}} (“the spiritual journey has been completed”). } 

When\marginnote{1.12.1} he said this, \textsanskrit{Ambaṭṭha} became angry and upset with the Buddha because of being described as unqualified. He even attacked and badmouthed the Buddha himself, saying, “The ascetic Gotama will be worsted!” He said to the Buddha, “Mister Gotama, the Sakyans are rude, harsh, touchy, and argumentative.\footnote{The PTS reading \textit{rabhasa} means “violent, aggressive”. But the commentary reads \textit{bhassa}, explained as “speaking much”. Moreover, the story below does not demonstrate violence. } Primitive they are, and primitive they remain! They don’t honor, respect, revere, worship, or venerate brahmins.\footnote{\textsanskrit{Ambaṭṭha} despises the Sakyans as “primitives” (\textit{ibbha}) who do not respect Vedic culture. The word \textit{ibbha} (“primitive”) stems from a non-Aryan word for “elephant” (\textit{ibha}). It originally referred to the native inhabitants who tamed elephants; see eg. \textsanskrit{Chāndogya} \textsanskrit{Upaniṣad} 1.10. At \href{https://suttacentral.net/snp3.1/en/sujato\#18.4}{Snp 3.1:18.4} the Buddha describes his own people as “natives” (\textit{niketino}), those who have a long connection with the land. } It is neither proper nor appropriate that the Sakyans—primitives that they are—don’t honor, respect, revere, worship, or venerate brahmins.” 

And\marginnote{1.12.9} that’s how \textsanskrit{Ambaṭṭha} denigrated the Sakyans with the word “primitives” for the first time. 

\subsection*{2.2. The Second Use of the Word “Primitives” }

“But\marginnote{1.13.1} \textsanskrit{Ambaṭṭha}, how have the Sakyans wronged you?” 

“This\marginnote{1.13.2} one time, Mister Gotama, I went to Kapilavatthu on some business for my tutor, the brahmin \textsanskrit{Pokkharasāti}. I approached the Sakyans in their town hall. Now at that time several Sakyans and Sakyan princes were sitting on high seats, poking each other with their fingers, giggling and playing together. In fact, they even presumed to giggle at me, and didn’t invite me to a seat. It is neither proper nor appropriate that the Sakyans—primitives that they are—don’t honor, respect, revere, worship, or venerate brahmins.” 

And\marginnote{1.13.6} that’s how \textsanskrit{Ambaṭṭha} denigrated the Sakyans with the word “primitives” for the second time. 

\subsection*{2.3. The Third Use of the Word “Primitives” }

“Even\marginnote{1.14.1} a little quail, \textsanskrit{Ambaṭṭha}, speaks as she likes in her own nest. Kapilavatthu is the Sakyans’ own place, \textsanskrit{Ambaṭṭha}. It’s not worthy of the Venerable \textsanskrit{Ambaṭṭha} to lose his temper over such a small thing.”\footnote{The Buddha’s use of \textit{\textsanskrit{āyasmā}} is noteworthy here: he is taking a conciliatory tone. } 

“Mister\marginnote{1.14.3} Gotama, there are these four classes: aristocrats, brahmins, peasants, and menials. Three of these classes—aristocrats, peasants, and menials—in fact succeed only when serving the brahmins. It is neither proper nor appropriate that the Sakyans—primitives that they are—don’t honor, respect, revere, worship, or venerate brahmins.” 

And\marginnote{1.14.9} that’s how \textsanskrit{Ambaṭṭha} denigrated the Sakyans with the word “primitives” for the third time. 

\subsection*{2.4. The Word “Son of a Slavegirl” is Used }

Then\marginnote{1.15.1} it occurred to the Buddha, “This \textsanskrit{Ambaṭṭha} puts the Sakyans down way too much by calling them primitives. Why don’t I ask him about his own clan?” 

So\marginnote{1.15.3} the Buddha said to him, “What is your clan, \textsanskrit{Ambaṭṭha}?”\footnote{The \textsanskrit{Ambaṭṭhas} were a people in the north-west of greater India (eg. \textsanskrit{Mahābhārata} 7.4.5c, 7.132.23a). They were evidently the Abastanians whose rout at the hands of Alexander is recorded by Arrian (\emph{The Anabasis of Alexander}, chapter 15). They were probably located near what is today the northern Sindh province in Pakistan. Later texts such as \textsanskrit{Manusmṛti} 1.8 say that an \textit{\textsanskrit{ambaṣṭha}} is born of a brahmin father and \textit{\textsanskrit{vaiśya}} mother. } 

“I\marginnote{1.15.5} am a \textsanskrit{Kaṇhāyana}, Mister Gotama.”\footnote{\textit{\textsanskrit{Kaṇhāyana}} means “descendant of the dark one (\textit{\textsanskrit{kaṇha}})”. Since no clan of that name is attested it is perhaps a confusion with the \textsanskrit{Kāṇvāyanas} of Rig Veda 8.55.4. But the confusion, if it is such, has an old history, for Rig Veda 1.117.8 refers to “Dark \textsanskrit{Kaṇva}” (\textit{\textsanskrit{Śyāva} \textsanskrit{Kaṇva}}). } 

“But,\marginnote{1.15.6} recollecting the ancient name and clan of your mother and father, the Sakyans were the children of the masters, while you’re descended from the son of a slavegirl of the Sakyans.\footnote{Normally I take \textit{ayyaputta} as a simple honorific, but here the sense is not that the Sakyans were the masters, but were descended from them. } But the Sakyans regard King \textsanskrit{Okkāka} as their grandfather.\footnote{\textsanskrit{Okkāka} (Sanskrit \textsanskrit{Ikṣvāku}) was the legendary son of the first man, Manu, and the founder of the solar dynasty of Kosala. It is a Munda name, which may be associated with the introduction of cane sugar (\textit{\textsanskrit{ikṣuḥ}}) from eastern Asia, a theory endorsed by the 9th century Jain scholar Jinasena (Natubhai Shah, \emph{Jainism, the World of Conquerors}, 2004, vol. 1, pg. 15). } 

Once\marginnote{1.15.8} upon a time, King \textsanskrit{Okkāka}, wishing to divert the royal succession to the son of his most beloved queen, banished the elder princes from the realm—\textsanskrit{Okkāmukha}, \textsanskrit{Karakaṇḍa}, Hatthinika, and \textsanskrit{Sinisūra}. They made their home beside a lotus pond on the slopes of the Himalayas, where there was a large grove of \textit{sakhua} trees.\footnote{The words for “teak” (\textit{\textsanskrit{sāka}}) and “sal” (\textit{\textsanskrit{sāla}}) have evidently been confused from the Munda root \textit{sarja} (both appear at \href{https://suttacentral.net/mn93/en/sujato\#11.6}{MN 93:11.6}). But teak does not grow so far north, so the sal must be meant here. To maintain the pun I use \textit{sakhua}, which is an alternate Hindi name for the sal tree. This story suggests that when they settled in their northern home in the shadow of the Himalayas, harvesting sal was a primary source of wealth. Compare Gilgamesh, for whom Lebanese cedar was the foundation of his royal capital. } For fear of breaking their line of birth, they slept with their own (\textit{saka}) sisters.\footnote{“Own” is \textit{saka}, the second pun on the Sakyan name. Incest is, of course, common among royal families for exactly the reason stated here. Marriage between cousins persisted even in the Buddha’s day. | For \textit{sambheda} in the sense of “dissolving, leaking”, see \href{https://suttacentral.net/an2.9/en/sujato\#1.5}{AN 2.9:1.5} = \href{https://suttacentral.net/dn26/en/sujato\#20.2}{DN 26:20.2}, \href{https://suttacentral.net/an5.103/en/sujato\#6.4}{AN 5.103:6.4}, \href{https://suttacentral.net/an10.45/en/sujato\#4.1}{AN 10.45:4.1}. } 

Then\marginnote{1.15.12} King \textsanskrit{Okkāka} addressed his ministers and councillors, ‘Where, sirs, have the princes settled now?’\footnote{For this sense of \textit{sammati}, see \href{https://suttacentral.net/sn11.9/en/sujato}{SN 11.9}, \href{https://suttacentral.net/sn11.10/en/sujato}{SN 11.10}. } 

‘Sire,\marginnote{1.15.14} there is a lotus pond on the slopes of the Himalayas, by a large grove of \textit{sakhua} trees. They’ve settled there. For fear of breaking their line of birth, they are sleeping with their own sisters.’ 

Then,\marginnote{1.15.16} \textsanskrit{Ambaṭṭha}, King \textsanskrit{Okkāka} expressed this heartfelt sentiment: ‘The princes are indeed Sakyans! The princes are indeed the best Sakyans!’\footnote{This draws on both the puns above. But the commentary also explains \textit{sakya} here as “capable” (\textit{\textsanskrit{samatthā}}, \textit{\textsanskrit{paṭibalā}}) in reference to their survival against all odds, thus connecting Sakya with \textit{sakka} (“able”). } From that day on the Sakyans were recognized and he was their founder. 

Now,\marginnote{1.16.1} King \textsanskrit{Okkāka} had a slavegirl named \textsanskrit{Disā}.\footnote{Vedic \textit{\textsanskrit{dāsa}} (“slave, bondservant”) refers to the “dark-wombed” (\textit{\textsanskrit{kṛṣṇayoni}}, Rig Veda 2.20.7) foes of the Aryan peoples (Rig Veda 10.22.8) who upon defeat were enslaved (Rig Veda 10.62.10). The name \textit{\textsanskrit{disā}} therefore probably means “foe” (Sanskrit \textit{\textsanskrit{dviṣa}}). } She gave birth to a boy named “Black”.\footnote{The passage wavers between treating \textit{\textsanskrit{kaṇha}} (Sanskrit \textit{\textsanskrit{kṛṣṇa}}, i.e. Krishna) as a personal name, a description, and a word for a goblin. I try to capture this ambiguity by using variations of “black boy”. | The passage does not say who the father was. According to \textsanskrit{Arthaśāstra} 3.13, a female slave is protected against sexual harassment by the master, but should she have a child by him, both mother and child are to be set free, and if the sex was not consensual, he must pay her a fine. } When he was born, Black Boy said: ‘Wash me, mum, bathe me! Get this filth off of me! I will be useful for you!’\footnote{Like Siddhattha, he spoke as soon as he was born. The boy was no common child, but had a larger destiny. His words are a dramatic contrast with Siddhattha’s words of confident proclamation, and his birth which was devoid of filth or impurity. } Whereas these days when people see goblins they recognize them as goblins, in those days they recognized goblins as ‘blackboys’. 

They\marginnote{1.16.7} said: ‘He spoke as soon as he was born—a blackboy is born! A goblin is born!’ From that day on the \textsanskrit{Kaṇhāyanas} were proclaimed, and he was their founder. That’s how, recollecting the ancient name and clan of your mother and father, the Sakyans were the children of the masters, while you’re descended from the son of a slavegirl of the Sakyans.” 

When\marginnote{1.17.1} he said this, those young students said to him, “Mister Gotama, please don’t put \textsanskrit{Ambaṭṭha} down too much by calling him the son of a slavegirl.\footnote{Lineage was important to brahmins, but the \textsanskrit{Brāhmaṇa} and \textsanskrit{Upaniṣad} literature shows that, as here, many were more concerned with conduct and wisdom than with birth. } He’s well-born, a gentleman, learned, who enunciates well, and is astute. He is capable of debating with Mister Gotama about this.” 

So\marginnote{1.18.1} the Buddha said to them, “Well, young students, if you think that \textsanskrit{Ambaṭṭha} is ill-born, not a gentleman, unlearned, a poor speaker, witless, and not capable of debating with me about this, then leave him aside and you can debate with me. But if you think that he’s well-born, a gentleman, learned, who enunciates well, is astute, and is capable of debating with me about this, then you should stand aside and let him debate with me.” 

“He\marginnote{1.19.1} is capable of debating you. We will be silent, and let \textsanskrit{Ambaṭṭha} debate with Mister Gotama about this.” 

So\marginnote{1.20.1} the Buddha said to \textsanskrit{Ambaṭṭha}, “Well, \textsanskrit{Ambaṭṭha}, there’s a legitimate question that comes up. You won’t like it, but you ought to answer anyway. If you fail to answer—by dodging the issue, remaining silent, or leaving—your head will explode into seven pieces right here.\footnote{The threat of losing one’s head is found at eg. \textsanskrit{Bṛhadāraṇyaka} \textsanskrit{Upaniṣad} 1.3.24, or at 3.9.26 when it actually did fall off. I cannot trace the detail of heads being split in seven to any early Sanskrit texts, but it is found in later texts such as \textsanskrit{Rāmāyaṇa} 7.26.44c and \textsanskrit{Mahābhārata} 14.7.2c. } What do you think, \textsanskrit{Ambaṭṭha}? According to what you have heard from elderly and senior brahmins, the tutors of tutors, what is the origin of the \textsanskrit{Kaṇhāyanas}, and who is their founder?” 

When\marginnote{1.20.6} he said this, \textsanskrit{Ambaṭṭha} kept silent. 

For\marginnote{1.20.7} a second time, the Buddha put the question, and for a second time \textsanskrit{Ambaṭṭha} kept silent. 

So\marginnote{1.20.10} the Buddha said to him, “Answer now, \textsanskrit{Ambaṭṭha}. Now is not the time for silence. If someone fails to answer a legitimate question when asked three times by the Buddha, their head explodes into seven pieces there and then.” 

Now\marginnote{1.21.1} at that time the spirit \textsanskrit{Vajirapāṇī}, holding a massive iron spear, burning, blazing, and glowing, stood in the air above \textsanskrit{Ambaṭṭha}, thinking,\footnote{\textsanskrit{Vajirapāṇī} (“lightning-bolt in hand”) appears here and in the parallel passage at \href{https://suttacentral.net/mn35/en/sujato\#14.1}{MN 35:14.1}. The synonymous Vajrahasta (Pali \textit{vajirahattha}, \href{https://suttacentral.net/dn20/en/sujato\#12.1}{DN 20:12.1}) is a frequent epithet of Indra in the Vedas (eg. Rig Veda 1.173.10a: \textit{indro vajrahastaḥ}), confirming the commentary’s identification with Sakka. Much later, Mahayana texts adopted the name for a fierce Bodhisattva who was protector of the Dhamma. } “If this \textsanskrit{Ambaṭṭha} doesn’t answer when asked a third time, I’ll blow his head into seven pieces there and then!” And both the Buddha and \textsanskrit{Ambaṭṭha} could see \textsanskrit{Vajirapāṇī}. 

\textsanskrit{Ambaṭṭha}\marginnote{1.21.4} was terrified, shocked, and awestruck. Looking to the Buddha for shelter, protection, and refuge, he sat down close by the Buddha and said, “What did you say? Please repeat the question.” 

“What\marginnote{1.21.7} do you think, \textsanskrit{Ambaṭṭha}? According to what you have heard from elderly and senior brahmins, the tutors of tutors, what is the origin of the \textsanskrit{Kaṇhāyanas}, and who is their founder?” 

“I\marginnote{1.21.9} have heard, Mister Gotama, that it is just as you say. That’s the origin of the \textsanskrit{Kaṇhāyanas}, and that’s who their founder is.” 

\subsection*{2.5. The Discussion of \textsanskrit{Ambaṭṭha}’s Heritage }

When\marginnote{1.22.1} he said this, those young students made an uproar, “It turns out \textsanskrit{Ambaṭṭha} is ill-born, not a gentleman, son of a Sakyan slavegirl, and that the Sakyans are sons of his masters! And it seems that the ascetic Gotama spoke only the truth, though we presumed to rebuke him!” 

Then\marginnote{1.23.1} it occurred to the Buddha, “These young students put \textsanskrit{Ambaṭṭha} down too much by calling him the son of a slavegirl. Why don’t I get him out of this?” 

So\marginnote{1.23.3} the Buddha said to the young students, “Young students, please don’t put \textsanskrit{Ambaṭṭha} down too much by calling him the son of a slavegirl. That Black Boy was an eminent sage.\footnote{The contemptuous senses of “black boy” represent the conservative brahmanical view, presented not as endorsement, but as a rhetorical means to undermine \textsanskrit{Ambaṭṭha}’s pride. The Buddha now shows how a man of a supposedly low birth rose to great spiritual eminence. } He went to a southern country and memorized the Divine Spell. Then he approached King \textsanskrit{Okkāka} and asked for the hand of his daughter \textsanskrit{Maddarūpī}.\footnote{“Divine Spell” is \textit{brahmamanta}, a term of unique occurrence in Pali. In modern Hinduism it is used for a verse of praise for \textsanskrit{Brahmā}, but that is not what is meant here. \textsanskrit{Kaṇha} is one of several “dark hermits” who accrued mighty and lineage-busting powers in the south. } 

The\marginnote{1.23.7} king said to him, ‘Who the hell is this son of a slavegirl to ask for the hand of my daughter!’ Angry and upset he fastened a razor-tipped arrow.\footnote{The Hindu deity Krishna won the hand of his seventh wife \textsanskrit{Lakṣmaṇā}, also known as \textsanskrit{Madrī}, at an archery contest. This detail is too precise to be a coincidence, and proves there must be some shared basis between the two figures. } But he wasn’t able to either shoot it or to relax it. 

Then\marginnote{1.23.10} the ministers and councillors approached the sage Black Boy and said: ‘Spare the king, sir, spare him!’ 

‘The\marginnote{1.23.13} king will be safe. But if he shoots the arrow downwards, there will be an earthquake across the entire realm.’\footnote{This draws on the ancient belief that the king’s acts affect the natural order of things. } 

‘Spare\marginnote{1.23.14} the king, sir, and spare the country!’ 

‘Both\marginnote{1.23.15} king and country will be safe. But if he shoots the arrow upwards, there will be no rain in the entire realm for seven years.’ 

‘Spare\marginnote{1.23.16} the king, sir, spare the country, and let there be rain!’\footnote{This sequence seems to be an etiological myth explaining certain rites of kingship and succession, providing an origin story for this prayer. } 

‘Both\marginnote{1.23.17} king and country will be safe, and the rain will fall. And if the king shoots the crown prince with an arrow, he will be safe and unruffled.’\footnote{National prosperity is ensured through symbolic regicide. This example was omitted from Frazer’s accounts of such substitute sacrifices. Here there is a double substitution: the prince substitutes for the king, then a threat substitutes for the act of killing. This suggests that, even from the legendary perspective of this story within a story, the rite was an ancient one that had evolved through multiple stages. } 

So\marginnote{1.23.18} the ministers said to \textsanskrit{Okkāka}:\footnote{The use of the bare personal name for the king is unusual. } ‘\textsanskrit{Okkāka} must shoot the crown prince with an arrow. He will be safe and unruffled.’ 

So\marginnote{1.23.20} King \textsanskrit{Okkāka} shot the crown prince with an arrow. And he was safe and unruffled. Then the king was terrified, shocked, and awestruck. Scared by the divine punishment, he gave the hand of his daughter \textsanskrit{Maddarūpī}.\footnote{“Divine punishment” is \textit{\textsanskrit{brahmadaṇḍa}}, harking back to the Divine Spell (\textit{brahmamantra}). The Buddha had his own version of the \textit{\textsanskrit{brahmadaṇḍa}}, which was to give the silent treatment (\href{https://suttacentral.net/dn16/en/sujato\#6.4.1}{DN 16:6.4.1}). } 

Young\marginnote{1.23.22} students, please don’t put \textsanskrit{Ambaṭṭha} down too much by calling him the son of a slavegirl. That Black Boy was an eminent sage.” 

\section*{3. The Supremacy of the Aristocrats }

Then\marginnote{1.24.1} the Buddha addressed \textsanskrit{Ambaṭṭha}, “What do you think, \textsanskrit{Ambaṭṭha}? Suppose an aristocrat boy was to sleep with a brahmin girl, and they had a son. Would he receive a seat and water from the brahmins?” 

“He\marginnote{1.24.5} would, Mister Gotama.” 

“And\marginnote{1.24.6} would the brahmins feed him at an offering of food for ancestors, an offering of a dish of milk-rice, a sacrifice, or a feast for guests?” 

“They\marginnote{1.24.7} would.” 

“And\marginnote{1.24.8} would the brahmins teach him the hymns or not?” 

“They\marginnote{1.24.9} would.” 

“And\marginnote{1.24.10} would he be kept from the women or not?”\footnote{In \href{https://suttacentral.net/mn56/en/sujato\#19.2}{MN 56:19.2} \textit{\textsanskrit{āvaṭa}}/\textit{\textsanskrit{anāvaṭa}} is used in reference to \textsanskrit{Upāli} “shutting his gate” against the Jains and opening it for the Buddhists. In \href{https://suttacentral.net/dn17/en/sujato\#1.23.2}{DN 17:1.23.2} \textit{\textsanskrit{anāvaṭa}} means “open to the public”. } 

“He\marginnote{1.24.11} would not.” 

“And\marginnote{1.24.12} would the aristocrats anoint him as king?” 

“No,\marginnote{1.24.13} Mister Gotama. Why is that? Because his maternity is unsuitable.” 

“What\marginnote{1.25.1} do you think, \textsanskrit{Ambaṭṭha}? Suppose a brahmin boy was to sleep with an aristocrat girl, and they had a son. Would he receive a seat and water from the brahmins?” 

“He\marginnote{1.25.4} would, Mister Gotama.” 

“And\marginnote{1.25.5} would the brahmins feed him at an offering of food for ancestors, an offering of a dish of milk-rice, a sacrifice, or a feast for guests?” 

“They\marginnote{1.25.6} would.” 

“And\marginnote{1.25.7} would the brahmins teach him the hymns or not?” 

“They\marginnote{1.25.8} would.” 

“And\marginnote{1.25.9} would he be kept from the women or not?” 

“He\marginnote{1.25.10} would not.” 

“And\marginnote{1.25.11} would the aristocrats anoint him as king?” 

“No,\marginnote{1.25.12} Mister Gotama. Why is that? Because his paternity is unsuitable.” 

“And\marginnote{1.26.1} so, \textsanskrit{Ambaṭṭha}, the aristocrats are superior and the brahmins inferior, whether comparing women with women or men with men. What do you think, \textsanskrit{Ambaṭṭha}? Suppose the brahmins for some reason were to shave a brahmin’s head, inflict him with a sack of ashes, and banish him from the nation or the city. Would he receive a seat and water from the brahmins?” 

“No,\marginnote{1.26.5} Mister Gotama.” 

“And\marginnote{1.26.6} would the brahmins feed him at an offering of food for ancestors, an offering of a dish of milk-rice, a sacrifice, or a feast for guests?” 

“No,\marginnote{1.26.7} Mister Gotama.” 

“And\marginnote{1.26.8} would the brahmins teach him the hymns or not?” 

“No,\marginnote{1.26.9} Mister Gotama.” 

“And\marginnote{1.26.10} would he be kept from the women or not?” 

“He\marginnote{1.26.11} would be.” 

“What\marginnote{1.27.1} do you think, \textsanskrit{Ambaṭṭha}? Suppose the aristocrats for some reason were to shave an aristocrat’s head, inflict him with a sack of ashes, and banish him from the nation or the city. Would he receive a seat and water from the brahmins?” 

“He\marginnote{1.27.4} would, Mister Gotama.” 

“And\marginnote{1.27.5} would the brahmins feed him at an offering of food for ancestors, an offering of a dish of milk-rice, a sacrifice, or a feast for guests?” 

“They\marginnote{1.27.6} would.” 

“And\marginnote{1.27.7} would the brahmins teach him the hymns or not?” 

“They\marginnote{1.27.8} would.” 

“And\marginnote{1.27.9} would he be kept from the women or not?” 

“He\marginnote{1.27.10} would not.” 

“At\marginnote{1.27.11} this point, \textsanskrit{Ambaṭṭha}, that aristocrat has reached rock bottom, with head shaven, inflicted with a sack of ashes, and banished from city or nation. Yet still the aristocrats are superior and the brahmins inferior. The divinity \textsanskrit{Sanaṅkumāra} also spoke this verse:\footnote{\textsanskrit{Sanaṅkumāra} (“Everyoung”) became a Hindu deity closely associated with the worship of Krishna. He first appears in the seventh chapter of the \textsanskrit{Chāndogya} \textsanskrit{Upaniṣad}. There he teaches the learned \textsanskrit{Nārada} what lies beyond the mere surface of words (\textit{\textsanskrit{nāma}}) by giving a progressive meditation that ultimately reveals the highest Self. Thus he is a perfect foil for \textsanskrit{Ambaṭṭha}. The occasion he spoke this verse is recorded at \href{https://suttacentral.net/sn6.11/en/sujato}{SN 6.11}, and it is repeated several times in the suttas. } 

\begin{verse}%
‘The\marginnote{1.28.2} aristocrat is best among people \\
who take clan as the standard. \\
But one accomplished in knowledge and conduct \\
is first among gods and humans.’ 

%
\end{verse}

That\marginnote{1.28.6} verse was well sung by the Divinity \textsanskrit{Sanaṅkumāra}, not poorly sung; well spoken, not poorly spoken, beneficial, not harmful, and it was approved by me. For I also say this: 

\begin{verse}%
The\marginnote{1.28.8} aristocrat is best among people \\
who take clan as the standard. \\
But one accomplished in knowledge and conduct \\
is first among gods and humans.” 

%
\end{verse}

\scendsection{The first recitation section. }

\section*{4. Knowledge and Conduct }

“But\marginnote{2.1.1} what, Mister Gotama, is that conduct, and what is that knowledge?”\footnote{To his credit, after that thorough humiliation, \textsanskrit{Ambaṭṭha} is ready to learn. } 

“\textsanskrit{Ambaṭṭha},\marginnote{2.1.2} in the supreme knowledge and conduct there is no discussion of genealogy or clan or pride—\footnote{Reading \textit{\textsanskrit{anuttarāya} \textsanskrit{vijjācaraṇasampadāya}} as locative, in agreement with \textit{yattha} below. } ‘You deserve me’ or ‘You don’t deserve me.’ Wherever there is giving and taking in marriage there is such discussion. Whoever is attached to questions of genealogy or clan or pride, or to giving and taking in marriage, is far from the supreme knowledge and conduct. The realization of supreme knowledge and conduct occurs when you’ve given up such things.”\footnote{The Buddha emphasizes that his “knowledge and conduct” rejects the notion of birth that is so essential to Brahmanism. } 

“But\marginnote{2.2.1} what, Mister Gotama, is that conduct, and what is that knowledge?” 

“\textsanskrit{Ambaṭṭha},\marginnote{2.2.2} it’s when a Realized One arises in the world, perfected, a fully awakened Buddha, accomplished in knowledge and conduct, holy, knower of the world, supreme guide for those who wish to train, teacher of gods and humans, awakened, blessed. He has realized with his own insight this world—with its gods, \textsanskrit{Māras}, and divinities, this population with its ascetics and brahmins, gods and humans—and he makes it known to others. He proclaims a teaching that is good in the beginning, good in the middle, and good in the end, meaningful and well-phrased. And he reveals a spiritual practice that’s entirely full and pure. A householder hears that teaching, or a householder’s child, or someone reborn in a good family. They gain faith in the Realized One and reflect …\footnote{The Pali text abbreviates the gradual training in this sutta and those that follow. The reader is expected to understand it as in \href{https://suttacentral.net/dn2/en/sujato}{DN 2}. Note, however, that the suttas sometimes have small differences in their perspective that make reconstruction tricky. } 

Quite\marginnote{2.2.8} secluded from sensual pleasures, secluded from unskillful qualities, they enter and remain in the first absorption … This pertains to their conduct. 

Furthermore,\marginnote{2.2.10} as the placing of the mind and keeping it connected are stilled, a mendicant enters and remains in the second absorption … This pertains to their conduct. 

Furthermore,\marginnote{2.2.12} with the fading away of rapture, they enter and remain in the third absorption … This pertains to their conduct. 

Furthermore,\marginnote{2.2.14} giving up pleasure and pain, and ending former happiness and sadness, they enter and remain in the fourth absorption … This pertains to their conduct. This is that conduct. 

When\marginnote{2.2.17} their mind has become immersed in \textsanskrit{samādhi} like this—purified, bright, flawless, rid of corruptions, pliable, workable, steady, and imperturbable—they project it and extend it toward knowledge and vision. This pertains to their knowledge. … They understand: ‘There is nothing further for this place.’ This pertains to their knowledge. This is that knowledge. 

This\marginnote{2.2.22} mendicant is said to be ‘accomplished in knowledge’, and also ‘accomplished in conduct’, and also ‘accomplished in knowledge and conduct’. And, \textsanskrit{Ambaṭṭha}, there is no accomplishment in knowledge and conduct that is better or finer than this. 

\section*{5. Four Causes of Quitting }

There\marginnote{2.3.1} are these four causes of quitting this supreme knowledge and conduct.\footnote{In later Theravada, \textit{\textsanskrit{apāyamukha}} refers to deeds that cause rebirth in lower realms. However this does not apply in the early texts; the acts described here are not evil. Rather, it means an “opening” (\textit{mukha}) for “departure” (\textit{\textsanskrit{apāya}}). } What four? Firstly, take some ascetic or brahmin who, not managing to obtain this supreme knowledge and conduct, plunges into a wilderness region carrying their stuff with a shoulder-pole, thinking they will get by eating fallen fruit.\footnote{A common practice of pre-Buddhist hermits, who avoided the slightest harm to plants. Buddhist mendicants may also not harm plants, but they rely on alms and only eat fallen fruit in case of famine. } In fact they succeed only in serving someone accomplished in knowledge and conduct.\footnote{The Buddha inverts \textsanskrit{Ambaṭṭha}’s earlier claim that the other three castes only succeed in serving brahmins (\href{https://suttacentral.net/dn3/en/sujato\#1.14.7}{DN 3:1.14.7}). } This is the first cause of quitting this supreme knowledge and conduct. 

Furthermore,\marginnote{2.3.7} take some ascetic or brahmin who, not managing to obtain this supreme knowledge and conduct or to get by eating fallen fruit, plunges into a wilderness region carrying a spade and basket, thinking they will get by eating tubers and fruit.\footnote{They are less strict than the previous ascetics, for they dig the soil and harm the plants. } In fact they succeed only in serving someone accomplished in knowledge and conduct. This is the second cause of quitting this supreme knowledge and conduct. 

Furthermore,\marginnote{2.3.11} take some ascetic or brahmin who, not managing to obtain this supreme knowledge and conduct, or to get by eating fallen fruit, or to get by eating tubers and fruit, sets up a fire chamber in the neighborhood of a village or town and dwells there serving the sacred flame. In fact they succeed only in serving someone accomplished in knowledge and conduct. This is the third cause of quitting this supreme knowledge and conduct. 

Furthermore,\marginnote{2.3.14} take some ascetic or brahmin who, not managing to obtain this supreme knowledge and conduct, or to get by eating fallen fruit, or to get by eating tubers and fruit, or to serve the sacred flame, sets up a four-doored fire chamber at the crossroads and dwells there, thinking: ‘When an ascetic or brahmin comes from the four quarters, I will honor them as best I can.’ In fact they succeed only in serving someone accomplished in knowledge and conduct. This is the fourth cause of quitting this supreme knowledge and conduct. These are the four causes of quitting this supreme knowledge and conduct. 

What\marginnote{2.4.1} do you think, \textsanskrit{Ambaṭṭha}? Is this supreme knowledge and conduct seen in your own tradition?”\footnote{“Tradition” renders \textit{\textsanskrit{sācariyaka}}, “that which stems from one’s own teacher”. } 

“No,\marginnote{2.4.3} Mister Gotama. Who am I and my tradition compared with the supreme knowledge and conduct? We are far from that.” 

“What\marginnote{2.4.6} do you think, \textsanskrit{Ambaṭṭha}? Since you’re not managing to obtain this supreme knowledge and conduct, have you with your tradition plunged into a wilderness region carrying your stuff with a shoulder-pole, thinking you will get by eating fallen fruit?” 

“No,\marginnote{2.4.9} Mister Gotama.” 

“What\marginnote{2.4.10} do you think, \textsanskrit{Ambaṭṭha}? Have you with your tradition … plunged into a wilderness region carrying a spade and basket, thinking you will get by eating tubers and fruit?” 

“No,\marginnote{2.4.13} Mister Gotama.” 

“What\marginnote{2.4.14} do you think, \textsanskrit{Ambaṭṭha}? Have you with your tradition … set up a fire chamber in the neighborhood of a village or town and dwelt there serving the sacred flame?” 

“No,\marginnote{2.4.16} Mister Gotama.” 

“What\marginnote{2.4.17} do you think, \textsanskrit{Ambaṭṭha}? Have you with your tradition … set up a four-doored fire chamber at the crossroads and dwelt there, thinking: ‘When an ascetic or brahmin comes from the four quarters, I will honor them as best I can’?” 

“No,\marginnote{2.4.20} Mister Gotama.” 

“So\marginnote{2.5.1} you with your tradition are not only inferior to the supreme knowledge and conduct, you are even inferior to the four causes of quitting the supreme knowledge and conduct. But you have been told this by your tutor, the brahmin \textsanskrit{Pokkharasāti}: ‘Who are these shavelings, fake ascetics, primitives, black spawn from the feet of our kinsman compared with conversation with the brahmins of the three knowledges?” Yet he himself has not even fulfilled one of the quittings! See, \textsanskrit{Ambaṭṭha}, how your tutor \textsanskrit{Pokkharasāti} has wronged you. 

\section*{6. Being Like the Sages of the Past }

\textsanskrit{Pokkharasāti}\marginnote{2.6.1} lives off an endowment provided by King Pasenadi of Kosala. But the king won’t even grant him an audience face to face. When he consults, he does so behind a curtain.\footnote{This practice is not elsewhere attested in early Pali. } Why wouldn’t the king grant a face to face audience with someone who’d receive his legitimate presentation of food? See, \textsanskrit{Ambaṭṭha}, how your tutor \textsanskrit{Pokkharasāti} has wronged you. 

What\marginnote{2.7.1} do you think, \textsanskrit{Ambaṭṭha}? Suppose King Pasenadi was holding consultations with warrior-chiefs or chieftains while sitting on an elephant’s neck or on horseback, or while standing on the mat in a chariot.\footnote{\textit{Ugga} is a rare word whose root sense is “mighty”, but here it must be a noun. Given that it is a military man who consults with the king, I translate as “warrior-chief”. | \textit{\textsanskrit{Rājañña}} is used occasionally in the suttas; it is an archaic synonym for \textit{khattiya}. } And suppose he’d get down from that place and stand aside. Then along would come a worker or their bondservant, who’d stand in the same place and continue the consultation: ‘This is what King Pasenadi says, and this too is what the king says.’ Though he spoke the king’s words and gave the king’s advice,\footnote{Taking this and the next as one sentence, despite the punctuation of the \textsanskrit{Mahāsaṅgīti} text. } does that qualify him to be the king or the king’s minister?” 

“No,\marginnote{2.7.8} Mister Gotama.” 

“In\marginnote{2.8.1} the same way, \textsanskrit{Ambaṭṭha}, the ancient seers of the brahmins were \textsanskrit{Aṭṭhaka}, \textsanskrit{Vāmaka}, \textsanskrit{Vāmadeva}, \textsanskrit{Vessāmitta}, Yamadaggi, \textsanskrit{Aṅgīrasa}, \textsanskrit{Bhāradvāja}, \textsanskrit{Vāseṭṭha}, Kassapa, and Bhagu. They were the authors and propagators of the hymns. Their hymnal was sung and propagated and compiled in ancient times; and these days, brahmins continue to sing and chant it, chanting what was chanted and teaching what was taught.\footnote{The “hymns” (\textit{\textsanskrit{mantā}}) are the verses of the Rig Veda. The ten names here all correspond with Vedic authors according to the Brahmanical tradition (for details, see note on \href{https://suttacentral.net/dn13/en/sujato\#13.1}{DN 13:13.1}). Note that in Sanskrit the names of the rishis are distinguished from the lineage holders, which take the patronymic. For example, \textsanskrit{Bharadvāja} is the rishi, the \textsanskrit{Bhāradvājas} are his descendants; \textsanskrit{Vasiṣṭha} is the rishi, the \textsanskrit{Vāsiṣṭhas} are his descendants. Pali texts do not make this distinction, but use the patronymic, although the two forms are not always readily distinguishable. | “Seer” is \textit{isi} (Sanskrit \textit{\textsanskrit{ṛṣi}}). It is of uncertain etymology, but was taken to mean that they had “seen” the Vedas (\textit{\textsanskrit{mantradraṣṭa}}) or directly “heard” them from \textsanskrit{Brahmā} through divine inspiration, rather than “composing” them like ordinary authors. Here, however, the Buddha says they were “authors” (\textit{\textsanskrit{kattāro}}). The Buddha adopted \textit{isi} in the sense “enlightened sage”. } You might imagine that, since you’ve learned their hymns by heart in your own tradition, that makes you a seer or someone on the path to becoming a seer. But that is not possible. 

What\marginnote{2.9.1} do you think, \textsanskrit{Ambaṭṭha}? According to what you have heard from elderly and senior brahmins, the tutors of tutors, did those ancient brahmin seers—nicely bathed and anointed, with hair and beard dressed, bedecked with jewels, earrings, and bracelets, dressed in white—amuse themselves, supplied and provided with the five kinds of sensual stimulation, like you do today in your tradition?” 

“No,\marginnote{2.9.5} Mister Gotama.” 

“Did\marginnote{2.10.1} they eat boiled fine rice, garnished with clean meat, with the dark grains picked out, served with many soups and sauces, like you do today in your tradition?” 

“No,\marginnote{2.10.3} Mister Gotama.” 

“Did\marginnote{2.10.4} they amuse themselves with girls wearing thongs that show off their curves, like you do today in your tradition?”\footnote{\textit{\textsanskrit{Veṭhakanatapassāhi}} is otherwise unattested. At \href{https://suttacentral.net/mn55/en/sujato\#12.4}{MN 55:12.4} \textit{\textsanskrit{veṭhaka}} evidently means “collar”. In the \textsanskrit{Lokuttaravāda} \textsanskrit{Bhikṣuṇī} Vinaya, the brazen nun \textsanskrit{Thullānandā} gets out of the water and wraps herself in a \textit{\textsanskrit{veṭhaka}}, which here seems synonymous with \textit{\textsanskrit{paṭṭaka}}, a strip of cloth. It is allowable if used to tie a basket (\href{https://suttacentral.net/san-lo-bi-pn3}{Lo Bi Pn 3}). \textit{Nata} is “curve”, \textit{passa} is “side, flank”. Walshe has “flounces and furbelows”, Rhys Davids has “fringes and furbelows round their loins”. These are prissy descriptions of what is evidently stripper gear. } 

“No,\marginnote{2.10.6} Mister Gotama.” 

“Did\marginnote{2.10.7} they drive about in chariots drawn by mares with plaited manes, whipping and lashing them onward with long goads, like you do today in your tradition?”\footnote{The Buddha calls back to earlier in the sutta, where \textsanskrit{Ambaṭṭha} drove a mare-drawn chariot (\href{https://suttacentral.net/dn3/en/sujato\#1.6.1}{DN 3:1.6.1}). | The verbs here (\textit{vitudenti vitacchenti}) are elsewhere applied to the pecking and slashing of vultures, crows, or hawks (\href{https://suttacentral.net/sn19.1/en/sujato\#3.2}{SN 19.1:3.2}, \href{https://suttacentral.net/mn54/en/sujato\#16.2}{MN 54:16.2}, etc.). The Buddha was disgusted with this maltreatment of the mares. } 

“No,\marginnote{2.10.9} Mister Gotama.” 

“Did\marginnote{2.10.10} they get men with long swords to guard them in fortresses with moats dug and barriers in place, like you do today in your tradition?”\footnote{Remembering that \textsanskrit{Pokkharasāti} lived in a wealthy property that was a royal endowment. Just as today, excessive wealth breeds insecurity. } 

“No,\marginnote{2.10.12} Mister Gotama.” 

“So,\marginnote{2.10.13} \textsanskrit{Ambaṭṭha}, in your own tradition you are neither seer nor someone on the path to becoming a seer. Whoever has any doubt or uncertainty about me, let them ask me and I will clear up their doubts with my answer.”\footnote{The Buddha has been hard on \textsanskrit{Ambaṭṭha}, but he is not unfair. He invites the same level of scrutiny for himself. } 

\section*{7. Seeing the Two Marks }

Then\marginnote{2.11.1} the Buddha came out of his dwelling and proceeded to begin walking mindfully,\footnote{This transition occurs nowhere else. } and \textsanskrit{Ambaṭṭha} did likewise. Then while walking beside the Buddha, \textsanskrit{Ambaṭṭha} scrutinized his body for the thirty-two marks of a great man.\footnote{Finally he remembers what his teacher \textsanskrit{Pokkharasāti} told him in \href{https://suttacentral.net/dn3/en/sujato\#1.5.2}{DN 3:1.5.2}: he will know the Buddha by his marks. } He saw all of them except for two, which he had doubts about: whether the private parts are covered in a foreskin, and the largeness of the tongue. 

Then\marginnote{2.12.1} it occurred to the Buddha, “This student \textsanskrit{Ambaṭṭha} sees all the marks except for two, which he has doubts about: whether the private parts are covered in a foreskin, and the largeness of the tongue.” Then the Buddha used his psychic power to will that \textsanskrit{Ambaṭṭha} would see his private parts covered in a foreskin.\footnote{This exceedingly strange “miracle” is also found at \href{https://suttacentral.net/mn91/en/sujato\#7.1}{MN 91:7.1}, \href{https://suttacentral.net/mn92/en/sujato\#14.1}{MN 92:14.1}, and \href{https://suttacentral.net/snp3.7/en/sujato\#11.5}{Snp 3.7:11.5}. } And he stuck out his tongue and stroked back and forth on his ear holes and nostrils, and covered his entire forehead with his tongue. 

Then\marginnote{2.12.7} \textsanskrit{Ambaṭṭha} thought, “The ascetic Gotama possesses the thirty-two marks completely, lacking none.” 

He\marginnote{2.12.9} said to the Buddha, “Well, now, sir, I must go. I have many duties, and much to do.” 

“Please,\marginnote{2.12.11} \textsanskrit{Ambaṭṭha}, go at your convenience.” Then \textsanskrit{Ambaṭṭha} mounted his chariot drawn by mares and left. 

Now\marginnote{2.13.1} at that time the brahmin \textsanskrit{Pokkharasāti} had come out from \textsanskrit{Ukkaṭṭhā} together with a large group of brahmins and was sitting in his own park just waiting for \textsanskrit{Ambaṭṭha}. Then \textsanskrit{Ambaṭṭha} entered the park. He went by carriage as far as the terrain allowed, then descended and approached the brahmin \textsanskrit{Pokkharasāti} on foot. He bowed and sat down to one side, and \textsanskrit{Pokkharasāti} said to him: 

“I\marginnote{2.14.2} hope, dear \textsanskrit{Ambaṭṭha}, you saw Mister Gotama?” 

“I\marginnote{2.14.3} saw him, sir.” 

“Well,\marginnote{2.14.4} does he live up to his reputation or not?” 

“He\marginnote{2.14.6} does, sir. Mister Gotama possesses the thirty-two marks completely, lacking none.” 

“And\marginnote{2.14.8} did you have some discussion with him?” 

“I\marginnote{2.14.9} did.” 

“And\marginnote{2.14.10} what kind of discussion did you have with him?” Then \textsanskrit{Ambaṭṭha} informed \textsanskrit{Pokkharasāti} of all they had discussed. 

Then\marginnote{2.15.1} \textsanskrit{Pokkharasāti} said to \textsanskrit{Ambaṭṭha}, “Oh, our bloody fake scholar, our fake learned man, who pretends to be proficient in the three Vedas! A man who behaves like this ought, when their body breaks up, after death, to be reborn in a place of loss, a bad place, the underworld, hell.\footnote{The diminutive ending for \textit{\textsanskrit{paṇḍitaka}} is the same as in \textit{\textsanskrit{samaṇaka}}, which \textsanskrit{Ambaṭṭha} used of the Buddha. | For \textit{re} (“bloody”), compare \textit{cara pi re} at \href{https://suttacentral.net/pli-tv-bu-vb-pc70/en/sujato\#1.35}{Bu Pc 70:1.35} and \textit{he je \textsanskrit{kāḷī}} at \href{https://suttacentral.net/mn21/en/sujato\#9.13}{MN 21:9.13}. Hard as the Buddha was on \textsanskrit{Ambaṭṭha}, his own teacher was harder. } It’s only because you repeatedly attacked Mister Gotama like that that he kept bringing up charges against us!”\footnote{\textsanskrit{Pokkharasāti} shows his astuteness, for in many other dialogues the Buddha engaged with brahmins perfectly politely, as he does in the next sutta (\href{https://suttacentral.net/dn4/en/sujato}{DN 4}). } Angry and upset, he kicked \textsanskrit{Ambaṭṭha} over,\footnote{Illustrating the lack of restraint of even a senior brahmin teacher. } and wanted to go and see the Buddha right away. 

\section*{8. \textsanskrit{Pokkharasāti} Visits the Buddha }

Then\marginnote{2.16.1} those brahmins said to \textsanskrit{Pokkharasāti}, “It’s much too late to visit the ascetic Gotama today. You can visit him tomorrow.”\footnote{Given \textsanskrit{Pokkharasāti}’s mood, this was probably a diplomatic move. } 

So\marginnote{2.16.4} \textsanskrit{Pokkharasāti} had delicious fresh and cooked foods prepared in his own home. Then he mounted a carriage and, with attendants carrying torches, set out from \textsanskrit{Ukkaṭṭhā} for the forest near \textsanskrit{Icchānaṅgala}.\footnote{\textit{\textsanskrit{Khādanīya}} and \textit{\textsanskrit{bhojanīya}} are food categories commonly mentioned in Pali. Etymologically they stem from “hard and soft”. \textit{\textsanskrit{Bhojanīya}} is defined in \href{https://suttacentral.net/pli-tv-bu-vb-pc37/en/sujato\#2.1.10}{Bu Pc 37:2.1.10} as grain, porridge, flour products, fish, and meat, thus being foods that are typically eaten cooked and “mooshed up” in with the fingers in the bowl. \textit{\textsanskrit{Khādanīya}} is not so readily defined, being essentially everything not included in other categories. But it would have included such “crunchy” things as fruit and vegetables, which may be eaten uncooked. } He went by carriage as far as the terrain allowed, then descended and entered the monastery on foot. He went up to the Buddha and exchanged greetings with him. When the greetings and polite conversation were over, he sat down to one side and said to the Buddha, “Master Gotama, has my resident pupil, the student \textsanskrit{Ambaṭṭha}, come here?” 

“Yes\marginnote{2.17.3} he has, brahmin.” 

“And\marginnote{2.17.4} did you have some discussion with him?” 

“I\marginnote{2.17.5} did.” 

“And\marginnote{2.17.6} what kind of discussion did you have with him?”\footnote{\textsanskrit{Pokkharasāti} makes sure he hears both sides of the story. } Then the Buddha informed \textsanskrit{Pokkharasāti} of all they had discussed. 

Then\marginnote{2.17.8} \textsanskrit{Pokkharasāti} said to the Buddha, “\textsanskrit{Ambaṭṭha} is a fool, Mister Gotama. Please forgive him.”\footnote{So far has \textsanskrit{Ambaṭṭha} fallen from the learned sage we were introduced to at the start of the sutta. } 

“May\marginnote{2.17.10} the student \textsanskrit{Ambaṭṭha} be happy, brahmin.”\footnote{The Buddha bears no ill will. \textit{\textsanskrit{Sukhī} hotu} is one of the most recognizable Pali phrases, but in early texts it is spoken only a few times: by the Buddha at \href{https://suttacentral.net/dn21/en/sujato\#1.8.8}{DN 21:1.8.8} and \href{https://suttacentral.net/snp5.1/en/sujato\#54.1}{Snp 5.1:54.1}; by Punabbasu’s Mother at \href{https://suttacentral.net/sn10.7/en/sujato\#10.1}{SN 10.7:10.1}; and by various women at \href{https://suttacentral.net/pli-tv-bu-vb-ss5/en/sujato\#1.4.8}{Bu Ss 5:1.4.8}. } 

Then\marginnote{2.18.1} \textsanskrit{Pokkharasāti} scrutinized the Buddha’s body for the thirty-two marks of a great man. He saw all of them except for two, which he had doubts about: whether the private parts are covered in a foreskin, and the largeness of the tongue. 

Then\marginnote{2.18.5} it occurred to the Buddha, “\textsanskrit{Pokkharasāti} sees all the marks except for two, which he has doubts about: whether the private parts are covered in a foreskin, and the largeness of the tongue.” Then the Buddha used his psychic power to will that \textsanskrit{Pokkharasāti} would see his private parts covered in a foreskin. And he stuck out his tongue and stroked back and forth on his ear holes and nostrils, and covered his entire forehead with his tongue. 

\textsanskrit{Pokkharasāti}\marginnote{2.19.1} thought, “The ascetic Gotama possesses the thirty-two marks completely, lacking none.” 

He\marginnote{2.19.3} said to the Buddha, “Would Mister Gotama together with the mendicant \textsanskrit{Saṅgha} please accept today’s meal from me?” The Buddha consented with silence. 

Then,\marginnote{2.20.1} knowing that the Buddha had consented, \textsanskrit{Pokkharasāti} announced the time to him, “It’s time, Mister Gotama, the meal is ready.” Then the Buddha robed up in the morning and, taking his bowl and robe, went to the home of \textsanskrit{Pokkharasāti} together with the mendicant \textsanskrit{Saṅgha}, where he sat on the seat spread out.\footnote{“Robed up” because inside the monastery, monks would normally wear just a lower robe, and would don the upper and (sometimes) outer robes when visiting a layperson’s home. } Then \textsanskrit{Pokkharasāti} served and satisfied the Buddha with his own hands with delicious fresh and cooked foods, while his young students served the \textsanskrit{Saṅgha}. When the Buddha had eaten and washed his hand and bowl, \textsanskrit{Pokkharasāti} took a low seat and sat to one side. 

Then\marginnote{2.21.1} the Buddha taught him step by step, with a talk on giving, ethical conduct, and heaven. He explained the drawbacks of sensual pleasures, so sordid and corrupt, and the benefit of renunciation.\footnote{While all these teachings feature commonly in the suttas, there is no text that depicts this framework in detail. } And when the Buddha knew that \textsanskrit{Pokkharasāti}’s mind was ready, pliable, rid of hindrances, elated, and confident he explained the special teaching of the Buddhas: suffering, its origin, its cessation, and the path.\footnote{This is the briefest expression of the four noble truths. } Just as a clean cloth rid of stains would properly absorb dye, in that very seat the stainless, immaculate vision of the Dhamma arose in the brahmin \textsanskrit{Pokkharasāti}:\footnote{This indicates that he became a stream-enterer (\textit{\textsanskrit{sotāpanna}}), the first of four stages of Awakening. Such details of personal attainment are typically found in the narrative rather than the teaching attributed to the Buddha, and hence were added by redactors at some point. They vary considerably in different versions. In this case, the parallel at DA 20 says that he became a stream-enterer and later a non-returner. T 20 said that he understood the teaching and went for refuge, and agrees that he became a non-returner before his death. } “Everything that has a beginning has an end.”\footnote{This is the insight into universal impermanence and dependent origination. } 

\section*{9. \textsanskrit{Pokkharasāti} Declares Himself a Lay Follower }

Then\marginnote{2.22.1} \textsanskrit{Pokkharasāti} saw, attained, understood, and fathomed the Dhamma. He went beyond doubt, got rid of indecision, and became self-assured and independent of others regarding the Teacher’s instructions. He said to the Buddha, “Excellent, Mister Gotama! Excellent! As if he were righting the overturned, or revealing the hidden, or pointing out the path to the lost, or lighting a lamp in the dark so people with clear eyes can see what’s there, just so has Mister Gotama made the Teaching clear in many ways. Together with my children, wives, retinue, and ministers, I go for refuge to Mister Gotama, to the teaching, and to the mendicant \textsanskrit{Saṅgha}. From this day forth, may Mister Gotama remember me as a lay follower who has gone for refuge for life. 

Just\marginnote{2.22.6} as Mister Gotama visits other devoted families in \textsanskrit{Ukkaṭṭhā}, may he visit mine.\footnote{When wandering for alms, mendicants would often roam randomly through the village (\textit{\textsanskrit{sapadānacārī}}). However if an invitation such as this were issued, the mendicant may visit that place for a meal. It was considered a special ascetic practice to refuse such invitations. The same invitation was issued by Lohicca to \textsanskrit{Mahākaccāna} at \href{https://suttacentral.net/sn35.132/en/sujato\#14.7}{SN 35.132:14.7}. } The brahmin boys and girls there will bow to you, rise in your presence, give you a seat and water, and gain confidence in their hearts. That will be for their lasting welfare and happiness.”\footnote{\textit{\textsanskrit{Māṇavikā}} is also mentioned at \href{https://suttacentral.net/mn56/en/sujato\#27.1}{MN 56:27.1} and \href{https://suttacentral.net/ud2.6/en/sujato\#1.3}{Ud 2.6:1.3} of a young married woman; and at \href{https://suttacentral.net/an5.192/en/sujato\#8.5}{AN 5.192:8.5} of a baby being born. Thus it does not seem that it meant “female student of the Vedas”. } 

“That’s\marginnote{2.22.8} nice of you to say, householder.”\footnote{\textit{\textsanskrit{Kalyāṇaṁ} vuccati} is a politely ambiguous phrase. It is spoken twice elsewhere in the Pali, and both times the mendicant who said it immediately departed and never returned (\href{https://suttacentral.net/sn41.3/en/sujato\#7.13}{SN 41.3:7.13}, \href{https://suttacentral.net/sn41.4/en/sujato\#6.7}{SN 41.4:6.7}). } 

%
\chapter*{{\suttatitleacronym DN 4}{\suttatitletranslation With Soṇadaṇḍa }{\suttatitleroot Soṇadaṇḍasutta}}
\addcontentsline{toc}{chapter}{\tocacronym{DN 4} \toctranslation{With Soṇadaṇḍa } \tocroot{Soṇadaṇḍasutta}}
\markboth{With Soṇadaṇḍa }{Soṇadaṇḍasutta}
\extramarks{DN 4}{DN 4}

\section*{1. The Brahmins and Householders of \textsanskrit{Campā} }

\scevam{So\marginnote{1.1} I have heard.\footnote{This sutta shows how the conversion of \textsanskrit{Pokkharasāti} in \href{https://suttacentral.net/dn3/en/sujato}{DN 3} affected the brahmins as far away as \textsanskrit{Campā}. } }At one time the Buddha was wandering in the land of the \textsanskrit{Aṅgas} together with a large \textsanskrit{Saṅgha} of five hundred mendicants when he arrived at \textsanskrit{Campā},\footnote{\textsanskrit{Campā} is modern Champapuri near Bhagalpur in Bihar state, not far from West Bengal. It is near the eastern-most reach of the Buddha’s journeys. \textsanskrit{Campā} was the capital of \textsanskrit{Aṅga}, one of the sixteen “great nations” (\textit{\textsanskrit{mahājanapadā}}). It was a flourishing trade center at which Northern Black Polished Ware has been found, and became a sacred city for the Jains. } where he stayed by the banks of the \textsanskrit{Gaggarā} Lotus Pond.\footnote{\textit{\textsanskrit{Gaggarā}}, an onomatopoeic reduplication (“gargle”), is the name of a number of rivers and whirlpools in Sanskrit (cp. the modern Ghaggar River in north-west India). } 

Now\marginnote{1.4} at that time the brahmin \textsanskrit{Soṇadaṇḍa} was living in \textsanskrit{Campā}. It was a crown property given by King Seniya \textsanskrit{Bimbisāra} of Magadha, teeming with living creatures, full of hay, wood, water, and grain, a royal park endowed to a brahmin.\footnote{Here we see how the endowment of \textit{brahmadeyya} helped the king of Magadha establish his influence over the \textsanskrit{Aṅgas}. } 

The\marginnote{2.1} brahmins and householders of \textsanskrit{Campā} heard:\footnote{“Householders” (\textit{gahapati}) is literal; it means land owners. Thus the “brahmins and householders” (not “brahmin householders”) were the wealthy class. } 

“It\marginnote{2.2} seems the ascetic Gotama—a Sakyan, gone forth from a Sakyan family—has arrived at \textsanskrit{Campā} and is staying on the banks of the \textsanskrit{Gaggarā} Lotus Pond. He has this good reputation: ‘That Blessed One is perfected, a fully awakened Buddha, accomplished in knowledge and conduct, holy, knower of the world, supreme guide for those who wish to train, teacher of gods and humans, awakened, blessed.’ He has realized with his own insight this world—with its gods, \textsanskrit{Māras}, and divinities, this population with its ascetics and brahmins, gods and humans—and he makes it known to others. He proclaims a teaching that is good in the beginning, good in the middle, and good in the end, meaningful and well-phrased. And he reveals a spiritual practice that’s entirely full and pure. It’s good to see such perfected ones.” Then, having departed \textsanskrit{Campā}, they formed into companies and headed to the \textsanskrit{Gaggarā} Lotus Pond. 

Now\marginnote{3.1} at that time the brahmin \textsanskrit{Soṇadaṇḍa} had retired to the upper floor of his stilt longhouse for his midday nap. He saw the brahmins and householders heading for the lotus pond, and addressed his butler, “My butler, why are the brahmins and householders headed for the \textsanskrit{Gaggarā} Lotus Pond?” 

“The\marginnote{3.5} ascetic Gotama has arrived at \textsanskrit{Campā} and is staying on the banks of the \textsanskrit{Gaggarā} Lotus Pond. He has this good reputation: ‘That Blessed One is perfected, a fully awakened Buddha, accomplished in knowledge and conduct, holy, knower of the world, supreme guide for those who wish to train, teacher of gods and humans, awakened, blessed.’ They’re going to see that Mister Gotama.” 

“Well\marginnote{3.9} then, go to the brahmins and householders and say to them: ‘Sirs, the brahmin \textsanskrit{Soṇadaṇḍa} asks you to wait, as he will also go to see the ascetic Gotama.’” 

“Yes,\marginnote{3.12} sir,” replied the butler, and did as he was asked. 

\section*{2. The Qualities of \textsanskrit{Soṇadaṇḍa} }

Now\marginnote{4.1} at that time around five hundred brahmins from abroad were residing in \textsanskrit{Campā} on some business. They heard that the brahmin \textsanskrit{Soṇadaṇḍa} was going to see the ascetic Gotama. They approached \textsanskrit{Soṇadaṇḍa} and said to him, “Is it really true that you are going to see the ascetic Gotama?” 

“Yes,\marginnote{4.6} gentlemen, it is true.” 

“Please\marginnote{5.1} don’t, mister \textsanskrit{Soṇadaṇḍa}! It’s not appropriate for you to go to see the ascetic Gotama.\footnote{Both the repetition below and the parallel at \href{https://suttacentral.net/mn95/en/sujato\#8.3}{MN 95:8.3} include the phrase “it’s appropriate that he comes to see you”. It may have been omitted here by mistake. } For if you do so, your reputation will diminish and his will increase. For this reason it’s not appropriate for you to go to see the ascetic Gotama; it’s appropriate that he comes to see you. 

You\marginnote{5.6} are well born on both your mother’s and father’s side, of pure descent, with irrefutable and impeccable genealogy back to the seventh paternal generation.\footnote{\textit{\textsanskrit{Jātivāda}} is sometimes translated as “doctrine of birth”, but the context here shows this cannot be the case. It refers to the genealogical records of the family lineage. } For this reason it’s not appropriate for you to go to see the ascetic Gotama; it’s appropriate that he comes to see you. 

You’re\marginnote{5.9} rich, affluent, and wealthy. … 

You\marginnote{5.10} recite and remember the hymns, and have mastered the three Vedas, together with their vocabularies and ritual performance, their phonology and word classification, and the testaments as fifth. You know them word-by-word, and their grammar. You are well versed in cosmology and the marks of a great man. … 

You\marginnote{5.11} are attractive, good-looking, lovely, of surpassing beauty. You are magnificent and splendid as the Divinity, remarkable to behold. …\footnote{For \textsanskrit{Mahāsaṅgīti} \textit{\textsanskrit{vacchasī}} read \textit{\textsanskrit{vaccasī}} (Sanskrit \textit{varcasin}), “possessing splendor”. } 

You\marginnote{5.12} are ethical, mature in ethical conduct. … 

You’re\marginnote{5.13} a good speaker who enunciates well, with a polished, clear, and articulate voice that expresses the meaning. … 

You\marginnote{5.14} tutor the tutors of many, and teach three hundred young students to recite the hymns. Many students come from various districts and countries for the sake of the hymns, wishing to learn the hymns. …\footnote{Notice that the royal endowment was not just for a luxury residence, it was the site of a major international college. Kings invested in education. } 

You’re\marginnote{5.15} old, elderly and senior, advanced in years, and have reached the final stage of life. The ascetic Gotama is young, and has newly gone forth. … 

You’re\marginnote{5.17} honored, respected, revered, venerated, and esteemed by King \textsanskrit{Bimbisāra} of Magadha … 

and\marginnote{5.18} the brahmin \textsanskrit{Pokkharasāti}. … 

You\marginnote{5.19} live in \textsanskrit{Campā}, a crown property given by King Seniya \textsanskrit{Bimbisāra} of Magadha, teeming with living creatures, full of hay, wood, water, and grain, a royal park endowed to a brahmin. For this reason, too, it’s not appropriate for you to go to see the ascetic Gotama; it’s appropriate that he comes to see you.” 

\section*{3. The Qualities of the Buddha }

When\marginnote{6.1} they had spoken, \textsanskrit{Soṇadaṇḍa} said to those brahmins: 

“Well\marginnote{6.2} then, gentlemen, listen to why it’s appropriate for me to go to see the ascetic Gotama, and it’s not appropriate for him to come to see me. He is well born on both his mother’s and father’s side, of pure descent, with irrefutable and impeccable genealogy back to the seventh paternal generation. For this reason it’s not appropriate for the ascetic Gotama to come to see me; rather, it’s appropriate for me to go to see him. 

When\marginnote{6.7} he went forth he abandoned a large family circle. … 

When\marginnote{6.8} he went forth he abandoned abundant gold, both coined and uncoined, stored in dungeons and towers. … 

He\marginnote{6.9} went forth from the lay life to homelessness while still a youth, young, with pristine black hair, blessed with youth, in the prime of life. … 

Though\marginnote{6.10} his mother and father wished otherwise, weeping with tearful faces, he shaved off his hair and beard, dressed in ocher robes, and went forth from the lay life to homelessness. …\footnote{Later tradition says that the young Siddhattha sneaked out of his home to avoid creating such a scene, but the early texts say he left despite his parents’ weeping. } 

He\marginnote{6.11} is attractive, good-looking, lovely, of surpassing beauty. He is magnificent and splendid as the Divinity, remarkable to behold. … 

He\marginnote{6.12} is ethical, possessing ethical conduct that is noble and skillful. … 

He’s\marginnote{6.13} a good speaker who enunciates well, with a polished, clear, and articulate voice that expresses the meaning. … 

He’s\marginnote{6.14} a tutor of tutors. … 

He\marginnote{6.15} has ended sensual desire, and is rid of caprice. … 

He\marginnote{6.16} teaches the efficacy of deeds and action. He doesn’t wish any harm upon the community of brahmins. …\footnote{In contrast with some of the other ascetics in \href{https://suttacentral.net/dn2/en/sujato}{DN 2}. } 

He\marginnote{6.17} went forth from an eminent family of unbroken aristocratic lineage. … 

He\marginnote{6.18} went forth from a rich, affluent, and wealthy family. … 

People\marginnote{6.19} come from distant lands and distant countries to question him. … 

Many\marginnote{6.20} thousands of deities have gone for refuge for life to him. … 

He\marginnote{6.21} has this good reputation: ‘That Blessed One is perfected, a fully awakened Buddha, accomplished in knowledge and conduct, holy, knower of the world, supreme guide for those who wish to train, teacher of gods and humans, awakened, blessed.’ … 

He\marginnote{6.23} has the thirty-two marks of a great man. … 

He\marginnote{6.24} is welcoming, congenial, polite, smiling, open, the first to speak. …\footnote{This description is applied to the Buddha at \href{https://suttacentral.net/dn5/en/sujato\#7.24}{DN 5:7.24}, and to the corrupt but superficially charming monks Assaji and Punabbasuka at \href{https://suttacentral.net/pli-tv-bu-vb-ss13/en/sujato\#1.3.4}{Bu Ss 13:1.3.4}. } 

He’s\marginnote{6.25} honored, respected, revered, venerated, and esteemed by the four assemblies. …\footnote{This would be the assemblies of aristocrats, brahmins, householders, and ascetics (\href{https://suttacentral.net/an5.213/en/sujato\#3.1}{AN 5.213:3.1}), rather than the Buddha’s four assemblies of monks, nuns, laymen, and laywomen (\href{https://suttacentral.net/an4.129/en/sujato}{AN 4.129}). } 

Many\marginnote{6.26} gods and humans are devoted to him. … 

While\marginnote{6.27} he is residing in a village or town, non-human entities do not harass them. … 

He\marginnote{6.28} leads an order and a community, and tutors a community, and is said to be the best of the various religious founders. He didn’t come by his fame in the same ways as those other ascetics and brahmins.\footnote{\textit{Titthakara}, literally “ford-maker”, is a term restricted to those such as the Buddha who founded a religion, or \textsanskrit{Mahāvīra} who was a major reformer on the same level as a founder. } Rather, he came by his fame due to his supreme knowledge and conduct. … 

King\marginnote{6.30} Seniya \textsanskrit{Bimbisāra} of Magadha and his wives and children have gone for refuge for life to the ascetic Gotama. …\footnote{\textsanskrit{Bimbisāra}’s refuge is at \href{https://suttacentral.net/pli-tv-kd1/en/sujato\#22.11.4}{Kd 1:22.11.4}. } 

King\marginnote{6.31} Pasenadi of Kosala and his wives and children have gone for refuge for life to the ascetic Gotama. …\footnote{Pasenadi’s refuge is at \href{https://suttacentral.net/sn3.1/en/sujato\#14.5}{SN 3.1:14.5}. } 

The\marginnote{6.32} brahmin \textsanskrit{Pokkharasāti} and his wives and children have gone for refuge for life to the ascetic Gotama. …\footnote{This sutta must have been before the events of \href{https://suttacentral.net/dn2/en/sujato}{DN 2}, but after those of \href{https://suttacentral.net/dn3/en/sujato}{DN 3}. } 

He’s\marginnote{6.33} honored, respected, revered, venerated, and esteemed by King \textsanskrit{Bimbisāra} of Magadha … 

King\marginnote{6.34} Pasenadi of Kosala … 

and\marginnote{6.35} the brahmin \textsanskrit{Pokkharasāti}. 

The\marginnote{6.36} ascetic Gotama has arrived at \textsanskrit{Campā} and is staying at the \textsanskrit{Gaggarā} Lotus Pond. Any ascetic or brahmin who comes to stay in our village district is our guest, and should be honored and respected as such. For this reason, too, it’s not appropriate for Mister Gotama to come to see me; rather, it’s appropriate for me to go to see him. This is the extent of Mister Gotama’s praise that I have memorized. But his praises are not confined to this, for the praise of Mister Gotama is limitless.” 

When\marginnote{6.45} he had spoken, those brahmins said to him, “According to \textsanskrit{Soṇadaṇḍa}’s praises, if Mister Gotama were staying within a hundred leagues, it’d be worthwhile for a faithful gentleman to go to see him, even if he had to carry his own provisions in a knapsack.”\footnote{“Knapsack” is \textit{\textsanskrit{puṭosa}}, where \textit{\textsanskrit{puṭa}} is “bag” and \textit{osa} is “food” (Sanskrit \textit{avasa}). The compound is a dative-dependent \textit{tappurisa} with reversal of the usual order, yielding the sense “bag for food”. } 

“Well\marginnote{6.47} then, gentlemen, let’s all go to see the ascetic Gotama.” 

\section*{4. \textsanskrit{Soṇadaṇḍa} Has Second Thoughts }

Then\marginnote{7.1} \textsanskrit{Soṇadaṇḍa} together with a large group of brahmins went to see the Buddha. 

But\marginnote{8.1} as he reached the far side of the forest, this thought came to mind, “Suppose I was to ask the ascetic Gotama a question. He might say to me: ‘Brahmin, you shouldn’t ask your question like that. This is how you should ask it.’ And the assembly might disparage me for that: ‘\textsanskrit{Soṇadaṇḍa} is foolish and incompetent. He’s not able to ask the ascetic Gotama a rational question.’\footnote{Even before he meets him, \textsanskrit{Soṇadaṇḍa} has extensive knowledge of the Buddha and faith in him. It’s also notable how \textsanskrit{Soṇadaṇḍa} gives \textsanskrit{Pokkharasāti} a special status. Perhaps \textsanskrit{Pokkharasāti} told \textsanskrit{Soṇadaṇḍa} of the encounter with \textsanskrit{Ambaṭṭha} in \href{https://suttacentral.net/dn3/en/sujato}{DN 3}, which would explain his hesitation. | Note the use of the term \textit{yoniso} here, which I render “rational”. The basic meaning of the term is “womb, source”, and it is used to mean “with reason”, which here has the sense “pertinent, relevant”. } And when you’re disparaged by the assembly, your reputation diminishes. When your reputation diminishes, your wealth also diminishes. But my wealth relies on my reputation. 

Or\marginnote{8.9} if the ascetic Gotama asks me a question, I might not satisfy him with my answer. He might say to me: ‘Brahmin, you shouldn’t answer the question like that. This is how you should answer it.’ And the assembly might disparage me for that: ‘\textsanskrit{Soṇadaṇḍa} is foolish and incompetent. He’s not able to satisfy the ascetic Gotama’s mind with his answer.’ And when you’re disparaged by the assembly, your reputation diminishes. When your reputation diminishes, your wealth also diminishes. But my wealth relies on my reputation. 

On\marginnote{8.16} the other hand, if I were to turn back after having come so far without having seen the ascetic Gotama, the assembly might disparage me for that: ‘\textsanskrit{Soṇadaṇḍa} is foolish and incompetent. He’s stuck-up and scared. He doesn’t dare to go and see the ascetic Gotama. For how on earth can he turn back after having come so far without having seen the ascetic Gotama!’ And when you’re disparaged by the assembly, your reputation diminishes. When your reputation diminishes, your wealth also diminishes. But my wealth relies on my reputation.” 

Then\marginnote{9.1} \textsanskrit{Soṇadaṇḍa} went up to the Buddha, and exchanged greetings with him.\footnote{For all his previous avowal of faith, \textsanskrit{Soṇadaṇḍa} is merely polite, not reverential. This foreshadows the ending. } When the greetings and polite conversation were over, he sat down to one side. Before sitting down to one side, some of the brahmins and householders of \textsanskrit{Campā} bowed, some exchanged greetings and polite conversation, some held up their joined palms toward the Buddha, some announced their name and clan, while some kept silent. 

But\marginnote{10.1} while sitting there, \textsanskrit{Soṇadaṇḍa} continued to be plagued by many second thoughts. He thought, “If only the ascetic Gotama would ask me about my own tutor’s scriptural heritage of the three Vedas! Then I could definitely satisfy his mind with my answer.” 

\section*{5. What Makes a Brahmin }

Then\marginnote{11.1} the Buddha, knowing \textsanskrit{Soṇadaṇḍa}’s train of thought, thought, “This brahmin \textsanskrit{Soṇadaṇḍa} is troubled by his own thoughts. Why don’t I ask him about his own tutor’s scriptural heritage of the three Vedas?”\footnote{The Buddha goes out of his way to make him comfortable. \textsanskrit{Soṇadaṇḍa} is nervous and overly eager to please, but he is not contemptuous and grinding a personal ax like \textsanskrit{Ambaṭṭha}. } 

So\marginnote{11.4} he said to \textsanskrit{Soṇadaṇḍa}, “Brahmin, how many factors must a brahmin possess for the brahmins to describe him as a brahmin; and so that when he says ‘I am a brahmin’ he speaks rightly, without falling into falsehood?” 

Then\marginnote{12.1} \textsanskrit{Soṇadaṇḍa} thought, “The ascetic Gotama has asked me about exactly what I wanted, what I wished for, what I desired, what I yearned for; that is, my own scriptural heritage. I can definitely satisfy his mind with my answer.” 

Then\marginnote{13.1} \textsanskrit{Soṇadaṇḍa} straightened his back, looked around the assembly, and said to the Buddha, “Mister Gotama, a brahmin must possess five factors for the brahmins to describe him as a brahmin; and so that when he says ‘I am a brahmin’ he speaks rightly, without falling into falsehood. What five? It’s when a brahmin is well born on both his mother’s and father’s side, of pure descent, with irrefutable and impeccable genealogy back to the seventh paternal generation. He recites and remembers the hymns, and has mastered the three Vedas, together with their vocabularies and ritual performance, their phonology and word classification, and the testaments as fifth. He knows them word-by-word, and their grammar. He is well versed in cosmology and the marks of a great man. He is attractive, good-looking, lovely, of surpassing beauty. He is magnificent and splendid as the Divinity, remarkable to behold. He is ethical, mature in ethical conduct. He’s astute and clever, being the first or second to hold the sacrificial ladle.\footnote{\textit{Suja} (Sanskrit \textit{sruc}) was a large wooden ladle for pouring ghee on to the sacred flame. } These are the five factors which a brahmin must possess for the brahmins to describe him as a brahmin; and so that when he says ‘I am a brahmin’ he speaks rightly, without falling into falsehood.” 

“But\marginnote{14.1} brahmin, is it possible to set aside one of these five factors and still rightly describe someone as a brahmin?”\footnote{The Buddha uses “Socratic method”. He has the other person articulate a view, then leads them to refine their view, leading to a clearer vision of the truth. } 

“It\marginnote{14.3} is possible, Mister Gotama. We could leave appearance out of the five factors. For what does appearance matter? A brahmin must possess the remaining four factors for the brahmins to rightly describe him as a brahmin.” 

“But\marginnote{15.1} brahmin, is it possible to set aside one of these four factors and still rightly describe someone as a brahmin?” 

“It\marginnote{15.3} is possible, Mister Gotama. We could leave the hymns out of the four factors. For what do the hymns matter? A brahmin must possess the remaining three factors for the brahmins to rightly describe him as a brahmin.” 

“But\marginnote{16.1} brahmin, is it possible to set aside one of these three factors and still rightly describe someone as a brahmin?” 

“It\marginnote{16.3} is possible, Mister Gotama. We could leave birth out of the three factors. For what does birth matter? It’s when a brahmin is ethical, mature in ethical conduct; and he’s astute and clever, being the first or second to hold the sacrificial ladle. A brahmin must possess these two factors for the brahmins to rightly describe him as a brahmin.” 

When\marginnote{17.1} he had spoken, those brahmins said to him, “Please don’t say that, mister \textsanskrit{Soṇadaṇḍa}, please don’t say that! You’re just condemning appearance, the hymns, and birth! You’re totally going over to the ascetic Gotama’s doctrine!” 

So\marginnote{18.1} the Buddha said to them, “Well, brahmins, if you think that \textsanskrit{Soṇadaṇḍa} is unlearned, a poor speaker, witless, and not capable of debating with me about this, then leave him aside and you can debate with me. But if you think that he’s learned, a good speaker who enunciates well, astute, and capable of debating with me about this, then you should stand aside and let him debate with me.” 

When\marginnote{19.1} he said this, \textsanskrit{Soṇadaṇḍa} said to the Buddha, “Let it be, Mister Gotama, be silent. I myself will respond to them in a legitimate manner.” Then he said to those brahmins, “Don’t say this, good sirs, don’t say this: ‘You’re just condemning appearance, the hymns, and birth! You’re totally going over to the ascetic Gotama’s doctrine!’ I’m not condemning appearance, hymns, or birth.” 

Now\marginnote{20.1} at that time \textsanskrit{Soṇadaṇḍa}’s nephew, the young student \textsanskrit{Aṅgaka} was sitting in that assembly. Then \textsanskrit{Soṇadaṇḍa} said to those brahmins, “Gentlemen, do you see my nephew, the young student \textsanskrit{Aṅgaka}?” 

“Yes,\marginnote{20.4} sir.” 

“\textsanskrit{Aṅgaka}\marginnote{20.5} is attractive, good-looking, lovely, of surpassing beauty. He is magnificent and splendid as the Divinity, remarkable to behold. There’s no-one in this assembly so good-looking, apart from the ascetic Gotama. \textsanskrit{Aṅgaka} recites and remembers the hymns, and has mastered the three Vedas, together with their vocabularies and ritual performance, their phonology and word classification, and the testaments as fifth. He knows them word-by-word, and their grammar. He is well versed in cosmology and the marks of a great man. And I am the one who teaches him the hymns. \textsanskrit{Aṅgaka} is well born on both his mother’s and father’s side, of pure descent, with irrefutable and impeccable genealogy back to the seventh paternal generation. And I know his mother and father. But if \textsanskrit{Aṅgaka} were to kill living creatures, steal, commit adultery, lie, and drink liquor, then what’s the use of his appearance, his hymns, or his birth? It’s when a brahmin is ethical, mature in ethical conduct; and he’s astute and clever, being the first or second to hold the sacrificial ladle. A brahmin must possess these two factors for the brahmins to rightly describe him as a brahmin.” 

\section*{6. The Discussion of Ethics and Wisdom }

“But\marginnote{21.1} brahmin, is it possible to set aside one of these two factors and still rightly describe someone as a brahmin?” 

“No,\marginnote{21.3} Mister Gotama. For wisdom is cleansed by ethics, and ethics are cleansed by wisdom. Ethics and wisdom always go together. An ethical person is wise, and a wise person ethical. And ethics and wisdom are said to be the best things in the world. It’s just like when you clean one hand with the other, or clean one foot with the other. In the same way, wisdom is cleansed by ethics, and ethics are cleansed by wisdom. Ethics and wisdom always go together. An ethical person is wise, and a wise person ethical. And ethics and wisdom are said to be the best things in the world.” 

“That’s\marginnote{22.1} so true, brahmin, that’s so true! For wisdom is cleansed by ethics, and ethics are cleansed by wisdom.\footnote{When the Buddha convinces \textsanskrit{Soṇadaṇḍa}, he does not crow over it or take pleasure in his discomfort, but praises his wisdom and affirms his conclusions. } Ethics and wisdom always go together. An ethical person is wise, and a wise person ethical. And ethics and wisdom are said to be the best things in the world. It’s just like when you clean one hand with the other, or clean one foot with the other. In the same way, wisdom is cleansed by ethics, and ethics are cleansed by wisdom. Ethics and wisdom always go together. An ethical person is wise, and a wise person ethical. And ethics and wisdom are said to be the best things in the world. 

But\marginnote{22.10} what, brahmin, is that ethical conduct?\footnote{The Buddha has taken pains to establish common ground, but \textsanskrit{Soṇadaṇḍa}’s description of wisdom is lacking, so the Buddha prompts a deeper inquiry. } And what is that wisdom?” 

“That’s\marginnote{22.12} all I know about this matter, Mister Gotama. May Mister Gotama himself please clarify the meaning of this.” 

“Well\marginnote{23.1} then, brahmin, listen and apply your mind well, I will speak.” 

“Yes\marginnote{23.2} sir,” \textsanskrit{Soṇadaṇḍa} replied. The Buddha said this: 

“It’s\marginnote{23.4} when a Realized One arises in the world, perfected, a fully awakened Buddha … That’s how a mendicant is accomplished in ethics. This, brahmin, is that ethical conduct. … They enter and remain in the first absorption …\footnote{Normally, the Buddha frames his teaching as ethics, meditation, and wisdom, but here he adapts to \textsanskrit{Soṇadaṇḍa}’s framing and places meditation under wisdom. } second absorption … third absorption … fourth absorption … They project and extend the mind toward knowledge and vision … This pertains to their wisdom. … They understand: ‘… there is nothing further for this place.’ This pertains to their wisdom. This, brahmin, is that wisdom.” 

\section*{7. \textsanskrit{Soṇadaṇḍa} Declares Himself a Lay Follower }

When\marginnote{24.1} he had spoken, \textsanskrit{Soṇadaṇḍa} said to the Buddha, “Excellent, Mister Gotama! Excellent! As if he were righting the overturned, or revealing the hidden, or pointing out the path to the lost, or lighting a lamp in the dark so people with clear eyes can see what’s there, Mister Gotama has made the Teaching clear in many ways. I go for refuge to Mister Gotama, to the teaching, and to the mendicant \textsanskrit{Saṅgha}. From this day forth, may Mister Gotama remember me as a lay follower who has gone for refuge for life. Would you and the mendicant \textsanskrit{Saṅgha} please accept a meal from me tomorrow?” The Buddha consented with silence. 

Then,\marginnote{24.8} knowing that the Buddha had consented, \textsanskrit{Soṇadaṇḍa} got up from his seat, bowed, and respectfully circled the Buddha, keeping him on his right, before leaving. And when the night had passed \textsanskrit{Soṇadaṇḍa} had delicious fresh and cooked foods prepared in his own home. Then he had the Buddha informed of the time, saying, “It’s time, Mister Gotama, the meal is ready.” Then the Buddha robed up in the morning and, taking his bowl and robe, went to the home of \textsanskrit{Soṇadaṇḍa} together with the mendicant \textsanskrit{Saṅgha}, where he sat on the seat spread out. Then \textsanskrit{Soṇadaṇḍa} served and satisfied the mendicant \textsanskrit{Saṅgha} headed by the Buddha with his own hands with delicious fresh and cooked foods. 

When\marginnote{26.1} the Buddha had eaten and washed his hand and bowl, \textsanskrit{Soṇadaṇḍa} took a low seat and sat to one side. Seated to one side he said to the Buddha: “Mister Gotama, if, when I have gone to an assembly, I rise from my seat and bow to the Buddha, that assembly might disparage me for that. And when you’re disparaged by the assembly, your reputation diminishes. When your reputation diminishes, your wealth also diminishes. But my wealth relies on my reputation. If, when I have gone to an assembly, I raise my joined palms, please take it that I have risen from my seat. And if I undo my turban, please take it that I have bowed. And Mister Gotama, if, when I am in a carriage, I get down from my carriage and bow to the Buddha, that assembly might disparage me for that. If, when I am in a carriage, I hold up my goad, please take it that I have got down from my carriage. And if I lower my sunshade, please take it that I have bowed.”\footnote{\textsanskrit{Soṇadaṇḍa}’s attachment to his reputation betrays his lack of inner confidence and stands in contrast with \textsanskrit{Pokkharasāti}. Perhaps it may also be read as a regional characteristic, since Buddhism  was less well established in the \textsanskrit{Aṅga} region than in central Magadha. } 

Then\marginnote{27.1} the Buddha educated, encouraged, fired up, and inspired the brahmin \textsanskrit{Soṇadaṇḍa} with a Dhamma talk, after which he got up from his seat and left. 

%
\chapter*{{\suttatitleacronym DN 5}{\suttatitletranslation With Kūṭadanta }{\suttatitleroot Kūṭadantasutta}}
\addcontentsline{toc}{chapter}{\tocacronym{DN 5} \toctranslation{With Kūṭadanta } \tocroot{Kūṭadantasutta}}
\markboth{With Kūṭadanta }{Kūṭadantasutta}
\extramarks{DN 5}{DN 5}

\section*{1. The Brahmins and Householders of \textsanskrit{Khāṇumata} }

\scevam{So\marginnote{1.1} I have heard.\footnote{When asked about sacrifice, the Buddha tells a story of how a king was persuaded to forgo cruelty and institute a constructive social policy of welfare instead. } }At one time the Buddha was wandering in the land of the Magadhans together with a large \textsanskrit{Saṅgha} of five hundred mendicants when he arrived at a village of the Magadhan brahmins named \textsanskrit{Khāṇumata}.\footnote{\textsanskrit{Khāṇumata} is only mentioned here. It means “stumpy”, perhaps indicating that it was, or had recently been, a rough out-of-the-way place. } There he stayed nearby at \textsanskrit{Ambalaṭṭhikā}.\footnote{The commentary says this was similar to the \textsanskrit{Ambalaṭṭhikā} of \href{https://suttacentral.net/dn1/en/sujato\#1.2.1}{DN 1:1.2.1}. } 

Now\marginnote{1.4} at that time the brahmin \textsanskrit{Kūṭadanta} was living in \textsanskrit{Khāṇumata}. It was a crown property given by King Seniya \textsanskrit{Bimbisāra} of Magadha, teeming with living creatures, full of hay, wood, water, and grain, a royal park endowed to a brahmin.\footnote{Like the town \textsanskrit{Khāṇumata}, the brahmin \textsanskrit{Kūṭadanta} appears only here. His name means “pointy teeth”, but there is no evidence that he was a vampire. } Now at that time \textsanskrit{Kūṭadanta} had prepared a great sacrifice. Bulls, bullocks, heifers, goats and rams—seven hundred of each—had been led to the post for the sacrifice.\footnote{At \href{https://suttacentral.net/snp2.7/en/sujato\#26.1}{Snp 2.7:26.1}, the sacrifice of cows in particular was said to be the nadir of decline for brahmins. When similar sacrifices are described elsewhere in the canon, the number is five hundred rather than seven hundred (\href{https://suttacentral.net/sn3.9/en/sujato\#1.2}{SN 3.9:1.2}, \href{https://suttacentral.net/an7.47/en/sujato\#1.2}{AN 7.47:1.2}). } 

The\marginnote{2.1} brahmins and householders of \textsanskrit{Khāṇumata} heard: 

“It\marginnote{2.2} seems the ascetic Gotama—a Sakyan, gone forth from a Sakyan family—has arrived at \textsanskrit{Khāṇumata} and is staying at \textsanskrit{Ambalaṭṭhikā}. He has this good reputation: ‘That Blessed One is perfected, a fully awakened Buddha, accomplished in knowledge and conduct, holy, knower of the world, supreme guide for those who wish to train, teacher of gods and humans, awakened, blessed.’ He has realized with his own insight this world—with its gods, \textsanskrit{Māras}, and divinities, this population with its ascetics and brahmins, gods and humans—and he makes it known to others. He proclaims a teaching that is good in the beginning, good in the middle, and good in the end, meaningful and well-phrased. And he reveals a spiritual practice that’s entirely full and pure. It’s good to see such perfected ones.” 

Then,\marginnote{2.8} having departed \textsanskrit{Khāṇumata}, they formed into companies and headed to \textsanskrit{Ambalaṭṭhikā}. 

Now\marginnote{3.1} at that time the brahmin \textsanskrit{Kūṭadanta} had retired to the upper floor of his stilt longhouse for his midday nap. He saw the brahmins and householders heading for \textsanskrit{Ambalaṭṭhikā}, and addressed his butler, “My butler, why are the brahmins and householders headed for \textsanskrit{Ambalaṭṭhikā}?” 

“The\marginnote{3.5} ascetic Gotama has arrived at \textsanskrit{Khāṇumata} and is staying at \textsanskrit{Ambalaṭṭhikā}. He has this good reputation: ‘That Blessed One is perfected, a fully awakened Buddha, accomplished in knowledge and conduct, holy, knower of the world, supreme guide for those who wish to train, teacher of gods and humans, awakened, blessed.’ They’re going to see that Mister Gotama.” 

Then\marginnote{4.1} \textsanskrit{Kūṭadanta} thought, “I’ve heard that the ascetic Gotama knows how to accomplish the sacrifice with three modes and sixteen accessories.\footnote{News of the Buddha had spread in Brahmanical circles. In \href{https://suttacentral.net/dn3/en/sujato}{DN 3} we saw the Buddha use his rhetorical technique of adapting his teaching to reframe Brahmanical doctrines in order to establish a common ground. Here we see the downside to such techniques, as the details of the reframing have been lost. } I don’t know about that, but I wish to perform a great sacrifice. Why don’t I ask him how to accomplish the sacrifice with three modes and sixteen accessories?”\footnote{No such sacrifice has been identified in Brahmanical texts. } 

Then\marginnote{4.7} \textsanskrit{Kūṭadanta} addressed his butler, “Well then, go to the brahmins and householders and say to them: ‘Sirs, the brahmin \textsanskrit{Kūṭadanta} asks you to wait, as he will also go to see the ascetic Gotama.’” 

“Yes,\marginnote{4.11} sir,” replied the butler, and did as he was asked. 

\section*{2. The Qualities of \textsanskrit{Kūṭadanta} }

Now\marginnote{5.1} at that time several hundred brahmins were residing in \textsanskrit{Khāṇumata} thinking to participate in \textsanskrit{Kūṭadanta}’s sacrifice. They heard that \textsanskrit{Kūṭadanta} was going to see the ascetic Gotama. They approached \textsanskrit{Kūṭadanta} and said to him: 

“Is\marginnote{5.6} it really true that you are going to see the ascetic Gotama?” 

“Yes,\marginnote{5.7} gentlemen, it is true.” 

“Please\marginnote{6.1} don’t! It’s not appropriate for you to go to see the ascetic Gotama. For if you do so, your reputation will diminish and his will increase. For this reason it’s not appropriate for you to go to see the ascetic Gotama; it’s appropriate that he comes to see you. 

You\marginnote{6.6} are well born on both your mother’s and father’s side, of pure descent, with irrefutable and impeccable genealogy back to the seventh paternal generation. For this reason it’s not appropriate for you to go to see the ascetic Gotama; it’s appropriate that he comes to see you. 

You’re\marginnote{6.9} rich, affluent, and wealthy, with lots of property and assets, and lots of money and grain … 

You\marginnote{6.10} recite and remember the hymns, and have mastered the three Vedas, together with their vocabularies and ritual performance, their phonology and word classification, and the testaments as fifth. You know them word-by-word, and their grammar. You are well versed in cosmology and the marks of a great man. … 

You\marginnote{6.11} are attractive, good-looking, lovely, of surpassing beauty. You are magnificent and splendid as the Divinity, remarkable to behold. … 

You\marginnote{6.12} are ethical, mature in ethical conduct. … 

You’re\marginnote{6.13} a good speaker who enunciates well, with a polished, clear, and articulate voice that expresses the meaning. … 

You\marginnote{6.14} teach the tutors of many, and teach three hundred young students to recite the hymns. Many students come from various districts and countries for the sake of the hymns, wishing to learn the hymns. … 

You’re\marginnote{6.15} old, elderly and senior, advanced in years, and have reached the final stage of life. The ascetic Gotama is young, and has newly gone forth. … 

You’re\marginnote{6.17} honored, respected, revered, venerated, and esteemed by King \textsanskrit{Bimbisāra} of Magadha … 

and\marginnote{6.18} the brahmin \textsanskrit{Pokkharasāti}. … 

You\marginnote{6.19} live in \textsanskrit{Khāṇumata}, a crown property given by King Seniya \textsanskrit{Bimbisāra} of Magadha, teeming with living creatures, full of hay, wood, water, and grain, a royal park endowed to a brahmin. For this reason it’s not appropriate for you to go to see the ascetic Gotama; it’s appropriate that he comes to see you.” 

\section*{3. The Qualities of the Buddha }

When\marginnote{7.1} they had spoken, \textsanskrit{Kūṭadanta} said to those brahmins: 

“Well\marginnote{7.2} then, gentlemen, listen to why it’s appropriate for me to go to see the ascetic Gotama, and it’s not appropriate for him to come to see me. He is well born on both his mother’s and father’s side, of pure descent, with irrefutable and impeccable genealogy back to the seventh paternal generation. For this reason it’s not appropriate for the ascetic Gotama to come to see me; rather, it’s appropriate for me to go to see him. 

When\marginnote{7.7} he went forth he abandoned a large family circle. … 

When\marginnote{7.8} he went forth he abandoned abundant gold, both coined and uncoined, stored in dungeons and towers. … 

He\marginnote{7.9} went forth from the lay life to homelessness while still a youth, young, with pristine black hair, blessed with youth, in the prime of life. … 

Though\marginnote{7.10} his mother and father wished otherwise, weeping with tearful faces, he shaved off his hair and beard, dressed in ocher robes, and went forth from the lay life to homelessness. … 

He\marginnote{7.11} is attractive, good-looking, lovely, of surpassing beauty. He is magnificent and splendid as the Divinity, remarkable to behold. … 

He\marginnote{7.12} is ethical, possessing ethical conduct that is noble and skillful. … 

He’s\marginnote{7.13} a good speaker who enunciates well, with a polished, clear, and articulate voice that expresses the meaning. … 

He’s\marginnote{7.14} a tutor of tutors. … 

He\marginnote{7.15} has ended sensual desire, and is rid of caprice. … 

He\marginnote{7.16} teaches the efficacy of deeds and action. He doesn’t wish any harm upon the community of brahmins. … 

He\marginnote{7.17} went forth from an eminent family of unbroken aristocratic lineage. … 

He\marginnote{7.18} went forth from a rich, affluent, and wealthy family. … 

People\marginnote{7.19} come from distant lands and distant countries to question him. … 

Many\marginnote{7.20} thousands of deities have gone for refuge for life to him. … 

He\marginnote{7.21} has this good reputation: ‘That Blessed One is perfected, a fully awakened Buddha, accomplished in knowledge and conduct, holy, knower of the world, supreme guide for those who wish to train, teacher of gods and humans, awakened, blessed.’ … 

He\marginnote{7.23} has the thirty-two marks of a great man. … 

He\marginnote{7.24} is welcoming, congenial, polite, smiling, open, the first to speak. … 

He’s\marginnote{7.25} honored, respected, revered, venerated, and esteemed by the four assemblies. … 

Many\marginnote{7.26} gods and humans are devoted to him. … 

While\marginnote{7.27} he is residing in a village or town, non-human entities do not harass them. … 

He\marginnote{7.28} leads an order and a community, and tutors a community, and is said to be the best of the various religious founders. He didn’t come by his fame in the same ways as those other ascetics and brahmins. Rather, he came by his fame due to his supreme knowledge and conduct. … 

King\marginnote{7.30} Seniya \textsanskrit{Bimbisāra} of Magadha and his wives and children have gone for refuge for life to the ascetic Gotama. … 

King\marginnote{7.31} Pasenadi of Kosala and his wives and children have gone for refuge for life to the ascetic Gotama. … 

The\marginnote{7.32} brahmin \textsanskrit{Pokkharasāti} and his wives and children have gone for refuge for life to the ascetic Gotama. … 

He’s\marginnote{7.33} honored, respected, revered, venerated, and esteemed by King \textsanskrit{Bimbisāra} of Magadha … 

King\marginnote{7.34} Pasenadi of Kosala … 

and\marginnote{7.35} the brahmin \textsanskrit{Pokkharasāti}. 

The\marginnote{7.36} ascetic Gotama has arrived at \textsanskrit{Khāṇumata} and is staying at \textsanskrit{Ambalaṭṭhikā}. Any ascetic or brahmin who comes to stay in our village district is our guest, and should be honored and respected as such. For this reason, too, it’s not appropriate for Mister Gotama to come to see me, rather, it’s appropriate for me to go to see him. This is the extent of Mister Gotama’s praise that I have memorized. But his praises are not confined to this, for the praise of Mister Gotama is limitless.” 

When\marginnote{7.45} he had spoken, those brahmins said to him, “According to \textsanskrit{Kūṭadanta}’s praises, if Mister Gotama were staying within a hundred leagues, it’d be worthwhile for a faithful gentleman to go to see him, even if he had to carry his own provisions in a knapsack.” 

“Well\marginnote{7.47} then, gentlemen, let’s all go to see the ascetic Gotama.” 

\section*{4. The Story of King \textsanskrit{Mahāvijita}’s Sacrifice }

Then\marginnote{8.1} \textsanskrit{Kūṭadanta} together with a large group of brahmins went to see the Buddha and exchanged greetings with him. When the greetings and polite conversation were over, he sat down to one side. Before sitting down to one side, some of the brahmins and householders of \textsanskrit{Khāṇumata} bowed, some exchanged greetings and polite conversation, some held up their joined palms toward the Buddha, some announced their name and clan, while some kept silent. 

\textsanskrit{Kūṭadanta}\marginnote{9.1} said to the Buddha, “Mister Gotama, I’ve heard that you know how to accomplish the sacrifice with three modes and sixteen accessories. I don’t know about that, but I wish to perform a great sacrifice. Please teach me how to accomplish the sacrifice with three modes and sixteen accessories.” 

“Well\marginnote{9.7} then, brahmin, listen and apply your mind well, I will speak.”\footnote{\textsanskrit{Kūṭadanta}’s proposal that the Buddha advise him on the sacrifice of 700 animals is outrageous. Nonetheless, the Buddha responds politely since \textsanskrit{Kūṭadanta} is being polite. While it is tempting to see the sacrifice of animals by supposedly virtuous priests as sheer hypocrisy, the fact of sacrifice remains one of the most widespread and mysterious of human religious practices. In an empathetic work that addresses this squarely, Roberto Calasso’s \emph{Ardor} sees the vast complex of Vedic ritual and theory as making plain the fact of killing so that the guilt may be contained, in contrast with our modern culture of killing on an industrial scale while hiding it out of sight. } 

“Yes\marginnote{9.8} sir,” \textsanskrit{Kūṭadanta} replied. The Buddha said this: “Once upon a time, brahmin, there was a king named \textsanskrit{Mahāvijita}. He was rich, affluent, and wealthy, with lots of gold and silver, lots of property and assets, lots of money and grain, and a full treasury and storehouses.\footnote{\textsanskrit{Mahāvijita} means “Great Dominion”. He seems to be only known from this story. The idiom \textit{\textsanskrit{bhūtapubbaṁ}} (literally “so it was in the past”) introduces legendary narratives of usually dubious historicity, like the English idiom “once upon a time”. } Then as King \textsanskrit{Mahāvijita} was in private retreat this thought came to his mind:\footnote{In the Pali, meaningful thoughts often occur to people when withdrawn in seclusion. This doesn’t necessary mean they were in formal meditation. } ‘I have achieved human wealth, and reign after conquering this vast territory. Why don’t I hold a large sacrifice? That will be for my lasting welfare and happiness.’\footnote{The great sacrifices, especially the horse sacrifice, ensured royal authority. Their very scale and wastefulness showed off the wealth of the king. } 

Then\marginnote{10.4} he summoned the brahmin high priest and said to him:\footnote{“High priest” is \textit{purohita}. He was a family chaplain advising and consecrating the royal family. The closeness of the relationship is shown by the fact that the royal family would take the lineage name of the \textit{purohita}. } ‘Just now, brahmin, as I was in private retreat this thought came to mind, “I have achieved human wealth, and reign after conquering this vast territory. Why don’t I perform a great sacrifice? That will be for my lasting welfare and happiness.” Brahmin, I wish to perform a great sacrifice. Please instruct me. It will be for my lasting welfare and happiness.’ 

When\marginnote{11.1} he had spoken, the brahmin high priest said to him: ‘Sir, the king’s realm is harried and oppressed. Raiding of villages, towns, and cities has been seen, and infesting of highways.\footnote{Then, as today, government policy was driven by the perception of rising crime rate. This whole passage is one of the Buddha’s most important statements on public policy. It is expressed through storytelling, giving a good example of how myths were invoked—and subverted—as rationales for current policy. } But if the king were to extract more taxes while his realm is thus harried and oppressed, he would not be doing his duty.\footnote{“Taxes” is \textit{bali}. He would have had to press his people for the extra funds to hold the sacrifice. } 

Now\marginnote{11.4} the king might think, “I’ll eradicate this plague of savages by execution or imprisonment or confiscation or condemnation or banishment!” But that’s not the right way to eradicate this plague of savages.\footnote{“Plague of savages” (\textit{\textsanskrit{dassukhīla}}) is the only occurrence of Sanskrit \textit{dasyu} in early Pali. The \textit{dasyu} were inveterate foes of the Aryans in the Vedic period. Bereft of civilizing rites, scriptures, and observances, they were no children of Manu (Rig Veda 10.22.8). Their wiles (\textit{\textsanskrit{māyā}}) made them a potent threat (Rig Veda 4.16.9, 8.14.14, 10.73.5). Indra was invoked to ensure their destruction (Rig Veda \emph{passim}; Atharva Veda 2.14.5, 4.32.3, 20.21.4, 20.37.5, 20.42.2). Legend has it that Agni and Soma first supported the \textit{dasyu} before being won over by Indra (Śatapatha \textsanskrit{Brāhmaṇa} 1.6.3.13; see also 6.4.2.4). By the time of the Buddha the \textit{dasyu} have vanished except as a legendary foe of the past. Where the Brahmanical texts advocate the pitiless destruction of the \textit{dasyu}, the brahmin high priest in the Buddhist text advocates an inclusive policy of social welfare. } Those who remain after the killing will return to harass the king’s realm.\footnote{The priest knows that the king will respond better to a pragmatic argument than a moral one. } 

Rather,\marginnote{11.7} here is a plan, relying on which the plague of savages will be properly uprooted.\footnote{Effective social policy requires a forward-thinking plan, not just reacting to grievances. } So let the king provide seed and fodder for those in the realm who work in growing crops and raising cattle.\footnote{The king should spend his own resources to support his citizens in the various occupations. } Let the king provide funding for those who work in trade. Let the king guarantee food and wages for those in government service. Then the people, occupied with their own work, will not harass the realm.\footnote{Here the priest identifies a fundamental cause of social unrest and disorder. } The king’s revenues will be great.\footnote{The king spends out of pocket, but the economy flourishes, so tax revenues increase even though he has not raised taxes. This is the essence of Keynesian economic theory. } When the country is secured as a sanctuary, free of being harried and oppressed, the happy people, with joy in their hearts, dancing with children at their breast, will dwell as if their houses were wide open.’\footnote{\textit{Khema} means a place of safety and sanctuary, where both humans and animals have no fear. } 

The\marginnote{11.14} king agreed with the high priest’s advice and followed his recommendation.\footnote{A good leader listens to advice. } 

Then\marginnote{11.19} the king summoned the brahmin high priest and said to him: ‘I have eradicated the plague of savages. And relying on your plan my revenue is now great. Since the country is secured as a sanctuary, free of being harried and oppressed, the happy people, with joy in their hearts, dancing with children at their breast, dwell as if their houses were wide open. Brahmin, I wish to perform a great sacrifice. Please instruct me. It will be for my lasting welfare and happiness.’ 

\subsection*{4.1. The Four Accessories }

‘In\marginnote{12.1} that case, let the king announce this throughout the realm to the aristocrat vassals of both town and country; the ministers and councillors of both town and country; the well-to-do brahmins of both town and country; and the well-off householders of both town and country.\footnote{Here the phrase “of both town and country” qualifies each group. At \href{https://suttacentral.net/an4.70/en/sujato\#1.3}{AN 4.70:1.3}, however, “brahmins and householders” and “people of town and country” are separate groups of people. Elsewhere the context does not always decide between these two possibilities. Generally the idiom aims at inclusivity, as opposed to here where the king is consulting the rich and powerful, so I treat them as two separate groups, thus including the common folk. } “I wish to perform a great sacrifice. Please grant your approval, good sirs; it will be for my lasting welfare and happiness.” 

The\marginnote{12.3} king agreed with the high priest’s advice and followed his recommendation. And all of the people who were thus informed responded by saying: ‘May the king perform a sacrifice! It is time for a sacrifice, great king.’ And so these four consenting factions became accessories to the sacrifice.\footnote{“Consenting factions” is \textit{\textsanskrit{anumatipakkhā}}. The king governs with the consent of his people, although only the landowning classes are considered. } 

\subsection*{4.2. The Eight Accessories }

King\marginnote{13.1} \textsanskrit{Mahāvijita} possessed eight factors.\footnote{Royal authority is not based just on birth, conquest, ritual, or power, but on quality of character. } 

He\marginnote{13.2} was well born on both his mother’s and father’s side, of pure descent, with irrefutable and impeccable genealogy back to the seventh paternal generation. 

He\marginnote{13.3} was attractive, good-looking, lovely, of surpassing beauty. He was magnificent and splendid as the Divinity, remarkable to behold. 

He\marginnote{13.4} was rich, affluent, and wealthy, with lots of gold and silver, lots of property and assets, lots of money and grain, and a full treasury and storehouses. 

He\marginnote{13.5} was powerful, having an army of four divisions that was obedient and carried out instructions. He’d probably prevail over his enemies just with his reputation.\footnote{Read \textit{sahati} (“prevails”) over the several variants. } 

He\marginnote{13.6} was faithful, generous, a donor, his door always open. He was a well-spring of support, making merit with ascetics and brahmins, for paupers, vagrants, supplicants, and beggars. 

He\marginnote{13.7} was very learned in diverse fields of learning. He understood the meaning of diverse statements, saying:\footnote{Showing the importance of comprehension over blind adherence to tradition. } ‘This is what that statement means; that is what this statement means.’ 

He\marginnote{13.9} was astute, competent, and intelligent, able to think issues through as they bear upon the past, future, and present.\footnote{Meditators focus on the present, but that does not mean they cannot think about the past or future; it just means they are not trapped in useless thoughts. } 

These\marginnote{13.10} are the eight factors that King \textsanskrit{Mahāvijita} possessed. And so these eight factors also became accessories to the sacrifice. 

\subsection*{4.3. Four More Accessories }

And\marginnote{14.1} the brahmin high priest had four factors.\footnote{These are four of the five qualities that \textsanskrit{Soṇadaṇḍa} identifies as the qualities of a brahmin at \href{https://suttacentral.net/dn4/en/sujato\#13.2}{DN 4:13.2}. Missing is appearance, which is the first factor that \textsanskrit{Soṇadaṇḍa} admits is unnecessary. } 

He\marginnote{14.2} was well born on both his mother’s and father’s side, of pure descent, with irrefutable and impeccable genealogy back to the seventh paternal generation. 

He\marginnote{14.3} recited and remembered the hymns, and had mastered the three Vedas, together with their vocabularies and ritual performance, their phonology and word classification, and the testaments as fifth. He knew them word-by-word, and their grammar. He was well versed in cosmology and the marks of a great man. 

He\marginnote{14.4} was ethical, mature in ethical conduct. 

He\marginnote{14.5} was astute and clever, being the first or second to hold the sacrificial ladle. 

These\marginnote{14.6} are the four factors that the brahmin high priest possessed. And so these four factors also became accessories to the sacrifice. 

\subsection*{4.4. The Three Modes }

Next,\marginnote{15.1} before the sacrifice, the brahmin high priest taught the three modes to the king. ‘Now, though the king wants to perform a great sacrifice, he might have certain regrets, thinking: “I shall lose a great fortune,” or\footnote{Compare the three factors of a donor’s mind-state at \href{https://suttacentral.net/an6.37/en/sujato\#2.4}{AN 6.37:2.4}. } “I am losing a great fortune,” or “I have lost a great fortune.” But the king should not harbor such regrets.’ 

These\marginnote{15.8} are the three modes that the brahmin high priest taught to the king before the sacrifice. 

\subsection*{4.5. The Ten Respects }

Next,\marginnote{16.1} before the sacrifice, the brahmin high priest dispelled the king’s regret regarding the recipients in ten respects:\footnote{What a recipient does with a gift is beyond the donor’s control. } 

‘There\marginnote{16.2} will come to the sacrifice those who kill living creatures and those who refrain from killing living creatures. As to those who kill living creatures, the outcome of that is theirs alone. But as to those who refrain from killing living creatures, it is for their sakes that the king should sacrifice, relinquish, rejoice, and gain confidence in his heart.\footnote{PTS edition acknowledges \textit{sajjata} only as a variant reading, but it is in the commentary, so should be accepted in the text. It is from √\textit{sajj} (relinquish). } 

There\marginnote{16.5} will come to the sacrifice those who steal … commit sexual misconduct … lie … use divisive speech … use harsh speech … talk nonsense … are covetous … have ill will … have wrong view and those who have right view. As to those who have wrong view, the outcome of that is theirs alone. But as to those who have right view, it is for their sakes that the king should sacrifice, relinquish, rejoice, and gain confidence in his heart.’ 

These\marginnote{16.16} are the ten respects in which the high priest dispelled the king’s regret regarding the recipients before the sacrifice. 

\subsection*{4.6. The Sixteen Respects }

Next,\marginnote{17.1} while the king was performing the great sacrifice, the brahmin high priest educated, encouraged, fired up, and inspired the king’s mind in sixteen respects: 

‘Now,\marginnote{17.2} while the king is performing the great sacrifice, someone might say, “King \textsanskrit{Mahāvijita} performs a great sacrifice, but he did not announce it to the aristocrat vassals of town and country.\footnote{Another lesson in leadership: the importance of communication. } That’s the kind of great sacrifice that this king performs.” Those who speak against the king in this way have no legitimacy. For the king did indeed announce it to the aristocrat vassals of town and country. Let the king know this as a reason to sacrifice, relinquish, rejoice, and gain confidence in his heart. 

While\marginnote{17.8} the king is performing the great sacrifice, someone might say, “King \textsanskrit{Mahāvijita} performs a great sacrifice, but he did not announce it to the aristocrat vassals; the ministers and councillors; the well-to-do brahmins; and the well-off householders. That’s the kind of great sacrifice that this king performs.” Those who speak against the king in this way have no legitimacy. For the king did indeed announce it to all these people.\footnote{Due to abbreviation, the text only mentions householders here, but clearly all are intended. } Let the king know this too as a reason to sacrifice, relinquish, rejoice, and gain confidence in his heart. 

While\marginnote{17.13} the king is performing the great sacrifice, someone might say that he does not possess the eight factors. Those who speak against the king in this way have no legitimacy. For the king does indeed possess the eight factors. Let the king know this too as a reason to sacrifice, relinquish, rejoice, and gain confidence in his heart. 

While\marginnote{17.30} the king is performing the great sacrifice, someone might say that the high priest does not possess the four factors. Those who speak against the king in this way have no legitimacy. For the high priest does indeed possess the four factors. Let the king know this too as a reason to sacrifice, relinquish, rejoice, and gain confidence in his heart.’ 

These\marginnote{17.45} are the sixteen respects in which the high priest educated, encouraged, fired up, and inspired the king’s mind while he was performing the sacrifice. 

And\marginnote{18.1} brahmin, in that sacrifice no cattle were killed, no goats or sheep were killed, and no chickens or pigs were killed. There was no slaughter of various kinds of creatures. No trees were felled for the sacrificial post. No grass was reaped to strew over the place of sacrifice. No bondservants, servants, or workers did their jobs under threat of punishment and danger, weeping with tearful faces. Those who wished to work did so, while those who did not wish to did not.\footnote{A leader gets results through inspiration, not fear. } They did the work they wanted to, and did not do what they didn’t want to. The sacrifice was completed with just ghee, oil, butter, curds, honey, and molasses.\footnote{These were regarded as valuable yet harmless products. } 

Then\marginnote{19.1} the aristocrat vassals, ministers and councillors, well-to-do brahmins, and well-off householders came to the king bringing abundant wealth and said, ‘Sire, this abundant wealth is specially for you alone; may Your Highness accept it!’\footnote{The king’s generosity and sincerity brings out the best in the others. } 

‘There’s\marginnote{19.3} enough raised for me through regular taxes. Let this be for you; and here, take even more!’ 

When\marginnote{19.5} the king turned them down, they withdrew to one side to think up a plan, ‘It wouldn’t be proper for us to take this abundant wealth back to our own homes. King \textsanskrit{Mahāvijita} is performing a great sacrifice. Let us make an offering as an auxiliary sacrifice.’ 

Then\marginnote{20.1} the aristocrat vassals of town and country set up gifts to the east of the sacrificial enclosure. The ministers and councillors of town and country set up gifts to the south of the sacrificial enclosure. The well-to-do brahmins of town and country set up gifts to the west of the sacrificial enclosure. The well-off householders of town and country set up gifts to the north of the sacrificial enclosure. 

And\marginnote{20.5} brahmin, in that sacrifice too no cattle were killed, no goats were killed, and no chickens or pigs were killed. There was no slaughter of various kinds of creatures. No trees were felled for the sacrificial post. No grass was reaped to strew over the place of sacrifice. No bondservants, servants, or workers did their jobs under threat of punishment and danger, weeping with tearful faces. Those who wished to work did so, while those who did not wish to did not. They did the work they wanted to, and did not do what they didn’t want to. The sacrifice was completed with just ghee, oil, butter, curds, honey, and molasses. 

And\marginnote{20.10} so there were four consenting factions, eight factors possessed by King \textsanskrit{Mahāvijita}, four factors possessed by the high priest, and three modes. Brahmin, this is called the sacrifice accomplished with three modes and sixteen accessories.” 

When\marginnote{21.1} he said this, those brahmins made an uproar,\footnote{Here ends the Buddha’s legendary account of the past. } “Hooray for such sacrifice! Hooray for the accomplishment of such sacrifice!” 

But\marginnote{21.3} the brahmin \textsanskrit{Kūṭadanta} sat in silence.\footnote{The other brahmins are satisfied, but \textsanskrit{Kūṭadanta} senses there is more to it. } So those brahmins said to him, “How can you not applaud the ascetic Gotama’s fine words?” 

“It’s\marginnote{21.6} not that I don’t applaud what he said. If anyone didn’t applaud such fine words, their head would explode! 

But,\marginnote{21.8} gentlemen, it occurs to me that the ascetic Gotama does not say: ‘So I have heard’ or ‘It ought to be like this.’\footnote{“So I have heard” (\textit{\textsanskrit{evaṁ} me \textsanskrit{sutaṁ}}) is the standard opening for Buddhist suttas. This tag was used to indicate that the speaker was not present at the events, but “heard” about them. This is in contrast with the phrase “I heard and learned this in the presence” (\textit{\textsanskrit{sammukhā} \textsanskrit{sutaṁ}, \textsanskrit{sammukhā} \textsanskrit{paṭiggahitaṁ}}), which is used when reporting a teaching heard directly from the Buddha, eg. \href{https://suttacentral.net/sn55.52/en/sujato\#5.1}{SN 55.52:5.1}, \href{https://suttacentral.net/sn22.90/en/sujato\#9.1}{SN 22.90:9.1}, \href{https://suttacentral.net/mn47/en/sujato\#10.7}{MN 47:10.7}, etc. } Rather, he just says: ‘So it was then, this is how it was then.’ 

It\marginnote{21.13} occurs to me that the ascetic Gotama at that time must have been King \textsanskrit{Mahāvijita}, the owner of the sacrifice, or else the brahmin high priest who facilitated the sacrifice for him. 

Does\marginnote{21.15} Mister Gotama recall having performed such a sacrifice, or having facilitated it, and then, when his body broke up, after death, being reborn in a good place, a heavenly realm?” 

“I\marginnote{21.16} do recall that, brahmin. For I myself was the brahmin high priest at that time who facilitated the sacrifice.”\footnote{This qualifies the story of \textsanskrit{Mahāvijita} as an early canonical \textsanskrit{Jātaka}. There are a small number of such stories in the early suttas, only some of which overlap with the later \textsanskrit{Jātaka} collections, the story of \textsanskrit{Mahāvijita} not being among them. } 

\section*{5. A Regular Gift as an Ongoing Family Sacrifice. }

“But\marginnote{22.1} Mister Gotama, apart from that sacrifice accomplished with three modes and sixteen accessories, is there any other sacrifice that has fewer requirements and undertakings, yet is more fruitful and beneficial?”\footnote{\textsanskrit{Kūṭadanta} is hoping for a better return on his investment. Throughout the suttas, we find a strain of what might be called “spiritual economics”. } 

“There\marginnote{22.2} is, brahmin.” 

“But\marginnote{22.3} what is it?” 

“The\marginnote{22.4} regular gifts as ongoing family sacrifice given specially to ethical renunciates;\footnote{Mentioned in a similar context at \href{https://suttacentral.net/an4.40/en/sujato\#2.2}{AN 4.40:2.2}. } this sacrifice, brahmin, has fewer requirements and undertakings, yet is more fruitful and beneficial.” 

“What\marginnote{23.1} is the cause, Mister Gotama, what is the reason why those regular gifts as ongoing family sacrifice have fewer requirements and undertakings, yet are more fruitful and beneficial, compared with the sacrifice accomplished with three modes and sixteen accessories?” 

“Because\marginnote{23.2} neither perfected ones nor those who are on the path to perfection will attend such a sacrifice. Why is that? Because beatings and throttlings are seen there.\footnote{This contradicts the description given above. } 

But\marginnote{23.5} the regular gifts as ongoing family sacrifice given specially to ethical renunciates; perfected ones and those who are on the path to perfection will attend such a sacrifice. Why is that? Because no beatings and throttlings are seen there. 

This\marginnote{23.9} is the cause, brahmin, this is the reason why those regular gifts as ongoing family sacrifice have fewer requirements and undertakings, yet are more fruitful and beneficial, compared with the sacrifice accomplished with three modes and sixteen accessories.” 

“But\marginnote{24.1} Mister Gotama, apart from that sacrifice accomplished with three modes and sixteen accessories and those regular gifts as ongoing family sacrifice, is there any other sacrifice that has fewer requirements and undertakings, yet is more fruitful and beneficial?” 

“There\marginnote{24.2} is, brahmin.” 

“But\marginnote{24.3} what is it?” 

“When\marginnote{24.4} someone gives a dwelling specially for the \textsanskrit{Saṅgha} of the four quarters.”\footnote{This means that the dwelling could be used by any \textsanskrit{Saṅgha} member, as opposed to being given to a particular monastic or group. The gift of a dwelling is regarded as the best kind of offering to the \textsanskrit{Saṅgha}. } 

“But\marginnote{25.1} is there any other sacrifice that has fewer requirements and undertakings, yet is more fruitful and beneficial?” 

“When\marginnote{25.4} someone with confident heart goes for refuge to the Buddha, the teaching, and the \textsanskrit{Saṅgha}.” 

“But\marginnote{26.1} is there any other sacrifice that has fewer requirements and undertakings, yet is more fruitful and beneficial?” 

“When\marginnote{26.4} someone with a confident heart undertakes the training rules to refrain from killing living creatures, stealing, sexual misconduct, lying, and beer, wine, and liquor intoxicants.”\footnote{Thus far the Buddha has described the regular practice of a Buddhist lay person. } 

“But\marginnote{27.1} is there any other sacrifice that has fewer requirements and undertakings, yet is more fruitful and beneficial?” 

“There\marginnote{27.2} is, brahmin. 

It’s\marginnote{27.4} when a Realized One arises in the world, perfected, a fully awakened Buddha … That’s how a mendicant is accomplished in ethics. … They enter and remain in the first absorption … This sacrifice has fewer requirements and undertakings than the former, yet is more fruitful and beneficial. … 

They\marginnote{27.8} enter and remain in the second absorption … third absorption … fourth absorption. This sacrifice has fewer requirements and undertakings than the former, yet is more fruitful and beneficial. … 

They\marginnote{27.12} project and extend the mind toward knowledge and vision … This sacrifice has fewer requirements and undertakings than the former, yet is more fruitful and beneficial. 

They\marginnote{27.14} understand: ‘… there is nothing further for this place.’ This sacrifice has fewer requirements and undertakings than the former, yet is more fruitful and beneficial. And, brahmin, there is no other accomplishment of sacrifice which is better and finer than this.”\footnote{The entire path may be described as a “sacrifice”. } 

\section*{6. \textsanskrit{Kūṭadanta} Declares Himself a Lay Follower }

When\marginnote{28.1} he had spoken, \textsanskrit{Kūṭadanta} said to the Buddha, “Excellent, Mister Gotama! Excellent! As if he were righting the overturned, or revealing the hidden, or pointing out the path to the lost, or lighting a lamp in the dark so people with clear eyes can see what’s there, Mister Gotama has made the Teaching clear in many ways. I go for refuge to Mister Gotama, to the teaching, and to the mendicant \textsanskrit{Saṅgha}. From this day forth, may Mister Gotama remember me as a lay follower who has gone for refuge for life. 

And\marginnote{28.6} these bulls, bullocks, heifers, goats, and rams—seven hundred of each—I release them, I grant them life! Let them eat green grass and drink cool water, and may a cool breeze blow upon them!”\footnote{Releasing animals remains a Buddhist practice today. } 

\section*{7. The Realization of the Fruit of Stream-Entry }

Then\marginnote{29.1} the Buddha taught \textsanskrit{Kūṭadanta} step by step, with a talk on giving, ethical conduct, and heaven. He explained the drawbacks of sensual pleasures, so sordid and corrupt, and the benefit of renunciation. And when he knew that \textsanskrit{Kūṭadanta}’s mind was ready, pliable, rid of hindrances, elated, and confident he explained the special teaching of the Buddhas: suffering, its origin, its cessation, and the path. Just as a clean cloth rid of stains would properly absorb dye, in that very seat the stainless, immaculate vision of the Dhamma arose in the brahmin \textsanskrit{Kūṭadanta}: “Everything that has a beginning has an end.” 

Then\marginnote{30.1} \textsanskrit{Kūṭadanta} saw, attained, understood, and fathomed the Dhamma. He went beyond doubt, got rid of indecision, and became self-assured and independent of others regarding the Teacher’s instructions. He said to the Buddha, “Would Mister Gotama together with the mendicant \textsanskrit{Saṅgha} please accept tomorrow’s meal from me?” The Buddha consented with silence. 

Then,\marginnote{30.4} knowing that the Buddha had consented, \textsanskrit{Kūṭadanta} got up from his seat, bowed, and respectfully circled the Buddha, keeping him on his right, before leaving. And when the night had passed \textsanskrit{Kūṭadanta} had delicious fresh and cooked foods prepared in his own sacrificial enclosure. Then he had the Buddha informed of the time, saying, “It’s time, Mister Gotama, the meal is ready.” 

Then\marginnote{30.7} the Buddha robed up in the morning and, taking his bowl and robe, went to the sacrificial enclosure of \textsanskrit{Kūṭadanta} together with the mendicant \textsanskrit{Saṅgha}, where he sat on the seat spread out. 

Then\marginnote{30.8} \textsanskrit{Kūṭadanta} served and satisfied the mendicant \textsanskrit{Saṅgha} headed by the Buddha with his own hands with delicious fresh and cooked foods. When the Buddha had eaten and washed his hand and bowl, \textsanskrit{Kūṭadanta} took a low seat and sat to one side. Then the Buddha educated, encouraged, fired up, and inspired him with a Dhamma talk, after which he got up from his seat and left. 

%
\chapter*{{\suttatitleacronym DN 6}{\suttatitletranslation With Mahāli }{\suttatitleroot Mahālisutta}}
\addcontentsline{toc}{chapter}{\tocacronym{DN 6} \toctranslation{With Mahāli } \tocroot{Mahālisutta}}
\markboth{With Mahāli }{Mahālisutta}
\extramarks{DN 6}{DN 6}

\section*{1. On the Brahmin Emissaries }

\scevam{So\marginnote{1.1} I have heard. }At one time the Buddha was staying near \textsanskrit{Vesālī}, at the Great Wood, in the hall with the peaked roof.\footnote{Lying some 60 kilometers north of \textsanskrit{Pāṭaliputra} (Patna), \textsanskrit{Vesālī} was the largest city in the Vajji Federation, a republican league in the region north of the Ganges. } Now at that time several brahmin emissaries from Kosala and Magadha were residing in \textsanskrit{Vesālī} on some business.\footnote{In the earlier suttas of this chapter, we have seen how news of the Buddha spread, evidently following \textsanskrit{Pokkharasāti}’s conversion. Here we see an example of the kind of meeting at which such news would be discussed. | The phrase “brahmin emissaries” (\textit{\textsanskrit{brāhmaṇadūtā}}) does not seem to occur elsewhere and is not explained in the commentary. Perhaps they were emissaries of the kings, meeting in a neutral location. Or perhaps they were emissaries of their respective communities of brahmins. } They heard: 

“It\marginnote{1.5} seems the ascetic Gotama—a Sakyan, gone forth from a Sakyan family—is staying near \textsanskrit{Vesālī}, at the Great Wood, in the hall with the peaked roof.\footnote{This monastery features prominently as the Buddha’s usual place of residence near \textsanskrit{Vesālī}. } He has this good reputation: ‘That Blessed One is perfected, a fully awakened Buddha, accomplished in knowledge and conduct, holy, knower of the world, supreme guide for those who wish to train, teacher of gods and humans, awakened, blessed.’ He has realized with his own insight this world—with its gods, \textsanskrit{Māras}, and divinities, this population with its ascetics and brahmins, gods and humans—and he makes it known to others. He proclaims a teaching that is good in the beginning, good in the middle, and good in the end, meaningful and well-phrased. And he reveals a spiritual practice that’s entirely full and pure. It’s good to see such perfected ones.” 

Then\marginnote{2.1} they went to the hall with the peaked roof in the Great Wood to see the Buddha. 

Now,\marginnote{2.2} at that time Venerable \textsanskrit{Nāgita} was the Buddha’s attendant. The brahmin emissaries went up to him and said, “Mister \textsanskrit{Nāgita}, where is Mister Gotama at present? For we want to see him.” 

“It’s\marginnote{2.6} the wrong time to see the Buddha; he is on retreat.”\footnote{At some times the Buddha would go on retreat and ask that no-one visit him except to bring food; this sometimes happened at the Great Wood (\href{https://suttacentral.net/sn54.9/en/sujato\#2.1}{SN 54.9:2.1}). He also had the habit of withdrawing into the wood itself for meditation (\href{https://suttacentral.net/an5.58/en/sujato\#1.3}{AN 5.58:1.3}). At this time, however, he was simply staying in a nearby hut, so it seems \textsanskrit{Nāgita} is being over-zealous. } 

So\marginnote{2.7} right there the brahmin emissaries sat down to one side, thinking, “We’ll go only after we’ve seen Mister Gotama.” 

\section*{2. On \textsanskrit{Oṭṭhaddha} the Licchavi }

\textsanskrit{Oṭṭhaddha}\marginnote{3.1} the Licchavi together with a large assembly of Licchavis also approached \textsanskrit{Nāgita} at the hall with the peaked roof. He bowed, stood to one side, and said to \textsanskrit{Nāgita},\footnote{\textsanskrit{Oṭṭhaddha} mean “hare-lip” and is evidently a nickname or epithet. Throughout, the Buddha refers to him by his personal name, \textsanskrit{Mahāli}. And it is under that name we meet him again in \href{https://suttacentral.net/sn11.13/en/sujato}{SN 11.13} and \href{https://suttacentral.net/sn22.60/en/sujato}{SN 22.60}. | The Licchavis, whose name is derived from “bear”, dominated the Vajji Federation. Note that the \textsanskrit{Mahāsaṅgīti} edition here spells the masculine singular as \textit{\textsanskrit{licchavī}}, whereas normally it is \textit{licchavi}. } “Mister \textsanskrit{Nāgita}, where is the Blessed One at present, the perfected one, the fully awakened Buddha? For we want to see him.” 

“It’s\marginnote{3.4} the wrong time to see the Buddha; he is on retreat.” 

So\marginnote{3.5} right there \textsanskrit{Oṭṭhaddha} also sat down to one side, thinking, “I’ll go only after I’ve seen the Blessed One, the perfected one, the fully awakened Buddha.” 

Then\marginnote{4.1} the novice \textsanskrit{Sīha} approached \textsanskrit{Nāgita}. He bowed, stood to one side, and said to \textsanskrit{Nāgita},\footnote{This \textsanskrit{Sīha} is unknown elsewhere. } “Honorable Kassapa, these several brahmin emissaries from Kosala and Magadha, and also \textsanskrit{Oṭṭhaddha} the Licchavi together with a large assembly of Licchavis, have come here to see the Buddha. It’d be good if these people got to see the Buddha.”\footnote{Kassapa is \textsanskrit{Nāgita}’s clan name; either he was a brahmin or a \textit{khattiya} whose family chaplain (\textit{purohita}) was a Kassapa. } 

“Well\marginnote{4.3} then, \textsanskrit{Sīha}, tell the Buddha yourself.”\footnote{In trying to protect the Buddha, \textsanskrit{Nāgita} was inflexible and lacking compassion. When given good advice by \textsanskrit{Sīha}, he responded gracelessly, fobbing off responsibility to a junior. No wonder he was replaced by Ānanda. } 

“Yes,\marginnote{4.4} sir,” replied \textsanskrit{Sīha}. He went to the Buddha, bowed, stood to one side, and told him of the people waiting to see him, adding: “Sir, it’d be good if these people got to see the Buddha.” 

“Well\marginnote{4.7} then, \textsanskrit{Sīha}, spread out a seat in the shade of the dwelling.” 

“Yes,\marginnote{4.8} sir,” replied \textsanskrit{Sīha}, and he did so. 

Then\marginnote{4.9} the Buddha came out of his dwelling and sat in the shade of the dwelling on the seat spread out.\footnote{This is still a common place for forest monks to receive guests. } Then the brahmin emissaries went up to the Buddha, and exchanged greetings with him.\footnote{This is the last we hear of these emissaries. } When the greetings and polite conversation were over, they sat down to one side. 

\textsanskrit{Oṭṭhaddha}\marginnote{5.3} the Licchavi together with a large assembly of Licchavis also went up to the Buddha, bowed, and sat down to one side. \textsanskrit{Oṭṭhaddha} said to the Buddha, “Sir, a few days ago Sunakkhatta the Licchavi came to me and said:\footnote{Sunakkhatta features in several suttas, through which his journey may be traced. In \href{https://suttacentral.net/mn105/en/sujato}{MN 105} he meets the Buddha and gains faith; here in \href{https://suttacentral.net/dn6/en/sujato}{DN 6} he is becoming dissatisfied; in \href{https://suttacentral.net/dn24/en/sujato}{DN 24} he rejects the Buddha; and in \href{https://suttacentral.net/mn12/en/sujato}{MN 12} he attacks the Buddha after disrobing. } ‘\textsanskrit{Mahāli}, soon I will have been living in dependence on the Buddha for three years. I see heavenly sights that are pleasant, sensual, and arousing, but I don’t hear heavenly sounds that are pleasant, sensual, and arousing.’\footnote{This refers to “clairvoyance” and “clairaudience”, sometimes translated as the “divine eye” and “divine ear”. Despite being included in the Gradual Training, they are not a goal of Buddhist practice. Rather, they are unnecessary but potentially useful, as they reveal dimensions of being inaccessible to ordinary consciousness. Sunakkhatta, however, was evidently just interested in having pleasant supersensory experiences. } The heavenly sounds that Sunakkhatta cannot hear: do such sounds really exist or not?” 

\subsection*{2.1. One-Sided Immersion }

“Such\marginnote{5.7} sounds really do exist, but Sunakkhatta cannot hear them.”\footnote{This must have wounded his pride. } 

“What\marginnote{6.1} is the cause, sir, what is the reason why Sunakkhatta cannot hear them, even though they really do exist?” 

“\textsanskrit{Mahāli},\marginnote{6.2} take a mendicant who has developed immersion to the eastern quarter in one aspect: so as to see heavenly sights but not to hear heavenly sounds.\footnote{This description of meditation is unique in the Pali canon. The Buddha answers \textsanskrit{Mahāli}’s question directly, even though the premise betrays Sunakkhatta’s limited understanding. When a questioner is sincere, answering directly shows respect and builds trust. } When they have developed immersion for that purpose, they see heavenly sights but don’t hear heavenly sounds. Why is that? Because that is how it is for a mendicant who develops immersion in that way. 

Furthermore,\marginnote{7.1} take a mendicant who has developed immersion to the southern quarter in one aspect … western quarter … northern quarter … above, below, across … That is how it is for a mendicant who develops immersion in that way. 

Take\marginnote{8.1} a mendicant who has developed immersion to the eastern quarter in one aspect: so as to hear heavenly sounds but not to see heavenly sights. When they have developed immersion for that purpose, they hear heavenly sounds but don’t see heavenly sights. Why is that? Because that is how it is for a mendicant who develops immersion in that way. 

Furthermore,\marginnote{9.1} take a mendicant who has developed immersion to the southern quarter in one aspect … western quarter … northern quarter … above, below, across … That is how it is for a mendicant who develops immersion in that way. 

Take\marginnote{10.1} a mendicant who has developed immersion to the eastern quarter in both aspects: so as to hear heavenly sounds and see heavenly sights. When they have developed immersion for that purpose, they see heavenly sights and hear heavenly sounds. Why is that? Because that is how it is for a mendicant who develops immersion in that way. 

Furthermore,\marginnote{11.1} take a mendicant who has developed immersion to the southern quarter in both aspects … western quarter … northern quarter … above, below, across … That is how it is for a mendicant who develops immersion in that way. This is the cause, \textsanskrit{Mahāli}, this is the reason why Sunakkhatta cannot hear heavenly sounds that are pleasant, sensual, and arousing, even though they really do exist.” 

“Surely\marginnote{12.1} the mendicants must lead the spiritual life under the Buddha for the sake of realizing such a development of immersion?” 

“No,\marginnote{12.2} \textsanskrit{Mahāli}, the mendicants don’t lead the spiritual life under me for the sake of realizing such a development of immersion.\footnote{Having directly answered the original question, the Buddha reframed the issue on request. } There are other things that are finer, for the sake of which the mendicants lead the spiritual life under me.” 

\subsection*{2.2. The Four Noble Fruits }

“But\marginnote{13.1} sir, what are those finer things?” 

“Firstly,\marginnote{13.2} \textsanskrit{Mahāli}, with the ending of three fetters a mendicant is a stream-enterer, not liable to be reborn in the underworld, bound for awakening.\footnote{This is the first description of the four stages of awakening which are featured throughout the Pali canon. The three fetters are identity view, doubt, and misapprehension of precepts and observances (\href{https://suttacentral.net/mn2/en/sujato\#11.3}{MN 2:11.3}). } This is one of the finer things for the sake of which the mendicants lead the spiritual life under me. 

Furthermore,\marginnote{13.4} a mendicant—with the ending of three fetters, and the weakening of greed, hate, and delusion—is a once-returner. They come back to this world once only, then make an end of suffering. This too is one of the finer things. 

Furthermore,\marginnote{13.6} with the ending of the five lower fetters, a mendicant is reborn spontaneously and will become extinguished there, not liable to return from that world.\footnote{The five lower fetters are the three mentioned above, plus sensual desire and ill will (\href{https://suttacentral.net/an10.13/en/sujato\#1.5}{AN 10.13:1.5}). This is the non-returner, who spends their last life in an exalted \textsanskrit{Brahmā} realm. | A “spontaneous” rebirth is one that occurs without gestation in the womb, like most \textit{devas}, or for that matter, Boltzmann brains. } This too is one of the finer things. 

Furthermore,\marginnote{13.8} a mendicant has realized the undefiled freedom of heart and freedom by wisdom in this very life, and lives having realized it with their own insight due to the ending of defilements.\footnote{This is the arahant, the “worthy” or “perfected” one. Elsewhere it is said they abandon the five higher fetters: desire for rebirth in the realm of luminous form, desire for rebirth in the formless realm, conceit, restlessness, and ignorance (\href{https://suttacentral.net/an10.13/en/sujato\#2.2}{AN 10.13:2.2}). } This too is one of the finer things. 

These\marginnote{13.10} are the finer things, for the sake of which the mendicants lead the spiritual life under me.” 

\subsection*{2.3. The Noble Eightfold Path }

“But,\marginnote{14.1} sir, is there a path and a practice for realizing these things?” 

“There\marginnote{14.2} is, \textsanskrit{Mahāli}.” 

“Well,\marginnote{14.3} what is it?” 

“It\marginnote{14.4} is simply this noble eightfold path, that is:\footnote{This is the most fundamental of the Buddha’s teachings on the path, declared in his first sermon (\href{https://suttacentral.net/sn56.11/en/sujato}{SN 56.11}). It reappears in \href{https://suttacentral.net/dn8/en/sujato\#13.5}{DN 8:13.5}, \href{https://suttacentral.net/dn19/en/sujato\#61.7}{DN 19:61.7}, and \href{https://suttacentral.net/dn22/en/sujato\#21.2}{DN 22:21.2}. } right view, right thought, right speech, right action, right livelihood, right effort, right mindfulness, and right immersion.\footnote{The eight factors map roughly on to the Gradual Training thus: hearing the Dhamma gives rise to right view; the choice to renounce is right thought; ethics includes right speech, action, and livelihood; undertaking seclusion and sense restraint is right effort; developing meditation is right mindfulness; and gaining the four \textit{\textsanskrit{jhānas}} is right immersion. Realization of the Dhamma completes the circle by deepening conceptual right view to liberating insight. Sometimes this is expressed by adding two further factors, right knowledge and right liberation. } This is the path and the practice for realizing these things. 

\subsection*{2.4. On the Two Renunciates }

This\marginnote{15.1} one time, \textsanskrit{Mahāli}, I was staying near \textsanskrit{Kosambī}, in Ghosita’s Monastery.\footnote{The Buddha retells the events recorded in the next sutta, \href{https://suttacentral.net/dn7/en/sujato}{DN 7}. } Then two renunciates—the wanderer \textsanskrit{Muṇḍiya} and \textsanskrit{Jāliya}, the pupil of the wood-bowl ascetic—came and exchanged greetings with me.\footnote{\textsanskrit{Muṇḍiya} means “shaven one”; his name is spelled Mandissa in some manuscripts. He appears only in this passage. \textsanskrit{Jāliya} returns in \href{https://suttacentral.net/dn24/en/sujato\#2.4.1}{DN 24:2.4.1}, which recounts the farcical events following Sunakkhatta’s disrobal. There he takes the Buddha’s part against the delusional \textsanskrit{Pāṭikaputta} favored by Sunakkhatta. } When the greetings and polite conversation were over, they stood to one side and said to me: ‘Reverend Gotama, are the soul and the body the same thing, or they are different things?’\footnote{The term “soul” (\textit{\textsanskrit{jīva}}) was favored by the \textit{\textsanskrit{samaṇas}}, as opposed to the “self” (\textit{\textsanskrit{attā}}) of the brahmins. Both are rejected by the Buddha as forms of “metaphysical” self: they postulate the absolute, eternal existence of entities that cannot be established empirically. The repeated demonstrative pronouns (\textit{\textsanskrit{taṁ} \textsanskrit{jīvaṁ} \textsanskrit{taṁ} \textsanskrit{sarīraṁ}}) assert an emphatic identity. } 

‘Well\marginnote{16.1} then, reverends, listen and apply your mind well, I will speak.’ 

‘Yes,\marginnote{16.2} reverend,’ they replied. 

I\marginnote{16.3} said this: ‘Take the case when a Realized One arises in the world, perfected, a fully awakened Buddha … That’s how a mendicant is accomplished in ethics. … 

They\marginnote{16.6} enter and remain in the first absorption. When a mendicant knows and sees like this, would it be appropriate to say of them: “The soul and the body are the same thing” or “The soul and the body are different things”?’ 

‘It\marginnote{16.10} would, reverend.’\footnote{They evidently believed that the experience of \textsanskrit{jhāna} would grant insight into this dilemma. But it is a loaded question: it assumes that the soul is real and that what needs determining is its relation to the body. } 

‘But\marginnote{16.12} reverends, I know and see like this. Nevertheless, I do not say: “The soul and the body are the same thing” or “The soul and the body are different things”. … 

They\marginnote{17.1} enter and remain in the second absorption … third absorption … fourth absorption. When a mendicant knows and sees like this, would it be appropriate to say of them: “The soul and the body are the same thing” or “The soul and the body are different things”?’ 

‘It\marginnote{17.6} would, reverend.’ 

‘But\marginnote{17.8} reverends, I know and see like this. Nevertheless, I do not say: “The soul and the body are the same thing” or “The soul and the body are different things”. … 

They\marginnote{18.1} project and extend the mind toward knowledge and vision … When a mendicant knows and sees like this, would it be appropriate to say of them: “The soul and the body are the same thing” or “The soul and the body are different things”?’ 

‘It\marginnote{18.5} would, reverend.’ 

‘But\marginnote{18.7} reverends, I know and see like this. Nevertheless, I do not say: “The soul and the body are the same thing” or “The soul and the body are different things”. … 

They\marginnote{19.1} understand: “… there is nothing further for this place.” When a mendicant knows and sees like this, would it be appropriate to say of them: “The soul and the body are the same thing” or “The soul and the body are different things”?’ 

‘It\marginnote{19.4} would not, reverend.’\footnote{Until this point, none of the experiences described are fundamentally incompatible with the notion of an eternal metaphysical self. Buddhists believe that non-Buddhists, before and after the Buddha, are quite capable of realizing such states. However, they would tend to interpret them in line with their previous beliefs, thus reinforcing their theories of self. Faced with the end of all rebirth, however, no theory of eternal self can stand. } 

‘But\marginnote{19.6} reverends, I know and see like this. Nevertheless, I do not say: “The soul and the body are the same thing” or “The soul and the body are different things”.’” 

That\marginnote{19.9} is what the Buddha said. Satisfied, \textsanskrit{Oṭṭhaddha} the Licchavi approved what the Buddha said. 

%
\chapter*{{\suttatitleacronym DN 7}{\suttatitletranslation With Jāliya }{\suttatitleroot Jāliyasutta}}
\addcontentsline{toc}{chapter}{\tocacronym{DN 7} \toctranslation{With Jāliya } \tocroot{Jāliyasutta}}
\markboth{With Jāliya }{Jāliyasutta}
\extramarks{DN 7}{DN 7}

\scevam{So\marginnote{1.1} I have heard. }At one time the Buddha was staying near \textsanskrit{Kosambī}, in Ghosita’s Monastery.\footnote{This sutta depicts the events that were subsequently related by the Buddha in the previous sutta, \href{https://suttacentral.net/dn6/en/sujato}{DN 6}. } 

Now\marginnote{1.3} at that time two renunciates—the wanderer \textsanskrit{Muṇḍiya} and \textsanskrit{Jāliya}, the pupil of the wood-bowl ascetic—came to the Buddha and exchanged greetings with him. When the greetings and polite conversation were over, they stood to one side and said to the Buddha, “Reverend Gotama, are the soul and the body the same thing, or they are different things?” 

“Well\marginnote{1.7} then, reverends, listen and apply your mind well, I will speak.” 

“Yes,\marginnote{1.8} reverend,” they replied. The Buddha said this: 

“Take\marginnote{2.1} the case when a Realized One arises in the world, perfected, a fully awakened Buddha … That’s how a mendicant is accomplished in ethics. … 

They\marginnote{2.3} enter and remain in the first absorption … When a mendicant knows and sees like this, would it be appropriate to say of them: ‘The soul and the body are the same thing’ or ‘The soul and the body are different things’?” 

“It\marginnote{2.7} would, reverend.” 

“But\marginnote{2.9} reverends, I know and see like this. Nevertheless, I do not say: ‘The soul and the body are the same thing’ or ‘The soul and the body are different things’. … 

They\marginnote{3.1} enter and remain in the second absorption … third absorption … fourth absorption. When a mendicant knows and sees like this, would it be appropriate to say of them: ‘The soul and the body are the same thing’ or ‘The soul and the body are different things’?” 

“It\marginnote{3.6} would, reverend.” 

“But\marginnote{3.8} reverends, I know and see like this. Nevertheless, I do not say: ‘The soul and the body are the same thing’ or ‘The soul and the body are different things’. … 

They\marginnote{4.1} project and extend the mind toward knowledge and vision … When a mendicant knows and sees like this, would it be appropriate to say of them: ‘The soul and the body are the same thing’ or ‘The soul and the body are different things’?” 

“It\marginnote{4.4} would, reverend.” 

“But\marginnote{4.6} reverends, I know and see like this. Nevertheless, I do not say: ‘The soul and the body are the same thing’ or ‘The soul and the body are different things’. … 

They\marginnote{5.1} understand: ‘… there is nothing further for this place.’ When a mendicant knows and sees like this, would it be appropriate to say of them: ‘The soul and the body are the same thing’ or ‘The soul and the body are different things’?” 

“It\marginnote{5.4} would not, reverend.” 

“But\marginnote{5.6} reverends, I know and see like this. Nevertheless, I do not say: ‘The soul and the body are the same thing’ or ‘The soul and the body are different things’.” 

That\marginnote{5.9} is what the Buddha said. Satisfied, the two renunciates approved what the Buddha said. 

%
\chapter*{{\suttatitleacronym DN 8}{\suttatitletranslation The Lion’s Roar to the Naked Ascetic Kassapa }{\suttatitleroot Mahāsīhanādasutta}}
\addcontentsline{toc}{chapter}{\tocacronym{DN 8} \toctranslation{The Lion’s Roar to the Naked Ascetic Kassapa } \tocroot{Mahāsīhanādasutta}}
\markboth{The Lion’s Roar to the Naked Ascetic Kassapa }{Mahāsīhanādasutta}
\extramarks{DN 8}{DN 8}

\scevam{So\marginnote{1.1} I have heard. }At one time the Buddha was staying near \textsanskrit{Ujuññā}, in the deer park at \textsanskrit{Kaṇṇakatthala}.\footnote{\textsanskrit{Ujuññā} was a Kosalan town at which King Pasenadi visited the Buddha in \href{https://suttacentral.net/mn90/en/sujato}{MN 90}. | “Deer parks” were nature reservations where the animals were safe from hunters. } 

Then\marginnote{1.3} the naked ascetic Kassapa went up to the Buddha and exchanged greetings with him.\footnote{Naked ascetics are still found in India today. Some Jains went naked, but if he were a Jain he would have been introduced as such. Kassapa is an ancient clan name of the brahmins, and we meet four naked ascetics named Kassapa in the canon (here, \href{https://suttacentral.net/sn12.17/en/sujato}{SN 12.17}, \href{https://suttacentral.net/sn41.9/en/sujato}{SN 41.9}, and \href{https://suttacentral.net/mn124/en/sujato}{MN 124}). They cannot be the same person, for at the end of each account it is said they went forth and attained arahantship. } When the greetings and polite conversation were over, he stood to one side and said to the Buddha: 

“Mister\marginnote{2.1} Gotama, I have heard the following: ‘The ascetic Gotama criticizes all fervent mortification. He categorically condemns and denounces all fervent mortifiers who live rough.’\footnote{In his first sermon, the Buddha rejected extremes of self-mortification. There the term was \textit{attakilamatha} (“self-mortification”), whereas here it is \textit{tapas} (“heat, burning, fervor”). These refer to the same practices, but \textit{tapas} points to the fervent ardor of the practitioner, generating an inner heat that “burns off” the corrupting traces of \textit{kamma} and defilements. This topic is also discussed in \href{https://suttacentral.net/an10.94/en/sujato}{AN 10.94}. } Do those who say this repeat what the Buddha has said, and not misrepresent him with an untruth? Is their explanation in line with the teaching? Are there any legitimate grounds for rebuttal and criticism? For we don’t want to misrepresent Mister Gotama.” 

“Kassapa,\marginnote{3.1} those who say this do not repeat what I have said. They misrepresent me with what is false, baseless, and untrue. With clairvoyance that is purified and superhuman, I see some fervent mortifier who lives rough reborn in a place of loss, a bad place, the underworld, hell. But I see another fervent mortifier who lives rough reborn in a good place, a heavenly realm.\footnote{While the self-mortification itself may be useless, the person who practices it may have other good qualities. The Buddha is cautioning against rash judgement. } 

I\marginnote{3.4} see some fervent mortifier who takes it easy reborn in a place of loss. But I see another fervent mortifier who takes it easy reborn in a good place, a heavenly realm. Since I truly understand the coming and going, passing away and rebirth of these fervent mortifiers in this way, how could I criticize all forms of mortification, or categorically condemn and denounce those fervent mortifiers who live rough? 

There\marginnote{4.1} are some clever ascetics and brahmins who are subtle, accomplished in the doctrines of others, hair-splitters. You’d think they live to demolish convictions with their intellect. They agree with me in some matters and disagree in others. Some of the things that they applaud, I also applaud. Some of the things that they don’t applaud, I also don’t applaud. But some of the things that they applaud, I don’t applaud. And some of the things that they don’t applaud, I do applaud. 

Some\marginnote{4.7} of the things that I applaud, others also applaud. Some of the things that I don’t applaud, they also don’t applaud. But some of the things that I don’t applaud, others do applaud. And some of the things that I do applaud, others don’t applaud. 

\section*{1. Examination }

I\marginnote{5.1} go up to them and say: ‘Let us leave aside those matters on which we disagree.\footnote{Again we see the Buddha’s preferred method of establishing common ground first, then building an argument from there. } But there are some matters on which we agree. Regarding these, sensible people, pursuing, pressing, and grilling, would compare teacher with teacher or community with community: 

“There\marginnote{5.4} are things that are unskillful, blameworthy, not to be cultivated, unworthy of the noble ones, and dark—and are reckoned as such. Who proceeds having totally given these things up: the ascetic Gotama, or the tutors of other communities?”’\footnote{Rather than logical hair-splitting, the Buddha recommends looking at a person’s conduct. } 

It’s\marginnote{6.1} possible that they might say: ‘The ascetic Gotama proceeds having totally given those unskillful things up, compared with the tutors of other communities.’ And that’s how, when sensible people pursue the matter, they will mostly praise us. 

In\marginnote{7.1} addition, sensible people, pursuing, pressing, and grilling, would compare teacher with teacher or community with community: ‘There are things that are skillful, blameless, worth cultivating, worthy of the noble ones, and bright—and are reckoned as such. Who proceeds having totally undertaken these things: the ascetic Gotama, or the tutors of other communities?’ 

It’s\marginnote{8.1} possible that they might say: ‘The ascetic Gotama proceeds having totally undertaken these things, compared with the tutors of other communities.’ And that’s how, when sensible people pursue the matter, they will mostly praise us. 

In\marginnote{9.1} addition, sensible people, pursuing, pressing, and grilling, would compare teacher with teacher or community with community: ‘There are things that are unskillful, blameworthy, not to be cultivated, unworthy of the noble ones, and dark—and are reckoned as such. Who proceeds having totally given these things up: the ascetic Gotama’s disciples, or the disciples of other tutors?’ 

It’s\marginnote{10.1} possible that they might say: ‘The ascetic Gotama’s disciples proceed having totally given those unskillful things up, compared with the disciples of other tutors.’ And that’s how, when sensible people pursue the matter, they will mostly praise us. 

In\marginnote{11.1} addition, sensible people, pursuing, pressing, and grilling, would compare teacher with teacher or community with community: ‘There are things that are skillful, blameless, worth cultivating, worthy of the noble ones, and bright—and are reckoned as such. Who proceeds having totally undertaken these things: the ascetic Gotama’s disciples, or the disciples of other tutors?’ 

It’s\marginnote{12.1} possible that they might say: ‘The ascetic Gotama’s disciples proceed having totally undertaken those skillful things, compared with the disciples of other tutors.’ And that’s how, when sensible people pursue the matter, they will mostly praise us. 

\section*{2. The Noble Eightfold Path }

There\marginnote{13.1} is, Kassapa, a path, there is a practice, practicing in accordance with which you will know and see for yourself: ‘Only the ascetic Gotama’s words are timely, true, and meaningful, in line with the teaching and training.’\footnote{Here \textit{-va} has an exclusive sense (= \textit{eva}). Compare \href{https://suttacentral.net/dhp274/en/sujato}{Dhp 274}: \textit{eseva maggo \textsanskrit{natthañño}} (“This is the path, there is no other”). } And what is that path? It is simply this noble eightfold path, that is: right view, right thought, right speech, right action, right livelihood, right effort, right mindfulness, and right immersion. This is the path, this is the practice, practicing in accordance with which you will know and see for yourself: ‘Only the ascetic Gotama’s words are timely, true, and meaningful, in line with the teaching and training.’”\footnote{Compare \href{https://suttacentral.net/dn16/en/sujato\#5.27.1}{DN 16:5.27.1}. } 

\section*{3. The Courses of Fervent Mortification }

When\marginnote{14.1} he had spoken, Kassapa said to the Buddha: 

“Reverend\marginnote{14.2} Gotama, those ascetics and brahmins consider these courses of fervent mortification to be what makes someone a true ascetic or brahmin.\footnote{What follows is a description of ascetic practices undertaken by the Jains and similar groups. | The phrase “course of fervent mortification” (\textit{tapopakkama}) is unique to this sutta. \textit{Pakkama} means “stepping out”. } They go naked, ignoring conventions. They lick their hands, and don’t come or wait when called. They don’t consent to food brought to them, or food prepared on their behalf, or an invitation for a meal.\footnote{Buddhist mendicants may not receive food in their hands, nor lick them while eating. Followers of the practices listed here would have walked steadily and randomly for alms, accepting only what was given at the time. } They don’t receive anything from a pot or bowl; or from someone who keeps sheep, or who has a weapon or a shovel in their home; or where a couple is eating; or where there is a woman who is pregnant, breastfeeding, or who lives with a man; or where there’s a dog waiting or flies buzzing. They accept no fish or meat or beer or wine, and drink no fermented gruel.\footnote{Keeping sheep (\textit{\textsanskrit{eḷaka}}, for slaughter) goes against the Jain principle of non-violence, as does keeping weapons (\textit{\textsanskrit{daṇḍa}}). | A \textit{musala} often means “pestle”, but it can also be a “shovel”; at \href{https://suttacentral.net/mn81/en/sujato\#18.12}{MN 81:18.12} it is regarded as a virtue to not use one to dig the soil (which is regarded as being alive in Jainism). | \textit{Thusodaka} is an alcoholic porridge fermented from grain-husks, mentioned alongside \textit{\textsanskrit{sovīraka}} in the Pali commentaries and \textsanskrit{Carakasaṁhitā} 27g.191. } They go to just one house for alms, taking just one mouthful, or two houses and two mouthfuls, up to seven houses and seven mouthfuls. They feed on one saucer a day, two saucers a day, up to seven saucers a day. They eat once a day, once every second day, up to once a week, and so on, even up to once a fortnight. They live committed to the practice of eating food at set intervals. 

Those\marginnote{14.8} ascetics and brahmins also consider these courses of fervent mortification to be what makes someone a true ascetic or brahmin. They eat herbs, millet, wild rice, poor rice, water lettuce, rice bran, scum from boiling rice, sesame flour, grass, or cow dung. They survive on forest roots and fruits, or eating fallen fruit.\footnote{It is not easy to meaningfully distinguish the various kinds of grain. } 

Those\marginnote{14.10} ascetics and brahmins also consider these courses of fervent mortification to be what makes someone a true ascetic or brahmin. They wear robes of sunn hemp, mixed hemp, corpse-wrapping cloth, rags, lodh tree bark, antelope hide (whole or in strips), kusa grass, bark, wood-chips, human hair, horse-tail hair, or owls’ wings.\footnote{All are extremely uncomfortable. Christian ascetics wore a “hair shirt” in order to “mortify the flesh” . } They tear out hair and beard, committed to this practice.\footnote{Jain ascetics tear out their hair at ordination, rather than shaving. } They constantly stand, refusing seats.\footnote{Remaining in one posture for months or years at a time is one of the most difficult practices. } They squat, committed to persisting in the squatting position. They lie on a mat of thorns, making a mat of thorns their bed. They make their bed on a plank, or the bare ground. They lie only on one side. They wear dust and dirt.\footnote{Strict Jain ascetics did not bathe. } They stay in the open air. They sleep wherever they lay their mat. They eat unnatural things, committed to the practice of eating unnatural foods.\footnote{At \href{https://suttacentral.net/pli-tv-kd6/en/sujato\#14.6.3}{Kd 6:14.6.3} the four “great unnaturals” (or “filthy edibles”, \textit{\textsanskrit{mahāvikaṭa}}) are said to be feces, urine, ash, and clay. At \href{https://suttacentral.net/mn12/en/sujato\#49.3}{MN 12:49.3} the Buddha said he ate the “unnatural things” of feces and urine when undertaking ascetic practices. } They don’t drink, committed to the practice of not drinking liquids. They’re devoted to ritual bathing three times a day, including the evening.”\footnote{This seems out of place here. It was a Brahmanical practice (\href{https://suttacentral.net/sn7.21/en/sujato}{SN 7.21}), as the Jains refused to bathe at all. Indeed, bathing three times a day in the Indian climate would, for most of the year, be quite pleasant. } 

\section*{4. The Uselessness of Fervent Mortification }

“Kassapa,\marginnote{15.1} someone may practice all those forms of mortification, but if they haven’t developed and realized any accomplishment in ethics, mind, and wisdom, they are far from being a true ascetic or brahmin.\footnote{The term “accomplishment in mind” (\textit{\textsanskrit{cittasampadā}}) is equivalent to “accomplishment in immersion” (\textit{\textsanskrit{samādhisampadā}}). More generally, when \textit{citta} is used in the context of meditation, it is normally a synonym of \textit{\textsanskrit{samādhi}}. } But take a mendicant who develops a heart of love, free of enmity and ill will. And they realize the undefiled freedom of heart and freedom by wisdom in this very life, and live having realized it with their own insight due to the ending of defilements.\footnote{\textit{\textsanskrit{Mettā}} is universal love and good will free from attachment. As well as being a foundation for good character and healthy emotional development, it serves to lead the mind into deep meditation of \textit{\textsanskrit{jhāna}}. } When they achieve this, they’re a mendicant who is called a ‘true ascetic’ and also ‘a true brahmin’. …” 

When\marginnote{16.1} he had spoken, Kassapa said to the Buddha, “It’s hard, Mister Gotama, to be a true ascetic or a true brahmin.” 

“It’s\marginnote{16.3} typical, Kassapa, in this world to think that it’s hard to be a true ascetic or brahmin. But someone might practice all those forms of mortification. And if it was only because of just that much, only because of that course of fervent mortification that it was so very hard to be a true ascetic or brahmin, it wouldn’t be appropriate to say that it’s hard to be a true ascetic or brahmin. 

For\marginnote{16.7} it would be quite possible for a householder or a householder’s child—or even the bonded maid who carries the water-jar—to practice all those forms of mortification. 

It’s\marginnote{16.9} because there’s something other than just that much, something other than that course of fervent mortification that it’s so very hard to be a true ascetic or brahmin. And that’s why it is appropriate to say that it’s hard to be a true ascetic or brahmin. Take a mendicant who develops a heart of love, free of enmity and ill will. And they realize the undefiled freedom of heart and freedom by wisdom in this very life, and live having realized it with their own insight due to the ending of defilements.\footnote{The Buddha was criticized for going soft after abandoning austere practices, but here he flips the script, arguing that it is inner transformation that is really hard, not outer shows of mortification. } When they achieve this, they’re a mendicant who is called a ‘true ascetic’ and also ‘a true brahmin’. …” 

When\marginnote{17.1} he had spoken, Kassapa said to the Buddha, “It’s hard, Mister Gotama, to know a true ascetic or a true brahmin.” 

“It’s\marginnote{17.3} typical in this world to think that it’s hard to know a true ascetic or brahmin. But someone might practice all those forms of mortification. And if it was only by just that much, only by that  course of fervent mortification that it was so very hard to know a true ascetic or brahmin, it wouldn’t be appropriate to say that it’s hard to know a true ascetic or brahmin. 

For\marginnote{17.7} it would be quite possible for a householder or a householder’s child—or even the bonded maid who carries the water-jar—to know that someone is practicing all those forms of mortification. 

It’s\marginnote{17.9} because there’s something other than just that much, something other than that course of fervent mortification that it’s so very hard to know a true ascetic or brahmin. And that’s why it is appropriate to say that it’s hard to know a true ascetic or brahmin. Take a mendicant who develops a heart of love, free of enmity and ill will. And they realize the undefiled freedom of heart and freedom by wisdom in this very life, and live having realized it with their own insight due to the ending of defilements. When they achieve this, they’re a mendicant who is called a ‘true ascetic’ and also ‘a true brahmin’.” 

\section*{5. The Accomplishment of Ethics, Mind, and Wisdom }

When\marginnote{18.1} he had spoken, Kassapa said to the Buddha, “But Mister Gotama, what is that accomplishment in ethics, in mind, and in wisdom?”\footnote{Note that the heading for this section in the \textsanskrit{Mahāsaṅgīti} edition uses \textit{\textsanskrit{samādhi}} rather than \textit{citta}. Headings were added by later editors, and are not part of the original text. } 

“It’s\marginnote{18.3} when a Realized One arises in the world, perfected, a fully awakened Buddha … Seeing danger in the slightest fault, a mendicant keeps the rules they’ve undertaken. They act skillfully by body and speech. They’re purified in livelihood and accomplished in ethical conduct. They guard the sense doors, have mindfulness and situational awareness, and are content. 

And\marginnote{18.5} how is a mendicant accomplished in ethics? It’s when a mendicant gives up killing living creatures. They renounce the rod and the sword. They’re scrupulous and kind, living full of sympathy for all living beings. This pertains to their accomplishment in ethics. … 

There\marginnote{19.1} are some ascetics and brahmins who, while enjoying food given in faith, still earn a living by low lore, by wrong livelihood. … They refrain from such low lore, such wrong livelihood. This pertains to their accomplishment in ethics. 

A\marginnote{19.5} mendicant thus accomplished in ethics sees no danger in any quarter in regards to their ethical restraint. It’s like a king who has defeated his enemies. He sees no danger from his foes in any quarter. In the same way, a mendicant thus accomplished in ethics sees no danger in any quarter in regards to their ethical restraint. When they have this entire spectrum of noble ethics, they experience a blameless happiness inside themselves. That’s how a mendicant is accomplished in ethics. This, Kassapa, is that accomplishment in ethics. … They enter and remain in the first absorption … This pertains to their accomplishment in mind. … They enter and remain in the second absorption … third absorption … fourth absorption. This pertains to their accomplishment in mind. This, Kassapa, is that accomplishment in mind. 

When\marginnote{20.1} their mind is immersed like this, they project it and extend it toward knowledge and vision … This pertains to their accomplishment in wisdom. … They understand: ‘… there is nothing further for this place.’ This pertains to their accomplishment in wisdom. This, Kassapa, is that accomplishment in wisdom. 

And,\marginnote{20.7} Kassapa, there is no accomplishment in ethics, mind, and wisdom that is better or finer than this. 

\section*{6. The Lion’s Roar }

There\marginnote{21.1} are, Kassapa, some ascetics and brahmins who teach ethics. They praise ethical conduct in many ways. But as far as the highest noble ethics goes, I don’t see anyone who’s my equal, still less my superior. Rather, I am the one who is superior when it comes to the higher ethics.\footnote{This is the ethical practices as described in the Gradual Training. } 

There\marginnote{21.5} are, Kassapa, some ascetics and brahmins who teach mortification in disgust of sin.\footnote{“Mortification in disgust of sin” renders \textit{\textsanskrit{tapojigucchā}}; \textit{tapo} is “fervent mortification” and \textit{\textsanskrit{jigucchā}} is “loathing, disgust”. It captures the severity with which practitioners regarded the “evil” or “sin” with which they were infected, like a quasi-physical stain on the soul, and the burning flame of white-hot pain required to cauterize their spiritual wounds. } They praise fervent mortification in disgust of sin in many ways. But as far as the highest noble fervent mortification in disgust of sin goes, I don’t see anyone who’s my equal, still less my superior. Rather, I am the one who is superior when it comes to the higher disgust of sin.\footnote{The Buddha repurposes the concept of “disgust of sin”, which here stands in the place of meditation (\textit{\textsanskrit{samādhi}}). } 

There\marginnote{21.9} are, Kassapa, some ascetics and brahmins who teach wisdom. They praise wisdom in many ways. But as far as the highest noble wisdom goes, I don’t see anyone who’s my equal, still less my superior. Rather, I am the one who is superior when it comes to the higher wisdom. 

There\marginnote{21.13} are, Kassapa, some ascetics and brahmins who teach freedom. They praise freedom in many ways. But as far as the highest noble freedom goes, I don’t see anyone who’s my equal, still less my superior. Rather, I am the one who is superior when it comes to the higher freedom. 

It’s\marginnote{22.1} possible that wanderers of other religions might say: ‘The ascetic Gotama only roars his lion’s roar in an empty hut, not in an assembly.’\footnote{A “lion’s roar” is an unapologetic proclamation of spiritual supremacy. } They should be told, ‘Not so!’ What should be said is this: ‘The ascetic Gotama roars his lion’s roar, and he roars it in the assemblies.’ 

It’s\marginnote{22.5} possible that wanderers of other religions might say: ‘The ascetic Gotama roars his lion’s roar, and he roars it in the assemblies. But he doesn’t roar it boldly.’ They should be told, ‘Not so!’ What should be said is this: ‘The ascetic Gotama roars his lion’s roar, he roars it in the assemblies, and he roars it boldly.’ 

It’s\marginnote{22.9} possible that wanderers of other religions might say: ‘The ascetic Gotama roars his lion’s roar, he roars it in the assemblies, and he roars it boldly. But they don’t question him. … Or he doesn’t answer their questions. … Or his answers are not satisfactory. … Or they don’t think him worth listening to. … Or they’re not confident after listening. … Or they don’t show their confidence. … Or they don’t practice accordingly. … Or they don’t succeed in their practice.’ They should be told, ‘Not so!’ What should be said is this: ‘The ascetic Gotama roars his lion’s roar; he roars it in the assemblies; he roars it boldly; they question him; he answers their questions; his answers are satisfactory; they think him worth listening to; they’re confident after listening; they show their confidence; they practice accordingly; and they succeed in their practice.’ 

\section*{7. The Probation For One Previously Ordained }

Kassapa,\marginnote{23.1} this one time I was staying near \textsanskrit{Rājagaha}, on the Vulture’s Peak Mountain. There a certain fervent celibate named Nigrodha asked me about the higher disgust of sin.\footnote{The Buddha is referring to the events of the Udumbarikasutta (\href{https://suttacentral.net/dn25/en/sujato}{DN 25}). There Nigrodha is referred to as a “wanderer” (\textit{\textsanskrit{paribbājaka}}) who according to the commentary was clothed. The term \textit{\textsanskrit{tapabrahmacārī}} here is unique and is not explained in the commentary. I think it means he was a celibate student of a Brahmanical teacher. } I answered his question. He was extremely happy with my answer.” 

“Sir,\marginnote{23.5} who wouldn’t be extremely happy after hearing the Buddha’s teaching? For I too am extremely happy after hearing the Buddha’s teaching! Excellent, sir! Excellent! As if he were righting the overturned, or revealing the hidden, or pointing out the path to the lost, or lighting a lamp in the dark so people with clear eyes can see what’s there, so too the Buddha has made the teaching clear in many ways. I go for refuge to the Buddha, to the teaching, and to the mendicant \textsanskrit{Saṅgha}. Sir, may I receive the going forth, the ordination in the Buddha’s presence?” 

“Kassapa,\marginnote{24.1} if someone formerly ordained in another sect wishes to take the going forth, the ordination in this teaching and training, they must spend four months on probation. When four months have passed, if the mendicants are satisfied, they’ll give the going forth, the ordination into monkhood.\footnote{This probation is laid down in the Vinaya at \href{https://suttacentral.net/pli-tv-kd1/en/sujato\#38.1.5}{Kd 1:38.1.5}. The candidate shaves, dons the robes, takes refuge, and asks for probation. They must show good conduct and restraint, diligence in duties, and enthusiasm for the Buddha’s teachings and practice. } However, I have recognized individual differences in this matter.”\footnote{In addition to individual exceptions, there are general exceptions for dreadlocked ascetics, since they believe in kamma, and for the Buddha’s relatives. } 

“Sir,\marginnote{24.3} if four months probation are required in such a case, I’ll spend four years on probation. When four years have passed, if the mendicants are satisfied, let them give me the going forth, the ordination into monkhood.” 

And\marginnote{24.4} the naked ascetic Kassapa received the going forth, the ordination in the Buddha’s presence. Not long after his ordination, Venerable Kassapa, living alone, withdrawn, diligent, keen, and resolute, soon realized the supreme end of the spiritual path in this very life. He lived having achieved with his own insight the goal for which gentlemen rightly go forth from the lay life to homelessness. 

He\marginnote{24.6} understood: “Rebirth is ended; the spiritual journey has been completed; what had to be done has been done; there is nothing further for this place.” And Venerable Kassapa became one of the perfected. 

%
\chapter*{{\suttatitleacronym DN 9}{\suttatitletranslation With Poṭṭhapāda }{\suttatitleroot Poṭṭhapādasutta}}
\addcontentsline{toc}{chapter}{\tocacronym{DN 9} \toctranslation{With Poṭṭhapāda } \tocroot{Poṭṭhapādasutta}}
\markboth{With Poṭṭhapāda }{Poṭṭhapādasutta}
\extramarks{DN 9}{DN 9}

\section*{1. On the Wanderer \textsanskrit{Poṭṭhapāda} }

\scevam{So\marginnote{1.1} I have heard. }At one time the Buddha was staying near \textsanskrit{Sāvatthī} in Jeta’s Grove, \textsanskrit{Anāthapiṇḍika}’s monastery. 

Now\marginnote{1.3} at that time the wanderer \textsanskrit{Poṭṭhapāda} was residing together with three hundred wanderers in \textsanskrit{Mallikā}’s single-halled monastery for philosophical debates, hedged by pale-moon ebony trees.\footnote{\textsanskrit{Poṭṭhapāda} appears only here; he was named for a month of the lunar calendar (August/September). | \textsanskrit{Mallikā} was the chief queen of Pasenadi, and her hall is mentioned in a similar context at \href{https://suttacentral.net/mn78/en/sujato\#1.3}{MN 78:1.3}. The commentary explains that the brahmins, Jains, and others would assemble there to “debate their beliefs” (\textit{\textsanskrit{samayaṁ} pavadanti}). Evidently the monastery grounds had accommodation for many ascetics of different beliefs, but only “one hall” where they would gather for debate. We hear many times of such debates, but here we catch a glimpse of a place that was set up to facilitate them. For \textit{-\textsanskrit{ācīra}}, read in the sense of “boundary, hedge” (commentary: \textit{\textsanskrit{timbarūrukkhapantiyā} \textsanskrit{parikkhittattā}}; cf. Sanskrit \textit{\textsanskrit{prācīra}}, “enclosure, hedge, fence, wall”). } Then the Buddha robed up in the morning and, taking his bowl and robe, entered \textsanskrit{Sāvatthī} for alms. 

Then\marginnote{2.1} it occurred to him, “It’s too early to wander for alms in \textsanskrit{Sāvatthī}.\footnote{According to the commentary, when he approached the vicinity of the city gate, he decided to check the position of the sun and noticed that it was too early to enter. The commentary and sub-commentary explain that it only sounds like the Buddha was in doubt, for Buddhas deliberate before deciding on a course of action. } Why don’t I go to \textsanskrit{Mallikā}’s monastery to visit the wanderer \textsanskrit{Poṭṭhapāda}?”\footnote{This can be understood as answering the criticism voiced in \href{https://suttacentral.net/dn8/en/sujato\#22.2}{DN 8:22.2}, that the Buddha was afraid to speak in an assembly. } So that’s what he did. 

Now\marginnote{3.1} at that time, \textsanskrit{Poṭṭhapāda} was sitting together with a large assembly of wanderers making an uproar, a dreadful racket. They engaged in all kinds of low talk, such as\footnote{In contrast with the silence of the Buddha’s community at \href{https://suttacentral.net/dn2/en/sujato\#10.7}{DN 2:10.7}. } talk about kings, bandits, and ministers; talk about armies, threats, and wars; talk about food, drink, clothes, and beds; talk about garlands and fragrances; talk about family, vehicles, villages, towns, cities, and countries; talk about women and heroes; street talk and well talk; talk about the departed; motley talk; tales of land and sea; and talk about being reborn in this or that place. 

\textsanskrit{Poṭṭhapāda}\marginnote{4.1} saw the Buddha coming off in the distance, and hushed his own assembly, “Be quiet, good sirs, don’t make a sound. Here comes the ascetic Gotama. The venerable likes quiet and praises quiet. Hopefully if he sees that our assembly is quiet he’ll see fit to approach.”\footnote{The Buddha encourages quiet for the sake of mental development; \textsanskrit{Poṭṭhapāda} does it for the sake of reputation. } Then those wanderers fell silent. 

Then\marginnote{5.1} the Buddha approached \textsanskrit{Poṭṭhapāda}, who said to him, “Let the Blessed One come, sir!\footnote{\textsanskrit{Poṭṭhapāda}’s address is almost overly deferential. } Welcome to the Blessed One, sir! It’s been a long time since you took the opportunity to come here. Please, sir, sit down, this seat is ready.” 

The\marginnote{5.7} Buddha sat on the seat spread out, while \textsanskrit{Poṭṭhapāda} took a low seat and sat to one side. The Buddha said to him, “\textsanskrit{Poṭṭhapāda}, what were you sitting talking about just now? What conversation was left unfinished?”\footnote{Always polite, the Buddha begins by showing an interest in them. } 

\subsection*{1.1. On the Cessation of Perception }

When\marginnote{6.1} he said this, the wanderer \textsanskrit{Poṭṭhapāda} said to the Buddha, “Sir, leave aside what we were sitting talking about just now. It won’t be hard for you to hear about that later. 

Sir,\marginnote{6.4} a few days ago several ascetics and brahmins who follow various other religions were sitting together at the debating hall, and a discussion about the cessation of perception came up among them:\footnote{\textit{\textsanskrit{Abhisaññā}} does not appear elsewhere. Here the prefix \textit{abhi-} means not “higher”, but rather “about, concerning”. Compare \textit{abhidhamma} at \href{https://suttacentral.net/mn32/en/sujato\#8.6}{MN 32:8.6}: \textit{dve \textsanskrit{bhikkhū} \textsanskrit{abhidhammakathaṁ} kathenti} (“two mendicants engage in discussion about the teaching”); also \textit{abhivinaya} at \href{https://suttacentral.net/an3.140/en/sujato\#4.4}{AN 3.140:4.4}. The commentary here says \textit{abhi-} is a mere particle, so it need not be translated. } ‘How does the cessation of perception happen?’\footnote{This discussion appears to have been directly sparked by the difficult passage in \textsanskrit{Bṛhadāraṇyaka} \textsanskrit{Upaniṣad} 2.4.12 and 4.5.13. The sage \textsanskrit{Yājñavalkya}, teaching his wife \textsanskrit{Maitreyī}, says that the true Self is a sheer mass of “consciousness” (\textit{\textsanskrit{vijñāna}}, Pali \textit{\textsanskrit{viññāṇa}}), which is “great, endless, infinite reality”. After realizing this, he says, there is no “perception” (\textit{\textsanskrit{saṁjñā}}, Pali \textit{\textsanskrit{saññā}}), a statement that bewilders even the wise \textsanskrit{Maitreyī}. He explains that only in an apparent state of duality (\textit{dvaitamiva}) do the separate functions of sense consciousness operate. When all is realized as the Self, how, he asks, can one know that owing to which all this is known? He is implicitly distinguishing between \textit{\textsanskrit{viññāna}} as “infinite” (= \textit{vi-}) knowing and \textit{\textsanskrit{saññā}} as “constrained” (\textit{\textsanskrit{saṁ}-}) knowing. \textsanskrit{Yājñavalkya} says the separate Self emerges with these elements and vanishes with them (\textit{etebhyo \textsanskrit{bhūtebhyaḥ} \textsanskrit{samutthāya} \textsanskrit{tānyevānu} \textsanskrit{vinaśyati}}), but he does not explain how or why this happens, which is the question the theorists here attempt to address. } 

Some\marginnote{6.6} of them said: ‘A person’s perceptions arise and cease without cause or reason.\footnote{Here perception is not identified with the “person” (\textit{purisa}), but rather belongs to them (cf. \textit{\textsanskrit{etaṁ} mama}, “this is mine”). In the discussion to follow, the Buddha only directly addresses this theory, while the remainder are included by inference. } When they arise, you become percipient. When they cease, you become non-percipient.’ That’s how some describe the cessation of perception. 

But\marginnote{6.11} someone else says: ‘That’s not how it is, good sirs!\footnote{This idiom is also at \href{https://suttacentral.net/sn47.19/en/sujato\#1.10}{SN 47.19:1.10}. } Perception is a person’s self,\footnote{The self is defined as perception (\textit{eso me \textsanskrit{attā}}), one of the five aggregates. Compare the various theories of the self and perception at \href{https://suttacentral.net/dn1/en/sujato\#2.38.0}{DN 1:2.38.0}. } which enters and departs. When it enters, you become percipient. When it departs, you become non-percipient.’\footnote{Implying that at such times a person lacks a “self”. This is perhaps related to \textsanskrit{Yājñavalkya}’s: “That man, when born, acquiring a body, is connected with ills (the bodily organs); and when he dies, departing, he discards those ills” (\textsanskrit{Bṛhadāraṇyaka} \textsanskrit{Upaniṣad} 4.3.8). } That’s how some describe the cessation of perception. 

But\marginnote{6.18} someone else says: ‘That’s not how it is, good sirs! There are ascetics and brahmins of great power and might.\footnote{The commentary says these were devotees of \textit{\textsanskrit{āthabbaṇa}}, i.e the practices preserved in the Atharvaveda. This “fourth Veda” is mentioned only once by name in the early Pali (\href{https://suttacentral.net/snp4.14/en/sujato\#13.1}{Snp 4.14:13.1}), where, as here, it is associated with the performance of magic and the casting of spells. The commentary fairly drips with contempt: “Allegedly, the \textsanskrit{Āthabbaṇa} practitioners cast a spell, showing a creature’s head as if cut off, or their hand as if cut off, or as if dead. Then they show them back to normal; imagining so, they say, ‘From cessation they have arisen.’” } They insert and extract a person’s perception.\footnote{See \href{https://suttacentral.net/dn29/en/sujato\#16.20}{DN 29:16.20} for \textit{\textsanskrit{upakaḍḍhati}} and \textit{\textsanskrit{apakaḍḍhati}} in this sense. } When they insert it, you become percipient. When they extract it, you become non-percipient.’ That’s how some describe the cessation of perception. 

But\marginnote{6.25} someone else says: ‘That’s not how it is, good sirs! There are deities of great power and might. They insert and extract a person’s perception. When they insert it, you become percipient. When they extract it, you become non-percipient.’ That’s how some describe the cessation of perception. 

That\marginnote{6.32} reminded me of the Buddha: ‘Surely it must be the Blessed One, the Holy One who is so very skilled in such matters.’ The Buddha is skilled and well-versed concerning the cessation of perception.\footnote{\textit{\textsanskrit{Pakataññū}} is not elsewhere attested in the suttas, but it is found in the Vinaya, for example at \href{https://suttacentral.net/pli-tv-bu-vb-pc72/en/sujato\#1.8}{Bu Pc 72:1.8}). } How does the cessation of perception happen?” 

\subsection*{1.2. Perception Arises With a Cause }

“Regarding\marginnote{7.1} this, \textsanskrit{Poṭṭhapāda}, those ascetics and brahmins who say that a person’s perceptions arise and cease without cause or reason are wrong from the start. Why is that? Because a person’s perceptions arise and cease with cause and reason. With training, certain perceptions arise and certain perceptions cease.\footnote{Here the Buddha argues that, since it is possible to change the nature of one’s own mind through practice, such changes cannot be random. } 

And\marginnote{7.6} what is that training?” said the Buddha.\footnote{It is is unusual if not unique to add “said the Buddha” (\textit{\textsanskrit{bhagavā} avoca}) in such a context. Normally, once a speaker starts, the text does not insert extra tags identifying the speaker without a reason, such as an interruption for a question. } 

“It’s\marginnote{7.7} when a Realized One arises in the world, perfected, a fully awakened Buddha … That’s how a mendicant is accomplished in ethics. …\footnote{Sinhalese manuscripts, followed by PTS, include the paragraphs on the metaphor of the king’s security (\href{https://suttacentral.net/dn2/en/sujato\#63.1}{DN 2:63.1}) and on guarding the sense doors here. } 

Seeing\marginnote{9.1} that the hindrances have been given up in them, joy springs up. Being joyful, rapture springs up. When the mind is full of rapture, the body becomes tranquil. When the body is tranquil, they feel bliss. And when blissful, the mind becomes immersed. 

Quite\marginnote{10.1} secluded from sensual pleasures, secluded from unskillful qualities, they enter and remain in the first absorption, which has the rapture and bliss born of seclusion, while placing the mind and keeping it connected. The sensual perception that they had previously ceases.\footnote{Progress through the \textit{\textsanskrit{jhānas}} is explained in terms of the refining of perceptions. } At that time they have a subtle and true perception of the rapture and bliss born of seclusion.\footnote{“Subtle and true” is \textit{sukhumasacca}, a term that appears only here. } That’s how, with training, certain perceptions arise and certain perceptions cease. And this is that training,” said the Buddha. 

“Furthermore,\marginnote{11.1} as the placing of the mind and keeping it connected are stilled, a mendicant enters and remains in the second absorption, which has the rapture and bliss born of immersion, with internal clarity and mind at one, without placing the mind and keeping it connected. The subtle and true perception of the rapture and bliss born of seclusion that they had previously ceases. At that time they have a subtle and true perception of the rapture and bliss born of immersion. That’s how, with training, certain perceptions arise and certain perceptions cease. And this is that training,” said the Buddha. 

“Furthermore,\marginnote{12.1} with the fading away of rapture, a mendicant enters and remains in the third absorption, where they meditate with equanimity, mindful and aware, personally experiencing the bliss of which the noble ones declare, ‘Equanimous and mindful, one meditates in bliss.’ The subtle and true perception of the rapture and bliss born of immersion that they had previously ceases. At that time they have a subtle and true perception of bliss with equanimity. That’s how, with training, certain perceptions arise and certain perceptions cease. And this is that training,” said the Buddha. 

“Furthermore,\marginnote{13.1} giving up pleasure and pain, and ending former happiness and sadness, a mendicant enters and remains in the fourth absorption, without pleasure or pain, with pure equanimity and mindfulness. The subtle and true perception of bliss with equanimity that they had previously ceases. At that time they have a subtle and true perception of neutral feeling. That’s how, with training, certain perceptions arise and certain perceptions cease. And this is that training,” said the Buddha. 

“Furthermore,\marginnote{14.1} a mendicant, going totally beyond perceptions of form, with the ending of perceptions of impingement, not focusing on perceptions of diversity, aware that ‘space is infinite’, enters and remains in the dimension of infinite space.\footnote{We have encountered the “formless attainments” (\textit{\textsanskrit{arūpasamāpatti}}) before, where they formed a refined basis for attachment to self (\href{https://suttacentral.net/dn1/en/sujato\#3.13.4}{DN 1:3.13.4}). Here they appear as part of the gradual refinement of consciousness through the cessation of increasingly subtle perceptions. } The perception of luminous form that they had previously ceases.\footnote{This is the vision of light that later came to be called a “sign” (\textit{nimitta}). In the first four \textit{\textsanskrit{jhānas}} this persists as a “subtle” (\textit{sukhuma}) reflection or echo of the “substantial” (\textit{\textsanskrit{olārika}}) material basis of meditation, such as the breath or the parts of the body. Even though it is a purely mental phenomenon, it is still “form” (\textit{\textsanskrit{rūpa}}) since it has physical properties like light or extension. } At that time they have a subtle and true perception of the dimension of infinite space.\footnote{The “light” (\textit{\textsanskrit{obhāsa}}) of \textit{\textsanskrit{jhāna}} grows from “limited” (\textit{paritta}) to “limitless” (\textit{\textsanskrit{appamāṇa}}, \href{https://suttacentral.net/mn128/en/sujato\#29.1}{MN 128:29.1}). Then the perception of even this limitless light vanishes, leaving only infinite space. } That’s how, with training, certain perceptions arise and certain perceptions cease. And this is that training,” said the Buddha. 

“Furthermore,\marginnote{15.1} a mendicant, going totally beyond the dimension of infinite space, aware that ‘consciousness is infinite’, enters and remains in the dimension of infinite consciousness.\footnote{Perception of infinite space fades away leaving only the infinite consciousness that is aware. } The subtle and true perception of the dimension of infinite space that they had previously ceases. At that time they have a subtle and true perception of the dimension of infinite consciousness. That’s how, with training, certain perceptions arise and certain perceptions cease. And this is that training,” said the Buddha. 

“Furthermore,\marginnote{16.1} a mendicant, going totally beyond the dimension of infinite consciousness, aware that ‘there is nothing at all’, enters and remains in the dimension of nothingness.\footnote{The meditator is no longer even aware of infinite consciousness, but of the even vaster nothingness. } The subtle and true perception of the dimension of infinite consciousness that they had previously ceases. At that time they have a subtle and true perception of the dimension of nothingness. That’s how, with training, certain perceptions arise and certain perceptions cease.\footnote{The last of the four formless attainments is the dimension of neither perception nor non-perception. Since this, by definition, lies beyond the scope of perception, it is not included here. } And this is that training,” said the Buddha. 

“\textsanskrit{Poṭṭhapāda},\marginnote{17.1} from the time a mendicant here takes charge of their own perception, they proceed from one stage to the next, progressively reaching the peak of perception.\footnote{For \textit{\textsanskrit{sakasaññī}} (“takes charge of their own perception”), see \href{https://suttacentral.net/pli-tv-bu-vb-pj2/en/sujato\#6.2.2}{Bu Pj 2:6.2.2}. One relevant factor in determining whether an object has been stolen is if the accused “perceives it as their own”. Here it means that the meditator understands that they can evolve their own perceptions through meditation. } Standing on the peak of perception they think, ‘Intentionality is bad for me, it’s better to be free of it. For if I were to intend and choose, these perceptions would cease in me, and other coarser perceptions would arise.\footnote{“Intend and choose” is \textit{\textsanskrit{ceteyyaṁ} \textsanskrit{abhisaṅkhareyyaṁ}} (1st singular optative). These synonyms are used in the sense of a subtle sense of will or intentionality that underlies such attainments (\href{https://suttacentral.net/mn52/en/sujato\#14.3}{MN 52:14.3}) and which must be let go lest they generate rebirth (\href{https://suttacentral.net/mn140/en/sujato\#22.10}{MN 140:22.10}). } Why don’t I neither make a choice nor form an intention?’ They neither make a choice nor form an intention. Those perceptions cease in them, and other coarser perceptions don’t arise. They touch cessation. And that, \textsanskrit{Poṭṭhapāda}, is how the progressive cessation of perception is attained with awareness.\footnote{Compare with “progressive cessation” (\href{https://suttacentral.net/an9.31/en/sujato}{AN 9.31}), “progressive tranquilizing of conditions” (\href{https://suttacentral.net/sn36.15/en/sujato}{SN 36.15}), “progressive meditations” (\href{https://suttacentral.net/an9.32/en/sujato}{AN 9.32}), etc. | “Awareness” is \textit{\textsanskrit{sampajāna}}, which we have encountered previously as “situational awareness” in daily activities, or as the “awareness” in the third \textit{\textsanskrit{jhāna}}. Here it refers to a reflective capacity to understand the nature of deep meditation in terms of causality. The term was adopted by \textsanskrit{Patañjalī}, who defined \textit{\textsanskrit{saṁprajñātasamādhi}} in a way that is clearly drawn from the Buddhist definition of \textit{\textsanskrit{jhāna}}. It is attained with \textit{vitakka}, \textit{\textsanskrit{vicāra}}, \textit{\textsanskrit{ānanda}} (“bliss”), and \textit{\textsanskrit{āsmitā}}. According to the commentary, this last term is “experience of the one self”, \textit{\textsanskrit{ekātmikā} \textsanskrit{saṁvid}}) and so is probably adapted from the Buddhist factor of \textit{\textsanskrit{ekaggatā}}.  (\textsanskrit{Yogasūtra} 1.17). } 

What\marginnote{18.1} do you think, \textsanskrit{Poṭṭhapāda}? Have you ever heard of this before?” 

“No,\marginnote{18.3} sir.\footnote{Taking the kernel of \textsanskrit{Yājñavalkya}’s theory of the cessation of limited perception, the Buddha has expanded it in psychological and practical detail, while leaving out the metaphysical assumption of the Self. } This is how I understand what the Buddha said:\footnote{\textsanskrit{Poṭṭhapāda} shows the Buddha that he has been paying attention. Notice how it is culturally assumed that it is possible to retain and repeat the exact content of the teaching. } ‘From the time a mendicant here takes charge of their own perception, they proceed from one stage to the next, progressively reaching the peak of perception. Standing on the peak of perception they think, “Intentionality is bad for me, it’s better to be free of it. For if I were to intend and choose, these perceptions would cease in me, and other coarser perceptions would arise. Why don’t I neither make a choice nor form an intention?” Those perceptions cease in them, and other coarser perceptions don’t arise. They touch cessation. And that is how the progressive cessation of perception is attained with awareness.’” 

“That’s\marginnote{18.13} right, \textsanskrit{Poṭṭhapāda}.” 

“Does\marginnote{19.1} the Buddha describe just one peak of perception, or many?” 

“I\marginnote{19.2} describe the peak of perception as both one and many.” 

“But\marginnote{19.3} sir, how do you describe it as one peak and as many?” 

“I\marginnote{19.4} describe the peak of perception according to the specific manner in which one touches cessation.\footnote{The meaning of this is not clear to me. It might mean that insight can be developed based on any of the stages of meditation, so for that person their peak of perception is different to another’s. The commentary says that it refers to different meditation subjects, or simply to different occasions of meditation. } That’s how I describe the peak of perception as both one and many.” 

“But\marginnote{20.1} sir, does perception arise first and knowledge afterwards? Or does knowledge arise first and perception afterwards? Or do they both arise at the same time?”\footnote{\textsanskrit{Poṭṭhapāda}’s distinction between “perception” (\textit{\textsanskrit{saññā}}) and “knowledge” (\textit{\textsanskrit{ñāṇa}}) echoes \textsanskrit{Yājñavalkya}’s distinction between “perception” (\textit{\textsanskrit{saṁjā}}) and “consciousness” (\textit{\textsanskrit{vijñāna}}, or in the repeated passage at 4.5.13, \textit{\textsanskrit{prajñāna}}). } 

“Perception\marginnote{20.2} arises first and knowledge afterwards. The arising of perception leads to the arising of knowledge.\footnote{Perception has been described in terms of the progress through more refined meditations (\textit{\textsanskrit{samādhi}}). Only then does the “knowledge” (\textit{\textsanskrit{ñāṇa}}) of insight arise. } They understand, ‘My knowledge arose from a specific condition.’\footnote{The term “specific condition” (\textit{\textsanskrit{idappaccayā}}) is well known from dependent origination (eg. \href{https://suttacentral.net/sn12.20/en/sujato\#2.3}{SN 12.20:2.3}), where it applies to the general situation of ongoing existence in transmigration. When developing insight, a meditator takes their own meditation experience as their primary locus. This is then generalized to an understanding of the nature of conscious existence. } That is a way to understand how perception arises first and knowledge afterwards; that the arising of perception leads to the arising of knowledge.” 

\subsection*{1.3. Perception and the Self }

“Sir,\marginnote{21.1} is perception a person’s self, or are perception and self different things?”\footnote{This draws from the initial presentation of different theories of the person and perception (\href{https://suttacentral.net/dn9/en/sujato\#6.4}{DN 9:6.4}). } 

“But\marginnote{21.2} \textsanskrit{Poṭṭhapāda}, do you believe in a self?”\footnote{“Believe” is \textit{paccesi}, literally “fall back on”. It implies that an idea is something one relies on or takes as fundamental. The Buddha wants to know where \textsanskrit{Poṭṭhapāda} stands before exploring this topic. } 

“I\marginnote{21.3} believe in a solid self, sir, which is formed, made up of the four principal states, and consumes solid food.”\footnote{Despite his previous questions about perception and the self, \textsanskrit{Poṭṭhapāda} takes his stand on a purely materialist view, identifying the “self” with the organic “substantial” body. } 

“Suppose\marginnote{21.4} there were such a solid self, \textsanskrit{Poṭṭhapāda}. In that case, perception would be one thing, the self another.\footnote{Again, the Buddha does not rush to tell \textsanskrit{Poṭṭhapāda} he is right or wrong, but rather draws out the implications of his statement. } Here is another way to understand how perception and self are different things. So long as that solid self remains, still some perceptions arise in a person and others cease.\footnote{Here \textit{\textsanskrit{tiṭṭhateva}} is not “leaving aside”, but “remains”. Compare the similar construction at \href{https://suttacentral.net/mn107/en/sujato\#13.1}{MN 107:13.1}. } That is a way to understand how perception and self are different things.” 

“Sir,\marginnote{22.1} I believe in a mind-made self which is whole in its major and minor limbs, not deficient in any faculty.”\footnote{This is the “subtle” (\textit{sukhuma}) body, corresponding with the form experienced in the four \textsanskrit{jhānas}. All manuscripts appear to be missing the expected \textit{\textsanskrit{rūpī}} in this passage, but it occurs in the corresponding passage on “reincarnation” below. } 

“Suppose\marginnote{22.2} there were such a mind-made self, \textsanskrit{Poṭṭhapāda}. In that case, perception would be one thing, the self another. Here is another way to understand how perception and self are different things. So long as that mind-made self remains, still some perceptions arise in a person and others cease. That too is a way to understand how perception and self are different things.” 

“Sir,\marginnote{23.1} I believe in a formless self which is made of perception.”\footnote{This identifies the self as that which is experienced in the formless attainments. \textsanskrit{Poṭṭhapāda} is simply cycling through possible self theories without really thinking through the implications. } 

“Suppose\marginnote{23.2} there were such a formless self, \textsanskrit{Poṭṭhapāda}. In that case, perception would be one thing, the self another. Here is another way to understand how perception and self are different things. So long as that formless self remains, still some perceptions arise in a person and others cease. That too is a way to understand how perception and self are different things.” 

“But,\marginnote{24.1} sir, am I able to know whether\footnote{\textsanskrit{Poṭṭhapāda} can only attest a belief in various theories and still does not know how to assess them for himself. } perception is a person’s self, or whether perception and self are different things?” 

“It’s\marginnote{24.3} hard for you to understand this, since you have a different view, creed, and belief, unless you dedicate yourself to practice with the guidance of tradition.”\footnote{The text shifts from \textit{\textsanskrit{añña}} “other” (eg. \textit{\textsanskrit{aññadiṭṭhikena}}) to \textit{\textsanskrit{aññatra}} (\textit{\textsanskrit{aññatr}’\textsanskrit{āyogena}}), which normally means “apart from”. Most authorities follow the commentary in taking \textit{\textsanskrit{aññatra}} here in the sense of “other”. However I think the change of sense is deliberate; the Buddha is not discouraging them, merely informing them what it will take. } 

“Well,\marginnote{25.1} if that’s the case, sir, then what do you make of this: ‘The cosmos is eternal. This is the only truth, anything else is futile’?”\footnote{This is the famous list of ten “undeclared points”, which are found throughout the suttas (eg. \href{https://suttacentral.net/mn25/en/sujato\#10.21}{MN 25:10.21}, \href{https://suttacentral.net/mn63/en/sujato\#2.3}{MN 63:2.3}, \href{https://suttacentral.net/mn72/en/sujato\#3.1}{MN 72:3.1}, and the whole of SN 44). They seem to have functioned as a kind of checklist by which philosophers could be evaluated and classified. | The word \textit{loka} occurs in a number of senses, but here it refers to the entire “cosmos” of countless worlds. } 

“This\marginnote{25.4} has not been declared by me, \textsanskrit{Poṭṭhapāda}.” 

“Then\marginnote{26.1} what do you make of this: ‘The cosmos is not eternal. This is the only truth, anything else is futile’?” 

“This\marginnote{26.2} too has not been declared by me.” 

“Then\marginnote{27.1} what do you make of this: ‘The cosmos is finite …’ … ‘The cosmos is infinite …’ … ‘The soul and the body are the same thing …’ … ‘The soul and the body are different things …’ … ‘A realized one still exists after death …’ … ‘A realized one no longer exists after death …’ … ‘A realized one both still exists and no longer exists after death …’ … ‘A Realized One neither still exists nor no longer exists after death. This is the only truth, anything else is futile’?” 

“This\marginnote{27.9} too has not been declared by me.” 

“Why\marginnote{28.1} haven’t these things been declared by the Buddha?” 

“Because\marginnote{28.2} they’re not beneficial or relevant to the fundamentals of the spiritual life. They don’t lead to disillusionment, dispassion, cessation, peace, insight, awakening, and extinguishment. That’s why I haven’t declared them.” 

“Then\marginnote{29.1} what has been declared by the Buddha?” 

“I\marginnote{29.2} have declared this: ‘This is suffering’ … ‘This is the origin of suffering’ … ‘This is the cessation of suffering’ … ‘This is the practice that leads to the cessation of suffering’.” 

“Why\marginnote{30.1} have these things been declared by the Buddha?” 

“Because\marginnote{30.2} they are beneficial and relevant to the fundamentals of the spiritual life. They lead to disillusionment, dispassion, cessation, peace, insight, awakening, and extinguishment. That’s why I have declared them.” 

“That’s\marginnote{30.4} so true, Blessed One! That’s so true, Holy One! Please, sir, go at your convenience.” Then the Buddha got up from his seat and left. 

Soon\marginnote{31.1} after the Buddha left, those wanderers beset \textsanskrit{Poṭṭhapāda} on all sides with sneering and jeering.\footnote{This phrase recurs at \href{https://suttacentral.net/sn21.9/en/sujato\#1.4}{SN 21.9:1.4} and \href{https://suttacentral.net/an3.64/en/sujato\#11.1}{AN 3.64:11.1}, with some variant readings. } “No matter what the ascetic Gotama says, \textsanskrit{Poṭṭhapāda} agrees with him: ‘That’s so true, Blessed One! That’s so true, Holy One!’ We understand that the ascetic Gotama didn’t give any categorical teaching at all regarding whether the cosmos is eternal and so on.” 

When\marginnote{31.6} they said this, \textsanskrit{Poṭṭhapāda} said to them, “I too understand that the ascetic Gotama didn’t give any categorical teaching at all regarding whether the cosmos is eternal and so on. Nevertheless, the practice that he describes is true, real, and accurate. It is the regularity of natural principles, the invariance of natural principles. So how could a sensible person such as I not agree that what was well spoken by the ascetic Gotama was in fact well spoken?” 

\section*{2. On Citta \textsanskrit{Hatthisāriputta} }

Then\marginnote{32.1} after two or three days had passed, Citta \textsanskrit{Hatthisāriputta} and \textsanskrit{Poṭṭhapāda} went to see the Buddha. Citta \textsanskrit{Hatthisāriputta} bowed and sat down to one side.\footnote{The commentary says Citta was the son of an elephant trainer. Here he shows greater respect to the Buddha than does \textsanskrit{Poṭṭhapāda}. } But the wanderer \textsanskrit{Poṭṭhapāda} exchanged greetings with the Buddha, and when the greetings and polite conversation were over, he sat down to one side. \textsanskrit{Poṭṭhapāda} told the Buddha what had happened after he left. The Buddha said: 

“All\marginnote{33.1} those wanderers, \textsanskrit{Poṭṭhapāda}, are blind and sightless. You are the only one whose eyes are clear. For I have taught and pointed out teachings that are categorical\footnote{This point seems to be lost on a number of modern commentators, who infer from passages such as the ten undeclared points that the Buddha refused to make any definitive declarations at all. The Buddha, rather, was a \textit{\textsanskrit{vibhajjavādin}} (\href{https://suttacentral.net/mn99/en/sujato\#4.4}{MN 99:4.4}, \href{https://suttacentral.net/an10.94/en/sujato\#4.7}{AN 10.94:4.7}), “one who speaks after analysis”. } and also teachings that are not categorical. 

And\marginnote{33.5} what teachings have I taught and pointed out as not categorical? ‘The cosmos is eternal’ … ‘The cosmos is not eternal’ … ‘The cosmos is finite’ … ‘The cosmos is infinite’ … ‘The soul is the same thing as the body’ … ‘The soul and the body are different things’ … ‘A realized one still exists after death’ … ‘A realized one no longer exists after death’ … ‘A realized one both still exists and no longer exists after death’ … ‘A realized one neither still exists nor no longer exists after death.’ 

And\marginnote{33.16} why have I taught and pointed out such teachings as not categorical? Because those things aren’t beneficial or relevant to the fundamentals of the spiritual life. They don’t lead to disillusionment, dispassion, cessation, peace, insight, awakening, and extinguishment. That’s why I have taught and pointed out such teachings as not categorical. 

\subsection*{2.1. Teachings That Are Categorical }

And\marginnote{33.20} what teachings have I taught and pointed out as categorical? ‘This is suffering’ … ‘This is the origin of suffering’ … ‘This is the cessation of suffering’ … ‘This is the practice that leads to the cessation of suffering’. 

And\marginnote{33.25} why have I taught and pointed out such teachings as categorical? Because they are beneficial and relevant to the fundamentals of the spiritual life. They lead to disillusionment, dispassion, cessation, peace, insight, awakening, and extinguishment. That’s why I have taught and pointed out such teachings as categorical. 

There\marginnote{34.1} are some ascetics and brahmins who have this doctrine and view: ‘The self is perfectly happy and free of disease after death.’\footnote{See \href{https://suttacentral.net/dn1/en/sujato\#2.38.2}{DN 1:2.38.2}. } I go up to them and say, ‘Is it really true that this is the venerables’ view?’\footnote{The Buddha does not rely on rumor; he begins by checking his facts with those concerned. Not only does this affirm his commitment to truth, it shows respect and establishes a common ground from which the argument proceeds. } And they answer, ‘Yes’. I say to them, ‘But do you meditate knowing and seeing a perfectly happy world?’\footnote{The verb \textit{viharati} means “dwell”, and functions as an auxiliary verb implying duration. In spiritual contexts it often means “a period or state of meditation”. This first question is asking whether they see such a state in a regular meditation practice. } Asked this, they say, ‘No.’ 

I\marginnote{34.10} say to them, ‘But have you perceived a perfectly happy self for a single day or night, or even half a day or night?’\footnote{Perhaps they might not be able to develop a meditation for seeing that self, but at some point they may have had some sort of perception or vision or insight. } Asked this, they say, ‘No.’ 

I\marginnote{34.13} say to them, ‘But do you know a path and a practice to realize a perfectly happy world?’\footnote{Since they have no experience, they might at least have an idea how to reach that experience. } Asked this, they say, ‘No.’ 

I\marginnote{34.17} say to them, ‘But have you ever heard the voice of the deities reborn in a perfectly happy world saying, “Practice well, dear sirs, practice sincerely so as to realize a perfectly happy world.\footnote{They haven’t even heard a report about it. } For this is how we practiced, and we were reborn in a perfectly happy world”?’ Asked this, they say, ‘No.’ 

What\marginnote{34.22} do you think, \textsanskrit{Poṭṭhapāda}? This being so, doesn’t what they say turn out to have no demonstrable basis?”\footnote{“No demonstrable basis” renders \textit{\textsanskrit{appāṭihīrakataṁ}}. This is related to \textit{\textsanskrit{pāṭihāra}}, which is usually understood as “miracle, wonder”. But the root sense is “demonstration” and the sense of “display of wonder” is secondary. } 

“Clearly\marginnote{34.24} that’s the case, sir.” 

“Suppose,\marginnote{35.1} \textsanskrit{Poṭṭhapāda}, a man were to say: ‘Whoever the finest lady in the land is, it is her that I want, her that I desire!’\footnote{\textit{\textsanskrit{Janapadakalyāṇī}} is typically rendered as “the most beautiful lady in the land”. At \href{https://suttacentral.net/sn47.20/en/sujato\#2.2}{SN 47.20:2.2} we learn that she is a dazzling singer and dancer. And while she was  famed for her beauty (\href{https://suttacentral.net/ud3.2/en/sujato\#9.1}{Ud 3.2:9.1}), the word \textit{\textsanskrit{kalyāṇa}} normally means “(morally) good, fine, lovely” and does not refer solely to her appearance. } They’d say to him, ‘Mister, that finest lady in the land who you desire—do you know whether she’s an aristocrat, a brahmin, a peasant, or a menial?’ Asked this, he’d say, ‘No.’ They’d say to him, ‘Mister, that finest lady in the land who you desire—do you know her name or clan? Whether she’s tall or short or medium? Whether her skin is black, brown, or tawny? What village, town, or city she comes from?’ Asked this, he’d say, ‘No.’ They’d say to him, ‘Mister, do you desire someone who you’ve never even known or seen?’ Asked this, he’d say, ‘Yes.’ 

What\marginnote{35.12} do you think, \textsanskrit{Poṭṭhapāda}? This being so, doesn’t that man’s statement turn out to have no demonstrable basis?” 

“Clearly\marginnote{35.14} that’s the case, sir.” 

“In\marginnote{36.1} the same way, the ascetics and brahmins who have that doctrine and view … 

Doesn’t\marginnote{36.22} what they say turn out to have no demonstrable basis?” 

“Clearly\marginnote{36.23} that’s the case, sir.” 

“Suppose\marginnote{37.1} a man was to build a ladder at the crossroads for climbing up to a stilt longhouse.\footnote{\textit{Nisseni} only occurs elsewhere in \href{https://suttacentral.net/pli-tv-bu-vb-ss6/en/sujato\#2.3.6}{Bu Ss 6:2.3.6} and \href{https://suttacentral.net/pli-tv-bu-vb-ss7/en/sujato\#2.67}{Bu Ss 7:2.67}, where it is something carried, i.e. a ladder rather than a flight of stairs. } They’d say to him, ‘Mister, that stilt longhouse that you’re building a ladder for—do you know whether it’s to the north, south, east, or west? Or whether it’s tall or short or medium?’ Asked this, he’d say, ‘No.’ They’d say to him, ‘Mister, are you building a ladder for a longhouse that you’ve never even known or seen?’ Asked this, he’d say, ‘Yes.’ 

What\marginnote{37.8} do you think, \textsanskrit{Poṭṭhapāda}? This being so, doesn’t that man’s statement turn out to have no demonstrable basis?” 

“Clearly\marginnote{37.10} that’s the case, sir.” 

“In\marginnote{38.1} the same way, the ascetics and brahmins who have those various doctrines and views … 

Doesn’t\marginnote{38.21} what they say turn out to have no demonstrable basis?” 

“Clearly\marginnote{38.22} that’s the case, sir.” 

\subsection*{2.2. Three Kinds of Reincarnation }

“\textsanskrit{Poṭṭhapāda},\marginnote{39.1} there are these three kinds of reincarnation in a life-form:\footnote{\textit{\textsanskrit{Attapaṭilābha}} is literally “re-acquisition of self”, where \textit{\textsanskrit{attā}} is explained by the commentary as \textit{\textsanskrit{attabhāva}}, the “state of the self” or “life-form” that is acquired at rebirth, i.e. the body (\textit{\textsanskrit{sarīra}}), whether material or immaterial. } a solid  life-form, a mind-made life-form, and a formless life-form.\footnote{These recap the three theses of perception and the self posted by \textsanskrit{Poṭṭhapāda} from \href{https://suttacentral.net/dn9/en/sujato\#22.1}{DN 9:22.1}. } And what is reincarnation in a solid life-form? It is formed, made up of the four principal states, and consumes solid food. What is reincarnation in a mind-made life-form? It is formed, mind-made, whole in its major and minor limbs, not deficient in any faculty. What is reincarnation in a formless life-form? It is formless, made of perception. 

I\marginnote{40.1} teach the Dhamma for the giving up of reincarnation in these three kinds of life-form. ‘When you practice accordingly, corrupting qualities will be given up in you and cleansing qualities will grow. You’ll enter and remain in the fullness and abundance of wisdom, having realized it with your own insight in this very life.’ \textsanskrit{Poṭṭhapāda}, you might think: ‘Corrupting qualities will be given up and cleansing qualities will grow. One will enter and remain in the fullness and abundance of wisdom, having realized it with one’s own insight in this very life. But such a life is suffering.’\footnote{Compare with the similar sentiment at \href{https://suttacentral.net/sn22.2/en/sujato\#10.1}{SN 22.2:10.1}. } But you should not see it like this. Corrupting qualities will be given up and cleansing qualities will grow. One will enter and remain in the fullness and abundance of wisdom, having realized it with one’s own insight in this very life. And there will be only joy and happiness, tranquility, mindfulness and awareness. Such a life is blissful. 

\textsanskrit{Poṭṭhapāda},\marginnote{43.1} if others should ask us, ‘But reverends, what is that reincarnation in a solid life-form for the giving up of which you teach?’ We’d answer like this, ‘\emph{This} is that reincarnation in a solid life-form.’ 

If\marginnote{44.1} others should ask us, ‘But reverends, what is that reincarnation in a mind-made life-form?’ We’d answer like this, ‘\emph{This} is that reincarnation in a mind-made life-form.’\footnote{The Buddha points to the experience to demonstrate what he is talking about, rather than offering a long theoretical explanation. \textit{\textsanskrit{Ayaṁ}} is a pronoun of presence, used to indicate what is apparent before the subject. } 

If\marginnote{45.1} others should ask us, ‘But reverends, what is that reincarnation in a formless life-form?’ We’d answer like this, ‘\emph{This} is that reincarnation in a formless life-form.’ 

What\marginnote{45.5} do you think, \textsanskrit{Poṭṭhapāda}? This being so, doesn’t that statement turn out to have a demonstrable basis?” 

“Clearly\marginnote{45.7} that’s the case, sir.” 

“Suppose\marginnote{46.1} a man were to build a ladder for climbing up to a stilt longhouse right underneath that longhouse. They’d say to him, ‘Mister, that stilt longhouse that you’re building a ladder for—do you know whether it’s to the north, south, east, or west? Or whether it’s tall or short or medium?’ He’d say, ‘This is that stilt longhouse for which I’m building a ladder, right underneath it.’ 

What\marginnote{46.6} do you think, \textsanskrit{Poṭṭhapāda}? This being so, doesn’t that man’s statement turn out to have a demonstrable basis?” 

“Clearly\marginnote{46.8} that’s the case, sir.” 

When\marginnote{48.1} the Buddha had spoken, Citta \textsanskrit{Hatthisāriputta} said, “Sir, when reincarnated in a solid life-form, are the mind-made and formless life-forms fictitious,\footnote{Citta is asking an ontological question, assuming that these three states are existent realities of the self. } and only the solid life-form real? When reincarnated in a mind-made life-form, are the solid and formless life-forms fictitious, and only the mind-made life-form real? When reincarnated in a formless life-form, are the solid and mind-made life-forms fictitious, and only the formless life-form real?” 

“When\marginnote{49.1} reincarnated in a solid life-form, it’s not referred to as a mind-made or formless life-form,\footnote{The Buddha reframes the question as one of conventional description. He is describing states in which one might be reborn, not underlying ontologies. } only as a solid life-form. When reincarnated in a mind-made life-form, it’s not referred to as a solid or formless life-form, only as a mind-made life-form. When reincarnated in a formless life-form, it’s not referred to as a solid or mind-made life-form, only as a formless life-form. 

Citta,\marginnote{49.7} suppose they were to ask you, ‘Did you exist in the past?\footnote{This anticipates one of the great philosophical debates of sectarian Buddhists which gave rise to the \textsanskrit{Sarvāstivāda}, the school whose core doctrine was that “all exists (in the past, future, and present)”. The Buddha describes past, future, and present with the three grammatical tenses. } Will you exist in the future? Do you exist now?’ How would you answer?” 

“Sir,\marginnote{49.12} if they were to ask me this, I’d answer like this, ‘I did exist in the past. I will exist in the future. I do exist now.’ That’s how I’d answer.” 

“But\marginnote{50.1} Citta, suppose they were to ask you, ‘Is the reincarnation you had in the past your only real one, and those of the future and present fictitious? Is the reincarnation you will have in the future your only real one, and those of the past and present fictitious? Is the reincarnation you have now your only real one, and those of the past and future fictitious?’ How would you answer?” 

“Sir,\marginnote{50.6} if they were to ask me this, I’d answer like this, ‘The reincarnation I had in the past was real at that time, and those of the future and present fictitious. The reincarnation I will have in the future will be real at the time, and those of the past and present fictitious. The reincarnation I have now is real at this time, and those of the past and future fictitious.’ That’s how I’d answer.” 

“In\marginnote{51.1} the same way, while in any one of the three reincarnations, it’s not referred to as the other two, only under its own name. 

From\marginnote{52.1} a cow comes milk, from milk comes curds, from curds come butter, from butter comes ghee, and from ghee comes cream of ghee. And the cream of ghee is said to be the best of these.\footnote{Compare Śatapatha \textsanskrit{Brāhmaṇa} 3.3.3.2. } While it’s milk, it’s not referred to as curds, butter, ghee, or cream of ghee. It’s only referred to as milk. While it’s curd or butter or ghee or cream of ghee, it’s not referred to as anything else, only under its own name. In the same way, while in any one of the three reincarnations, it’s not referred to as the other two, only under its own name. These are the world’s common usages, terms, means of communication, and descriptions, which the Realized One uses to communicate without getting stuck on them.”\footnote{Words such as “self” have a conventional usage and in that context are perfectly fine. But what that “self” refers to is constantly changing, as it is reincarnated in different states. It is like a river which keeps the same name even though the water is always changing. If, driven by attachment, we assume there is a metaphysical reality underlying the conventional “self”, we step beyond what can be empirically verified. Note, however, that the Buddha is \emph{not} asserting that there are two levels of truth, conventional and ultimate, a distinction not found in early Buddhism. | Compare \href{https://suttacentral.net/mn139/en/sujato\#3.9}{MN 139:3.9}, \href{https://suttacentral.net/mn74/en/sujato\#13.1}{MN 74:13.1}. } 

When\marginnote{54.1} he had spoken, the wanderer \textsanskrit{Poṭṭhapāda} said to the Buddha, “Excellent, sir! Excellent! As if he were righting the overturned, or revealing the hidden, or pointing out the path to the lost, or lighting a lamp in the dark so people with clear eyes can see what’s there, so too the Buddha has made the teaching clear in many ways. I go for refuge to the Buddha, to the teaching, and to the mendicant \textsanskrit{Saṅgha}. From this day forth, may the Buddha remember me as a lay follower who has gone for refuge for life.” 

\subsection*{2.3. The Ordination of Citta \textsanskrit{Hatthisāriputta} }

But\marginnote{55.1} Citta \textsanskrit{Hatthisāriputta} said to the Buddha, “Excellent, sir! Excellent! As if he were righting the overturned, or revealing the hidden, or pointing out the path to the lost, or lighting a lamp in the dark so people with clear eyes can see what’s there, so too the Buddha has made the teaching clear in many ways. I go for refuge to the Buddha, to the teaching, and to the mendicant \textsanskrit{Saṅgha}. Sir, may I receive the going forth, the ordination in the Buddha’s presence?” 

And\marginnote{56.1} Citta \textsanskrit{Hatthisāriputta} received the going forth, the ordination in the Buddha’s presence. Not long after his ordination, Venerable Citta \textsanskrit{Hatthisāriputta}, living alone, withdrawn, diligent, keen, and resolute, soon realized the supreme end of the spiritual path in this very life. He lived having achieved with his own insight the goal for which gentlemen rightly go forth from the lay life to homelessness.\footnote{In \href{https://suttacentral.net/an6.60/en/sujato}{AN 6.60} we find Citta \textsanskrit{Hatthisāriputta}, still a somewhat junior monk, rudely interrupting his seniors. After admonition he disrobed, but he ordained again and later achieved arahantship. We can reconcile these two accounts by recognizing that the phrase \textit{acira} “not long after” is a conventional term, which might be several years. Thus the events of AN 6.60 occurred some time between his ordination and awakening. } He understood: “Rebirth is ended; the spiritual journey has been completed; what had to be done has been done; there is nothing further for this place.” And Venerable Citta \textsanskrit{Hatthisāriputta} became one of the perfected. 

%
\chapter*{{\suttatitleacronym DN 10}{\suttatitletranslation With Subha }{\suttatitleroot Subhasutta}}
\addcontentsline{toc}{chapter}{\tocacronym{DN 10} \toctranslation{With Subha } \tocroot{Subhasutta}}
\markboth{With Subha }{Subhasutta}
\extramarks{DN 10}{DN 10}

\scevam{So\marginnote{1.1.1} I have heard. }At one time Venerable Ānanda was staying near \textsanskrit{Sāvatthī} in Jeta’s Grove, \textsanskrit{Anāthapiṇḍika}’s monastery. It was not long after the Buddha had become fully quenched.\footnote{Ānanda’s role became more prominent as a leader of the \textsanskrit{Saṅgha} in the years after the Buddha’s passing. This sutta shows the continued propagation of the Buddha’s teachings after his death. } 

Now\marginnote{1.1.3} at that time the student Subha, Todeyya’s son, was residing in \textsanskrit{Sāvatthī} on some business.\footnote{The same Subha earlier met the Buddha in \href{https://suttacentral.net/mn99/en/sujato}{MN 99} and again in \href{https://suttacentral.net/mn135/en/sujato}{MN 135}, where he asked about kamma. His father Todeyya was a prominent brahmin, often mentioned alongside \textsanskrit{Pokkharasāti}. The two apparently lived not far from each other, as, according to the commentary, Todeyya was named for his village of Tudi outside of \textsanskrit{Sāvatthī} (see \textsanskrit{Pāṇini}’s \textsanskrit{Aṣṭādhyāyī} 4.3.94). | These events suggest a certain, albeit tenuous, timeframe for the significant conversion of influential brahmins initiated by \textsanskrit{Pokkharasāti} in \href{https://suttacentral.net/dn3/en/sujato}{DN 3}. Here, Subha is active after the Buddha’s death, suggesting his age is aligned with that of Ānanda, a generation younger than the Buddha. If this is so, Subha’s first meeting with the Buddha would have taken place no earlier than the middle period of his teaching, perhaps twenty years before the \textsanskrit{Parinibbāna} (MN 99). There he mentions \textsanskrit{Pokkharasāti}’s hostility to the claims of ascetics, so this must precede \textsanskrit{Pokkharasāti}’s conversion in DN 3 by a considerable period. If we are on the right track, the conversion of \textsanskrit{Pokkharasāti}, and the events that flowed from that, must have happened late in the Buddha’s career, perhaps in the final decade of his life. } Then he addressed a certain young student, “Here, young student, go to the ascetic Ānanda and in my name bow with your head to his feet. Ask him if he is healthy and well, nimble, strong, and living comfortably. And then say: ‘Sir, please visit the student Subha, Todeyya’s son, at his home out of sympathy.’” 

“Yes,\marginnote{1.3.2} sir,” replied the young student, and did as he was asked. 

When\marginnote{1.4.1} he had spoken, Venerable Ānanda said to him, “It’s not the right time, young student. I have drunk a dose of medicine today.\footnote{Ānanda was getting old. } But hopefully tomorrow I’ll get a chance to visit him.” 

“Yes,\marginnote{1.4.5} sir,” replied the young student. He went back to Subha, and told him what had happened, adding, “This much, sir, I managed to do. At least Mister Ānanda will take the opportunity to visit tomorrow.” 

Then\marginnote{1.5.1} when the night had passed, Ānanda robed up in the morning and, taking his bowl and robe, went with Venerable Cetaka as his second monk to Subha’s home, where he sat on the seat spread out. Then Subha went up to Ānanda, and exchanged greetings with him.\footnote{Cetaka is mentioned only here. The commentary says he was named for his home country of \textsanskrit{Cetī}, which is roughly the modern region of Bundelkhand, about 500 km south-west of \textsanskrit{Sāvatthī}. } When the greetings and polite conversation were over, he sat down to one side and said to Ānanda: 

“Mister\marginnote{1.5.3} Ānanda, you were Mister Gotama’s attendant. You were close to him, living in his presence. You ought to know what things Mister Gotama praised, and in which he encouraged, settled, and grounded all these people. What were those things?” 

“Student,\marginnote{1.6.1} the Buddha praised three spectrums of practice, and that’s what he encouraged, settled, and grounded all these people in. What three? The entire spectrum of noble ethics, immersion, and wisdom.\footnote{What follows has much the same content as \href{https://suttacentral.net/dn2/en/sujato}{DN 2}, but arranged under these three heads rather than as successively refined happiness. } These are the three  spectrums of practice that the Buddha praised.” 

\section*{1. The Entire Spectrum of Ethics }

“But\marginnote{1.6.6} what was that entire spectrum of noble ethics that the Buddha praised?” 

“Student,\marginnote{1.7.1} it’s when a Realized One arises in the world, perfected, a fully awakened Buddha, accomplished in knowledge and conduct, holy, knower of the world, supreme guide for those who wish to train, teacher of gods and humans, awakened, blessed. He has realized with his own insight this world—with its gods, \textsanskrit{Māras}, and divinities, this population with its ascetics and brahmins, gods and humans—and he makes it known to others. He proclaims a teaching that is good in the beginning, good in the middle, and good in the end, meaningful and well-phrased. And he reveals a spiritual practice that’s entirely full and pure. A householder hears that teaching, or a householder’s child, or someone reborn in a good family. They gain faith in the Realized One and reflect: ‘Life at home is cramped and dirty, life gone forth is wide open. It’s not easy for someone living at home to lead the spiritual life utterly full and pure, like a polished shell. Why don’t I shave off my hair and beard, dress in ocher robes, and go forth from the lay life to homelessness?’ After some time they give up a large or small fortune, and a large or small family circle. They shave off hair and beard, dress in ocher robes, and go forth from the lay life to homelessness. Once they’ve gone forth, they live restrained in the monastic code, conducting themselves well and resorting for alms in suitable places. Seeing danger in the slightest fault, they keep the rules they’ve undertaken. They act skillfully by body and speech. They’re purified in livelihood and accomplished in ethical conduct. They guard the sense doors, have mindfulness and situational awareness, and are content. 

And\marginnote{1.11.1} how is a mendicant accomplished in ethics? It’s when a mendicant gives up killing living creatures. They renounce the rod and the sword. They’re scrupulous and kind, living full of sympathy for all living beings. … 

This\marginnote{1.12.1{-}1.27} pertains to their ethics. 

There\marginnote{1.28.1} are some ascetics and brahmins who, while enjoying food given in faith, still earn a living by low lore, by wrong livelihood. This includes rites for propitiation, for granting wishes, for ghosts, for the earth, for rain, for property settlement, and for preparing and consecrating house sites, and rites involving rinsing and bathing, and oblations. It also includes administering emetics, purgatives, expectorants, and phlegmagogues; administering ear-oils, eye restoratives, nasal medicine, ointments, and counter-ointments; surgery with needle and scalpel, treating children, prescribing root medicines, and binding on herbs. They refrain from such low lore, such wrong livelihood. … This pertains to their ethics. 

A\marginnote{1.29.1} mendicant thus accomplished in ethics sees no danger in any quarter in regards to their ethical restraint. It’s like a king who has defeated his enemies. He sees no danger from his foes in any quarter. A mendicant thus accomplished in ethics sees no danger in any quarter in regards to their ethical restraint. When they have this entire spectrum of noble ethics, they experience a blameless happiness inside themselves. That’s how a mendicant is accomplished in ethics. 

This\marginnote{1.30.1} is that entire spectrum of noble ethics that the Buddha praised. But there is still more to be done.” 

“It’s\marginnote{1.30.3} incredible, Mister Ānanda, it’s amazing, This entire spectrum of noble ethics is complete, not lacking anything! Such a complete spectrum of ethics cannot be seen among the other ascetics and brahmins. Were other ascetics and brahmins to see such an entire spectrum of noble ethics in themselves, they’d be delighted with just that much: ‘This is sufficient; enough has been done. We’ve reached the goal of our ascetic life. There is nothing more to be done.’ And yet you say: ‘But there is still more to be done.’ 

\section*{2. The Spectrum of Immersion }

But\marginnote{2.1.1} what, Mister Ānanda, was that noble spectrum of immersion that the Buddha praised?”\footnote{While \textit{\textsanskrit{samādhi}} proper is the deep immersion in meditation, here it is a category that pertains to developing such states. } 

“And\marginnote{2.2.1} how, student, does a mendicant guard the sense doors? When a mendicant sees a sight with their eyes, they don’t get caught up in the features and details. If the faculty of sight were left unrestrained, bad unskillful qualities of covetousness and displeasure would become overwhelming. For this reason, they practice restraint, protecting the faculty of sight, and achieving its restraint. When they hear a sound with their ears … When they smell an odor with their nose … When they taste a flavor with their tongue … When they feel a touch with their body … When they know an idea with their mind, they don’t get caught up in the features and details. If the faculty of mind were left unrestrained, bad unskillful qualities of covetousness and displeasure would become overwhelming. For this reason, they practice restraint, protecting the faculty of mind, and achieving its restraint. When they have this noble sense restraint, they experience an unsullied bliss inside themselves. That’s how a mendicant guards the sense doors. 

And\marginnote{2.3.1} how does a mendicant have mindfulness and situational awareness? It’s when a mendicant acts with situational awareness when going out and coming back; when looking ahead and aside; when bending and extending the limbs; when bearing the outer robe, bowl and robes; when eating, drinking, chewing, and tasting; when urinating and defecating; when walking, standing, sitting, sleeping, waking, speaking, and keeping silent. That’s how a mendicant has mindfulness and situational awareness. 

And\marginnote{2.4.1} how is a mendicant content? It’s when a mendicant is content with robes to look after the body and almsfood to look after the belly. Wherever they go, they set out taking only these things. They’re like a bird: wherever it flies, wings are its only burden. In the same way, a mendicant is content with robes to look after the body and almsfood to look after the belly. Wherever they go, they set out taking only these things. That’s how a mendicant is content. 

When\marginnote{2.5.1} they have this entire spectrum of noble ethics, this noble sense restraint, this noble mindfulness and situational awareness, and this noble contentment, they frequent a secluded lodging—a wilderness, the root of a tree, a hill, a ravine, a mountain cave, a charnel ground, a forest, the open air, a heap of straw. After the meal, they return from almsround, sit down cross-legged, set their body straight, and establish mindfulness in their presence. 

Giving\marginnote{2.6.1} up covetousness for the world, they meditate with a heart rid of covetousness, cleansing the mind of covetousness. Giving up ill will and malevolence, they meditate with a mind rid of ill will, full of sympathy for all living beings, cleansing the mind of ill will. Giving up dullness and drowsiness, they meditate with a mind rid of dullness and drowsiness, perceiving light, mindful and aware, cleansing the mind of dullness and drowsiness. Giving up restlessness and remorse, they meditate without restlessness, their mind peaceful inside, cleansing the mind of restlessness and remorse. Giving up doubt, they meditate having gone beyond doubt, not undecided about skillful qualities, cleansing the mind of doubt. 

Suppose\marginnote{2.7.1} a man who has gotten into debt were to apply himself to work, and his efforts proved successful. He would pay off the original loan and have enough left over to support his partner. Thinking about this, he’d be filled with joy and happiness. 

Suppose\marginnote{2.8.1} there was a person who was sick, suffering, gravely ill. They’d lose their appetite and get physically weak. But after some time they’d recover from that illness, and regain their appetite and their strength. Thinking about this, they’d be filled with joy and happiness. 

Suppose\marginnote{2.9.1} a person was imprisoned in a jail. But after some time they were released from jail, safe and sound, with no loss of wealth. Thinking about this, they’d be filled with joy and happiness. 

Suppose\marginnote{2.10.1} a person was a bondservant. They would not be their own master, but indentured to another, unable to go where they wish. But after some time they’d be freed from servitude. They would be their own master, not indentured to another, an emancipated individual able to go where they wish. Thinking about this, they’d be filled with joy and happiness. 

Suppose\marginnote{2.11.1} there was a person with wealth and property who was traveling along a desert road, which was perilous, with nothing to eat. But after some time they crossed over the desert safely, arriving within a village, a sanctuary free of peril. Thinking about this, they’d be filled with joy and happiness. 

In\marginnote{2.12.1} the same way, as long as these five hindrances are not given up inside themselves, a mendicant regards them as a debt, a disease, a prison, slavery, and a desert crossing. 

But\marginnote{2.12.2} when these five hindrances are given up inside themselves, a mendicant regards this as freedom from debt, good health, release from prison, emancipation, and a place of sanctuary at last. 

Seeing\marginnote{2.12.4} that the hindrances have been given up in them, joy springs up. Being joyful, rapture springs up. When the mind is full of rapture, the body becomes tranquil. When the body is tranquil, they feel bliss. And when blissful, the mind becomes immersed. 

Quite\marginnote{2.13.1} secluded from sensual pleasures, secluded from unskillful qualities, they enter and remain in the first absorption, which has the rapture and bliss born of seclusion, while placing the mind and keeping it connected. They drench, steep, fill, and spread their body with rapture and bliss born of seclusion. There’s no part of the body that’s not spread with rapture and bliss born of seclusion. 

It’s\marginnote{2.14.1} like when a deft bathroom attendant or their apprentice pours bath powder into a bronze dish, sprinkling it little by little with water. They knead it until the ball of bath powder is soaked and saturated with moisture, spread through inside and out; yet no moisture oozes out. 

In\marginnote{2.14.2} the same way, a mendicant drenches, steeps, fills, and spreads their body with rapture and bliss born of seclusion. There’s no part of the body that’s not spread with rapture and bliss born of seclusion. This pertains to their immersion. 

Furthermore,\marginnote{2.15.1} as the placing of the mind and keeping it connected are stilled, a mendicant enters and remains in the second absorption, which has the rapture and bliss born of immersion, with internal clarity and mind at one, without placing the mind and keeping it connected. They drench, steep, fill, and spread their body with rapture and bliss born of immersion. There’s no part of the body that’s not spread with rapture and bliss born of immersion. 

It’s\marginnote{2.16.1} like a deep lake fed by spring water. There’s no inlet to the east, west, north, or south, and the heavens would not properly bestow showers from time to time. But the stream of cool water welling up in the lake drenches, steeps, fills, and spreads throughout the lake. There’s no part of the lake that’s not spread through with cool water. 

In\marginnote{2.16.2} the same way, a mendicant drenches, steeps, fills, and spreads their body with rapture and bliss born of immersion. There’s no part of the body that’s not spread with rapture and bliss born of immersion. This pertains to their immersion. 

Furthermore,\marginnote{2.17.1} with the fading away of rapture, a mendicant enters and remains in the third absorption, where they meditate with equanimity, mindful and aware, personally experiencing the bliss of which the noble ones declare, ‘Equanimous and mindful, one meditates in bliss.’ They drench, steep, fill, and spread their body with bliss free of rapture. There’s no part of the body that’s not spread with bliss free of rapture. 

It’s\marginnote{2.17.3} like a pool with blue water lilies, or pink or white lotuses. Some of them sprout and grow in the water without rising above it, thriving underwater. From the tip to the root they’re drenched, steeped, filled, and soaked with cool water. There’s no part of them that’s not soaked with cool water. 

In\marginnote{2.17.4} the same way, a mendicant drenches, steeps, fills, and spreads their body with bliss free of rapture. There’s no part of the body that’s not spread with bliss free of rapture. This pertains to their immersion. 

Furthermore,\marginnote{2.18.1} giving up pleasure and pain, and ending former happiness and sadness, a mendicant enters and remains in the fourth absorption, without pleasure or pain, with pure equanimity and mindfulness. They sit spreading their body through with pure bright mind. There’s no part of the body that’s not spread with pure bright mind. 

It’s\marginnote{2.18.4} like someone sitting wrapped from head to foot with white cloth. There’s no part of the body that’s not spread over with white cloth. 

In\marginnote{2.18.5} the same way, a mendicant sits spreading their body through with pure bright mind. There's no part of their body that's not spread with pure bright mind. This pertains to their immersion. 

This\marginnote{2.19.1} is that noble spectrum of immersion that the Buddha praised. But there is still more to be done.” 

“It’s\marginnote{2.19.3} incredible, Mister Ānanda, it’s amazing! This noble spectrum of immersion is complete, not lacking anything! Such a complete spectrum of immersion cannot be seen among the other ascetics and brahmins. Were other ascetics and brahmins to see such a complete spectrum of noble immersion in themselves, they’d be delighted with just that much: ‘This is sufficient; enough has been done. We’ve reached the goal of our ascetic life. There is nothing more to be done.’ And yet you say: ‘But there is still more to be done.’ 

\section*{3. The Spectrum of Wisdom }

But\marginnote{2.20.1} what, Mister Ānanda, was that spectrum of noble wisdom that the Buddha praised?” 

“When\marginnote{2.21.1} their mind has become immersed in \textsanskrit{samādhi} like this—purified, bright, flawless, rid of corruptions, pliable, workable, steady, and imperturbable—they project it and extend it toward knowledge and vision. They understand: ‘This body of mine is formed. It’s made up of the four principal states, produced by mother and father, built up from rice and porridge, liable to impermanence, to wearing away and erosion, to breaking up and destruction. And this consciousness of mine is attached to it, tied to it.’ 

Suppose\marginnote{2.22.1} there was a beryl gem that was naturally beautiful, eight-faceted, well-worked, transparent, clear, and unclouded, endowed with all good qualities. And it was strung with a thread of blue, yellow, red, white, or golden brown. And someone with clear eyes were to take it in their hand and examine it: ‘This beryl gem is naturally beautiful, eight-faceted, well-worked, transparent, clear, and unclouded, endowed with all good qualities. And it’s strung with a thread of blue, yellow, red, white, or golden brown.’ 

In\marginnote{2.22.3} the same way, when their mind has become immersed in \textsanskrit{samādhi} like this—purified, bright, flawless, rid of corruptions, pliable, workable, steady, and imperturbable—they project it and extend it toward knowledge and vision. This pertains to their wisdom. 

When\marginnote{2.23.1} their mind has become immersed in \textsanskrit{samādhi} like this—purified, bright, flawless, rid of corruptions, pliable, workable, steady, and imperturbable—they project it and extend it toward the creation of a mind-made body. From this body they create another body—formed, mind-made, whole in its major and minor limbs, not deficient in any faculty. 

Suppose\marginnote{2.24.1} a person was to draw a reed out from its sheath. They’d think: ‘This is the reed, this is the sheath. The reed and the sheath are different things. The reed has been drawn out from the sheath.’ Or suppose a person was to draw a sword out from its scabbard. They’d think: ‘This is the sword, this is the scabbard. The sword and the scabbard are different things. The sword has been drawn out from the scabbard.’ Or suppose a person was to draw a snake out from its slough. They’d think: ‘This is the snake, this is the slough. The snake and the slough are different things. The snake has been drawn out from the slough.’ 

In\marginnote{2.24.10} the same way, when their mind has become immersed in \textsanskrit{samādhi} like this—purified, bright, flawless, rid of corruptions, pliable, workable, steady, and imperturbable—they project it and extend it toward the creation of a mind-made body. This pertains to their wisdom. 

When\marginnote{2.25.1} their mind has become immersed in \textsanskrit{samādhi} like this—purified, bright, flawless, rid of corruptions, pliable, workable, steady, and imperturbable—they project it and extend it toward psychic power. They wield the many kinds of psychic power: multiplying themselves and becoming one again; appearing and disappearing; going unobstructed through a wall, a rampart, or a mountain as if through space; diving in and out of the earth as if it were water; walking on water as if it were earth; flying cross-legged through the sky like a bird; touching and stroking with the hand the sun and moon, so mighty and powerful; controlling the body as far as the realm of divinity. 

Suppose\marginnote{2.26.1} a deft potter or their apprentice had some well-prepared clay. They could produce any kind of pot that they like. Or suppose a deft ivory-carver or their apprentice had some well-prepared ivory. They could produce any kind of ivory item that they like. Or suppose a deft goldsmith or their apprentice had some well-prepared gold. They could produce any kind of gold item that they like. 

In\marginnote{2.26.4} the same way, when their mind has become immersed in \textsanskrit{samādhi} like this—purified, bright, flawless, rid of corruptions, pliable, workable, steady, and imperturbable—they project it and extend it toward psychic power. This pertains to their wisdom. 

When\marginnote{2.27.1} their mind has become immersed in \textsanskrit{samādhi} like this—purified, bright, flawless, rid of corruptions, pliable, workable, steady, and imperturbable—they project it and extend it toward clairaudience. With clairaudience that is purified and superhuman, they hear both kinds of sounds, human and heavenly, whether near or far. Suppose there was a person traveling along the road. They’d hear the sound of drums, clay drums, horns, kettledrums, and tom-toms. They’d think: ‘That’s the sound of drums,’ and ‘that’s the sound of clay drums,’ and ‘that’s the sound of horns, kettledrums, and tom-toms.’ 

In\marginnote{2.27.4} the same way, when their mind has become immersed in \textsanskrit{samādhi} like this—purified, bright, flawless, rid of corruptions, pliable, workable, steady, and imperturbable—they project it and extend it toward clairaudience. This pertains to their wisdom. 

When\marginnote{2.28.1} their mind has become immersed in \textsanskrit{samādhi} like this—purified, bright, flawless, rid of corruptions, pliable, workable, steady, and imperturbable—they project it and extend it toward comprehending the minds of others. They understand mind with greed as ‘mind with greed’, and mind without greed as ‘mind without greed’. They understand mind with hate … mind without hate … mind with delusion … mind without delusion … constricted mind … scattered mind … expansive mind … unexpansive mind … mind that is not supreme … mind that is supreme … immersed mind … unimmersed mind … freed mind … They understand unfreed mind as ‘unfreed mind’. 

Suppose\marginnote{2.29.1} there was a woman or man who was young, youthful, and fond of adornments, and they check their own reflection in a clean bright mirror or a clear bowl of water. If they had a spot they’d know ‘I have a spot,’ and if they had no spots they’d know ‘I have no spots.’ 

In\marginnote{2.29.2} the same way, when their mind has become immersed in \textsanskrit{samādhi} like this—purified, bright, flawless, rid of corruptions, pliable, workable, steady, and imperturbable—they project it and extend it toward comprehending the minds of others. This pertains to their wisdom. 

When\marginnote{2.30.1} their mind has become immersed in \textsanskrit{samādhi} like this—purified, bright, flawless, rid of corruptions, pliable, workable, steady, and imperturbable—they project it and extend it toward recollection of past lives. They recollect many kinds of past lives, that is, one, two, three, four, five, ten, twenty, thirty, forty, fifty, a hundred, a thousand, a hundred thousand rebirths; many eons of the world contracting, many eons of the world expanding, many eons of the world contracting and expanding. They remember: ‘There, I was named this, my clan was that, I looked like this, and that was my food. This was how I felt pleasure and pain, and that was how my life ended. When I passed away from that place I was reborn somewhere else. There, too, I was named this, my clan was that, I looked like this, and that was my food. This was how I felt pleasure and pain, and that was how my life ended. When I passed away from that place I was reborn here.’ And so they recollect their many kinds of past lives, with features and details. 

Suppose\marginnote{2.31.1} a person was to leave their home village and go to another village. From that village they’d go to yet another village. And from that village they’d return to their home village. They’d think: ‘I went from my home village to another village. There I stood like this, sat like that, spoke like this, or kept silent like that. From that village I went to yet another village. There too I stood like this, sat like that, spoke like this, or kept silent like that. And from that village I returned to my home village.’ 

In\marginnote{2.31.2} the same way, when their mind has become immersed in \textsanskrit{samādhi} like this—purified, bright, flawless, rid of corruptions, pliable, workable, steady, and imperturbable—they project it and extend it toward recollection of past lives. This pertains to their wisdom. 

When\marginnote{2.32.1} their mind has become immersed in \textsanskrit{samādhi} like this—purified, bright, flawless, rid of corruptions, pliable, workable, steady, and imperturbable—they project it and extend it toward knowledge of the death and rebirth of sentient beings. With clairvoyance that is purified and superhuman, they see sentient beings passing away and being reborn—inferior and superior, beautiful and ugly, in a good place or a bad place. They understand how sentient beings are reborn according to their deeds. ‘These dear beings did bad things by way of body, speech, and mind. They denounced the noble ones; they had wrong view; and they chose to act out of that wrong view. When their body breaks up, after death, they’re reborn in a place of loss, a bad place, the underworld, hell. These dear beings, however, did good things by way of body, speech, and mind. They never denounced the noble ones; they had right view; and they chose to act out of that right view. When their body breaks up, after death, they’re reborn in a good place, a heavenly realm.’ And so, with clairvoyance that is purified and superhuman, they see sentient beings passing away and being reborn—inferior and superior, beautiful and ugly, in a good place or a bad place. They understand how sentient beings are reborn according to their deeds. 

Suppose\marginnote{2.33.1} there was a stilt longhouse at the central square. A person with clear eyes standing there might see people entering and leaving a house, walking along the streets and paths, and sitting at the central square. They’d think: ‘These are people entering and leaving a house, walking along the streets and paths, and sitting at the central square.’ 

In\marginnote{2.33.2} the same way, when their mind has become immersed in \textsanskrit{samādhi} like this—purified, bright, flawless, rid of corruptions, pliable, workable, steady, and imperturbable—they project and extend it toward knowledge of the death and rebirth of sentient beings. This pertains to their wisdom. 

When\marginnote{2.34.1} their mind has become immersed in \textsanskrit{samādhi} like this—purified, bright, flawless, rid of corruptions, pliable, workable, steady, and imperturbable—they project it and extend it toward knowledge of the ending of defilements. They truly understand: ‘This is suffering’ … ‘This is the origin of suffering’ … ‘This is the cessation of suffering’ … ‘This is the practice that leads to the cessation of suffering’. They truly understand: ‘These are defilements’ … ‘This is the origin of defilements’ … ‘This is the cessation of defilements’ … ‘This is the practice that leads to the cessation of defilements’. Knowing and seeing like this, their mind is freed from the defilements of sensuality, desire to be reborn, and ignorance. When they’re freed, they know they’re freed. 

They\marginnote{2.35.3} understand: ‘Rebirth is ended, the spiritual journey has been completed, what had to be done has been done, there is nothing further for this place.’ 

Suppose\marginnote{2.36.1} that in a mountain glen there was a lake that was transparent, clear, and unclouded. A person with clear eyes standing on the bank would see the clams and mussels, and pebbles and gravel, and schools of fish swimming about or staying still. They’d think: ‘This lake is transparent, clear, and unclouded. And here are the clams and mussels, and pebbles and gravel, and schools of fish swimming about or staying still.’ 

In\marginnote{2.36.2} the same way, when their mind has become immersed in \textsanskrit{samādhi} like this—purified, bright, flawless, rid of corruptions, pliable, workable, steady, and imperturbable—they project it and extend it toward knowledge of the ending of defilements. This pertains to their wisdom. 

This\marginnote{2.37.1} is that spectrum of noble wisdom that the Buddha praised. And there is nothing more to be done.” 

“It’s\marginnote{2.37.3} incredible, Mister Ānanda, it’s amazing! This noble spectrum of wisdom is complete, not lacking anything! Such a complete spectrum of wisdom cannot be seen among the other ascetics and brahmins. And there is nothing more to be done. Excellent, Mister Ānanda! Excellent! As if he were righting the overturned, or revealing the hidden, or pointing out the path to the lost, or lighting a lamp in the dark so people with clear eyes can see what’s there, Mister Ānanda has made the teaching clear in many ways. I go for refuge to Mister Gotama, to the teaching, and to the mendicant \textsanskrit{Saṅgha}.\footnote{He had already gone to the Buddha for refuge in \href{https://suttacentral.net/mn99/en/sujato\#28.4}{MN 99:28.4} and \href{https://suttacentral.net/mn135/en/sujato\#21.4}{MN 135:21.4}. } From this day forth, may Mister Ānanda remember me as a lay follower who has gone for refuge for life.” 

%
\chapter*{{\suttatitleacronym DN 11}{\suttatitletranslation With Kevaḍḍha }{\suttatitleroot Kevaṭṭasutta}}
\addcontentsline{toc}{chapter}{\tocacronym{DN 11} \toctranslation{With Kevaḍḍha } \tocroot{Kevaṭṭasutta}}
\markboth{With Kevaḍḍha }{Kevaṭṭasutta}
\extramarks{DN 11}{DN 11}

\scevam{So\marginnote{1.1} I have heard. }At one time the Buddha was staying near \textsanskrit{Nāḷandā} in \textsanskrit{Pāvārika}’s mango grove.\footnote{This was the scene for some controversial discussions with Jains (\href{https://suttacentral.net/mn56/en/sujato}{MN 56}, \href{https://suttacentral.net/sn42.8/en/sujato}{SN 42.8}), and \textsanskrit{Sāriputta}’s touching declaration of faith shortly before his passing (\href{https://suttacentral.net/sn47.12/en/sujato}{SN 47.12}, \href{https://suttacentral.net/dn16/en/sujato\#1.16.1}{DN 16:1.16.1}, \href{https://suttacentral.net/dn28/en/sujato}{DN 28}). It is probably the \textsanskrit{Pāvā} (modern Pawapuri) at which \textsanskrit{Mahāvīra} died according to the Jain tradition. } 

Then\marginnote{1.3} the householder \textsanskrit{Kevaḍḍha} went up to the Buddha, bowed, sat down to one side, and said to him,\footnote{\textsanskrit{Kevaḍḍha} is mentioned only here. Manuscripts spell his name variously as Kevaddha or \textsanskrit{Kevaṭṭa} (“fisherman”). The Chinese form \langlzh{堅固} means “sturdy” (from \textit{\textsanskrit{dṛḍha}}) and thus supports \textsanskrit{Kevaḍḍha}. } “Sir, this \textsanskrit{Nāḷandā} is successful and prosperous, populous, full of people. Please direct a mendicant to perform a superhuman demonstration of psychic power.\footnote{I have been asked to do the same thing for the same reason. } Then \textsanskrit{Nāḷandā} will become even more devoted to the Buddha!” 

When\marginnote{1.7} he said this, the Buddha said, “\textsanskrit{Kevaḍḍha}, I do not teach Dhamma to the mendicants like this: ‘Come now, mendicants, perform a superhuman demonstration of psychic power for the white-clothed laypeople.’”\footnote{In fact it is forbidden in \href{https://suttacentral.net/pli-tv-kd15/en/sujato\#8.2.23}{Kd 15:8.2.23}. } 

For\marginnote{2.1} a second time, \textsanskrit{Kevaḍḍha} made the same request, “Sir, I am not teaching you the Dhamma,\footnote{The reading \textit{\textsanskrit{dhaṁsemi}} is dubious. An old Burmese manuscript has the reading \textit{\textsanskrit{dhammaṁ} desemi}, which echoes the Buddha just above. Note too that \textsanskrit{Kevaḍḍha} urges the Buddha to “direct” the monks (\textit{\textsanskrit{samādisatu}}, from the same root as \textit{desemi}). I think the tension is deliberate: \textsanskrit{Kevaḍḍha} says he isn’t telling the Buddha how to teach, but he absolutely is. Other readings convey the sense “attack, insult”, but this seems out of place. } but nonetheless I say: ‘Sir, this \textsanskrit{Nāḷandā} is successful and prosperous, populous, full of people. Please direct a mendicant to perform a superhuman demonstration of psychic power. Then \textsanskrit{Nāḷandā} will become even more devoted to the Buddha!’” But for a second time, the Buddha gave the same answer. 

For\marginnote{3.1} a third time, \textsanskrit{Kevaḍḍha} made the same request, at which the Buddha said the following. 

\section*{1. The Demonstration of Psychic Power }

“\textsanskrit{Kevaḍḍha},\marginnote{3.8} there are three kinds of demonstration, which I declare having realized them with my own insight.\footnote{As noted previously, the basic sense of \textit{\textsanskrit{pāṭihāriya}} is “demonstration”, and as the context here shows, it may or may not involve a “demonstration of wonders” i.e. a “miracle”. } What three? The demonstration of psychic power, the demonstration of revealing, and the demonstration of instruction.\footnote{These three are mentioned frequently in the suttas. Only the last is endorsed by the Buddha, as it leads to genuine growth. } 

And\marginnote{4.1} what is the demonstration of psychic power? It’s a mendicant who wields the many kinds of psychic power: multiplying themselves and becoming one again; appearing and disappearing; going unobstructed through a wall, a rampart, or a mountain as if through space; diving in and out of the earth as if it were water; walking on water as if it were earth; flying cross-legged through the sky like a bird; touching and stroking with the hand the sun and moon, so mighty and powerful; controlling the body as far as the realm of divinity. 

Someone\marginnote{4.3} with faith and confidence sees that mendicant performing those superhuman feats. 

They\marginnote{4.4} tell someone else who lacks faith and confidence: ‘Oh, how incredible, how amazing! The ascetic has such psychic power and might!\footnote{Their priors have been confirmed. } I saw him myself, performing all these superhuman feats!’ 

But\marginnote{5.1} the one lacking faith and confidence would say to them:\footnote{Note how a skeptical mindset sticks closer to the truth. } ‘There’s a spell named \textsanskrit{Gandhārī}.\footnote{“Spell” is \textit{\textsanskrit{vijjā}} (Sanskrit \textit{\textsanskrit{vidyā}}, “(potent) knowledge”, cf. English “wicca”, “wizard”, “witch”).  The commentary says it was practiced by the seers of \textsanskrit{Gandhāra} (north-west Pakistan). Jain tradition also knows a \textsanskrit{Gandhārī} mantra, but attribute it to certain \textit{\textsanskrit{vidyādhara}} deities. Sanskrit tradition similarly knows of a \textit{\textsanskrit{vidyādevī}} (“lore goddess”) named \textsanskrit{Gandhārī}. \textsanskrit{Gandhāra} was an ancient land of learning, and a convenient location for exotic magics. } Using that a mendicant can perform such superhuman feats.’ 

What\marginnote{5.4} do you think, \textsanskrit{Kevaḍḍha}? Wouldn’t someone lacking faith speak like that?” 

“They\marginnote{5.6} would, sir.” 

“Seeing\marginnote{5.7} this drawback in psychic power, I’m horrified, repelled, and disgusted by demonstrations of psychic power.\footnote{These feats that have nothing to do with spiritual growth, hence they may be produced or perhaps faked by a variety of means. } 

\section*{2. The Demonstration of Revealing }

And\marginnote{6.1} what is the demonstration of revealing?\footnote{“Revealing” is \textit{\textsanskrit{ādesana}}, from root \textit{dis} “to indicate, show, or point”. } It’s when a mendicant reveals the mind, mentality, thoughts, and reflections of other beings and individuals:\footnote{The Pali terms here are \textit{citta}, \textit{cetasika}, \textit{vitakka}, and \textit{\textsanskrit{vicāra}}. } ‘This is what you’re thinking, such is your thought, and thus is your state of mind.’\footnote{Here we have \textit{mano} (twice) and \textit{citta}. } 

Someone\marginnote{6.4} with faith and confidence sees that mendicant revealing another person’s thoughts. They tell someone else who lacks faith and confidence: ‘Oh, how incredible, how amazing! The ascetic has such psychic power and might! I saw him myself, revealing the thoughts of another person!’ 

But\marginnote{7.1} the one lacking faith and confidence would say to them: ‘There’s a spell named \textsanskrit{Māṇikā}.\footnote{From \textit{\textsanskrit{maṇi}}, “gem”. Magical gems are a common feature of Indian storytelling. Buddhist stories often feature the “wish-granting gem” (\textit{\textsanskrit{cintāmaṇi}}), which according to the commentary is meant here. } Using that a mendicant can reveal another person’s thoughts.’ 

What\marginnote{7.5} do you think, \textsanskrit{Kevaḍḍha}? Wouldn’t someone lacking faith speak like that?” 

“They\marginnote{7.7} would, sir.” 

“Seeing\marginnote{7.8} this drawback in revealing, I’m horrified, repelled, and disgusted by demonstrations of revealing. 

\section*{3. The Demonstration of Instruction }

And\marginnote{8.1} what is the demonstration of instruction? It’s when a mendicant instructs others like this: ‘Think like this, not like that. Focus your mind like this, not like that. Give up this, and live having achieved that.’ This is called the demonstration of instruction. 

Furthermore,\marginnote{9{-}66.1} a Realized One arises in the world … That’s how a mendicant is accomplished in ethics. … They enter and remain in the first absorption … This is called the demonstration of instruction. 

They\marginnote{9{-}66.5} enter and remain in the second absorption … third absorption … fourth absorption. This too is called the demonstration of instruction. 

They\marginnote{9{-}66.9} project and extend the mind toward knowledge and vision … This too is called the demonstration of instruction. 

They\marginnote{9{-}66.11} understand: ‘… there is nothing further for this place.’ This too is called the demonstration of instruction. 

These,\marginnote{67.1} \textsanskrit{Kevaḍḍha}, are the three kinds of demonstration, which I declare having realized them with my own insight. 

\section*{4. On the Mendicant in Search of the Cessation of Being }

Once\marginnote{67.3} upon a time, \textsanskrit{Kevaḍḍha}, a mendicant in this very \textsanskrit{Saṅgha} had the following thought,\footnote{This story is presented as an actual event, but is phrased like a fable. } ‘Where do these four principal states cease without anything left over, namely, the elements of earth, water, fire, and air?’\footnote{The question is about meditation, not the annihilation of the material world. The first four \textit{\textsanskrit{jhānas}} are based on the “subtle form” (\textit{\textsanskrit{sukhumarūpa}}) that manifests as light in deep meditation. He is asking how to go beyond this to the formless attainments. } 

Then\marginnote{68.1} that mendicant attained a state of immersion such that a path to the gods appeared.\footnote{The mendicant has already well developed the \textit{\textsanskrit{jhānas}}. | “Controlling the body as far as the \textsanskrit{Brahmā} realm” is one of the “demonstrations of psychic power” listed above. } Then he approached the gods of the four great kings and said,\footnote{The “gods of the Four Great Kings” are deities born in a realm subject to the overlords known as the Four Great Kings. These deities inhabit the lowest of the celestial realms. } ‘Reverends, where do these four principal states cease without anything left over, namely, the elements of earth, water, fire, and air?’ 

When\marginnote{68.4} he said this, those gods said to him, ‘Mendicant, we too do not know this. But the four great kings are our superiors.\footnote{These are powerful spirits who guard the four quarters. In \href{https://suttacentral.net/an8.36/en/sujato}{AN 8.36} it is explained that they, like the other leading gods mentioned below, achieved their station due to their greater generosity and morality. } They might know.’ 

Then\marginnote{69.1} he approached the four great kings and asked the same question. But they also said to him, ‘Mendicant, we too do not know this. But the gods of the thirty-three …\footnote{The “thirty-three” enjoy refined sensual delights. The number is a reduplication of the trinity. In Buddhist texts they are not enumerated, but \textsanskrit{Yājñavalkya} reckons them as eight Vasus, eleven Rudras, twelve Ādityas, plus Indra and \textsanskrit{Prajāpati} (\textsanskrit{Bṛhadāraṇyaka} \textsanskrit{Upaniṣad} 3.9.2). The final pair are elsewhere said to be Dyaus (“Heaven” = Zeus) and \textsanskrit{Pṛthivī} (“Earth”), or the twin \textsanskrit{Aśvins}. } Sakka, lord of gods …\footnote{Conventionally known as “lord of gods”, but in fact the ruler only of the relatively lowly realm of the thirty-three. He is Vedic Indra, heroic slayer of the dragon \textsanskrit{Vṛtra}, and is the most personally known god in the Pali Canon. } the gods of Yama …\footnote{Gods in this realm (spelled \textit{\textsanskrit{yāma}}, “of Yama”) are subjects of the god of the dead, Yama. } the god named \textsanskrit{Suyāma} … the Joyful gods … the god named Santussita … the gods who love to imagine … the god named Sunimmita … the gods who control the creation of others … the god named \textsanskrit{Vasavattī} … the gods of the Divinity’s host are our superiors. They might know.’ 

Then\marginnote{80.1} that mendicant attained a state of immersion such that a path to divinity appeared. Then he approached the gods of the Divinity’s host and said, ‘Reverends, where do these four principal states cease without anything left over, namely, the elements of earth, water, fire, and air?’ But they also said to him, ‘Mendicant, we too do not know this.\footnote{The previous deities achieved their station by mere morality and generosity, not by \textit{\textsanskrit{jhāna}}. The gods of \textsanskrit{Brahmā}’s Host practiced the first \textit{\textsanskrit{jhāna}}, but they do not know what lies beyond. } But there is the Divinity, the Great Divinity, the Vanquisher, the Unvanquished, the Universal Seer, the Wielder of Power, God Almighty, the Maker, the Creator, the First, the Begetter, the Controller, the Father of those who have been born and those yet to be born. He is our superior.\footnote{The same passage appears in \href{https://suttacentral.net/dn1/en/sujato\#2.5.2}{DN 1:2.5.2}, where it also had a satirical tone, poking fun at the pomposity of religious titles. } He might know.’\footnote{Even \textsanskrit{Brahmā}’s community are not confident. } 

‘But\marginnote{80.10} reverends, where is that Divinity now?’ ‘We also don’t know where he is or what way he lies. But by the signs that are seen—light arising and radiance appearing—we know that Divinity will appear. For this is the precursor for the appearance of the Divinity, namely light arising and radiance appearing.’\footnote{This passage may be one of the sources for the later use of \textit{nimitta} to mean the appearance of light that signifies the approach of \textit{\textsanskrit{jhāna}}. } Not long afterwards, the Great Divinity appeared. 

Then\marginnote{81.2} that mendicant approached the Great Divinity and said to him, ‘Reverend, where do these four principal states cease without anything left over, namely, the elements of earth, water, fire, and air?’ The Great Divinity said to him, ‘I am the Divinity, the Great Divinity, the Vanquisher, the Unvanquished, the Universal Seer, the Wielder of Power, God Almighty, the Maker, the Creator, the First, the Begetter, the Controller, the Father of those who have been born and those yet to be born.’\footnote{He puffs his own chest, but like the ascetic teachers of \href{https://suttacentral.net/dn2/en/sujato}{DN 2}, he does not answer the question. } 

For\marginnote{82.1} a second time, that mendicant said to the Great Divinity, ‘Reverend, I am not asking you whether you are\footnote{He addresses \textsanskrit{Brahmā} with \textit{\textsanskrit{āvuso}}. This is often translated as “friend”, but the root is \textit{\textsanskrit{āyu}} (“age”) and it is respectful not familiar. } the Divinity, the Great Divinity, the Vanquisher, the Unvanquished, the Universal Seer, the Wielder of Power, God Almighty, the Maker, the Creator, the First, the Begetter, the Controller, the Father of those who have been born and those yet to be born. I am asking where these four principal states cease without anything left over.’ 

For\marginnote{82.6} a second time, the Great Divinity said to him, ‘I am the Divinity, the Great Divinity, the Vanquisher, the Unvanquished, the Universal Seer, the Wielder of Power, God Almighty, the Maker, the Creator, the First, the Begetter, the Controller, the Father of those who have been born and those yet to be born.’ For a third time, that mendicant said to the Great Divinity,\footnote{Rather than trying to engage with \textsanskrit{Brahmā}’s agenda, he keeps restating his question. This is a skillful way of curbing narcissism. } ‘Reverend, I am not asking you whether you are the Divinity, the Great Divinity, the Vanquisher, the Unvanquished, the Universal Seer, the Wielder of Power, God Almighty, the Maker, the Creator, the First, the Begetter, the Controller, the Father of those who have been born and those yet to be born. I am asking where these four principal states cease without anything left over.’ 

Then\marginnote{83.6} the Great Divinity took that mendicant by the arm, led him off to one side, and said to him,\footnote{\textsanskrit{Brahmā} is embarrassed to reveal his ignorance. Perhaps a satire of \textsanskrit{Bṛhadāraṇyaka} \textsanskrit{Upaniṣad} 3.2.13, where \textsanskrit{Yājñavalkya} takes \textsanskrit{Jāratkārava} \textsanskrit{Ārtabhāga} by the hand and leads him aside for a secret discussion. } ‘Mendicant, these gods think that there is nothing at all that I don’t know and see and understand and realize. That’s why I didn’t answer in front of them. But I too do not know where these four principal states cease with nothing left over.\footnote{At least he is honest about his lack of knowledge, even if not publicly. } Therefore, mendicant, the misdeed is yours alone, the mistake is yours alone, in that you passed over the Buddha and searched elsewhere for an answer to this question. Mendicant, go to the Buddha and ask him this question. You should remember it in line with his answer.’ 

Then\marginnote{84.1} that mendicant, as easily as a strong person would extend or contract their arm, vanished from the realm of divinity and reappeared in front of me. Then he bowed, sat down to one side, and said to me, ‘Sir, where do these four principal states cease without anything left over, namely, the elements of earth, water, fire, and air?’ 

\subsection*{4.1. The Simile of the Land-Spotting Bird }

When\marginnote{85.1} he said this, I said to him: 

‘Once\marginnote{85.2} upon a time, mendicant, some sea-merchants set sail for the ocean deeps, taking with them a land-spotting bird. When their ship was out of sight of land, they released the bird. It flew right away to the east, the west, the north, the south, upwards, and in-between. If it saw land on any side, it went there and stayed. But if it saw no land on any side it returned to the ship. 

In\marginnote{85.7} the same way, after failing to get an answer to this question even after searching as far as the realm of divinity, you’ve returned to me. Mendicant, this is not how the question should be asked: “Sir, where do these four principal states cease without anything left over, namely, the elements of earth, water, fire, and air?” 

This\marginnote{85.10} is how the question should be asked:\footnote{The following verses are difficult because they speak of a kind of consciousness at the start and the cessation of consciousness at the end. The simplest way to resolve this is to assume there are two distinct questions. } 

\begin{verse}%
“Where\marginnote{85.11} do water and earth,\footnote{This is a rephrasing of the original question, asking where the four “form” \textit{\textsanskrit{jhānas}} end. } \\
fire and air find no footing? \\
Where do long and short,\footnote{Here starts the second question, asking the deeper question of how all these things end. A similar list of descriptors elsewhere describes things that are not stolen (\href{https://suttacentral.net/snp3.9/en/sujato\#45.1}{Snp 3.9:45.1}) or the kinds of sentient beings (\href{https://suttacentral.net/snp1.8/en/sujato\#4.3}{Snp 1.8:4.3}). These are aspects of how “form” manifests in desirable or undesirable ways. } \\
fine and coarse, beautiful and ugly;\footnote{The first four terms in this verse are identical with the first four terms in \textsanskrit{Yājñavalkya}’s description of the immutable Brahman as “neither coarse nor fine, neither short nor long” at \textsanskrit{Bṛhadāraṇyaka} \textsanskrit{Upaniṣad} 3.8.8. } \\
where do name and form \\
cease with nothing left over?” 

%
\end{verse}

And\marginnote{85.17} the answer to that is: 

\begin{verse}%
“Consciousness\marginnote{85.18} where nothing appears,\footnote{“Where nothing appears” (\textit{\textsanskrit{anidassanaṁ}}) here is a synonym for “formless” (see eg. \href{https://suttacentral.net/mn21/en/sujato\#14.8}{MN 21:14.8}, “space is formless and invisible”, \textit{\textsanskrit{ākāso} \textsanskrit{arūpī} anidassano}). Normally the colors and images seen in the “form” absorptions are described as “visible” (eg. \href{https://suttacentral.net/dn16/en/sujato\#3.29.1}{DN 16:3.29.1}), so this indicates the formless attainments. } \\
infinite, luminous all-round—\footnote{“Infinite” (\textit{ananta}) is the direct qualifier of “consciousness”, but in the Pali it is shifted to the next line to fit the meter. It indicates the second of the formless attainments. \textsanskrit{Yājñavalkya} describes consciousness as infinite in the famous passage at \textsanskrit{Bṛhadāraṇyaka} \textsanskrit{Upaniṣad} 2.4.12. | \textit{\textsanskrit{Pabhaṁ}} means “luminous”, as with the deities that are “self-luminous” (\textit{\textsanskrit{sayaṁpabhā}}, \href{https://suttacentral.net/dn27/en/sujato\#10.3}{DN 27:10.3}). \textit{Sabbato \textsanskrit{pabhaṁ}} (“luminous all-round”) is synonymous with \textit{\textsanskrit{pariyodāta}} (“bright”, literally “white all over”), a stock descriptor of the mind of fourth \textit{\textsanskrit{jhāna}}, on which the formless states are based. } \\
that’s where water and earth, \\
fire and air find no footing.\footnote{I read these verses as broken into two statements. The first part, ending here, speaks of the formless attainments as “infinite consciousness”, agreeing with the highest of the Brahmanical meditative sages. The following verses go further to speak of the cessation of consciousness. } 

And\marginnote{85.22} that is where long and short, \\
fine and coarse, beautiful and ugly; \\
that’s where name and form \\
cease with nothing left over—\\
with the cessation of consciousness,\footnote{According to dependent origination, when consciousness ceases, name and form cease, and with it the manifestation of all things desirable and undesirable in the world. } \\
that’s where they cease.”’” 

%
\end{verse}

That\marginnote{85.28} is what the Buddha said. Satisfied, the householder \textsanskrit{Kevaḍḍha} approved what the Buddha said. 

%
\chapter*{{\suttatitleacronym DN 12}{\suttatitletranslation With Lohicca }{\suttatitleroot Lohiccasutta}}
\addcontentsline{toc}{chapter}{\tocacronym{DN 12} \toctranslation{With Lohicca } \tocroot{Lohiccasutta}}
\markboth{With Lohicca }{Lohiccasutta}
\extramarks{DN 12}{DN 12}

\scevam{So\marginnote{1.1} I have heard. }At one time the Buddha was wandering in the land of the Kosalans together with a large \textsanskrit{Saṅgha} of five hundred mendicants when he arrived at \textsanskrit{Sālavatikā}.\footnote{This is the only appearance of a place called \textsanskrit{Sālavatikā} (“Abounding in Sal Trees”). A courtesan of \textsanskrit{Rājagaha} named \textsanskrit{Sālavatī} appears in \href{https://suttacentral.net/pli-tv-kd8/en/sujato\#1.3.1}{Kd 8:1.3.1}; she might perhaps have come from there. } 

Now\marginnote{1.3} at that time the brahmin Lohicca was living in \textsanskrit{Sālavatikā}. It was a crown property given by King Pasenadi of Kosala, teeming with living creatures, full of hay, wood, water, and grain, a royal park endowed to a brahmin.\footnote{A certain brahmin Lohicca also appears in \href{https://suttacentral.net/sn35.132/en/sujato}{SN 35.132}. But since that is set far away in \textsanskrit{Avantī} after the Buddha’s passing, and since in both cases he is said to have taken refuge, it seems likely these are different people. } 

Now\marginnote{2.1} at that time Lohicca had the following harmful misconception: “Should an ascetic or brahmin achieve some skillful quality, they ought not inform anyone else. For what can one person do for another?\footnote{A similar view is sometimes unjustly imputed to the \textsanskrit{Theravādins}, that they are only interested in their own liberation. } Suppose someone cut off an old bond, only to create another new bond. That’s the consequence of such a wicked, greedy deed, I say. For what can one person do for another?” 

Lohicca\marginnote{3.1} heard: 

“It\marginnote{3.2} seems the ascetic Gotama—a Sakyan, gone forth from a Sakyan family—has arrived at \textsanskrit{Sālavatikā}, together with a large \textsanskrit{Saṅgha} of five hundred mendicants. He has this good reputation: ‘That Blessed One is perfected, a fully awakened Buddha, accomplished in knowledge and conduct, holy, knower of the world, supreme guide for those who wish to train, teacher of gods and humans, awakened, blessed.’ He has realized with his own insight this world—with its gods, \textsanskrit{Māras}, and divinities, this population with its ascetics and brahmins, gods and humans—and he makes it known to others. He proclaims a teaching that is good in the beginning, good in the middle, and good in the end, meaningful and well-phrased. And he reveals a spiritual practice that’s entirely full and pure. It’s good to see such perfected ones.” 

Then\marginnote{4.1} Lohicca addressed his barber Rosika,\footnote{While \textit{\textsanskrit{nhāpita}} would seem to be identical with \textit{\textsanskrit{nhāpaka}} (“bathroom attendant”), it is usually translated per \href{https://suttacentral.net/ja395/en/sujato}{Ja 395} where it means “barber”. } “Here, dear Rosika, go to the ascetic Gotama and in my name bow with your head to his feet. Ask him if he is healthy and well, nimble, strong, and living comfortably. And then ask him whether he, together with the mendicant \textsanskrit{Saṅgha}, might please accept tomorrow’s meal from the brahmin Lohicca.”\footnote{Note the unusual use of \textit{kira} in this idiom, found in \href{https://suttacentral.net/sn35.133/en/sujato\#2.3}{SN 35.133:2.3}, \href{https://suttacentral.net/mn85/en/sujato\#3.5}{MN 85:3.5}, \href{https://suttacentral.net/mn127/en/sujato\#2.5}{MN 127:2.5}, and \href{https://suttacentral.net/ud2.8/en/sujato\#6.8}{Ud 2.8:6.8}. I think this expresses polite deference, and render with “might” rather than “would”. } 

“Yes,\marginnote{5.1} sir,” Rosika replied. He did as he was asked, and the Buddha consented with silence. 

Then,\marginnote{6.1} knowing that the Buddha had consented, Rosika got up from his seat, went to Lohicca, and said to him, “I gave the Buddha your message, and he accepted.” 

And\marginnote{7.1} when the night had passed Lohicca had delicious fresh and cooked foods prepared in his own home. Then he had the Buddha informed of the time, saying, “Here, dear Rosika, go to the ascetic Gotama and announce the time, saying: ‘It’s time, Mister Gotama, the meal is ready.’” 

“Yes,\marginnote{7.4} sir,” Rosika replied. He did as he was asked. 

Then\marginnote{7.7} the Buddha robed up in the morning and, taking his bowl and robe, went to \textsanskrit{Sālavatikā} together with the \textsanskrit{Saṅgha} of mendicants. Now, Rosika was following behind the Buddha, and told him of Lohicca’s views, adding, “Sir, please dissuade him from that harmful misconception.”\footnote{It seems the barber was not only a trusted confidant, but an intelligent man with sincere concern for Lohicca’s well being. } 

“Hopefully\marginnote{8.10} that’ll happen, Rosika, hopefully that’ll happen.”\footnote{The Buddha is modest as to his chances. } 

Then\marginnote{9.1} the Buddha approached Lohicca’s home, where he sat on the seat spread out. Then Lohicca served and satisfied the mendicant \textsanskrit{Saṅgha} headed by the Buddha with his own hands with delicious fresh and cooked foods. 

\section*{1. Questioning Lohicca }

When\marginnote{9.4} the Buddha had eaten and washed his hand and bowl, Lohicca took a low seat and sat to one side. 

The\marginnote{9.5} Buddha said to him, “Is it really true, Lohicca, that you have such a harmful misconception:\footnote{Here the Buddha takes the initiative. “Harmful misconception” is \textit{\textsanskrit{pāpakaṁ} \textsanskrit{diṭṭhigataṁ}}. } ‘Should an ascetic or brahmin achieve some skillful quality, they ought not inform anyone else. For what can one person do for another? Suppose someone cut off an old bond, only to create another new bond. That’s the consequence of such a wicked, greedy deed, I say. For what can one person do for another?’” 

“Yes,\marginnote{9.10} Mister Gotama.” 

“What\marginnote{10.1} do you think, Lohicca? Do you reside in \textsanskrit{Sālavatikā}?” 

“Yes,\marginnote{10.3} Mister Gotama.” 

“Lohicca,\marginnote{10.4} suppose someone were to say: ‘The brahmin Lohicca resides in \textsanskrit{Sālavatikā}. He alone should enjoy the revenues produced in \textsanskrit{Sālavatikā} and not share them with anyone else.’ Would the person who spoke like that make it difficult for those whose living depends on you or not?” 

“They\marginnote{10.8} would, Mister Gotama.” 

“But\marginnote{10.9} is someone who creates difficulties for others acting kindly or unkindly?” 

“Unkindly,\marginnote{10.10} sir.” 

“But\marginnote{10.11} does an unkind person have love in their heart or hostility?” 

“Hostility,\marginnote{10.12} sir.” 

“And\marginnote{10.13} when the heart is full of hostility, is there right view or wrong view?” 

“Wrong\marginnote{10.14} view, Mister Gotama.” 

“An\marginnote{10.15} individual with wrong view is reborn in one of two places, I say: hell or the animal realm. 

What\marginnote{11.1} do you think, Lohicca? Does King Pasenadi reign over \textsanskrit{Kāsi} and Kosala?”\footnote{Kosala is the native realm of Pasenadi. \textsanskrit{Kāsi} had formerly been an independent kingdom, but was taken over by Pasenadi’s father \textsanskrit{Mahākosala}. Towards the end of the Buddha’s life it was contested between Kosala and Magadha (\href{https://suttacentral.net/sn3.14/en/sujato}{SN 3.14}, \href{https://suttacentral.net/sn3.15/en/sujato}{SN 3.15}). Ultimately it became part of the greater Magadhan empire. } 

“Yes,\marginnote{11.3} Mister Gotama.” 

“Lohicca,\marginnote{11.4} suppose someone were to say: ‘King Pasenadi reigns over \textsanskrit{Kāsi} and Kosala. He alone should enjoy the revenues produced in \textsanskrit{Kāsi} and Kosala and not share them with anyone else.’ Would the person who spoke like that make it difficult for yourself and others whose living depends on King Pasenadi or not?” 

“They\marginnote{11.8} would, Mister Gotama.” 

“But\marginnote{11.9} is someone who creates difficulties for others acting kindly or unkindly?” 

“Unkindly,\marginnote{11.10} sir.” 

“But\marginnote{11.11} does an unkind person have love in their heart or hostility?” 

“Hostility,\marginnote{11.12} sir.” 

“And\marginnote{11.13} when the heart is full of hostility, is there right view or wrong view?” 

“Wrong\marginnote{11.14} view, Mister Gotama.” 

“An\marginnote{11.15} individual with wrong view is reborn in one of two places, I say: hell or the animal realm. 

So\marginnote{12.1} it seems, Lohicca, that should someone say such a thing either of Lohicca or of King Pasenadi, that is wrong view. 

In\marginnote{13.3} the same way, suppose someone were to say: ‘Should an ascetic or brahmin achieve some skillful quality, they ought not inform anyone else. For what can one person do for another? Suppose someone cut off an old bond, only to create another new bond. That’s the consequence of such a wicked, greedy deed, I say. For what can one person do for another?’ 

Now,\marginnote{13.7} there are gentlemen who, relying on the teaching and training proclaimed by the Realized One, achieve a high distinction such as the following: they realize the fruit of stream-entry, the fruit of once-return, the fruit of non-return, or perfection. And in addition, there are those who ripen the seeds for rebirth in a heavenly state. The person who spoke like that makes it difficult for them. They’re acting unkindly, their heart is full of hostility, and they have wrong view.\footnote{It is only through sharing what good things we know that we can support each other. } An individual with wrong view is reborn in one of two places, I say: hell or the animal realm. 

\section*{2. Three Teachers Who Deserve to Be Reprimanded }

Lohicca,\marginnote{16.1} there are three kinds of teachers in the world who deserve to be reprimanded.\footnote{Even though he has characterized Lohicca’s view as harmful, the Buddha goes out of his way to show that it is not entirely wrong. There are cases where it is best to avoid teaching. } When someone reprimands such teachers, the reprimand is true, correct, legitimate, and blameless. What three? 

Firstly,\marginnote{16.4} take a teacher who has not reached the goal of the ascetic life for which they went forth from the lay life to homelessness. They teach their disciples: ‘This is for your welfare. This is for your happiness.’ But their disciples don’t want to listen. They don’t actively listen or try to understand. They proceed having turned away from the teacher’s instruction. That teacher deserves to be reprimanded: ‘Venerable, you haven’t reached the goal of the ascetic life; and when you teach disciples they proceed having turned away from the teacher’s instruction. It’s like a man who makes advances on a woman though she pulls away, or embraces her though she turns her back.\footnote{The genders of this passage are made clear through the use of feminine nouns. Making unwelcome advances was seen as an obvious example of something wrong. } That’s the consequence of such a wicked, greedy deed, I say. For what can one do for another?’ This is the first kind of teacher who deserves to be reprimanded. 

Furthermore,\marginnote{17.1} take a teacher who has not reached the goal of the ascetic life for which they went forth from the lay life to homelessness. They teach their disciples: ‘This is for your welfare. This is for your happiness.’ Their disciples do want to listen. They actively listen and try to understand. They don’t proceed having turned away from the teacher’s instruction.\footnote{At \href{https://suttacentral.net/dn29/en/sujato\#5.2}{DN 29:5.2} the opposite sense is expressed as \textit{\textsanskrit{samādāya} \textsanskrit{taṁ} \textsanskrit{dhammaṁ} vattati}. } That teacher deserves to be reprimanded: ‘Venerable, you haven’t reached the goal of the ascetic life; and when you teach disciples they don’t proceed having turned away from the teacher’s instruction. It’s like someone who abandons their own field and presumes to weed someone else’s field. That’s the consequence of such a wicked, greedy deed, I say. For what can one do for another?’ This is the second kind of teacher who deserves to be reprimanded. 

Furthermore,\marginnote{18.1} take a teacher who has reached the goal of the ascetic life for which they went forth from the lay life to homelessness. They teach their disciples: ‘This is for your welfare. This is for your happiness.’ But their disciples don’t want to listen. They don’t actively listen or try to understand. They proceed having turned away from the teacher’s instruction. That teacher deserves to be reprimanded: ‘Venerable, you have reached the goal of the ascetic life; yet when you teach disciples they proceed having turned away from the teacher’s instruction. Suppose someone cut off an old bond, only to create another new bond.\footnote{The Buddha adopts Lohicca’s formulation, but applies it in a specific sense, not as a generalization. } That’s the consequence of such a wicked, greedy deed, I say. For what can one person do for another?’ This is the third kind of teacher who deserves to be reprimanded. 

These\marginnote{18.14} are the three kinds of teachers in the world who deserve to be reprimanded. When someone reprimands such teachers, the reprimand is true, correct, legitimate, and blameless.” 

\section*{3. A Teacher Who Does Not Deserve to Be Reprimanded }

When\marginnote{19.1} he had spoken, Lohicca said to the Buddha, “But Mister Gotama, is there a teacher in the world who does not deserve to be reprimanded?”\footnote{Now that the Buddha has established a degree of overlap between their views, Lohicca wants to hear more. } 

“There\marginnote{19.3} is, Lohicca.”\footnote{Again, the Buddha answers directly and simply. } 

“But\marginnote{19.4} who is that teacher?” 

“It’s\marginnote{20{-}55.1} when a Realized One arises in the world, perfected, a fully awakened Buddha … That’s how a mendicant is accomplished in ethics. … They enter and remain in the first absorption … A teacher under whom a disciple achieves such a high distinction is one who does not deserve to be reprimanded. When someone reprimands such a teacher, the reprimand is false, baseless, illegitimate, and blameworthy. 

They\marginnote{56{-}62.1} enter and remain in the second absorption … third absorption … fourth absorption. A teacher under whom a disciple achieves such a high distinction is one who does not deserve to be reprimanded. … 

They\marginnote{63{-}77.1} project and extend the mind toward knowledge and vision … A teacher under whom a disciple achieves such a high distinction is one who does not deserve to be reprimanded. … 

They\marginnote{63{-}77.4} understand: ‘… there is nothing further for this place.’ A teacher under whom a disciple achieves such a high distinction is one who does not deserve to be reprimanded. When someone reprimands such a teacher, the reprimand is false, baseless, illegitimate, and blameworthy.” 

When\marginnote{78.1} he had spoken, Lohicca said to the Buddha: 

“Suppose,\marginnote{78.2} Mister Gotama, a person was falling over a cliff, and another person were to grab them by the hair, pull them up, and place them on firm ground.\footnote{\textit{Naraka} means “cliff” or “abyss” in early Pali (\href{https://suttacentral.net/mn49/en/sujato\#5.9}{MN 49:5.9}, \href{https://suttacentral.net/mn86/en/sujato\#6.15}{MN 86:6.15}, \href{https://suttacentral.net/snp3.11/en/sujato\#28.4}{Snp 3.11:28.4}, \href{https://suttacentral.net/thag16.8/en/sujato\#4.2}{Thag 16.8:4.2}). It does not have the sense “hell” until the late canonical period; the early Pali term for hell is \textit{niraya}. Notably, \textit{naraka} is not strongly attested in the sense of “hell” for pre-Buddhist Sanskrit either, although we do find \textit{\textsanskrit{nāraka}}, apparently in the sense of “hell being”, at Atharva Veda 12.4.36c and Śukla Yajur Veda 30.5. } In the same way, when I was falling off a cliff Mister Gotama pulled me up and placed me on safe ground. 

Excellent,\marginnote{78.4} Mister Gotama! Excellent! As if he were righting the overturned, or revealing the hidden, or pointing out the path to the lost, or lighting a lamp in the dark so people with clear eyes can see what’s there, Mister Gotama has made the Teaching clear in many ways. I go for refuge to Mister Gotama, to the teaching, and to the mendicant \textsanskrit{Saṅgha}. From this day forth, may Mister Gotama remember me as a lay follower who has gone for refuge for life.” 

%
\chapter*{{\suttatitleacronym DN 13}{\suttatitletranslation Experts in the Three Vedas }{\suttatitleroot Tevijjasutta}}
\addcontentsline{toc}{chapter}{\tocacronym{DN 13} \toctranslation{Experts in the Three Vedas } \tocroot{Tevijjasutta}}
\markboth{Experts in the Three Vedas }{Tevijjasutta}
\extramarks{DN 13}{DN 13}

\scevam{So\marginnote{1.1} I have heard. }At one time the Buddha was wandering in the land of the Kosalans together with a large \textsanskrit{Saṅgha} of five hundred mendicants when he arrived at a village of the Kosalan brahmins named \textsanskrit{Manasākaṭa}.\footnote{\textsanskrit{Manasākaṭa} is mentioned only here. } He stayed in a mango grove on a bank of the river \textsanskrit{Aciravatī} to the north of \textsanskrit{Manasākaṭa}.\footnote{\textsanskrit{Aciravatī} is called Rapti today. It was one of the great rivers that flowed from the Himalayas through Kosala into the Ganges. } 

Now\marginnote{2.1} at that time several very well-known well-to-do brahmins were residing in \textsanskrit{Manasākaṭa}. They included the brahmins \textsanskrit{Caṅkī}, \textsanskrit{Tārukkha}, \textsanskrit{Pokkharasāti}, \textsanskrit{Jānussoṇi}, Todeyya, and others. 

Then\marginnote{3.1} as the students \textsanskrit{Vāseṭṭha} and \textsanskrit{Bhāradvāja} were going for a walk they began a discussion regarding what is the path and what is not the path.\footnote{Similar discussions are found in \href{https://suttacentral.net/mn98/en/sujato}{MN 98} = \href{https://suttacentral.net/snp3.9/en/sujato}{Snp 3.9} and \href{https://suttacentral.net/dn27/en/sujato}{DN 27}. | The compound \textit{\textsanskrit{maggāmagga}} can be read either as “what is the path and what is not the path” (per commentary, \textit{magge ca amagge ca}), or as “the variety of paths” (compare \textit{\textsanskrit{phalāphala}}, “all sorts of fruit”). Here, however, they are concerned to distinguish one path as correct. } 

\textsanskrit{Vāseṭṭha}\marginnote{4.1} said this: “This is the only straight path, the direct route that delivers one who practices it to the company of Divinity; namely, that explained by the brahmin \textsanskrit{Pokkharasāti}.”\footnote{This must have been earlier than \textsanskrit{Pokkharasāti}’s conversion at \href{https://suttacentral.net/dn3/en/sujato\#2.22.1}{DN 3:2.22.1}. \textsanskrit{Pokkharasāti} was a family man who denied the reality of superhuman meditative attainments (\href{https://suttacentral.net/mn99/en/sujato\#10.4}{MN 99:10.4}) and emphasized ethics and duties over lineage (\href{https://suttacentral.net/mn98/en/sujato\#3.7}{MN 98:3.7}), which agrees with him being cited on ethical matters at Āpastamba Dharmasūtra 1.6.19.7 and 1.10.28. | \textit{\textsanskrit{Brahmasahabyatā}} does not mean “union with \textsanskrit{Brahmā}” but rather “rebirth as one of the members of Brahma’s retinue” (see eg. \href{https://suttacentral.net/an5.34/en/sujato\#9.4}{AN 5.34:9.4}). The non-dualist concept that the limited personal self merges with the cosmic divinity is expressed in Pali, rather, with such phrases as \textit{so \textsanskrit{attā} so loko} (“the self is identical with the cosmos”). | For \textit{\textsanskrit{añjasa}} (“direct route”) see \href{https://suttacentral.net/sn12.65/en/sujato\#7.1}{SN 12.65:7.1} and note. } 

\textsanskrit{Bhāradvāja}\marginnote{5.1} said this: “This is the only straight path, the direct route that delivers one who practices it to the company of Divinity; namely, that explained by the brahmin \textsanskrit{Tārukkha}.”\footnote{In Pali we never meet \textsanskrit{Tārukkha} and he is only mentioned in his absence. He evidently advocated that lineage rather than conduct made one a brahmin (\href{https://suttacentral.net/mn98/en/sujato\#3.4}{MN 98:3.4}). There is a \textsanskrit{Tārukṣya} whose view was that union (with \textsanskrit{Brahmā}) arose with the conjunction of speech and breath; this was discussed alongside the views of many other brahmins (Aitareya \textsanskrit{Āraṇyaka} 3.1.6.1; \textsanskrit{Śāṅkhāyana} \textsanskrit{Āraṇyaka} 7.19). In Rig Veda 8.46.32 a certain \textsanskrit{Balbūtha} \textsanskrit{Tarukṣa} the \textsanskrit{Dāsa} makes an offering to a sage. \textsanskrit{Sāyaṇa}, the Vedic commentator, says that \textsanskrit{Balbūtha} \textsanskrit{Tarukṣa} was a guardian of cows, evidently alluding to the Aitareya \textsanskrit{Āraṇyaka}, which describes \textsanskrit{Tārukṣya} as a guardian of his teacher’s cows, thus locating \textsanskrit{Tārukṣya} in the lineage of \textsanskrit{Tarukṣa}. \textsanskrit{Hiraṇyakeśīgṛhyasūtra} 2.8.19 also mentions him as a teacher, there spelled \textsanskrit{Tarukṣa}. } 

But\marginnote{6.1} neither was able to persuade the other. So \textsanskrit{Vāseṭṭha} said to \textsanskrit{Bhāradvāja}, “\textsanskrit{Bhāradvāja}, the ascetic Gotama—a Sakyan, gone forth from a Sakyan family—is staying in a mango grove on a bank of the river \textsanskrit{Aciravatī} to the north of \textsanskrit{Manasākaṭa}. He has this good reputation: ‘That Blessed One is perfected, a fully awakened Buddha, accomplished in knowledge and conduct, holy, knower of the world, supreme guide for those who wish to train, teacher of gods and humans, awakened, blessed.’ Come, let’s go to see him and ask him about this matter. As he answers, so we’ll remember it.” 

“Yes,\marginnote{7.7} sir,” replied \textsanskrit{Bhāradvāja}. 

\section*{1. What is the Path and What is Not the Path }

So\marginnote{8.1} they went to the Buddha and exchanged greetings with him. When the greetings and polite conversation were over, they sat down to one side and \textsanskrit{Vāseṭṭha} told him of their conversation, adding: “In this matter we have a dispute, a disagreement, a difference of opinion.” 

“So,\marginnote{9.1} \textsanskrit{Vāseṭṭha}, it seems that you say that the straight path is that explained by \textsanskrit{Pokkharasāti}, while \textsanskrit{Bhāradvāja} says that the straight path is that explained by \textsanskrit{Tārukkha}. But what exactly is your disagreement about?” 

“About\marginnote{10.1} what is the path and what is not the path, Mister Gotama. Even though brahmins describe different paths—the Adhvaryu brahmins, the \textsanskrit{Taittirīya} brahmins, the \textsanskrit{Chāndogya} brahmins, the \textsanskrit{Cāndrāyaṇa} brahmins, and the \textsanskrit{Bahvṛca} brahmins—all of them still lead someone who practices them to the company of Divinity.\footnote{Identified by Wijesekera (\emph{A Pali Reference to \textsanskrit{Brāhmaṇa}-\textsanskrit{Caraṇas}}, Adyar Library Bulletin, vol 20, 1956; reprinted in \emph{Buddhist and Vedic Studies}) and Jayatilleke (\emph{Early Buddhist Theory of Knowledge}, p. 480). I use the familiar Sanskrit forms, as the Pali has several dubious spellings and variants. Their texts and corresponding Vedas are respectively: Adhvaryu = Śatapatha \textsanskrit{Brāhmaṇa} (incl. \textsanskrit{Bṛhadāraṇyaka} \textsanskrit{Upaniṣad}; White Yajur Veda); \textsanskrit{Taittirīya} = \textsanskrit{Taittirīya} \textsanskrit{Brāhmaṇa} (Black Yajur Veda); \textsanskrit{Chāndogya} = \textsanskrit{Chāndogya} \textsanskrit{Brāhmaṇa} (\textsanskrit{Sāman} Veda); \textsanskrit{Cāndrāyaṇa} (omitted from MS edition) = \textsanskrit{Kauṣītaki} \textsanskrit{Brāhmaṇa} (Rig Veda; spelling established by Wijesekera; see below at \href{https://suttacentral.net/dn13/en/sujato\#16.2}{DN 13:16.2}); \textsanskrit{Bahvṛca} = \textsanskrit{Bahvṛca} \textsanskrit{Brāhmaṇa} (Rig Veda; incorporated in Aitareya and \textsanskrit{Kauśītaki}.) This is the only time the Pali canon mentions these schools, but in some cases we can identify them with brahmins in the canon. Examples include the murmuring \textsanskrit{Chāndogya} brahmin (\href{https://suttacentral.net/ud1.4/en/sujato}{Ud 1.4}); or the Buddha’s former teachers, who evidently hailed from the Addhariya tradition of the Śatapatha \textsanskrit{Brāhmaṇa} (\href{https://suttacentral.net/mn26/en/sujato\#15.1}{MN 26:15.1}ff.). | Jayatilleke notes that the Śatapatha describes its own adherents as Adhvaryu (\textit{\textsanskrit{addhariyā}}), those priests of the Yajur Veda responsible for the physical acts at the ritual. } 

It’s\marginnote{10.3} like a village or town that has many different roads nearby, yet all of them meet at that village.\footnote{Earlier they were arguing over which one of the paths was correct, whereas now they have shifted to a more universalist “many roads up the same mountain” position. When speaking with each other they saw each others’ views as contradictory, but when speaking with an outsider they adopted a more conciliatory position. } In the same way, even though brahmins describe different paths—the Adhvaryu brahmins, the \textsanskrit{Taittirīya} brahmins, the \textsanskrit{Chāndogya} brahmins, the \textsanskrit{Cāndrāyaṇa} brahmins, and the \textsanskrit{Bahvṛca} brahmins—all of them still lead someone who practices them to the company of Divinity.” 

\section*{2. Questioning \textsanskrit{Vāseṭṭha} }

“Do\marginnote{11.1} you say, ‘they lead someone’, \textsanskrit{Vāseṭṭha}?” 

“I\marginnote{11.2} do, Mister Gotama.” 

“Do\marginnote{11.3} you say, ‘they lead someone’, \textsanskrit{Vāseṭṭha}?” 

“I\marginnote{11.4} do, Mister Gotama.” 

“Do\marginnote{11.5} you say, ‘they lead someone’, \textsanskrit{Vāseṭṭha}?” 

“I\marginnote{11.6} do, Mister Gotama.” 

“Well,\marginnote{12.1} of the brahmins who are proficient in the three Vedas, \textsanskrit{Vāseṭṭha}, is there even a single one who has seen the Divinity with their own eyes?” 

“No,\marginnote{12.2} Mister Gotama.” 

“Well,\marginnote{12.3} has even a single one of their tutors seen the Divinity with their own eyes?” 

“No,\marginnote{12.4} Mister Gotama.” 

“Well,\marginnote{12.5} has even a single one of their tutors’ tutors seen the Divinity with their own eyes?” 

“No,\marginnote{12.6} Mister Gotama.” 

“Well,\marginnote{12.7} has anyone back to the seventh generation of tutors seen the Divinity with their own eyes?” 

“No,\marginnote{12.8} Mister Gotama.” 

“Well,\marginnote{13.1} what of the ancient seers of the brahmins proficient in the three Vedas, namely \textsanskrit{Aṭṭhaka}, \textsanskrit{Vāmaka}, \textsanskrit{Vāmadeva}, \textsanskrit{Vessāmitta}, Yamadaggi, \textsanskrit{Aṅgīrasa}, \textsanskrit{Bhāradvāja}, \textsanskrit{Vāseṭṭha}, Kassapa, and Bhagu? They were the authors and propagators of the hymns. Their hymnal was sung and propagated and compiled in ancient times; and these days, brahmins continue to sing and chant it, chanting what was chanted and teaching what was taught.\footnote{The ten names in Pali include the seven authors of the so-called “family books” of the Rig Veda (\textsanskrit{Maṇḍalas} 2–8). As founders of poetic lineages, we often find works by their descendants, which are not always confined to their dedicated family book. Poems by the other three authors are mostly outside the family books. Thus the sages listed here cover most of the Rig Veda, although the Vedic tradition records many other authors as well. | \textsanskrit{Aṭṭhaka} = Atri Bhauma (\textsanskrit{Maṇḍala} 5, rather than \textsanskrit{Aṣṭaka} \textsanskrit{Vaiśvāmitra} of 10.104); \textsanskrit{Vāmaka} = Vamra(ka) \textsanskrit{Vaikhānasa} (10.99; see 9.66); \textsanskrit{Vāmadeva} = \textsanskrit{Vāmadeva} Gautama (\textsanskrit{Maṇḍala} 4); \textsanskrit{Vessāmitta} = \textsanskrit{Viśvāmitra} \textsanskrit{Gāthina} (\textsanskrit{Maṇḍala} 3); Yamadaggi = Jamadagni \textsanskrit{Bhārgava}, a descendant of \textsanskrit{Bhṛgu} (several hymns mostly in \textsanskrit{Maṇḍalas} 9 and 10); \textsanskrit{Aṅgīrasa} = \textsanskrit{Aṅgirasa}, identified with Agni as the founder of a lineage of poet-singers (\textsanskrit{Maṇḍala} 8); \textsanskrit{Bhāradvāja} = \textsanskrit{Bharadvāja} \textsanskrit{Bārhaspatya} (\textsanskrit{Maṇḍala} 6); \textsanskrit{Vāseṭṭha} = \textsanskrit{Vasiṣṭha} \textsanskrit{Maitrāvaruṇi} (\textsanskrit{Maṇḍala} 7); Kassapa = \textsanskrit{Kaśyapa} \textsanskrit{Mārīca} (several hymns mostly in \textsanskrit{Maṇḍalas} 9 and 10); Bhagu = \textsanskrit{Bhṛgu}, the bringer of fire from heaven whose adoptive descendant was \textsanskrit{Gṛtsamada} \textsanskrit{Bhārgava} Śaunaka (\textsanskrit{Maṇḍala} 2). } Did they say: ‘We know and see where the Divinity is or what way he lies’?”\footnote{Unlike the Buddhist monk at \href{https://suttacentral.net/dn11/en/sujato\#80.1}{DN 11:80.1}. } 

“No,\marginnote{13.4} Mister Gotama.” 

“So\marginnote{14.1} it seems that none of those brahmins have seen the Divinity with their own eyes, and not even the ancient seers claimed to know where he is. Yet the brahmins proficient in the three Vedas say: ‘We teach the path to the company of that which we neither know nor see. This is the only straight path, the direct route that delivers one who practices it to the company of Divinity.’ 

What\marginnote{14.10} do you think, \textsanskrit{Vāseṭṭha}? This being so, doesn’t their statement turn out to have no demonstrable basis?”\footnote{“No demonstrable basis” is \textit{\textsanskrit{appāṭihīrakataṁ}}. } 

“Clearly\marginnote{14.12} that’s the case, Mister Gotama.” 

“Good,\marginnote{15.1} \textsanskrit{Vāseṭṭha}. For it is impossible that they should teach the path to that which they neither know nor see. 

Suppose\marginnote{15.2} there was a queue of blind men, each holding the one in front: the first one does not see, the middle one does not see, and the last one does not see.\footnote{The “blind following the blind” is also at \href{https://suttacentral.net/mn95/en/sujato\#13.24}{MN 95:13.24} and \href{https://suttacentral.net/mn99/en/sujato\#9.25}{MN 99:9.25}. \textsanskrit{Maitrī} \textsanskrit{Upaniṣad} 7.8–9 turns it around, saying that the blind teach false doctrines aimed at destroying the Vedas, “the doctrine of not-self” (\textit{\textsanskrit{nairātmyavāda}}), an obvious reference to Buddhists. We also find it at \textsanskrit{Kaṭha} \textsanskrit{Upaniṣad} 1.2.5, \textsanskrit{Mahābhārata} 2.38.3, and the Jain \textsanskrit{Sūyagaḍa} 1.1.2.19. } In the same way, it seems to me that the brahmins’ statement turns out to be comparable to a queue of blind men: the first one does not see, the middle one does not see, and the last one does not see. Their statement turns out to be a joke—mere words, vacuous and hollow. 

What\marginnote{16.1} do you think, \textsanskrit{Vāseṭṭha}? Do the brahmins proficient in the three Vedas see the sun and moon just as other folk do? And do they pray to them and exalt them, following their course from where they rise to where they set with joined palms held in worship?”\footnote{\textsanskrit{Kuṣītaka} worshiped the rising and setting sun and moon, turning himself to follow their course (\textsanskrit{Kauṣītaki} \textsanskrit{Upaniṣad} 2.7–9). He founded the \textsanskrit{Kauṣītaki} lineage, referred to above as “those who follow the course of the moon” (\textit{\textsanskrit{cāndrāyaṇa}}). } 

“Yes,\marginnote{16.3} Mister Gotama.” 

“What\marginnote{17.1} do you think, \textsanskrit{Vāseṭṭha}? Though this is so, are the brahmins proficient in the three Vedas able to teach the path to the company of the sun and moon, saying: ‘This is the only straight path, the direct route that delivers one who practices it to the company of the sun and moon’?”\footnote{Indra taught \textsanskrit{Bhāradvāja} a new threefold knowledge by which he might become immortal and realize companionship (\textit{\textsanskrit{sāyujya}}) with the sun (\textsanskrit{Taittirīya} \textsanskrit{Brāhmaṇa} 3.10.11.15). } 

“No,\marginnote{17.4} Mister Gotama.” 

“So\marginnote{18.1} it seems that even though the brahmins proficient in the three Vedas see the sun and moon, they are not able to teach the path to the company of the sun and moon. 

But\marginnote{18.3} it seems that even though they have not seen the Divinity with their own eyes, they still claim to teach the path to the company of that which they neither know nor see. 

What\marginnote{18.11} do you think, \textsanskrit{Vāseṭṭha}? This being so, doesn’t their statement turn out to have no demonstrable basis?” 

“Clearly\marginnote{18.13} that’s the case, Mister Gotama.” 

“Good,\marginnote{18.14} \textsanskrit{Vāseṭṭha}. For it is impossible that they should teach the path to that which they neither know nor see. 

\subsection*{2.1. The Simile of the Finest Lady in the Land }

Suppose\marginnote{19.1} a man were to say, ‘Whoever the finest lady in the land is, it is her that I want, her that I desire!’ 

They’d\marginnote{19.3} say to him, ‘Mister, that finest lady in the land who you desire—do you know whether she’s an aristocrat, a brahmin, a peasant, or a menial?’ Asked this, he’d say, ‘No.’ 

They’d\marginnote{19.7} say to him, ‘Mister, that finest lady in the land who you desire—do you know her name or clan? Whether she’s tall or short or medium? Whether her skin is black, brown, or tawny? What village, town, or city she comes from?’ 

Asked\marginnote{19.9} this, he’d say, ‘No.’ 

They’d\marginnote{19.10} say to him, ‘Mister, do you desire someone who you’ve never even known or seen?’ 

Asked\marginnote{19.12} this, he’d say, ‘Yes.’ 

What\marginnote{19.13} do you think, \textsanskrit{Vāseṭṭha}? This being so, doesn’t that man’s statement turn out to have no demonstrable basis?” 

“Clearly\marginnote{19.15} that’s the case, Mister Gotama.” 

“In\marginnote{20.1} the same way, doesn’t the statement of those brahmins turn out to have no demonstrable basis?” 

“Clearly\marginnote{20.8} that’s the case, Mister Gotama.” 

“Good,\marginnote{20.9} \textsanskrit{Vāseṭṭha}. For it is impossible that they should teach the path to that which they neither know nor see. 

\subsection*{2.2. The Simile of the Ladder }

Suppose\marginnote{21.1} a man was to build a ladder at the crossroads for climbing up to a stilt longhouse. 

They’d\marginnote{21.2} say to him, ‘Mister, that stilt longhouse that you’re building a ladder for—do you know whether it’s to the north, south, east, or west? Or whether it’s tall or short or medium?’ 

Asked\marginnote{21.4} this, he’d say, ‘No.’ 

They’d\marginnote{22.1} say to him, ‘Mister, are you building a ladder for a longhouse that you’ve never even known or seen?’ 

Asked\marginnote{22.3} this, he’d say, ‘Yes.’ 

What\marginnote{22.4} do you think, \textsanskrit{Vāseṭṭha}? This being so, doesn’t that man’s statement turn out to have no demonstrable basis?” 

“Clearly\marginnote{22.6} that’s the case, Mister Gotama.” 

“In\marginnote{22.7} the same way, doesn’t the statement of those brahmins turn out to have no demonstrable basis?” 

“Clearly\marginnote{22.13} that’s the case, Mister Gotama.” 

“Good,\marginnote{23.1} \textsanskrit{Vāseṭṭha}. For it is impossible that they should teach the path to that which they neither know nor see. 

\subsection*{2.3. The Simile of the River \textsanskrit{Aciravatī} }

Suppose\marginnote{24.1} the river \textsanskrit{Aciravatī} was full to the brim so a crow could drink from it. Then along comes a person who wants to cross over to the far shore. Standing on the near shore, they’d call out to the far shore, ‘Come here, far shore! Come here, far shore!’ 

What\marginnote{24.5} do you think, \textsanskrit{Vāseṭṭha}? Would the far shore of the \textsanskrit{Aciravatī} river come over to the near shore because of that man’s call, request, desire, or expectation?” 

“No,\marginnote{24.7} Mister Gotama.” 

“In\marginnote{25.1} the same way, \textsanskrit{Vāseṭṭha}, the brahmins proficient in the three Vedas proceed having given up those things that make one a true brahmin, and having undertaken those things that make one not a true brahmin. Yet they say:\footnote{The bulk of the Vedic texts consist of prayers and invocations to various gods. } ‘We call upon Indra! We call upon Soma! We call upon \textsanskrit{Varuṇa}! We call upon \textsanskrit{Īsāna}! We call upon the Progenitor! We call upon the Divinity! We call upon Mahinda! We call upon Yama!’\footnote{Soma is the ritual drug of exhilaration, identified with the moon. | Vedic \textsanskrit{Varuṇa} was the god of command, the king of tough rule. | \textsanskrit{Īsāna} (Sanskrit \textsanskrit{Īśāna}) was created by \textsanskrit{Pajāpati} as “ruler”, said to be the sun (Śatapatha \textsanskrit{Brāhmaṇa} 6.1.3). He was later identified with Rudra and Śiva. | \textsanskrit{Pajāpati} (“progenitor”) was the lonely god of creation; the heat of his fervent exertions (\textit{tapas}) created the world and all things in it (Śatapatha \textsanskrit{Brāhmaṇa} 6). | \textsanskrit{Brahmā}, like \textsanskrit{Pajāpati}, is also a creator god, but is the divine power immanent within creation, rather than the forgotten instigator of the past. | The \textsanskrit{Mahāsaṅgīti} reading Mahiddhi (“great power”) does not seem to correspond with any particular Vedic deity. Accept the BJT reading Mahinda (Sanskrit Mahendra), the “Great Indra”, said to be a title of Indra bestowed after slaying the dragon \textsanskrit{Vṛtra} (Śatapatha \textsanskrit{Brāhmaṇa} 2.5.4.9). | Yama (“twin”, with his incestuous sister \textsanskrit{Yamī}) was born immortal but chose mortality, becoming the god of the dead. } 

So\marginnote{25.3} long as they proceed in this way it’s impossible that they will, when the body breaks up, after death, be reborn in the company of Divinity. 

Suppose\marginnote{26.1} the river \textsanskrit{Aciravatī} was full to the brim so a crow could drink from it. Then along comes a person who wants to cross over to the far shore. But while still on the near shore, their arms are tied tightly behind their back with a strong chain. 

What\marginnote{26.4} do you think, \textsanskrit{Vāseṭṭha}? Could that person cross over to the far shore?” 

“No,\marginnote{26.6} Mister Gotama.” 

“In\marginnote{27.1} the same way, the five kinds of sensual stimulation are called ‘chains’ and ‘fetters’ in the training of the Noble One. What five? Sights known by the eye, which are likable, desirable, agreeable, pleasant, sensual, and arousing. Sounds known by the ear … Smells known by the nose … Tastes known by the tongue … Touches known by the body, which are likable, desirable, agreeable, pleasant, sensual, and arousing. 

These\marginnote{27.8} are the five kinds of sensual stimulation that are called ‘chains’ and ‘fetters’ in the training of the Noble One. The brahmins proficient in the three Vedas enjoy these five kinds of sensual stimulation tied, infatuated, attached, blind to the drawbacks, and not understanding the escape. So long as they enjoy them it’s impossible that they will, when the body breaks up, after death, be reborn in the company of Divinity. 

Suppose\marginnote{29.1} the river \textsanskrit{Aciravatī} was full to the brim so a crow could drink from it. Then along comes a person who wants to cross over to the far shore. But they’d lie down wrapped in cloth from head to foot. 

What\marginnote{29.4} do you think, \textsanskrit{Vāseṭṭha}? Could that person cross over to the far shore?” 

“No,\marginnote{29.6} Mister Gotama.” 

“In\marginnote{30.1} the same way, the five hindrances are called ‘obstacles’ and ‘hindrances’ and ‘encasings’ and ‘shrouds’ in the training of the Noble One. What five? The hindrances of sensual desire, ill will, dullness and drowsiness, restlessness and remorse, and doubt. These five hindrances are called ‘obstacles’ and ‘hindrances’ and ‘encasings’ and ‘shrouds’ in the training of the Noble One. 

The\marginnote{30.5} brahmins proficient in the three Vedas are obstructed, hindered, encased, and shrouded by these five hindrances. So long as they are so obstructed it’s impossible that they will, when the body breaks up, after death, be reborn in the company of Divinity. 

\section*{3. Converging }

What\marginnote{31.1} do you think, \textsanskrit{Vāseṭṭha}? Have you heard that the brahmins who are elderly and senior, the tutors of tutors, say whether the Divinity is encumbered with possessions or not?” 

“That\marginnote{31.3} he is not, Mister Gotama.” 

“Is\marginnote{31.4} his heart full of enmity or not?” 

“It\marginnote{31.5} is not.” 

“Is\marginnote{31.6} his heart full of ill will or not?” 

“It\marginnote{31.7} is not.” 

“Is\marginnote{31.8} his heart corrupted or not?” 

“It\marginnote{31.9} is not.” 

“Does\marginnote{31.10} he wield power or not?” 

“He\marginnote{31.11} does.” 

“What\marginnote{32.1} do you think, \textsanskrit{Vāseṭṭha}? Are the brahmins proficient in the three Vedas encumbered with possessions or not?” 

“They\marginnote{32.3} are.” 

“Are\marginnote{32.4} their hearts full of enmity or not?” 

“They\marginnote{32.5} are.” 

“Are\marginnote{32.6} their hearts full of ill will or not?” 

“They\marginnote{32.7} are.” 

“Are\marginnote{32.8} their hearts corrupted or not?” 

“They\marginnote{32.9} are.” 

“Do\marginnote{32.10} they wield power or not?” 

“They\marginnote{32.11} do not.” 

“So\marginnote{33.1} it seems that the brahmins proficient in the three Vedas are encumbered with possessions, but the Divinity is not. But would brahmins who are encumbered with possessions come together and converge with the Divinity, who isn’t encumbered with possessions?” 

“No,\marginnote{33.3} Mister Gotama.” 

“Good,\marginnote{34.1} \textsanskrit{Vāseṭṭha}! It’s impossible that the brahmins who are encumbered with possessions will, when the body breaks up, after death, be reborn in the company of Divinity, who isn’t encumbered with possessions. 

And\marginnote{35.1} it seems that the brahmins have enmity, ill will, corruption, and do not wield power, while the Divinity is the opposite in all these things. But would brahmins who are opposite to the Divinity in all things come together and converge with him?” 

“No,\marginnote{35.5} Mister Gotama.” 

“Good,\marginnote{36.1} \textsanskrit{Vāseṭṭha}! It’s impossible that such brahmins will, when the body breaks up, after death, be reborn in the company of Divinity. 

But\marginnote{36.2} here the brahmins proficient in the three Vedas sink down where they have sat, only to drift apart, while imagining they’re crossing over to drier ground.\footnote{I wonder if this is a satire on the idea of \textit{\textsanskrit{upaniṣad}} (“sitting near”); even as they affirm their commitment to their texts, they drift apart (\textit{\textsanskrit{visāra}}) into separate schools and ideologies. } That’s why the three Vedas of the brahmins are called a ‘salted land’ and a ‘barren land’ and a ‘disaster’.” 

When\marginnote{37.1} he said this, \textsanskrit{Vāseṭṭha} said to the Buddha, “I have heard, Mister Gotama, that you know the path to company with Divinity.” 

“What\marginnote{37.3} do you think, \textsanskrit{Vāseṭṭha}? Is the village of \textsanskrit{Manasākaṭa} nearby?” 

“Yes\marginnote{37.5} it is.” 

“What\marginnote{37.6} do you think, \textsanskrit{Vāseṭṭha}? Suppose a person was born and raised in \textsanskrit{Manasākaṭa}. And as soon as they left the town some people asked them for the road to \textsanskrit{Manasākaṭa}. Would they be slow or hesitant to answer?” 

“No,\marginnote{37.10} Mister Gotama. Why is that? Because they were born and raised in \textsanskrit{Manasākaṭa}. They’re well acquainted with all the roads to the village.” 

“Still,\marginnote{38.1} it’s possible they might be slow or hesitant to answer. But the Realized One is never slow or hesitant when questioned about the realm of divinity or the practice that leads to the realm of divinity. I understand the Divinity, the realm of divinity, and the practice that leads to the realm of divinity, practicing in accordance with which one is reborn in the realm of divinity.” 

When\marginnote{39.1} he said this, \textsanskrit{Vāseṭṭha} said to the Buddha, “I have heard, Mister Gotama, that you teach the path to company with Divinity.\footnote{The close \textit{-ti} has apparently confused some editors; \textsanskrit{Vāseṭṭha} is quoting what he has heard. } Please teach us that path and elevate this generation of brahmins.” 

“Well\marginnote{39.4} then, \textsanskrit{Vāseṭṭha}, listen and apply your mind well, I will speak.” 

“Yes,\marginnote{39.5} sir,” replied \textsanskrit{Vāseṭṭha}. 

\section*{4. Teaching the Path to the Divinity }

The\marginnote{40{-}75.1} Buddha said this: 

“It’s\marginnote{40{-}75.2} when a Realized One arises in the world, perfected, a fully awakened Buddha … That’s how a mendicant is accomplished in ethics. … Seeing that the hindrances have been given up in them, joy springs up. Being joyful, rapture springs up. When the mind is full of rapture, the body becomes tranquil. When the body is tranquil, they feel bliss. And when blissful, the mind becomes immersed. 

They\marginnote{76.1} meditate spreading a heart full of love to one direction, and to the second, and to the third, and to the fourth. In the same way above, below, across, everywhere, all around, they spread a heart full of love to the whole world—abundant, expansive, limitless, free of enmity and ill will.\footnote{Here the four \textit{\textsanskrit{brahmavihāras}} (“meditations of \textsanskrit{Brahmā}”) stand in place of the four \textit{\textsanskrit{jhānas}}. \textit{\textsanskrit{Brahmavihāras}} are simply one of the means by which \textit{\textsanskrit{jhānas}} may be attained, chosen here to fit the stated goal of teaching the path to \textsanskrit{Brahmā}. The suttas treat them as pre-Buddhist, but they have not been traced as a group in pre-Buddhist texts. However, they are shared with later non-Buddhist texts such as  \textsanskrit{Yogasūtra} 1.33 and the Jain \textsanskrit{Tattvārthasūtra} 7.11. | “Love” (\textit{\textsanskrit{mettā}}) is a universal positive regard and well-wishing free of personal desires or attachments. It ultimately derives from the Vedic in the sense of “union”; Mitra was the god of alliances (Rig Veda 3.59). } 

Suppose\marginnote{77.1} there was a powerful horn blower. They’d easily make themselves heard in the four quarters. In the same way, when the heart’s release by love has been developed like this, any limited deeds they’ve done don’t remain or persist there.\footnote{The mind in \textit{\textsanskrit{jhāna}} is so powerful that it effectively overrides any limited kamma, including ordinary good or bad deeds. Unless they have committed serious crimes with a fixed kammic result such as matricide, etc., the meditator will be reborn in a \textsanskrit{Brahmā} realm. } This is a path to company with Divinity. 

Furthermore,\marginnote{78.1} a mendicant meditates spreading a heart full of compassion …\footnote{“Compassion” (\textit{\textsanskrit{karuṇā}}) is the quality of empathy with the suffering of another or oneself and the wish to remove it. } 

They\marginnote{78.2} meditate spreading a heart full of rejoicing …\footnote{“Rejoicing” (\textit{\textsanskrit{muditā}}) is joyful celebration in the success of others or oneself, free of jealousy or cynicism. } 

They\marginnote{78.3} meditate spreading a heart full of equanimity to one direction, and to the second, and to the third, and to the fourth. In the same way above, below, across, everywhere, all around, they spread a heart full of equanimity to the whole world—abundant, expansive, limitless, free of enmity and ill will.\footnote{Equanimity (\textit{\textsanskrit{upekkhā}}) is literally “close watching”, not interfering but standing ready when needed. It is not indifference, which is why it emerges only at the end, after the positive emotions are developed. } 

Suppose\marginnote{79.1} there was a powerful horn blower. They’d easily make themselves heard in the four quarters. In the same way, when the heart’s release by equanimity has been developed and cultivated like this, any limited deeds they’ve done don’t remain or persist there. This too is a path to company with Divinity.\footnote{A brahmin student who for their whole life practices harmlessness for all beings—except at holy places—attains the world of \textsanskrit{Brahmā} (\textsanskrit{Chāndogya} \textsanskrit{Upaniṣad} 8.15.1). The exception for holy places is, of course, to allow for the sacrifice. } 

What\marginnote{80.1} do you think, \textsanskrit{Vāseṭṭha}? When a mendicant meditates like this, are they encumbered with possessions or not?” 

“They\marginnote{80.3} are not.” 

“Is\marginnote{80.4} their heart full of enmity or not?” 

“It\marginnote{80.5} is not.” 

“Is\marginnote{80.6} their heart full of ill will or not?” 

“It\marginnote{80.7} is not.” 

“Is\marginnote{80.8} their heart corrupted or not?” 

“It\marginnote{80.9} is not.” 

“Do\marginnote{80.10} they wield power or not?” 

“They\marginnote{80.11} do.” 

“So\marginnote{81.1} it seems that that mendicant is not encumbered with possessions, and neither is the Divinity. Would a mendicant who is not encumbered with possessions come together and converge with the Divinity, who isn’t encumbered with possessions?” 

“Yes,\marginnote{81.3} Mister Gotama.” 

“Good,\marginnote{81.4} \textsanskrit{Vāseṭṭha}! It’s quite possible that a mendicant who is not encumbered with possessions will, when the body breaks up, after death, be reborn in the company of Divinity, who isn’t encumbered with possessions. 

And\marginnote{81.5} it seems that that mendicant has no enmity, ill will, corruption, and does wield power, while the Divinity is the same in all these things. Would a mendicant who is the same as the Divinity in all things come together and converge with him?” 

“Yes,\marginnote{81.9} Mister Gotama.” 

“Good,\marginnote{81.10} \textsanskrit{Vāseṭṭha}! It’s quite possible that that mendicant will, when the body breaks up, after death, be reborn in the company of Divinity.”\footnote{Here the entire concluding section of the Gradual Training dealing with wisdom is omitted, as the aim is limited to teaching rebirth with \textsanskrit{Brahmā} to Brahmanical laypeople, rather than teaching liberation. \textsanskrit{Vāseṭṭha} and \textsanskrit{Bhāradvāja} later applied to ordain, in which time the Buddha taught the full path to Nibbana (\href{https://suttacentral.net/dn27/en/sujato\#7.8}{DN 27:7.8}). } 

When\marginnote{82.1} he had spoken, \textsanskrit{Vāseṭṭha} and \textsanskrit{Bhāradvāja} said to him, “Excellent, Mister Gotama! Excellent! As if he were righting the overturned, or revealing the hidden, or pointing out the path to the lost, or lighting a lamp in the dark so people with clear eyes can see what’s there, Mister Gotama has made the teaching clear in many ways. We go for refuge to Mister Gotama, to the teaching, and to the mendicant \textsanskrit{Saṅgha}.\footnote{They also went for refuge in similar circumstances at \href{https://suttacentral.net/mn98/en/sujato\#14.4}{MN 98:14.4} = \href{https://suttacentral.net/snp3.9/en/sujato\#69.3}{Snp 3.9:69.3}. According to the commentary, that was the first time they went for refuge, while this was the second time. This makes sense in terms of the progress of the teachings, for there they discuss what makes a brahmin, whereas here they ask the more subtle question how to achieve rebirth with \textsanskrit{Brahmā}. The discussion, too, is on a more detailed level, with a more explicit criticism of the brahmins. Nonetheless, it is difficult to square the details of the narratives, for the opening of this sutta depicts \textsanskrit{Vāseṭṭha} and \textsanskrit{Bhāradvāja} speaking about the Buddha by reputation rather than as devotees who have previously met him and gone for refuge. } From this day forth, may Mister Gotama remember us as lay followers who have gone for refuge for life.” 

%
\backmatter%
%
\chapter*{Colophon}
\addcontentsline{toc}{chapter}{Colophon}
\markboth{Colophon}{Colophon}

\section*{The Translator}

Bhikkhu Sujato was born as Anthony Aidan Best on 4/11/1966 in Perth, Western Australia. He grew up in the pleasant suburbs of Mt Lawley and Attadale alongside his sister Nicola, who was the good child. His mother, Margaret Lorraine Huntsman née Pinder, said “he’ll either be a priest or a poet”, while his father, Anthony Thomas Best, advised him to “never do anything for money”. He attended Aquinas College, a Catholic school, where he decided to become an atheist. At the University of WA he studied philosophy, aiming to learn what he wanted to do with his life. Finding that what he wanted to do was play guitar, he dropped out. His main band was named Martha’s Vineyard, which achieved modest success in the indie circuit. 

A seemingly random encounter with a roadside joey took him to Thailand, where he entered his first meditation retreat at Wat Ram Poeng, Chieng Mai in 1992. Feeling the call to the Buddha’s path, he took full ordination in Wat Pa Nanachat in 1994, where his teachers were Ajahn Pasanno and Ajahn Jayasaro. In 1997 he returned to Perth to study with Ajahn Brahm at Bodhinyana Monastery. 

He spent several years practicing in seclusion in Malaysia and Thailand before establishing Santi Forest Monastery in Bundanoon, NSW, in 2003. There he was instrumental in supporting the establishment of the Theravada bhikkhuni order in Australia and advocating for women’s rights. He continues to teach in Australia and globally, with a special concern for the moral implications of climate change and other forms of environmental destruction. He has published a series of books of original and groundbreaking research on early Buddhism. 

In 2005 he founded SuttaCentral together with Rod Bucknell and John Kelly. In 2015, seeing the need for a complete, accurate, plain English translation of the Pali texts, he undertook the task, spending nearly three years in isolation on the isle of Qi Mei off the coast of the nation of Taiwan. He completed the four main \textsanskrit{Nikāyas} in 2018, and the early books of the Khuddaka \textsanskrit{Nikāya} were complete by 2021. All this work is dedicated to the public domain and is entirely free of copyright encumbrance. 

In 2019 he returned to Sydney where he established Lokanta Vihara (The Monastery at the End of the World). 

\section*{Creation Process}

Primary source was the digital \textsanskrit{Mahāsaṅgīti} edition of the Pali \textsanskrit{Tipiṭaka}. Translated from the Pali, with reference to several English translations, especially those of Bhikkhu Bodhi. Older translations by Maurice Walshe and T.W. and C.A.F. Rhys Davids were also consulted.

\section*{The Translation}

This translation was part of a project to translate the four Pali \textsanskrit{Nikāyas} with the following aims: plain, approachable English; consistent terminology; accurate rendition of the Pali; free of copyright. It was made during 2016–2018 while Bhikkhu Sujato was staying in Qimei, Taiwan.

\section*{About SuttaCentral}

SuttaCentral publishes early Buddhist texts. Since 2005 we have provided root texts in Pali, Chinese, Sanskrit, Tibetan, and other languages, parallels between these texts, and translations in many modern languages. Building on the work of generations of scholars, we offer our contribution freely.

SuttaCentral is driven by volunteer contributions, and in addition we employ professional developers. We offer a sponsorship program for high quality translations from the original languages. Financial support for SuttaCentral is handled by the SuttaCentral Development Trust, a charitable trust registered in Australia.

\section*{About Bilara}

“Bilara” means “cat” in Pali, and it is the name of our Computer Assisted Translation (CAT) software. Bilara is a web app that enables translators to translate early Buddhist texts into their own language. These translations are published on SuttaCentral with the root text and translation side by side.

\section*{About SuttaCentral Editions}

The SuttaCentral Editions project makes high quality books from selected Bilara translations. These are published in formats including HTML, EPUB, PDF, and print.

You are welcome to print any of our Editions.

%
\end{document}