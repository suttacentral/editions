\documentclass[12pt,openany]{book}%
\usepackage{lastpage}%
%
\usepackage[inner=1in, outer=1in, top=.7in, bottom=1in, papersize={6in,9in}, headheight=13pt]{geometry}
\usepackage{polyglossia}
\usepackage[12pt]{moresize}
\usepackage{soul}%
\usepackage{microtype}
\usepackage{tocbasic}
\usepackage{realscripts}
\usepackage{epigraph}%
\usepackage{setspace}%
\usepackage{sectsty}
\usepackage{fontspec}
\usepackage{marginnote}
\usepackage[bottom]{footmisc}
\usepackage{enumitem}
\usepackage{fancyhdr}
\usepackage{extramarks}
\usepackage{graphicx}
\usepackage{verse}
\usepackage{relsize}
\usepackage{etoolbox}
\usepackage[a-3u]{pdfx}

\hypersetup{
colorlinks=true,
urlcolor=black,
linkcolor=black,
citecolor=black
}

% use a small amount of tracking on small caps
\SetTracking[ spacing = {25*,166, } ]{ encoding = *, shape = sc }{ 25 }

% add a blank page
\newcommand{\blankpage}{
\newpage
\thispagestyle{empty}
\mbox{}
\newpage
}

% define languages
\setdefaultlanguage[]{english}
\setotherlanguage[script=Latin]{sanskrit}

%\usepackage{pagegrid}
%\pagegridsetup{top-left, step=.25in}

% define fonts
% use if arno sanskrit is unavailable
%\setmainfont{Gentium Plus}
%\newfontfamily\Semiboldsubheadfont[]{Gentium Plus}
%\newfontfamily\Semiboldnormalfont[]{Gentium Plus}
%\newfontfamily\Lightfont[]{Gentium Plus}
%\newfontfamily\Marginalfont[]{Gentium Plus}
%\newfontfamily\Allsmallcapsfont[RawFeature=+c2sc]{Gentium Plus}
%\newfontfamily\Noligaturefont[Renderer=Basic]{Gentium Plus}
%\newfontfamily\Noligaturecaptionfont[Renderer=Basic]{Gentium Plus}
%\newfontfamily\Fleuronfont[Ornament=1]{Gentium Plus}

% use if arno sanskrit is available. display is applied to \chapter and \part, subhead to \section and \subsection. When specifying semibold, the italic must be defined.
\setmainfont[Numbers=OldStyle]{Arno Pro}
\newfontfamily\Semibolddisplayfont[BoldItalicFont = Arno Pro Semibold Italic Display]{Arno Pro Semibold Display} %
\newfontfamily\Semiboldsubheadfont[BoldItalicFont = Arno Pro Semibold Italic Subhead]{Arno Pro Semibold Subhead}
\newfontfamily\Semiboldnormalfont[BoldItalicFont = Arno Pro Semibold Italic]{Arno Pro Semibold}
\newfontfamily\Marginalfont[RawFeature=+subs]{Arno Pro Regular}
\newfontfamily\Allsmallcapsfont[RawFeature=+c2sc]{Arno Pro}
\newfontfamily\Noligaturefont[Renderer=Basic]{Arno Pro}
\newfontfamily\Noligaturecaptionfont[Renderer=Basic]{Arno Pro Caption}

% chinese fonts
\newfontfamily\cjk{Noto Serif TC}
\newcommand*{\langlzh}[1]{\cjk{#1}\normalfont}%

% logo
\newfontfamily\Logofont{sclogo.ttf}
\newcommand*{\sclogo}[1]{\large\Logofont{#1}}

% use subscript numerals for margin notes
\renewcommand*{\marginfont}{\Marginalfont}

% ensure margin notes have consistent vertical alignment
\renewcommand*{\marginnotevadjust}{-.17em}

% use compact lists
\setitemize{noitemsep,leftmargin=1em}
\setenumerate{noitemsep,leftmargin=1em}
\setdescription{noitemsep, style=unboxed, leftmargin=0em}

% style ToC
\DeclareTOCStyleEntries[
  raggedentrytext,
  linefill=\hfill,
  pagenumberwidth=.5in,
  pagenumberformat=\normalfont,
  entryformat=\normalfont
]{tocline}{chapter,section}


  \setlength\topsep{0pt}%
  \setlength\parskip{0pt}%

% define new \centerpars command for use in ToC. This ensures centering, proper wrapping, and no page break after
\def\startcenter{%
  \par
  \begingroup
  \leftskip=0pt plus 1fil
  \rightskip=\leftskip
  \parindent=0pt
  \parfillskip=0pt
}
\def\stopcenter{%
  \par
  \endgroup
}
\long\def\centerpars#1{\startcenter#1\stopcenter}

% redefine part, so that it adds a toc entry without page number
\let\oldcontentsline\contentsline
\newcommand{\nopagecontentsline}[3]{\oldcontentsline{#1}{#2}{}}

    \makeatletter
\renewcommand*\l@part[2]{%
  \ifnum \c@tocdepth >-2\relax
    \addpenalty{-\@highpenalty}%
    \addvspace{0em \@plus\p@}%
    \setlength\@tempdima{3em}%
    \begingroup
      \parindent \z@ \rightskip \@pnumwidth
      \parfillskip -\@pnumwidth
      {\leavevmode
       \setstretch{.85}\large\scshape\centerpars{#1}\vspace*{-1em}\llap{#2}}\par
       \nobreak
         \global\@nobreaktrue
         \everypar{\global\@nobreakfalse\everypar{}}%
    \endgroup
  \fi}
\makeatother

\makeatletter
\def\@pnumwidth{2em}
\makeatother

% define new sectioning command, which is only used in volumes where the pannasa is found in some parts but not others, especially in an and sn

\newcommand*{\pannasa}[1]{\clearpage\thispagestyle{empty}\begin{center}\vspace*{14em}\setstretch{.85}\huge\itshape\scshape\MakeLowercase{#1}\end{center}}

    \makeatletter
\newcommand*\l@pannasa[2]{%
  \ifnum \c@tocdepth >-2\relax
    \addpenalty{-\@highpenalty}%
    \addvspace{.5em \@plus\p@}%
    \setlength\@tempdima{3em}%
    \begingroup
      \parindent \z@ \rightskip \@pnumwidth
      \parfillskip -\@pnumwidth
      {\leavevmode
       \setstretch{.85}\large\itshape\scshape\lowercase{\centerpars{#1}}\vspace*{-1em}\llap{#2}}\par
       \nobreak
         \global\@nobreaktrue
         \everypar{\global\@nobreakfalse\everypar{}}%
    \endgroup
  \fi}
\makeatother

% don't put page number on first page of toc (relies on etoolbox)
\patchcmd{\chapter}{plain}{empty}{}{}

% global line height
\setstretch{1.05}

% allow linebreak after em-dash
\catcode`\—=13
\protected\def—{\unskip\textemdash\allowbreak}

% style headings with secsty. chapter and section are defined per-edition
\partfont{\setstretch{.85}\normalfont\centering\textsc}
\subsectionfont{\setstretch{.85}\Semiboldsubheadfont}%
\subsubsectionfont{\setstretch{.85}\Semiboldnormalfont}

% style elements of suttatitle
\newcommand*{\suttatitleacronym}[1]{\smaller[2]{#1}\vspace*{.3em}}
\newcommand*{\suttatitletranslation}[1]{\linebreak{#1}}
\newcommand*{\suttatitleroot}[1]{\linebreak\smaller[2]\itshape{#1}}

\DeclareTOCStyleEntries[
  indent=3.3em,
  dynindent,
  beforeskip=.2em plus -2pt minus -1pt,
]{tocline}{section}

\DeclareTOCStyleEntries[
  indent=0em,
  dynindent,
  beforeskip=.4em plus -2pt minus -1pt,
]{tocline}{chapter}

\newcommand*{\tocacronym}[1]{\hspace*{-3.3em}{#1}\quad}
\newcommand*{\toctranslation}[1]{#1}
\newcommand*{\tocroot}[1]{(\textit{#1})}
\newcommand*{\tocchapterline}[1]{\bfseries\itshape{#1}}


% redefine paragraph and subparagraph headings to not be inline
\makeatletter
% Change the style of paragraph headings %
\renewcommand\paragraph{\@startsection{paragraph}{4}{\z@}%
            {-2.5ex\@plus -1ex \@minus -.25ex}%
            {1.25ex \@plus .25ex}%
            {\noindent\Semiboldnormalfont\normalsize}}

% Change the style of subparagraph headings %
\renewcommand\subparagraph{\@startsection{subparagraph}{5}{\z@}%
            {-2.5ex\@plus -1ex \@minus -.25ex}%
            {1.25ex \@plus .25ex}%
            {\noindent\Semiboldnormalfont\small}}
\makeatother

% use etoolbox to suppress page numbers on \part
\patchcmd{\part}{\thispagestyle{plain}}{\thispagestyle{empty}}
  {}{\errmessage{Cannot patch \string\part}}

% and to reduce margins on quotation
\patchcmd{\quotation}{\rightmargin}{\leftmargin 1.2em \rightmargin}{}{}
\AtBeginEnvironment{quotation}{\small}

% titlepage
\newcommand*{\titlepageTranslationTitle}[1]{{\begin{center}\begin{large}{#1}\end{large}\end{center}}}
\newcommand*{\titlepageCreatorName}[1]{{\begin{center}\begin{normalsize}{#1}\end{normalsize}\end{center}}}

% halftitlepage
\newcommand*{\halftitlepageTranslationTitle}[1]{\setstretch{2.5}{\begin{Huge}\uppercase{\so{#1}}\end{Huge}}}
\newcommand*{\halftitlepageTranslationSubtitle}[1]{\setstretch{1.2}{\begin{large}{#1}\end{large}}}
\newcommand*{\halftitlepageFleuron}[1]{{\begin{large}\Fleuronfont{{#1}}\end{large}}}
\newcommand*{\halftitlepageByline}[1]{{\begin{normalsize}\textit{{#1}}\end{normalsize}}}
\newcommand*{\halftitlepageCreatorName}[1]{{\begin{LARGE}{\textsc{#1}}\end{LARGE}}}
\newcommand*{\halftitlepageVolumeNumber}[1]{{\begin{normalsize}{\Allsmallcapsfont{\textsc{#1}}}\end{normalsize}}}
\newcommand*{\halftitlepageVolumeAcronym}[1]{{\begin{normalsize}{#1}\end{normalsize}}}
\newcommand*{\halftitlepageVolumeTranslationTitle}[1]{{\begin{Large}{\textsc{#1}}\end{Large}}}
\newcommand*{\halftitlepageVolumeRootTitle}[1]{{\begin{normalsize}{\Allsmallcapsfont{\textsc{\itshape #1}}}\end{normalsize}}}
\newcommand*{\halftitlepagePublisher}[1]{{\begin{large}{\Noligaturecaptionfont\textsc{#1}}\end{large}}}

% epigraph
\renewcommand{\epigraphflush}{center}
\renewcommand*{\epigraphwidth}{.85\textwidth}
\newcommand*{\epigraphTranslatedTitle}[1]{\vspace*{.5em}\footnotesize\textsc{#1}\\}%
\newcommand*{\epigraphRootTitle}[1]{\footnotesize\textit{#1}\\}%
\newcommand*{\epigraphReference}[1]{\footnotesize{#1}}%

% custom commands for html styling classes
\newcommand*{\scnamo}[1]{\begin{center}\textit{#1}\end{center}}
\newcommand*{\scendsection}[1]{\begin{center}\textit{#1}\end{center}}
\newcommand*{\scendsutta}[1]{\begin{center}\textit{#1}\end{center}}
\newcommand*{\scendbook}[1]{\begin{center}\uppercase{#1}\end{center}}
\newcommand*{\scendkanda}[1]{\begin{center}\textbf{#1}\end{center}}
\newcommand*{\scend}[1]{\begin{center}\textit{#1}\end{center}}
\newcommand*{\scuddanaintro}[1]{\textit{#1}}
\newcommand*{\scendvagga}[1]{\begin{center}\textbf{#1}\end{center}}
\newcommand*{\scrule}[1]{\textbf{#1}}
\newcommand*{\scadd}[1]{\textit{#1}}
\newcommand*{\scevam}[1]{\textsc{#1}}
\newcommand*{\scspeaker}[1]{\hspace{2em}\textit{#1}}
\newcommand*{\scbyline}[1]{\begin{flushright}\textit{#1}\end{flushright}\bigskip}

% custom command for thematic break = hr
\newcommand*{\thematicbreak}{\begin{center}\rule[.5ex]{6em}{.4pt}\begin{normalsize}\quad\Fleuronfont{•}\quad\end{normalsize}\rule[.5ex]{6em}{.4pt}\end{center}}

% manage and style page header and footer. "fancy" has header and footer, "plain" has footer only

\pagestyle{fancy}
\fancyhf{}
\fancyfoot[RE,LO]{\thepage}
\fancyfoot[LE,RO]{\footnotesize\lastleftxmark}
\fancyhead[CE]{\setstretch{.85}\Noligaturefont\MakeLowercase{\textsc{\firstrightmark}}}
\fancyhead[CO]{\setstretch{.85}\Noligaturefont\MakeLowercase{\textsc{\firstleftmark}}}
\renewcommand{\headrulewidth}{0pt}
\fancypagestyle{plain}{ %
\fancyhf{} % remove everything
\fancyfoot[RE,LO]{\thepage}
\fancyfoot[LE,RO]{\footnotesize\lastleftxmark}
\renewcommand{\headrulewidth}{0pt}
\renewcommand{\footrulewidth}{0pt}}

% style footnotes
\setlength{\skip\footins}{1em}

\makeatletter
\newcommand{\@makefntextcustom}[1]{%
    \parindent 0em%
    \thefootnote.\enskip #1%
}
\renewcommand{\@makefntext}[1]{\@makefntextcustom{#1}}
\makeatother

% hang quotes (requires microtype)
\microtypesetup{
  protrusion = true,
  expansion  = true,
  tracking   = true,
  factor     = 1000,
  patch      = all,
  final
}

% Custom protrusion rules to allow hanging punctuation
\SetProtrusion
{ encoding = *}
{
% char   right left
  {-} = {    , 500 },
  % Double Quotes
  \textquotedblleft
      = {1000,     },
  \textquotedblright
      = {    , 1000},
  \quotedblbase
      = {1000,     },
  % Single Quotes
  \textquoteleft
      = {1000,     },
  \textquoteright
      = {    , 1000},
  \quotesinglbase
      = {1000,     }
}

% make latex use actual font em for parindent, not Computer Modern Roman
\AtBeginDocument{\setlength{\parindent}{1em}}%
%

% Default values; a bit sloppier than normal
\tolerance 1414
\hbadness 1414
\emergencystretch 1.5em
\hfuzz 0.3pt
\clubpenalty = 10000
\widowpenalty = 10000
\displaywidowpenalty = 10000
\hfuzz \vfuzz
 \raggedbottom%

\title{Long Discourses}
\author{Bhikkhu Sujato}
\date{}%
% define a different fleuron for each edition
\newfontfamily\Fleuronfont[Ornament=16]{Arno Pro}

% Define heading styles per edition for chapter and section. Suttatitle can be either of these, depending on the volume. 

\let\oldfrontmatter\frontmatter
\renewcommand{\frontmatter}{%
\chapterfont{\setstretch{.85}\normalfont\centering}%
\sectionfont{\setstretch{.85}\Semiboldsubheadfont}%
\oldfrontmatter}

\let\oldmainmatter\mainmatter
\renewcommand{\mainmatter}{%
\chapterfont{\setstretch{.85}\normalfont\centering}%
\sectionfont{\setstretch{.85}\Semiboldsubheadfont}%
\oldmainmatter}

\let\oldbackmatter\backmatter
\renewcommand{\backmatter}{%
\chapterfont{\setstretch{.85}\normalfont\centering}%
\sectionfont{\setstretch{.85}\Semiboldsubheadfont}%
\oldbackmatter}
%
%
\begin{document}%
\normalsize%
\frontmatter%
\setlength{\parindent}{0cm}

\pagestyle{empty}

\maketitle

\blankpage%
\begin{center}

\vspace*{2.2em}

\halftitlepageTranslationTitle{Long Discourses}

\vspace*{1em}

\halftitlepageTranslationSubtitle{A faithful translation of the Dīgha Nikāya}

\vspace*{2em}

\halftitlepageFleuron{•}

\vspace*{2em}

\halftitlepageByline{translated and introduced by}

\vspace*{.5em}

\halftitlepageCreatorName{Bhikkhu Sujato}

\vspace*{4em}

\halftitlepageVolumeNumber{Volume 1}

\smallskip

\halftitlepageVolumeAcronym{DN 1–13}

\smallskip

\halftitlepageVolumeTranslationTitle{The Chapter on the Entire Spectrum of Ethics}

\smallskip

\halftitlepageVolumeRootTitle{Sīlakkhandhavagga}

\vspace*{\fill}

\sclogo{0}
 \halftitlepagePublisher{SuttaCentral}

\end{center}

\newpage
%
\setstretch{1.05}

\begin{footnotesize}

\textit{Long Discourses} is a translation of the Dīghanikāya by Bhikkhu Sujato.

\medskip

Creative Commons Zero (CC0)

To the extent possible under law, Bhikkhu Sujato has waived all copyright and related or neighboring rights to \textit{Long Discourses}.

\medskip

This work is published from Australia.

\begin{center}
\textit{This translation is an expression of an ancient spiritual text that has been passed down by the Buddhist tradition for the benefit of all sentient beings. It is dedicated to the public domain via Creative Commons Zero (CC0). You are encouraged to copy, reproduce, adapt, alter, or otherwise make use of this translation. The translator respectfully requests that any use be in accordance with the values and principles of the Buddhist community.}
\end{center}

\medskip

\begin{description}
    \item[Web publication date] 2018
    \item[This edition] 2022-11-30 08:48:21
    \item[Publication type] paperback
    \item[Edition] ed5
    \item[Number of volumes] 3
    \item[Publication ISBN] 978-1-76132-052-1
    \item[Publication URL] https://suttacentral.net/editions/dn/en/sujato
    \item[Source URL] https://github.com/suttacentral/bilara-data/tree/published/translation/en/sujato/sutta/dn
    \item[Publication number] scpub2
\end{description}

\medskip

Published by SuttaCentral

\medskip

\textit{SuttaCentral,\\
c/o Alwis \& Alwis Pty Ltd\\
Kaurna Country,\\
Suite 12,\\
198 Greenhill Road,\\
Eastwood,\\
SA 5063,\\
Australia}

\end{footnotesize}

\newpage

\setlength{\parindent}{1.5em}%%
\newpage

\vspace*{\fill}

\begin{center}
\epigraph{These are the principles—deep, hard to see, hard to understand, peaceful, sublime, beyond the scope of logic, subtle, comprehensible to the astute—which the Realized One makes known after realizing them with his own insight.}
{
\epigraphTranslatedTitle{“The Prime Net”}
\epigraphRootTitle{\textsanskrit{Brahmajālasutta}}
\epigraphReference{\textsanskrit{Dīgha} \textsanskrit{Nikāya} 1}
}
\end{center}

\vspace*{2in}

\vspace*{\fill}

\blankpage%

\setlength{\parindent}{1em}
%
\tableofcontents
\newpage
\pagestyle{fancy}
%
\chapter*{The SuttaCentral Editions Series}
\addcontentsline{toc}{chapter}{The SuttaCentral Editions Series}
\markboth{The SuttaCentral Editions Series}{The SuttaCentral Editions Series}

Since 2005 SuttaCentral has provided access to the texts, translations, and parallels of early Buddhist texts. In 2018 we started creating and publishing our own translations of these seminal spiritual classics. The “Editions” series now makes selected translations available as books in various forms, including print, PDF, and EPUB.

Editions are selected from our most complete, well-crafted, and reliable translations. They aim to bring these texts to a wider audience in forms that reward mindful reading. Care is taken with every detail of the production, and we aim to meet or exceed professional best standards in every way. These are the core scriptures underlying the entire Buddhist tradition, and we believe that they deserve to be preserved and made available in highest quality without compromise.

SuttaCentral is a charitable organization. Our work is accomplished by volunteers and through the generosity of our donors. Everything we create is offered to all of humanity free of any copyright or licensing restrictions. 

%
\chapter*{Preface to \emph{Long Discourses}}
\addcontentsline{toc}{chapter}{Preface to \emph{Long Discourses}}
\markboth{Preface to \emph{Long Discourses}}{Preface to \emph{Long Discourses}}

I grew up in Perth, Western Australia. It’s a city that is often described as “nice”, a somewhat backhanded compliment. The weather is bright and sunny, it’s safe and prosperous, life is good. But it’s not a place where anything particularly happens. Certainly not anything meaningful or interesting to anyone outside of Perth.

As a musician, I would sing songs about New York, about Paris, about Memphis or Singapore or even Darlinghurst. I didn’t know those places, but I knew that they were meaningful places, places deserving of a song. My own life, by contrast, seemed entirely on the surface. The bright sun and clear skies of Perth had no poetry, it banished all the shadows, everything was just so bland. There was nothing to sing about.

You’re sensing a plot twist coming up, and you’re right. In those days—the early 80s—the Perth indie music scene produced its finest band, the Triffids. The singer Dave McComb wrote about things that had happened to me: “he swam out to the edge of the reef, there were cuts along his skin.” I knew what that was like, not because someone had told me, but because I’d done it myself. Suddenly I was living in a world of meaning. I realized that my place, and therefore my life, was just as real and just as meaningful as anything else. The bleaching light of Perth was its meaning, the lack of shadows was its shadow.

When I came to Buddhism, it all seemed so exotic, so distant. I was made to chant in this strange language “Pali”, which I’d never even heard of. It took me a while to even realize that Pali really was an actual language, not just a mystical invocation. The monks I met were strange and incomprehensible: who would choose such a life? It had a depth that made my own paltry life pale in comparison.

As I began to study Buddhism in depth, grappling with deep matters, I discovered a range of other scholars and practitioners to learn from. There were the meditation masters of the Thai forest tradition in which I had ordained—Ajahn Chah, Ajahn Mun, Ajahn Thate, Ajahn Lee Dhammadharo. I grew to find sustenance also in the great scholar-monks of the modern Theravada—Venerables \textsanskrit{Ñāṇatiloka}, \textsanskrit{Ñāṇapoṇika}, \textsanskrit{Ñāṇamoḷī}, \textsanskrit{Guṇaratana}, Bodhi, Narada, \textsanskrit{Kaṭukurunde} \textsanskrit{Ñāṇananda}, Buddhadasa, and many others. I struggled to learn the broader history and nature of the Buddhist schools and traditions from scholars such as I.B. Horner, T.W. Rhys Davids, A.K. Warder and Étienne Lamotte. The knowledge and understanding of all these people seemed so lofty, so confident and capable. I devoured everything I could get my hands on.

It never really occurred to me that I might have something to add. I could hardly even manage to master the basics. The masters of the Buddhist tradition appeared as peerless savants, holders of an ancient and impenetrable wisdom.

If you’re sensing another plot twist, you’re right again. Around 1994 I was still a young resident at Wat Nanachat in north-east Thailand when we received a guest, an elderly English gentleman who introduced himself as Maurice Walshe. Of course I knew that name very well: he had translated the \textsanskrit{Dīgha} \textsanskrit{Nikāya}. I was so excited to meet one of my heroes. He was a charming and witty man, and it was an honor for me to meet him and spend some time together. I am always grateful to him because he made me realize that the Buddhist tradition was created and formed by ordinary people. He had studied Pali, but did not regard himself as an accomplished scholar. He undertook the translation at the behest of Venerable Ānandamaitreya—another figure of legend for me. Maurice was very humble about his abilities and his achievement. And it was no false modesty; while his translation was eminently readable, it was not especially accurate. But he did it. And in doing so, helped the Dhamma take one more step forward.

It was after meeting Rod Bucknell and John Kelly, the co-founders of SuttaCentral, in 2004, that I started making my own contributions to the Dhamma through SuttaCentral. Modest as they were, I realized that my talents and skills could help others, as I had been helped. It took a long while, much learning and many trials, but eventually I dared imagine that maybe I could make my own contribution to the corpus of Pali translations. It would surely be imperfect and inadequate, but perhaps I had something to give.

Those of us who have enjoyed the sweet taste of the Dhamma owe a debt of gratitude to all those who have made it possible; to all the teachers, the supporters, the donors, the monks and nuns and layfolk, the scholars, the meditators, the builders and cooks and plumbers and weavers, the artists and storytellers, the repairers of leaky roofs and the kindlers of lamps. There is not a single one who has the capacity to hold the whole tradition. But I believe that there is not a single one who has nothing to offer.

Allow me to indulge in a further recollection of my days in the indie music scene. One of the songs that has stuck with me is \textit{Song of the Siren}, written by Tim Buckley, but known from the version by This Mortal Coil. In three short verses it tells the story of the protagonist lost on “shipless oceans”, who was drawn in and given shelter by one they came to love. Just as they thought they were safe, the beloved seemed to turn away, leaving them “broken lovelorn on your rocks”. Despairing and confused, they considered ending it all. Until at last, they realized: now it was their turn. They could not live forever relying on the other to offer shelter and protection. When they were lost, they had been saved, and now they called to the other, “swim to me, let me enfold you”.

As a person of faith, I believe that the Buddha was a perfected human being. The Buddhist tradition, on the other hand, is made up of people who are usually notably imperfect. Sometimes we feel inspiration and uplift, other times disappointment or disillusionment. I reached a point of frustration when I knew that, for all the efforts of many people, we were still not able to make all the Suttas available in translation for free. It seemed wrong, and I didn’t know what to do. It was then that I realized that it was my turn to offer shelter.

%
\chapter*{A Reader’s Guide to the Pali Suttas}
\addcontentsline{toc}{chapter}{A Reader’s Guide to the Pali Suttas}
\markboth{A Reader’s Guide to the Pali Suttas}{A Reader’s Guide to the Pali Suttas}

\scbyline{Bhikkhu Sujato, 2019}

The suttas of the Pali Canon (\textsanskrit{Tipiṭaka}), especially the four main \textit{\textsanskrit{nikāyas}}, are essential reading for anyone who wishes to understand the Buddha and his teaching. They have been preserved and passed down in the Pali language by the \textsanskrit{Theravāda} tradition of Buddhism as the word of the Buddha.

These texts were originally passed down orally, by generations of monks and nuns who memorized them and recited them together. Around 30 BCE they were written down in the \textsanskrit{Āluvihāra} in Sri Lanka, and subsequently were transmitted in manuscripts of palm leaves.

From the 19th century, the manuscripts were edited and published as modern editions in sets of books. In addition, the Pali text was translated into a number of modern languages, including Thai, Burmese, Sinhalese, and English.

The word \textsanskrit{Tipiṭaka} means “Three Baskets”. The Basket of Discourses is traditionally listed as the second of the three. The four \textit{\textsanskrit{nikāyas}} make up the bulk of the Basket of Discourses. Here is how they are situated within the canon as a whole.

\begin{itemize}%
\item Vinaya \textsanskrit{Piṭaka} (Basket of Monastic Law)%
\item Sutta \textsanskrit{Piṭaka} (Basket of Discourses)
\begin{itemize}%
\item \textbf{\textsanskrit{Dīgha} \textsanskrit{Nikāya}} (Long Discourses)%
\item \textbf{Majjhima \textsanskrit{Nikāya}} (Middle Discourses)%
\item \textbf{\textsanskrit{Saṁyutta} \textsanskrit{Nikāya}} (Linked Discourses)%
\item \textbf{\textsanskrit{Aṅguttara} \textsanskrit{Nikāya}} (Numbered Discourses)%
\item Khuddaka \textsanskrit{Nikāya} (Minor Discourses)%
\end{itemize}

%
\item Abhidhamma \textsanskrit{Piṭaka} (Basket of Systematic Treatises)%
\end{itemize}

Similar collections are found in ancient Chinese translations, and substantial portions of them are also in Sanskrit and Tibetan. The diverse collections of scriptures arose among the Buddhist communities who spread across greater India in the centuries following the Buddha, especially under the Buddhist emperor, Ashoka. These missions are documented in the ancient chronicles of Sri Lanka as well as the Vinaya commentaries in Pali and Chinese, and have been partially corroborated by modern archaeology.

SuttaCentral hosts almost all of these texts and provides comprehensive parallels showing the relations between them. A comparative understanding based on the full spectrum of these texts is essential for any study of early Buddhism. The Chinese Buddhist canon, in particular, contains a vast amount of translations of early texts, and in terms of quantity it outweighs the Pali texts by some margin.

For many reasons, though, the Pali texts will always retain a special place for those who wish to understand what the Buddha taught.

\begin{itemize}%
\item They are the only complete set of scriptures of an early school of Buddhism.%
\item They are by far the largest body of texts to survive in an early Indic dialect.%
\item They are accompanied by an extensive and detailed set of ancient commentaries (\textit{\textsanskrit{aṭṭhakathā}}).%
\item They are, for the most part, linguistically clear, well-edited, and readily comprehensible.%
\end{itemize}

Moreover, the Pali texts are the core scriptures of a living tradition, the \textsanskrit{Theravāda} school found in Sri Lanka, Thailand, Myanmar, Bangladesh, Cambodia, Laos, India, China, and Vietnam. To this day they are recited, taught, studied, and practiced daily, and are regarded in the traditions as being a reliable witness to the teachings of the Buddha himself.

Among the Pali texts, it is the four \textit{\textsanskrit{nikāyas}} that command the most attention. It is here that we find extensive and definitive explanations of Buddhist teachings, as well as the living personality of the Buddha and his immediate disciples.

By comparison, the Vinaya \textsanskrit{Piṭaka} details the life of the monastic communities, and in addition it reveals much about the history and social background; but it contains only a few teaching passages. The Abhidhamma \textsanskrit{Piṭaka} is made up of systematic treatises that were composed in the centuries following the Buddha’s passing. And the books of the Khuddaka \textsanskrit{Nikāya} are very mixed. There are six fairly short books that are supplements to the main four \textit{\textsanskrit{nikāyas}}, mostly in verse: the \href{dhp}{Dhammapada}, \href{ud}{Udāna}, \href{iti}{Itivuttaka}, \href{snp}{Sutta Nipāta}, \href{thag}{Theragāthā}, and \href{thig}{Therīgāthā}. However, most of the other books in the Khuddaka are later, and represent a phase of Buddhism a few centuries after the Buddha.

\section*{About These Guides}

I have prepared these guides in order to support a student who wishes to develop a deeper understanding of the \textit{\textsanskrit{nikāyas}}. They accompany my translations of the four \textit{\textsanskrit{nikāyas}} as found on SuttaCentral. This general guide is meant to be read first, as it covers a variety of issues that are common to all the \textit{\textsanskrit{nikāyas}}. The four \textit{\textsanskrit{nikāyas}} are a highly unified body of texts, sharing most of the significant doctrinal passages. The general information presented here is fleshed out in individual essays on each of the four \textit{\textsanskrit{nikāyas}}, which highlight the shifts in emphasis and orientation from one collection to the next. These may be read in any order. While the guides for the specific \textit{\textsanskrit{nikāyas}} naturally focus on the texts in that \textit{\textsanskrit{nikāya}}, this is not adhered to rigidly, and they may refer also to passages found elsewhere.

Summaries of major doctrinal themes may be found mostly in the \textit{\textsanskrit{nikāya}} guides, especially that for the \textsanskrit{Saṁyutta} \textsanskrit{Nikāya}, rather than here. However, I would urge a degree of caution when it comes to summaries, including my own. The true joy of the suttas is in the undigested teachings, in that raw moment when the Buddha encounters a person in suffering and helps them, not by giving them a digested abstract, but by reaching out to them as people. Summaries and surveys are best treated as starting points for discovery, rather than as definitive treatises.

I almost completely avoid sideways glances at the various Chinese and other parallels. Understanding these relations is critical, and the entire basis of SuttaCentral is founded on this fact. But the number of texts is very large, and the complexity of the subject is daunting. I fear that if I were to deal with parallels in any kind of depth, these essays would never be completed; and if they were, they would have become unreadably complex. Hence I have set myself a more manageable scope, sticking to the Pali texts, on the understanding that most things probably apply to the parallels as well. The reader can easily check the parallels on SuttaCentral if they wish.

Among students of the suttas, the names of these collections are often abbreviated to “\textsanskrit{Dīgha}”, “Majjhima”, and so on, just as the word “Sutta” is often omitted from sutta titles. Strictly speaking, it’d be best to use the Pali title when referring to the original text, and the translated title when referring to the translation; but this distinction is often overlooked.

\section*{An Approachable Translation}

In 2015 I determined to create freely available translations of the main Pali discourses, so that all of these teachings might be made freely available in a clear, consistent, and accurate rendition. My aim was to translate the four main \textit{\textsanskrit{nikāyas}} as well as the 6 early books of the Khuddaka \textsanskrit{Nikāya}: \textsanskrit{Theragāthā}, \textsanskrit{Therīgāthā}, \textsanskrit{Udāna}, Itivuttaka, Dhammapada, and \textsanskrit{Suttanipāta}. I did this so that these astonishing works of ancient spiritual insight might enjoy the wider audience they so richly deserve.

In considering my translation style, I reflected on the standard trope that introduces the prose suttas: a person “approaches” the Buddha to ask a question or hear a teaching. It’s one of those passages that became so standard that we usually just pass it by. But it is no small thing to “approach” a spiritual teacher. It takes time, effort, curiosity, and courage; many of those people would have been more than a little nervous.

How, then, would the Buddha respond when approached? Would he have been archaic and obscure? Would he use words in odd, alienating ways? Would you need to have another monk by your side, whispering notes into your ear every second sentence—“He said this; but what he really meant was…”?

I think not. I think that the Buddha would have spoken clearly, kindly, and with no more complication than was necessary. I think that he would have respected the effort that people made to “approach” his teachings, and he would have tried the best he could, given the limitations of language and comprehension, to explain the Dhamma so that people could understand it.

An approachable translation expresses the meaning of the text in a manner that is simple, friendly, and idiomatic. It should not just be technically correct, it should sound like something someone might actually say.

Which means that it should strive to dispense entirely with the formalisms, technicalities, and Indic idioms that has dominated Buddhist translations, into which English has been coerced by translators who were writing for Indologists, linguists, and Buddhist philosophers. Such translations are a death by a thousand papercuts; with each obscurity the reader is distanced, taken out of the text, pushed into a mode of acting on the text, rather than being drawn into it.

That is not how those who listened to the Buddha would have experienced it. They were not being annoyed by the grit of dubious diction, nor were they being constantly nagged to check the footnotes. They were drawn inwards and upwards, fully experiencing the transformative power of the Dhamma as it came to life in the words of the Awakened One. We cannot hope to recapture this experience fully; but at least we can try to not make things worse than they need to be.

At each step of the way I asked myself, “Would an ordinary person, with little or no understanding of Buddhism, be able to read this and understand what it is actually saying?” To this end, I have favored the simpler word over the more complex; the direct phrasing rather than the oblique; the active voice rather than the passive; the informal rather than the formal; and the explicit rather than the implicit.

Still, it should not be thought that these are loose adaptations or simplifications. There is a place for re-imaginings of ancient texts, and for versions that strip the complexity to focus on the main point. But my work is intended as a full and accurate translation, one that omits nothing of substance. It is just that I try to express this without undue complexity.

I still feel I am a long way from achieving my goal. No-one is more aware than the translator of the compromises and losses along the way. Consistency, clarity, correctness, and beauty all make their competing claims, and only rarely, it seems, can all be met. It is a work in progress, and I will probably be making corrections and adjustments for many years to come.

I have been especially influenced in this approach by my fellow monks, Ajahn Brahm and Ajahn Brahmali. It is from Ajahn Brahm that I have learned the virtue of plain English; of the kindness of speaking such that people actually understand. And with Ajahn Brahmali, who has been working on Vinaya translations at the same time, I have had many illuminating discussions about the meaning of various words and phrases. He said one thing that stuck in my mind: a translation should mean \emph{something}. Even if you’re not sure what the text means, we can be sure that it had some meaning, so to translate it based purely on lexical correspondences is to not really translate it at all. Say what you think the text means, and if you make a mistake, fix it.

\section*{Plan Your Route or Wander in the Garden}

The Buddha’s teaching is a graduated one, leading from simple principles to profound realizations. This pattern is found within almost all of the discourses in one way or another. However it does \emph{not} apply to the collections of discourses. From collection to collection or discourse to discourse there is no graduation in difficulty, no build up of assumed knowledge in the student.

On the contrary, the \textsanskrit{Dīgha} \textsanskrit{Nikāya} begins with the Brahmajala Sutta, while the Majjhima \textsanskrit{Nikāya} begins with the \textsanskrit{Mūlapariyāya} Sutta, both of which are among the most profound and difficult discourses in the whole canon. A student who dives in unwarily will suddenly find themselves in very deep waters indeed.

If we wish to build up knowledge step by step, we can’t rely on simply reading the suttas in order. Students often find it helpful to use a structured reading guide such as that offered here. On SuttaCentral, we offer several other approaches.

Having said which, there’s nothing wrong with simply plunging in at random, so long as you don’t expect everything to make sense at first. Take your time and enjoy wandering about. Don’t worry too much about things that seem odd or unexpected. Usually you’ll find that obscure or difficult ideas are explained somewhere else; discovering those unexpected connections is one of the joys of reading the suttas.

In these introductory essays, you will find many references to the suttas. You don’t need to look up each reference to understand the essays. But if you do, you will get a reasonable survey of many important texts, and learn how to find the passage that you need. I suggest reading each essay on its own first, and then a second time, looking up and reading the sutta references as you go.

\section*{Looking Up References}

When you delve into sutta reading, you’ll notice that texts and passages are referenced in sometimes confusing ways. On SuttaCentral we employ a simple and widely adopted form of semantic referencing. By “semantic” references, we mean that the reference numbers are based on meaningful divisions in the texts themselves.

For the four \textit{\textsanskrit{nikāyas}}, this means:

\begin{itemize}%
\item In the \textsanskrit{Dīgha} and Majjhima, texts are referenced by the ID for the collection (DN and MN respectively) and the sutta number, counted in a simple sequence from the beginning.%
\item In the \textsanskrit{Saṁyutta}, the collection ID (SN), \textit{\textsanskrit{saṁyutta}} (thematic group), and sutta.%
\item In the \textsanskrit{Aṅguttara}, the collection ID (AN), \textit{\textsanskrit{nipāta}} (numerical group), and sutta.%
\end{itemize}

More granular referencing is provided by section numbers. These follow pre-existing conventions:

\begin{itemize}%
\item In the \textsanskrit{Dīgha}, the section numbers of the PTS Pali edition, which have been widely adopted in translations.%
\item In the Majjhima, the section numbers introduced by Bhikkhu \textsanskrit{Ñāṇamoḷi}, and used in Bhikkhu Bodhi’s edition.%
\item In the \textsanskrit{Saṁyutta} and \textsanskrit{Aṅguttara}, the paragraph numbers as found in the \textsanskrit{Mahāsaṅgīti} Pali text.%
\end{itemize}

Each of these is further subdivided so that each section contains a number of “segments”, a short piece of text usually about a sentence or so long. In my translations, the segments are matched with the underlying Pali text.

In our system, the numbers following a colon represent the section and segment numbers, that is, the subdivisions within a sutta. So, for example:

\begin{itemize}%
\item DN 3 means “the third discourse in the collection of Long Discourses (\textsanskrit{Dīgha} \textsanskrit{Nikāya})”.%
\item MN 43:3 means “discourse 43, section 3 in the collection of Middle Discourses (Majjhima \textsanskrit{Nikāya})”.%
\item MN 43:3.7 means “discourse 43, section 3, segment 7 in the collection of Middle Discourses (Majjhima \textsanskrit{Nikāya})”.%
\item SN 12.2:6.2 means “discourse 2, section 6, segment 2 in the 12th \textit{\textsanskrit{saṁyutta}} of the collection of Linked Discourses (\textsanskrit{Saṁyutta} \textsanskrit{Nikāya})”.%
\end{itemize}

You may encounter various other referencing systems. In academic works, texts are often referenced by volume and page of the Pali Text Society (PTS) edition of the original Pali. This is a regrettable and clumsy convention, since it binds references to a specific paper edition. I hope it is swiftly abandoned in favor of proper semantic references. However, the PTS volume/page numbers are displayed on SuttaCentral in case you need to look up a legacy reference.

Traditionally, the texts were—and often still are—referenced the long way: by \textit{\textsanskrit{nikāya}}, then \textit{\textsanskrit{saṁyutta}} or \textit{\textsanskrit{nipāta}} and/or \textit{\textsanskrit{paṇṇasaka}} (as applicable), then \textit{vagga} and \textit{sutta}. This system is helpful when using manuscripts, as you can narrow your search step by step through the manuscript to find what you need. On the web, or even in books, however, it is unnecessary. Nevertheless, you can use the traditional navigation structure in our sidebar if you wish.

\section*{Elements of Structure}

As students of Buddhist texts we are interested in the content, in learning what the Buddha and his disciples had to say and how they lived. However, due to the manner in which the texts are arranged, we quickly discover that it’s not easy to know how different texts relate to each other. So while it may seem dry, it is worth spending a little time to consider the \emph{structure} of the texts.

Early Buddhist texts were organized, not for reading, but for oral recitation and memorization. The overriding concern was to divide the texts into chunks that could be memorized and recited together. Since the texts were preserved in memory, they were largely “random access”: a skilled student could instantly recall a passage from anywhere in the texts, without having to flip through the pages or look up an index. In this way, the earliest system of organization is a little similar to how we find information today through a search engine.

It follows from this that we cannot expect early Buddhist texts to be structured sequentially like a modern book. But this does not mean that the collections are random or chaotic. They follow their own logic, which can be discerned if the texts are approached sympathetically.

Here are some of the structural or formal elements you will encounter in the early Buddhist texts.

\subsection*{Imagery and Narrative}

The suttas frequently employ an ABA pattern. A statement is made; a simile is given; and the statement is repeated.

This formal pattern is highly effective in reinforcing learning. First we get the basic idea. But abstract philosophical or psychological statements are hard to understand and remember without any context, so the Buddha illuminates his teaching with a simile. He ends by driving the message home once again.

The range of similes in the suttas is truly astounding. The Buddha had an uncanny ability to effortlessly summon an apt comparison from anything that he saw around him. The similes also convey a great deal of incidental detail regarding life and culture in the Buddha’s day, and, more importantly, they show how the Dhamma teaching makes sense in its context. Most of the classic Buddhist images that are familiar today trace their roots to similes used by the Buddha in the early texts.

Sometimes the similes are extended to a brief parable or fable. Curiously enough, however, we rarely see the Buddha engage in story-telling of any length.

Where narratives are developed in some detail, they are typically as part of the background story (\textit{\textsanskrit{nidāna}}) rather than in the Buddha’s teaching as such. It is an elementary principle of historical scholarship that the background story is of a somewhat later date than the main doctrinal material. Such stories vary considerably in the parallels, showing that the traditions treated narrative more flexibly than doctrine.

\subsection*{Repetition}

It won’t take long before you notice that the suttas tend to be repetitious. \emph{Very} repetitious. This can be a major hurdle for a new reader, so let’s take a little time to consider what is happening.

Like so many patterns found in nature, the repetitions are \emph{fractal}. That is, they occur at every level: the word, the phrase, the sentence, the paragraph, the passage, the whole text, even the group of texts. This shows that the repetition is not something alien to these texts, not something forced on them by an over-zealous editor, but is intrinsic from the beginning.

But why? The thing to remember is that the texts were formed in an oral tradition. And in an oral tradition, repetition works very differently than it does in writing. When you \emph{read} a repetition, it can be annoying; it feels like a waste of time, and you want to skip over it. But when \emph{reciting}, repetition has exactly the opposite effect. It is soothing and relaxing. The parts that are different take more work, you have to exercise your memory; but when the repeated passage comes around—like the chorus of a song—you relax into the flow of the chanting. Repetitions give the reciter space to be at ease and contemplate. Reciting a highly repetitive text becomes a form of meditation, where you reflect the meaning and apply it to your experience as you recite.

But in addition to this spiritual aspect, repetition has a definite practical purpose: preservation. By saying the same thing again and again, identically or with small variations, the reciters were constantly checking their memories, ensuring the accuracy of the texts. And if a text was lost, there is always another similar passage somewhere else. Thus the repetitions ensured the long-term survival of the Dhamma by creating backups of important information in multiple places, retained in the minds of Buddhist practitioners.

Understanding the historical role of repetitions, however, doesn’t help us when we just want to read a sutta. What are we to do? Well, there is no single way to read a sutta. Some people prefer to read them in full, contemplating each repetition. Others read them more briefly, getting to the important point. You’ll figure out a way that suits you. But when you understand the role of repetitions, hopefully you will not find them such an obstruction.

\subsection*{Abbreviation}

The flipside of repetition is abbreviation. Since the repetitions are so abundant, they are often abbreviated. Such abbreviation is not a modern invention; it is found throughout the manuscripts, and indeed there is no edition that fully spells out all the repetitions. The Pali texts have their own convention for indicating abbreviations, marked with the syllable \textit{pe}, itself an abbreviation of \textit{\textsanskrit{peyyāla}}.

Generally speaking, the abbreviations in the Pali editions, and the occasional instructions on how to spell out the full text, are sensible and fairly consistent between editions. Modern translations follow the manuscript tradition, but not slavishly. Sometimes the translation will spell out abbreviated passages, or else abbreviate passages spelled out in the original.

Abbreviations are both “internal” or “external”. By internal abbreviation, I mean that there is enough information in the text itself to fully reconstruct it. Typically only the first and last items in a list are spelled out in full, and for the rest, only the key terms are mentioned. Here is an example from \href{/sn22.137}{SN 22.137}:

\begin{quotation}%
Form is impermanent; you should give up desire for it. Feeling … Perception … Choices … Consciousness is impermanent; you should give up desire for it.

%
\end{quotation}

In external abbreviations, an abbreviated passage cannot be fully reconstructed from the context, but requires looking up another text to fill in the blanks. This is another example of how the oral tradition differs from written texts. A reciter would obviously know, say, the formula for the four noble truths, so there is no need to write it every time; just enough to jog the memory. But in modern editions, especially on the web, a reader can access a specific text from anywhere, and may never have encountered the abbreviated passage before. For this reason I tried to reduce the number of external abbreviations in my translations.

\subsection*{Titles}

Buddhist manuscripts rarely have titles at the start like modern texts. Rather, the title is recorded at the end. In modern editions, these titles have been added at the start for clarity.

In many cases, especially in the titles of suttas and \textit{vaggas}, what we actually have in the manuscript is not really a title as such, but a key word found in the summary verse (\textit{\textsanskrit{uddāna}}) found at the end of a \textit{vagga} or other division. These verses were inserted by the redactors of the canon in order to help keep the contents organized, much like a Table of Contents. However, the summary verses do vary to some extent between editions, so the titles of suttas are not always consistent. In addition, some suttas are assigned more than one title in the text itself—for example \href{/dn1}{DN 1} \textit{The Prime Net} (\textit{Brahmajalasutta})—or there are spelling variations. So take care, for it is quite common to find different titles for the same text.

\subsection*{Textual Divisions}

\subsubsection*{Vagga (“Chapter”)}

The \textit{vagga} is the most widespread and distinctive structural unit in early Buddhist texts. It usually consists of ten texts, which may be ten discourses, ten verses, ten rules, and so on. The number ten is adhered to fairly consistently, although on occasion a \textit{vagga} may contain more or less than ten.

The \textit{vagga} is often little more than a convenient grouping to help organize the discourses neatly. In such cases, a \textit{vagga} is usually just named after its first discourse.

However, it is also common to find that a \textit{vagga} collects discourses with a loose thematic unity. For example, in the \href{dn{-}silakkhandhavagga}{Chapter on the Full Spectrum of Ethics} (\textit{\textsanskrit{Sīlakkhandhavagga}}) of the \textsanskrit{Dīgha} \textsanskrit{Nikāya} (thirteen discourses in this case), almost all the texts deal with the “gradual training” of ethics, meditation, and wisdom.

In some cases, a \textit{vagga} in Pali may parallel a similar \textit{vagga} in another language. For example, the famous \href{snp{-}atthakavagga}{Chapter of the Eights} (\textit{\textsanskrit{Aṭṭhakavagga}}) of the Sutta \textsanskrit{Nipāta} exists in Chinese translation, though the Sutta \textsanskrit{Nipāta} as a whole does not. Similarly, the \textit{\textsanskrit{Sīlakkhandhavagga}} of the \textsanskrit{Dīgha} \textsanskrit{Nikāya} has parallels in both the Dharmaguptaka (Chinese) and \textsanskrit{Sarvāstivāda} (Sanskrit) texts of the \textsanskrit{Dīgha}.

Occasionally the word \textit{vagga} is applied to a larger textual unit, one that includes a number of sections, each of which composed of “little” \textit{vaggas}. Examples of such nested \textit{vagga} structures include the \textsanskrit{Saṁyutta} \textsanskrit{Nikāya} and the Khandhakas of the Pali Vinaya.

\subsubsection*{\textsanskrit{Paṇṇāsaka} (“Fifty”)}

The word \textit{\textsanskrit{paṇṇāsa}} means “fifty”, and a \textit{\textsanskrit{paṇṇāsaka}} is a group of approximately fifty suttas, or five \textit{vaggas}. It is used to organize collections that contain many \textit{vaggas}. Most of the collections with large numbers of discourses use this structure, for example the “\href{mn{-}mulapannasa}{Root Fifty}” of the Majjhima \textsanskrit{Nikāya} (\textit{\textsanskrit{Mūlapaṇṇāsa}}). The \textit{\textsanskrit{paṇṇāsaka}} is purely for convenience and does not correspond to any meaningful division of the text.

\subsubsection*{\textsanskrit{Nipāta} (“Group”)}

The literal meaning of \textit{\textsanskrit{nipāta}} is “fallen down”, and it is a generic term for texts that have been placed together. In the \textsanskrit{Aṅguttara}, it is used for each division of texts gathered together by number: the \href{an1}{group of discourses consisting of one item}, and so on. Elsewhere it is used, for example, in the title of the \href{snp}{Sutta Nipāta}, the “Group of Discourses”.

\subsubsection*{\textsanskrit{Saṁyutta} (“Collection of Linked Discourses”)}

Whereas the \textit{\textsanskrit{nipāta}} is quite generic, the \textit{\textsanskrit{saṁyutta}} has a more specific meaning: texts collected according to a similar theme or subject matter. The \textsanskrit{Saṁyutta} \textsanskrit{Nikāya} consists of 56 such collections. For example, the fourteenth \textit{\textsanskrit{saṁyutta}} contains \href{sn14}{39 discourses on the topic of the elements}.

\subsubsection*{\textsanskrit{Nikāya} or Āgama (“Division”)}

The largest structural unit, usually known as \textit{\textsanskrit{nikāya}} in the Pali tradition of the \textsanskrit{Theravāda}, and as \textit{\textsanskrit{āgama}} in the northern traditions. However, the term \textit{\textsanskrit{āgama}}, while it has fallen into disuse in modern \textsanskrit{Theravāda}, is found quite commonly in the Pali commentaries.

Collections similar to the four \textit{\textsanskrit{nikāyas}} as found in the Pali are found in all the other schools. However, while the overall nature of the collections is similar, and they are organized in similar ways, the detailed content varies considerably. It frequently happens that a sutta found in the Majjhima of one school, for example, may be found in the \textsanskrit{Saṁyutta} or \textsanskrit{Dīgha} of another school. In addition, the internal sequence of texts is quite different. Thus it seems that the \textit{\textsanskrit{nikāyas}} or \textit{\textsanskrit{āgamas}} functioned more as organizational guidelines than as fixed units.

The fifth Pali \textit{\textsanskrit{nikāya}}, the Khuddaka \textsanskrit{Nikāya}, is more flexible and varies more between traditions. It seems it originated as a place for collecting verses and minor texts not gathered elsewhere. However the Pali collection became a handy place to include later texts, so it has now become the biggest of all the \textit{\textsanskrit{nikāyas}}. While there are occasional references to a similar collection in the northern schools, none exist in that form today. Nevertheless, many of the individual texts of the Khuddaka have parallels, especially the Dhammapada, which survives in many different editions.

\section*{In the Buddha’s Day: A Time of Change}

Each discourse begins with a brief statement saying that at “one time” the Buddha was staying at a particular place. In this way the redactors of the texts were concerned to locate the Buddha and his teachings in a specific historical and cultural context. Modern scholars have been able to reconstruct a fairly reliable picture of the Buddha’s life and times, relying on the early Buddhist texts, as well as what information may be gleaned from Brahmanical and Jaina scriptures.

Archeology is, unfortunately, of limited use, as there are few archeological remnants from the Buddha’s day. In fact, before the time of Ashoka—perhaps 150 years after the Buddha—there are very few remains at all of ancient India, until the time of the Indic Valley civilization, many centuries earlier. For the period we are interested in, what has been found consists of some pottery and similar small implements, as well as a few remnants of fortifications around \textsanskrit{Kosambī}. The paucity of evidence is due to two main reasons. The first is that buildings at the time were mostly of wood or other perishable materials. And the second is that archeological work in India has been very spotty and incomplete.

The Buddha lived in the 5th century BCE in the Ganges plain in northern India. The exact dates of his birth and death are uncertain, but modern scholarly opinion tends to place his birth around 480 BCE and his death 80 years later at around 400 BCE. He was born in Lumbini and grew up in Kapilavatthu, both of which belong to the Sakyan republic, straddling the modern border of India and Nepal. His family name was Gotama; the earliest texts do not mention his personal name, but tradition says it was Siddhattha.

After his Awakening, the Buddha traveled about the Gangetic plain. The area he traversed was part of the cultural/political region known as the “sixteen nations” (\textit{janapada}). This spanned from modern Delhi to the north-west, the Bangladesh border to the east, the Himalayan foothills to the north, the Deccan to the south, and Ujjain to the south-west. Most of his time was spent around the cities of \textsanskrit{Sāvatthī} in the kingdom of Kosala and \textsanskrit{Rājagaha} in the kingdom of Magadha. Despite the proliferation of local legends in most Buddhist countries, the Buddha never ventured outside this area.

It was not just the Buddha who was restricted to this region. It seems that trade and other close cultural contacts normally took place within this region, too. Occasional references to places further afield—southern India or the Greeks—were vague and often legendary. It was in the century after the Buddha passed away that the kingdoms of northern India were unified and regular international trade routes were opened, first to Europe, and, a couple of centuries later, to south-east Asia and China.

\section*{Economics and Politics}

Though cities and urban life feature prominently in the texts, they are still on a small scale. The economy was largely rural, with farming playing a prominent role.

However there are lists of occupations in the Pali canon that show a diverse range of employment—accountants, jewelers, builders, soldiers, doctors, government officials, and many more.

The rise in diversity of employment was linked to the growth of cities, which in turn is associated with the appearance of new technologies. The archeological record, though thin, has furnished us with records of two significant innovations: forges for iron, and a kind of fine pottery known as Northern Black Polished Ware. These new developments attest to a growing mastery in the industrial use of fire, something that the suttas mention in several memorable similes.

Technological innovation drove economic growth. We frequently hear of wealthy individuals, employing large staffs and managing properties or businesses. There was enough economic surplus to support a large class of spiritual seekers. Such ascetics made no material contributions to society; their value lay in spiritual and ethical development.

These technological and economic shifts were mirrored in the political sphere. There were two major forms of governance. Traditional clans such as the Sakyans or the Vajjians followed an ancient restricted form of democracy, where decisions were made in a town council, and the clan elected a leading member as temporary ruler. Other nations, like Kosala and Magadha, had formed a more familiar kind of kingdom, with an absolute hereditary monarch. While the Buddha evidently favored the democratic ideals under which he grew up, and after which he modeled the governance of the Sangha, it was the kingdoms that were growing in economic and military dominance. During the Buddha’s lifetime, there were repeated skirmishes between Kosala and Magadha, vying for dominion over the ancient sacred city of Varanasi.

Of even greater significance, towards the end of his life, Magadha was announcing its intentions to invade the Vajjian republic. History attests to the success of this policy: in the decades following the Buddha’s death, Magadha conquered virtually all of the sixteen nations, establishing an unquestioned supremacy over the region, and establishing pan-Indian trade networks. So powerful was the resulting kingdom that Alexander the Great’s troops rebelled at the mere rumor of Magadhan military might.

\section*{Social Life}

The growing complexity of economic and political life required corresponding changes in social roles and responsibilities. Like any society in a time of change, people in the time of the Buddha were trying to balance their traditional values and structures with the new realities. It seems that people were for the most part reasonably well off. Still, poverty and famine, along with injustice, banditry, and economic uncertainty, were present and posed a very real threat. We hear frequent laments about how unpredictable wealth is, whether the older forms of wealth in cattle and land, or the newer forms in money and career.

The Buddha was not a social revolutionary and did not overtly agitate for an overthrow of social systems, even those he recognized as unjust. Typically he would argue for a more just and fair implementation of existing norms. For example, rather than saying all societies should be democratic, he spoke of the moral duty of a king to look after his people.

A man was expected to earn a living so as to maintain and protect his family. He should use his earnings to provide his family with both essentials and luxuries, and to treat workers with kindness and decency, while not neglecting to assign some funds for savings, and some for donations to charity.

A woman’s traditional role was to marry and bear children. Aside from this, her options were limited. We rarely hear of professional women aside from sex workers. In this context, the opportunity to become a nun allowed women to pursue their own spiritual and intellectual development, to act as leaders and teachers, and to be respected and supported in that role.

Many of the more extreme manifestations of sexual discrimination are not found in the early texts. We find no mention of child brides, immolation of widows, or the essential subjugation of women to men.

India had not yet developed a full-fledged caste system. But there was a much simpler fourfold division of society:

\begin{description}%
\item[Aristocrats (\textit{khattiya})] Owners of land (\textit{khetta}), typically wealthy and powerful, engaged in statecraft, war, and production. The Buddha was from an aristocratic clan. The aristocrats placed themselves at the highest tier of society.%
\item[Brahmins (\textit{\textsanskrit{brāhmaṇa}})] Members of a hereditary priestly class. The brahmins were custodians of sacred texts called “Vedas”, and performed rituals and ceremonies to their various deities. However by the time of the Buddha many brahmins were simply engaged in ordinary worldly livelihoods and their religious role was secondary. They believed themselves to be the children of God (\textsanskrit{Brahmā}).%
\item[Merchants (\textit{vessa}):] Engaged in trade and commerce.%
\item[Workers (\textit{sudda})] Performed physical labor.%
\end{description}

Not everyone fit into this neat scheme. We hear reference to outcastes and various tribal peoples. In addition, there were slaves or bonded servants. Finally, the ascetics (\textit{\textsanskrit{samaṇa}}) such as the Buddha saw themselves as having left behind all such notions of caste.

\section*{The Many Spiritual Paths of Ancient India}

Change in the Buddha’s day was not limited to the worldly sphere. The religious life of ancient India was equally dynamic. For this reason it would be a mistake to assume that India in the time of the Buddha was primarily a Hindu society. Some of the elements that make up modern Hinduism may be discerned, but Indian religion, like spiritual and religious practice everywhere, has always been in a state of flux and evolution.

In the time of the Buddha, and indeed even to this day, the ancient pre-Buddhist Vedas formed the basis for the spiritual life of the brahmins and those who followed them. Rituals such as the \textit{agnihotra}, the daily pouring of ghee onto the fire as an offering to the fire-god Agni—originated long before the Aryan people even came to India, and continue to be performed today.

Nevertheless, many of the old gods featured in the Vedas had vanished by the time of the Buddha, and many of the famed deities of later Hinduism had not yet appeared. Those who do appear take on a markedly different aspect; prominent gods such as Vishnu (Pali: \textit{\textsanskrit{Veṇhu}}) or Shiva (Pali: \textit{Siva}) appear in minor roles, and a warrior like Sakka (AKA Indra) appears as an apostle of peace. There were no temples, no images, and no cult of devotion (\textit{bhakti}). There is no mention of a system of \textit{avatars}, nor of familiar concepts from modern Hindu-inspired spirituality such as \textit{\textsanskrit{śakti}}, \textit{\textsanskrit{kuṇḍalinī}}, chakras, or yoga exercises.

Moreover, when we look at the aspects of modern Hinduism that were present at the time, many of them are completely separate from each other. No-one considered, for example, the worship of a local dragon (\textit{\textsanskrit{nāga}}) to have anything to do with the rites of the brahmins. The outstanding feature of Hinduism—the great synthesis of religious and philosophical ideas and practices, from simple animism to profound non-dualism—had not yet been undertaken. Different strands of religious life were quite distinct and were not considered to be part of the same path.

Thus historians do not refer to the brahmanical religion of the time as Hinduism, but rather as Vedism or Brahmanism. It was nearly a thousand years later that the movement recognizable as modern Hinduism became prominent in India. To be sure, much of Hinduism is drawn from the Vedas, in the same way that much of Catholicism is drawn from the Hebrew scriptures that Christians call the Old Testament. But were you to meet Abraham or Noah and address them as “Catholics”, they would not know what you are talking about. And the Indians of the Buddha’s time would have been equally unfamiliar with the very idea of “Hinduism”.

All this notwithstanding, there is an oft-repeated claim to the effect that the Buddha “was born, lived, and died a Hindu”, attributed to the great pioneer of Indology, Thomas Rhys Davids. While it is true that he did write this, it was in an early work, page 116 of \textit{Buddhism: its history and literature}, a lecture series published in 1896. But by 1912 his views had changed, for on page 83 of \textit{Buddhism: Being a sketch of the life and teachings of Gautama, the Buddha}, he said:

\begin{quotation}%
Gautama was born, and brought up, and lived, and died a typical Indian. Hinduism had not yet, in his time, arisen.

%
\end{quotation}

Rhys Davids emphasizes that the Buddha did not have an antagonistic relation to the Brahmanical religion. His purpose was not religious reform, but freedom from suffering. However, on page 85 of the same work he comments:

\begin{quotation}%
In the long run the two systems were quite incompatible. … Gautama’s whole training lay indeed outside of the ritualistic lore of the Brahmanas and the brahmins. The local deities of his clan were not Vedic.

%
\end{quotation}

The lesson here is that we must avoid reading modern conditions back into ancient times. The peoples of ancient India had their own rich, complex, and many-faceted spiritual lives. We can only begin to understand them, and to understand how the Buddha related to them, when we set aside our modern preconceptions and preoccupations and listen to what they had to say for themselves.

An outstanding feature of early Buddhist texts is interreligious dialogue. The Buddha did not live in a Buddhist culture. We frequently encounter the Buddha and his disciples discussing various aspects of spiritual philosophy and practice with followers of other spiritual paths, or with people of no particular path. Sometimes they would come to the Buddha seeking to learn or even to attack. And it is not uncommon to find the Buddha and his disciples actively seeking out followers of other spiritual paths simply to engage in conversation. In this, the early texts are quite different from later Buddhist literature, which almost always consists of Buddhists speaking with other Buddhists.

While many of these people ended up declaring themselves followers of the Buddha, this was not the purpose of the dialogue. The Buddha did not debate simply to win an argument, but out of compassion, to help alleviate suffering.

Amid the complex sets of religious practices, we may discern three major domains.

\subsection*{Animism}

In the villages and towns of ancient India, the most widespread folk religion was a belief in the omnipresent reality of spirits in nature. Such deities might embody aspects of the weather, or live in plants or rivers or caves; they might promote abundance, or take ferocious and threatening forms. They were unreliable, but could be wooed through simple offerings of rice, flowers, or ghee.

Animist beliefs were derived from local legends and rituals, not from religious philosophy. However, the higher religious paths such as Buddhism or Jainism, rather than repressing such beliefs, were happy to assign them a minor role in the scheme of things, so long as they eliminated harmful practices like human or animal sacrifice.

Throughout the Buddhist texts, we hear of \textit{yakkhas} (spirits), \textit{\textsanskrit{nāgas}} (dragons), \textit{gandhabbas} (fairies), \textit{garudas} (phoenixes), and many more. The Buddhist attitude towards such beings might best be described as “good neighborliness”. Neither they, nor any higher beings, are worshiped or looked to for salvation. Rather, they are treated with respect and dignity, for who knows? If they are real, it would be good to have them on your side.

\subsection*{Brahmanism}

The caste who called themselves “brahmins” inherited an ancient body of sacred lore known as the Vedas. This consisted of sets of oral scriptures, among which the Ṛg Veda was primary. In the early Buddhist texts there are three Vedas: Ṛg, \textsanskrit{Sāma}, and Yajur; the Atharva is mentioned, but was not yet considered to be a Veda.

The Ṛg Veda grew out of the cultural and religious milieu of the ancient Indo-European peoples. It shares a common heritage with the Avestan texts of Iranian Zoroastrianism, and more distantly, the mythologies of Europe.

It seems that Indo-European peoples moved into India around a millennium before the Buddha, with distinct clans maintaining sets of sacred lore. In the early centuries of the first millennium BCE, in the area known as the Kuru country (modern Delhi), the clans were unified into the classical brahmanical kingdom whose story is echoed in the \textsanskrit{Mahābharata}. The Ṛg Veda was forged from the books of the clans, wrapped in opening and closing chapters emphasizing unity. By the time of the Buddha, the brahmanical culture and language had already become strongly established in the regions further south and east where the Buddha lived.

The brahmins insisted on the holiness of their caste, the efficacy of their rituals, the truth of their scriptures, and the omnipotence of their deity. The Buddha rejected all these claims out of hand.

However, Brahmanical traditions were far from a unified monolith. We see a strong strand of questioning of tradition, of seeking out new ways, of earnest seeking of the truth; and such attitudes are just as strong in the Brahmanical texts as the Buddhist.

Brahmins were typically family men, living a settled life, and expecting a degree of social respect and standing due to their learning and caste. But some brahmins had adopted an ascetic lifestyle, apparently influenced by the \textit{\textsanskrit{samaṇas}}.

In the generations preceding the Buddha, brahmanical philosophy had reached a peak in the \textsanskrit{Upaniṣads}, with their sophisticated debates and mystic philosophy of the essential unity of self and cosmos. These texts form the immediate dialectical context of the Buddha’s philosophy. \textsanskrit{Yājñavalkya}, a key \textsanskrit{Upaniṣadic} philosopher, lived around Mithila, in the same region traversed by the Buddha, no more than a century or two before him. Some early \textsanskrit{Upaniṣads} are apparently referred to in \href{/dn13}{DN 13} \textit{The Three Knowledges} (\textit{Tevijjasutta}), and the \textsanskrit{Upaniṣadic} doctrine of “self” (\textit{\textsanskrit{ātman}}) was prominently rejected by the Buddha in his most distinctive teaching: not-self (\textit{\textsanskrit{anattā}}).

\subsection*{The \textsanskrit{Samaṇas}}

Quite distinct from the brahmins, and often in opposition to them, was a complex set of religious movements known as the \textit{\textsanskrit{samaṇas}} or “ascetics”. Six prominent ascetic schools were acknowledged in the time of the Buddha. The Buddha counted himself as an ascetic, too, in view of the many similarities between his own movement and theirs.

Like the Buddhist mendicants, the other \textit{\textsanskrit{samaṇas}} were typically celibate renunciates, living either in solitude or in monastic communities, and relying on alms for food. The most famous movement—and the only one to survive until today—was Jainism, which flourished under their leader \textsanskrit{Mahāvīra}, known as \textsanskrit{Nigaṇṭha} \textsanskrit{Nātaputta} in the Buddhist texts.

The ascetics shared an iconoclastic attitude, and all rejected the brahmanical system \textit{in toto}. However they varied amongst each other, as shown in their teachings attested at \href{/dn2}{DN 2} \textit{The Fruits of the Ascetic Life} (\textit{\textsanskrit{Sāmaññaphalasutta}}) as well as \href{/mn60}{MN 60} and \href{/mn76}{MN 76}. Some emphasized austerities and self-mortification, others rationality and debate. Some advocated ardent effort, others a resigned fatalism. Some taught rebirth, while others asserted that this material world was the only reality.

While their doctrines may appear florid and obscure, and their practices sometimes bizarre and pointless, the ascetic movements are a lasting testament to the diversity, vigor, and innovation of religious life in ancient India.

\section*{Cosmology}

A recurring theme in many of the religious strands of India is a concern for cosmology. A religious philosophy was expected to paint a picture of the world on a large scale and indicate humanity’s role within it. Like all aspects of religious life, such cosmologies were partly shared across traditions and in part were unique to each tradition.

Some traditions asserted a materialist cosmology, rejecting the notion that one would be reborn in any other state, and asserting that only this life was real.

Most cosmologies, however, posited multiple realms of existence. Beings would come and go from these different stations. Some were pleasant and desirable, while others were not. As to why this was so, different reasons were given.

\begin{itemize}%
\item Some ascetics argued that beings transmigrated due to destiny or chance.%
\item Mainstream Brahmanical traditions said it was due to the performance of rituals and sacrifices to the gods.%
\item Some said that rebirth was determined by intentional actions, whether moral or immoral.%
\end{itemize}

The latter view was held by some ascetic schools, such as Buddhism and Jainism, and some of the more advanced and innovative threads of Brahmanism. These traditions shared a conception of transmigration that in many ways is quite similar. Three common elements can be discerned:

\begin{enumerate}%
\item All sentient beings are reborn countless times in process called \textit{\textsanskrit{saṁsāra}} (“transmigration”).%
\item This process is driven by ethical choices (\textit{kamma}). Good deeds lead to a pleasant rebirth; bad deeds lead to a painful rebirth.%
\item True salvation is not found in any realm of existence, but only in liberation from transmigration itself.%
\end{enumerate}

While these aspects of the cosmology were shared, the details differed in both theory and practice.

Jain and Brahmanical theory proposed that transmigration was undergone by a soul or self which could attain freedom. For the Jains, the individual soul (\textit{\textsanskrit{jīva}}) attains eternal purity and bliss. For the most sophisticated among the brahmins, the individual self (\textit{\textsanskrit{ātman}}) realizes its true nature as identical with the divinity that is the cosmos itself (Sanskrit: \textit{tat tvam asi}; Pali: \textit{eso hasmasmi}; \textit{so \textsanskrit{attā} so loko}).

The Buddha rejected all such metaphysical notions of self or soul. Instead, he explained transmigration as an ongoing process of changing conditions, formulated as the famous twelve links of dependent origination (\textit{\textsanskrit{paṭicca} \textsanskrit{samuppāda}}).

In the practical application of their theory, Jains believed that the way to salvation was to firstly avoid harming any sentient beings, even unintentionally, and then to burn off past \textit{kamma} through painful self-mortification. Such practices are described frequently and in detail, attesting to their prominence in early Indian spiritual life.

The brahmins, as seen in the suttas, did not have such a clear and unambiguous path to a highest goal, and indeed are depicted as arguing among themselves as to the correct path. This reflects the historical situation, where the earlier, simpler, and more worldly goals of Vedic Brahmanism were growing into a more sophisticated \textsanskrit{Upaniṣadic} form.

For the \textsanskrit{Upaniṣads}, the key to salvation was understanding. It is only one who understands the rituals and philosophies correctly (\textit{ya evam veda}) who will experience their full benefit. In the centuries prior to the Buddha, this path of wisdom had developed into a profound contemplative culture, expressed in the ecstatic and mystical passages of the \textsanskrit{Upaniṣads}.

The Buddha shared the Jain concern for avoiding harm, but rejected the practice of extreme austerities. Rather than bodily torment, he emphasized mental development.

Certain Brahmanical lineages had developed meditation to a high degree, but meditative states were still conceived in metaphysical terms. The Buddha adopted such meditations for their value in purifying the mind, but interpreted them in purely psychological terms, rejecting metaphysics entirely.

One of the benefits of advanced meditation was that a practitioner would develop the ability to perceive many past lives and many realms into which beings may be reborn. In this, we may distinguish between the core doctrinal texts, which typically speak of rebirth in general terms as good or bad destinies, and the narrative portions, which depicted the realms of rebirth in terms familiar from popular Indian cosmology.

The early texts do not attempt to systematize these realms in any great detail. Indeed, the various deities and realms mentioned defy any simple categorization. Later Buddhism developed a theory of various realms, sometimes called the 31 planes of existence, but this does not fully represent the situation as found in the early texts.

Here is a general overview of the most important realms found in the suttas. It is crucial to remember that, in the Buddhist view, all of these, even the most high, are impermanent and do not constitute true freedom. They are not separate metaphysical planes, but mere stations in which consciousness may spend some time during its long journey.

\begin{description}%
\item[\textsanskrit{Brahmā} realms] The highest heavens, which correspond to attainments of absorption meditation (\textit{\textsanskrit{jhāna}}), and may only be attained by \textit{\textsanskrit{jhāna}} practitioners. The \textsanskrit{Brahmā} realms include the realms of luminous form (\textit{\textsanskrit{rūpaloka}}) and the formless realms (\textit{\textsanskrit{arūpaloka}}). The former are attained by means of the four primary absorptions. In this context, the word “form” refers to the refined and radiant echo or reflection of the original meditation subject upon which these states are based. The formless states lie beyond this, and are realized when even that subtle luminosity disappears.%
\item[Heavens of sensual pleasures] Many of these are mentioned, most commonly the realm of the Thirty-Three, governed by Sakka. Various beings from Indian animist beliefs are said to inhabit the lower tiers of these realms.%
\item[Human realm] The most important realm, where Buddhas appear and the spiritual path is taught.%
\item[Lower realms] These include the animal realm, the ghost realm, and the hells. The realm of the \textit{asuras}—titans or demons—is usually placed here in the later cosmologies, but the early texts seem to treat it as one of the heavens.%
\end{description}

The Buddha taught that doing good and avoiding bad was the path to rebirth in one of the fortunate realms, which include the human realm and all higher realms. However, the course of transmigration is long and unpredictable, so no heaven realm provides a sure refuge.

Far from teaching rebirth as a solace for naive followers unable to face the inevitability of death, rebirth is depicted in traumatic and terrifying terms: the tears that one has shed in the endless course of transmigration are greater than all the waters in all the oceans of the world. Thus the true significance of doing good deeds is not merely to get a better rebirth, but to lay the foundations for higher spiritual development, primarily through meditation.

In the core teaching of the four noble truths, the origin of all suffering is traced to the craving that is connected with rebirth (\textit{\textsanskrit{yāyaṁ} \textsanskrit{taṇhā} \textsanskrit{ponobbhavikā}}). The practice of the noble eightfold path is the only thing that enables one to let go of that craving and be free of suffering. This is what the Buddha called “extinguishment” or “quenching” (Pali: \textit{\textsanskrit{nibbāna}}; Sanskrit \textit{\textsanskrit{nirvāṇa}}).

\section*{On the Pali Commentaries}

The Pali canonical texts are accompanied by an extensive and detailed set of commentaries (\textit{\textsanskrit{aṭṭhakathā}}) and subcommentaries (\textit{\textsanskrit{ṭīkā}}). These texts are, for most people, even more mysterious than the canon itself, so let me say a few words on them.

The main commentaries were compiled by the monk Buddhaghosa at the \textsanskrit{Mahāvihāra} monastery in Anuradhapura, then the capital of Sri Lanka, in the 5th century. Buddhaghosa inherited a series of older commentaries in the old Sinhalese language, now lost. These had been compiled over the centuries in Sri Lanka, mostly between around 200 BCE–200 CE; that is to say, the main content of the commentaries was closed several centuries before Buddhaghosa.

It was all a bit messy, with text in Pali and commentaries in Sinhala, and a variety of different commentarial texts. Buddhaghosa aimed to streamline the situation by combining all the old commentaries into a single system, translated into Pali.

Buddhaghosa’s work remains as an extraordinary accomplishment of traditional scholarship. He had an almost preternatural mastery of his materials, and the clarity and rigor of his writings make light work of what must have been an exceedingly difficult task. It is crucial to remember that he saw his work as that of an editor, compiler, and translator. That is what he claimed to be doing, and from everything we know about his work, he was a scholar of integrity who did exactly what he said. When he felt a need to express his own opinion he said so; but such interventions were rare and hesitant. The commentaries are the record of discussions and explanations of the Pali texts handed down in the \textsanskrit{Mahāvihāra} tradition, not the opinions of Buddhaghosa.

While Buddhaghosa compiled commentaries on the major texts, he left some incomplete. It is not always certain which commentaries were by him; but in any case later scholars completed his work. Subsequently, subcommentaries were written to clarify obscure points in the commentaries.

In modern \textsanskrit{Theravāda}, the commentaries have become a sadly and unnecessarily divisive issue. Some people take the entire tradition uncritically and regard the commentaries as essentially infallible. Others flip to an extreme of suspecting anything in the commentaries, rewriting \textsanskrit{Theravādin} history as a conspiracy of the commentaries. But any serious scholar knows that the commentaries are often helpful, even indispensable, on countless difficult and obscure points. Without them, there is no way we would have been able to create the accurate dictionaries and translations that we have today. Yet they cannot be relied on blindly, for, like any resource, they are fallible, and must be read with a careful and critical eye. On some doctrinal issues, the position of the commentaries had shifted considerably from the stance in the suttas, and not in illuminating ways.

I once read some advice from a Burmese Sayadaw—I am afraid this was many years ago and I have forgotten who it was—on how to use the commentaries. He said—and I paraphrase—something like this. First read the sutta. Try to understand it. Read it and meditate on it again and again. If there’s anything you don’t understand, see if it can be explained elsewhere in the suttas. If, at the end of the day, you still cannot understand it, check the commentary. If it answers the question, good. But if, after equally careful study, the commentary is still unclear, then check the subcommentary.

This has always seemed like sound advice to me, and I have tried to follow it. The purpose of the commentary is to help explain the suttas. Where the suttas are clear—and mostly they are—there is no need to refer to the commentary. The only extra thing I would add is that, in addition to the commentaries and subcommentaries, we now also have Chinese and Sanskrit parallels to help us understand difficult passages.

In these guides, I almost completely leave aside the commentarial explanations. In several places the explanations I have given differ from those in the commentaries. I am aware of this, and have written on most of these things elsewhere, but I do not want to burden the guides by re-litigating every controversy. I don’t contradict the commentaries out of ignorance or stubbornness, but because after many years of study, contemplation, discussion, and practice, I have come to see some things differently.

\section*{A Brief and Incomplete Textual History}

The significance of the \textit{\textsanskrit{nikāyas}} was recognized by European scholars early on. I will discuss specifics of the editions and translations in the essays on the individual \textit{\textsanskrit{nikāyas}}, and here offer some general remarks.

During the 19th century, European scholars became aware of the Pali tradition, seeing in it a reliable source of information for the Buddha, his times and his teachings. An English civil servant in Sri Lanka, Thomas Rhys Davids, learned Pali from the monks, initially to help him better understand Sri Lankan legal practices. Recognizing the significance of these texts, he returned to England and established the Pali Text Society (PTS), largely funded by Asian donors. They obtained palm-leaf manuscripts, on the basis of which the PTS prepared print editions of the main Pali texts.

The PTS editions introduced a number of ideas from European scholarship. Most obviously, they used a set of conventions for presenting Indic scripts with European letters. This system is lossless, so texts may be automatically changed from one script to another. It enables easy comparison between the editions of the Pali canon from different countries, which traditionally had been written in diverse local scripts. They also introduced titles at the start of texts, punctuation and capitalization, page numbers, footnotes, variant readings, and various other modern innovations.

One innovation that was not pursued consistently was the use of chapter and section numbers. These were added to the PTS Pali editions of the \textsanskrit{Dīgha} \textsanskrit{Nikāya} and the Vinaya, and are used in subsequent translations. However most of the PTS editions lack such sections, with the unfortunate consequence that academic referencing of Pali texts is still based on the volume and page of the PTS edition, a system that is neither practical nor precise.

The PTS editions were ground-breaking and have exerted an unparalleled influence on modern Buddhism, both east and west. Asian scholars have been well aware of them, with the consequence that it is probably hard to find any printed edition from the 20th century that is completely free of their influence. Nevertheless, the PTS texts are not particularly reliable. They were put together over a considerable period of time, with scant resources and few workers. The editors used whatever manuscripts they had to hand, and, apart from a general preference for Sri Lankan readings, it is hard to discern a consistent or clear methodology in their choices of readings. The limitations of these editions are well known among experts in the field, and in some cases updated and improved editions have been published.

For my translation of the \textit{\textsanskrit{nikāyas}}, I preferred to use the \textsanskrit{Mahāsaṅgīti} edition. This is essentially a digital representation of the Burmese textual tradition of the 6th Council, itself based on the 19th century 5th Council text. It is based on the digital edition prepared by the Vipassana Research Institute, with extensive proofreading and corrections by the Dhamma Society of Bangkok. The \textsanskrit{Mahāsaṅgīti} is a consistent and carefully edited digital text, and for that reason was chosen as the root Pali text for SuttaCentral. But it should not be assumed that it is the most authentic. On the contrary, it preserves the Burmese readings, which tend to correct the text in conformity with the Pali grammars. Nevertheless, in almost all such cases there is no difference in the meaning, just minor differences in spelling.

Like most translators, when editions vary I did not adhere to one edition, but simply selected what seems to be the best reading in each case. I referred to the PTS editions fairly often. More rarely, I consulted the romanized Buddha Jayanthi edition found on GRETIL; note, however, that the digital edition is widely regarded as being inferior to the original in Sinhalese script. Occasionally I also consulted the Rama 5 edition in Thai script. I also consulted previous translations, especially those of Bhikkhu Bodhi.

In problematic cases I cross-checked the Pali against the Sanskrit and Chinese parallels; I did not make use of Tibetan sources. However in every case the overriding intention was to accurately represent the Pali text. Only in a very few exceptional cases did I rely on the Sanskrit or Chinese parallels to correct the Pali.

%
\chapter*{The Long Discourses: Dhamma as literature and compilation}
\addcontentsline{toc}{chapter}{The Long Discourses: Dhamma as literature and compilation}
\markboth{The Long Discourses: Dhamma as literature and compilation}{The Long Discourses: Dhamma as literature and compilation}

\scbyline{Bhikkhu Sujato, 2019}

The \textsanskrit{Dīgha} \textsanskrit{Nikāya} is the first of the four main divisions in the Sutta \textsanskrit{Piṭaka} of the Pali Canon (\textit{\textsanskrit{tipiṭaka}}). It is translated here as \textit{Long Discourses}. As the title suggests, its discourses are somewhat longer than those of other \textit{\textsanskrit{nikāyas}}. There are, however, only 34 discourses in the collection, so despite the length of the individual discourses, the collection as a whole is the shortest of the \textit{\textsanskrit{nikāyas}}.

It is distinguished from the other \textit{\textsanskrit{nikāyas}} by its more developed and elaborate literary forms. Outgrowing the bare and direct style of most of the early texts, here the extra length offers space for narratives and doctrinal expositions to find a fuller expression. This is an early hint at how the literary form of Buddhist texts was to develop in later years, moving towards expansiveness and abundance.

It is no coincidence that these elaborate texts are often addressed to the brahmins, who were the self-proclaimed spiritual leaders of the time. The brahmins were the custodians of the most sophisticated texts in ancient India up to this time, the Vedic literature. It seems that one aim of the \textsanskrit{Dīgha} was to impress such learned men. These discourses offer a wide range of examples of how the Buddha related to those of other religious paths.

Another overriding theme of the \textsanskrit{Dīgha} is the passing away of the Buddha. The centerpiece of the collection is \href{https://suttacentral.net/dn16}{DN 16}, \textit{The Great Discourse on the Buddha’s Extinguishment} (\textit{\textsanskrit{Mahāparinibbānasutta}}), a discourse of unrivaled importance. This presents the last journey of the Buddha, wandering in unhurried stages from town to town, each step bringing him closer to his passing. In the very length of the text, recording so many details of the journey, we can sense a longing to draw out those last precious days as far as possible.

\section*{How the \textsanskrit{Dīgha} is Organized}

The 34 discourses are grouped in three \textit{vaggas}. The first \textit{vagga} consists of thirteen discourses, each of which includes a lengthy passage on the spiritual practice of a monastic, known as the Gradual Training (\textit{\textsanskrit{anupubbasikkhā}}).

In the second \textit{vagga} we find a number of discourses of a more biographical nature. \href{https://suttacentral.net/dn14}{DN 14} \textit{The Great Discourse on the Harvest of Deeds} (\textit{\textsanskrit{Mahāpadānasutta}}) tells of past Buddhas, while \href{https://suttacentral.net/dn16}{DN 16} \textit{\textsanskrit{Mahāparinibbāna}} tells of Gotama’s last days. In addition, a number of other discourses in this section are closely related to the \textit{\textsanskrit{Mahāparinibbāna}}. I will discuss this cycle further below.

The final \textit{vagga} is more miscellaneous. It includes long poetic sections, doctrinal compilations—some of which are precursors to the Abhidhamma—and narratives that are often humorous and occasionally border on farce.

As usual in the \textit{\textsanskrit{nikāyas}}, there is no overall sequence of the teaching and many details of organization appear quite arbitrary. Still, we can discern a purpose in the arrangement of a few of the major discourses. These details are unique to the Theravadin tradition, so should be seen as reflecting their concerns, rather than the fundamental principles of the \textsanskrit{Dīgha}.

The first discourse, \href{https://suttacentral.net/dn1}{DN 1} \textit{The Prime Net} (\textit{Brahmajalasutta}), sets out a scheme of wrong views, and thus acts as a filter for the Dhamma, screening out possible misinterpretations. It seems that this arrangement was connected with the events of the so-called “Third Council” under King Ashoka, at a time when the \textsanskrit{Saṅgha} was overrun with imposters who were not genuine Buddhists. The second discourse, \href{https://suttacentral.net/dn2}{DN 2} \textit{The Fruits of the Ascetic Life} (\textit{\textsanskrit{Sāmaññaphalasutta}}), addresses a fundamental question: why do people follow a life of renunciation? In answering this, it sets forth the Gradual Training, a distinctively Buddhist path to peace.

The middle of the collection is dominated by discourses that deal in one way or another with the cosmic significance of the Buddha (\href{https://suttacentral.net/dn14}{DN 14}, \href{https://suttacentral.net/dn16}{DN 16}, \href{https://suttacentral.net/dn17}{DN 17}, \href{https://suttacentral.net/dn18}{DN 18}, \href{https://suttacentral.net/dn19}{DN 19}, \href{https://suttacentral.net/dn20}{DN 20}, \href{https://suttacentral.net/dn21}{DN 21}; to these may be added \href{https://suttacentral.net/dn26}{DN 26}, \href{https://suttacentral.net/dn27}{DN 27}, \href{https://suttacentral.net/dn30}{DN 30}, and \href{https://suttacentral.net/dn32}{DN 32}). Where the biographical texts of the Majjhima emphasize the practical and the personal, the specifics of how \emph{our} Buddha lived, these discourses exist in an arena of mythic grandeur. Time and space are expanded as the poignant and personal details of the \textit{\textsanskrit{Mahāparinibbāna}} are set among a series of mythological texts that show the potency of the Buddha and his teachings in the deep past, in the apocalyptic future, and in the present among the orders of gods.

The central event in all this is the death of the Buddha. Historically this was a traumatic crisis for the Buddhist community, and many feared that the Dhamma would not survive. By lifting attention from the present trauma and pointing to a longer meaning, these suttas show that the Dhamma need not die with the Buddha. The events of the \textit{\textsanskrit{Mahāparinibbāna}} spurred the \textsanskrit{Saṅgha} to hold the First Council, where the discourses were collected and organized to ensure their preservation. And these are, of course, the very scriptures that we are reading. In this way, these narratives tell the story of their own origin.

The \textsanskrit{Dīgha} finishes with mostly doctrinal compilations (\href{https://suttacentral.net/dn28}{DN 28}, \href{https://suttacentral.net/dn29}{DN 29}, \href{https://suttacentral.net/dn33}{DN 33}, \href{https://suttacentral.net/dn34}{DN 34}). If the beginning of the \textsanskrit{Dīgha} tells us \emph{why} the teachings matter and the middle tells us \emph{how} they came to be, the ending tells us \emph{what} they are. It is a rather curious thing that in the \textsanskrit{Dīgha}, many of the doctrines that we think of as fundamental to the Buddha’s teachings occur only rarely. These discourses rectify this situation, ensuring that the students of the \textsanskrit{Dīgha} had access to a wide range of teachings. The last two discourses, in particular, are clearly compiled as handy mnemonics for memorizing sets of doctrinal teachings.

\section*{The Gradual Training}

The Gradual Training sets out the steps taken by a Buddhist renunciate on their path. It begins with the arising of a Buddha in the world. Hearing the Buddha’s teaching, a person reflects on how best it can be applied to their own life. Realizing that “the household life is cramped and dirty, but the life of one gone forth is wide open”, they give up their worldly possessions and attachments, don the ochre robe of a Buddhist mendicant, and undertake a life of morality, simplicity, and meditation. Proceeding step by step to ever more advanced practices, they eventually enter into deep meditative absorption (\textit{\textsanskrit{jhāna}}) before realizing the four noble truths and finding true freedom.

The Gradual Training is an expansion of the threefold training (\textit{tisso \textsanskrit{sikkhā}}): ethics (\textit{\textsanskrit{sīla}}), meditative immersion (\textit{\textsanskrit{samādhi}}), and wisdom (\textit{\textsanskrit{paññā}}). At \href{https://suttacentral.net/an3.89}{AN 3.89} the three trainings are defined:

\begin{itemize}%
\item ethics (in a monastic context) requires keeping the monastic rules;%
\item meditative immersion is the four absorptions;%
\item wisdom is the understanding of the four noble truths.%
\end{itemize}

This teaching is distributed widely throughout the early Buddhist texts. In the \textsanskrit{Dīgha}, for example, it’s found in the \textit{\textsanskrit{Mahāparinibbāna}} as a standard teaching repeated by the Buddha at many of the stops on his journey. A series of shorter discourses on this subject may be found in the \textsanskrit{Samaṇa} Vagga of the \textsanskrit{Aṅguttara} (\href{https://suttacentral.net/an3{-}samanavagga}{AN 3.81–91}).

This brief overview of the path is explained more fully in the Gradual Training, which explains each of the three trainings in considerable detail. This longer exposition appears to have been the original teaching on the overall lifestyle, practices, and aims of the Buddha’s mendicant followers. It seems that the Buddha preferred to encourage his monastics by exhorting them to follow the highest ideals of conduct and meditation. Only reluctantly did he set up the legal system of the Vinaya texts, with its procedures and punishments.

The Gradual Training is found, in somewhat varying forms, in the Majjhima (\href{https://suttacentral.net/mn27}{MN 27}, \href{https://suttacentral.net/mn51}{MN 51}, \href{https://suttacentral.net/mn38}{MN 38}, \href{https://suttacentral.net/mn39}{MN 39}, \href{https://suttacentral.net/mn53}{MN 53}, \href{https://suttacentral.net/mn107}{MN 107}, \href{https://suttacentral.net/mn125}{MN 125}), the \textsanskrit{Aṅguttara} (\href{https://suttacentral.net/an4.198}{AN 4.198}, \href{https://suttacentral.net/an10.99}{AN 10.99}), and even the Abhidhamma (\href{https://suttacentral.net/vb12}{Vb 12}, \href{https://suttacentral.net/pp2.4\#114}{Pp 2.4:114}). Curiously enough, however, it is not found among the collected discourses on the path found in the last book of the \textsanskrit{Saṁyutta}. While virtually all of the practices of the Gradual Training are found in the \textsanskrit{Saṁyutta}, the overall framework is not found.

The \textsanskrit{Dīgha} makes up for this lack by placing a \textit{vagga} of thirteen discourses right at the start featuring the Gradual Training. This is called the \textsanskrit{Sīlakkhandhavagga}, the “Chapter on the Aggregate of Ethics”. Despite the title, however, these texts treat the full training on ethics, meditative immersion, and wisdom.

While the content is similar in each place that the Gradual Training appears, the \textsanskrit{Dīgha} versions feature a pronounced emphasis on beauty and pleasure. The stages of the path are illustrated by similes that are as lovely as they are apt, while each step of the path is said to be accompanied by a deepening sense of pleasure and happiness. The Gradual Training is not a path of suffering, but one of grace and joy and freedom.

Due to the repetition, the texts invariably abbreviate all the expositions except for the first two discourses. It should be remembered, however, that this is merely a consequence of how the Pali tradition arranged these texts. In the Sanskrit and Chinese \textsanskrit{Dīrghāgamas}, the texts in this section are arranged differently, and different suttas are either expanded or abbreviated accordingly.

While the focus is firmly on monastic life, the general principles hold good for everyone, and indeed at \href{https://suttacentral.net/mn53}{MN 53} \textit{A Trainee} (\textit{Sekhasutta}), Ānanda teaches essentially the same path to a lay audience. In the \textsanskrit{Sīlakkhandhavagga}, many discourses are in fact addressed to lay people, most of whom are brahmins.

The question of King \textsanskrit{Ajātasattu} in \href{https://suttacentral.net/dn2}{DN 2} provides the key to understanding why this is so. He points out that in worldly life, each trade can be seen to have its own benefit. But what is the benefit of the renunciate life? While other ascetics falter before this question, the Buddha presents the Gradual Training. He shows how the life gone forth is not one of pain and distress, nor one of delayed gratification, but one that shows real benefits in this life. It is about the power and transformative potential of inner development and meditation. In contrast, the household path offers only limited happiness, with much uncertainty and stress, while the paths of other ascetics are unclear, ineffective, or painful, and the brahmins can only offer rituals and prayers of dubious efficacy. Thus the Gradual Training explains why there is a need for the \textsanskrit{Saṅgha} at all.

Just as the Gradual Training is built from the kernel of the threefold training, the code of monastic ethics is built from the core principles of basic precepts. It is divided into three sections. The first section begins with the most fundamental precept for everyone in Buddhism: non-violence, to refrain from killing any creature, however small. It continues with items found in such common teachings as the five precepts and the ten paths of skillful action. But it adds items that specially pertain to monastic life, such as avoiding luxuries and ownership of property. The second section on ethics expands these specifically monastic and renunciate precepts in much greater detail, while the final section deals with right livelihood. A Buddhist monastic, who relies on alms food given in faith, should not make a living by other means, especially through superstitious and magical practices.

The Gradual Training builds on these ethical foundations as the mendicant undertakes a series of practices designed to quell the busyness and activity of the mind. They rein in their senses, avoiding things that are overly stimulating. They focus on remaining mindful and aware throughout all their activities. They aim at contentment, being satisfied with a few simple possessions.

Only when all these have been developed does the mendicant resort to seclusion for meditation. Going to the forest, they undertake mindfulness meditation and give up the five hindrances that prevent peace of mind. These hindrances are one of the core meditation teachings in the suttas, regarded as the key obstacle to absorption. They are:

\begin{description}%
\item[Sensual desire] Includes any kind of craving, greed, or desire for sensual experience. It includes powerful forms such as sexual desire as well as more subtle kinds of attachment.%
\item[Ill will] Anything from outright hatred to subtle forms of annoyance and aversion come under this hindrance.%
\item[Dullness and drowsiness] When the mind begins to settle down in meditation, it commonly becomes sleepy or dull.%
\item[Restlessness and remorse] Restlessness is always looking for some future experience, while remorse keeps digging up the past, especially moments of regret.%
\item[Doubt] It is normal and healthy to doubt when it comes to things that we do not know. But if we do not understand the elements of what is right and what is wrong, doubt will subtly undermine our meditation.%
\end{description}

Experiencing an ever-deepening peace and bliss, they ultimately enter a series of profoundly still states of meditative immersion known as the four absorptions (\textit{\textsanskrit{jhānas}}).

The absorptions are the fundamental meditation practice in early Buddhism and are essential to all stages of Awakening. They occur in many contexts, but it is here, in the Gradual Training, that they emerge most naturally from the life and practice undertaken by the mendicants. This context was so central to early Buddhists that when they compiled the early Abhidhamma text, the \textsanskrit{Vibhaṅga}, the chapter on Absorption begins with the Gradual Training. It is true, there are lay followers in the early texts who were said to have practiced absorption. But it is equally true that when the Buddha taught how to attain such profound peace, he emphasized the power of deep renunciation.

It has become common in certain modern forms of Buddhism to assert that absorptions are not an essential part of the path. Others say that, while important, the absorptions are relatively shallow states of concentration that may be easily attained on a short retreat. Suffice to say, neither of these views finds support in the early texts. The absorptions are essential, profound, and difficult to attain. Even with the full strength of renunciation, many mendicants in the Buddha’s day struggled to realize them. Nevertheless, it is a special quality of the Dhamma that each step along the path is accompanied by deepening peace and joy, and letting go gets easier the further one travels. This is what makes the realization of even such profound and subtle states possible.

Emerging from the absorptions, the mendicant harnesses the power of a deeply purified mind to realize a series of special forms of knowledge or insight. These culminate in the realization of the four noble truths:

\begin{enumerate}%
\item Suffering (\textit{dukkha}).%
\item The origin of suffering, i.e. craving (\textit{samudaya}).%
\item The cessation of suffering, i.e. \textsanskrit{Nibbāna} (\textit{nirodha}).%
\item The practice that leads to the end of suffering (\textit{magga}).%
\end{enumerate}

Suffering is the spur that drives us to undertake spiritual practice. Only when we have some experience of suffering will we look for an escape. And when encountering the Buddha’s teaching, a seeker recognizes that the Dhamma speaks to that which matters in their own life, offering a powerful and pragmatic solution. But wallowing in suffering gets you nowhere. When you understand that this suffering is real, but has causes and conditions that you can do something about, it sparks faith and the resolve to act. The path itself is one of unfolding happiness and receding pain; the truth of the ending of suffering is experienced at every step. This culminates in the experience of profound meditative stillness, called absorption (\textit{\textsanskrit{jhāna}}) or immersion (\textit{\textsanskrit{samādhi}}). In such states, having let go of sensual desire, the five external senses cease (\textit{vivicc’eva \textsanskrit{kāmehi}}) and the mind feels a peace and happiness unlike anything it has known before. Empowered by the clarity and brilliance of absorption, the reality of suffering and its cause becomes apparent. This signifies that one has realized the first stage of awakening, that of the stream-enterer (\textit{\textsanskrit{sotāpanna}}).

Stream-entry occurs when all the factors of the path—from the arousing of faith to the practice of absorption and deep insight—have been developed to a sufficient degree. At this point one has a profound insight into the nature of reality, letting go three of the ten fetters that bind a person to rebirth. In the Gradual Training the understanding of the four noble truths is usually followed by the understanding of the end of the defilements (\textit{\textsanskrit{āsava}}), which signifies the attainment of full perfection (\textit{\textsanskrit{arahattā}}). The remaining fetters are given up at this point, which is the final stage of the path: full awakening and freedom.

\section*{How to Build a Long Discourse}

There are over a thousand discourses recorded in each of the \textsanskrit{Aṅguttara} and the \textsanskrit{Saṁyutta} \textsanskrit{Nikāyas}, but only 34 long texts recorded in the \textsanskrit{Dīgha}. The relatively short texts of the \textsanskrit{Aṅguttara} and \textsanskrit{Saṁyutta} are reminiscent of the pre-Buddhist \textsanskrit{Upaniṣads}, especially the \textsanskrit{Bṛhadāraṇyaka} and Chandogya. These consist of a series of mostly independent passages, each episode covering no more than a few pages, and assembled into a much larger text. They are recollections of concise and focused teachings at certain times and places by certain people. It would seem, then, from the overwhelming majority of contemporary texts both Buddhist and Brahmanical, that the short discourse or dialogue was the standard format.

How, then, were these long texts constructed? Why? And for whom? Let us approach these questions by briefly considering a few different forms employed in the \textsanskrit{Dīgha}.

\subsection*{Inherently Complex Subjects}

Some discourses are long because the subject matter is inherently complex and demands a lengthy explanation. Of course, the Buddha was a master of presenting subjects in both pithy and detailed forms. Nevertheless, there are a few discourses whose subject matter requires an extensive treatment.

The most prominent example of this is the Gradual Training. In some cases—for example \href{https://suttacentral.net/dn6}{DN 6} \textit{With \textsanskrit{Mahāli}} and \href{https://suttacentral.net/dn7}{DN 7} \textit{With \textsanskrit{Jāliya}}—the discourse consists of little more than this passage, with a simple narrative background and some short extra teachings. So it seems that the presence of the long Gradual Training section was itself enough to qualify a discourse as “long”. Since this passage aims to provide a detailed guide to the whole of the renunciate spiritual life, from hearing the teaching to full awakening, the length is inherent in the subject matter. True, it is taught more briefly elsewhere, but even in those cases it tends to be somewhat long, and there was clearly a tendency to make it more inclusive.

In other cases the Gradual Training is taught in the middle of a discourse that is already quite extensive. Such is the case with \href{https://suttacentral.net/dn1}{DN 1} \textit{Brahmajala}, although here, uniquely, it is only the first section on Ethics that is taught. But the bulk of the text sets forth a network of 62 kinds of wrong view. Here, the nature of the subject matter is so extensive and complex that a shorter exposition would not do it justice. Indeed, when this teaching is mentioned in shorter discourses (\href{https://suttacentral.net/sn41.3}{SN 41.3}), it is not summarized, but the reader is referred rather to the full text.

\subsection*{Compilations}

Far more common than inherently lengthy teachings are the compilations. In such cases, a long text provides an occasion or background framework within which a series of short passages are collected. Such passages usually occur in identical or near-identical form in the \textsanskrit{Aṅguttara} or the \textsanskrit{Saṁyutta}, and occasionally the Majjhima. Compiling them here enables the reciters of the \textsanskrit{Dīgha} to learn a wide range of doctrines, and provides an essential backup, preserving the texts in case the shorter discourses are lost.

In a few instances, such short passages are not found elsewhere in exactly the same form. Whether that is because they were unique to the \textsanskrit{Dīgha}, or because the parallel passages have become lost, is hard to say.

How do we end up with parallel passages in so many different places?

Clearly the Buddha taught very often and, like all teachers, repeated his message many times. Such repeated teachings would have been collected in various places. This would be the case with important and generic teachings found throughout the Buddhist literature, like the four noble truths or the four absorptions.

In some cases, though, this is unlikely or impossible. For example, we sometimes find the exact same event on the same occasion—with the same teaching, the same location, and the same people—occurring in more than one text. In such instances it is clear that there is, in fact, just one passage, and it has been copied into two or more places.

Generally speaking, it is prudent to assume that such passages existed as short discourses before being collected into the larger forms. This is because, as noted above, the short discourse is the dominant form, and rests closest to the oral tradition. It is a principle observed everywhere through early Buddhist texts that the redactors preferred to add rather than subtract. Thus texts commonly become longer over time, and rarely shorter.

Examples of compilation are very common, and almost every discourse in the \textsanskrit{Dīgha} does this to some extent. Here are just a few examples.

\href{https://suttacentral.net/dn16}{DN 16} \textit{\textsanskrit{Mahāparinibbāna}} includes a wide range of collected passages. In some cases events pertinent to the narrative may have occurred there originally and been extracted later, while in other cases the included passages seem strangely extraneous to the context and were no doubt added in at some point.

Venerable \textsanskrit{Sāriputta} is said to be the main author of several such long compilations. He is the teacher in \href{https://suttacentral.net/dn33}{DN 33} \textit{Reciting in Concert} (\textit{\textsanskrit{Saṅgītisutta}}) and \href{https://suttacentral.net/dn34}{DN 34} \textit{Up to Ten} (\textit{Dasuttarasutta}), which consist almost entirely of short passages collected from elsewhere in the suttas and arranged by number. In \href{https://suttacentral.net/dn28}{DN 28} \textit{Inspiring Confidence} (\textit{\textsanskrit{Sampasādanīyasutta}}) he expresses his great faith in the Buddha, and cites a long series of passages to display the Buddha’s glory.

A more sophisticated form of compilation is found in \href{https://suttacentral.net/dn22}{DN 22} \textit{The Longer Discourse on Mindfulness Meditation} (\textit{\textsanskrit{Mahāsatipaṭṭhānasutta}}), the most important meditation discourse in 20th century \textsanskrit{Theravāda}. It gives a detailed account of the four kinds of mindfulness meditation. These are taught in brief in many places, but the details are found only here and at the mostly identical \href{https://suttacentral.net/mn10}{MN 10} \textit{Mindfulness Meditation} (\textit{\textsanskrit{Satipaṭṭhānasutta}}). Whereas many compilations simply list a series of different teachings, here the text is very systematic, organizing the compiled passages under the four heads. These meditation passages are mostly not found elsewhere in the \textsanskrit{Dīgha}, and were no doubt added to ensure the \textsanskrit{Dīgha} reciters preserved the full range of meditation teachings.

To the already lengthy discourse at \href{https://suttacentral.net/mn10}{MN 10} is added a full exposition on the four noble truths, sourced from \href{https://suttacentral.net/mn141}{MN 141} \textit{The Analysis of the Truths} (\textit{\textsanskrit{Saccavibhaṅgasutta}}). In Burmese editions, this extended section later made its way back into the text of \href{https://suttacentral.net/mn10}{MN 10}. Since SuttaCentral’s text is a Burmese one, we include this in our Pali, but mark it as an addition.

\subsection*{Narratives: backgrounds, parables, and myths}

Unusually for early Buddhist texts, the \textsanskrit{Dīgha} includes several lengthy narratives. Most obviously this includes \href{https://suttacentral.net/dn16}{DN 16} \textit{\textsanskrit{Mahāparinibbāna}}. But it also includes a number of other narratives.

In common with the discourses of other collections, we often find a simple narrative background that gives context to the teaching. However, in some cases this is developed in much greater detail as the narratives come to play a more sophisticated literary role than mere setting.

\href{https://suttacentral.net/dn2}{DN 2} \textit{\textsanskrit{Sāmaññaphala}} opens with King \textsanskrit{Ajātasattu} of Magadha exclaiming over the beauty of the moonlit night and asking his ministers for advice as to which ascetic teacher he should visit. From the \textit{\textsanskrit{Mahāparinibbāna}} and discourses elsewhere we know that \textsanskrit{Ajātasattu} was a warlike king, so this setting immediately establishes a sense of wonder. The narrative unfolds gracefully, avoiding the excess of ornament so typical of later Indian narratives, and holding the key to its mystery close to its chest. Only at the end of the text do we learn the dreadful secret that plagues the king’s heart. Thus the narrative portions imbue the teachings—on the doctrines of other teachers as contrasted with the Buddha’s Gradual Training—with a tragic pathos.

In addition to backgrounds, we also find narratives that are told as stories in the discourses themselves. These include short parables like the tale of the monk who mistakenly sought among the gods for an answer to his question (\href{https://suttacentral.net/dn11}{DN 11}). In \href{https://suttacentral.net/dn23}{DN 23} the monk \textsanskrit{Kumāra} Kassapa debates with the skeptic \textsanskrit{Pāyāsi}, illustrating his arguments with a series of tales alternatively humorous and gruesome. Such parables are found not infrequently elsewhere in the suttas, but in the \textsanskrit{Dīgha} certain stories expand beyond this and approach the stature of myth. This includes some of the texts in the \textit{\textsanskrit{Mahāparinibbāna}} cycle, such as \href{https://suttacentral.net/dn17}{DN 17} \textit{\textsanskrit{Mahāsudassanasutta}} and \href{https://suttacentral.net/dn14}{DN 14} \textit{\textsanskrit{Mahāpadānasutta}}.

To forestall a common misunderstanding, in the study of religion, “myth” does not mean “something believed to be true that is actually false”, as it does in popular culture. Rather, a myth is a sacred story. Some sacred stories are true, some are inventions. But this is a matter for historians and is irrelevant to the mythology itself. The purpose of myth is to tell a story that creates meaning for those who participate in it, so they can understand their own lives in the context of the story being expressed.

The \textsanskrit{Dīgha} contains truly mythic texts in \href{https://suttacentral.net/dn26}{DN 26} \textit{The Wheel-Turning Monarch} (\textit{\textsanskrit{Cakkavattisīhanādasutta}}) and \href{https://suttacentral.net/dn27}{DN 27} \textit{The Origin of the World} (\textit{\textsanskrit{Aggaññasutta}}). These set forth a myth of origins, replacing conventional creation mythology with an evolutionary account of how the world came to be the way it is. In these stories, human choices play a critical role in how the environment evolves, and in how it will all fall apart. The \textit{\textsanskrit{Aggañña}} depicts climate change quite explicitly, showing how human activity affects the plants, the weather, and the natural ecosystem of which we are a part (see also \href{https://suttacentral.net/an3.56}{AN 3.56}).

The mythology is essentially cyclic. There is no absolute beginning, just another turning of the wheel. Thus even when the world falls apart and civilization collapses, there will be a new renaissance, far in the future, and ultimately another Buddha will arise. He is named as Metteyya (Sanskrit: \textit{Maitreya}), who in the early texts appears only in \href{https://suttacentral.net/dn26}{DN 26} \textit{\textsanskrit{Cakkavattisīhanāda}}. He went on to become one of the most important figures in \textsanskrit{Mahāyāna} Buddhism, and many Buddhists even today still await his coming with hope. Yet \href{https://suttacentral.net/dn26}{DN 26} is not taught in order to encourage devotees to dedicate themselves to Metteyya, but to illustrate the impermanence and uncertainty of our lives. The Buddha always taught that we should practice as best we can to understand the Dhamma in this life.

\section*{The \textsanskrit{Mahāparinibbāna} Cycle}

In several instances, episodes mentioned in brief in the \textit{\textsanskrit{Mahāparinibbāna}} have been spun off and expanded to become individual discourses in their own right. Thus the \textit{\textsanskrit{Mahāparinibbāna}} dominates much of the \textsanskrit{Dīgha}, not just through its own length and thematic weight, but through its influence and connections with other discourses.

In my view, this cycle of suttas was likely composed by Ānanda and his students, beginning this great literary work with the \textit{\textsanskrit{Mahāparinibbānasutta}} itself, and gradually branching off into related works. The cycle as a whole shows not only Ānanda’s characteristic personal love and devotion for the Buddha, but also reveals a concern for what is to come, for the fate of the Dhamma in the years after the Buddha’s passing. One distinctive unifying detail of these discourses is that they do not end with the standard phrase saying that the listeners rejoiced in the teachings, but instead finish directly with a teaching or a verse on the subject of impermanence or the long-lasting of the dispensation. Ānanda survived the Buddha for several decades, and his legacy was the establishment of the texts, thus preserving the memory of his beloved Teacher for future generations.

\begin{description}%
\item[\href{https://suttacentral.net/dn16}{DN 16} \textit{The Great Discourse on the Buddha’s Extinguishment} (\textit{\textsanskrit{Mahāparinibbānasutta}})] Beginning with King \textsanskrit{Ajātasattu} of Magadha declaring his intent to invade the Vajjis, and ending with the peaceful distribution of the Buddha’s relics to the potentially warring nations and clans, the story of the Buddha’s last journeys is as politically revealing as it is spiritually moving. Throughout, the theme of impermanence unifies the diverse events and teachings. The weight of constructing such an epic shows, however, in the considerable differences between extant versions of the text. Many of the extra repetitious sections—such as the apparently superfluous sets of eight that follow the eight causes of earthquakes—are not found in all parallels. It seems that over time, more and more material was added, and at certain points portions of the text were split off to form other discourses in the cycle.%
\item[\href{https://suttacentral.net/dn17}{DN 17} \textit{King \textsanskrit{Mahāsudassana}}] In a small scene of the \textit{\textsanskrit{Mahāparinibbāna}}, Ānanda encourages the Buddha to pass away in a well-known city, not in the obscure village of \textsanskrit{Kusinārā}. The Buddha rebukes him, saying that in the past it had been a great city. The Sanskrit (\textsanskrit{Sarvāstivāda}) versions of the \textit{\textsanskrit{Mahāparinibbāna}} include a shorter account of the story of King \textsanskrit{Mahāsudassana} in their \textit{\textsanskrit{Mahāparinibbāna}} itself, but in the Pali it has become greatly extended and formed into its own long discourse. The discourse itself is fabulous, full of extended passages on the crystal balustrades and other wonders of \textsanskrit{Mahāsudassana}’s palace. But at its heart is a very human story: the love of the queen for her king, and the pain of letting go. The struggle that the queen undergoes to fully understand that her king must pass mirrors the struggles of Ānanda in the \textit{\textsanskrit{Mahāparinibbāna}} as he comes to terms with the passing of his beloved Teacher.%
\item[\href{https://suttacentral.net/dn18}{DN 18} \textit{With Janavasabha}] Like \href{https://suttacentral.net/dn17}{DN 17}, this begins with a short passage extracted from the \textit{\textsanskrit{Mahāparinibbāna}}, to which has been added an extended narrative. During the journey in the \textit{\textsanskrit{Mahāparinibbāna}}, Ānanda asks the Buddha to reveal the fate after death of devotees in the town of \textsanskrit{Nādika}. Characteristically, it ends with the Buddha showing how people may know for themselves their own spiritual progress. This short passage is preserved as an independent discourse also in \href{https://suttacentral.net/sn55.10}{SN 55.10}. In \href{https://suttacentral.net/dn18}{DN 18}, however, the discourse continues with a long story of the doings of the gods, as told by the spirit Janavasabha. It culminates by saying that this discourses was learned by the Buddha from Janavasabha, and from there was taught to Ānanda, and he informed the assemblies of monks, nuns, laymen, and lay women, resulting in the Buddha’s dispensation being famous and successful among gods and men. This corroborates the idea that these discourses, shaped by Ānanda, were aimed at ensuring the long-lasting of Buddhism.%
\item[\href{https://suttacentral.net/dn28}{DN 28} \textit{Inspiring Confidence} (\textit{\textsanskrit{Sampasādanīyasuttasutta}})] The \textit{\textsanskrit{Mahāparinibbāna}} records an incident where \textsanskrit{Sāriputta}, the Buddha’s foremost disciple, comes to him and makes a “lion’s roar” of his faith in the Buddha, based on his understanding of Dhamma. This is recorded as an independent discourse at \href{https://suttacentral.net/sn47.12}{SN 47.12}. We also have a short discourse at \href{https://suttacentral.net/sn47.13}{SN 47.13} that tells of \textsanskrit{Sāriputta}’s death. This echoes the themes of the \textit{\textsanskrit{Mahāparinibbāna}}, even including the famous saying that one should be one’s own refuge. Clearly this must have happened during the journey recorded in the \textit{\textsanskrit{Mahāparinibbāna}}. Oddly, however, it is not included in \href{https://suttacentral.net/dn16}{DN 16}, and in addition, it situates the Buddha in \textsanskrit{Sāvatthī}, far from the track of his journey. Regardless, the passage on the lion’s roar was expanded into its own extensive discourse, with \textsanskrit{Sāriputta} expounding at length on various inspiring qualities of the Buddha. This gives an opportunity to list many standard doctrinal teachings. Like \href{https://suttacentral.net/dn18}{DN 18}, the sutta ends with an exhortation to share the teaching.%
\end{description}

In addition to texts that have a direct literary and narrative connection with the \textit{\textsanskrit{Mahāparinibbāna}}, there is a further series of discourses that share a more indirect or thematic connection.

\begin{description}%
\item[\href{https://suttacentral.net/dn14}{DN 14} \textit{The Great Discourse on the Harvest of Deeds} (\textit{\textsanskrit{Mahāpadānasuttasutta}})] The Buddha gives biographical details of six past Buddhas, as well as a lengthy account of the life of one of them, \textsanskrit{Vipassī}. This discourse establishes the historical Buddha Gotama as one of a series of world teachers that stretches back into the deep past, and whose dispensations all follow similar patterns.%
\item[\href{https://suttacentral.net/dn29}{DN 29} \textit{An Impressive Discourse} (\textit{\textsanskrit{Pāsādikasutta}})] This begins with the story of the passing away of the Jain leader \textsanskrit{Mahāvīra} (\textsanskrit{Nigaṇṭha} \textsanskrit{Nātaputta}). In the Buddhist texts, this is depicted as a disaster for the Jains, as they fell apart in conflict right away. Whether this is historically accurate or not, the text shows the Buddha taking the opportunity to teach the qualities that make a religious movement last long after the passing of the founder. Discourses in response to this are found at \href{https://suttacentral.net/dn16}{DN 16}, \href{https://suttacentral.net/dn29}{DN 29}, \href{https://suttacentral.net/dn33}{DN 33}, and \href{https://suttacentral.net/mn104}{MN 104}. In the current sutta, contrasting his own dispensation with what he claims was the inadequacy of Jain techings, the Buddha declares that the faith and practice of his followers is well-grounded, since it is based on genuine Awakening.%
\item[\href{https://suttacentral.net/dn30}{DN 30} \textit{The Marks of a Great Man} (\textit{\textsanskrit{Lakkhaṇasutta}})] The early texts refer several times to a mysterious set of bodily characteristics known as the “marks of a great man”. These are said to fulfill a brahmanical prophecy that one who possesses such marks will either become a universal emperor or a fully awakened Buddha. Normally when the suttas present something as a brahmanical teaching, it is in fact found in brahmanical texts. But in this case no trace of such a doctrine has been found, so the origins of this mythological idea are obscure. The story of the two paths is a classic mythological theme, found in the oldest known myth, the story of Gilgamesh. The marks of a great man exist as a curious counterpoint to the rational teachings found in most of the suttas. In this particular sutta, the Buddha is said to have explained each mark as a consequence of a specific kind of kammic deed. The literary and verse styles betray this as a late composition, and it has no real parallel in other collections. Nevertheless, it remains as a testament to the evolution of the idea of the Buddha, relating his spiritual qualities to his physical presence.%
\item[\href{https://suttacentral.net/dn33}{DN 33} \textit{Reciting in Concert} (\textit{\textsanskrit{Saṅgītisutta}})] Like \href{https://suttacentral.net/dn29}{DN 29}, this discourse is set after the death of \textsanskrit{Mahāvīra}. Speaking to the Mallians of \textsanskrit{Pāvā}—who appear also in the \textit{\textsanskrit{Mahāparinibbāna}}—the Buddha asks \textsanskrit{Sāriputta} to speak on his behalf. This echoes the theme of \href{https://suttacentral.net/dn28}{DN 28} and \href{https://suttacentral.net/dn29}{DN 29}, that it is the disciples who will be responsible for the continuation of the teachings. \textsanskrit{Sāriputta} gives an extensive systematic presentation of doctrines, using the \textsanskrit{Aṅguttara} principle of organizing teachings by number. Indeed, a study of this discourse can serve as an introduction to the teachings found in the \textsanskrit{Aṅguttara} \textsanskrit{Nikāya}. The monastics are encouraged to recite these teachings in concert, so that they may be preserved and the dispensation continued for a long time. This discourse anticipates the systematic tendencies of the Abhidhamma, and indeed one of the \textsanskrit{Sarvāstivādin} Abhidhamma texts (\textit{\textsanskrit{Saṅgītiparyāya}}) consists of an expansion and commentary on this discourse.%
\item[\href{https://suttacentral.net/dn34}{DN 34} \textit{Up to Ten} (\textit{Dasuttarasutta})] This is similar to the \textit{\textsanskrit{Saṅgīti}}, but with a briefer narrative context and even more systematic style. Here the Buddha no longer appears, and the discourse is simply spoken by \textsanskrit{Sāriputta}.%
\end{description}

This does not exhaust the scope of the \textit{\textsanskrit{Mahāparinibbāna}} cycle, for it is not confined to the \textsanskrit{Dīgha}. We have already mentioned that several shorter suttas contain episodes either found in the \textit{\textsanskrit{Mahāparinibbāna}} or related to it. And the story does not end with the Buddha’s death. The \textit{\textsanskrit{Mahāparinibbāna}} tells of the funeral arrangements and events following the Buddha’s passing. In several versions apart from the Pali, this story continues directly into the account of the First Council. This narrative is the 21st chapter of the Vinaya Khandhakas, and indeed the \textit{\textsanskrit{Mahāparinibbāna}} is found in the Vinaya of several schools. It is, in fact, one continuous narrative, and one of the many purposes of the \textit{\textsanskrit{Mahāparinibbāna}} is to authorize the actions of the \textsanskrit{Saṅgha} at the First Council, establishing the fundamental Buddhist scriptures in an organized and definitive manner. The First Council narrative was then extended to the Second Council, which echoes many of the same themes and ideas.

These stories of the end of the Buddha’s life and teaching are also echoed in the first chapter of the Vinaya Khandhakas, which tells the story of the Buddha’s awakening, first teaching, and establishing of his community of followers. These are not just separate episodes in the Buddha’s life. The texts as we have them frequently echo ideas, turns of phrase, events, and people, all of which show that they were edited and composed as a coherent whole. Taken together, they make up a framework of a magnificent mythology: the life and death of the greatest spiritual teacher that the world has ever known.

\section*{A Brief Textual History}

The \textsanskrit{Dīgha} \textsanskrit{Nikāya} was edited by T.W. Rhys Davids and J.E. Carpenter on the basis of manuscripts in Sinhalese, Burmese, and Thai scripts, and published in three volumes in Latin script by the Pali Text Society from 1890 to 1910.

The first translation followed in 1899–1921 by T.W. and C.A.F. Rhys Davids, and was published under the “Sacred Books of the Buddhists” series under the title \textit{Dialogues of the Buddha}. This was a milestone in the publication of Buddhist texts, and marked the first occasion a full \textit{\textsanskrit{nikāya}} was available in English. The translation endeavored to retain something of the literary flavor of the texts, and is accompanied by introductory essays and notes that are often useful and sometimes brilliant. But it is far from perfect, and contains many errors of both reading and interpretation. Today the insights of Rhys Davids remain valuable especially in the area of history and society.

An updated translation by Maurice Walshe was published by Wisdom Publications in 1987 under the title \textit{Thus Have I Heard: The Long Discourses of the Buddha}, a title that in later editions was changed to \textit{The Long Discourses of the Buddha}. The Walshe edition benefited from many decades of study and practice of Dhamma in the west. Avoiding the archaic stylings of the older translations, it remains a clear and approachable translation, with a far more accurate handling of doctrinal terms and passages. But it too is far from perfect. It relies heavily on the Rhys Davids translation, and while it corrects many errors, it sometimes repeats errors found in the older translation. Worse, it not infrequently introduces new errors.

In addition, there have been many translations of individual discourses and passages. Of these, the following were specially useful for my work:

\begin{itemize}%
\item For \href{https://suttacentral.net/dn1}{DN 1}, \href{https://suttacentral.net/dn2}{DN 2}, and \href{https://suttacentral.net/dn15}{DN 15}, the translations of text and commentary by Bhikkhu Bodhi.%
\item For \href{https://suttacentral.net/dn16}{DN 16}, the translation by Bhikkhu Ānandajoti.%
\item For the verses of \href{https://suttacentral.net/dn30}{DN 30}, the translations and studies by K.R. Norman.%
\item For \href{https://suttacentral.net/dn31}{DN 31}, the translation by John Kelly, Sue Sawyer, and Victoria Yareham.%
\end{itemize}

%
\chapter*{Acknowledgements}
\addcontentsline{toc}{chapter}{Acknowledgements}
\markboth{Acknowledgements}{Acknowledgements}

I remember with gratitude all those from whom I have learned the Dhamma, especially Ajahn Brahm and Bhikkhu Bodhi, the two monks who more than anyone else showed me the depth, meaning, and practical value of the Suttas.

Special thanks to Dustin and Keiko Cheah and family, who sponsored my stay in Qi Mei while I made this translation.

Thanks also for Blake Walshe, who provided essential software support for my translation work.

Throughout the process of translation, I have frequently sought feedback and suggestions from the community on the SuttaCentral community on our forum, “Discuss and Discover”. I want to thank all those who have made suggestions and contributed to my understanding, as well as to the moderators who have made the forum possible. A special thanks is due to \textsanskrit{Sabbamittā}, a true friend of all, who has tirelessly and precisely checked my work.

Finally my everlasting thanks to all those people, far too many to mention, who have supported SuttaCentral, and those who have supported my life as a monastic. None of this would be possible without you.

%
\chapter*{Summary of Contents}
\addcontentsline{toc}{chapter}{Summary of Contents}
\markboth{Summary of Contents}{Summary of Contents}

\begin{description}%
\item[\href{\#dn{-}silakkhandhavagga}{None}] The Chapter Containing the Section on Ethics (\textsanskrit{Sīlakkhandhavagga}) is a chapter of 13 discourses. Each of these contains a long passage on the Gradual Training in ethics, meditation, and wisdom. The chapter is named after the first of these sections. The two other known versions of the \textsanskrit{Dīrghāgama} (in Chinese and Sanskrit) also contain a similar chapter. Despite the monastic nature of the central teaching, most of these discourses are presented in dialog with lay people, with a strong emphasis on the relation between the Buddha’s teachings and other contemporary movements.%
\item[\href{\#dn1}{None}] While others may praise or criticize the Buddha, they tend to focus on trivial details. The Buddha presents an analysis of 62 kinds of wrong view, seeing through which one becomes detached from meaningless speculations.%
\item[\href{\#dn2}{None}] The newly crowned King \textsanskrit{Ajātasattu} is disturbed by the violent means by which he achieved the crown. He visits the Buddha to find peace of mind, and asks him about the benefits of spiritual practice. This is one of the greatest literary and spiritual texts of early Buddhism.%
\item[\href{\#dn3}{None}] A young brahmin student attacks the Buddha’s family, but is put in his place.%
\item[\href{\#dn4}{None}] A reputed brahmin visits the Buddha, despite the reservations of other brahmins. They discuss the true meaning of a brahmin, and the Buddha skillfully draws him around to his own point of view.%
\item[\href{\#dn5}{None}] A brahmin wishes to undertake a great sacrifice, and asks for the Buddha’s advice. The Buddha tells a legend of the past, in which a king is persuaded to give up violent sacrifice, and instead to devote his resources to supporting the needy citizens of his realm. However, even such a beneficial and non-violent sacrifice pales in comparison to the spiritual sacrifice of giving up attachments.%
\item[\href{\#dn6}{None}] The Buddha explains to a diverse group of lay people how the results of meditation depend on the manner of development.%
\item[\href{\#dn7}{None}] This discourse is mostly quoted by the Buddha in the previous.%
\item[\href{\#dn8}{None}] The Buddha is challenged by a naked ascetic on the topic of spiritual austerities. He points out that it is quite possible to perform all kinds of austere practices without having any inner purity of mind.%
\item[\href{\#dn9}{None}] The Buddha discusses with a wanderer the nature of perception and how it evolves through deeper states of meditation. None of these, however, should be identified with a self or soul.%
\item[\href{\#dn10}{None}] Shortly after the Buddha’s death, Venerable Ānanda is invited to explain the core teachings.%
\item[\href{\#dn11}{None}] The Buddha refuses to perform miracles, explaining that this is not the right way to inspire faith. He goes on to tell the story of a monk whose misguided quest for answers led him as far as \textsanskrit{Brahmā}.%
\item[\href{\#dn12}{None}] A brahmin has fallen into the idea that there is no point in trying to offer spiritual help to others. The Buddha goes to see him, and persuades him of the genuine benefits of spiritual teaching.%
\item[\href{\#dn13}{None}] A number of brahmins are discussing the true path to \textsanskrit{Brahmā}. Contesting the claims to authority based on the Vedas, the Buddha insists that only personal experience can lead to the truth.%
\end{description}

%
\mainmatter%
\pagestyle{fancy}%
\addtocontents{toc}{\let\protect\contentsline\protect\nopagecontentsline}
\part*{The Chapter on the Entire Spectrum of Ethics }
\addcontentsline{toc}{part}{The Chapter on the Entire Spectrum of Ethics }
\markboth{}{}
\addtocontents{toc}{\let\protect\contentsline\protect\oldcontentsline}

%
\chapter*{{\suttatitleacronym DN 1}{\suttatitletranslation The Prime Net }{\suttatitleroot Brahmajālasutta}}
\addcontentsline{toc}{chapter}{\tocacronym{DN 1} \toctranslation{The Prime Net } \tocroot{Brahmajālasutta}}
\markboth{The Prime Net }{Brahmajālasutta}
\extramarks{DN 1}{DN 1}

\section*{1. Talk on Wanderers }

\scevam{So\marginnote{1.1.1} I have heard. }At one time the Buddha was traveling along the road between \textsanskrit{Rājagaha} and \textsanskrit{Nālanda} together with a large \textsanskrit{Saṅgha} of around five hundred mendicants. The wanderer Suppiya was also traveling along the same road, together with his pupil, the brahmin student Brahmadatta. Meanwhile, Suppiya criticized the Buddha, the teaching, and the \textsanskrit{Saṅgha} in many ways, but his pupil Brahmadatta praised them in many ways. And so both teacher and pupil followed behind the Buddha and the \textsanskrit{Saṅgha} of mendicants directly contradicting each other. 

Then\marginnote{1.2.1} the Buddha took up residence for the night in the royal rest-house in \textsanskrit{Ambalaṭṭhikā} together with the \textsanskrit{Saṅgha} of mendicants. And Suppiya and Brahmadatta did likewise. There too, Suppiya criticized the Buddha, the teaching, and the \textsanskrit{Saṅgha} in many ways, but his pupil Brahmadatta praised them in many ways. And so both teacher and pupil kept on directly contradicting each other. 

Then\marginnote{1.3.1} several mendicants rose at the crack of dawn and sat together in the pavilion, where the topic of evaluation came up: 

“It’s\marginnote{1.3.2} incredible, reverends, it’s amazing how the diverse convictions of sentient beings have been clearly comprehended by the Blessed One, who knows and sees, the perfected one, the fully awakened Buddha. For this Suppiya criticizes the Buddha, the teaching, and the \textsanskrit{Saṅgha} in many ways, while his pupil Brahmadatta praises them in many ways. And so both teacher and pupil followed behind the Buddha and the \textsanskrit{Saṅgha} of mendicants directly contradicting each other.” 

When\marginnote{1.4.1} the Buddha found out about this discussion on evaluation among the mendicants, he went to the pavilion, where he sat on the seat spread out and addressed the mendicants, “Mendicants, what were you sitting talking about just now? What conversation was left unfinished?” 

The\marginnote{1.4.3} mendicants told him what had happened, adding, “This was our conversation that was unfinished when the Buddha arrived.” 

“Mendicants,\marginnote{1.5.1} if others criticize me, the teaching, or the \textsanskrit{Saṅgha}, don’t make yourselves resentful, bitter, and exasperated. You’ll get angry and upset, which would be an obstacle for you alone. If others were to criticize me, the teaching, or the \textsanskrit{Saṅgha}, and you got angry and upset, would you be able to understand whether they spoke well or poorly?” 

“No,\marginnote{1.5.4} sir.” 

“If\marginnote{1.6.1} others criticize me, the teaching, or the \textsanskrit{Saṅgha}, you should explain that what is untrue is in fact untrue: ‘This is why that’s untrue, this is why that’s false. There’s no such thing in us, it’s not found among us.’ 

If\marginnote{1.6.3} others praise me, the teaching, or the \textsanskrit{Saṅgha}, don’t make yourselves thrilled, elated, and excited. You’ll get thrilled, elated, and excited, which would be an obstacle for you alone. If others praise me, the teaching, or the \textsanskrit{Saṅgha}, you should acknowledge that what is true is in fact true: ‘This is why that’s true, this is why that’s correct. There is such a thing in us, it is found among us.’ 

\section*{2. Ethics }

\subsection*{2.1. The Shorter Section on Ethics }

When\marginnote{1.7.1} an ordinary person speaks praise of the Realized One, they speak only of trivial, insignificant details of mere ethics. And what are the trivial, insignificant details of mere ethics that an ordinary person speaks of? 

‘The\marginnote{1.8.1} ascetic Gotama has given up killing living creatures. He has renounced the rod and the sword. He’s scrupulous and kind, living full of compassion for all living beings.’ Such is an ordinary person’s praise of the Realized One. 

‘The\marginnote{1.8.3} ascetic Gotama has given up stealing. He takes only what’s given, and expects only what’s given. He keeps himself clean by not thieving.’ Such is an ordinary person’s praise of the Realized One. 

‘The\marginnote{1.8.5} ascetic Gotama has given up unchastity. He is celibate, set apart, avoiding the common practice of sex.’ Such is an ordinary person’s praise of the Realized One. 

‘The\marginnote{1.9.1} ascetic Gotama has given up lying. He speaks the truth and sticks to the truth. He’s honest and trustworthy, and doesn’t trick the world with his words.’ Such is an ordinary person’s praise of the Realized One. 

‘The\marginnote{1.9.3} ascetic Gotama has given up divisive speech. He doesn’t repeat in one place what he heard in another so as to divide people against each other. Instead, he reconciles those who are divided, supporting unity, delighting in harmony, loving harmony, speaking words that promote harmony.’ Such is an ordinary person’s praise of the Realized One. 

‘The\marginnote{1.9.5} ascetic Gotama has given up harsh speech. He speaks in a way that’s mellow, pleasing to the ear, lovely, going to the heart, polite, likable and agreeable to the people.’ Such is an ordinary person’s praise of the Realized One. 

‘The\marginnote{1.9.7} ascetic Gotama has given up talking nonsense. His words are timely, true, and meaningful, in line with the teaching and training. He says things at the right time which are valuable, reasonable, succinct, and beneficial.’ Such is an ordinary person’s praise of the Realized One. 

‘The\marginnote{1.10.1} ascetic Gotama refrains from injuring plants and seeds.’ 

‘He\marginnote{1.10.3} eats in one part of the day, abstaining from eating at night and food at the wrong time.’ 

‘He\marginnote{1.10.4} refrains from dancing, singing, music, and seeing shows.’ 

‘He\marginnote{1.10.5} refrains from beautifying and adorning himself with garlands, perfumes, and makeup.’ 

‘He\marginnote{1.10.6} refrains from high and luxurious beds.’ 

‘He\marginnote{1.10.7} refrains from receiving gold and money, raw grains, raw meat, women and girls, male and female bondservants, goats and sheep, chickens and pigs, elephants, cows, horses, and mares, and fields and land.’ 

‘He\marginnote{1.10.16} refrains from running errands and messages; buying and selling; falsifying weights, metals, or measures; bribery, fraud, cheating, and duplicity; mutilation, murder, abduction, banditry, plunder, and violence.’ Such is an ordinary person’s praise of the Realized One. 

\scendsection{The shorter section on ethics is finished. }

\subsection*{2.2. The Middle Section on Ethics }

‘There\marginnote{1.11.1} are some ascetics and brahmins who, while enjoying food given in faith, still engage in injuring plants and seeds. These include plants propagated from roots, stems, cuttings, or joints; and those from regular seeds as the fifth. The ascetic Gotama refrains from such injury to plants and seeds.’ Such is an ordinary person’s praise of the Realized One. 

‘There\marginnote{1.12.1} are some ascetics and brahmins who, while enjoying food given in faith, still engage in storing up goods for their own use. This includes such things as food, drink, clothes, vehicles, bedding, fragrance, and material possessions. The ascetic Gotama refrains from storing up such goods.’ Such is an ordinary person’s praise of the Realized One. 

‘There\marginnote{1.13.1} are some ascetics and brahmins who, while enjoying food given in faith, still engage in seeing shows. This includes such things as dancing, singing, music, performances, and storytelling; clapping, gongs, and kettledrums; art exhibitions and acrobatic displays; battles of elephants, horses, buffaloes, bulls, goats, rams, chickens, and quails; staff-fights, boxing, and wrestling; combat, roll calls of the armed forces, battle-formations, and regimental reviews. The ascetic Gotama refrains from such shows.’ Such is an ordinary person’s praise of the Realized One. 

‘There\marginnote{1.14.1} are some ascetics and brahmins who, while enjoying food given in faith, still engage in gambling that causes negligence. This includes such things as checkers, draughts, checkers in the air, hopscotch, spillikins, board-games, tip-cat, drawing straws, dice, leaf-flutes, toy plows, somersaults, pinwheels, toy measures, toy carts, toy bows, guessing words from syllables, and guessing another’s thoughts. The ascetic Gotama refrains from such gambling.’ Such is an ordinary person’s praise of the Realized One. 

‘There\marginnote{1.15.1} are some ascetics and brahmins who, while enjoying food given in faith, still make use of high and luxurious bedding. This includes such things as sofas, couches, woolen covers—shag-piled, colorful, white, embroidered with flowers, quilted, embroidered with animals, double- or single-fringed—and silk covers studded with gems, as well as silken sheets, woven carpets, rugs for elephants, horses, or chariots, antelope hide rugs, and spreads of fine deer hide, with a canopy above and red cushions at both ends. The ascetic Gotama refrains from such bedding.’ Such is an ordinary person’s praise of the Realized One. 

‘There\marginnote{1.16.1} are some ascetics and brahmins who, while enjoying food given in faith, still engage in beautifying and adorning themselves with garlands, fragrance, and makeup. This includes such things as applying beauty products by anointing, massaging, bathing, and rubbing; mirrors, ointments, garlands, fragrances, and makeup; face-powder, foundation, bracelets, headbands, fancy walking-sticks or containers, rapiers, parasols, fancy sandals, turbans, jewelry, chowries, and long-fringed white robes. The ascetic Gotama refrains from such beautification and adornment.’ Such is an ordinary person’s praise of the Realized One. 

‘There\marginnote{1.17.1} are some ascetics and brahmins who, while enjoying food given in faith, still engage in unworthy talk. This includes such topics as talk about kings, bandits, and ministers; talk about armies, threats, and wars; talk about food, drink, clothes, and beds; talk about garlands and fragrances; talk about family, vehicles, villages, towns, cities, and countries; talk about women and heroes; street talk and well talk; talk about the departed; motley talk; tales of land and sea; and talk about being reborn in this or that state of existence. The ascetic Gotama refrains from such unworthy talk.’ Such is an ordinary person’s praise of the Realized One. 

‘There\marginnote{1.18.1} are some ascetics and brahmins who, while enjoying food given in faith, still engage in arguments. They say such things as: “You don’t understand this teaching and training. I understand this teaching and training. What, you understand this teaching and training? You’re practicing wrong. I’m practicing right. I stay on topic, you don’t. You said last what you should have said first. You said first what you should have said last. What you’ve thought so much about has been disproved. Your doctrine is refuted. Go on, save your doctrine! You’re trapped; get yourself out of this—if you can!” The ascetic Gotama refrains from such argumentative talk.’ Such is an ordinary person’s praise of the Realized One. 

‘There\marginnote{1.19.1} are some ascetics and brahmins who, while enjoying food given in faith, still engage in running errands and messages. This includes running errands for rulers, ministers, aristocrats, brahmins, householders, or princes who say: “Go here, go there. Take this, bring that from there.” The ascetic Gotama refrains from such errands.’ Such is an ordinary person’s praise of the Realized One. 

‘There\marginnote{1.20.1} are some ascetics and brahmins who, while enjoying food given in faith, still engage in deceit, flattery, hinting, and belittling, and using material possessions to chase after other material possessions. The ascetic Gotama refrains from such deceit and flattery.’ Such is an ordinary person’s praise of the Realized One. 

\scendsection{The middle section on ethics is finished. }

\subsection*{2.3. The Large Section on Ethics }

‘There\marginnote{1.21.1} are some ascetics and brahmins who, while enjoying food given in faith, still earn a living by unworthy branches of knowledge, by wrong livelihood. This includes such fields as limb-reading, omenology, divining celestial portents, interpreting dreams, divining bodily marks, divining holes in cloth gnawed by mice, fire offerings, ladle offerings, offerings of husks, rice powder, rice, ghee, or oil; offerings from the mouth, blood sacrifices, palmistry; geomancy for building sites, fields, and cemeteries; exorcisms, earth magic, snake charming, poisons; the crafts of the scorpion, the rat, the bird, and the crow; prophesying life span, chanting for protection, and deciphering animal cries. The ascetic Gotama refrains from such unworthy branches of knowledge, such wrong livelihood.’ Such is an ordinary person’s praise of the Realized One. 

‘There\marginnote{1.22.1} are some ascetics and brahmins who, while enjoying food given in faith, still earn a living by unworthy branches of knowledge, by wrong livelihood. This includes reading the marks of gems, cloth, clubs, swords, spears, arrows, weapons, women, men, boys, girls, male and female bondservants, elephants, horses, buffaloes, bulls, cows, goats, rams, chickens, quails, monitor lizards, rabbits, tortoises, or deer. The ascetic Gotama refrains from such unworthy branches of knowledge, such wrong livelihood.’ Such is an ordinary person’s praise of the Realized One. 

‘There\marginnote{1.23.1} are some ascetics and brahmins who, while enjoying food given in faith, still earn a living by unworthy branches of knowledge, by wrong livelihood. This includes making predictions that the king will march forth or march back; or that our king will attack and the enemy king will retreat, or vice versa; or that our king will triumph and the enemy king will be defeated, or vice versa; and so there will be victory for one and defeat for the other. The ascetic Gotama refrains from such unworthy branches of knowledge, such wrong livelihood.’ Such is an ordinary person’s praise of the Realized One. 

‘There\marginnote{1.24.1} are some ascetics and brahmins who, while enjoying food given in faith, still earn a living by unworthy branches of knowledge, by wrong livelihood. This includes making predictions that there will be an eclipse of the moon, or sun, or stars; that the sun, moon, and stars will be in conjunction or in opposition; that there will be a meteor shower, a fiery sky, an earthquake, thunder; that there will be a rising, a setting, a darkening, a brightening of the moon, sun, and stars. And it also includes making predictions about the results of all such phenomena. The ascetic Gotama refrains from such unworthy branches of knowledge, such wrong livelihood.’ Such is an ordinary person’s praise of the Realized One. 

‘There\marginnote{1.25.1} are some ascetics and brahmins who, while enjoying food given in faith, still earn a living by unworthy branches of knowledge, by wrong livelihood. This includes predicting whether there will be plenty of rain or drought; plenty to eat or famine; an abundant harvest or a bad harvest; security or peril; sickness or health. It also includes such occupations as computing, accounting, calculating, poetry, and cosmology. The ascetic Gotama refrains from such unworthy branches of knowledge, such wrong livelihood.’ Such is an ordinary person’s praise of the Realized One. 

‘There\marginnote{1.26.1} are some ascetics and brahmins who, while enjoying food given in faith, still earn a living by unworthy branches of knowledge, by wrong livelihood. This includes making arrangements for giving and taking in marriage; for engagement and divorce; and for scattering rice inwards or outwards at the wedding ceremony. It also includes casting spells for good or bad luck, treating impacted fetuses, binding the tongue, or locking the jaws; charms for the hands and ears; questioning a mirror, a girl, or a god as an oracle; worshiping the sun, worshiping the Great One, breathing fire, and invoking Siri, the goddess of luck. The ascetic Gotama refrains from such unworthy branches of knowledge, such wrong livelihood.’ Such is an ordinary person’s praise of the Realized One. 

‘There\marginnote{1.27.1} are some ascetics and brahmins who, while enjoying food given in faith, still earn a living by unworthy branches of knowledge, by wrong livelihood. This includes rites for propitiation, for granting wishes, for ghosts, for the earth, for rain, for property settlement, and for preparing and consecrating house sites, and rites involving rinsing and bathing, and oblations. It also includes administering emetics, purgatives, expectorants, and phlegmagogues; administering ear-oils, eye restoratives, nasal medicine, ointments, and counter-ointments; surgery with needle and scalpel, treating children, prescribing root medicines, and binding on herbs. The ascetic Gotama refrains from such unworthy branches of knowledge, such wrong livelihood.’ Such is an ordinary person’s praise of the Realized One. 

These\marginnote{1.27.5} are the trivial, insignificant details of mere ethics that an ordinary person speaks of when they speak praise of the Realized One. 

\scendsection{The longer section on ethics is finished. }

\section*{3. Views }

\subsection*{3.1. Theories About the Past }

There\marginnote{1.28.1} are other principles—deep, hard to see, hard to understand, peaceful, sublime, beyond the scope of logic, subtle, comprehensible to the astute—which the Realized One makes known after realizing them with his own insight. Those who genuinely praise the Realized One would rightly speak of these things. And what are these principles? 

There\marginnote{1.29.1} are some ascetics and brahmins who theorize about the past, and assert various hypotheses concerning the past on eighteen grounds. And what are the eighteen grounds on which they rely? 

\subsubsection*{3.1.1. Eternalism }

There\marginnote{1.30.1} are some ascetics and brahmins who are eternalists, who assert that the self and the cosmos are eternal on four grounds. And what are the four grounds on which they rely? 

It’s\marginnote{1.31.1} when some ascetic or brahmin—by dint of keen, resolute, committed, and diligent effort, and right focus—experiences an immersion of the heart of such a kind that they recollect their many kinds of past lives. That is: one, two, three, four, five, ten, twenty, thirty, forty, fifty, a hundred, a thousand, a hundred thousand rebirths. They remember: ‘There, I was named this, my clan was that, I looked like this, and that was my food. This was how I felt pleasure and pain, and that was how my life ended. When I passed away from that place I was reborn somewhere else. There, too, I was named this, my clan was that, I looked like this, and that was my food. This was how I felt pleasure and pain, and that was how my life ended. When I passed away from that place I was reborn here.’ And so they recollect their many kinds of past lives, with features and details. 

They\marginnote{1.31.3} say: ‘The self and the cosmos are eternal, barren, steady as a mountain peak, standing firm like a pillar. They remain the same for all eternity, while these sentient beings wander and transmigrate and pass away and rearise. Why is that? Because by dint of keen, resolute, committed, and diligent effort, and right focus I experience an immersion of the heart of such a kind that I recollect my many kinds of past lives, with features and details. 

Because\marginnote{1.31.9} of this I know: 

“The\marginnote{1.31.10} self and the cosmos are eternal, barren, steady as a mountain peak, standing firm like a pillar. They remain the same for all eternity, while these sentient beings wander and transmigrate and pass away and rearise.’ This is the first ground on which some ascetics and brahmins rely to assert that the self and the cosmos are eternal. 

And\marginnote{1.32.1} what is the second ground on which they rely? It’s when some ascetic or brahmin—by dint of keen, resolute, committed, and diligent effort, and right focus—experiences an immersion of the heart of such a kind that they recollect their many kinds of past lives. That is: one eon of the cosmos contracting and expanding; two, three, four, five, or ten eons of the cosmos contracting and expanding. They remember: ‘There, I was named this, my clan was that, I looked like this, and that was my food. This was how I felt pleasure and pain, and that was how my life ended. When I passed away from that place I was reborn somewhere else. There, too, I was named this, my clan was that, I looked like this, and that was my food. This was how I felt pleasure and pain, and that was how my life ended. When I passed away from that place I was reborn here.’ And so they recollect their many kinds of past lives, with features and details. 

They\marginnote{1.32.4} say: ‘The self and the cosmos are eternal, barren, steady as a mountain peak, standing firm like a pillar. They remain the same for all eternity, while these sentient beings wander and transmigrate and pass away and rearise. Why is that? Because by dint of keen, resolute, committed, and diligent effort, and right focus I experience an immersion of the heart of such a kind that I recollect my many kinds of past lives, with features and details. 

Because\marginnote{1.32.10} of this I know: 

“The\marginnote{1.32.11} self and the cosmos are eternal, barren, steady as a mountain peak, standing firm like a pillar. They remain the same for all eternity, while these sentient beings wander and transmigrate and pass away and rearise.”’ This is the second ground on which some ascetics and brahmins rely to assert that the self and the cosmos are eternal. 

And\marginnote{1.33.1} what is the third ground on which they rely? It’s when some ascetic or brahmin—by dint of keen, resolute, committed, and diligent effort, and right focus—experiences an immersion of the heart of such a kind that they recollect their many kinds of past lives. That is: ten eons of the cosmos contracting and expanding; twenty, thirty, or forty eons of the cosmos contracting and expanding. They remember: ‘There, I was named this, my clan was that, I looked like this, and that was my food. This was how I felt pleasure and pain, and that was how my life ended. When I passed away from that place I was reborn somewhere else. There, too, I was named this, my clan was that, I looked like this, and that was my food. This was how I felt pleasure and pain, and that was how my life ended. When I passed away from that place I was reborn here.’ And so they recollect their many kinds of past lives, with features and details. 

They\marginnote{1.33.4} say: ‘The self and the cosmos are eternal, barren, steady as a mountain peak, standing firm like a pillar. They remain the same for all eternity, while these sentient beings wander and transmigrate and pass away and rearise. Why is that? Because by dint of keen, resolute, committed, and diligent effort, and right focus I experience an immersion of the heart of such a kind that I recollect my many kinds of past lives, with features and details. 

Because\marginnote{1.33.10} of this I know: 

“The\marginnote{1.33.11} self and the cosmos are eternal, barren, steady as a mountain peak, standing firm like a pillar. They remain the same for all eternity, while these sentient beings wander and transmigrate and pass away and rearise.”’ This is the third ground on which some ascetics and brahmins rely to assert that the self and the cosmos are eternal. 

And\marginnote{1.34.1} what is the fourth ground on which they rely? It’s when some ascetic or brahmin relies on logic and inquiry. They speak of what they have worked out by logic, following a line of inquiry, expressing their own perspective: ‘The self and the cosmos are eternal, barren, steady as a mountain peak, standing firm like a pillar. They remain the same for all eternity, while these sentient beings wander and transmigrate and pass away and rearise.’ This is the fourth ground on which some ascetics and brahmins rely to assert that the self and the cosmos are eternal. 

These\marginnote{1.35.1} are the four grounds on which those ascetics and brahmins assert that the self and the cosmos are eternal. Any ascetics and brahmins who assert that the self and the cosmos are eternal do so on one or other of these four grounds. Outside of this there is none. 

The\marginnote{1.36.1} Realized One understands this: ‘If you hold on to and attach to these grounds for views it leads to such and such a destiny in the next life.’ He understands this, and what goes beyond this. Yet since he does not misapprehend that understanding, he has realized extinguishment within himself. Having truly understood the origin, ending, gratification, drawback, and escape from feelings, the Realized One is freed through not grasping. 

These\marginnote{1.37.1} are the principles—deep, hard to see, hard to understand, peaceful, sublime, beyond the scope of logic, subtle, comprehensible to the astute—which the Realized One makes known after realizing them with his own insight. And those who genuinely praise the Realized One would rightly speak of these things. 

\subsubsection*{3.1.2. Partial Eternalism }

There\marginnote{2.1.1} are some ascetics and brahmins who are partial eternalists, who assert that the self and the cosmos are partially eternal and partially not eternal on four grounds. And what are the four grounds on which they rely? 

There\marginnote{2.2.1} comes a time when, after a very long period has passed, this cosmos contracts. As the cosmos contracts, sentient beings are mostly headed for the realm of streaming radiance. There they are mind-made, feeding on rapture, self-luminous, moving through the sky, steadily glorious, and they remain like that for a very long time. 

There\marginnote{2.3.1} comes a time when, after a very long period has passed, this cosmos expands. As it expands an empty mansion of \textsanskrit{Brahmā} appears. Then a certain sentient being—due to the running out of their life-span or merit—passes away from that host of radiant deities and is reborn in that empty mansion of \textsanskrit{Brahmā}. There they are mind-made, feeding on rapture, self-luminous, moving through the sky, steadily glorious, and they remain like that for a very long time. 

But\marginnote{2.4.1} after staying there all alone for a long time, they become dissatisfied and anxious: ‘Oh, if only another being would come to this state of existence.’ Then other sentient beings—due to the running out of their life-span or merit—pass away from that host of radiant deities and are reborn in that empty mansion of \textsanskrit{Brahmā} in company with that being. There they too are mind-made, feeding on rapture, self-luminous, moving through the sky, steadily glorious, and they remain like that for a very long time. 

Now,\marginnote{2.5.1} the being who was reborn there first thinks: ‘I am \textsanskrit{Brahmā}, the Great \textsanskrit{Brahmā}, the Undefeated, the Champion, the Universal Seer, the Wielder of Power, the Lord God, the Maker, the Author, the First, the Begetter, the Controller, the Father of those who have been born and those yet to be born. These beings were created by me! Why is that? Because first I thought: 

“Oh,\marginnote{2.5.6} if only another being would come to this state of existence.” Such was my heart’s wish, and then these creatures came to this state of existence.’ 

And\marginnote{2.5.8} the beings who were reborn there later also think: ‘This must be \textsanskrit{Brahmā}, the Great \textsanskrit{Brahmā}, the Undefeated, the Champion, the Universal Seer, the Wielder of Power, the Lord God, the Maker, the Author, the First, the Begetter, the Controller, the Father of those who have been born and those yet to be born. And we have been created by him. Why is that? Because we see that he was reborn here first, and we arrived later.’ 

And\marginnote{2.6.1} the being who was reborn first is more long-lived, beautiful, and illustrious than those who arrived later. 

It’s\marginnote{2.6.3} possible that one of those beings passes away from that host and is reborn in this state of existence. Having done so, they go forth from the lay life to homelessness. By dint of keen, resolute, committed, and diligent effort, and right focus, they experience an immersion of the heart of such a kind that they recollect that past life, but no further. 

They\marginnote{2.6.6} say: ‘He who is \textsanskrit{Brahmā}—the Great \textsanskrit{Brahmā}, the Undefeated, the Champion, the Universal Seer, the Wielder of Power, the Lord God, the Maker, the Author, the First, the Begetter, the Controller, the Father of those who have been born and those yet to be born—is permanent, everlasting, eternal, imperishable, remaining the same for all eternity. We who were created by that \textsanskrit{Brahmā} are impermanent, not lasting, short-lived, perishable, and have come to this state of existence. This is the first ground on which some ascetics and brahmins rely to assert that the self and the cosmos are partially eternal. 

And\marginnote{2.7.1} what is the second ground on which they rely? There are gods named ‘depraved by play.’ They spend too much time laughing, playing, and making merry. And in doing so, they lose their mindfulness, and they pass away from that host of gods. 

It’s\marginnote{2.8.1} possible that one of those beings passes away from that host and is reborn in this state of existence. Having done so, they go forth from the lay life to homelessness. By dint of keen, resolute, committed, and diligent effort, and right focus, they experience an immersion of the heart of such a kind that they recollect that past life, but no further. 

They\marginnote{2.9.1} say: ‘The gods not depraved by play don’t spend too much time laughing, playing, and making merry. So they don’t lose their mindfulness, and don’t pass away from that host of gods. They are permanent, everlasting, eternal, imperishable, remaining the same for all eternity. But we who were depraved by play spent too much time laughing, playing, and making merry. In doing so, we lost our mindfulness, and passed away from that host of gods. We are impermanent, not lasting, short-lived, perishable, and have come to this state of existence.’ This is the second ground on which some ascetics and brahmins rely to assert that the self and the cosmos are partially eternal. 

And\marginnote{2.10.1} what is the third ground on which they rely? There are gods named ‘malevolent’. They spend too much time gazing at each other, so they grow angry with each other, and their bodies and minds get tired. They pass away from that host of gods. 

It’s\marginnote{2.11.1} possible that one of those beings passes away from that host and is reborn in this state of existence. Having done so, they go forth from the lay life to homelessness. By dint of keen, resolute, committed, and diligent effort, and right focus, they experience an immersion of the heart of such a kind that they recollect that past life, but no further. 

They\marginnote{2.12.1} say: ‘The gods who are not malevolent don’t spend too much time gazing at each other, so they don’t grow angry with each other, their bodies and minds don’t get tired, and they don’t pass away from that host of gods. They are permanent, everlasting, eternal, imperishable, remaining the same for all eternity. But we who were malevolent spent too much time gazing at each other, we grew angry with each other, our bodies and minds got tired, and we passed away from that host of gods. We are impermanent, not lasting, short-lived, perishable, and have come to this state of existence.’ This is the third ground on which some ascetics and brahmins rely to assert that the self and the cosmos are partially eternal. 

And\marginnote{2.13.1} what is the fourth ground on which they rely? It’s when some ascetic or brahmin relies on logic and inquiry. They speak of what they have worked out by logic, following a line of inquiry, expressing their own perspective: ‘That which is called “the eye” or “the ear” or “the nose” or “the tongue” or “the body”: that self is impermanent, not lasting, transient, perishable. That which is called “mind” or “sentience” or “consciousness”: that self is permanent, everlasting, eternal, imperishable, remaining the same for all eternity.’ This is the fourth ground on which some ascetics and brahmins rely to assert that the self and the cosmos are partially eternal. 

These\marginnote{2.14.1} are the four grounds on which those ascetics and brahmins assert that the self and the cosmos are partially eternal and partially not eternal. Any ascetics and brahmins who assert that the self and the cosmos are partially eternal and partially not eternal do so on one or other of these four grounds. Outside of this there is none. 

The\marginnote{2.15.1} Realized One understands this: ‘If you hold on to and attach to these grounds for views it leads to such and such a destiny in the next life.’ He understands this, and what goes beyond this. Yet since he does not misapprehend that understanding, he has realized extinguishment within himself. Having truly understood the origin, ending, gratification, drawback, and escape from feelings, the Realized One is freed through not grasping. 

These\marginnote{2.15.5} are the principles—deep, hard to see, hard to understand, peaceful, sublime, beyond the scope of logic, subtle, comprehensible to the astute—which the Realized One makes known after realizing them with his own insight. And those who genuinely praise the Realized One would rightly speak of these things. 

\subsubsection*{3.1.3. The Cosmos is Finite or Infinite }

There\marginnote{2.16.1} are some ascetics and brahmins who theorize about size, and assert that the cosmos is finite or infinite on four grounds. And what are the four grounds on which they rely? 

It’s\marginnote{2.17.1} when some ascetic or brahmin—by dint of keen, resolute, committed, and diligent effort, and right focus—experiences an immersion of the heart of such a kind that they meditate perceiving the cosmos as finite. 

They\marginnote{2.17.2} say: ‘The cosmos is finite and bounded. Why is that? Because by dint of keen, resolute, committed, and diligent effort, and right focus I experience an immersion of the heart of such a kind that I meditate perceiving the cosmos as finite. Because of this I know: 

“The\marginnote{2.17.7} cosmos is finite and bounded.”’ This is the first ground on which some ascetics and brahmins rely to assert that the cosmos is finite or infinite. 

And\marginnote{2.18.1} what is the second ground on which they rely? It’s when some ascetic or brahmin—by dint of keen, resolute, committed, and diligent effort, and right focus—experiences an immersion of the heart of such a kind that they meditate perceiving the cosmos as infinite. 

They\marginnote{2.18.3} say: ‘The cosmos is infinite and unbounded. The ascetics and brahmins who say that the cosmos is finite are wrong. The cosmos is infinite and unbounded. Why is that? Because by dint of keen, resolute, committed, and diligent effort, and right focus I experience an immersion of the heart of such a kind that I meditate perceiving the cosmos as infinite. Because of this I know: 

“The\marginnote{2.18.11} cosmos is infinite and unbounded.”’ This is the second ground on which some ascetics and brahmins rely to assert that the cosmos is finite or infinite. 

And\marginnote{2.19.1} what is the third ground on which they rely? It’s when some ascetic or brahmin—by dint of keen, resolute, committed, and diligent effort, and right focus—experiences an immersion of the heart of such a kind that they meditate perceiving the cosmos as finite vertically but infinite horizontally. 

They\marginnote{2.19.3} say: ‘The cosmos is both finite and infinite. The ascetics and brahmins who say that the cosmos is finite are wrong, and so are those who say that the cosmos is infinite. The cosmos is both finite and infinite. Why is that? Because by dint of keen, resolute, committed, and diligent effort, and right focus I experience an immersion of the heart of such a kind that I meditate perceiving the cosmos as finite vertically but infinite horizontally. Because of this I know: 

“The\marginnote{2.19.13} cosmos is both finite and infinite.”’ This is the third ground on which some ascetics and brahmins rely to assert that the cosmos is finite or infinite. 

And\marginnote{2.20.1} what is the fourth ground on which they rely? It’s when some ascetic or brahmin relies on logic and inquiry. They speak of what they have worked out by logic, following a line of inquiry, expressing their own perspective: ‘The cosmos is neither finite nor infinite. The ascetics and brahmins who say that the cosmos is finite are wrong, as are those who say that the cosmos is infinite, and also those who say that the cosmos is both finite and infinite. The cosmos is neither finite nor infinite.’ This is the fourth ground on which some ascetics and brahmins rely to assert that the cosmos is finite or infinite. 

These\marginnote{2.21.1} are the four grounds on which those ascetics and brahmins assert that the cosmos is finite or infinite. Any ascetics and brahmins who assert that the cosmos is finite or infinite do so on one or other of these four grounds. Outside of this there is none. 

The\marginnote{2.22.1} Realized One understands this: ‘If you hold on to and attach to these grounds for views it leads to such and such a destiny in the next life.’ He understands this, and what goes beyond this. Yet since he does not misapprehend that understanding, he has realized extinguishment within himself. Having truly understood the origin, ending, gratification, drawback, and escape from feelings, the Realized One is freed through not grasping. 

These\marginnote{2.22.5} are the principles—deep, hard to see, hard to understand, peaceful, sublime, beyond the scope of logic, subtle, comprehensible to the astute—which the Realized One makes known after realizing them with his own insight. And those who genuinely praise the Realized One would rightly speak of these things. 

\subsubsection*{3.1.4. Equivocators }

There\marginnote{2.23.1} are some ascetics and brahmins who are equivocators. Whenever they’re asked a question, they resort to evasiveness and equivocation on four grounds. And what are the four grounds on which they rely? 

It’s\marginnote{2.24.1} when some ascetic or brahmin doesn’t truly understand what is skillful and what is unskillful. They think: ‘I don’t truly understand what is skillful and what is unskillful. If I were to declare that something was skillful or unskillful I might be wrong. That would be stressful for me, and that stress would be an obstacle.’ So from fear and disgust with false speech they avoid stating whether something is skillful or unskillful. Whenever they’re asked a question, they resort to evasiveness and equivocation: ‘I don’t say it’s like this. I don’t say it’s like that. I don’t say it’s otherwise. I don’t say it’s not so. And I don’t deny it’s not so.’ This is the first ground on which some ascetics and brahmins rely when resorting to evasiveness and equivocation. 

And\marginnote{2.25.1} what is the second ground on which they rely? It’s when some ascetic or brahmin doesn’t truly understand what is skillful and what is unskillful. They think: ‘I don’t truly understand what is skillful and what is unskillful. If I were to declare that something was skillful or unskillful I might feel desire or greed or hate or repulsion. That would be grasping on my part. That would be stressful for me, and that stress would be an obstacle.’ So from fear and disgust with grasping they avoid stating whether something is skillful or unskillful. Whenever they’re asked a question, they resort to evasiveness and equivocation: ‘I don’t say it’s like this. I don’t say it’s like that. I don’t say it’s otherwise. I don’t say it’s not so. And I don’t deny it’s not so.’ This is the second ground on which some ascetics and brahmins rely when resorting to evasiveness and equivocation. 

And\marginnote{2.26.1} what is the third ground on which they rely? It’s when some ascetic or brahmin doesn’t truly understand what is skillful and what is unskillful. They think: ‘I don’t truly understand what is skillful and what is unskillful. Suppose I were to declare that something was skillful or unskillful. There are clever ascetics and brahmins who are subtle, accomplished in the doctrines of others, hair-splitters. You’d think they live to demolish convictions with their intellect. They might pursue, press, and grill me about that. I’d be stumped by such a grilling. That would be stressful for me, and that stress would be an obstacle.’ So from fear and disgust with examination they avoid stating whether something is skillful or unskillful. Whenever they’re asked a question, they resort to evasiveness and equivocation: ‘I don’t say it’s like this. I don’t say it’s like that. I don’t say it’s otherwise. I don’t say it’s not so. And I don’t deny it’s not so.’ This is the third ground on which some ascetics and brahmins rely when resorting to evasiveness and equivocation. 

And\marginnote{2.27.1} what is the fourth ground on which they rely? It’s when some ascetic or brahmin is dull and stupid. Because of that, whenever they’re asked a question, they resort to evasiveness and equivocation: ‘Suppose you were to ask me whether there is another world. If I believed there was, I would say so. But I don’t say it’s like this. I don’t say it’s like that. I don’t say it’s otherwise. I don’t say it’s not so. And I don’t deny it’s not so. Suppose you were to ask me whether there is no other world … whether there both is and is not another world … whether there neither is nor is not another world … whether there are beings who are reborn spontaneously … whether there are not beings who are reborn spontaneously … whether there both are and are not beings who are reborn spontaneously … whether there neither are nor are not beings who are reborn spontaneously … whether there is fruit and result of good and bad deeds … whether there is not fruit and result of good and bad deeds … whether there both is and is not fruit and result of good and bad deeds … whether there neither is nor is not fruit and result of good and bad deeds … whether a Realized One exists after death … whether a Realized One doesn’t exist after death … whether a Realized One both exists and doesn’t exist after death … whether a Realized One neither exists nor doesn’t exist after death. If I believed there was, I would say so. But I don’t say it’s like this. I don’t say it’s like that. I don’t say it’s otherwise. I don’t say it’s not so. And I don’t deny it’s not so.’ This is the fourth ground on which some ascetics and brahmins rely when resorting to evasiveness and equivocation. 

These\marginnote{2.28.1} are the four grounds on which those ascetics and brahmins who are equivocators resort to evasiveness and equivocation whenever they’re asked a question. Any ascetics and brahmins who resort to equivocation do so on one or other of these four grounds. Outside of this there is none. The Realized One understands this … And those who genuinely praise the Realized One would rightly speak of these things. 

\subsubsection*{3.1.5. Doctrines of Origination by Chance }

There\marginnote{2.30.1} are some ascetics and brahmins who theorize about chance. They assert that the self and the cosmos arose by chance on two grounds. And what are the two grounds on which they rely? 

There\marginnote{2.31.1} are gods named ‘non-percipient beings’. When perception arises they pass away from that host of gods. It’s possible that one of those beings passes away from that host and is reborn in this state of existence. Having done so, they go forth from the lay life to homelessness. By dint of keen, resolute, committed, and diligent effort, and right focus, they experience an immersion of the heart of such a kind that they recollect the arising of perception, but no further. They say: ‘The self and the cosmos arose by chance. Why is that? Because formerly I didn’t exist. Now, having not been, I’ve sprung into existence.’ This is the first ground on which some ascetics and brahmins rely to assert that the self and the cosmos arose by chance. 

And\marginnote{2.32.1} what is the second ground on which they rely? It’s when some ascetic or brahmin relies on logic and inquiry. They speak of what they have worked out by logic, following a line of inquiry, expressing their own perspective: ‘The self and the cosmos arose by chance.’ This is the second ground on which some ascetics and brahmins rely to assert that the self and the cosmos arose by chance. 

These\marginnote{2.33.1} are the two grounds on which those ascetics and brahmins who theorize about chance assert that the self and the cosmos arose by chance. Any ascetics and brahmins who theorize about chance do so on one or other of these two grounds. Outside of this there is none. The Realized One understands this … And those who genuinely praise the Realized One would rightly speak of these things. 

These\marginnote{2.35.1} are the eighteen grounds on which those ascetics and brahmins who theorize about the past assert various hypotheses concerning the past. Any ascetics and brahmins who theorize about the past do so on one or other of these eighteen grounds. Outside of this there is none. 

The\marginnote{2.36.1} Realized One understands this: ‘If you hold on to and attach to these grounds for views it leads to such and such a destiny in the next life.’ He understands this, and what goes beyond this. Yet since he does not misapprehend that understanding, he has realized extinguishment within himself. Having truly understood the origin, ending, gratification, drawback, and escape from feelings, the Realized One is freed through not grasping. 

These\marginnote{2.36.5} are the principles—deep, hard to see, hard to understand, peaceful, sublime, beyond the scope of logic, subtle, comprehensible to the astute—which the Realized One makes known after realizing them with his own insight. And those who genuinely praise the Realized One would rightly speak of these things. 

\subsection*{3.2. Theories About the Future }

There\marginnote{2.37.1} are some ascetics and brahmins who theorize about the future, and assert various hypotheses concerning the future on forty-four grounds. And what are the forty-four grounds on which they rely? 

\subsubsection*{3.2.1. Percipient Life After Death }

There\marginnote{2.38.1} are some ascetics and brahmins who say there is life after death, and assert that the self lives on after death in a percipient form on sixteen grounds. And what are the sixteen grounds on which they rely? 

They\marginnote{2.38.3} assert: ‘The self is well and percipient after death, and it has form … 

formless\marginnote{2.38.4} … 

both\marginnote{2.38.5} formed and formless … 

neither\marginnote{2.38.6} formed nor formless … 

finite\marginnote{2.38.7} … 

infinite\marginnote{2.38.8} … 

both\marginnote{2.38.9} finite and infinite … 

neither\marginnote{2.38.10} finite nor infinite … 

of\marginnote{2.38.11} unified perception … 

of\marginnote{2.38.12} diverse perception … 

of\marginnote{2.38.13} limited perception … 

of\marginnote{2.38.14} limitless perception … 

experiences\marginnote{2.38.15} nothing but happiness … 

experiences\marginnote{2.38.16} nothing but suffering … 

experiences\marginnote{2.38.17} both happiness and suffering … 

experiences\marginnote{2.38.18} neither happiness nor suffering.’ 

These\marginnote{2.39.1} are the sixteen grounds on which those ascetics and brahmins assert that the self lives on after death in a percipient form. Any ascetics and brahmins who assert that the self lives on after death in a percipient form do so on one or other of these sixteen grounds. Outside of this there is none. The Realized One understands this … And those who genuinely praise the Realized One would rightly speak of these things. 

\subsubsection*{3.2.2. Non-Percipient Life After Death }

There\marginnote{3.1.1} are some ascetics and brahmins who say there is life after death, and assert that the self lives on after death in a non-percipient form on eight grounds. And what are the eight grounds on which they rely? 

They\marginnote{3.2.1} assert: ‘The self is well and non-percipient after death, and it has form … 

formless\marginnote{3.2.2} … 

both\marginnote{3.2.3} formed and formless … 

neither\marginnote{3.2.4} formed nor formless … 

finite\marginnote{3.2.5} … 

infinite\marginnote{3.2.6} … 

both\marginnote{3.2.7} finite and infinite … 

neither\marginnote{3.2.8} finite nor infinite.’ 

These\marginnote{3.3.1} are the eight grounds on which those ascetics and brahmins assert that the self lives on after death in a non-percipient form. Any ascetics and brahmins who assert that the self lives on after death in a non-percipient form do so on one or other of these eight grounds. Outside of this there is none. The Realized One understands this … And those who genuinely praise the Realized One would rightly speak of these things. 

\subsubsection*{3.2.3. Neither Percipient Nor Non-Percipient Life After Death }

There\marginnote{3.5.1} are some ascetics and brahmins who say there is life after death, and assert that the self lives on after death in a neither percipient nor non-percipient form on eight grounds. And what are the eight grounds on which they rely? 

They\marginnote{3.6.1} assert: ‘The self is well and neither percipient nor non-percipient after death, and it has form … 

formless\marginnote{3.6.2} … 

both\marginnote{3.6.3} formed and formless … 

neither\marginnote{3.6.4} formed nor formless … 

finite\marginnote{3.6.5} … 

infinite\marginnote{3.6.6} … 

both\marginnote{3.6.7} finite and infinite … 

neither\marginnote{3.6.8} finite nor infinite.’ 

These\marginnote{3.7.1} are the eight grounds on which those ascetics and brahmins assert that the self lives on after death in a neither percipient nor non-percipient form. Any ascetics and brahmins who assert that the self lives on after death in a neither percipient nor non-percipient form do so on one or other of these eight grounds. Outside of this there is none. The Realized One understands this … And those who genuinely praise the Realized One would rightly speak of these things. 

\subsubsection*{3.2.4. Annihilationism }

There\marginnote{3.9.1} are some ascetics and brahmins who are annihilationists. They assert the annihilation, eradication, and obliteration of an existing being on seven grounds. And what are the seven grounds on which they rely? 

There\marginnote{3.10.1} are some ascetics and brahmins who have this doctrine and view: ‘This self has form, made up of the four primary elements, and produced by mother and father. Since it’s annihilated and destroyed when the body breaks up, and doesn’t exist after death, that’s how this self becomes rightly annihilated.’ That is how some assert the annihilation of an existing being. 

But\marginnote{3.11.1} someone else says to them: ‘\emph{That} self of which you speak does exist, I don’t deny it. But that’s not how \emph{this} self becomes rightly annihilated. There is another self that is divine, formed, sensual, consuming solid food. You don’t know or see that. But I know it and see it. Since this self is annihilated and destroyed when the body breaks up, and doesn’t exist after death, that’s how this self becomes rightly annihilated.’ That is how some assert the annihilation of an existing being. 

But\marginnote{3.12.1} someone else says to them: ‘\emph{That} self of which you speak does exist, I don’t deny it. But that’s not how \emph{this} self becomes rightly annihilated. There is another self that is divine, formed, mind-made, complete in all its various parts, not deficient in any faculty. You don’t know or see that. But I know it and see it. Since this self is annihilated and destroyed when the body breaks up, and doesn’t exist after death, that’s how this self becomes rightly annihilated.’ That is how some assert the annihilation of an existing being. 

But\marginnote{3.13.1} someone else says to them: ‘\emph{That} self of which you speak does exist, I don’t deny it. But that’s not how \emph{this} self becomes rightly annihilated. There is another self which has gone totally beyond perceptions of form. With the ending of perceptions of impingement, not focusing on perceptions of diversity, aware that “space is infinite”, it’s reborn in the dimension of infinite space. You don’t know or see that. But I know it and see it. Since this self is annihilated and destroyed when the body breaks up, and doesn’t exist after death, that’s how this self becomes rightly annihilated.’ That is how some assert the annihilation of an existing being. 

But\marginnote{3.14.1} someone else says to them: ‘\emph{That} self of which you speak does exist, I don’t deny it. But that’s not how \emph{this} self becomes rightly annihilated. There is another self which has gone totally beyond the dimension of infinite space. Aware that “consciousness is infinite”, it’s reborn in the dimension of infinite consciousness. You don’t know or see that. But I know it and see it. Since this self is annihilated and destroyed when the body breaks up, and doesn’t exist after death, that’s how this self becomes rightly annihilated.’ That is how some assert the annihilation of an existing being. 

But\marginnote{3.15.1} someone else says to them: ‘\emph{That} self of which you speak does exist, I don’t deny it. But that’s not how \emph{this} self becomes rightly annihilated. There is another self that has gone totally beyond the dimension of infinite consciousness. Aware that “there is nothing at all”, it’s been reborn in the dimension of nothingness. You don’t know or see that. But I know it and see it. Since this self is annihilated and destroyed when the body breaks up, and doesn’t exist after death, that’s how this self becomes rightly annihilated.’ That is how some assert the annihilation of an existing being. 

But\marginnote{3.16.1} someone else says to them: ‘\emph{That} self of which you speak does exist, I don’t deny it. But that’s not how \emph{this} self becomes rightly annihilated. There is another self that has gone totally beyond the dimension of nothingness. Aware that “this is peaceful, this is sublime”, it’s been reborn in the dimension of neither perception nor non-perception. You don’t know or see that. But I know it and see it. Since this self is annihilated and destroyed when the body breaks up, and doesn’t exist after death, that’s how this self becomes rightly annihilated.’ That is how some assert the annihilation of an existing being. 

These\marginnote{3.17.1} are the seven grounds on which those ascetics and brahmins assert the annihilation, eradication, and obliteration of an existing being. Any ascetics and brahmins who assert the annihilation, eradication, and obliteration of an existing being do so on one or other of these seven grounds. Outside of this there is none. The Realized One understands this … And those who genuinely praise the Realized One would rightly speak of these things. 

\subsubsection*{3.2.5. Extinguishment in the Present Life }

There\marginnote{3.19.1} are some ascetics and brahmins who speak of extinguishment in the present life. They assert the ultimate extinguishment of an existing being in the present life on five grounds. And what are the five grounds on which they rely? 

There\marginnote{3.20.1} are some ascetics and brahmins who have this doctrine and view: ‘When this self amuses itself, supplied and provided with the five kinds of sensual stimulation, that’s how this self attains ultimate extinguishment in the present life.’ That is how some assert the extinguishment of an existing being in the present life. 

But\marginnote{3.21.1} someone else says to them: ‘\emph{That} self of which you speak does exist, I don’t deny it. But that’s not how \emph{this} self attains ultimate extinguishment in the present life. Why is that? Because sensual pleasures are impermanent, suffering, and perishable. Their decay and perishing give rise to sorrow, lamentation, pain, sadness, and distress. Quite secluded from sensual pleasures, secluded from unskillful qualities, this self enters and remains in the first absorption, which has the rapture and bliss born of seclusion, while placing the mind and keeping it connected. That’s how this self attains ultimate extinguishment in the present life.’ That is how some assert the extinguishment of an existing being in the present life. 

But\marginnote{3.22.1} someone else says to them: ‘\emph{That} self of which you speak does exist, I don’t deny it. But that’s not how \emph{this} self attains ultimate extinguishment in the present life. Why is that? Because the placing of the mind and the keeping it connected there are coarse. But when the placing of the mind and keeping it connected are stilled, this self enters and remains in the second absorption, which has the rapture and bliss born of immersion, with internal clarity and confidence, and unified mind, without placing the mind and keeping it connected. That’s how this self attains ultimate extinguishment in the present life.’ That is how some assert the extinguishment of an existing being in the present life. 

But\marginnote{3.23.1} someone else says to them: ‘\emph{That} self of which you speak does exist, I don’t deny it. But that’s not how \emph{this} self attains ultimate extinguishment in the present life. Why is that? Because the rapture and emotional excitement there are coarse. But with the fading away of rapture, this self enters and remains in the third absorption, where it meditates with equanimity, mindful and aware, personally experiencing the bliss of which the noble ones declare, “Equanimous and mindful, one meditates in bliss”. That’s how this self attains ultimate extinguishment in the present life.’ That is how some assert the extinguishment of an existing being in the present life. 

But\marginnote{3.24.1} someone else says to them: ‘\emph{That} self of which you speak does exist, I don’t deny it. But that’s not how \emph{this} self attains ultimate extinguishment in the present life. Why is that? Because the bliss and enjoyment there are coarse. But giving up pleasure and pain, and ending former happiness and sadness, this self enters and remains in the fourth absorption, without pleasure or pain, with pure equanimity and mindfulness. That’s how this self attains ultimate extinguishment in the present life.’ That is how some assert the extinguishment of an existing being in the present life. 

These\marginnote{3.25.1} are the five grounds on which those ascetics and brahmins assert the ultimate extinguishment of an existing being in the present life. Any ascetics and brahmins who assert the ultimate extinguishment of an existing being in the present life do so on one or other of these five grounds. Outside of this there is none. The Realized One understands this … And those who genuinely praise the Realized One would rightly speak of these things. 

These\marginnote{3.27.1} are the forty-four grounds on which those ascetics and brahmins who theorize about the future assert various hypotheses concerning the future. Any ascetics and brahmins who theorize about the future do so on one or other of these forty-four grounds. Outside of this there is none. The Realized One understands this … And those who genuinely praise the Realized One would rightly speak of these things. 

These\marginnote{3.29.1} are the sixty-two grounds on which those ascetics and brahmins who theorize about the past and the future assert various hypotheses concerning the past and the future. 

Any\marginnote{3.29.2} ascetics and brahmins who theorize about the past or the future do so on one or other of these sixty-two grounds. Outside of this there is none. 

The\marginnote{3.30.1} Realized One understands this: ‘If you hold on to and attach to these grounds for views it leads to such and such a destiny in the next life.’ He understands this, and what goes beyond this. Yet since he does not misapprehend that understanding, he has realized extinguishment within himself. Having truly understood the origin, ending, gratification, drawback, and escape from feelings, the Realized One is freed through not grasping. 

These\marginnote{3.31.1} are the principles—deep, hard to see, hard to understand, peaceful, sublime, beyond the scope of logic, subtle, comprehensible to the astute—which the Realized One makes known after realizing them with his own insight. And those who genuinely praise the Realized One would rightly speak of these things. 

\section*{4. The Grounds For Assertions About the Self and the Cosmos }

\subsection*{4.1. Anxiety and Evasiveness }

Now,\marginnote{3.32.1} these things are only the feeling of those who do not know or see, the agitation and evasiveness of those under the sway of craving. Namely, when those ascetics and brahmins assert that the self and the cosmos are eternal on four grounds … 

partially\marginnote{3.33.1} eternal on four grounds … 

finite\marginnote{3.34.1} or infinite on four grounds … 

or\marginnote{3.35.1} they resort to equivocation on four grounds … 

or\marginnote{3.36.1} they assert that the self and the cosmos arose by chance on two grounds … 

they\marginnote{3.37.1} theorize about the past on these eighteen grounds … 

or\marginnote{3.38.1} they assert that the self lives on after death in a percipient form on sixteen grounds … 

or\marginnote{3.39.1} that the self lives on after death in a non-percipient form on eight grounds … 

or\marginnote{3.40.1} that the self lives on after death in a neither percipient nor non-percipient form on eight grounds … 

or\marginnote{3.41.1} they assert the annihilation of an existing being on seven grounds … 

or\marginnote{3.42.1} they assert the ultimate extinguishment of an existing being in the present life on five grounds … 

they\marginnote{3.43.1} theorize about the future on these forty-four grounds … 

When\marginnote{3.44.1} those ascetics and brahmins theorize about the past and the future on these sixty-two grounds, these things are only the feeling of those who do not know or see, the agitation and evasiveness of those under the sway of craving. 

\subsection*{4.2. Conditioned by Contact }

Now,\marginnote{3.45.1} these things are conditioned by contact. Namely, when those ascetics and brahmins assert that the self and the cosmos are eternal on four grounds … 

partially\marginnote{3.46.1} eternal on four grounds … 

finite\marginnote{3.47.1} or infinite on four grounds … 

or\marginnote{3.48.1} they resort to equivocation on four grounds … 

or\marginnote{3.49.1} they assert that the self and the cosmos arose by chance on two grounds … 

they\marginnote{3.50.1} theorize about the past on these eighteen grounds … 

or\marginnote{3.51.1} they assert that the self lives on after death in a percipient form on sixteen grounds … 

or\marginnote{3.52.1} that the self lives on after death in a non-percipient form on eight grounds … 

or\marginnote{3.53.1} that the self lives on after death in a neither percipient nor non-percipient form on eight grounds … 

or\marginnote{3.54.1} they assert the annihilation of an existing being on seven grounds … 

or\marginnote{3.55.1} they assert the ultimate extinguishment of an existing being in the present life on five grounds … 

they\marginnote{3.56.1} theorize about the future on these forty-four grounds … 

When\marginnote{3.57.1} those ascetics and brahmins theorize about the past and the future on these sixty-two grounds, that too is conditioned by contact. 

\subsection*{4.3. Not Possible }

Now,\marginnote{3.70.1} when those ascetics and brahmins theorize about the past and the future on these sixty-two grounds, it is not possible that they should experience these things without contact. 

\subsection*{4.4. Dependent Origination }

Now,\marginnote{3.71.1} when those ascetics and brahmins theorize about the past and the future on these sixty-two grounds, all of them experience this by repeated contact through the six fields of contact. Their feeling is a condition for craving. Craving is a condition for grasping. Grasping is a condition for continued existence. Continued existence is a condition for rebirth. Rebirth is a condition for old age and death, sorrow, lamentation, pain, sadness, and distress to come to be. 

\section*{5. The End of the Round }

When\marginnote{3.72.1} a mendicant truly understands the six fields of contact’s origin, ending, gratification, drawback, and escape, they understand what lies beyond all these things. 

All\marginnote{3.72.2} of these ascetics and brahmins who theorize about the past or the future are trapped in the net of these sixty-two grounds, so that wherever they emerge they are caught and trapped in this very net. 

Suppose\marginnote{3.72.3} a deft fisherman or his apprentice were to cast a fine-meshed net over a small pond. They’d think: ‘Any sizable creatures in this pond will be trapped in the net. Wherever they emerge they are caught and trapped in this very net.’ In the same way, all of these ascetics and brahmins who theorize about the past or the future are trapped in the net of these sixty-two grounds, so that wherever they emerge they are caught and trapped in this very net. 

The\marginnote{3.73.1} Realized One’s body remains, but his conduit to rebirth has been cut off. As long as his body remains he will be seen by gods and humans. But when his body breaks up, after life has ended, gods and humans will see him no more. 

When\marginnote{3.73.4} the stalk of a bunch of mangoes is cut, all the mangoes attached to the stalk will follow along. In the same way, the Realized One’s body remains, but his conduit to rebirth has been cut off. As long as his body remains he will be seen by gods and humans. But when his body breaks up, after life has ended, gods and humans will see him no more.” 

When\marginnote{3.74.1} he had spoken, Venerable Ānanda said to the Buddha, “It’s incredible, sir, it’s amazing! What is the name of this exposition of the teaching?” 

“Well,\marginnote{3.74.3} then, Ānanda, you may remember this exposition of the teaching as ‘The Net of Meaning’, or else ‘The Net of the Teaching’, or else ‘The Prime Net’, or else ‘The Net of Views’, or else ‘The Supreme Victory in Battle’.” 

That\marginnote{3.74.4} is what the Buddha said. Satisfied, the mendicants were happy with what the Buddha said. And while this discourse was being spoken, the galaxy shook. 

%
\chapter*{{\suttatitleacronym DN 2}{\suttatitletranslation The Fruits of the Ascetic Life }{\suttatitleroot Sāmaññaphalasutta}}
\addcontentsline{toc}{chapter}{\tocacronym{DN 2} \toctranslation{The Fruits of the Ascetic Life } \tocroot{Sāmaññaphalasutta}}
\markboth{The Fruits of the Ascetic Life }{Sāmaññaphalasutta}
\extramarks{DN 2}{DN 2}

\section*{1. A Discussion With the King’s Ministers }

\scevam{So\marginnote{1.1} I have heard. }At one time the Buddha was staying near \textsanskrit{Rājagaha} in the Mango Grove of \textsanskrit{Jīvaka} \textsanskrit{Komārabhacca}, together with a large \textsanskrit{Saṅgha} of 1,250 mendicants. 

Now,\marginnote{1.3} at that time it was the sabbath—the Komudi full moon on the fifteenth day of the fourth month—and King \textsanskrit{Ajātasattu} Vedehiputta of Magadha was sitting upstairs in the royal longhouse surrounded by his ministers. 

Then\marginnote{1.4} \textsanskrit{Ajātasattu} expressed this heartfelt sentiment, “Oh, sirs, this moonlit night is so very delightful, so beautiful, so glorious, so lovely, so striking. Now, what ascetic or brahmin might I pay homage to today, paying homage to whom my mind might find peace?” 

When\marginnote{2.1} he had spoken, one of the king’s ministers said to him, “Sire, \textsanskrit{Pūraṇa} Kassapa leads an order and a community, and teaches a community. He’s a well-known and famous religious founder, regarded as holy by many people. He is of long standing, long gone forth; he is advanced in years and has reached the final stage of life. Let Your Majesty pay homage to him. Hopefully in so doing your mind will find peace.” But when he had spoken, the king kept silent. 

Another\marginnote{3.1} of the king’s ministers said to him, “Sire, Makkhali \textsanskrit{Gosāla} leads an order and a community, and teaches a community. He’s a well-known and famous religious founder, regarded as holy by many people. He is of long standing, long gone forth; he is advanced in years and has reached the final stage of life. Let Your Majesty pay homage to him. Hopefully in so doing your mind will find peace.” But when he had spoken, the king kept silent. 

Another\marginnote{4.1} of the king’s ministers said to him, “Sire, Ajita Kesakambala leads an order and a community, and teaches a community. He’s a well-known and famous religious founder, regarded as holy by many people. He is of long standing, long gone forth; he is advanced in years and has reached the final stage of life. Let Your Majesty pay homage to him. Hopefully in so doing your mind will find peace.” But when he had spoken, the king kept silent. 

Another\marginnote{5.1} of the king’s ministers said to him, “Sire, Pakudha \textsanskrit{Kaccāyana} leads an order and a community, and teaches a community. He’s a well-known and famous religious founder, regarded as holy by many people. He is of long standing, long gone forth; he is advanced in years and has reached the final stage of life. Let Your Majesty pay homage to him. Hopefully in so doing your mind will find peace.” But when he had spoken, the king kept silent. 

Another\marginnote{6.1} of the king’s ministers said to him, “Sire, \textsanskrit{Sañjaya} \textsanskrit{Belaṭṭhiputta} leads an order and a community, and teaches a community. He’s a well-known and famous religious founder, regarded as holy by many people. He is of long standing, long gone forth; he is advanced in years and has reached the final stage of life. Let Your Majesty pay homage to him. Hopefully in so doing your mind will find peace.” But when he had spoken, the king kept silent. 

Another\marginnote{7.1} of the king’s ministers said to him, “Sire, \textsanskrit{Nigaṇṭha} \textsanskrit{Nātaputta} leads an order and a community, and teaches a community. He’s a well-known and famous religious founder, regarded as holy by many people. He is of long standing, long gone forth; he is advanced in years and has reached the final stage of life. Let Your Majesty pay homage to him. Hopefully in so doing your mind will find peace.” But when he had spoken, the king kept silent. 

\section*{2. A Discussion With \textsanskrit{Jīvaka} \textsanskrit{Komārabhacca} }

Now\marginnote{8.1} at that time \textsanskrit{Jīvaka} \textsanskrit{Komārabhacca} was sitting silently not far from the king. Then the king said to him, “But my dear \textsanskrit{Jīvaka}, why are you silent?” 

“Sire,\marginnote{8.4} the Blessed One, the perfected one, the fully awakened Buddha is staying in my mango grove together with a large \textsanskrit{Saṅgha} of 1,250 mendicants. He has this good reputation: ‘That Blessed One is perfected, a fully awakened Buddha, accomplished in knowledge and conduct, holy, knower of the world, supreme guide for those who wish to train, teacher of gods and humans, awakened, blessed.’ Let Your Majesty pay homage to him. Hopefully in so doing your mind will find peace.” 

“Well\marginnote{8.9} then, my dear \textsanskrit{Jīvaka}, have the elephants readied.” 

“Yes,\marginnote{9.1} Your Majesty,” replied \textsanskrit{Jīvaka}. He had around five hundred female elephants readied, in addition to the king’s bull elephant for riding. Then he informed the king, “The elephants are ready, sire. Please go at your convenience.” 

Then\marginnote{10.1} King \textsanskrit{Ajātasattu} had women mounted on each of the five hundred female elephants, while he mounted his bull elephant. With attendants carrying torches, he set out in full royal pomp from \textsanskrit{Rājagaha} to \textsanskrit{Jīvaka}’s mango grove. 

But\marginnote{10.2} as he drew near the mango grove, the king became frightened, scared, his hair standing on end. He said to \textsanskrit{Jīvaka}, “My dear \textsanskrit{Jīvaka}, I hope you’re not deceiving me! I hope you’re not betraying me! I hope you’re not turning me over to my enemies! For how on earth can there be no sound of coughing or clearing throats or any noise in such a large \textsanskrit{Saṅgha} of 1,250 mendicants?” 

“Do\marginnote{10.8} not fear, great king, do not fear! I am not deceiving you, or betraying you, or turning you over to your enemies. Go forward, great king, go forward! Those are lamps shining in the pavilion.” 

\section*{3. The Question About the Fruits of the Ascetic Life }

Then\marginnote{11.1} King \textsanskrit{Ajātasattu} rode on the elephant as far as the terrain allowed, then descended and approached the pavilion door on foot, where he asked \textsanskrit{Jīvaka}, “But my dear \textsanskrit{Jīvaka}, where is the Buddha?” 

“That\marginnote{11.3} is the Buddha, great king, that is the Buddha! He’s sitting against the central column facing east, in front of the \textsanskrit{Saṅgha} of mendicants.” 

Then\marginnote{12.1} the king went up to the Buddha and stood to one side. He looked around the \textsanskrit{Saṅgha} of monks, who were so very silent, like a still, clear lake, and expressed this heartfelt sentiment, “May my son, Prince \textsanskrit{Udāyibhadda}, be blessed with such peace as the \textsanskrit{Saṅgha} of mendicants now enjoys!” 

“Has\marginnote{12.4} your mind gone to one you love, great king?” 

“I\marginnote{12.5} love my son, sir, Prince \textsanskrit{Udāyibhadda}. May he be blessed with such peace as the \textsanskrit{Saṅgha} of mendicants now enjoys!” 

Then\marginnote{13.1} the king bowed to the Buddha, raised his joined palms toward the \textsanskrit{Saṅgha}, and sat down to one side. He said to the Buddha, “Sir, I’d like to ask you about a certain point, if you’d take the time to answer.” 

“Ask\marginnote{13.5} what you wish, great king.” 

“Sir,\marginnote{14.1} there are many different professional fields. These include elephant riders, cavalry, charioteers, archers, bannermen, adjutants, food servers, warrior-chiefs, princes, chargers, great warriors, heroes, leather-clad soldiers, and sons of bondservants. They also include bakers, barbers, bathroom attendants, cooks, garland-makers, dyers, weavers, basket-makers, potters, accountants, finger-talliers, or those following any similar professions. All these live off the fruits of their profession which are apparent in the present life. With that they bring happiness and joy to themselves, their parents, their children and partners, and their friends and colleagues. And they establish an uplifting religious donation for ascetics and brahmins that’s conducive to heaven, ripens in happiness, and leads to heaven. Sir, can you point out a fruit of the ascetic life that’s likewise apparent in the present life?” 

“Great\marginnote{15.1} king, do you recall having asked this question of other ascetics and brahmins?” 

“I\marginnote{15.2} do, sir.” 

“If\marginnote{15.3} you wouldn’t mind, great king, tell me how they answered.” 

“It’s\marginnote{15.4} no trouble when someone such as the Blessed One is sitting here.” 

“Well,\marginnote{15.5} speak then, great king.” 

\subsection*{3.1. The Doctrine of \textsanskrit{Pūraṇa} Kassapa }

“One\marginnote{16.1} time, sir, I approached \textsanskrit{Pūraṇa} Kassapa and exchanged greetings with him. When the greetings and polite conversation were over, I sat down to one side, and asked him the same question. 

He\marginnote{17.1} said to me: ‘Great king, the one who acts does nothing wrong when they punish, mutilate, torture, aggrieve, oppress, intimidate, or when they encourage others to do the same. They do nothing wrong when they kill, steal, break into houses, plunder wealth, steal from isolated buildings, commit highway robbery, commit adultery, and lie. If you were to reduce all the living creatures of this earth to one heap and mass of flesh with a razor-edged chakram, no evil comes of that, and no outcome of evil. If you were to go along the south bank of the Ganges killing, mutilating, and torturing, and encouraging others to do the same, no evil comes of that, and no outcome of evil. If you were to go along the north bank of the Ganges giving and sacrificing and encouraging others to do the same, no merit comes of that, and no outcome of merit. In giving, self-control, restraint, and truthfulness there is no merit or outcome of merit.’ 

And\marginnote{18.1} so, when I asked \textsanskrit{Pūraṇa} Kassapa about the fruits of the ascetic life apparent in the present life, he answered with the doctrine of inaction. It was like someone who, when asked about a mango, answered with a breadfruit, or when asked about a breadfruit, answered with a mango. I thought: ‘How could one such as I presume to rebuke an ascetic or brahmin living in my realm?’ So I neither approved nor dismissed that statement of \textsanskrit{Pūraṇa} Kassapa. I was displeased, but did not express my displeasure. Neither accepting what he said nor contradicting it, I got up from my seat and left. 

\subsection*{3.2. The Doctrine of Makkhali \textsanskrit{Gosāla} }

One\marginnote{19.1} time, sir, I approached Makkhali \textsanskrit{Gosāla} and exchanged greetings with him. When the greetings and polite conversation were over, I sat down to one side, and asked him the same question. 

He\marginnote{20.1} said: ‘Great king, there is no cause or reason for the corruption of sentient beings. Sentient beings are corrupted without cause or reason. There’s no cause or reason for the purification of sentient beings. Sentient beings are purified without cause or reason. One does not act of one’s own volition, one does not act of another’s volition, one does not act from a person’s volition. There is no power, no energy, no human strength or vigor. All sentient beings, all living creatures, all beings, all souls lack control, power, and energy. Molded by destiny, circumstance, and nature, they experience pleasure and pain in the six classes of rebirth. There are 1.4 million main wombs, and 6,000, and 600. There are 500 deeds, and five, and three. There are deeds and half-deeds. There are 62 paths, 62 sub-eons, six classes of rebirth, and eight stages in a person’s life. There are 4,900 \textsanskrit{Ājīvaka} ascetics, 4,900 wanderers, and 4,900 naked ascetics. There are 2,000 faculties, 3,000 hells, and 36 realms of dust. There are seven percipient embryos, seven non-percipient embryos, and seven embryos without attachments. There are seven gods, seven humans, and seven goblins. There are seven lakes, seven winds, 700 winds, seven cliffs, and 700 cliffs. There are seven dreams and 700 dreams. There are 8.4 million great eons through which the foolish and the astute transmigrate before making an end of suffering. And here there is no such thing as this: “By this precept or observance or mortification or spiritual life I shall force unripened deeds to bear their fruit, or eliminate old deeds by experiencing their results little by little,” for that cannot be. Pleasure and pain are allotted. Transmigration lasts only for a limited period, so there’s no increase or decrease, no getting better or worse. It’s like how, when you toss a ball of string, it rolls away unraveling. In the same way, after transmigrating the foolish and the astute will make an end of suffering.’ 

And\marginnote{21.1} so, when I asked Makkhali \textsanskrit{Gosāla} about the fruits of the ascetic life apparent in the present life, he answered with the doctrine of purification through transmigration. It was like someone who, when asked about a mango, answered with a breadfruit, or when asked about a breadfruit, answered with a mango. I thought: ‘How could one such as I presume to rebuke an ascetic or brahmin living in my realm?’ So I neither approved nor dismissed that statement of Makkhali \textsanskrit{Gosāla}. I was displeased, but did not express my displeasure. Neither accepting what he said nor contradicting it, I got up from my seat and left. 

\subsection*{3.3. The Doctrine of Ajita Kesakambala }

One\marginnote{22.1} time, sir, I approached Ajita Kesakambala and exchanged greetings with him. When the greetings and polite conversation were over, I sat down to one side, and asked him the same question. 

He\marginnote{23.1} said: ‘Great king, there is no meaning in giving, sacrifice, or offerings. There’s no fruit or result of good and bad deeds. There’s no afterlife. There’s no such thing as mother and father, or beings that are reborn spontaneously. And there’s no ascetic or brahmin who is well attained and practiced, and who describes the afterlife after realizing it with their own insight. This person is made up of the four primary elements. When they die, the earth in their body merges and coalesces with the main mass of earth. The water in their body merges and coalesces with the main mass of water. The fire in their body merges and coalesces with the main mass of fire. The air in their body merges and coalesces with the main mass of air. The faculties are transferred to space. Four men with a bier carry away the corpse. Their footprints show the way to the cemetery. The bones become bleached. Offerings dedicated to the gods end in ashes. Giving is a doctrine of morons. When anyone affirms a positive teaching it’s just hollow, false nonsense. Both the foolish and the astute are annihilated and destroyed when their body breaks up, and don’t exist after death.’ 

And\marginnote{24.1} so, when I asked Ajita Kesakambala about the fruits of the ascetic life apparent in the present life, he answered with the doctrine of annihilationism. It was like someone who, when asked about a mango, answered with a breadfruit, or when asked about a breadfruit, answered with a mango. I thought: ‘How could one such as I presume to rebuke an ascetic or brahmin living in my realm?’ So I neither approved nor dismissed that statement of Ajita Kesakambala. I was displeased, but did not express my displeasure. Neither accepting what he said nor contradicting it, I got up from my seat and left. 

\subsection*{3.4. The Doctrine of Pakudha \textsanskrit{Kaccāyana} }

One\marginnote{25.1} time, sir, I approached Pakudha \textsanskrit{Kaccāyana} and exchanged greetings with him. When the greetings and polite conversation were over, I sat down to one side, and asked him the same question. 

He\marginnote{26.1} said: ‘Great king, these seven substances are not made, not derived, not created, without a creator, barren, steady as a mountain peak, standing firm like a pillar. They don’t move or deteriorate or obstruct each other. They’re unable to cause pleasure, pain, or neutral feeling to each other. What seven? The substances of earth, water, fire, air; pleasure, pain, and the soul is the seventh. These seven substances are not made, not derived, not created, without a creator, barren, steady as a mountain peak, standing firm like a pillar. They don’t move or deteriorate or obstruct each other. They’re unable to cause pleasure, pain, or neutral feeling to each other. And here there is no-one who kills or who makes others kill; no-one who learns or who educates others; no-one who understands or who helps others understand. If you chop off someone’s head with a sharp sword, you don’t take anyone’s life. The sword simply passes through the gap between the seven substances.’ 

And\marginnote{27.1} so, when I asked Pakudha \textsanskrit{Kaccāyana} about the fruits of the ascetic life apparent in the present life, he answered with something else entirely. It was like someone who, when asked about a mango, answered with a breadfruit, or when asked about a breadfruit, answered with a mango. I thought: ‘How could one such as I presume to rebuke an ascetic or brahmin living in my realm?’ So I neither approved nor dismissed that statement of Pakudha \textsanskrit{Kaccāyana}. I was displeased, but did not express my displeasure. Neither accepting what he said nor contradicting it, I got up from my seat and left. 

\subsection*{3.5. The Doctrine of \textsanskrit{Nigaṇṭha} \textsanskrit{Nātaputta} }

One\marginnote{28.1} time, sir, I approached \textsanskrit{Nigaṇṭha} \textsanskrit{Nātaputta} and exchanged greetings with him. When the greetings and polite conversation were over, I sat down to one side, and asked him the same question. 

He\marginnote{29.1} said: ‘Great king, consider a Jain ascetic who is restrained in the fourfold restraint. And how is a Jain ascetic restrained in the fourfold restraint? It’s when a Jain ascetic is obstructed by all water, devoted to all water, shaking off all water, pervaded by all water. That’s how a Jain ascetic is restrained in the fourfold restraint. When a Jain ascetic is restrained in the fourfold restraint, they’re called a knotless one who is self-realized, self-controlled, and steadfast.’ 

And\marginnote{30.1} so, when I asked \textsanskrit{Nigaṇṭha} \textsanskrit{Nāṭaputta} about the fruits of the ascetic life apparent in the present life, he answered with the fourfold restraint. It was like someone who, when asked about a mango, answered with a breadfruit, or when asked about a breadfruit, answered with a mango. I thought: ‘How could one such as I presume to rebuke an ascetic or brahmin living in my realm?’ So I neither approved nor dismissed that statement of \textsanskrit{Nigaṇṭha} \textsanskrit{Nāṭaputta}. I was displeased, but did not express my displeasure. Neither accepting what he said nor contradicting it, I got up from my seat and left. 

\subsection*{3.6. The Doctrine of \textsanskrit{Sañjaya} \textsanskrit{Belaṭṭhiputta} }

One\marginnote{31.1} time, sir, I approached \textsanskrit{Sañjaya} \textsanskrit{Belaṭṭhiputta} and exchanged greetings with him. When the greetings and polite conversation were over, I sat down to one side, and asked him the same question. 

He\marginnote{32.1} said: ‘Suppose you were to ask me whether there is another world. If I believed there was, I would say so. But I don’t say it’s like this. I don’t say it’s like that. I don’t say it’s otherwise. I don’t say it’s not so. And I don’t deny it’s not so. Suppose you were to ask me whether there is no other world … whether there both is and is not another world … whether there neither is nor is not another world … whether there are beings who are reborn spontaneously … whether there are no beings who are reborn spontaneously … whether there both are and are not beings who are reborn spontaneously … whether there neither are nor are not beings who are reborn spontaneously … whether there is fruit and result of good and bad deeds … whether there is no fruit and result of good and bad deeds … whether there both is and is not fruit and result of good and bad deeds … whether there neither is nor is not fruit and result of good and bad deeds … whether a Realized One exists after death … whether a Realized One doesn’t exist after death … whether a Realized One both exists and doesn’t exist after death … whether a Realized One neither exists nor doesn’t exist after death. If I believed there was, I would say so. But I don’t say it’s like this. I don’t say it’s like that. I don’t say it’s otherwise. I don’t say it’s not so. And I don’t deny it’s not so.’ 

And\marginnote{33.1} so, when I asked \textsanskrit{Sañjaya} \textsanskrit{Belaṭṭhiputta} about the fruits of the ascetic life apparent in the present life, he answered with evasiveness. It was like someone who, when asked about a mango, answered with a breadfruit, or when asked about a breadfruit, answered with a mango. I thought: ‘This is the most foolish and stupid of all these ascetics and brahmins! How on earth can he answer with evasiveness when asked about the fruits of the ascetic life apparent in the present life?’ I thought: ‘How could one such as I presume to rebuke an ascetic or brahmin living in my realm?’ So I neither approved nor dismissed that statement of \textsanskrit{Sañjaya} \textsanskrit{Belaṭṭhiputta}. I was displeased, but did not express my displeasure. Neither accepting what he said nor contradicting it, I got up from my seat and left. 

\section*{4. The Fruits of the Ascetic Life }

\subsection*{4.1. The First Fruit of the Ascetic Life }

And\marginnote{34.1} so I ask the Buddha: Sir, there are many different professional fields. These include elephant riders, cavalry, charioteers, archers, bannermen, adjutants, food servers, warrior-chiefs, princes, chargers, great warriors, heroes, leather-clad soldiers, and sons of bondservants. They also include bakers, barbers, bathroom attendants, cooks, garland-makers, dyers, weavers, basket-makers, potters, accountants, finger-talliers, or those following any similar professions. All these live off the fruits of their profession which are apparent in the present life. With that they bring happiness and joy to themselves, their parents, their children and partners, and their friends and colleagues. And they establish an uplifting religious donation for ascetics and brahmins that’s conducive to heaven, ripens in happiness, and leads to heaven. Sir, can you point out a fruit of the ascetic life that’s likewise apparent in the present life?” 

“I\marginnote{34.7} can, great king. Well then, I’ll ask you about this in return, and you can answer as you like. What do you think, great king? Suppose you had a person who was a bondservant, a worker. They get up before you and go to bed after you, and are obliging, behaving nicely and speaking politely, and gazing up at your face. They’d think: ‘The outcome and result of good deeds is just so incredible, so amazing! For this King \textsanskrit{Ajātasattu} is a human being, and so am I. Yet he amuses himself, supplied and provided with the five kinds of sensual stimulation as if he were a god. Whereas I’m his bondservant, his worker. I get up before him and go to bed after him, and am obliging, behaving nicely and speaking politely, and gazing up at his face. I should do good deeds. Why don’t I shave off my hair and beard, dress in ocher robes, and go forth from the lay life to homelessness?’ 

After\marginnote{35.10} some time, that is what they do. Having gone forth they’d live restrained in body, speech, and mind, living content with nothing more than food and clothes, delighting in seclusion. And suppose your men were to report all this to you. Would you say to them: ‘Bring that person to me! Let them once more be my bondservant, my worker’?” 

“No,\marginnote{36.1} sir. Rather, I would bow to them, rise in their presence, and offer them a seat. I’d invite them to accept robes, almsfood, lodgings, and medicines and supplies for the sick. And I’d organize their lawful guarding and protection.” 

“What\marginnote{36.3} do you think, great king? If this is so, is there a fruit of the ascetic life apparent in the present life or not?” 

“Clearly,\marginnote{36.5} sir, there is.” 

“This\marginnote{36.6} is the first fruit of the ascetic life that’s apparent in the present life, which I point out to you.” 

\subsection*{4.2. The Second Fruit of the Ascetic Life }

“But\marginnote{37.1} sir, can you point out another fruit of the ascetic life that’s likewise apparent in the present life?” 

“I\marginnote{37.2} can, great king. Well then, I’ll ask you about this in return, and you can answer as you like. What do you think, great king? Suppose you had a person who was a farmer, a householder, a hard worker, someone who builds up their capital. They’d think: ‘The outcome and result of good deeds is just so incredible, so amazing! For this King \textsanskrit{Ajātasattu} is a human being, and so am I. Yet he amuses himself, supplied and provided with the five kinds of sensual stimulation as if he were a god. Whereas I’m a farmer, a householder, a hard worker, someone who builds up their capital. I should do good deeds. Why don’t I shave off my hair and beard, dress in ocher robes, and go forth from the lay life to homelessness?’ 

After\marginnote{37.13} some time they give up a large or small fortune, and a large or small family circle. They’d shave off hair and beard, dress in ocher robes, and go forth from the lay life to homelessness. Having gone forth they’d live restrained in body, speech, and mind, living content with nothing more than food and clothes, delighting in seclusion. And suppose your men were to report all this to you. Would you say to them: ‘Bring that person to me! Let them once more be a farmer, a householder, a hard worker, someone who builds up their capital’?” 

“No,\marginnote{38.1} sir. Rather, I would bow to them, rise in their presence, and offer them a seat. I’d invite them to accept robes, almsfood, lodgings, and medicines and supplies for the sick. And I’d organize their lawful guarding and protection.” 

“What\marginnote{38.3} do you think, great king? If this is so, is there a fruit of the ascetic life apparent in the present life or not?” 

“Clearly,\marginnote{38.5} sir, there is.” 

“This\marginnote{38.6} is the second fruit of the ascetic life that’s apparent in the present life, which I point out to you.” 

\subsection*{4.3. The Finer Fruits of the Ascetic Life }

“But\marginnote{39.1} sir, can you point out a fruit of the ascetic life that’s apparent in the present life which is better and finer than these?” 

“I\marginnote{39.2} can, great king. Well then, listen and pay close attention, I will speak.” 

“Yes,\marginnote{39.4} sir,” replied the king. 

The\marginnote{39.5} Buddha said this: 

“Consider\marginnote{40.1} when a Realized One arises in the world, perfected, a fully awakened Buddha, accomplished in knowledge and conduct, holy, knower of the world, supreme guide for those who wish to train, teacher of gods and humans, awakened, blessed. He has realized with his own insight this world—with its gods, \textsanskrit{Māras} and \textsanskrit{Brahmās}, this population with its ascetics and brahmins, gods and humans—and he makes it known to others. He teaches Dhamma that’s good in the beginning, good in the middle, and good in the end, meaningful and well-phrased. And he reveals a spiritual practice that’s entirely full and pure. 

A\marginnote{41.1} householder hears that teaching, or a householder’s child, or someone reborn in some clan. They gain faith in the Realized One, and reflect: ‘Living in a house is cramped and dirty, but the life of one gone forth is wide open. It’s not easy for someone living at home to lead the spiritual life utterly full and pure, like a polished shell. Why don’t I shave off my hair and beard, dress in ocher robes, and go forth from the lay life to homelessness?’ 

After\marginnote{41.7} some time they give up a large or small fortune, and a large or small family circle. They shave off hair and beard, dress in ocher robes, and go forth from the lay life to homelessness. 

Once\marginnote{42.1} they’ve gone forth, they live restrained in the monastic code, conducting themselves well and seeking alms in suitable places. Seeing danger in the slightest fault, they keep the rules they’ve undertaken. They act skillfully by body and speech. They’re purified in livelihood and accomplished in ethical conduct. They guard the sense doors, have mindfulness and situational awareness, and are content. 

\subsubsection*{4.3.1. Ethics }

\paragraph*{4.3.1.1. The Shorter Section on Ethics }

And\marginnote{43.1} how, great king, is a mendicant accomplished in ethics? It’s when a mendicant gives up killing living creatures, renouncing the rod and the sword. They’re scrupulous and kind, living full of compassion for all living beings. This pertains to their ethics. 

They\marginnote{43.4} give up stealing. They take only what’s given, and expect only what’s given. They keep themselves clean by not thieving. This pertains to their ethics. 

They\marginnote{43.6} give up unchastity. They are celibate, set apart, avoiding the common practice of sex. This pertains to their ethics. 

They\marginnote{44.1} give up lying. They speak the truth and stick to the truth. They’re honest and trustworthy, and don’t trick the world with their words. This pertains to their ethics. 

They\marginnote{44.3} give up divisive speech. They don’t repeat in one place what they heard in another so as to divide people against each other. Instead, they reconcile those who are divided, supporting unity, delighting in harmony, loving harmony, speaking words that promote harmony. This pertains to their ethics. 

They\marginnote{44.5} give up harsh speech. They speak in a way that’s mellow, pleasing to the ear, lovely, going to the heart, polite, likable and agreeable to the people. This pertains to their ethics. 

They\marginnote{44.7} give up talking nonsense. Their words are timely, true, and meaningful, in line with the teaching and training. They say things at the right time which are valuable, reasonable, succinct, and beneficial. This pertains to their ethics. 

They\marginnote{45.1} refrain from injuring plants and seeds. They eat in one part of the day, abstaining from eating at night and food at the wrong time. They avoid dancing, singing, music, and seeing shows. They refrain from beautifying and adorning themselves with garlands, fragrance, and makeup. They avoid high and luxurious beds. They avoid receiving gold and money, raw grains, raw meat, women and girls, male and female bondservants, goats and sheep, chickens and pigs, elephants, cows, horses, and mares, and fields and land. They refrain from running errands and messages; buying and selling; falsifying weights, metals, or measures; bribery, fraud, cheating, and duplicity; mutilation, murder, abduction, banditry, plunder, and violence. This pertains to their ethics. 

\scendsection{The shorter section on ethics is finished. }

\paragraph*{4.3.1.2. The Middle Section on Ethics }

There\marginnote{46.1} are some ascetics and brahmins who, while enjoying food given in faith, still engage in injuring plants and seeds. These include plants propagated from roots, stems, cuttings, or joints; and those from regular seeds as the fifth. They refrain from such injury to plants and seeds. This pertains to their ethics. 

There\marginnote{47.1} are some ascetics and brahmins who, while enjoying food given in faith, still engage in storing up goods for their own use. This includes such things as food, drink, clothes, vehicles, bedding, fragrance, and material possessions. They refrain from storing up such goods. This pertains to their ethics. 

There\marginnote{48.1} are some ascetics and brahmins who, while enjoying food given in faith, still engage in seeing shows. This includes such things as dancing, singing, music, performances, and storytelling; clapping, gongs, and kettledrums; art exhibitions and acrobatic displays; battles of elephants, horses, buffaloes, bulls, goats, rams, chickens, and quails; staff-fights, boxing, and wrestling; combat, roll calls of the armed forces, battle-formations, and regimental reviews. They refrain from such shows. This pertains to their ethics. 

There\marginnote{49.1} are some ascetics and brahmins who, while enjoying food given in faith, still engage in gambling that causes negligence. This includes such things as checkers, draughts, checkers in the air, hopscotch, spillikins, board-games, tip-cat, drawing straws, dice, leaf-flutes, toy plows, somersaults, pinwheels, toy measures, toy carts, toy bows, guessing words from syllables, and guessing another’s thoughts. They refrain from such gambling. This pertains to their ethics. 

There\marginnote{50.1} are some ascetics and brahmins who, while enjoying food given in faith, still make use of high and luxurious bedding. This includes such things as sofas, couches, woolen covers—shag-piled, colorful, white, embroidered with flowers, quilted, embroidered with animals, double-or single-fringed—and silk covers studded with gems, as well as silken sheets, woven carpets, rugs for elephants, horses, or chariots, antelope hide rugs, and spreads of fine deer hide, with a canopy above and red cushions at both ends. They refrain from such bedding. This pertains to their ethics. 

There\marginnote{51.1} are some ascetics and brahmins who, while enjoying food given in faith, still engage in beautifying and adorning themselves with garlands, fragrance, and makeup. This includes such things as applying beauty products by anointing, massaging, bathing, and rubbing; mirrors, ointments, garlands, fragrances, and makeup; face-powder, foundation, bracelets, headbands, fancy walking-sticks or containers, rapiers, parasols, fancy sandals, turbans, jewelry, chowries, and long-fringed white robes. They refrain from such beautification and adornment. This pertains to their ethics. 

There\marginnote{52.1} are some ascetics and brahmins who, while enjoying food given in faith, still engage in unworthy talk. This includes such topics as talk about kings, bandits, and ministers; talk about armies, threats, and wars; talk about food, drink, clothes, and beds; talk about garlands and fragrances; talk about family, vehicles, villages, towns, cities, and countries; talk about women and heroes; street talk and well talk; talk about the departed; motley talk; tales of land and sea; and talk about being reborn in this or that state of existence. They refrain from such unworthy talk. This pertains to their ethics. 

There\marginnote{53.1} are some ascetics and brahmins who, while enjoying food given in faith, still engage in arguments. They say such things as: ‘You don’t understand this teaching and training. I understand this teaching and training. What, you understand this teaching and training? You’re practicing wrong. I’m practicing right. I stay on topic, you don’t. You said last what you should have said first. You said first what you should have said last. What you’ve thought so much about has been disproved. Your doctrine is refuted. Go on, save your doctrine! You’re trapped; get yourself out of this—if you can!’ They refrain from such argumentative talk. This pertains to their ethics. 

There\marginnote{54.1} are some ascetics and brahmins who, while enjoying food given in faith, still engage in running errands and messages. This includes running errands for rulers, ministers, aristocrats, brahmins, householders, or princes who say: ‘Go here, go there. Take this, bring that from there.’ They refrain from such errands. This pertains to their ethics. 

There\marginnote{55.1} are some ascetics and brahmins who, while enjoying food given in faith, still engage in deceit, flattery, hinting, and belittling, and using material possessions to chase after other material possessions. They refrain from such deceit and flattery. This pertains to their ethics. 

\scendsection{The middle section on ethics is finished. }

\paragraph*{4.3.1.3. The Long Section on Ethics }

There\marginnote{56.1} are some ascetics and brahmins who, while enjoying food given in faith, still earn a living by unworthy branches of knowledge, by wrong livelihood. This includes such fields as limb-reading, omenology, divining celestial portents, interpreting dreams, divining bodily marks, divining holes in cloth gnawed by mice, fire offerings, ladle offerings, offerings of husks, rice powder, rice, ghee, or oil; offerings from the mouth, blood sacrifices, palmistry; geomancy for building sites, fields, and cemeteries; exorcisms, earth magic, snake charming, poisons; the crafts of the scorpion, the rat, the bird, and the crow; prophesying life span, chanting for protection, and deciphering animal cries. They refrain from such unworthy branches of knowledge, such wrong livelihood. This pertains to their ethics. 

There\marginnote{57.1} are some ascetics and brahmins who, while enjoying food given in faith, still earn a living by unworthy branches of knowledge, by wrong livelihood. This includes reading the marks of gems, cloth, clubs, swords, spears, arrows, weapons, women, men, boys, girls, male and female bondservants, elephants, horses, buffaloes, bulls, cows, goats, rams, chickens, quails, monitor lizards, rabbits, tortoises, or deer. They refrain from such unworthy branches of knowledge, such wrong livelihood. This pertains to their ethics. 

There\marginnote{58.1} are some ascetics and brahmins who, while enjoying food given in faith, still earn a living by unworthy branches of knowledge, by wrong livelihood. This includes making predictions that the king will march forth or march back; or that our king will attack and the enemy king will retreat, or vice versa; or that our king will triumph and the enemy king will be defeated, or vice versa; and so there will be victory for one and defeat for the other. They refrain from such unworthy branches of knowledge, such wrong livelihood. This pertains to their ethics. 

There\marginnote{59.1} are some ascetics and brahmins who, while enjoying food given in faith, still earn a living by unworthy branches of knowledge, by wrong livelihood. This includes making predictions that there will be an eclipse of the moon, or sun, or stars; that the sun, moon, and stars will be in conjunction or in opposition; that there will be a meteor shower, a fiery sky, an earthquake, thunder; that there will be a rising, a setting, a darkening, a brightening of the moon, sun, and stars. And it also includes making predictions about the results of all such phenomena. They refrain from such unworthy branches of knowledge, such wrong livelihood. This pertains to their ethics. 

There\marginnote{60.1} are some ascetics and brahmins who, while enjoying food given in faith, still earn a living by unworthy branches of knowledge, by wrong livelihood. This includes predicting whether there will be plenty of rain or drought; plenty to eat or famine; an abundant harvest or a bad harvest; security or peril; sickness or health. It also includes such occupations as computing, accounting, calculating, poetry, and cosmology. They refrain from such unworthy branches of knowledge, such wrong livelihood. This pertains to their ethics. 

There\marginnote{61.1} are some ascetics and brahmins who, while enjoying food given in faith, still earn a living by unworthy branches of knowledge, by wrong livelihood. This includes making arrangements for giving and taking in marriage; for engagement and divorce; and for scattering rice inwards or outwards at the wedding ceremony. It also includes casting spells for good or bad luck, treating impacted fetuses, binding the tongue, or locking the jaws; charms for the hands and ears; questioning a mirror, a girl, or a god as an oracle; worshiping the sun, worshiping the Great One, breathing fire, and invoking Siri, the goddess of luck. They refrain from such unworthy branches of knowledge, such wrong livelihood. This pertains to their ethics. 

There\marginnote{62.1} are some ascetics and brahmins who, while enjoying food given in faith, still earn a living by unworthy branches of knowledge, by wrong livelihood. This includes rites for propitiation, for granting wishes, for ghosts, for the earth, for rain, for property settlement, and for preparing and consecrating house sites, and rites involving rinsing and bathing, and oblations. It also includes administering emetics, purgatives, expectorants, and phlegmagogues; administering ear-oils, eye restoratives, nasal medicine, ointments, and counter-ointments; surgery with needle and scalpel, treating children, prescribing root medicines, and binding on herbs. They refrain from such unworthy branches of knowledge, such wrong livelihood. This pertains to their ethics. 

A\marginnote{63.1} mendicant thus accomplished in ethics sees no danger in any quarter in regards to their ethical restraint. It’s like a king who has defeated his enemies. He sees no danger from his foes in any quarter. In the same way, a mendicant thus accomplished in ethics sees no danger in any quarter in regards to their ethical restraint. When they have this entire spectrum of noble ethics, they experience a blameless happiness inside themselves. That’s how a mendicant is accomplished in ethics. 

\scendsection{The longer section on ethics is finished. }

\subsubsection*{4.3.2. Immersion }

\paragraph*{4.3.2.1. Sense Restraint }

And\marginnote{64.1} how does a mendicant guard the sense doors? When a mendicant sees a sight with their eyes, they don’t get caught up in the features and details. If the faculty of sight were left unrestrained, bad unskillful qualities of desire and aversion would become overwhelming. For this reason, they practice restraint, protecting the faculty of sight, and achieving its restraint. When they hear a sound with their ears … When they smell an odor with their nose … When they taste a flavor with their tongue … When they feel a touch with their body … When they know a thought with their mind, they don’t get caught up in the features and details. If the faculty of mind were left unrestrained, bad unskillful qualities of desire and aversion would become overwhelming. For this reason, they practice restraint, protecting the faculty of mind, and achieving its restraint. When they have this noble sense restraint, they experience an unsullied bliss inside themselves. That’s how a mendicant guards the sense doors. 

\paragraph*{4.3.2.2. Mindfulness and Situational Awareness }

And\marginnote{65.1} how does a mendicant have mindfulness and situational awareness? It’s when a mendicant acts with situational awareness when going out and coming back; when looking ahead and aside; when bending and extending the limbs; when bearing the outer robe, bowl and robes; when eating, drinking, chewing, and tasting; when urinating and defecating; when walking, standing, sitting, sleeping, waking, speaking, and keeping silent. That’s how a mendicant has mindfulness and situational awareness. 

\paragraph*{4.3.2.3. Contentment }

And\marginnote{66.1} how is a mendicant content? It’s when a mendicant is content with robes to look after the body and almsfood to look after the belly. Wherever they go, they set out taking only these things. They’re like a bird: wherever it flies, wings are its only burden. In the same way, a mendicant is content with robes to look after the body and almsfood to look after the belly. Wherever they go, they set out taking only these things. That’s how a mendicant is content. 

\paragraph*{4.3.2.4. Giving Up the Hindrances }

When\marginnote{67.1} they have this noble spectrum of ethics, this noble sense restraint, this noble mindfulness and situational awareness, and this noble contentment, they frequent a secluded lodging—a wilderness, the root of a tree, a hill, a ravine, a mountain cave, a charnel ground, a forest, the open air, a heap of straw. After the meal, they return from almsround, sit down cross-legged with their body straight, and establish mindfulness right there. 

Giving\marginnote{68.1} up desire for the world, they meditate with a heart rid of desire, cleansing the mind of desire. Giving up ill will and malevolence, they meditate with a mind rid of ill will, full of compassion for all living beings, cleansing the mind of ill will. Giving up dullness and drowsiness, they meditate with a mind rid of dullness and drowsiness, perceiving light, mindful and aware, cleansing the mind of dullness and drowsiness. Giving up restlessness and remorse, they meditate without restlessness, their mind peaceful inside, cleansing the mind of restlessness and remorse. Giving up doubt, they meditate having gone beyond doubt, not undecided about skillful qualities, cleansing the mind of doubt. 

Suppose\marginnote{69.1} a man who has gotten into debt were to apply himself to work, and his efforts proved successful. He would pay off the original loan and have enough left over to support his partner. Thinking about this, he’d be filled with joy and happiness. 

Suppose\marginnote{70.1} there was a person who was sick, suffering, gravely ill. They’d lose their appetite and get physically weak. But after some time they’d recover from that illness, and regain their appetite and their strength. Thinking about this, they’d be filled with joy and happiness. 

Suppose\marginnote{71.1} a person was imprisoned in a jail. But after some time they were released from jail, safe and sound, with no loss of wealth. Thinking about this, they’d be filled with joy and happiness. 

Suppose\marginnote{72.1} a person was a bondservant. They belonged to someone else and were unable to go where they wish. But after some time they’d be freed from servitude and become their own master, an emancipated individual able to go where they wish. Thinking about this, they’d be filled with joy and happiness. 

Suppose\marginnote{73.1} there was a person with wealth and property who was traveling along a desert road, which was perilous, with nothing to eat. But after some time they crossed over the desert safely, arriving within a village, a sanctuary free of peril. Thinking about this, they’d be filled with joy and happiness. 

In\marginnote{74.1} the same way, as long as these five hindrances are not given up inside themselves, a mendicant regards them thus as a debt, a disease, a prison, slavery, and a desert crossing. 

But\marginnote{74.2} when these five hindrances are given up inside themselves, a mendicant regards this as freedom from debt, good health, release from prison, emancipation, and sanctuary. 

Seeing\marginnote{74.4} that the hindrances have been given up in them, joy springs up. Being joyful, rapture springs up. When the mind is full of rapture, the body becomes tranquil. When the body is tranquil, they feel bliss. And when blissful, the mind becomes immersed. 

\paragraph*{4.3.2.5. First Absorption }

Quite\marginnote{75.1} secluded from sensual pleasures, secluded from unskillful qualities, they enter and remain in the first absorption, which has the rapture and bliss born of seclusion, while placing the mind and keeping it connected. They drench, steep, fill, and spread their body with rapture and bliss born of seclusion. There’s no part of the body that’s not spread with rapture and bliss born of seclusion. 

It’s\marginnote{76.1} like when a deft bathroom attendant or their apprentice pours bath powder into a bronze dish, sprinkling it little by little with water. They knead it until the ball of bath powder is soaked and saturated with moisture, spread through inside and out; yet no moisture oozes out. In the same way, a mendicant drenches, steeps, fills, and spreads their body with rapture and bliss born of seclusion. There’s no part of the body that’s not spread with rapture and bliss born of seclusion. This, great king, is a fruit of the ascetic life that’s apparent in the present life which is better and finer than the former ones. 

\paragraph*{4.3.2.6. Second Absorption }

Furthermore,\marginnote{77.1} as the placing of the mind and keeping it connected are stilled, a mendicant enters and remains in the second absorption, which has the rapture and bliss born of immersion, with internal clarity and confidence, and unified mind, without applying the mind and keeping it connected. In the same way, a mendicant drenches, steeps, fills, and spreads their body with rapture and bliss born of immersion. There’s no part of the body that’s not spread with rapture and bliss born of immersion. 

It’s\marginnote{78.1} like a deep lake fed by spring water. There’s no inlet to the east, west, north, or south, and no rainfall to replenish it from time to time. But the stream of cool water welling up in the lake drenches, steeps, fills, and spreads throughout the lake. There’s no part of the lake that’s not spread through with cool water. 

In\marginnote{78.3} the same way, a mendicant drenches, steeps, fills, and spreads their body with rapture and bliss born of immersion. There’s no part of the body that’s not spread with rapture and bliss born of immersion. This too, great king, is a fruit of the ascetic life that’s apparent in the present life which is better and finer than the former ones. 

\paragraph*{4.3.2.7. Third Absorption }

Furthermore,\marginnote{79.1} with the fading away of rapture, a mendicant enters and remains in the third absorption, where they meditate with equanimity, mindful and aware, personally experiencing the bliss of which the noble ones declare, ‘Equanimous and mindful, one meditates in bliss.’ They drench, steep, fill, and spread their body with bliss free of rapture. There’s no part of the body that’s not spread with bliss free of rapture. 

It’s\marginnote{80.1} like a pool with blue water lilies, or pink or white lotuses. Some of them sprout and grow in the water without rising above it, thriving underwater. From the tip to the root they’re drenched, steeped, filled, and soaked with cool water. There’s no part of them that’s not soaked with cool water. In the same way, a mendicant drenches, steeps, fills, and spreads their body with bliss free of rapture. There’s no part of the body that’s not spread with bliss free of rapture. This too, great king, is a fruit of the ascetic life that’s apparent in the present life which is better and finer than the former ones. 

\paragraph*{4.3.2.8. Fourth Absorption }

Furthermore,\marginnote{81.1} giving up pleasure and pain, and ending former happiness and sadness, a mendicant enters and remains in the fourth absorption, without pleasure or pain, with pure equanimity and mindfulness. They sit spreading their body through with pure bright mind. There’s no part of the body that’s not spread with pure bright mind. 

It’s\marginnote{82.1} like someone sitting wrapped from head to foot with white cloth. There’s no part of the body that’s not spread over with white cloth. In the same way, they sit spreading their body through with pure bright mind. There’s no part of the body that’s not spread with pure bright mind. This too, great king, is a fruit of the ascetic life that’s apparent in the present life which is better and finer than the former ones. 

\subsubsection*{4.3.3. The Eight Knowledges }

\paragraph*{4.3.3.1. Knowledge and Vision }

When\marginnote{83.1} their mind has become immersed in \textsanskrit{samādhi} like this—purified, bright, flawless, rid of corruptions, pliable, workable, steady, and imperturbable—they extend it and project it toward knowledge and vision. They understand: ‘This body of mine is physical. It’s made up of the four primary elements, produced by mother and father, built up from rice and porridge, liable to impermanence, to wearing away and erosion, to breaking up and destruction. And this consciousness of mine is attached to it, tied to it.’ 

Suppose\marginnote{84.1} there was a beryl gem that was naturally beautiful, eight-faceted, well-worked, transparent, clear, and unclouded, endowed with all good qualities. And it was strung with a thread of blue, yellow, red, white, or golden brown. And someone with good eyesight were to take it in their hand and check it: ‘This beryl gem is naturally beautiful, eight-faceted, well-worked, transparent, clear, and unclouded, endowed with all good qualities. And it’s strung with a thread of blue, yellow, red, white, or golden brown.’ 

In\marginnote{84.6} the same way, when their mind has become immersed in \textsanskrit{samādhi} like this—purified, bright, flawless, rid of corruptions, pliable, workable, steady, and imperturbable—they extend it and project it toward knowledge and vision. This too, great king, is a fruit of the ascetic life that’s apparent in the present life which is better and finer than the former ones. 

\paragraph*{4.3.3.2. Mind-Made Body }

When\marginnote{85.1} their mind has become immersed in \textsanskrit{samādhi} like this—purified, bright, flawless, rid of corruptions, pliable, workable, steady, and imperturbable—they extend it and project it toward the creation of a mind-made body. From this body they create another body, physical, mind-made, complete in all its various parts, not deficient in any faculty. 

Suppose\marginnote{86.1} a person was to draw a reed out from its sheath. They’d think: ‘This is the reed, this is the sheath. The reed and the sheath are different things. The reed has been drawn out from the sheath.’ Or suppose a person was to draw a sword out from its scabbard. They’d think: ‘This is the sword, this is the scabbard. The sword and the scabbard are different things. The sword has been drawn out from the scabbard.’ Or suppose a person was to draw a snake out from its slough. They’d think: ‘This is the snake, this is the slough. The snake and the slough are different things. The snake has been drawn out from the slough.’ 

In\marginnote{86.10} the same way, when their mind has become immersed in \textsanskrit{samādhi} like this—purified, bright, flawless, rid of corruptions, pliable, workable, steady, and imperturbable—they extend it and project it toward the creation of a mind-made body. From this body they create another body, physical, mind-made, complete in all its various parts, not deficient in any faculty. This too, great king, is a fruit of the ascetic life that’s apparent in the present life which is better and finer than the former ones. 

\paragraph*{4.3.3.3. Psychic Powers }

When\marginnote{87.1} their mind has become immersed in \textsanskrit{samādhi} like this—purified, bright, flawless, rid of corruptions, pliable, workable, steady, and imperturbable—they extend it and project it toward psychic power. They wield the many kinds of psychic power: multiplying themselves and becoming one again; going unimpeded through a wall, a rampart, or a mountain as if through space; diving in and out of the earth as if it were water; walking on water as if it were earth; flying cross-legged through the sky like a bird; touching and stroking with the hand the sun and moon, so mighty and powerful; controlling the body as far as the \textsanskrit{Brahmā} realm. 

Suppose\marginnote{88.1} an expert potter or their apprentice had some well-prepared clay. They could produce any kind of pot that they like. Or suppose a deft ivory-carver or their apprentice had some well-prepared ivory. They could produce any kind of ivory item that they like. Or suppose a deft goldsmith or their apprentice had some well-prepared gold. They could produce any kind of gold item that they like. 

In\marginnote{88.4} the same way, when their mind has become immersed in \textsanskrit{samādhi} like this—purified, bright, flawless, rid of corruptions, pliable, workable, steady, and imperturbable—they extend it and project it toward psychic power. This too, great king, is a fruit of the ascetic life that’s apparent in the present life which is better and finer than the former ones. 

\paragraph*{4.3.3.4. Clairaudience }

When\marginnote{89.1} their mind has become immersed in \textsanskrit{samādhi} like this—purified, bright, flawless, rid of corruptions, pliable, workable, steady, and imperturbable—they extend it and project it toward clairaudience. With clairaudience that is purified and superhuman, they hear both kinds of sounds, human and divine, whether near or far. 

Suppose\marginnote{90.1} there was a person traveling along the road. They’d hear the sound of drums, clay drums, horns, kettledrums, and tom-toms. They’d think: ‘That’s the sound of drums,’ and ‘that’s the sound of clay drums,’ and ‘that’s the sound of horns, kettledrums, and tom-toms.’ 

In\marginnote{90.2} the same way, when their mind has become immersed in \textsanskrit{samādhi} like this—purified, bright, flawless, rid of corruptions, pliable, workable, steady, and imperturbable—they extend it and project it toward clairaudience. With clairaudience that is purified and superhuman, they hear both kinds of sounds, human and divine, whether near or far. This too, great king, is a fruit of the ascetic life that’s apparent in the present life which is better and finer than the former ones. 

\paragraph*{4.3.3.5. Comprehending the Minds of Others }

When\marginnote{91.1} their mind has become immersed in \textsanskrit{samādhi} like this—purified, bright, flawless, rid of corruptions, pliable, workable, steady, and imperturbable—they extend it and project it toward comprehending the minds of others. They understand the minds of other beings and individuals, having comprehended them with their own mind. They understand mind with greed as ‘mind with greed’, and mind without greed as ‘mind without greed’. They understand mind with hate … mind without hate … mind with delusion … mind without delusion … constricted mind … scattered mind … expansive mind … unexpansive mind … mind that is not supreme … mind that is supreme … immersed mind … unimmersed mind … freed mind … They understand unfreed mind as ‘unfreed mind’. 

Suppose\marginnote{92.1} there was a woman or man who was young, youthful, and fond of adornments, and they check their own reflection in a clean bright mirror or a clear bowl of water. If they had a spot they’d know ‘I have a spot,’ and if they had no spots they’d know ‘I have no spots.’ In the same way, when their mind has become immersed in \textsanskrit{samādhi} like this—purified, bright, flawless, rid of corruptions, pliable, workable, steady, and imperturbable—they extend it and project it toward comprehending the minds of others. They understand the minds of other beings and individuals, having comprehended them with their own mind. This too, great king, is a fruit of the ascetic life that’s apparent in the present life which is better and finer than the former ones. 

\paragraph*{4.3.3.6. Recollection of Past Lives }

When\marginnote{93.1} their mind has become immersed in \textsanskrit{samādhi} like this—purified, bright, flawless, rid of corruptions, pliable, workable, steady, and imperturbable—they extend it and project it toward recollection of past lives. They recollect many kinds of past lives, that is, one, two, three, four, five, ten, twenty, thirty, forty, fifty, a hundred, a thousand, a hundred thousand rebirths; many eons of the world contracting, many eons of the world expanding, many eons of the world contracting and expanding. They remember: ‘There, I was named this, my clan was that, I looked like this, and that was my food. This was how I felt pleasure and pain, and that was how my life ended. When I passed away from that place I was reborn somewhere else. There, too, I was named this, my clan was that, I looked like this, and that was my food. This was how I felt pleasure and pain, and that was how my life ended. When I passed away from that place I was reborn here.’ And so they recollect their many kinds of past lives, with features and details. 

Suppose\marginnote{94.1} a person was to leave their home village and go to another village. From that village they’d go to yet another village. And from that village they’d return to their home village. They’d think: ‘I went from my home village to another village. There I stood like this, sat like that, spoke like this, or kept silent like that. From that village I went to yet another village. There too I stood like this, sat like that, spoke like this, or kept silent like that. And from that village I returned to my home village.’ 

In\marginnote{94.2} the same way, when their mind has become immersed in \textsanskrit{samādhi} like this—purified, bright, flawless, rid of corruptions, pliable, workable, steady, and imperturbable—they extend it and project it toward recollection of past lives. This too, great king, is a fruit of the ascetic life that’s apparent in the present life which is better and finer than the former ones. 

\paragraph*{4.3.3.7. Clairvoyance }

When\marginnote{95.1} their mind has become immersed in \textsanskrit{samādhi} like this—purified, bright, flawless, rid of corruptions, pliable, workable, steady, and imperturbable—they extend it and project it toward knowledge of the death and rebirth of sentient beings. With clairvoyance that is purified and superhuman, they see sentient beings passing away and being reborn—inferior and superior, beautiful and ugly, in a good place or a bad place. They understand how sentient beings are reborn according to their deeds: ‘These dear beings did bad things by way of body, speech, and mind. They spoke ill of the noble ones; they had wrong view; and they acted out of that wrong view. When their body breaks up, after death, they’re reborn in a place of loss, a bad place, the underworld, hell. These dear beings, however, did good things by way of body, speech, and mind. They never spoke ill of the noble ones; they had right view; and they acted out of that right view. When their body breaks up, after death, they’re reborn in a good place, a heavenly realm.’ And so, with clairvoyance that is purified and superhuman, they see sentient beings passing away and being reborn—inferior and superior, beautiful and ugly, in a good place or a bad place. They understand how sentient beings are reborn according to their deeds. 

Suppose\marginnote{96.1} there was a stilt longhouse at the central square. A person with good eyesight standing there might see people entering and leaving a house, walking along the streets and paths, and sitting at the central square. They’d think: ‘These are people entering and leaving a house, walking along the streets and paths, and sitting at the central square.’ 

In\marginnote{96.2} the same way, when their mind has become immersed in \textsanskrit{samādhi} like this—purified, bright, flawless, rid of corruptions, pliable, workable, steady, and imperturbable—they extend and project it toward knowledge of the death and rebirth of sentient beings. This too, great king, is a fruit of the ascetic life that’s apparent in the present life which is better and finer than the former ones. 

\paragraph*{4.3.3.8. Ending of Defilements }

When\marginnote{97.1} their mind has become immersed in \textsanskrit{samādhi} like this—purified, bright, flawless, rid of corruptions, pliable, workable, steady, and imperturbable—they extend it and project it toward knowledge of the ending of defilements. They truly understand: ‘This is suffering’ … ‘This is the origin of suffering’ … ‘This is the cessation of suffering’ … ‘This is the practice that leads to the cessation of suffering’. They truly understand: ‘These are defilements’ … ‘This is the origin of defilements’ … ‘This is the cessation of defilements’ … ‘This is the practice that leads to the cessation of defilements’. Knowing and seeing like this, their mind is freed from the defilements of sensuality, desire to be reborn, and ignorance. When they’re freed, they know they’re freed. They understand: ‘Rebirth is ended, the spiritual journey has been completed, what had to be done has been done, there is no return to any state of existence.’ 

Suppose\marginnote{98.1} that in a mountain glen there was a lake that was transparent, clear, and unclouded. A person with good eyesight standing on the bank would see the clams and mussels, and pebbles and gravel, and schools of fish swimming about or staying still. They’d think: ‘This lake is transparent, clear, and unclouded. And here are the clams and mussels, and pebbles and gravel, and schools of fish swimming about or staying still.’ 

In\marginnote{98.2} the same way, when their mind has become immersed in \textsanskrit{samādhi} like this—purified, bright, flawless, rid of corruptions, pliable, workable, steady, and imperturbable—they extend it and project it toward knowledge of the ending of defilements. This too, great king, is a fruit of the ascetic life that’s apparent in the present life which is better and finer than the former ones. And, great king, there is no other fruit of the ascetic life apparent in the present life which is better and finer than this.” 

\section*{5. \textsanskrit{Ajātasattu} Declares Himself a Lay Follower }

When\marginnote{99.1} the Buddha had spoken, King \textsanskrit{Ajātasattu} said to him, “Excellent, sir! Excellent! As if he were righting the overturned, or revealing the hidden, or pointing out the path to the lost, or lighting a lamp in the dark so people with good eyes can see what’s there, the Buddha has made the teaching clear in many ways. I go for refuge to the Buddha, to the teaching, and to the mendicant \textsanskrit{Saṅgha}. From this day forth, may the Buddha remember me as a lay follower who has gone for refuge for life. 

I\marginnote{99.6} have made a mistake, sir. It was foolish, stupid, and unskillful of me to take the life of my father, a just and principled king, for the sake of authority. Please, sir, accept my mistake for what it is, so I will restrain myself in future.” 

“Indeed,\marginnote{100.1} great king, you made a mistake. It was foolish, stupid, and unskillful of you to take the life of your father, a just and principled king, for the sake of sovereignty. But since you have recognized your mistake for what it is, and have dealt with it properly, I accept it. For it is growth in the training of the Noble One to recognize a mistake for what it is, deal with it properly, and commit to restraint in the future.” 

When\marginnote{101.1} the Buddha had spoken, King \textsanskrit{Ajātasattu} said to him, “Well, now, sir, I must go. I have many duties, and much to do.” 

“Please,\marginnote{101.3} great king, go at your convenience.” 

Then\marginnote{101.4} the king, having approved and agreed with what the Buddha said, got up from his seat, bowed, and respectfully circled him, keeping him on his right, before leaving. 

Soon\marginnote{102.1} after the king had left, the Buddha addressed the mendicants, “The king is broken, mendicants, he is ruined. If he had not taken the life of his father, a just and principled king, the stainless, immaculate vision of the Dhamma would have arisen in him in that very seat.” 

That\marginnote{102.5} is what the Buddha said. Satisfied, the mendicants were happy with what the Buddha said. 

%
\chapter*{{\suttatitleacronym DN 3}{\suttatitletranslation With Ambaṭṭha }{\suttatitleroot Ambaṭṭhasutta}}
\addcontentsline{toc}{chapter}{\tocacronym{DN 3} \toctranslation{With Ambaṭṭha } \tocroot{Ambaṭṭhasutta}}
\markboth{With Ambaṭṭha }{Ambaṭṭhasutta}
\extramarks{DN 3}{DN 3}

\scevam{So\marginnote{1.1.1} I have heard. }At one time the Buddha was wandering in the land of the Kosalans together with a large \textsanskrit{Saṅgha} of around five hundred mendicants when he arrived at a village of the Kosalan brahmins named \textsanskrit{Icchānaṅgala}. He stayed in a forest near \textsanskrit{Icchānaṅgala}. 

\section*{1. The Section on \textsanskrit{Pokkharasāti} }

Now\marginnote{1.2.1} at that time the brahmin \textsanskrit{Pokkharasāti} was living in \textsanskrit{Ukkaṭṭhā}. It was a crown property given by King Pasenadi of Kosala, teeming with living creatures, full of hay, wood, water, and grain, a royal endowment of the highest quality. \textsanskrit{Pokkharasāti} heard: 

“It\marginnote{1.2.3} seems the ascetic Gotama—a Sakyan, gone forth from a Sakyan family—has arrived at \textsanskrit{Icchānaṅgala} and is staying in a forest nearby. He has this good reputation: ‘That Blessed One is perfected, a fully awakened Buddha, accomplished in knowledge and conduct, holy, knower of the world, supreme guide for those who wish to train, teacher of gods and humans, awakened, blessed.’ He has realized with his own insight this world—with its gods, \textsanskrit{Māras} and \textsanskrit{Brahmās}, this population with its ascetics and brahmins, gods and humans—and he makes it known to others. He teaches Dhamma that’s good in the beginning, good in the middle, and good in the end, meaningful and well-phrased. And he reveals a spiritual practice that’s entirely full and pure. It’s good to see such perfected ones.” 

\section*{2. The Brahmin Student \textsanskrit{Ambaṭṭha} }

Now\marginnote{1.3.1} at that time \textsanskrit{Pokkharasāti} had a student named \textsanskrit{Ambaṭṭha}. He was one who recited and remembered the hymns, and had mastered in the three Vedas, together with their vocabularies, ritual, phonology and etymology, and the testament as fifth. He knew philology and grammar, and was well versed in cosmology and the marks of a great man. He had been authorized as a master in his own teacher’s scriptural heritage of the three Vedas with the words: “What I know, you know. And what you know, I know.” 

Then\marginnote{1.4.1} \textsanskrit{Pokkharasāti} addressed \textsanskrit{Ambaṭṭha}, “Dear \textsanskrit{Ambaṭṭha}, the ascetic Gotama—a Sakyan, gone forth from a Sakyan family—has arrived at \textsanskrit{Icchānaṅgala} and is staying in a forest nearby. … It’s good to see such perfected ones. Please, dear \textsanskrit{Ambaṭṭha}, go to the ascetic Gotama and find out whether or not he lives up to his reputation. Through you I shall learn about Master Gotama.” 

“But\marginnote{1.5.1} sir, how shall I find out whether or not the ascetic Gotama lives up to his reputation?” 

“Dear\marginnote{1.5.2} \textsanskrit{Ambaṭṭha}, the thirty-two marks of a great man have been handed down in our hymns. A great man who possesses these has only two possible destinies, no other. If he stays at home he becomes a king, a wheel-turning monarch, a just and principled king. His dominion extends to all four sides, he achieves stability in the country, and he possesses the seven treasures. He has the following seven treasures: the wheel, the elephant, the horse, the jewel, the woman, the treasurer, and the counselor as the seventh treasure. He has over a thousand sons who are valiant and heroic, crushing the armies of his enemies. After conquering this land girt by sea, he reigns by principle, without rod or sword. But if he goes forth from the lay life to homelessness, he becomes a perfected one, a fully awakened Buddha, who draws back the veil from the world. But, dear \textsanskrit{Ambaṭṭha}, I am the one who gives the hymns, and you are the one who receives them.” 

“Yes,\marginnote{1.6.1} sir,” replied \textsanskrit{Ambaṭṭha}. He got up from his seat, bowed, and respectfully circled \textsanskrit{Pokkharasāti}, keeping him to his right. He mounted a mare-drawn chariot and, together with several students, set out for the forest near \textsanskrit{Icchānaṅgala}. He went by carriage as far as the terrain allowed, then descended and entered the monastery on foot. 

At\marginnote{1.7.1} that time several mendicants were walking mindfully in the open air. Then the student \textsanskrit{Ambaṭṭha} went up to those mendicants and said, “Gentlemen, where is Master Gotama at present? For we have come here to see him.” 

Then\marginnote{1.8.1} those mendicants thought, “This \textsanskrit{Ambaṭṭha} is from a well-known family, and he is the pupil of the well-known brahmin \textsanskrit{Pokkharasāti}. The Buddha won’t mind having a discussion together with such gentlemen.” 

They\marginnote{1.8.4} said to \textsanskrit{Ambaṭṭha}, “\textsanskrit{Ambaṭṭha}, that’s his dwelling, with the door closed. Approach it quietly, without hurrying; go onto the porch, clear your throat, and knock with the latch. The Buddha will open the door.” 

So\marginnote{1.9.1} he approached the Buddha’s dwelling and knocked, and the Buddha opened the door. \textsanskrit{Ambaṭṭha} and the other students entered the dwelling. The other students exchanged greetings with the Buddha, and when the greetings and polite conversation were over, sat down to one side. But while the Buddha was sitting, \textsanskrit{Ambaṭṭha} spoke some polite words or other while walking around or standing. 

So\marginnote{1.9.4} the Buddha said to him, “\textsanskrit{Ambaṭṭha}, is this how you hold a discussion with elderly and senior brahmins, the teachers of teachers: walking around or standing while I’m sitting, speaking some polite words or other?” 

\subsection*{2.1. The First Use of the Word “Riffraff” }

“No,\marginnote{1.10.1} Master Gotama. For it is proper for one brahmin to converse with another while both are walking, standing, sitting, or lying down. But as to these shavelings, fake ascetics, riffraff, black spawn from the feet of our kinsman, I converse with them as I do with Master Gotama.” 

“But\marginnote{1.11.1} \textsanskrit{Ambaṭṭha}, you must have come here for some purpose. You should focus on that. Though this \textsanskrit{Ambaṭṭha} is unqualified, he thinks he’s qualified. What is that but lack of qualifications?” 

When\marginnote{1.12.1} he said this, \textsanskrit{Ambaṭṭha} became angry and upset with the Buddha because of being described as unqualified. He even attacked and badmouthed the Buddha himself, saying, “The ascetic Gotama will be worsted!” He said to the Buddha, “Master Gotama, the Sakyan clan are rude, harsh, touchy, and argumentative. Riffraff they are, and riffraff they remain! They don’t honor, respect, revere, worship, or venerate brahmins. It is neither proper nor appropriate that the Sakyans—riffraff that they are—don’t honor, respect, revere, worship, or venerate brahmins.” 

And\marginnote{1.12.9} that’s how \textsanskrit{Ambaṭṭha} denigrated the Sakyans with the word “riffraff” for the first time. 

\subsection*{2.2. The Second Use of the Word “Riffraff” }

“But\marginnote{1.13.1} \textsanskrit{Ambaṭṭha}, how have the Sakyans wronged you?” 

“This\marginnote{1.13.2} one time, Master Gotama, I went to Kapilavatthu on some business for my teacher, the brahmin \textsanskrit{Pokkharasāti}. I approached the Sakyans in their meeting hall. Now at that time several Sakyans and Sakyan princes were sitting on high seats, poking each other with their fingers, giggling and playing together. In fact, they even presumed to giggle at me, and didn’t invite me to a seat. It is neither proper nor appropriate that the Sakyans—riffraff that they are—don’t honor, respect, revere, worship, or venerate brahmins.” 

And\marginnote{1.13.6} that’s how \textsanskrit{Ambaṭṭha} denigrated the Sakyans with the word “riffraff” for the second time. 

\subsection*{2.3. The Third Use of the Word “Riffraff” }

“Even\marginnote{1.14.1} a little quail, \textsanskrit{Ambaṭṭha}, speaks as she likes in her own nest. Kapilavatthu is the Sakyans own place, \textsanskrit{Ambaṭṭha}. It’s not worthy of the Venerable \textsanskrit{Ambaṭṭha} to lose his temper over such a small thing.” 

“Master\marginnote{1.14.3} Gotama, there are these four castes: aristocrats, brahmins, merchants, and workers. Three of these castes—aristocrats, merchants, and workers—in fact succeed only in serving the brahmins. It is neither proper nor appropriate that the Sakyans—riffraff that they are—don’t honor, respect, revere, worship, or venerate brahmins.” 

And\marginnote{1.14.9} that’s how \textsanskrit{Ambaṭṭha} denigrated the Sakyans with the word “riffraff” for the third time. 

\subsection*{2.4. The Word “Son of Bondservants” is Used }

Then\marginnote{1.15.1} it occurred to the Buddha, “This \textsanskrit{Ambaṭṭha} puts the Sakyans down too much by calling them riffraff. Why don’t I ask him about his own clan?” 

So\marginnote{1.15.3} the Buddha said to him, “What is your clan, \textsanskrit{Ambaṭṭha}?” 

“I\marginnote{1.15.5} am a \textsanskrit{Kaṇhāyana}, Master Gotama.” 

“But,\marginnote{1.15.6} recollecting the ancient name and clan of your mother and father, the Sakyans were the children of the masters, while you’re descended from the son of a female bondservant of the Sakyans. But the Sakyans regard King \textsanskrit{Okkāka} as their grandfather. 

Once\marginnote{1.15.8} upon a time, King \textsanskrit{Okkāka}, wishing to divert the royal succession to the son of his most beloved queen, banished the elder princes from the realm—\textsanskrit{Okkāmukha}, \textsanskrit{Karakaṇḍa}, Hatthinika, and \textsanskrit{Sinisūra}. They made their home beside a lotus pond on the slopes of the Himalayas, where there was a large teak grove. For fear of diluting their lineage, they slept with their own sisters. 

Then\marginnote{1.15.12} King \textsanskrit{Okkāka} addressed his ministers and counselors, ‘Where, sirs, have the princes settled now?’ 

‘Sire,\marginnote{1.15.14} there is a lotus pond on the slopes of the Himalayas, by a large grove of \textit{\textsanskrit{sāka}}, the teak tree. They’ve settled there. For fear of diluting their lineage, they are sleeping with their own (\textit{saka}) sisters.’ 

Then,\marginnote{1.15.16} \textsanskrit{Ambaṭṭha}, King \textsanskrit{Okkāka} expressed this heartfelt sentiment: ‘The princes are indeed Sakyans! The princes are indeed the best Sakyans!’ From that day on the Sakyans were recognized, and he was their founder. 

Now,\marginnote{1.16.1} King \textsanskrit{Okkāka} had a female bondservant named \textsanskrit{Disā}. She gave birth to a black boy. When he was born, Black Boy said: ‘Wash me, mum, bathe me! Get this filth off of me! I will be useful for you!’ Whereas these days when people see goblins they know them as goblins, in those days they knew goblins as ‘blackboys’. 

They\marginnote{1.16.7} said: ‘He spoke as soon as he was born—a blackboy is born! A goblin is born!’ From that day on the \textsanskrit{Kaṇhāyanas} were recognized, and he was their founder. That’s how, recollecting the ancient name and clan of your mother and father, the Sakyans were the children of the masters, while you’re descended from the son of a female bondservant of the Sakyans.” 

When\marginnote{1.17.1} he said this, those students said to him, “Master Gotama, please don’t put \textsanskrit{Ambaṭṭha} down too much by calling him the son of a bondservant. He’s well-born, a gentleman, learned, a good speaker, and astute. He’s capable of having a dialogue with Master Gotama about this.” 

So\marginnote{1.18.1} the Buddha said to them, “Well, students, if you think that \textsanskrit{Ambaṭṭha} is ill-born, not a gentleman, unlearned, a poor speaker, witless, and not capable of having a dialogue with me about this, then leave him aside and you can have a dialogue with me. But if you think that he’s well-born, a gentleman, learned, a good speaker, astute, and capable of having a dialogue with me about this, then you should stand aside and let him have a dialogue with me.” 

“He\marginnote{1.19.1} is capable of having a dialogue. We will be silent, and let \textsanskrit{Ambaṭṭha} have a dialogue with Master Gotama.” 

So\marginnote{1.20.1} the Buddha said to \textsanskrit{Ambaṭṭha}, “Well, \textsanskrit{Ambaṭṭha}, there’s a legitimate question that comes up. You won’t like it, but you ought to answer anyway. If you don’t answer, but dodge the issue, remain silent, or simply leave, your head will explode into seven pieces right here. What do you think, \textsanskrit{Ambaṭṭha}? According to what you have heard from elderly and senior brahmins, the teachers of teachers, what is the origin of the \textsanskrit{Kaṇhāyanas}, and who is their founder?” 

When\marginnote{1.20.6} he said this, \textsanskrit{Ambaṭṭha} kept silent. 

For\marginnote{1.20.7} a second time, the Buddha put the question, and for a second time \textsanskrit{Ambaṭṭha} kept silent. 

So\marginnote{1.20.10} the Buddha said to him, “Answer now, \textsanskrit{Ambaṭṭha}. Now is not the time for silence. If someone fails to answer a legitimate question when asked three times by the Buddha, their head explodes into seven pieces there and then.” 

Now\marginnote{1.21.1} at that time the spirit \textsanskrit{Vajirapāṇi}, holding a massive iron spear, burning, blazing, and glowing, stood in the sky above \textsanskrit{Ambaṭṭha}, thinking, “If this \textsanskrit{Ambaṭṭha} doesn’t answer when asked a third time, I’ll blow his head into seven pieces there and then!” And both the Buddha and \textsanskrit{Ambaṭṭha} could see \textsanskrit{Vajirapāṇi}. 

\textsanskrit{Ambaṭṭha}\marginnote{1.21.4} was terrified, shocked, and awestruck. Looking to the Buddha for shelter, protection, and refuge, he sat down close by the Buddha and said, “What did you say? Please repeat the question.” 

“What\marginnote{1.21.7} do you think, \textsanskrit{Ambaṭṭha}? According to what you have heard from elderly and senior brahmins, the teachers of teachers, what is the origin of the \textsanskrit{Kaṇhāyanas}, and who is their founder?” 

“I\marginnote{1.21.9} have heard, Master Gotama, that it is just as you say. That’s the origin of the \textsanskrit{Kaṇhāyanas}, and that’s who their founder is.” 

\subsection*{2.5. The Discussion of \textsanskrit{Ambaṭṭha}’s Heritage }

When\marginnote{1.22.1} he said this, those students made an uproar, “It turns out \textsanskrit{Ambaṭṭha} is ill-born, not a gentleman, son of a Sakyan bondservant, and that the Sakyans are sons of his masters! And it seems that the ascetic Gotama spoke only the truth, though we presumed to rebuke him!” 

Then\marginnote{1.23.1} it occurred to the Buddha, “These students put \textsanskrit{Ambaṭṭha} down too much by calling him the son of a bondservant. Why don’t I get him out of this?” 

So\marginnote{1.23.3} the Buddha said to the students, “Students, please don’t put \textsanskrit{Ambaṭṭha} down too much by calling him the son of a bondservant. That Black Boy was an eminent sage. He went to a southern country and memorized the Prime Spell. Then he approached King \textsanskrit{Okkāka} and asked for the hand of his daughter \textsanskrit{Maddarūpī}. 

The\marginnote{1.23.7} king said to him, ‘Who the hell is this son of a bondservant to ask for the hand of my daughter!’ Angry and upset he fastened a razor-tipped arrow. But he wasn’t able to either shoot it or to relax it. 

Then\marginnote{1.23.10} the ministers and counselors approached the sage Black Boy and said: ‘Spare the king, sir, spare him!’ 

‘The\marginnote{1.23.13} king will be safe. But if he shoots the arrow downwards, there will be an earthquake across the entire realm.’ 

‘Spare\marginnote{1.23.14} the king, sir, and spare the country!’ 

‘Both\marginnote{1.23.15} king and country will be safe. But if he shoots the arrow upwards, there will be no rain in the entire realm for seven years.’ 

‘Spare\marginnote{1.23.16} the king, sir, spare the country, and let there be rain!’ 

‘Both\marginnote{1.23.17} king and country will be safe, and the rain will fall. And if the king aims the arrow at the crown prince, he will be safe and untouched.’ 

So\marginnote{1.23.18} the ministers said to \textsanskrit{Okkāka}: ‘\textsanskrit{Okkāka} must aim the arrow at the crown prince. He will be safe and untouched.’ 

So\marginnote{1.23.20} King \textsanskrit{Okkāka} aimed the arrow at the crown prince. And he was safe and untouched. Then the king was terrified, shocked, and awestruck. Scared by the prime punishment, he gave the hand of his daughter \textsanskrit{Maddarūpī}. 

Students,\marginnote{1.23.22} please don’t put \textsanskrit{Ambaṭṭha} down too much by calling him the son of a bondservant. That Black Boy was an eminent sage.” 

\section*{3. The Supremacy of the Aristocrats }

Then\marginnote{1.24.1} the Buddha addressed \textsanskrit{Ambaṭṭha}, “What do you think, \textsanskrit{Ambaṭṭha}? Suppose an aristocrat boy was to sleep with a brahmin girl, and they had a son. Would he receive a seat and water from the brahmins?” 

“He\marginnote{1.24.5} would, Master Gotama.” 

“And\marginnote{1.24.6} would the brahmins feed him at an offering of food for ancestors, an offering of a dish of milk-rice, a sacrifice, or a feast for guests?” 

“They\marginnote{1.24.7} would.” 

“And\marginnote{1.24.8} would the brahmins teach him the hymns or not?” 

“They\marginnote{1.24.9} would.” 

“And\marginnote{1.24.10} would he be kept from the women or not?” 

“He\marginnote{1.24.11} would not.” 

“And\marginnote{1.24.12} would the aristocrats anoint him as king?” 

“No,\marginnote{1.24.13} Master Gotama. Why is that? Because his maternity is unsuitable.” 

“What\marginnote{1.25.1} do you think, \textsanskrit{Ambaṭṭha}? Suppose a brahmin boy was to sleep with an aristocrat girl, and they had a son. Would he receive a seat and water from the brahmins?” 

“He\marginnote{1.25.4} would, Master Gotama.” 

“And\marginnote{1.25.5} would the brahmins feed him at an offering of food for ancestors, an offering of a dish of milk-rice, a sacrifice, or a feast for guests?” 

“They\marginnote{1.25.6} would.” 

“And\marginnote{1.25.7} would the brahmins teach him the hymns or not?” 

“They\marginnote{1.25.8} would.” 

“And\marginnote{1.25.9} would he be kept from the women or not?” 

“He\marginnote{1.25.10} would not.” 

“And\marginnote{1.25.11} would the aristocrats anoint him as king?” 

“No,\marginnote{1.25.12} Master Gotama. Why is that? Because his paternity is unsuitable.” 

“And\marginnote{1.26.1} so, \textsanskrit{Ambaṭṭha}, the aristocrats are superior and the brahmins inferior, whether comparing women with women or men with men. What do you think, \textsanskrit{Ambaṭṭha}? Suppose the brahmins for some reason were to shave a brahmin’s head, inflict him with a sack of ashes, and banish him from the nation or the city. Would he receive a seat and water from the brahmins?” 

“No,\marginnote{1.26.5} Master Gotama.” 

“And\marginnote{1.26.6} would the brahmins feed him at an offering of food for ancestors, an offering of a dish of milk-rice, a sacrifice, or a feast for guests?” 

“No,\marginnote{1.26.7} Master Gotama.” 

“And\marginnote{1.26.8} would the brahmins teach him the hymns or not?” 

“No,\marginnote{1.26.9} Master Gotama.” 

“And\marginnote{1.26.10} would he be kept from the women or not?” 

“He\marginnote{1.26.11} would be.” 

“What\marginnote{1.27.1} do you think, \textsanskrit{Ambaṭṭha}? Suppose the aristocrats for some reason were to shave an aristocrat’s head, inflict him with a sack of ashes, and banish him from the nation or the city. Would he receive a seat and water from the brahmins?” 

“He\marginnote{1.27.4} would, Master Gotama.” 

“And\marginnote{1.27.5} would the brahmins feed him at an offering of food for ancestors, an offering of a dish of milk-rice, a sacrifice, or a feast for guests?” 

“They\marginnote{1.27.6} would.” 

“And\marginnote{1.27.7} would the brahmins teach him the hymns or not?” 

“They\marginnote{1.27.8} would.” 

“And\marginnote{1.27.9} would he be kept from the women or not?” 

“He\marginnote{1.27.10} would not.” 

“At\marginnote{1.27.11} this point, \textsanskrit{Ambaṭṭha}, that aristocrat has reached rock bottom, with head shaven, inflicted with a sack of ashes, and banished from city or nation. Yet still the aristocrats are superior and the brahmins inferior. \textsanskrit{Brahmā} \textsanskrit{Sanaṅkumāra} also spoke this verse: 

\begin{verse}%
‘The\marginnote{1.28.2} aristocrat is first among people \\
who take clan as the standard. \\
But one accomplished in knowledge and conduct \\
is first among gods and humans.’ 

%
\end{verse}

That\marginnote{1.28.6} verse was well sung by \textsanskrit{Brahmā} \textsanskrit{Sanaṅkumāra}, not poorly sung; well spoken, not poorly spoken, beneficial, not harmful, and it was approved by me. For I also say this: 

\begin{verse}%
The\marginnote{1.28.8} aristocrat is first among people \\
who take clan as the standard. \\
But one accomplished in knowledge and conduct \\
is first among gods and humans.” 

%
\end{verse}

\section*{4. Knowledge and Conduct }

“But\marginnote{2.1.1} what, Master Gotama, is that conduct, and what is that knowledge?” 

“\textsanskrit{Ambaṭṭha},\marginnote{2.1.2} in the supreme knowledge and conduct there is no discussion of ancestry or clan or pride—‘You deserve me’ or ‘You don’t deserve me.’ Wherever there is giving and taking in marriage there is such discussion. Whoever is attached to questions of ancestry or clan or pride, or to giving and taking in marriage, is far from the supreme knowledge and conduct. The realization of supreme knowledge and conduct occurs when you’ve given up such things.” 

“But\marginnote{2.2.1} what, Master Gotama, is that conduct, and what is that knowledge?” 

“\textsanskrit{Ambaṭṭha},\marginnote{2.2.2} it’s when a Realized One arises in the world, perfected, a fully awakened Buddha, accomplished in knowledge and conduct, holy, knower of the world, supreme guide for those who wish to train, teacher of gods and humans, awakened, blessed. He has realized with his own insight this world—with its gods, \textsanskrit{Māras} and \textsanskrit{Brahmās}, this population with its ascetics and brahmins, gods and humans—and he makes it known to others. He teaches Dhamma that’s good in the beginning, good in the middle, and good in the end, meaningful and well-phrased. And he reveals a spiritual practice that’s entirely full and pure. A householder hears that teaching, or a householder’s child, or someone reborn in some clan. They gain faith in the Realized One, and reflect … 

Quite\marginnote{2.2.8} secluded from sensual pleasures, secluded from unskillful qualities, they enter and remain in the first absorption … This pertains to their conduct. 

Furthermore,\marginnote{2.2.10} as the placing of the mind and keeping it connected are stilled, a mendicant enters and remains in the second absorption … This pertains to their conduct. 

Furthermore,\marginnote{2.2.12} with the fading away of rapture, they enter and remain in the third absorption … This pertains to their conduct. 

Furthermore,\marginnote{2.2.14} giving up pleasure and pain, and ending former happiness and sadness, they enter and remain in the fourth absorption … This pertains to their conduct. This is that conduct. 

When\marginnote{2.2.17} their mind has become immersed in \textsanskrit{samādhi} like this—purified, bright, flawless, rid of corruptions, pliable, workable, steady, and imperturbable—they extend it and project it toward knowledge and vision. This pertains to their knowledge. … They understand: ‘There is no return to any state of existence.’ This pertains to their knowledge. This is that knowledge. 

This\marginnote{2.2.22} mendicant is said to be ‘accomplished in knowledge’, and also ‘accomplished in conduct’, and also ‘accomplished in knowledge and conduct’. And, \textsanskrit{Ambaṭṭha}, there is no accomplishment in knowledge and conduct that is better or finer than this. 

\section*{5. Four Drains }

There\marginnote{2.3.1} are these four drains that affect this supreme knowledge and conduct. What four? Firstly, take some ascetic or brahmin who, not managing to obtain this supreme knowledge and conduct, plunges into a wilderness region carrying their stuff with a shoulder-pole, thinking they will get by eating fallen fruit. In fact they succeed only in serving someone accomplished in knowledge and conduct. This is the first drain that affects this supreme knowledge and conduct. 

Furthermore,\marginnote{2.3.7} take some ascetic or brahmin who, not managing to obtain this supreme knowledge and conduct or to get by eating fallen fruit, plunges into a wilderness region carrying a spade and basket, thinking they will get by eating tubers and fruit. In fact they succeed only in serving someone accomplished in knowledge and conduct. This is the second drain that affects this supreme knowledge and conduct. 

Furthermore,\marginnote{2.3.11} take some ascetic or brahmin who, not managing to obtain this supreme knowledge and conduct, or to get by eating fallen fruit, or to get by eating tubers and fruit, sets up a fire chamber in the neighborhood of a village or town and dwells there serving the sacred flame. In fact they succeed only in serving someone accomplished in knowledge and conduct. This is the third drain that affects this supreme knowledge and conduct. 

Furthermore,\marginnote{2.3.14} take some ascetic or brahmin who, not managing to obtain this supreme knowledge and conduct, or to get by eating fallen fruit, or to get by eating tubers and fruit, or to serve the sacred flame, sets up a fire chamber in the central square and dwells there, thinking: ‘When an ascetic or brahmin comes from the four quarters, I will honor them as best I can.’ In fact they succeed only in serving someone accomplished in knowledge and conduct. This is the fourth drain that affects this supreme knowledge and conduct. These are the four drains that affect this supreme knowledge and conduct. 

What\marginnote{2.4.1} do you think, \textsanskrit{Ambaṭṭha}? Is this supreme knowledge and conduct seen in your own tradition?” 

“No,\marginnote{2.4.3} Master Gotama. Who am I and my tradition compared with the supreme knowledge and conduct? We are far from that.” 

“What\marginnote{2.4.6} do you think, \textsanskrit{Ambaṭṭha}? Since you’re not managing to obtain this supreme knowledge and conduct, have you with your tradition plunged into a wilderness region carrying your stuff with a shoulder-pole, thinking you will get by eating fallen fruit?” 

“No,\marginnote{2.4.9} Master Gotama.” 

“What\marginnote{2.4.10} do you think, \textsanskrit{Ambaṭṭha}? Have you with your tradition … plunged into a wilderness region carrying a spade and basket, thinking you will get by eating tubers and fruit?” 

“No,\marginnote{2.4.13} Master Gotama.” 

“What\marginnote{2.4.14} do you think, \textsanskrit{Ambaṭṭha}? Have you with your tradition … set up a fire chamber in the neighborhood of a village or town and dwelt there serving the sacred flame?” 

“No,\marginnote{2.4.16} Master Gotama.” 

“What\marginnote{2.4.17} do you think, \textsanskrit{Ambaṭṭha}? Have you with your tradition … set up a fire chamber in the central square and dwelt there, thinking: ‘When an ascetic or brahmin comes from the four quarters, I will honor them as best I can’?” 

“No,\marginnote{2.4.20} Master Gotama.” 

“So\marginnote{2.5.1} you with your tradition are not only inferior to the supreme knowledge and conduct, you are even inferior to the four drains that affect the supreme knowledge and conduct. But you have been told this by your teacher, the brahmin \textsanskrit{Pokkharasāti}: ‘Who are these shavelings, fake ascetics, riffraff, black spawn from the feet of our kinsman compared with conversation with the brahmins of the three knowledges?” Yet he himself has not even fulfilled one of the drains! See, \textsanskrit{Ambaṭṭha}, how your teacher \textsanskrit{Pokkharasāti} has wronged you. 

\section*{6. Being Like the Sages of the Past }

But\marginnote{2.6.1} \textsanskrit{Pokkharasāti} lives off an endowment provided by King Pasenadi of Kosala. But the king won’t even grant him an audience face to face. When he consults, he does so behind a curtain. Why wouldn’t the king grant a face to face audience with someone who’d receive his legitimate presentation of food? See, \textsanskrit{Ambaṭṭha}, how your teacher \textsanskrit{Pokkharasāti} has wronged you. 

What\marginnote{2.7.1} do you think, \textsanskrit{Ambaṭṭha}? Suppose King Pasenadi was holding consultations with warrior-chiefs or chieftains while sitting on an elephant’s neck or on horseback, or while standing on the mat in a chariot. And suppose he’d get down from that place and stand aside. Then along would come a worker or their bondservant, who’d stand in the same place and continue the consultation: ‘This is what King Pasenadi says, and this too is what the king says.’ Though he spoke the king’s words and gave the king’s advice, does that qualify him to be the king or the king’s minister?” 

“No,\marginnote{2.7.8} Master Gotama.” 

“In\marginnote{2.8.1} the same way, \textsanskrit{Ambaṭṭha}, the brahmin seers of the past were \textsanskrit{Aṭṭhaka}, \textsanskrit{Vāmaka}, \textsanskrit{Vāmadeva}, \textsanskrit{Vessāmitta}, Yamadaggi, \textsanskrit{Aṅgīrasa}, \textsanskrit{Bhāradvāja}, \textsanskrit{Vāseṭṭha}, Kassapa, and Bhagu. They were the authors and propagators of the hymns. Their hymnal was sung and propagated and compiled in ancient times; and these days, brahmins continue to sing and chant it, chanting what was chanted and teaching what was taught. You might imagine that, since you’ve learned their hymns by heart in your own tradition, that makes you a hermit or someone on the path to becoming a hermit. But that is not possible. 

What\marginnote{2.9.1} do you think, \textsanskrit{Ambaṭṭha}? According to what you have heard from elderly and senior brahmins, the teachers of teachers, did those ancient brahmin hermits—nicely bathed and anointed, with hair and beard dressed, bedecked with jewels, earrings, and bracelets, dressed in white—amuse themselves, supplied and provided with the five kinds of sensual stimulation, like you do today in your tradition?” 

“No,\marginnote{2.9.5} Master Gotama.” 

“Did\marginnote{2.10.1} they eat boiled fine rice, garnished with clean meat, with the dark grains picked out, served with many soups and sauces, like you do today in your tradition?” 

“No,\marginnote{2.10.3} Master Gotama.” 

“Did\marginnote{2.10.4} they amuse themselves with girls wearing thongs that show off their curves, like you do today in your tradition?” 

“No,\marginnote{2.10.6} Master Gotama.” 

“Did\marginnote{2.10.7} they drive about in chariots drawn by mares with plaited manes, whipping and lashing them onward with long goads, like you do today in your tradition?” 

“No,\marginnote{2.10.9} Master Gotama.” 

“Did\marginnote{2.10.10} they get men with long swords to guard them in fortresses with moats dug and barriers in place, like you do today in your tradition?” 

“No,\marginnote{2.10.12} Master Gotama.” 

“So,\marginnote{2.10.13} \textsanskrit{Ambaṭṭha}, in your own tradition you are neither hermit nor someone on the path to becoming a hermit. Whoever has any doubt or uncertainty about me, let them ask me and I will clear up their doubts with my answer.” 

\section*{7. Seeing the Two Marks }

Then\marginnote{2.11.1} the Buddha came out of his dwelling and proceeded to begin walking mindfully, and \textsanskrit{Ambaṭṭha} did likewise. Then while walking beside the Buddha, \textsanskrit{Ambaṭṭha} scrutinized his body for the thirty-two marks of a great man. He saw all of them except for two, which he had doubts about: whether the private parts are covered in a foreskin, and the largeness of the tongue. 

Then\marginnote{2.12.1} it occurred to the Buddha, “This brahmin student \textsanskrit{Ambaṭṭha} sees all the marks except for two, which he has doubts about: whether the private parts are covered in a foreskin, and the largeness of the tongue.” Then the Buddha used his psychic power to will that \textsanskrit{Ambaṭṭha} would see his private parts covered in a foreskin. And he stuck out his tongue and stroked back and forth on his ear holes and nostrils, and covered his entire forehead with his tongue. 

Then\marginnote{2.12.7} \textsanskrit{Ambaṭṭha} thought, “The ascetic Gotama possesses the thirty-two marks completely, lacking none.” 

He\marginnote{2.12.9} said to the Buddha, “Well, now, sir, I must go. I have many duties, and much to do.” 

“Please,\marginnote{2.12.11} \textsanskrit{Ambaṭṭha}, go at your convenience.” Then \textsanskrit{Ambaṭṭha} mounted his mare-drawn chariot and left. 

Now\marginnote{2.13.1} at that time the brahmin \textsanskrit{Pokkharasāti} had come out from \textsanskrit{Ukkaṭṭhā} together with a large group of brahmins and was sitting in his own park just waiting for \textsanskrit{Ambaṭṭha}. Then \textsanskrit{Ambaṭṭha} entered the park. He went by carriage as far as the terrain allowed, then descended and approached the brahmin \textsanskrit{Pokkharasāti} on foot. He bowed and sat down to one side, and \textsanskrit{Pokkharasāti} said to him: 

“I\marginnote{2.14.2} hope, dear \textsanskrit{Ambaṭṭha}, you saw the Master Gotama?” 

“I\marginnote{2.14.3} saw him, sir.” 

“Well,\marginnote{2.14.4} does he live up to his reputation or not?” 

“He\marginnote{2.14.6} does, sir. Master Gotama possesses the thirty-two marks completely, lacking none.” 

“And\marginnote{2.14.8} did you have some discussion with him?” 

“I\marginnote{2.14.9} did.” 

“And\marginnote{2.14.10} what kind of discussion did you have with him?” Then \textsanskrit{Ambaṭṭha} informed \textsanskrit{Pokkharasāti} of all they had discussed. 

Then\marginnote{2.15.1} \textsanskrit{Pokkharasāti} said to \textsanskrit{Ambaṭṭha}, “Oh, our bloody fake scholar, our fake learned man, who pretends to be proficient in the three Vedas! A man who behaves like this ought, when their body breaks up, after death, to be reborn in a place of loss, a bad place, the underworld, hell. It’s only because you repeatedly attacked Master Gotama like that that he kept bringing up charges against us!” Angry and upset, he kicked \textsanskrit{Ambaṭṭha} over, and wanted to go and see the Buddha right away. 

\section*{8. \textsanskrit{Pokkharasāti} Visits the Buddha }

Then\marginnote{2.16.1} those brahmins said to \textsanskrit{Pokkharasāti}, “It’s much too late to visit the ascetic Gotama today. You can visit him tomorrow.” 

So\marginnote{2.16.4} \textsanskrit{Pokkharasāti} had a variety of delicious foods prepared in his own home. Then he mounted a carriage and, with attendants carrying torches, set out from \textsanskrit{Ukkaṭṭhā} for the forest near \textsanskrit{Icchānaṅgala}. He went by carriage as far as the terrain allowed, then descended and entered the monastery on foot. He went up to the Buddha and exchanged greetings with him. When the greetings and polite conversation were over, he sat down to one side and said to the Buddha, “Master Gotama, has my pupil, the student \textsanskrit{Ambaṭṭha}, come here?” 

“Yes\marginnote{2.17.3} he has, brahmin.” 

“And\marginnote{2.17.4} did you have some discussion with him?” 

“I\marginnote{2.17.5} did.” 

“And\marginnote{2.17.6} what kind of discussion did you have with him?” Then the Buddha informed \textsanskrit{Pokkharasāti} of all they had discussed. 

Then\marginnote{2.17.8} \textsanskrit{Pokkharasāti} said to the Buddha, “\textsanskrit{Ambaṭṭha} is a fool, Master Gotama. Please forgive him.” 

“May\marginnote{2.17.10} the student \textsanskrit{Ambaṭṭha} be happy, brahmin.” 

Then\marginnote{2.18.1} \textsanskrit{Pokkharasāti} scrutinized the Buddha’s body for the thirty-two marks of a great man. He saw all of them except for two, which he had doubts about: whether the private parts are covered in a foreskin, and the largeness of the tongue. 

Then\marginnote{2.18.5} it occurred to the Buddha, “\textsanskrit{Pokkharasāti} sees all the marks except for two, which he has doubts about: whether the private parts are covered in a foreskin, and the largeness of the tongue.” Then the Buddha used his psychic power to will that \textsanskrit{Pokkharasāti} would see his private parts covered in a foreskin. And he stuck out his tongue and stroked back and forth on his ear holes and nostrils, and covered his entire forehead with his tongue. 

\textsanskrit{Pokkharasāti}\marginnote{2.19.1} thought, “The ascetic Gotama possesses the thirty-two marks completely, lacking none.” 

He\marginnote{2.19.3} said to the Buddha, “Would Master Gotama together with the mendicant \textsanskrit{Saṅgha} please accept today’s meal from me?” The Buddha consented in silence. 

Then,\marginnote{2.20.1} knowing that the Buddha had consented, \textsanskrit{Pokkharasāti} announced the time to him, “It’s time, Master Gotama, the meal is ready.” Then the Buddha robed up in the morning and, taking his bowl and robe, went to the home of \textsanskrit{Pokkharasāti} together with the mendicant \textsanskrit{Saṅgha}, where he sat on the seat spread out. Then \textsanskrit{Pokkharasāti} served and satisfied the Buddha with his own hands with a variety of delicious foods, while his students served the \textsanskrit{Saṅgha}. When the Buddha had eaten and washed his hand and bowl, \textsanskrit{Pokkharasāti} took a low seat and sat to one side. 

Then\marginnote{2.21.1} the Buddha taught him step by step, with a talk on giving, ethical conduct, and heaven. He explained the drawbacks of sensual pleasures, so sordid and corrupt, and the benefit of renunciation. And when the Buddha knew that \textsanskrit{Pokkharasāti}’s mind was ready, pliable, rid of hindrances, elated, and confident he explained the special teaching of the Buddhas: suffering, its origin, its cessation, and the path. Just as a clean cloth rid of stains would properly absorb dye, in that very seat the stainless, immaculate vision of the Dhamma arose in the brahmin \textsanskrit{Pokkharasāti}: “Everything that has a beginning has an end.” 

\section*{9. \textsanskrit{Pokkharasāti} Declares Himself a Lay Follower }

Then\marginnote{2.22.1} \textsanskrit{Pokkharasāti} saw, attained, understood, and fathomed the Dhamma. He went beyond doubt, got rid of indecision, and became self-assured and independent of others regarding the Teacher’s instructions. He said to the Buddha, “Excellent, Master Gotama! Excellent! As if he were righting the overturned, or revealing the hidden, or pointing out the path to the lost, or lighting a lamp in the dark so people with good eyes can see what’s there, just so has Master Gotama made the Teaching clear in many ways. Together with my children, wives, retinue, and ministers, I go for refuge to Master Gotama, to the teaching, and to the mendicant \textsanskrit{Saṅgha}. From this day forth, may Master Gotama remember me as a lay follower who has gone for refuge for life. 

Just\marginnote{2.22.6} as Master Gotama visits other devoted families in \textsanskrit{Ukkaṭṭhā}, may he visit mine. The brahmin boys and girls there will bow to you, rise in your presence, give you a seat and water, and gain confidence in their hearts. That will be for their lasting welfare and happiness.” 

“That’s\marginnote{2.22.8} good of you to say, householder.” 

%
\chapter*{{\suttatitleacronym DN 4}{\suttatitletranslation With Soṇadaṇḍa }{\suttatitleroot Soṇadaṇḍasutta}}
\addcontentsline{toc}{chapter}{\tocacronym{DN 4} \toctranslation{With Soṇadaṇḍa } \tocroot{Soṇadaṇḍasutta}}
\markboth{With Soṇadaṇḍa }{Soṇadaṇḍasutta}
\extramarks{DN 4}{DN 4}

\section*{1. The Brahmins and Householders of \textsanskrit{Campā} }

\scevam{So\marginnote{1.1} I have heard. }At one time the Buddha was wandering in the land of the \textsanskrit{Aṅgas} together with a large \textsanskrit{Saṅgha} of around five hundred mendicants when he arrived at \textsanskrit{Campā}, where he stayed by the banks of the \textsanskrit{Gaggarā} Lotus Pond. 

Now\marginnote{1.4} at that time the brahmin \textsanskrit{Soṇadaṇḍa} was living in \textsanskrit{Campā}. It was a crown property given by King Seniya \textsanskrit{Bimbisāra} of Magadha, teeming with living creatures, full of hay, wood, water, and grain, a royal endowment of the highest quality. 

The\marginnote{2.1} brahmins and householders of \textsanskrit{Campā} heard: 

“It\marginnote{2.2} seems the ascetic Gotama—a Sakyan, gone forth from a Sakyan family—has arrived at \textsanskrit{Campā} and is staying on the banks of the \textsanskrit{Gaggarā} Lotus Pond. He has this good reputation: ‘That Blessed One is perfected, a fully awakened Buddha, accomplished in knowledge and conduct, holy, knower of the world, supreme guide for those who wish to train, teacher of gods and humans, awakened, blessed.’ He has realized with his own insight this world—with its gods, \textsanskrit{Māras} and \textsanskrit{Brahmās}, this population with its ascetics and brahmins, gods and humans—and he makes it known to others. He teaches Dhamma that’s good in the beginning, good in the middle, and good in the end, meaningful and well-phrased. And he reveals a spiritual practice that’s entirely full and pure. It’s good to see such perfected ones.” Then, having departed \textsanskrit{Campā}, they formed into companies and headed to the \textsanskrit{Gaggarā} Lotus Pond. 

Now\marginnote{3.1} at that time the brahmin \textsanskrit{Soṇadaṇḍa} had retired to the upper floor of his stilt longhouse for his midday nap. He saw the brahmins and householders heading for the lotus pond, and addressed his steward, “My steward, why are the brahmins and householders headed for the \textsanskrit{Gaggarā} Lotus Pond?” 

“The\marginnote{3.5} ascetic Gotama has arrived at \textsanskrit{Campā} and is staying on the banks of the \textsanskrit{Gaggarā} Lotus Pond. He has this good reputation: ‘That Blessed One is perfected, a fully awakened Buddha, accomplished in knowledge and conduct, holy, knower of the world, supreme guide for those who wish to train, teacher of gods and humans, awakened, blessed.’ They’re going to see that Master Gotama.” 

“Well\marginnote{3.9} then, go to the brahmins and householders and say to them: ‘Sirs, the brahmin \textsanskrit{Soṇadaṇḍa} asks you to wait, as he will also go to see the ascetic Gotama.’” 

“Yes,\marginnote{3.12} sir,” replied the steward, and did as he was asked. 

\section*{2. The Qualities of \textsanskrit{Soṇadaṇḍa} }

Now\marginnote{4.1} at that time around five hundred brahmins from abroad were residing in \textsanskrit{Campā} on some business. They heard that the brahmin \textsanskrit{Soṇadaṇḍa} was going to see the ascetic Gotama. They approached \textsanskrit{Soṇadaṇḍa} and said to him, “Is it really true that you are going to see the ascetic Gotama?” 

“Yes,\marginnote{4.6} gentlemen, it is true.” 

“Please\marginnote{5.1} don’t, master \textsanskrit{Soṇadaṇḍa}! It’s not appropriate for you to go to see the ascetic Gotama. For if you do so, your reputation will diminish and his will increase. For this reason it’s not appropriate for you to go to see the ascetic Gotama; it’s appropriate that he comes to see you. 

You\marginnote{5.6} are well born on both your mother’s and father’s side, of pure descent, irrefutable and impeccable in questions of ancestry back to the seventh paternal generation. For this reason it’s not appropriate for you to go to see the ascetic Gotama; it’s appropriate that he comes to see you. 

You’re\marginnote{5.9} rich, affluent, and wealthy. … 

You\marginnote{5.10} recite and remember the hymns, and have mastered the three Vedas, together with their vocabularies, ritual, phonology and etymology, and the testament as fifth. You know philology and grammar, and are well versed in cosmology and the marks of a great man. … 

You\marginnote{5.11} are attractive, good-looking, lovely, of surpassing beauty. You are magnificent, splendid, remarkable to behold. … 

You\marginnote{5.12} are ethical, mature in ethical conduct. … 

You’re\marginnote{5.13} a good speaker, with a polished, clear, and articulate voice that expresses the meaning. … 

You\marginnote{5.14} teach the teachers of many, and teach three hundred students to recite the hymns. Many students come from various districts and countries for the sake of the hymns, wishing to learn the hymns. … 

You’re\marginnote{5.15} old, elderly and senior, advanced in years, and have reached the final stage of life. The ascetic Gotama is young, and has newly gone forth. … 

You’re\marginnote{5.17} honored, respected, revered, venerated, and esteemed by King \textsanskrit{Bimbisāra} of Magadha … 

and\marginnote{5.18} the brahmin \textsanskrit{Pokkharasāti}. … 

You\marginnote{5.19} live in \textsanskrit{Campā}, a crown property given by King Seniya \textsanskrit{Bimbisāra} of Magadha, teeming with living creatures, full of hay, wood, water, and grain, a royal endowment of the highest quality. For this reason, too, it’s not appropriate for you to go to see the ascetic Gotama; it’s appropriate that he comes to see you. 

\section*{3. The Qualities of the Buddha }

When\marginnote{6.1} they had spoken, \textsanskrit{Soṇadaṇḍa} said to those brahmins: 

“Well\marginnote{6.2} then, gentlemen, listen to why it’s appropriate for me to go to see the ascetic Gotama, and it’s not appropriate for him to come to see me. He is well born on both his mother’s and father’s side, of pure descent, irrefutable and impeccable in questions of ancestry back to the seventh paternal generation. For this reason it’s not appropriate for the ascetic Gotama to come to see me; rather, it’s appropriate for me to go to see him. 

When\marginnote{6.7} he went forth he abandoned a large family circle. … 

When\marginnote{6.8} he went forth he abandoned abundant gold coin and bullion stored in dungeons and towers. … 

He\marginnote{6.9} went forth from the lay life to homelessness while still a youth, young, black-haired, blessed with youth, in the prime of life. … 

Though\marginnote{6.10} his mother and father wished otherwise, weeping with tearful faces, he shaved off his hair and beard, dressed in ocher robes, and went forth from the lay life to homelessness. … 

He\marginnote{6.11} is attractive, good-looking, lovely, of surpassing beauty. He is magnificent, splendid, remarkable to behold. … 

He\marginnote{6.12} is ethical, possessing ethical conduct that is noble and skillful. … 

He’s\marginnote{6.13} a good speaker, with a polished, clear, and articulate voice that expresses the meaning. … 

He’s\marginnote{6.14} a teacher of teachers. … 

He\marginnote{6.15} has ended sensual desire, and is rid of caprice. … 

He\marginnote{6.16} teaches the efficacy of deeds and action. He doesn’t wish any harm upon the community of brahmins. … 

He\marginnote{6.17} went forth from an eminent family of unbroken aristocratic lineage. … 

He\marginnote{6.18} went forth from a rich, affluent, and wealthy family. … 

People\marginnote{6.19} come from distant lands and distant countries to question him. … 

Many\marginnote{6.20} thousands of deities have gone for refuge for life to him. … 

He\marginnote{6.21} has this good reputation: ‘That Blessed One is perfected, a fully awakened Buddha, accomplished in knowledge and conduct, holy, knower of the world, supreme guide for those who wish to train, teacher of gods and humans, awakened, blessed.’ … 

He\marginnote{6.23} has the thirty-two marks of a great man. … 

He\marginnote{6.24} is welcoming, congenial, polite, smiling, open, the first to speak. … 

He’s\marginnote{6.25} honored, respected, revered, venerated, and esteemed by the four assemblies. … 

Many\marginnote{6.26} gods and humans are devoted to him. … 

While\marginnote{6.27} he is residing in a village or town, non-human entities do not harass them. … 

He\marginnote{6.28} leads an order and a community, and teaches a community, and is said to be the best of the various religious founders. He didn’t come by his fame in the same ways as those other ascetics and brahmins. Rather, he came by his fame due to his supreme knowledge and conduct. … 

King\marginnote{6.30} Seniya \textsanskrit{Bimbisāra} of Magadha and his wives and children have gone for refuge for life to the ascetic Gotama. … 

King\marginnote{6.31} Pasenadi of Kosala and his wives and children have gone for refuge for life to the ascetic Gotama. … 

The\marginnote{6.32} brahmin \textsanskrit{Pokkharasāti} and his wives and children have gone for refuge for life to the ascetic Gotama. … 

He’s\marginnote{6.33} honored, respected, revered, venerated, and esteemed by King \textsanskrit{Bimbisāra} of Magadha … 

King\marginnote{6.34} Pasenadi of Kosala … 

and\marginnote{6.35} the brahmin \textsanskrit{Pokkharasāti}. 

The\marginnote{6.36} ascetic Gotama has arrived at \textsanskrit{Campā} and is staying at the \textsanskrit{Gaggarā} Lotus Pond. Any ascetic or brahmin who comes to stay in our village district is our guest, and should be honored and respected as such. For this reason, too, it’s not appropriate for Master Gotama to come to see me; rather, it’s appropriate for me to go to see him. This is the extent of Master Gotama’s praise that I have learned. But his praises are not confined to this, for the praise of Master Gotama is limitless.” 

When\marginnote{6.45} he had spoken, those brahmins said to him, “According to \textsanskrit{Soṇadaṇḍa}’s praises, if Master Gotama were staying within a hundred leagues, it’d be worthwhile for a faithful gentleman to go to see him, even if he had to carry his own provisions in a shoulder bag.” 

“Well\marginnote{6.47} then, gentlemen, let’s all go to see the ascetic Gotama.” 

\section*{4. \textsanskrit{Soṇadaṇḍa}’s Second Thoughts }

Then\marginnote{7.1} \textsanskrit{Soṇadaṇḍa} together with a large group of brahmins went to see the Buddha. 

But\marginnote{8.1} as he reached the far side of the forest, this thought came to mind, “Suppose I was to ask the ascetic Gotama a question. He might say to me: ‘Brahmin, you shouldn’t ask your question like that. This is how you should ask it.’ And the assembly might disparage me for that: ‘\textsanskrit{Soṇadaṇḍa} is foolish and incompetent. He’s not able to properly ask the ascetic Gotama a question.’ And when you’re disparaged by the assembly, your reputation diminishes. When your reputation diminishes, your wealth also diminishes. But my wealth relies on my reputation. 

Or\marginnote{8.9} if the ascetic Gotama asks me a question, I might not satisfy him with my answer. He might say to me: ‘Brahmin, you shouldn’t answer the question like that. This is how you should answer it.’ And the assembly might disparage me for that: ‘\textsanskrit{Soṇadaṇḍa} is foolish and incompetent. He’s not able to satisfy the ascetic Gotama’s mind with his answer.’ And when you’re disparaged by the assembly, your reputation diminishes. When your reputation diminishes, your wealth also diminishes. But my wealth relies on my reputation. 

On\marginnote{8.16} the other hand, if I were to turn back after having come so far without having seen the ascetic Gotama, the assembly might disparage me for that: ‘\textsanskrit{Soṇadaṇḍa} is foolish and incompetent. He’s stuck-up and scared. He doesn’t dare to go and see the ascetic Gotama. For how on earth can he turn back after having come so far without having seen the ascetic Gotama!’ And when you’re disparaged by the assembly, your reputation diminishes. When your reputation diminishes, your wealth also diminishes. But my wealth relies on my reputation.” 

Then\marginnote{9.1} \textsanskrit{Soṇadaṇḍa} went up to the Buddha, and exchanged greetings with him. When the greetings and polite conversation were over, he sat down to one side. Before sitting down to one side, some of the brahmins and householders of \textsanskrit{Campā} bowed, some exchanged greetings and polite conversation, some held up their joined palms toward the Buddha, some announced their name and clan, while some kept silent. 

But\marginnote{10.1} while sitting there, \textsanskrit{Soṇadaṇḍa} continued to be plagued by many second thoughts. He thought, “If only the ascetic Gotama would ask me about my own teacher’s scriptural heritage of the three Vedas! Then I could definitely satisfy his mind with my answer.” 

\section*{5. What Makes a Brahmin }

Then\marginnote{11.1} the Buddha, knowing what \textsanskrit{Soṇadaṇḍa} was thinking, thought, “This brahmin \textsanskrit{Soṇadaṇḍa} is worried by his own thoughts. Why don’t I ask him about his own teacher’s scriptural heritage of the three Vedas?” 

So\marginnote{11.4} he said to \textsanskrit{Soṇadaṇḍa}, “Brahmin, how many factors must a brahmin possess for the brahmins to describe him as a brahmin; and so that when he says ‘I am a brahmin’ he speaks rightly, without falling into falsehood?” 

Then\marginnote{12.1} \textsanskrit{Soṇadaṇḍa} thought, “The ascetic Gotama has asked me about exactly what I wanted, what I wished for, what I desired, what I yearned for; that is, my own scriptural heritage. I can definitely satisfy his mind with my answer.” 

Then\marginnote{13.1} \textsanskrit{Soṇadaṇḍa} straightened his back, looked around the assembly, and said to the Buddha, “Master Gotama, a brahmin must possess five factors for the brahmins to describe him as a brahmin; and so that when he says ‘I am a brahmin’ he speaks rightly, without falling into falsehood. What five? It’s when a brahmin is well born on both his mother’s and father’s side, of pure descent, irrefutable and impeccable in questions of ancestry back to the seventh paternal generation. He recites and remembers the hymns, and has mastered the three Vedas, together with their vocabularies, ritual, phonology and etymology, and the testament as fifth. He knows philology and grammar, and is well versed in cosmology and the marks of a great man. He is attractive, good-looking, lovely, of surpassing beauty. He is magnificent, splendid, remarkable to behold. He is ethical, mature in ethical conduct. He’s astute and clever, being the first or second to hold the sacrificial ladle. These are the five factors which a brahmin must possess for the brahmins to describe him as a brahmin; and so that when he says ‘I am a brahmin’ he speaks rightly, without falling into falsehood.” 

“But\marginnote{14.1} brahmin, is it possible to set aside one of these five factors and still rightly describe someone as a brahmin?” 

“It\marginnote{14.3} is possible, Master Gotama. We could leave appearance out of the five factors. For what does appearance matter? A brahmin must possess the remaining four factors for the brahmins to rightly describe him as a brahmin.” 

“But\marginnote{15.1} brahmin, is it possible to set aside one of these four factors and still rightly describe someone as a brahmin?” 

“It\marginnote{15.3} is possible, Master Gotama. We could leave the hymns out of the five factors. For what do the hymns matter? A brahmin must possess the remaining three factors for the brahmins to rightly describe him as a brahmin.” 

“But\marginnote{16.1} brahmin, is it possible to set aside one of these three factors and still rightly describe someone as a brahmin?” 

“It\marginnote{16.3} is possible, Master Gotama. We could leave birth out of the five factors. For what does birth matter? It’s when a brahmin is ethical, mature in ethical conduct; and he’s astute and clever, being the first or second to hold the sacrificial ladle. A brahmin must possess these two factors for the brahmins to rightly describe him as a brahmin.” 

When\marginnote{17.1} he had spoken, those brahmins said to him, “Please don’t say that, Master \textsanskrit{Soṇadaṇḍa}, please don’t say that! You’re just condemning appearance, the hymns, and birth! You’re totally going over to the ascetic Gotama’s doctrine!” 

So\marginnote{18.1} the Buddha said to them, “Well, brahmins, if you think that \textsanskrit{Soṇadaṇḍa} is unlearned, a poor speaker, witless, and not capable of having a dialogue with me about this, then leave him aside and you can have a dialogue with me. But if you think that he’s learned, a good speaker, astute, and capable of having a dialogue with me about this, then you should stand aside and let him have a dialogue with me.” 

When\marginnote{19.1} he said this, \textsanskrit{Soṇadaṇḍa} said to the Buddha, “Let it be, Master Gotama, be silent. I myself will respond to them in a legitimate manner.” Then he said to those brahmins, “Don’t say this, gentlemen, don’t say this: ‘You’re just condemning appearance, the hymns, and birth! You’re totally going over to the ascetic Gotama’s doctrine!’ I’m not condemning appearance, hymns, or birth.” 

Now\marginnote{20.1} at that time \textsanskrit{Soṇadaṇḍa}’s nephew, the student \textsanskrit{Aṅgaka} was sitting in that assembly. Then \textsanskrit{Soṇadaṇḍa} said to those brahmins, “Gentlemen, do you see my nephew, the student \textsanskrit{Aṅgaka}?” 

“Yes,\marginnote{20.4} sir.” 

“\textsanskrit{Aṅgaka}\marginnote{20.5} is attractive, good-looking, lovely, of surpassing beauty. He is magnificent, splendid, remarkable to behold. There’s no-one in this assembly so good-looking, apart from the ascetic Gotama. \textsanskrit{Aṅgaka} recites and remembers the hymns, and has mastered the three Vedas, together with their vocabularies, ritual, phonology and etymology, and the testament as fifth. He knows philology and grammar, and is well versed in cosmology and the marks of a great man. And I am the one who teaches him the hymns. \textsanskrit{Aṅgaka} is well born on both his mother’s and father’s side, of pure descent, irrefutable and impeccable in questions of ancestry back to the seventh paternal generation. And I know his mother and father. But if \textsanskrit{Aṅgaka} were to kill living creatures, steal, commit adultery, lie, and drink alcohol, then what’s the use of his appearance, his hymns, or his birth? It’s when a brahmin is ethical, mature in ethical conduct; and he’s astute and clever, being the first or second to hold the sacrificial ladle. A brahmin must possess these two factors for the brahmins to rightly describe him as a brahmin.” 

\section*{6. The Discussion of Ethics and Wisdom }

“But\marginnote{21.1} brahmin, is it possible to set aside one of these two factors and still rightly describe someone as a brahmin?” 

“No,\marginnote{21.3} Master Gotama. For wisdom is cleansed by ethics, and ethics are cleansed by wisdom. Ethics and wisdom always go together. An ethical person is wise, and a wise person ethical. And ethics and wisdom are said to be the best things in the world. It’s just like when you clean one hand with the other, or clean one foot with the other. In the same way, wisdom is cleansed by ethics, and ethics are cleansed by wisdom. Ethics and wisdom always go together. An ethical person is wise, and a wise person ethical. And ethics and wisdom are said to be the best things in the world.” 

“That’s\marginnote{22.1} so true, brahmin, that’s so true! For wisdom is cleansed by ethics, and ethics are cleansed by wisdom. Ethics and wisdom always go together. An ethical person is wise, and a wise person ethical. And ethics and wisdom are said to be the best things in the world. It’s just like when you clean one hand with the other, or clean one foot with the other. In the same way, wisdom is cleansed by ethics, and ethics are cleansed by wisdom. Ethics and wisdom always go together. An ethical person is wise, and a wise person ethical. And ethics and wisdom are said to be the best things in the world. 

But\marginnote{22.10} what, brahmin, is that ethical conduct? And what is that wisdom?” 

“That’s\marginnote{22.12} all I know about this matter, Master Gotama. May Master Gotama himself please clarify the meaning of this.” 

“Well\marginnote{23.1} then, brahmin, listen and pay close attention, I will speak.” 

“Yes\marginnote{23.2} sir,” \textsanskrit{Soṇadaṇḍa} replied. The Buddha said this: 

“It’s\marginnote{23.4} when a Realized One arises in the world, perfected, a fully awakened Buddha … That’s how a mendicant is accomplished in ethics. This, brahmin, is that ethical conduct. … They enter and remain in the first absorption … second absorption … third absorption … fourth absorption … They extend and project the mind toward knowledge and vision … This pertains to their wisdom. … They understand: ‘… there is no return to any state of existence.’ This pertains to their wisdom. This, brahmin, is that wisdom.” 

\section*{7. \textsanskrit{Soṇadaṇḍa} Declares Himself a Lay Follower }

When\marginnote{24.1} he had spoken, \textsanskrit{Soṇadaṇḍa} said to the Buddha, “Excellent, Master Gotama! Excellent! As if he were righting the overturned, or revealing the hidden, or pointing out the path to the lost, or lighting a lamp in the dark so people with good eyes can see what’s there, Master Gotama has made the Teaching clear in many ways. I go for refuge to Master Gotama, to the teaching, and to the mendicant \textsanskrit{Saṅgha}. From this day forth, may Master Gotama remember me as a lay follower who has gone for refuge for life. Would you and the Order of monks please accept a meal from me tomorrow?” The Buddha consented in silence. 

Then,\marginnote{24.8} knowing that the Buddha had consented, \textsanskrit{Soṇadaṇḍa} got up from his seat, bowed, and respectfully circled the Buddha, keeping him on his right, before leaving. And when the night had passed \textsanskrit{Soṇadaṇḍa} had a variety of delicious foods prepared in his own home. Then he had the Buddha informed of the time, saying, “It’s time, Master Gotama, the meal is ready.” Then the Buddha robed up in the morning and, taking his bowl and robe, went to the home of \textsanskrit{Soṇadaṇḍa} together with the mendicant \textsanskrit{Saṅgha}, where he sat on the seat spread out. Then \textsanskrit{Soṇadaṇḍa} served and satisfied the mendicant \textsanskrit{Saṅgha} headed by the Buddha with his own hands with a variety of delicious foods. 

When\marginnote{26.1} the Buddha had eaten and washed his hand and bowl, \textsanskrit{Soṇadaṇḍa} took a low seat and sat to one side. Seated to one side he said to the Buddha: “Master Gotama, if, when I have gone to an assembly, I rise from my seat and bow to the Buddha, that assembly might disparage me for that. And when you’re disparaged by the assembly, your reputation diminishes. When your reputation diminishes, your wealth also diminishes. But my wealth relies on my reputation. If, when I have gone to an assembly, I raise my joined palms, please take it that I have risen from my seat. And if I undo my turban, please take it that I have bowed. And Master Gotama, if, when I am in a carriage, I rise from my seat and bow to the Buddha, that assembly might disparage me for that. If, when I am in a carriage, I hold up my goad, please take it that I have got down from my carriage. And if I lower my sunshade, please take it that I have bowed.” 

Then\marginnote{27.1} the Buddha educated, encouraged, fired up, and inspired the brahmin \textsanskrit{Soṇadaṇḍa} with a Dhamma talk, after which he got up from his seat and left. 

%
\chapter*{{\suttatitleacronym DN 5}{\suttatitletranslation With Kūṭadanta }{\suttatitleroot Kūṭadantasutta}}
\addcontentsline{toc}{chapter}{\tocacronym{DN 5} \toctranslation{With Kūṭadanta } \tocroot{Kūṭadantasutta}}
\markboth{With Kūṭadanta }{Kūṭadantasutta}
\extramarks{DN 5}{DN 5}

\section*{1. The Brahmins and Householders of \textsanskrit{Khāṇumata} }

\scevam{So\marginnote{1.1} I have heard. }At one time the Buddha was wandering in the land of the Magadhans together with a large \textsanskrit{Saṅgha} of around five hundred mendicants when he arrived at a village of the Magadhan brahmins named \textsanskrit{Khāṇumata}. There he stayed nearby at \textsanskrit{Ambalaṭṭhikā}. 

Now\marginnote{1.4} at that time the brahmin \textsanskrit{Kūṭadanta} was living in \textsanskrit{Khāṇumata}. It was a crown property given by King Seniya \textsanskrit{Bimbisāra} of Magadha, teeming with living creatures, full of hay, wood, water, and grain, a royal endowment of the highest quality. Now at that time \textsanskrit{Kūṭadanta} had prepared a great sacrifice. Bulls, bullocks, heifers, goats and rams—seven hundred of each—had been led to the post for the sacrifice. 

The\marginnote{2.1} brahmins and householders of \textsanskrit{Khāṇumataka} heard: 

“It\marginnote{2.2} seems the ascetic Gotama—a Sakyan, gone forth from a Sakyan family—has arrived at \textsanskrit{Khāṇumataka} and is staying in a forest nearby. He has this good reputation: ‘That Blessed One is perfected, a fully awakened Buddha, accomplished in knowledge and conduct, holy, knower of the world, supreme guide for those who wish to train, teacher of gods and humans, awakened, blessed.’ He has realized with his own insight this world—with its gods, \textsanskrit{Māras} and \textsanskrit{Brahmās}, this population with its ascetics and brahmins, gods and humans—and he makes it known to others. He teaches Dhamma that’s good in the beginning, good in the middle, and good in the end, meaningful and well-phrased. And he reveals a spiritual practice that’s entirely full and pure. It’s good to see such perfected ones.” 

Then,\marginnote{2.8} having departed \textsanskrit{Khāṇumataka}, they formed into companies and headed to \textsanskrit{Ambalaṭṭhikā}. 

Now\marginnote{3.1} at that time the brahmin \textsanskrit{Kūṭadanta} had retired to the upper floor of his stilt longhouse for his midday nap. He saw the brahmins and householders heading for \textsanskrit{Ambalaṭṭhikā}, and addressed his steward, “My steward, why are the brahmins and householders headed for \textsanskrit{Ambalaṭṭhikā}?” 

“The\marginnote{3.5} ascetic Gotama has arrived at \textsanskrit{Khāṇumataka} and is staying at \textsanskrit{Ambalaṭṭhikā}. He has this good reputation: ‘That Blessed One is perfected, a fully awakened Buddha, accomplished in knowledge and conduct, holy, knower of the world, supreme guide for those who wish to train, teacher of gods and humans, awakened, blessed.’ They’re going to see that Master Gotama.” 

Then\marginnote{4.1} \textsanskrit{Kūṭadanta} thought, “I’ve heard that the ascetic Gotama knows how to accomplish the sacrifice with three modes and sixteen accessories. I don’t know about that, but I wish to perform a great sacrifice. Why don’t I ask him how to accomplish the sacrifice with three modes and sixteen accessories?” 

Then\marginnote{4.7} \textsanskrit{Kūṭadanta} addressed his steward, “Well then, go to the brahmins and householders and say to them: ‘Sirs, the brahmin \textsanskrit{Kūṭadanta} asks you to wait, as he will also go to see the ascetic Gotama.’” 

“Yes,\marginnote{4.11} sir,” replied the steward, and did as he was asked. 

\section*{2. The Qualities of \textsanskrit{Kūṭadanta} }

Now\marginnote{5.1} at that time several hundred brahmins were residing in \textsanskrit{Khāṇumata} thinking to participate in \textsanskrit{Kūṭadanta}’s sacrifice. They heard that \textsanskrit{Kūṭadanta} was going to see the ascetic Gotama. They approached \textsanskrit{Kūṭadanta} and said to him: 

“Is\marginnote{5.6} it really true that you are going to see the ascetic Gotama?” 

“Yes,\marginnote{5.7} gentlemen, it is true.” 

“Please\marginnote{6.1} don’t! It’s not appropriate for you to go to see the ascetic Gotama. For if you do so, your reputation will diminish and his will increase. For this reason it’s not appropriate for you to go to see the ascetic Gotama; it’s appropriate that he comes to see you. 

You\marginnote{6.6} are well born on both your mother’s and father’s side, of pure descent, irrefutable and impeccable in questions of ancestry back to the seventh paternal generation. For this reason it’s not appropriate for you to go to see the ascetic Gotama; it’s appropriate that he comes to see you. 

You’re\marginnote{6.9} rich, affluent, and wealthy, with lots of property and assets, and lots of money and grain … 

You\marginnote{6.10} recite and remember the hymns, and have mastered the three Vedas, together with their vocabularies, ritual, phonology and etymology, and the testament as fifth. You know philology and grammar, and are well versed in cosmology and the marks of a great man. … 

You\marginnote{6.11} are attractive, good-looking, lovely, of surpassing beauty. You are magnificent, splendid, remarkable to behold. … 

You\marginnote{6.12} are ethical, mature in ethical conduct. … 

You’re\marginnote{6.13} a good speaker, with a polished, clear, and articulate voice that expresses the meaning. … 

You\marginnote{6.14} teach the teachers of many, and teach three hundred students to recite the hymns. Many students come from various districts and countries for the sake of the hymns, wishing to learn the hymns. … 

You’re\marginnote{6.15} old, elderly and senior, advanced in years, and have reached the final stage of life. The ascetic Gotama is young, and has newly gone forth. … 

You’re\marginnote{6.17} honored, respected, revered, venerated, and esteemed by King \textsanskrit{Bimbisāra} of Magadha … 

and\marginnote{6.18} the brahmin \textsanskrit{Pokkharasāti}. … 

You\marginnote{6.19} live in \textsanskrit{Khāṇumata}, a crown property given by King Seniya \textsanskrit{Bimbisāra} of Magadha, teeming with living creatures, full of hay, wood, water, and grain, a royal endowment of the highest quality. For this reason it’s not appropriate for you to go to see the ascetic Gotama; it’s appropriate that he comes to see you.” 

\section*{3. The Qualities of the Buddha }

When\marginnote{7.1} they had spoken, \textsanskrit{Kūṭadanta} said to those brahmins: 

“Well\marginnote{7.2} then, gentlemen, listen to why it’s appropriate for me to go to see the ascetic Gotama, and it’s not appropriate for him to come to see me. He is well born on both his mother’s and father’s side, of pure descent, irrefutable and impeccable in questions of ancestry back to the seventh paternal generation. For this reason it’s not appropriate for the ascetic Gotama to come to see me; rather, it’s appropriate for me to go to see him. 

When\marginnote{7.7} he went forth he abandoned a large family circle. … 

When\marginnote{7.8} he went forth he abandoned abundant gold coin and bullion stored in dungeons and towers. … 

He\marginnote{7.9} went forth from the lay life to homelessness while still a youth, young, black-haired, blessed with youth, in the prime of life. … 

Though\marginnote{7.10} his mother and father wished otherwise, weeping with tearful faces, he shaved off his hair and beard, dressed in ocher robes, and went forth from the lay life to homelessness. … 

He\marginnote{7.11} is attractive, good-looking, lovely, of surpassing beauty. He is magnificent, splendid, remarkable to behold. … 

He\marginnote{7.12} is ethical, possessing ethical conduct that is noble and skillful. … 

He’s\marginnote{7.13} a good speaker, with a polished, clear, and articulate voice that expresses the meaning. … 

He’s\marginnote{7.14} a teacher of teachers. … 

He\marginnote{7.15} has ended sensual desire, and is rid of caprice. … 

He\marginnote{7.16} teaches the efficacy of deeds and action. He doesn’t wish any harm upon the community of brahmins. … 

He\marginnote{7.17} went forth from an eminent family of unbroken aristocratic lineage. … 

He\marginnote{7.18} went forth from a rich, affluent, and wealthy family. … 

People\marginnote{7.19} come from distant lands and distant countries to question him. … 

Many\marginnote{7.20} thousands of deities have gone for refuge for life to him. … 

He\marginnote{7.21} has this good reputation: ‘That Blessed One is perfected, a fully awakened Buddha, accomplished in knowledge and conduct, holy, knower of the world, supreme guide for those who wish to train, teacher of gods and humans, awakened, blessed.’ … 

He\marginnote{7.23} has the thirty-two marks of a great man. … 

He\marginnote{7.24} is welcoming, congenial, polite, smiling, open, the first to speak. … 

He’s\marginnote{7.25} honored, respected, revered, venerated, and esteemed by the four assemblies. … 

Many\marginnote{7.26} gods and humans are devoted to him. … 

While\marginnote{7.27} he is residing in a village or town, non-human entities do not harass them. … 

He\marginnote{7.28} leads an order and a community, and teaches a community, and is said to be the best of the various religious founders. He didn’t come by his fame in the same ways as those other ascetics and brahmins. Rather, he came by his fame due to his supreme knowledge and conduct. … 

King\marginnote{7.30} Seniya \textsanskrit{Bimbisāra} of Magadha and his wives and children have gone for refuge for life to the ascetic Gotama. … 

King\marginnote{7.31} Pasenadi of Kosala and his wives and children have gone for refuge for life to the ascetic Gotama. … 

The\marginnote{7.32} brahmin \textsanskrit{Pokkharasāti} and his wives and children have gone for refuge for life to the ascetic Gotama. … 

He’s\marginnote{7.33} honored, respected, revered, venerated, and esteemed by King \textsanskrit{Bimbisāra} of Magadha … 

King\marginnote{7.34} Pasenadi of Kosala … 

and\marginnote{7.35} the brahmin \textsanskrit{Pokkharasāti}. 

The\marginnote{7.36} ascetic Gotama has arrived at \textsanskrit{Khāṇumata} and is staying at \textsanskrit{Ambalaṭṭhikā}. Any ascetic or brahmin who comes to stay in our village district is our guest, and should be honored and respected as such. For this reason, too, it’s not appropriate for Master Gotama to come to see me, rather, it’s appropriate for me to go to see him. This is the extent of Master Gotama’s praise that I have learned. But his praises are not confined to this, for the praise of Master Gotama is limitless.” 

When\marginnote{7.45} he had spoken, those brahmins said to him, “According to \textsanskrit{Kūṭadanta}’s praises, if Master Gotama were staying within a hundred leagues, it’d be worthwhile for a faithful gentleman to go to see him, even if he had to carry his own provisions in a shoulder bag.” 

“Well\marginnote{7.47} then, gentlemen, let’s all go to see the ascetic Gotama.” 

\section*{4. The Story of King \textsanskrit{Mahāvijita}’s Sacrifice }

Then\marginnote{8.1} \textsanskrit{Kūṭadanta} together with a large group of brahmins went to see the Buddha and exchanged greetings with him. When the greetings and polite conversation were over, he sat down to one side. Before sitting down to one side, some of the brahmins and householders of \textsanskrit{Khāṇumataka} bowed, some exchanged greetings and polite conversation, some held up their joined palms toward the Buddha, some announced their name and clan, while some kept silent. 

\textsanskrit{Kūṭadanta}\marginnote{9.1} said to the Buddha, “Master Gotama, I’ve heard that you know how to accomplish the sacrifice with three modes and sixteen accessories. I don’t know about that, but I wish to perform a great sacrifice. Please teach me how to accomplish the sacrifice with three modes and sixteen accessories.” 

“Well\marginnote{9.7} then, brahmin, listen and pay close attention, I will speak.” 

“Yes\marginnote{9.8} sir,” \textsanskrit{Kūṭadanta} replied. The Buddha said this: Once upon a time, brahmin, there was a king named \textsanskrit{Mahāvijita}. He was rich, affluent, and wealthy, with lots of gold and silver, lots of property and assets, lots of money and grain, and a full treasury and storehouses. Then as King \textsanskrit{Mahāvijita} was in private retreat this thought came to his mind: ‘I have achieved human wealth, and reign after conquering this vast territory. Why don’t I hold a large sacrifice? That will be for my lasting welfare and happiness.’ 

Then\marginnote{10.4} he summoned the brahmin high priest and said to him: ‘Just now, brahmin, as I was in private retreat this thought came to mind, “I have achieved human wealth, and reign after conquering this vast territory. Why don’t I perform a great sacrifice? That will be for my lasting welfare and happiness.” Brahmin, I wish to perform a great sacrifice. Please instruct me. It will be for my lasting welfare and happiness.’ 

When\marginnote{11.1} he had spoken, the brahmin high priest said to him: ‘Sir, the king’s realm is harried and oppressed. Bandits have been seen raiding villages, towns, and cities, and infesting the highways. But if the king were to extract more taxes while his realm is thus harried and oppressed, he would not be doing his duty. 

Now\marginnote{11.4} the king might think, “I’ll eradicate this outlaw threat by execution or imprisonment or confiscation or condemnation or banishment!” But that’s not the right way to eradicate this barbarian obstacle. Those who remain after the killing will return to harass the king’s realm. 

Rather,\marginnote{11.7} here is a plan, relying on which the outlaw threat will be properly uprooted. So let the king provide seed and fodder for those in the realm who work in farming and raising cattle. Let the king provide funding for those who work in trade. Let the king guarantee food and wages for those in government service. Then the people, occupied with their own work, will not harass the realm. The king’s revenues will be great. When the country is secured as a sanctuary, free of being harried and oppressed, the happy people, with joy in their hearts, dancing with children at their breast, will dwell as if their houses were wide open.’ 

The\marginnote{11.14} king agreed with the high priest’s advice and followed his recommendation. 

Then\marginnote{11.19} the king summoned the brahmin high priest and said to him: ‘I have eradicated the outlaw threat. And relying on your plan my revenue is now great. Since the country is secured as a sanctuary, free of being harried and oppressed, the happy people, with joy in their hearts, dancing with children at their breast, dwell as if their houses were wide open. Brahmin, I wish to perform a great sacrifice. Please instruct me. It will be for my lasting welfare and happiness.’ 

\subsection*{4.1. The Four Accessories }

‘In\marginnote{12.1} that case, let the king announce this throughout the realm to the aristocrat vassals, ministers and counselors, well-to-do brahmins, and well-off householders, both of town and country: “I wish to perform a great sacrifice. Please grant your approval, gentlemen; it will be for my lasting welfare and happiness.” 

The\marginnote{12.3} king agreed with the high priest’s advice and followed his recommendation. And all of the people who were thus informed responded by saying: ‘May the king perform a sacrifice! It is time for a sacrifice, great king.’ And so these four consenting factions became accessories to the sacrifice. 

\subsection*{4.2. The Eight Accessories }

King\marginnote{13.1} \textsanskrit{Mahāvijita} possessed eight factors. 

He\marginnote{13.2} was well born on both his mother’s and father’s side, of pure descent, irrefutable and impeccable in questions of ancestry back to the seventh paternal generation. 

He\marginnote{13.3} was attractive, good-looking, lovely, of surpassing beauty. He was magnificent, splendid, remarkable to behold. 

He\marginnote{13.4} was rich, affluent, and wealthy, with lots of gold and silver, lots of property and assets, lots of money and grain, and a full treasury and storehouses. 

He\marginnote{13.5} was powerful, having an army of four divisions that was obedient and carried out instructions. He’d probably prevail over his enemies just with his reputation. 

He\marginnote{13.6} was faithful, generous, a donor, his door always open. He was a well-spring of support, making merit with ascetics and brahmins, for paupers, vagrants, travelers, and beggars. 

He\marginnote{13.7} was very learned in diverse fields of learning. He understood the meaning of diverse statements, saying: ‘This is what that statement means; that is what this statement means.’ 

He\marginnote{13.9} was astute, competent, and intelligent, able to think issues through as they bear upon the past, future, and present. 

These\marginnote{13.10} are the eight factors that King \textsanskrit{Mahāvijita} possessed. And so these eight factors also became accessories to the sacrifice. 

\subsection*{4.3. Four More Accessories }

And\marginnote{14.1} the brahmin high priest had four factors. 

He\marginnote{14.2} was well born on both his mother’s and father’s side, of pure descent, irrefutable and impeccable in questions of ancestry back to the seventh paternal generation. 

He\marginnote{14.3} recited and remembered the hymns, and had mastered the three Vedas, together with their vocabularies, ritual, phonology and etymology, and the testament as fifth. He knew philology and grammar, and was well versed in cosmology and the marks of a great man. 

He\marginnote{14.4} was ethical, mature in ethical conduct. 

He\marginnote{14.5} was astute and clever, being the first or second to hold the sacrificial ladle. 

These\marginnote{14.6} are the four factors that the brahmin high priest possessed. And so these four factors also became accessories to the sacrifice. 

\subsection*{4.4. The Three Modes }

Next,\marginnote{15.1} before the sacrifice, the brahmin high priest taught the three modes to the king. ‘Now, though the king wants to perform a great sacrifice, he might have certain regrets, thinking: “I shall lose a great fortune,” or “I am losing a great fortune,” or “I have lost a great fortune.” But the king should not harbor such regrets.’ 

These\marginnote{15.8} are the three modes that the brahmin high priest taught to the king before the sacrifice. 

\subsection*{4.5. The Ten Respects }

Next,\marginnote{16.1} before the sacrifice, the brahmin high priest dispelled the king’s regret regarding the recipients in ten respects: 

‘There\marginnote{16.2} will come to the sacrifice those who kill living creatures and those who refrain from killing living creatures. As to those who kill living creatures, the outcome of that is theirs alone. But as to those who refrain from killing living creatures, it is for their sakes that the king should sacrifice, relinquish, rejoice, and gain confidence in his heart. 

There\marginnote{16.5} will come to the sacrifice those who steal … commit sexual misconduct … lie … use divisive speech … use harsh speech … talk nonsense … are covetous … have ill will … have wrong view and those who have right view. As to those who have wrong view, the outcome of that is theirs alone. But as to those who have right view, it is for their sakes that the king should sacrifice, relinquish, rejoice, and gain confidence in his heart.’ 

These\marginnote{16.16} are the ten respects in which the high priest dispelled the king’s regret regarding the recipients before the sacrifice. 

\subsection*{4.6. The Sixteen Respects }

Next,\marginnote{17.1} while the king was performing the great sacrifice, the brahmin high priest educated, encouraged, fired up, and inspired the king’s mind in sixteen respects: 

‘Now,\marginnote{17.2} while the king is performing the great sacrifice, someone might say, “King \textsanskrit{Mahāvijita} performs a great sacrifice, but he did not announce it to the aristocrat vassals of town and country. That’s the kind of great sacrifice that this king performs.” Those who speak against the king in this way have no legitimacy. For the king did indeed announce it to the aristocrat vassals of town and country. Let the king know this as a reason to sacrifice, relinquish, rejoice, and gain confidence in his heart. 

While\marginnote{17.8} the king is performing the great sacrifice, someone might say, “King \textsanskrit{Mahāvijita} performs a great sacrifice, but he did not announce it to the ministers and counselors, well-to-do brahmins, and well-off householders, both of town and country. That’s the kind of great sacrifice that this king performs.” Those who speak against the king in this way have no legitimacy. For the king did indeed announce it to all these people. Let the king know this too as a reason to sacrifice, relinquish, rejoice, and gain confidence in his heart. 

While\marginnote{17.13} the king is performing the great sacrifice, someone might say that he does not possess the eight factors. Those who speak against the king in this way have no legitimacy. For the king does indeed possess the eight factors. Let the king know this too as a reason to sacrifice, relinquish, rejoice, and gain confidence in his heart. 

While\marginnote{17.30} the king is performing the great sacrifice, someone might say that the high priest does not possess the four factors. Those who speak against the king in this way have no legitimacy. For the high priest does indeed possess the four factors. Let the king know this too as a reason to sacrifice, relinquish, rejoice, and gain confidence in his heart.’ 

These\marginnote{17.45} are the sixteen respects in which the high priest educated, encouraged, fired up, and inspired the king’s mind while he was performing the sacrifice. 

And\marginnote{18.1} brahmin, in that sacrifice no cattle were killed, no goats were killed, and no chickens or pigs were killed. There was no slaughter of various kinds of creatures. No trees were felled for the sacrificial post. No grass was reaped to strew over the place of sacrifice. No bondservants, employees, or workers did their jobs under threat of punishment and danger, weeping with tearful faces. Those who wished to work did so, while those who did not wish to did not. They did the work they wanted to, and did not do what they didn’t want to. The sacrifice was completed with just ghee, oil, butter, curds, honey, and molasses. 

Then\marginnote{19.1} the aristocrat vassals, ministers and counselors, well-to-do brahmins, and well-off householders of both town and country came to the king bringing abundant wealth and said, ‘Sire, this abundant wealth is specially for you alone; may Your Highness accept it!’ 

‘There’s\marginnote{19.3} enough raised for me through regular taxes. Let this be for you; and here, take even more!’ 

When\marginnote{19.5} the king turned them down, they withdrew to one side to think up a plan, ‘It wouldn’t be proper for us to take this abundant wealth back to our own homes. King \textsanskrit{Mahāvijita} is performing a great sacrifice. Let us make an offering as an auxiliary sacrifice.’ 

Then\marginnote{20.1} the aristocrat vassals of town and country set up gifts to the east of the sacrificial pit. The ministers and counselors of town and country set up gifts to the south of the sacrificial pit. The well-to-do brahmins of town and country set up gifts to the west of the sacrificial pit. The well-off householders of town and country set up gifts to the north of the sacrificial pit. 

And\marginnote{20.5} brahmin, in that sacrifice too no cattle were killed, no goats were killed, and no chickens or pigs were killed. There was no slaughter of various kinds of creatures. No trees were felled for the sacrificial post. No grass was reaped to strew over the place of sacrifice. No bondservants, employees, or workers did their jobs under threat of punishment and danger, weeping with tearful faces. Those who wished to work did so, while those who did not wish to did not. They did the work they wanted to, and did not do what they didn’t want to. The sacrifice was completed with just ghee, oil, butter, curds, honey, and molasses. 

And\marginnote{20.10} so there were four consenting factions, eight factors possessed by King \textsanskrit{Mahāvijita}, four factors possessed by the high priest, and three modes. Brahmin, this is called the sacrifice accomplished with three modes and sixteen accessories.” 

When\marginnote{21.1} he said this, those brahmins made an uproar, “Hooray for such sacrifice! Hooray for the accomplishment of such sacrifice!” 

But\marginnote{21.3} the brahmin \textsanskrit{Kūṭadanta} sat in silence. So those brahmins said to him, “How can you not applaud the ascetic Gotama’s fine words?” 

“It’s\marginnote{21.6} not that I don’t applaud what he said. If anyone didn’t applaud such fine words, their head would explode! 

But,\marginnote{21.8} gentlemen, it occurs to me that the ascetic Gotama does not say: ‘So I have heard’ or ‘It ought to be like this.’ Rather, he just says: ‘So it was then, this is how it was then.’ 

It\marginnote{21.13} occurs to me that the ascetic Gotama at that time must have been King \textsanskrit{Mahāvijita}, the owner of the sacrifice, or else the brahmin high priest who facilitated the sacrifice for him. 

Does\marginnote{21.15} Master Gotama recall having performed such a sacrifice, or having facilitated it, and then, when his body broke up, after death, being reborn in a good place, a heavenly realm?” 

“I\marginnote{21.16} do recall that, brahmin. For at that time I was the brahmin high priest who facilitated the sacrifice.” 

\section*{5. A Regular Gift as an Ongoing Family Sacrifice. }

“But\marginnote{22.1} Master Gotama, apart from that sacrifice accomplished with three modes and sixteen accessories, is there any other sacrifice that has fewer requirements and undertakings, yet is more fruitful and beneficial?” 

“There\marginnote{22.2} is, brahmin.” 

“But\marginnote{22.3} what is it?” 

“The\marginnote{22.4} regular gifts as ongoing family sacrifice given specially to ethical renunciates; this sacrifice, brahmin, has fewer requirements and undertakings, yet is more fruitful and beneficial.” 

“What\marginnote{23.1} is the cause, Master Gotama, what is the reason why those regular gifts as ongoing family sacrifice have fewer requirements and undertakings, yet are more fruitful and beneficial, compared with the sacrifice accomplished with three modes and sixteen accessories?” 

“Because\marginnote{23.2} neither perfected ones nor those who are on the path to perfection will attend such a sacrifice. Why is that? Because beatings and throttlings are seen there. 

But\marginnote{23.5} the regular gifts as ongoing family sacrifice given specially to ethical renunciates; perfected ones and those who are on the path to perfection will attend such a sacrifice. Why is that? Because no beatings and throttlings are seen there. 

This\marginnote{23.9} is the cause, brahmin, this is the reason why those regular gifts as ongoing family sacrifice have fewer requirements and undertakings, yet are more fruitful and beneficial, compared with the sacrifice accomplished with three modes and sixteen accessories.” 

“But\marginnote{24.1} Master Gotama, apart from that sacrifice accomplished with three modes and sixteen accessories and those regular gifts as ongoing family sacrifice, is there any other sacrifice that has fewer requirements and undertakings, yet is more fruitful and beneficial?” 

“There\marginnote{24.2} is, brahmin.” 

“But\marginnote{24.3} what is it?” 

“When\marginnote{24.4} someone gives a dwelling specially for the \textsanskrit{Saṅgha} of the four quarters.” 

“But\marginnote{25.1} is there any other sacrifice that has fewer requirements and undertakings, yet is more fruitful and beneficial?” 

“When\marginnote{25.4} someone with confident heart goes for refuge to the Buddha, the teaching, and the \textsanskrit{Saṅgha}.” 

“But\marginnote{26.1} is there any other sacrifice that has fewer requirements and undertakings, yet is more fruitful and beneficial?” 

“When\marginnote{26.4} someone with a confident heart undertakes the training rules to refrain from killing living creatures, stealing, sexual misconduct, lying, and alcoholic drinks that cause negligence.” 

“But\marginnote{27.1} is there any other sacrifice that has fewer requirements and undertakings, yet is more fruitful and beneficial?” 

“There\marginnote{27.2} is, brahmin. 

It’s\marginnote{27.4} when a Realized One arises in the world, perfected, a fully awakened Buddha … That’s how a mendicant is accomplished in ethics. … They enter and remain in the first absorption … This sacrifice has fewer requirements and undertakings than the former, yet is more fruitful and beneficial. … 

They\marginnote{27.8} enter and remain in the second absorption … third absorption … fourth absorption. This sacrifice has fewer requirements and undertakings than the former, yet is more fruitful and beneficial. … 

They\marginnote{27.12} extend and project the mind toward knowledge and vision … This sacrifice has fewer requirements and undertakings than the former, yet is more fruitful and beneficial. 

They\marginnote{27.14} understand: ‘… there is no return to any state of existence.’ This sacrifice has fewer requirements and undertakings than the former, yet is more fruitful and beneficial. And, brahmin, there is no other accomplishment of sacrifice which is better and finer than this.” 

\section*{6. \textsanskrit{Kūṭadanta} Declares Himself a Lay Follower }

When\marginnote{28.1} he had spoken, \textsanskrit{Kūṭadanta} said to the Buddha, “Excellent, Master Gotama! Excellent! As if he were righting the overturned, or revealing the hidden, or pointing out the path to the lost, or lighting a lamp in the dark so people with good eyes can see what’s there, Master Gotama has made the Teaching clear in many ways. I go for refuge to Master Gotama, to the teaching, and to the mendicant \textsanskrit{Saṅgha}. From this day forth, may Master Gotama remember me as a lay follower who has gone for refuge for life. 

And\marginnote{28.6} these bulls, bullocks, heifers, goats, and rams—seven hundred of each—I release them, I grant them life! Let them eat green grass and drink cool water, and may a cool breeze blow upon them!” 

\section*{7. The Realization of the Fruit of Stream-Entry }

Then\marginnote{29.1} the Buddha taught \textsanskrit{Kūṭadanta} step by step, with a talk on giving, ethical conduct, and heaven. He explained the drawbacks of sensual pleasures, so sordid and corrupt, and the benefit of renunciation. And when he knew that \textsanskrit{Kūṭadanta}’s mind was ready, pliable, rid of hindrances, elated, and confident he explained the special teaching of the Buddhas: suffering, its origin, its cessation, and the path. Just as a clean cloth rid of stains would properly absorb dye, in that very seat the stainless, immaculate vision of the Dhamma arose in the brahmin \textsanskrit{Kūṭadanta}: “Everything that has a beginning has an end.” 

Then\marginnote{30.1} \textsanskrit{Kūṭadanta} saw, attained, understood, and fathomed the Dhamma. He went beyond doubt, got rid of indecision, and became self-assured and independent of others regarding the Teacher’s instructions. He said to the Buddha, “Would Master Gotama together with the mendicant \textsanskrit{Saṅgha} please accept tomorrow’s meal from me?” The Buddha consented in silence. 

Then,\marginnote{30.4} knowing that the Buddha had consented, \textsanskrit{Kūṭadanta} got up from his seat, bowed, and respectfully circled the Buddha, keeping him on his right, before leaving. And when the night had passed \textsanskrit{Kūṭadanta} had a variety of delicious foods prepared in his own home. Then he had the Buddha informed of the time, saying, “It’s time, Master Gotama, the meal is ready.” 

Then\marginnote{30.7} the Buddha robed up in the morning and, taking his bowl and robe, went to the home of \textsanskrit{Kūṭadanta} together with the mendicant \textsanskrit{Saṅgha}, where he sat on the seat spread out. 

Then\marginnote{30.8} \textsanskrit{Kūṭadanta} served and satisfied the mendicant \textsanskrit{Saṅgha} headed by the Buddha with his own hands with a variety of delicious foods. When the Buddha had eaten and washed his hand and bowl, \textsanskrit{Kūṭadanta} took a low seat and sat to one side. Then the Buddha educated, encouraged, fired up, and inspired him with a Dhamma talk, after which he got up from his seat and left. 

%
\chapter*{{\suttatitleacronym DN 6}{\suttatitletranslation With Mahāli }{\suttatitleroot Mahālisutta}}
\addcontentsline{toc}{chapter}{\tocacronym{DN 6} \toctranslation{With Mahāli } \tocroot{Mahālisutta}}
\markboth{With Mahāli }{Mahālisutta}
\extramarks{DN 6}{DN 6}

\section*{1. On the Brahmin Emissaries }

\scevam{So\marginnote{1.1} I have heard. }At one time the Buddha was staying near \textsanskrit{Vesālī}, at the Great Wood, in the hall with the peaked roof. Now at that time several brahmin emissaries from Kosala and Magadha were residing in \textsanskrit{Vesālī} on some business. They heard: 

“It\marginnote{1.5} seems the ascetic Gotama—a Sakyan, gone forth from a Sakyan family—is staying near \textsanskrit{Vesālī}, at the Great Wood, in the hall with the peaked roof. He has this good reputation: ‘That Blessed One is perfected, a fully awakened Buddha, accomplished in knowledge and conduct, holy, knower of the world, supreme guide for those who wish to train, teacher of gods and humans, awakened, blessed.’ He has realized with his own insight this world—with its gods, \textsanskrit{Māras} and \textsanskrit{Brahmās}, this population with its ascetics and brahmins, gods and humans—and he makes it known to others. He teaches Dhamma that’s good in the beginning, good in the middle, and good in the end, meaningful and well-phrased. And he reveals a spiritual practice that’s entirely full and pure. It’s good to see such perfected ones.” 

Then\marginnote{2.1} they went to the hall with the peaked roof in the Great Wood to see the Buddha. 

Now,\marginnote{2.2} at that time Venerable \textsanskrit{Nāgita} was the Buddha’s attendant. The brahmin emissaries went up to him and said, “Master \textsanskrit{Nāgita}, where is Master Gotama at present? For we want to see him.” 

“It’s\marginnote{2.6} the wrong time to see the Buddha; he is on retreat.” 

So\marginnote{2.7} right there the brahmin emissaries sat down to one side, thinking, “We’ll go only after we’ve seen Master Gotama.” 

\section*{2. On \textsanskrit{Oṭṭhaddha} the Licchavi }

\textsanskrit{Oṭṭhaddha}\marginnote{3.1} the Licchavi together with a large assembly of Licchavis also approached \textsanskrit{Nāgita} at the hall with the peaked roof. He bowed, stood to one side, and said to \textsanskrit{Nāgita}, “Master \textsanskrit{Nāgita}, where is the Blessed One at present, the perfected one, the fully awakened Buddha? For we want to see him.” 

“It’s\marginnote{3.4} the wrong time to see the Buddha; he is on retreat.” 

So\marginnote{3.5} right there \textsanskrit{Oṭṭhaddha} also sat down to one side, thinking, “I’ll go only after I’ve seen the Blessed One, the perfected one, the fully awakened Buddha.” 

Then\marginnote{4.1} the novice \textsanskrit{Sīha} approached \textsanskrit{Nāgita}. He bowed, stood to one side, and said to \textsanskrit{Nāgita}, “Sir, Kassapa, these several brahmin emissaries from Kosala and Magadha, and also \textsanskrit{Oṭṭhaddha} the Licchavi together with a large assembly of Licchavis, have come here to see the Buddha. It’d be good if these people got to see the Buddha.” 

“Well\marginnote{4.3} then, \textsanskrit{Sīha}, tell the Buddha yourself.” 

“Yes,\marginnote{4.4} sir,” replied \textsanskrit{Sīha}. He went to the Buddha, bowed, stood to one side, and told him of the people waiting to see him, adding: “Sir, it’d be good if these people got to see the Buddha.” 

“Well\marginnote{4.7} then, \textsanskrit{Sīha}, spread out a seat in the shade of the dwelling.” 

“Yes,\marginnote{4.8} sir,” replied \textsanskrit{Sīha}, and he did so. 

Then\marginnote{4.9} the Buddha came out of his dwelling and sat in the shade of the dwelling on the seat spread out. Then the brahmin emissaries went up to the Buddha, and exchanged greetings with him. When the greetings and polite conversation were over, they sat down to one side. 

\textsanskrit{Oṭṭhaddha}\marginnote{5.3} the Licchavi together with a large assembly of Licchavis also went up to the Buddha, bowed, and sat down to one side. \textsanskrit{Oṭṭhaddha} said to the Buddha, “Sir, a few days ago Sunakkhatta the Licchavi came to me and said: ‘\textsanskrit{Mahāli}, soon I will have been living in dependence on the Buddha for three years. I see heavenly sights that are pleasant, sensual, and arousing, but I don’t hear heavenly sounds that are pleasant, sensual, and arousing.’ The heavenly sounds that Sunakkhatta cannot hear: do such sounds really exist or not?” 

\subsection*{2.1. One-Sided Immersion }

“Such\marginnote{5.7} sounds really do exist, but Sunakkhatta cannot hear them.” 

“What\marginnote{6.1} is the cause, sir, what is the reason why Sunakkhatta cannot hear them, even though they really do exist?” 

“\textsanskrit{Mahāli},\marginnote{6.2} take a mendicant who has developed one-sided immersion to the eastern quarter so as to see heavenly sights but not to hear heavenly sounds. When they have developed immersion for that purpose, they see heavenly sights but don’t hear heavenly sounds. Why is that? Because that is how it is for a mendicant who develops immersion in that way. 

Furthermore,\marginnote{7.1} take a mendicant who has developed one-sided immersion to the southern quarter … western quarter … northern quarter … above, below, across … That is how it is for a mendicant who develops immersion in that way. 

Take\marginnote{8.1} a mendicant who has developed one-sided immersion to the eastern quarter so as to hear heavenly sounds but not to see heavenly sights. When they have developed immersion for that purpose, they hear heavenly sounds but don’t see heavenly sights. Why is that? Because that is how it is for a mendicant who develops immersion in that way. 

Furthermore,\marginnote{9.1} take a mendicant who has developed one-sided immersion to the southern quarter … western quarter … northern quarter … above, below, across … That is how it is for a mendicant who develops immersion in that way. 

Take\marginnote{10.1} a mendicant who has developed two-sided immersion to the eastern quarter so as to both hear heavenly sounds and see heavenly sights. When they have developed immersion for that purpose, they both see heavenly sights and hear heavenly sounds. Why is that? Because that is how it is for a mendicant who develops immersion in that way. 

Furthermore,\marginnote{11.1} take a mendicant who has developed two-sided immersion to the southern quarter … western quarter … northern quarter … above, below, across … That is how it is for a mendicant who develops immersion in that way. This is the cause, \textsanskrit{Mahāli}, this is the reason why Sunakkhatta cannot hear heavenly sounds that are pleasant, sensual, and arousing, even though they really do exist.” 

“Surely\marginnote{12.1} the mendicants must lead the spiritual life under the Buddha for the sake of realizing such a development of immersion?” 

“No,\marginnote{12.2} \textsanskrit{Mahāli}, the mendicants don’t lead the spiritual life under me for the sake of realizing such a development of immersion. There are other things that are finer, for the sake of which the mendicants lead the spiritual life under me.” 

\subsection*{2.2. The Four Noble Fruits }

“But\marginnote{13.1} sir, what are those finer things?” 

“Firstly,\marginnote{13.2} \textsanskrit{Mahāli}, with the ending of three fetters a mendicant is a stream-enterer, not liable to be reborn in the underworld, bound for awakening. This is one of the finer things for the sake of which the mendicants lead the spiritual life under me. 

Furthermore,\marginnote{13.4} a mendicant—with the ending of three fetters, and the weakening of greed, hate, and delusion—is a once-returner. They come back to this world once only, then make an end of suffering. This too is one of the finer things. 

Furthermore,\marginnote{13.6} with the ending of the five lower fetters, a mendicant is reborn spontaneously and will become extinguished there, not liable to return from that world. This too is one of the finer things. 

Furthermore,\marginnote{13.8} a mendicant has realized the undefiled freedom of heart and freedom by wisdom in this very life, and lives having realized it with their own insight due to the ending of defilements. This too is one of the finer things. 

These\marginnote{13.10} are the finer things, for the sake of which the mendicants lead the spiritual life under me.” 

\subsection*{2.3. The Noble Eightfold Path }

“But,\marginnote{14.1} sir, is there a path and a practice for realizing these things?” 

“There\marginnote{14.2} is, \textsanskrit{Mahāli}.” 

“Well,\marginnote{14.3} what is it?” 

“It\marginnote{14.4} is simply this noble eightfold path, that is: right view, right thought, right speech, right action, right livelihood, right effort, right mindfulness, and right immersion. This is the path and the practice for realizing these things. 

\subsection*{2.4. On the Two Renunciates }

This\marginnote{15.1} one time, \textsanskrit{Mahāli}, I was staying near Kosambi, in Ghosita’s Monastery. Then two renunciates—the wanderer \textsanskrit{Muṇḍiya} and \textsanskrit{Jāliya} the pupil of \textsanskrit{Dārupattika}—came and exchanged greetings with me. When the greetings and polite conversation were over, they stood to one side and said to me: ‘Reverend Gotama, are the soul and the body the same thing, or they are different things?’ 

‘Well\marginnote{16.1} then, reverends, listen and pay close attention, I will speak.’ 

‘Yes,\marginnote{16.2} reverend,’ they replied. 

I\marginnote{16.3} said this: ‘Take the case when a Realized One arises in the world, perfected, a fully awakened Buddha … That’s how a mendicant is accomplished in ethics. … 

They\marginnote{16.6} enter and remain in the first absorption. When a mendicant knows and sees like this, would it be appropriate to say of them: “The soul and the body are the same thing” or “The soul and the body are different things”?’ 

‘It\marginnote{16.10} would, reverend.’ 

‘But\marginnote{16.12} reverends, I know and see like this. Nevertheless, I do not say: “The soul and the body are the same thing” or “The soul and the body are different things”. … 

They\marginnote{17.1} enter and remain in the second absorption … third absorption … fourth absorption. When a mendicant knows and sees like this, would it be appropriate to say of them: “The soul and the body are the same thing” or “The soul and the body are different things”?’ 

‘It\marginnote{17.6} would, reverend.’ 

‘But\marginnote{17.8} reverends, I know and see like this. Nevertheless, I do not say: “The soul and the body are the same thing” or “The soul and the body are different things”. … 

They\marginnote{18.1} extend and project the mind toward knowledge and vision … When a mendicant knows and sees like this, would it be appropriate to say of them: “The soul and the body are the same thing” or “The soul and the body are different things”?’ 

‘It\marginnote{18.5} would, reverend.’ 

‘But\marginnote{18.7} reverends, I know and see like this. Nevertheless, I do not say: “The soul and the body are the same thing” or “The soul and the body are different things”. … 

They\marginnote{19.1} understand: “… there is no return to any state of existence.” When a mendicant knows and sees like this, would it be appropriate to say of them: “The soul and the body are the same thing” or “The soul and the body are different things”?’ 

‘It\marginnote{19.4} would not, reverend.’ 

‘But\marginnote{19.6} reverends, I know and see like this. Nevertheless, I do not say: “The soul and the body are the same thing” or “The soul and the body are different things”.’” 

That\marginnote{19.9} is what the Buddha said. Satisfied, \textsanskrit{Oṭṭhaddha} the Licchavi was happy with what the Buddha said. 

%
\chapter*{{\suttatitleacronym DN 7}{\suttatitletranslation With Jāliya }{\suttatitleroot Jāliyasutta}}
\addcontentsline{toc}{chapter}{\tocacronym{DN 7} \toctranslation{With Jāliya } \tocroot{Jāliyasutta}}
\markboth{With Jāliya }{Jāliyasutta}
\extramarks{DN 7}{DN 7}

\scevam{So\marginnote{1.1} I have heard. }At one time the Buddha was staying near Kosambi, in Ghosita’s Monastery. 

Now\marginnote{1.3} at that time two renunciates—the wanderer \textsanskrit{Muṇḍiya} and \textsanskrit{Jāliya} the pupil of \textsanskrit{Dārupattika}—came to the Buddha and exchanged greetings with him. When the greetings and polite conversation were over, they stood to one side and said to the Buddha, “Reverend Gotama, are the soul and the body the same thing, or they are different things?” 

“Well\marginnote{1.7} then, reverends, listen and pay close attention, I will speak.” 

“Yes,\marginnote{1.8} reverend,” they replied. The Buddha said this: 

“Take\marginnote{2.1} the case when a Realized One arises in the world, perfected, a fully awakened Buddha … That’s how a mendicant is accomplished in ethics. … 

They\marginnote{2.3} enter and remain in the first absorption … When a mendicant knows and sees like this, would it be appropriate to say of them: ‘The soul and the body are the same thing’ or ‘The soul and the body are different things’?” 

“It\marginnote{2.7} would, reverend.” 

“But\marginnote{2.9} reverends, I know and see like this. Nevertheless, I do not say: ‘The soul and the body are the same thing’ or ‘The soul and the body are different things’. … 

They\marginnote{3.1} enter and remain in the second absorption … third absorption … fourth absorption. When a mendicant knows and sees like this, would it be appropriate to say of them: ‘The soul and the body are the same thing’ or ‘The soul and the body are different things’?” 

“It\marginnote{3.6} would, reverend.” 

“But\marginnote{3.8} reverends, I know and see like this. Nevertheless, I do not say: ‘The soul and the body are the same thing’ or ‘The soul and the body are different things’. … 

They\marginnote{4.1} extend and project the mind toward knowledge and vision … When a mendicant knows and sees like this, would it be appropriate to say of them: ‘The soul and the body are the same thing’ or ‘The soul and the body are different things’?” 

“It\marginnote{4.4} would, reverend.” 

“But\marginnote{4.6} reverends, I know and see like this. Nevertheless, I do not say: ‘The soul and the body are the same thing’ or ‘The soul and the body are different things’. … 

They\marginnote{5.1} understand: ‘… there is no return to any state of existence.’ When a mendicant knows and sees like this, would it be appropriate to say of them: ‘The soul and the body are the same thing’ or ‘The soul and the body are different things’?” 

“It\marginnote{5.4} would not, reverend.” 

“But\marginnote{5.6} reverends, I know and see like this. Nevertheless, I do not say: ‘The soul and the body are the same thing’ or ‘The soul and the body are different things’.” 

That\marginnote{5.9} is what the Buddha said. Satisfied, the two renunciates were happy with what the Buddha said. 

%
\chapter*{{\suttatitleacronym DN 8}{\suttatitletranslation The Longer Discourse on the Lion’s Roar }{\suttatitleroot Mahāsīhanādasutta}}
\addcontentsline{toc}{chapter}{\tocacronym{DN 8} \toctranslation{The Longer Discourse on the Lion’s Roar } \tocroot{Mahāsīhanādasutta}}
\markboth{The Longer Discourse on the Lion’s Roar }{Mahāsīhanādasutta}
\extramarks{DN 8}{DN 8}

\scevam{So\marginnote{1.1} I have heard. }At one time the Buddha was staying near \textsanskrit{Ujuñña}, in the deer park at \textsanskrit{Kaṇṇakatthala}. 

Then\marginnote{1.3} the naked ascetic Kassapa went up to the Buddha and exchanged greetings with him. When the greetings and polite conversation were over, he stood to one side, and said to the Buddha: 

“Master\marginnote{2.1} Gotama, I have heard the following: ‘The ascetic Gotama criticizes all forms of mortification. He categorically condemns and denounces those self-mortifiers who live rough.’ Do those who say this repeat what the Buddha has said, and not misrepresent him with an untruth? Is their explanation in line with the teaching? Are there any legitimate grounds for rebuke and criticism? For we don’t want to misrepresent Master Gotama.” 

“Kassapa,\marginnote{3.1} those who say this do not repeat what I have said. They misrepresent me with what is false, baseless, and untrue. With clairvoyance that is purified and superhuman, I see some self-mortifier who lives rough reborn in a place of loss, a bad place, the underworld, hell. But I see another self-mortifier who lives rough reborn in a good place, a heavenly realm. 

I\marginnote{3.4} see some self-mortifier who takes it easy reborn in a place of loss. But I see another self-mortifier who takes it easy reborn in a good place, a heavenly realm. Since I truly understand the coming and going, passing away and rebirth of these self-mortifiers in this way, how could I criticize all forms of mortification, or categorically condemn and denounce those self-mortifiers who live rough? 

There\marginnote{4.1} are some clever ascetics and brahmins who are subtle, accomplished in the doctrines of others, hair-splitters. You’d think they live to demolish convictions with their intellect. They agree with me in some matters and disagree in others. Some of the things that they applaud, I also applaud. Some of the things that they don’t applaud, I also don’t applaud. But some of the things that they applaud, I don’t applaud. And some of the things that they don’t applaud, I do applaud. 

Some\marginnote{4.7} of the things that I applaud, others also applaud. Some of the things that I don’t applaud, they also don’t applaud. But some of the things that I don’t applaud, others do applaud. And some of the things that I do applaud, others don’t applaud. 

\section*{1. Examination }

I\marginnote{5.1} go up to them and say: ‘Let us leave aside those matters on which we disagree. But there are some matters on which we agree. Regarding these, sensible people, pursuing, pressing, and grilling, would compare teacher with teacher or community with community: 

“There\marginnote{5.4} are things that are unskillful, blameworthy, not to be cultivated, unworthy of the noble ones, and dark—and are reckoned as such. Who behaves like they’ve totally given these things up: the ascetic Gotama, or the teachers of other communities?”’ 

It’s\marginnote{6.1} possible that they might say: ‘The ascetic Gotama behaves like he’s totally given those unskillful things up, compared with the teachers of other communities.’ And that’s how, when sensible people pursue the matter, they will mostly praise us. 

In\marginnote{7.1} addition, sensible people, engaging, pressing, and grilling, would compare teacher with teacher or community with community: ‘There are things that are skillful, blameless, worth cultivating, worthy of the noble ones, and bright—and are reckoned as such. Who proceeds having totally undertaken these things: the ascetic Gotama, or the teachers of other communities?’ 

It’s\marginnote{8.1} possible that they might say: ‘The ascetic Gotama proceeds having totally undertaken these things, compared with the teachers of other communities.’ And that’s how, when sensible people pursue the matter, they will mostly praise us. 

In\marginnote{9.1} addition, sensible people, pursuing, pressing, and grilling, would compare teacher with teacher or community with community: ‘There are things that are unskillful, blameworthy, not to be cultivated, unworthy of the noble ones, and dark—and are reckoned as such. Who behaves like they’ve totally given these things up: the ascetic Gotama’s disciples, or the disciples of other teachers?’ 

It’s\marginnote{10.1} possible that they might say: ‘The ascetic Gotama’s disciples behave like they’ve totally given those unskillful things up, compared with the disciples of other teachers.’ And that’s how, when sensible people pursue the matter, they will mostly praise us. 

In\marginnote{11.1} addition, sensible people, pursuing, pressing, and grilling, would compare teacher with teacher or community with community: ‘There are things that are skillful, blameless, worth cultivating, worthy of the noble ones, and bright—and are reckoned as such. Who proceeds having totally undertaken these things: the ascetic Gotama’s disciples, or the disciples of other teachers?’ 

It’s\marginnote{12.1} possible that they might say: ‘The ascetic Gotama’s disciples proceed having totally undertaken those skillful things, compared with the disciples of other teachers.’ And that’s how, when sensible people pursue the matter, they will mostly praise us. 

\section*{2. The Noble Eightfold Path }

There\marginnote{13.1} is, Kassapa, a path, there is a practice, practicing in accordance with which you will know and see for yourself: ‘Only the ascetic Gotama’s words are timely, true, and meaningful, in line with the teaching and training.’ And what is that path? It is simply this noble eightfold path, that is: right view, right thought, right speech, right action, right livelihood, right effort, right mindfulness, and right immersion. This is the path, this is the practice, practicing in accordance with which you will know and see for yourself: ‘Only the ascetic Gotama’s words are timely, true, and meaningful, in line with the teaching and training.’” 

\section*{3. Practicing Self-Mortification }

When\marginnote{14.1} he had spoken, Kassapa said to the Buddha: 

“Reverend\marginnote{14.2} Gotama, those ascetics and brahmins consider these practices of self-mortification to be what makes someone a true ascetic or brahmin. They go naked, ignoring conventions. They lick their hands, and don’t come or wait when called. They don’t consent to food brought to them, or food prepared on purpose for them, or an invitation for a meal. They don’t receive anything from a pot or bowl; or from someone who keeps sheep, or who has a weapon or a shovel in their home; or where a couple is eating; or where there is a woman who is pregnant, breastfeeding, or who has a man in her home; or where there’s a dog waiting or flies buzzing. They accept no fish or meat or liquor or wine, and drink no beer. They go to just one house for alms, taking just one mouthful, or two houses and two mouthfuls, up to seven houses and seven mouthfuls. They feed on one saucer a day, two saucers a day, up to seven saucers a day. They eat once a day, once every second day, up to once a week, and so on, even up to once a fortnight. They live committed to the practice of eating food at set intervals. 

Those\marginnote{14.8} ascetics and brahmins also consider these practices of self-mortification to be what makes someone a true ascetic or brahmin. They eat herbs, millet, wild rice, poor rice, water lettuce, rice bran, scum from boiling rice, sesame flour, grass, or cow dung. They survive on forest roots and fruits, or eating fallen fruit. 

Those\marginnote{14.10} ascetics and brahmins also consider these practices of mortification to be what makes someone a true ascetic or brahmin. They wear robes of sunn hemp, mixed hemp, corpse-wrapping cloth, rags, lodh tree bark, antelope hide (whole or in strips), kusa grass, bark, wood-chips, human hair, horse-tail hair, or owls’ wings. They tear out hair and beard, committed to this practice. They constantly stand, refusing seats. They squat, committed to persisting in the squatting position. They lie on a mat of thorns, making a mat of thorns their bed. They make their bed on a plank, or the bare ground. They lie only on one side. They wear dust and dirt. They stay in the open air. They sleep wherever they lay their mat. They eat unnatural things, committed to the practice of eating unnatural foods. They don’t drink, committed to the practice of not drinking liquids. They’re committed to the practice of immersion in water three times a day, including the evening.” 

\section*{4. The Uselessness of Self-Mortification }

“Kassapa,\marginnote{15.1} someone may practice all those forms of self-mortification, but if they haven’t developed and realized any accomplishment in ethics, mind, and wisdom, they are far from being a true ascetic or brahmin. But take a mendicant who develops a heart of love, free of enmity and ill will. And they realize the undefiled freedom of heart and freedom by wisdom in this very life, and live having realized it with their own insight due to the ending of defilements. When they achieve this, they’re called a mendicant who is a ‘true ascetic’ and also ‘a true brahmin’. 

When\marginnote{16.1} he had spoken, Kassapa said to the Buddha, “It’s hard, Master Gotama, to be a true ascetic or a true brahmin.” 

“It’s\marginnote{16.3} typical, Kassapa, in this world to think that it’s hard to be a true ascetic or brahmin. But someone might practice all those forms of self-mortification. And if it was only because of just that much, only because of that self-mortification that it was so very hard to be a true ascetic or brahmin, it wouldn’t be appropriate to say that it’s hard to be a true ascetic or brahmin. 

For\marginnote{16.7} it would be quite possible for a householder or a householder’s child—or even the bonded maid who carries the water-jar—to practice all those forms of self-mortification. 

It’s\marginnote{16.9} because there’s something other than just that much, something other than that self-mortification that it’s so very hard to be a true ascetic or brahmin. And that’s why it is appropriate to say that it’s hard to be a true ascetic or brahmin. Take a mendicant who develops a heart of love, free of enmity and ill will. And they realize the undefiled freedom of heart and freedom by wisdom in this very life, and live having realized it with their own insight due to the ending of defilements. When they achieve this, they’re called a mendicant who is a ‘true ascetic’ and also ‘a true brahmin’. 

When\marginnote{17.1} he had spoken, Kassapa said to the Buddha, “It’s hard, Master Gotama, to know a true ascetic or a true brahmin.” 

“It’s\marginnote{17.3} typical in this world to think that it’s hard to know a true ascetic or brahmin. But someone might practice all those forms of self-mortification. And if it was only by just that much, only by that self-mortification that it was so very hard to know a true ascetic or brahmin, it wouldn’t be appropriate to say that it’s hard to know a true ascetic or brahmin. 

For\marginnote{17.7} it would be quite possible for a householder or a householder’s child—or even the bonded maid who carries the water-jar—to know that someone is practicing all those forms of self-mortification. 

It’s\marginnote{17.9} because there’s something other than just that much, something other than that self-mortification that it’s so very hard to know a true ascetic or brahmin. And that’s why it is appropriate to say that it’s hard to know a true ascetic or brahmin. Take a mendicant who develops a heart of love, free of enmity and ill will. And they realize the undefiled freedom of heart and freedom by wisdom in this very life, and live having realized it with their own insight due to the ending of defilements. When they achieve this, they’re called a mendicant who is a ‘true ascetic’ and also ‘a true brahmin’.” 

\section*{5. The Accomplishment of Ethics, Immersion, and Wisdom }

When\marginnote{18.1} he had spoken, Kassapa said to the Buddha, “But Master Gotama, what is that accomplishment in ethics, in mind, and in wisdom?” 

“It’s\marginnote{18.3} when a Realized One arises in the world, perfected, a fully awakened Buddha … Seeing danger in the slightest fault, a mendicant keeps the rules they’ve undertaken. They act skillfully by body and speech. They’re purified in livelihood and accomplished in ethical conduct. They guard the sense doors, have mindfulness and situational awareness, and are content. 

And\marginnote{18.5} how is a mendicant accomplished in ethics? It’s when a mendicant gives up killing living creatures. They renounce the rod and the sword. They’re scrupulous and kind, living full of compassion for all living beings. This pertains to their accomplishment in ethics. … 

There\marginnote{19.1} are some ascetics and brahmins who, while enjoying food given in faith, still earn a living by unworthy branches of knowledge, by wrong livelihood. … They refrain from such unworthy branches of knowledge, such wrong livelihood. This pertains to their accomplishment in ethics. 

A\marginnote{19.5} mendicant thus accomplished in ethics sees no danger in any quarter in regards to their ethical restraint. It’s like a king who has defeated his enemies. He sees no danger from his foes in any quarter. In the same way, a mendicant thus accomplished in ethics sees no danger in any quarter in regards to their ethical restraint. When they have this entire spectrum of noble ethics, they experience a blameless happiness inside themselves. That’s how a mendicant is accomplished in ethics. This, Kassapa, is that accomplishment in ethics. … They enter and remain in the first absorption … This pertains to their accomplishment in mind. … They enter and remain in the second absorption … third absorption … fourth absorption. This pertains to their accomplishment in mind. This, Kassapa, is that accomplishment in mind. 

When\marginnote{20.1} their mind is immersed like this, they extend and project it toward knowledge and vision … This pertains to their accomplishment in wisdom. … They understand: ‘… there is no return to any state of existence.’ This pertains to their accomplishment in wisdom. This, Kassapa, is that accomplishment in wisdom. 

And,\marginnote{20.7} Kassapa, there is no accomplishment in ethics, mind, and wisdom that is better or finer than this. 

\section*{6. The Lion’s Roar }

There\marginnote{21.1} are, Kassapa, some ascetics and brahmins who teach ethics. They praise ethical conduct in many ways. But as far as the highest noble ethics goes, I don’t see anyone who’s my equal, still less my superior. Rather, I am the one who is superior when it comes to the higher ethics. 

There\marginnote{21.5} are, Kassapa, some ascetics and brahmins who teach mortification in disgust of sin. They praise mortification in disgust of sin in many ways. But as far as the highest noble mortification in disgust of sin goes, I don’t see anyone who’s my equal, still less my superior. Rather, I am the one who is superior when it comes to the higher mortification in disgust of sin. 

There\marginnote{21.9} are, Kassapa, some ascetics and brahmins who teach wisdom. They praise wisdom in many ways. But as far as the highest noble wisdom goes, I don’t see anyone who’s my equal, still less my superior. Rather, I am the one who is superior when it comes to the higher wisdom. 

There\marginnote{21.13} are, Kassapa, some ascetics and brahmins who teach freedom. They praise freedom in many ways. But as far as the highest noble freedom goes, I don’t see anyone who’s my equal, still less my superior. Rather, I am the one who is superior when it comes to the higher freedom. 

It’s\marginnote{22.1} possible that wanderers who follow other paths might say: ‘The ascetic Gotama only roars his lion’s roar in an empty hut, not in an assembly.’ They should be told, ‘Not so!’ What should be said is this: ‘The ascetic Gotama roars his lion’s roar, and he roars it in the assemblies.’ 

It’s\marginnote{22.5} possible that wanderers who follow other paths might say: ‘The ascetic Gotama roars his lion’s roar, and he roars it in the assemblies. But he doesn’t roar it boldly.’ They should be told, ‘Not so!’ What should be said is this: ‘The ascetic Gotama roars his lion’s roar, he roars it in the assemblies, and he roars it boldly.’ 

It’s\marginnote{22.9} possible that wanderers who follow other paths might say: ‘The ascetic Gotama roars his lion’s roar, he roars it in the assemblies, and he roars it boldly. But they don’t question him. … Or he doesn’t answer their questions. … Or his answers are not satisfactory. … Or they don’t think him worth listening to. … Or they’re not confident after listening. … Or they don’t show their confidence. … Or they don’t practice accordingly. … Or they don’t succeed in their practice.’ They should be told, ‘Not so!’ What should be said is this: ‘The ascetic Gotama roars his lion’s roar; he roars it in the assemblies; he roars it boldly; they question him; he answers their questions; his answers are satisfactory; they think him worth listening to; they’re confident after listening; they show their confidence; they practice accordingly; and they succeed in their practice.’ 

\section*{7. The Probation For One Previously Ordained }

Kassapa,\marginnote{23.1} this one time I was staying near \textsanskrit{Rājagaha}, on the Vulture’s Peak Mountain. There a certain practitioner of self-mortification named Nigrodha asked me about the higher mortification in disgust of sin. I answered his question. He was extremely happy with my answer.” 

“Sir,\marginnote{23.5} who wouldn’t be extremely happy after hearing the Buddha’s teaching? For I too am extremely happy after hearing the Buddha’s teaching! Excellent, sir! Excellent! As if he were righting the overturned, or revealing the hidden, or pointing out the path to the lost, or lighting a lamp in the dark so people with good eyes can see what’s there, so too the Buddha has made the teaching clear in many ways. I go for refuge to the Buddha, to the teaching, and to the mendicant \textsanskrit{Saṅgha}. Sir, may I receive the going forth, the ordination in the Buddha’s presence?” 

“Kassapa,\marginnote{24.1} if someone formerly ordained in another sect wishes to take the going forth, the ordination in this teaching and training, they must spend four months on probation. When four months have passed, if the mendicants are satisfied, they’ll give the going forth, the ordination into monkhood. However, I have recognized individual differences in this matter.” 

“Sir,\marginnote{24.3} if four months probation are required in such a case, I’ll spend four years on probation. When four years have passed, if the mendicants are satisfied, let them give me the going forth, the ordination into monkhood.” 

And\marginnote{24.4} the naked ascetic Kassapa received the going forth, the ordination in the Buddha’s presence. Not long after his ordination, Venerable Kassapa, living alone, withdrawn, diligent, keen, and resolute, soon realized the supreme end of the spiritual path in this very life. He lived having achieved with his own insight the goal for which gentlemen rightly go forth from the lay life to homelessness. 

He\marginnote{24.6} understood: “Rebirth is ended; the spiritual journey has been completed; what had to be done has been done; there is no return to any state of existence.” And Venerable Kassapa became one of the perfected. 

%
\chapter*{{\suttatitleacronym DN 9}{\suttatitletranslation With Poṭṭhapāda }{\suttatitleroot Poṭṭhapādasutta}}
\addcontentsline{toc}{chapter}{\tocacronym{DN 9} \toctranslation{With Poṭṭhapāda } \tocroot{Poṭṭhapādasutta}}
\markboth{With Poṭṭhapāda }{Poṭṭhapādasutta}
\extramarks{DN 9}{DN 9}

\section*{1. On the Wanderer \textsanskrit{Poṭṭhapāda} }

\scevam{So\marginnote{1.1} I have heard. }At one time the Buddha was staying near \textsanskrit{Sāvatthī} in Jeta’s Grove, \textsanskrit{Anāthapiṇḍika}’s monastery. 

Now\marginnote{1.3} at that time the wanderer \textsanskrit{Poṭṭhapāda} was residing together with three hundred wanderers in \textsanskrit{Mallikā}’s single-halled monastery for group debates, set among the flaking pale-moon ebony trees. Then the Buddha robed up in the morning and, taking his bowl and robe, entered \textsanskrit{Sāvatthī} for alms. 

Then\marginnote{2.1} it occurred to him, “It’s too early to wander for alms in \textsanskrit{Sāvatthī}. Why don’t I go to \textsanskrit{Mallikā}’s monastery to visit the wanderer \textsanskrit{Poṭṭhapāda}?” So that’s what he did. 

Now\marginnote{3.1} at that time, \textsanskrit{Poṭṭhapāda} was sitting together with a large assembly of wanderers making an uproar, a dreadful racket. They engaged in all kinds of unworthy talk, such as talk about kings, bandits, and ministers; talk about armies, threats, and wars; talk about food, drink, clothes, and beds; talk about garlands and fragrances; talk about family, vehicles, villages, towns, cities, and countries; talk about women and heroes; street talk and well talk; talk about the departed; motley talk; tales of land and sea; and talk about being reborn in this or that state of existence. 

\textsanskrit{Poṭṭhapāda}\marginnote{4.1} saw the Buddha coming off in the distance, and hushed his own assembly, “Be quiet, good sirs, don’t make a sound. Here comes the ascetic Gotama. The venerable likes quiet and praises quiet. Hopefully if he sees that our assembly is quiet he’ll see fit to approach.” Then those wanderers fell silent. 

Then\marginnote{5.1} the Buddha approached \textsanskrit{Poṭṭhapāda}, who said to him, “Come, Blessed One! Welcome, Blessed One! It’s been a long time since you took the opportunity to come here. Please, sir, sit down, this seat is ready.” 

The\marginnote{5.7} Buddha sat on the seat spread out, while \textsanskrit{Poṭṭhapāda} took a low seat and sat to one side. The Buddha said to him, “\textsanskrit{Poṭṭhapāda}, what were you sitting talking about just now? What conversation was left unfinished?” 

\subsection*{1.1. On the Cessation of Perception }

When\marginnote{6.1} he said this, the wanderer \textsanskrit{Poṭṭhapāda} said to the Buddha, “Sir, leave aside what we were sitting talking about just now. It won’t be hard for you to hear about that later. 

Sir,\marginnote{6.4} a few days ago several ascetics and brahmins who follow various other paths were sitting together at the debating hall, and this discussion came up among them: ‘How does the cessation of perception happen?’ 

Some\marginnote{6.6} of them said: ‘A person’s perceptions arise and cease without cause or reason. When they arise, you become percipient. When they cease, you become non-percipient.’ That’s how some describe the cessation of perception. 

But\marginnote{6.11} someone else says: ‘That’s not how it is, gentlemen ! Perception is a person’s self, When it enters, you become percipient. When it departs, you become non-percipient.’ That’s how some describe the cessation of perception. 

But\marginnote{6.18} someone else says: ‘That’s not how it is, gentlemen ! There are ascetics and brahmins of great power and might. They insert and extract a person’s perception. When they insert it, you become percipient. When they extract it, you become non-percipient.’ That’s how some describe the cessation of perception. 

But\marginnote{6.25} someone else says: ‘That’s not how it is, gentlemen ! There are deities of great power and might. They insert and extract a person’s perception. When they insert it, you become percipient. When they extract it, you become non-percipient.’ That’s how some describe the cessation of perception. 

That\marginnote{6.32} reminded me of the Buddha: ‘Surely it must be the Blessed One, the Holy One who is so skilled in such matters.’ The Buddha is skilled and well-versed in the cessation of perception. How does the cessation of perception happen?” 

\subsection*{1.2. Perception Arises With a Cause }

“Regarding\marginnote{7.1} this, \textsanskrit{Poṭṭhapāda}, those ascetics and brahmins who say that a person’s perceptions arise and cease without cause or reason are wrong from the start. Why is that? Because a person’s perceptions arise and cease with cause and reason. With training, certain perceptions arise and certain perceptions cease. 

And\marginnote{7.6} what is that training?” said the Buddha. 

“It’s\marginnote{7.7} when a Realized One arises in the world, perfected, a fully awakened Buddha … That’s how a mendicant is accomplished in ethics. … Seeing that the hindrances have been given up in them, joy springs up. Being joyful, rapture springs up. When the mind is full of rapture, the body becomes tranquil. When the body is tranquil, they feel bliss. And when blissful, the mind becomes immersed. Quite secluded from sensual pleasures, secluded from unskillful qualities, they enter and remain in the first absorption, which has the rapture and bliss born of seclusion, while placing the mind and keeping it connected. The sensual perception that they had previously ceases. At that time they have a subtle and true perception of the rapture and bliss born of seclusion. That’s how, with training, certain perceptions arise and certain perceptions cease. And this is that training,” said the Buddha. 

“Furthermore,\marginnote{11.1} as the placing of the mind and keeping it connected are stilled, a mendicant enters and remains in the second absorption, which has the rapture and bliss born of immersion, with internal clarity and confidence, and unified mind, without placing the mind and keeping it connected. The subtle and true perception of the rapture and bliss born of seclusion that they had previously ceases. At that time they have a subtle and true perception of the rapture and bliss born of immersion. That’s how, with training, certain perceptions arise and certain perceptions cease. And this is that training,” said the Buddha. 

“Furthermore,\marginnote{12.1} with the fading away of rapture, a mendicant enters and remains in the third absorption, where they meditate with equanimity, mindful and aware, personally experiencing the bliss of which the noble ones declare, ‘Equanimous and mindful, one meditates in bliss.’ The subtle and true perception of the rapture and bliss born of immersion that they had previously ceases. At that time they have a subtle and true perception of equanimous bliss. That’s how, with training, certain perceptions arise and certain perceptions cease. And this is that training,” said the Buddha. 

“Furthermore,\marginnote{13.1} giving up pleasure and pain, and ending former happiness and sadness, a mendicant enters and remains in the fourth absorption, without pleasure or pain, with pure equanimity and mindfulness. The subtle and true perception of equanimous bliss that they had previously ceases. At that time they have a subtle and true perception of neutral feeling. That’s how, with training, certain perceptions arise and certain perceptions cease. And this is that training,” said the Buddha. 

“Furthermore,\marginnote{14.1} a mendicant, going totally beyond perceptions of form, with the ending of perceptions of impingement, not focusing on perceptions of diversity, aware that ‘space is infinite’, enters and remains in the dimension of infinite space. The perception of luminous form that they had previously ceases. At that time they have a subtle and true perception of the dimension of infinite space. That’s how, with training, certain perceptions arise and certain perceptions cease. And this is that training,” said the Buddha. 

“Furthermore,\marginnote{15.1} a mendicant, going totally beyond the dimension of infinite space, aware that ‘consciousness is infinite’, enters and remains in the dimension of infinite consciousness. The subtle and true perception of the dimension of infinite space that they had previously ceases. At that time they have a subtle and true perception of the dimension of infinite consciousness. That’s how, with training, certain perceptions arise and certain perceptions cease. And this is that training,” said the Buddha. 

“Furthermore,\marginnote{16.1} a mendicant, going totally beyond the dimension of infinite consciousness, aware that ‘there is nothing at all’, enters and remains in the dimension of nothingness. The subtle and true perception of the dimension of infinite consciousness that they had previously ceases. At that time they have a subtle and true perception of the dimension of nothingness. That’s how, with training, certain perceptions arise and certain perceptions cease. And this is that training,” said the Buddha. 

“\textsanskrit{Poṭṭhapāda},\marginnote{17.1} from the time a mendicant here takes responsibility for their own perception, they proceed from one stage to the next, gradually reaching the peak of perception. Standing on the peak of perception they think, ‘Intentionality is bad for me, it’s better to be free of it. For if I were to intend and choose, these perceptions would cease in me, and other coarser perceptions would arise. Why don’t I neither make a choice nor form an intention?’ They neither make a choice nor form an intention. Those perceptions cease in them, and other coarser perceptions don’t arise. They touch cessation. And that, \textsanskrit{Poṭṭhapāda}, is how the gradual cessation of perception is attained with awareness. 

What\marginnote{18.1} do you think, \textsanskrit{Poṭṭhapāda}? Have you ever heard of this before?” 

“No,\marginnote{18.3} sir. This is how I understand what the Buddha said: ‘From the time a mendicant here takes responsibility for their own perception, they proceed from one stage to the next, gradually reaching the peak of perception. Standing on the peak of perception they think, “Intentionality is bad for me, it’s better to be free of it. For if I were to intend and choose, these perceptions would cease in me, and other coarser perceptions would arise. Why don’t I neither make a choice nor form an intention?” Those perceptions cease in them, and other coarser perceptions don’t arise. They touch cessation. And that is how the gradual cessation of perception is attained with awareness.’” 

“That’s\marginnote{18.13} right, \textsanskrit{Poṭṭhapāda}.” 

“Does\marginnote{19.1} the Buddha describe just one peak of perception, or many?” 

“I\marginnote{19.2} describe the peak of perception as both one and many.” 

“But\marginnote{19.3} sir, how do you describe it as one peak and as many?” 

“I\marginnote{19.4} describe the peak of perception according to the specific manner in which one touches cessation. That’s how I describe the peak of perception as both one and many.” 

“But\marginnote{20.1} sir, does perception arise first and knowledge afterwards? Or does knowledge arise first and perception afterwards? Or do they both arise at the same time?” 

“Perception\marginnote{20.2} arises first and knowledge afterwards. The arising of perception leads to the arising of knowledge. They understand, ‘My knowledge arose from a specific condition.’ That is a way to understand how perception arises first and knowledge afterwards; that the arising of perception leads to the arising of knowledge.” 

\subsection*{1.3. Perception and the Self }

“Sir,\marginnote{21.1} is perception a person’s self, or are perception and self different things?” 

“But\marginnote{21.2} \textsanskrit{Poṭṭhapāda}, do you believe in a self?” 

“I\marginnote{21.3} believe in a substantial self, sir, which has form, made up of the four primary elements, and consumes solid food.” 

“Suppose\marginnote{21.4} there were such a substantial self, \textsanskrit{Poṭṭhapāda}. In that case, perception would be one thing, the self another. Here is another way to understand how perception and self are different things. So long as that substantial self remains, still some perceptions arise in a person and others cease. That is a way to understand how perception and self are different things.” 

“Sir,\marginnote{22.1} I believe in a mind-made self which is complete in all its various parts, not deficient in any faculty.” 

“Suppose\marginnote{22.2} there were such a mind-made self, \textsanskrit{Poṭṭhapāda}. In that case, perception would be one thing, the self another. Here is another way to understand how perception and self are different things. So long as that mind-made self remains, still some perceptions arise in a person and others cease. That too is a way to understand how perception and self are different things.” 

“Sir,\marginnote{23.1} I believe in a formless self which is made of perception.” 

“Suppose\marginnote{23.2} there were such a formless self, \textsanskrit{Poṭṭhapāda}. In that case, perception would be one thing, the self another. Here is another way to understand how perception and self are different things. So long as that formless self remains, still some perceptions arise in a person and others cease. That too is a way to understand how perception and self are different things.” 

“But,\marginnote{24.1} sir, am I able to know whether perception is a person’s self, or whether perception and self are different things?” 

“It’s\marginnote{24.3} hard for you to understand this, since you have a different view, creed, preference, practice, and tradition.” 

“Well,\marginnote{25.1} if that’s the case, sir, then is this right: ‘The cosmos is eternal. This is the only truth, anything else is wrong’?” 

“This\marginnote{25.4} has not been declared by me, \textsanskrit{Poṭṭhapāda}.” 

“Then\marginnote{26.1} is this right: ‘The cosmos is not eternal. This is the only truth, anything else is wrong’?” 

“This\marginnote{26.2} too has not been declared by me.” 

“Then\marginnote{27.1} is this right: ‘The cosmos is finite …’ … ‘The cosmos is infinite …’ … ‘The soul and the body are the same thing …’ … ‘The soul and the body are different things …’ … ‘A Realized One exists after death …’ … ‘A Realized One doesn’t exist after death …’ … ‘A Realized One both exists and doesn’t exist after death …’ … ‘A Realized One neither exists nor doesn’t exist after death. This is the only truth, anything else is wrong’?” 

“This\marginnote{27.9} too has not been declared by me.” 

“Why\marginnote{28.1} haven’t these things been declared by the Buddha?” 

“Because\marginnote{28.2} they’re not beneficial or relevant to the fundamentals of the spiritual life. They don’t lead to disillusionment, dispassion, cessation, peace, insight, awakening, and extinguishment. That’s why I haven’t declared them.” 

“Then\marginnote{29.1} what has been declared by the Buddha?” 

“I\marginnote{29.2} have declared this: ‘This is suffering’ … ‘This is the origin of suffering’ … ‘This is the cessation of suffering’ … ‘This is the practice that leads to the cessation of suffering’.” 

“Why\marginnote{30.1} have these things been declared by the Buddha?” 

“Because\marginnote{30.2} they are beneficial and relevant to the fundamentals of the spiritual life. They lead to disillusionment, dispassion, cessation, peace, insight, awakening, and extinguishment. That’s why I have declared them.” 

“That’s\marginnote{30.4} so true, Blessed One! That’s so true, Holy One! Please, sir, go at your convenience.” Then the Buddha got up from his seat and left. 

Soon\marginnote{31.1} after the Buddha left, those wanderers gave \textsanskrit{Poṭṭhapāda} a comprehensive tongue-lashing, “No matter what the ascetic Gotama says, \textsanskrit{Poṭṭhapāda} agrees with him: ‘That’s so true, Blessed One! That’s so true, Holy One!’ We understand that the ascetic Gotama didn’t make any definitive statement at all regarding whether the cosmos is eternal and so on.” 

When\marginnote{31.6} they said this, \textsanskrit{Poṭṭhapāda} said to them, “I too understand that the ascetic Gotama didn’t make any definitive statement at all regarding whether the cosmos is eternal and so on. Nevertheless, the practice that he describes is true, real, and accurate. It is the regularity of natural principles, the invariance of natural principles. So how could a sensible person such as I not agree that what was well spoken by the ascetic Gotama was in fact well spoken?” 

\section*{2. On Citta \textsanskrit{Hatthisāriputta} }

Then\marginnote{32.1} after two or three days had passed, Citta \textsanskrit{Hatthisāriputta} and \textsanskrit{Poṭṭhapāda} went to see the Buddha. Citta \textsanskrit{Hatthisāriputta} bowed and sat down to one side. But the wanderer \textsanskrit{Poṭṭhapāda} exchanged greetings with the Buddha, and when the greetings and polite conversation were over, he sat down to one side. \textsanskrit{Poṭṭhapāda} told the Buddha what had happened after he left. The Buddha said: 

“All\marginnote{33.1} those wanderers, \textsanskrit{Poṭṭhapāda}, are blind and sightless. You are the only one who sees. For I have taught and pointed out both things that are definitive and things that are not definitive. 

And\marginnote{33.5} what things have I taught and pointed out that are not definitive? ‘The cosmos is eternal’ … ‘The cosmos is not eternal’ … ‘The cosmos is finite’ … ‘The cosmos is infinite’ … ‘The soul is the same thing as the body’ … ‘The soul and the body are different things’ … ‘A Realized One exists after death’ … ‘A Realized One doesn’t exist after death’ … ‘A Realized One both exists and doesn’t exist after death’ … ‘A Realized One neither exists nor doesn’t exist after death.’ 

And\marginnote{33.16} why haven’t I taught and pointed out such things that are not definitive? Because those things aren’t beneficial or relevant to the fundamentals of the spiritual life. They don’t lead to disillusionment, dispassion, cessation, peace, insight, awakening, and extinguishment. That’s why I haven’t taught and pointed them out. 

\subsection*{2.1. Things That Are Definitive }

And\marginnote{33.20} what things have I taught and pointed out that are definitive? ‘This is suffering’ … ‘This is the origin of suffering’ … ‘This is the cessation of suffering’ … ‘This is the practice that leads to the cessation of suffering’.” 

And\marginnote{33.25} why have I taught and pointed out such things that are definitive? Because they are beneficial and relevant to the fundamentals of the spiritual life. They lead to disillusionment, dispassion, cessation, peace, insight, awakening, and extinguishment. That’s why I have taught and pointed them out. 

There\marginnote{34.1} are some ascetics and brahmins who have this doctrine and view: ‘The self is exclusively happy and is well after death.’ I go up to them and say, ‘Is it really true that this is the venerables’ view?’ And they answer, ‘Yes’. I say to them, ‘But do you meditate knowing and seeing an exclusively happy world?’ Asked this, they say, ‘No.’ 

I\marginnote{34.10} say to them, ‘But have you perceived an exclusively happy self for a single day or night, or even half a day or night?’ Asked this, they say, ‘No.’ 

I\marginnote{34.13} say to them, ‘But do you know a path and a practice to realize an exclusively happy world?’ Asked this, they say, ‘No.’ 

I\marginnote{34.17} say to them, ‘But have you ever heard the voice of the deities reborn in an exclusively happy world saying, “Practice well, dear sirs, practice directly so as to realize an exclusively happy world. For this is how we practiced, and we were reborn in an exclusively happy world”?’ Asked this, they say, ‘No.’ 

What\marginnote{34.22} do you think, \textsanskrit{Poṭṭhapāda}? This being so, doesn’t what they say turn out to have no demonstrable basis?” 

“Clearly\marginnote{34.24} that’s the case, sir.” 

“Suppose,\marginnote{35.1} \textsanskrit{Poṭṭhapāda}, a man were to say: ‘Whoever the finest lady in the land is, it is her that I want, her that I desire!’ They’d say to him, ‘Mister, that finest lady in the land who you desire—do you know whether she’s an aristocrat, a brahmin, a merchant, or a worker?’ Asked this, he’d say, ‘No.’ They’d say to him, ‘Mister, that finest lady in the land who you desire—do you know her name or clan? Whether she’s tall or short or medium? Whether her skin is black, brown, or tawny? What village, town, or city she comes from?’ Asked this, he’d say, ‘No.’ They’d say to him, ‘Mister, do you desire someone who you’ve never even known or seen?’ Asked this, he’d say, ‘Yes.’ 

What\marginnote{35.12} do you think, \textsanskrit{Poṭṭhapāda}? This being so, doesn’t that man’s statement turn out to have no demonstrable basis?” 

“Clearly\marginnote{35.14} that’s the case, sir.” 

“In\marginnote{36.1} the same way, the ascetics and brahmins who have those various doctrines and views … 

Doesn’t\marginnote{36.22} what they say turn out to have no demonstrable basis?” 

“Clearly\marginnote{36.23} that’s the case, sir.” 

“Suppose\marginnote{37.1} a man was to build a ladder at the crossroads for climbing up to a stilt longhouse. They’d say to him, ‘Mister, that stilt longhouse that you’re building a ladder for—do you know whether it’s to the north, south, east, or west? Or whether it’s tall or short or medium?’ Asked this, he’d say, ‘No.’ They’d say to him, ‘Mister, are you building a ladder for a longhouse that you’ve never even known or seen?’ Asked this, he’d say, ‘Yes.’ 

What\marginnote{37.8} do you think, \textsanskrit{Poṭṭhapāda}? This being so, doesn’t that man’s statement turn out to have no demonstrable basis?” 

“Clearly\marginnote{37.10} that’s the case, sir.” 

“In\marginnote{38.1} the same way, the ascetics and brahmins who have those various doctrines and views … 

Doesn’t\marginnote{38.21} what they say turn out to have no demonstrable basis?” 

“Clearly\marginnote{38.22} that’s the case, sir.” 

\subsection*{2.2. Three Kinds of Reincarnation }

“\textsanskrit{Poṭṭhapāda},\marginnote{39.1} there are these three kinds of reincarnation: a substantial reincarnation, a mind-made reincarnation, and a formless reincarnation. And what is a substantial reincarnation? It has form, made up of the four primary elements, and consumes solid food. What is a mind-made reincarnation? It has form, mind-made, complete in all its various parts, not deficient in any faculty. What is a formless reincarnation? It is formless, made of perception. 

I\marginnote{40.1} teach the Dhamma for the giving up of these three kinds of reincarnation: ‘When you practice accordingly, corrupting qualities will be given up in you and cleansing qualities will grow. You’ll enter and remain in the fullness and abundance of wisdom, having realized it with your own insight in this very life.’ \textsanskrit{Poṭṭhapāda}, you might think: ‘Corrupting qualities will be given up and cleansing qualities will grow. One will enter and remain in the fullness and abundance of wisdom, having realized it with one’s own insight in this very life. But such a life is suffering.’ But you should not see it like this. Corrupting qualities will be given up and cleansing qualities will grow. One will enter and remain in the fullness and abundance of wisdom, having realized it with one’s own insight in this very life. And there will be only joy and happiness, tranquility, mindfulness and awareness. Such a life is blissful. 

If\marginnote{43.1} others should ask us, ‘But reverends, what is that substantial reincarnation?’ We’d answer like this, ‘This is that substantial reincarnation.’ 

If\marginnote{44.1} others should ask us, ‘But reverends, what is that mind-made reincarnation?’ We’d answer like this, ‘This is that mind-made reincarnation.’ 

If\marginnote{45.1} others should ask us, ‘But reverends, what is that formless reincarnation?’ We’d answer like this, ‘This is that formless reincarnation.’ 

What\marginnote{45.5} do you think, \textsanskrit{Poṭṭhapāda}? This being so, doesn’t that statement turn out to have a demonstrable basis?” 

“Clearly\marginnote{45.7} that’s the case, sir.” 

“Suppose\marginnote{46.1} a man were to build a ladder for climbing up to a stilt longhouse right underneath that longhouse. They’d say to him, ‘Mister, that stilt longhouse that you’re building a ladder for—do you know whether it’s to the north, south, east, or west? Or whether it’s tall or short or medium?’ He’d say, ‘This is that stilt longhouse for which I’m building a ladder, right underneath it.’ 

What\marginnote{46.6} do you think, \textsanskrit{Poṭṭhapāda}? This being so, doesn’t that man’s statement turn out to have a demonstrable basis?” 

“Clearly\marginnote{46.8} that’s the case, sir.” 

When\marginnote{48.1} the Buddha had spoken, Citta \textsanskrit{Hatthisāriputta} said, “Sir, while in a substantial reincarnation, are the mind-made and formless reincarnations fictitious, and only the substantial reincarnation real? While in a mind-made reincarnation, are the substantial and formless reincarnations fictitious, and only the mind-made reincarnation real? While in a formless reincarnation, are the substantial and mind-made reincarnations fictitious, and only the formless reincarnation real?” 

“While\marginnote{49.1} in a substantial reincarnation, it’s not referred to as a mind-made or formless reincarnation, only as a substantial reincarnation. While in a mind-made reincarnation, it’s not referred to as a substantial or formless reincarnation, only as a mind-made reincarnation. While in a formless reincarnation, it’s not referred to as a substantial or mind-made reincarnation, only as a formless reincarnation. 

Citta,\marginnote{49.7} suppose they were to ask you, ‘Did you exist in the past? Will you exist in the future? Do you exist now?’ How would you answer?” 

“Sir,\marginnote{49.12} if they were to ask me this, I’d answer like this, ‘I existed in the past. I will exist in the future. I exist now.’ That’s how I’d answer.” 

“But\marginnote{50.1} Citta, suppose they were to ask you, ‘Is the reincarnation you had in the past your only real one, and those of the future and present fictitious? Is the reincarnation you will have in the future your only real one, and those of the past and present fictitious? Is the reincarnation you have now your only real one, and those of the past and future fictitious?’ How would you answer?” 

“Sir,\marginnote{50.6} if they were to ask me this, I’d answer like this, ‘The reincarnation I had in the past was real at that time, and those of the future and present fictitious. The reincarnation I will have in the future will be real at the time, and those of the past and present fictitious. The reincarnation I have now is real at this time, and those of the past and future fictitious.’ That’s how I’d answer.” 

“In\marginnote{51.1} the same way, while in any one of the three reincarnations, it’s not referred to as the other two, only under its own name. 

From\marginnote{52.1} a cow comes milk, from milk comes curds, from curds come butter, from butter comes ghee, and from ghee comes cream of ghee. And the cream of ghee is said to be the best of these. While it’s milk, it’s not referred to as curds, butter, ghee, or cream of ghee. It’s only referred to as milk. While it’s curd or butter or ghee or cream of ghee, it’s not referred to as anything else, only under its own name. In the same way, while in any one of the three reincarnations, it’s not referred to as the other two, only under its own name. These are the world’s usages, terms, expressions, and descriptions, which the Realized One uses without misapprehending them.” 

When\marginnote{54.1} he had spoken, the wanderer \textsanskrit{Poṭṭhapāda} said to the Buddha, “Excellent, sir! Excellent! As if he were righting the overturned, or revealing the hidden, or pointing out the path to the lost, or lighting a lamp in the dark so people with good eyes can see what’s there, so too the Buddha has made the teaching clear in many ways. I go for refuge to the Buddha, to the teaching, and to the mendicant \textsanskrit{Saṅgha}. From this day forth, may the Buddha remember me as a lay follower who has gone for refuge for life.” 

\subsection*{2.3. The Ordination of Citta \textsanskrit{Hatthisāriputta} }

But\marginnote{55.1} Citta \textsanskrit{Hatthisāriputta} said to the Buddha, “Excellent, sir! Excellent! As if he were righting the overturned, or revealing the hidden, or pointing out the path to the lost, or lighting a lamp in the dark so people with good eyes can see what’s there, so too the Buddha has made the teaching clear in many ways. I go for refuge to the Buddha, to the teaching, and to the mendicant \textsanskrit{Saṅgha}. Sir, may I receive the going forth, the ordination in the Buddha’s presence?” 

And\marginnote{56.1} Citta \textsanskrit{Hatthisāriputta} received the going forth, the ordination in the Buddha’s presence. Not long after his ordination, Venerable Citta \textsanskrit{Hatthisāriputta}, living alone, withdrawn, diligent, keen, and resolute, soon realized the supreme end of the spiritual path in this very life. He lived having achieved with his own insight the goal for which gentlemen rightly go forth from the lay life to homelessness. He understood: “Rebirth is ended; the spiritual journey has been completed; what had to be done has been done; there is no return to any state of existence.” And Venerable Citta \textsanskrit{Hatthisāriputta} became one of the perfected. 

%
\chapter*{{\suttatitleacronym DN 10}{\suttatitletranslation With Subha }{\suttatitleroot Subhasutta}}
\addcontentsline{toc}{chapter}{\tocacronym{DN 10} \toctranslation{With Subha } \tocroot{Subhasutta}}
\markboth{With Subha }{Subhasutta}
\extramarks{DN 10}{DN 10}

\scevam{So\marginnote{1.1.1} I have heard. }At one time Venerable Ānanda was staying near \textsanskrit{Sāvatthī} in Jeta’s Grove, \textsanskrit{Anāthapiṇḍika}’s monastery. It was not long after the Buddha had become fully extinguished. 

Now\marginnote{1.1.3} at that time the brahmin student Subha, Todeyya’s son, was residing in \textsanskrit{Sāvatthī} on some business. Then he addressed a certain student, “Here, student, go to the ascetic Ānanda and in my name bow with your head to his feet. Ask him if he is healthy and well, nimble, strong, and living comfortably. And then say: ‘Sir, please visit the student Subha, Todeyya’s son, at his home out of compassion.’” 

“Yes,\marginnote{1.3.2} sir,” replied the student, and did as he was asked. 

When\marginnote{1.4.1} he had spoken, Venerable Ānanda said to him, “It’s not the right time, student. I’ve drunk sufficient refreshments for today. But hopefully tomorrow I’ll get a chance to visit him.” 

“Yes,\marginnote{1.4.5} sir,” replied the student. He went back to Subha, and told him what had happened, adding, “This much, sir, I managed to do. At least Master Ānanda will take the opportunity to visit tomorrow.” 

Then\marginnote{1.5.1} when the night had passed, Ānanda robed up in the morning and, taking his bowl and robe, went with Venerable Cetaka as his second monk to Subha’s home, where he sat on the seat spread out. Then Subha went up to Ānanda, and exchanged greetings with him. When the greetings and polite conversation were over, he sat down to one side and said to Ānanda: 

“Master\marginnote{1.5.3} Ānanda, you were Master Gotama’s attendant. You were close to him, living in his presence. You ought to know what things Master Gotama praised, and in which he encouraged, settled, and grounded all these people. What were those things?” 

“Student,\marginnote{1.6.1} the Buddha praised three sets of things, and that’s what he encouraged, settled, and grounded all these people in. What three? The entire spectrum of noble ethics, immersion, and wisdom. These are the three sets of things that the Buddha praised.” 

\section*{1. The Entire Spectrum of Ethics }

“But\marginnote{1.6.6} what was that noble spectrum of ethics that the Buddha praised?” 

“Student,\marginnote{1.7.1} it’s when a Realized One arises in the world, perfected, a fully awakened Buddha, accomplished in knowledge and conduct, holy, knower of the world, supreme guide for those who wish to train, teacher of gods and humans, awakened, blessed. He has realized with his own insight this world—with its gods, \textsanskrit{Māras} and \textsanskrit{Brahmās}, this population with its ascetics and brahmins, gods and humans—and he makes it known to others. He teaches Dhamma that’s good in the beginning, good in the middle, and good in the end, meaningful and well-phrased. And he reveals a spiritual practice that’s entirely full and pure. A householder hears that teaching, or a householder’s child, or someone reborn in some clan. They gain faith in the Realized One, and reflect: ‘Living in a house is cramped and dirty, but the life of one gone forth is wide open. It’s not easy for someone living at home to lead the spiritual life utterly full and pure, like a polished shell. Why don’t I shave off my hair and beard, dress in ocher robes, and go forth from the lay life to homelessness?’ After some time they give up a large or small fortune, and a large or small family circle. They shave off hair and beard, dress in ocher robes, and go forth from the lay life to homelessness. Once they’ve gone forth, they live restrained in the monastic code, conducting themselves well and seeking alms in suitable places. Seeing danger in the slightest fault, they keep the rules they’ve undertaken. They act skillfully by body and speech. They’re purified in livelihood and accomplished in ethical conduct. They guard the sense doors, have mindfulness and situational awareness, and are content. 

And\marginnote{1.11.1} how is a mendicant accomplished in ethics? It’s when a mendicant gives up killing living creatures. They renounce the rod and the sword. They’re scrupulous and kind, living full of compassion for all living beings. … 

This\marginnote{1.12.1{-}1.27} pertains to their ethics. 

There\marginnote{1.28.1} are some ascetics and brahmins who, while enjoying food given in faith, still earn a living by unworthy branches of knowledge, by wrong livelihood. This includes rites for propitiation, for granting wishes, for ghosts, for the earth, for rain, for property settlement, and for preparing and consecrating house sites, and rites involving rinsing and bathing, and oblations. It also includes administering emetics, purgatives, expectorants, and phlegmagogues; administering ear-oils, eye restoratives, nasal medicine, ointments, and counter-ointments; surgery with needle and scalpel, treating children, prescribing root medicines, and binding on herbs. They refrain from such unworthy branches of knowledge, such wrong livelihood. … This pertains to their ethics. 

A\marginnote{1.29.1} mendicant thus accomplished in ethics sees no danger in any quarter in regards to their ethical restraint. It’s like a king who has defeated his enemies. He sees no danger from his foes in any quarter. A mendicant thus accomplished in ethics sees no danger in any quarter in regards to their ethical restraint. When they have this entire spectrum of noble ethics, they experience a blameless happiness inside themselves. That’s how a mendicant is accomplished in ethics. 

This\marginnote{1.30.1} is that noble spectrum of ethics that the Buddha praised. But there is still more to be done.” 

“It’s\marginnote{1.30.3} incredible, Master Ānanda, it’s amazing, This noble spectrum of ethics is complete, not lacking anything! Such a complete spectrum of ethics cannot be seen among the other ascetics and brahmins. Were other ascetics and brahmins to see such a complete spectrum of noble ethics in themselves, they’d be delighted with just that much: ‘At this point it’s enough; at this point our work is done. We’ve reached the goal of our ascetic life. There is nothing more to be done.’ And yet you say: ‘But there is still more to be done.’ 

\section*{2. The Spectrum of Immersion }

But\marginnote{2.1.1} what, Master Ānanda, was that noble spectrum of immersion that the Buddha praised?” 

“And\marginnote{2.2.1} how, student, does a mendicant guard the sense doors? When a mendicant sees a sight with their eyes, they don’t get caught up in the features and details. If the faculty of sight were left unrestrained, bad unskillful qualities of desire and aversion would become overwhelming. For this reason, they practice restraint, protecting the faculty of sight, and achieving its restraint. When they hear a sound with their ears … When they smell an odor with their nose … When they taste a flavor with their tongue … When they feel a touch with their body … When they know a thought with their mind, they don’t get caught up in the features and details. If the faculty of mind were left unrestrained, bad unskillful qualities of desire and aversion would become overwhelming. For this reason, they practice restraint, protecting the faculty of mind, and achieving its restraint. When they have this noble sense restraint, they experience an unsullied bliss inside themselves. That’s how a mendicant guards the sense doors. 

And\marginnote{2.3.1} how does a mendicant have mindfulness and situational awareness? It’s when a mendicant acts with situational awareness when going out and coming back; when looking ahead and aside; when bending and extending the limbs; when bearing the outer robe, bowl and robes; when eating, drinking, chewing, and tasting; when urinating and defecating; when walking, standing, sitting, sleeping, waking, speaking, and keeping silent. That’s how a mendicant has mindfulness and situational awareness. 

And\marginnote{2.4.1} how is a mendicant content? It’s when a mendicant is content with robes to look after the body and almsfood to look after the belly. Wherever they go, they set out taking only these things. They’re like a bird: wherever it flies, wings are its only burden. In the same way, a mendicant is content with robes to look after the body and almsfood to look after the belly. Wherever they go, they set out taking only these things. That’s how a mendicant is content. 

When\marginnote{2.5.1} they have this noble spectrum of ethics, this noble sense restraint, this noble mindfulness and situational awareness, and this noble contentment, they frequent a secluded lodging—a wilderness, the root of a tree, a hill, a ravine, a mountain cave, a charnel ground, a forest, the open air, a heap of straw. After the meal, they return from almsround, sit down cross-legged with their body straight, and establish mindfulness right there. 

Giving\marginnote{2.6.1} up desire for the world, they meditate with a heart rid of desire, cleansing the mind of desire. Giving up ill will and malevolence, they meditate with a mind rid of ill will, full of compassion for all living beings, cleansing the mind of ill will. Giving up dullness and drowsiness, they meditate with a mind rid of dullness and drowsiness, perceiving light, mindful and aware, cleansing the mind of dullness and drowsiness. Giving up restlessness and remorse, they meditate without restlessness, their mind peaceful inside, cleansing the mind of restlessness and remorse. Giving up doubt, they meditate having gone beyond doubt, not undecided about skillful qualities, cleansing the mind of doubt. 

Suppose\marginnote{2.7.1} a man who has gotten into debt were to apply himself to work, and his efforts proved successful. He would pay off the original loan and have enough left over to support his partner. Thinking about this, he’d be filled with joy and happiness. 

Suppose\marginnote{2.8.1} there was a person who was sick, suffering, gravely ill. They’d lose their appetite and get physically weak. But after some time they’d recover from that illness, and regain their appetite and their strength. Thinking about this, they’d be filled with joy and happiness. 

Suppose\marginnote{2.9.1} a person was imprisoned in a jail. But after some time they were released from jail, safe and sound, with no loss of wealth. Thinking about this, they’d be filled with joy and happiness. 

Suppose\marginnote{2.10.1} a person was a bondservant. They belonged to someone else and were unable to go where they wish. But after some time they’d be freed from servitude and become their own master, an emancipated individual able to go where they wish. Thinking about this, they’d be filled with joy and happiness. 

Suppose\marginnote{2.11.1} there was a person with wealth and property who was traveling along a desert road, which was perilous, with nothing to eat. But after some time they crossed over the desert safely, arriving within a village, a sanctuary free of peril. Thinking about this, they’d be filled with joy and happiness. 

In\marginnote{2.12.1} the same way, as long as these five hindrances are not given up inside themselves, a mendicant regards them as a debt, a disease, a prison, slavery, and a desert crossing. 

But\marginnote{2.12.2} when these five hindrances are given up inside themselves, a mendicant regards this as freedom from debt, good health, release from prison, emancipation, and sanctuary. 

Seeing\marginnote{2.12.4} that the hindrances have been given up in them, joy springs up. Being joyful, rapture springs up. When the mind is full of rapture, the body becomes tranquil. When the body is tranquil, they feel bliss. And when blissful, the mind becomes immersed. 

Quite\marginnote{2.13.1} secluded from sensual pleasures, secluded from unskillful qualities, they enter and remain in the first absorption, which has the rapture and bliss born of seclusion, while placing the mind and keeping it connected. They drench, steep, fill, and spread their body with rapture and bliss born of seclusion. There’s no part of the body that’s not spread with rapture and bliss born of seclusion. 

It’s\marginnote{2.14.1} like when a deft bathroom attendant or their apprentice pours bath powder into a bronze dish, sprinkling it little by little with water. They knead it until the ball of bath powder is soaked and saturated with moisture, spread through inside and out; yet no moisture oozes out. 

In\marginnote{2.14.2} the same way, a mendicant drenches, steeps, fills, and spreads their body with rapture and bliss born of seclusion. There’s no part of the body that’s not spread with rapture and bliss born of seclusion. This pertains to their immersion. 

Furthermore,\marginnote{2.15.1} as the placing of the mind and keeping it connected are stilled, a mendicant enters and remains in the second absorption, which has the rapture and bliss born of immersion, with internal clarity and confidence, and unified mind, without placing the mind and keeping it connected. They drench, steep, fill, and spread their body with rapture and bliss born of immersion. There’s no part of the body that’s not spread with rapture and bliss born of immersion. 

It’s\marginnote{2.16.1} like a deep lake fed by spring water. There’s no inlet to the east, west, north, or south, and no rainfall to replenish it from time to time. But the stream of cool water welling up in the lake drenches, steeps, fills, and spreads throughout the lake. There’s no part of the lake that’s not spread through with cool water. 

In\marginnote{2.16.2} the same way, a mendicant drenches, steeps, fills, and spreads their body with rapture and bliss born of immersion. There’s no part of the body that’s not spread with rapture and bliss born of immersion. This pertains to their immersion. 

Furthermore,\marginnote{2.17.1} with the fading away of rapture, a mendicant enters and remains in the third absorption, where they meditate with equanimity, mindful and aware, personally experiencing the bliss of which the noble ones declare, ‘Equanimous and mindful, one meditates in bliss.’ They drench, steep, fill, and spread their body with bliss free of rapture. There’s no part of the body that’s not spread with bliss free of rapture. 

It’s\marginnote{2.17.3} like a pool with blue water lilies, or pink or white lotuses. Some of them sprout and grow in the water without rising above it, thriving underwater. From the tip to the root they’re drenched, steeped, filled, and soaked with cool water. There’s no part of them that’s not soaked with cool water. 

In\marginnote{2.17.4} the same way, a mendicant drenches, steeps, fills, and spreads their body with bliss free of rapture. There’s no part of the body that’s not spread with bliss free of rapture. This pertains to their immersion. 

Furthermore,\marginnote{2.18.1} giving up pleasure and pain, and ending former happiness and sadness, a mendicant enters and remains in the fourth absorption, without pleasure or pain, with pure equanimity and mindfulness. They sit spreading their body through with pure bright mind. There’s no part of the body that’s not spread with pure bright mind. 

It’s\marginnote{2.18.4} like someone sitting wrapped from head to foot with white cloth. There’s no part of the body that’s not spread over with white cloth. 

In\marginnote{2.18.5} the same way, a mendicant sits spreading their body through with pure bright mind. There's no part of their body that's not spread with pure bright mind. This pertains to their immersion. 

This\marginnote{2.19.1} is that noble spectrum of immersion that the Buddha praised. But there is still more to be done.” 

“It’s\marginnote{2.19.3} incredible, Master Ānanda, it’s amazing! This noble spectrum of immersion is complete, not lacking anything! Such a complete spectrum of immersion cannot be seen among the other ascetics and brahmins. Were other ascetics and brahmins to see such a complete spectrum of noble immersion in themselves, they’d be delighted with just that much: ‘At this point it’s enough; at this point our work is done. We’ve reached the goal of our ascetic life. There is nothing more to be done.’ And yet you say: ‘But there is still more to be done.’ 

\section*{3. The Spectrum of Wisdom }

But\marginnote{2.20.1} what, Master Ānanda, was that noble spectrum of wisdom that the Buddha praised?” 

“When\marginnote{2.21.1} their mind has become immersed in \textsanskrit{samādhi} like this—purified, bright, flawless, rid of corruptions, pliable, workable, steady, and imperturbable—they extend it and project it toward knowledge and vision. They understand: ‘This body of mine is physical. It’s made up of the four primary elements, produced by mother and father, built up from rice and porridge, liable to impermanence, to wearing away and erosion, to breaking up and destruction. And this consciousness of mine is attached to it, tied to it.’ 

Suppose\marginnote{2.22.1} there was a beryl gem that was naturally beautiful, eight-faceted, well-worked, transparent, clear, and unclouded, endowed with all good qualities. And it was strung with a thread of blue, yellow, red, white, or golden brown. And someone with good eyesight were to take it in their hand and examine it: ‘This beryl gem is naturally beautiful, eight-faceted, well-worked, transparent, clear, and unclouded, endowed with all good qualities. And it’s strung with a thread of blue, yellow, red, white, or golden brown.’ 

In\marginnote{2.22.3} the same way, when their mind has become immersed in \textsanskrit{samādhi} like this—purified, bright, flawless, rid of corruptions, pliable, workable, steady, and imperturbable—they extend it and project it toward knowledge and vision. This pertains to their wisdom. 

When\marginnote{2.23.1} their mind has become immersed in \textsanskrit{samādhi} like this—purified, bright, flawless, rid of corruptions, pliable, workable, steady, and imperturbable—they extend it and project it toward the creation of a mind-made body. From this body they create another body, physical, mind-made, complete in all its various parts, not deficient in any faculty. 

Suppose\marginnote{2.24.1} a person was to draw a reed out from its sheath. They’d think: ‘This is the reed, this is the sheath. The reed and the sheath are different things. The reed has been drawn out from the sheath.’ Or suppose a person was to draw a sword out from its scabbard. They’d think: ‘This is the sword, this is the scabbard. The sword and the scabbard are different things. The sword has been drawn out from the scabbard.’ Or suppose a person was to draw a snake out from its slough. They’d think: ‘This is the snake, this is the slough. The snake and the slough are different things. The snake has been drawn out from the slough.’ 

In\marginnote{2.24.10} the same way, when their mind has become immersed in \textsanskrit{samādhi} like this—purified, bright, flawless, rid of corruptions, pliable, workable, steady, and imperturbable—they extend it and project it toward the creation of a mind-made body. This pertains to their wisdom. 

When\marginnote{2.25.1} their mind has become immersed in \textsanskrit{samādhi} like this—purified, bright, flawless, rid of corruptions, pliable, workable, steady, and imperturbable—they extend it and project it toward psychic power. They wield the many kinds of psychic power: multiplying themselves and becoming one again; going unimpeded through a wall, a rampart, or a mountain as if through space; diving in and out of the earth as if it were water; walking on water as if it were earth; flying cross-legged through the sky like a bird; touching and stroking with the hand the sun and moon, so mighty and powerful; controlling the body as far as the \textsanskrit{Brahmā} realm. 

Suppose\marginnote{2.26.1} an expert potter or their apprentice had some well-prepared clay. They could produce any kind of pot that they like. Or suppose an expert ivory-carver or their apprentice had some well-prepared ivory. They could produce any kind of ivory item that they like. Or suppose an expert goldsmith or their apprentice had some well-prepared gold. They could produce any kind of gold item that they like. 

In\marginnote{2.26.4} the same way, when their mind has become immersed in \textsanskrit{samādhi} like this—purified, bright, flawless, rid of corruptions, pliable, workable, steady, and imperturbable—they extend it and project it toward psychic power. This pertains to their wisdom. 

When\marginnote{2.27.1} their mind has become immersed in \textsanskrit{samādhi} like this—purified, bright, flawless, rid of corruptions, pliable, workable, steady, and imperturbable—they extend it and project it toward clairaudience. With clairaudience that is purified and superhuman, they hear both kinds of sounds, human and divine, whether near or far. Suppose there was a person traveling along the road. They’d hear the sound of drums, clay drums, horns, kettledrums, and tom-toms. They’d think: ‘That’s the sound of drums,’ and ‘that’s the sound of clay drums,’ and ‘that’s the sound of horns, kettledrums, and tom-toms.’ 

In\marginnote{2.27.4} the same way, when their mind has become immersed in \textsanskrit{samādhi} like this—purified, bright, flawless, rid of corruptions, pliable, workable, steady, and imperturbable—they extend it and project it toward clairaudience. This pertains to their wisdom. 

When\marginnote{2.28.1} their mind has become immersed in \textsanskrit{samādhi} like this—purified, bright, flawless, rid of corruptions, pliable, workable, steady, and imperturbable—they extend it and project it toward comprehending the minds of others. They understand mind with greed as ‘mind with greed’, and mind without greed as ‘mind without greed’. They understand mind with hate … mind without hate … mind with delusion … mind without delusion … constricted mind … scattered mind … expansive mind … unexpansive mind … mind that is not supreme … mind that is supreme … immersed mind … unimmersed mind … freed mind … They understand unfreed mind as ‘unfreed mind’. 

Suppose\marginnote{2.29.1} there was a woman or man who was young, youthful, and fond of adornments, and they check their own reflection in a clean bright mirror or a clear bowl of water. If they had a spot they’d know ‘I have a spot,’ and if they had no spots they’d know ‘I have no spots.’ 

In\marginnote{2.29.2} the same way, when their mind has become immersed in \textsanskrit{samādhi} like this—purified, bright, flawless, rid of corruptions, pliable, workable, steady, and imperturbable—they extend it and project it toward comprehending the minds of others. This pertains to their wisdom. 

When\marginnote{2.30.1} their mind has become immersed in \textsanskrit{samādhi} like this—purified, bright, flawless, rid of corruptions, pliable, workable, steady, and imperturbable—they extend it and project it toward recollection of past lives. They recollect many kinds of past lives, that is, one, two, three, four, five, ten, twenty, thirty, forty, fifty, a hundred, a thousand, a hundred thousand rebirths; many eons of the world contracting, many eons of the world expanding, many eons of the world contracting and expanding. They remember: ‘There, I was named this, my clan was that, I looked like this, and that was my food. This was how I felt pleasure and pain, and that was how my life ended. When I passed away from that place I was reborn somewhere else. There, too, I was named this, my clan was that, I looked like this, and that was my food. This was how I felt pleasure and pain, and that was how my life ended. When I passed away from that place I was reborn here.’ And so they recollect their many kinds of past lives, with features and details. 

Suppose\marginnote{2.31.1} a person was to leave their home village and go to another village. From that village they’d go to yet another village. And from that village they’d return to their home village. They’d think: ‘I went from my home village to another village. There I stood like this, sat like that, spoke like this, or kept silent like that. From that village I went to yet another village. There too I stood like this, sat like that, spoke like this, or kept silent like that. And from that village I returned to my home village.’ 

In\marginnote{2.31.2} the same way, when their mind has become immersed in \textsanskrit{samādhi} like this—purified, bright, flawless, rid of corruptions, pliable, workable, steady, and imperturbable—they extend it and project it toward recollection of past lives. This pertains to their wisdom. 

When\marginnote{2.32.1} their mind has become immersed in \textsanskrit{samādhi} like this—purified, bright, flawless, rid of corruptions, pliable, workable, steady, and imperturbable—they extend it and project it toward knowledge of the death and rebirth of sentient beings. With clairvoyance that is purified and superhuman, they see sentient beings passing away and being reborn—inferior and superior, beautiful and ugly, in a good place or a bad place. They understand how sentient beings are reborn according to their deeds. ‘These dear beings did bad things by way of body, speech, and mind. They spoke ill of the noble ones; they had wrong view; and they chose to act out of that wrong view. When their body breaks up, after death, they’re reborn in a place of loss, a bad place, the underworld, hell. These dear beings, however, did good things by way of body, speech, and mind. They never spoke ill of the noble ones; they had right view; and they chose to act out of that right view. When their body breaks up, after death, they’re reborn in a good place, a heavenly realm.’ And so, with clairvoyance that is purified and superhuman, they see sentient beings passing away and being reborn—inferior and superior, beautiful and ugly, in a good place or a bad place. They understand how sentient beings are reborn according to their deeds. 

Suppose\marginnote{2.33.1} there was a stilt longhouse at the central square. A person with good eyesight standing there might see people entering and leaving a house, walking along the streets and paths, and sitting at the central square. They’d think: ‘These are people entering and leaving a house, walking along the streets and paths, and sitting at the central square.’ 

In\marginnote{2.33.2} the same way, when their mind has become immersed in \textsanskrit{samādhi} like this—purified, bright, flawless, rid of corruptions, pliable, workable, steady, and imperturbable—they extend and project it toward knowledge of the death and rebirth of sentient beings. This pertains to their wisdom. 

When\marginnote{2.34.1} their mind has become immersed in \textsanskrit{samādhi} like this—purified, bright, flawless, rid of corruptions, pliable, workable, steady, and imperturbable—they extend it and project it toward knowledge of the ending of defilements. They truly understand: ‘This is suffering’ … ‘This is the origin of suffering’ … ‘This is the cessation of suffering’ … ‘This is the practice that leads to the cessation of suffering’. They truly understand: ‘These are defilements’ … ‘This is the origin of defilements’ … ‘This is the cessation of defilements’ … ‘This is the practice that leads to the cessation of defilements’. Knowing and seeing like this, their mind is freed from the defilements of sensuality, desire to be reborn, and ignorance. When they’re freed, they know they’re freed. 

They\marginnote{2.35.3} understand: ‘Rebirth is ended, the spiritual journey has been completed, what had to be done has been done, there is no return to any state of existence.’ 

Suppose\marginnote{2.36.1} that in a mountain glen there was a lake that was transparent, clear, and unclouded. A person with good eyesight standing on the bank would see the clams and mussels, and pebbles and gravel, and schools of fish swimming about or staying still. They’d think: ‘This lake is transparent, clear, and unclouded. And here are the clams and mussels, and pebbles and gravel, and schools of fish swimming about or staying still.’ 

In\marginnote{2.36.2} the same way, when their mind has become immersed in \textsanskrit{samādhi} like this—purified, bright, flawless, rid of corruptions, pliable, workable, steady, and imperturbable—they extend it and project it toward knowledge of the ending of defilements. This pertains to their wisdom. 

This\marginnote{2.37.1} is that noble spectrum of wisdom that the Buddha praised. And there is nothing more to be done.” 

“It’s\marginnote{2.37.3} incredible, Master Ānanda, it’s amazing! This noble spectrum of wisdom is complete, not lacking anything! Such a complete spectrum of wisdom cannot be seen among the other ascetics and brahmins. And there is nothing more to be done. Excellent, Master Ānanda! Excellent! As if he were righting the overturned, or revealing the hidden, or pointing out the path to the lost, or lighting a lamp in the dark so people with good eyes can see what’s there, Master Ānanda has made the teaching clear in many ways. I go for refuge to Master Gotama, to the teaching, and to the mendicant \textsanskrit{Saṅgha}. From this day forth, may Master Ānanda remember me as a lay follower who has gone for refuge for life.” 

%
\chapter*{{\suttatitleacronym DN 11}{\suttatitletranslation With Kevaddha }{\suttatitleroot Kevaṭṭasutta}}
\addcontentsline{toc}{chapter}{\tocacronym{DN 11} \toctranslation{With Kevaddha } \tocroot{Kevaṭṭasutta}}
\markboth{With Kevaddha }{Kevaṭṭasutta}
\extramarks{DN 11}{DN 11}

\scevam{So\marginnote{1.1} I have heard. }At one time the Buddha was staying near \textsanskrit{Nālandā} in \textsanskrit{Pāvārika}’s mango grove. 

Then\marginnote{1.3} the householder Kevaddha went up to the Buddha, bowed, sat down to one side, and said to him, “Sir, this \textsanskrit{Nāḷandā} is successful and prosperous and full of people. Sir, please direct a mendicant to perform a demonstration of superhuman psychic power. Then \textsanskrit{Nāḷandā} will become even more devoted to the Buddha!” 

When\marginnote{1.7} he said this, the Buddha said, “Kevaddha, I do not teach the mendicants like this: ‘Come now, mendicants, perform a demonstration of superhuman psychic power for the white-clothed laypeople.’” 

For\marginnote{2.1} a second time, Kevaddha made the same request, and the Buddha gave the same answer. 

For\marginnote{3.1} a third time, Kevaddha made the same request, and the Buddha said the following. 

\section*{1. The Demonstration of Psychic Power }

“Kevaddha,\marginnote{3.8} there are three kinds of demonstration, which I declare having realized them with my own insight. What three? The demonstration of psychic power, the demonstration of revealing, and the demonstration of instruction. 

And\marginnote{4.1} what is the demonstration of psychic power? It’s a mendicant who wields the many kinds of psychic power: multiplying themselves and becoming one again; going unimpeded through a wall, a rampart, or a mountain as if through space; diving in and out of the earth as if it were water; walking on water as if it were earth; flying cross-legged through the sky like a bird; touching and stroking with the hand the sun and moon, so mighty and powerful; controlling the body as far as the \textsanskrit{Brahmā} realm. 

Then\marginnote{4.3} someone with faith and confidence sees that mendicant performing those superhuman feats. 

They\marginnote{4.4} tell someone else who lacks faith and confidence: ‘It’s incredible, it’s amazing! The ascetic has such psychic power and might! I saw him myself, performing all these superhuman feats!’ 

But\marginnote{5.1} the one lacking faith and confidence would say to them: ‘There’s a spell named \textsanskrit{Gandhārī}. Using that a mendicant can perform such superhuman feats.’ 

What\marginnote{5.4} do you think, Kevaddha? Wouldn’t someone lacking faith speak like that?” 

“They\marginnote{5.6} would, sir.” 

“Seeing\marginnote{5.7} this drawback in psychic power, I’m horrified, repelled, and disgusted by demonstrations of psychic power. 

\section*{2. The Demonstration of Revealing }

And\marginnote{6.1} what is the demonstration of revealing? In one case, someone reveals the mind, mentality, thoughts, and reflections of other beings and individuals: ‘This is what you’re thinking, such is your thought, and thus is your state of mind.’ 

Then\marginnote{6.4} someone with faith and confidence sees that mendicant revealing another person’s thoughts. They tell someone else who lacks faith and confidence: ‘It’s incredible, it’s amazing! The ascetic has such psychic power and might! I saw him myself, revealing the thoughts of another person!’ 

But\marginnote{7.1} the one lacking faith and confidence would say to them: ‘There’s a spell named \textsanskrit{Māṇikā}. Using that a mendicant can reveal another person’s thoughts.’ 

What\marginnote{7.5} do you think, Kevaddha? Wouldn’t someone lacking faith speak like that?” 

“They\marginnote{7.7} would, sir.” 

“Seeing\marginnote{7.8} this drawback in revealing, I’m horrified, repelled, and disgusted by demonstrations of revealing. 

\section*{3. The Demonstration of Instruction }

And\marginnote{8.1} what is the demonstration of instruction? It’s when a mendicant instructs others like this: ‘Think like this, not like that. Focus your mind like this, not like that. Give up this, and live having achieved that.’ This is called the demonstration of instruction. 

Furthermore,\marginnote{9{-}66.1} a Realized One arises in the world … That’s how a mendicant is accomplished in ethics. … They enter and remain in the first absorption … This is called the demonstration of instruction. 

They\marginnote{9{-}66.5} enter and remain in the second absorption … third absorption … fourth absorption. This too is called the demonstration of instruction. 

They\marginnote{9{-}66.9} extend and project the mind toward knowledge and vision … This too is called the demonstration of instruction. 

They\marginnote{9{-}66.11} understand: ‘… there is no return to any state of existence.’ This too is called the demonstration of instruction. 

These,\marginnote{67.1} Kevaddha, are the three kinds of demonstration, which I declare having realized them with my own insight. 

\section*{4. On the Mendicant In Search of the Cessation of Being }

Once\marginnote{67.3} it so happened, Kevaddha, that a mendicant in this very \textsanskrit{Saṅgha} had the following thought, ‘Where do these four primary elements cease without anything left over, namely, the elements of earth, water, fire, and air?’ 

Then\marginnote{68.1} that mendicant attained a state of immersion such that a path to the gods appeared. Then he approached the Gods of the Four Great Kings and said, ‘Reverends, where do these four primary elements cease without anything left over, namely, the elements of earth, water, fire, and air?’ 

When\marginnote{68.4} he said this, those gods said to him, ‘Mendicant, we too do not know this. But the Four Great Kings are our superiors. They might know.’ 

Then\marginnote{69.1} he approached the Four Great Kings and asked the same question. But they also said to him, ‘Mendicant, we too do not know this. But the gods of the Thirty-Three … Sakka, lord of gods … the gods of \textsanskrit{Yāmā} … the god named \textsanskrit{Suyāma} … the Joyful gods … the god named Santussita … the gods who delight in creation … the god named Sunimmita … the gods who control the creation of others … the god named \textsanskrit{Vasavattī} … the gods of \textsanskrit{Brahmā}’s Host. They might know.’ 

Then\marginnote{80.1} that mendicant attained a state of immersion such that a path to \textsanskrit{Brahmā} appeared. Then he approached those gods and said, ‘Reverends, where do these four primary elements cease without anything left over, namely, the elements of earth, water, fire, and air?’ But they also said to him, ‘Mendicant, we too do not know this. But there is \textsanskrit{Brahmā}, the Great \textsanskrit{Brahmā}, the Undefeated, the Champion, the Universal Seer, the Wielder of Power, the Lord God, the Maker, the Author, the First, the Begetter, the Controller, the Father of those who have been born and those yet to be born. He is our superior. He might know.’ 

‘But\marginnote{80.10} reverends, where is that \textsanskrit{Brahmā} now?’ ‘We also don’t know where he is or what way he lies. But by the signs that are seen—light arising and radiance appearing—we know that \textsanskrit{Brahmā} will appear. For this is the precursor for the appearance of \textsanskrit{Brahmā}, namely light arising and radiance appearing.’ Not long afterwards, the Great \textsanskrit{Brahmā} appeared. 

Then\marginnote{81.2} that mendicant approached the Great \textsanskrit{Brahmā} and said to him, ‘Reverend, where do these four primary elements cease without anything left over, namely, the elements of earth, water, fire, and air?’ The Great \textsanskrit{Brahmā} said to him, ‘I am \textsanskrit{Brahmā}, the Great \textsanskrit{Brahmā}, the Undefeated, the Champion, the Universal Seer, the Wielder of Power, the Lord God, the Maker, the Author, the First, the Begetter, the Controller, the Father of those who have been born and those yet to be born.’ 

For\marginnote{82.1} a second time, that mendicant said to the Great \textsanskrit{Brahmā}, ‘Reverend, I am not asking you whether you are \textsanskrit{Brahmā}, the Great \textsanskrit{Brahmā}, the Undefeated, the Champion, the Universal Seer, the Wielder of Power, the Lord God, the Maker, the Author, the First, the Begetter, the Controller, the Father of those who have been born and those yet to be born. I am asking where these four primary elements cease without anything left over.’ 

For\marginnote{82.6} a second time, the Great \textsanskrit{Brahmā} said to him, ‘I am \textsanskrit{Brahmā}, the Great \textsanskrit{Brahmā}, the Undefeated, the Champion, the Universal Seer, the Wielder of Power, the Lord God, the Maker, the Author, the First, the Begetter, the Controller, the Father of those who have been born and those yet to be born.’ For a third time, that mendicant said to the Great \textsanskrit{Brahmā}, ‘Reverend, I am not asking you whether you are \textsanskrit{Brahmā}, the Great \textsanskrit{Brahmā}, the Undefeated, the Champion, the Universal Seer, the Wielder of Power, the Lord God, the Maker, the Author, the First, the Begetter, the Controller, the Father of those who have been born and those yet to be born. I am asking where these four primary elements cease without anything left over.’ 

Then\marginnote{83.6} the Great \textsanskrit{Brahmā} took that mendicant by the arm, led him off to one side, and said to him, ‘Mendicant, these gods think that there is nothing at all that I don’t know and see and understand and realize. That’s why I didn’t answer in front of them. But I too do not know where these four primary elements cease with nothing left over. Therefore, mendicant, the misdeed is yours alone, the mistake is yours alone, in that you passed over the Buddha and searched elsewhere for an answer to this question. Mendicant, go to the Buddha and ask him this question. You should remember it in line with his answer.’ 

Then\marginnote{84.1} that mendicant, as easily as a strong person would extend or contract their arm, vanished from the \textsanskrit{Brahmā} realm and reappeared in front of me. Then he bowed, sat down to one side, and said to me, ‘Sir, where do these four primary elements cease without anything left over, namely, the elements of earth, water, fire, and air?’ 

\subsection*{4.1. The Simile of the Land-Spotting Bird }

When\marginnote{85.1} he said this, I said to him: 

‘Once\marginnote{85.2} upon a time, mendicant, some sea-merchants set sail for the ocean deeps, taking with them a land-spotting bird. When their ship was out of sight of land, they released the bird. It flew right away to the east, the west, the north, the south, upwards, and in-between. If it saw land on any side, it went there and stayed. But if it saw no land on any side it returned to the ship. 

In\marginnote{85.7} the same way, after failing to get an answer to this question even after searching as far as the \textsanskrit{Brahmā} realm, you’ve returned to me. Mendicant, this is not how the question should be asked: “Sir, where do these four primary elements cease without anything left over, namely, the elements of earth, water, fire, and air?” 

This\marginnote{85.10} is how the question should be asked: 

\begin{verse}%
“Where\marginnote{85.11} do water and earth, \\
fire and air find no footing; \\
where do long and short, \\
fine and coarse, beautiful and ugly; \\
where do name and form \\
cease with nothing left over?” 

%
\end{verse}

And\marginnote{85.17} the answer to that is: 

\begin{verse}%
“Consciousness\marginnote{85.18} that’s invisible, \\
infinite, entirely given up: \\
that’s where water and earth, \\
fire and air find no footing. 

And\marginnote{85.22} that’s where long and short, \\
fine and coarse, beautiful and ugly—\\
that’s where name and form \\
cease with nothing left over. \\
With the cessation of consciousness, \\
that’s where they cease.”’” 

%
\end{verse}

That\marginnote{85.28} is what the Buddha said. Satisfied, the householder Kevaddha was happy with what the Buddha said. 

%
\chapter*{{\suttatitleacronym DN 12}{\suttatitletranslation With Lohicca }{\suttatitleroot Lohiccasutta}}
\addcontentsline{toc}{chapter}{\tocacronym{DN 12} \toctranslation{With Lohicca } \tocroot{Lohiccasutta}}
\markboth{With Lohicca }{Lohiccasutta}
\extramarks{DN 12}{DN 12}

\scevam{So\marginnote{1.1} I have heard. }At one time the Buddha was wandering in the land of the Kosalans together with a large \textsanskrit{Saṅgha} of five hundred mendicants when he arrived at \textsanskrit{Sālavatikā}. 

Now\marginnote{1.3} at that time the brahmin Lohicca was living in \textsanskrit{Sālavatikā}. It was a crown property given by King Pasenadi of Kosala, teeming with living creatures, full of hay, wood, water, and grain, a royal endowment of the highest quality. 

Now\marginnote{2.1} at that time Lohicca had the following harmful misconception: “Should an ascetic or brahmin achieve some skillful quality, they ought not inform anyone else. For what can one person do for another? Suppose someone cut off an old bond, only to create another new bond. That’s the consequence of such a wicked, greedy deed, I say. For what can one person do for another?” 

Lohicca\marginnote{3.1} heard: 

“It\marginnote{3.2} seems the ascetic Gotama—a Sakyan, gone forth from a Sakyan family—has arrived at \textsanskrit{Sālavatikā}, together with a large \textsanskrit{Saṅgha} of five hundred mendicants. He has this good reputation: ‘That Blessed One is perfected, a fully awakened Buddha, accomplished in knowledge and conduct, holy, knower of the world, supreme guide for those who wish to train, teacher of gods and humans, awakened, blessed.’ He has realized with his own insight this world—with its gods, \textsanskrit{Māras} and \textsanskrit{Brahmās}, this population with its ascetics and brahmins, gods and humans—and he makes it known to others. He teaches Dhamma that’s good in the beginning, good in the middle, and good in the end, meaningful and well-phrased. And he reveals a spiritual practice that’s entirely full and pure. It’s good to see such perfected ones.” 

Then\marginnote{4.1} Lohicca addressed his barber Rosika, “Here, dear Rosika, go to the ascetic Gotama and in my name bow with your head to his feet. Ask him if he is healthy and well, nimble, strong, and living comfortably. And then ask him whether he, together with the mendicant \textsanskrit{Saṅgha}, might please accept tomorrow’s meal from the brahmin Lohicca.” 

“Yes,\marginnote{5.1} sir,” Rosika replied. He did as he was asked, and the Buddha consented in silence. 

Then,\marginnote{6.1} knowing that the Buddha had consented, Rosika got up from his seat, went to Lohicca, and said to him, “I gave the Buddha your message, and he accepted.” 

And\marginnote{7.1} when the night had passed Lohicca had a variety of delicious foods prepared in his own home. Then he had the Buddha informed of the time, saying, “Here, dear Rosika, go to the ascetic Gotama and announce the time, saying: ‘It’s time, Master Gotama, the meal is ready.’” 

“Yes,\marginnote{7.4} sir,” Rosika replied. He did as he was asked. 

Then\marginnote{7.7} the Buddha robed up in the morning and, taking his bowl and robe, went to \textsanskrit{Sālavatikā} together with the \textsanskrit{Saṅgha} of mendicants. Now, Rosika was following behind the Buddha, and told him of Lohicca’s views, adding, “Sir, please dissuade him from that harmful misconception.” 

“Hopefully\marginnote{8.10} that’ll happen, Rosika, hopefully that’ll happen.” 

Then\marginnote{9.1} the Buddha approached Lohicca’s home, where he sat on the seat spread out. Then Lohicca served and satisfied the mendicant \textsanskrit{Saṅgha} headed by the Buddha with his own hands with a variety of delicious foods. 

\section*{1. Questioning Lohicca }

When\marginnote{9.4} the Buddha had eaten and washed his hand and bowl, Lohicca took a low seat and sat to one side. 

The\marginnote{9.5} Buddha said to him, “Is it really true, Lohicca, that you have such a harmful misconception: ‘Should an ascetic or brahmin achieve some skillful quality, they ought not inform anyone else. For what can one person do for another? Suppose someone cut off an old bond, only to create another new bond. That’s the consequence of such a wicked, greedy deed, I say. For what can one person do for another?’” 

“Yes,\marginnote{9.10} Master Gotama.” 

“What\marginnote{10.1} do you think, Lohicca? Do you reside in \textsanskrit{Sālavatikā}?” 

“Yes,\marginnote{10.3} Master Gotama.” 

“Lohicca,\marginnote{10.4} suppose someone were to say: ‘The brahmin Lohicca reigns over \textsanskrit{Sālavatikā}. He alone should enjoy the revenues produced in \textsanskrit{Sālavatikā} and not share them with anyone else.’ Would the person who spoke like that make it difficult for those whose living depends on you or not?” 

“They\marginnote{10.8} would, Master Gotama.” 

“But\marginnote{10.9} is someone who creates difficulties for others acting kindly or unkindly?” 

“Unkindly,\marginnote{10.10} sir.” 

“But\marginnote{10.11} does an unkind person have love in their heart or hostility?” 

“Hostility,\marginnote{10.12} sir.” 

“And\marginnote{10.13} when the heart is full of hostility, is there right view or wrong view?” 

“Wrong\marginnote{10.14} view, Master Gotama.” 

“An\marginnote{10.15} individual with wrong view is reborn in one of two places, I say: hell or the animal realm. 

What\marginnote{11.1} do you think, Lohicca? Does King Pasenadi reign over \textsanskrit{Kāsī} and Kosala?” 

“Yes,\marginnote{11.3} Master Gotama.” 

“Lohicca,\marginnote{11.4} suppose someone were to say: ‘King Pasenadi reigns over \textsanskrit{Kāsī} and Kosala. He alone should enjoy the revenues produced in \textsanskrit{Kāsī} and Kosala and not share them with anyone else.’ Would the person who spoke like that make it difficult for yourself and others whose living depends on King Pasenadi or not?” 

“They\marginnote{11.8} would, Master Gotama.” 

“But\marginnote{11.9} is someone who creates difficulties for others acting kindly or unkindly?” 

“Unkindly,\marginnote{11.10} sir.” 

“But\marginnote{11.11} does an unkind person have love in their heart or hostility?” 

“Hostility,\marginnote{11.12} sir.” 

“And\marginnote{11.13} when the heart is full of hostility, is there right view or wrong view?” 

“Wrong\marginnote{11.14} view, Master Gotama.” 

“An\marginnote{11.15} individual with wrong view is reborn in one of two places, I say: hell or the animal realm. 

So\marginnote{12.1} it seems, Lohicca, that should someone say such a thing either of Lohicca or of King Pasenadi, that is wrong view. 

In\marginnote{13.3} the same way, suppose someone were to say: ‘Should an ascetic or brahmin achieve some skillful quality, they ought not inform anyone else. For what can one person do for another? Suppose someone cut off an old bond, only to create another new bond. That’s the consequence of such a wicked, greedy deed, I say. For what can one person do for another?’ 

Now,\marginnote{13.7} there are gentlemen who, relying on the teaching and training proclaimed by the Realized One, achieve a high distinction such as the following: they realize the fruit of stream-entry, the fruit of once-return, the fruit of non-return, or the fruit of perfection. And in addition, there are those who ripen the seeds for rebirth in a heavenly state. The person who spoke like that makes it difficult for them. They’re acting unkindly, their heart is full of hostility, and they have wrong view. An individual with wrong view is reborn in one of two places, I say: hell or the animal realm. 

\section*{2. Three Teachers Who Deserve to Be Reprimanded }

Lohicca,\marginnote{16.1} there are three kinds of teachers in the world who deserve to be reprimanded. When someone reprimands such teachers, the reprimand is true, substantive, legitimate, and blameless. What three? 

Firstly,\marginnote{16.4} take a teacher who has not reached the goal of the ascetic life for which they went forth from the lay life to homelessness. They teach their disciples: ‘This is for your welfare. This is for your happiness.’ But their disciples don’t want to listen. They don’t pay attention or apply their minds to understand. They proceed having turned away from the teacher’s instruction. That teacher deserves to be reprimanded: ‘Venerable, you haven’t reached the goal of the ascetic life; and when you teach disciples they proceed having turned away from the teacher’s instruction. It’s like a man who makes advances on a woman though she pulls away, or embraces her though she turns her back. That’s the consequence of such a wicked, greedy deed, I say. For what can one do for another?’ This is the first kind of teacher who deserves to be reprimanded. 

Furthermore,\marginnote{17.1} take a teacher who has not reached the goal of the ascetic life for which they went forth from the lay life to homelessness. They teach their disciples: ‘This is for your welfare. This is for your happiness.’ Their disciples do want to listen. They pay attention and apply their minds to understand. They don’t proceed having turned away from the teacher’s instruction. That teacher deserves to be reprimanded: ‘Venerable, you haven’t reached the goal of the ascetic life; and when you teach disciples they don’t proceed having turned away from the teacher’s instruction. It’s like someone who abandons their own field and presumes to weed someone else’s field. That’s the consequence of such a wicked, greedy deed, I say. For what can one do for another?’ This is the second kind of teacher who deserves to be reprimanded. 

Furthermore,\marginnote{18.1} take a teacher who has reached the goal of the ascetic life for which they went forth from the lay life to homelessness. They teach their disciples: ‘This is for your welfare. This is for your happiness.’ But their disciples don’t want to listen. They don’t pay attention or apply their minds to understand. They proceed having turned away from the teacher’s instruction. That teacher deserves to be reprimanded: ‘Venerable, you have reached the goal of the ascetic life; yet when you teach disciples they proceed having turned away from the teacher’s instruction. Suppose someone cut off an old bond, only to create another new bond. That’s the consequence of such a wicked, greedy deed, I say. For what can one person do for another?’ This is the third kind of teacher who deserves to be reprimanded. 

These\marginnote{18.14} are the three kinds of teachers in the world who deserve to be reprimanded. When someone reprimands such teachers, the reprimand is true, substantive, legitimate, and blameless.” 

\section*{3. A Teacher Who Does Not Deserve to Be Reprimanded }

When\marginnote{19.1} he had spoken, Lohicca said to the Buddha, “But Master Gotama, is there a teacher in the world who does not deserve to be reprimanded?” 

“There\marginnote{19.3} is, Lohicca.” 

“But\marginnote{19.4} who is that teacher?” 

“It’s\marginnote{20{-}55.1} when a Realized One arises in the world, perfected, a fully awakened Buddha … That’s how a mendicant is accomplished in ethics. … They enter and remain in the first absorption … A teacher under whom a disciple achieves such a high distinction is one who does not deserve to be reprimanded. When someone reprimands such a teacher, the reprimand is false, baseless, illegitimate, and blameworthy. 

They\marginnote{56{-}62.1} enter and remain in the second absorption … third absorption … fourth absorption. A teacher under whom a disciple achieves such a high distinction is one who does not deserve to be reprimanded. … 

They\marginnote{63{-}77.1} extend and project the mind toward knowledge and vision … A teacher under whom a disciple achieves such a high distinction is one who does not deserve to be reprimanded. … 

They\marginnote{63{-}77.4} understand: ‘… there is no return to any state of existence.’ A teacher under whom a disciple achieves such a high distinction is one who does not deserve to be reprimanded. When someone reprimands such a teacher, the reprimand is false, baseless, illegitimate, and blameworthy.” 

When\marginnote{78.1} he had spoken, Lohicca said to the Buddha: 

“Suppose,\marginnote{78.2} Master Gotama, a person was on the verge of falling off a cliff, and another person were to grab them by the hair, pull them up, and place them on firm ground. In the same way, when I was falling off a cliff Master Gotama pulled me up and placed me on safe ground. 

Excellent,\marginnote{78.4} Master Gotama! Excellent! As if he were righting the overturned, or revealing the hidden, or pointing out the path to the lost, or lighting a lamp in the dark so people with good eyes can see what’s there, Master Gotama has made the Teaching clear in many ways. I go for refuge to Master Gotama, to the teaching, and to the mendicant \textsanskrit{Saṅgha}. From this day forth, may Master Gotama remember me as a lay follower who has gone for refuge for life.” 

%
\chapter*{{\suttatitleacronym DN 13}{\suttatitletranslation The Three Knowledges }{\suttatitleroot Tevijjasutta}}
\addcontentsline{toc}{chapter}{\tocacronym{DN 13} \toctranslation{The Three Knowledges } \tocroot{Tevijjasutta}}
\markboth{The Three Knowledges }{Tevijjasutta}
\extramarks{DN 13}{DN 13}

\scevam{So\marginnote{1.1} I have heard. }At one time the Buddha was wandering in the land of the Kosalans together with a large \textsanskrit{Saṅgha} of five hundred mendicants when he arrived at a village of the Kosalan brahmins named \textsanskrit{Manasākaṭa}. He stayed in a mango grove on a bank of the river \textsanskrit{Aciravatī} to the north of \textsanskrit{Manasākaṭa}. 

Now\marginnote{2.1} at that time several very well-known well-to-do brahmins were residing in \textsanskrit{Manasākaṭa}. They included the brahmins \textsanskrit{Caṅkī}, \textsanskrit{Tārukkha}, \textsanskrit{Pokkharasāti}, \textsanskrit{Jāṇussoṇi}, Todeyya, and others. 

Then\marginnote{3.1} as the students \textsanskrit{Vāseṭṭha} and \textsanskrit{Bhāradvāja} were going for a walk they began a discussion regarding the variety of paths. 

\textsanskrit{Vāseṭṭha}\marginnote{4.1} said this: “This is the only straight path, the direct route that leads someone who practices it to the company of \textsanskrit{Brahmā}; namely, that explained by the brahmin \textsanskrit{Pokkharasāti}.” 

\textsanskrit{Bhāradvāja}\marginnote{5.1} said this: “This is the only straight path, the direct route that leads someone who practices it to the company of \textsanskrit{Brahmā}; namely, that explained by the brahmin \textsanskrit{Tārukkha}.” 

But\marginnote{6.1} neither was able to persuade the other. So \textsanskrit{Vāseṭṭha} said to \textsanskrit{Bhāradvāja}, “\textsanskrit{Bhāradvāja}, the ascetic Gotama—a Sakyan, gone forth from a Sakyan family—is staying in a mango grove on a bank of the river \textsanskrit{Aciravatī} to the north of \textsanskrit{Manasākaṭa}. He has this good reputation: ‘That Blessed One is perfected, a fully awakened Buddha, accomplished in knowledge and conduct, holy, knower of the world, supreme guide for those who wish to train, teacher of gods and humans, awakened, blessed.’ Come, let’s go to see him and ask him about this matter. As he answers, so we’ll remember it.” 

“Yes,\marginnote{7.7} sir,” replied \textsanskrit{Bhāradvāja}. 

\section*{1. The Variety of Paths }

So\marginnote{8.1} they went to the Buddha and exchanged greetings with him. When the greetings and polite conversation were over, they sat down to one side and \textsanskrit{Vāseṭṭha} told him of their conversation, adding: “In this matter we have a dispute, a disagreement, a difference of opinion.” 

“So,\marginnote{9.1} \textsanskrit{Vāseṭṭha}, it seems that you say that the straight path is that explained by \textsanskrit{Pokkharasāti}, while \textsanskrit{Bhāradvāja} says that the straight path is that explained by \textsanskrit{Tārukkha}. But what exactly is your disagreement about?” 

“About\marginnote{10.1} the variety of paths, Master Gotama. Even though brahmins describe different paths—the Addhariya brahmins, the Tittiriya brahmins, the Chandoka brahmins, and the Bavhadija brahmins—all of them lead someone who practices them to the company of \textsanskrit{Brahmā}. 

It’s\marginnote{10.3} like a village or town that has many different roads nearby, yet all of them meet at that village. In the same way, even though brahmins describe different paths—the Addhariya brahmins, the Tittiriya brahmins, the Chandoka brahmins, and the Bavhadija brahmins—all of them lead someone who practices them to the company of \textsanskrit{Brahmā}.” 

\section*{2. Questioning \textsanskrit{Vāseṭṭha} }

“Do\marginnote{11.1} you say, ‘they lead someone’, \textsanskrit{Vāseṭṭha}?” 

“I\marginnote{11.2} do, Master Gotama.” 

“Do\marginnote{11.3} you say, ‘they lead someone’, \textsanskrit{Vāseṭṭha}?” 

“I\marginnote{11.4} do, Master Gotama.” 

“Do\marginnote{11.5} you say, ‘they lead someone’, \textsanskrit{Vāseṭṭha}?” 

“I\marginnote{11.6} do, Master Gotama.” 

“Well,\marginnote{12.1} of the brahmins who are proficient in the three Vedas, \textsanskrit{Vāseṭṭha}, is there even a single one who has seen \textsanskrit{Brahmā} with their own eyes?” 

“No,\marginnote{12.2} Master Gotama.” 

“Well,\marginnote{12.3} has even a single one of their teachers seen \textsanskrit{Brahmā} with their own eyes?” 

“No,\marginnote{12.4} Master Gotama.” 

“Well,\marginnote{12.5} has even a single one of their teachers’ teachers seen \textsanskrit{Brahmā} with their own eyes?” 

“No,\marginnote{12.6} Master Gotama.” 

“Well,\marginnote{12.7} has anyone back to the seventh generation of teachers seen \textsanskrit{Brahmā} with their own eyes?” 

“No,\marginnote{12.8} Master Gotama.” 

“Well,\marginnote{13.1} what of the ancient hermits of the brahmins, namely \textsanskrit{Aṭṭhaka}, \textsanskrit{Vāmaka}, \textsanskrit{Vāmadeva}, \textsanskrit{Vessāmitta}, Yamadaggi, \textsanskrit{Aṅgīrasa}, \textsanskrit{Bhāradvāja}, \textsanskrit{Vāseṭṭha}, Kassapa, and Bhagu? They were the authors and propagators of the hymns. Their hymnal was sung and propagated and compiled in ancient times; and these days, brahmins continue to sing and chant it, chanting what was chanted and teaching what was taught. Did they say: ‘We know and see where \textsanskrit{Brahmā} is or what way he lies’?” 

“No,\marginnote{13.4} Master Gotama.” 

“So\marginnote{14.1} it seems that none of the brahmins have seen \textsanskrit{Brahmā} with their own eyes, and not even the ancient hermits claimed to know where he is. Yet the brahmins proficient in the three Vedas say: ‘We teach the path to the company of that which we neither know nor see. This is the only straight path, the direct route that leads someone who practices it to the company of \textsanskrit{Brahmā}.’ 

What\marginnote{14.10} do you think, \textsanskrit{Vāseṭṭha}? This being so, doesn’t their statement turn out to have no demonstrable basis?” 

“Clearly\marginnote{14.12} that’s the case, Master Gotama.” 

“Good,\marginnote{15.1} \textsanskrit{Vāseṭṭha}. For it is impossible that they should teach the path to that which they neither know nor see. 

Suppose\marginnote{15.2} there was a queue of blind men, each holding the one in front: the first one does not see, the middle one does not see, and the last one does not see. In the same way, it seems to me that the brahmins’ statement turns out to be comparable to a queue of blind men: the first one does not see, the middle one does not see, and the last one does not see. Their statement turns out to be a joke—mere words, void and hollow. 

What\marginnote{16.1} do you think, \textsanskrit{Vāseṭṭha}? Do the brahmins proficient in the three Vedas see the sun and moon just as other folk do? And do they pray to them and exalt them, following their course from where they rise to where they set with joined palms held in worship?” 

“Yes,\marginnote{16.3} Master Gotama.” 

“What\marginnote{17.1} do you think, \textsanskrit{Vāseṭṭha}? Though this is so, are the brahmins proficient in the three Vedas able to teach the path to the company of the sun and moon, saying: ‘This is the only straight path, the direct route that leads someone who practices it to the company of the sun and moon’?” 

“No,\marginnote{17.4} Master Gotama.” 

“So\marginnote{18.1} it seems that even though the brahmins proficient in the three Vedas see the sun and moon, they are not able to teach the path to the company of the sun and moon. 

But\marginnote{18.3} it seems that even though they have not seen \textsanskrit{Brahmā} with their own eyes, they still claim to teach the path to the company of that which they neither know nor see. 

What\marginnote{18.11} do you think, \textsanskrit{Vāseṭṭha}? This being so, doesn’t their statement turn out to have no demonstrable basis?” 

“Clearly\marginnote{18.13} that’s the case, Master Gotama.” 

“Good,\marginnote{18.14} \textsanskrit{Vāseṭṭha}. For it is impossible that they should teach the path to that which they neither know nor see. 

\subsection*{2.1. The Simile of the Finest Lady in the Land }

Suppose\marginnote{19.1} a man were to say, ‘Whoever the finest lady in the land is, it is her that I want, her that I desire!’ 

They’d\marginnote{19.3} say to him, ‘Mister, that finest lady in the land who you desire—do you know whether she’s an aristocrat, a brahmin, a merchant, or a worker?’ Asked this, he’d say, ‘No.’ 

They’d\marginnote{19.7} say to him, ‘Mister, that finest lady in the land who you desire—do you know her name or clan? Whether she’s tall or short or medium? Whether her skin is black, brown, or tawny? What village, town, or city she comes from?’ 

Asked\marginnote{19.9} this, he’d say, ‘No.’ 

They’d\marginnote{19.10} say to him, ‘Mister, do you desire someone who you’ve never even known or seen?’ 

Asked\marginnote{19.12} this, he’d say, ‘Yes.’ 

What\marginnote{19.13} do you think, \textsanskrit{Vāseṭṭha}? This being so, doesn’t that man’s statement turn out to have no demonstrable basis?” 

“Clearly\marginnote{19.15} that’s the case, sir.” 

“In\marginnote{20.1} the same way, doesn’t the statement of those brahmins turn out to have no demonstrable basis?” 

“Clearly\marginnote{20.8} that’s the case, Master Gotama.” 

“Good,\marginnote{20.9} \textsanskrit{Vāseṭṭha}. For it is impossible that they should teach the path to that which they neither know nor see. 

\subsection*{2.2. The Simile of the Ladder }

Suppose\marginnote{21.1} a man was to build a ladder at the crossroads for climbing up to a stilt longhouse. 

They’d\marginnote{21.2} say to him, ‘Mister, that stilt longhouse that you’re building a ladder for—do you know whether it’s to the north, south, east, or west? Or whether it’s tall or short or medium?’ 

Asked\marginnote{21.4} this, he’d say, ‘No.’ 

They’d\marginnote{22.1} say to him, ‘Mister, are you building a ladder for a longhouse that you’ve never even known or seen?’ 

Asked\marginnote{22.3} this, he’d say, ‘Yes.’ 

What\marginnote{22.4} do you think, \textsanskrit{Vāseṭṭha}? This being so, doesn’t that man’s statement turn out to have no demonstrable basis?” 

“Clearly\marginnote{22.6} that’s the case, sir.” 

“In\marginnote{22.7} the same way, doesn’t the statement of those brahmins turn out to have no demonstrable basis?” 

“Clearly\marginnote{22.13} that’s the case, Master Gotama.” 

“Good,\marginnote{23.1} \textsanskrit{Vāseṭṭha}. For it is impossible that they should teach the path to that which they neither know nor see. 

\subsection*{2.3. The Simile of the River \textsanskrit{Aciravatī} }

Suppose\marginnote{24.1} the river \textsanskrit{Aciravatī} was full to the brim so a crow could drink from it. Then along comes a person who wants to cross over to the far shore. Standing on the near shore, they’d call out to the far shore, ‘Come here, far shore! Come here, far shore!’ 

What\marginnote{24.5} do you think, \textsanskrit{Vāseṭṭha}? Would the far shore of the \textsanskrit{Aciravatī} river come over to the near shore because of that man’s call, request, desire, or expectation?” 

“No,\marginnote{24.7} Master Gotama.” 

“In\marginnote{25.1} the same way, \textsanskrit{Vāseṭṭha}, the brahmins proficient in the three Vedas proceed having given up those things that make one a true brahmin, and having undertaken those things that make one not a true brahmin. Yet they say: ‘We call upon Indra! We call upon Soma! We call upon \textsanskrit{Varuṇa}! We call upon \textsanskrit{Īsāna}! We call upon \textsanskrit{Pajāpati}! We call upon \textsanskrit{Brahmā}! We call upon Mahiddhi! We call upon Yama!’ 

So\marginnote{25.3} long as they proceed in this way it’s impossible that they will, when the body breaks up, after death, be reborn in the company of \textsanskrit{Brahmā}. 

Suppose\marginnote{26.1} the river \textsanskrit{Aciravatī} was full to the brim so a crow could drink from it. Then along comes a person who wants to cross over to the far shore. But while still on the near shore, their arms are tied tightly behind their back with a strong chain. 

What\marginnote{26.4} do you think, \textsanskrit{Vāseṭṭha}? Could that person cross over to the far shore?” 

“No,\marginnote{26.6} Master Gotama.” 

“In\marginnote{27.1} the same way, the five kinds of sensual stimulation are called ‘chains’ and ‘fetters’ in the training of the Noble One. What five? Sights known by the eye that are likable, desirable, agreeable, pleasant, sensual, and arousing. Sounds known by the ear … Smells known by the nose … Tastes known by the tongue … Touches known by the body that are likable, desirable, agreeable, pleasant, sensual, and arousing. 

These\marginnote{27.8} are the five kinds of sensual stimulation that are called ‘chains’ and ‘fetters’ in the training of the Noble One. The brahmins proficient in the three Vedas enjoy these five kinds of sensual stimulation tied, infatuated, attached, blind to the drawbacks, and not understanding the escape. So long as they enjoy them it’s impossible that they will, when the body breaks up, after death, be reborn in the company of \textsanskrit{Brahmā}. 

Suppose\marginnote{29.1} the river \textsanskrit{Aciravatī} was full to the brim so a crow could drink from it. Then along comes a person who wants to cross over to the far shore. But they’d lie down wrapped in cloth from head to foot. 

What\marginnote{29.4} do you think, \textsanskrit{Vāseṭṭha}? Could that person cross over to the far shore?” 

“No,\marginnote{29.6} Master Gotama.” 

“In\marginnote{30.1} the same way, the five hindrances are called ‘obstacles’ and ‘hindrances’ and ‘coverings’ and ‘shrouds’ in the training of the Noble One. What five? The hindrances of sensual desire, ill will, dullness and drowsiness, restlessness and remorse, and doubt. These five hindrances are called ‘obstacles’ and ‘hindrances’ and ‘coverings’ and ‘shrouds’ in the training of the Noble One. 

The\marginnote{30.5} brahmins proficient in the three Vedas are obstructed, shrouded, covered, and engulfed by these five hindrances. So long as they are so obstructed it’s impossible that they will, when the body breaks up, after death, be reborn in the company of \textsanskrit{Brahmā}. 

\section*{3. Converging }

What\marginnote{31.1} do you think, \textsanskrit{Vāseṭṭha}? Have you heard that the brahmins who are elderly and senior, the teachers of teachers, say whether \textsanskrit{Brahmā} is possessive or not?” 

“That\marginnote{31.3} he is not, Master Gotama.” 

“Is\marginnote{31.4} his heart full of enmity or not?” 

“It\marginnote{31.5} is not.” 

“Is\marginnote{31.6} his heart full of ill will or not?” 

“It\marginnote{31.7} is not.” 

“Is\marginnote{31.8} his heart corrupted or not?” 

“It\marginnote{31.9} is not.” 

“Does\marginnote{31.10} he wield power or not?” 

“He\marginnote{31.11} does.” 

“What\marginnote{32.1} do you think, \textsanskrit{Vāseṭṭha}? Are the brahmins proficient in the three Vedas possessive or not?” 

“They\marginnote{32.3} are.” 

“Are\marginnote{32.4} their hearts full of enmity or not?” 

“They\marginnote{32.5} are.” 

“Are\marginnote{32.6} their hearts full of ill will or not?” 

“They\marginnote{32.7} are.” 

“Are\marginnote{32.8} their hearts corrupted or not?” 

“They\marginnote{32.9} are.” 

“Do\marginnote{32.10} they wield power or not?” 

“They\marginnote{32.11} do not.” 

“So\marginnote{33.1} it seems that the brahmins proficient in the three Vedas are possessive, but \textsanskrit{Brahmā} is not. But would brahmins who are possessive come together and converge with \textsanskrit{Brahmā}, who isn’t possessive?” 

“No,\marginnote{33.3} Master Gotama.” 

“Good,\marginnote{34.1} \textsanskrit{Vāseṭṭha}! It’s impossible that the brahmins who are possessive will, when the body breaks up, after death, be reborn in the company of \textsanskrit{Brahmā}, who isn’t possessive. 

And\marginnote{35.1} it seems that the brahmins have enmity, ill will, corruption, and do not wield power, while \textsanskrit{Brahmā} is the opposite in all these things. But would brahmins who are opposite to \textsanskrit{Brahmā} in all things come together and converge with him?” 

“No,\marginnote{35.5} Master Gotama.” 

“Good,\marginnote{36.1} \textsanskrit{Vāseṭṭha}! It’s impossible that such brahmins will, when the body breaks up, after death, be reborn in the company of \textsanskrit{Brahmā}. 

But\marginnote{36.2} here the brahmins proficient in the three Vedas sink down where they have sat, only to be torn apart; all the while imagining that they’re crossing over to drier ground. That’s why the three Vedas of the brahmins are called a ‘salted land’ and a ‘barren land’ and a ‘disaster’.” 

When\marginnote{37.1} he said this, \textsanskrit{Vāseṭṭha} said to the Buddha, “I have heard, Master Gotama, that you know the path to company with \textsanskrit{Brahmā}.” 

“What\marginnote{37.3} do you think, \textsanskrit{Vāseṭṭha}? Is the village of \textsanskrit{Manasākaṭa} nearby?” 

“Yes\marginnote{37.5} it is.” 

“What\marginnote{37.6} do you think, \textsanskrit{Vāseṭṭha}? Suppose a person was born and raised in \textsanskrit{Manasākaṭa}. And as soon as they left the town some people asked them for the road to \textsanskrit{Manasākaṭa}. Would they be slow or hesitant to answer?” 

“No,\marginnote{37.10} Master Gotama. Why is that? Because they were born and raised in \textsanskrit{Manasākaṭa}. They’re well acquainted with all the roads to the village.” 

“Still,\marginnote{38.1} it’s possible they might be slow or hesitant to answer. But the Realized One is never slow or hesitant when questioned about the \textsanskrit{Brahmā} realm or the practice that leads to the \textsanskrit{Brahmā} realm. I understand \textsanskrit{Brahmā}, the \textsanskrit{Brahmā} realm, and the practice that leads to the \textsanskrit{Brahmā} realm, practicing in accordance with which one is reborn in the \textsanskrit{Brahmā} realm.” 

When\marginnote{39.1} he said this, \textsanskrit{Vāseṭṭha} said to the Buddha, “I have heard, Master Gotama, that you teach the path to company with \textsanskrit{Brahmā}. Please teach us that path and elevate this generation of brahmins.” 

“Well\marginnote{39.4} then, \textsanskrit{Vāseṭṭha}, listen and pay close attention, I will speak.” 

“Yes,\marginnote{39.5} sir,” replied \textsanskrit{Vāseṭṭha}. 

\section*{4. Teaching the Path to \textsanskrit{Brahmā} }

The\marginnote{40{-}75.1} Buddha said this: 

“It’s\marginnote{40{-}75.2} when a Realized One arises in the world, perfected, a fully awakened Buddha … That’s how a mendicant is accomplished in ethics. … Seeing that the hindrances have been given up in them, joy springs up. Being joyful, rapture springs up. When the mind is full of rapture, the body becomes tranquil. When the body is tranquil, they feel bliss. And when blissful, the mind becomes immersed. 

They\marginnote{76.1} meditate spreading a heart full of love to one direction, and to the second, and to the third, and to the fourth. In the same way above, below, across, everywhere, all around, they spread a heart full of love to the whole world—abundant, expansive, limitless, free of enmity and ill will. 

Suppose\marginnote{77.1} there was a powerful horn blower. They’d easily make themselves heard in the four quarters. In the same way, when the heart’s release by love has been developed like this, any limited deeds they’ve done don’t remain or persist there. This is a path to companionship with \textsanskrit{Brahmā}. 

Furthermore,\marginnote{78.1} a mendicant meditates spreading a heart full of compassion … 

They\marginnote{78.2} meditate spreading a heart full of rejoicing … 

They\marginnote{78.3} meditate spreading a heart full of equanimity to one direction, and to the second, and to the third, and to the fourth. In the same way above, below, across, everywhere, all around, they spread a heart full of equanimity to the whole world—abundant, expansive, limitless, free of enmity and ill will. 

Suppose\marginnote{79.1} there was a powerful horn blower. They’d easily make themselves heard in the four quarters. In the same way, when the heart’s release by equanimity has been developed and cultivated like this, any limited deeds they’ve done don’t remain or persist there. This too is a path to companionship with \textsanskrit{Brahmā}. 

What\marginnote{80.1} do you think, \textsanskrit{Vāseṭṭha}? When a mendicant meditates like this, are they possessive or not?” 

“They\marginnote{80.3} are not.” 

“Is\marginnote{80.4} their heart full of enmity or not?” 

“It\marginnote{80.5} is not.” 

“Is\marginnote{80.6} their heart full of ill will or not?” 

“It\marginnote{80.7} is not.” 

“Is\marginnote{80.8} their heart corrupted or not?” 

“It\marginnote{80.9} is not.” 

“Do\marginnote{80.10} they wield power or not?” 

“They\marginnote{80.11} do.” 

“So\marginnote{81.1} it seems that that mendicant is not possessive, and neither is \textsanskrit{Brahmā}. Would a mendicant who is not possessive come together and converge with \textsanskrit{Brahmā}, who isn’t possessive?” 

“Yes,\marginnote{81.3} Master Gotama.” 

“Good,\marginnote{81.4} \textsanskrit{Vāseṭṭha}! It’s possible that a mendicant who is not possessive will, when the body breaks up, after death, be reborn in the company of \textsanskrit{Brahmā}, who isn’t possessive. 

And\marginnote{81.5} it seems that that mendicant has no enmity, ill will, corruption, and does wield power, while \textsanskrit{Brahmā} is the same in all these things. Would a mendicant who is the same as \textsanskrit{Brahmā} in all things come together and converge with him?” 

“Yes,\marginnote{81.9} Master Gotama.” 

“Good,\marginnote{81.10} \textsanskrit{Vāseṭṭha}! It’s possible that that mendicant will, when the body breaks up, after death, be reborn in the company of \textsanskrit{Brahmā}.” 

When\marginnote{82.1} he had spoken, \textsanskrit{Vāseṭṭha} and \textsanskrit{Bhāradvāja} said to him, “Excellent, Master Gotama! Excellent! As if he were righting the overturned, or revealing the hidden, or pointing out the path to the lost, or lighting a lamp in the dark so people with good eyes can see what’s there, Master Gotama has made the teaching clear in many ways. We go for refuge to Master Gotama, to the teaching, and to the mendicant \textsanskrit{Saṅgha}. From this day forth, may Master Gotama remember us as lay followers who have gone for refuge for life.” 

%
\backmatter%
\chapter*{Colophon}
\addcontentsline{toc}{chapter}{Colophon}
\markboth{Colophon}{Colophon}

\section*{The Translator}

Bhikkhu Sujato was born as Anthony Aidan Best on 4/11/1966 in Perth, Western Australia. He grew up in the pleasant suburbs of Mt Lawley and Attadale alongside his sister Nicola, who was the good child. His mother, Margaret Lorraine Huntsman née Pinder, said “he’ll either be a priest or a poet”, while his father, Anthony Thomas Best, advised him to “never do anything for money”. He attended Aquinas College, a Catholic school, where he decided to become an atheist. At the University of WA he studied philosophy, aiming to learn what he wanted to do with his life. Finding that what he wanted to do was play guitar, he dropped out. His main band was named Martha’s Vineyard, which achieved modest success in the indie circuit. 

A seemingly random encounter with a roadside joey took him to Thailand, where he entered his first meditation retreat at Wat Ram Poeng, Chieng Mai in 1992. Feeling the call to the Buddha’s path, he took full ordination in Wat Pa Nanachat in 1994, where his teachers were Ajahn Pasanno and Ajahn Jayasaro. In 1997 he returned to Perth to study with Ajahn Brahm at Bodhinyana Monastery. 

He spent several years practicing in seclusion in Malaysia and Thailand before establishing Santi Forest Monastery in Bundanoon, NSW, in 2003. There he was instrumental in supporting the establishment of the Theravada bhikkhuni order in Australia and advocating for women’s rights. He continues to teach in Australia and globally, with a special concern for the moral implications of climate change and other forms of environmental destruction. He has published a series of books of original and groundbreaking research on early Buddhism. 

In 2005 he founded SuttaCentral together with Rod Bucknell and John Kelly. In 2015, seeing the need for a complete, accurate, plain English translation of the Pali texts, he undertook the task, spending nearly three years in isolation on the isle of Qi Mei off the coast of the nation of Taiwan. He completed the four main \textsanskrit{Nikāyas} in 2018, and the early books of the Khuddaka \textsanskrit{Nikāya} were complete by 2021. All this work is dedicated to the public domain and is entirely free of copyright encumbrance. 

In 2019 he returned to Sydney where he established Lokanta Vihara (The Monastery at the End of the World). 

\section*{Creation Process}

Primary source was the digital \textsanskrit{Mahāsaṅgīti} edition of the Pali \textsanskrit{Tipiṭaka}. Translated from the Pali, with reference to several English translations, especially those of Bhikkhu Bodhi. Older translations by Maurice Walshe and T.W. and C.A.F. Rhys Davids were also consulted.

\section*{The Translation}

This translation was part of a project to translate the four Pali \textsanskrit{Nikāyas} with the following aims: plain, approachable English; consistent terminology; accurate rendition of the Pali; free of copyright. It was made during 2016–2018 while Bhikkhu Sujato was staying in Qimei, Taiwan.

\section*{About SuttaCentral}

SuttaCentral publishes early Buddhist texts. Since 2005 we have provided root texts in Pali, Chinese, Sanskrit, Tibetan, and other languages, parallels between these texts, and translations in many modern languages. We build on the work of generations of scholars, and offer our contribution freely.

SuttaCentral is driven by volunteer contributions, and in addition we employ professional developers. We offer a sponsorship program for high quality translations from the original languages. Financial support for SuttaCentral is handled by the SuttaCentral Development Trust, a charitable trust registered in Australia.

\section*{About Bilara}

“Bilara” means “cat” in Pali, and it is the name of our Computer Assisted Translation (CAT) software. Bilara is a web app that enables translators to translate early Buddhist texts into their own language. These translations are published on SuttaCentral with the root text and translation side by side.

\section*{About SuttaCentral Editions}

The SuttaCentral Editions project makes high quality books from selected Bilara translations. These are published in formats including HTML, EPUB, PDF, and print.

If you want to print any of our Editions, please let us know and we will help prepare a file to your specifications.

%
\end{document}