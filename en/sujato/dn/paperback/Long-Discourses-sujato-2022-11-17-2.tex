\documentclass[12pt,openany]{book}%
\usepackage{lastpage}%
%
\usepackage[inner=1in, outer=1in, top=.7in, bottom=1in, papersize={6in,9in}, headheight=13pt]{geometry}
\usepackage{polyglossia}
\usepackage[12pt]{moresize}
\usepackage{soul}%
\usepackage{microtype}
\usepackage{tocbasic}
\usepackage{realscripts}
\usepackage{epigraph}%
\usepackage{setspace}%
\usepackage{sectsty}
\usepackage{fontspec}
\usepackage{marginnote}
\usepackage[bottom]{footmisc}
\usepackage{enumitem}
\usepackage{fancyhdr}
\usepackage{extramarks}
\usepackage{graphicx}
\usepackage{verse}
\usepackage{relsize}
\usepackage{etoolbox}
\usepackage[a-3u]{pdfx}

\hypersetup{
colorlinks=true,
urlcolor=black,
linkcolor=black,
citecolor=black
}

% use a small amount of tracking on small caps
\SetTracking[ spacing = {25*,166, } ]{ encoding = *, shape = sc }{ 25 }

% add a blank page
\newcommand{\blankpage}{
\newpage
\thispagestyle{empty}
\mbox{}
\newpage
}

% define languages
\setdefaultlanguage[]{english}
\setotherlanguage[script=Latin]{sanskrit}

%\usepackage{pagegrid}
%\pagegridsetup{top-left, step=.25in}

% define fonts
% use if arno sanskrit is unavailable
%\setmainfont{Gentium Plus}
%\newfontfamily\Semiboldsubheadfont[]{Gentium Plus}
%\newfontfamily\Semiboldnormalfont[]{Gentium Plus}
%\newfontfamily\Lightfont[]{Gentium Plus}
%\newfontfamily\Marginalfont[]{Gentium Plus}
%\newfontfamily\Allsmallcapsfont[RawFeature=+c2sc]{Gentium Plus}
%\newfontfamily\Noligaturefont[Renderer=Basic]{Gentium Plus}
%\newfontfamily\Noligaturecaptionfont[Renderer=Basic]{Gentium Plus}
%\newfontfamily\Fleuronfont[Ornament=1]{Gentium Plus}

% use if arno sanskrit is available. display is applied to \chapter and \part, subhead to \section and \subsection. When specifying semibold, the italic must be defined.
\setmainfont[Numbers=OldStyle]{Arno Pro}
\newfontfamily\Semibolddisplayfont[BoldItalicFont = Arno Pro Semibold Italic Display]{Arno Pro Semibold Display} %
\newfontfamily\Semiboldsubheadfont[BoldItalicFont = Arno Pro Semibold Italic Subhead]{Arno Pro Semibold Subhead}
\newfontfamily\Semiboldnormalfont[BoldItalicFont = Arno Pro Semibold Italic]{Arno Pro Semibold}
\newfontfamily\Marginalfont[RawFeature=+subs]{Arno Pro Regular}
\newfontfamily\Allsmallcapsfont[RawFeature=+c2sc]{Arno Pro}
\newfontfamily\Noligaturefont[Renderer=Basic]{Arno Pro}
\newfontfamily\Noligaturecaptionfont[Renderer=Basic]{Arno Pro Caption}

% chinese fonts
\newfontfamily\cjk{Noto Serif TC}
\newcommand*{\langlzh}[1]{\cjk{#1}\normalfont}%

% logo
\newfontfamily\Logofont{sclogo.ttf}
\newcommand*{\sclogo}[1]{\large\Logofont{#1}}

% use subscript numerals for margin notes
\renewcommand*{\marginfont}{\Marginalfont}

% ensure margin notes have consistent vertical alignment
\renewcommand*{\marginnotevadjust}{-.17em}

% use compact lists
\setitemize{noitemsep,leftmargin=1em}
\setenumerate{noitemsep,leftmargin=1em}
\setdescription{noitemsep, style=unboxed, leftmargin=0em}

% style ToC
\DeclareTOCStyleEntries[
  raggedentrytext,
  linefill=\hfill,
  pagenumberwidth=.5in,
  pagenumberformat=\normalfont,
  entryformat=\normalfont
]{tocline}{chapter,section}


  \setlength\topsep{0pt}%
  \setlength\parskip{0pt}%

% define new \centerpars command for use in ToC. This ensures centering, proper wrapping, and no page break after
\def\startcenter{%
  \par
  \begingroup
  \leftskip=0pt plus 1fil
  \rightskip=\leftskip
  \parindent=0pt
  \parfillskip=0pt
}
\def\stopcenter{%
  \par
  \endgroup
}
\long\def\centerpars#1{\startcenter#1\stopcenter}

% redefine part, so that it adds a toc entry without page number
\let\oldcontentsline\contentsline
\newcommand{\nopagecontentsline}[3]{\oldcontentsline{#1}{#2}{}}

    \makeatletter
\renewcommand*\l@part[2]{%
  \ifnum \c@tocdepth >-2\relax
    \addpenalty{-\@highpenalty}%
    \addvspace{0em \@plus\p@}%
    \setlength\@tempdima{3em}%
    \begingroup
      \parindent \z@ \rightskip \@pnumwidth
      \parfillskip -\@pnumwidth
      {\leavevmode
       \setstretch{.85}\large\scshape\centerpars{#1}\vspace*{-1em}\llap{#2}}\par
       \nobreak
         \global\@nobreaktrue
         \everypar{\global\@nobreakfalse\everypar{}}%
    \endgroup
  \fi}
\makeatother

\makeatletter
\def\@pnumwidth{2em}
\makeatother

% define new sectioning command, which is only used in volumes where the pannasa is found in some parts but not others, especially in an and sn

\newcommand*{\pannasa}[1]{\clearpage\thispagestyle{empty}\begin{center}\vspace*{14em}\setstretch{.85}\huge\itshape\scshape\MakeLowercase{#1}\end{center}}

    \makeatletter
\newcommand*\l@pannasa[2]{%
  \ifnum \c@tocdepth >-2\relax
    \addpenalty{-\@highpenalty}%
    \addvspace{.5em \@plus\p@}%
    \setlength\@tempdima{3em}%
    \begingroup
      \parindent \z@ \rightskip \@pnumwidth
      \parfillskip -\@pnumwidth
      {\leavevmode
       \setstretch{.85}\large\itshape\scshape\lowercase{\centerpars{#1}}\vspace*{-1em}\llap{#2}}\par
       \nobreak
         \global\@nobreaktrue
         \everypar{\global\@nobreakfalse\everypar{}}%
    \endgroup
  \fi}
\makeatother

% don't put page number on first page of toc (relies on etoolbox)
\patchcmd{\chapter}{plain}{empty}{}{}

% global line height
\setstretch{1.05}

% allow linebreak after em-dash
\catcode`\—=13
\protected\def—{\unskip\textemdash\allowbreak}

% style headings with secsty. chapter and section are defined per-edition
\partfont{\setstretch{.85}\normalfont\centering\textsc}
\subsectionfont{\setstretch{.85}\Semiboldsubheadfont}%
\subsubsectionfont{\setstretch{.85}\Semiboldnormalfont}

% style elements of suttatitle
\newcommand*{\suttatitleacronym}[1]{\smaller[2]{#1}\vspace*{.3em}}
\newcommand*{\suttatitletranslation}[1]{\linebreak{#1}}
\newcommand*{\suttatitleroot}[1]{\linebreak\smaller[2]\itshape{#1}}

\DeclareTOCStyleEntries[
  indent=3.3em,
  dynindent,
  beforeskip=.2em plus -2pt minus -1pt,
]{tocline}{section}

\DeclareTOCStyleEntries[
  indent=0em,
  dynindent,
  beforeskip=.4em plus -2pt minus -1pt,
]{tocline}{chapter}

\newcommand*{\tocacronym}[1]{\hspace*{-3.3em}{#1}\quad}
\newcommand*{\toctranslation}[1]{#1}
\newcommand*{\tocroot}[1]{(\textit{#1})}
\newcommand*{\tocchapterline}[1]{\bfseries\itshape{#1}}


% redefine paragraph and subparagraph headings to not be inline
\makeatletter
% Change the style of paragraph headings %
\renewcommand\paragraph{\@startsection{paragraph}{4}{\z@}%
            {-2.5ex\@plus -1ex \@minus -.25ex}%
            {1.25ex \@plus .25ex}%
            {\noindent\Semiboldnormalfont\normalsize}}

% Change the style of subparagraph headings %
\renewcommand\subparagraph{\@startsection{subparagraph}{5}{\z@}%
            {-2.5ex\@plus -1ex \@minus -.25ex}%
            {1.25ex \@plus .25ex}%
            {\noindent\Semiboldnormalfont\small}}
\makeatother

% use etoolbox to suppress page numbers on \part
\patchcmd{\part}{\thispagestyle{plain}}{\thispagestyle{empty}}
  {}{\errmessage{Cannot patch \string\part}}

% and to reduce margins on quotation
\patchcmd{\quotation}{\rightmargin}{\leftmargin 1.2em \rightmargin}{}{}
\AtBeginEnvironment{quotation}{\small}

% titlepage
\newcommand*{\titlepageTranslationTitle}[1]{{\begin{center}\begin{large}{#1}\end{large}\end{center}}}
\newcommand*{\titlepageCreatorName}[1]{{\begin{center}\begin{normalsize}{#1}\end{normalsize}\end{center}}}

% halftitlepage
\newcommand*{\halftitlepageTranslationTitle}[1]{\setstretch{2.5}{\begin{Huge}\uppercase{\so{#1}}\end{Huge}}}
\newcommand*{\halftitlepageTranslationSubtitle}[1]{\setstretch{1.2}{\begin{large}{#1}\end{large}}}
\newcommand*{\halftitlepageFleuron}[1]{{\begin{large}\Fleuronfont{{#1}}\end{large}}}
\newcommand*{\halftitlepageByline}[1]{{\begin{normalsize}\textit{{#1}}\end{normalsize}}}
\newcommand*{\halftitlepageCreatorName}[1]{{\begin{LARGE}{\textsc{#1}}\end{LARGE}}}
\newcommand*{\halftitlepageVolumeNumber}[1]{{\begin{normalsize}{\Allsmallcapsfont{\textsc{#1}}}\end{normalsize}}}
\newcommand*{\halftitlepageVolumeAcronym}[1]{{\begin{normalsize}{#1}\end{normalsize}}}
\newcommand*{\halftitlepageVolumeTranslationTitle}[1]{{\begin{Large}{\textsc{#1}}\end{Large}}}
\newcommand*{\halftitlepageVolumeRootTitle}[1]{{\begin{normalsize}{\Allsmallcapsfont{\textsc{\itshape #1}}}\end{normalsize}}}
\newcommand*{\halftitlepagePublisher}[1]{{\begin{large}{\Noligaturecaptionfont\textsc{#1}}\end{large}}}

% epigraph
\renewcommand{\epigraphflush}{center}
\renewcommand*{\epigraphwidth}{.85\textwidth}
\newcommand*{\epigraphTranslatedTitle}[1]{\vspace*{.5em}\footnotesize\textsc{#1}\\}%
\newcommand*{\epigraphRootTitle}[1]{\footnotesize\textit{#1}\\}%
\newcommand*{\epigraphReference}[1]{\footnotesize{#1}}%

% custom commands for html styling classes
\newcommand*{\scnamo}[1]{\begin{center}\textit{#1}\end{center}}
\newcommand*{\scendsection}[1]{\begin{center}\textit{#1}\end{center}}
\newcommand*{\scendsutta}[1]{\begin{center}\textit{#1}\end{center}}
\newcommand*{\scendbook}[1]{\begin{center}\uppercase{#1}\end{center}}
\newcommand*{\scendkanda}[1]{\begin{center}\textbf{#1}\end{center}}
\newcommand*{\scend}[1]{\begin{center}\textit{#1}\end{center}}
\newcommand*{\scuddanaintro}[1]{\textit{#1}}
\newcommand*{\scendvagga}[1]{\begin{center}\textbf{#1}\end{center}}
\newcommand*{\scrule}[1]{\textbf{#1}}
\newcommand*{\scadd}[1]{\textit{#1}}
\newcommand*{\scevam}[1]{\textsc{#1}}
\newcommand*{\scspeaker}[1]{\hspace{2em}\textit{#1}}
\newcommand*{\scbyline}[1]{\begin{flushright}\textit{#1}\end{flushright}\bigskip}

% custom command for thematic break = hr
\newcommand*{\thematicbreak}{\begin{center}\rule[.5ex]{6em}{.4pt}\begin{normalsize}\quad\Fleuronfont{•}\quad\end{normalsize}\rule[.5ex]{6em}{.4pt}\end{center}}

% manage and style page header and footer. "fancy" has header and footer, "plain" has footer only

\pagestyle{fancy}
\fancyhf{}
\fancyfoot[RE,LO]{\thepage}
\fancyfoot[LE,RO]{\footnotesize\lastleftxmark}
\fancyhead[CE]{\setstretch{.85}\Noligaturefont\MakeLowercase{\textsc{\firstrightmark}}}
\fancyhead[CO]{\setstretch{.85}\Noligaturefont\MakeLowercase{\textsc{\firstleftmark}}}
\renewcommand{\headrulewidth}{0pt}
\fancypagestyle{plain}{ %
\fancyhf{} % remove everything
\fancyfoot[RE,LO]{\thepage}
\fancyfoot[LE,RO]{\footnotesize\lastleftxmark}
\renewcommand{\headrulewidth}{0pt}
\renewcommand{\footrulewidth}{0pt}}

% style footnotes
\setlength{\skip\footins}{1em}

\makeatletter
\newcommand{\@makefntextcustom}[1]{%
    \parindent 0em%
    \thefootnote.\enskip #1%
}
\renewcommand{\@makefntext}[1]{\@makefntextcustom{#1}}
\makeatother

% hang quotes (requires microtype)
\microtypesetup{
  protrusion = true,
  expansion  = true,
  tracking   = true,
  factor     = 1000,
  patch      = all,
  final
}

% Custom protrusion rules to allow hanging punctuation
\SetProtrusion
{ encoding = *}
{
% char   right left
  {-} = {    , 500 },
  % Double Quotes
  \textquotedblleft
      = {1000,     },
  \textquotedblright
      = {    , 1000},
  \quotedblbase
      = {1000,     },
  % Single Quotes
  \textquoteleft
      = {1000,     },
  \textquoteright
      = {    , 1000},
  \quotesinglbase
      = {1000,     }
}

% make latex use actual font em for parindent, not Computer Modern Roman
\AtBeginDocument{\setlength{\parindent}{1em}}%
%

% Default values; a bit sloppier than normal
\tolerance 1414
\hbadness 1414
\emergencystretch 1.5em
\hfuzz 0.3pt
\clubpenalty = 10000
\widowpenalty = 10000
\displaywidowpenalty = 10000
\hfuzz \vfuzz
 \raggedbottom%

\title{Long Discourses}
\author{Bhikkhu Sujato}
\date{}%
% define a different fleuron for each edition
\newfontfamily\Fleuronfont[Ornament=16]{Arno Pro}

% Define heading styles per edition for chapter and section. Suttatitle can be either of these, depending on the volume. 

\let\oldfrontmatter\frontmatter
\renewcommand{\frontmatter}{%
\chapterfont{\setstretch{.85}\normalfont\centering}%
\sectionfont{\setstretch{.85}\Semiboldsubheadfont}%
\oldfrontmatter}

\let\oldmainmatter\mainmatter
\renewcommand{\mainmatter}{%
\chapterfont{\setstretch{.85}\normalfont\centering}%
\sectionfont{\setstretch{.85}\Semiboldsubheadfont}%
\oldmainmatter}

\let\oldbackmatter\backmatter
\renewcommand{\backmatter}{%
\chapterfont{\setstretch{.85}\normalfont\centering}%
\sectionfont{\setstretch{.85}\Semiboldsubheadfont}%
\oldbackmatter}
%
%
\begin{document}%
\normalsize%
\frontmatter%
\setlength{\parindent}{0cm}

\pagestyle{empty}

\maketitle

\blankpage%
\begin{center}

\vspace*{2.2em}

\halftitlepageTranslationTitle{Long Discourses}

\vspace*{1em}

\halftitlepageTranslationSubtitle{A faithful translation of the Dīgha Nikāya}

\vspace*{2em}

\halftitlepageFleuron{•}

\vspace*{2em}

\halftitlepageByline{translated and introduced by}

\vspace*{.5em}

\halftitlepageCreatorName{Bhikkhu Sujato}

\vspace*{4em}

\halftitlepageVolumeNumber{Volume 2}

\smallskip

\halftitlepageVolumeAcronym{DN 14–23}

\smallskip

\halftitlepageVolumeTranslationTitle{The Great Chapter}

\smallskip

\halftitlepageVolumeRootTitle{Mahāvagga}

\vspace*{\fill}

\sclogo{0}
 \halftitlepagePublisher{SuttaCentral}

\end{center}

\newpage
%
\setstretch{1.05}

\begin{footnotesize}

\textit{Long Discourses} is a translation of the Dīghanikāya by Bhikkhu Sujato.

\medskip

Creative Commons Zero (CC0)

To the extent possible under law, Bhikkhu Sujato has waived all copyright and related or neighboring rights to \textit{Long Discourses}.

\medskip

This work is published from Australia.

\begin{center}
\textit{This translation is an expression of an ancient spiritual text that has been passed down by the Buddhist tradition for the benefit of all sentient beings. It is dedicated to the public domain via Creative Commons Zero (CC0). You are encouraged to copy, reproduce, adapt, alter, or otherwise make use of this translation. The translator respectfully requests that any use be in accordance with the values and principles of the Buddhist community.}
\end{center}

\medskip

\begin{description}
    \item[Web publication date] 2018
    \item[This edition] 2022-11-17 09:04:40
    \item[Publication type] paperback
    \item[Edition] ed5
    \item[Number of volumes] 3
    \item[Publication ISBN] 978-1-76132-052-1
    \item[Publication URL] https://suttacentral.net/editions/dn/en/sujato
    \item[Source URL] https://github.com/suttacentral/bilara-data/tree/published/translation/en/sujato/sutta/dn
    \item[Publication number] scpub2
\end{description}

\medskip

Published by SuttaCentral

\medskip

\textit{SuttaCentral,\\
c/o Alwis \& Alwis Pty Ltd\\
Kaurna Country,\\
Suite 12,\\
198 Greenhill Road,\\
Eastwood,\\
SA 5063,\\
Australia}

\end{footnotesize}

\newpage

\setlength{\parindent}{1.5em}%%
\tableofcontents
\newpage
\pagestyle{fancy}
%
\chapter*{Summary of Contents}
\addcontentsline{toc}{chapter}{Summary of Contents}
\markboth{Summary of Contents}{Summary of Contents}

\begin{description}%
\item[The Great Chapter (\textit{\textsanskrit{Mahāvagga}})] This chapter contains a diverse range of discourses. Several focus on the events surrounding the Buddha’s death, while others range into fabulous scenarios set among the gods, and still others are grounded in detailed discussions of doctrine.%
\item[DN 14: The Great Discourse on the Harvest of Deeds (\textit{\textsanskrit{Mahāpadānasutta}})] The Buddha teaches about the six Buddhas of the past, and tells a lengthy account of one of those, \textsanskrit{Vipassī}.%
\item[DN 15: The Great Discourse on Causation (\textit{\textsanskrit{Mahānidānasutta}})] Rejecting Venerable Ānanda’s claim to easily understand dependent origination, the Buddha presents a complex and demanding analysis, revealing hidden nuances and implications of this central teaching.%
\item[DN 16: The Great Discourse on the Buddha’s Extinguishment (\textit{\textsanskrit{Mahāparinibbānasutta}})] The longest of all discourses, this extended narrative tells of the events surrounding the Buddha’s death. Full of vivid and moving details, it is an ideal entry point into knowing the Buddha as a person, and understanding how the Buddhist community coped with his passing.%
\item[DN 17: King \textsanskrit{Mahāsudassana} (\textit{\textsanskrit{Mahāsudassanasutta}})] An elaborate story of a past life of the Buddha as a legendary king who renounced all to practice meditation.%
\item[DN 18: With Janavasabha (\textit{\textsanskrit{Janavasabhasutta}})] Beginning with an account of the fates of disciples who had recently passed away, the scene shifts to a discussion of Dhamma held by the gods.%
\item[DN 19: The Great Steward (\textit{\textsanskrit{Mahāgovindasutta}})] A minor deity informs the Buddha of the conversations and business of the gods.%
\item[DN 20: The Great Congregation (\textit{\textsanskrit{Mahāsamayasutta}})] When deities from all realms gather in homage to the Buddha, he gives a series of verses describing them. These verses, which are commonly chanted in Theravadin countries, give one of the most detailed descriptions of the deities worshiped at the the time of the Buddha.%
\item[DN 21: Sakka’s Questions (\textit{\textsanskrit{Sakkapañhasutta}})] After hearing a love song from a god of music, the Buddha engages in a deep discussion with Sakka on the conditioned origin of attachment and suffering.%
\item[DN 22: The Longer Discourse on Mindfulness Meditation (\textit{\textsanskrit{Mahāsatipaṭṭhānasutta}})] The Buddha details the seventh factor of the noble eightfold path, mindfulness meditation. This discourse is essentially identical to MN 10, with the addition of an extended section on the four noble truths derived from MN 141.%
\item[DN 23: With \textsanskrit{Pāyāsi} (\textit{\textsanskrit{Pāyāsisutta}})] This is a long and entertaining debate between a monk and a skeptic, who went to elaborate and bizarre lengths to prove that there is no such thing as an afterlife. The discourse contains a colorful series of parables and examples.%
\end{description}

%
\mainmatter%
\pagestyle{fancy}%
\addtocontents{toc}{\let\protect\contentsline\protect\nopagecontentsline}
\part*{The Great Chapter }
\addcontentsline{toc}{part}{The Great Chapter }
\markboth{}{}
\addtocontents{toc}{\let\protect\contentsline\protect\oldcontentsline}

%
\chapter*{{\suttatitleacronym DN 14}{\suttatitletranslation The Great Discourse on the Harvest of Deeds }{\suttatitleroot Mahāpadānasutta}}
\addcontentsline{toc}{chapter}{\tocacronym{DN 14} \toctranslation{The Great Discourse on the Harvest of Deeds } \tocroot{Mahāpadānasutta}}
\markboth{The Great Discourse on the Harvest of Deeds }{Mahāpadānasutta}
\extramarks{DN 14}{DN 14}

\section*{1. On Past Lives }

\scevam{So\marginnote{1.1.1} I have heard. }At one time the Buddha was staying near \textsanskrit{Sāvatthī} in Jeta’s Grove, \textsanskrit{Anāthapiṇḍika}’s monastery, in the hut by the kari tree. 

Then\marginnote{1.1.3} after the meal, on return from almsround, several senior mendicants sat together in the pavilion by the kari tree and this Dhamma talk on the subject of past lives came up among them, “So it was in a past life; such it was in a past life.” 

With\marginnote{1.2.1} clairaudience that is purified and superhuman, the Buddha heard that discussion among the mendicants. So he got up from his seat and went to the pavilion, where he sat on the seat spread out and addressed the mendicants, “Mendicants, what were you sitting talking about just now? What conversation was left unfinished?” 

The\marginnote{1.2.4} mendicants told him what had happened, adding, “This is the conversation that was unfinished when the Buddha arrived.” 

“Would\marginnote{1.3.1} you like to hear a Dhamma talk on the subject of past lives?” 

“Now\marginnote{1.3.2} is the time, Blessed One! Now is the time, Holy One! Let the Buddha give a Dhamma talk on the subject of past lives. The mendicants will listen and remember it.” 

“Well\marginnote{1.3.4} then, mendicants, listen and pay close attention, I will speak.” 

“Yes,\marginnote{1.3.5} sir,” they replied. The Buddha said this: 

“Ninety-one\marginnote{1.4.1} eons ago, the Buddha \textsanskrit{Vipassī} arose in the world, perfected and fully awakened. Thirty-one eons ago, the Buddha \textsanskrit{Sikhī} arose in the world, perfected and fully awakened. In the same thirty-first eon, the Buddha \textsanskrit{Vessabhū} arose in the world, perfected and fully awakened. In the present fortunate eon, the Buddhas Kakusandha, \textsanskrit{Koṇāgamana}, and Kassapa arose in the world, perfected and fully awakened. And in the present fortunate eon, I have arisen in the world, perfected and fully awakened. 

The\marginnote{1.5.1} Buddhas \textsanskrit{Vipassī}, \textsanskrit{Sikhī}, and \textsanskrit{Vessabhū} were born as aristocrats into aristocrat families. The Buddhas Kakusandha, \textsanskrit{Koṇāgamana}, and Kassapa were born as brahmins into brahmin families. I was born as an aristocrat into an aristocrat family. 

\textsanskrit{Koṇḍañña}\marginnote{1.6.1} was the clan of \textsanskrit{Vipassī}, \textsanskrit{Sikhī}, and \textsanskrit{Vessabhū}. Kassapa was the clan of Kakusandha, \textsanskrit{Koṇāgamana}, and Kassapa. Gotama is my clan. 

\textsanskrit{Vipassī}\marginnote{1.7.1} lived for 80,000 years. \textsanskrit{Sikhī} lived for 70,000 years. \textsanskrit{Vessabhū} lived for 60,000 years. Kakusandha lived for 40,000 years. \textsanskrit{Koṇāgamana} lived for 30,000 years. Kassapa lived for 20,000 years. For me these days the life-span is short, brief, and fleeting. A long-lived person lives for a century or a little more. 

\textsanskrit{Vipassī}\marginnote{1.8.1} was awakened at the root of a trumpet flower tree. \textsanskrit{Sikhī} was awakened at the root of a white-mango tree. \textsanskrit{Vessabhū} was awakened at the root of a sal tree. Kakusandha was awakened at the root of an acacia tree. \textsanskrit{Koṇāgamana} was awakened at the root of a cluster fig tree. Kassapa was awakened at the root of a banyan tree. I was awakened at the root of a peepul tree. 

\textsanskrit{Vipassī}\marginnote{1.9.1} had a fine pair of chief disciples named \textsanskrit{Khaṇḍa} and Tissa. \textsanskrit{Sikhī} had a fine pair of chief disciples named \textsanskrit{Abhibhū} and Sambhava. \textsanskrit{Vessabhū} had a fine pair of chief disciples named \textsanskrit{Soṇa} and Uttara. Kakusandha had a fine pair of chief disciples named Vidhura and \textsanskrit{Sañjīva}. \textsanskrit{Koṇāgamana} had a fine pair of chief disciples named Bhiyyosa and Uttara. Kassapa had a fine pair of chief disciples named Tissa and \textsanskrit{Bhāradvāja}. I have a fine pair of chief disciples named \textsanskrit{Sāriputta} and \textsanskrit{Moggallāna}. 

\textsanskrit{Vipassī}\marginnote{1.10.1} had three gatherings of disciples—one of 6,800,000, one of 100,000, and one of 80,000—all of them mendicants who had ended their defilements. 

\textsanskrit{Sikhī}\marginnote{1.10.2} had three gatherings of disciples—one of 100,000, one of 80,000, and one of 70,000—all of them mendicants who had ended their defilements. 

\textsanskrit{Vessabhū}\marginnote{1.10.3} had three gatherings of disciples—one of 80,000, one of 70,000, and one of 60,000—all of them mendicants who had ended their defilements. 

Kakusandha\marginnote{1.10.4} had one gathering of disciples—40,000 mendicants who had ended their defilements. 

\textsanskrit{Koṇāgamana}\marginnote{1.10.5} had one gathering of disciples—30,000 mendicants who had ended their defilements. 

Kassapa\marginnote{1.10.6} had one gathering of disciples—20,000 mendicants who had ended their defilements. 

I\marginnote{1.10.7} have had one gathering of disciples—1,250 mendicants who had ended their defilements. 

\textsanskrit{Vipassī}\marginnote{1.11.1} had as chief attendant a mendicant named Asoka. \textsanskrit{Sikhī} had as chief attendant a mendicant named \textsanskrit{Khemaṅkara}. \textsanskrit{Vessabhū} had as chief attendant a mendicant named Upasanta. Kakusandha had as chief attendant a mendicant named Buddhija. \textsanskrit{Koṇāgamana} had as chief attendant a mendicant named Sotthija. Kassapa had as chief attendant a mendicant named Sabbamitta. I have as chief attendant a mendicant named Ānanda. 

\textsanskrit{Vipassī}’s\marginnote{1.12.1} father was King Bandhuma, his birth mother was Queen \textsanskrit{Bandhumatī}, and their capital city was named \textsanskrit{Bandhumatī}. 

\textsanskrit{Sikhī}’s\marginnote{1.12.4} father was King \textsanskrit{Aruṇa}, his birth mother was Queen \textsanskrit{Pabhāvatī}, and their capital city was named \textsanskrit{Aruṇavatī}. 

\textsanskrit{Vessabhū}’s\marginnote{1.12.7} father was King \textsanskrit{Suppatīta}, his birth mother was Queen \textsanskrit{Vassavatī}, and their capital city was named Anoma. 

Kakusandha’s\marginnote{1.12.10} father was the brahmin Aggidatta, and his birth mother was the brahmin lady \textsanskrit{Visākhā}. At that time the king was Khema, whose capital city was named \textsanskrit{Khemavatī}. 

\textsanskrit{Koṇāgamana}’s\marginnote{1.12.14} father was the brahmin \textsanskrit{Yaññadatta}, and his birth mother was the brahmin lady \textsanskrit{Uttarā}. At that time the king was Sobha, whose capital city was named \textsanskrit{Sobhavatī}. 

Kassapa’s\marginnote{1.12.18} father was the brahmin Brahmadatta, and his birth mother was the brahmin lady \textsanskrit{Dhanavatī}. At that time the king was \textsanskrit{Kikī}, whose capital city was named Benares. 

In\marginnote{1.12.22} this life, my father was King Suddhodana, my birth mother was Queen \textsanskrit{Māyā}, and our capital city was Kapilavatthu.” 

That\marginnote{1.12.25} is what the Buddha said. When he had spoken, the Holy One got up from his seat and entered his dwelling. 

Soon\marginnote{1.13.1} after the Buddha left, those mendicants discussed among themselves: 

“It’s\marginnote{1.13.2} incredible, reverends, it’s amazing, the power and might of a Realized One! For he is able to recollect the caste, names, clans, life-span, chief disciples, and gatherings of disciples of the Buddhas of the past who have become completely extinguished, cut off proliferation, cut off the track, finished off the cycle, and transcended suffering. He knows the caste they were born in, and also their names, clans, conduct, qualities, wisdom, meditation, and freedom. 

Is\marginnote{1.13.5} it because the Realized One has clearly comprehended the principle of the teachings that he can recollect all these things? Or did deities tell him?” But this conversation among those mendicants was left unfinished. 

Then\marginnote{1.14.1} in the late afternoon, the Buddha came out of retreat and went to the pavilion by the kari tree, where he sat on the seat spread out and addressed the mendicants, “Mendicants, what were you sitting talking about just now? What conversation was left unfinished?” 

The\marginnote{1.14.4} mendicants told him what had happened, adding, “This was our conversation that was unfinished when the Buddha arrived.” 

“It\marginnote{1.15.1} is because the Realized One has clearly comprehended the principle of the teachings that he can recollect all these things. And the deities also told me. 

Would\marginnote{1.15.5} you like to hear a further Dhamma talk on the subject of past lives?” 

“Now\marginnote{1.15.6} is the time, Blessed One! Now is the time, Holy One! Let the Buddha give a further Dhamma talk on the subject of past lives. The mendicants will listen and remember it.” 

“Well\marginnote{1.15.8} then, mendicants, listen and pay close attention, I will speak.” 

“Yes,\marginnote{1.15.9} sir,” they replied. The Buddha said this: 

“Ninety-one\marginnote{1.16.1} eons ago, the Buddha \textsanskrit{Vipassī} arose in the world, perfected and fully awakened. He was born as an aristocrat into an aristocrat family. His clan was \textsanskrit{Koṇḍañña}. He lived for 80,000 years. He was awakened at the root of a trumpet flower tree. He had a fine pair of chief disciples named \textsanskrit{Khaṇḍa} and Tissa. He had three gatherings of disciples—one of 6,800,000, one of 100,000, and one of 80,000—all of them mendicants who had ended their defilements. He had as chief attendant a mendicant named Asoka. His father was King Bandhuma, his birth mother was Queen \textsanskrit{Bandhumatī}, and their capital city was named \textsanskrit{Bandhumatī}. 

\section*{2. What’s Normal For One Intent on Awakening }

When\marginnote{1.17.1} \textsanskrit{Vipassī}, the being intent on awakening, passed away from the host of Joyful Gods, he was conceived in his mother’s womb, mindful and aware. This is normal in such a case. 

It’s\marginnote{1.17.3} normal that, when the being intent on awakening passes away from the host of Joyful Gods, he is conceived in his mother’s womb. And then—in this world with its gods, \textsanskrit{Māras} and \textsanskrit{Brahmās}, this population with its ascetics and brahmins, gods and humans—an immeasurable, magnificent light appears, surpassing the glory of the gods. Even in the boundless desolation of interstellar space—so utterly dark that even the light of the moon and the sun, so mighty and powerful, makes no impression—an immeasurable, magnificent light appears, surpassing the glory of the gods. And the sentient beings reborn there recognize each other by that light: ‘So, it seems other sentient beings have been reborn here!’ And this galaxy shakes and rocks and trembles. And an immeasurable, magnificent light appears in the world, surpassing the glory of the gods. This is normal in such a case. 

It’s\marginnote{1.17.9} normal that, when the being intent on awakening is conceived in his mother’s belly, four gods approach to guard the four quarters, so that no human or non-human or anyone at all shall harm the being intent on awakening or his mother. This is normal in such a case. 

It’s\marginnote{1.18.1} normal that, when the being intent on awakening is conceived in his mother’s belly, she becomes naturally ethical. She refrains from killing living creatures, stealing, sexual misconduct, lying, and alcoholic drinks that cause negligence. This is normal in such a case. 

It’s\marginnote{1.19.1} normal that, when the being intent on awakening is conceived in his mother’s belly, she no longer feels sexual desire for men, and she cannot be violated by a man of lustful intent. This is normal in such a case. 

It’s\marginnote{1.20.1} normal that, when the being intent on awakening is conceived in his mother’s belly, she obtains the five kinds of sensual stimulation and amuses herself, supplied and provided with them. This is normal in such a case. 

It’s\marginnote{1.21.1} normal that, when the being intent on awakening is conceived in his mother’s belly, no afflictions beset her. She’s happy and free of bodily fatigue. And she sees the being intent on awakening in her womb, complete with all his various parts, not deficient in any faculty. Suppose there was a beryl gem that was naturally beautiful, eight-faceted, well-worked, transparent, clear, and unclouded, endowed with all good qualities. And it was strung with a thread of blue, yellow, red, white, or golden brown. And someone with good eyesight were to take it in their hand and examine it: ‘This beryl gem is naturally beautiful, eight-faceted, well-worked, transparent, clear, and unclouded, endowed with all good qualities. And it’s strung with a thread of blue, yellow, red, white, or golden brown.’ 

In\marginnote{1.21.4} the same way, when the being intent on awakening is conceived in his mother’s belly, no afflictions beset her. She’s happy and free of bodily fatigue. And she sees the being intent on awakening in her womb, complete with all his various parts, not deficient in any faculty. This is normal in such a case. 

It’s\marginnote{1.22.1} normal that, seven days after the being intent on awakening is born, his mother passes away and is reborn in the host of Joyful Gods. This is normal in such a case. 

It’s\marginnote{1.23.1} normal that, while other women carry the infant in the womb for nine or ten months before giving birth, not so the mother of the being intent on awakening. She gives birth after exactly ten months. This is normal in such a case. 

It’s\marginnote{1.24.1} normal that, while other women give birth while sitting or lying down, not so the mother of the being intent on awakening. She only gives birth standing up. This is normal in such a case. 

It’s\marginnote{1.25.1} normal that, when the being intent on awakening emerges from his mother’s womb, gods receive him first, then humans. This is normal in such a case. 

It’s\marginnote{1.26.1} normal that, when the being intent on awakening emerges from his mother’s womb, before he reaches the ground, four gods receive him and place him before his mother, saying: ‘Rejoice, O Queen! An illustrious child is born to you.’ This is normal in such a case. 

It’s\marginnote{1.27.1} normal that, when the being intent on awakening emerges from his mother’s womb, he emerges already clean, unsoiled by waters, mucus, blood, or any other kind of impurity, pure and clean. Suppose a jewel-treasure was placed on a cloth from \textsanskrit{Kāsī}. The jewel would not soil the cloth, nor would the cloth soil the jewel. Why is that? Because of the cleanliness of them both. 

In\marginnote{1.27.5} the same way, when the being intent on awakening emerges from his mother’s womb, he emerges already clean, unsoiled by waters, mucus, blood, or any other kind of impurity, pure and clean. This is normal in such a case. 

It’s\marginnote{1.28.1} normal that, when the being intent on awakening emerges from his mother’s womb, two streams of water appear in the sky, one cool, one warm, for bathing the being intent on awakening and his mother. This is normal in such a case. 

It’s\marginnote{1.29.1} normal that, as soon as he’s born, the being intent on awakening stands firm with his own feet on the ground. Facing north, he takes seven strides with a white parasol held above him, surveys all quarters, and makes this dramatic proclamation: ‘I am the foremost in the world! I am the eldest in the world! I am the first in the world! This is my last rebirth. Now there are no more future lives.’ This is normal in such a case. 

It’s\marginnote{1.30.1} normal that, when the being intent on awakening emerges from his mother’s womb, then—in this world with its gods, \textsanskrit{Māras} and \textsanskrit{Brahmās}, this population with its ascetics and brahmins, gods and humans—an immeasurable, magnificent light appears, surpassing the glory of the gods. Even in the boundless desolation of interstellar space—so utterly dark that even the light of the moon and the sun, so mighty and powerful, makes no impression—an immeasurable, magnificent light appears, surpassing the glory of the gods. And the sentient beings reborn there recognize each other by that light: ‘So, it seems other sentient beings have been reborn here!’ And this galaxy shakes and rocks and trembles. And an immeasurable, magnificent light appears in the world, surpassing the glory of the gods. This is normal in such a case. 

\section*{3. The Thirty-Two Marks of a Great Man }

When\marginnote{1.31.1} Prince \textsanskrit{Vipassī} was born, they announced it to King Bandhuma, ‘Sire, your son is born! Let your majesty examine him!’ When the king had examined the prince, he had the brahmin soothsayers summoned and said to them, ‘Gentlemen, please examine the prince.’ When they had examined him they said to the king, ‘Rejoice, O King! An illustrious son is born to you. You are fortunate, so very fortunate, to have a son such as this born in this family! For the prince has the thirty-two marks of a great man. A great man who possesses these has only two possible destinies, no other. If he stays at home he becomes a king, a wheel-turning monarch, a just and principled king. His dominion extends to all four sides, he achieves stability in the country, and he possesses the seven treasures. He has the following seven treasures: the wheel, the elephant, the horse, the jewel, the woman, the treasurer, and the counselor as the seventh treasure. He has over a thousand sons who are valiant and heroic, crushing the armies of his enemies. After conquering this land girt by sea, he reigns by principle, without rod or sword. But if he goes forth from the lay life to homelessness, he becomes a perfected one, a fully awakened Buddha, who draws back the veil from the world. 

And\marginnote{1.32.1} what are the marks which he possesses? After conquering this land girt by sea, he reigns by principle, without rod or sword. 

He\marginnote{1.32.7} has well-planted feet. 

On\marginnote{1.32.8} the soles of his feet there are thousand-spoked wheels, with rims and hubs, complete in every detail. 

He\marginnote{1.32.9} has projecting heels. 

He\marginnote{1.32.10} has long fingers. 

His\marginnote{1.32.11} hands and feet are tender. 

His\marginnote{1.32.12} hands and feet cling gracefully. 

His\marginnote{1.32.13} feet are arched. 

His\marginnote{1.32.14} calves are like those of an antelope. 

When\marginnote{1.32.15} standing upright and not bending over, the palms of both hands touch the knees. 

His\marginnote{1.32.16} private parts are covered in a foreskin. 

He\marginnote{1.32.17} is gold colored; his skin has a golden sheen. 

He\marginnote{1.32.18} has delicate skin, so delicate that dust and dirt don’t stick to his body. 

His\marginnote{1.32.19} hairs grow one per pore. 

His\marginnote{1.32.20} hairs stand up; they’re blue-black and curl clockwise. 

His\marginnote{1.32.21} body is tall and straight. 

He\marginnote{1.32.22} has bulging muscles in seven places. 

His\marginnote{1.32.23} chest is like that of a lion. 

The\marginnote{1.32.24} gap between the shoulder-blades is filled in. 

He\marginnote{1.32.25} has the proportional circumference of a banyan tree: the span of his arms equals the height of his body. 

His\marginnote{1.32.26} torso is cylindrical. 

He\marginnote{1.32.27} has an excellent sense of taste. 

His\marginnote{1.32.28} jaw is like that of a lion. 

He\marginnote{1.32.29} has forty teeth. 

His\marginnote{1.32.30} teeth are even. 

His\marginnote{1.32.31} teeth have no gaps. 

His\marginnote{1.32.32} teeth are perfectly white. 

He\marginnote{1.32.33} has a large tongue. 

He\marginnote{1.32.34} has the voice of \textsanskrit{Brahmā}, like a cuckoo’s call. 

His\marginnote{1.32.35} eyes are deep blue. 

He\marginnote{1.32.36} has eyelashes like a cow’s. 

Between\marginnote{1.32.37} his eyebrows there grows a tuft, soft and white like cotton-wool. 

His\marginnote{1.32.38} head is shaped like a turban. 

These\marginnote{1.33.1} are the thirty-two marks of a great man that the prince has. A great man who possesses these has only two possible destinies, no other. If he stays at home he becomes a king, a wheel-turning monarch. But if he goes forth from the lay life to homelessness, he becomes a perfected one, a fully awakened Buddha, who draws back the veil from the world.’ 

\section*{4. How He Came to be Known as \textsanskrit{Vipassī} }

Then\marginnote{1.33.9} King Bandhuma had the brahmin soothsayers dressed in fresh clothes and satisfied all their needs. Then the king appointed nurses for Prince \textsanskrit{Vipassī}. Some suckled him, some bathed him, some held him, and some carried him on their hip. From when he was born, a white parasol was held over him night and day, with the thought, ‘Don’t let cold, heat, grass, dust, or damp bother him.’ He was dear and beloved by many people, like a blue water lily, or a pink or white lotus. He was always passed from hip to hip. 

From\marginnote{1.35.1} when he was born, his voice was charming, graceful, sweet, and lovely. It was as sweet as the song of a cuckoo-bird found in the Himalayas. 

From\marginnote{1.36.1} when he was born, Prince \textsanskrit{Vipassī} had the power of clairvoyance which manifested as a result of past deeds. He could see for a league all around both by day and by night. 

And\marginnote{1.37.1} he was unblinkingly watchful, like the gods of the Thirty-Three. And because it was said that he was unblinkingly watchful, he came to be known as ‘\textsanskrit{Vipassī}’. 

Then\marginnote{1.37.3} while King Bandhuma was sitting in judgment, he’d sit Prince \textsanskrit{Vipassī} in his lap and explain the case to him. And sitting there in his father’s lap, \textsanskrit{Vipassī} would thoroughly consider the case and draw a conclusion using a logical procedure. So this was all the more reason for him to be known as ‘\textsanskrit{Vipassī}’. 

Then\marginnote{1.38.1} King Bandhuma had three stilt longhouses built for him—one for the winter, one for the summer, and one for the rainy season, and provided him with the five kinds of sensual stimulation. Prince \textsanskrit{Vipassī} stayed in a stilt longhouse without coming downstairs for the four months of the rainy season, where he was entertained by musicians—none of them men. 

\section*{5. The Old Man }

Then,\marginnote{2.1.1} after many thousand years had passed, Prince \textsanskrit{Vipassī} addressed his charioteer, ‘My dear charioteer, harness the finest chariots. We will go to a park and see the scenery.’ 

‘Yes,\marginnote{2.1.3} sir,’ replied the charioteer. He harnessed the chariots and informed the prince, ‘Sire, the finest chariots are harnessed. Please go at your convenience.’ Then Prince \textsanskrit{Vipassī} mounted a fine carriage and, along with other fine carriages, set out for the park. 

Along\marginnote{2.2.1} the way he saw an elderly man, bent double, crooked, leaning on a staff, trembling as he walked, ailing, past his prime. He addressed his charioteer, ‘My dear charioteer, what has that man done? For his hair and his body are unlike those of other men.’ 

‘That,\marginnote{2.2.5} Your Majesty, is called an old man.’ 

‘But\marginnote{2.2.6} why is he called an old man?’ 

‘He’s\marginnote{2.2.7} called an old man because now he has not long to live.’ 

‘But\marginnote{2.2.8} my dear charioteer, am I liable to grow old? Am I not exempt from old age?’ 

‘Everyone\marginnote{2.2.9} is liable to grow old, Your Majesty, including you. No-one is exempt from old age.’ 

‘Well\marginnote{2.2.10} then, my dear charioteer, that’s enough of the park for today. Let’s return to the royal compound.’ 

‘Yes,\marginnote{2.2.11} Your Majesty,’ replied the charioteer and did so. 

Back\marginnote{2.2.12} at the royal compound, the prince brooded, miserable and sad: ‘Damn this thing called rebirth, since old age will come to anyone who’s born.’ 

Then\marginnote{2.3.1} King Bandhuma summoned the charioteer and said, ‘My dear charioteer, I hope the prince enjoyed himself at the park? I hope he was happy there?’ 

‘No,\marginnote{2.3.3} Your Majesty, the prince didn’t enjoy himself at the park.’ 

‘But\marginnote{2.3.4} what did he see on the way to the park?’ And the charioteer told the king about seeing the old man and the prince’s reaction. 

\section*{6. The Sick Man }

Then\marginnote{2.4.1} King Bandhuma thought, ‘Prince \textsanskrit{Vipassī} must not renounce the throne. He must not go forth from the lay life to homelessness. And the words of the brahmin soothsayers must not come true.’ To this end he provided the prince with even more of the five kinds of sensual stimulation, with which the prince amused himself. 

Then,\marginnote{2.5.1} after many thousand years had passed, Prince \textsanskrit{Vipassī} had his charioteer drive him to the park once more. 

Along\marginnote{2.6.1} the way he saw a man who was ill, suffering, gravely ill, collapsed in his own urine and feces, being picked up by some and put down by others. He addressed his charioteer, ‘My dear charioteer, what has that man done? For his eyes and his voice are unlike those of other men.’ 

‘That,\marginnote{2.6.5} Your Majesty, is called a sick man.’ 

‘But\marginnote{2.6.6} why is he called a sick man?’ 

‘He’s\marginnote{2.6.7} called an sick man; hopefully he will recover from that illness.’ 

‘But\marginnote{2.6.8} my dear charioteer, am I liable to fall sick? Am I not exempt from sickness?’ 

‘Everyone\marginnote{2.6.9} is liable to fall sick, Your Majesty, including you. No-one is exempt from sickness.’ 

‘Well\marginnote{2.6.10} then, my dear charioteer, that’s enough of the park for today. Let’s return to the royal compound.’ 

‘Yes,\marginnote{2.6.11} Your Majesty,’ replied the charioteer and did so. 

Back\marginnote{2.6.12} at the royal compound, the prince brooded, miserable and sad: ‘Damn this thing called rebirth, since old age and sickness will come to anyone who’s born.’ 

Then\marginnote{2.7.1} King Bandhuma summoned the charioteer and said, ‘My dear charioteer, I hope the prince enjoyed himself at the park? I hope he was happy there?’ 

‘No,\marginnote{2.7.3} Your Majesty, the prince didn’t enjoy himself at the park.’ 

‘But\marginnote{2.7.4} what did he see on the way to the park?’ And the charioteer told the king about seeing the sick man and the prince’s reaction. 

\section*{7. The Dead Man }

Then\marginnote{2.8.1} King Bandhuma thought, ‘Prince \textsanskrit{Vipassī} must not renounce the throne. He must not go forth from the lay life to homelessness. And the words of the brahmin soothsayers must not come true.’ To this end he provided the prince with even more of the five kinds of sensual stimulation, with which the prince amused himself. 

Then,\marginnote{2.9.2} after many thousand years had passed, Prince \textsanskrit{Vipassī} had his charioteer drive him to the park once more. 

Along\marginnote{2.10.1} the way he saw a large crowd gathered making a bier out of garments of different colors. He addressed his charioteer, ‘My dear charioteer, why is that crowd making a bier?’ 

‘That,\marginnote{2.10.4} Your Majesty, is for someone who’s departed.’ 

‘Well\marginnote{2.10.5} then, drive the chariot up to the departed.’ 

‘Yes,\marginnote{2.10.6} Your Majesty,’ replied the charioteer, and did so. 

When\marginnote{2.10.7} the prince saw the corpse of the departed, he addressed the charioteer, ‘But why is he called departed?’ 

‘He’s\marginnote{2.10.9} called departed because now his mother and father, his relatives and kin shall see him no more, and he shall never again see them.’ 

‘But\marginnote{2.10.10} my dear charioteer, am I liable to die? Am I not exempt from death? Will the king and queen and my other relatives and kin see me no more? And shall I never again see them?’ 

‘Everyone\marginnote{2.10.13} is liable to die, Your Majesty, including you. No-one is exempt from death. The king and queen and your other relatives and kin shall see you no more, and you shall never again see them.’ 

‘Well\marginnote{2.10.16} then, my dear charioteer, that’s enough of the park for today. Let’s return to the royal compound.’ 

‘Yes,\marginnote{2.10.17} Your Majesty,’ replied the charioteer and did so. 

Back\marginnote{2.10.18} at the royal compound, the prince brooded, miserable and sad: ‘Damn this thing called rebirth, since old age, sickness, and death will come to anyone who’s born.’ 

Then\marginnote{2.11.1} King Bandhuma summoned the charioteer and said, ‘My dear charioteer, I hope the prince enjoyed himself at the park? I hope he was happy there?’ 

‘No,\marginnote{2.11.3} Your Majesty, the prince didn’t enjoy himself at the park.’ 

‘But\marginnote{2.11.4} what did he see on the way to the park?’ And the charioteer told the king about seeing the dead man and the prince’s reaction. 

\section*{8. The Renunciate }

Then\marginnote{2.12.1} King Bandhuma thought, ‘Prince \textsanskrit{Vipassī} must not renounce the throne. He must not go forth from the lay life to homelessness. And the words of the brahmin soothsayers must not come true.’ To this end he provided the prince with even more of the five kinds of sensual stimulation, with which the prince amused himself. 

Then,\marginnote{2.13.2} after many thousand years had passed, Prince \textsanskrit{Vipassī} had his charioteer drive him to the park once more. 

Along\marginnote{2.14.1} the way he saw a man, a renunciate with shaven head, wearing an ocher robe. He addressed his charioteer, ‘My dear charioteer, what has that man done? For his head and his clothes are unlike those of other men.’ 

‘That,\marginnote{2.14.5} Your Majesty, is called a renunciate.’ 

‘But\marginnote{2.14.6} why is he called a renunciate?’ 

‘He\marginnote{2.14.7} is called a renunciate because he celebrates principled and fair conduct, skillful actions, good deeds, harmlessness, and compassion for living creatures.’ 

‘Then\marginnote{2.14.8} I celebrate the one called a renunciate, who celebrates principled and fair conduct, skillful actions, good deeds, harmlessness, and compassion for living creatures! Well then, drive the chariot up to that renunciate.’ 

‘Yes,\marginnote{2.14.10} Your Majesty,’ replied the charioteer, and did so. 

Then\marginnote{2.14.11} Prince \textsanskrit{Vipassī} said to that renunciate, ‘My good man, what have you done? For your head and your clothes are unlike those of other men.’ 

‘Sire,\marginnote{2.14.14} I am what is called a renunciate.’ 

‘But\marginnote{2.14.15} why are you called a renunciate?’ 

‘I\marginnote{2.14.16} am called a renunciate because I celebrate principled and fair conduct, skillful actions, good deeds, harmlessness, and compassion for living creatures.’ 

‘Then\marginnote{2.14.17} I celebrate the one called a renunciate, who celebrates principled and fair conduct, skillful actions, good deeds, harmlessness, and compassion for living creatures!’ 

\section*{9. The Going Forth }

Then\marginnote{2.15.1} the prince addressed the charioteer, ‘Well then, my dear charioteer, take the chariot and return to the royal compound. I shall shave off my hair and beard right here, dress in ocher robes, and go forth from the lay life to homelessness.’ 

‘Yes,\marginnote{2.15.4} Your Majesty,’ replied the charioteer and did so. 

Then\marginnote{2.15.5} Prince \textsanskrit{Vipassī} shaved off his hair and beard, dressed in ocher robes, and went forth from the lay life to homelessness. 

\section*{10. A Great Crowd Goes Forth }

A\marginnote{2.16.1} large crowd of 84,000 people in the capital of \textsanskrit{Bandhumatī} heard that \textsanskrit{Vipassī} had gone forth. It occurred to them, ‘This must be no ordinary teaching and training, no ordinary going forth in which Prince \textsanskrit{Vipassī} has gone forth. If even the prince goes forth, why don’t we do the same?’ 

Then\marginnote{2.16.6} that great crowd of 84,000 people shaved off their hair and beard, dressed in ocher robes, and followed the one intent on awakening, \textsanskrit{Vipassī}, by going forth from the lay life to homelessness. Escorted by that assembly, \textsanskrit{Vipassī} wandered on tour among the villages, towns, and capital cities. 

Then\marginnote{2.17.1} as he was in private retreat this thought came to his mind, ‘It’s not appropriate for me to live in a crowd. Why don’t I live alone, withdrawn from the group?’ After some time he withdrew from the group to live alone. The 84,000 went one way, but \textsanskrit{Vipassī} went another. 

\section*{11. \textsanskrit{Vipassī}’s Reflections }

Then\marginnote{2.18.1} as \textsanskrit{Vipassī}, the one intent on awakening, was in private retreat this thought came to his mind, ‘Alas, this world has fallen into trouble. It’s born, grows old, dies, passes away, and is reborn, yet it doesn’t understand how to escape from this suffering, from old age and death. Oh, when will an escape be found from this suffering, from old age and death?’ 

Then\marginnote{2.18.4} \textsanskrit{Vipassī} thought, ‘When what exists is there old age and death? What is a condition for old age and death?’ Then, through proper attention, \textsanskrit{Vipassī} comprehended with wisdom, ‘When rebirth exists there’s old age and death. Rebirth is a condition for old age and death.’ 

Then\marginnote{2.18.8} \textsanskrit{Vipassī} thought, ‘When what exists is there rebirth? What is a condition for rebirth?’ Then, through proper attention, \textsanskrit{Vipassī} comprehended with wisdom, ‘When continued existence exists there’s rebirth. Continued existence is a condition for rebirth.’ 

Then\marginnote{2.18.12} \textsanskrit{Vipassī} thought, ‘When what exists is there continued existence? What is a condition for continued existence?’ Then, through proper attention, \textsanskrit{Vipassī} comprehended with wisdom, ‘When grasping exists there’s continued existence. Grasping is a condition for continued existence.’ 

Then\marginnote{2.18.16} \textsanskrit{Vipassī} thought, ‘When what exists is there grasping? What is a condition for grasping?’ Then, through proper attention, \textsanskrit{Vipassī} comprehended with wisdom, ‘When craving exists there’s grasping. Craving is a condition for grasping.’ 

Then\marginnote{2.18.20} \textsanskrit{Vipassī} thought, ‘When what exists is there craving? What is a condition for craving?’ Then, through proper attention, \textsanskrit{Vipassī} comprehended with wisdom, ‘When feeling exists there’s craving. Feeling is a condition for craving.’ 

Then\marginnote{2.18.24} \textsanskrit{Vipassī} thought, ‘When what exists is there feeling? What is a condition for feeling?’ Then, through proper attention, \textsanskrit{Vipassī} comprehended with wisdom, ‘When contact exists there’s feeling. Contact is a condition for feeling.’ 

Then\marginnote{2.18.28} \textsanskrit{Vipassī} thought, ‘When what exists is there contact? What is a condition for contact?’ Then, through proper attention, \textsanskrit{Vipassī} comprehended with wisdom, ‘When the six sense fields exist there’s contact. The six sense fields are a condition for contact.’ 

Then\marginnote{2.18.32} \textsanskrit{Vipassī} thought, ‘When what exists are there the six sense fields? What is a condition for the six sense fields?’ Then, through proper attention, \textsanskrit{Vipassī} comprehended with wisdom, ‘When name and form exist there are the six sense fields. Name and form are a condition for the six sense fields.’ 

Then\marginnote{2.18.36} \textsanskrit{Vipassī} thought, ‘When what exists are there name and form? What is a condition for name and form?’ Then, through proper attention, \textsanskrit{Vipassī} comprehended with wisdom, ‘When consciousness exists there are name and form. Consciousness is a condition for name and form.’ 

Then\marginnote{2.18.40} \textsanskrit{Vipassī} thought, ‘When what exists is there consciousness? What is a condition for consciousness?’ Then, through proper attention, \textsanskrit{Vipassī} comprehended with wisdom, ‘When name and form exist there’s consciousness. Name and form are a condition for consciousness.’ 

Then\marginnote{2.19.1} \textsanskrit{Vipassī} thought, ‘This consciousness turns back from name and form, and doesn’t go beyond that.’ It is to this extent that one may be reborn, grow old, die, pass away, or reappear. That is: Name and form are conditions for consciousness. Consciousness is a condition for name and form. Name and form are conditions for the six sense fields. The six sense fields are conditions for contact. Contact is a condition for feeling. Feeling is a condition for craving. Craving is a condition for grasping. Grasping is a condition for continued existence. Continued existence is a condition for rebirth. Rebirth is a condition for old age and death, sorrow, lamentation, pain, sadness, and distress to come to be. That is how this entire mass of suffering originates.’ 

‘Origination,\marginnote{2.19.6} origination.’ Such was the vision, knowledge, wisdom, realization, and light that arose in \textsanskrit{Vipassī}, the one intent on awakening, regarding teachings not learned before from another. 

Then\marginnote{2.20.1} \textsanskrit{Vipassī} thought, ‘When what doesn’t exist is there no old age and death? When what ceases do old age and death cease?’ Then, through proper attention, \textsanskrit{Vipassī} comprehended with wisdom, ‘When rebirth doesn’t exist there’s no old age and death. When rebirth ceases, old age and death cease.’ 

Then\marginnote{2.20.5} \textsanskrit{Vipassī} thought, ‘When what doesn’t exist is there no rebirth? When what ceases does rebirth cease?’ Then, through proper attention, \textsanskrit{Vipassī} comprehended with wisdom, ‘When continued existence doesn’t exist there’s no rebirth. When continued existence ceases, rebirth ceases.’ 

Then\marginnote{2.20.9} \textsanskrit{Vipassī} thought, ‘When what doesn’t exist is there no continued existence? When what ceases does continued existence cease?’ Then, through proper attention, \textsanskrit{Vipassī} comprehended with wisdom, ‘When grasping doesn’t exist there’s no continued existence. When grasping ceases, continued existence ceases.’ 

Then\marginnote{2.20.13} \textsanskrit{Vipassī} thought, ‘When what doesn’t exist is there no grasping? When what ceases does grasping cease?’ Then, through proper attention, \textsanskrit{Vipassī} comprehended with wisdom, ‘When craving doesn’t exist there’s no grasping. When craving ceases, grasping ceases.’ 

Then\marginnote{2.20.17} \textsanskrit{Vipassī} thought, ‘When what doesn’t exist is there no craving? When what ceases does craving cease?’ Then, through proper attention, \textsanskrit{Vipassī} comprehended with wisdom, ‘When feeling doesn’t exist there’s no craving. When feeling ceases, craving ceases.’ 

Then\marginnote{2.20.21} \textsanskrit{Vipassī} thought, ‘When what doesn’t exist is there no feeling? When what ceases does feeling cease?’ Then, through proper attention, \textsanskrit{Vipassī} comprehended with wisdom, ‘When contact doesn’t exist there’s no feeling. When contact ceases, feeling ceases.’ 

Then\marginnote{2.20.25} \textsanskrit{Vipassī} thought, ‘When what doesn’t exist is there no contact? When what ceases does contact cease?’ Then, through proper attention, \textsanskrit{Vipassī} comprehended with wisdom, ‘When the six sense fields don’t exist there’s no contact. When the six sense fields cease, contact ceases.’ 

Then\marginnote{2.20.29} \textsanskrit{Vipassī} thought, ‘When what doesn’t exist are there no six sense fields? When what ceases do the six sense fields cease?’ Then, through proper attention, \textsanskrit{Vipassī} comprehended with wisdom, ‘When name and form don’t exist there are no six sense fields. When name and form cease, the six sense fields cease.’ 

Then\marginnote{2.20.33} \textsanskrit{Vipassī} thought, ‘When what doesn’t exist are there no name and form? When what ceases do name and form cease?’ Then, through proper attention, \textsanskrit{Vipassī} comprehended with wisdom, ‘When consciousness doesn’t exist there are no name and form. When consciousness ceases, name and form cease.’ 

Then\marginnote{2.20.37} \textsanskrit{Vipassī} thought, ‘When what doesn’t exist is there no consciousness? When what ceases does consciousness cease?’ Then, through proper attention, \textsanskrit{Vipassī} comprehended with wisdom, ‘When name and form don’t exist there’s no consciousness. When name and form cease, consciousness ceases.’ 

Then\marginnote{2.21.1} \textsanskrit{Vipassī} thought, ‘I have discovered the path to awakening. That is: When name and form cease, consciousness ceases. When consciousness ceases, name and form cease. When name and form cease, the six sense fields cease. When the six sense fields cease, contact ceases. When contact ceases, feeling ceases. When feeling ceases, craving ceases. When craving ceases, grasping ceases. When grasping ceases, continued existence ceases. When continued existence ceases, rebirth ceases. When rebirth ceases, old age and death, sorrow, lamentation, pain, sadness, and distress cease. That is how this entire mass of suffering ceases.’ 

‘Cessation,\marginnote{2.21.5} cessation.’ Such was the vision, knowledge, wisdom, realization, and light that arose in \textsanskrit{Vipassī}, the one intent on awakening, regarding teachings not learned before from another. 

Some\marginnote{2.22.1} time later \textsanskrit{Vipassī} meditated observing rise and fall in the five grasping aggregates. ‘Such is form, such is the origin of form, such is the ending of form. Such is feeling, such is the origin of feeling, such is the ending of feeling. Such is perception, such is the origin of perception, such is the ending of perception. Such are choices, such is the origin of choices, such is the ending of choices. Such is consciousness, such is the origin of consciousness, such is the ending of consciousness.’ Meditating like this his mind was soon freed from defilements by not grasping. 

\section*{12. The Appeal of \textsanskrit{Brahmā} }

Then\marginnote{3.1.1} the Blessed One \textsanskrit{Vipassī}, the perfected one, the fully awakened Buddha, thought, ‘Why don’t I teach the Dhamma?’ 

Then\marginnote{3.1.3} he thought, ‘This principle I have discovered is deep, hard to see, hard to understand, peaceful, sublime, beyond the scope of logic, subtle, comprehensible to the astute. But people like attachment, they love it and enjoy it. It’s hard for them to see this thing; that is, specific conditionality, dependent origination. It’s also hard for them to see this thing; that is, the stilling of all activities, the letting go of all attachments, the ending of craving, fading away, cessation, extinguishment. And if I were to teach the Dhamma, others might not understand me, which would be wearying and troublesome for me.’ 

And\marginnote{3.2.1} then these verses, which were neither supernaturally inspired, nor learned before in the past, occurred to him: 

\begin{verse}%
‘I’ve\marginnote{3.2.2} struggled hard to realize this, \\
enough with trying to explain it! \\
This teaching is not easily understood \\
by those mired in greed and hate. 

Those\marginnote{3.2.6} besotted by greed can’t see \\
what’s subtle, going against the stream, \\
deep, hard to see, and very fine, \\
for they’re veiled in a mass of darkness.’ 

%
\end{verse}

So,\marginnote{3.2.10} as the Buddha \textsanskrit{Vipassī} reflected like this, his mind inclined to remaining passive, not to teaching the Dhamma. 

Then\marginnote{3.2.11} a certain Great \textsanskrit{Brahmā}, knowing what the Buddha \textsanskrit{Vipassī} was thinking, thought, ‘Oh my goodness! The world will be lost, the world will perish! For the mind of the Blessed One \textsanskrit{Vipassī}, the perfected one, the fully awakened Buddha, inclines to remaining passive, not to teaching the Dhamma.’ Then, as easily as a strong person would extend or contract their arm, he vanished from the \textsanskrit{Brahmā} realm and reappeared in front of the Buddha \textsanskrit{Vipassī}. He arranged his robe over one shoulder, knelt on his right knee, raised his joined palms toward the Buddha \textsanskrit{Vipassī}, and said, ‘Sir, let the Blessed One teach the Dhamma! Let the Holy One teach the Dhamma! There are beings with little dust in their eyes. They’re in decline because they haven’t heard the teaching. There will be those who understand the teaching!’ 

When\marginnote{3.4.1} he said this, the Buddha \textsanskrit{Vipassī} said to him, ‘I too thought this, \textsanskrit{Brahmā}, “Why don’t I teach the Dhamma?” Then it occurred to me, “If I were to teach the Dhamma, others might not understand me, which would be wearying and troublesome for me.” 

So,\marginnote{3.4.19} as I reflected like this, my mind inclined to remaining passive, not to teaching the Dhamma.’ 

For\marginnote{3.5.1} a second time, and a third time that Great \textsanskrit{Brahmā} begged the Buddha to teach. 

Then,\marginnote{3.6.1} understanding \textsanskrit{Brahmā}’s invitation, the Buddha \textsanskrit{Vipassī} surveyed the world with the eye of a Buddha, because of his compassion for sentient beings. And he saw sentient beings with little dust in their eyes, and some with much dust in their eyes; with keen faculties and with weak faculties, with good qualities and with bad qualities, easy to teach and hard to teach. And some of them lived seeing the danger in the fault to do with the next world, while others did not. It’s like a pool with blue water lilies, or pink or white lotuses. Some of them sprout and grow in the water without rising above it, thriving underwater. Some of them sprout and grow in the water reaching the water’s surface. And some of them sprout and grow in the water but rise up above the water and stand with no water clinging to them. 

In\marginnote{3.6.4} the same way, the Buddha \textsanskrit{Vipassī} saw sentient beings with little dust in their eyes, and some with much dust in their eyes. 

Then\marginnote{3.7.1} that Great \textsanskrit{Brahmā}, knowing what the Buddha \textsanskrit{Vipassī} was thinking, addressed him in verse: 

\begin{verse}%
‘Standing\marginnote{3.7.2} high on a rocky mountain, \\
you can see the people all around. \\
In just the same way, all-seer, wise one, \\
having ascended the Temple of Truth, \\
rid of sorrow, look upon the people \\
swamped with sorrow, oppressed by rebirth and old age. 

Rise,\marginnote{3.7.8} hero! Victor in battle, leader of the caravan, \\
wander the world without obligation. \\
Let the Blessed One teach the Dhamma! \\
There will be those who understand!’ 

%
\end{verse}

Then\marginnote{3.7.12} the Buddha \textsanskrit{Vipassī} addressed that Great \textsanskrit{Brahmā} in verse: 

\begin{verse}%
‘Flung\marginnote{3.7.13} open are the doors to the deathless! \\
Let those with ears to hear commit to faith. \\
Thinking it would be troublesome, \textsanskrit{Brahmā}, I did not teach \\
the sophisticated, sublime Dhamma among humans.’ 

%
\end{verse}

Then\marginnote{3.7.17} the Great \textsanskrit{Brahmā}, knowing that his request for the Buddha \textsanskrit{Vipassī} to teach the Dhamma had been granted, bowed and respectfully circled him, keeping him on his right, before vanishing right there. 

\section*{13. The Chief Disciples }

Then\marginnote{3.8.1} the Blessed One \textsanskrit{Vipassī}, the perfected one, the fully awakened Buddha, thought, ‘Who should I teach first of all? Who will quickly understand this teaching?’ Then he thought, ‘That \textsanskrit{Khaṇḍa}, the king’s son, and Tissa, the high priest’s son, are astute, competent, clever, and have long had little dust in their eyes. Why don’t I teach them first of all? They will quickly understand this teaching.’ 

Then,\marginnote{3.9.1} as easily as a strong person would extend or contract their arm, he vanished from the tree of awakening and reappeared near the capital city of \textsanskrit{Bandhumatī}, in the deer park named Sanctuary. 

Then\marginnote{3.9.2} the Buddha \textsanskrit{Vipassī} addressed the park keeper, ‘My dear park keeper, please enter the city and say this to the king’s son \textsanskrit{Khaṇḍa} and the high priest’s son Tissa: “Sirs, the Blessed One \textsanskrit{Vipassī}, the perfected one, the fully awakened Buddha, has arrived at \textsanskrit{Bandhumatī} and is staying in the deer park named Sanctuary. He wishes to see you.”’ 

‘Yes,\marginnote{3.9.5} sir,’ replied the park keeper, and did as he was asked. 

Then\marginnote{3.10.1} the king’s son \textsanskrit{Khaṇḍa} and the high priest’s son Tissa had the finest carriages harnessed. Then they mounted a fine carriage and, along with other fine carriages, set out from \textsanskrit{Bandhumatī} for the Sanctuary. They went by carriage as far as the terrain allowed, then descended and approached the Buddha \textsanskrit{Vipassī} on foot. They bowed and sat down to one side. 

The\marginnote{3.11.1} Buddha \textsanskrit{Vipassī} taught them step by step, with a talk on giving, ethical conduct, and heaven. He explained the drawbacks of sensual pleasures, so sordid and corrupt, and the benefit of renunciation. And when he knew that their minds were ready, pliable, rid of hindrances, elated, and confident he explained the special teaching of the Buddhas: suffering, its origin, its cessation, and the path. Just as a clean cloth rid of stains would properly absorb dye, in that very seat the stainless, immaculate vision of the Dhamma arose in the king’s son \textsanskrit{Khaṇḍa} and the high priest’s son Tissa: ‘Everything that has a beginning has an end.’ 

They\marginnote{3.12.1} saw, attained, understood, and fathomed the Dhamma. They went beyond doubt, got rid of indecision, and became self-assured and independent of others regarding the Teacher’s instructions. They said to the Buddha \textsanskrit{Vipassī}, ‘Excellent, sir! Excellent! As if he were righting the overturned, or revealing the hidden, or pointing out the path to the lost, or lighting a lamp in the dark so people with good eyes can see what’s there, the Buddha has made the teaching clear in many ways. We go for refuge to the Blessed One and to the teaching. Sir, may we receive the going forth and ordination in the Buddha’s presence?’ 

And\marginnote{3.13.1} they received the going forth, the ordination in the Buddha \textsanskrit{Vipassī}’s presence. Then the Buddha \textsanskrit{Vipassī} educated, encouraged, fired up, and inspired them with a Dhamma talk. He explained the drawbacks of conditioned phenomena, so sordid and corrupt, and the benefit of extinguishment. Being taught like this their minds were soon freed from defilements by not grasping. 

\section*{14. The Going Forth of the Large Crowd }

A\marginnote{3.14.1} large crowd of 84,000 people in the capital of \textsanskrit{Bandhumatī} heard that the Blessed One \textsanskrit{Vipassī}, the perfected one, the fully awakened Buddha, had arrived at \textsanskrit{Bandhumatī} and was staying in the deer park named Sanctuary. And they heard that the king’s son \textsanskrit{Khaṇḍa} and the high priest’s son Tissa had shaved off their hair and beard, dressed in ocher robes, and gone forth from the lay life to homelessness in the Buddha’s presence. It occurred to them, ‘This must be no ordinary teaching and training, no ordinary going forth in which the king’s son \textsanskrit{Khaṇḍa} and the high priest’s son Tissa have gone forth. If even they go forth, why don’t we do the same?’ Then those 84,000 people left \textsanskrit{Bandhumatī} for the deer park named Sanctuary, where they approached the Buddha \textsanskrit{Vipassī}, bowed and sat down to one side. 

The\marginnote{3.15.1} Buddha \textsanskrit{Vipassī} taught them step by step, with a talk on giving, ethical conduct, and heaven. He explained the drawbacks of sensual pleasures, so sordid and corrupt, and the benefit of renunciation. And when he knew that their minds were ready, pliable, rid of hindrances, elated, and confident he explained the special teaching of the Buddhas: suffering, its origin, its cessation, and the path. Just as a clean cloth rid of stains would properly absorb dye, in that very seat the stainless, immaculate vision of the Dhamma arose in those 84,000 people: ‘Everything that has a beginning has an end.’ 

They\marginnote{3.16.1} saw, attained, understood, and fathomed the Dhamma. They went beyond doubt, got rid of indecision, and became self-assured and independent of others regarding the Teacher’s instructions. They said to the Buddha \textsanskrit{Vipassī}, ‘Excellent, sir! Excellent!’ And just like \textsanskrit{Khaṇḍa} and Tissa they asked for and received ordination. Then the Buddha taught them further. 

Being\marginnote{3.17.1} taught like this their minds were soon freed from defilements by not grasping. 

\section*{15. The 84,000 Who Had Gone Forth Previously }

The\marginnote{3.18.1} 84,000 people who had gone forth previously also heard: ‘It seems the Blessed One \textsanskrit{Vipassī}, the perfected one, the fully awakened Buddha, has arrived at \textsanskrit{Bandhumatī} and is staying in the deer park named Sanctuary. And he is teaching the Dhamma!’ Then they too went to see the Buddha \textsanskrit{Vipassī}, realized the Dhamma, went forth, and became freed from defilements. 

\section*{16. The Allowance to Wander }

Now\marginnote{3.22.1} at that time a large \textsanskrit{Saṅgha} of 6,800,000 mendicants were residing at \textsanskrit{Bandhumatī}. As the Buddha \textsanskrit{Vipassī} was in private retreat this thought came to his mind, ‘The \textsanskrit{Saṅgha} residing at \textsanskrit{Bandhumatī} now is large. What if I was to urge them: 

“Wander\marginnote{3.22.4} forth, mendicants, for the welfare and happiness of the people, out of compassion for the world, for the benefit, welfare, and happiness of gods and humans. Let not two go by one road. Teach the Dhamma that’s good in the beginning, good in the middle, and good in the end, meaningful and well-phrased. And reveal a spiritual practice that’s entirely full and pure. There are beings with little dust in their eyes. They’re in decline because they haven’t heard the teaching. There will be those who understand the teaching! But when six years have passed, you must all come to \textsanskrit{Bandhumatī} to recite the monastic code.”’ 

Then\marginnote{3.23.1} a certain Great \textsanskrit{Brahmā}, knowing what the Buddha \textsanskrit{Vipassī} was thinking, as easily as a strong person would extend or contract their arm, vanished from the \textsanskrit{Brahmā} realm and reappeared in front of the Buddha \textsanskrit{Vipassī}. He arranged his robe over one shoulder, raised his joined palms toward the Buddha \textsanskrit{Vipassī}, and said, ‘That’s so true, Blessed One! That’s so true, Holy One! The \textsanskrit{Saṅgha} residing at \textsanskrit{Bandhumatī} now is large. Please urge them to wander, as you thought. And sir, I’ll make sure that when six years have passed the mendicants will return to \textsanskrit{Bandhumatī} to recite the monastic code.’ 

That’s\marginnote{3.23.10} what that Great \textsanskrit{Brahmā} said. Then he bowed and respectfully circled the Buddha \textsanskrit{Vipassī}, keeping him on his right side, before vanishing right there. 

Then\marginnote{3.24.1} in the late afternoon, the Buddha \textsanskrit{Vipassī} came out of retreat and addressed the mendicants, telling them all that had happened. Then he said, 

‘Wander\marginnote{3.26.1} forth, mendicants, for the welfare and happiness of the people, out of compassion for the world, for the benefit, welfare, and happiness of gods and humans. Let not two go by one road. Teach the Dhamma that’s good in the beginning, good in the middle, and good in the end, meaningful and well-phrased. And reveal a spiritual practice that’s entirely full and pure. There are beings with little dust in their eyes. They’re in decline because they haven’t heard the teaching. There will be those who understand the teaching! But when six years have passed, you must all come to \textsanskrit{Bandhumatī} to recite the monastic code.’ 

Then\marginnote{3.26.7} most of the mendicants departed to wander the country that very day. 

Now\marginnote{3.27.1} at that time there were 84,000 monasteries in India. And when the first year came to an end the deities raised the cry: ‘Good sirs, the first year has ended. Now five years remain. When five years have passed, you must all go to \textsanskrit{Bandhumatī} to recite the monastic code.’ 

And\marginnote{3.27.6} when the second year … the third year … the fourth year … the fifth year came to an end, the deities raised the cry: ‘Good sirs, the fifth year has ended. Now one year remains. When one year has passed, you must all go to \textsanskrit{Bandhumatī} to recite the monastic code.’ 

And\marginnote{3.27.13} when the sixth year came to an end the deities raised the cry: ‘Good sirs, the sixth year has ended. Now is the time that you must go to \textsanskrit{Bandhumatī} to recite the monastic code.’ Then that very day the mendicants went to \textsanskrit{Bandhumatī} to recite the monastic code. Some went by their own psychic power, and some by the psychic power of the deities. 

And\marginnote{3.28.1} there the Blessed One \textsanskrit{Vipassī}, the perfected one, the fully awakened Buddha, recited the monastic code thus: 

\begin{verse}%
‘Patient\marginnote{3.28.2} acceptance is the ultimate austerity. \\
Extinguishment is the ultimate, say the Buddhas. \\
No true renunciate injures another, \\
nor does an ascetic hurt another. 

Not\marginnote{3.28.6} to do any evil; \\
to embrace the good; \\
to purify one’s mind: \\
this is the instruction of the Buddhas. 

Not\marginnote{3.28.10} speaking ill nor doing harm; \\
restraint in the monastic code; \\
moderation in eating; \\
staying in remote lodgings; \\
commitment to the higher mind—\\
this is the instruction of the Buddhas.’ 

%
\end{verse}

\section*{17. Being Informed by Deities }

At\marginnote{3.29.1} one time, mendicants, I was staying near \textsanskrit{Ukkaṭṭhā}, in the Subhaga Forest at the root of a magnificent sal tree. As I was in private retreat this thought came to mind, ‘It’s not easy to find an abode of sentient beings where I haven’t previously abided in all this long time, except for the gods of the pure abodes. Why don’t I go to see them?’ 

Then,\marginnote{3.29.5} as easily as a strong person would extend or contract their arm, I vanished from the Subhaga Forest and reappeared with the Aviha gods. 

In\marginnote{3.29.6} that order of gods, many thousands, many hundreds of thousands of deities approached me, bowed, stood to one side, and said to me, ‘Ninety-one eons ago, good sir, the Buddha \textsanskrit{Vipassī} arose in the world, perfected and fully awakened. He was born as an aristocrat into an aristocrat family. \textsanskrit{Koṇḍañña} was his clan. He lived for 80,000 years. He was awakened at the root of a trumpet flower tree. He had a fine pair of chief disciples named \textsanskrit{Khaṇḍa} and Tissa. He had three gatherings of disciples—one of 6,800,000, one of 100,000, and one of 80,000—all of them mendicants who had ended their defilements. He had as chief attendant a mendicant named Asoka. His father was King Bandhuma, his birth mother was Queen \textsanskrit{Bandhumatī}, and their capital city was named \textsanskrit{Bandhumatī}. And such was his renunciation, such his going forth, such his striving, such his awakening, and such his rolling forth of the wheel of Dhamma. And good sir, after leading the spiritual life under that Buddha \textsanskrit{Vipassī} we lost our desire for sensual pleasures and were reborn here.’ 

And\marginnote{3.29.20} other deities came and similarly recounted the details of the Buddhas \textsanskrit{Sikhī}, \textsanskrit{Vessabhū}, Kakusandha, \textsanskrit{Koṇāgamana}, and Kassapa. 

In\marginnote{3.30.1} that order of gods, many hundreds of deities approached me, bowed, stood to one side, and said to me, ‘In the present fortunate eon, good sir, you have arisen in the world, perfected and fully awakened. You were born as an aristocrat into an aristocrat family. Gotama is your clan. For you the life-span is short, brief, and fleeting. A long-lived person lives for a century or a little more. You were awakened at the root of a peepul tree. You have a fine pair of chief disciples named \textsanskrit{Sāriputta} and \textsanskrit{Moggallāna}. You have had one gathering of disciples—1,250 mendicants who had ended their defilements. You have as chief attendant a mendicant named Ānanda. Your father was King Suddhodana, your birth mother was Queen \textsanskrit{Māyā}, and your capital city was Kapilavatthu. And such was your renunciation, such your going forth, such your striving, such your awakening, and such your rolling forth of the wheel of Dhamma. And good sir, after leading the spiritual life under you we lost our desire for sensual pleasures and were reborn here.’ 

Then\marginnote{3.31.1} together with the Aviha gods I went to see the Atappa gods … the Gods Fair to See … and the Fair Seeing Gods. Then together with all these gods I went to see the Gods of \textsanskrit{Akaniṭṭha}, where we had a similar conversation. 

And\marginnote{3.33.1} that is how the Realized One is able to recollect the caste, names, clans, life-span, chief disciples, and gatherings of disciples of the Buddhas of the past who have become completely extinguished, cut off proliferation, cut off the track, finished off the cycle, and transcended suffering. It is both because I have clearly comprehended the principle of the teachings, and also because the deities told me.” 

That\marginnote{3.33.5} is what the Buddha said. Satisfied, the mendicants were happy with what the Buddha said. 

%
\chapter*{{\suttatitleacronym DN 15}{\suttatitletranslation The Great Discourse on Causation }{\suttatitleroot Mahānidānasutta}}
\addcontentsline{toc}{chapter}{\tocacronym{DN 15} \toctranslation{The Great Discourse on Causation } \tocroot{Mahānidānasutta}}
\markboth{The Great Discourse on Causation }{Mahānidānasutta}
\extramarks{DN 15}{DN 15}

\section*{1. Dependent Origination }

\scevam{So\marginnote{1.1} I have heard. }At one time the Buddha was staying in the land of the Kurus, near the Kuru town named \textsanskrit{Kammāsadamma}. 

Then\marginnote{1.3} Venerable Ānanda went up to the Buddha, bowed, sat down to one side, and said to him, “It’s incredible, sir, it’s amazing, in that this dependent origination is deep and appears deep, yet to me it seems as plain as can be.” 

“Don’t\marginnote{1.6} say that, Ānanda, don’t say that! This dependent origination is deep and appears deep. It is because of not understanding and not penetrating this teaching that this population has become tangled like string, knotted like a ball of thread, and matted like rushes and reeds, and it doesn’t escape the places of loss, the bad places, the underworld, transmigration. 

When\marginnote{2.1} asked, ‘Is there a specific condition for old age and death?’ you should answer, ‘There is.’ If they say, ‘What is a condition for old age and death?’ you should answer, ‘Rebirth is a condition for old age and death.’ 

When\marginnote{2.3} asked, ‘Is there a specific condition for rebirth?’ you should answer, ‘There is.’ If they say, ‘What is a condition for rebirth?’ you should answer, ‘Continued existence is a condition for rebirth.’ 

When\marginnote{2.5} asked, ‘Is there a specific condition for continued existence?’ you should answer, ‘There is.’ If they say, ‘What is a condition for continued existence?’ you should answer, ‘Grasping is a condition for continued existence.’ 

When\marginnote{2.7} asked, ‘Is there a specific condition for grasping?’ you should answer, ‘There is.’ If they say, ‘What is a condition for grasping?’ you should answer, ‘Craving is a condition for grasping.’ 

When\marginnote{2.9} asked, ‘Is there a specific condition for craving?’ you should answer, ‘There is.’ If they say, ‘What is a condition for craving?’ you should answer, ‘Feeling is a condition for craving.’ 

When\marginnote{2.11} asked, ‘Is there a specific condition for feeling?’ you should answer, ‘There is.’ If they say, ‘What is a condition for feeling?’ you should answer, ‘Contact is a condition for feeling.’ 

When\marginnote{2.13} asked, ‘Is there a specific condition for contact?’ you should answer, ‘There is.’ If they say, ‘What is a condition for contact?’ you should answer, ‘Name and form are conditions for contact.’ 

When\marginnote{2.15} asked, ‘Is there a specific condition for name and form?’ you should answer, ‘There is.’ If they say, ‘What is a condition for name and form?’ you should answer, ‘Consciousness is a condition for name and form.’ 

When\marginnote{2.17} asked, ‘Is there a specific condition for consciousness?’ you should answer, ‘There is.’ If they say, ‘What is a condition for consciousness?’ you should answer, ‘Name and form are conditions for consciousness.’ 

So:\marginnote{3.1} name and form are conditions for consciousness. Consciousness is a condition for name and form. Name and form are conditions for contact. Contact is a condition for feeling. Feeling is a condition for craving. Craving is a condition for grasping. Grasping is a condition for continued existence. Continued existence is a condition for rebirth. Rebirth is a condition for old age and death, sorrow, lamentation, pain, sadness, and distress to come to be. That is how this entire mass of suffering originates. 

‘Rebirth\marginnote{4.1} is a condition for old age and death’—that’s what I said. And this is a way to understand how this is so. Suppose there were totally and utterly no rebirth for anyone anywhere. That is, there were no rebirth of sentient beings into their various realms—of gods, fairies, spirits, creatures, humans, quadrupeds, birds, or reptiles, each into their own realm. When there’s no rebirth at all, with the cessation of rebirth, would old age and death still be found?” 

“No,\marginnote{4.4} sir.” 

“That’s\marginnote{4.5} why this is the cause, source, origin, and reason of old age and death, namely rebirth. 

‘Continued\marginnote{5.1} existence is a condition for rebirth’—that’s what I said. And this is a way to understand how this is so. Suppose there were totally and utterly no continued existence for anyone anywhere. That is, continued existence in the sensual realm, the realm of luminous form, or the formless realm. When there’s no continued existence at all, with the cessation of continued existence, would rebirth still be found?” 

“No,\marginnote{5.4} sir.” 

“That’s\marginnote{5.5} why this is the cause, source, origin, and reason of rebirth, namely continued existence. 

‘Grasping\marginnote{6.1} is a condition for continued existence’—that’s what I said. And this is a way to understand how this is so. Suppose there were totally and utterly no grasping for anyone anywhere. That is, grasping at sensual pleasures, views, precepts and observances, and theories of a self. When there’s no grasping at all, with the cessation of grasping, would continued existence still be found?” 

“No,\marginnote{6.4} sir.” 

“That’s\marginnote{6.5} why this is the cause, source, origin, and reason of continued existence, namely grasping. 

‘Craving\marginnote{7.1} is a condition for grasping’—that’s what I said. And this is a way to understand how this is so. Suppose there were totally and utterly no craving for anyone anywhere. That is, craving for sights, sounds, smells, tastes, touches, and thoughts. When there’s no craving at all, with the cessation of craving, would grasping still be found?” 

“No,\marginnote{7.4} sir.” 

“That’s\marginnote{7.5} why this is the cause, source, origin, and reason of grasping, namely craving. 

‘Feeling\marginnote{8.1} is a condition for craving’—that’s what I said. And this is a way to understand how this is so. Suppose there were totally and utterly no feeling for anyone anywhere. That is, feeling born of contact through the eye, ear, nose, tongue, body, and mind. When there’s no feeling at all, with the cessation of feeling, would craving still be found?” 

“No,\marginnote{8.4} sir.” 

“That’s\marginnote{8.5} why this is the cause, source, origin, and reason of craving, namely feeling. 

So\marginnote{9.1} it is, Ānanda, that feeling is a cause of craving. Craving is a cause of seeking. Seeking is a cause of gaining material possessions. Gaining material possessions is a cause of assessing. Assessing is a cause of desire and lust. Desire and lust is a cause of attachment. Attachment is a cause of ownership. Ownership is a cause of stinginess. Stinginess is a cause of safeguarding. Owing to safeguarding, many bad, unskillful things come to be: taking up the rod and the sword, quarrels, arguments, and disputes, accusations, divisive speech, and lies. 

‘Owing\marginnote{10.1} to safeguarding, many bad, unskillful things come to be: taking up the rod and the sword, quarrels, arguments, and disputes, accusations, divisive speech, and lies’—that’s what I said. And this is a way to understand how this is so. Suppose there were totally and utterly no safeguarding for anyone anywhere. When there’s no safeguarding at all, with the cessation of safeguarding, would those many bad, unskillful things still come to be?” 

“No,\marginnote{10.3} sir.” 

“That’s\marginnote{10.4} why this is the cause, source, origin, and reason for the origination of those many bad, unskillful things, namely safeguarding. 

‘Stinginess\marginnote{11.1} is a cause of safeguarding’—that’s what I said. And this is a way to understand how this is so. Suppose there were totally and utterly no stinginess for anyone anywhere. When there’s no stinginess at all, with the cessation of stinginess, would safeguarding still be found?” 

“No,\marginnote{11.3} sir.” 

“That’s\marginnote{11.4} why this is the cause, source, origin, and reason of safeguarding, namely stinginess. 

‘Ownership\marginnote{12.1} is a cause of stinginess’—that’s what I said. And this is a way to understand how this is so. Suppose there were totally and utterly no ownership for anyone anywhere. When there’s no ownership at all, with the cessation of ownership, would stinginess still be found?” 

“No,\marginnote{12.3} sir.” 

“That’s\marginnote{12.4} why this is the cause, source, origin, and reason of stinginess, namely ownership. 

‘Attachment\marginnote{13.1} is a cause of ownership’—that’s what I said. And this is a way to understand how this is so. Suppose there were totally and utterly no attachment for anyone anywhere. When there’s no attachment at all, with the cessation of attachment, would ownership still be found?” 

“No,\marginnote{13.3} sir.” 

“That’s\marginnote{13.4} why this is the cause, source, origin, and reason of ownership, namely attachment. 

‘Desire\marginnote{14.1} and lust is a cause of attachment’—that’s what I said. And this is a way to understand how this is so. Suppose there were totally and utterly no desire and lust for anyone anywhere. When there’s no desire and lust at all, with the cessation of desire and lust, would attachment still be found?” 

“No,\marginnote{14.3} sir.” 

“That’s\marginnote{14.4} why this is the cause, source, origin, and reason of attachment, namely desire and lust. 

‘Assessing\marginnote{15.1} is a cause of desire and lust’—that’s what I said. And this is a way to understand how this is so. Suppose there were totally and utterly no assessing for anyone anywhere. When there’s no assessing at all, with the cessation of assessing, would desire and lust still be found?” 

“No,\marginnote{15.3} sir.” 

“That’s\marginnote{15.4} why this is the cause, source, origin, and reason of desire and lust, namely assessing. 

‘Gaining\marginnote{16.1} material possessions is a cause of assessing’—that’s what I said. And this is a way to understand how this is so. Suppose there were totally and utterly no gaining of material possessions for anyone anywhere. When there’s no gaining of material possessions at all, with the cessation of gaining material possessions, would assessing still be found?” 

“No,\marginnote{16.3} sir.” 

“That’s\marginnote{16.4} why this is the cause, source, origin, and reason of assessing, namely the gaining of material possessions. 

‘Seeking\marginnote{17.1} is a cause of gaining material possessions’—that’s what I said. And this is a way to understand how this is so. Suppose there were totally and utterly no seeking for anyone anywhere. When there’s no seeking at all, with the cessation of seeking, would the gaining of material possessions still be found?” 

“No,\marginnote{17.3} sir.” 

“That’s\marginnote{17.4} why this is the cause, source, origin, and reason of gaining material possessions, namely seeking. 

‘Craving\marginnote{18.1} is a cause of seeking’—that’s what I said. And this is a way to understand how this is so. Suppose there were totally and utterly no craving for anyone anywhere. That is, craving for sensual pleasures, craving for continued existence, and craving to end existence. When there’s no craving at all, with the cessation of craving, would seeking still be found?” 

“No,\marginnote{18.4} sir.” 

“That’s\marginnote{18.5} why this is the cause, source, origin, and reason of seeking, namely craving. And so, Ānanda, these two things are united by the two aspects of feeling. 

‘Contact\marginnote{19.1} is a condition for feeling’—that’s what I said. And this is a way to understand how this is so. Suppose there were totally and utterly no contact for anyone anywhere. That is, contact through the eye, ear, nose, tongue, body, and mind. When there’s no contact at all, with the cessation of contact, would feeling still be found?” 

“No,\marginnote{19.4} sir.” 

“That’s\marginnote{19.5} why this is the cause, source, origin, and reason of feeling, namely contact. 

‘Name\marginnote{20.1} and form are conditions for contact’—that’s what I said. And this is a way to understand how this is so. Suppose there were none of the features, attributes, signs, and details by which the category of mental phenomena is found. Would linguistic contact still be found in the category of physical phenomena?” 

“No,\marginnote{20.3} sir.” 

“Suppose\marginnote{20.4} there were none of the features, attributes, signs, and details by which the category of physical phenomena is found. Would impingement contact still be found in the category of mental phenomena?” 

“No,\marginnote{20.5} sir.” 

“Suppose\marginnote{20.6} there were none of the features, attributes, signs, and details by which the categories of mental or physical phenomena are found. Would either linguistic contact or impingement contact still be found?” 

“No,\marginnote{20.7} sir.” 

“Suppose\marginnote{20.8} there were none of the features, attributes, signs, and details by which name and form are found. Would contact still be found?” 

“No,\marginnote{20.9} sir.” 

“That’s\marginnote{20.10} why this is the cause, source, origin, and reason of contact, namely name and form. 

‘Consciousness\marginnote{21.1} is a condition for name and form’—that’s what I said. And this is a way to understand how this is so. If consciousness were not conceived in the mother’s womb, would name and form coagulate there?” 

“No,\marginnote{21.3} sir.” 

“If\marginnote{21.4} consciousness, after being conceived in the mother’s womb, were to be miscarried, would name and form be born into this state of existence?” 

“No,\marginnote{21.5} sir.” 

“If\marginnote{21.6} the consciousness of a young boy or girl were to be cut off, would name and form achieve growth, increase, and maturity?” 

“No,\marginnote{21.7} sir.” 

“That’s\marginnote{21.8} why this is the cause, source, origin, and reason of name and form, namely consciousness. 

‘Name\marginnote{22.1} and form are conditions for consciousness’—that’s what I said. And this is a way to understand how this is so. If consciousness were not to become established in name and form, would the coming to be of the origin of suffering—of rebirth, old age, and death in the future—be found?” 

“No,\marginnote{22.3} sir.” 

“That’s\marginnote{22.4} why this is the cause, source, origin, and reason of consciousness, namely name and form. This is the extent to which one may be reborn, grow old, die, pass away, or reappear. This is how far the scope of language, terminology, and description extends; how far the sphere of wisdom extends; how far the cycle of rebirths proceeds so that this state of existence is to be found; namely, name and form together with consciousness. 

\section*{2. Describing the Self }

How\marginnote{23.1} do those who describe the self describe it? They describe it as physical and limited: ‘My self is physical and limited.’ Or they describe it as physical and infinite: ‘My self is physical and infinite.’ Or they describe it as formless and limited: ‘My self is formless and limited.’ Or they describe it as formless and infinite: ‘My self is formless and infinite.’ 

Now,\marginnote{24.1} take those who describe the self as physical and limited. They describe the self as physical and limited in the present; or in some future life; or else they think: ‘Though it is not like that, I will ensure it is provided with what it needs to become like that.’ This being so, it’s appropriate to say that a view of self as physical and limited underlies them. 

Now,\marginnote{24.4} take those who describe the self as physical and infinite … formless and limited … formless and infinite. They describe the self as formless and infinite in the present; or in some future life; or else they think: ‘Though it is not like that, I will ensure it is provided with what it needs to become like that.’ This being so, it’s appropriate to say that a view of self as formless and infinite underlies them. That’s how those who describe the self describe it. 

\section*{3. Not Describing the Self }

How\marginnote{25.1} do those who don’t describe the self not describe it? They don’t describe it as physical and limited … physical and infinite … formless and limited … formless and infinite: ‘My self is formless and infinite.’ 

Now,\marginnote{26.1} take those who don’t describe the self as physical and limited … physical and infinite … formless and limited … formless and infinite. They don’t describe the self as formless and infinite in the present; or in some future life; and they don’t think: ‘Though it is not like that, I will ensure it is provided with what it needs to become like that.’ This being so, it’s appropriate to say that a view of self as formless and infinite doesn’t underlie them. That’s how those who don’t describe the self don’t describe it. 

\section*{4. Regarding a Self }

How\marginnote{27.1} do those who regard the self regard it? They regard feeling as self: ‘Feeling is my self.’ Or they regard it like this: ‘Feeling is definitely not my self. My self does not experience feeling.’ Or they regard it like this: ‘Feeling is definitely not my self. But it’s not that my self does not experience feeling. My self feels, for my self is liable to feel.’ 

Now,\marginnote{28.1} as to those who say: ‘Feeling is my self.’ You should say this to them: ‘Reverend, there are three feelings: pleasant, painful, and neutral. Which one of these do you regard as self?’ Ānanda, at a time when you feel a pleasant feeling, you don’t feel a painful or neutral feeling; you only feel a pleasant feeling. At a time when you feel a painful feeling, you don’t feel a pleasant or neutral feeling; you only feel a painful feeling. At a time when you feel a neutral feeling, you don’t feel a pleasant or painful feeling; you only feel a neutral feeling. 

Pleasant\marginnote{29.1} feelings, painful feelings, and neutral feelings are all impermanent, conditioned, dependently originated, liable to end, vanish, fade away, and cease. When feeling a pleasant feeling they think: ‘This is my self.’ When their pleasant feeling ceases they think: ‘My self has disappeared.’ When feeling a painful feeling they think: ‘This is my self.’ When their painful feeling ceases they think: ‘My self has disappeared.’ When feeling a neutral feeling they think: ‘This is my self.’ When their neutral feeling ceases they think: ‘My self has disappeared.’ So those who say ‘feeling is my self’ regard as self that which is evidently impermanent, a mixture of pleasure and pain, and liable to rise and fall. That’s why it’s not acceptable to regard feeling as self. 

Now,\marginnote{30.1} as to those who say: ‘Feeling is definitely not my self. My self does not experience feeling.’ You should say this to them, ‘But reverend, where there is nothing felt at all, would the thought “I am” occur there?’” 

“No,\marginnote{30.4} sir.” 

“That’s\marginnote{30.5} why it’s not acceptable to regard self as that which does not experience feeling. 

Now,\marginnote{31.1} as to those who say: ‘Feeling is definitely not my self. But it’s not that my self does not experience feeling. My self feels, for my self is liable to feel.’ You should say this to them, ‘Suppose feelings were to totally and utterly cease without anything left over. When there’s no feeling at all, with the cessation of feeling, would the thought “I am this” occur there?’” 

“No,\marginnote{31.6} sir.” 

“That’s\marginnote{31.7} why it’s not acceptable to regard self as that which is liable to feel. 

Not\marginnote{32.1} regarding anything in this way, they don’t grasp at anything in the world. Not grasping, they’re not anxious. Not being anxious, they personally become extinguished. They understand: ‘Rebirth is ended, the spiritual journey has been completed, what had to be done has been done, there is no return to any state of existence.’ 

It\marginnote{32.5} wouldn’t be appropriate to say that a mendicant whose mind is freed like this holds the following views: ‘A Realized One exists after death’; ‘A Realized One doesn’t exist after death’; ‘A Realized One both exists and doesn’t exist after death’; ‘A Realized One neither exists nor doesn’t exist after death’. 

Why\marginnote{32.10} is that? A mendicant is freed by directly knowing this: how far language and the scope of language extend; how far terminology and the scope of terminology extend; how far description and the scope of description extend; how far wisdom and the sphere of wisdom extend; how far the cycle of rebirths and its continuation extend. It wouldn’t be appropriate to say that a mendicant freed by directly knowing this holds the view: ‘There is no such thing as knowing and seeing.’ 

\section*{5. Planes of Consciousness }

Ānanda,\marginnote{33.1} there are seven planes of consciousness and two dimensions. What seven? 

There\marginnote{33.3} are sentient beings that are diverse in body and diverse in perception, such as human beings, some gods, and some beings in the underworld. This is the first plane of consciousness. 

There\marginnote{33.5} are sentient beings that are diverse in body and unified in perception, such as the gods reborn in \textsanskrit{Brahmā}’s Host through the first absorption. This is the second plane of consciousness. 

There\marginnote{33.7} are sentient beings that are unified in body and diverse in perception, such as the gods of streaming radiance. This is the third plane of consciousness. 

There\marginnote{33.9} are sentient beings that are unified in body and unified in perception, such as the gods replete with glory. This is the fourth plane of consciousness. 

There\marginnote{33.11} are sentient beings that have gone totally beyond perceptions of form. With the ending of perceptions of impingement, not focusing on perceptions of diversity, aware that ‘space is infinite’, they have been reborn in the dimension of infinite space. This is the fifth plane of consciousness. 

There\marginnote{33.13} are sentient beings that have gone totally beyond the dimension of infinite space. Aware that ‘consciousness is infinite’, they have been reborn in the dimension of infinite consciousness. This is the sixth plane of consciousness. 

There\marginnote{33.15} are sentient beings that have gone totally beyond the dimension of infinite consciousness. Aware that ‘there is nothing at all’, they have been reborn in the dimension of nothingness. This is the seventh plane of consciousness. 

Then\marginnote{33.17} there’s the dimension of non-percipient beings, and secondly, the dimension of neither perception nor non-perception. 

Now,\marginnote{34.1} regarding these seven planes of consciousness and two dimensions, is it appropriate for someone who understands them—and their origin, ending, gratification, drawback, and escape—to take pleasure in them?” 

“No,\marginnote{34.3} sir.” 

“When\marginnote{34.10} a mendicant, having truly understood the origin, ending, gratification, drawback, and escape regarding these seven planes of consciousness and these two dimensions, is freed by not grasping, they’re called a mendicant who is freed by wisdom. 

\section*{6. The Eight Liberations }

Ānanda,\marginnote{35.1} there are these eight liberations. What eight? 

Having\marginnote{35.3} physical form, they see visions. This is the first liberation. 

Not\marginnote{35.5} perceiving form internally, they see visions externally. This is the second liberation. 

They’re\marginnote{35.7} focused only on beauty. This is the third liberation. 

Going\marginnote{35.9} totally beyond perceptions of form, with the ending of perceptions of impingement, not focusing on perceptions of diversity, aware that ‘space is infinite’, they enter and remain in the dimension of infinite space. This is the fourth liberation. 

Going\marginnote{35.11} totally beyond the dimension of infinite space, aware that ‘consciousness is infinite’, they enter and remain in the dimension of infinite consciousness. This is the fifth liberation. 

Going\marginnote{35.13} totally beyond the dimension of infinite consciousness, aware that ‘there is nothing at all’, they enter and remain in the dimension of nothingness. This is the sixth liberation. 

Going\marginnote{35.15} totally beyond the dimension of nothingness, they enter and remain in the dimension of neither perception nor non-perception. This is the seventh liberation. 

Going\marginnote{35.17} totally beyond the dimension of neither perception nor non-perception, they enter and remain in the cessation of perception and feeling. This is the eighth liberation. 

These\marginnote{35.19} are the eight liberations. 

When\marginnote{36.1} a mendicant enters into and withdraws from these eight liberations—in forward order, in reverse order, and in forward and reverse order—wherever they wish, whenever they wish, and for as long as they wish; and when they realize the undefiled freedom of heart and freedom by wisdom in this very life, and live having realized it with their own insight due to the ending of defilements, they’re called a mendicant who is freed both ways. And, Ānanda, there is no other freedom both ways that is better or finer than this.” 

That\marginnote{36.4} is what the Buddha said. Satisfied, Venerable Ānanda was happy with what the Buddha said. 

%
\chapter*{{\suttatitleacronym DN 16}{\suttatitletranslation The Great Discourse on the Buddha’s Extinguishment }{\suttatitleroot Mahāparinibbānasutta}}
\addcontentsline{toc}{chapter}{\tocacronym{DN 16} \toctranslation{The Great Discourse on the Buddha’s Extinguishment } \tocroot{Mahāparinibbānasutta}}
\markboth{The Great Discourse on the Buddha’s Extinguishment }{Mahāparinibbānasutta}
\extramarks{DN 16}{DN 16}

\scevam{So\marginnote{1.1.1} I have heard. }At one time the Buddha was staying near \textsanskrit{Rājagaha}, on the Vulture’s Peak Mountain. Now at that time King \textsanskrit{Ajātasattu} Vedehiputta of \textsanskrit{Māgadha} wanted to invade the Vajjis. He declared: “I shall wipe out these Vajjis, so mighty and powerful! I shall destroy them, and lay ruin and devastation upon them!” 

And\marginnote{1.2.1} then King \textsanskrit{Ajātasattu} addressed \textsanskrit{Vassakāra} the brahmin minister of \textsanskrit{Māgadha}, “Please, brahmin, go to the Buddha, and in my name bow with your head to his feet. Ask him if he is healthy and well, nimble, strong, and living comfortably. And then say: ‘Sir, King \textsanskrit{Ajātasattu} Vedehiputta of \textsanskrit{Māgadha} wants to invade the Vajjis. He says, “I shall wipe out these Vajjis, so mighty and powerful! I shall destroy them, and lay ruin and devastation upon them!”’ Remember well how the Buddha answers and tell it to me. For Realized Ones say nothing that is not so.” 

\section*{1. The Brahmin \textsanskrit{Vassakāra} }

“Yes,\marginnote{1.3.1} sir,” \textsanskrit{Vassakāra} replied. He had the finest carriages harnessed. Then he mounted a fine carriage and, along with other fine carriages, set out from \textsanskrit{Rājagaha} for the Vulture’s Peak Mountain. He went by carriage as far as the terrain allowed, then descended and approached the Buddha on foot, and exchanged greetings with him. 

When\marginnote{1.3.3} the greetings and polite conversation were over, he sat down to one side and said to the Buddha, “Master Gotama, King \textsanskrit{Ajātasattu} Vedehiputta of \textsanskrit{Māgadha} bows with his head to your feet. He asks if you are healthy and well, nimble, strong, and living comfortably. Master Gotama, King \textsanskrit{Ajātasattu} wants to invade the Vajjis. He has declared: ‘I shall wipe out these Vajjis, so mighty and powerful! I shall destroy them, and lay ruin and devastation upon them!’” 

\section*{2. Principles That Prevent Decline }

Now\marginnote{1.4.1} at that time Venerable Ānanda was standing behind the Buddha fanning him. Then the Buddha said to him, “Ānanda, have you heard that the Vajjis meet frequently and have many meetings?” 

“I\marginnote{1.4.4} have heard that, sir.” 

“As\marginnote{1.4.5} long as the Vajjis meet frequently and have many meetings, they can expect growth, not decline. 

Ānanda,\marginnote{1.4.6} have you heard that the Vajjis meet in harmony, leave in harmony, and carry on their business in harmony?” 

“I\marginnote{1.4.7} have heard that, sir.” 

“As\marginnote{1.4.8} long as the Vajjis meet in harmony, leave in harmony, and carry on their business in harmony, they can expect growth, not decline. 

Ānanda,\marginnote{1.4.9} have you heard that the Vajjis don’t make new decrees or abolish existing decrees, but proceed having undertaken the traditional Vajjian principles as they have been decreed?” 

“I\marginnote{1.4.10} have heard that, sir.” 

“As\marginnote{1.4.11} long as the Vajjis don’t make new decrees or abolish existing decrees, but proceed having undertaken the traditional Vajjian principles as they have been decreed, they can expect growth, not decline. 

Ānanda,\marginnote{1.4.12} have you heard that the Vajjis honor, respect, esteem, and venerate Vajjian elders, and think them worth listening to?” 

“I\marginnote{1.4.13} have heard that, sir.” 

“As\marginnote{1.4.14} long as the Vajjis honor, respect, esteem, and venerate Vajjian elders, and think them worth listening to, they can expect growth, not decline. 

Ānanda,\marginnote{1.4.15} have you heard that the Vajjis don’t rape or abduct women or girls from their families and force them to live with them?” 

“I\marginnote{1.4.16} have heard that, sir.” 

“As\marginnote{1.4.17} long as the Vajjis don’t rape or abduct women or girls from their families and force them to live with them, they can expect growth, not decline. 

Ānanda,\marginnote{1.4.18} have you heard that the Vajjis honor, respect, esteem, and venerate the Vajjian shrines, whether inner or outer, not neglecting the proper spirit-offerings that were given and made in the past?” 

“I\marginnote{1.4.19} have heard that, sir.” 

“As\marginnote{1.4.20} long as the Vajjis honor, respect, esteem, and venerate the Vajjian shrines, whether inner or outer, not neglecting the proper spirit-offerings that were given and made in the past, they can expect growth, not decline. 

Ānanda,\marginnote{1.4.21} have you heard that the Vajjis organize proper protection, shelter, and security for perfected ones, so that more perfected ones might come to the realm and those already here may live in comfort?” 

“I\marginnote{1.4.22} have heard that, sir.” 

“As\marginnote{1.4.23} long as the Vajjis organize proper protection, shelter, and security for perfected ones, so that more perfected ones might come to the realm and those already here may live in comfort, they can expect growth, not decline.” 

Then\marginnote{1.5.1} the Buddha said to \textsanskrit{Vassakāra}, “Brahmin, this one time I was staying near \textsanskrit{Vesālī} at the \textsanskrit{Sārandada} woodland shrine. There I taught the Vajjis these seven principles that prevent decline. As long as these seven principles that prevent decline last among the Vajjis, and as long as the Vajjis are seen following them, they can expect growth, not decline.” 

When\marginnote{1.5.5} the Buddha had spoken, \textsanskrit{Vassakāra} said to him, “Master Gotama, if the Vajjis follow even a single one of these principles they can expect growth, not decline. How much more so all seven! King \textsanskrit{Ajātasattu} cannot defeat the Vajjis in war, unless by diplomacy or by sowing dissension. Well, now, Master Gotama, I must go. I have many duties, and much to do.” 

“Please,\marginnote{1.5.10} brahmin, go at your convenience.” Then \textsanskrit{Vassakāra} the brahmin, having approved and agreed with what the Buddha said, got up from his seat and left. 

\section*{3. Principles That Prevent Decline Among the Mendicants }

Soon\marginnote{1.6.1} after he had left, the Buddha said to Ānanda, “Go, Ānanda, gather all the mendicants staying in the vicinity of \textsanskrit{Rājagaha} together in the assembly hall.” 

“Yes,\marginnote{1.6.3} sir,” replied Ānanda. He did what the Buddha asked. Then he went back, bowed, stood to one side, and said to him, “Sir, the mendicant \textsanskrit{Saṅgha} has assembled. Please, sir, go at your convenience.” 

Then\marginnote{1.6.5} the Buddha went to the assembly hall, where he sat on the seat spread out and addressed the mendicants: “Mendicants, I will teach you these seven principles that prevent decline. Listen and pay close attention, I will speak.” 

“Yes,\marginnote{1.6.9} sir,” they replied. The Buddha said this: 

“As\marginnote{1.6.11} long as the mendicants meet frequently and have many meetings, they can expect growth, not decline. 

As\marginnote{1.6.12} long as the mendicants meet in harmony, leave in harmony, and carry on their business in harmony, they can expect growth, not decline. 

As\marginnote{1.6.13} long as the mendicants don’t make new decrees or abolish existing decrees, but undertake and follow the training rules as they have been decreed, they can expect growth, not decline. 

As\marginnote{1.6.14} long as the mendicants honor, respect, esteem, and venerate the senior mendicants—of long standing, long gone forth, fathers and leaders of the \textsanskrit{Saṅgha}—and think them worth listening to, they can expect growth, not decline. 

As\marginnote{1.6.15} long as the mendicants don’t fall under the sway of arisen craving for future lives, they can expect growth, not decline. 

As\marginnote{1.6.16} long as the mendicants take care to live in wilderness lodgings, they can expect growth, not decline. 

As\marginnote{1.6.17} long as the mendicants individually establish mindfulness, so that more good-hearted spiritual companions might come, and those that have already come may live comfortably, they can expect growth, not decline. 

As\marginnote{1.6.18} long as these seven principles that prevent decline last among the mendicants, and as long as the mendicants are seen following them, they can expect growth, not decline. 

I\marginnote{1.7.1} will teach you seven more principles that prevent decline. … 

As\marginnote{1.7.4} long as the mendicants don’t relish work, loving it and liking to relish it, they can expect growth, not decline. 

As\marginnote{1.7.5} long as they don’t enjoy talk … 

sleep\marginnote{1.7.6} … 

company\marginnote{1.7.7} … 

they\marginnote{1.7.8} don’t have wicked desires, falling under the sway of wicked desires … 

they\marginnote{1.7.9} don’t have bad friends, companions, and associates … 

they\marginnote{1.7.10} don’t stop half-way after achieving some insignificant distinction, they can expect growth, not decline. 

As\marginnote{1.8.1} long as these seven principles that prevent decline last among the mendicants, and as long as the mendicants are seen following them, they can expect growth, not decline. 

I\marginnote{1.8.2} will teach you seven more principles that prevent decline. … As long as the mendicants are faithful … conscientious … prudent … learned … energetic … mindful … wise, they can expect growth, not decline. As long as these seven principles that prevent decline last among the mendicants, and as long as the mendicants are seen following them, they can expect growth, not decline. 

I\marginnote{1.8.11} will teach you seven more principles that prevent decline. … 

As\marginnote{1.9.1} long as the mendicants develop the awakening factors of mindfulness … investigation of principles … energy … rapture … tranquility … immersion … equanimity, they can expect growth, not decline. 

As\marginnote{1.10.1} long as these seven principles that prevent decline last among the mendicants, and as long as the mendicants are seen following them, they can expect growth, not decline. 

I\marginnote{1.10.2} will teach you seven more principles that prevent decline. … 

As\marginnote{1.10.5} long as the mendicants develop the perceptions of impermanence … not-self … ugliness … drawbacks … giving up … fading away … cessation, they can expect growth, not decline. 

As\marginnote{1.10.12} long as these seven principles that prevent decline last among the mendicants, and as long as the mendicants are seen following them, they can expect growth, not decline. 

I\marginnote{1.11.1} will teach you six principles that prevent decline. … 

As\marginnote{1.11.4} long as the mendicants consistently treat their spiritual companions with bodily kindness … verbal kindness … and mental kindness both in public and in private, they can expect growth, not decline. 

As\marginnote{1.11.7} long as the mendicants share without reservation any material possessions they have gained by legitimate means, even the food placed in the alms-bowl, using them in common with their ethical spiritual companions, they can expect growth, not decline. 

As\marginnote{1.11.8} long as the mendicants live according to the precepts shared with their spiritual companions, both in public and in private—such precepts as are unbroken, impeccable, spotless, and unmarred, liberating, praised by sensible people, not mistaken, and leading to immersion—they can expect growth, not decline. 

As\marginnote{1.11.9} long as the mendicants live according to the view shared with their spiritual companions, both in public and in private—the view that is noble and emancipating, and leads one who practices it to the complete end of suffering—they can expect growth, not decline. 

As\marginnote{1.11.10} long as these six principles that prevent decline last among the mendicants, and as long as the mendicants are seen following them, they can expect growth, not decline.” 

And\marginnote{1.12.1} while staying there at the Vulture’s Peak the Buddha often gave this Dhamma talk to the mendicants: 

“Such\marginnote{1.12.2} is ethics, such is immersion, such is wisdom. When immersion is imbued with ethics it’s very fruitful and beneficial. When wisdom is imbued with immersion it’s very fruitful and beneficial. When the mind is imbued with wisdom it is rightly freed from the defilements, namely, the defilements of sensuality, desire to be reborn, and ignorance.” 

When\marginnote{1.13.1} the Buddha had stayed in \textsanskrit{Rājagaha} as long as he wished, he addressed Venerable Ānanda, “Come, Ānanda, let’s go to \textsanskrit{Ambalaṭṭhikā}.” 

“Yes,\marginnote{1.13.3} sir,” Ānanda replied. Then the Buddha together with a large \textsanskrit{Saṅgha} of mendicants arrived at \textsanskrit{Ambalaṭṭhikā}, where he stayed in the royal rest-house. And while staying there, too, he often gave this Dhamma talk to the mendicants: 

“Such\marginnote{1.14.3} is ethics, such is immersion, such is wisdom. When immersion is imbued with ethics it’s very fruitful and beneficial. When wisdom is imbued with immersion it’s very fruitful and beneficial. When the mind is imbued with wisdom it is rightly freed from the defilements, namely, the defilements of sensuality, desire to be reborn, and ignorance.” 

When\marginnote{1.15.1} the Buddha had stayed in \textsanskrit{Ambalaṭṭhikā} as long as he wished, he addressed Venerable Ānanda, “Come, Ānanda, let’s go to \textsanskrit{Nāḷandā}.” 

“Yes,\marginnote{1.15.3} sir,” Ānanda replied. Then the Buddha together with a large \textsanskrit{Saṅgha} of mendicants arrived at \textsanskrit{Nāḷandā}, where he stayed in \textsanskrit{Pāvārika}’s mango grove. 

\section*{4. \textsanskrit{Sāriputta}’s Lion’s Roar }

Then\marginnote{1.16.1} \textsanskrit{Sāriputta} went up to the Buddha, bowed, sat down to one side, and said to him, “Sir, I have such confidence in the Buddha that I believe there’s no other ascetic or brahmin—whether past, future, or present—whose direct knowledge is superior to the Buddha when it comes to awakening.” 

“That’s\marginnote{1.16.4} a grand and dramatic statement, \textsanskrit{Sāriputta}. You’ve roared a definitive, categorical lion’s roar, saying: ‘I have such confidence in the Buddha that I believe there’s no other ascetic or brahmin—whether past, future, or present—whose direct knowledge is superior to the Buddha when it comes to awakening.’ 

What\marginnote{1.16.7} about all the perfected ones, the fully awakened Buddhas who lived in the past? Have you comprehended their minds to know that those Buddhas had such ethics, or such qualities, or such wisdom, or such meditation, or such freedom?” 

“No,\marginnote{1.16.9} sir.” 

“And\marginnote{1.16.10} what about all the perfected ones, the fully awakened Buddhas who will live in the future? Have you comprehended their minds to know that those Buddhas will have such ethics, or such qualities, or such wisdom, or such meditation, or such freedom?” 

“No,\marginnote{1.16.12} sir.” 

“And\marginnote{1.16.13} what about me, the perfected one, the fully awakened Buddha at present? Have you comprehended my mind to know that I have such ethics, or such teachings, or such wisdom, or such meditation, or such freedom?” 

“No,\marginnote{1.16.15} sir.” 

“Well\marginnote{1.16.16} then, \textsanskrit{Sāriputta}, given that you don’t comprehend the minds of Buddhas past, future, or present, what exactly are you doing, making such a grand and dramatic statement, roaring such a definitive, categorical lion’s roar?” 

“Sir,\marginnote{1.17.1} though I don’t comprehend the minds of Buddhas past, future, and present, still I understand this by inference from the teaching. Suppose there was a king’s frontier citadel with fortified embankments, ramparts, and arches, and a single gate. And it has a gatekeeper who is astute, competent, and intelligent. He keeps strangers out and lets known people in. As he walks around the patrol path, he doesn’t see a hole or cleft in the wall, not even one big enough for a cat to slip out. He thinks: ‘Whatever sizable creatures enter or leave the citadel, all of them do so via this gate.’ 

In\marginnote{1.17.8} the same way, I understand this by inference from the teaching: ‘All the perfected ones, fully awakened Buddhas—whether past, future, or present—give up the five hindrances, corruptions of the heart that weaken wisdom. Their mind is firmly established in the four kinds of mindfulness meditation. They correctly develop the seven awakening factors. And they wake up to the supreme perfect awakening.’” 

And\marginnote{1.18.1} while staying at \textsanskrit{Nāḷandā}, too, the Buddha often gave this Dhamma talk to the mendicants: 

“Such\marginnote{1.18.2} is ethics, such is immersion, such is wisdom. When immersion is imbued with ethics it’s very fruitful and beneficial. When wisdom is imbued with immersion it’s very fruitful and beneficial. When the mind is imbued with wisdom it is rightly freed from the defilements, namely, the defilements of sensuality, desire to be reborn, and ignorance.” 

\section*{5. The Drawbacks of Unethical Conduct }

When\marginnote{1.19.1} the Buddha had stayed in \textsanskrit{Nāḷandā} as long as he wished, he addressed Venerable Ānanda, “Come, Ānanda, let’s go to \textsanskrit{Pāṭali} Village.” 

“Yes,\marginnote{1.19.3} sir,” Ānanda replied. Then the Buddha together with a large \textsanskrit{Saṅgha} of mendicants arrived at \textsanskrit{Pāṭali} Village. 

The\marginnote{1.20.1} lay followers of \textsanskrit{Pāṭali} Village heard that he had arrived. So they went to see him, bowed, sat down to one side, and said to him, “Sir, please consent to come to our guest house.” The Buddha consented in silence. 

Then,\marginnote{1.21.1} knowing that the Buddha had consented, the lay followers of \textsanskrit{Pāṭali} Village got up from their seat, bowed, and respectfully circled the Buddha, keeping him on their right. Then they went to the guest house, where they spread carpets all over, prepared seats, set up a water jar, and placed a lamp. Then they went back to the Buddha, bowed, stood to one side, and told him of their preparations, saying: “Please, sir, come at your convenience.” 

In\marginnote{1.22.1} the morning, the Buddha robed up and, taking his bowl and robe, went to the guest house together with the \textsanskrit{Saṅgha} of mendicants. Having washed his feet he entered the guest house and sat against the central column facing east. The \textsanskrit{Saṅgha} of mendicants also washed their feet, entered the guest house, and sat against the west wall facing east, with the Buddha right in front of them. The lay followers of \textsanskrit{Pāṭali} Village also washed their feet, entered the guest house, and sat against the east wall facing west, with the Buddha right in front of them. 

Then\marginnote{1.23.1} the Buddha addressed them: 

“Householders,\marginnote{1.23.2} there are these five drawbacks for an unethical person because of their failure in ethics. What five? 

Firstly,\marginnote{1.23.4} an unethical person loses substantial wealth on account of negligence. This is the first drawback for an unethical person because of their failure in ethics. 

Furthermore,\marginnote{1.23.6} an unethical person gets a bad reputation. This is the second drawback. 

Furthermore,\marginnote{1.23.8} an unethical person enters any kind of assembly timid and embarrassed, whether it’s an assembly of aristocrats, brahmins, householders, or ascetics. This is the third drawback. 

Furthermore,\marginnote{1.23.10} an unethical person feels lost when they die. This is the fourth drawback. 

Furthermore,\marginnote{1.23.12} an unethical person, when their body breaks up, after death, is reborn in a place of loss, a bad place, the underworld, hell. This is the fifth drawback. 

These\marginnote{1.23.14} are the five drawbacks for an unethical person because of their failure in ethics. 

\section*{6. The Benefits of Ethical Conduct }

There\marginnote{1.24.1} are these five benefits for an ethical person because of their accomplishment in ethics. What five? 

Firstly,\marginnote{1.24.3} an ethical person gains substantial wealth on account of diligence. This is the first benefit. 

Furthermore,\marginnote{1.24.5} an ethical person gets a good reputation. This is the second benefit. 

Furthermore,\marginnote{1.24.7} an ethical person enters any kind of assembly bold and self-assured, whether it’s an assembly of aristocrats, brahmins, householders, or ascetics. This is the third benefit. 

Furthermore,\marginnote{1.24.9} an ethical person dies not feeling lost. This is the fourth benefit. 

Furthermore,\marginnote{1.24.11} when an ethical person’s body breaks up, after death, they’re reborn in a good place, a heavenly realm. This is the fifth benefit. 

These\marginnote{1.24.13} are the five benefits for an ethical person because of their accomplishment in ethics.” 

The\marginnote{1.25.1} Buddha spent most of the night educating, encouraging, firing up, and inspiring the lay followers of \textsanskrit{Pāṭali} Village with a Dhamma talk. Then he dismissed them, “The night is getting late, householders. Please go at your convenience.” 

“Yes,\marginnote{1.25.3} sir,” replied the lay followers of \textsanskrit{Pāṭali} Village. They got up from their seat, bowed, and respectfully circled the Buddha, keeping him on their right, before leaving. Soon after they left the Buddha entered a private cubicle. 

\section*{7. Building a Citadel }

Now\marginnote{1.26.1} at that time the Magadhan ministers Sunidha and \textsanskrit{Vassakāra} were building a citadel at \textsanskrit{Pāṭali} Village to keep the Vajjis out. At that time thousands of deities were taking possession of building sites in \textsanskrit{Pāṭali} Village. Illustrious rulers or royal ministers inclined to build houses at sites possessed by illustrious deities. Middling rulers or royal ministers inclined to build houses at sites possessed by middling deities. Lesser rulers or royal ministers inclined to build houses at sites possessed by lesser deities. 

With\marginnote{1.27.3} clairvoyance that is purified and superhuman, the Buddha saw those deities taking possession of building sites in \textsanskrit{Pāṭali} Village. The Buddha rose at the crack of dawn and addressed Ānanda, “Ānanda, who is building a citadel at \textsanskrit{Pāṭali} Village?” 

“Sir,\marginnote{1.27.6} the Magadhan ministers Sunidha and \textsanskrit{Vassakāra} are building a citadel to keep the Vajjis out.” 

“It’s\marginnote{1.28.1} as if they were building the citadel in consultation with the gods of the Thirty-Three. With clairvoyance that is purified and superhuman, I saw those deities taking possession of building sites. Illustrious rulers or royal ministers inclined to build houses at sites possessed by illustrious deities. Middling rulers or royal ministers inclined to build houses at sites possessed by middling deities. Lesser rulers or royal ministers inclined to build houses at sites possessed by lesser deities. As far as the civilized region extends, as far as the trading zone extends, this will be the chief city: the \textsanskrit{Pāṭaliputta} trade center. But \textsanskrit{Pāṭaliputta} will face three threats: from fire, flood, and dissension.” 

Then\marginnote{1.29.1} the Magadhan ministers Sunidha and \textsanskrit{Vassakāra} approached the Buddha, and exchanged greetings with him. When the greetings and polite conversation were over, they stood to one side and said, “Would Master Gotama together with the mendicant \textsanskrit{Saṅgha} please accept today’s meal from me?” The Buddha consented in silence. 

Then,\marginnote{1.30.1} knowing that the Buddha had consented, they went to their own guest house, where they had a variety of delicious foods prepared. Then they had the Buddha informed of the time, saying, “It’s time, Master Gotama, the meal is ready.” 

Then\marginnote{1.30.3} the Buddha robed up in the morning and, taking his bowl and robe, went to their guest house together with the mendicant \textsanskrit{Saṅgha}, where he sat on the seat spread out. Then Sunidha and \textsanskrit{Vassakāra} served and satisfied the mendicant \textsanskrit{Saṅgha} headed by the Buddha with their own hands with a variety of delicious foods. When the Buddha had eaten and washed his hand and bowl, Sunidha and \textsanskrit{Vassakāra} took a low seat and sat to one side. 

The\marginnote{1.31.1} Buddha expressed his appreciation with these verses: 

\begin{verse}%
“In\marginnote{1.31.2} the place he makes his dwelling, \\
having fed the astute \\
and the virtuous here, \\
the restrained spiritual practitioners, 

he\marginnote{1.31.6} should dedicate an offering \\
to the deities there. \\
Venerated, they venerate him; \\
honored, they honor him. 

After\marginnote{1.31.10} that they have compassion for him, \\
like a mother for the child at her breast. \\
A man beloved of the deities \\
always sees nice things.” 

%
\end{verse}

When\marginnote{1.31.14} the Buddha had expressed his appreciation to Sunidha and \textsanskrit{Vassakāra} with these verses, he got up from his seat and left. 

Sunidha\marginnote{1.32.1} and \textsanskrit{Vassakāra} followed behind the Buddha, thinking, “The gate through which the ascetic Gotama departs today shall be named the Gotama Gate. The ford at which he crosses the Ganges River shall be named the Gotama Ford.” 

Then\marginnote{1.32.4} the gate through which the Buddha departed was named the Gotama Gate. 

Then\marginnote{1.32.5} the Buddha came to the Ganges River. 

Now\marginnote{1.33.1} at that time the Ganges was full to the brim so a crow could drink from it. Wanting to cross from the near to the far shore, some people were seeking a boat, some a dinghy, while some were tying up a raft. But, as easily as a strong person would extend or contract their arm, the Buddha, together with the mendicant \textsanskrit{Saṅgha}, vanished from the near shore and landed on the far shore. 

He\marginnote{1.34.1} saw all those people wanting to cross over. Knowing the meaning of this, on that occasion the Buddha expressed this heartfelt sentiment: 

\begin{verse}%
“Those\marginnote{1.34.3} who cross a deluge or stream \\
have built a bridge and left the marshes behind. \\
While some people are still tying a raft, \\
intelligent people have crossed over.” 

%
\end{verse}

\section*{8. Talk on the Noble Truths }

Then\marginnote{2.1.1} the Buddha said to Venerable Ānanda, “Come, Ānanda, let’s go to \textsanskrit{Koṭigāma}.” 

“Yes,\marginnote{2.1.3} sir,” Ānanda replied. Then the Buddha together with a large \textsanskrit{Saṅgha} of mendicants arrived at \textsanskrit{Koṭigāma}, and stayed there. 

There\marginnote{2.2.1} he addressed the mendicants: 

“Mendicants,\marginnote{2.2.2} not understanding and not penetrating four noble truths, both you and I have wandered and transmigrated for such a very long time. What four? The noble truths of suffering, the origin of suffering, the cessation of suffering, and the practice that leads to the cessation of suffering. These noble truths of suffering, origin, cessation, and the path have been understood and comprehended. Craving for continued existence has been cut off; the conduit to rebirth is ended; now there are no more future lives.” 

That\marginnote{2.3.1} is what the Buddha said. Then the Holy One, the Teacher, went on to say: 

\begin{verse}%
“Because\marginnote{2.3.3} of not truly seeing \\
the four noble truths, \\
we have transmigrated for a long time \\
from one rebirth to the next. 

But\marginnote{2.3.7} now that these truths have been seen, \\
the conduit to rebirth is eradicated. \\
The root of suffering is cut off, \\
now there are no more future lives.” 

%
\end{verse}

And\marginnote{2.4.1} while staying at \textsanskrit{Koṭigāma}, too, the Buddha often gave this Dhamma talk to the mendicants: 

“Such\marginnote{2.4.2} is ethics, such is immersion, such is wisdom. When immersion is imbued with ethics it’s very fruitful and beneficial. When wisdom is imbued with immersion it’s very fruitful and beneficial. When the mind is imbued with wisdom it is rightly freed from the defilements, namely, the defilements of sensuality, desire to be reborn, and ignorance.” 

\section*{9. The Deaths in \textsanskrit{Nādika} }

When\marginnote{2.5.1} the Buddha had stayed in \textsanskrit{Koṭigāma} as long as he wished, he said to Ānanda, “Come, Ānanda, let’s go to \textsanskrit{Nādika}.” 

“Yes,\marginnote{2.5.3} sir,” Ānanda replied. Then the Buddha together with a large \textsanskrit{Saṅgha} of mendicants arrived at \textsanskrit{Nādika}, where he stayed in the brick house. 

Then\marginnote{2.6.1} Venerable Ānanda went up to the Buddha, bowed, sat down to one side, and said to him, “Sir, the monk named \textsanskrit{Sāḷha} has passed away in \textsanskrit{Nādika}. Where has he been reborn in his next life? The nun named \textsanskrit{Nandā}, the layman named Sudatta, and the laywoman named \textsanskrit{Sujātā} have passed away in \textsanskrit{Nādika}. Where have they been reborn in the next life? The laymen named \textsanskrit{Kakkaṭa}, \textsanskrit{Kaḷibha}, Nikata, \textsanskrit{Kaṭissaha}, \textsanskrit{Tuṭṭha}, \textsanskrit{Santuṭṭha}, Bhadda, and Subhadda have passed away in \textsanskrit{Nādika}. Where have they been reborn in the next life?” 

“Ānanda,\marginnote{2.7.1} the monk \textsanskrit{Sāḷha} had realized the undefiled freedom of heart and freedom by wisdom in this very life, having realized it with his own insight due to the ending of defilements. 

The\marginnote{2.7.2} nun \textsanskrit{Nandā} had ended the five lower fetters. She’s been reborn spontaneously, and will be extinguished there, not liable to return from that world. 

The\marginnote{2.7.3} layman Sudatta had ended three fetters, and weakened greed, hate, and delusion. He’s a once-returner; he will come back to this world once only, then make an end of suffering. 

The\marginnote{2.7.4} laywoman \textsanskrit{Sujātā} had ended three fetters. She’s a stream-enterer, not liable to be reborn in the underworld, bound for awakening. 

The\marginnote{2.7.5} laymen \textsanskrit{Kakkaṭa}, \textsanskrit{Kaḷibha}, Nikata, \textsanskrit{Kaṭissaha}, \textsanskrit{Tuṭṭha}, \textsanskrit{Santuṭṭha}, Bhadda, and and Subhadda had ended the five lower fetters. They’ve been reborn spontaneously, and will be extinguished there, not liable to return from that world. 

Over\marginnote{2.7.13} fifty laymen in \textsanskrit{Nādika} have passed away having ended the five lower fetters. They’ve been reborn spontaneously, and will be extinguished there, not liable to return from that world. 

More\marginnote{2.7.14} than ninety laymen in \textsanskrit{Nādika} have passed away having ended three fetters, and weakened greed, hate, and delusion. They’re once-returners, who will come back to this world once only, then make an end of suffering. 

In\marginnote{2.7.15} excess of five hundred laymen in \textsanskrit{Nādika} have passed away having ended three fetters. They’re stream-enterers, not liable to be reborn in the underworld, bound for awakening. 

\section*{10. The Mirror of the Teaching }

It’s\marginnote{2.8.1} hardly surprising that a human being should pass away. But if you should come and ask me about it each and every time someone passes away, that would be a bother for me. 

So\marginnote{2.8.3} Ānanda, I will teach you the explanation of the Dhamma called ‘the mirror of the teaching’. A noble disciple who has this may declare of themselves: ‘I’ve finished with rebirth in hell, the animal realm, and the ghost realm. I’ve finished with all places of loss, bad places, the underworld. I am a stream-enterer! I’m not liable to be reborn in the underworld, and am bound for awakening.’ 

And\marginnote{2.9.1} what is that mirror of the teaching? 

It’s\marginnote{2.9.3} when a noble disciple has experiential confidence in the Buddha: ‘That Blessed One is perfected, a fully awakened Buddha, accomplished in knowledge and conduct, holy, knower of the world, supreme guide for those who wish to train, teacher of gods and humans, awakened, blessed.’ 

They\marginnote{2.9.5} have experiential confidence in the teaching: ‘The teaching is well explained by the Buddha—visible in this very life, immediately effective, inviting inspection, relevant, so that sensible people can know it for themselves.’ 

They\marginnote{2.9.7} have experiential confidence in the \textsanskrit{Saṅgha}: ‘The \textsanskrit{Saṅgha} of the Buddha’s disciples is practicing the way that’s good, direct, methodical, and proper. It consists of the four pairs, the eight individuals. This is the \textsanskrit{Saṅgha} of the Buddha’s disciples that is worthy of offerings dedicated to the gods, worthy of hospitality, worthy of a religious donation, worthy of greeting with joined palms, and is the supreme field of merit for the world.’ 

And\marginnote{2.9.9} a noble disciple’s ethical conduct is loved by the noble ones, unbroken, impeccable, spotless, and unmarred, liberating, praised by sensible people, not mistaken, and leading to immersion. 

This\marginnote{2.9.10} is that mirror of the teaching.” 

And\marginnote{2.10.1} while staying there in \textsanskrit{Nādika} the Buddha often gave this Dhamma talk to the mendicants: 

“Such\marginnote{2.10.2} is ethics, such is immersion, such is wisdom. When immersion is imbued with ethics it’s very fruitful and beneficial. When wisdom is imbued with immersion it’s very fruitful and beneficial. When the mind is imbued with wisdom it is rightly freed from the defilements, namely, the defilements of sensuality, desire to be reborn, and ignorance.” 

When\marginnote{2.11.1} the Buddha had stayed in \textsanskrit{Nādika} as long as he wished, he addressed Venerable Ānanda, “Come, Ānanda, let’s go to \textsanskrit{Vesālī}.” 

“Yes,\marginnote{2.11.3} sir,” Ānanda replied. Then the Buddha together with a large \textsanskrit{Saṅgha} of mendicants arrived at \textsanskrit{Vesālī}, where he stayed in \textsanskrit{Ambapālī}’s mango grove. 

There\marginnote{2.12.1} the Buddha addressed the mendicants: 

“Mendicants,\marginnote{2.12.2} a mendicant should live mindful and aware. This is my instruction to you. 

And\marginnote{2.12.4} how is a mendicant mindful? It’s when a mendicant meditates by observing an aspect of the body—keen, aware, and mindful, rid of desire and aversion for the world. They meditate observing an aspect of feelings … mind … principles—keen, aware, and mindful, rid of desire and aversion for the world. That’s how a mendicant is mindful. 

And\marginnote{2.13.1} how is a mendicant aware? It’s when a mendicant acts with situational awareness when going out and coming back; when looking ahead and aside; when bending and extending the limbs; when bearing the outer robe, bowl and robes; when eating, drinking, chewing, and tasting; when urinating and defecating; when walking, standing, sitting, sleeping, waking, speaking, and keeping silent. That’s how a mendicant is aware. A mendicant should live mindful and aware. This is my instruction to you.” 

\section*{11. \textsanskrit{Ambapālī} the Courtesan }

\textsanskrit{Ambapālī}\marginnote{2.14.1} the courtesan heard that the Buddha had arrived and was staying in her mango grove. She had the finest carriages harnessed. Then she mounted a fine carriage and, along with other fine carriages, set out from \textsanskrit{Vesālī} for her own park. She went by carriage as far as the terrain allowed, then descended and approached the Buddha on foot. She bowed and sat down to one side. The Buddha educated, encouraged, fired up, and inspired her with a Dhamma talk. 

Then\marginnote{2.14.5} she said to the Buddha, “Sir, may the Buddha together with the mendicant \textsanskrit{Saṅgha} please accept tomorrow’s meal from me.” The Buddha consented in silence. Then, knowing that the Buddha had consented, \textsanskrit{Ambapālī} got up from her seat, bowed, and respectfully circled the Buddha, keeping him on her right, before leaving. 

The\marginnote{2.15.1} Licchavis of \textsanskrit{Vesālī} also heard that the Buddha had arrived and was staying in \textsanskrit{Ambapālī}’s mango grove. They had the finest carriages harnessed. Then they mounted a fine carriage and, along with other fine carriages, set out from \textsanskrit{Vesālī}. Some of the Licchavis were in blue, of blue color, clad in blue, adorned with blue. And some were similarly colored in yellow, red, or white. 

Then\marginnote{2.16.1} \textsanskrit{Ambapālī} the courtesan collided with those Licchavi youths, axle to axle, wheel to wheel, yoke to yoke. The Licchavis said to her, “What the hell, \textsanskrit{Ambapālī}, are you doing colliding with us axle to axle, wheel to wheel, yoke to yoke?” 

“Well,\marginnote{2.16.4} masters, it’s because I’ve invited the Buddha for tomorrow’s meal together with the mendicant \textsanskrit{Saṅgha}.” 

“Girl,\marginnote{2.16.5} give us that meal for a hundred thousand!” 

“Masters,\marginnote{2.16.6} even if you were to give me \textsanskrit{Vesālī} with her fiefdoms, I still wouldn’t give that meal to you.” 

Then\marginnote{2.16.7} the Licchavis snapped their fingers, saying, “We’ve been beaten by the aunty! We’ve been beaten by the aunty!” Then they continued on to \textsanskrit{Ambapālī}’s grove. 

The\marginnote{2.17.1} Buddha saw them coming off in the distance, and addressed the mendicants: “Any of the mendicants who’ve never seen the gods of the Thirty-Three, just have a look at the assembly of Licchavis. See the assembly of Licchavis, check them out: they’re just like the Thirty-Three!” 

The\marginnote{2.18.1} Licchavis went by carriage as far as the terrain allowed, then descended and approached the Buddha on foot. They bowed to the Buddha, sat down to one side, and the Buddha educated, encouraged, fired up, and inspired them with a Dhamma talk. 

Then\marginnote{2.18.3} they said to the Buddha, “Sir, may the Buddha together with the mendicant \textsanskrit{Saṅgha} please accept tomorrow’s meal from us.” 

Then\marginnote{2.18.5} the Buddha said to the Licchavis, “I have already accepted tomorrow’s meal from \textsanskrit{Ambapālī} the courtesan.” 

Then\marginnote{2.18.7} the Licchavis snapped their fingers, saying, “We’ve been beaten by the aunty! We’ve been beaten by the aunty!” 

And\marginnote{2.18.9} then those Licchavis approved and agreed with what the Buddha said. They got up from their seat, bowed, and respectfully circled the Buddha, keeping him on their right, before leaving. 

And\marginnote{2.19.1} when the night had passed \textsanskrit{Ambapālī} had a variety of delicious foods prepared in her own park. Then she had the Buddha informed of the time, saying, “Sir, it’s time. The meal is ready.” 

Then\marginnote{2.19.3} the Buddha robed up in the morning and, taking his bowl and robe, went to the home of \textsanskrit{Ambapālī} together with the mendicant \textsanskrit{Saṅgha}, where he sat on the seat spread out. Then \textsanskrit{Ambapālī} served and satisfied the mendicant \textsanskrit{Saṅgha} headed by the Buddha with her own hands with a variety of delicious foods. 

When\marginnote{2.19.5} the Buddha had eaten and washed his hands and bowl, \textsanskrit{Ambapālī} took a low seat, sat to one side, and said to the Buddha, “Sir, I present this park to the mendicant \textsanskrit{Saṅgha} headed by the Buddha.” 

The\marginnote{2.19.8} Buddha accepted the park. 

Then\marginnote{2.19.9} the Buddha educated, encouraged, fired up, and inspired her with a Dhamma talk, after which he got up from his seat and left. 

And\marginnote{2.20.1} while staying at \textsanskrit{Vesālī}, too, the Buddha often gave this Dhamma talk to the mendicants: 

“Such\marginnote{2.20.2} is ethics, such is immersion, such is wisdom. When immersion is imbued with ethics it’s very fruitful and beneficial. When wisdom is imbued with immersion it’s very fruitful and beneficial. When the mind is imbued with wisdom it is rightly freed from the defilements, namely, the defilements of sensuality, desire to be reborn, and ignorance.” 

\section*{12. Commencing the Rains at Beluva }

When\marginnote{2.21.1} the Buddha had stayed in \textsanskrit{Ambapālī}’s grove as long as he wished, he addressed Venerable Ānanda, “Come, Ānanda, let’s go to the little village of Beluva.” 

“Yes,\marginnote{2.21.3} sir,” Ānanda replied. Then the Buddha together with a large \textsanskrit{Saṅgha} of mendicants arrived at the little village of Beluva, and stayed there. 

There\marginnote{2.22.1} the Buddha addressed the mendicants: “Mendicants, please enter the rainy season residence with whatever friends or acquaintances you have around \textsanskrit{Vesālī}. I’ll commence the rainy season residence right here in the little village of Beluva.” 

“Yes,\marginnote{2.22.4} sir,” those mendicants replied. They did as the Buddha said, while the Buddha commenced the rainy season residence right there in the little village of Beluva. 

After\marginnote{2.23.1} the Buddha had commenced the rainy season residence, he fell severely ill, struck by dreadful pains, close to death. But he endured unbothered, with mindfulness and situational awareness. Then it occurred to the Buddha, “It would not be appropriate for me to become fully extinguished before informing my attendants and taking leave of the mendicant \textsanskrit{Saṅgha}. Why don’t I forcefully suppress this illness, stabilize the life force, and live on?” 

So\marginnote{2.24.1} that is what he did. Then the Buddha’s illness died down. 

Soon\marginnote{2.24.3} after the Buddha had recovered from that sickness, he came out from his dwelling and sat in the shade of the porch on the seat spread out. Then Venerable Ānanda went up to the Buddha, bowed, sat down to one side, and said to him, “Sir, it’s fantastic that the Buddha is comfortable and well. Because when the Buddha was sick, my body felt like it was drugged. I was disorientated, and the teachings weren’t clear to me. Still, at least I was consoled by the thought that the Buddha won’t become fully extinguished without making some statement regarding the \textsanskrit{Saṅgha} of mendicants.” 

“But\marginnote{2.25.1} what could the mendicant \textsanskrit{Saṅgha} expect from me, Ānanda? I’ve taught the Dhamma without making any distinction between secret and public teachings. The Realized One doesn’t have the closed fist of a teacher when it comes to the teachings. If there’s anyone who thinks: ‘I’ll take charge of the \textsanskrit{Saṅgha} of mendicants,’ or ‘the \textsanskrit{Saṅgha} of mendicants is meant for me,’ let them make a statement regarding the \textsanskrit{Saṅgha}. But the Realized One doesn’t think like this, so why should he make some statement regarding the \textsanskrit{Saṅgha}? 

I’m\marginnote{2.25.9} now old, elderly and senior. I’m advanced in years and have reached the final stage of life. I’m currently eighty years old. Just as a decrepit cart keeps going by relying on straps, in the same way, the Realized One’s body keeps going by relying on straps, or so you’d think. Sometimes the Realized One, not focusing on any signs, and with the cessation of certain feelings, enters and remains in the signless immersion of the heart. Only then does the Realized One’s body become more comfortable. 

So\marginnote{2.26.1} Ānanda, live as your own island, your own refuge, with no other refuge. Let the teaching be your island and your refuge, with no other refuge. And how does a mendicant do this? It’s when a mendicant meditates by observing an aspect of the body—keen, aware, and mindful, rid of desire and aversion for the world. They meditate observing an aspect of feelings … mind … principles—keen, aware, and mindful, rid of desire and aversion for the world. That’s how a mendicant is their own island, their own refuge, with no other refuge. That’s how the teaching is their island and their refuge, with no other refuge. 

Whether\marginnote{2.26.8} now or after I have passed, any who shall live as their own island, their own refuge, with no other refuge; with the teaching as their island and their refuge, with no other refuge—those mendicants of mine who want to train shall be among the best of the best.” 

\section*{13. An Obvious Hint }

Then\marginnote{3.1.1} the Buddha robed up in the morning and, taking his bowl and robe, entered \textsanskrit{Vesālī} for alms. Then, after the meal, on his return from almsround, he addressed Venerable Ānanda: “Ānanda, get your sitting cloth. Let’s go to the \textsanskrit{Cāpāla} shrine for the day’s meditation.” 

“Yes,\marginnote{3.1.5} sir,” replied Ānanda. Taking his sitting cloth he followed behind the Buddha. 

Then\marginnote{3.2.1} the Buddha went up to the \textsanskrit{Cāpāla} shrine, where he sat on the seat spread out. Ānanda bowed to the Buddha and sat down to one side. 

The\marginnote{3.2.3} Buddha said to him: “Ānanda, \textsanskrit{Vesālī} is lovely. And the Udena, Gotamaka, Sattamba, Bahuputta, \textsanskrit{Sārandada}, and \textsanskrit{Cāpāla} Tree-shrines are all lovely. 

Whoever\marginnote{3.3.1} has developed and cultivated the four bases of psychic power—made them a vehicle and a basis, kept them up, consolidated them, and properly implemented them—may, if they wish, live on for the eon or what’s left of the eon. The Realized One has developed and cultivated the four bases of psychic power, made them a vehicle and a basis, kept them up, consolidated them, and properly implemented them. If he wished, the Realized One could live on for the eon or what’s left of the eon.” 

But\marginnote{3.4.1} Ānanda didn’t get it, even though the Buddha dropped such an obvious hint, such a clear sign. He didn’t beg the Buddha: “Sir, may the Blessed One please remain for the eon! May the Holy One please remain for the eon! That would be for the welfare and happiness of the people, out of compassion for the world, for the benefit, welfare, and happiness of gods and humans.” For his mind was as if possessed by \textsanskrit{Māra}. 

For\marginnote{3.5.1} a second time … And for a third time, the Buddha said to Ānanda: “Ānanda, \textsanskrit{Vesālī} is lovely. And the Udena, Gotamaka, Sattamba, Bahuputta, \textsanskrit{Sārandada}, and \textsanskrit{Cāpāla} Tree-shrines are all lovely. Whoever has developed and cultivated the four bases of psychic power—made them a vehicle and a basis, kept them up, consolidated them, and properly implemented them—may, if they wish, live on for the eon, or what’s left of it. The Realized One has developed and cultivated the four bases of psychic power, made them a vehicle and a basis, kept them up, consolidated them, and properly implemented them. If he wished, the Realized One could live on for the eon, or what’s left of it.” 

But\marginnote{3.5.6} Ānanda didn’t get it, even though the Buddha dropped such an obvious hint, such a clear sign. He didn’t beg the Buddha: “Sir, may the Blessed One please remain for the eon! May the Holy One please remain for the eon! That would be for the welfare and happiness of the people, out of compassion for the world, for the benefit, welfare, and happiness of gods and humans.” For his mind was as if possessed by \textsanskrit{Māra}. 

Then\marginnote{3.6.1} the Buddha got up and said to Venerable Ānanda, “Go now, Ānanda, at your convenience.” 

“Yes,\marginnote{3.6.4} sir,” replied Ānanda. He rose from his seat, bowed, and respectfully circled the Buddha, keeping him on his right, before sitting at the root of a tree close by. 

\section*{14. The Appeal of \textsanskrit{Māra} }

And\marginnote{3.7.1} then, not long after Ānanda had left, \textsanskrit{Māra} the Wicked went up to the Buddha, stood to one side, and said to him: 

“Sir,\marginnote{3.7.2} may the Blessed One now become fully extinguished! May the Holy One now become fully extinguished! Now is the time for the Buddha to become fully extinguished. Sir, you once made this statement: ‘Wicked One, I will not become fully extinguished until I have monk disciples who are competent, educated, assured, learned, have memorized the teachings, and practice in line with the teachings. Not until they practice properly, living in line with the teaching. Not until they’ve learned their tradition, and explain, teach, assert, establish, disclose, analyze, and make it clear. Not until they can legitimately and completely refute the doctrines of others that come up, and teach with a demonstrable basis.’ 

Today\marginnote{3.8.1} you do have such monk disciples. May the Blessed One now become fully extinguished! May the Holy One now become fully extinguished! Now is the time for the Buddha to become fully extinguished. 

Sir,\marginnote{3.8.3} you once made this statement: ‘Wicked One, I will not become fully extinguished until I have nun disciples who are competent, educated, assured, learned …’ 

Today\marginnote{3.8.5} you do have such nun disciples. May the Blessed One now become fully extinguished! May the Holy One now become fully extinguished! Now is the time for the Buddha to become fully extinguished. 

Sir,\marginnote{3.8.7} you once made this statement: ‘Wicked One, I will not become fully extinguished until I have layman disciples who are competent, educated, assured, learned …’ 

Today\marginnote{3.8.9} you do have such layman disciples. May the Blessed One now become fully extinguished! May the Holy One now become fully extinguished! Now is the time for the Buddha to become fully extinguished. 

Sir,\marginnote{3.8.11} you once made this statement: ‘Wicked One, I will not become fully extinguished until I have laywoman disciples who are competent, educated, assured, learned …’ 

Today\marginnote{3.8.13} you do have such laywoman disciples. May the Blessed One now become fully extinguished! May the Holy One now become fully extinguished! Now is the time for the Buddha to become fully extinguished. 

Sir,\marginnote{3.8.15} you once made this statement: ‘Wicked One, I will not become fully extinguished until my spiritual path is successful and prosperous, extensive, popular, widespread, and well proclaimed wherever there are gods and humans.’ 

Today\marginnote{3.8.17} your spiritual path is successful and prosperous, extensive, popular, widespread, and well proclaimed wherever there are gods and humans. May the Blessed One now become fully extinguished! May the Holy One now become fully extinguished! Now is the time for the Buddha to become fully extinguished.” 

When\marginnote{3.9.1} this was said, the Buddha said to \textsanskrit{Māra}, “Relax, Wicked One. The final extinguishment of the Realized One will be soon. Three months from now the Realized One will finally be extinguished.” 

\section*{15. Surrendering the Life Force }

So\marginnote{3.10.1} at the \textsanskrit{Cāpāla} Tree-shrine the Buddha, mindful and aware, surrendered the life force. When he did so there was a great earthquake, awe-inspiring and hair-raising, and thunder cracked the sky. Then, understanding this matter, on that occasion the Buddha expressed this heartfelt sentiment: 

\begin{verse}%
“Weighing\marginnote{3.10.4} up the incomparable against an extension of life, \\
the sage surrendered the life force. \\
Happy inside, serene, \\
he burst out of this self-made chain like a suit of armor.” 

%
\end{verse}

\section*{16. The Causes of Earthquakes }

Then\marginnote{3.11.1} Venerable Ānanda thought, “How incredible, how amazing! That was a really big earthquake! That was really a very big earthquake; awe-inspiring and hair-raising, and thunder cracked the sky! What’s the cause, what’s the reason for a great earthquake?” 

Then\marginnote{3.12.1} Venerable Ānanda went up to the Buddha, bowed, sat down to one side, and said to him, “How incredible, sir, how amazing! That was a really big earthquake! That was really a very big earthquake; awe-inspiring and hair-raising, and thunder cracked the sky! What’s the cause, what’s the reason for a great earthquake?” 

“Ānanda,\marginnote{3.13.1} there are these eight causes and reasons for a great earthquake. What eight? 

This\marginnote{3.13.3} great earth is grounded on water, the water is grounded on air, and the air stands in space. At a time when a great wind blows, it stirs the water, and the water stirs the earth. This is the first cause and reason for a great earthquake. 

Furthermore,\marginnote{3.14.1} there is an ascetic or brahmin with psychic power who has achieved mastery of the mind, or a god who is mighty and powerful. They’ve developed a limited perception of earth and a limitless perception of water. They make the earth shake and rock and tremble. This is the second cause and reason for a great earthquake. 

Furthermore,\marginnote{3.15.1} when the being intent on awakening passes away from the host of Joyful Gods, he’s conceived in his mother’s belly, mindful and aware. Then the earth shakes and rocks and trembles. This is the third cause and reason for a great earthquake. 

Furthermore,\marginnote{3.16.1} when the being intent on awakening comes out of his mother’s belly mindful and aware, the earth shakes and rocks and trembles. This is the fourth cause and reason for a great earthquake. 

Furthermore,\marginnote{3.17.1} when the Realized One realizes the supreme perfect awakening, the earth shakes and rocks and trembles. This is the fifth cause and reason for a great earthquake. 

Furthermore,\marginnote{3.18.1} when the Realized One rolls forth the supreme Wheel of Dhamma, the earth shakes and rocks and trembles. This is the sixth cause and reason for a great earthquake. 

Furthermore,\marginnote{3.19.1} when the Realized One, mindful and aware, surrenders the life force, the earth shakes and rocks and trembles. This is the seventh cause and reason for a great earthquake. 

Furthermore,\marginnote{3.20.1} when the Realized One becomes fully extinguished through the element of extinguishment with nothing left over, the earth shakes and rocks and trembles. This is the eighth cause and reason for a great earthquake. 

These\marginnote{3.20.3} are the eight causes and reasons for a great earthquake. 

\section*{17. Eight Assemblies }

There\marginnote{3.21.1} are, Ānanda, these eight assemblies. What eight? The assemblies of aristocrats, brahmins, householders, and ascetics. An assembly of the gods of the Four Great Kings. An assembly of the gods of the Thirty-Three. An assembly of \textsanskrit{Māras}. An assembly of \textsanskrit{Brahmās}. 

I\marginnote{3.22.1} recall having approached an assembly of hundreds of aristocrats. There I used to sit with them, converse, and engage in discussion. And my appearance and voice became just like theirs. I educated, encouraged, fired up, and inspired them with a Dhamma talk. But when I spoke they didn’t know: ‘Who is this that speaks? Is it a god or a human?’ And when my Dhamma talk was finished I vanished. But when I vanished they didn’t know: ‘Who was that who vanished? Was it a god or a human?’ 

I\marginnote{3.23.1} recall having approached an assembly of hundreds of brahmins … householders … ascetics … the gods of the Four Great Kings … the gods of the Thirty-Three … \textsanskrit{Māras} … \textsanskrit{Brahmās}. There too I used to sit with them, converse, and engage in discussion. And my appearance and voice became just like theirs. I educated, encouraged, fired up, and inspired them with a Dhamma talk. But when I spoke they didn’t know: ‘Who is this that speaks? Is it a god or a human?’ And when my Dhamma talk was finished I vanished. But when I vanished they didn’t know: ‘Who was that who vanished? Was it a god or a human?’ 

These\marginnote{3.23.17} are the eight assemblies. 

\section*{18. Eight Dimensions of Mastery }

Ānanda,\marginnote{3.24.1} there are these eight dimensions of mastery. What eight? 

Perceiving\marginnote{3.25.1} form internally, someone sees visions externally, limited, both pretty and ugly. Mastering them, they perceive: ‘I know and see.’ This is the first dimension of mastery. 

Perceiving\marginnote{3.26.1} form internally, someone sees visions externally, limitless, both pretty and ugly. Mastering them, they perceive: ‘I know and see.’ This is the second dimension of mastery. 

Not\marginnote{3.27.1} perceiving form internally, someone sees visions externally, limited, both pretty and ugly. Mastering them, they perceive: ‘I know and see.’ This is the third dimension of mastery. 

Not\marginnote{3.28.1} perceiving form internally, someone sees visions externally, limitless, both pretty and ugly. Mastering them, they perceive: ‘I know and see.’ This is the fourth dimension of mastery. 

Not\marginnote{3.29.1} perceiving form internally, someone sees visions externally that are blue, with blue color, blue hue, and blue tint. They’re like a flax flower that’s blue, with blue color, blue hue, and blue tint. Or a cloth from \textsanskrit{Bāraṇasī} that’s smoothed on both sides, blue, with blue color, blue hue, and blue tint. In the same way, not perceiving form internally, someone sees visions externally, blue, with blue color, blue hue, and blue tint. Mastering them, they perceive: ‘I know and see.’ This is the fifth dimension of mastery. 

Not\marginnote{3.30.1} perceiving form internally, someone sees visions externally that are yellow, with yellow color, yellow hue, and yellow tint. They’re like a champak flower that’s yellow, with yellow color, yellow hue, and yellow tint. Or a cloth from \textsanskrit{Bāraṇasī} that’s smoothed on both sides, yellow, with yellow color, yellow hue, and yellow tint. In the same way, not perceiving form internally, someone sees visions externally that are yellow, with yellow color, yellow hue, and yellow tint. Mastering them, they perceive: ‘I know and see.’ This is the sixth dimension of mastery. 

Not\marginnote{3.31.1} perceiving form internally, someone sees visions externally that are red, with red color, red hue, and red tint. They’re like a scarlet mallow flower that’s red, with red color, red hue, and red tint. Or a cloth from \textsanskrit{Bāraṇasī} that’s smoothed on both sides, red, with red color, red hue, and red tint. In the same way, not perceiving form internally, someone sees visions externally that are red, with red color, red hue, and red tint. Mastering them, they perceive: ‘I know and see.’ This is the seventh dimension of mastery. 

Not\marginnote{3.32.1} perceiving form internally, someone sees visions externally that are white, with white color, white hue, and white tint. They’re like the morning star that’s white, with white color, white hue, and white tint. Or a cloth from \textsanskrit{Bāraṇasī} that’s smoothed on both sides, white, with white color, white hue, and white tint. In the same way, not perceiving form internally, someone sees visions externally that are white, with white color, white hue, and white tint. Mastering them, they perceive: ‘I know and see.’ This is the eighth dimension of mastery. 

These\marginnote{3.32.6} are the eight dimensions of mastery. 

\section*{19. The Eight Liberations }

Ānanda,\marginnote{3.33.1} there are these eight liberations. What eight? 

Having\marginnote{3.33.3} physical form, they see visions. This is the first liberation. 

Not\marginnote{3.33.5} perceiving form internally, they see visions externally. This is the second liberation. 

They’re\marginnote{3.33.7} focused only on beauty. This is the third liberation. 

Going\marginnote{3.33.9} totally beyond perceptions of form, with the ending of perceptions of impingement, not focusing on perceptions of diversity, aware that ‘space is infinite’, they enter and remain in the dimension of infinite space. This is the fourth liberation. 

Going\marginnote{3.33.11} totally beyond the dimension of infinite space, aware that ‘consciousness is infinite’, they enter and remain in the dimension of infinite consciousness. This is the fifth liberation. 

Going\marginnote{3.33.13} totally beyond the dimension of infinite consciousness, aware that ‘there is nothing at all’, they enter and remain in the dimension of nothingness. This is the sixth liberation. 

Going\marginnote{3.33.15} totally beyond the dimension of nothingness, they enter and remain in the dimension of neither perception nor non-perception. This is the seventh liberation. 

Going\marginnote{3.33.17} totally beyond the dimension of neither perception nor non-perception, they enter and remain in the cessation of perception and feeling. This is the eighth liberation. 

These\marginnote{3.33.19} are the eight liberations. 

Ānanda,\marginnote{3.34.1} this one time, when I was first awakened, I was staying near \textsanskrit{Uruvelā} at the goatherd’s banyan tree on the bank of the \textsanskrit{Nerañjarā} River. Then \textsanskrit{Māra} the wicked approached me, stood to one side, and said: ‘Sir, may the Blessed One now become fully extinguished! May the Holy One now become fully extinguished! Now is the time for the Buddha to become fully extinguished.’ When he had spoken, I said to \textsanskrit{Māra}: 

‘Wicked\marginnote{3.35.1} One, I will not become fully extinguished until I have monk disciples … nun disciples … layman disciples … laywoman disciples who are competent, educated, assured, learned. 

Not\marginnote{3.35.5} until my spiritual path is successful and prosperous, extensive, popular, widespread, and well proclaimed wherever there are gods and humans.’ 

Today,\marginnote{3.36.1} just now at the \textsanskrit{Cāpāla} shrine \textsanskrit{Māra} the Wicked approached me once more with the same request, reminding me of my former statement, and saying that those conditions had been fulfilled. 

When\marginnote{3.37.1} he had spoken, I said to \textsanskrit{Māra}: ‘Relax, Wicked One. The final extinguishment of the Realized One will be soon. Three months from now the Realized One will finally be extinguished.’ So today, just now at the \textsanskrit{Cāpāla} Tree-shrine, mindful and aware, I surrendered the life force.” 

\section*{20. The Appeal of Ānanda }

When\marginnote{3.38.1} he said this, Venerable Ānanda said to the Buddha, “Sir, may the Blessed One please remain for the eon! May the Holy One please remain for the eon! That would be for the welfare and happiness of the people, out of compassion for the world, for the benefit, welfare, and happiness of gods and humans.” 

“Enough\marginnote{3.38.3} now, Ānanda. Do not beg the Realized One. Now is not the time to beg the Realized One.” 

For\marginnote{3.39.1} a second time … For a third time, Ānanda said to the Buddha, “Sir, may the Blessed One please remain for the eon! May the Holy One please remain for the eon! That would be for the welfare and happiness of the people, out of compassion for the world, for the benefit, welfare, and happiness of gods and humans.” 

“Ānanda,\marginnote{3.39.4} do you have faith in the Realized One’s awakening?” 

“Yes,\marginnote{3.39.5} sir.” 

“Then\marginnote{3.39.6} why do you keep pressing me up to the third time?” 

“Sir,\marginnote{3.40.1} I have heard and learned this in the presence of the Buddha: ‘Whoever has developed and cultivated the four bases of psychic power—made them a vehicle and a basis, kept them up, consolidated them, and properly implemented them—may, if they wish, live on for the eon or what’s left of the eon. The Realized One has developed and cultivated the four bases of psychic power, made them a vehicle and a basis, kept them up, consolidated them, and properly implemented them. If he wished, the Realized One could live on for the eon or what’s left of the eon.’” 

“Do\marginnote{3.40.4} you have faith, Ānanda?” 

“Yes,\marginnote{3.40.5} sir.” 

“Therefore,\marginnote{3.40.6} Ānanda, the misdeed is yours alone, the mistake is yours alone. For even though the Realized One dropped such an obvious hint, such a clear sign, you didn’t beg me to remain for the eon, or what’s left of it. If you had begged me, I would have refused you twice, but consented on the third time. Therefore, Ānanda, the misdeed is yours alone, the mistake is yours alone. 

Ānanda,\marginnote{3.41.1} this one time I was staying near \textsanskrit{Rājagaha}, on the Vulture’s Peak Mountain. There I said to you: ‘Ānanda, \textsanskrit{Rājagaha} is lovely, and so is the Vulture’s Peak. Whoever has developed and cultivated the four bases of psychic power—made them a vehicle and a basis, kept them up, consolidated them, and properly implemented them—may, if they wish, live on for the eon or what’s left of the eon. The Realized One has developed and cultivated the four bases of psychic power, made them a vehicle and a basis, kept them up, consolidated them, and properly implemented them. If he wished, the Realized One could live on for the eon or what’s left of the eon.’ But you didn’t get it, even though I dropped such an obvious hint, such a clear sign. You didn’t beg me to remain for the eon, or what’s left of it. If you had begged me, I would have refused you twice, but consented on the third time. Therefore, Ānanda, the misdeed is yours alone, the mistake is yours alone. 

Ānanda,\marginnote{3.42.1} this one time I was staying right there near \textsanskrit{Rājagaha}, at the Gotama banyan tree … at Bandit’s Cliff … in the \textsanskrit{Sattapaṇṇi} cave on the slopes of Vebhara … at the Black rock on the slopes of Isigili … in the Cool Grove, under the Snake’s Hood Grotto … in the Hot Springs Monastery … in the Bamboo Grove, the squirrels’ feeding ground … in \textsanskrit{Jīvaka}’s mango grove … in the Maddakucchi deer park … 

And\marginnote{3.43.1} in each place I said to you: ‘Ānanda, \textsanskrit{Rājagaha} is lovely, and so are all these places. … If he wished, the Realized One could live on for the eon or what’s left of the eon.’ But you didn’t get it, even though I dropped such an obvious hint, such a clear sign. You didn’t beg me to remain for the eon, or what’s left of it. 

Ānanda,\marginnote{3.45.1} this one time I was staying right here near \textsanskrit{Vesālī}, at the Udena shrine … at the Gotamaka shrine … at the Sattamba shrine … at the Many Sons shrine … at the \textsanskrit{Sārandada} shrine … and just now, today at the \textsanskrit{Cāpāla} shrine. There I said to you: ‘Ānanda, \textsanskrit{Vesālī} is lovely. And the Udena, Gotamaka, Sattamba, Bahuputta, \textsanskrit{Sārandada}, and \textsanskrit{Cāpāla} Tree-shrines are all lovely. Whoever has developed and cultivated the four bases of psychic power—made them a vehicle and a basis, kept them up, consolidated them, and properly implemented them—may, if they wish, live on for the eon or what’s left of the eon. The Realized One has developed and cultivated the four bases of psychic power, made them a vehicle and a basis, kept them up, consolidated them, and properly implemented them. If he wished, the Realized One could live on for the eon or what’s left of the eon.’ But you didn’t get it, even though I dropped such an obvious hint, such a clear sign. You didn’t beg me to remain for the eon, or what’s left of it, saying: ‘Sir, may the Blessed One please remain for the eon! May the Holy One please remain for the eon! That would be for the welfare and happiness of the people, out of compassion for the world, for the benefit, welfare, and happiness of gods and humans.’ 

If\marginnote{3.47.7} you had begged me, I would have refused you twice, but consented on the third time. Therefore, Ānanda, the misdeed is yours alone, the mistake is yours alone. 

Did\marginnote{3.48.1} I not prepare for this when I explained that we must be parted and separated from all we hold dear and beloved? How could it possibly be so that what is born, created, conditioned, and liable to wear out should not wear out? The Realized One has discarded, eliminated, released, given up, relinquished, and surrendered the life force. He has definitively stated: ‘The final extinguishment of the Realized One will be soon. Three months from now the Realized One will finally be extinguished.’ It’s not possible for the Realized One, for the sake of life, to take back the life force once it has been given up like that. 

Come,\marginnote{3.48.8} Ānanda, let’s go to the Great Wood, the hall with the peaked roof.” 

“Yes,\marginnote{3.48.9} sir,” Ānanda replied. 

So\marginnote{3.49.1} the Buddha went with Ānanda to the hall with the peaked roof, and said to him, “Go, Ānanda, gather all the mendicants staying in the vicinity of \textsanskrit{Vesālī} together in the assembly hall.” 

“Yes,\marginnote{3.49.3} sir,” replied Ānanda. He did what the Buddha asked, went up to him, bowed, stood to one side, and said to him, “Sir, the mendicant \textsanskrit{Saṅgha} has assembled. Please, sir, go at your convenience.” 

Then\marginnote{3.50.1} the Buddha went to the assembly hall, where he sat on the seat spread out and addressed the mendicants: 

“So,\marginnote{3.50.3} mendicants, having carefully memorized those things I have taught you from my direct knowledge, you should cultivate, develop, and make much of them so that this spiritual practice may last for a long time. That would be for the welfare and happiness of the people, out of compassion for the world, for the benefit, welfare, and happiness of gods and humans. And what are those things I have taught from my direct knowledge? They are: the four kinds of mindfulness meditation, the four right efforts, the four bases of psychic power, the five faculties, the five powers, the seven awakening factors, and the noble eightfold path. 

These\marginnote{3.51.1} are the things I have taught from my direct knowledge. Having carefully memorized them, you should cultivate, develop, and make much of them so that this spiritual practice may last for a long time. That would be for the welfare and happiness of the people, out of compassion for the world, for the benefit, welfare, and happiness of gods and humans.” 

Then\marginnote{3.51.2} the Buddha said to the mendicants: 

“Come\marginnote{3.51.3} now, mendicants, I say to you all: ‘Conditions fall apart. Persist with diligence.’ The final extinguishment of the Realized One will be soon. Three months from now the Realized One will finally be extinguished.” 

That\marginnote{3.51.7} is what the Buddha said. Then the Holy One, the Teacher, went on to say: 

\begin{verse}%
“I’ve\marginnote{3.51.9} reached a ripe old age, \\
and little of my life is left. \\
Having given it up, I’ll depart; \\
I’ve made a refuge for myself. 

Diligent\marginnote{3.51.13} and mindful, \\
be of good virtues, mendicants! \\
With well-settled thoughts, \\
take good care of your minds. 

Whoever\marginnote{3.51.17} meditates diligently \\
in this teaching and training, \\
giving up transmigration through rebirths, \\
will make an end to suffering.” 

%
\end{verse}

\section*{21. The Elephant Look }

Then\marginnote{4.1.1} the Buddha robed up in the morning and, taking his bowl and robe, entered \textsanskrit{Vesālī} for alms. Then, after the meal, on his return from almsround, he turned his whole body, the way that elephants do, to look back at \textsanskrit{Vesālī}. He said to Venerable Ānanda: “Ānanda, this will be the last time the Realized One sees \textsanskrit{Vesālī}. Come, Ānanda, let’s go to \textsanskrit{Bhaṇḍagāma}.” 

“Yes,\marginnote{4.1.5} sir,” Ānanda replied. 

Then\marginnote{4.2.1} the Buddha together with a large \textsanskrit{Saṅgha} of mendicants arrived at \textsanskrit{Bhaṇḍagāma}, and stayed there. There the Buddha addressed the mendicants: 

“Mendicants,\marginnote{4.2.4} not understanding and not penetrating four things, both you and I have wandered and transmigrated for such a very long time. What four? Noble ethics, immersion, wisdom, and freedom. These noble ethics, immersion, wisdom, and freedom have been understood and comprehended. Craving for continued existence has been cut off; the conduit to rebirth is ended; now there are no more future lives.” 

That\marginnote{4.3.1} is what the Buddha said. Then the Holy One, the Teacher, went on to say: 

\begin{verse}%
“Ethics,\marginnote{4.3.3} immersion, and wisdom, \\
and the supreme freedom: \\
these things have been understood \\
by Gotama the renowned. 

And\marginnote{4.3.7} so the Buddha, having insight, \\
explained this teaching to the mendicants. \\
The teacher made an end of suffering, \\
seeing clearly, he is extinguished.” 

%
\end{verse}

And\marginnote{4.4.1} while staying there, too, he often gave this Dhamma talk to the mendicants: 

“Such\marginnote{4.4.2} is ethics, such is immersion, such is wisdom. When immersion is imbued with ethics it’s very fruitful and beneficial. When wisdom is imbued with immersion it’s very fruitful and beneficial. When the mind is imbued with wisdom it is rightly freed from the defilements, namely, the defilements of sensuality, desire to be reborn, and ignorance.” 

\section*{22. The Four Great References }

When\marginnote{4.5.1} the Buddha had stayed in \textsanskrit{Bhaṇḍagāma} as long as he wished, he addressed Ānanda, “Come, Ānanda, let’s go to \textsanskrit{Hatthigāma}.”… 

“Let’s\marginnote{4.5.3} go to \textsanskrit{Ambagāma}.”… 

“Let’s\marginnote{4.5.4} go to \textsanskrit{Jambugāma}.”… 

“Let’s\marginnote{4.5.5} go to Bhoganagara.” 

“Yes,\marginnote{4.6.1} sir,” Ānanda replied. Then the Buddha together with a large \textsanskrit{Saṅgha} of mendicants arrived at Bhoganagara, where he stayed at the Ānanda shrine. 

There\marginnote{4.7.2} the Buddha addressed the mendicants: “Mendicants, I will teach you the four great references. Listen and pay close attention, I will speak.” 

“Yes,\marginnote{4.7.5} sir,” they replied. The Buddha said this: 

“Take\marginnote{4.8.1} a mendicant who says: ‘Reverend, I have heard and learned this in the presence of the Buddha: this is the teaching, this is the monastic law, this is the Teacher’s instruction.’ You should neither approve nor dismiss that mendicant’s statement. Instead, you should carefully memorize those words and phrases, then check if they’re included in the discourses or found in the monastic law. If they’re not included in the discourses or found in the monastic law, you should draw the conclusion: ‘Clearly this is not the word of the Buddha. It has been incorrectly memorized by that mendicant.’ And so you should reject it. If they are included in the discourses or found in the monastic law, you should draw the conclusion: ‘Clearly this is the word of the Buddha. It has been correctly memorized by that mendicant.’ You should remember it. This is the first great reference. 

Take\marginnote{4.9.1} another mendicant who says: ‘In such-and-such monastery lives a \textsanskrit{Saṅgha} with seniors and leaders. I’ve heard and learned this in the presence of that \textsanskrit{Saṅgha}: this is the teaching, this is the monastic law, this is the Teacher’s instruction.’ You should neither approve nor dismiss that mendicant’s statement. Instead, you should carefully memorize those words and phrases, then check if they’re included in the discourses or found in the monastic law. If they’re not included in the discourses or found in the monastic law, you should draw the conclusion: ‘Clearly this is not the word of the Buddha. It has been incorrectly memorized by that \textsanskrit{Saṅgha}.’ And so you should reject it. If they are included in the discourses or found in the monastic law, you should draw the conclusion: ‘Clearly this is the word of the Buddha. It has been correctly memorized by that \textsanskrit{Saṅgha}.’ You should remember it. This is the second great reference. 

Take\marginnote{4.10.1} another mendicant who says: ‘In such-and-such monastery there are several senior mendicants who are very learned, knowledgeable in the scriptures, who have memorized the teachings, the monastic law, and the outlines. I’ve heard and learned this in the presence of those senior mendicants: this is the teaching, this is the monastic law, this is the Teacher’s instruction.’ You should neither approve nor dismiss that mendicant’s statement. Instead, you should carefully memorize those words and phrases, then check if they’re included in the discourses or found in the monastic law. If they’re not included in the discourses or found in the monastic law, you should draw the conclusion: ‘Clearly this is not the word of the Buddha. It has not been correctly memorized by those senior mendicants.’ And so you should reject it. If they are included in the discourses and found in the monastic law, you should draw the conclusion: ‘Clearly this is the word of the Buddha. It has been correctly memorized by those senior mendicants.’ You should remember it. This is the third great reference. 

Take\marginnote{4.11.1} another mendicant who says: ‘In such-and-such monastery there is a single senior mendicant who is very learned and knowledgeable in the scriptures, who has memorized the teachings, the monastic law, and the outlines. I’ve heard and learned this in the presence of that senior mendicant: this is the teaching, this is the monastic law, this is the Teacher’s instruction.’ You should neither approve nor dismiss that mendicant’s statement. Instead, you should carefully memorize those words and phrases, then check if they’re included in the discourses or found in the monastic law. If they’re not included in the discourses or found in the monastic law, you should draw the conclusion: ‘Clearly this is not the word of the Buddha. It has been incorrectly memorized by that senior mendicant.’ And so you should reject it. If they are included in the discourses and found in the monastic law, you should draw the conclusion: ‘Clearly this is the word of the Buddha. It has been correctly memorized by that senior mendicant.’ You should remember it. This is the fourth great reference. 

These\marginnote{4.11.15} are the four great references. You should remember them.” 

And\marginnote{4.12.1} while staying at the Ānanda shrine, too, the Buddha often gave this Dhamma talk to the mendicants: 

“Such\marginnote{4.12.2} is ethics, such is immersion, such is wisdom. When immersion is imbued with ethics it’s very fruitful and beneficial. When wisdom is imbued with immersion it’s very fruitful and beneficial. When the mind is imbued with wisdom it is rightly freed from the defilements, namely, the defilements of sensuality, desire to be reborn, and ignorance.” 

\section*{23. On Cunda the Smith }

When\marginnote{4.13.1} the Buddha had stayed in Bhoganagara as long as he wished, he addressed Ānanda, “Come, Ānanda, let’s go to \textsanskrit{Pāvā}.” 

“Yes,\marginnote{4.13.3} sir,” Ānanda replied. Then the Buddha together with a large \textsanskrit{Saṅgha} of mendicants arrived at \textsanskrit{Pāvā}, where he stayed in Cunda the smith’s mango grove. 

Cunda\marginnote{4.14.1} heard that the Buddha had arrived and was staying in his mango grove. Then he went to the Buddha, bowed, and sat down to one side. The Buddha educated, encouraged, fired up, and inspired him with a Dhamma talk. Then Cunda said to the Buddha, “Sir, may the Buddha together with the mendicant \textsanskrit{Saṅgha} please accept tomorrow’s meal from me.” The Buddha consented in silence. 

Then,\marginnote{4.16.1} knowing that the Buddha had consented, Cunda got up from his seat, bowed, and respectfully circled the Buddha, keeping him on his right, before leaving. 

And\marginnote{4.17.1} when the night had passed Cunda had a variety of delicious foods prepared in his own home, and plenty of pork on the turn. Then he had the Buddha informed of the time, saying, “Sir, it’s time. The meal is ready.” 

Then\marginnote{4.18.1} the Buddha robed up in the morning and, taking his bowl and robe, went to the home of Cunda together with the mendicant \textsanskrit{Saṅgha}, where he sat on the seat spread out and addressed Cunda, “Cunda, please serve me with the pork on the turn that you’ve prepared. And serve the mendicant \textsanskrit{Saṅgha} with the other foods.” 

“Yes,\marginnote{4.18.5} sir,” replied Cunda, and did as he was asked. 

Then\marginnote{4.19.1} the Buddha addressed Cunda, “Cunda, any pork on the turn that’s left over, you should bury it in a pond. I don’t see anyone in this world—with its gods, \textsanskrit{Māras}, and \textsanskrit{Brahmās}, this population with its ascetics and brahmins, its gods and humans—who could properly digest it except for the Realized One.” 

“Yes,\marginnote{4.19.4} sir,” replied Cunda. He did as he was asked, then came back to the Buddha, bowed, and sat down to one side. Then the Buddha educated, encouraged, fired up, and inspired him with a Dhamma talk, after which he got up from his seat and left. 

After\marginnote{4.20.1} the Buddha had eaten Cunda’s meal, he fell severely ill with bloody dysentery, struck by dreadful pains, close to death. But he endured unbothered, with mindfulness and situational awareness. Then he addressed Ānanda, “Come, Ānanda, let’s go to \textsanskrit{Kusinārā}.” 

“Yes,\marginnote{4.20.5} sir,” Ānanda replied. 

\begin{verse}%
I’ve\marginnote{4.20.6} heard that after eating \\
the meal of Cunda the smith, \\
the wise one fell severely ill, \\
with pains, close to death. 

A\marginnote{4.20.10} severe sickness struck the Teacher \\
who had eaten the pork on the turn. \\
While still purging the Buddha said: \\
“I’ll go to the citadel of \textsanskrit{Kusinārā}.” 

%
\end{verse}

\section*{24. Bringing a Drink }

Then\marginnote{4.21.1} the Buddha left the road and went to the root of a certain tree, where he addressed Ānanda, “Please, Ānanda, fold my outer robe in four and spread it out for me. I am tired and will sit down.” 

“Yes,\marginnote{4.21.3} sir,” replied Ānanda, and did as he was asked. The Buddha sat on the seat spread out. 

When\marginnote{4.22.1} he was seated he said to Venerable Ānanda, “Please, Ānanda, fetch me some water. I am thirsty and will drink.” 

When\marginnote{4.22.3} he said this, Venerable Ānanda said to the Buddha, “Sir, just now around five hundred carts have passed by. The shallow water has been churned up by their wheels, and it flows cloudy and murky. The \textsanskrit{Kakutthā} river is not far away, with clear, sweet, cool water, clean, with smooth banks, delightful. There the Buddha can drink and cool his limbs.” 

For\marginnote{4.23.1} a second time, the Buddha asked Ānanda for a drink, and for a second time Ānanda suggested going to the \textsanskrit{Kakutthā} river. 

And\marginnote{4.24.1} for a third time, the Buddha said to Ānanda, “Please, Ānanda, fetch me some water. I am thirsty and will drink.” 

“Yes,\marginnote{4.24.3} sir,” replied Ānanda. Taking his bowl he went to the river. Now, though the shallow water in that creek had been churned up by wheels, and flowed cloudy and murky, when Ānanda approached it flowed transparent, clear, and unclouded. 

Then\marginnote{4.25.1} Ānanda thought, “It’s incredible, it’s amazing! The Realized One has such psychic power and might! For though the shallow water in that creek had been churned up by wheels, and flowed cloudy and murky, when I approached it flowed transparent, clear, and unclouded.” Gathering a bowl of drinking water he went back to the Buddha, and said to him, “It’s incredible, sir, it’s amazing! The Realized One has such psychic power and might! Just now, though the shallow water in that creek had been churned up by wheels, and flowed cloudy and murky, when I approached it flowed transparent, clear, and unclouded. Drink the water, Blessed One! Drink the water, Holy One!” So the Buddha drank the water. 

\section*{25. On Pukkusa the Malla }

Now\marginnote{4.26.1} at that time Pukkusa the Malla, a disciple of \textsanskrit{Āḷāra} \textsanskrit{Kālāma}, was traveling along the road from \textsanskrit{Kusinārā} and \textsanskrit{Pāvā}. He saw the Buddha sitting at the root of a certain tree. He went up to him, bowed, sat down to one side, and said, “It’s incredible, sir, it’s amazing! Those who have gone forth remain in such peaceful meditations. 

Once\marginnote{4.27.1} it so happened that \textsanskrit{Āḷāra} \textsanskrit{Kālāma}, while traveling along a road, left the road and sat at the root of a nearby tree for the day’s meditation. Then around five hundred carts passed by right next to \textsanskrit{Āḷāra} \textsanskrit{Kālāma}. Then a certain person coming behind those carts went up to \textsanskrit{Āḷāra} \textsanskrit{Kālāma} and said to him: ‘Sir, didn’t you see the five hundred carts pass by?’ 

‘No,\marginnote{4.27.5} friend, I didn’t see them.’ 

‘But\marginnote{4.27.6} sir, didn’t you hear a sound?’ 

‘No,\marginnote{4.27.7} friend, I didn’t hear a sound.’ 

‘But\marginnote{4.27.8} sir, were you asleep?’ 

‘No,\marginnote{4.27.9} friend, I wasn’t asleep.’ 

‘But\marginnote{4.27.10} sir, were you conscious?’ 

‘Yes,\marginnote{4.27.11} friend.’ ‘So, sir, while conscious and awake you neither saw nor heard a sound as five hundred carts passed by right next to you? Why sir, even your outer robe is covered with dust!’ 

‘Yes,\marginnote{4.27.14} friend.’ 

Then\marginnote{4.27.15} that person thought: ‘It’s incredible, it’s amazing! Those who have gone forth remain in such peaceful meditations, in that, while conscious and awake he neither saw nor heard a sound as five hundred carts passed by right next to him.’ And after declaring his lofty confidence in \textsanskrit{Āḷāra} \textsanskrit{Kālāma}, he left.” 

“What\marginnote{4.28.1} do you think, Pukkusa? Which is harder and more challenging to do while conscious and awake: to neither see nor hear a sound as five hundred carts pass by right next to you? Or to neither see nor hear a sound as it’s raining and pouring, lightning’s flashing, and thunder’s cracking?” 

“What\marginnote{4.29.1} do five hundred carts matter, or six hundred, or seven hundred, or eight hundred, or nine hundred, or a thousand, or even a hundred thousand carts? It’s far harder and more challenging to neither see nor hear a sound as it’s raining and pouring, lightning’s flashing, and thunder’s cracking!” 

“This\marginnote{4.30.1} one time, Pukkusa, I was staying near \textsanskrit{Ātumā} in a threshing-hut. At that time it was raining and pouring, lightning was flashing, and thunder was cracking. And not far from the threshing-hut two farmers who were brothers were killed, as well as four oxen. Then a large crowd came from \textsanskrit{Ātumā} to the place where that happened. 

Now\marginnote{4.31.1} at that time I came out of the threshing-hut and was walking mindfully in the open near the door of the hut. Then having left that crowd, a certain person approached me, bowed, and stood to one side. I said to them, ‘Why, friend, has this crowd gathered?’ 

‘Just\marginnote{4.32.2} now, sir, it was raining and pouring, lightning was flashing, and thunder was cracking. And two farmers who were brothers were killed, as well as four oxen. Then this crowd gathered here. But sir, where were you?’ 

‘I\marginnote{4.32.5} was right here, friend.’ 

‘But\marginnote{4.32.6} sir, did you see?’ 

‘No,\marginnote{4.32.7} friend, I didn’t see anything.’ 

‘But\marginnote{4.32.8} sir, didn’t you hear a sound?’ 

‘No,\marginnote{4.32.9} friend, I didn’t hear a sound.’ 

‘But\marginnote{4.32.10} sir, were you asleep?’ 

‘No,\marginnote{4.32.11} friend, I wasn’t asleep.’ 

‘But\marginnote{4.32.12} sir, were you conscious?’ 

‘Yes,\marginnote{4.32.13} friend.’ 

‘So,\marginnote{4.32.14} sir, while conscious and awake you neither saw nor heard a sound as it was raining and pouring, lightning was flashing, and thunder was cracking?’ 

‘Yes,\marginnote{4.32.15} friend.’ 

Then\marginnote{4.33.1} that person thought: ‘It’s incredible, it’s amazing! Those who have gone forth remain in such peaceful meditations, in that, while conscious and awake he neither saw nor heard a sound as it was raining and pouring, lightning was flashing, and thunder was cracking.’ And after declaring their lofty confidence in me, they bowed and respectfully circled me, keeping me on their right, before leaving.” 

When\marginnote{4.34.1} he said this, Pukkusa said to him, “Any confidence I had in \textsanskrit{Āḷāra} \textsanskrit{Kālāma} I sweep away as in a strong wind, or float away as down a swift stream. Excellent, sir! Excellent! As if he were righting the overturned, or revealing the hidden, or pointing out the path to the lost, or lighting a lamp in the dark so people with good eyes can see what’s there, the Buddha has made the teaching clear in many ways. I go for refuge to the Buddha, to the teaching, and to the mendicant \textsanskrit{Saṅgha}. From this day forth, may the Buddha remember me as a lay follower who has gone for refuge for life.” 

Then\marginnote{4.35.1} Pukkusa addressed a certain man, “Please, my man, fetch a pair of ready to wear garments the color of rose-gold.” 

“Yes,\marginnote{4.35.3} sir,” replied that man, and did as he was asked. Then Pukkusa brought the garments to the Buddha, “Sir, please accept this pair of ready to wear garments the color of rose-gold from me out of compassion.” 

“Well\marginnote{4.35.6} then, Pukkusa, clothe me in one, and Ānanda in the other.” 

“Yes,\marginnote{4.35.7} sir,” replied Pukkusa, and did so. 

Then\marginnote{4.36.1} the Buddha educated, encouraged, fired up, and inspired Pukkusa the Malla with a Dhamma talk, after which he got up from his seat, bowed, and respectfully circled the Buddha before leaving. 

Then,\marginnote{4.37.1} not long after Pukkusa had left, Ānanda placed the pair of garments the color of rose-gold on the Buddha’s body. But when placed on the Buddha’s body they seemed to lose their shine. Then Ānanda said to the Buddha, “It’s incredible, sir, it’s amazing, how pure and bright is the color of the Realized One’s skin. When this pair of ready to wear garments  the color of rose-gold is placed on the Buddha’s body they seem to lose their lustre.” 

“That’s\marginnote{4.37.6} so true, Ānanda, that’s so true! There are two times when the color of the Realized One’s skin becomes extra pure and bright. What two? The night when a Realized One understands the supreme perfect awakening; and the night he becomes fully extinguished through the element of extinguishment with nothing left over. These are the are two times when the color of the Realized One’s skin becomes extra pure and bright. 

Today,\marginnote{4.38.4} Ānanda, in the last watch of the night, between a pair of sal trees in the sal forest of the Mallas at Upavattana near \textsanskrit{Kusinārā}, shall be the Realized One’s full extinguishment. Come, Ānanda, let’s go to the \textsanskrit{Kakutthā} River.” 

“Yes,\marginnote{4.38.6} sir,” Ānanda replied. 

\begin{verse}%
A\marginnote{4.38.7} pair of garments the color of rose-gold \\
was presented by Pukkusa; \\
when the teacher was clothed with them, \\
his golden skin glowed bright. 

%
\end{verse}

Then\marginnote{4.39.1} the Buddha together with a large \textsanskrit{Saṅgha} of mendicants went to the \textsanskrit{Kakutthā} River. He plunged into the river and bathed and drank. And when he had emerged, he went to the mango grove, where he addressed Venerable Cundaka, “Please, Cundaka, fold my outer robe in four and spread it out for me. I am tired and will lie down.” 

“Yes,\marginnote{4.40.1} sir,” replied Cundaka, and did as he was asked. And then the Buddha laid down in the lion’s posture—on the right side, placing one foot on top of the other—mindful and aware, and focused on the time of getting up. But Cundaka sat down right there in front of the Buddha. 

\begin{verse}%
Having\marginnote{4.41.1} gone to \textsanskrit{Kakutthā} Creek, \\
whose water was transparent, sweet, and clear, \\
the Teacher, being tired, plunged in, \\
the Realized One, without compare in the world. 

And\marginnote{4.41.5} after bathing and drinking the Teacher emerged. \\
Before the group of mendicants, in the middle, the Buddha, \\
the Teacher who rolled forth the present dispensation, \\
the great hermit went to the mango grove. 

He\marginnote{4.41.9} addressed the mendicant named Cundaka: \\
“Spread out my folded robe so I can lie down.” \\
The evolved one urged Cunda, \\
who quickly spread the folded robe. \\
The Teacher lay down so tired, \\
while Cunda sat there before him. 

%
\end{verse}

Then\marginnote{4.42.1} the Buddha said to Venerable Ānanda: 

“Now\marginnote{4.42.2} it may happen, Ānanda, that others may give rise to some regret for Cunda the smith: ‘It’s your loss, friend Cunda, it’s your misfortune, in that the Realized One became fully extinguished after eating his last almsmeal from you.’ You should dispel remorse in Cunda the smith like this: ‘You’re fortunate, friend Cunda, you’re so very fortunate, in that the Realized One became fully extinguished after eating his last almsmeal from you. I have heard and learned this in the presence of the Buddha. 

There\marginnote{4.42.8} are two almsmeal offerings that have identical fruit and result, and are more fruitful and beneficial than other almsmeal offerings. What two? The almsmeal after eating which a Realized One understands the supreme perfect awakening; and the almsmeal after eating which he becomes fully extinguished through the element of extinguishment with nothing left over. These two almsmeal offerings have identical fruit and result, and are more fruitful and beneficial than other almsmeal offerings. 

You’ve\marginnote{4.42.12} accumulated a deed that leads to long life, beauty, happiness, fame, heaven, and sovereignty.’ That’s how you should dispel remorse in Cunda the smith.” 

Then,\marginnote{4.43.1} understanding this matter, on that occasion the Buddha expressed this heartfelt sentiment: 

\begin{verse}%
“A\marginnote{4.43.2} giver’s merit grows; \\
enmity doesn’t build up when you have self-control. \\
A skillful person gives up bad things—\\
with the end of greed, hate, and delusion, they’re extinguished.” 

%
\end{verse}

\section*{26. The Pair of Sal Trees }

Then\marginnote{5.1.1} the Buddha said to Ānanda, “Come, Ānanda, let’s go to the far shore of the Golden River, and on to the sal forest of the Mallas at Upavattana near \textsanskrit{Kusinārā}.” 

“Yes,\marginnote{5.1.3} sir,” Ānanda replied. And that’s where they went. Then the Buddha addressed Ānanda, “Please, Ānanda, set up a cot for me between the twin sal trees, with my head to the north. I am tired and will lie down.” 

“Yes,\marginnote{5.1.6} sir,” replied Ānanda, and did as he was asked. And then the Buddha laid down in the lion’s posture—on the right side, placing one foot on top of the other—mindful and aware. 

Now\marginnote{5.2.1} at that time the twin sal trees were in full blossom with flowers out of season. They sprinkled and bestrewed the Realized One’s body in honor of the Realized One. And the flowers of the heavenly Flame Tree fell from the sky, and they too sprinkled and bestrewed the Realized One’s body in honor of the Realized One. And heavenly sandalwood powder fell from the sky, and it too sprinkled and bestrewed the Realized One’s body in honor of the Realized One. And heavenly music played in the sky in honor of the Realized One. And heavenly choirs sang in the sky in honor of the Realized One. 

Then\marginnote{5.3.1} the Buddha pointed out to Ānanda what was happening, adding: “That’s not how the Realized One is honored, respected, revered, venerated, and esteemed. Any monk or nun or male or female lay follower who practices in line with the teachings, practicing properly, living in line with the teachings—they honor, respect, revere, venerate, and esteem the Realized One with the highest honor. So Ānanda, you should train like this: ‘We shall practice in line with the teachings, practicing properly, living in line with the teaching.’ 

\section*{27. The Monk \textsanskrit{Upavāṇa} }

Now\marginnote{5.4.1} at that time Venerable \textsanskrit{Upavāṇa} was standing in front of the Buddha fanning him. Then the Buddha made him move, “Move over, mendicant, don’t stand in front of me.” 

Ānanda\marginnote{5.4.4} thought, “This Venerable \textsanskrit{Upavāṇa} has been the Buddha’s attendant for a long time, close to him, living in his presence. Yet in his final hour the Buddha makes him move, saying: ‘Move over, mendicant, don’t stand in front of me.’ What is the cause, what is the reason for this?” 

Then\marginnote{5.5.1} Ānanda said to the Buddha, “This Venerable \textsanskrit{Upavāṇa} has been the Buddha’s attendant for a long time, close to him, living in his presence. Yet in his final hour the Buddha makes him move, saying: ‘Move over, mendicant, don’t stand in front of me.’ What is the cause, sir, what is the reason for this?” 

“Most\marginnote{5.5.7} of the deities from ten solar systems have gathered to see the Realized One. For twelve leagues all around this sal grove there’s no spot, not even a fraction of a hair’s tip, that’s not crowded full of illustrious deities. The deities are complaining: ‘We’ve come such a long way to see the Realized One! Only rarely do Realized Ones arise in the world, perfected ones, fully awakened Buddhas. This very day, in the last watch of the night, the Realized One will become fully extinguished. And this illustrious mendicant is standing in front of the Buddha blocking the view. We won’t get to see the Realized One in his final hour!’” 

“But\marginnote{5.6.1} sir, what kind of deities are you thinking of?” 

“There\marginnote{5.6.2} are, Ānanda, deities—both in the sky and on the earth—who are percipient of the earth. With hair disheveled and arms raised, they fall down like their feet were chopped off, rolling back and forth, lamenting: ‘Too soon the Blessed One will become fully extinguished! Too soon the Holy One will become fully extinguished! Too soon the seer will vanish from the world!’ 

But\marginnote{5.6.6} the deities who are free of desire endure, mindful and aware, thinking: ‘Conditions are impermanent. How could it possibly be otherwise?’” 

\section*{28. The Four Inspiring Places }

“Previously,\marginnote{5.7.1} sir, when mendicants had completed the rainy season residence in various districts they came to see the Realized One. We got to see the esteemed mendicants, and to pay homage to them. But when the Buddha has passed, we won’t get to see the esteemed mendicants or to pay homage to them.” 

“Ānanda,\marginnote{5.8.1} a faithful gentleman should go to see these four inspiring places. What four? Thinking: ‘Here the Realized One was born!’—that is an inspiring place. Thinking: ‘Here the Realized One became awakened as a supreme fully awakened Buddha!’—that is an inspiring place. Thinking: ‘Here the supreme Wheel of Dhamma was rolled forth by the Realized One!’—that is an inspiring place. Thinking: ‘Here the Realized One became fully extinguished through the element of extinguishment with nothing left over!’—that is an inspiring place. These are the four inspiring places that a faithful gentleman should go to see. 

Faithful\marginnote{5.8.8} monks, nuns, laymen, and laywomen will come, and think: ‘Here the Realized One was born!’ and ‘Here the Realized One became awakened as a supreme fully awakened Buddha!’ and ‘Here the supreme Wheel of Dhamma was rolled forth by the Realized One!’ and ‘Here the Realized One became fully extinguished through the element of extinguishment with nothing left over!’ Anyone who passes away while on pilgrimage to these shrines will, when their body breaks up, after death, be reborn in a good place, a heavenly realm.” 

\section*{29. Ānanda’s Questions }

“Sir,\marginnote{5.9.1} how do we proceed when it comes to females?” 

“Without\marginnote{5.9.2} seeing, Ānanda.” 

“But\marginnote{5.9.3} when seeing, how to proceed?” 

“Without\marginnote{5.9.4} getting into conversation, Ānanda.” 

“But\marginnote{5.9.5} when in a conversation, how to proceed?” 

“Be\marginnote{5.9.6} mindful, Ānanda.” 

“Sir,\marginnote{5.10.1} how do we proceed when it comes to the Realized One’s corpse?” 

“Don’t\marginnote{5.10.2} get involved in the rites for venerating the Realized One’s corpse, Ānanda. Please, Ānanda, you must all strive and practice for your own goal! Meditate diligent, keen, and resolute for your own goal! There are astute aristocrats, brahmins, and householders who are devoted to the Realized One. They will perform the rites for venerating the Realized One’s corpse.” 

“But\marginnote{5.11.1} sir, how to proceed when it comes to the Realized One’s corpse?” 

“Proceed\marginnote{5.11.2} in the same way as they do for the corpse of a wheel-turning monarch.” 

“But\marginnote{5.11.3} how do they proceed with a wheel-turning monarch’s corpse?” 

“They\marginnote{5.11.4} wrap a wheel-turning monarch’s corpse with unworn cloth, then with uncarded cotton, then again with unworn cloth. In this way they wrap the corpse with five hundred double-layers. Then they place it in an iron case filled with oil and close it up with another case. Then, having built a funeral pyre out of all kinds of fragrant substances, they cremate the corpse. They build a monument for the wheel-turning monarch at the crossroads. That’s how they proceed with a wheel-turning monarch’s corpse. Proceed in the same way with the Realized One’s corpse. A monument for the Realized One is to be built at the crossroads. When someone there lifts up garlands or fragrance or powder, or bows, or inspires confidence in their heart, that will be for their lasting welfare and happiness. 

\section*{30. Persons Worthy of Monument }

Ānanda,\marginnote{5.12.1} these four are worthy of a monument. What four? A Realized One, a perfected one, a fully awakened Buddha; a Buddha awakened for themselves; a disciple of a Realized One; and a wheel-turning monarch. 

And\marginnote{5.12.4} for what reason is a Realized One worthy of a monument? So that many people will inspire confidence in their hearts, thinking: ‘This is the monument for that Blessed One, perfected and fully awakened!’ And having done so, when their body breaks up, after death, they are reborn in a good place, a heavenly realm. It is for this reason that a Realized One is worthy of a monument. 

And\marginnote{5.12.8} for what reason is a Buddha awakened for themselves worthy of a monument? So that many people will inspire confidence in their hearts, thinking: ‘This is the monument for that Buddha awakened for themselves!’ And having done so, when their body breaks up, after death, they are reborn in a good place, a heavenly realm. It is for this reason that a Buddha awakened for themselves is worthy of a monument. 

And\marginnote{5.12.12} for what reason is a Realized One’s disciple worthy of a monument? So that many people will inspire confidence in their hearts, thinking: ‘This is the monument for that Blessed One’s disciple!’ And having done so, when their body breaks up, after death, they are reborn in a good place, a heavenly realm. It is for this reason that a Realized One’s disciple is worthy of a monument. 

And\marginnote{5.12.16} for what reason is a wheel-turning monarch worthy of a monument? So that many people will inspire confidence in their hearts, thinking: ‘This is the monument for that just and principled king!’ And having done so, when their body breaks up, after death, they are reborn in a good place, a heavenly realm. It is for this reason that a wheel-turning monarch is worthy of a monument. 

These\marginnote{5.12.20} four are worthy of a monument.” 

\section*{31. Ānanda’s Incredible Qualities }

Then\marginnote{5.13.1} Venerable Ānanda entered a dwelling, and stood there leaning against the door-jamb and crying, “Oh! I’m still only a trainee with work left to do; and my Teacher’s about to become fully extinguished, he who is so kind to me!” 

Then\marginnote{5.13.3} the Buddha said to the mendicants, “Mendicants, where is Ānanda?” 

“Sir,\marginnote{5.13.5} Ānanda has entered a dwelling, and stands there leaning against the door-jamb and crying: ‘Oh! I’m still only a trainee with work left to do; and my Teacher’s about to become fully extinguished, he who is so kind to me!’” 

So\marginnote{5.13.7} the Buddha addressed a certain monk, “Please, monk, in my name tell Ānanda that the teacher summons him.” 

“Yes,\marginnote{5.13.10} sir,” that monk replied. He went to Ānanda and said to him, “Reverend Ānanda, the teacher summons you.” 

“Yes,\marginnote{5.14.1} reverend,” Ānanda replied. He went to the Buddha, bowed, and sat down to one side. The Buddha said to him: 

“Enough,\marginnote{5.14.2} Ānanda! Do not grieve, do not lament. Did I not prepare for this when I explained that we must be parted and separated from all we hold dear and beloved? How could it possibly be so that what is born, created, conditioned, and liable to wear out should not wear out, even the Realized One’s body? For a long time, Ānanda, you’ve treated the Realized One with deeds of body, speech, and mind that are loving, beneficial, pleasant, whole-hearted, and limitless. You have done good deeds, Ānanda. Devote yourself to meditation, and you will soon be free of defilements.” 

Then\marginnote{5.15.1} the Buddha said to the mendicants: 

“The\marginnote{5.15.2} Buddhas of the past or the future have attendants who are no better than Ānanda is for me. Ānanda is astute, he is intelligent. He knows the time for monks, nuns, laymen, laywomen, king’s ministers, religious founders, and the disciples of religious founders to visit the Realized One. 

There\marginnote{5.16.1} are these four incredible and amazing things about Ānanda. What four? If an assembly of monks goes to see Ānanda, they’re uplifted by seeing him and uplifted by hearing him speak. And when he falls silent, they’ve never had enough. If an assembly of nuns … laymen … or laywomen goes to see Ānanda, they’re uplifted by seeing him and uplifted by hearing him speak. And when he falls silent, they’ve never had enough. These are the four incredible and amazing things about Ānanda. 

There\marginnote{5.16.16} are these four incredible and amazing things about a wheel-turning monarch. What four? If an assembly of aristocrats goes to see a wheel-turning monarch, they’re uplifted by seeing him and uplifted by hearing him speak. And when he falls silent, they’ve never had enough. If an assembly of brahmins … householders … or ascetics goes to see a wheel-turning monarch, they’re uplifted by seeing him and uplifted by hearing him speak. And when he falls silent, they’ve never had enough. 

In\marginnote{5.16.26} the same way, there are those four incredible and amazing things about Ānanda.” 

\section*{32. Teaching the Discourse on \textsanskrit{Mahāsudassana} }

When\marginnote{5.17.1} he said this, Venerable Ānanda said to the Buddha: 

“Sir,\marginnote{5.17.2} please don’t become fully extinguished in this little hamlet, this jungle hamlet, this branch hamlet. There are other great cities such as \textsanskrit{Campā}, \textsanskrit{Rājagaha}, \textsanskrit{Sāvatthī}, \textsanskrit{Sāketa}, \textsanskrit{Kosambī}, and Benares. Let the Buddha become fully extinguished there. There are many well-to-do aristocrats, brahmins, and householders there who are devoted to the Buddha. They will perform the rites of venerating the Realized One’s corpse.” 

“Don’t\marginnote{5.17.8} say that Ānanda! Don’t say that this is a little hamlet, a jungle hamlet, a branch hamlet. 

Once\marginnote{5.18.1} upon a time there was a king named \textsanskrit{Mahāsudassana} who was a wheel-turning monarch, a just and principled king. His dominion extended to all four sides, he achieved stability in the country, and he possessed the seven treasures. His capital was this \textsanskrit{Kusinārā}, which at the time was named \textsanskrit{Kusāvatī}. It stretched for twelve leagues from east to west, and seven leagues from north to south. The royal capital of \textsanskrit{Kusāvatī} was successful, prosperous, populous, full of people, with plenty of food. It was just like \textsanskrit{Āḷakamandā}, the royal capital of the gods, which is successful, prosperous, populous, full of spirits, with plenty of food. \textsanskrit{Kusāvatī} was never free of ten sounds by day or night, namely: the sound of elephants, horses, chariots, drums, clay drums, arched harps, singing, horns, gongs, and handbells; and the cry: ‘Eat, drink, be merry!’ as the tenth. 

Go,\marginnote{5.19.1} Ānanda, into \textsanskrit{Kusinārā} and inform the Mallas: ‘This very day, \textsanskrit{Vāseṭṭhas}, in the last watch of the night, the Realized One will become fully extinguished. Come forth, \textsanskrit{Vāseṭṭhas}! Come forth, \textsanskrit{Vāseṭṭhas}! Don’t regret it later, thinking: ‘The Realized One became fully extinguished in our own village district, but we didn’t get a chance to see him in his final hour.’” 

“Yes,\marginnote{5.19.6} sir,” replied Ānanda. Then he robed up and, taking his bowl and robe, entered \textsanskrit{Kusinārā} with a companion. 

\section*{33. The Mallas Pay Homage }

Now\marginnote{5.20.1} at that time the Mallas of \textsanskrit{Kusinārā} were sitting together at the meeting hall on some business. Ānanda went up to them, and announced: “This very day, \textsanskrit{Vāseṭṭhas}, in the last watch of the night, the Realized One will become fully extinguished. Come forth, \textsanskrit{Vāseṭṭhas}! Come forth, \textsanskrit{Vāseṭṭhas}! Don’t regret it later, thinking: ‘The Realized One became fully extinguished in our own village district, but we didn’t get a chance to see him in his final hour.’” 

When\marginnote{5.21.1} they heard what Ānanda had to say, the Mallas, their sons, daughters-in-law, and wives became distraught, saddened, and grief-stricken. And some, with hair disheveled and arms raised, falling down like their feet were chopped off, rolling back and forth, lamented, “Too soon the Blessed One will become fully extinguished! Too soon the Holy One will become fully extinguished! Too soon the seer will vanish from the world!” 

Then\marginnote{5.21.3} the Mallas, their sons, daughters-in-law, and wives, distraught, saddened, and grief-stricken went to the Mallian sal grove at Upavattana and approached Ānanda. 

Then\marginnote{5.22.1} Ānanda thought, “If I have the Mallas pay homage to the Buddha one by one, they won’t be finished before first light. I’d better separate them family by family and then have them pay homage, saying: ‘Sir, the Malla named so-and-so with children, wives, retinue, and ministers bows with his head at your feet.’” And so that’s what he did. So by this means Ānanda got the Mallas to finish paying homage to the Buddha in the first watch of the night. 

\section*{34. On Subhadda the Wanderer }

Now\marginnote{5.23.1} at that time a wanderer named Subhadda was residing near \textsanskrit{Kusinārā}. He heard that on that very day, in the last watch of the night, the ascetic Gotama would become fully extinguished. He thought: “I have heard that brahmins of the past who were elderly and senior, the teachers of teachers, said: ‘Only rarely do Realized Ones arise in the world, perfected ones, fully awakened Buddhas.’ And this very day, in the last watch of the night, the ascetic Gotama will become fully extinguished. This state of uncertainty has come up in me. I am quite confident that the Buddha is capable of teaching me so that I can give up this state of uncertainty.” 

Then\marginnote{5.24.1} Subhadda went to the Mallian sal grove at Upavattana, approached Ānanda, and said to him, “Master Ānanda, I have heard that brahmins of the past who were elderly and senior, the teachers of teachers, said: ‘Only rarely do Realized Ones arise in the world, perfected ones, fully awakened Buddhas.’ And this very day, in the last watch of the night, the ascetic Gotama will become fully extinguished. This state of uncertainty has come up in me. I am quite confident that the Buddha is capable of teaching me so that I can give up this state of uncertainty. Master Ānanda, please let me see the ascetic Gotama.” 

When\marginnote{5.24.8} he had spoken, Ānanda said, “Enough, Reverend Subhadda, do not trouble the Realized One. He is tired.” 

For\marginnote{5.24.10} a second time, and a third time, Subhadda asked Ānanda, and a third time Ānanda refused. 

The\marginnote{5.25.1} Buddha heard that discussion between Ānanda and Subhadda. He said to Ānanda, “Enough, Ānanda, don’t obstruct Subhadda; let him see the Realized One. For whatever he asks me, he will only be looking for understanding, not trouble. And he will quickly understand any answer I give to his question.” 

So\marginnote{5.25.6} Ānanda said to the wanderer Subhadda, “Go, Reverend Subhadda, the Buddha is taking the time for you.” 

Then\marginnote{5.26.1} the wanderer Subhadda went up to the Buddha, and exchanged greetings with him. When the greetings and polite conversation were over, he sat down to one side and said to the Buddha: 

“Master\marginnote{5.26.2} Gotama, there are those ascetics and brahmins who lead an order and a community, and teach a community. They’re well-known and famous religious founders, regarded as holy by many people. Namely: \textsanskrit{Pūraṇa} Kassapa, Makkhali \textsanskrit{Gosāla}, \textsanskrit{Nigaṇṭha} \textsanskrit{Nāṭaputta}, \textsanskrit{Sañjaya} \textsanskrit{Belaṭṭhiputta}, Pakudha \textsanskrit{Kaccāyana}, and Ajita Kesakambala. According to their own claims, did all of them have direct knowledge, or none of them, or only some?” 

“Enough,\marginnote{5.26.5} Subhadda, let that be. I shall teach you the Dhamma. Listen and pay close attention, I will speak.” 

“Yes,\marginnote{5.26.9} sir,” Subhadda replied. The Buddha said this: 

“Subhadda,\marginnote{5.27.1} in whatever teaching and training the noble eightfold path is not found, there is no true ascetic found, no second ascetic, no third ascetic, and no fourth ascetic. In whatever teaching and training the noble eightfold path is found, there is a true ascetic found, a second ascetic, a third ascetic, and a fourth ascetic. In this teaching and training the noble eightfold path is found. Only here is there a true ascetic, here a second ascetic, here a third ascetic, and here a fourth ascetic. Other sects are empty of ascetics. 

Were\marginnote{5.27.4} these mendicants to practice well, the world would not be empty of perfected ones. 

\begin{verse}%
I\marginnote{5.27.5} was twenty-nine years of age, Subaddha, \\
when I went forth to discover what is skillful. \\
It’s been over fifty years \\
since I went forth. \\
I am the one who points out the proper teaching: \\
Outside of here there is no true ascetic. 

%
\end{verse}

Were\marginnote{5.27.11} these mendicants to practice well, the world would not be empty of perfected ones.” 

When\marginnote{5.28.1} he had spoken, Subhadda said to the Buddha, “Excellent, sir! Excellent! As if he were righting the overturned, or revealing the hidden, or pointing out the path to the lost, or lighting a lamp in the dark so people with good eyes can see what’s there, the Buddha has made the teaching clear in many ways. I go for refuge to the Buddha, to the teaching, and to the mendicant \textsanskrit{Saṅgha}. Sir, may I receive the going forth, the ordination in the Buddha’s presence?” 

“Subhadda,\marginnote{5.29.1} if someone formerly ordained in another sect wishes to take the going forth, the ordination in this teaching and training, they must spend four months on probation. When four months have passed, if the mendicants are satisfied, they’ll give the going forth, the ordination into monkhood. However, I have recognized individual differences in this matter.” 

“Sir,\marginnote{5.29.3} if four months probation are required in such a case, I’ll spend four years on probation. When four years have passed, if the mendicants are satisfied, let them give me the going forth, the ordination into monkhood.” 

Then\marginnote{5.29.4} the Buddha said to Ānanda, “Well then, Ānanda, give Subhadda the going forth.” 

“Yes,\marginnote{5.29.6} sir,” Ānanda replied. 

Then\marginnote{5.30.1} Subhadda said to Ānanda, “You’re so fortunate, Reverand Ānanda, so very fortunate, to be anointed here in the Teacher’s presence as his pupil!” And the wanderer Subhadda received the going forth, the ordination in the Buddha’s presence. Not long after his ordination, Venerable Subhadda, living alone, withdrawn, diligent, keen, and resolute, soon realized the supreme end of the spiritual path in this very life. He lived having achieved with his own insight the goal for which gentlemen rightly go forth from the lay life to homelessness. 

He\marginnote{5.30.6} understood: “Rebirth is ended; the spiritual journey has been completed; what had to be done has been done; there is no return to any state of existence.” And Venerable Subhadda became one of the perfected. He was the last personal disciple of the Buddha. 

\section*{35. The Buddha’s Last Words }

Then\marginnote{6.1.1} the Buddha addressed Venerable Ānanda: 

“Now,\marginnote{6.1.2} Ānanda, some of you might think: ‘The teacher’s dispensation has passed. Now we have no Teacher.’ But you should not see it like this. The teaching and training that I have taught and pointed out for you shall be your Teacher after my passing. 

After\marginnote{6.2.1} my passing, mendicants ought not address each other as ‘reverend’, as they do today. A more senior mendicant ought to address a more junior mendicant by name or clan, or by saying ‘reverend’. A more junior mendicant ought to address a more senior mendicant using ‘sir’ or ‘venerable’. 

If\marginnote{6.3.1} it wishes, after my passing the \textsanskrit{Saṅgha} may abolish the lesser and minor training rules. 

After\marginnote{6.4.1} my passing, give the prime punishment to the mendicant Channa.” 

“But\marginnote{6.4.2} sir, what is the prime punishment?” 

“Channa\marginnote{6.4.3} may say what he likes, but the mendicants should not advise or instruct him.” 

Then\marginnote{6.5.1} the Buddha said to the mendicants, “Perhaps even a single mendicant has doubt or uncertainty regarding the Buddha, the teaching, the \textsanskrit{Saṅgha}, the path, or the practice. So ask, mendicants! Don’t regret it later, thinking: ‘We were in the Teacher’s presence and we weren’t able to ask the Buddha a question.’” 

When\marginnote{6.5.4} this was said, the mendicants kept silent. 

For\marginnote{6.6.1} a second time, and a third time the Buddha addressed the mendicants: “Perhaps even a single mendicant has doubt or uncertainty regarding the Buddha, the teaching, the \textsanskrit{Saṅgha}, the path, or the practice. So ask, mendicants! Don’t regret it later, thinking: ‘We were in the Teacher’s presence and we weren’t able to ask the Buddha a question.’” 

For\marginnote{6.6.5} a third time, the mendicants kept silent. Then the Buddha said to the mendicants, 

“Mendicants,\marginnote{6.6.7} perhaps you don’t ask out of respect for the Teacher. So let a friend tell a friend.” 

When\marginnote{6.6.8} this was said, the mendicants kept silent. 

Then\marginnote{6.6.9} Venerable Ānanda said to the Buddha, “It’s incredible, sir, it’s amazing! I am quite confident that there’s not even a single mendicant in this \textsanskrit{Saṅgha} who has doubt or uncertainty regarding the Buddha, the teaching, the \textsanskrit{Saṅgha}, the path, or the practice.” 

“Ānanda,\marginnote{6.6.11} you speak from faith. But the Realized One knows that there’s not even a single mendicant in this \textsanskrit{Saṅgha} who has doubt or uncertainty regarding the Buddha, the teaching, the \textsanskrit{Saṅgha}, the path, or the practice. Even the last of these five hundred mendicants is a stream-enterer, not liable to be reborn in the underworld, bound for awakening.” 

Then\marginnote{6.7.1} the Buddha said to the mendicants: “Come now, mendicants, I say to you all: ‘Conditions fall apart. Persist with diligence.’” 

These\marginnote{6.7.4} were the Realized One’s last words. 

\section*{36. The Full Extinguishment }

Then\marginnote{6.8.1} the Buddha entered the first absorption. Emerging from that, he entered the second absorption. Emerging from that, he successively entered into and emerged from the third absorption, the fourth absorption, the dimension of infinite space, the dimension of infinite consciousness, the dimension of nothingness, and the dimension of neither perception nor non-perception. Then he entered the cessation of perception and feeling. 

Then\marginnote{6.8.2} Venerable Ānanda said to Venerable Anuruddha, “Venerable Anuruddha, has the Buddha become fully extinguished?” 

“No,\marginnote{6.8.4} Reverend Ānanda. He has entered the cessation of perception and feeling.” 

Then\marginnote{6.9.1} the Buddha emerged from the cessation of perception and feeling and entered the dimension of neither perception nor non-perception. Emerging from that, he successively entered into and emerged from the dimension of nothingness, the dimension of infinite consciousness, the dimension of infinite space, the fourth absorption, the third absorption, the second absorption, and the first absorption. Emerging from that, he successively entered into and emerged from the second absorption and the third absorption. Then he entered the fourth absorption. Emerging from that the Buddha immediately became fully extinguished. 

When\marginnote{6.10.1} the Buddha became fully extinguished, along with the full extinguishment there was a great earthquake, awe-inspiring and hair-raising, and thunder cracked the sky. When the Buddha became fully extinguished, \textsanskrit{Brahmā} Sahampati recited this verse: 

\begin{verse}%
“All\marginnote{6.10.3} creatures in this world \\
must lay down this bag of bones. \\
For even a Teacher such as this, \\
unrivaled in the world, \\
the Realized One, attained to power, \\
the Buddha became fully extinguished.” 

%
\end{verse}

When\marginnote{6.10.9} the Buddha became fully extinguished, Sakka, lord of gods, recited this verse: 

\begin{verse}%
“Oh!\marginnote{6.10.10} Conditions are impermanent, \\
their nature is to rise and fall; \\
having arisen, they cease; \\
their stilling is true bliss.” 

%
\end{verse}

When\marginnote{6.10.14} the Buddha became fully extinguished, Venerable Anuruddha recited this verse: 

\begin{verse}%
“There\marginnote{6.10.15} was no more breathing \\
for the poised one of steady heart. \\
Imperturbable, committed to peace, \\
the sage has done his time. 

He\marginnote{6.10.19} put up with painful feelings \\
without flinching. \\
The liberation of his heart \\
was like the extinguishing of a lamp.” 

%
\end{verse}

When\marginnote{6.10.23} the Buddha became fully extinguished, Venerable Ānanda recited this verse: 

\begin{verse}%
“Then\marginnote{6.10.24} there was terror! \\
Then they had goosebumps! \\
When the Buddha, endowed with all fine qualities, \\
became fully extinguished.” 

%
\end{verse}

When\marginnote{6.10.28} the Buddha became fully extinguished, some of the mendicants there, with arms raised, falling down like their feet were chopped off, rolling back and forth, lamented: “Too soon the Blessed One has become fully extinguished! Too soon the Holy One has become fully extinguished! Too soon the seer has vanished from the world!” But the mendicants who were free of desire endured, mindful and aware, thinking, “Conditions are impermanent. How could it possibly be otherwise?” 

Then\marginnote{6.11.1} Anuruddha addressed the mendicants: “Enough, reverends, do not grieve or lament. Did the Buddha not prepare us for this when he explained that we must be parted and separated from all we hold dear and beloved? How could it possibly be so that what is born, created, conditioned, and liable to wear out should not wear out? The deities are complaining.” 

“But\marginnote{6.11.7} sir, what kind of deities are you thinking of?” 

“There\marginnote{6.11.8} are, Ānanda, deities—both in the sky and on the earth—who are percipient of the earth. With hair disheveled and arms raised, they fall down like their feet were chopped off, rolling back and forth, lamenting: ‘Too soon the Blessed One has become fully extinguished! Too soon the Holy One has become fully extinguished! Too soon the seer has vanished from the world!’ But the deities who are free of desire endure, mindful and aware, thinking: ‘Conditions are impermanent. How could it possibly be otherwise?’” 

Ānanda\marginnote{6.11.14} and Anuruddha spent the rest of the night talking about Dhamma. 

Then\marginnote{6.12.1} Anuruddha said to Ānanda, “Go, Ānanda, into \textsanskrit{Kusinārā} and inform the Mallas: ‘\textsanskrit{Vāseṭṭhas}, the Buddha has become fully extinguished. Please come at your convenience.’” 

“Yes,\marginnote{6.12.5} sir,” replied Ānanda. Then, in the morning, he robed up and, taking his bowl and robe, entered \textsanskrit{Kusinārā} with a companion. 

Now\marginnote{6.12.6} at that time the Mallas of \textsanskrit{Kusinārā} were sitting together at the meeting hall on some business. Ānanda went up to them, and announced, “\textsanskrit{Vāseṭṭhas}, the Buddha has become fully extinguished. Please come at your convenience.” 

When\marginnote{6.12.10} they heard what Ānanda had to say, the Mallas, their sons, daughters-in-law, and wives became distraught, saddened, and grief-stricken. And some, with hair disheveled and arms raised, falling down like their feet were chopped off, rolling back and forth, lamented, “Too soon the Blessed One has become fully extinguished! Too soon the Holy One has become fully extinguished! Too soon the seer has vanished from the world!” 

\section*{37. The Rites of Venerating the Buddha’s Corpse }

Then\marginnote{6.13.1} the Mallas ordered their men, “So then, my men, collect fragrances and garlands, and all the musical instruments in \textsanskrit{Kusinārā}.” 

Then—taking\marginnote{6.13.3} those fragrances and garlands, all the musical instruments, and five hundred pairs of garments—they went to the Mallian sal grove at Upavattana and approached the Buddha’s corpse. They spent the day honoring, respecting, revering, and venerating the Buddha’s corpse with dance and song and music and garlands and fragrances, and making awnings and setting up pavilions. 

Then\marginnote{6.13.4} they thought, “It’s too late to cremate the Buddha’s corpse today. Let’s do it tomorrow.” But they spent the next day the same way, and so too the third, fourth, fifth, and sixth days. 

Then\marginnote{6.14.1} on the seventh day they thought, “Honoring, respecting, revering, and venerating the Buddha’s corpse with dance and song and music and garlands and fragrances, let us carry it to the south of the town, and cremate it there outside the town.” 

Now\marginnote{6.14.3} at that time eight of the leading Mallas, having bathed their heads and dressed in unworn clothes, said, “We shall lift the Buddha’s corpse.” But they were unable to do so. 

The\marginnote{6.14.5} Mallas said to Anuruddha, “What is the cause, Venerable Anuruddha, what is the reason why these eight Mallian leaders are unable to lift the Buddha’s corpse?” 

“\textsanskrit{Vāseṭṭhas},\marginnote{6.14.8} you have one plan, but the deities have a different one.” 

“But\marginnote{6.15.1} sir, what is the deities’ plan?” 

“You\marginnote{6.15.2} plan to carry the Buddha’s corpse to the south of the town while venerating it with dance and song and music and garlands and fragrances, and cremate it there outside the town. The deities plan to carry the Buddha’s corpse to the north of the town while venerating it with heavenly dance and song and music and garlands and fragrances. Then they plan to enter the town by the northern gate, carry it through the center of the town, leave by the eastern gate, and cremate it there at the Mallian shrine named \textsanskrit{Makuṭabandhana}.” 

“Sir,\marginnote{6.15.6} let it be as the deities plan.” 

Now\marginnote{6.16.1} at that time the whole of \textsanskrit{Kusinārā} was covered knee-deep with the flowers of the Flame Tree, without gaps even on the filth and rubbish heaps. Then the deities and the Mallas of \textsanskrit{Kusinārā} carried the Buddha’s corpse to the north of the town while venerating it with heavenly and human dance and song and music and garlands and fragrances. Then they entered the town by the northern gate, carried it through the center of the town, left by the eastern gate, and deposited the corpse there at the Mallian shrine named \textsanskrit{Makuṭabandhana}. 

Then\marginnote{6.17.1} the Mallas said to Anuruddha, “Sir, how do we proceed when it comes to the Realized One’s corpse?” 

“Proceed\marginnote{6.17.3} in the same way as they do for the corpse of a wheel-turning monarch.” 

“But\marginnote{6.17.4} how do they proceed with a wheel-turning monarch’s corpse?” 

“They\marginnote{6.17.5} wrap a wheel-turning monarch’s corpse with unworn cloth, then with uncarded cotton, then again with unworn cloth. In this way they wrap the corpse with five hundred double-layers. Then they place it in an iron case filled with oil and close it up with another case. Then, having built a funeral pyre out of all kinds of fragrant substances, they cremate the corpse. They build a monument for the wheel-turning monarch at the crossroads. That’s how they proceed with a wheel-turning monarch’s corpse. Proceed in the same way with the Realized One’s corpse. A monument for the Realized One is to be built at the crossroads. When someone there lifts up garlands or fragrance or powder, or bows, or inspires confidence in their heart, that will be for their lasting welfare and happiness.” 

Then\marginnote{6.18.1} the Mallas ordered their men, “So then, my men, collect uncarded cotton.” 

So\marginnote{6.18.3} the Mallas wrapped the Buddha’s corpse, and placed it in an iron case filled with oil. Then, having built a funeral pyre out of all kinds of fragrant substances, they lifted the corpse on to the pyre. 

\section*{38. \textsanskrit{Mahākassapa}’s Arrival }

Now\marginnote{6.19.1} at that time Venerable \textsanskrit{Mahākassapa} was traveling along the road from \textsanskrit{Pāvā} to \textsanskrit{Kusinārā} together with a large \textsanskrit{Saṅgha} of around five hundred mendicants. Then he left the road and sat at the root of a tree. 

Now\marginnote{6.19.3} at that time a certain \textsanskrit{Ājīvaka} ascetic had picked up a Flame Tree flower in \textsanskrit{Kusinārā} and was traveling along the road to \textsanskrit{Pāvā}. \textsanskrit{Mahākassapa} saw him coming off in the distance and said to him, “Reverend, might you know about our Teacher?” 

“Yes,\marginnote{6.19.6} reverend. Seven days ago the ascetic Gotama became fully extinguished. From there I picked up this Flame Tree flower.” Some of the mendicants there, with arms raised, falling down like their feet were chopped off, rolling back and forth, lamented, “Too soon the Blessed One has become fully extinguished! Too soon the Holy One has become fully extinguished! Too soon the seer has vanished from the world!” But the mendicants who were free of desire endured, mindful and aware, thinking, “Conditions are impermanent. How could it possibly be otherwise?” 

Now\marginnote{6.20.1} at that time a monk named Subhadda, who had gone forth when old, was sitting in that assembly. He said to those mendicants, “Enough, reverends, do not grieve or lament. We’re well rid of that Great Ascetic harassing us: ‘This is allowable for you; this is not allowable for you.’ Well, now we shall do what we want and not do what we don’t want.” 

Then\marginnote{6.20.6} Venerable \textsanskrit{Mahākassapa} addressed the mendicants, “Enough, reverends, do not grieve or lament. Did the Buddha not prepare us for this when he explained that we must be parted and separated from all we hold dear and beloved? How could it possibly be so that what is born, created, conditioned, and liable to wear out should not wear out, even the Realized One’s body?” 

Now\marginnote{6.21.1} at that time four of the leading Mallas, having bathed their heads and dressed in unworn clothes, said, “We shall light the Buddha’s funeral pyre.” But they were unable to do so. 

The\marginnote{6.21.3} Mallas said to Anuruddha, “What is the cause, Venerable Anuruddha, what is the reason why these four Mallian leaders are unable to light the Buddha’s funeral pyre?” 

“\textsanskrit{Vāseṭṭhas},\marginnote{6.21.6} the deities have a different plan.” 

“But\marginnote{6.21.7} sir, what is the deities’ plan?” 

“The\marginnote{6.21.8} deities’ plan is this: Venerable \textsanskrit{Mahākassapa} is traveling along the road from \textsanskrit{Pāvā} to \textsanskrit{Kusinārā} together with a large \textsanskrit{Saṅgha} of around five hundred mendicants. The Buddha’s funeral pyre shall not burn until he bows with his head at the Buddha’s feet.” 

“Sir,\marginnote{6.21.11} let it be as the deities plan.” 

Then\marginnote{6.22.1} Venerable \textsanskrit{Mahākassapa} came to the Mallian shrine named \textsanskrit{Makuṭabandhana} at \textsanskrit{Kusinārā} and approached the Buddha’s funeral pyre. Arranging his robe over one shoulder and raising his joined palms, he respectfully circled the Buddha three times, keeping him on his right, and bowed with his head at the Buddha’s feet. And the five hundred mendicants did likewise. And when \textsanskrit{Mahākassapa} and the five hundred mendicants bowed the Buddha’s funeral pyre burst into flames all by itself. 

And\marginnote{6.23.1} when the Buddha’s corpse was cremated no ash or soot was found from outer or inner skin, flesh, sinews, or synovial fluid. Only the relics remained. It’s like when ghee or oil blaze and burn, and neither ashes nor soot are found. In the same way, when the Buddha’s corpse was cremated no ash or soot was found from outer or inner skin, flesh, sinews, or synovial fluid. Only the relics remained. And of those five hundred pairs of garments only two were not burnt: the innermost and the outermost. But when the Buddha’s corpse was consumed the funeral pyre was extinguished by a stream of water that appeared in the sky, by water dripping from the sal trees, and by the Mallas’ fragrant water. 

Then\marginnote{6.23.10} the Mallas made a cage of spears for the Buddha’s relics in the meeting hall and surrounded it with a buttress of bows. For seven days they honored, respected, revered, and venerated them with dance and song and music and garlands and fragrances. 

\section*{39. Distributing the Relics }

King\marginnote{6.24.1} \textsanskrit{Ajātasattu} of Magadha heard that the Buddha had become fully extinguished at \textsanskrit{Kusinārā}. He sent an envoy to the Mallas of \textsanskrit{Kusinārā}: “The Buddha was an aristocrat, and so am I. I too deserve a share of the Buddha’s relics. I will build a monument for them and conduct a memorial service.” 

The\marginnote{6.24.5} Licchavis of \textsanskrit{Vesālī} also heard that the Buddha had become fully extinguished at \textsanskrit{Kusinārā}. They sent an envoy to the Mallas of \textsanskrit{Kusinārā}: “The Buddha was an aristocrat, and so are we. We too deserve a share of the Buddha’s relics. We will build a monument for them and conduct a memorial service.” 

The\marginnote{6.24.9} Sakyans of Kapilavatthu also heard that the Buddha had become fully extinguished at \textsanskrit{Kusinārā}. They sent an envoy to the Mallas of \textsanskrit{Kusinārā}: “The Buddha was our foremost relative. We too deserve a share of the Buddha’s relics. We will build a monument for them and conduct a memorial service.” 

The\marginnote{6.24.13} Bulas of Allakappa also heard that the Buddha had become fully extinguished at \textsanskrit{Kusinārā}. They sent an envoy to the Mallas of \textsanskrit{Kusinārā}: “The Buddha was an aristocrat, and so are we. We too deserve a share of the Buddha’s relics. We will build a monument for them and conduct a memorial service.” 

The\marginnote{6.24.17} \textsanskrit{Koḷiyans} of \textsanskrit{Rāmagāma} also heard that the Buddha had become fully extinguished at \textsanskrit{Kusinārā}. They sent an envoy to the Mallas of \textsanskrit{Kusinārā}: “The Buddha was an aristocrat, and so are we. We too deserve a share of the Buddha’s relics. We will build a monument for them and conduct a memorial service.” 

The\marginnote{6.24.21} brahmin of \textsanskrit{Veṭhadīpa} also heard that the Buddha had become fully extinguished at \textsanskrit{Kusinārā}. He sent an envoy to the Mallas of \textsanskrit{Kusinārā}: “The Buddha was an aristocrat, and I am a brahmin. I too deserve a share of the Buddha’s relics. I will build a monument for them and conduct a memorial service.” 

The\marginnote{6.24.25} Mallas of \textsanskrit{Pāvā} also heard that the Buddha had become fully extinguished at \textsanskrit{Kusinārā}. They sent an envoy to the Mallas of \textsanskrit{Kusinārā}: “The Buddha was an aristocrat, and so are we. We too deserve a share of the Buddha’s relics. We will build a monument for them and conduct a memorial service.” 

When\marginnote{6.25.1} they had spoken, the Mallas of \textsanskrit{Kusinārā} said to those various groups: “The Buddha became fully extinguished in our village district. We will not give away a share of his relics.” 

Then\marginnote{6.25.3} \textsanskrit{Doṇa} the brahmin said to those various groups: 

\begin{verse}%
“Hear,\marginnote{6.25.4} sirs, a single word from me. \\
Our Buddha’s teaching was acceptance. \\
It would not be good to fight over \\
a share of the supreme person’s relics. 

Let\marginnote{6.25.8} us make eight portions, good sirs, \\
rejoicing in unity and harmony. \\
Let there be monuments far and wide, \\
so many folk may gain faith in the Seer!” 

%
\end{verse}

“Well\marginnote{6.25.12} then, brahmin, you yourself should fairly divide the Buddha’s relics in eight portions.” 

“Yes,\marginnote{6.25.13} sirs,” replied \textsanskrit{Doṇa} to those various groups. He divided the relics as asked and said to them, “Sirs, please give me the urn, and I shall build a monument for it and conduct a memorial service.” So they gave \textsanskrit{Doṇa} the urn. 

The\marginnote{6.26.1} Moras of Pippalivana heard that the Buddha had become fully extinguished at \textsanskrit{Kusinārā}. They sent an envoy to the Mallas of \textsanskrit{Kusinārā}: “The Buddha was an aristocrat, and so are we. We too deserve a share of the Buddha’s relics. We will build a monument for them and conduct a memorial service.” 

“There\marginnote{6.26.5} is no portion of the Buddha’s relics left, they have already been portioned out. Here, take the embers.” So they took the embers. 

\section*{40. Venerating the Relics }

Then\marginnote{6.27.1} King \textsanskrit{Ajātasattu} of Magadha, the Licchavis of \textsanskrit{Vesālī}, the Sakyans of Kapilavatthu, the Bulas of Allakappa, the \textsanskrit{Koḷiyans} of \textsanskrit{Rāmagāma}, the brahmin of \textsanskrit{Veṭhadīpa}, the Mallas of \textsanskrit{Pāvā}, the Mallas of \textsanskrit{Kusinārā}, the brahmin \textsanskrit{Doṇa}, and the Moriyas of Pippalivana built monuments for them and conducted memorial services. Thus there were eight monuments for the relics, a ninth for the urn, and a tenth for the embers. That is how it was in those days. 

\begin{verse}%
There\marginnote{6.28.1} were eight shares of the Seer’s relics. \\
Seven were worshipped throughout India. \\
But one share of the most excellent of men \\
was worshipped in \textsanskrit{Rāmagāma} by a dragon king. 

One\marginnote{6.28.5} tooth is venerated by the gods of the Three and Thirty, \\
and one is worshipped in the city of \textsanskrit{Gandhāra}; \\
another one in the realm of the \textsanskrit{Kaliṅga} King, \\
and one is worshipped by a dragon king. 

Through\marginnote{6.28.9} their glory this rich earth \\
is adorned with the best of offerings. \\
Thus the Seer’s corpse \\
is well honored by the honorable. 

It’s\marginnote{6.28.13} venerated by lords of gods, dragons, and spirits; \\
and likewise venerated by the finest lords of men. \\
Honor it with joined palms when you get the chance, \\
for a Buddha is rare even in a hundred eons. 

Altogether\marginnote{6.28.17} forty even teeth, \\
and the body hair and head hair, \\
were carried off individually by gods \\
across the universe. 

%
\end{verse}

%
\chapter*{{\suttatitleacronym DN 17}{\suttatitletranslation King Mahāsudassana }{\suttatitleroot Mahāsudassanasutta}}
\addcontentsline{toc}{chapter}{\tocacronym{DN 17} \toctranslation{King Mahāsudassana } \tocroot{Mahāsudassanasutta}}
\markboth{King Mahāsudassana }{Mahāsudassanasutta}
\extramarks{DN 17}{DN 17}

\scevam{So\marginnote{1.1.1} I have heard. }At one time the Buddha was staying between a pair of sal trees in the sal forest of the Mallas at Upavattana near \textsanskrit{Kusinārā} at the time of his final extinguishment. 

Then\marginnote{1.2.1} Venerable Ānanda went up to the Buddha, bowed, sat down to one side, and said to him, “Sir, please don’t become fully extinguished in this little hamlet, this jungle hamlet, this branch hamlet. There are other great cities such as \textsanskrit{Campā}, \textsanskrit{Rājagaha}, \textsanskrit{Sāvatthī}, \textsanskrit{Sāketa}, \textsanskrit{Kosambī}, and Benares. Let the Buddha become fully extinguished there. There are many well-to-do aristocrats, brahmins, and householders there who are devoted to the Buddha. They will perform the rites of venerating the Realized One’s corpse.” 

“Don’t\marginnote{1.3.1} say that, Ānanda! Don’t say that this is a little hamlet, a jungle hamlet, a branch hamlet. 

\section*{1. The Capital City of \textsanskrit{Kusāvatī} }

Once\marginnote{1.3.4} upon a time there was a king named \textsanskrit{Mahāsudassana} whose dominion extended to all four sides, and who achieved stability in the country. His capital was this \textsanskrit{Kusinārā}, which at the time was named \textsanskrit{Kusāvatī}. It stretched for twelve leagues from east to west, and seven leagues from north to south. The royal capital of \textsanskrit{Kusāvatī} was successful, prosperous, populous, full of people, with plenty of food. It was just like \textsanskrit{Āḷakamandā}, the royal capital of the gods, which is successful, prosperous, populous, full of spirits, with plenty of food. 

\textsanskrit{Kusāvatī}\marginnote{1.3.10} was never free of ten sounds by day or night, namely: the sound of elephants, horses, chariots, drums, clay drums, arched harps, singing, horns, gongs, and handbells; and the cry, ‘Eat, drink, be merry!’ as the tenth. 

\textsanskrit{Kusāvatī}\marginnote{1.4.1} was encircled by seven ramparts: one made of gold, one made of silver, one made of beryl, one made of crystal, one made of ruby, one made of emerald, and one made of all precious things. 

It\marginnote{1.5.1} had four gates, made of gold, silver, beryl, and crystal. At each gate there were seven pillars, three fathoms deep and four fathoms high, made of gold, silver, beryl, crystal, ruby, emerald, and all precious things. 

It\marginnote{1.6.1} was surrounded by seven rows of palm trees, made of gold, silver, beryl, crystal, ruby, emerald, and all precious things. The golden palms had trunks of gold, and leaves and fruits of silver. The silver palms had trunks of silver, and leaves and fruits of gold. The beryl palms had trunks of beryl, and leaves and fruits of crystal. The crystal palms had trunks of crystal, and leaves and fruits of beryl. The ruby palms had trunks of ruby, and leaves and fruits of emerald. The emerald palms had trunks of emerald, and leaves and fruits of ruby. The palms of all precious things had trunks of all precious things, and leaves and fruits of all precious things. When those rows of palm trees were blown by the wind they sounded graceful, tantalizing, sensuous, lovely, and intoxicating, like a quintet made up of skilled musicians who had practiced well and kept excellent rhythm. And any addicts, carousers, or drunkards in \textsanskrit{Kusāvatī} at that time were entertained by that sound. 

\section*{2. The Seven Treasures }

\subsection*{2.1. The Wheel-Treasure }

King\marginnote{1.7.1} \textsanskrit{Mahāsudassana} possessed seven treasures and four blessings. What seven? 

On\marginnote{1.7.3} a fifteenth day sabbath, King \textsanskrit{Mahāsudassana} had bathed his head and gone upstairs in the royal longhouse to observe the sabbath. And the heavenly wheel-treasure appeared to him, with a thousand spokes, with rim and hub, complete in every detail. Seeing this, the king thought, ‘I have heard that when the heavenly wheel-treasure appears to a king in this way, he becomes a wheel-turning monarch. Am I then a wheel-turning monarch?’ 

Then\marginnote{1.8.1} King \textsanskrit{Mahāsudassana}, rising from his seat and arranging his robe over one shoulder, took a ceremonial vase in his left hand and besprinkled the wheel-treasure with his right hand, saying: ‘Roll forth, O wheel-treasure! Triumph, O wheel-treasure!’ 

Then\marginnote{1.8.3} the wheel-treasure rolled towards the east. And the king followed it together with his army of four divisions. In whatever place the wheel-treasure stood still, there the king came to stay together with his army. 

And\marginnote{1.9.1} any opposing rulers of the eastern quarter came to him and said, ‘Come, great king! Welcome, great king! We are yours, great king, instruct us.’ 

The\marginnote{1.9.3} king said, ‘Do not kill living creatures. Do not steal. Do not commit sexual misconduct. Do not lie. Do not drink alcohol. Maintain the current level of taxation.’ And so the opposing rulers of the eastern quarter became his vassals. 

Then\marginnote{1.10.1} the wheel-treasure, having plunged into the eastern ocean and emerged again, rolled towards the south. … 

Having\marginnote{1.10.2} plunged into the southern ocean and emerged again, it rolled towards the west. … 

Having\marginnote{1.10.3} plunged into the western ocean and emerged again, it rolled towards the north, followed by the king together with his army of four divisions. In whatever place the wheel-treasure stood still, there the king came to stay together with his army. 

And\marginnote{1.10.5} any opposing rulers of the northern quarter came to him and said, ‘Come, great king! Welcome, great king! We are yours, great king, instruct us.’ 

The\marginnote{1.10.7} king said, ‘Do not kill living creatures. Do not steal. Do not commit sexual misconduct. Do not lie. Do not drink alcohol. Maintain the current level of taxation.’ And so the opposing rulers of the northern quarter became his vassals. 

And\marginnote{1.11.1} then the wheel-treasure, having triumphed over this land surrounded by ocean, returned to the royal capital of \textsanskrit{Kusāvatī}. There it stood still by the gate to \textsanskrit{Mahāsudassana}’s royal compound at the High Court as if fixed to an axle, illuminating the royal compound. Such is the wheel-treasure that appeared to King \textsanskrit{Mahāsudassana}. 

\subsection*{2.2. The Elephant-Treasure }

Next,\marginnote{1.12.1} the elephant-treasure appeared to King \textsanskrit{Mahāsudassana}. It was an all-white sky-walker with psychic power, touching the ground in seven places, a king of elephants named Sabbath. Seeing him, the king was impressed, ‘This would truly be a fine elephant vehicle, if he would submit to taming.’ Then the elephant-treasure submitted to taming, as if he was a fine thoroughbred elephant that had been tamed for a long time. 

Once\marginnote{1.12.6} it so happened that King \textsanskrit{Mahāsudassana}, testing that same elephant-treasure, mounted him in the morning and traversed the land surrounded by ocean before returning to the royal capital in time for breakfast. Such is the elephant-treasure that appeared to King \textsanskrit{Mahāsudassana}. 

\subsection*{2.3. The Horse-Treasure }

Next,\marginnote{1.13.1} the horse-treasure appeared to King \textsanskrit{Mahāsudassana}. It was an all-white sky-walker with psychic power, with head of black and mane like woven reeds, a royal steed named Thundercloud. Seeing him, the king was impressed, ‘This would truly be a fine horse vehicle, if he would submit to taming.’ Then the horse-treasure submitted to taming, as if he was a fine thoroughbred horse that had been tamed for a long time. 

Once\marginnote{1.13.6} it so happened that King \textsanskrit{Mahāsudassana}, testing that same horse-treasure, mounted him in the morning and traversed the land surrounded by ocean before returning to the royal capital in time for breakfast. Such is the horse-treasure that appeared to King \textsanskrit{Mahāsudassana}. 

\subsection*{2.4. The Jewel-Treasure }

Next,\marginnote{1.14.1} the jewel-treasure appeared to King \textsanskrit{Mahāsudassana}. It was a beryl gem that was naturally beautiful, eight-faceted, well-worked, transparent, clear, and unclouded, endowed with all good qualities. And the radiance of that jewel spread all-round for a league. 

Once\marginnote{1.14.4} it so happened that King \textsanskrit{Mahāsudassana}, testing that same jewel-treasure, mobilized his army of four divisions and, with the jewel hoisted on his banner, set out in the dark of the night. Then the villagers around them set off to work, thinking that it was day. Such is the jewel-treasure that appeared to King \textsanskrit{Mahāsudassana}. 

\subsection*{2.5. The Woman-Treasure }

Next,\marginnote{1.15.1} the woman-treasure appeared to King \textsanskrit{Mahāsudassana}. She was attractive, good-looking, lovely, of surpassing beauty. She was neither too tall nor too short; neither too thin nor too fat; neither too dark nor too light. She outdid human beauty without reaching divine beauty. And her touch was like a tuft of cotton-wool or kapok. When it was cool her limbs were warm, and when it was warm her limbs were cool. The fragrance of sandal floated from her body, and lotus from her mouth. She got up before the king and went to bed after him, and was obliging, behaving nicely and speaking politely. The woman-treasure did not betray the wheel-turning monarch even in thought, still less in deed. Such is the woman-treasure that appeared to King \textsanskrit{Mahāsudassana}. 

\subsection*{2.6. The Householder-Treasure }

Next,\marginnote{1.16.1} the householder-treasure appeared to King \textsanskrit{Mahāsudassana}. The power of clairvoyance manifested in him as a result of past deeds, by which he sees hidden treasure, both owned and ownerless. 

He\marginnote{1.16.3} approached the king and said, ‘Relax, sire. I will take care of the treasury.’ 

Once\marginnote{1.16.5} it so happened that the wheel-turning monarch, testing that same householder-treasure, boarded a boat and sailed to the middle of the Ganges river. Then he said to the householder-treasure, ‘Householder, I need gold coins and bullion.’ 

‘Well\marginnote{1.16.7} then, great king, draw the boat up to one shore.’ 

‘It’s\marginnote{1.16.8} right here, householder, that I need gold coins and bullion.’ 

Then\marginnote{1.16.9} that householder-treasure, immersing both hands in the water, pulled up a pot full of gold coin and bullion, and said to the king, ‘Is this sufficient, great king? Has enough been done, great king, enough offered?’ 

The\marginnote{1.16.11} king said, ‘That is sufficient, householder. Enough has been done, enough offered.’ 

Such\marginnote{1.16.13} is the householder-treasure that appeared to King \textsanskrit{Mahāsudassana}. 

\subsection*{2.7. The Counselor-Treasure }

Next,\marginnote{1.18.1} the counselor-treasure appeared to King \textsanskrit{Mahāsudassana}. He was astute, competent, intelligent, and capable of getting the king to appoint who should be appointed, dismiss who should be dismissed, and retain who should be retained. 

He\marginnote{1.18.3} approached the king and said, ‘Relax, sire. I shall issue instructions.’ 

Such\marginnote{1.18.5} is the counselor-treasure that appeared to King \textsanskrit{Mahāsudassana}. 

These\marginnote{1.18.6} are the seven treasures possessed by King \textsanskrit{Mahāsudassana}. 

\section*{3. The Four Blessings }

King\marginnote{1.18.8} \textsanskrit{Mahāsudassana} possessed four blessings. And what are the four blessings? 

He\marginnote{1.18.10} was attractive, good-looking, lovely, of surpassing beauty, more so than other people. This is the first blessing. 

Furthermore,\marginnote{1.19.1} he was long-lived, more so than other people. This is the second blessing. 

Furthermore,\marginnote{1.20.1} he was rarely ill or unwell, and his stomach digested well, being neither too hot nor too cold, more so than other people. This is the third blessing. 

Furthermore,\marginnote{1.21.1} he was as dear and beloved to the brahmins and householders as a father is to his children. And the brahmins and householders were as dear to the king as children are to their father. 

Once\marginnote{1.21.7} it so happened that King \textsanskrit{Mahāsudassana} went with his army of four divisions to visit a park. Then the brahmins and householders went up to him and said, ‘Slow down, Your Majesty, so we may see you longer!’ And the king addressed his charioteer, ‘Drive slowly, charioteer, so I can see the brahmins and householders longer!’ This is the fourth blessing. 

These\marginnote{1.21.13} are the four blessings possessed by King \textsanskrit{Mahāsudassana}. 

\section*{4. Lotus Ponds in the Palace of Principle }

Then\marginnote{1.22.1} King \textsanskrit{Mahāsudassana} thought, ‘Why don’t I have lotus ponds built between the palms, at intervals of a hundred bow lengths?’ 

So\marginnote{1.22.3} that’s what he did. The lotus ponds were lined with tiles of four colors, made of gold, silver, beryl, and crystal. 

And\marginnote{1.22.6} four flights of stairs of four colors descended into each lotus pond, made of gold, silver, beryl, and crystal. The golden stairs had posts of gold, and banisters and finials of silver. The silver stairs had posts of silver, and banisters and finials of gold. The beryl stairs had posts of beryl, and banisters and finials of crystal. The crystal stairs had posts of crystal, and banisters and finials of beryl. Those lotus ponds were surrounded by two balustrades, made of gold and silver. The golden balustrades had posts of gold, and banisters and finials of silver. The silver balustrades had posts of silver, and banisters and finials of gold. 

Then\marginnote{1.23.1} King \textsanskrit{Mahāsudassana} thought, ‘Why don’t I plant flowers in the lotus ponds such as blue water lilies, and lotuses of pink, yellow, and white, blooming all year round, and accessible to the public?’ So that’s what he did. 

Then\marginnote{1.23.4} King \textsanskrit{Mahāsudassana} thought, ‘Why don’t I appoint bath attendants to help bathe the people who come to bathe in the lotus ponds?’ So that’s what he did. 

Then\marginnote{1.23.7} King \textsanskrit{Mahāsudassana} thought, ‘Why don’t I set up charities on the banks of the lotus ponds, so that those in need of food, drink, clothes, vehicles, beds, women, gold, or silver can get what they need?’ So that’s what he did. 

Then\marginnote{1.24.1} the brahmins and householders came to the king bringing abundant wealth and said, ‘Sire, this abundant wealth is specially for you alone; may Your Highness accept it!’ 

‘There’s\marginnote{1.24.3} enough raised for me through regular taxes. Let this be for you; and here, take even more!’ 

When\marginnote{1.24.4} the king turned them down, they withdrew to one side to think up a plan, ‘It wouldn’t be proper for us to take this abundant wealth back to our own homes. Why don’t we build a home for King \textsanskrit{Mahāsudassana}?’ 

They\marginnote{1.24.7} went up to the king and said, ‘We shall have a home built for you, sire!’ King \textsanskrit{Mahāsudassana} consented in silence. 

And\marginnote{1.25.1} then Sakka, lord of gods, knowing what the king was thinking, addressed the god Vissakamma, ‘Come, dear Vissakamma, build a palace named Principle as a home for King \textsanskrit{Mahāsudassana}.’ 

‘Yes,\marginnote{1.25.3} lord,’ replied Vissakamma. Then, as easily as a strong person would extend or contract their arm, he vanished from the gods of the Thirty-Three and appeared in front of King \textsanskrit{Mahāsudassana}. 

Vissakamma\marginnote{1.25.4} said to the king, ‘I shall build a palace named Principle as a home for you, sire.’ King \textsanskrit{Mahāsudassana} consented in silence. 

And\marginnote{1.25.7} so that’s what Vissakamma did. 

The\marginnote{1.25.8} Palace of Principle stretched for a league from east to west, and half a league from north to south. It was lined with tiles of four colors, three fathoms high, made of gold, silver, beryl, and crystal. 

It\marginnote{1.26.1} had 84,000 pillars of four colors, made of gold, silver, beryl, and crystal. It was covered with panels of four colors, made of gold, silver, beryl, and crystal. 

It\marginnote{1.26.5} had twenty-four staircases of four colors, made of gold, silver, beryl, and crystal. The golden stairs had posts of gold, and banisters and finials of silver. The silver stairs had posts of silver, and banisters and finials of gold. The beryl stairs had posts of beryl, and banisters and finials of crystal. The crystal stairs had posts of crystal, and banisters and finials of beryl. 

It\marginnote{1.26.11} had 84,000 chambers of four colors, made of gold, silver, beryl, and crystal. In each chamber a couch was spread: in the golden chamber a couch of silver; in the silver chamber a couch of gold; in the beryl chamber a couch of ivory; in the crystal chamber a couch of hardwood. At the door of the golden chamber stood a palm tree of silver, with trunk of silver, and leaves and fruits of gold. At the door of the silver chamber stood a palm tree of gold, with trunk of gold, and leaves and fruits of silver. At the door of the beryl chamber stood a palm tree of crystal, with trunk of crystal, and leaves and fruits of beryl. At the door of the crystal chamber stood a palm tree of beryl, with trunk of beryl, and leaves and fruits of crystal. 

Then\marginnote{1.27.1} King \textsanskrit{Mahāsudassana} thought, ‘Why don’t I build a grove of golden palm trees at the door to the great foyer, where I can sit for the day?’ So that’s what he did. 

The\marginnote{1.28.1} Palace of Principle was surrounded by two balustrades, made of gold and silver. The golden balustrades had posts of gold, and banisters and finials of silver. The silver balustrades had posts of silver, and banisters and finials of gold. 

The\marginnote{1.29.1} Palace of Principle was surrounded by two nets of bells, made of gold and silver. The golden net had bells of silver, and the silver net had bells of gold. When those nets of bells were blown by the wind they sounded graceful, tantalizing, sensuous, lovely, and intoxicating, like a quintet made up of skilled musicians who had practiced well and kept excellent rhythm. And any addicts, carousers, or drunkards in \textsanskrit{Kusāvatī} at that time were entertained by that sound. When it was finished, the palace was hard to look at, dazzling to the eyes, like the sun rising in a clear and cloudless sky in the last month of the rainy season. 

Then\marginnote{1.30.1} King \textsanskrit{Mahāsudassana} thought, ‘Why don’t I build a lotus pond named Principle in front of the palace?’ So that’s what he did. The Lotus Pond of Principle stretched for a league from east to west, and half a league from north to south. It was lined with tiles of four colors, made of gold, silver, beryl, and crystal. 

It\marginnote{1.31.1} had twenty-four staircases of four colors, made of gold, silver, beryl, and crystal. The golden stairs had posts of gold, and banisters and finials of silver. The silver stairs had posts of silver, and banisters and finials of gold. The beryl stairs had posts of beryl, and banisters and finials of crystal. The crystal stairs had posts of crystal, and banisters and finials of beryl. 

It\marginnote{1.31.7} was surrounded by two balustrades, made of gold and silver. The golden balustrades had posts of gold, and banisters and finials of silver. The silver balustrades had posts of silver, and banisters and finials of gold. 

It\marginnote{1.32.1} was surrounded by seven rows of palm trees, made of gold, silver, beryl, crystal, ruby, emerald, and all precious things. The golden palms had trunks of gold, and leaves and fruits of silver. The silver palms had trunks of silver, and leaves and fruits of gold. The beryl palms had trunks of beryl, and leaves and fruits of crystal. The crystal palms had trunks of crystal, and leaves and fruits of beryl. The ruby palms had trunks of ruby, and leaves and fruits of emerald. The emerald palms had trunks of emerald, and leaves and fruits of ruby. The palms of all precious things had trunks of all precious things, and leaves and fruits of all precious things. When those rows of palm trees were blown by the wind they sounded graceful, tantalizing, sensuous, lovely, and intoxicating, like a quintet made up of skilled musicians who had practiced well and kept excellent rhythm. And any addicts, carousers, or drunkards in \textsanskrit{Kusāvatī} at that time were entertained by that sound. 

When\marginnote{1.33.1} the palace and its lotus pond were finished, King \textsanskrit{Mahāsudassana} served those who were reckoned as true ascetics and brahmins with all they desired. Then he ascended the Palace of Principle. 

\section*{5. Attaining Absorption }

Then\marginnote{2.1.1} King \textsanskrit{Mahāsudassana} thought, ‘Of what deed of mine is this the fruit and result, that I am now so mighty and powerful?’ 

Then\marginnote{2.1.3} King \textsanskrit{Mahāsudassana} thought, ‘It is the fruit and result of three kinds of deeds: giving, self-control, and restraint.’ 

Then\marginnote{2.2.1} he went to the great foyer, stood at the door, and expressed this heartfelt sentiment: ‘Stop here, sensual, malicious, and cruel thoughts—no further!’ 

Then\marginnote{2.3.1} he entered the great foyer and sat on the golden couch. Quite secluded from sensual pleasures, secluded from unskillful qualities, he entered and remained in the first absorption, which has the rapture and bliss born of seclusion, while placing the mind and keeping it connected. As the placing of the mind and keeping it connected were stilled, he entered and remained in the second absorption, which has the rapture and bliss born of immersion, with internal clarity and confidence, and unified mind, without placing the mind and keeping it connected. And with the fading away of rapture, he entered and remained in the third absorption, where he meditated with equanimity, mindful and aware, personally experiencing the bliss of which the noble ones declare, ‘Equanimous and mindful, one meditates in bliss.’ With the giving up of pleasure and pain, and the ending of former happiness and sadness, he entered and remained in the fourth absorption, without pleasure or pain, with pure equanimity and mindfulness. 

Then\marginnote{2.4.1} King \textsanskrit{Mahāsudassana} left the great foyer and entered the golden chamber, where he sat on the golden couch. He meditated spreading a heart full of love to one direction, and to the second, and to the third, and to the fourth. In the same way he spread a heart full of love above, below, across, everywhere, all around, to everyone in the world—abundant, expansive, limitless, free of enmity and ill will. He meditated spreading a heart full of compassion … He meditated spreading a heart full of rejoicing … He meditated spreading a heart full of equanimity to one direction, and to the second, and to the third, and to the fourth. In the same way above, below, across, everywhere, all around, he spread a heart full of equanimity to the whole world—abundant, expansive, limitless, free of enmity and ill will. 

\section*{6. Of All Cities }

King\marginnote{2.5.1} \textsanskrit{Mahāsudassana} had 84,000 cities, with the royal capital of \textsanskrit{Kusāvatī} foremost. He had 84,000 palaces, with the Palace of Principle foremost. He had 84,000 chambers, with the great foyer foremost. He had 84,000 couches made of gold, silver, ivory, and hardwood. They were spread with woollen covers—shag-piled, pure white, or embroidered with flowers—and spread with a fine deer hide, with a canopy above and red pillows at both ends. He had 84,000 bull elephants with gold adornments and banners, covered with gold netting, with the royal bull elephant named Sabbath foremost. He had 84,000 horses with gold adornments and banners, covered with gold netting, with the royal steed named Thundercloud foremost. He had 84,000 chariots upholstered with the hide of lions, tigers, and leopards, and cream rugs, with gold adornments and banners, covered with gold netting, with the chariot named Triumph foremost. He had 84,000 jewels, with the jewel-treasure foremost. He had 84,000 women, with Queen \textsanskrit{Subhaddā} foremost. He had 84,000 householders, with the householder-treasure foremost. He had 84,000 aristocrat vassals, with the counselor-treasure foremost. He had 84,000 milk-cows with silken reins and bronze pails. He had 8,400,000,000 fine cloths of linen, cotton, silk, and wool. He had 84,000 servings of food, which were presented to him as offerings in the morning and evening. 

Now\marginnote{2.6.1} at that time his 84,000 royal elephants came to attend on him in the morning and evening. Then King \textsanskrit{Mahāsudassana} thought, ‘What if instead half of the elephants took turns to attend on me at the end of each century?’ He instructed the counselor-treasure to do this, and so it was done. 

\section*{7. The Visit of Queen \textsanskrit{Subhaddā} }

Then,\marginnote{2.7.1} after many years, many hundred years, many thousand years had passed, Queen \textsanskrit{Subhaddā} thought, ‘It is long since I have seen the king. Why don’t I go to see him?’ 

So\marginnote{2.7.3} the queen addressed the ladies of the harem, ‘Come, bathe your heads and dress in yellow. It is long since we saw the king, and we shall go to see him.’ 

‘Yes,\marginnote{2.7.6} ma’am,’ replied the ladies of the harem. They did as she asked and returned to the queen. 

Then\marginnote{2.8.1} the queen addressed the counselor-treasure, ‘Dear counselor-treasure, please ready the army of four divisions. It is long since we saw the king, and we shall go to see him.’ 

‘Yes,\marginnote{2.8.3} my queen,’ he replied, and did as he was asked. He informed the queen, ‘My queen, the army of four divisions is ready, please go at your convenience.’ 

Then\marginnote{2.8.6} Queen \textsanskrit{Subhaddā} together with the ladies of the harem went with the army to the Palace of Principle. She ascended the palace and went to the great foyer, where she stood leaning against a door-post. 

Hearing\marginnote{2.8.8} them, the king thought, ‘What’s that, it sounds like a big crowd!’ Coming out of the foyer he saw Queen \textsanskrit{Subhaddā} leaning against a door-post and said to her, ‘Please stay there, my queen, don’t enter in here.’ 

Then\marginnote{2.9.1} he addressed a certain man, ‘Come, mister, bring the golden couch from the great foyer and set it up in the golden palm grove.’ 

‘Yes,\marginnote{2.9.3} Your Majesty,’ that man replied, and did as he was asked. The king laid down in the lion’s posture—on the right side, placing one foot on top of the other—mindful and aware. 

Then\marginnote{2.10.1} Queen \textsanskrit{Subhaddā} thought, ‘The king’s faculties are so very clear, and the complexion of his skin is pure and bright. Let him not pass away!’ She said to him, ‘Sire, you have 84,000 cities, with the royal capital of \textsanskrit{Kusāvatī} foremost. Arouse desire for these! Take an interest in life!’ 

And\marginnote{2.10.5} she likewise urged the king to live on by taking an interest in all his possessions as described above. 

When\marginnote{2.11.1} the queen had spoken, the king said to her, ‘For a long time, my queen, you have spoken to me with loving, desirable, pleasant, and agreeable words. And yet in my final hour, your words are undesirable, unpleasant, and disagreeable!’ 

‘Then\marginnote{2.11.4} how exactly, Your Majesty, am I to speak to you?’ 

‘Like\marginnote{2.11.5} this, my queen: “Sire, we must be parted and separated from all we hold dear and beloved. Don’t pass away with concerns. Such concern is suffering, and it’s criticized. Sire, you have 84,000 cities, with the royal capital of \textsanskrit{Kusāvatī} foremost. Give up desire for these! Take no interest in life!”’ And so on for all the king’s possessions. 

When\marginnote{2.12.1} the king had spoken, Queen \textsanskrit{Subhaddā} cried and burst out in tears. Wiping away her tears, the queen said to the king: ‘Sire, we must be parted and separated from all we hold dear and beloved. Don’t pass away with concerns. Such concern is suffering, and it’s criticized. Sire, you have 84,000 cities, with the royal capital of \textsanskrit{Kusāvatī} foremost. Give up desire for these! Take no interest in life!’ And she continued, listing all the king’s possessions. 

\section*{8. Rebirth in the \textsanskrit{Brahmā} Realm }

Not\marginnote{2.13.1} long after that, King \textsanskrit{Mahāsudassana} passed away. And the feeling he had close to death was like a householder or their child falling asleep after eating a delectable meal. 

When\marginnote{2.13.3} he passed away King \textsanskrit{Mahāsudassana} was reborn in a good place, a \textsanskrit{Brahmā} realm. Ānanda, King \textsanskrit{Mahāsudassana} played children’s games for 84,000 years. He ruled as viceroy for 84,000 years. He ruled as king for 84,000 years. He led the spiritual life as a layman in the Palace of Principle for 84,000 years. And having developed the four \textsanskrit{Brahmā} meditations, when his body broke up, after death, he was reborn in a good place, a \textsanskrit{Brahmā} realm. 

Now,\marginnote{2.14.1} Ānanda, you might think: ‘Surely King \textsanskrit{Mahāsudassana} must have been someone else at that time?’ But you should not see it like that. I myself was King \textsanskrit{Mahāsudassana} at that time. 

Mine\marginnote{2.14.4} were the 84,000 cities, with the royal capital of \textsanskrit{Kusāvatī} foremost. And mine were all the other possessions. 

Of\marginnote{2.15.1} those 84,000 cities, I only stayed in one, the capital \textsanskrit{Kusāvatī}. Of those 84,000 mansions, I only dwelt in one, the Palace of Principle. Of those 84,000 chambers, I only dwelt in the great foyer. Of those 84,000 couches, I only used one, made of gold or silver or ivory or heartwood. Of those 84,000 bull elephants, I only rode one, the royal bull elephant named Sabbath. Of those 84,000 horses, I only rode one, the royal horse named Thundercloud. Of those 84,000 chariots, I only rode one, the chariot named Triumph. Of those 84,000 women, I was only served by one, a maiden of the aristocratic or merchant classes. Of those 8,400,000,000 cloths, I only wore one pair, made of fine linen, cotton, silk, or wool. Of those 84,000 servings of food, I only had one, eating at most a serving of rice and suitable sauce. 

See,\marginnote{2.16.1} Ānanda! All those conditioned phenomena have passed, ceased, and perished. So impermanent are conditions, so unstable are conditions, so unreliable are conditions. This is quite enough for you to become disillusioned, dispassionate, and freed regarding all conditions. 

Six\marginnote{2.17.1} times, Ānanda, I recall having laid down my body at this place. And the seventh time was as a wheel-turning monarch, a just and principled king, at which time my dominion extended to all four sides, I achieved stability in the country, and I possessed the seven treasures. But Ānanda, I do not see any place in this world with its gods, \textsanskrit{Māras}, and \textsanskrit{Brahmās}, this population with its ascetics and brahmins, its gods and humans where the Realized One would lay down his body for the eighth time.” 

That\marginnote{2.17.3} is what the Buddha said. Then the Holy One, the Teacher, went on to say: 

\begin{verse}%
“Oh!\marginnote{2.17.5} Conditions are impermanent, \\
their nature is to rise and fall; \\
having arisen, they cease; \\
their stilling is true bliss.” 

%
\end{verse}

%
\chapter*{{\suttatitleacronym DN 18}{\suttatitletranslation With Janavasabha }{\suttatitleroot Janavasabhasutta}}
\addcontentsline{toc}{chapter}{\tocacronym{DN 18} \toctranslation{With Janavasabha } \tocroot{Janavasabhasutta}}
\markboth{With Janavasabha }{Janavasabhasutta}
\extramarks{DN 18}{DN 18}

\section*{1. Declaring the Rebirths of People From \textsanskrit{Nādika} and Elsewhere }

\scevam{So\marginnote{1.1} I have heard. }At one time the Buddha was staying at \textsanskrit{Nādika} in the brick house. 

Now\marginnote{1.3} at that time the Buddha was explaining the rebirths of devotees all over the nations; the \textsanskrit{Kāsis} and Kosalans, Vajjis and Mallas, \textsanskrit{Cetīs} and \textsanskrit{Vaṁsas}, Kurus and \textsanskrit{Pañcālas}, Macchas and \textsanskrit{Sūrasenas}: 

“This\marginnote{1.4} one was reborn here, while that one was reborn there. 

Over\marginnote{1.5} fifty devotees in \textsanskrit{Nādika} have passed away having ended the five lower fetters. They’ve been reborn spontaneously, and will be extinguished there, not liable to return from that world. 

More\marginnote{1.6} than ninety devotees in \textsanskrit{Nādika} have passed away having ended three fetters, and weakened greed, hate, and delusion. They’re once-returners, who will come back to this world once only, then make an end of suffering. 

In\marginnote{1.7} excess of five hundred devotees in \textsanskrit{Nādika} have passed away having ended three fetters. They’re stream-enterers, not liable to be reborn in the underworld, bound for awakening.” 

When\marginnote{2.1} the devotees of \textsanskrit{Nādika} heard about the Buddha’s answers to those questions, they became uplifted and overjoyed, full of rapture and happiness. 

Venerable\marginnote{3.1} Ānanda heard of the Buddha’s statements and the \textsanskrit{Nādikans}’ happiness. 

\section*{2. Ānanda’s Suggestion }

Then\marginnote{4.1} Venerable Ānanda thought, “But there were also Magadhan devotees—many, and of long standing too—who have passed away. You’d think that \textsanskrit{Aṅga} and Magadha were empty of devotees who have passed away! But they too had confidence in the Buddha, the teaching, and the \textsanskrit{Saṅgha}, and had fulfilled their ethics. The Buddha hasn’t declared their passing. It would be good to do so, for many people would gain confidence, and so be reborn in a good place. 

That\marginnote{4.7} King Seniya \textsanskrit{Bimbisāra} of Magadha was a just and principled king who benefited the brahmins and householders of town and country. People still sing his praises: ‘That just and principled king, who made us so happy, has passed away. Life was good under his dominion.’ He too had confidence in the Buddha, the teaching, and the \textsanskrit{Saṅgha}, and had fulfilled his ethics. People say: ‘Until his dying day, King \textsanskrit{Bimbisāra} sang the Buddha’s praises!’ The Buddha hasn’t declared his passing. It would be good to do so, for many people would gain confidence, and so be reborn in a good place. 

Besides,\marginnote{4.15} the Buddha was awakened in Magadha; so why hasn’t he declared the rebirth of the Magadhan devotees? If he fails to do so, they will be dejected.” 

After\marginnote{5.1} pondering the fate of the Magadhan devotees alone in private, Ānanda rose at the crack of dawn and went to see the Buddha. He bowed, sat down to one side, and told the Buddha of his concerns, finishing by saying, “Why hasn’t the Buddha declared the rebirth of the Magadhan devotees? If he fails to do so, they will be dejected.” Then Ānanda, after making this suggestion regarding the Magadhan devotees, got up from his seat, bowed, and respectfully circled the Buddha, keeping him on his right, before leaving. 

Soon\marginnote{7.1} after Ānanda had left, the Buddha robed up in the morning and, taking his bowl and robe, entered \textsanskrit{Nādika} for alms. He wandered for alms in \textsanskrit{Nādika}. After the meal, on his return from almsround, he washed his feet and entered the brick house. He paid heed, paid attention, and concentrated wholeheartedly on the fate of Magadhan devotees, and sat on the seat spread out, thinking, “I shall know their destiny, where they are reborn in the next life.” And he saw where they had been reborn. 

Then\marginnote{7.6} in the late afternoon, the Buddha came out of retreat. Emerging from the brick house, he sat on the seat spread out in the shade of the porch. 

Then\marginnote{8.1} Venerable Ānanda went up to the Buddha, bowed, sat down to one side, and said to him, “Sir, you look so serene; your face seems to shine owing to the clarity of your faculties. Have you been abiding in a peaceful meditation today, sir?” 

The\marginnote{9.1} Buddha then recounted what had happened since speaking to Ānanda, revealing that he had seen the destiny of the Magadhan devotees. He continued: 

\section*{3. Janavasabha the Spirit }

“Then,\marginnote{9.5} Ānanda an invisible spirit called out: ‘I am Janavasabha, Blessed One! I am Janavasabha, Holy One!’ Ānanda, do you recall having previously heard such a name as Janavasabha?” 

“No,\marginnote{9.9} sir. But when I heard the word, I got goosebumps! I thought, ‘This must be no ordinary spirit to bear such an exalted name as Janavasabha.’” 

“After\marginnote{10.1} making himself heard while invisible, Ānanda, a very beautiful spirit appeared in front of me. And for a second time he called out: ‘I am \textsanskrit{Bimbisāra}, Blessed One! I am \textsanskrit{Bimbisāra}, Holy One! This is the seventh time I have been reborn in the company of the Great King \textsanskrit{Vessavaṇa}. After passing away from there, I am now able to become a king of non-humans. 

\begin{verse}%
Seven\marginnote{10.6} from here, seven from there—\\
fourteen transmigrations in all. \\
That’s how many past lives \\
I can recollect. 

%
\end{verse}

For\marginnote{10.10} a long time I’ve known that I won’t be reborn in the underworld, but that I still hope to become a once-returner.’ 

‘It’s\marginnote{10.11} incredible and amazing that you, the venerable spirit Janavasabha, should say: 

“For\marginnote{10.12} a long time I’ve been aware that I won’t be reborn in the underworld” and also “But I still hope to become a once-returner.” But from what source do you know that you’ve achieved such a high distinction?’ 

‘None\marginnote{11.1} other than the Blessed One’s instruction! None other than the Holy One’s instruction! From the day I had absolute devotion to the Buddha I have known that I won’t be reborn in the underworld, but that I still hope to become a once-returner. Just now, sir, I had been sent out by the great king \textsanskrit{Vessavaṇa} to the great king \textsanskrit{Virūḷhaka}’s presence on some business, and on the way I saw the Buddha giving his attention to the fate of the Magadhan devotees. But it comes as no surprise that I have heard and learned the fate of the Magadhan devotees in the presence of the great king \textsanskrit{Vessavaṇa} as he was speaking to his assembly. It occurred to me, “I shall see the Buddha and inform him of this.” These are the two reasons I’ve come to see the Buddha. 

\section*{4. The Council of the Gods }

Sir,\marginnote{12.1} it was more than a few days ago—on the fifteenth day sabbath on the full moon day at the entry to the rainy season—when all the gods of the Thirty-Three were sitting together in the Hall of Justice. A large assembly of gods was sitting all around, and the Four Great Kings were seated at the four quarters. 

The\marginnote{12.3} Great King \textsanskrit{Dhataraṭṭha} was seated to the east, facing west, in front of his gods. The Great King \textsanskrit{Virūḷhaka} was seated to the south, facing north, in front of his gods. The Great King \textsanskrit{Virūpakkha} was seated to the west, facing east, in front of his gods. The Great King \textsanskrit{Vessavaṇa} was seated to the north, facing south, in front of his gods. When the gods of the Thirty-Three have a gathering like this, that is how they are seated. After that come our seats. 

Sir,\marginnote{12.9} those gods who had been recently reborn in the company of the Thirty-Three after leading the spiritual life under the Buddha outshone the other gods in beauty and glory. The gods of the Thirty-Three became uplifted and overjoyed at that, full of rapture and happiness, saying, “The heavenly hosts swell, while the demon hosts dwindle!” 

Seeing\marginnote{13.1} the joy of those gods, Sakka, lord of gods, celebrated with these verses: 

\begin{verse}%
“The\marginnote{13.2} gods rejoice—\\
the Thirty-Three with their Lord—\\
revering the Realized One, \\
and the natural excellence of the teaching; 

and\marginnote{13.6} seeing the new gods, \\
so beautiful and glorious, \\
who have come here after leading \\
the spiritual life under the Buddha! 

They\marginnote{13.10} outshine the others \\
in beauty, glory, and lifespan. \\
Here are the distinguished disciples \\
of he whose wisdom is vast. 

Seeing\marginnote{13.14} this, they delight—\\
the Thirty-Three with their Lord—\\
revering the Realized One, \\
and the natural excellence of the teaching.” 

%
\end{verse}

The\marginnote{13.18} gods of the Thirty-Three became even more uplifted and overjoyed at that, saying: “The heavenly hosts swell, while the demon hosts dwindle!” 

Then\marginnote{14.1} the gods of the Thirty-Three, having considered and deliberated on the matter for which they were seated together in the Hall of Justice, advised and instructed the Four Great Kings on the subject. And each stood at his own seat without departing. 

\begin{verse}%
The\marginnote{14.3} Kings were instructed, \\
and heeded good advice. \\
With clear and peaceful minds, \\
they stood by their own seats. 

%
\end{verse}

Then\marginnote{15.1} in the northern quarter a magnificent light arose and radiance appeared, surpassing the glory of the gods. Then Sakka, lord of gods, addressed the gods of the Thirty-Three, “As indicated by the signs—light arising and radiance appearing—\textsanskrit{Brahmā} will appear. For this is the precursor for the appearance of \textsanskrit{Brahmā}, namely light arising and radiance appearing.” 

\begin{verse}%
As\marginnote{15.4} indicated by the signs, \\
\textsanskrit{Brahmā} will appear. \\
For this is the sign of \textsanskrit{Brahmā}: \\
a light vast and great. 

%
\end{verse}

\section*{5. On \textsanskrit{Sanaṅkumāra} }

Then\marginnote{16.1} the gods of the Thirty-Three sat in their own seats, saying, “We shall find out what has caused that light, and only when we have realized it shall we go to it.” And the Four Great Kings did likewise. 

Hearing\marginnote{16.5} that, the gods of the Thirty-Three agreed in unison, “We shall find out what has caused that light, and only when we have realized it shall we go to it.” 

When\marginnote{17.1} \textsanskrit{Brahmā} \textsanskrit{Sanaṅkumāra} appears to the gods of the Thirty-Three, he does so after manifesting in a solid corporeal form. For the gods of the Thirty-Three aren’t able to see a \textsanskrit{Brahmā}’s normal appearance. When \textsanskrit{Brahmā} \textsanskrit{Sanaṅkumāra} appears to the gods of the Thirty-Three, he outshines the other gods in beauty and glory, as a golden statue outshines the human form. 

When\marginnote{17.6} \textsanskrit{Brahmā} \textsanskrit{Sanaṅkumāra} appears to the gods of the Thirty-Three, not a single god in that assembly greets him by bowing down or rising up or inviting him to a seat. They all sit silently on their couches with their joined palms raised, thinking, “Now \textsanskrit{Brahmā} \textsanskrit{Sanaṅkumāra} will sit on the couch of whatever god he chooses.” And the god on whose couch \textsanskrit{Brahmā} sits is overjoyed and brimming with happiness, like a king on the day of his coronation. 

Then\marginnote{18.3} \textsanskrit{Brahmā} \textsanskrit{Sanaṅkumāra} manifested in a solid corporeal form, taking on the appearance of the youth \textsanskrit{Pañcasikha}, and appeared to the gods of the Thirty-Three. Rising into the air, he sat cross-legged in the sky, like a strong man might sit cross-legged on a well-appointed couch or on level ground. Seeing the joy of those gods, \textsanskrit{Brahmā} \textsanskrit{Sanaṅkumāra} celebrated with these verses: 

\begin{verse}%
“The\marginnote{18.7} gods rejoice—\\
the Thirty-Three with their Lord—\\
revering the Realized One, \\
and the natural excellence of the teaching; 

and\marginnote{18.11} seeing the new gods, \\
so beautiful and glorious, \\
who have come here after leading \\
the spiritual life under the Buddha! 

They\marginnote{18.15} outshine the others \\
in beauty, glory, and lifespan. \\
Here are the distinguished disciples \\
of he whose wisdom is vast. 

Seeing\marginnote{18.19} this, they delight—\\
the Thirty-Three with their Lord—\\
revering the Realized One, \\
and the natural excellence of the teaching!” 

%
\end{verse}

That\marginnote{19.1} is the topic on which \textsanskrit{Brahmā} \textsanskrit{Sanaṅkumāra} spoke. And while he was speaking on that topic, his voice had eight qualities: it was clear, comprehensible, charming, audible, lucid, undistorted, deep, and resonant. He makes sure his voice is intelligible as far as the assembly goes, but it doesn’t extend outside the assembly. When someone has a voice like this, they’re said to have the voice of \textsanskrit{Brahmā}. 

Then\marginnote{20.1} \textsanskrit{Brahmā} \textsanskrit{Sanaṅkumāra}, having manifested thirty-three corporeal forms, sat down on the couches of each of the gods of the Thirty-Three and addressed them, “What do the good gods of the Thirty-Three think about how much the Buddha has acted for the welfare and happiness of the people, out of compassion for the world, for the benefit, welfare, and happiness of gods and humans? For consider those who have gone for refuge to the Buddha, the teaching, and the \textsanskrit{Saṅgha}, and have fulfilled their ethics. When their bodies break up, after death, some are reborn in the company of the Gods Who Control the Creations of Others, some with the Gods Who Love to Create, some with the Joyful Gods, some with the Gods of Yama, some with the Gods of the Thirty-Three, and some with the Gods of the Four Great Kings. And at the very least they swell the hosts of the fairies.” 

That\marginnote{21.1} is the topic on which \textsanskrit{Brahmā} \textsanskrit{Sanaṅkumāra} spoke. And while he was speaking on that topic, each of the gods fancied, “The one sitting on my couch is the only one speaking.” 

\begin{verse}%
When\marginnote{21.4} one is speaking, \\
all the forms speak. \\
When one sits in silence, \\
they all remain silent. 

But\marginnote{21.8} those gods imagine—\\
the Thirty-Three with their Lord—\\
that the one on their seat \\
is the only one to speak. 

%
\end{verse}

The\marginnote{22.1} \textsanskrit{Brahmā} \textsanskrit{Sanaṅkumāra} merged into one corporeal form. Then he sat on the couch of Sakka, lord of gods, and addressed the gods of the Thirty-Three: 

\section*{6. Developing the Bases of Psychic Power }

“What\marginnote{22.3} do the good gods of the Thirty-Three think about how much the four bases of psychic power have been clearly described by the Blessed One—the one who knows and sees, the perfected one, the fully awakened Buddha—for the multiplication, generation, and transformation of corporeal forms through psychic power? What four? It’s when a mendicant develops the basis of psychic power that has immersion due to enthusiasm, and active effort. They develop the basis of psychic power that has immersion due to energy, and active effort. They develop the basis of psychic power that has immersion due to mental development, and active effort. They develop the basis of psychic power that has immersion due to inquiry, and active effort. These are the four bases of psychic power that have been clearly described by the Buddha, for the multiplication, generation, and transformation of corporeal forms through psychic power. 

All\marginnote{22.10} the ascetics and brahmins in the past, future, or present who wield the many kinds of psychic power do so by developing and cultivating these four bases of psychic power. Gentlemen, do you see such psychic might and power in me?” 

“Yes,\marginnote{22.14} Great \textsanskrit{Brahmā}.” 

“I\marginnote{22.15} too became so mighty and powerful by developing and cultivating these four bases of psychic power.” 

That\marginnote{23.1} is the topic on which \textsanskrit{Brahmā} \textsanskrit{Sanaṅkumāra} spoke. And having spoken about that, he addressed the gods of the Thirty-Three: 

\section*{7. The Three Openings }

“What\marginnote{23.4} do the good gods of the Thirty-Three think about how much the Buddha has understood the three opportunities for achieving happiness? What three? 

First,\marginnote{23.6} take someone who lives mixed up with sensual pleasures and unskillful qualities. After some time they hear the teaching of the noble ones, properly attend to how it applies to them, and practice accordingly. They live aloof from sensual pleasures and unskillful qualities. That gives rise to pleasure, and more than pleasure, happiness, like the joy that’s born from gladness. This is the first opportunity for achieving happiness. 

Next,\marginnote{24.1} take someone whose coarse physical, verbal, and mental processes have not died down. After some time they hear the teaching of the noble ones, properly attend to how it applies to them, and practice accordingly. Their coarse physical, verbal, and mental processes die down. That gives rise to pleasure, and more than pleasure, happiness, like the joy that’s born from gladness. This is the second opportunity for achieving happiness. 

Next,\marginnote{25.1} take someone who doesn’t truly understand what is skillful and what is unskillful, what is blameworthy and what is blameless, what should be cultivated and what should not be cultivated, what is inferior and what is superior, and what is on the side of dark and the side of bright. After some time they hear the teaching of the noble ones, properly attend to how it applies to them, and practice accordingly. They truly understand what is skillful and what is unskillful, and so on. Knowing and seeing like this, ignorance is given up and knowledge arises. That gives rise to pleasure, and more than pleasure, happiness, like the joy that’s born from gladness. This is the third opportunity for achieving happiness. 

These\marginnote{25.11} are the three opportunities for achieving happiness that have been understood by the Buddha.” 

That\marginnote{26.1} is the topic on which \textsanskrit{Brahmā} \textsanskrit{Sanaṅkumāra} spoke. And having spoken about that, he addressed the gods of the Thirty-Three: 

\section*{8. Mindfulness Meditation }

“What\marginnote{26.4} do the good gods of the Thirty-Three think about how much the Buddha has clearly described the four kinds of mindfulness meditation for achieving what is skillful? What four? 

It’s\marginnote{26.6} when a mendicant meditates by observing an aspect of the body internally—keen, aware, and mindful, rid of desire and aversion for the world. As they meditate in this way, they become rightly immersed in that, and rightly serene. Then they give rise to knowledge and vision of other people’s bodies externally. 

They\marginnote{26.9} meditate observing an aspect of feelings internally … Then they give rise to knowledge and vision of other people’s feelings externally. 

They\marginnote{26.11} meditate observing an aspect of the mind internally … Then they give rise to knowledge and vision of other people’s minds externally. 

They\marginnote{26.13} meditate observing an aspect of principles internally—keen, aware, and mindful, rid of desire and aversion for the world. As they meditate in this way, they become rightly immersed in that, and rightly serene. Then they give rise to knowledge and vision of other people’s principles externally. 

These\marginnote{26.16} are the four kinds of mindfulness meditation that the Buddha has clearly described for achieving what is skillful.” 

That\marginnote{27.1} is the topic on which \textsanskrit{Brahmā} \textsanskrit{Sanaṅkumāra} spoke. And having spoken about that, he addressed the gods of the Thirty-Three: 

\section*{9. Seven Prerequisites of Immersion }

“What\marginnote{27.4} do the good gods of the Thirty-Three think about how much the Buddha has clearly described the seven prerequisites of immersion for the development and fulfillment of right immersion? What seven? Right view, right thought, right speech, right action, right livelihood, right effort, and right mindfulness. Unification of mind with these seven factors as prerequisites is called noble right immersion ‘with its vital conditions’ and ‘with its prerequisites’. 

Right\marginnote{27.8} view gives rise to right thought. Right thought gives rise to right speech. Right speech gives rise to right action. Right action gives rise to right livelihood. Right livelihood gives rise to right effort. Right effort gives rise to right mindfulness. Right mindfulness gives rise to right immersion. Right immersion gives rise to right knowledge. Right knowledge gives rise to right freedom. 

If\marginnote{27.9} anything should be rightly described as ‘a teaching that’s well explained by the Buddha, visible in this very life, immediately effective, inviting inspection, relevant, so that sensible people can know it for themselves; and the doors to the deathless are flung open,’ it’s this. For the teaching is well explained by the Buddha—visible in this very life, immediately effective, inviting inspection, relevant, so that sensible people can know it for themselves—and the doors of the deathless are flung open. 

Whoever\marginnote{27.12} has experiential confidence in the Buddha, the teaching, and the \textsanskrit{Saṅgha}, and has the ethical conduct loved by the noble ones; and whoever is spontaneously reborn, and is trained in the teaching; in excess of 2,400,000 such Magadhan devotees have passed away having ended three fetters. They’re stream-enterers, not liable to be reborn in the underworld, bound for awakening. And there are once-returners here, too. 

\begin{verse}%
And\marginnote{27.14} as for other people \\
who I think have shared in merit—\\
I couldn’t even number them, \\
for fear of speaking falsely.” 

%
\end{verse}

That\marginnote{28.1} is the topic on which \textsanskrit{Brahmā} \textsanskrit{Sanaṅkumāra} spoke. And while he was speaking on that topic, this thought came to the great king \textsanskrit{Vessavaṇa}, “It’s incredible, it’s amazing! That there should be such a magnificent Teacher, and such a magnificent exposition of the teaching! And that such achievements of high distinction should be made known!” 

And\marginnote{28.3} then \textsanskrit{Brahmā} \textsanskrit{Sanaṅkumāra}, knowing what the great king \textsanskrit{Vessavaṇa} was thinking, said to him, “What does Great King \textsanskrit{Vessavaṇa} think? In the past, too, there was such a magnificent Teacher, and such a magnificent exposition of the teaching! And such achievements of high distinction were made known! In the future, too, there will be such a magnificent Teacher, and such a magnificent exposition of the teaching! And such achievements of high distinction will be made known!” 

That,\marginnote{29.1} sir, is the topic on which \textsanskrit{Brahmā} \textsanskrit{Sanaṅkumāra} spoke to the gods of the Thirty-Three. And the great king \textsanskrit{Vessavaṇa}, having heard and learned it in the presence of \textsanskrit{Brahmā} as he was speaking on that topic, informed his own assembly.’” 

And\marginnote{29.2} the spirit Janavasabha, having heard and learned it in the presence of the great king \textsanskrit{Vessavaṇa} as he was speaking on that topic to his own assembly, informed the Buddha. And the Buddha, having heard and learned it in the presence of the spirit Janavasabha, and also from his own direct knowledge, informed Venerable Ānanda. And Venerable Ānanda, having heard and learned it in the presence of the Buddha, informed the monks, nuns, laymen, and laywomen. And that’s how this spiritual life has become successful and prosperous, extensive, popular, widespread, and well proclaimed wherever there are gods and humans. 

%
\chapter*{{\suttatitleacronym DN 19}{\suttatitletranslation The Great Steward }{\suttatitleroot Mahāgovindasutta}}
\addcontentsline{toc}{chapter}{\tocacronym{DN 19} \toctranslation{The Great Steward } \tocroot{Mahāgovindasutta}}
\markboth{The Great Steward }{Mahāgovindasutta}
\extramarks{DN 19}{DN 19}

\scevam{So\marginnote{1.1} I have heard. }At one time the Buddha was staying near \textsanskrit{Rājagaha}, on the Vulture’s Peak Mountain. 

Then,\marginnote{1.3} late at night, the fairy \textsanskrit{Pañcasikha}, lighting up the entire Vulture’s Peak, went up to the Buddha, bowed, stood to one side, and said to him, “Sir, I would tell you of what I heard and learned directly from the gods of the Thirty-Three.” 

“Tell\marginnote{1.5} me, \textsanskrit{Pañcasikha},” said the Buddha. 

\section*{1. The Council of the Gods }

“Sir,\marginnote{2.1} it was more than a few days ago—on the fifteenth day sabbath on the full moon day at the invitation to admonish held at the end of the rainy season—when all the gods of the Thirty-Three were sitting together in the Hall of Justice. A large assembly of gods was sitting all around, and the Four Great Kings were seated at the four quarters. 

The\marginnote{2.3} Great King \textsanskrit{Dhataraṭṭha} was seated to the east, facing west, in front of his gods. The Great King \textsanskrit{Virūḷhaka} was seated to the south, facing north, in front of his gods. The Great King \textsanskrit{Virūpakkha} was seated to the west, facing east, in front of his gods. The Great King \textsanskrit{Vessavaṇa} was seated to the north, facing south, in front of his gods. 

When\marginnote{2.7} the gods of the Thirty-Three have a gathering like this, that is how they are seated. After that come our seats. 

Sir,\marginnote{3.1} those gods who had been recently reborn in the company of the Thirty-Three after leading the spiritual life under the Buddha outshine the other gods in beauty and glory. The gods of the Thirty-Three became uplifted and overjoyed at that, full of rapture and happiness, saying, ‘The heavenly hosts swell, while the demon hosts dwindle!’ 

Seeing\marginnote{3.4} the joy of those gods, Sakka, lord of gods, celebrated with these verses: 

\begin{verse}%
‘The\marginnote{3.5} gods rejoice—\\
the Thirty-Three with their Lord—\\
revering the Realized One, \\
and the natural excellence of the teaching; 

and\marginnote{3.9} seeing the new gods, \\
so beautiful and glorious, \\
who have come here after leading \\
the spiritual life under the Buddha! 

They\marginnote{3.13} outshine the others \\
in beauty, glory, and lifespan. \\
Here are the distinguished disciples \\
of he whose wisdom is vast. 

Seeing\marginnote{3.17} this, they delight—\\
the Thirty-Three with their Lord—\\
revering the Realized One, \\
and the natural excellence of the teaching!’ 

%
\end{verse}

The\marginnote{3.21} gods of the Thirty-Three became even more uplifted and overjoyed at that, full of rapture and happiness, saying, ‘The heavenly hosts swell, while the demon hosts dwindle!’ 

\section*{2. Eight Genuine Praises }

Seeing\marginnote{4.1} the joy of those gods, Sakka, lord of gods, addressed them, ‘Gentlemen, would you like to hear eight genuine praises of the Buddha?’ 

‘Indeed\marginnote{4.3} we would, sir.’ 

Then\marginnote{4.4} Sakka proffered these eight genuine praises of the Buddha: 

‘What\marginnote{5.1} do the good gods of the Thirty-Three think about how much the Buddha has acted for the welfare and happiness of the people, out of compassion for the world, for the benefit, welfare, and happiness of gods and humans? I don’t see any Teacher, past or present, who has such compassion for the world, apart from the Buddha. 

Also,\marginnote{6.1} the Buddha has explained the teaching well—visible in this very life, immediately effective, inviting inspection, relevant, so that sensible people can know it for themselves. I don’t see any Teacher, past or present, who explains such a relevant teaching, apart from the Buddha. 

Also,\marginnote{7.1} the Buddha has clearly described what is skillful and what is unskillful, what is blameworthy and what is blameless, what should be cultivated and what should not be cultivated, what is inferior and what is superior, and what is on the side of dark and the side of bright. I don’t see any Teacher, past or present, who so clearly describes all these things, apart from the Buddha. 

Also,\marginnote{8.1} the Buddha has clearly described the practice that leads to extinguishment for his disciples. And extinguishment and the practice come together, as the waters of the Ganges come together and converge with the waters of the Yamuna. I don’t see any Teacher, past or present, who so clearly describes the practice that leads to extinguishment for his disciples, apart from the Buddha. 

Also,\marginnote{9.1} possessions and popularity have accrued to the Buddha, so much that you’d think it would thrill even the aristocrats. But he takes his food free of vanity. I don’t see any Teacher, past or present, who takes their food so free of vanity, apart from the Buddha. 

Also,\marginnote{10.1} the Buddha has gained companions, both trainees who are practicing, and those with defilements ended who have completed their journey. The Buddha is committed to the joy of solitude, but doesn’t send them away. I don’t see any Teacher, past or present, so committed to the joy of solitude, apart from the Buddha. 

Also,\marginnote{11.1} the Buddha does as he says, and says as he does, thus: he does as he says, and says as he does. I don’t see any Teacher, past or present, who so practices in line with the teaching, apart from the Buddha. 

Also,\marginnote{12.1} the Buddha has gone beyond doubt and got rid of indecision. He has achieved all he wished for regarding the fundamental purpose of the spiritual life. I don’t see any Teacher, past or present, who has achieved these things, apart from the Buddha.’ 

These\marginnote{12.3} are the eight genuine praises of the Buddha proffered by Sakka. Hearing them, the gods of the Thirty-Three became even more uplifted and overjoyed. 

Then\marginnote{13.1} some gods thought, ‘If only four fully awakened Buddhas might arise in the world and teach the Dhamma, just like the Blessed One! That would be for the welfare and happiness of the people, out of compassion for the world, for the benefit, welfare, and happiness of gods and humans!’ 

Other\marginnote{13.4} gods thought, ‘Let alone four fully awakened Buddhas; if only three fully awakened Buddhas, or two fully awakened Buddhas might arise in the world and teach the Dhamma, just like the Blessed One! That would be for the welfare and happiness of the people, out of compassion for the world, for the benefit, welfare, and happiness of gods and humans!’ 

When\marginnote{14.1} they said this, Sakka said, ‘It’s impossible, gentlemen, for two perfected ones, fully awakened Buddhas to arise in the same solar system at the same time. May that Blessed One be healthy and well, and remain with us for a long time! That would be for the welfare and happiness of the people, out of compassion for the world, for the benefit, welfare, and happiness of gods and humans!’ 

Then\marginnote{14.5} the gods of the Thirty-Three, having considered and deliberated on the matter for which they were seated together in the Hall of Justice, advised and instructed the Four Great Kings on the subject. And each stood at their own seat without departing. 

\begin{verse}%
The\marginnote{14.7} Kings were instructed, \\
and heeded good advice. \\
With clear and peaceful minds, \\
they stood by their own seats. 

%
\end{verse}

Then\marginnote{15.1} in the northern quarter a magnificent light arose and radiance appeared, surpassing the glory of the gods. Then Sakka, lord of gods, addressed the gods of the Thirty-Three, ‘As indicated by the signs—light arising and radiance appearing—\textsanskrit{Brahmā} will appear. For this is the precursor for the appearance of \textsanskrit{Brahmā}, namely light arising and radiance appearing.’ 

\begin{verse}%
As\marginnote{15.5} indicated by the signs, \\
\textsanskrit{Brahmā} will appear. \\
For this is the sign of \textsanskrit{Brahmā}: \\
a light vast and great. 

%
\end{verse}

\section*{3. On \textsanskrit{Sanaṅkumāra} }

Then\marginnote{16.1} the gods of the Thirty-Three sat in their own seats, saying, ‘We shall find out what has caused that light, and only when we have realized it shall we go to it.’ And the Four Great Kings did likewise. Hearing that, the gods of the Thirty-Three agreed in unison, ‘We shall find out what has caused that light, and only when we have realized it shall we go to it.’ 

When\marginnote{16.7} \textsanskrit{Brahmā} \textsanskrit{Sanaṅkumāra} appears to the gods of the Thirty-Three, he does so after manifesting in a solid corporeal form, for the gods of the Thirty-Three aren’t able to see a \textsanskrit{Brahmā}’s normal appearance. When \textsanskrit{Brahmā} \textsanskrit{Sanaṅkumāra} appears to the gods of the Thirty-Three, he outshines the other gods in beauty and glory, as a golden statue outshines the human form. When \textsanskrit{Brahmā} \textsanskrit{Sanaṅkumāra} appears to the gods of the Thirty-Three, not a single god in that assembly greets him by bowing down or rising up or inviting him to a seat. They all sit silently on their couches with their joined palms raised, thinking, ‘Now \textsanskrit{Brahmā} \textsanskrit{Sanaṅkumāra} will sit on the couch of whatever god he chooses.’ And the god on whose couch \textsanskrit{Brahmā} sits is overjoyed and brimming with happiness, like a king on the day of his coronation. 

Seeing\marginnote{17.3} the joy of those gods, \textsanskrit{Brahmā} \textsanskrit{Sanaṅkumāra} celebrated with these verses: 

\begin{verse}%
‘The\marginnote{17.4} gods rejoice—\\
the Thirty-Three with their Lord—\\
revering the Realized One, \\
and the natural excellence of the teaching; 

and\marginnote{17.8} seeing the new gods, \\
so beautiful and glorious, \\
who have come here after leading \\
the spiritual life under the Buddha! 

They\marginnote{17.12} outshine the others \\
in beauty, glory, and lifespan. \\
Here are the distinguished disciples \\
of he whose wisdom is vast. 

Seeing\marginnote{17.16} this, they delight—\\
the Thirty-Three with their Lord—\\
revering the Realized One, \\
and the natural excellence of the teaching!’ 

%
\end{verse}

That\marginnote{18.1} is the topic on which \textsanskrit{Brahmā} \textsanskrit{Sanaṅkumāra} spoke. And while he was speaking on that topic, his voice had eight qualities: it was clear, comprehensible, charming, audible, lucid, undistorted, deep, and resonant. He makes sure his voice is intelligible as far as the assembly goes, but it doesn’t extend outside the assembly. When someone has a voice like this, they’re said to have the voice of \textsanskrit{Brahmā}. 

Then\marginnote{19.1} the gods of the Thirty-Three said to \textsanskrit{Brahmā} \textsanskrit{Sanaṅkumāra}, ‘Good, Great \textsanskrit{Brahmā}! Having assessed this, we rejoice. And there are the eight genuine praises of the Buddha spoken by Sakka—having assessed them, too, we rejoice.’ 

\section*{4. Eight Genuine Praises }

Then\marginnote{19.6} \textsanskrit{Brahmā} said to Sakka, ‘It would be good, lord of gods, if I could also hear the eight genuine praises of the Buddha.’ 

Saying,\marginnote{19.8} ‘Yes, Great \textsanskrit{Brahmā},’ Sakka repeated the eight genuine praises for him. 

Hearing\marginnote{28.1} them, \textsanskrit{Brahmā} \textsanskrit{Sanaṅkumāra} was uplifted and overjoyed, full of rapture and happiness. Then \textsanskrit{Brahmā} \textsanskrit{Sanaṅkumāra} manifested in a solid corporeal form, taking on the appearance of the youth \textsanskrit{Pañcasikha}, and appeared to the gods of the Thirty-Three. Rising into the air, he sat cross-legged in the sky, like a strong man might sit cross-legged on a well-appointed couch or on level ground. There he addressed the gods of the Thirty-Three: 

\section*{5. The Story of the Steward }

‘What\marginnote{28.8} do the gods of the Thirty-Three think about the extent of the Buddha’s great wisdom? 

Once\marginnote{29.1} upon a time, there was a king named Disampati. He had a brahmin high priest named the Steward. Disampati’s son was the prince named \textsanskrit{Reṇu}, while the Steward’s son was the student named \textsanskrit{Jotipāla}. There were \textsanskrit{Reṇu} the prince, \textsanskrit{Jotipāla} the student, and six other aristocrats; these eight became friends. 

In\marginnote{29.6} due course the brahmin Steward passed away. At his passing, King Disampati lamented, “At a time when I have relinquished all my duties to the brahmin Steward and amuse myself, supplied and provided with the five kinds of sensual stimulation, he passes away!” 

When\marginnote{29.9} he said this, Prince \textsanskrit{Reṇu} said to him, “Sire, don’t lament too much at the Steward’s passing. He has a son named \textsanskrit{Jotipāla}, who is even more astute and expert than his father. He should manage the affairs that were managed by his father.” 

“Is\marginnote{29.13} that so, my prince?” 

“Yes,\marginnote{29.14} sire.” 

\section*{6. The Story of the Great Steward }

So\marginnote{30.1} King Disampati addressed one of his men, “Please, mister, go to the student \textsanskrit{Jotipāla}, and say to him, ‘Best wishes, \textsanskrit{Jotipāla}! You are summoned by King Disampati; he wants to see you.’” 

“Yes,\marginnote{30.4} Your Majesty,” replied that man, and did as he was asked. Then \textsanskrit{Jotipāla} went to the king and exchanged greetings with him. 

When\marginnote{30.7} the greetings and polite conversation were over, he sat down to one side, and the king said to him, “May you, \textsanskrit{Jotipāla}, manage my affairs—please don’t turn me down! I shall appoint you to your father’s position, and anoint you as Steward.” 

“Yes,\marginnote{30.10} sir,” replied \textsanskrit{Jotipāla}. 

So\marginnote{31.1} the king anointed him as Steward and appointed him to his father’s position. After his appointment, the Steward \textsanskrit{Jotipāla} managed both the affairs that his father had managed, and other affairs that his father had not managed. He organized both the works that his father had organized, and other works that his father had not organized. When people noticed this they said, “The brahmin is indeed a Steward, a Great Steward!” And that’s how the student \textsanskrit{Jotipāla} came to be known as the Great Steward. 

\subsection*{6.1. Dividing the Realm }

Then\marginnote{32.1} the Great Steward went to the six aristocrats and said, “King Disampati is old, elderly and senior, advanced in years, and has reached the final stage of life. Who knows how long he has to live? It’s likely that when he passes away the king-makers will anoint Prince \textsanskrit{Reṇu} as king. Come, sirs, go to Prince \textsanskrit{Reṇu} and say, ‘Prince \textsanskrit{Reṇu}, we are your friends, dear, beloved, and cherished. We have shared your joys and sorrows. King Disampati is old, elderly and senior, advanced in years, and has reached the final stage of life. Who knows how long he has to live? It’s likely that when he passes away the king-makers will anoint you as king. If you should gain kingship, share it with us.’” 

“Yes,\marginnote{33.1} sir,” replied the six aristocrats. They went to Prince \textsanskrit{Reṇu} and put the proposal to him. 

The\marginnote{33.7} prince replied, “Who else, sirs, in my realm ought to prosper if not you? If I gain kingship, I will share it with you all.” 

In\marginnote{34.1} due course King Disampati passed away. At his passing, the king-makers anointed Prince \textsanskrit{Reṇu} as king. But after being anointed, King \textsanskrit{Reṇu} amused himself, supplied and provided with the five kinds of sensual stimulation. 

Then\marginnote{34.4} the Great Steward went to the six aristocrats and said, “King Disampati has passed away. But after being anointed, King \textsanskrit{Reṇu} amused himself, supplied and provided with the five kinds of sensual stimulation. Who knows the intoxicating power of sensual pleasures? Come, sirs, go to Prince \textsanskrit{Reṇu} and say, ‘Sir, King Disampati has passed away, and you have been anointed as king. Do you remember what you said?’” 

“Yes,\marginnote{34.10} sir,” replied the six aristocrats. They went to King \textsanskrit{Reṇu} and said, “Sir, King Disampati has passed away, and you have been anointed as king. Do you remember what you said?” 

“I\marginnote{34.12} remember, sirs. Who is able to neatly divide into seven equal parts this great land, so broad in the north and narrow as the front of a cart in the south?” 

“Who\marginnote{34.14} else, sir, if not the Great Steward?” 

So\marginnote{35.1} King \textsanskrit{Reṇu} addressed one of his men, “Please, mister, go to the brahmin Great Steward and say that King \textsanskrit{Reṇu} summons him.” 

“Yes,\marginnote{35.4} Your Majesty,” replied that man, and did as he was asked. Then the Great Steward went to the king and exchanged greetings with him. 

When\marginnote{35.7} the greetings and polite conversation were over, he sat down to one side, and the king said to him, “Come, let the good Steward neatly divide into seven equal parts this great land, so broad in the north and narrow as the front of a cart in the south.” 

“Yes,\marginnote{35.9} sir,” replied the Great Steward, and did as he was asked. All were set up like the fronts of carts, with King \textsanskrit{Reṇu}’s nation in the center. 

\begin{verse}%
Dantapura\marginnote{36.3} for the \textsanskrit{Kaliṅgas}; \\
Potana for the Assakas; \\
Mahissati for the Avantis; \\
Roruka for the \textsanskrit{Sovīras}; 

Mithila\marginnote{36.7} for the Videhas; \\
\textsanskrit{Campā} was made for the \textsanskrit{Aṅgas}; \\
and Varanasi for the \textsanskrit{Kāsīs}: \\
these were laid out by the Steward. 

%
\end{verse}

Then\marginnote{36.11} those six aristocrats were delighted with their respective gains, having achieved all they wished for, “We have received exactly what we wanted, what we wished for, what we desired, what we yearned for.” 

\begin{verse}%
\textsanskrit{Sattabhū}\marginnote{36.13} and Brahmadatta, \\
\textsanskrit{Vessabhū} and Bharata, \\
\textsanskrit{Reṇu} and the two \textsanskrit{Dhataraṭṭhas}: \\
these are the seven \textsanskrit{Bhāratas}. 

%
\end{verse}

\scendsection{The first recitation section is finished. }

\subsection*{6.2. A Good Reputation }

Then\marginnote{37.1} the six aristocrats approached the Great Steward and said, “Steward, just as you are King \textsanskrit{Reṇu}’s friend, dear, beloved, and cherished, you are also our friend. Would you manage our affairs? Please don’t turn us down!” 

“Yes,\marginnote{37.5} sirs,” replied the Great Steward. Then the Great Steward managed the realms of the seven kings. And he taught seven well-to-do brahmins, and seven hundred bathed initiates to recite the hymns. 

After\marginnote{38.1} some time he got this good reputation, “The Great Steward sees \textsanskrit{Brahmā} in person! The Great Steward discusses, converses, and consults with \textsanskrit{Brahmā} in person!” 

The\marginnote{38.3} Great Steward thought, “I have the reputation of seeing \textsanskrit{Brahmā} in person, and discussing with him in person. But I don’t. I have heard that brahmins of the past who were elderly and senior, the teachers of teachers, said: ‘Whoever goes on retreat for the four months of the rainy season and practices the absorption on compassion sees \textsanskrit{Brahmā} and discusses with him.’ Why don’t I do that?” 

So\marginnote{39.1} the Great Steward went to King \textsanskrit{Reṇu} and told him of the situation, saying, “Sir, I wish to go on retreat for the four months of the rainy season and practice the absorption on compassion. No one should approach me, except for the one who brings my meal.” 

“Please\marginnote{39.9} do so, Steward, at your convenience.” 

Then\marginnote{40.1} the Great Steward went to the six aristocrats to put the same proposal, and received the same reply. 

He\marginnote{41.1} also went to the seven well-to-do brahmins and seven hundred bathed initiates and put to them the same proposal, adding, “Sirs, recite the hymns in detail as you have learned and memorized them, and teach each other how to recite.” 

And\marginnote{41.9} they too said, “Please do so, Steward, at your convenience.” 

Then\marginnote{42.1} the Great Steward went to his forty equal wives to put the same proposal to them, and received the same reply. 

Then\marginnote{43.1} the Great Steward had a new meeting hall built to the east of his citadel, where he went on retreat for the four months of the rainy season and practiced the absorption on compassion. And no one approached him except the one who brought him meals. 

But\marginnote{43.3} then, when the four months had passed, the Great Steward became dissatisfied and anxious, “I have heard that brahmins of the past said that whoever goes on retreat for the four months of the rainy season and practices the absorption on compassion sees \textsanskrit{Brahmā} and discusses with him. But I neither see \textsanskrit{Brahmā} nor discuss with him.” 

\subsection*{6.3. A Discussion With \textsanskrit{Brahmā} }

And\marginnote{44.1} then \textsanskrit{Brahmā} \textsanskrit{Sanaṅkumāra}, knowing what the Great Steward was thinking, as easily as a strong person would extend or contract their arm, vanished from the \textsanskrit{Brahmā} realm and reappeared in the Great Steward’s presence. At that, the Great Steward became frightened, scared, his hair standing on end, as he had never seen such a sight before. So he addressed \textsanskrit{Brahmā} \textsanskrit{Sanaṅkumāra} in verse: 

\begin{verse}%
“Who\marginnote{44.4} might you be, sir, \\
so beautiful, glorious, majestic? \\
Not knowing, I ask—\\
how am I to know who you are?” 

“In\marginnote{44.8} the \textsanskrit{Brahmā} realm they know me \\
as ‘The Eternal Youth’. \\
All the gods know me thus, \\
and so you should know me, Steward.” 

“A\marginnote{44.12} \textsanskrit{Brahmā} deserves a seat and water, \\
foot-salve, and sweet cakes. \\
Sir, I ask you to please accept \\
these gifts of hospitality.” 

“I\marginnote{44.16} accept the gifts of hospitality \\
of which you speak. \\
I grant you the opportunity \\
to ask whatever you desire—\\
about welfare and benefit in this life, \\
or happiness in lives to come.” 

%
\end{verse}

Then\marginnote{45.1} the Great Steward thought, “\textsanskrit{Brahmā} \textsanskrit{Sanaṅkumāra} has granted me an opportunity. Should I ask him about what is beneficial for this life or lives to come?” 

Then\marginnote{45.4} he thought, “I’m a skilled in what is beneficial for this life, and others even ask me about it. Why don’t I ask \textsanskrit{Brahmā} about the benefit that specifically applies to lives to come?” So he addressed \textsanskrit{Brahmā} \textsanskrit{Sanaṅkumāra} in verse: 

\begin{verse}%
“I’m\marginnote{45.8} in doubt, so I ask \textsanskrit{Brahmā}—who is free of doubt—\\
about things one may learn from another. \\
Standing on what, training in what \\
may a mortal reach the deathless \textsanskrit{Brahmā} realm?” 

“He\marginnote{45.12} among men, O brahmin, has given up possessiveness, \\
become one, compassionate, \\
free of putrefaction, and refraining from sex. \\
Standing on that, training in that \\
a mortal may reach the deathless \textsanskrit{Brahmā} realm.” 

%
\end{verse}

“Sir,\marginnote{46.1} I understand what ‘giving up possessiveness’ means. It’s when someone gives up a large or small fortune, and a large or small family circle. They shave off hair and beard, dress in ocher robes, and go forth from the lay life to homelessness. That’s how I understand ‘giving up possessiveness’. 

Sir,\marginnote{46.4} I understand what ‘oneness’ means. It’s when someone frequents a secluded lodging—a wilderness, the root of a tree, a hill, a ravine, a mountain cave, a charnel ground, a forest, the open air, a heap of straw. That’s how I understand ‘oneness’. 

Sir,\marginnote{46.7} I understand what ‘compassionate’ means. It’s when someone meditates spreading a heart full of compassion to one direction, and to the second, and to the third, and to the fourth. In the same way above, below, across, everywhere, all around, they spread a heart full of compassion to the whole world—abundant, expansive, limitless, free of enmity and ill will. That’s how I understand ‘compassionate’. 

But\marginnote{46.10} I don’t understand what you say about putrefaction. 

\begin{verse}%
What\marginnote{46.11} among men, O \textsanskrit{Brahmā}, is putrefaction? \\
I don’t understand, so tell me, wise one: \\
wrapped in what do people stink, \\
headed for hell, shut out of the \textsanskrit{Brahmā} realm?” 

“Anger,\marginnote{46.15} lies, fraud, and deceit, \\
miserliness, vanity, jealousy, \\
desire, stinginess, harassing others, \\
greed, hate, vanity, and delusion—\\
those bound to such things have putrefaction; \\
they’re headed for hell, shut out of the \textsanskrit{Brahmā} realm.” 

%
\end{verse}

“As\marginnote{46.21} I understand what you say about putrefaction, it’s not easy to quell while living at home. I shall go forth from the lay life to homelessness!” 

“Please\marginnote{46.23} do so, Steward, at your convenience.” 

\subsection*{6.4. Informing King \textsanskrit{Reṇu} }

So\marginnote{47.1} the Great Steward went to King \textsanskrit{Reṇu} and said, “Sir, please now find another high priest to manage the affairs of state for you. I wish to go forth from the lay life to homelessness. As I understand what \textsanskrit{Brahmā} says about putrefaction, it’s not easy to quell while living at home. I shall go forth from the lay life to homelessness. 

\begin{verse}%
I\marginnote{47.6} announce to King \textsanskrit{Reṇu}, \\
the lord of the land: \\
you must learn how to rule, \\
for I no longer care for my ministry.” 

“If\marginnote{47.10} you’re lacking any pleasures, \\
I’ll supply them for you. \\
I’ll protect you from any harm, \\
for I command the nation’s army. \\
You are my father, I am your son! \\
O Steward, please don’t leave!” 

“I’m\marginnote{47.16} lacking no pleasures, \\
and no-one is harming me. \\
I’ve heard a non-human voice, \\
so I no longer care for lay life.” 

“What\marginnote{47.20} was that non-human like? \\
What did he say to you, \\
hearing which you would abandon \\
our house and all our people?” 

“Before\marginnote{47.24} entering this retreat, \\
I only liked to sacrifice. \\
I kindled the sacred flame, \\
strewn about with kusa grass. 

But\marginnote{47.28} then \textsanskrit{Brahmā} the Eternal Youth \\
appeared to me from the \textsanskrit{Brahmā} realm. \\
He answered my question, \\
hearing which I no longer care for lay life.” 

“I\marginnote{47.32} have faith, O Steward, \\
in that of which you speak. \\
Having heard a non-human voice, \\
what else could you do? 

We\marginnote{47.36} will follow your example, \\
Steward, be my Teacher! \\
Like a gem of beryl—\\
flawless, immaculate, beautiful—\\
that’s how pure we shall live, \\
in the Steward’s dispensation. 

%
\end{verse}

If\marginnote{47.42} the Steward is going forth from the lay life to homelessness, we shall do so too. Your destiny shall be ours.” 

\subsection*{6.5. Informing the Six Aristocrats }

Then\marginnote{48.1} the Great Steward went to the six aristocrats and said, “Sirs, please now find another high priest to manage the affairs of state for you. I wish to go forth from the lay life to homelessness. As I understand what \textsanskrit{Brahmā} says about putrefaction, it’s not easy to quell while living at home. I shall go forth from the lay life to homelessness!” 

Then\marginnote{49.1} the six aristocrats withdrew to one side and thought up a plan, “These brahmins are really greedy for wealth. Why don’t we try to persuade him with wealth?” 

They\marginnote{49.4} returned to the Great Steward and said, “In these seven kingdoms there is abundant wealth. We’ll get you as much as you want.” 

“Enough,\marginnote{49.6} sirs. I already have abundant wealth, owing to my lords. Giving up all that, I shall go forth.” 

Then\marginnote{49.10} the six aristocrats withdrew to one side and thought up a plan, “These brahmins are really greedy for women. Why don’t we try to persuade him with women?” 

They\marginnote{49.13} returned to the Great Steward and said, “In these seven kingdoms there are many women. We’ll get you as many as you want.” 

“Enough,\marginnote{49.15} sirs. I already have forty equal wives. Giving up all them, I shall go forth.” 

“If\marginnote{50.1} the Steward is going forth from the lay life to homelessness, we shall do so too. Your destiny shall be ours.” 

\begin{verse}%
“If\marginnote{50.2} you all give up sensual pleasures, \\
to which ordinary people are attached, \\
exert yourselves, being strong, \\
and possessing the power of patience. 

This\marginnote{50.6} path is the straight path, \\
this path is supreme. \\
Guarded by the good, the true teaching \\
leads to rebirth in the \textsanskrit{Brahmā} realm.” 

%
\end{verse}

“Well\marginnote{51.1} then, sir, please wait for seven years. When seven years have passed, we shall go forth with you. Your destiny shall be ours.” 

“Seven\marginnote{51.3} years is too long, sirs. I cannot wait that long. Who knows what will happen to the living? We are heading to the next life. We must be thoughtful and wake up! We must do what’s good and lead the spiritual life, for no-one born can escape death. I shall go forth.” 

“Well\marginnote{52.1} then, sir, please wait for six years, five years, four years, three years, two years, one year, seven months, six months, five months, four months, three months, two months, one month, or even a fortnight. When a fortnight has passed, we shall go forth. Your destiny shall be ours.” 

“A\marginnote{55.1} fortnight is too long, sirs. I cannot wait that long. Who knows what will happen to the living? We are heading to the next life. We must be thoughtful and wake up! We must do what’s good and lead the spiritual life, for no-one born can escape death. As I understand what \textsanskrit{Brahmā} says about putrefaction, it’s not easy to quell while living at home. I shall go forth from the lay life to homelessness.” 

“Well\marginnote{55.6} then, sir, please wait for a week, so that we can instruct our sons and brothers in kingship. When a week has passed, we shall go forth. Your destiny shall be ours.” 

“A\marginnote{55.7} week is not too long, sirs. I will wait that long.” 

\subsection*{6.6. Informing the Brahmins }

Then\marginnote{56.1} the Great Steward also went to the seven well-to-do brahmins and seven hundred bathed initiates and said, “Sirs, please now find another teacher to teach you to recite the hymns. I wish to go forth from the lay life to homelessness. As I understand what \textsanskrit{Brahmā} says about putrefaction, it’s not easy to quell while living at home. I shall go forth from the lay life to homelessness.” 

“Please\marginnote{56.6} don’t go forth from the lay life to homelessness! The life of one gone forth is of little influence or profit, whereas the life of a brahmin is of great influence and profit.” 

“Please,\marginnote{56.9} sirs, don’t say that. Who has greater influence and profit than myself? For now I am like a king to kings, like \textsanskrit{Brahmā} to brahmins, like a deity to householders. Giving up all that, I shall go forth. As I understand what \textsanskrit{Brahmā} says about putrefaction, it’s not easy to quell while living at home. I shall go forth from the lay life to homelessness.” 

“If\marginnote{56.15} the Steward is going forth from the lay life to homelessness, we shall do so too. Your destiny shall be ours.” 

\subsection*{6.7. Informing the Wives }

Then\marginnote{57.1} the Great Steward went to his forty equal wives and said, “Ladies, please do whatever you wish, whether returning to your own families, or finding another husband. I wish to go forth from the lay life to homelessness. As I understand what \textsanskrit{Brahmā} says about putrefaction, it’s not easy to quell while living at home. I shall go forth from the lay life to homelessness.” 

“You\marginnote{57.6} are the only family we want! You are the only husband we want! If you are going forth from the lay life to homelessness, we shall do so too. Your destiny shall be ours.” 

\subsection*{6.8. The Great Steward Goes Forth }

When\marginnote{58.1} a week had passed, the Great Steward shaved off his hair and beard, dressed in ocher robes, and went forth from the lay life to homelessness. And when he had gone forth, the seven anointed aristocrat kings, the seven brahmins with seven hundred initiates, the forty equal wives, and many thousands of aristocrats, brahmins, householders, and many harem women shaved off their hair and beards, dressed in ocher robes, and went forth from the lay life to homelessness. 

Escorted\marginnote{58.3} by that assembly, the Great Steward wandered on tour among the villages, towns, and capital cities. And at that time, whenever he arrived at a village or town, he was like a king to kings, like \textsanskrit{Brahmā} to brahmins, like a deity to householders. And whenever people sneezed or tripped over they’d say: “Homage to the Great Steward! Homage to the high priest for the seven!” 

And\marginnote{59.1} the Great Steward meditated spreading a heart full of love to one direction, and to the second, and to the third, and to the fourth. In the same way above, below, across, everywhere, all around, he spread a heart full of love to the whole world—abundant, expansive, limitless, free of enmity and ill will. He meditated spreading a heart full of compassion … rejoicing … equanimity to one direction, and to the second, and to the third, and to the fourth. In the same way above, below, across, everywhere, all around, he spread a heart full of equanimity to the whole world—abundant, expansive, limitless, free of enmity and ill will. And he taught his disciples the path to rebirth in the company of \textsanskrit{Brahmā}. 

Those\marginnote{60.1} of his disciples who completely understood the Great Steward’s instructions, at the breaking up of the body, after death, were reborn in the \textsanskrit{Brahmā} realm. Of those disciples who only partly understood the Great Steward’s instructions, some were reborn in the company of the Gods Who Control the Creations of Others, while some were reborn in the company of the Gods Who Love to Create, or the Joyful Gods, or the Gods of Yama, or the Gods of the Thirty-Three, or the Gods of the Four Great Kings. And at the very least they swelled the hosts of the fairies. 

And\marginnote{60.10} so the going forth of all those gentlemen was not in vain, was not wasted, but was fruitful and fertile.’ 

Do\marginnote{60.11} you remember this, Blessed One?” 

“I\marginnote{61.1} remember, \textsanskrit{Pañcasikha}. I myself was the brahmin Great Steward at that time. And I taught those disciples the path to rebirth in the company of \textsanskrit{Brahmā}. But that spiritual path of mine doesn’t lead to disillusionment, dispassion, cessation, peace, insight, awakening, and extinguishment. It only leads as far as rebirth in the \textsanskrit{Brahmā} realm. 

But\marginnote{61.5} this spiritual path does lead to disillusionment, dispassion, cessation, peace, insight, awakening, and extinguishment. And what is the spiritual path that leads to extinguishment? It is simply this noble eightfold path, that is: right view, right thought, right speech, right action, right livelihood, right effort, right mindfulness, and right immersion. This is the spiritual path that leads to disillusionment, dispassion, cessation, peace, insight, awakening, and extinguishment. 

Those\marginnote{62.1} of my disciples who completely understand my instructions realize the undefiled freedom of heart and freedom by wisdom in this very life. And they live having realized it with their own insight due to the ending of defilements. 

Of\marginnote{62.2} those disciples who only partly understand my instructions, some, with the ending of the five lower fetters, become reborn spontaneously. They are extinguished there, and are not liable to return from that world. 

Some,\marginnote{62.3} with the ending of three fetters, and the weakening of greed, hate, and delusion, become once-returners. They come back to this world once only, then make an end of suffering. 

And\marginnote{62.4} some, with the ending of three fetters, become stream-enterers, not liable to be reborn in the underworld, bound for awakening. 

And\marginnote{62.5} so the going forth of all those gentlemen was not in vain, was not wasted, but was fruitful and fertile.” 

That\marginnote{62.6} is what the Buddha said. Delighted, the fairy \textsanskrit{Pañcasikha} approved and agreed with what the Buddha said. He bowed and respectfully circled the Buddha, keeping him on his right, before vanishing right there. 

%
\chapter*{{\suttatitleacronym DN 20}{\suttatitletranslation The Great Congregation }{\suttatitleroot Mahāsamayasutta}}
\addcontentsline{toc}{chapter}{\tocacronym{DN 20} \toctranslation{The Great Congregation } \tocroot{Mahāsamayasutta}}
\markboth{The Great Congregation }{Mahāsamayasutta}
\extramarks{DN 20}{DN 20}

\scevam{So\marginnote{1.1} I have heard. }At one time the Buddha was staying in the land of the Sakyans, near Kapilavatthu in the Great Wood, together with a large \textsanskrit{Saṅgha} of around five hundred mendicants, all of whom were perfected ones. And most of the deities from ten solar systems had gathered to see the Buddha and the \textsanskrit{Saṅgha} of mendicants. 

Then\marginnote{2.1} four deities of the Pure Abodes, aware of what was happening, thought: “Why don’t we go to the Buddha and each recite a verse in his presence?” 

Then,\marginnote{3.1} as easily as a strong person would extend or contract their arm, they vanished from the Pure Abodes and reappeared in front of the Buddha. They bowed to the Buddha and stood to one side. Standing to one side, one deity recited this verse in the Buddha’s presence: 

\begin{verse}%
“There’s\marginnote{3.4} a great congregation in the woods, \\
a host of gods have assembled. \\
We’ve come to this righteous congregation \\
to see the invincible \textsanskrit{Saṅgha}!” 

%
\end{verse}

Then\marginnote{3.8} another deity recited this verse in the Buddha’s presence: 

\begin{verse}%
“The\marginnote{3.9} mendicants there are immersed in \textsanskrit{samādhi}, \\
they’ve straightened out their own minds. \\
Like a charioteer who has taken the reins, \\
the astute ones protect their senses.” 

%
\end{verse}

Then\marginnote{3.13} another deity recited this verse in the Buddha’s presence: 

\begin{verse}%
“Having\marginnote{3.14} cut the stake and cut the bar, \\
they’re unstirred, with boundary pillar uprooted. \\
They live pure and immaculate, \\
the young dragons tamed by the seer.” 

%
\end{verse}

Then\marginnote{3.18} another deity recited this verse in the Buddha’s presence: 

\begin{verse}%
“Anyone\marginnote{3.19} who has gone to the Buddha for refuge \\
won’t go to a plane of loss. \\
After giving up this human body, \\
they swell the hosts of gods.” 

%
\end{verse}

\section*{1. The Gathering of Deities }

Then\marginnote{4.1} the Buddha said to the mendicants: 

“Mendicants,\marginnote{4.2} most of the deities from ten solar systems have gathered to see the Realized One and the mendicant \textsanskrit{Saṅgha}. The Buddhas of the past had, and the Buddhas of the future will have, gatherings of deities that are at most like the gathering for me now. I shall declare the names of the heavenly hosts; I shall extol the names of the heavenly hosts; I shall teach the names of the heavenly hosts. Listen and pay close attention, I will speak.” 

“Yes,\marginnote{4.9} sir,” they replied. 

The\marginnote{4.10} Buddha said this: 

\begin{verse}%
“I\marginnote{5.1} shall invoke a paean of praise! \\
Where the earth-gods dwell, \\
there, in mountain caves, \\
resolute and composed, 

dwell\marginnote{5.5} many like lonely lions, \\
who have mastered their fears. \\
Their minds are bright and pure, \\
clear and undisturbed.” 

The\marginnote{5.9} teacher knew that over five hundred \\
were in the wood at Kapilavatthu. \\
Therefore he addressed \\
the disciples who love the teaching: 

“The\marginnote{5.13} heavenly hosts have come forth; \\
mendicants, you should be aware of them.” \\
Those monks became keen, \\
hearing the Buddha’s instruction. 

Knowledge\marginnote{6.1} manifested in them, \\
seeing those non-human beings. \\
Some saw a hundred, \\
a thousand, even seventy thousand, 

while\marginnote{6.5} some saw a hundred thousand \\
non-human beings. \\
But some saw an endless number \\
spread out in every direction. 

And\marginnote{6.9} all that was known \\
and distinguished by the Seer. \\
Therefore he addressed \\
the disciples who love the teaching: 

“The\marginnote{6.13} heavenly hosts have come forth; \\
mendicants, you should be aware of them. \\
I shall extol them for you, \\
with lyrics in proper order. 

There\marginnote{7.1} are seven thousand spirits, \\
earth-gods of Kapilavatthu. \\
They’re powerful and brilliant, \\
so beautiful and glorious. \\
Rejoicing, they’ve come forth \\
to the meeting of mendicants in the wood. 

From\marginnote{7.7} the Himalayas there are six thousand \\
spirits of different colors. \\
They’re powerful and brilliant, \\
so beautiful and glorious. \\
Rejoicing, they’ve come forth \\
to the meeting of mendicants in the wood. 

From\marginnote{7.13} Mount \textsanskrit{Sātā} there are three thousand \\
spirits of different colors. \\
They’re powerful and brilliant, \\
so beautiful and glorious. \\
Rejoicing, they’ve come forth \\
to the meeting of mendicants in the wood. 

And\marginnote{7.19} thus there are sixteen thousand \\
spirits of different colors. \\
They’re powerful and brilliant, \\
so beautiful and glorious. \\
Rejoicing, they’ve come forth \\
to the meeting of mendicants in the wood. 

From\marginnote{8.1} \textsanskrit{Vessamittā} there are five hundred \\
spirits of different colors. \\
They’re powerful and brilliant, \\
so beautiful and glorious. \\
Rejoicing, they’ve come forth \\
to the meeting of mendicants in the wood. 

And\marginnote{8.7} there’s \textsanskrit{Kumbhīra} of \textsanskrit{Rājagaha}, \\
whose home is on Mount Vepulla. \\
Attended by more than \\
a hundred thousand spirits, \\
\textsanskrit{Kumbhīra} of \textsanskrit{Rājagaha} \\
also came to the meeting in the wood. 

King\marginnote{9.1} \textsanskrit{Dhataraṭṭha} rules \\
the eastern quarter. \\
Lord of the fairies, \\
he’s a great king, glorious. 

And\marginnote{9.5} he has many mighty sons \\
all of them named Indra. \\
They’re powerful and brilliant, \\
so beautiful and glorious. \\
Rejoicing, they’ve come forth \\
to the meeting of mendicants in the wood. 

King\marginnote{9.11} \textsanskrit{Virūḷha} rules \\
the southern quarter. \\
Lord of the goblins, \\
he’s a great king, glorious. 

And\marginnote{9.15} he has many mighty sons \\
all of them named Indra. \\
They’re powerful and brilliant, \\
so beautiful and glorious. \\
Rejoicing, they’ve come forth \\
to the meeting of mendicants in the wood. 

King\marginnote{9.21} \textsanskrit{Virūpakkha} rules \\
the western quarter. \\
Lord of the dragons, \\
he’s a great king, glorious. 

And\marginnote{9.25} he has many mighty sons \\
all of them named Indra. \\
They’re powerful and brilliant, \\
so beautiful and glorious. \\
Rejoicing, they’ve come forth \\
to the meeting of mendicants in the wood. 

King\marginnote{9.31} Kuvera rules \\
the northern quarter. \\
Lord of spirits, \\
he’s a great king, glorious. 

And\marginnote{9.35} he has many mighty sons \\
all of them named Indra. \\
They’re powerful and brilliant, \\
so beautiful and glorious. \\
Rejoicing, they’ve come forth \\
to the meeting of mendicants in the wood. 

\textsanskrit{Dhataraṭṭha}\marginnote{9.41} in the east, \\
\textsanskrit{Virūḷhaka} to the south, \\
\textsanskrit{Virūpakkha} to the west, \\
and Kuvera in the north. 

These\marginnote{9.45} Four Great Kings, \\
all around in the four quarters, \\
stood there dazzling \\
in the wood at Kapilavatthu. 

Their\marginnote{10.1} deceitful bondservants came, \\
so treacherous and crafty—\\
the deceivers \textsanskrit{Kuṭeṇḍu}, \textsanskrit{Viṭeṇḍu}, \\
with \textsanskrit{Viṭucca} and \textsanskrit{Viṭuḍa}. 

And\marginnote{10.5} Candana and \textsanskrit{Kāmaseṭṭha}, \\
\textsanskrit{Kinnughaṇḍu} and \textsanskrit{Nighaṇḍu}, \\
\textsanskrit{Panāda} and \textsanskrit{Opamañña}, \\
and \textsanskrit{Mātali}, the god’s charioteer. 

Cittasena\marginnote{10.9} the fairy came too, \\
and the kings Nala and Janesabha, \\
as well as \textsanskrit{Pañcasikha}, \\
Timbaru, and \textsanskrit{Suriyavacchasā}. 

These\marginnote{10.13} and other kings there were, \\
the fairies with their kings. \\
Rejoicing, they’ve come forth \\
to the meeting of mendicants in the wood. 

Then\marginnote{11.1} came dragons from \textsanskrit{Nābhasa}, \\
and \textsanskrit{Vesālī}, with the Tacchakas. \\
The Kambalas and Assataras came \\
from \textsanskrit{Pāyāga} with their kin. 

From\marginnote{11.5} \textsanskrit{Yamunā} the \textsanskrit{Dhataraṭṭha} \\
dragons came, so glorious. \\
And \textsanskrit{Erāvaṇa} the great dragon \\
also came to the meeting in the wood. 

Those\marginnote{11.9} who seize the dragon kings by force—\\
divine, twice-born birds with piercing vision—\\
swoop down to the wood from the sky; \\
their name is ‘Rainbow Phoenix’. 

But\marginnote{11.13} the dragon kings remained fearless, \\
for the Buddha kept them safe from the phoenixes. \\
Introducing each other with gentle words, \\
the dragons and phoenixes took the Buddha as their refuge. 

Defeated\marginnote{12.1} by Vajirahattha, \\
the demons live in the ocean. \\
They’re brothers of \textsanskrit{Vāsava}, \\
powerful and glorious. 

There’s\marginnote{12.5} the terrifying \textsanskrit{Kālakañjas}, \\
the \textsanskrit{Dānaveghasa} demons, \\
Vepacitti and Sucitti, \\
\textsanskrit{Pahārāda} with Namuci, 

and\marginnote{12.9} a hundred of Bali’s sons, \\
all named after Veroca. \\
Bali’s army armed themselves \\
and went up to the auspicious \textsanskrit{Rāhu}, saying: \\
‘Now is the time, sir, \\
for the meeting of mendicants in the wood.’ 

The\marginnote{13.1} gods of Water and Earth, \\
and Fire and Wind came there. \\
The \textsanskrit{Varuṇa} and \textsanskrit{Vāruṇa} gods, \\
and Soma together with Yasa. 

A\marginnote{13.5} host of the gods of Love \\
and Compassion came, so glorious. \\
These ten hosts of gods \\
shone in all different colors. 

They’re\marginnote{13.9} powerful and brilliant, \\
so beautiful and glorious. \\
Rejoicing, they’ve come forth \\
to the meeting of mendicants in the wood. 

The\marginnote{14.1} \textsanskrit{Veṇhu} and Sahali gods, \\
and Asama, and the twin Yamas came. \\
The gods living on the moon came, \\
with the Moon before them. 

The\marginnote{14.5} gods living on the sun came, \\
with the Sun before them. \\
And with the stars before them \\
came the silly gods of clouds. 

And\marginnote{14.9} Sakka came, the stronghold-giver, \\
known as \textsanskrit{Vāsava}, the first of the Vasus. \\
These ten hosts of gods \\
shone in all different colors. 

They’re\marginnote{14.13} powerful and brilliant, \\
so beautiful and glorious. \\
Rejoicing, they’ve come forth \\
to the meeting of mendicants in the wood. 

Then\marginnote{15.1} came the \textsanskrit{Sahabhū} gods, \\
blazing like a crested flame; \\
and the \textsanskrit{Ariṭṭhakas} and Rojas too, \\
and the gods hued blue as flax. 

The\marginnote{15.5} \textsanskrit{Varuṇas} and Sahadhammas, \\
the Accutas and Anejakas, \\
the \textsanskrit{Sūleyyas} and Ruciras all came, \\
as did the \textsanskrit{Vāsavanesi} gods. \\
These ten hosts of gods \\
shone in all different colors. 

They’re\marginnote{15.11} powerful and brilliant, \\
so beautiful and glorious. \\
Rejoicing, they’ve come forth \\
to the meeting of mendicants in the wood. 

The\marginnote{16.1} \textsanskrit{Samānas}, \textsanskrit{Mahāsamānas}, \\
\textsanskrit{Mānusas}, and \textsanskrit{Mānusuttamas} all came, \\
and the gods depraved by play, \\
and those who are malevolent. 

Then\marginnote{16.5} came the Hari gods, \\
and the \textsanskrit{Lohitavāsīs}. \\
The \textsanskrit{Pāragas} and \textsanskrit{Mahāpāragas} came, \\
such glorious gods. \\
These ten hosts of gods \\
shone in all different colors. 

They’re\marginnote{16.11} powerful and brilliant, \\
so beautiful and glorious. \\
Rejoicing, they’ve come forth \\
to the meeting of mendicants in the wood. 

The\marginnote{17.1} Sukkas, Karumhas, and \textsanskrit{Aruṇas} \\
came along with the Veghanasas. \\
And the \textsanskrit{Odātagayhas} came as chiefs \\
of the \textsanskrit{Vicakkhaṇa} gods. 

The\marginnote{17.5} \textsanskrit{Sadāmattas} and \textsanskrit{Hāragajas}, \\
and the glorious Missakas; \\
Pajjuna came thundering, \\
he who rains on all quarters. 

These\marginnote{17.9} ten hosts of gods \\
shone in all different colors. \\
They’re powerful and brilliant, \\
so beautiful and glorious. \\
Rejoicing, they’ve come forth \\
to the meeting of mendicants in the wood. 

The\marginnote{18.1} Khemiyas, Tusitas, \textsanskrit{Yāmas}, \\
and the glorious \textsanskrit{Kaṭṭhakas} came; \\
The \textsanskrit{Lambītakas}, \textsanskrit{Lāmaseṭṭhas}, \\
those called the Jotis and Āsavas. \\
The Gods Who Love to Create came too, \\
and those who delight in the Creations of Others. 

These\marginnote{18.7} ten hosts of gods \\
shone in all different colors. \\
They’re powerful and brilliant, \\
so beautiful and glorious. \\
Rejoicing, they’ve come forth \\
to the meeting of mendicants in the wood. 

These\marginnote{19.1} sixty hosts of gods \\
shone in all different colors. \\
They came in order of their names, \\
these and others likewise, thinking: 

‘We\marginnote{19.5} shall see those rid of rebirth, kind, \\
the undefiled ones who have crossed the flood, \\
and the dragon who brought them across, \\
who like the Moon has overcome darkness.’ 

\textsanskrit{Subrahmā}\marginnote{20.1} and Paramatta came, \\
with sons of those powerful ones. \\
\textsanskrit{Sanaṅkumāra} and Tissa \\
also came to the meeting in the wood. 

Of\marginnote{20.5} a thousand \textsanskrit{Brahmā} realms, \\
the Great \textsanskrit{Brahmā} stands forth. \\
He has arisen, resplendent, \\
his formidable body so glorious. 

The\marginnote{20.9} ten \textsanskrit{Issarās} came there, \\
each one of them wielding power, \\
and in the middle of them came \\
\textsanskrit{Hārita} with his following.” 

When\marginnote{21.1} they had all come forth—\\
the gods with their Lord, and the \textsanskrit{Brahmās}—\\
\textsanskrit{Māra}’s army came forth too: \\
see the stupidity of the Dark Lord! 

“Come,\marginnote{21.5} seize them and bind them,” he said, \\
“let them be bound by desire! \\
Surround them on all sides, \\
don’t let any escape!” 

And\marginnote{21.9} so there the great general \\
sent forth his dark army. \\
He struck the ground with his fist \\
to make a horrifying sound 

like\marginnote{21.13} a storm cloud shedding rain, \\
thundering and flashing. \\
But then he retreated, \\
furious, out of control. 

And\marginnote{22.1} all that was known \\
and distinguished by the Seer. \\
Therefore he addressed \\
the disciples who love the teaching: 

“\textsanskrit{Māra}’s\marginnote{22.5} army has arrived; \\
mendicants, you should be aware of them.” \\
Those monks became keen, \\
hearing the Buddha’s instruction. \\
The army fled from those free of passion, \\
and not a single hair was stirred! 

“All\marginnote{22.11} are triumphant in battle, \\
so fearless and glorious. \\
They rejoice with all the spirits, \\
the disciples well-known among men.” 

%
\end{verse}

%
\chapter*{{\suttatitleacronym DN 21}{\suttatitletranslation Sakka’s Questions }{\suttatitleroot Sakkapañhasutta}}
\addcontentsline{toc}{chapter}{\tocacronym{DN 21} \toctranslation{Sakka’s Questions } \tocroot{Sakkapañhasutta}}
\markboth{Sakka’s Questions }{Sakkapañhasutta}
\extramarks{DN 21}{DN 21}

\scevam{So\marginnote{1.1.1} I have heard. }At one time the Buddha was staying in the land of the Magadhans; east of \textsanskrit{Rājagaha} there’s a brahmin village named \textsanskrit{Ambasaṇḍā}, north of which, on Mount Vediyaka, is the \textsanskrit{Indasāla} cave. 

Now\marginnote{1.1.3} at that time Sakka, the lord of gods, became eager to see the Buddha. He thought, “Where is the Blessed One at present, the perfected one, the fully awakened Buddha?” 

He\marginnote{1.1.6} saw that the Buddha was at the \textsanskrit{Indasāla} cave, and addressed the gods of the Thirty-Three, “Good sirs, the Buddha is staying in the land of the Magadhans at the \textsanskrit{Indasāla} cave. What if we were to go and see that Blessed One, the perfected one, the fully awakened Buddha?” 

“Yes,\marginnote{1.1.10} lord,” replied the gods. 

Then\marginnote{1.2.1} Sakka addressed the fairy \textsanskrit{Pañcasikha}, “Dear \textsanskrit{Pañcasikha}, the Buddha is staying in the land of the Magadhans at the \textsanskrit{Indasāla} cave. What if we were to go and see that Blessed One, the perfected one, the fully awakened Buddha?” 

“Yes,\marginnote{1.2.4} lord,” replied the fairy \textsanskrit{Pañcasikha}. Taking his arched harp made from the pale timber of wood-apple, he went as Sakka’s attendant. 

Then\marginnote{1.2.5} Sakka went at the head of a retinue consisting of the gods of the Thirty-Three and the fairy \textsanskrit{Pañcasikha}. As easily as a strong person would extend or contract their arm, he vanished from the heaven of the gods of the Thirty-Three and landed on Mount Vediyaka north of \textsanskrit{Ambasaṇḍā}. 

Now\marginnote{1.3.1} at that time a dazzling light appeared over Mount Vediyaka and \textsanskrit{Ambasaṇḍā}, as happens through the glory of the gods. People in the villages round about, terrified, shocked, and awestruck, said, “Mount Vediyaka must be on fire today, blazing and burning! Oh why has such a dazzling light appeared over Mount Vediyaka and \textsanskrit{Ambasaṇḍā}?” 

Then\marginnote{1.4.1} Sakka addressed the fairy \textsanskrit{Pañcasikha}, “My dear \textsanskrit{Pañcasikha}, it’s hard for one like me to get near the Realized Ones while they are on retreat practicing absorption, enjoying absorption. But if you were to charm the Buddha first, then I could go to see him.” 

“Yes,\marginnote{1.4.4} lord,” replied the fairy \textsanskrit{Pañcasikha}. Taking his arched harp made from the pale timber of wood-apple, he went to the \textsanskrit{Indasāla} cave. When he had drawn near, he stood to one side, thinking, “This is neither too far nor too near; and he’ll hear my voice.” 

\section*{1. \textsanskrit{Pañcasikha}’s Song }

Standing\marginnote{1.4.8} to one side, \textsanskrit{Pañcasikha} played his arched harp, and sang these verses on the Buddha, the teaching, the \textsanskrit{Saṅgha}, the perfected ones, and sensual love. 

\begin{verse}%
“My\marginnote{1.5.1} lady \textsanskrit{Suriyavaccasā}, oh my Sunshine—\\
I pay homage to your father Timbaru, \\
through whom was born a lady so fine, \\
to fill me with a joy I never knew. 

As\marginnote{1.5.5} sweet as a breeze to one who’s sweating, \\
or when thirsty, a sweet and cooling drink, \\
so dear is your shining beauty to me, \\
just like the teaching is to all the saints! 

Like\marginnote{1.5.9} a cure when you’re struck by fever dire, \\
or food to ease the hunger pain, \\
come on, darling, please put out my fire, \\
quench me like water on a flame. 

As\marginnote{1.5.13} elephants burning in the heat of summer, \\
sink down in a lotus pond to rest, \\
so cool, full of petals and of pollen—\\
that’s how I would plunge into your breast. 

Like\marginnote{1.5.17} elephants bursting bonds in rutting season, \\
beating off the pricks of lance and pikes—\\
I just don’t understand what is the reason \\
I’m so crazy for your shapely thighs! 

For\marginnote{1.5.21} you, my heart is full of passion, \\
I’m in an altered state of mind. \\
There is no going back, I’m just not able, \\
I’m like a fish that’s hooked up on the line. 

Come\marginnote{1.5.25} on, my darling, hold me, fair of thighs! \\
Embrace me, with your so bashful eyes! \\
Take me in your arms, my lovely lady, \\
that’s all I’d ever want or could desire. 

Ah,\marginnote{1.5.29} then my desire was such a small thing, \\
my sweet, with your curling wavy hair; \\
now, like to arahants an offering, \\
it’s grown so very much from there. 

Whatever\marginnote{1.5.33} the merit I have forged \\
by giving to such perfected beings—\\
may that, my altogether gorgeous, \\
ripen in togetherness with you. 

Whatever\marginnote{1.5.37} the merit I have forged \\
in this vast territory, \\
may that, my altogether gorgeous, \\
ripen in togetherness with you. 

Absorbed,\marginnote{1.5.41} the Sakyan meditates, \\
unified, alert, and mindful, \\
the sage seeks the deathless state—\\
like me, oh my Sunshine, aiming for you! 

And\marginnote{1.5.45} just like the sage would be rejoicing, \\
were he to awaken to the truth, \\
so I’d be rejoicing, lady, \\
were I to end up as one with you. 

If\marginnote{1.5.49} Sakka were to grant me just one wish, \\
as Lord of the holy Thirty-Three, \\
my darling, you’re the only one I’d wish for, \\
so strong is the love I hold for you. 

Like\marginnote{1.5.53} a freshly blossoming sal tree \\
is your father, my lady so wise. \\
I pay homage to him, bowing down humbly, \\
to he whose daughter is of such a kind.” 

%
\end{verse}

When\marginnote{1.6.1} \textsanskrit{Pañcasikha} had spoken, the Buddha said to him, “\textsanskrit{Pañcasikha}, the sound of the strings and the sound of your voice blend well together, so that neither overpowers the other. But when did you compose these verses on the Buddha, the teaching, the \textsanskrit{Saṅgha}, the perfected ones, and sensual love?” 

“This\marginnote{1.6.5} one time, sir, when you were first awakened, you were staying near \textsanskrit{Uruvelā} at the goatherd’s banyan tree on the bank of the \textsanskrit{Nerañjarā} River. And at that time I was in love with the lady named \textsanskrit{Bhaddā} \textsanskrit{Suriyavacchasā}, the daughter of the fairy king Timbaru. But the sister desired another—it was \textsanskrit{Mātali} the charioteer’s son named \textsanskrit{Sikhaḍḍī} who she loved. Since I couldn’t win that sister by any means, I took my arched harp to Timbaru’s home, where I played those verses. 

When\marginnote{1.7.10} I finished, \textsanskrit{Suriyavacchasā} said to me, ‘Dear sir, I have not personally seen the Buddha. But I did hear about him when I went to dance for the gods of the Thirty-Three in the Hall of Justice. Since you extol the Buddha, let us meet up today.’ And that’s when I met up with that sister. But we have not met since.” 

\section*{2. The Approach of Sakka }

Then\marginnote{1.8.1} Sakka, lord of gods, thought, “\textsanskrit{Pañcasikha} is exchanging pleasantries with the Buddha.” 

So\marginnote{1.8.3} he addressed \textsanskrit{Pañcasikha}, “My dear \textsanskrit{Pañcasikha}, please bow to the Buddha for me, saying: ‘Sir, Sakka, lord of gods, with his ministers and retinue, bows with his head at your feet.’” 

“Yes,\marginnote{1.8.6} lord,” replied \textsanskrit{Pañcasikha}. He bowed to the Buddha and said, “Sir, Sakka, lord of gods, with his ministers and retinue, bows with his head at your feet.” 

“So\marginnote{1.8.8} may Sakka with his ministers and retinue be happy, \textsanskrit{Pañcasikha},” said the Buddha, “for all want to be happy—whether gods, humans, demons, dragons, fairies, or any of the other diverse creatures there may be.” 

For\marginnote{1.8.10} that is how the Realized Ones salute such illustrious spirits. And being saluted by the Buddha, Sakka entered the \textsanskrit{Indasāla} cave, bowed to the Buddha, and stood to one side. And the gods of the Thirty-Three did likewise, as did \textsanskrit{Pañcasikha}. 

And\marginnote{1.9.1} at that time the uneven places were evened out, the cramped places were opened up, the darkness vanished in the cave and light appeared, as happens through the glory of the gods. 

Then\marginnote{1.9.2} the Buddha said to Sakka, “It’s incredible and amazing that you, the venerable Kosiya, who has so many duties and so much to do, should come here.” 

“For\marginnote{1.9.4} a long time I’ve wanted to come and see the Buddha, but I wasn’t able, being prevented by my many duties and responsibilities for the gods of the Thirty-Three. This one time, sir, the Buddha was staying near \textsanskrit{Sāvatthī} in the frankincense-tree hut. Then I went to \textsanskrit{Sāvatthī} to see the Buddha. But at that time the Buddha was sitting immersed in some kind of meditation. And a divine maiden of Great King \textsanskrit{Vessavaṇa} named \textsanskrit{Bhūjati} was attending on the Buddha, standing there paying homage to him with joined palms. 

So\marginnote{1.10.2} I said to her, ‘Sister, please bow to the Buddha for me, saying: “Sir, Sakka, lord of gods, with his ministers and retinue, bows with his head at your feet.”’ 

When\marginnote{1.10.5} I said this, she said to me, ‘It’s the wrong time to see the Buddha, as he’s in retreat.’ 

‘Well\marginnote{1.10.8} then, sister, please convey my message when the Buddha emerges from that immersion.’ I hope that sister bowed to you? Do you remember what she said?” 

“She\marginnote{1.10.12} did bow, lord of gods, and I remember what she said. I also remember that it was the sound of your chariot wheels that pulled me out of that immersion.” 

“Sir,\marginnote{1.11.1} I have heard and learned this in the presence of the gods who were reborn in the host of the Thirty-Three before me: ‘When a Realized One arises in the world, perfected and fully awakened, the heavenly hosts swell, while the demon hosts dwindle.’ And I have seen this with my own eyes. 

\subsection*{2.1. The Story of \textsanskrit{Gopikā} }

Right\marginnote{1.11.5} here in Kapilavatthu there was a Sakyan lady named \textsanskrit{Gopikā} who had confidence in the Buddha, the teaching, and the \textsanskrit{Saṅgha}, and had fulfilled her ethics. Losing her attachment to femininity, she developed masculinity. When her body broke up, after death, she was reborn in a good place, a heavenly realm. In the company of the gods of the Thirty-Three she became one of my sons. There they knew him as the god Gopaka. 

Meanwhile\marginnote{1.11.10} three others, mendicants who had led the spiritual life under the Buddha, were reborn in the inferior fairy realm. There they amused themselves, supplied and provided with the five kinds of sensual stimulation, and became my servants and attendants. 

At\marginnote{1.11.12} that, Gopaka scolded them, ‘Where on earth were you at, good sirs, when you heard the Buddha’s teaching! For while I was still a woman I had confidence in the Buddha, the teaching, and the \textsanskrit{Saṅgha}, and had fulfilled my ethics. I lost my attachment to femininity and developed masculinity. When my body broke up, after death, I was reborn in a good place, a heavenly realm. In the company of the gods of the Thirty-Three I became one of Sakka’s sons. Here they know me as the god Gopaka. But you, having led the spiritual life under the Buddha, were reborn in the inferior fairy realm.’ 

When\marginnote{1.11.19} scolded by Gopaka, two of those gods in that very life gained mindfulness leading to the host of \textsanskrit{Brahmā}’s Ministers. But one god remained attached to sensuality. 

\begin{verse}%
‘I\marginnote{1.12.1} was a laywoman disciple of the seer, \\
and my name was \textsanskrit{Gopikā}. \\
I was devoted to the Buddha and the teaching, \\
and I faithfully served the \textsanskrit{Saṅgha}. 

Because\marginnote{1.12.5} of the excellence of the Buddha’s teaching, \\
I’m now a mighty, splendid son of Sakka, \\
reborn among the Three and Thirty. \\
And here they know me as Gopaka. 

Then\marginnote{1.12.9} I saw some mendicants who I’d seen before, \\
dwelling in the host of fairies. \\
When I used to be a human, \\
they were disciples of Gotama. 

I\marginnote{1.12.13} served them with food and drink, \\
and clasped their feet in my own home. \\
Where on earth were they at \\
when they learned the Buddha’s teachings? 

For\marginnote{1.12.17} each must know for themselves the teaching \\
so well-taught, realized by the seer. \\
I was one who followed you, \\
having heard the fine words of the noble ones. 

I’m\marginnote{1.12.21} now a mighty, splendid son of Sakka, \\
reborn among the Three and Thirty. \\
But you followed the best of men, \\
and led the supreme spiritual life, 

but\marginnote{1.12.25} still you’re born in this lesser realm, \\
a rebirth not befitting. \\
It’s a sorry sight I see, good sirs, \\
fellow Buddhists in a lesser realm. 

Reborn\marginnote{1.12.29} in the host of fairies, \\
you only wait upon the gods. \\
Meanwhile, I dwelt in a house—\\
but see my distinction now! 

Having\marginnote{1.12.33} been a woman now I’m a male god, \\
blessed with heavenly sensual pleasures.’ \\
Scolded by that disciple of Gotama, \\
when they understood Gopaka, they were struck with urgency. 

‘Let’s\marginnote{1.12.37} strive, let’s try hard—\\
we won’t serve others any more!’ \\
Two of them roused up energy, \\
recalling the Buddha’s instructions. 

Right\marginnote{1.12.41} away they became dispassionate, \\
seeing the drawbacks in sensual pleasures. \\
The fetters and bonds of sensual pleasures—\\
the ties of the Wicked One so hard to break—

they\marginnote{1.12.45} burst them like a bull elephant his ropes, \\
and passed right over the Thirty-Three. \\
The gods with Indra and \textsanskrit{Pajāpati} \\
were all gathered in the Hall of Justice. 

As\marginnote{1.12.49} they sat there, they passed over them, \\
the heroes desireless, practicing purity. \\
Seeing them, \textsanskrit{Vāsava} was struck with a sense of urgency; \\
the master of gods in the midst of the group said, 

‘These\marginnote{1.12.53} were born in the lesser fairy realm, \\
but now they pass us by!’ \\
Heeding the speech of one so moved, \\
Gopaka addressed \textsanskrit{Vāsava}, 

‘There\marginnote{1.12.57} is a Buddha, a lord of men, in the world. \\
Known as the Sakyan Sage, he’s mastered the senses. \\
Those sons of his were bereft of mindfulness; \\
but when scolded by me they gained it back. 

Of\marginnote{1.12.61} the three, there is one who remains \\
dwelling in the host of fairies. \\
But two, recollecting the path to awakening, \\
serene, spurn even the gods.’ 

The\marginnote{1.12.65} teaching’s explained in such a way \\
that not a single disciple doubts it. \\
We venerate the Buddha, the victor, lord of men, \\
who has crossed the flood and cut off doubt. 

They\marginnote{1.12.69} attained to distinction fitting \\
the extent to which they understood the teaching here. \\
Two of them mastered the distinction of \\
the host of \textsanskrit{Brahmā}’s Ministers. 

We\marginnote{1.12.73} have come here, dear sir, \\
to realize this same teaching. \\
If the Buddha would give me a chance, \\
I would ask a question, dear sir.” 

%
\end{verse}

Then\marginnote{1.13.1} the Buddha thought, “For a long time now this spirit has led a pure life. Any question he asks me will be beneficial, not useless. And he will quickly understand any answer I give to his question.” 

So\marginnote{1.13.4} the Buddha addressed Sakka in verse: 

\begin{verse}%
“Ask\marginnote{1.13.5} me your question, \textsanskrit{Vāsava}, \\
whatever you want. \\
I’ll solve each and every \\
question you have.” 

%
\end{verse}

\scendsection{The first recitation section is finished. }

Having\marginnote{2.1.1} been granted an opportunity by the Buddha, Sakka asked the first question. 

“Dear\marginnote{2.1.2} sir, what fetters bind the gods, humans, demons, dragons, fairies—and any of the other diverse creatures—so that, though they wish to be free of enmity, violence, hostility, and hate, they still have enmity, violence, hostility, and hate?” 

Such\marginnote{2.1.4} was Sakka’s question to the Buddha. And the Buddha answered him: 

“Lord\marginnote{2.1.6} of gods, the fetters of jealousy and stinginess bind the gods, humans, demons, dragons, fairies—and any of the other diverse creatures—so that, though they wish to be free of enmity, violence, hostility, and hate, they still have enmity, violence, hostility, and hate.” 

Such\marginnote{2.1.8} was the Buddha’s answer to Sakka. Delighted, Sakka approved and agreed with what the Buddha said, saying, “That’s so true, Blessed One! That’s so true, Holy One! Hearing the Buddha’s answer, I’ve gone beyond doubt and got rid of indecision.” 

And\marginnote{2.2.1} then, having approved and agreed with what the Buddha said, Sakka asked another question: 

“But\marginnote{2.2.2} dear sir, what is the source, origin, birthplace, and inception of jealousy and stinginess? When what exists is there jealousy and stinginess? When what doesn’t exist is there no jealousy and stinginess?” 

“The\marginnote{2.2.5} liked and the disliked, lord of gods, are the source of jealousy and stinginess. When the liked and the disliked exist there is jealousy and stinginess. When the liked and the disliked don’t exist there is no jealousy and stinginess.” 

“But\marginnote{2.2.8} dear sir, what is the source of what is liked and disliked?” 

“Desire\marginnote{2.2.11} is the source of what is liked and disliked.” 

“But\marginnote{2.2.14} what is the source of desire?” 

“Thought\marginnote{2.2.17} is the source of desire.” 

“But\marginnote{2.2.20} what is the source of thought?” 

“Concepts\marginnote{2.2.23} of identity that emerge from the proliferation of perceptions are the source of thoughts.” 

“But\marginnote{2.3.1} how does a mendicant fittingly practice for the cessation of concepts of identity that emerge from the proliferation of perceptions?” 

\subsection*{2.2. Meditation on Feelings }

“Lord\marginnote{2.3.3} of gods, there are two kinds of happiness, I say: that which you should cultivate, and that which you should not cultivate. There are two kinds of sadness, I say: that which you should cultivate, and that which you should not cultivate. There are two kinds of equanimity, I say: that which you should cultivate, and that which you should not cultivate. 

Why\marginnote{2.3.9} did I say that there are two kinds of happiness? Well, should you know of a happiness: ‘When I cultivate this kind of happiness, unskillful qualities grow, and skillful qualities decline.’ You should not cultivate that kind of happiness. Whereas, should you know of a happiness: ‘When I cultivate this kind of happiness, unskillful qualities decline, and skillful qualities grow.’ You should cultivate that kind of happiness. And that which is free of placing the mind and keeping it connected is better than that which still involves placing the mind and keeping it connected. That’s why I said there are two kinds of happiness. 

Why\marginnote{2.3.17} did I say that there are two kinds of sadness? Well, should you know of a sadness: ‘When I cultivate this kind of sadness, unskillful qualities grow, and skillful qualities decline.’ You should not cultivate that kind of sadness. Whereas, should you know of a sadness: ‘When I cultivate this kind of sadness, unskillful qualities decline, and skillful qualities grow.’ You should cultivate that kind of sadness. And that which is free of placing the mind and keeping it connected is better than that which still involves placing the mind and keeping it connected. That’s why I said there are two kinds of sadness. 

Why\marginnote{2.3.25} did I say that there are two kinds of equanimity? Well, should you know of an equanimity: ‘When I cultivate this kind of equanimity, unskillful qualities grow, and skillful qualities decline.’ You should not cultivate that kind of equanimity. Whereas, should you know of an equanimity: ‘When I cultivate this kind of equanimity, unskillful qualities decline, and skillful qualities grow.’ You should cultivate that kind of equanimity. And that which is free of placing the mind and keeping it connected is better than that which still involves placing the mind and keeping it connected. That’s why I said there are two kinds of equanimity. 

That’s\marginnote{2.3.33} how a mendicant fittlingly practices for the cessation of concepts of identity that emerge from the proliferation of perceptions.” 

Such\marginnote{2.3.34} was the Buddha’s answer to Sakka. Delighted, Sakka approved and agreed with what the Buddha said, saying, “That’s so true, Blessed One! That’s so true, Holy One! Hearing the Buddha’s answer, I’ve gone beyond doubt and got rid of indecision.” 

\subsection*{2.3. Restraint in the Monastic Code }

And\marginnote{2.4.1} then Sakka asked another question: 

“But\marginnote{2.4.2} dear sir, how does a mendicant practice for restraint in the monastic code?” 

“Lord\marginnote{2.4.3} of gods, I say that there are two kinds of bodily behavior: that which you should cultivate, and that which you should not cultivate. I say that there are two kinds of verbal behavior: that which you should cultivate, and that which you should not cultivate. There are two kinds of search, I say: that which you should cultivate, and that which you should not cultivate. 

Why\marginnote{2.4.9} did I say that there are two kinds of bodily behavior? Well, should you know of a bodily conduct: ‘When I cultivate this kind of bodily conduct, unskillful qualities grow, and skillful qualities decline.’ You should not cultivate that kind of bodily conduct. Whereas, should you know of a bodily conduct: ‘When I cultivate this kind of bodily conduct, unskillful qualities decline, and skillful qualities grow.’ You should cultivate that kind of bodily conduct. That’s why I said there are two kinds of bodily behavior. 

Why\marginnote{2.4.17} did I say that there are two kinds of verbal behavior? Well, should you know of a kind of verbal behavior that it causes unskillful qualities to grow while skillful qualities decline, you should not cultivate it. Whereas, should you know of a kind of verbal behavior that it causes unskillful qualities to decline while skillful qualities grow, you should cultivate it. That’s why I said there are two kinds of verbal behavior. 

Why\marginnote{2.4.24} did I say that there are two kinds of search? Well, should you know of a kind of search that it causes unskillful qualities to grow while skillful qualities decline, you should not cultivate it. Whereas, should you know of a kind of search that it causes unskillful qualities to decline while skillful qualities grow, you should cultivate it. That’s why I said there are two kinds of search. 

That’s\marginnote{2.4.31} how a mendicant practices for restraint in the monastic code.” 

Such\marginnote{2.4.32} was the Buddha’s answer to Sakka. Delighted, Sakka approved and agreed with what the Buddha said, saying, “That’s so true, Blessed One! That’s so true, Holy One! Hearing the Buddha’s answer, I’ve gone beyond doubt and got rid of indecision.” 

\subsection*{2.4. Sense Restraint }

And\marginnote{2.5.1} then Sakka asked another question: 

“But\marginnote{2.5.2} dear sir, how does a mendicant practice for restraint of the sense faculties?” 

“Lord\marginnote{2.5.3} of gods, I say that there are two kinds of sight known by the eye: that which you should cultivate, and that which you should not cultivate. There are two kinds of sound known by the ear … smells known by the nose … tastes known by the tongue … touches known by the body … thoughts known by the mind: that which you should cultivate, and that which you should not cultivate.” 

When\marginnote{2.5.15} the Buddha said this, Sakka said to him: 

“Sir,\marginnote{2.5.16} this is how I understand the detailed meaning of the Buddha’s brief statement: You should not cultivate the kind of sight known by the eye which causes unskillful qualities to grow while skillful qualities decline. And you should cultivate the kind of sight known by the eye which causes unskillful qualities to decline while skillful qualities grow. You should not cultivate the kind of sound, smell, taste, touch, or thought known by the mind which causes unskillful qualities to grow while skillful qualities decline. And you should cultivate the kind of thought known by the mind which causes unskillful qualities to decline while skillful qualities grow. 

Sir,\marginnote{2.5.25} that’s how I understand the detailed meaning of the Buddha’s brief statement. Hearing the Buddha’s answer, I’ve gone beyond doubt and got rid of indecision.” 

And\marginnote{2.6.1} then Sakka asked another question: 

“Dear\marginnote{2.6.2} sir, do all ascetics and brahmins have the same doctrine, ethics, desires, and attachments?” 

“No,\marginnote{2.6.3} lord of gods, they do not.” 

“Why\marginnote{2.6.4} not?” 

“The\marginnote{2.6.5} world has many and diverse elements. Whatever element sentient beings insist on in this world of many and diverse elements, they obstinately stick to it, insisting that: ‘This is the only truth, other ideas are silly.’ That’s why not all ascetics and brahmins have the same doctrine, ethics, desires, and attachments.” 

“Dear\marginnote{2.6.9} sir, have all ascetics and brahmins reached the ultimate end, the ultimate sanctuary, the ultimate spiritual life, the ultimate goal?” 

“No,\marginnote{2.6.10} lord of gods, they have not.” 

“Why\marginnote{2.6.11} not?” 

“Those\marginnote{2.6.12} mendicants who are freed through the ending of craving have reached the ultimate end, the ultimate sanctuary, the ultimate spiritual life, the ultimate goal. That’s why not all ascetics and brahmins have reached the ultimate end, the ultimate sanctuary, the ultimate spiritual life, the ultimate goal.” 

Such\marginnote{2.6.14} was the Buddha’s answer to Sakka. Delighted, Sakka approved and agreed with what the Buddha said, saying, “That’s so true, Blessed One! That’s so true, Holy One! Hearing the Buddha’s answer, I’ve gone beyond doubt and got rid of indecision.” 

And\marginnote{2.6.18} then Sakka asked another question: 

“Turbulence,\marginnote{2.7.1} sir, is a disease, a boil, a dart. Turbulence drags a person to be reborn in life after life. That’s why a person finds themselves in states high and low. Elsewhere, among other ascetics and brahmins, I wasn’t even given a chance to ask these questions that the Buddha has answered. The dart of doubt and uncertainty has lain within me for a long time, but the Buddha has plucked it out.” 

“Lord\marginnote{2.7.5} of gods, do you recall having asked this question of other ascetics and brahmins?” 

“I\marginnote{2.7.6} do, sir.” 

“If\marginnote{2.7.7} you wouldn’t mind, lord of gods, tell me how they answered.” 

“It’s\marginnote{2.7.8} no trouble when someone such as the Blessed One is sitting here.” 

“Well,\marginnote{2.7.9} speak then, lord of gods.” 

“Sir,\marginnote{2.7.10} I approached those who I imagined were ascetics and brahmins living in the wilderness, in remote lodgings. But they were stumped by my question, and they even questioned me in return: ‘What is the venerable’s name?’ So I answered them: ‘Dear sir, I am Sakka, lord of gods.’ So they asked me another question: ‘But lord of gods, what deed brought you to this position?’ So I taught them the Dhamma as I had learned and memorized it. And they were pleased with just that much: ‘We have seen Sakka, lord of gods! And he answered our questions!’ Invariably, they become my disciples, I don’t become theirs. But sir, I am the Buddha’s disciple, a stream-enterer, not liable to be reborn in the underworld, bound for awakening.” 

\subsection*{2.5. On Feeling Happy }

“Lord\marginnote{2.7.22} of gods, do you recall ever feeling such joy and happiness before?” 

“I\marginnote{2.7.23} do, sir.” 

“But\marginnote{2.7.24} how?” 

“Once\marginnote{2.7.25} upon a time, sir, a battle was fought between the gods and the demons. In that battle the gods won and the demons lost. It occurred to me as victor, ‘Now the gods shall enjoy both the nectar of the gods and the nectar of the demons.’ But sir, that joy and happiness is in the sphere of the rod and the sword. It doesn’t lead to disillusionment, dispassion, cessation, peace, insight, awakening, and extinguishment. But the joy and happiness I feel listening to the Buddha’s teaching is not in the sphere of the rod and the sword. It does lead to disillusionment, dispassion, cessation, peace, insight, awakening, and extinguishment.” 

“But\marginnote{2.8.1} lord of gods, what reason do you see for speaking of such joy and happiness?” 

“I\marginnote{2.8.2} see six reasons to speak of such joy and happiness, sir. 

\begin{verse}%
While\marginnote{2.8.3} staying right here, \\
remaining in the godly form, \\
I have gained an extended life: \\
know this, dear sir. 

%
\end{verse}

This\marginnote{2.8.7} is the first reason. 

\begin{verse}%
When\marginnote{2.8.8} I fall from the heavenly host, \\
leaving behind the non-human life, \\
I shall consciously go to a new womb, \\
wherever my mind delights. 

%
\end{verse}

This\marginnote{2.8.12} is the second reason. 

\begin{verse}%
Living\marginnote{2.8.13} happily under the guidance \\
of the one of unclouded wisdom, \\
I shall practice according to method, \\
aware and mindful. 

%
\end{verse}

This\marginnote{2.8.17} is the third reason. 

\begin{verse}%
And\marginnote{2.8.18} if awakening should arise \\
as I practice according to the method, \\
I shall live as one who understands, \\
and my end shall come right there. 

%
\end{verse}

This\marginnote{2.8.22} is the fourth reason. 

\begin{verse}%
When\marginnote{2.8.23} I fall from the human realm, \\
leaving behind the human life, \\
I shall become a god again, \\
in the supreme heaven realm. 

%
\end{verse}

This\marginnote{2.8.27} is the fifth reason. 

\begin{verse}%
They\marginnote{2.8.28} are the finest of gods, \\
the glorious \textsanskrit{Akaniṭṭhas}. \\
So long as my final life goes on, \\
there my home will be. 

%
\end{verse}

This\marginnote{2.8.32} is the sixth reason. 

Seeing\marginnote{2.8.33} these six reasons I speak of such joy and happiness. 

\begin{verse}%
My\marginnote{2.9.1} wishes unfulfilled, \\
doubting and undecided, \\
I wandered for such a long time, \\
in search of the Realized One. 

I\marginnote{2.9.5} imagined that ascetics \\
living in seclusion \\
must surely be awakened, \\
so I went to sit near them. 

‘How\marginnote{2.9.9} is there success? \\
How is there failure?’ \\
But they were stumped by such questions \\
about the path and practice. 

And\marginnote{2.9.13} when they found out that I \\
was Sakka, come from the gods, \\
they questioned me instead about \\
the deed that brought me to this state. 

I\marginnote{2.9.17} taught them the Dhamma \\
as I had learned it among men. \\
They were delighted with that, saying: \\
‘We’ve seen \textsanskrit{Vāsava}!’ 

Now\marginnote{2.9.21} since I’ve seen the Buddha, \\
who helps us overcome doubt, \\
today, free of fear, \\
I pay homage to the awakened one. 

Destroyer\marginnote{2.9.25} of the dart of craving, \\
the Buddha is unrivaled. \\
I bow to the great hero, \\
the Buddha, kinsman of the Sun. 

Just\marginnote{2.9.29} as \textsanskrit{Brahmā} is worshipped \\
by the gods, dear sir, \\
today we shall worship you—\\
come, let us bow to you! 

You\marginnote{2.9.33} alone are the Awakened! \\
You are the Teacher supreme! \\
In the world with its gods, \\
you have no counterpart.” 

%
\end{verse}

Then\marginnote{2.10.1} Sakka addressed the fairy \textsanskrit{Pañcasikha}, “Dear \textsanskrit{Pañcasikha}, you were very helpful to me, since you first charmed the Buddha, after which I went to see him. I shall appoint you to your father’s position—you shall be king of the fairies. And I give you \textsanskrit{Bhaddā} \textsanskrit{Suriyavaccasā}, who you love so much.” 

Then\marginnote{2.10.5} Sakka, touching the ground with his hand, expressed this heartfelt sentiment three times: 

“Homage\marginnote{2.10.6} to that Blessed One, the perfected one, the fully awakened Buddha! 

Homage\marginnote{2.10.7} to that Blessed One, the perfected one, the fully awakened Buddha! 

Homage\marginnote{2.10.8} to that Blessed One, the perfected one, the fully awakened Buddha!” 

And\marginnote{2.10.9} while this discourse was being spoken, the stainless, immaculate vision of the Dhamma arose in Sakka, lord of gods: “Everything that has a beginning has an end.” And also for another 80,000 deities. 

Such\marginnote{2.10.12} were the questions Sakka was invited to ask, and which were answered by the Buddha. And that’s why the name of this discussion is “Sakka’s Questions”. 

%
\chapter*{{\suttatitleacronym DN 22}{\suttatitletranslation The Longer Discourse on Mindfulness Meditation }{\suttatitleroot Mahāsatipaṭṭhānasutta}}
\addcontentsline{toc}{chapter}{\tocacronym{DN 22} \toctranslation{The Longer Discourse on Mindfulness Meditation } \tocroot{Mahāsatipaṭṭhānasutta}}
\markboth{The Longer Discourse on Mindfulness Meditation }{Mahāsatipaṭṭhānasutta}
\extramarks{DN 22}{DN 22}

\scevam{So\marginnote{1.1} I have heard. }At one time the Buddha was staying in the land of the Kurus, near the Kuru town named \textsanskrit{Kammāsadamma}. There the Buddha addressed the mendicants, “Mendicants!” 

“Venerable\marginnote{1.5} sir,” they replied. The Buddha said this: 

“Mendicants,\marginnote{1.7} the four kinds of mindfulness meditation are the path to convergence. They are in order to purify sentient beings, to get past sorrow and crying, to make an end of pain and sadness, to end the cycle of suffering, and to realize extinguishment. 

What\marginnote{1.8} four? It’s when a mendicant meditates by observing an aspect of the body—keen, aware, and mindful, rid of desire and aversion for the world. They meditate observing an aspect of feelings—keen, aware, and mindful, rid of desire and aversion for the world. They meditate observing an aspect of the mind—keen, aware, and mindful, rid of desire and aversion for the world. They meditate observing an aspect of principles—keen, aware, and mindful, rid of desire and aversion for the world. 

\section*{1. Observing the Body }

\subsection*{1.1. Mindfulness of Breathing }

And\marginnote{2.1} how does a mendicant meditate observing an aspect of the body? 

It’s\marginnote{2.2} when a mendicant—gone to a wilderness, or to the root of a tree, or to an empty hut—sits down cross-legged, with their body straight, and focuses their mindfulness right there. Just mindful, they breathe in. Mindful, they breathe out. 

When\marginnote{2.4} breathing in heavily they know: ‘I’m breathing in heavily.’ When breathing out heavily they know: ‘I’m breathing out heavily.’ 

When\marginnote{2.5} breathing in lightly they know: ‘I’m breathing in lightly.’ When breathing out lightly they know: ‘I’m breathing out lightly.’ 

They\marginnote{2.6} practice breathing in experiencing the whole body. They practice breathing out experiencing the whole body. 

They\marginnote{2.7} practice breathing in stilling the body’s motion. They practice breathing out stilling the body’s motion. 

It’s\marginnote{2.8} like a deft carpenter or carpenter’s apprentice. When making a deep cut they know: ‘I’m making a deep cut,’ and when making a shallow cut they know: ‘I’m making a shallow cut.’ 

And\marginnote{2.11} so they meditate observing an aspect of the body internally, externally, and both internally and externally. They meditate observing the body as liable to originate, as liable to vanish, and as liable to both originate and vanish. Or mindfulness is established that the body exists, to the extent necessary for knowledge and mindfulness. They meditate independent, not grasping at anything in the world. 

That’s\marginnote{2.14} how a mendicant meditates by observing an aspect of the body. 

\subsection*{1.2. The Postures }

Furthermore,\marginnote{3.1} when a mendicant is walking they know: ‘I am walking.’ When standing they know: ‘I am standing.’ When sitting they know: ‘I am sitting.’ And when lying down they know: ‘I am lying down.’ Whatever posture their body is in, they know it. 

And\marginnote{3.3} so they meditate observing an aspect of the body internally, externally, and both internally and externally. They meditate observing the body as liable to originate, as liable to vanish, and as liable to both originate and vanish. Or mindfulness is established that the body exists, to the extent necessary for knowledge and mindfulness. They meditate independent, not grasping at anything in the world. 

That\marginnote{3.6} too is how a mendicant meditates by observing an aspect of the body. 

\subsection*{1.3. Situational Awareness }

Furthermore,\marginnote{4.1} a mendicant acts with situational awareness when going out and coming back; when looking ahead and aside; when bending and extending the limbs; when bearing the outer robe, bowl, and robes; when eating, drinking, chewing, and tasting; when urinating and defecating; when walking, standing, sitting, sleeping, waking, speaking, and keeping silent. 

And\marginnote{4.2} so they meditate observing an aspect of the body internally … 

That\marginnote{4.3} too is how a mendicant meditates by observing an aspect of the body. 

\subsection*{1.4. Focusing on the Repulsive }

Furthermore,\marginnote{5.1} a mendicant examines their own body, up from the soles of the feet and down from the tips of the hairs, wrapped in skin and full of many kinds of filth. ‘In this body there is head hair, body hair, nails, teeth, skin, flesh, sinews, bones, bone marrow, kidneys, heart, liver, diaphragm, spleen, lungs, intestines, mesentery, undigested food, feces, bile, phlegm, pus, blood, sweat, fat, tears, grease, saliva, snot, synovial fluid, urine.’ 

It’s\marginnote{5.3} as if there were a bag with openings at both ends, filled with various kinds of grains, such as fine rice, wheat, mung beans, peas, sesame, and ordinary rice. And someone with good eyesight were to open it and examine the contents: ‘These grains are fine rice, these are wheat, these are mung beans, these are peas, these are sesame, and these are ordinary rice.’ 

And\marginnote{5.6} so they meditate observing an aspect of the body internally … 

That\marginnote{5.7} too is how a mendicant meditates by observing an aspect of the body. 

\subsection*{1.5. Focusing on the Elements }

Furthermore,\marginnote{6.1} a mendicant examines their own body, whatever its placement or posture, according to the elements: ‘In this body there is the earth element, the water element, the fire element, and the air element.’ 

It’s\marginnote{6.3} as if a deft butcher or butcher’s apprentice were to kill a cow and sit down at the crossroads with the meat cut into portions. 

And\marginnote{6.6} so they meditate observing an aspect of the body internally … 

That\marginnote{6.7} too is how a mendicant meditates by observing an aspect of the body. 

\subsection*{1.6. The Charnel Ground Contemplations }

Furthermore,\marginnote{7.1} suppose a mendicant were to see a corpse discarded in a charnel ground. And it had been dead for one, two, or three days, bloated, livid, and festering. They’d compare it with their own body: ‘This body is also of that same nature, that same kind, and cannot go beyond that.’ And so they meditate observing an aspect of the body internally … 

That\marginnote{7.5} too is how a mendicant meditates by observing an aspect of the body. 

Furthermore,\marginnote{8.1} suppose they were to see a corpse discarded in a charnel ground being devoured by crows, hawks, vultures, herons, dogs, tigers, leopards, jackals, and many kinds of little creatures. They’d compare it with their own body: ‘This body is also of that same nature, that same kind, and cannot go beyond that.’ And so they meditate observing an aspect of the body internally … 

That\marginnote{8.5} too is how a mendicant meditates by observing an aspect of the body. 

Furthermore,\marginnote{9.1} suppose they were to see a corpse discarded in a charnel ground, a skeleton with flesh and blood, held together by sinews … 

A\marginnote{9.2} skeleton without flesh but smeared with blood, and held together by sinews … 

A\marginnote{9.3} skeleton rid of flesh and blood, held together by sinews … 

Bones\marginnote{9.4} rid of sinews, scattered in every direction. Here a hand-bone, there a foot-bone, here a shin-bone, there a thigh-bone, here a hip-bone, there a rib-bone, here a back-bone, there an arm-bone, here a neck-bone, there a jaw-bone, here a tooth, there the skull … 

White\marginnote{10.1} bones, the color of shells … 

Decrepit\marginnote{10.2} bones, heaped in a pile … 

Bones\marginnote{10.3} rotted and crumbled to powder. They’d compare it with their own body: ‘This body is also of that same nature, that same kind, and cannot go beyond that.’ And so they meditate observing an aspect of the body internally, externally, and both internally and externally. They meditate observing the body as liable to originate, as liable to vanish, and as liable to both originate and vanish. Or mindfulness is established that the body exists, to the extent necessary for knowledge and mindfulness. They meditate independent, not grasping at anything in the world. 

That\marginnote{10.9} too is how a mendicant meditates by observing an aspect of the body. 

\section*{2. Observing the Feelings }

And\marginnote{11.1} how does a mendicant meditate observing an aspect of feelings? 

It’s\marginnote{11.2} when a mendicant who feels a pleasant feeling knows: ‘I feel a pleasant feeling.’ 

When\marginnote{11.3} they feel a painful feeling, they know: ‘I feel a painful feeling.’ 

When\marginnote{11.4} they feel a neutral feeling, they know: ‘I feel a neutral feeling.’ 

When\marginnote{11.5} they feel a material pleasant feeling, they know: ‘I feel a material pleasant feeling.’ 

When\marginnote{11.6} they feel a spiritual pleasant feeling, they know: ‘I feel a spiritual pleasant feeling.’ 

When\marginnote{11.7} they feel a material painful feeling, they know: ‘I feel a material painful feeling.’ 

When\marginnote{11.8} they feel a spiritual painful feeling, they know: ‘I feel a spiritual painful feeling.’ 

When\marginnote{11.9} they feel a material neutral feeling, they know: ‘I feel a material neutral feeling.’ 

When\marginnote{11.10} they feel a spiritual neutral feeling, they know: ‘I feel a spiritual neutral feeling.’ 

And\marginnote{11.11} so they meditate observing an aspect of feelings internally, externally, and both internally and externally. They meditate observing feelings as liable to originate, as liable to vanish, and as liable to both originate and vanish. Or mindfulness is established that feelings exist, to the extent necessary for knowledge and mindfulness. They meditate independent, not grasping at anything in the world. 

That’s\marginnote{11.14} how a mendicant meditates by observing an aspect of feelings. 

\section*{3. Observing the Mind }

And\marginnote{12.1} how does a mendicant meditate observing an aspect of the mind? 

It’s\marginnote{12.2} when a mendicant understands mind with greed as ‘mind with greed,’ and mind without greed as ‘mind without greed.’ They understand mind with hate as ‘mind with hate,’ and mind without hate as ‘mind without hate.’ They understand mind with delusion as ‘mind with delusion,’ and mind without delusion as ‘mind without delusion.’ They know constricted mind as ‘constricted mind,’ and scattered mind as ‘scattered mind.’ They know expansive mind as ‘expansive mind,’ and unexpansive mind as ‘unexpansive mind.’ They know mind that is not supreme as ‘mind that is not supreme,’ and mind that is supreme as ‘mind that is supreme.’ They know mind immersed in \textsanskrit{samādhi} as ‘mind immersed in meditation,’ and mind not immersed in \textsanskrit{samādhi} as ‘mind not immersed in meditation.’ They know freed mind as ‘freed mind,’ and unfreed mind as ‘unfreed mind.’ 

And\marginnote{12.18} so they meditate observing an aspect of the mind internally, externally, and both internally and externally. They meditate observing the mind as liable to originate, as liable to vanish, and as liable to both originate and vanish. Or mindfulness is established that the mind exists, to the extent necessary for knowledge and mindfulness. They meditate independent, not grasping at anything in the world. 

That’s\marginnote{12.21} how a mendicant meditates by observing an aspect of the mind. 

\section*{4. Observing Principles }

\subsection*{4.1. The Hindrances }

And\marginnote{13.1} how does a mendicant meditate observing an aspect of principles? 

It’s\marginnote{13.2} when a mendicant meditates by observing an aspect of principles with respect to the five hindrances. And how does a mendicant meditate observing an aspect of principles with respect to the five hindrances? 

It’s\marginnote{13.4} when a mendicant who has sensual desire in them understands: ‘I have sensual desire in me.’ When they don’t have sensual desire in them, they understand: ‘I don’t have sensual desire in me.’ They understand how sensual desire arises; how, when it’s already arisen, it’s given up; and how, once it’s given up, it doesn’t arise again in the future. 

When\marginnote{13.5} they have ill will in them, they understand: ‘I have ill will in me.’ When they don’t have ill will in them, they understand: ‘I don’t have ill will in me.’ They understand how ill will arises; how, when it’s already arisen, it’s given up; and how, once it’s given up, it doesn’t arise again in the future. 

When\marginnote{13.6} they have dullness and drowsiness in them, they understand: ‘I have dullness and drowsiness in me.’ When they don’t have dullness and drowsiness in them, they understand: ‘I don’t have dullness and drowsiness in me.’ They understand how dullness and drowsiness arise; how, when they’ve already arisen, they’re given up; and how, once they’re given up, they don’t arise again in the future. 

When\marginnote{13.7} they have restlessness and remorse in them, they understand: ‘I have restlessness and remorse in me.’ When they don’t have restlessness and remorse in them, they understand: ‘I don’t have restlessness and remorse in me.’ They understand how restlessness and remorse arise; how, when they’ve already arisen, they’re given up; and how, once they’re given up, they don’t arise again in the future. 

When\marginnote{13.8} they have doubt in them, they understand: ‘I have doubt in me.’ When they don’t have doubt in them, they understand: ‘I don’t have doubt in me.’ They understand how doubt arises; how, when it’s already arisen, it’s given up; and how, once it’s given up, it doesn’t arise again in the future. 

And\marginnote{13.9} so they meditate observing an aspect of principles internally, externally, and both internally and externally. They meditate observing the principles as liable to originate, as liable to vanish, and as liable to both originate and vanish. Or mindfulness is established that principles exist, to the extent necessary for knowledge and mindfulness. They meditate independent, not grasping at anything in the world. 

That’s\marginnote{13.12} how a mendicant meditates by observing an aspect of principles with respect to the five hindrances. 

\subsection*{4.2. The Aggregates }

Furthermore,\marginnote{14.1} a mendicant meditates by observing an aspect of principles with respect to the five grasping aggregates. And how does a mendicant meditate observing an aspect of principles with respect to the five grasping aggregates? 

It’s\marginnote{14.3} when a mendicant contemplates: Such is form, such is the origin of form, such is the ending of form. Such is feeling, such is the origin of feeling, such is the ending of feeling. Such is perception, such is the origin of perception, such is the ending of perception. Such are choices, such is the origin of choices, such is the ending of choices. Such is consciousness, such is the origin of consciousness, such is the ending of consciousness.’ And so they meditate observing an aspect of principles internally … 

That’s\marginnote{14.12} how a mendicant meditates by observing an aspect of principles with respect to the five grasping aggregates. 

\subsection*{4.3. The Sense Fields }

Furthermore,\marginnote{15.1} a mendicant meditates by observing an aspect of principles with respect to the six interior and exterior sense fields. And how does a mendicant meditate observing an aspect of principles with respect to the six interior and exterior sense fields? 

It’s\marginnote{15.3} when a mendicant understands the eye, sights, and the fetter that arises dependent on both of these. They understand how the fetter that has not arisen comes to arise; how the arisen fetter comes to be abandoned; and how the abandoned fetter comes to not rise again in the future. 

They\marginnote{15.4} understand the ear, sounds, and the fetter … 

They\marginnote{15.5} understand the nose, smells, and the fetter … 

They\marginnote{15.6} understand the tongue, tastes, and the fetter … 

They\marginnote{15.7} understand the body, touches, and the fetter … 

They\marginnote{15.8} understand the mind, thoughts, and the fetter that arises dependent on both of these. They understand how the fetter that has not arisen comes to arise; how the arisen fetter comes to be abandoned; and how the abandoned fetter comes to not rise again in the future. 

And\marginnote{15.9} so they meditate observing an aspect of principles internally … 

That’s\marginnote{15.12} how a mendicant meditates by observing an aspect of principles with respect to the six internal and external sense fields. 

\subsection*{4.4. The Awakening Factors }

Furthermore,\marginnote{16.1} a mendicant meditates by observing an aspect of principles with respect to the seven awakening factors. And how does a mendicant meditate observing an aspect of principles with respect to the seven awakening factors? 

It’s\marginnote{16.3} when a mendicant who has the awakening factor of mindfulness in them understands: ‘I have the awakening factor of mindfulness in me.’ When they don’t have the awakening factor of mindfulness in them, they understand: ‘I don’t have the awakening factor of mindfulness in me.’ They understand how the awakening factor of mindfulness that has not arisen comes to arise; and how the awakening factor of mindfulness that has arisen becomes fulfilled by development. 

When\marginnote{16.4} they have the awakening factor of investigation of principles … energy … rapture … tranquility … immersion … equanimity in them, they understand: ‘I have the awakening factor of equanimity in me.’ When they don’t have the awakening factor of equanimity in them, they understand: ‘I don’t have the awakening factor of equanimity in me.’ They understand how the awakening factor of equanimity that has not arisen comes to arise; and how the awakening factor of equanimity that has arisen becomes fulfilled by development. 

And\marginnote{16.10} so they meditate observing an aspect of principles internally, externally, and both internally and externally. They meditate observing the principles as liable to originate, as liable to vanish, and as liable to both originate and vanish. Or mindfulness is established that principles exist, to the extent necessary for knowledge and mindfulness. They meditate independent, not grasping at anything in the world. 

That’s\marginnote{16.13} how a mendicant meditates by observing an aspect of principles with respect to the seven awakening factors. 

\subsection*{4.5. The Truths }

Furthermore,\marginnote{17.1} a mendicant meditates by observing an aspect of principles with respect to the four noble truths. And how does a mendicant meditate observing an aspect of principles with respect to the four noble truths? 

It’s\marginnote{17.3} when a mendicant truly understands: ‘This is suffering’ … ‘This is the origin of suffering’ … ‘This is the cessation of suffering’ … ‘This is the practice that leads to the cessation of suffering.’ 

\scendsection{The first recitation section is finished. }

\subsubsection*{4.5.1. The Truth of Suffering }

And\marginnote{18.1} what is the noble truth of suffering? 

Rebirth\marginnote{18.2} is suffering; old age is suffering; death is suffering; sorrow, lamentation, pain, sadness, and distress are suffering; association with the disliked is suffering; separation from the liked is suffering; not getting what you wish for is suffering. In brief, the five grasping aggregates are suffering. 

And\marginnote{18.3} what is rebirth? The rebirth, inception, conception, reincarnation, manifestation of the sets of phenomena, and acquisition of the sense fields of the various sentient beings in the various orders of sentient beings. This is called rebirth. 

And\marginnote{18.6} what is old age? The old age, decrepitude, broken teeth, grey hair, wrinkly skin, diminished vitality, and failing faculties of the various sentient beings in the various orders of sentient beings. This is called old age. 

And\marginnote{18.9} what is death? The passing away, perishing, disintegration, demise, mortality, death, decease, breaking up of the aggregates, laying to rest of the corpse, and cutting off of the life faculty of the various sentient beings in the various orders of sentient beings. This is called death. 

And\marginnote{18.12} what is sorrow? The sorrow, sorrowing, state of sorrow, inner sorrow, inner deep sorrow in someone who has undergone misfortune, who has experienced suffering. This is called sorrow. 

And\marginnote{18.15} what is lamentation? The wail, lament, wailing, lamenting, state of wailing and lamentation in someone who has undergone misfortune, who has experienced suffering. This is called lamentation. 

And\marginnote{18.18} what is pain? Physical pain, physical displeasure, the painful, unpleasant feeling that’s born from physical contact. This is called pain. 

And\marginnote{18.21} what is sadness? Mental pain, mental displeasure, the painful, unpleasant feeling that’s born from mental contact. This is called sadness. 

And\marginnote{18.24} what is distress? The stress, distress, state of stress and distress in someone who has undergone misfortune, who has experienced suffering. This is called distress. 

And\marginnote{18.27} what is meant by ‘association with the disliked is suffering’? There are sights, sounds, smells, tastes, touches, and thoughts that are unlikable, undesirable, and disagreeable. And there are those who want to harm, injure, disturb, and threaten you. The coming together with these, the joining, inclusion, mixing with them: this is what is meant by ‘association with the disliked is suffering’. 

And\marginnote{18.30} what is meant by ‘separation from the liked is suffering’? There are sights, sounds, smells, tastes, touches, and thoughts that are likable, desirable, and agreeable. And there are those who want to benefit, help, comfort, and protect you. The division from these, the disconnection, segregation, and parting from them: this is what is meant by ‘separation from the liked is suffering’. 

And\marginnote{18.33} what is meant by ‘not getting what you wish for is suffering’? In sentient beings who are liable to be reborn, such a wish arises: ‘Oh, if only we were not liable to be reborn! If only rebirth would not come to us!’ But you can’t get that by wishing. This is what is meant by ‘not getting what you wish for is suffering.’ In sentient beings who are liable to grow old … fall ill … die … experience sorrow, lamentation, pain, sadness, and distress, such a wish arises: ‘Oh, if only we were not liable to experience sorrow, lamentation, pain, sadness, and distress! If only sorrow, lamentation, pain, sadness, and distress would not come to us!’ But you can’t get that by wishing. This is what is meant by ‘not getting what you wish for is suffering.’ 

And\marginnote{18.48} what is meant by ‘in brief, the five grasping aggregates are suffering’? They are the grasping aggregates that consist of form, feeling, perception, choices, and consciousness. This is what is meant by ‘in brief, the five grasping aggregates are suffering’. 

This\marginnote{18.51} is called the noble truth of suffering. 

\subsubsection*{4.5.2. The Origin of Suffering }

And\marginnote{19.1} what is the noble truth of the origin of suffering? 

It’s\marginnote{19.2} the craving that leads to future lives, mixed up with relishing and greed, chasing pleasure in various realms. That is, craving for sensual pleasures, craving for continued existence, and craving to end existence. 

But\marginnote{19.4} where does that craving arise and where does it settle? Whatever in the world seems nice and pleasant, it is there that craving arises and settles. 

And\marginnote{19.6} what in the world seems nice and pleasant? The eye in the world seems nice and pleasant, and it is there that craving arises and settles. The ear … nose … tongue … body … mind in the world seems nice and pleasant, and it is there that craving arises and settles. 

Sights\marginnote{19.13} … sounds … smells … tastes … touches … thoughts in the world seem nice and pleasant, and it is there that craving arises and settles. 

Eye\marginnote{19.19} consciousness … ear consciousness … nose consciousness … tongue consciousness … body consciousness … mind consciousness in the world seems nice and pleasant, and it is there that craving arises and settles. 

Eye\marginnote{19.25} contact … ear contact … nose contact … tongue contact … body contact … mind contact in the world seems nice and pleasant, and it is there that craving arises and settles. 

Feeling\marginnote{19.31} born of eye contact … feeling born of ear contact … feeling born of nose contact … feeling born of tongue contact … feeling born of body contact … feeling born of mind contact in the world seems nice and pleasant, and it is there that craving arises and settles. 

Perception\marginnote{19.37} of sights … perception of sounds … perception of smells … perception of tastes … perception of touches … perception of thoughts in the world seems nice and pleasant, and it is there that craving arises and settles. 

Intention\marginnote{19.43} regarding sights … intention regarding sounds … intention regarding smells … intention regarding tastes … intention regarding touches … intention regarding thoughts in the world seems nice and pleasant, and it is there that craving arises and settles. 

Craving\marginnote{19.49} for sights … craving for sounds … craving for smells … craving for tastes … craving for touches … craving for thoughts in the world seems nice and pleasant, and it is there that craving arises and settles. 

Thoughts\marginnote{19.55} about sights … thoughts about sounds … thoughts about smells … thoughts about tastes … thoughts about touches … thoughts about thoughts in the world seem nice and pleasant, and it is there that craving arises and settles. 

Considerations\marginnote{19.61} regarding sights … considerations regarding sounds … considerations regarding smells … considerations regarding tastes … considerations regarding touches … considerations regarding thoughts in the world seem nice and pleasant, and it is there that craving arises and settles. 

This\marginnote{19.67} is called the noble truth of the origin of suffering. 

\subsubsection*{4.5.3. The Cessation of Suffering }

And\marginnote{20.1} what is the noble truth of the cessation of suffering? 

It’s\marginnote{20.2} the fading away and cessation of that very same craving with nothing left over; giving it away, letting it go, releasing it, and not adhering to it. 

Whatever\marginnote{20.3} in the world seems nice and pleasant, it is there that craving is given up and ceases. 

And\marginnote{20.5} what in the world seems nice and pleasant? The eye in the world seems nice and pleasant, and it is there that craving is given up and ceases. … 

Considerations\marginnote{20.60} regarding thoughts in the world seem nice and pleasant, and it is there that craving is given up and ceases. 

This\marginnote{20.66} is called the noble truth of the cessation of suffering. 

\subsubsection*{4.5.4. The Path }

And\marginnote{21.1} what is the noble truth of the practice that leads to the cessation of suffering? 

It\marginnote{21.2} is simply this noble eightfold path, that is: right view, right thought, right speech, right action, right livelihood, right effort, right mindfulness, and right immersion. 

And\marginnote{21.4} what is right view? Knowing about suffering, the origin of suffering, the cessation of suffering, and the practice that leads to the cessation of suffering. This is called right view. 

And\marginnote{21.7} what is right thought? Thoughts of renunciation, good will, and harmlessness. This is called right thought. 

And\marginnote{21.10} what is right speech? The refraining from lying, divisive speech, harsh speech, and talking nonsense. This is called right speech. 

And\marginnote{21.13} what is right action? Refraining from killing living creatures, stealing, and sexual misconduct. This is called right action. 

And\marginnote{21.16} what is right livelihood? It’s when a noble disciple gives up wrong livelihood and earns a living by right livelihood. This is called right livelihood. 

And\marginnote{21.19} what is right effort? It’s when a mendicant generates enthusiasm, tries, makes an effort, exerts the mind, and strives so that bad, unskillful qualities don’t arise. They generate enthusiasm, try, make an effort, exert the mind, and strive so that bad, unskillful qualities that have arisen are given up. They generate enthusiasm, try, make an effort, exert the mind, and strive so that skillful qualities arise. They generate enthusiasm, try, make an effort, exert the mind, and strive so that skillful qualities that have arisen remain, are not lost, but increase, mature, and are completed by development. This is called right effort. 

And\marginnote{21.25} what is right mindfulness? It’s when a mendicant meditates by observing an aspect of the body—keen, aware, and mindful, rid of desire and aversion for the world. They meditate observing an aspect of feelings—keen, aware, and mindful, rid of desire and aversion for the world. They meditate observing an aspect of the mind—keen, aware, and mindful, rid of desire and aversion for the world. They meditate observing an aspect of principles—keen, aware, and mindful, rid of desire and aversion for the world. This is called right mindfulness. 

And\marginnote{21.31} what is right immersion? It’s when a mendicant, quite secluded from sensual pleasures, secluded from unskillful qualities, enters and remains in the first absorption, which has the rapture and bliss born of seclusion, while placing the mind and keeping it connected. As the placing of the mind and keeping it connected are stilled, they enter and remain in the second absorption, which has the rapture and bliss born of immersion, with internal clarity and confidence, and unified mind, without placing the mind and keeping it connected. And with the fading away of rapture, they enter and remain in the third absorption, where they meditate with equanimity, mindful and aware, personally experiencing the bliss of which the noble ones declare, ‘Equanimous and mindful, one meditates in bliss.’ Giving up pleasure and pain, and ending former happiness and sadness, they enter and remain in the fourth absorption, without pleasure or pain, with pure equanimity and mindfulness. This is called right immersion. 

This\marginnote{21.37} is called the noble truth of the practice that leads to the cessation of suffering. 

And\marginnote{21.38} so they meditate observing an aspect of principles internally, externally, and both internally and externally. They meditate observing the principles as liable to originate, as liable to vanish, and as liable to both originate and vanish. Or mindfulness is established that principles exist, to the extent necessary for knowledge and mindfulness. They meditate independent, not grasping at anything in the world. 

That’s\marginnote{21.41} how a mendicant meditates by observing an aspect of principles with respect to the four noble truths. 

Anyone\marginnote{22.1} who develops these four kinds of mindfulness meditation in this way for seven years can expect one of two results: enlightenment in the present life, or if there’s something left over, non-return. 

Let\marginnote{22.3} alone seven years, anyone who develops these four kinds of mindfulness meditation in this way for six years … five years … four years … three years … two years … one year … seven months … six months … five months … four months … three months … two months … one month … a fortnight … Let alone a fortnight, anyone who develops these four kinds of mindfulness meditation in this way for seven days can expect one of two results: enlightenment in the present life, or if there’s something left over, non-return. 

‘The\marginnote{22.24} four kinds of mindfulness meditation are the path to convergence. They are in order to purify sentient beings, to get past sorrow and crying, to make an end of pain and sadness, to end the cycle of suffering, and to realize extinguishment.’ That’s what I said, and this is why I said it.” 

That\marginnote{22.26} is what the Buddha said. Satisfied, the mendicants were happy with what the Buddha said. 

%
\chapter*{{\suttatitleacronym DN 23}{\suttatitletranslation With Pāyāsi }{\suttatitleroot Pāyāsisutta}}
\addcontentsline{toc}{chapter}{\tocacronym{DN 23} \toctranslation{With Pāyāsi } \tocroot{Pāyāsisutta}}
\markboth{With Pāyāsi }{Pāyāsisutta}
\extramarks{DN 23}{DN 23}

\scevam{So\marginnote{1.1} I have heard. }At one time Venerable Kassapa the Prince was wandering in the land of the Kosalans together with a large \textsanskrit{Saṅgha} of five hundred mendicants when he arrived at a Kosalan citadel named \textsanskrit{Setavyā}. He stayed in the grove of Indian Rosewood to the north of \textsanskrit{Setavyā}. 

Now\marginnote{1.4} at that time the chieftain \textsanskrit{Pāyāsi} was living in \textsanskrit{Setavyā}. It was a crown property given by King Pasenadi of Kosala, teeming with living creatures, full of hay, wood, water, and grain, a royal endowment of the highest quality. 

\section*{1. On \textsanskrit{Pāyāsi} }

Now\marginnote{2.1} at that time \textsanskrit{Pāyāsi} had the following harmful misconception: “There’s no afterlife. No beings are reborn spontaneously. There’s no fruit or result of good and bad deeds.” 

The\marginnote{2.3} brahmins and householders of \textsanskrit{Setavyā} heard, “It seems the ascetic Kassapa the Prince—a disciple of the ascetic Gotama—is staying in the grove of Indian Rosewood to the north of \textsanskrit{Setavyā}. He has this good reputation: ‘He is astute, competent, intelligent, learned, a brilliant speaker, eloquent, mature, a perfected one.’ It’s good to see such perfected ones.” Then, having departed \textsanskrit{Setavyā}, they formed into companies and headed north to the grove. 

Now\marginnote{3.1} at that time the chieftain \textsanskrit{Pāyāsi} had retired to the upper floor of his stilt longhouse for his midday nap. He saw the brahmins and householders heading north towards the grove, and addressed his steward, “My steward, why are the brahmins and householders heading north towards the grove?” 

“The\marginnote{3.5} ascetic Kassapa the Prince—a disciple of the ascetic Gotama—is staying in the grove of Indian Rosewood to the north of \textsanskrit{Setavyā}. He has this good reputation: ‘He is astute, competent, intelligent, learned, a brilliant speaker, eloquent, mature, a perfected one.’ They’re going to see that Kassapa the Prince.” 

“Well\marginnote{3.9} then, go to the brahmins and householders and say to them: ‘Sirs, the chieftain \textsanskrit{Pāyāsi} asks you to wait, as he will also go to see the ascetic Kassapa the Prince.’ Before Kassapa the Prince convinces those foolish and incompetent brahmins and householders that there is an afterlife, there are beings reborn spontaneously, and there is a fruit or result of good and bad deeds—for none of these things are true!” 

“Yes,\marginnote{3.15} sir,” replied the steward, and did as he was asked. 

Then\marginnote{4.1} \textsanskrit{Pāyāsi} escorted by the brahmins and householders, went up to Kassapa the Prince, and exchanged greetings with him. When the greetings and polite conversation were over, he sat down to one side. Before sitting down to one side, some of the brahmins and householders of \textsanskrit{Setavyā} bowed, some exchanged greetings and polite conversation, some held up their joined palms toward Kassapa the Prince, some announced their name and clan, while some kept silent. 

\section*{2. Nihilism }

Seated\marginnote{5.1} to one side, the chieftain \textsanskrit{Pāyāsi} said to Venerable Kassapa the Prince, “Master Kassapa, this is my doctrine and view: ‘There’s no afterlife. No beings are reborn spontaneously. There’s no fruit or result of good and bad deeds.’” 

“Well,\marginnote{5.4} chieftain, I’ve never seen or heard of anyone holding such a doctrine or view. For how on earth can anyone say such a thing? 

\subsection*{2.1. The Simile of the Moon and Sun }

Well\marginnote{5.8} then, chieftain, I’ll ask you about this in return, and you can answer as you like. What do you think, chieftain? Are the moon and sun in this world or the other world? Are they gods or humans?” 

“They\marginnote{5.11} are in the other world, Master Kassapa, and they are gods, not humans.” 

“By\marginnote{5.12} this method it ought to be proven that there is an afterlife, there are beings reborn spontaneously, and there is a fruit or result of good and bad deeds.” 

“Even\marginnote{6.1} though Master Kassapa says this, still I think that there’s no afterlife, no beings are reborn spontaneously, and there’s no fruit or result of good and bad deeds.” 

“Is\marginnote{6.3} there a method by which you can prove what you say?” 

“There\marginnote{6.5} is, Master Kassapa.” 

“How,\marginnote{6.7} exactly, chieftain?” 

“Well,\marginnote{6.8} I have friends and colleagues, relatives and kin who kill living creatures, steal, and commit sexual misconduct. They use speech that’s false, divisive, harsh, or nonsensical. And they’re covetous, malicious, with wrong view. Some time later they become sick, suffering, gravely ill. When I know that they will not recover from their illness, I go to them and say, ‘Sirs, there are some ascetics and brahmins who have this doctrine and view: “Those who kill living creatures, steal, and commit sexual misconduct; use speech that’s false, divisive, harsh, or nonsensical; and are covetous, malicious, and have wrong view—when their body breaks up, after death, are reborn in a place of loss, a bad place, the underworld, hell.” You do all these things. If what those ascetics and brahmins say is true, when your body breaks up, after death, you’ll be reborn in a place of loss, a bad place, the underworld, hell. If that happens, sirs, come and tell me that there is an afterlife, there are beings reborn spontaneously, and there is a fruit or result of good and bad deeds. I trust you and believe you. Anything you see will be just as if I’ve seen it for myself.’ They agree to this. But they don’t come back to tell me, nor do they send a messenger. This is the method by which I prove that there’s no afterlife, no beings are reborn spontaneously, and there’s no fruit or result of good and bad deeds.” 

\subsection*{2.2. The Simile of the Bandit }

“Well\marginnote{7.1} then, chieftain, I’ll ask you about this in return, and you can answer as you like. What do you think, chieftain? Suppose they were to arrest a bandit, a criminal and present him to you, saying, ‘Sir, this is a bandit, a criminal. Punish him as you will.’ Then you’d say to them, ‘Well then, my men, tie this man’s arms tightly behind his back with a strong rope. Shave his head and march him from street to street and square to square to the beating of a harsh drum. Then take him out the south gate and there, at the place of execution to the south of the city, chop off his head.’ Saying, ‘Good,’ they’d do as they were told, sitting him down at the place of execution. Could that bandit get the executioners to wait, saying, ‘Please, good executioners! I have friends and colleagues, relatives and kin in such and such village or town. Wait until I’ve visited them, then I’ll come back’? Or would they just chop off his head as he prattled on?” 

“They’d\marginnote{7.13} just chop off his head.” 

“So\marginnote{7.14} even a human bandit couldn’t get his human executioners to stay his execution. What then of your friends and colleagues, relatives and kin who are reborn in a lower realm after doing bad things? Could they get the wardens of hell to wait, saying, ‘Please, good wardens of hell! Wait until I’ve gone to the chieftain \textsanskrit{Pāyāsi} to tell him that there is an afterlife, there are beings reborn spontaneously, and there is a fruit or result of good and bad deeds’? By this method, too, it ought to be proven that there is an afterlife, there are beings reborn spontaneously, and there is a fruit or result of good and bad deeds.” 

“Even\marginnote{8.1} though Master Kassapa says this, still I think that there’s no afterlife.” 

“Is\marginnote{8.3} there a method by which you can prove what you say?” 

“There\marginnote{8.5} is, Master Kassapa.” 

“How,\marginnote{8.7} exactly, chieftain?” 

“Well,\marginnote{8.8} I have friends and colleagues, relatives and kin who refrain from killing living creatures, stealing, and committing sexual misconduct. They refrain from speech that’s false, divisive, harsh, or nonsensical. And they’re content, kind-hearted, with right view. Some time later they become sick, suffering, gravely ill. When I know that they will not recover from their illness, I go to them and say, ‘Sirs, there are some ascetics and brahmins who have this doctrine and view: “Those who refrain from killing living creatures, stealing, and committing sexual misconduct; who refrain from speech that’s false, divisive, harsh, or nonsensical; and are content, kind-hearted, with right view—when their body breaks up, after death, are reborn in a good place, a heavenly realm.” You do all these things. If what those ascetics and brahmins say is true, when your body breaks up, after death, you’ll be reborn in a good place, a heavenly realm. If that happens, sirs, come and tell me that there is an afterlife. I trust you and believe you. Anything you see will be just as if I’ve seen it for myself.’ They agree to this. But they don’t come back to tell me, nor do they send a messenger. This is the method by which I prove that there’s no afterlife.” 

\subsection*{2.3. The Simile of the Sewer }

“Well\marginnote{9.1} then, chieftain, I shall give you a simile. For by means of a simile some sensible people understand the meaning of what is said. Suppose there were a man sunk over his head in a sewer. Then you were to order someone to pull him out of the sewer, and they’d agree to do so. Then you’d tell them to carefully scrape the dung off that man’s body with bamboo scrapers, and they’d agree to do so. Then you’d tell them to carefully scrub that man’s body down with pale clay three times, and they’d do so. Then you’d tell them to smear that man’s body with oil, and carefully wash him down with fine paste three times, and they’d do so. Then you’d tell them to dress that man’s hair and beard, and they’d do so. Then you’d tell them to provide that man with costly garlands, makeup, and clothes, and they’d do so. Then you’d tell them to bring that man up to the stilt longhouse and set him up with the five kinds of sensual stimulation, and they’d do so. 

What\marginnote{9.25} do you think, chieftain? Now that man is nicely bathed and anointed, with hair and beard dressed, bedecked with garlands and bracelets, dressed in white, supplied and provided with the five kinds of sensual stimulation upstairs in the royal longhouse. Would he want to dive back into that sewer again?” 

“No,\marginnote{9.27} Master Kassapa. Why is that? Because that sewer is filthy, stinking, disgusting, and repulsive, and it’s regarded as such.” 

“In\marginnote{9.30} the same way, chieftain, to the gods, human beings are filthy, stinking, disgusting, and repulsive, and are regarded as such. The smell of humans reaches the gods even a hundred leagues away. What then of your friends and colleagues, relatives and kin who are reborn in a higher realm after doing good things? Will they come back to tell you that there is an afterlife? By this method, too, it ought to be proven that there is an afterlife.” 

“Even\marginnote{10.1} though Master Kassapa says this, still I think that there’s no afterlife.” 

“Can\marginnote{10.3} you prove it?” 

“I\marginnote{10.4} can.” 

“How,\marginnote{10.5} exactly, chieftain?” 

“Well,\marginnote{10.6} I have friends and colleagues, relatives and kin who refrain from killing living creatures and so on. Some time later they become sick, suffering, gravely ill. When I know that they will not recover from their illness, I go to them and say, ‘Sirs, there are some ascetics and brahmins who have this doctrine and view: “Those who refrain from killing living creatures and so on are reborn in a good place, a heavenly realm, in the company of the gods of the Thirty-Three.” You do all these things. If what those ascetics and brahmins say is true, when your body breaks up, after death, you’ll be reborn in the company of the gods of the Thirty-Three. If that happens, sirs, come and tell me that there is an afterlife. I trust you and believe you. Anything you see will be just as if I’ve seen it for myself.’ They agree to this. But they don’t come back to tell me, nor do they send a messenger. This is how I prove that there’s no afterlife.” 

\subsection*{2.4. The Simile of the Gods of the Thirty-Three }

“Well\marginnote{11.1} then, chieftain, I’ll ask you about this in return, and you can answer as you like. A hundred human years are equivalent to one day and night for the gods of the Thirty-Three. Thirty such days make a month, and twelve months make a year. The gods of the Thirty-Three have a lifespan of a thousand such years. Now, as to your friends who are reborn in the company of the gods of the Thirty-Three after doing good things. If they think, ‘First I’ll amuse myself for two or three days, supplied and provided with the five kinds of heavenly sensual stimulation. Then I’ll go back to \textsanskrit{Pāyāsi} and tell him that there is an afterlife.’ Would they come back to tell you that there is an afterlife?” 

“No,\marginnote{11.9} Master Kassapa. For I would be long dead by then. But Master Kassapa, who has told you that the gods of the Thirty-Three exist, or that they have such a long life span? I don’t believe you.” 

\subsection*{2.5. Blind From Birth }

“Chieftain,\marginnote{11.16} suppose there was a person blind from birth. They couldn’t see sights that are dark or bright, or blue, yellow, red, or magenta. They couldn’t see even and uneven ground, or the stars, or the moon and sun. They’d say, ‘There’s no such thing as dark and bright sights, and no-one who sees them. There’s no such thing as blue, yellow, red, magenta, even and uneven ground, stars, moon and sun, and no-one who sees these things. I don’t know it or see it, therefore it doesn’t exist.’ Would they be speaking rightly?” 

“No,\marginnote{11.28} Master Kassapa. There are such things as dark and bright sights, and one who sees them. And those other things are real, too, as is the one who sees them. So it’s not right to say this: ‘I don’t know it or see it, therefore it doesn’t exist.’” 

“In\marginnote{11.36} the same way, chieftain, when you tell me you don’t believe me you seem like the blind man in the simile. You can’t see the other world the way you think, with the eye of the flesh. There are ascetics and brahmins who live in the wilderness, frequenting remote lodgings in the wilderness and the forest. Meditating diligent, keen, and resolute, they purify the divine eye, the power of clairvoyance. With clairvoyance that is purified and superhuman, they see this world and the other world, and sentient beings who are spontaneously reborn. That’s how to see the other world, not how you think, with the eye of the flesh. By this method, too, it ought to be proven that there is an afterlife.” 

“Even\marginnote{12.1} though Master Kassapa says this, still I think that there’s no afterlife.” 

“Can\marginnote{12.3} you prove it?” 

“I\marginnote{12.4} can.” 

“How,\marginnote{12.5} exactly, chieftain?” 

“Well,\marginnote{12.6} I see ascetics and brahmins who are ethical, of good character, who want to live and don’t want to die, who want to be happy and recoil from pain. I think to myself, ‘If those ascetics and brahmins knew that things were going to be better for them after death, they’d drink poison, slit their wrists, hang themselves, or throw themselves off a cliff. They mustn’t know that things are going to be better for them after death. That’s why they are ethical, of good character, wanting to live and not wanting to die, wanting to be happy and recoiling from pain.’ This is the method by which I prove that there’s no afterlife.” 

\subsection*{2.6. The Simile of the Pregnant Woman }

“Well\marginnote{13.1} then, chieftain, I shall give you a simile. For by means of a simile some sensible people understand the meaning of what is said. 

Once\marginnote{13.3} upon a time, a certain brahmin had two wives. One had a son ten or twelve years of age, while the other was pregnant and about to give birth. Then the brahmin passed away. 

So\marginnote{13.6} the youth said to his mother’s co-wife, ‘Madam, all the wealth, grain, silver, and gold is mine, and you get nothing. Transfer to me my father’s inheritance.’ 

But\marginnote{13.10} the brahmin lady said, ‘Wait, my dear, until I give birth. If it’s a boy, one portion shall be his. If it’s a girl, she will be your reward.’ 

But\marginnote{13.14} for a second time, and a third time, the youth insisted that the entire inheritance must be his. 

So\marginnote{13.26} the brahmin lady took a knife, went to her bedroom, and sliced open her belly, thinking, ‘Until I give birth—whether it’s a boy or a girl!’ She destroyed her own life and that of the fetus, as well as any wealth. 

Being\marginnote{13.29} foolish and incompetent, she sought an inheritance irrationally and fell to ruin and disaster. In the same way, chieftain, being foolish and incompetent, you’re seeking the other world irrationally and will fall to ruin and disaster, just like that brahmin lady. Good ascetics and brahmins don’t force what is unripe to ripen; rather, they wait for it to ripen. For the life of clever ascetics and brahmins is beneficial. So long as they remain, good ascetics and brahmins make much merit, and act for the welfare and happiness of the people, out of compassion for the world, for the benefit, welfare, and happiness of gods and humans. By this method, too, it ought to be proven that there is an afterlife.” 

“Even\marginnote{14.1} though Master Kassapa says this, still I think that there’s no afterlife.” 

“Can\marginnote{14.3} you prove it?” 

“I\marginnote{14.4} can.” 

“How,\marginnote{14.5} exactly, chieftain?” 

“Suppose\marginnote{14.6} they were to arrest a bandit, a criminal and present him to me, saying, ‘Sir, this is a bandit, a criminal. Punish him as you will.’ I say to them, ‘Well then, sirs, place this man in a pot while he’s still alive. Close up the mouth, bind it up with damp leather, and seal it with a thick coat of damp clay. Then lift it up on a stove and light the fire.’ They agree, and do what I ask. When we know that that man has passed away, we lift down the pot and break it open, uncover the mouth, and slowly peek inside, thinking, ‘Hopefully we’ll see his soul escaping.’ But we don’t see his soul escaping. This is how I prove that there’s no afterlife.” 

\subsection*{2.7. The Simile of the Dream }

“Well\marginnote{15.1} then, chieftain, I’ll ask you about this in return, and you can answer as you like. Do you recall ever having a midday nap and seeing delightful parks, woods, meadows, and lotus ponds in a dream?” 

“I\marginnote{15.3} do, sir.” 

“At\marginnote{15.4} that time were you guarded by hunchbacks, dwarves, midgets, and younglings?” 

“I\marginnote{15.5} was.” 

“But\marginnote{15.6} did they see your soul entering or leaving?” 

“No\marginnote{15.7} they did not.” 

“So\marginnote{15.8} if they couldn’t even see your soul entering or leaving while you were still alive, how could you see the soul of a dead man? By this method, too, it ought to be proven that there is an afterlife, there are beings reborn spontaneously, and there is a fruit or result of good and bad deeds.” 

“Even\marginnote{16.1} though Master Kassapa says this, still I think that there’s no afterlife.” 

“Can\marginnote{16.3} you prove it?” 

“I\marginnote{16.4} can.” 

“How,\marginnote{16.5} exactly, chieftain?” 

“Suppose\marginnote{16.6} they were to arrest a bandit, a criminal and present him to me, saying, ‘Sir, this is a bandit, a criminal. Punish him as you will.’ I say to them, ‘Well then, sirs, weigh this man with scales while he’s still alive. Then strangle him with a bowstring, and when he’s dead, weigh him again.’ They agree, and do what I ask. So long as they are alive, they’re lighter, softer, more flexible. But when they die they become heavier, stiffer, less flexible. This is how I prove that there’s no afterlife.” 

\subsection*{2.8. The Simile of the Hot Iron Ball }

“Well\marginnote{17.1} then, chieftain, I shall give you a simile. For by means of a simile some sensible people understand the meaning of what is said. Suppose a person was to heat an iron ball all day until it was burning, blazing, and glowing, and then they weigh it with scales. After some time, when it had cooled and become extinguished, they’d weigh it again. When would that iron ball be lighter, softer, and more workable—when it’s burning or when it’s cool?” 

“So\marginnote{17.6} long as the iron ball is full of heat and air—burning, blazing, and glowing—it’s lighter, softer, and more workable. But when it lacks heat and air—cooled and extinguished—it’s heavier, stiffer, and less workable.” 

“In\marginnote{17.8} the same way, so long as this body is full of life and warmth and consciousness it’s lighter, softer, and more flexible. But when it lacks life and warmth and consciousness it’s heavier, stiffer, and less flexible. By this method, too, it ought to be proven that there is an afterlife.” 

“Even\marginnote{18.1} though Master Kassapa says this, still I think that there’s no afterlife.” 

“Can\marginnote{18.3} you prove it?” 

“I\marginnote{18.4} can.” 

“How,\marginnote{18.5} exactly, chieftain?” 

“Suppose\marginnote{18.6} they were to arrest a bandit, a criminal and present him to me, saying, ‘Sir, this is a bandit, a criminal. Punish him as you will.’ I say to them, ‘Well then, sirs, take this man’s life without injuring his outer skin, inner skin, flesh, sinews, bones, or marrow. Hopefully we’ll see his soul escaping.’ They agree, and do what I ask. When he’s half-dead, I tell them to lay him on his back in hope of seeing his soul escape. They do so. But we don’t see his soul escaping. I tell them to lay him bent over, to lay him on his side, to lay him on the other side; to stand him upright, to stand him upside down; to strike him with fists, stones, rods, and swords; and to give him a good shaking in hope of seeing his soul escape. They do all these things. But we don’t see his soul escaping. For him the eye itself is present, and so are those sights. Yet he does not experience that sense-field. The ear itself is present, and so are those sounds. Yet he does not experience that sense-field. The nose itself is present, and so are those smells. Yet he does not experience that sense-field. The tongue itself is present, and so are those tastes. Yet he does not experience that sense-field. The body itself is present, and so are those touches. Yet he does not experience that sense-field. This is how I prove that there’s no afterlife.” 

\subsection*{2.9. The Simile of the Horn Blower }

“Well\marginnote{19.1} then, chieftain, I shall give you a simile. For by means of a simile some sensible people understand the meaning of what is said. 

Once\marginnote{19.3} upon a time, a certain horn blower took his horn and traveled to a borderland, where he went to a certain village. Standing in the middle of the village, he sounded his horn three times, then placed it on the ground and sat down to one side. 

Then\marginnote{19.5} the people of the borderland thought, ‘What is making this sound, so arousing, sensuous, intoxicating, infatuating, and captivating?’ They gathered around the horn blower and said, ‘Mister, what is making this sound, so arousing, sensuous, intoxicating, infatuating, and captivating?’ 

‘The\marginnote{19.9} sound is made by this, which is called a horn.’ 

They\marginnote{19.10} laid that horn on its back, saying, ‘Speak, good horn! Speak, good horn!’ But still the horn made no sound. 

Then\marginnote{19.13} they lay the horn bent over, they lay it on its side, they lay it on its other side; they stood it upright, they stood it upside down; they struck it with fists, stones, rods, and swords; and they gave it a good shake, saying, ‘Speak, good horn! Speak, good horn!’ But still the horn made no sound. 

So\marginnote{19.16} the horn blower thought, ‘How foolish are these borderland folk! For how can they seek the sound of a horn so irrationally?’ And as they looked on, he picked up the horn, sounded it three times, and took it away with him. 

Then\marginnote{19.19} the people of the borderland thought, ‘So, it seems, when what is called a horn is accompanied by a person, effort, and wind, it makes a sound. But when these things are absent it makes no sound.’ 

In\marginnote{19.21} the same way, so long as this body is full of life and warmth and consciousness it walks back and forth, stands, sits, and lies down. It sees sights with the eye, hears sounds with the ear, smells odors with the nose, tastes flavors with the tongue, feels touches with the body, and knows thoughts with the mind. But when it lacks life and warmth and consciousness it does none of these things. By this method, too, it ought to be proven that there is an afterlife.” 

“Even\marginnote{20.1} though Master Kassapa says this, still I think that there’s no afterlife.” 

“Can\marginnote{20.3} you prove it?” 

“I\marginnote{20.4} can.” 

“How,\marginnote{20.5} exactly, chieftain?” 

“Suppose\marginnote{20.6} they were to arrest a bandit, a criminal and present him to me, saying, ‘Sir, this is a bandit, a criminal. Punish him as you will.’ I say to them, ‘Well then, sirs, cut open this man’s outer skin. Hopefully we might see his soul.’ They cut open his outer skin, but we see no soul. I say to them, ‘Well then, sirs, cut open his inner skin, flesh, sinews, bones, or marrow. Hopefully we’ll see his soul.’ They do so, but we see no soul. This is how I prove that there’s no afterlife.” 

\subsection*{2.10. The Simile of the Fire-Worshiping Matted-Hair Ascetic }

“Well\marginnote{21.1} then, chieftain, I shall give you a simile. For by means of a simile some sensible people understand the meaning of what is said. 

Once\marginnote{21.3} upon a time, a certain fire-worshiping matted-hair ascetic settled in a leaf hut in a wilderness region. Then a caravan came out from a certain country. It stayed for one night not far from that ascetic’s hermitage, and then moved on. The ascetic thought, ‘Why don’t I go to that caravan’s campsite? Hopefully I’ll find something useful there.’ 

So\marginnote{21.8} he went, and he saw a little baby boy abandoned there. When he saw this he thought, ‘It’s not proper for me to look on while a human being dies. Why don’t I bring this boy back to my hermitage, nurse him, nourish him, and raise him?’ So that’s what he did. 

When\marginnote{21.13} the boy was ten or twelve years old, the ascetic had some business come up in the country. So he said to the boy, ‘My dear, I wish to go to the country. Serve the sacred flame. Do not extinguish it. But if you should extinguish it, here is the hatchet, the firewood, and the bundle of drill-sticks. Light the fire and serve it.’ And having instructed the boy, the ascetic went to the country. 

But\marginnote{21.20} the boy was so engrossed in his play, the fire went out. He thought, ‘My father told me to serve the sacred flame. Why don’t I light it again and serve it?’ 

So\marginnote{21.27} he chopped the bundle of drill-sticks with the hatchet, thinking, ‘Hopefully I’ll get a fire!’ But he still got no fire. 

He\marginnote{21.30} split the bundle of drill-sticks into two, three, four, five, ten, or a hundred parts. He chopped them into splinters, pounded them in a mortar, and swept them away in a strong wind, thinking, ‘Hopefully I’ll get a fire!’ But he still got no fire. 

Then\marginnote{21.33} the matted-hair ascetic, having concluded his business in the country, returned to his own hermitage, and said to the boy, ‘I trust, my dear, that the fire didn’t go out?’ And the boy told him what had happened. Then the ascetic thought, ‘How foolish is this boy, how incompetent! For how can he seek a fire so irrationally?’ 

So\marginnote{21.49} while the boy looked on, he took a bundle of fire-sticks, lit the fire, and said, ‘Dear boy, this is how to light a fire. Not the foolish and incompetent way you sought it so irrationally.’ In the same way, chieftain, being foolish and incompetent, you seek the other world irrationally. Let go of this harmful misconception, chieftain, let go of it! Don’t create lasting harm and suffering for yourself!” 

“Even\marginnote{22.1} though Master Kassapa says this, still I’m not able to let go of that harmful misconception. King Pasenadi of Kosala knows my views, and so do foreign kings. If I let go of this harmful misconception, people will say, ‘How foolish is the chieftain \textsanskrit{Pāyāsi}, how incompetent, that he should hold on to a mistake!’ I shall carry on with this view out of anger, contempt, and spite!” 

\subsection*{2.11. The Simile of the Two Caravan Leaders }

“Well\marginnote{23.1} then, chieftain, I shall give you a simile. For by means of a simile some sensible people understand the meaning of what is said. 

Once\marginnote{23.3} upon a time, a large caravan of a thousand wagons traveled from a country in the east to the west. Wherever they went they quickly used up the grass, wood, water, and the green foliage. Now, that caravan had two leaders, each in charge of five hundred wagons. They thought, ‘This is a large caravan of a thousand wagons. Wherever we go we quickly use up the grass, wood, water, and the green foliage. Why don’t we split the caravan in two halves?’ So that’s what they did. 

One\marginnote{23.12} caravan leader, having prepared much grass, wood, and water, started the caravan. After two or three days’ journey he saw a dark man with red eyes coming the other way in a donkey cart with muddy wheels. He was armored with a quiver and wreathed with yellow lotus, his clothes and hair all wet. Seeing him, he said, ‘Sir, where do you come from?’ 

‘From\marginnote{23.15} such and such a country.’ 

‘And\marginnote{23.16} where are you going?’ 

‘To\marginnote{23.17} the country named so and so.’ 

‘But\marginnote{23.18} has there been much rain in the desert up ahead?’ 

‘Indeed\marginnote{23.19} there has, sir. The paths are sprinkled with water, and there is much grass, wood, and water. Toss out your grass, wood, and water. Your wagons will move swiftly when lightly-laden, so don’t tire your draught teams.’ 

So\marginnote{23.21} the caravan leader addressed his drivers, ‘This man says that there has been much rain in the desert up ahead. He advises us to toss out the grass, wood, and water. The wagons will move swiftly when lightly-laden, and won’t tire our draught teams. So let’s toss out the grass, wood, and water and restart the caravan with lightly-laden wagons.’ 

‘Yes,\marginnote{23.26} sir,’ the drivers replied, and that’s what they did. 

But\marginnote{23.27} in the caravan’s first campsite they saw no grass, wood, or water. And in the second, third, fourth, fifth, sixth, and seventh campsites they saw no grass, wood, or water. And all fell to ruin and disaster. And the men and beasts in that caravan were all devoured by that non-human spirit. Only their bones remained. 

Now,\marginnote{23.37} when the second caravan leader knew that the first caravan was well underway, he prepared much grass, wood, and water and started the caravan. After two or three days’ journey he saw a dark man with red eyes coming the other way in a donkey cart with muddy wheels. He was armored with a quiver and wreathed with yellow lotus, his clothes and hair all wet. Seeing him, he said, ‘Sir, where do you come from?’ 

‘From\marginnote{23.41} such and such a country.’ 

‘And\marginnote{23.42} where are you going?’ 

‘To\marginnote{23.43} the country named so and so.’ 

‘But\marginnote{23.44} has there been much rain in the desert up ahead?’ 

‘Indeed\marginnote{23.45} there has, sir. The paths are sprinkled with water, and there is much grass, wood, and water. Toss out your grass, wood, and water. Your wagons will move swiftly when lightly-laden, so don’t tire your draught teams.’ 

So\marginnote{23.47} the caravan leader addressed his drivers, ‘This man says that there has been much rain in the desert up ahead. He advises us to toss out the grass, wood, and water. The wagons will move swiftly when lightly-laden, and won’t tire our draught teams. But this person is neither our friend nor relative. How can we proceed out of trust in him? We shouldn’t toss out any grass, wood, or water, but continue with our goods laden as before. We shall not toss out any old stock.’ 

‘Yes,\marginnote{23.54} sir,’ the drivers replied, and they restarted the caravan with the goods laden as before. 

And\marginnote{23.55} in the caravan’s first campsite they saw no grass, wood, or water. And in the second, third, fourth, fifth, sixth, and seventh campsites they saw no grass, wood, or water. And they saw the other caravan that had come to ruin. And they saw the bones of the men and beasts who had been devoured by that non-human spirit. 

So\marginnote{23.64} the caravan leader addressed his drivers, ‘This caravan came to ruin, as happens when guided by a foolish caravan leader. Well then, sirs, toss out any of our merchandise that’s of little value, and take what’s valuable from this caravan.’ 

‘Yes,\marginnote{23.67} sir’ replied the drivers, and that’s what they did. They crossed over the desert safely, as happens when guided by an astute caravan leader. 

In\marginnote{23.68} the same way, chieftain, being foolish and incompetent, you will come to ruin seeking the other world irrationally, like the first caravan leader. And those who think you’re worth listening to and trusting will also come to ruin, like the drivers. Let go of this harmful misconception, chieftain, let go of it! Don’t create lasting harm and suffering for yourself!” 

“Even\marginnote{24.1} though Master Kassapa says this, still I’m not able to let go of that harmful misconception. King Pasenadi of Kosala knows my views, and so do foreign kings. I shall carry on with this view out of anger, contempt, and spite!” 

\subsection*{2.12. The Simile of the Dung-Carrier }

“Well\marginnote{25.1} then, chieftain, I shall give you a simile. For by means of a simile some sensible people understand the meaning of what is said. 

Once\marginnote{25.3} upon a time, a certain swineherd went from his own village to another village. There he saw a large pile of dry dung abandoned. He thought, ‘This pile of dry dung can serve as food for my pigs. Why don’t I carry it off?’ So he spread out his upper robe, shoveled the dry dung onto it, tied it up into a bundle, lifted it on to his head, and went on his way. While on his way a large sudden storm poured down. Smeared with leaking, oozing dung down to his fingernails, he kept on carrying the load of dung. 

When\marginnote{25.11} people saw him they said, ‘Have you gone mad, sir? Have you lost your mind? For how can you, smeared with leaking, oozing dung down to your fingernails, keep on carrying that load of dung?’ 

‘You’re\marginnote{25.13} the mad ones, sirs! You’re the ones who’ve lost your minds! For this will serve as food for my pigs.’ 

In\marginnote{25.14} the same way, chieftain, you seem like the dung carrier in the simile. Let go of this harmful misconception, chieftain, let go of it! Don’t create lasting harm and suffering for yourself!” 

“Even\marginnote{26.1} though Master Kassapa says this, still I’m not able to let go of that harmful misconception. King Pasenadi of Kosala knows my views, and so do foreign kings. I shall carry on with this view out of anger, contempt, and spite!” 

\subsection*{2.13. The Simile of the Gamblers }

“Well\marginnote{27.1} then, chieftain, I shall give you a simile. For by means of a simile some sensible people understand the meaning of what is said. 

Once\marginnote{27.3} upon a time, two gamblers were playing with dice. One gambler, every time they made a bad throw, swallowed the dice. 

The\marginnote{27.5} second gambler saw him, and said, ‘Well, my friend, you’ve won it all! Give me the dice, I will offer them as sacrifice.’ 

‘Yes,\marginnote{27.7} my friend,’ the gambler replied, and gave them. 

Having\marginnote{27.8} soaked the dice in poison, the gambler said to the other, ‘Come, my friend, let’s play dice.’ 

‘Yes,\marginnote{27.10} my friend,’ the other gambler replied. 

And\marginnote{27.11} for a second time the gamblers played with dice. And for the second time, every time they made a bad throw, that gambler swallowed the dice. 

The\marginnote{27.13} second gambler saw him, and said, 

\begin{verse}%
‘The\marginnote{27.14} man swallows the dice without realizing \\
they’re smeared with burning poison. \\
Swallow, you bloody cheat, swallow! \\
Soon you’ll know the bitter fruit!’ 

%
\end{verse}

In\marginnote{27.18} the same way, chieftain, you seem like the gambler in the simile. Let go of this harmful misconception, chieftain, let go of it! Don’t create lasting harm and suffering for yourself!” 

“Even\marginnote{28.1} though Master Kassapa says this, still I’m not able to let go of that harmful misconception. King Pasenadi of Kosala knows my views, and so do foreign kings. I shall carry on with this view out of anger, contempt, and spite!” 

\subsection*{2.14. The Simile of the Man Who Carried Hemp }

“Well\marginnote{29.1} then, chieftain, I shall give you a simile. For by means of a simile some sensible people understand the meaning of what is said. 

Once\marginnote{29.3} upon a time, the inhabitants of a certain country emigrated. Then one friend said to another, ‘Come, my friend, let’s go to that country. Hopefully we’ll get some riches there!’ 

‘Yes,\marginnote{29.6} my friend,’ the other replied. 

They\marginnote{29.7} went to that country, and to a certain place in a village. There they saw a pile of abandoned sunn hemp. Seeing it, one friend said to the other, ‘This is a pile of abandoned sunn hemp. Well then, my friend, you make up a bundle of hemp, and I’ll make one too. Let’s both take a bundle of hemp and go on.’ 

‘Yes,\marginnote{29.9} my friend,’ he said. Carrying their bundles of hemp they went to another place in the village. 

There\marginnote{29.10} they saw much sunn hemp thread abandoned. Seeing it, one friend said to the other, ‘This pile of abandoned sunn hemp thread is just what we wanted the hemp for! Well then, my friend, let’s abandon our bundles of hemp, and both take a bundle of hemp thread and go on.’ 

‘I’ve\marginnote{29.13} already carried this bundle of hemp a long way, and it’s well tied up. It’s good enough for me, you understand.’ So one friend abandoned their bundle of hemp and picked up a bundle of hemp thread. 

They\marginnote{29.15} went to another place in the village. There they saw much sunn hemp cloth abandoned. Seeing it, one friend said to the other, ‘This pile of abandoned sunn hemp cloth is just what we wanted the hemp and hemp thread for! Well then, my friend, let’s abandon our bundles, and both take a bundle of hemp cloth and go on.’ 

‘I’ve\marginnote{29.19} already carried this bundle of hemp a long way, and it’s well tied up. It’s good enough for me, you understand.’ So one friend abandoned their bundle of hemp thread and picked up a bundle of hemp cloth. 

They\marginnote{29.21} went to another place in the village. There they saw a pile of flax, and by turn, linen thread, linen cloth, silk, silk thread, silk cloth, iron, copper, tin, lead, silver, and gold abandoned. Seeing it, one friend said to the other, ‘This pile of gold is just what we wanted all those other things for! Well then, my friend, let’s abandon our bundles, and both take a bundle of gold and go on.’ 

‘I’ve\marginnote{29.36} already carried this bundle of hemp a long way, and it’s well tied up. It’s good enough for me, you understand.’ So one friend abandoned their bundle of silver and picked up a bundle of gold. 

Then\marginnote{29.38} they returned to their own village. When one friend returned with a bundle of sunn hemp, they didn’t please their parents, their partners and children, or their friends and colleagues. And they got no pleasure and happiness on that account. But when the other friend returned with a bundle of gold, they pleased their parents, their partners and children, and their friends and colleagues. And they got much pleasure and happiness on that account. 

In\marginnote{29.41} the same way, chieftain, you seem like the hemp-carrier in the simile. Let go of this harmful misconception, chieftain, let go of it! Don’t create lasting harm and suffering for yourself!” 

\section*{3. Going for Refuge }

“I\marginnote{30.1} was delighted and satisfied with your very first simile, Master Kassapa! Nevertheless, I wanted to hear your various solutions to the problem, so I thought I’d oppose you in this way. Excellent, Master Kassapa! Excellent! As if he were righting the overturned, or revealing the hidden, or pointing out the path to the lost, or lighting a lamp in the dark so people with good eyes can see what’s there, Master Kassapa has made the teaching clear in many ways. I go for refuge to Master Gotama, to the teaching, and to the mendicant \textsanskrit{Saṅgha}. From this day forth, may Master Kassapa remember me as a lay follower who has gone for refuge for life. 

Master\marginnote{30.7} Kassapa, I wish to perform a great sacrifice. Please instruct me so it will be for my lasting welfare and happiness.” 

\section*{4. On Sacrifice }

“Chieftain,\marginnote{31.1} take the kind of sacrifice where cattle, goats and sheep, chickens and pigs, and various kinds of creatures are slaughtered. And the recipients have wrong view, wrong thought, wrong speech, wrong action, wrong livelihood, wrong effort, wrong mindfulness, and wrong immersion. That kind of sacrifice is not very fruitful or beneficial or splendid or bountiful. 

Suppose\marginnote{31.2} a farmer was to enter a wood taking seed and plough. And on that barren field, that barren ground, with uncleared stumps he sowed seeds that were broken, spoiled, weather-damaged, infertile, and ill kept. And the heavens don’t provide enough rain when needed. Would those seeds grow, increase, and mature, and would the farmer get abundant fruit?” 

“No,\marginnote{31.6} Master Kassapa.” 

“In\marginnote{31.7} the same way, chieftain, take the kind of sacrifice where cattle, goats and sheep, chickens and pigs, and various kinds of creatures are slaughtered. And the recipients have wrong view, wrong thought, wrong speech, wrong action, wrong livelihood, wrong effort, wrong mindfulness, and wrong immersion. That kind of sacrifice is not very fruitful or beneficial or splendid or bountiful. 

But\marginnote{31.8} take the kind of sacrifice where cattle, goats and sheep, chickens and pigs, and various kinds of creatures are not slaughtered. And the recipients have right view, right thought, right speech, right action, right livelihood, right effort, right mindfulness, and right immersion. That kind of sacrifice is very fruitful and beneficial and splendid and bountiful. 

Suppose\marginnote{31.9} a farmer was to enter a wood taking seed and plough. And on that fertile field, that fertile ground, with well-cleared stumps he sowed seeds that were intact, unspoiled, not weather-damaged, fertile, and well kept. And the heavens provide plenty of rain when needed. Would those seeds grow, increase, and mature, and would the farmer get abundant fruit?” 

“Yes,\marginnote{31.13} Master Kassapa.” 

“In\marginnote{31.14} the same way, chieftain, take the kind of sacrifice where cattle, goats and sheep, chickens and pigs, and various kinds of creatures are not slaughtered. And the recipients have right view, right thought, right speech, right action, right livelihood, right effort, right mindfulness, and right immersion. That kind of sacrifice is very fruitful and beneficial and splendid and bountiful.” 

\section*{5. On the Brahmin Student Uttara }

Then\marginnote{32.1} the chieftain \textsanskrit{Pāyāsi} set up an offering for ascetics and brahmins, for paupers, vagrants, travelers, and beggars. At that offering such food as rough gruel with pickles was given, and heavy clothes with knotted fringes. Now, it was a brahmin student named Uttara who organized that offering. 

When\marginnote{32.4} the offering was over he referred to it like this, “Through this offering may I be together with the chieftain \textsanskrit{Pāyāsi} in this world, but not in the next.” 

\textsanskrit{Pāyāsi}\marginnote{32.6} heard of this, so he summoned Uttara and said, “Is it really true, dear Uttara, that you referred to the offering in this way?” 

“Yes,\marginnote{32.12} sir.” 

“But\marginnote{32.13} why? Don’t we who seek merit expect some result from the offering?” 

“At\marginnote{32.16} your offering such food as rough gruel with pickles was given, which you wouldn’t even want to touch with your foot, much less eat. And also heavy clothes with knotted fringes, which you also wouldn’t want to touch with your foot, much less wear. Sir, you’re dear and beloved to me. But how can I reconcile one so dear with something so disagreeable?” 

“Well\marginnote{32.18} then, dear Uttara, set up an offering with the same kind of food that I eat, and the same kind of clothes that I wear.” 

“Yes,\marginnote{32.20} sir,” replied Uttara, and did so. 

So\marginnote{32.22} the chieftain \textsanskrit{Pāyāsi} gave gifts carelessly, thoughtlessly, not with his own hands, giving the dregs. When his body broke up, after death, he was reborn in company with the gods of the Four Great Kings, in an empty palace of acacia. But the brahmin student Uttara who organized the offering gave gifts carefully, thoughtfully, with his own hands, not giving the dregs. When his body broke up, after death, he was reborn in company with the gods of the Thirty-Three. 

\section*{6. The God \textsanskrit{Pāyāsi} }

Now\marginnote{33.1} at that time Venerable Gavampati would often go to that empty acacia palace for the day’s meditation. Then the god \textsanskrit{Pāyāsi} went up to him, bowed, and stood to one side. Gavampati said to him, “Who are you, reverend?” 

“Sir,\marginnote{33.4} I am the chieftain \textsanskrit{Pāyāsi}.” 

“Didn’t\marginnote{33.5} you have the view that there’s no afterlife, no beings are reborn spontaneously, and there’s no fruit or result of good and bad deeds?” 

“It’s\marginnote{33.7} true, sir, I did have such a view. But Venerable Kassapa the Prince dissuaded me from that harmful misconception.” 

“But\marginnote{33.10} the student named Uttara who organized that offering for you—where has he been reborn?” 

“Sir,\marginnote{33.11} Uttara gave gifts carefully, thoughtfully, with his own hands, not giving the dregs. When his body broke up, after death, he was reborn in company with the gods of the Thirty-Three. But I gave gifts carelessly, thoughtlessly, not with my own hands, giving the dregs. When my body broke up, after death, I was reborn in company with the gods of the Four Great Kings, in an empty palace of acacia. 

So,\marginnote{33.13} sir, when you’ve returned to the human realm, please announce this: ‘Give gifts carefully, thoughtfully, with your own hands, not giving the dregs. The chieftain \textsanskrit{Pāyāsi} gave gifts carelessly, thoughtlessly, not with his own hands, giving the dregs. When his body broke up, after death, he was reborn in company with the gods of the Four Great Kings, in an empty palace of acacia. But the brahmin student Uttara who organized the offering gave gifts carefully, thoughtfully, with his own hands, not giving the dregs. When his body broke up, after death, he was reborn in company with the gods of the Thirty-Three.’” 

So\marginnote{34.1} when Venerable Gavampati returned to the human realm he made that announcement. 

%
\backmatter%
\chapter*{Colophon}
\addcontentsline{toc}{chapter}{Colophon}
\markboth{Colophon}{Colophon}

\section*{The Translator}

Bhikkhu Sujato was born as Anthony Aidan Best on 4/11/1966 in Perth, Western Australia. He grew up in the pleasant suburbs of Mt Lawley and Attadale alongside his sister Nicola, who was the good child. His mother, Margaret Lorraine Huntsman née Pinder, said “he’ll either be a priest or a poet”, while his father, Anthony Thomas Best, advised him to “never do anything for money”. He attended Aquinas College, a Catholic school, where he decided to become an atheist. At the University of WA he studied philosophy, aiming to learn what he wanted to do with his life. Finding that what he wanted to do was play guitar, he dropped out. His main band was named Martha’s Vineyard, which achieved modest success in the indie circuit. 

A seemingly random encounter with a roadside joey took him to Thailand, where he entered his first meditation retreat at Wat Ram Poeng, Chieng Mai in 1992. Feeling the call to the Buddha’s path, he took full ordination in Wat Pa Nanachat in 1994, where his teachers were Ajahn Pasanno and Ajahn Jayasaro. In 1997 he returned to Perth to study with Ajahn Brahm at Bodhinyana Monastery. 

He spent several years practicing in seclusion in Malaysia and Thailand before establishing Santi Forest Monastery in Bundanoon, NSW, in 2003. There he was instrumental in supporting the establishment of the Theravada bhikkhuni order in Australia and advocating for women’s rights. He continues to teach in Australia and globally, with a special concern for the moral implications of climate change and other forms of environmental destruction. He has published a series of books of original and groundbreaking research on early Buddhism. 

In 2005 he founded SuttaCentral together with Rod Bucknell and John Kelly. In 2015, seeing the need for a complete, accurate, plain English translation of the Pali texts, he undertook the task, spending nearly three years in isolation on the isle of Qi Mei off the coast of the nation of Taiwan. He completed the four main \textsanskrit{Nikāyas} in 2018, and the early books of the Khuddaka \textsanskrit{Nikāya} were complete by 2021. All this work is dedicated to the public domain and is entirely free of copyright encumbrance. 

In 2019 he returned to Sydney where he established Lokanta Vihara (The Monastery at the End of the World). 

\section*{Creation Process}

Primary source was the digital \textsanskrit{Mahāsaṅgīti} edition of the Pali \textsanskrit{Tipiṭaka}. Translated from the Pali, with reference to several English translations, especially those of Bhikkhu Bodhi. Older translations by Maurice Walshe and T.W. and C.A.F. Rhys Davids were also consulted.

\section*{The Translation}

This translation was part of a project to translate the four Pali \textsanskrit{Nikāyas} with the following aims: plain, approachable English; consistent terminology; accurate rendition of the Pali; free of copyright. It was made during 2016–2018 while Bhikkhu Sujato was staying in Qimei, Taiwan.

\section*{About SuttaCentral}

SuttaCentral publishes early Buddhist texts. Since 2005 we have provided root texts in Pali, Chinese, Sanskrit, Tibetan, and other languages, parallels between these texts, and translations in many modern languages. We build on the work of generations of scholars, and offer our contribution freely.

SuttaCentral is driven by volunteer contributions, and in addition we employ professional developers. We offer a sponsorship program for high quality translations from the original languages. Financial support for SuttaCentral is handled by the SuttaCentral Development Trust, a charitable trust registered in Australia.

\section*{About Bilara}

“Bilara” means “cat” in Pali, and it is the name of our Computer Assisted Translation (CAT) software. Bilara is a web app that enables translators to translate early Buddhist texts into their own language. These translations are published on SuttaCentral with the root text and translation side by side.

\section*{About SuttaCentral Editions}

The SuttaCentral Editions project makes high quality books from selected Bilara translations. These are published in formats including HTML, EPUB, PDF, and print.

If you want to print any of our Editions, please let us know and we will help prepare a file to your specifications.

%
\end{document}