\documentclass[12pt,openany]{book}%
\usepackage{lastpage}%
%
\usepackage[inner=1in, outer=1in, top=.7in, bottom=1in, papersize={6in,9in}, headheight=13pt]{geometry}
\usepackage{polyglossia}
\usepackage[12pt]{moresize}
\usepackage{soul}%
\usepackage{microtype}
\usepackage{tocbasic}
\usepackage{realscripts}
\usepackage{epigraph}%
\usepackage{setspace}%
\usepackage{sectsty}
\usepackage{fontspec}
\usepackage{marginnote}
\usepackage[bottom]{footmisc}
\usepackage{enumitem}
\usepackage{fancyhdr}
\usepackage{extramarks}
\usepackage{graphicx}
\usepackage{verse}
\usepackage{relsize}
\usepackage{etoolbox}
\usepackage[a-3u]{pdfx}

\hypersetup{
colorlinks=true,
urlcolor=black,
linkcolor=black,
citecolor=black
}

% use a small amount of tracking on small caps
\SetTracking[ spacing = {25*,166, } ]{ encoding = *, shape = sc }{ 25 }

% add a blank page
\newcommand{\blankpage}{
\newpage
\thispagestyle{empty}
\mbox{}
\newpage
}

% define languages
\setdefaultlanguage[]{english}
\setotherlanguage[script=Latin]{sanskrit}

%\usepackage{pagegrid}
%\pagegridsetup{top-left, step=.25in}

% define fonts
% use if arno sanskrit is unavailable
%\setmainfont{Gentium Plus}
%\newfontfamily\Semiboldsubheadfont[]{Gentium Plus}
%\newfontfamily\Semiboldnormalfont[]{Gentium Plus}
%\newfontfamily\Lightfont[]{Gentium Plus}
%\newfontfamily\Marginalfont[]{Gentium Plus}
%\newfontfamily\Allsmallcapsfont[RawFeature=+c2sc]{Gentium Plus}
%\newfontfamily\Noligaturefont[Renderer=Basic]{Gentium Plus}
%\newfontfamily\Noligaturecaptionfont[Renderer=Basic]{Gentium Plus}
%\newfontfamily\Fleuronfont[Ornament=1]{Gentium Plus}

% use if arno sanskrit is available. display is applied to \chapter and \part, subhead to \section and \subsection. When specifying semibold, the italic must be defined.
\setmainfont[Numbers=OldStyle]{Arno Pro}
\newfontfamily\Semibolddisplayfont[BoldItalicFont = Arno Pro Semibold Italic Display]{Arno Pro Semibold Display} %
\newfontfamily\Semiboldsubheadfont[BoldItalicFont = Arno Pro Semibold Italic Subhead]{Arno Pro Semibold Subhead}
\newfontfamily\Semiboldnormalfont[BoldItalicFont = Arno Pro Semibold Italic]{Arno Pro Semibold}
\newfontfamily\Marginalfont[RawFeature=+subs]{Arno Pro Regular}
\newfontfamily\Allsmallcapsfont[RawFeature=+c2sc]{Arno Pro}
\newfontfamily\Noligaturefont[Renderer=Basic]{Arno Pro}
\newfontfamily\Noligaturecaptionfont[Renderer=Basic]{Arno Pro Caption}

% chinese fonts
\newfontfamily\cjk{Noto Serif TC}
\newcommand*{\langlzh}[1]{\cjk{#1}\normalfont}%

% logo
\newfontfamily\Logofont{sclogo.ttf}
\newcommand*{\sclogo}[1]{\large\Logofont{#1}}

% use subscript numerals for margin notes
\renewcommand*{\marginfont}{\Marginalfont}

% ensure margin notes have consistent vertical alignment
\renewcommand*{\marginnotevadjust}{-.17em}

% use compact lists
\setitemize{noitemsep,leftmargin=1em}
\setenumerate{noitemsep,leftmargin=1em}
\setdescription{noitemsep, style=unboxed, leftmargin=0em}

% style ToC
\DeclareTOCStyleEntries[
  raggedentrytext,
  linefill=\hfill,
  pagenumberwidth=.5in,
  pagenumberformat=\normalfont,
  entryformat=\normalfont
]{tocline}{chapter,section}


  \setlength\topsep{0pt}%
  \setlength\parskip{0pt}%

% define new \centerpars command for use in ToC. This ensures centering, proper wrapping, and no page break after
\def\startcenter{%
  \par
  \begingroup
  \leftskip=0pt plus 1fil
  \rightskip=\leftskip
  \parindent=0pt
  \parfillskip=0pt
}
\def\stopcenter{%
  \par
  \endgroup
}
\long\def\centerpars#1{\startcenter#1\stopcenter}

% redefine part, so that it adds a toc entry without page number
\let\oldcontentsline\contentsline
\newcommand{\nopagecontentsline}[3]{\oldcontentsline{#1}{#2}{}}

    \makeatletter
\renewcommand*\l@part[2]{%
  \ifnum \c@tocdepth >-2\relax
    \addpenalty{-\@highpenalty}%
    \addvspace{0em \@plus\p@}%
    \setlength\@tempdima{3em}%
    \begingroup
      \parindent \z@ \rightskip \@pnumwidth
      \parfillskip -\@pnumwidth
      {\leavevmode
       \setstretch{.85}\large\scshape\centerpars{#1}\vspace*{-1em}\llap{#2}}\par
       \nobreak
         \global\@nobreaktrue
         \everypar{\global\@nobreakfalse\everypar{}}%
    \endgroup
  \fi}
\makeatother

\makeatletter
\def\@pnumwidth{2em}
\makeatother

% define new sectioning command, which is only used in volumes where the pannasa is found in some parts but not others, especially in an and sn

\newcommand*{\pannasa}[1]{\clearpage\thispagestyle{empty}\begin{center}\vspace*{14em}\setstretch{.85}\huge\itshape\scshape\MakeLowercase{#1}\end{center}}

    \makeatletter
\newcommand*\l@pannasa[2]{%
  \ifnum \c@tocdepth >-2\relax
    \addpenalty{-\@highpenalty}%
    \addvspace{.5em \@plus\p@}%
    \setlength\@tempdima{3em}%
    \begingroup
      \parindent \z@ \rightskip \@pnumwidth
      \parfillskip -\@pnumwidth
      {\leavevmode
       \setstretch{.85}\large\itshape\scshape\lowercase{\centerpars{#1}}\vspace*{-1em}\llap{#2}}\par
       \nobreak
         \global\@nobreaktrue
         \everypar{\global\@nobreakfalse\everypar{}}%
    \endgroup
  \fi}
\makeatother

% don't put page number on first page of toc (relies on etoolbox)
\patchcmd{\chapter}{plain}{empty}{}{}

% global line height
\setstretch{1.05}

% allow linebreak after em-dash
\catcode`\—=13
\protected\def—{\unskip\textemdash\allowbreak}

% style headings with secsty. chapter and section are defined per-edition
\partfont{\setstretch{.85}\normalfont\centering\textsc}
\subsectionfont{\setstretch{.85}\Semiboldsubheadfont}%
\subsubsectionfont{\setstretch{.85}\Semiboldnormalfont}

% style elements of suttatitle
\newcommand*{\suttatitleacronym}[1]{\smaller[2]{#1}\vspace*{.3em}}
\newcommand*{\suttatitletranslation}[1]{\linebreak{#1}}
\newcommand*{\suttatitleroot}[1]{\linebreak\smaller[2]\itshape{#1}}

\DeclareTOCStyleEntries[
  indent=3.3em,
  dynindent,
  beforeskip=.2em plus -2pt minus -1pt,
]{tocline}{section}

\DeclareTOCStyleEntries[
  indent=0em,
  dynindent,
  beforeskip=.4em plus -2pt minus -1pt,
]{tocline}{chapter}

\newcommand*{\tocacronym}[1]{\hspace*{-3.3em}{#1}\quad}
\newcommand*{\toctranslation}[1]{#1}
\newcommand*{\tocroot}[1]{(\textit{#1})}
\newcommand*{\tocchapterline}[1]{\bfseries\itshape{#1}}


% redefine paragraph and subparagraph headings to not be inline
\makeatletter
% Change the style of paragraph headings %
\renewcommand\paragraph{\@startsection{paragraph}{4}{\z@}%
            {-2.5ex\@plus -1ex \@minus -.25ex}%
            {1.25ex \@plus .25ex}%
            {\noindent\Semiboldnormalfont\normalsize}}

% Change the style of subparagraph headings %
\renewcommand\subparagraph{\@startsection{subparagraph}{5}{\z@}%
            {-2.5ex\@plus -1ex \@minus -.25ex}%
            {1.25ex \@plus .25ex}%
            {\noindent\Semiboldnormalfont\small}}
\makeatother

% use etoolbox to suppress page numbers on \part
\patchcmd{\part}{\thispagestyle{plain}}{\thispagestyle{empty}}
  {}{\errmessage{Cannot patch \string\part}}

% and to reduce margins on quotation
\patchcmd{\quotation}{\rightmargin}{\leftmargin 1.2em \rightmargin}{}{}
\AtBeginEnvironment{quotation}{\small}

% titlepage
\newcommand*{\titlepageTranslationTitle}[1]{{\begin{center}\begin{large}{#1}\end{large}\end{center}}}
\newcommand*{\titlepageCreatorName}[1]{{\begin{center}\begin{normalsize}{#1}\end{normalsize}\end{center}}}

% halftitlepage
\newcommand*{\halftitlepageTranslationTitle}[1]{\setstretch{2.5}{\begin{Huge}\uppercase{\so{#1}}\end{Huge}}}
\newcommand*{\halftitlepageTranslationSubtitle}[1]{\setstretch{1.2}{\begin{large}{#1}\end{large}}}
\newcommand*{\halftitlepageFleuron}[1]{{\begin{large}\Fleuronfont{{#1}}\end{large}}}
\newcommand*{\halftitlepageByline}[1]{{\begin{normalsize}\textit{{#1}}\end{normalsize}}}
\newcommand*{\halftitlepageCreatorName}[1]{{\begin{LARGE}{\textsc{#1}}\end{LARGE}}}
\newcommand*{\halftitlepageVolumeNumber}[1]{{\begin{normalsize}{\Allsmallcapsfont{\textsc{#1}}}\end{normalsize}}}
\newcommand*{\halftitlepageVolumeAcronym}[1]{{\begin{normalsize}{#1}\end{normalsize}}}
\newcommand*{\halftitlepageVolumeTranslationTitle}[1]{{\begin{Large}{\textsc{#1}}\end{Large}}}
\newcommand*{\halftitlepageVolumeRootTitle}[1]{{\begin{normalsize}{\Allsmallcapsfont{\textsc{\itshape #1}}}\end{normalsize}}}
\newcommand*{\halftitlepagePublisher}[1]{{\begin{large}{\Noligaturecaptionfont\textsc{#1}}\end{large}}}

% epigraph
\renewcommand{\epigraphflush}{center}
\renewcommand*{\epigraphwidth}{.85\textwidth}
\newcommand*{\epigraphTranslatedTitle}[1]{\vspace*{.5em}\footnotesize\textsc{#1}\\}%
\newcommand*{\epigraphRootTitle}[1]{\footnotesize\textit{#1}\\}%
\newcommand*{\epigraphReference}[1]{\footnotesize{#1}}%

% custom commands for html styling classes
\newcommand*{\scnamo}[1]{\begin{center}\textit{#1}\end{center}}
\newcommand*{\scendsection}[1]{\begin{center}\textit{#1}\end{center}}
\newcommand*{\scendsutta}[1]{\begin{center}\textit{#1}\end{center}}
\newcommand*{\scendbook}[1]{\begin{center}\uppercase{#1}\end{center}}
\newcommand*{\scendkanda}[1]{\begin{center}\textbf{#1}\end{center}}
\newcommand*{\scend}[1]{\begin{center}\textit{#1}\end{center}}
\newcommand*{\scuddanaintro}[1]{\textit{#1}}
\newcommand*{\scendvagga}[1]{\begin{center}\textbf{#1}\end{center}}
\newcommand*{\scrule}[1]{\textbf{#1}}
\newcommand*{\scadd}[1]{\textit{#1}}
\newcommand*{\scevam}[1]{\textsc{#1}}
\newcommand*{\scspeaker}[1]{\hspace{2em}\textit{#1}}
\newcommand*{\scbyline}[1]{\begin{flushright}\textit{#1}\end{flushright}\bigskip}

% custom command for thematic break = hr
\newcommand*{\thematicbreak}{\begin{center}\rule[.5ex]{6em}{.4pt}\begin{normalsize}\quad\Fleuronfont{•}\quad\end{normalsize}\rule[.5ex]{6em}{.4pt}\end{center}}

% manage and style page header and footer. "fancy" has header and footer, "plain" has footer only

\pagestyle{fancy}
\fancyhf{}
\fancyfoot[RE,LO]{\thepage}
\fancyfoot[LE,RO]{\footnotesize\lastleftxmark}
\fancyhead[CE]{\setstretch{.85}\Noligaturefont\MakeLowercase{\textsc{\firstrightmark}}}
\fancyhead[CO]{\setstretch{.85}\Noligaturefont\MakeLowercase{\textsc{\firstleftmark}}}
\renewcommand{\headrulewidth}{0pt}
\fancypagestyle{plain}{ %
\fancyhf{} % remove everything
\fancyfoot[RE,LO]{\thepage}
\fancyfoot[LE,RO]{\footnotesize\lastleftxmark}
\renewcommand{\headrulewidth}{0pt}
\renewcommand{\footrulewidth}{0pt}}

% style footnotes
\setlength{\skip\footins}{1em}

\makeatletter
\newcommand{\@makefntextcustom}[1]{%
    \parindent 0em%
    \thefootnote.\enskip #1%
}
\renewcommand{\@makefntext}[1]{\@makefntextcustom{#1}}
\makeatother

% hang quotes (requires microtype)
\microtypesetup{
  protrusion = true,
  expansion  = true,
  tracking   = true,
  factor     = 1000,
  patch      = all,
  final
}

% Custom protrusion rules to allow hanging punctuation
\SetProtrusion
{ encoding = *}
{
% char   right left
  {-} = {    , 500 },
  % Double Quotes
  \textquotedblleft
      = {1000,     },
  \textquotedblright
      = {    , 1000},
  \quotedblbase
      = {1000,     },
  % Single Quotes
  \textquoteleft
      = {1000,     },
  \textquoteright
      = {    , 1000},
  \quotesinglbase
      = {1000,     }
}

% make latex use actual font em for parindent, not Computer Modern Roman
\AtBeginDocument{\setlength{\parindent}{1em}}%
%

% Default values; a bit sloppier than normal
\tolerance 1414
\hbadness 1414
\emergencystretch 1.5em
\hfuzz 0.3pt
\clubpenalty = 10000
\widowpenalty = 10000
\displaywidowpenalty = 10000
\hfuzz \vfuzz
 \raggedbottom%

\title{Long Discourses}
\author{Bhikkhu Sujato}
\date{}%
% define a different fleuron for each edition
\newfontfamily\Fleuronfont[Ornament=16]{Arno Pro}

% Define heading styles per edition for chapter and section. Suttatitle can be either of these, depending on the volume. 

\let\oldfrontmatter\frontmatter
\renewcommand{\frontmatter}{%
\chapterfont{\setstretch{.85}\normalfont\centering}%
\sectionfont{\setstretch{.85}\Semiboldsubheadfont}%
\oldfrontmatter}

\let\oldmainmatter\mainmatter
\renewcommand{\mainmatter}{%
\chapterfont{\setstretch{.85}\normalfont\centering}%
\sectionfont{\setstretch{.85}\Semiboldsubheadfont}%
\oldmainmatter}

\let\oldbackmatter\backmatter
\renewcommand{\backmatter}{%
\chapterfont{\setstretch{.85}\normalfont\centering}%
\sectionfont{\setstretch{.85}\Semiboldsubheadfont}%
\oldbackmatter}
%
%
\begin{document}%
\normalsize%
\frontmatter%
\setlength{\parindent}{0cm}

\pagestyle{empty}

\maketitle

\blankpage%
\begin{center}

\vspace*{2.2em}

\halftitlepageTranslationTitle{Long Discourses}

\vspace*{1em}

\halftitlepageTranslationSubtitle{A faithful translation of the Dīgha Nikāya}

\vspace*{2em}

\halftitlepageFleuron{•}

\vspace*{2em}

\halftitlepageByline{translated and introduced by}

\vspace*{.5em}

\halftitlepageCreatorName{Bhikkhu Sujato}

\vspace*{4em}

\halftitlepageVolumeNumber{Volume 3}

\smallskip

\halftitlepageVolumeAcronym{DN 24–34}

\smallskip

\halftitlepageVolumeTranslationTitle{The Chapter with Pāṭikaputta}

\smallskip

\halftitlepageVolumeRootTitle{Pāthikavagga}

\vspace*{\fill}

\sclogo{0}
 \halftitlepagePublisher{SuttaCentral}

\end{center}

\newpage
%
\setstretch{1.05}

\begin{footnotesize}

\textit{Long Discourses} is a translation of the Dīghanikāya by Bhikkhu Sujato.

\medskip

Creative Commons Zero (CC0)

To the extent possible under law, Bhikkhu Sujato has waived all copyright and related or neighboring rights to \textit{Long Discourses}.

\medskip

This work is published from Australia.

\begin{center}
\textit{This translation is an expression of an ancient spiritual text that has been passed down by the Buddhist tradition for the benefit of all sentient beings. It is dedicated to the public domain via Creative Commons Zero (CC0). You are encouraged to copy, reproduce, adapt, alter, or otherwise make use of this translation. The translator respectfully requests that any use be in accordance with the values and principles of the Buddhist community.}
\end{center}

\medskip

\begin{description}
    \item[Web publication date] 2018
    \item[This edition] 2022-11-08 07:11:28
    \item[Publication type] paperback
    \item[Edition] ed5
    \item[Number of volumes] 3
    \item[Publication ISBN] 978-1-76132-052-1
    \item[Publication URL] https://suttacentral.net/editions/dn/en/sujato
    \item[Source URL] https://github.com/suttacentral/bilara-data/tree/published/translation/en/sujato/sutta/dn
    \item[Publication number] scpub2
\end{description}

\medskip

Published by SuttaCentral

\medskip

\textit{SuttaCentral,\\
c/o Alwis \& Alwis Pty Ltd\\
Kaurna Country,\\
Suite 12,\\
198 Greenhill Road,\\
Eastwood,\\
SA 5063,\\
Australia}

\end{footnotesize}

\newpage

\setlength{\parindent}{1.5em}%%
\tableofcontents
\newpage
\pagestyle{fancy}
%
\chapter*{Summary of Contents}
\addcontentsline{toc}{chapter}{Summary of Contents}
\markboth{Summary of Contents}{Summary of Contents}

\begin{description}%
\item[The Chapter with \textsanskrit{Pāṭikaputta} (\textit{\textsanskrit{Pāthikavagga}})] Like the previous chapter, this contains a diverse range of discourses. It is named after the first discourse in the chapter. Among the discourses here are legendary accounts of the history and future of our world, which are extremely famous and influential in Buddhist circles.%
\item[DN 24: About \textsanskrit{Pāṭikaputta} (\textit{\textsanskrit{Pāthikasutta}})] When Sunakkhatta threatens to disrobe, the Buddha is unimpressed. Rejecting showy displays of asceticism or wondrous powers, he demonstrates his pre-eminence.%
\item[DN 25: The Lion’s Roar at \textsanskrit{Udumbarikā}’s Monastery (\textit{\textsanskrit{Udumbarikasutta}})] This discourse gives a specially good example of dialog between religions. The Buddha insists that he is not interested to make anyone give up their teacher or practices, but only to help people let go of suffering.%
\item[DN 26: The Wheel-Turning Monarch(\textit{\textsanskrit{Cakkavattisutta}})] In illustration of his dictum that one should rely on oneself, the Buddha gives a detailed account of the fall of a kingly lineage of the past, and the subsequent degeneration of society. This process, however, is not over, as the Buddha predicts that eventually society will fall into utter chaos. But far in the future, another Buddha, Metteyya, will arise in a time of peace and plenty.%
\item[DN 27: The Origin of the World (\textit{\textsanskrit{Aggaññasutta}})] In contrast with the brahmin’s self-serving mythologies of the past, the Buddha presents an account of evolution that shows how human choices are an integral part of the ecological balance, and how excessive greed destroys the order of nature.%
\item[DN 28: Inspiring Confidence (\textit{\textsanskrit{Sampasādanīyasutta}})] Shortly before he passes away, Venerable \textsanskrit{Sāriputta} visits the Buddha and utters a moving eulogy of his great teacher.%
\item[DN 29: An Impressive Discourse (\textit{\textsanskrit{Pāsādikasutta}})] Following the death of \textsanskrit{Nigaṇṭha} \textsanskrit{Nātaputta}, the leader of the Jains, the Buddha emphasizes the stability and maturity of his own community. He encourages the community to come together after his death and recite the teachings in harmony.%
\item[DN 30: The Marks of a Great Man (\textit{\textsanskrit{Lakkhaṇasutta}})] This presents the brahmanical prophecy of the Great Man, and explains the 32 marks in detail. This discourse contains some of the latest and most complex verse forms in the canon.%
\item[DN 31: Advice to \textsanskrit{Sigālaka} (\textit{\textsanskrit{Siṅgālasutta}})] The Buddha encounters a young man who honors his dead parents by performing rituals. The Buddha recasts the meaningless rites in terms of virtuous conduct. This is the most detailed discourse on ethics for lay people.%
\item[DN 32: The \textsanskrit{Āṭānāṭiya} Protection (\textit{\textsanskrit{Āṭānāṭiyasutta}})] Mighty spirits hold a congregation, and warn the Buddha that, since not all spirits are friendly, the mendicants should learn verses of protection.%
\item[DN 33: Reciting in Concert (\textit{\textsanskrit{Saṅgītisutta}})] The Buddha encourages Venerable \textsanskrit{Sāriputta} to teach the mendicants, and he offers an extended listing of Buddhist doctrines arranged in numerical sequence.%
\item[DN 34: Up to Ten (\textit{\textsanskrit{Dasuttarasutta}})] This is similar to the previous, but with a different manner of exposition. These two discourses anticipate some of the methods of the Abhidhamma.%
\end{description}

%
\mainmatter%
\pagestyle{fancy}%
\addtocontents{toc}{\let\protect\contentsline\protect\nopagecontentsline}
\part*{The Chapter with Pāṭikaputta }
\addcontentsline{toc}{part}{The Chapter with Pāṭikaputta }
\markboth{}{}
\addtocontents{toc}{\let\protect\contentsline\protect\oldcontentsline}

%
\chapter*{{\suttatitleacronym DN 24}{\suttatitletranslation About Pāṭikaputta }{\suttatitleroot Pāthikasutta}}
\addcontentsline{toc}{chapter}{\tocacronym{DN 24} \toctranslation{About Pāṭikaputta } \tocroot{Pāthikasutta}}
\markboth{About Pāṭikaputta }{Pāthikasutta}
\extramarks{DN 24}{DN 24}

\section*{1. The Story of Sunakkhatta }

\scevam{So\marginnote{1.1.1} I have heard. }At one time the Buddha was staying in the land of the Mallas, near the Mallian town named Anupiya. Then the Buddha robed up in the morning and, taking his bowl and robe, entered Anupiya for alms. Then it occurred to him, “It’s too early to wander for alms in Anupiya. Why don’t I go to the wanderer Bhaggavagotta’s monastery to visit him?” 

So\marginnote{1.2.1} that’s what he did. Then the wanderer Bhaggavagotta said to the Buddha, “Come, Blessed One! Welcome, Blessed One! It’s been a long time since you took the opportunity to come here. Please, sir, sit down, this seat is ready.” 

The\marginnote{1.2.7} Buddha sat on the seat spread out, while Bhaggavagotta took a low seat, sat to one side, and said to the Buddha, “Sir, a few days ago Sunakkhatta the Licchavi came to me and said: ‘Now, Bhaggava, I have rejected the Buddha. Now I no longer live dedicated to him.’ Sir, is what Sunakkhatta said true?” 

“Indeed\marginnote{1.2.14} it is, Bhaggava. 

A\marginnote{1.3.1} few days ago Sunakkhatta the Licchavi came to me, bowed, sat down to one side, and said: ‘Now I reject the Buddha! Now I shall no longer live dedicated to you.’ 

When\marginnote{1.3.4} Sunakkhatta said this, I said to him, ‘But Sunakkhatta, did I ever say to you: “Come, live dedicated to me”?’ 

‘No,\marginnote{1.3.7} sir.’ 

‘Or\marginnote{1.3.8} did you ever say to me: “Sir, I shall live dedicated to the Buddha”?’ 

‘No,\marginnote{1.3.10} sir.’ 

‘So\marginnote{1.3.11} it seems that I did not ask you to live dedicated to me, nor did you say you would live dedicated to me. In that case, you silly man, are you really in a position to be rejecting anything? See how far you have strayed!’ 

‘But\marginnote{1.4.1} sir, the Buddha never performs any superhuman demonstrations of psychic power for me.’ 

‘But\marginnote{1.4.2} Sunakkhatta, did I ever say to you: “Come, live dedicated to me and I will perform a superhuman demonstration of psychic power for you”?’ 

‘No,\marginnote{1.4.4} sir.’ 

‘Or\marginnote{1.4.5} did you ever say to me: “Sir, I shall live dedicated to the Buddha, and the Buddha will perform a superhuman demonstration of psychic power for me”?’ 

‘No,\marginnote{1.4.7} sir.’ 

‘So\marginnote{1.4.8} it seems that I did not ask this of you, and you did not require it of me. In that case, you silly man, are you really in a position to be rejecting anything? What do you think, Sunakkhatta? Whether or not there is a demonstration of psychic power, does my teaching lead someone who practices it to the goal of the complete ending of suffering?’ 

‘It\marginnote{1.4.15} does, sir.’ 

‘So\marginnote{1.4.16} it seems that whether or not there is a demonstration of psychic power, my teaching leads someone who practices it to the goal of the complete ending of suffering. In that case, what is the point of superhuman demonstrations of psychic power? See how far you have strayed, you silly man!’ 

‘But\marginnote{1.5.1} sir, the Buddha never describes the origin of the world to me.’ 

‘But\marginnote{1.5.2} Sunakkhatta, did I ever say to you: “Come, live dedicated to me and I will describe the origin of the world to you”?’ 

‘No,\marginnote{1.5.4} sir.’ 

‘Or\marginnote{1.5.5} did you ever say to me: “Sir, I shall live dedicated to the Buddha, and the Buddha will describe the origin of the world to me”?’ 

‘No,\marginnote{1.5.7} sir.’ 

‘So\marginnote{1.5.8} it seems that I did not ask this of you, and you did not require it of me. In that case, you silly man, are you really in a position to be rejecting anything? What do you think, Sunakkhatta? Whether or not the origin of the world is described, does my teaching lead someone who practices it to the goal of the complete ending of suffering?’ 

‘It\marginnote{1.5.15} does, sir.’ 

‘So\marginnote{1.5.16} it seems that whether or not the origin of the world is described, my teaching leads someone who practices it to the goal of the complete ending of suffering. In that case, what is the point of describing the origin of the world? See how far you have strayed, you silly man! 

In\marginnote{1.6.1} many ways, Sunakkhatta, you have praised me like this in the Vajjian capital: “That Blessed One is perfected, a fully awakened Buddha, accomplished in knowledge and conduct, holy, knower of the world, supreme guide for those who wish to train, teacher of gods and humans, awakened, blessed.” 

In\marginnote{1.6.4} many ways you have praised the teaching like this in the Vajjian capital: “The teaching is well explained by the Buddha—visible in this very life, immediately effective, inviting inspection, relevant, so that sensible people can know it for themselves.” 

In\marginnote{1.6.7} many ways you have praised the \textsanskrit{Saṅgha} like this in the Vajjian capital: “The \textsanskrit{Saṅgha} of the Buddha’s disciples is practicing the way that’s good, direct, methodical, and proper. It consists of the four pairs, the eight individuals. This is the \textsanskrit{Saṅgha} of the Buddha’s disciples that is worthy of offerings dedicated to the gods, worthy of hospitality, worthy of a religious donation, worthy of greeting with joined palms, and is the supreme field of merit for the world.” 

I\marginnote{1.6.10} declare this to you, Sunakkhatta, I announce this to you! There will be those who say that Sunakkhatta was unable to lead the spiritual life under the ascetic Gotama. That’s why he resigned the training and returned to a lesser life. That’s what they’ll say.’ 

Though\marginnote{1.6.13} I spoke to Sunakkhatta like this, he still left this teaching and training, like someone on the highway to hell. 

\section*{2. On Korakkhattiya }

Bhaggava,\marginnote{1.7.1} this one time I was staying in the land of the \textsanskrit{Thūlus} where they have a town named \textsanskrit{Uttarakā}. Then I robed up in the morning and, taking my bowl and robe, entered \textsanskrit{Uttarakā} for alms with Sunakkhatta the Licchavi as my second monk. Now at that time the naked ascetic Korakkhattiya had taken a vow to behave like a dog. When food is tossed on the ground, he gets down on all fours, eating and devouring it just with his mouth. 

Sunakkhatta\marginnote{1.7.4} saw him doing this and thought, ‘That ascetic is a true holy man!’ 

Then,\marginnote{1.7.7} knowing what Sunakkhatta was thinking, I said to him, ‘Don’t you claim to be an ascetic, a follower of the Sakyan, you silly man?’ 

‘But\marginnote{1.7.9} why does the Buddha say this to me?’ 

‘When\marginnote{1.7.11} you saw that naked ascetic Korakkhattiya, didn’t you think, “That ascetic is a true holy man!”?’ 

‘Yes,\marginnote{1.7.13} sir. But sir, are you jealous of the perfected ones?’ 

‘I’m\marginnote{1.7.15} not jealous of the perfected ones, you silly man. Rather, you should give up this harmful misconception that has arisen in you. Don’t create lasting harm and suffering for yourself! 

That\marginnote{1.7.18} naked ascetic Korakkhattiya, who you imagine to be a true holy man, will die of flatulence in seven days. And when he dies, he’ll be reborn in the very lowest rank of demons, named the \textsanskrit{Kālakañjas}. And they’ll throw him in the charnel ground on a clump of vetiver. If you wish, Sunakkhatta, go to Korakkhattiya and ask him whether he knows his own destiny. It’s possible that he will answer: “Reverend Sunakkhatta, I know my own destiny. I’ve been reborn in the very lowest rank of demons, named the \textsanskrit{Kālakañjas}.”’ 

So,\marginnote{1.8.1} Bhaggava, Sunakkhatta went to see Korakkhattiya and said to him, ‘Reverend Korakkhattiya, the ascetic Gotama has declared that you will die of flatulence in seven days. And when you die, you’ll be reborn in the very lowest rank of demons, named the \textsanskrit{Kālakañjas}. And when you die, they’ll throw you in the charnel ground on a clump of vetiver. But by eating just a little food and drinking just a little water, you’ll prove what the ascetic Gotama says to be false.’ 

Then\marginnote{1.8.8} Sunakkhatta counted up the days until the seventh day, as happens when you have no faith in the Realized One. But on the seventh day, the naked ascetic Korakkhattiya died of flatulence. And when he passed away, he was reborn in the very lowest rank of demons, named the \textsanskrit{Kālakañjas}. And when he passed away, they threw him in the charnel ground on a clump of vetiver. 

Sunakkhatta\marginnote{1.9.1} the Licchavi heard about this. So he went to see Korakkhattiya in the charnel ground on the clump of vetiver. There he struck Korakkhattiya with his fist three times, ‘Reverend Korakkhattiya, do you know your destiny?’ 

Then\marginnote{1.9.5} Korakkhattiya got up, rubbing his back with his hands, and said, ‘Reverend Sunakkhatta, I know my own destiny. I’ve been reborn in the very lowest rank of demons, named the \textsanskrit{Kālakañjas}.’ After speaking, he fell flat right there. 

Then\marginnote{1.10.1} Sunakkhatta came to me, bowed, and sat down to one side. I said to him, ‘What do you think, Sunakkhatta? Did the declaration I made about Korakkhattiya turn out to be correct, or not?’ 

‘It\marginnote{1.10.4} turned out to be correct.’ 

‘What\marginnote{1.10.5} do you think, Sunakkhatta? If that is so, has a superhuman demonstration of psychic power been performed or not?’ 

‘Clearly,\marginnote{1.10.7} sir, a superhuman demonstration of psychic power has been performed.’ 

‘Though\marginnote{1.10.8} I performed such a superhuman demonstration of psychic power you say this: “But sir, the Buddha never performs any superhuman demonstrations of psychic power for me.” See how far you have strayed!’ Though I spoke to Sunakkhatta like this, he still left this teaching and training, like someone on the highway to hell. 

\section*{3. On the Naked Ascetic \textsanskrit{Kaḷāramaṭṭaka} }

This\marginnote{1.11.1} one time, Bhaggava, I was staying near \textsanskrit{Vesālī}, at the Great Wood, in the hall with the peaked roof. Now at that time the naked ascetic \textsanskrit{Kaḷāramaṭṭaka} was residing in \textsanskrit{Vesālī}. And in the Vajjian capital he had reached the peak of material possessions and fame. He had undertaken these seven vows. ‘As long as I live, I will be a naked ascetic, not wearing clothes. As long as I live, I will be celibate, not having sex. As long as I live, I will consume only meat and alcohol, not eating rice and porridge. And I will not go past the following tree-shrines near \textsanskrit{Vesālī}: the Udena shrine to the east, the Gotamaka to the south, the Sattamba to the west, and the Bahuputta to the north.’ And it was due to undertaking these seven vows that he had reached the peak of material possessions and fame. 

So,\marginnote{1.12.1} Bhaggava, Sunakkhatta went to see \textsanskrit{Kaḷāramaṭṭaka} and asked him a question. But when it stumped him, he displayed annoyance, hate, and bitterness. So Sunakkhatta thought, ‘I’ve offended the holy man, the perfected one, the ascetic. I mustn’t create lasting harm and suffering for myself!’ 

Then\marginnote{1.13.1} Sunakkhatta came to me, bowed, and sat down to one side. I said to him, ‘Don’t you claim to be an ascetic, a follower of the Sakyan, you silly man?’ 

‘But\marginnote{1.13.3} why does the Buddha say this to me?’ 

‘Didn’t\marginnote{1.13.5} you go to see the naked ascetic \textsanskrit{Kaḷāramaṭṭaka} and ask him a question? But when it stumped him, you displayed annoyance, hate, and bitterness. Then you thought, “I’ve offended the holy man, the perfected one, the ascetic. I mustn’t create lasting harm and suffering for myself!”’ 

‘Yes,\marginnote{1.13.11} sir. But sir, are you jealous of perfected ones?’ 

‘I’m\marginnote{1.13.13} not jealous of the perfected ones, you silly man. Rather, you should give up this harmful misconception that has arisen in you. Don’t create lasting harm and suffering for yourself! 

That\marginnote{1.13.16} naked ascetic \textsanskrit{Kaḷāramaṭṭaka}, who you imagine to be a true holy man, will shortly be clothed, living with a partner, eating rice and porridge, having gone past all the shrines near \textsanskrit{Vesālī}. And he will die after losing all his fame.’ 

And\marginnote{1.13.18} that’s exactly what happened. 

Sunakkhatta\marginnote{1.14.1} heard about this. He came to me, bowed, and sat down to one side. I said to him, ‘What do you think, Sunakkhatta? Did the declaration I made about \textsanskrit{Kaḷāramaṭṭaka} turn out to be correct, or not?’ 

‘It\marginnote{1.14.6} turned out to be correct.’ 

‘What\marginnote{1.14.7} do you think, Sunakkhatta? If that is so, has a superhuman demonstration of psychic power been performed or not?’ 

‘Clearly,\marginnote{1.14.9} sir, a superhuman demonstration of psychic power has been performed.’ 

‘Though\marginnote{1.14.10} I perform such a superhuman demonstration of psychic power you say this: “But sir, the Buddha never performs any superhuman demonstrations of psychic power for me.” See how far you have strayed!’ Though I spoke to Sunakkhatta like this, he still left this teaching and training, like someone on the highway to hell. 

\section*{4. On the Naked Ascetic \textsanskrit{Pāṭikaputta} }

This\marginnote{1.15.1} one time, Bhaggava, I was staying right there near \textsanskrit{Vesālī}, at the Great Wood, in the hall with the peaked roof. Now at that time the naked ascetic \textsanskrit{Pāṭikaputta} was residing in \textsanskrit{Vesālī}. And in the Vajjian capital he had reached the peak of material possessions and fame. He was telling a crowd in \textsanskrit{Vesālī}: ‘Both the ascetic Gotama and I speak from knowledge. One who speaks from knowledge ought to display a superhuman demonstration of psychic power to another who speaks from knowledge. If the ascetic Gotama meets me half-way, there we should both perform a superhuman demonstration of psychic power. If he performs one demonstration of psychic power, I’ll perform two. If he performs two, I’ll perform four. If he performs four, I’ll perform eight. However many demonstrations of psychic power the ascetic Gotama performs, I’ll perform double.’ 

Then\marginnote{1.16.1} Sunakkhatta came to me, bowed, sat down to one side, and told me of all this. 

I\marginnote{1.16.12} said to him, ‘Sunakkhatta, the naked ascetic \textsanskrit{Pāṭikaputta} is not capable of coming into my presence, unless he gives up that statement and that intention, and lets go of that view. If he thinks he can come into my presence without giving up those things, his head may explode.’ 

‘Careful\marginnote{1.17.1} what you say, Blessed One! Careful what you say, Holy One!’ 

‘But\marginnote{1.17.2} why do you say this to me, Sunakkhatta?’ 

‘Sir,\marginnote{1.17.4} the Buddha has definitively asserted that \textsanskrit{Pāṭikaputta} is not capable of coming into the Buddha’s presence, otherwise his head may explode. But \textsanskrit{Pāṭikaputta} might come into the Buddha’s presence in disguise, proving the Buddha wrong.’ 

‘Sunakkhatta,\marginnote{1.18.1} would the Realized One make an ambiguous statement?’ 

‘But\marginnote{1.18.2} sir, did you make that statement after comprehending \textsanskrit{Pāṭikaputta}’s mind with your mind? Or did deities tell you about it?’ 

‘Both,\marginnote{1.18.10} Sunakkhatta. For Ajita the Licchavi general has recently passed away and been reborn in the host of the Thirty-Three. He came and told me this, “The naked ascetic \textsanskrit{Pāṭikaputta} is shameless, sir, he is a liar. For he has declared of me in the Vajjian capital: ‘Ajita the Licchavi general has been reborn in the Great Hell.’ 

But\marginnote{1.18.24} that is not true—I have been reborn in the host of the Thirty-Three. The naked ascetic \textsanskrit{Pāṭikaputta} is shameless, sir, he is a liar. \textsanskrit{Pāṭikaputta} is not capable of coming into the Buddha’s presence, otherwise his head may explode.” 

Thus\marginnote{1.18.31} I both made that statement after comprehending \textsanskrit{Pāṭikaputta}’s mind with my mind, and deities told me about it. 

So\marginnote{1.18.38} Sunakkhatta, I’ll wander for alms in \textsanskrit{Vesālī}. After the meal, on my return from almsround, I’ll go to \textsanskrit{Pāṭikaputta}’s monastery for the day’s meditation. Now you may tell him, if you so wish.’ 

\section*{5. On Demonstrations of Psychic Power }

Then,\marginnote{1.19.1} Bhaggava, I robed up in the morning and, taking my bowl and robe, entered \textsanskrit{Vesālī} for alms. After the meal, on my return from almsround, I went to \textsanskrit{Pāṭikaputta}’s monastery for the day’s meditation. Then Sunakkhatta rushed into \textsanskrit{Vesālī} to see the very well-known Licchavis and said to them, ‘Sirs, after his almsround, the Buddha has gone to \textsanskrit{Pāṭikaputta}’s monastery for the day’s meditation. Come forth, sirs, come forth! There will be a superhuman demonstration of psychic power by the holy ascetics!’ So the very well-known Licchavis thought, ‘It seems there will be a superhuman demonstration of psychic power by the holy ascetics! Let’s go!’ 

Then\marginnote{1.19.9} he went to see the very well-known well-to-do brahmins, rich householders, and ascetics and brahmins who follow various other paths, and said the same thing. They all said, ‘It seems there will be a superhuman demonstration of psychic power by the holy ascetics! Let’s go!’ 

Then\marginnote{1.19.15} all those very well-known people went to \textsanskrit{Pāṭikaputta}’s monastery. That assembly was large, Bhaggava; there were many hundreds, many thousands of them. 

\textsanskrit{Pāṭikaputta}\marginnote{1.20.1} heard, ‘It seems that very well-known Licchavis, well-to-do brahmins, rich householders, and ascetics and brahmins who follow various other paths have come forth. And the ascetic Gotama is sitting in my monastery for the day’s meditation.’ When he heard that, he became frightened, scared, his hair standing on end. In fear he went to the Pale-Moon Ebony Trunk Monastery of the wanderers. 

The\marginnote{1.20.6} assembly heard of this, and instructed a man, ‘Come, my man, go to see \textsanskrit{Pāṭikaputta} at the Pale-Moon Ebony Trunk Monastery and say to him, “Come forth, Reverend \textsanskrit{Pāṭikaputta}! All these very well-known people have come forth, and the ascetic Gotama is sitting in your monastery for the day’s meditation. For you stated this in the assembly at \textsanskrit{Vesālī}: ‘Both the ascetic Gotama and I speak from knowledge. One who speaks from knowledge ought to display a superhuman demonstration of psychic power to another who speaks from knowledge. If the ascetic Gotama meets me half-way, there we should both perform a superhuman demonstration of psychic power. If he performs one demonstration of psychic power, I’ll perform two. If he performs two, I’ll perform four. If he performs four, I’ll perform eight. However many demonstrations of psychic power the ascetic Gotama performs, I’ll perform double.’ Come forth, Reverend \textsanskrit{Pāṭikaputta}, half-way. The ascetic Gotama has come the first half, and is sitting in your monastery.”’ 

‘Yes,\marginnote{1.21.1} sir,’ replied that man, and delivered the message. 

When\marginnote{1.21.10} he had spoken, \textsanskrit{Pāṭika} said: ‘I’m coming, sir, I’m coming!’ But wriggle as he might, he couldn’t get up from his seat. Then that man said to \textsanskrit{Pāṭikaputta}, ‘What’s up, Reverend \textsanskrit{Pāṭikaputta}? Is your bottom stuck to the bench, or is the bench stuck to your bottom? You say “I’m coming, sir, I’m coming!” But wriggle as you might, you can’t get up from your seat.’ And as he was speaking, \textsanskrit{Pāṭika} said: ‘I’m coming, sir, I’m coming!’ But wriggle as he might, he couldn’t get up from his seat. 

When\marginnote{1.22.1} that man knew that \textsanskrit{Pāṭikaputta} had lost, he returned to the assembly and said, ‘\textsanskrit{Pāṭikaputta} has lost, sirs. He says “I’m coming, sir, I’m coming!” But wriggle as he might, he can’t get up from his seat.’ When he said this, I said to the assembly, ‘The naked ascetic \textsanskrit{Pāṭikaputta} is not capable of coming into my presence, unless he gives up that statement and that intention, and lets go of that view. If he thinks he can come into my presence without giving up those things, his head may explode.’ 

\scendsection{The first recitation section is finished. }

Then,\marginnote{2.1.1} Bhaggava, a certain Licchavi minister stood up and said to the assembly, ‘Well then, sirs, wait a moment, I’ll go. Hopefully I’ll be able to lead \textsanskrit{Pāṭikaputta} back to the assembly.’ 

So\marginnote{2.1.4} that minister went to see \textsanskrit{Pāṭikaputta} and said, ‘Come forth, Reverend \textsanskrit{Pāṭikaputta}! It’s best for you to come forth. All these very well-known people have come forth, and the ascetic Gotama is sitting in your monastery for the day’s meditation. You said you’d meet the ascetic Gotama half-way. The ascetic Gotama has come the first half, and is sitting in your monastery. The ascetic Gotama has told the assembly that you’re not capable of coming into his presence. Come forth, \textsanskrit{Pāṭikaputta}! When you come forth we’ll make you win and the ascetic Gotama lose.’ 

When\marginnote{2.2.1} he had spoken, \textsanskrit{Pāṭikaputta} said: ‘I’m coming, sir, I’m coming!’ But wriggle as he might, he couldn’t get up from his seat. Then the minister said to \textsanskrit{Pāṭikaputta}, ‘What’s up, Reverend \textsanskrit{Pāṭikaputta}? Is your bottom stuck to the bench, or is the bench stuck to your bottom? You say “I’m coming, sir, I’m coming!” But wriggle as you might, you can’t get up from your seat.’ And as he was speaking, \textsanskrit{Pāṭikaputta} said: ‘I’m coming, sir, I’m coming!’ But wriggle as he might, he couldn’t get up from his seat. 

When\marginnote{2.3.1} the Licchavi minister knew that \textsanskrit{Pāṭikaputta} had lost, he returned to the assembly and said, ‘\textsanskrit{Pāṭikaputta} has lost, sirs.’ When he said this, I said to the assembly, ‘\textsanskrit{Pāṭikaputta} is not capable of coming into my presence, otherwise his head may explode. Even if the good Licchavis were to think, “Let’s bind \textsanskrit{Pāṭikaputta} with straps and drag him with a pair of oxen!” But either the straps will break or \textsanskrit{Pāṭikaputta} will.’ 

Then,\marginnote{2.4.1} Bhaggava, \textsanskrit{Jāliya}, the pupil of the wood-bowl ascetic, stood up and said to the assembly, ‘Well then, sirs, wait a moment, I’ll go. Hopefully I’ll be able to lead \textsanskrit{Pāṭikaputta} back to the assembly.’ 

So\marginnote{2.4.4} \textsanskrit{Jāliya} went to see \textsanskrit{Pāṭikaputta} and said, ‘Come forth, Reverend \textsanskrit{Pāṭikaputta}! It’s best for you to come forth. All these very well-known people have come forth, and the ascetic Gotama is sitting in your monastery for the day’s meditation. You said you’d meet the ascetic Gotama half-way. The ascetic Gotama has come the first half, and is sitting in your monastery. The ascetic Gotama has told the assembly that you’re not capable of coming into his presence. And he said that even if the Licchavis try to bind you with straps and drag you with a pair of oxen, either the straps will break or you will. Come forth, \textsanskrit{Pāṭikaputta}! When you come forth we’ll make you win and the ascetic Gotama lose.’ 

When\marginnote{2.5.1} he had spoken, \textsanskrit{Pāṭikaputta} said: ‘I’m coming, sir, I’m coming!’ But wriggle as he might, he couldn’t get up from his seat. Then \textsanskrit{Jāliya} said to \textsanskrit{Pāṭikaputta}, ‘What’s up, Reverend \textsanskrit{Pāṭikaputta}? Is your bottom stuck to the bench, or is the bench stuck to your bottom? You say “I’m coming, sir, I’m coming!” But wriggle as you might, you can’t get up from your seat.’ And as he was speaking, \textsanskrit{Pāṭikaputta} said: ‘I’m coming, sir, I’m coming!’ But wriggle as he might, he couldn’t get up from his seat. 

When\marginnote{2.6.1} \textsanskrit{Jāliya} knew that \textsanskrit{Pāṭikaputta} had lost, he said to him, 

‘Once\marginnote{2.6.3} upon a time, Reverend \textsanskrit{Pāṭikaputta}, it occurred to a lion, king of beasts, “Why don’t I make my lair near a certain forest? Towards evening I can emerge from my den, yawn, look all around the four quarters, roar my lion’s roar three times, and set out on the hunt. Having slain the very best of the deer herd, and eaten the most tender flesh, I could return to my den.” 

And\marginnote{2.6.7} so that’s what he did. 

Now,\marginnote{2.6.10} there was an old jackal who had grown fat on the lion’s leavings, becoming arrogant and strong. He thought, “What does the lion, king of beasts, have that I don’t? Why don’t I also make my lair near a certain forest? Towards evening I can emerge from my den, yawn, look all around the four quarters, roar my lion’s roar three times, and set out on the hunt. Having slain the very best of the deer herd, and eaten the most tender flesh, I could return to my den.” 

And\marginnote{2.7.1} so that’s what he did. But when he tried to roar a lion’s roar, he only managed to squeal and yelp like a jackal. And what is a pathetic jackal’s squeal next to the roar of a lion? 

In\marginnote{2.7.3} the same way, reverend, while living on the harvest of the Holy One, enjoying the leftovers of the Holy One, you presume to attack the Realized One, the perfected one, the fully awakened Buddha! Who are the pathetic \textsanskrit{Pāṭikaputtas} to attack the Realized Ones, the perfected ones, the fully awakened Buddhas?’ 

When\marginnote{2.8.1} \textsanskrit{Jāliya} couldn’t get \textsanskrit{Pāṭikaputta} to shift from his seat even with this simile, he said to him: 

\begin{verse}%
‘Seeing\marginnote{2.8.3} himself as equal to the lion, \\
the jackal presumed “I’m the king of the beasts!” \\
But in reality he only managed to yelp, \\
and what’s a sad jackal’s squeal to the roar of a lion? 

%
\end{verse}

In\marginnote{2.8.7} the same way, reverend, while living on the harvest of the Holy One, you presume to attack him!’ 

When\marginnote{2.9.1} \textsanskrit{Jāliya} couldn’t get \textsanskrit{Pāṭikaputta} to shift from his seat even with this simile, he said to him: 

\begin{verse}%
‘Following\marginnote{2.9.3} in the steps of another, \\
seeing himself grown fat on scraps, \\
until he doesn’t even see himself, \\
the jackal presumes he’s a tiger. 

But\marginnote{2.9.7} in reality he only managed to yelp, \\
and what’s a sad jackal’s squeal to the roar of a lion? 

%
\end{verse}

In\marginnote{2.9.9} the same way, reverend, while living on the harvest of the Holy One, you presume to attack him!’ 

When\marginnote{2.10.1} \textsanskrit{Jāliya} couldn’t get \textsanskrit{Pāṭikaputta} to shift from his seat even with this simile, he said to him: 

\begin{verse}%
‘Gorged\marginnote{2.10.3} on frogs, and mice from the barn, \\
and carcasses tossed in the cemetery, \\
thriving in the great, empty wood, \\
the jackal presumed “I’m the king of the beasts!” \\
But in reality he only managed to yelp, \\
and what’s a sad jackal’s squeal to the roar of a lion? 

%
\end{verse}

In\marginnote{2.10.9} the same way, reverend, while living on the harvest of the Holy One, enjoying the leftovers of the Holy One, you presume to attack the Realized One, the perfected one, the fully awakened Buddha! Who are the pathetic \textsanskrit{Pāṭikaputtas} to attack the Realized Ones, the perfected ones, the fully awakened Buddhas?’ 

When\marginnote{2.11.1} \textsanskrit{Jāliya} couldn’t get \textsanskrit{Pāṭikaputta} to shift from his seat even with this simile, he returned to the assembly and said, ‘\textsanskrit{Pāṭikaputta} has lost, sirs. He says “I’m coming, sir, I’m coming!” But wriggle as he might, he can’t get up from his seat.’ 

When\marginnote{2.12.1} he said this, I said to the assembly, ‘The naked ascetic \textsanskrit{Pāṭikaputta} is not capable of coming into my presence, unless he gives up that statement and that intention, and lets go of that view. If he thinks he can come into my presence without giving up those things, his head may explode. The good Licchavis might even think, “Let’s bind \textsanskrit{Pāṭikaputta} with straps and drag him with a pair of oxen!” But either the straps will break or \textsanskrit{Pāṭikaputta} will. \textsanskrit{Pāṭikaputta} is not capable of coming into my presence, otherwise his head may explode.’ 

Then,\marginnote{2.13.1} Bhaggava, I educated, encouraged, fired up, and inspired that assembly with a Dhamma talk. I released that assembly from the great bondage, and lifted 84,000 beings from the great swamp. Next I entered upon the fire element, rose into the sky to the height of seven palm trees, and created a flame another seven palm trees high, blazing and smoking. Finally I landed at the Great Wood, in the hall with the peaked roof. 

Then\marginnote{2.13.2} Sunakkhatta came to me, bowed, and sat down to one side. I said to him, ‘What do you think, Sunakkhatta? Did the declaration I made about \textsanskrit{Pāṭikaputta} turn out to be correct, or not?’ 

‘It\marginnote{2.13.5} turned out to be correct.’ 

‘What\marginnote{2.13.6} do you think, Sunakkhatta? If that is so, has a superhuman demonstration of psychic power been performed or not?’ 

‘Clearly,\marginnote{2.13.8} sir, a superhuman demonstration of psychic power has been performed.’ 

‘Though\marginnote{2.13.9} I perform such a superhuman demonstration of psychic power you say this: “But sir, the Buddha never performs any superhuman demonstrations of psychic power for me.” See how far you have strayed!’ 

Though\marginnote{2.13.12} I spoke to Sunakkhatta like this, he still left this teaching and training, like someone on the highway to hell. 

\section*{6. On Describing the Origin of the World }

Bhaggava,\marginnote{2.14.1} I understand the origin of the world. I understand this, and what goes beyond it. Yet since I do not misapprehend that understanding, I have realized extinguishment within myself. Directly knowing this, the Realized One does not come to ruin. 

There\marginnote{2.14.3} are some ascetics and brahmins who describe the origin of the world in their tradition as created by the Lord God, by \textsanskrit{Brahmā}. I go up to them and say, ‘Is it really true that this is the venerables’ view?’ And they answer, ‘Yes’. I say to them, ‘But how do you describe in your tradition that the origin of the world came about as created by the Lord God, by \textsanskrit{Brahmā}?’ But they are stumped by my question, and they even question me in return. So I answer them, 

‘There\marginnote{2.14.11} comes a time when, reverends, after a very long period has passed, this cosmos contracts. As the cosmos contracts, sentient beings are mostly headed for the realm of streaming radiance. There they are mind-made, feeding on rapture, self-luminous, moving through the sky, steadily glorious, and they remain like that for a very long time. 

There\marginnote{2.15.1} comes a time when, after a very long period has passed, this cosmos expands. As it expands an empty mansion of \textsanskrit{Brahmā} appears. Then a certain sentient being—due to the running out of their life-span or merit—passes away from that host of radiant deities and is reborn in that empty mansion of \textsanskrit{Brahmā}. There they are mind-made, feeding on rapture, self-luminous, moving through the sky, steadily glorious, and they remain like that for a very long time. 

But\marginnote{2.16.1} after staying there all alone for a long time, they become dissatisfied and anxious, “Oh, if only another being would come to this state of existence.” 

Then\marginnote{2.16.3} other sentient beings—due to the running out of their life-span or merit—pass away from that host of radiant deities and are reborn in that empty mansion of \textsanskrit{Brahmā} in company with that being. There they too are mind-made, feeding on rapture, self-luminous, moving through the sky, steadily glorious, and they remain like that for a very long time. 

Now,\marginnote{2.17.1} the being who was reborn there first thinks, “I am \textsanskrit{Brahmā}, the Great \textsanskrit{Brahmā}, the Undefeated, the Champion, the Universal Seer, the Wielder of Power, the Lord God, the Maker, the Author, the First, the Begetter, the Controller, the Father of those who have been born and those yet to be born. Why is that? Because first I thought, ‘Oh, if only another being would come to this state of existence.’ Such was my heart’s wish, and then these creatures came to this state of existence.” 

And\marginnote{2.17.7} the beings who were reborn there later also think, “This must be \textsanskrit{Brahmā}, the Great \textsanskrit{Brahmā}, the Undefeated, the Champion, the Universal Seer, the Wielder of Power, the Lord God, the Maker, the Author, the First, the Begetter, the Controller, the Father of those who have been born and those yet to be born. And we have been created by him. Why is that? Because we see that he was reborn here first, and we arrived later.” 

And\marginnote{2.17.12} the being who was reborn first is more long-lived, beautiful, and illustrious than those who arrived later. 

It’s\marginnote{2.17.14} possible that one of those beings passes away from that host and is reborn in this state of existence. Having done so, they go forth from the lay life to homelessness. By dint of keen, resolute, committed, and diligent effort, and right focus, they experience an immersion of the heart of such a kind that they recollect that past life, but no further. 

They\marginnote{2.17.17} say: “He who is \textsanskrit{Brahmā}—the Great \textsanskrit{Brahmā}, the Undefeated, the Champion, the Universal Seer, the Wielder of Power, the Lord God, the Maker, the Author, the First, the Begetter, the Controller, the Father of those who have been born and those yet to be born—is permanent, everlasting, eternal, imperishable, remaining the same for all eternity. We who were created by that \textsanskrit{Brahmā} are impermanent, not lasting, short-lived, perishable, and have come to this state of existence.” This is how you describe in your tradition that the origin of the world came about as created by the Lord God, by \textsanskrit{Brahmā}.’ 

They\marginnote{2.17.21} say, ‘That is what we have heard, Reverend Gotama, just as you say.’ 

Bhaggava,\marginnote{2.17.23} I understand the origin of the world. I understand this, and what goes beyond it. Yet since I do not misapprehend that understanding, I have realized extinguishment within myself. Directly knowing this, the Realized One does not come to ruin. 

There\marginnote{2.18.1} are some ascetics and brahmins who describe the origin of the world in their tradition as due to those depraved by play. I go up to them and say, ‘Is it really true that this is the venerables’ view?’ And they answer, ‘Yes’. I say to them, ‘But how do you describe in your tradition that the origin of the world came about due to those depraved by play?’ But they are stumped by my question, and they even question me in return. So I answer them, 

‘Reverends,\marginnote{2.18.9} there are gods named “depraved by play”. They spend too much time laughing, playing, and making merry. And in doing so, they lose their mindfulness, and they pass away from that host of gods. 

It’s\marginnote{2.18.10} possible that one of those beings passes away from that host and is reborn in this state of existence. Having done so, they go forth from the lay life to homelessness. By dint of keen, resolute, committed, and diligent effort, and right focus, they experience an immersion of the heart of such a kind that they recollect that past life, but no further. 

They\marginnote{2.18.13} say, “The gods not depraved by play don’t spend too much time laughing, playing, and making merry. So they don’t lose their mindfulness, and don’t pass away from that host of gods. They are permanent, everlasting, eternal, imperishable, remaining the same for all eternity. But we who were depraved by play spent too much time laughing, playing, and making merry. In doing so, we lost our mindfulness, and passed away from that host of gods. We are impermanent, not lasting, short-lived, perishable, and have come to this state of existence.” This is how you describe in your tradition that the origin of the world came about due to those depraved by play.’ 

They\marginnote{2.18.19} say, ‘That is what we have heard, Reverend Gotama, just as you say.’ 

Bhaggava,\marginnote{2.18.21} I understand the origin of the world. Directly knowing this, the Realized One does not come to ruin. 

There\marginnote{2.19.1} are some ascetics and brahmins who describe the origin of the world in their tradition as due to those who are malevolent. I go up to them and say, ‘Is it really true that this is the venerables’ view?’ And they answer, ‘Yes’. I say to them, ‘But how do you describe in your tradition that the origin of the world came about due to those who are malevolent?’ But they are stumped by my question, and they even question me in return. So I answer them, 

‘Reverends,\marginnote{2.19.9} there are gods named “malevolent”. They spend too much time gazing at each other, so they grow angry with each other, and their bodies and minds get tired. They pass away from that host of gods. 

It’s\marginnote{2.19.10} possible that one of those beings passes away from that host and is reborn in this state of existence. Having done so, they go forth from the lay life to homelessness. By dint of keen, resolute, committed, and diligent effort, and right focus, they experience an immersion of the heart of such a kind that they recollect that past life, but no further. 

They\marginnote{2.19.13} say, “The gods who are not malevolent don’t spend too much time gazing at each other, so they don’t grow angry with each other, their bodies and minds don’t get tired, and they don’t pass away from that host of gods. They are permanent, everlasting, eternal, imperishable, remaining the same for all eternity. But we who were malevolent spent too much time gazing at each other, so our minds grew angry with each other, our bodies and minds got tired, and we passed away from that host of gods. We are impermanent, not lasting, short-lived, perishable, and have come to this state of existence.” This is how you describe in your tradition that the origin of the world came about due to those who are malevolent.’ 

They\marginnote{2.19.19} say, ‘That is what we have heard, Reverend Gotama, just as you say.’ 

Bhaggava,\marginnote{2.19.21} I understand the origin of the world. Directly knowing this, the Realized One does not come to ruin. 

There\marginnote{2.20.1} are some ascetics and brahmins who describe the origin of the world in their tradition as having arisen by chance. I go up to them and say, ‘Is it really true that this is the venerables’ view?’ And they answer, ‘Yes’. I say to them, ‘But how do you describe in your tradition that the origin of the world came about by chance?’ But they are stumped by my question, and they even question me in return. So I answer them, 

‘Reverends,\marginnote{2.20.9} there are gods named “non-percipient beings”. When perception arises they pass away from that host of gods. 

It’s\marginnote{2.20.11} possible that one of those beings passes away from that host and is reborn in this state of existence. Having done so, they go forth from the lay life to homelessness. By dint of keen, resolute, committed, and diligent effort, and right focus, they experience an immersion of the heart of such a kind that they recollect the arising of perception, but no further. 

They\marginnote{2.20.14} say, “The self and the cosmos arose by chance. Why is that? Because formerly I didn’t exist. Now from not being I’ve changed into being.” This is how you describe in your tradition that the origin of the world came about by chance.’ 

They\marginnote{2.20.19} say, ‘That is what we have heard, Reverend Gotama, just as you say.’ 

I\marginnote{2.20.21} understand this, and what goes beyond it. Yet since I do not misapprehend that understanding, I have realized extinguishment within myself. Directly knowing this, the Realized One does not come to ruin. 

Though\marginnote{2.21.1} I speak and explain like this, certain ascetics and brahmins misrepresent me with the false, hollow, lying, untruthful claim: ‘The ascetic Gotama has a distorted perspective, and so have his monks. 

He\marginnote{2.21.3} says, “When one enters and remains in the liberation of the beautiful, at that time one only perceives what is ugly.”’ 

But\marginnote{2.21.5} I don’t say that. I say this: ‘When one enters and remains in the liberation of the beautiful, at that time one only perceives what is beautiful.’” 

“They\marginnote{2.21.9} are the ones with a distorted perspective, sir, who regard the Buddha and the mendicants in this way. Sir, I am quite confident that the Buddha is capable of teaching me so that I can enter and remain in the liberation on the beautiful.” 

“It’s\marginnote{2.21.11} hard for you to enter and remain in the liberation on the beautiful, since you have a different view, creed, preference, practice, and tradition. Come now, Bhaggava, carefully preserve the confidence that you have in me.” 

“If\marginnote{2.21.13} it’s hard for me to enter and remain in the liberation on the beautiful, since I have a different view, creed, preference, practice, and tradition, I shall carefully preserve the confidence that I have in the Buddha.” 

That\marginnote{2.21.15} is what the Buddha said. Satisfied, the wanderer Bhaggavagotta was happy with what the Buddha said. 

%
\chapter*{{\suttatitleacronym DN 25}{\suttatitletranslation The Lion’s Roar at Udumbarikā’s Monastery }{\suttatitleroot Udumbarikasutta}}
\addcontentsline{toc}{chapter}{\tocacronym{DN 25} \toctranslation{The Lion’s Roar at Udumbarikā’s Monastery } \tocroot{Udumbarikasutta}}
\markboth{The Lion’s Roar at Udumbarikā’s Monastery }{Udumbarikasutta}
\extramarks{DN 25}{DN 25}

\section*{1. The Story of the Wanderer Nigrodha }

\scevam{So\marginnote{1.1} I have heard. }At one time the Buddha was staying near \textsanskrit{Rājagaha}, on the Vulture’s Peak Mountain. 

Now\marginnote{1.3} at that time the wanderer Nigrodha was residing in the lady \textsanskrit{Udumbarikā}’s monastery for wanderers, together with a large assembly of three thousand wanderers. Then the householder \textsanskrit{Sandhāna} left \textsanskrit{Rājagaha} in the middle of the day to see the Buddha. 

Then\marginnote{1.5} it occurred to him, “It’s the wrong time to see the Buddha, as he’s in retreat. And it’s the wrong time to see the esteemed mendicants, as they’re in retreat. Why don’t I visit the wanderer Nigrodha at the lady \textsanskrit{Udumbarikā}’s monastery for wanderers?” So he went to the monastery of the wanderers. 

Now\marginnote{2.1} at that time, Nigrodha was sitting together with a large assembly of wanderers making an uproar, a dreadful racket. They engaged in all kinds of unworthy talk, such as talk about kings, bandits, and ministers; talk about armies, threats, and wars; talk about food, drink, clothes, and beds; talk about garlands and fragrances; talk about family, vehicles, villages, towns, cities, and countries; talk about women and heroes; street talk and well talk; talk about the departed; motley talk; tales of land and sea; and talk about being reborn in this or that state of existence. 

Nigrodha\marginnote{3.1} saw \textsanskrit{Sandhāna} coming off in the distance, and hushed his own assembly: “Be quiet, good sirs, don’t make a sound. The householder \textsanskrit{Sandhāna}, a disciple of the ascetic Gotama, is coming. He is included among the white-clothed lay disciples of the ascetic Gotama, who is residing near \textsanskrit{Rājagaha}. Such venerables like the quiet, are educated to be quiet, and praise the quiet. Hopefully if he sees that our assembly is quiet he’ll see fit to approach.” Then those wanderers fell silent. 

Then\marginnote{4.1} \textsanskrit{Sandhāna} went up to the wanderer Nigrodha, and exchanged greetings with him. When the greetings and polite conversation were over, he sat down to one side and said to Nigrodha, “The way the wanderers make an uproar as they sit together and talk about all kinds of unworthy topics is one thing. It’s quite different to the way the Buddha frequents remote lodgings in the wilderness and the forest that are quiet and still, far from the madding crowd, remote from human settlements, and fit for retreat.” 

When\marginnote{5.1} \textsanskrit{Sandhāna} said this, Nigrodha said to him, “Surely, householder, you should know better! With whom does the ascetic Gotama converse? With whom does he engage in discussion? With whom does he achieve lucidity of wisdom? Staying in empty huts has destroyed the ascetic Gotama’s wisdom. Not frequenting assemblies, he is unable to hold a discussion. He just lurks on the periphery. He’s just like the nilgai antelope, circling around and lurking on the periphery. Please, householder, let the ascetic Gotama come to this assembly. I’ll sink him with just one question! I’ll roll him over and wrap him up like a hollow pot!” 

With\marginnote{6.1} clairaudience that is purified and superhuman, the Buddha heard this discussion between the householder \textsanskrit{Sandhāna} and the wanderer Nigrodha. Then the Buddha descended Vulture’s Peak Mountain and went to the peacocks’ feeding ground on the bank of the \textsanskrit{Sumāgadhā}, where he walked mindfully in the open air. 

Nigrodha\marginnote{6.3} saw him, and hushed his own assembly: “Be quiet, good sirs, don’t make a sound. The ascetic Gotama is walking mindfully on the bank of the \textsanskrit{Sumāgadhā}. The venerable likes quiet and praises quiet. Hopefully if he sees that our assembly is quiet he’ll see fit to approach. If he comes, I’ll ask him this question: ‘Sir, what teaching do you use to guide your disciples, through which they claim solace in the fundamental purpose of the spiritual life?’” Then those wanderers fell silent. 

\section*{2. Mortification in Disgust of Sin }

Then\marginnote{7.1} the Buddha went up to the wanderer Nigrodha, who said to him, “Come, Blessed One! Welcome, Blessed One! It’s been a long time since you took the opportunity to come here. Please, sir, sit down, this seat is ready.” The Buddha sat on the seat spread out, while Nigrodha took a low seat and sat to one side. The Buddha said to him, “Nigrodha, what were you sitting talking about just now? What conversation was left unfinished?” 

Nigrodha\marginnote{7.10} said, “Well, sir, I saw you walking mindfully and said: ‘If the ascetic Gotama comes, I’ll ask him this question: “Sir, what teaching do you use to guide your disciples, through which they claim solace in the fundamental purpose of the spiritual life?”’ This is the conversation that was unfinished when the Buddha arrived.” 

“It’s\marginnote{7.15} hard for you to understand this, Nigrodha, since you have a different view, creed, preference, practice, and tradition. Please ask me a question about the higher mortification in disgust of sin in your own tradition: ‘How are the conditions for the mortification in disgust of sin completed, and how are they incomplete?’” 

When\marginnote{7.18} he said this, those wanderers made an uproar, “It’s incredible, it’s amazing! The ascetic Gotama has such power and might! For he sets aside his own doctrine and invites discussion on the doctrine of others!” 

Then\marginnote{8.1} Nigrodha, having quieted those wanderers, said to the Buddha, “Sir, we teach mortification in disgust of sin, regarding it as essential and clinging to it. How are the conditions for the mortification in disgust of sin completed, and how are they incomplete?” 

“It’s\marginnote{8.4} when a mortifier goes naked, ignoring conventions. They lick their hands, and don’t come or wait when called. They don’t consent to food brought to them, or food prepared on purpose for them, or an invitation for a meal. They don’t receive anything from a pot or bowl; or from someone who keeps sheep, or who has a weapon or a shovel in their home; or where a couple is eating; or where there is a woman who is pregnant, breast-feeding, or who has a man in her home; or where there’s a dog waiting or flies buzzing. They accept no fish or meat or liquor or wine, and drink no beer. They go to just one house for alms, taking just one mouthful, or two houses and two mouthfuls, up to seven houses and seven mouthfuls. They feed on one saucer a day, two saucers a day, up to seven saucers a day. They eat once a day, once every second day, up to once a week, and so on, even up to once a fortnight. They live committed to the practice of eating food at set intervals. They eat herbs, millet, wild rice, poor rice, water lettuce, rice bran, scum from boiling rice, sesame flour, grass, or cow dung. They survive on forest roots and fruits, or eating fallen fruit. They wear robes of sunn hemp, mixed hemp, corpse-wrapping cloth, rags, lodh tree bark, antelope hide (whole or in strips), kusa grass, bark, wood-chips, human hair, horse-tail hair, or owls’ wings. They tear out their hair and beard, committed to this practice. They constantly stand, refusing seats. They squat, committed to persisting in the squatting position. They lie on a mat of thorns, making a mat of thorns their bed. They make their bed on a plank, or the bare ground. They lie only on one side. They wear dust and dirt. They stay in the open air. They sleep wherever they lay their mat. They eat unnatural things, committed to the practice of eating unnatural foods. They don’t drink, committed to the practice of not drinking liquids. They’re committed to the practice of immersion in water three times a day, including the evening. 

What\marginnote{8.22} do you think, Nigrodha? If this is so, is the mortification in disgust of sin complete, or incomplete?” 

“Clearly,\marginnote{8.24} sir, if that is so the mortification in disgust of sin is complete, not incomplete.” 

“But\marginnote{8.25} even such a completed mortification has many defects, I say.” 

\subsection*{2.1. Defects }

“But\marginnote{9.1} how does the Buddha say that even such a completed mortification has many defects?” 

“Firstly,\marginnote{9.2} a mortifier undertakes a practice of mortification. They’re happy with that, as they’ve got all they wished for. This is a defect in that mortifier. 

Furthermore,\marginnote{9.5} a mortifier undertakes a practice of mortification. They glorify themselves and put others down on account of that. This too is a defect in that mortifier. 

Furthermore,\marginnote{9.8} a mortifier undertakes a practice of mortification. They become indulgent and infatuated and fall into negligence on account of that. This too is a defect in that mortifier. 

Furthermore,\marginnote{10.1} a mortifier undertakes a practice of mortification. They generate possessions, honor, and popularity through that mortification. They’re happy with that, as they’ve got all they wished for. This too is a defect in that mortifier. 

Furthermore,\marginnote{10.4} a mortifier undertakes a practice of mortification. They generate possessions, honor, and popularity through that mortification. They glorify themselves and put others down on account of that. This too is a defect in that mortifier. 

Furthermore,\marginnote{10.7} a mortifier undertakes a practice of mortification. They generate possessions, honor, and popularity through that mortification. They become indulgent and infatuated and fall into negligence on account of that. This too is a defect in that mortifier. 

Furthermore,\marginnote{10.10} a mortifier becomes fussy about food, saying, ‘This agrees with me, this doesn’t agree with me.’ What doesn’t agree with them they reluctantly give up. But what does agree with them they eat tied, infatuated, attached, blind to the drawbacks, and not understanding the escape. This too is a defect in that mortifier. 

Furthermore,\marginnote{10.15} a mortifier undertakes a practice of mortification out of longing for possessions, honor, and popularity, thinking, ‘Kings, royal ministers, aristocrats, brahmins, householders, and sectarians will honor me!’ This too is a defect in that mortifier. 

Furthermore,\marginnote{11.1} a mortifier rebukes a certain ascetic or brahmin, ‘But what is this one doing, living in abundance! According to this ascetic’s doctrine, everything—plants propagated from roots, stems, cuttings, or joints; and those from regular seeds as the fifth—is crunched together like the thunder of a tooth-hammer!’ This too is a defect in that mortifier. 

Furthermore,\marginnote{11.5} a mortifier sees a certain ascetic or brahmin being honored, respected, esteemed, and venerated among good families. They think, ‘This one, who lives in abundance, is honored, respected, esteemed, and venerated among good families. But I, a self-mortifier who lives rough, am not honored, respected, esteemed, and venerated among good families.’ Thus they give rise to jealousy and stinginess regarding families. This too is a defect in that mortifier. 

Furthermore,\marginnote{11.10} a mortifier sits meditation only when people can see them. This too is a defect in that mortifier. 

Furthermore,\marginnote{11.12} a mortifier sneaks about among families, thinking, ‘This is part of my mortification; this is part of my mortification.’ This too is a defect in that mortifier. 

Furthermore,\marginnote{11.15} a mortifier sometimes behaves in an underhand manner. When asked whether something agrees with them, they say it does, even though it doesn’t. Or they say it doesn’t, even though it does. Thus they tell a deliberate lie. This too is a defect in that mortifier. 

Furthermore,\marginnote{12.1} a mortifier disagrees with the way that the Realized One or their disciple teaches Dhamma, even when they make a valid point. This too is a defect in that mortifier. 

Furthermore,\marginnote{12.3} a mortifier is irritable and hostile … offensive and contemptuous … jealous and stingy … devious and deceitful … obstinate and vain … they have wicked desires, falling under the sway of wicked desires … they have wrong view, being attached to an extremist view … they’re attached to their own views, holding them tight, and refusing to let go. This too is a defect in that mortifier. 

What\marginnote{12.15} do you think, Nigrodha? Are such mortifications defective or not?” 

“Clearly,\marginnote{12.17} sir, they’re defective. It’s possible that a mortifier might have all of these defects, let alone one or other of them.” 

\subsection*{2.2. On Reaching the Shoots }

“Firstly,\marginnote{13.1} Nigrodha, a mortifier undertakes a practice of mortification. But they’re not happy with that, as they still haven’t got all they wished for. So they’re pure on that point. 

Furthermore,\marginnote{13.4} a mortifier undertakes a practice of mortification. They don’t glorify themselves or put others down on account of that. So they’re pure on that point. 

They\marginnote{13.6} don’t become indulgent … 

Furthermore,\marginnote{13.8} a mortifier undertakes a practice of mortification. They generate possessions, honor, and popularity through that mortification. They’re not happy with that, as they still haven’t got all they wished for … 

They\marginnote{13.10} don’t glorify themselves and put others down on account of possessions, honor, and popularity … 

They\marginnote{13.12} don’t become indulgent because of it … So they’re pure on that point. 

Furthermore,\marginnote{13.14} a mortifier doesn’t become fussy about food, saying, ‘This agrees with me, this doesn’t agree with me.’ What doesn’t agree with them they readily give up. But what does agree with them they eat without being tied, infatuated, attached, seeing the drawbacks, and understanding the escape. So they’re pure on that point. 

Furthermore,\marginnote{13.19} a mortifier doesn’t undertake a practice of mortification out of longing for possessions, honor, and popularity … ‘Kings, royal ministers, aristocrats, brahmins, householders, and sectarians will honor me!’ So they’re pure on that point. 

Furthermore,\marginnote{13.22} a mortifier doesn’t rebuke a certain ascetic or brahmin, ‘But what is this one doing, living in abundance! According to this ascetic’s doctrine, everything—plants propagated from roots, stems, cuttings, or joints; and those from regular seeds as the fifth—is crunched together like the thunder of a tooth-hammer!’ So they’re pure on that point. 

Furthermore,\marginnote{14.1} a mortifier sees a certain ascetic or brahmin being honored, respected, esteemed, and venerated among good families. It never occurs to them, ‘This one, who lives in abundance, is honored, respected, esteemed, and venerated among good families. But I, a self-mortifier who lives rough, am not honored, respected, esteemed, and venerated among good families.’ Thus they don’t give rise to jealousy and stinginess regarding families. So they’re pure on that point. 

Furthermore,\marginnote{14.6} a mortifier doesn’t sit meditation only when people can see them. So they’re pure on that point. 

Furthermore,\marginnote{14.8} a mortifier doesn’t sneak about among families, thinking, ‘This is part of my mortification; this is part of my mortification.’ So they’re pure on that point. 

Furthermore,\marginnote{14.11} a mortifier never behaves in an underhand manner. When asked whether something agrees with them, they say it doesn’t when it doesn’t. Or they say it does when it does. Thus they don’t tell a deliberate lie. So they’re pure on that point. 

Furthermore,\marginnote{15.1} a mortifier agrees with the way that the Realized One or their disciple teaches Dhamma when they make a valid point. So they’re pure on that point. 

Furthermore,\marginnote{15.3} a mortifier is not irritable and hostile … offensive and contemptuous … jealous and stingy … devious and deceitful … obstinate and vain … they don’t have wicked desires … and wrong view … they’re not attached to their own views, holding them tight, and refusing to let go. So they’re pure on that point. 

What\marginnote{15.14} do you think, Nigrodha? If this is so, is the mortification in disgust of sin purified or not?” 

“Clearly,\marginnote{15.16} sir, it is purified. It has reached the peak and the pith.” 

“No,\marginnote{15.17} Nigrodha, at this point the mortification in disgust of sin has not yet reached the peak and the pith. Rather, it has only reached the shoots.” 

\subsection*{2.3. On Reaching the Bark }

“But\marginnote{16.1} at what point, sir, does the mortification in disgust of sin reach the peak and the pith? Please help me reach the peak and the pith!” 

“Nigrodha,\marginnote{16.3} take a mortifier who is restrained in the fourfold restraint. And how is a mortifier restrained in the fourfold restraint? It’s when a mortifier doesn’t kill living creatures, doesn’t get others to kill, and doesn’t approve of killing. They don’t steal, get others to steal, or approve of stealing. They don’t lie, get others to lie, or approve of lying. They don’t expect rewards from their practice, they don’t lead others to expect rewards, and they don’t approve of expecting rewards. That’s how a mortifier is restrained in the fourfold restraint. 

When\marginnote{16.10} a mortifier has the fourfold restraint, that is their mortification. They step forward, not falling back. They frequent a secluded lodging—a wilderness, the root of a tree, a hill, a ravine, a mountain cave, a charnel ground, a forest, the open air, a heap of straw. After the meal, they return from almsround, sit down cross-legged with their body straight, and establish mindfulness right there. Giving up desire for the world, they meditate with a heart rid of desire, cleansing the mind of desire. Giving up ill will and malevolence, they meditate with a mind rid of ill will, full of compassion for all living beings, cleansing the mind of ill will. Giving up dullness and drowsiness, they meditate with a mind rid of dullness and drowsiness, perceiving light, mindful and aware, cleansing the mind of dullness and drowsiness. Giving up restlessness and remorse, they meditate without restlessness, their mind peaceful inside, cleansing the mind of restlessness and remorse. Giving up doubt, they meditate having gone beyond doubt, not undecided about skillful qualities, cleansing the mind of doubt. 

They\marginnote{17.1} give up these five hindrances, corruptions of the heart that weaken wisdom. Then they meditate spreading a heart full of love to one direction, and to the second, and to the third, and to the fourth. In the same way above, below, across, everywhere, all around, they spread a heart full of love to the whole world—abundant, expansive, limitless, free of enmity and ill will. They meditate spreading a heart full of compassion … They meditate spreading a heart full of rejoicing … They meditate spreading a heart full of equanimity to one direction, and to the second, and to the third, and to the fourth. In the same way above, below, across, everywhere, all around, they spread a heart full of equanimity to the whole world—abundant, expansive, limitless, free of enmity and ill will. 

What\marginnote{17.6} do you think, Nigrodha? If this is so, is the mortification in disgust of sin purified or not?” 

“Clearly,\marginnote{17.8} sir, it is purified. It has reached the peak and the pith.” 

“No,\marginnote{17.9} Nigrodha, at this point the mortification in disgust of sin has not yet reached the peak and the pith. Rather, it has only reached the bark.” 

\subsection*{2.4. On Reaching the Softwood }

“But\marginnote{18.1} at what point, sir, does the mortification in disgust of sin reach the peak and the pith? Please help me reach the peak and the pith!” 

“Nigrodha,\marginnote{18.3} take a mortifier who is restrained in the fourfold restraint. They give up these five hindrances, corruptions of the heart that weaken wisdom. Then they meditate spreading a heart full of love … compassion … rejoicing … equanimity. 

They\marginnote{18.13} recollect many kinds of past lives, that is, one, two, three, four, five, ten, twenty, thirty, forty, fifty, a hundred, a thousand, a hundred thousand rebirths; many eons of the world contracting, many eons of the world expanding, many eons of the world contracting and expanding. They remember: ‘There, I was named this, my clan was that, I looked like this, and that was my food. This was how I felt pleasure and pain, and that was how my life ended. When I passed away from that place I was reborn somewhere else. There, too, I was named this, my clan was that, I looked like this, and that was my food. This was how I felt pleasure and pain, and that was how my life ended. When I passed away from that place I was reborn here.’ And so they recollect their many kinds of past lives, with features and details. 

What\marginnote{18.14} do you think, Nigrodha? If this is so, is the mortification in disgust of sin purified or not?” 

“Clearly,\marginnote{18.16} sir, it is purified. It has reached the peak and the pith.” 

“No,\marginnote{18.17} Nigrodha, at this point the mortification in disgust of sin has not yet reached the peak and the pith. Rather, it has only reached the softwood.” 

\section*{3. On Reaching the Heartwood }

“But\marginnote{19.1} at what point, sir, does the mortification in disgust of sin reach the peak and the pith? Please help me reach the peak and the pith!” 

“Nigrodha,\marginnote{19.3} take a mortifier who is restrained in the fourfold restraint. They give up these five hindrances, corruptions of the heart that weaken wisdom. Then they meditate spreading a heart full of love … equanimity … They recollect many kinds of past lives, with features and details. 

With\marginnote{19.12} clairvoyance that is purified and superhuman, they see sentient beings passing away and being reborn—inferior and superior, beautiful and ugly, in a good place or a bad place. They understand how sentient beings are reborn according to their deeds: ‘These dear beings did bad things by way of body, speech, and mind. They spoke ill of the noble ones; they had wrong view; and they chose to act out of that wrong view. When their body breaks up, after death, they’re reborn in a place of loss, a bad place, the underworld, hell. These dear beings, however, did good things by way of body, speech, and mind. They never spoke ill of the noble ones; they had right view; and they chose to act out of that right view. When their body breaks up, after death, they’re reborn in a good place, a heavenly realm.’ And so, with clairvoyance that is purified and superhuman, they see sentient beings passing away and being reborn—inferior and superior, beautiful and ugly, in a good place or a bad place. They understand how sentient beings are reborn according to their deeds. 

What\marginnote{19.13} do you think, Nigrodha? If this is so, is the mortification in disgust of sin purified or not?” 

“Clearly,\marginnote{19.15} sir, it is purified. It has reached the peak and the pith.” 

“Nigrodha,\marginnote{19.16} at this point the mortification in disgust of sin has reached the peak and the pith. Nigrodha, remember you said this to me: ‘Sir, what teaching do you use to guide your disciples, through which they claim solace in the fundamental purpose of the spiritual life?’ Well, there is something better and finer than this. That’s what I use to guide my disciples, through which they claim solace in the fundamental purpose of the spiritual life.” 

When\marginnote{19.20} he said this, those wanderers made an uproar, “In that case, we’re lost, and so is our tradition! We don’t know anything better or finer than that!” 

\section*{4. Nigrodha Feels Depressed }

Then\marginnote{20.1} the householder \textsanskrit{Sandhāna} realized, “Obviously, now these wanderers want to listen to what the Buddha says. They’re paying attention and applying their minds to understand!” 

So\marginnote{20.3} he said to the wanderer Nigrodha, “Nigrodha, remember you said this to me: ‘Surely, householder, you should know better! With whom does the ascetic Gotama converse? With whom does he engage in discussion? With whom does he achieve lucidity of wisdom? Staying in empty huts has destroyed the ascetic Gotama’s wisdom. Not frequenting assemblies, he is unable to hold a discussion. He just lurks on the periphery. He’s just like the nilgai antelope, circling around and lurking on the periphery. Please, householder, let the ascetic Gotama come to this assembly. I’ll sink him with just one question! I’ll roll him over and wrap him up like a hollow pot!’ Now the Blessed One, perfected and fully awakened, has arrived here. Why don’t you send him out of the assembly to the periphery like a nilgai antelope? Why don’t you sink him with just one question? Why don’t you roll him over and wrap him up like a hollow pot?” When he said this, Nigrodha sat silent, embarrassed, shoulders drooping, downcast, depressed, with nothing to say. 

Knowing\marginnote{21.1} this, the Buddha said to him, “Is it really true, Nigrodha—are those your words?” 

“It’s\marginnote{21.3} true, sir, those are my words. It was foolish, stupid, and unskillful of me.” 

“What\marginnote{21.4} do you think, Nigrodha? Have you heard that wanderers of the past who were elderly and senior, the teachers of teachers, said that when the perfected ones, the fully awakened Buddhas of the past came together, they made an uproar, a dreadful racket as they sat and talked about all kinds of unworthy topics, like you do in your tradition these days? Or did they say that the Buddhas frequented remote lodgings in the wilderness and the forest that are quiet and still, far from the madding crowd, remote from human settlements, and fit for retreat, like I do these days?” 

“I\marginnote{21.10} have heard that wanderers of the past who were elderly and senior, said that when the perfected ones, the fully awakened Buddhas of the past came together, they didn’t make an uproar, like I do in my tradition these days. They said that the Buddhas of the past frequented remote lodgings in the wilderness, like the Buddha does these days.” 

“Nigrodha,\marginnote{21.15} you are a sensible and mature man. Did it not occur to you: ‘The Blessed One is awakened, tamed, serene, crossed over, and extinguished. And he teaches Dhamma for awakening, taming, serenity, crossing over, and extinguishment’?” 

\section*{5. The Culmination of the Spiritual Path }

Nigrodha\marginnote{22.1} said, “I have made a mistake, sir. It was foolish, stupid, and unskillful of me to speak in that way. Please, sir, accept my mistake for what it is, so I will restrain myself in future.” 

“Indeed,\marginnote{22.4} Nigrodha, you made a mistake. It was foolish, stupid, and unskillful of you to speak in that way. But since you have recognized your mistake for what it is, and have dealt with it properly, I accept it. For it is growth in the training of the Noble One to recognize a mistake for what it is, deal with it properly, and commit to restraint in the future. Nigrodha, this is what I say: 

Let\marginnote{22.8} a sensible person come—neither devious nor deceitful, a person of integrity. I teach and instruct them. By practicing as instructed they will realize the supreme end of the spiritual path in this very life, in seven years. They will live having achieved with their own insight the goal for which gentlemen rightly go forth from the lay life to homelessness. Let alone seven years. Let a sensible person come—neither devious nor deceitful, a person of integrity. I teach and instruct them. By practicing as instructed they will realize the supreme end of the spiritual path in this very life, in six years … five years … four years … three years … two years … one year … seven months … six months … five months … four months … three months … two months … one month … a fortnight. Let alone a fortnight. Let a sensible person come—neither devious nor deceitful, a person of integrity. I teach and instruct them. By practicing as instructed they will realize the supreme end of the spiritual path in this very life, in seven days. 

\section*{6. The Wanderers Feel Depressed }

Nigrodha,\marginnote{23.1} you might think, ‘The ascetic Gotama speaks like this because he wants pupils.’ But you should not see it like this. Let your teacher remain your teacher. 

You\marginnote{23.5} might think, ‘The ascetic Gotama speaks like this because he wants us to give up our recitation.’ But you should not see it like this. Let your recitation remain as it is. 

You\marginnote{23.9} might think, ‘The ascetic Gotama speaks like this because he wants us to give up our livelihood.’ But you should not see it like this. Let your livelihood remain as it is. 

You\marginnote{23.13} might think, ‘The ascetic Gotama speaks like this because he wants us to start doing things that are unskillful and considered unskillful in our tradition.’ But you should not see it like this. Let those things that are unskillful and considered unskillful in your tradition remain as they are. 

You\marginnote{23.17} might think, ‘The ascetic Gotama speaks like this because he wants us to stop doing things that are skillful and considered skillful in our tradition.’ But you should not see it like this. Let those things that are skillful and considered skillful in your tradition remain as they are. 

I\marginnote{23.21} do not speak for any of these reasons. Nigrodha, there are things that are unskillful, corrupting, leading to future lives, hurtful, resulting in suffering and future rebirth, old age, and death. I teach Dhamma so that those things may be given up. When you practice accordingly, corrupting qualities will be given up in you and cleansing qualities will grow. You’ll enter and remain in the fullness and abundance of wisdom, having realized it with your own insight in this very life.” 

When\marginnote{24.1} this was said, those wanderers sat silent, dismayed, shoulders drooping, downcast, depressed, with nothing to say, as if their minds were possessed by \textsanskrit{Māra}. Then the Buddha thought, “All these foolish people have been touched by the Wicked One! For not even a single one thinks, ‘Come, let us lead the spiritual life under the ascetic Gotama for the sake of enlightenment—for what do seven days matter?’” 

Then\marginnote{24.6} the Buddha, having roared his lion’s roar in the lady \textsanskrit{Udumbarikā}’s monastery for wanderers, rose into the air and landed on Vulture’s Peak. Meanwhile, the householder \textsanskrit{Sandhāna} just went back to \textsanskrit{Rājagaha}. 

%
\chapter*{{\suttatitleacronym DN 26}{\suttatitletranslation The Wheel-Turning Monarch }{\suttatitleroot Cakkavattisutta}}
\addcontentsline{toc}{chapter}{\tocacronym{DN 26} \toctranslation{The Wheel-Turning Monarch } \tocroot{Cakkavattisutta}}
\markboth{The Wheel-Turning Monarch }{Cakkavattisutta}
\extramarks{DN 26}{DN 26}

\section*{1. Taking Refuge in Oneself }

\scevam{So\marginnote{1.1} I have heard. }At one time the Buddha was staying in the land of the Magadhans at \textsanskrit{Mātulā}. There the Buddha addressed the mendicants, “Mendicants!” 

“Venerable\marginnote{1.5} sir,” they replied. The Buddha said this: 

“Mendicants,\marginnote{1.7} live as your own island, your own refuge, with no other refuge. Let the teaching be your island and your refuge, with no other refuge. And how does a mendicant do this? They meditate observing an aspect of the body—keen, aware, and mindful, rid of desire and aversion for the world. They meditate observing an aspect of feelings … mind … principles—keen, aware, and mindful, rid of desire and aversion for the world. That’s how a mendicant lives as their own island, their own refuge, with no other refuge. That’s how they let the teaching be their island and their refuge, with no other refuge. 

You\marginnote{1.14} should roam inside your own territory, the domain of your fathers. If you roam inside your own territory, the domain of your fathers, \textsanskrit{Māra} won’t catch you or get hold of you. It is due to undertaking skillful qualities that this merit grows. 

\section*{2. King \textsanskrit{Daḷhanemi} }

Once\marginnote{2.1} upon a time, mendicants, there was a king named \textsanskrit{Daḷhanemi} who was a wheel-turning monarch, a just and principled king. His dominion extended to all four sides, he achieved stability in the country, and he possessed the seven treasures. He had the following seven treasures: the wheel, the elephant, the horse, the jewel, the woman, the treasurer, and the counselor as the seventh treasure. He had over a thousand sons who were valiant and heroic, crushing the armies of his enemies. After conquering this land girt by sea, he reigned by principle, without rod or sword. 

Then,\marginnote{3.1} after many years, many hundred years, many thousand years had passed, King \textsanskrit{Daḷhanemi} addressed one of his men, ‘My good man, when you see that the heavenly wheel-treasure has receded back from its place, please tell me.’ 

‘Yes,\marginnote{3.3} Your Majesty,’ replied that man. 

After\marginnote{3.4} many thousand years had passed, that man saw that the heavenly wheel-treasure had receded back from its place. So he went to King \textsanskrit{Daḷhanemi} and said, ‘Please sire, you should know that your heavenly wheel-treasure has receded back from its place.’ 

So\marginnote{3.6} the king summoned the crown prince and said, ‘Dear prince, my heavenly wheel-treasure has receded back from its place. I’ve heard that when this happens to a wheel-turning monarch, he does not have long to live. I have enjoyed human pleasures. Now it is time for me to seek heavenly pleasures. Come, dear prince, rule this land surrounded by ocean! I shall shave off my hair and beard, dress in ocher robes, and go forth from the lay life to homelessness.’ 

And\marginnote{3.13} so, after carefully instructing the crown prince in kingship, King \textsanskrit{Daḷhanemi} shaved off his hair and beard, dressed in ocher robes, and went forth from the lay life to homelessness. Seven days later the heavenly wheel-treasure vanished. 

Then\marginnote{4.1} a certain man approached the newly anointed aristocrat king and said, ‘Please sire, you should know that the heavenly wheel-treasure has vanished.’ At that the king was unhappy and experienced unhappiness. He went to the royal sage and said, ‘Please sire, you should know that the heavenly wheel-treasure has vanished.’ 

When\marginnote{4.6} he said this, the royal sage said to him, ‘Don’t be unhappy at the vanishing of the wheel-treasure. My dear, the wheel-treasure is not inherited from your father. Come now, my dear, implement the noble duties of a wheel-turning monarch. If you do so, it’s possible that—on a fifteenth day sabbath, having bathed your head and gone upstairs in the royal longhouse to observe the sabbath—the heavenly wheel-treasure will appear to you, with a thousand spokes, with rim and hub, complete in every detail.’ 

\subsection*{2.1. The Noble Duties of a Wheel-Turning Monarch }

‘But\marginnote{5.1} sire, what are the noble duties of a wheel-turning monarch?’ 

‘Well\marginnote{5.2} then, my dear, relying only on principle—honoring, respecting, and venerating principle, having principle as your flag, banner, and authority—provide just protection and security for your court, troops, aristocrats, vassals, brahmins and householders, people of town and country, ascetics and brahmins, beasts and birds. Do not let injustice prevail in the realm. Pay money to the penniless in the realm. 

And\marginnote{5.5} there are ascetics and brahmins in the realm who avoid intoxication and negligence, are settled in patience and gentleness, and who tame, calm, and extinguish themselves. From time to time you should go up to them and ask: “Sirs, what is skillful? What is unskillful? What is blameworthy? What is blameless? What should be cultivated? What should not be cultivated? Doing what leads to my lasting harm and suffering? Doing what leads to my lasting welfare and happiness?” Having heard them, you should reject what is unskillful and undertake and follow what is skillful. 

These\marginnote{5.8} are the noble duties of a wheel-turning monarch.’ 

\subsection*{2.2. The Wheel-Treasure Appears }

‘Yes,\marginnote{5.10} Your Majesty,’ replied the new king to the royal sage. And he implemented the noble duties of a wheel-turning monarch. 

While\marginnote{5.11} he was implementing them, on a fifteenth day sabbath, he had bathed his head and gone upstairs in the royal longhouse to observe the sabbath. And the heavenly wheel-treasure appeared to him, with a thousand spokes, with rim and hub, complete in every detail. Seeing this, the king thought, ‘I have heard that when the heavenly wheel-treasure appears to a king in this way, he becomes a wheel-turning monarch. Am I then a wheel-turning monarch?’ 

Then\marginnote{6.1} the anointed king, rising from his seat and arranging his robe over one shoulder, took a ceremonial vase in his left hand and besprinkled the wheel-treasure with his right hand, saying, ‘Roll forth, O wheel-treasure! Triumph, O wheel-treasure!’ 

Then\marginnote{6.3} the wheel-treasure rolled towards the east. And the king followed it together with his army of four divisions. In whatever place the wheel-treasure stood still, there the king came to stay together with his army. And any opposing rulers of the eastern quarter came to the wheel-turning monarch and said, ‘Come, great king! Welcome, great king! We are yours, great king, instruct us.’ The wheel-turning monarch said, ‘Do not kill living creatures. Do not steal. Do not commit sexual misconduct. Do not lie. Do not drink alcohol. Maintain the current level of taxation.’ And so the opposing rulers of the eastern quarter became his vassals. 

Then\marginnote{7.1} the wheel-treasure, having plunged into the eastern ocean and emerged again, rolled towards the south. … Having plunged into the southern ocean and emerged again, it rolled towards the west. … 

Having\marginnote{7.9} plunged into the western ocean and emerged again, it rolled towards the north, followed by the king together with his army of four divisions. In whatever place the wheel-treasure stood still, there the king came to stay together with his army. And any opposing rulers of the northern quarter came to the wheel-turning monarch and said, ‘Come, great king! Welcome, great king! We are yours, great king, instruct us.’ The wheel-turning monarch said, ‘Do not kill living creatures. Do not steal. Do not commit sexual misconduct. Do not lie. Do not drink alcohol. Maintain the current level of taxation.’ And so the rulers of the northern quarter became his vassals. 

And\marginnote{7.16} then the wheel-treasure, having triumphed over this land surrounded by ocean, returned to the royal capital. There it stood still by the gate to the royal compound at the High Court as if fixed to an axle, illuminating the royal compound. 

\section*{3. On Subsequent Wheel-Turning Monarchs }

And\marginnote{8.1} for a second time, and a third, a fourth, a fifth, a sixth, and a seventh time, a wheel-turning monarch was established in exactly the same way. And after many years the seventh wheel-turning monarch went forth, handing the realm over to the crown prince. 

Seven\marginnote{8.17} days later the heavenly wheel-treasure vanished. 

Then\marginnote{9.1} a certain man approached the newly anointed aristocrat king and said, ‘Please sire, you should know that the heavenly wheel-treasure has vanished.’ At that the king was unhappy and experienced unhappiness. But he didn’t go to the royal sage and ask about the noble duties of a wheel-turning monarch. He just governed the country according to his own ideas. So governed, the nations did not prosper like before, as they had when former kings implemented the noble duties of a wheel-turning monarch. 

Then\marginnote{9.7} the ministers and counselors, the treasury officials, military officers, guardsmen, and advisers gathered and said to the king, ‘Sire, when governed according to your own ideas, the nations do not prosper like before, as they did when former kings implemented the noble duties of a wheel-turning monarch. In your realm are found ministers and counselors, treasury officials, military officers, guardsmen, and advisers—both ourselves and others—who remember the noble duties of a wheel-turning monarch. Please, Your Majesty, ask us about the noble duties of a wheel-turning monarch. We will answer you.’ 

\section*{4. On the Period of Decline }

So\marginnote{10.1} the anointed king asked the assembled ministers and counselors, treasury officials, military officers, guardsmen, and advisers about the noble duties of a wheel-turning monarch. And they answered him. But after listening to them, he didn’t provide just protection and security. Nor did he pay money to the penniless in the realm. And so poverty grew widespread. 

When\marginnote{10.5} poverty was widespread, a certain person stole from others, with the intention to commit theft. They arrested him and presented him to the king, saying, ‘Your Majesty, this person stole from others with the intention to commit theft.’ 

The\marginnote{10.9} king said to that person, ‘Is it really true, mister, that you stole from others with the intention to commit theft?’ 

‘It’s\marginnote{10.11} true, sire.’ 

‘What\marginnote{10.12} was the reason?’ 

‘Sire,\marginnote{10.13} I can’t survive.’ 

So\marginnote{10.14} the king paid some money to that person, saying, ‘With this money, mister, keep yourself alive, and provide for your mother and father, partners and children. Work for a living, and establish an uplifting religious donation for ascetics and brahmins that’s conducive to heaven, ripens in happiness, and leads to heaven.’ 

‘Yes,\marginnote{10.16} Your Majesty,’ replied that man. 

But\marginnote{11.1} then another man stole something from others. They arrested him and presented him to the king, saying, ‘Your Majesty, this person stole from others.’ 

The\marginnote{11.5} king said to that person, ‘Is it really true, mister, that you stole from others?’ 

‘It’s\marginnote{11.7} true, sire.’ 

‘What\marginnote{11.8} was the reason?’ 

‘Sire,\marginnote{11.9} I can’t survive.’ 

So\marginnote{11.10} the king paid some money to that person, saying, ‘With this money, mister, keep yourself alive, and provide for your mother and father, partners and children. Work for a living, and establish an uplifting religious donation for ascetics and brahmins that’s conducive to heaven, ripens in happiness, and leads to heaven.’ 

‘Yes,\marginnote{11.12} Your Majesty,’ replied that man. 

People\marginnote{12.1} heard about this: ‘It seems the king is paying money to anyone who steals from others!’ It occurred to them, ‘Why don’t we steal from others?’ So then another man stole something from others. 

They\marginnote{12.6} arrested him and presented him to the king, saying, ‘Your Majesty, this person stole from others.’ 

The\marginnote{12.9} king said to that person, ‘Is it really true, mister, that you stole from others?’ 

‘It’s\marginnote{12.11} true, sire.’ 

‘What\marginnote{12.12} was the reason?’ 

‘Sire,\marginnote{12.13} I can’t survive.’ 

Then\marginnote{12.14} the king thought, ‘If I pay money to anyone who steals from others, it will only increase the stealing. I’d better make an end of this person, finish him off, and chop off his head.’ 

Then\marginnote{12.17} he ordered his men, ‘Well then, my men, tie this man’s arms tightly behind his back with a strong rope. Shave his head and march him from street to street and square to square to the beating of a harsh drum. Then take him out the south gate and make an end of him, finish him off, and chop off his head.’ 

‘Yes,\marginnote{12.19} Your Majesty,’ they replied, and did as he commanded. 

People\marginnote{13.1} heard about this: ‘It seems the king is chopping the head off anyone who steals from others!’ It occurred to them, ‘We’d better have sharp swords made. Then when we steal from others, we’ll make an end of them, finish them off, and chop off their heads.’ They had sharp swords made. Then they started to make raids on villages, towns, and cities, and to infest the highways. And they chopped the heads off anyone they stole from. 

And\marginnote{14.1} so, mendicants, from not paying money to the penniless, poverty became widespread. When poverty was widespread, theft became widespread. When theft was widespread, swords became widespread. When swords were widespread, killing living creatures became widespread. And for the sentient beings among whom killing was widespread, their lifespan and beauty declined. Those people lived for 80,000 years, but their children lived for 40,000 years. 

Among\marginnote{14.3} the people who lived for 40,000 years, a certain person stole something from others. They arrested him and presented him to the king, saying, ‘Your Majesty, this person stole from others.’ 

The\marginnote{14.7} king said to that person, ‘Is it really true, mister, that you stole from others?’ 

‘No,\marginnote{14.9} sire,’ he said, deliberately lying. 

And\marginnote{15.1} so, mendicants, from not paying money to the penniless, poverty, theft, swords, and killing became widespread. When killing was widespread, lying became widespread. And for the sentient beings among whom lying was widespread, their lifespan and beauty declined. Those people who lived for 40,000 years had children who lived for 20,000 years. 

Among\marginnote{15.3} the people who lived for 20,000 years, a certain person stole something from others. Someone else reported this to the king, ‘Your Majesty, such-and-such person stole from others,’ he said, going behind his back. 

And\marginnote{16.1} so, mendicants, from not paying money to the penniless, poverty, theft, swords, killing, and lying became widespread. When lying was widespread, backbiting became widespread. And for the sentient beings among whom backbiting was widespread, their lifespan and beauty declined. Those people who lived for 20,000 years had children who lived for 10,000 years. 

Among\marginnote{17.1} the people who lived for 10,000 years, some were more beautiful than others. And the ugly beings, coveting the beautiful ones, committed adultery with others’ wives. 

And\marginnote{17.3} so, mendicants, from not paying money to the penniless, poverty, theft, swords, killing, lying, and backbiting became widespread. When backbiting was widespread, sexual misconduct became widespread. And for the sentient beings among whom sexual misconduct was widespread, their lifespan and beauty declined. Those people who lived for 10,000 years had children who lived for 5,000 years. 

Among\marginnote{17.5} the people who lived for 5,000 years, two things became widespread: harsh speech and talking nonsense. For the sentient beings among whom these two things were widespread, their lifespan and beauty declined. Those people who lived for 5,000 years had some children who lived for 2,500 years, while others lived for 2,000 years. 

Among\marginnote{17.9} the people who lived for 2,500 years, desire and ill will became widespread. For the sentient beings among whom desire and ill will were widespread, their lifespan and beauty declined. Those people who lived for 2,500 years had children who lived for 1,000 years. 

Among\marginnote{17.12} the people who lived for 1,000 years, wrong view became widespread. For the sentient beings among whom wrong view was widespread, their lifespan and beauty declined. Those people who lived for 1,000 years had children who lived for five hundred years. 

Among\marginnote{17.15} the people who lived for five hundred years, three things became widespread: illicit desire, immoral greed, and wrong thoughts. For the sentient beings among whom these three things were widespread, their lifespan and beauty declined. Those people who lived for five hundred years had some children who lived for two hundred and fifty years, while others lived for two hundred years. 

Among\marginnote{17.19} the people who lived for two hundred and fifty years, three things became widespread: lack of due respect for mother and father, ascetics and brahmins, and failure to honor the elders in the family. 

And\marginnote{18.1} so, mendicants, from not paying money to the penniless, all these things became widespread—poverty, theft, swords, killing, lying, backbiting, sexual misconduct, harsh speech and talking nonsense, desire and ill will, wrong view, illicit desire, immoral greed, and wrong thoughts, and lack of due respect for mother and father, ascetics and brahmins, and failure to honor the elders in the family. For the sentient beings among whom these things were widespread, their lifespan and beauty declined. Those people who lived for two hundred and fifty years had children who lived for a hundred years. 

\section*{5. When People Live for Ten Years }

There\marginnote{19.1} will come a time, mendicants, when these people will have children who live for ten years. Among the people who live for ten years, girls will be marriageable at five. The following flavors will disappear: ghee, butter, oil, honey, molasses, and salt. The best kind of food will be finger millet, just as fine rice with meat is the best kind of food today. 

The\marginnote{19.8} ten ways of doing skillful deeds will totally disappear, and the ten ways of doing unskillful deeds will explode in popularity. Those people will not even have the word ‘skillful’, still less anyone who does what is skillful. And anyone who disrespects mother and father, ascetics and brahmins, and fails to honor the elders in the family will be venerated and praised, just as the opposite is venerated and praised today. 

There’ll\marginnote{20.1} be no recognition of the status of mother, aunts, or wives and partners of teachers and respected people. The world will become promiscuous, like goats and sheep, chickens and pigs, and dogs and jackals. 

They’ll\marginnote{20.3} be full of hostility towards each other, with acute ill will, malevolence, and thoughts of murder. Even a mother will feel like this for her child, and the child for its mother, father for child, child for father, brother for sister, and sister for brother. They’ll be just like a deer hunter when he sees a deer—full of hostility, ill will, malevolence, and thoughts of killing. 

Among\marginnote{21.1} the people who live for ten years, there will be an interregnum of swords lasting seven days. During that time they will see each other as beasts. Sharp swords will appear in their hands, with which they’ll take each other’s life, crying, ‘It’s a beast! It’s a beast!’ 

But\marginnote{21.5} then some of those beings will think, ‘Let us neither be perpetrators nor victims! Why don’t we hide in thick grass, thick jungle, thick trees, inaccessible riverlands, or rugged mountains and survive on forest roots and fruits?’ So that’s what they do. 

When\marginnote{21.8} those seven days have passed, having emerged from their hiding places and embraced each other, they will come together in one voice and cry, ‘How fantastic, dear being, you live! How fantastic, dear being, you live!’ 

\section*{6. The Period of Growth }

Then\marginnote{21.11} those beings will think, ‘It’s because we undertook unskillful things that we suffered such an extensive loss of our relatives. We’d better do what’s skillful. What skillful thing should we do? Why don’t we refrain from killing living creatures? Having undertaken this skillful thing we’ll live by it.’ So that’s what they do. Because of undertaking this skillful thing, their lifespan and beauty will grow. Those people who live for ten years will have children who live for twenty years. 

Then\marginnote{22.1} those beings will think, ‘Because of undertaking this skillful thing, our lifespan and beauty are growing. Why don’t we do even more skillful things? What skillful thing should we do? Why don’t we refrain from stealing … sexual misconduct … lying … backbiting … harsh speech … and talking nonsense. Why don’t we give up covetousness … ill will … wrong view … three things: illicit desire, immoral greed, and wrong thoughts. Why don’t we pay due respect to mother and father, ascetics and brahmins, honoring the elders in our families? Having undertaken this skillful thing we’ll live by it.’ So that’s what they do. 

Because\marginnote{22.18} of undertaking this skillful thing, their lifespan and beauty will grow. Those people who live for twenty years will have children who live for forty years. Those people who live for forty years will have children who live for eighty years, then a hundred and sixty years, three hundred and twenty years, six hundred and forty years, 2,000 years, 4,000 years, 8,000 years, 20,000 years, 40,000 years, and finally 80,000 years. Among the people who live for 80,000 years, girls will be marriageable at five hundred. 

\section*{7. The Time of King \textsanskrit{Saṅkha} }

Among\marginnote{23.5} the people who live for 80,000 years, there will be just three afflictions: greed, starvation, and old age. India will be successful and prosperous. The villages, towns, and capital cities will be no more than a chicken’s flight apart. And the land will be as crowded as hell, just full of people, like a thicket of rushes or reeds. The royal capital will be our Benares, but renamed Ketumati. And it will be successful, prosperous, populous, full of people, with plenty of food. There will be 84,000 cities in India, with the royal capital of Ketumati foremost. 

And\marginnote{24.1} in the royal capital of Ketumati a king named \textsanskrit{Saṅkha} will arise, a wheel-turning monarch, a just and principled king. His dominion will extend to all four sides, he will achieve stability in the country, and possess the seven treasures. He will have the following seven treasures: the wheel, the elephant, the horse, the jewel, the woman, the treasurer, and the counselor as the seventh treasure. He will have over a thousand sons who are valiant and heroic, crushing the armies of his enemies. After conquering this land girt by sea, he will reign by principle, without rod or sword. 

\section*{8. The Arising of the Buddha Metteyya }

And\marginnote{25.1} the Blessed One named Metteyya will arise in the world—perfected, a fully awakened Buddha, accomplished in knowledge and conduct, holy, knower of the world, supreme guide for those who wish to train, teacher of gods and humans, awakened, blessed—just as I have arisen today. He will realize with his own insight this world—with its gods, \textsanskrit{Māras} and \textsanskrit{Brahmās}, this population with its ascetics and brahmins, gods and humans—and make it known to others, just as I do today. He will teach the Dhamma that’s good in the beginning, good in the middle, and good in the end, meaningful and well-phrased. And he will reveal a spiritual practice that’s entirely full and pure, just as I do today. He will look after a \textsanskrit{Saṅgha} of many thousand mendicants, just as I look after a \textsanskrit{Saṅgha} of many hundreds today. 

Then\marginnote{26.1} King \textsanskrit{Saṅkha} will have the sacrificial post that had been built by King \textsanskrit{Mahāpanāda} raised up. After staying there, he will give it away to ascetics and brahmins, paupers, vagrants, travelers, and beggars. Then, having shaved off his hair and beard and dressed in ocher robes, he will go forth from the lay life to homelessness in the Buddha Metteyya’s presence. Soon after going forth, living withdrawn, diligent, keen, and resolute, he will realize the supreme end of the spiritual path in this very life. He will live having achieved with his own insight the goal for which gentlemen rightly go forth from the lay life to homelessness. 

Mendicants,\marginnote{27.1} live as your own island, your own refuge, with no other refuge. Let the teaching be your island and your refuge, with no other refuge. And how does a mendicant do this? It’s when a mendicant meditates by observing an aspect of the body—keen, aware, and mindful, rid of desire and aversion for the world. They meditate observing an aspect of feelings … mind … principles—keen, aware, and mindful, rid of desire and aversion for the world. That’s how a mendicant lives as their own island, their own refuge, with no other refuge. That’s how they let the teaching be their island and their refuge, with no other refuge. 

\section*{9. On Long Life and Beauty for Mendicants }

Mendicants,\marginnote{28.1} you should roam inside your own territory, the domain of your fathers. Doing so, you will grow in life span, beauty, happiness, wealth, and power. 

And\marginnote{28.3} what is long life for a mendicant? It’s when a mendicant develops the basis of psychic power that has immersion due to enthusiasm, and active effort. They develop the basis of psychic power that has immersion due to energy, and active effort. They develop the basis of psychic power that has immersion due to mental development, and active effort. They develop the basis of psychic power that has immersion due to inquiry, and active effort. Having developed and cultivated these four bases of psychic power they may, if they wish, live on for the eon or what’s left of the eon. This is long life for a mendicant. 

And\marginnote{28.10} what is beauty for a mendicant? It’s when a mendicant is ethical, restrained in the monastic code, conducting themselves well and seeking alms in suitable places. Seeing danger in the slightest fault, they keep the rules they’ve undertaken. This is beauty for a mendicant. 

And\marginnote{28.13} what is happiness for a mendicant? It’s when a mendicant, quite secluded from sensual pleasures, secluded from unskillful qualities, enters and remains in the first absorption, which has the rapture and bliss born of seclusion, while placing the mind and keeping it connected. As the placing of the mind and keeping it connected are stilled, they enter and remain in the second absorption … third absorption … fourth absorption. This is happiness for a mendicant. 

And\marginnote{28.19} what is wealth for a mendicant? It’s when a monk meditates spreading a heart full of love to one direction, and to the second, and to the third, and to the fourth. In the same way above, below, across, everywhere, all around, they spread a heart full of love to the whole world—abundant, expansive, limitless, free of enmity and ill will. They meditate spreading a heart full of compassion … rejoicing … equanimity to one direction, and to the second, and to the third, and to the fourth. In the same way above, below, across, everywhere, all around, they spread a heart full of equanimity to the whole world—abundant, expansive, limitless, free of enmity and ill will. This is wealth for a mendicant. 

And\marginnote{28.25} what is power for a mendicant? It’s when a mendicant realizes the undefiled freedom of heart and freedom by wisdom in this very life. And they live having realized it with their own insight due to the ending of defilements. This is power for a mendicant. 

Mendicants,\marginnote{28.28} I do not see a single power so hard to defeat as the power of \textsanskrit{Māra}. It is due to undertaking skillful qualities that this merit grows.” 

That\marginnote{28.30} is what the Buddha said. Satisfied, the mendicants were happy with what the Buddha said. 

%
\chapter*{{\suttatitleacronym DN 27}{\suttatitletranslation What Came First }{\suttatitleroot Aggaññasutta}}
\addcontentsline{toc}{chapter}{\tocacronym{DN 27} \toctranslation{What Came First } \tocroot{Aggaññasutta}}
\markboth{What Came First }{Aggaññasutta}
\extramarks{DN 27}{DN 27}

\scevam{So\marginnote{1.1} I have heard. }At one time the Buddha was staying near \textsanskrit{Sāvatthī} in the Eastern Monastery, in the stilt longhouse of \textsanskrit{Migāra}’s mother. 

Now\marginnote{1.3} at that time \textsanskrit{Vāseṭṭha} and \textsanskrit{Bhāradvāja} were living on probation among the mendicants in hopes of being ordained. Then in the late afternoon, the Buddha came downstairs from the longhouse and was walking mindfully in the open air, beneath the shade of the longhouse. 

\textsanskrit{Vāseṭṭha}\marginnote{2.1} saw him and said to \textsanskrit{Bhāradvāja}, “Reverend \textsanskrit{Bhāradvāja}, the Buddha is walking mindfully in the open air, beneath the shade of the longhouse. Come, reverend, let’s go to the Buddha. Hopefully we’ll get to hear a Dhamma talk from him.” 

“Yes,\marginnote{2.6} reverend,” replied \textsanskrit{Bhāradvāja}. 

So\marginnote{2.7} they went to the Buddha, bowed, and walked beside him. 

Then\marginnote{3.1} the Buddha said to \textsanskrit{Vāseṭṭha}, “\textsanskrit{Vāseṭṭha}, you are both brahmins by birth and clan, and have gone forth from the lay life to homelessness from a brahmin family. I hope you don’t have to suffer abuse and insults from the brahmins.” 

“Actually,\marginnote{3.3} sir, the brahmins do insult and abuse us with their typical insults to the fullest extent.” 

“But\marginnote{3.4} how do the brahmins insult you?” 

“Sir,\marginnote{3.5} the brahmins say: ‘Only brahmins are the first caste; other castes are inferior. Only brahmins are the light caste; other castes are dark. Only brahmins are purified, not others. Only brahmins are \textsanskrit{Brahmā}’s rightful sons, born of his mouth, born of \textsanskrit{Brahmā}, created by \textsanskrit{Brahmā}, heirs of \textsanskrit{Brahmā}. You’ve both abandoned the first caste to join an inferior caste, namely these shavelings, fake ascetics, riffraff, black spawn from the feet of our Kinsman. This is not right, it’s not proper!’ That’s how the brahmins insult us.” 

“Actually,\marginnote{4.1} \textsanskrit{Vāseṭṭha}, the brahmins are forgetting their tradition when they say this to you. For brahmin women are seen menstruating, being pregnant, giving birth, and breast-feeding. Yet even though they’re born from a brahmin womb they say: ‘Only brahmins are the first caste; other castes are inferior. Only brahmins are the light caste; other castes are dark. Only brahmins are purified, not others. Only brahmins are \textsanskrit{Brahmā}’s rightful sons, born of his mouth, born of \textsanskrit{Brahmā}, created by \textsanskrit{Brahmā}, heirs of \textsanskrit{Brahmā}.’ They misrepresent the brahmins, speak falsely, and make much bad karma. 

\section*{1. Purification in the Four Castes }

\textsanskrit{Vāseṭṭha},\marginnote{5.1} there are these four castes: aristocrats, brahmins, merchants, and workers. Some aristocrats kill living creatures, steal, and commit sexual misconduct. They use speech that’s false, divisive, harsh, and nonsensical. And they’re covetous, malicious, with wrong view. These things are unskillful, blameworthy, not to be cultivated, unworthy of the noble ones—and are reckoned as such. They are dark deeds with dark results, criticized by sensible people. Such things are seen in some aristocrats. And they are also seen among some brahmins, merchants, and workers. 

But\marginnote{6.1} some aristocrats refrain from killing living creatures, stealing, and committing sexual misconduct. They refrain from speech that’s false, divisive, harsh, and nonsensical. And they’re content, kind-hearted, with right view. These things are skillful, blameless, to be cultivated, worthy of the noble ones—and are reckoned as such. They are bright deeds with bright results, praised by sensible people. Such things are seen in some aristocrats. And they are also seen among some brahmins, merchants, and workers. 

Both\marginnote{7.1} these things occur like this, mixed up in these four castes—the dark and the bright, that which is praised and that which is criticized by sensible people. Yet of this the brahmins say: ‘Only brahmins are the first caste; other castes are inferior. Only brahmins are the light caste; other castes are dark. Only brahmins are purified, not others. Only brahmins are \textsanskrit{Brahmā}’s rightful sons, born of his mouth, born of \textsanskrit{Brahmā}, created by \textsanskrit{Brahmā}, heirs of \textsanskrit{Brahmā}.’ 

Sensible\marginnote{7.6} people don’t acknowledge this. Why is that? Because any mendicant from these four castes who is perfected—with defilements ended, who has completed the spiritual journey, done what had to be done, laid down the burden, achieved their own true goal, utterly ended the fetters of rebirth, and is rightly freed through enlightenment—is said to be foremost by virtue of principle, not without principle. For principle, \textsanskrit{Vāseṭṭha}, is the first thing for people in both this life and the next. 

And\marginnote{7.10} here’s a way to understand how this is so. 

King\marginnote{8.1} Pasenadi of Kosala knows that the ascetic Gotama has gone forth from the neighboring clan of the Sakyans. And the Sakyans are his vassals. The Sakyans show deference to King Pasenadi by bowing down, rising up, greeting him with joined palms, and observing proper etiquette for him. Now, King Pasenadi shows the same kind of deference to the Realized One. But he doesn’t think: ‘The ascetic Gotama is well-born, I am ill-born. He is powerful, I am weak. He is handsome, I am ugly. He is influential, I am insignificant.’ Rather, in showing such deference to the Realized One he is only honoring, respecting, and venerating principle. And here’s another way to understand how principle is the first thing for people in both this life and the next. 

\textsanskrit{Vāseṭṭha},\marginnote{9.1} you have different births, names, and clans, and have gone forth from the lay life to homelessness from different families. When they ask you what you are, you claim to be ascetics, followers of the Sakyan. But only when someone has faith in the Realized One—settled, rooted, and planted deep, strong, not to be shifted by any ascetic or brahmin or god or \textsanskrit{Māra} or \textsanskrit{Brahmā} or by anyone in the world—is it appropriate for them to say: ‘I am the Buddha’s true-born child, born from his mouth, born of principle, created by principle, heir to principle.’ Why is that? For these are terms for the Realized One: ‘the embodiment of truth’, and ‘the embodiment of holiness’, and ‘the one who has become the truth’, and ‘the one who has become holy’. 

There\marginnote{10.1} comes a time when, \textsanskrit{Vāseṭṭha}, after a very long period has passed, this cosmos contracts. As the cosmos contracts, sentient beings are mostly headed for the realm of streaming radiance. There they are mind-made, feeding on rapture, self-luminous, moving through the sky, steadily glorious, and they remain like that for a very long time. 

There\marginnote{10.4} comes a time when, after a very long period has passed, this cosmos expands. As the cosmos expands, sentient beings mostly pass away from that host of radiant deities and come back to this realm. Here they are mind-made, feeding on rapture, self-luminous, moving through the sky, steadily glorious, and they remain like that for a very long time. 

\section*{2. Solid Nectar Appears }

But\marginnote{11.1} the single mass of water at that time was utterly dark. The moon and sun were not found, nor were stars and constellations, day and night, months and fortnights, years and seasons, or male and female. Beings were simply known as ‘beings’. After a very long period had passed, solid nectar curdled in the water. It appeared just like the curd on top of hot milk-rice as it cools. It was beautiful, fragrant, and delicious, like ghee or butter. And it was as sweet as pure manuka honey. Now, one of those beings was reckless. Thinking, ‘Oh my, what might this be?’ they tasted the solid nectar with their finger. They enjoyed it, and craving was born in them. And other beings, following that being’s example, tasted solid nectar with their fingers. They too enjoyed it, and craving was born in them. 

\section*{3. The Moon and Sun Appear }

Then\marginnote{12.1} those beings started to eat the solid nectar, breaking it into lumps. But when they did this their luminosity vanished. And with the vanishing of their luminosity the moon and sun appeared, stars and constellations appeared, days and nights were distinguished, and so were months and fortnights, and years and seasons. To this extent the world had evolved once more. 

Then\marginnote{13.1} those beings eating the solid nectar, with that as their food and nourishment, remained for a very long time. But so long as they ate that solid nectar, their bodies became more solid and they diverged in appearance; some beautiful, some ugly. And the beautiful beings looked down on the ugly ones: ‘We’re more beautiful, they’re the ugly ones!’ And the vanity of the beautiful ones made the solid nectar vanish. They gathered together and bemoaned, ‘Oh, what a taste! Oh, what a taste!’ And even today when people get something tasty they say: ‘Oh, what a taste! Oh, what a taste!’ They’re just remembering an ancient primordial saying, but they don’t understand what it means. 

\section*{4. Ground-Sprouts }

When\marginnote{14.1} the solid nectar had vanished, ground-sprouts appeared to those beings. They appeared just like mushrooms. They were beautiful, fragrant, and delicious, like ghee or butter. And they were as sweet as pure manuka honey. 

Then\marginnote{14.5} those beings started to eat the ground-sprouts. With that as their food and nourishment, they remained for a very long time. But so long as they ate those ground-sprouts, their bodies became more solid and they diverged in appearance; some beautiful, some ugly. And the beautiful beings looked down on the ugly ones: ‘We’re more beautiful, they’re the ugly ones!’ And the vanity of the beautiful ones made the ground-sprouts vanish. 

\section*{5. Bursting Pods }

When\marginnote{14.13} the ground-sprouts had vanished, bursting pods appeared, like the fruit of the kadam tree. They were beautiful, fragrant, and delicious, like ghee or butter. And they were as sweet as pure manuka honey. 

Then\marginnote{15.1} those beings started to eat the bursting pods. With that as their food and nourishment, they remained for a very long time. But so long as they ate those bursting pods, their bodies became more solid and they diverged in appearance; some beautiful, some ugly. And the beautiful beings looked down on the ugly ones: ‘We’re more beautiful, they’re the ugly ones!’ And the vanity of the beautiful ones made the bursting pods vanish. 

They\marginnote{15.8} gathered together and bemoaned, ‘Oh, what we’ve lost! Oh, what we’ve lost—those bursting pods!’ And even today when people experience suffering they say: ‘Oh, what we’ve lost! Oh, what we’ve lost!’ They’re just remembering an ancient primordial saying, but they don’t understand what it means. 

\section*{6. Ripe Untilled Rice }

When\marginnote{16.1} the bursting pods had vanished, ripe untilled rice appeared to those beings. It had no powder or husk, pure and fragrant, with only the rice-grain. What they took for supper in the evening, by the morning had grown back and ripened. And what they took for breakfast in the morning had grown back and ripened by the evening, so the cutting didn’t show. Then those beings eating the ripe untilled rice, with that as their food and nourishment, remained for a very long time. 

\section*{7. Gender Appears }

But\marginnote{16.7} so long as they ate that ripe untilled rice, their bodies became more solid and they diverged in appearance. And female characteristics appeared on women, while male characteristics appeared on men. Women spent too much time gazing at men, and men at women. They became lustful, and their bodies burned with fever. Due to this fever they had sex with each other. 

Those\marginnote{16.11} who saw them having sex pelted them with dirt, ashes, or cow-dung, saying, ‘Get lost, filth! Get lost, filth! How on earth can one being do that to another?’ And even today people in some countries, when a bride is carried off, pelt her with dirt, ashes, or cow-dung. They’re just remembering an ancient primordial saying, but they don’t understand what it means. 

\section*{8. Sexual Intercourse }

What\marginnote{17.1} was reckoned as immoral at that time, these days is reckoned as moral. The beings who had sex together weren’t allowed to enter a village or town for one or two months. Ever since they excessively threw themselves into immorality, they started to make buildings to hide their immoral deeds. Then one of those beings of idle disposition thought, ‘Hey now, why should I be bothered to gather rice in the evening for supper, and in the morning for breakfast? Why don’t I gather rice for supper and breakfast all at once?’ 

So\marginnote{17.8} that’s what he did. Then one of the other beings approached that being and said, ‘Come, good being, we shall go to gather rice.’ ‘There’s no need, good being! I gathered rice for supper and breakfast all at once.’ So that being, following their example, gathered rice for two days all at once, thinking: ‘This seems fine.’ 

Then\marginnote{17.13} one of the other beings approached that being and said, ‘Come, good being, we shall go to gather rice.’ ‘There’s no need, good being! I gathered rice for two days all at once.’ So that being, following their example, gathered rice for four days all at once, thinking: ‘This seems fine.’ 

Then\marginnote{17.17} one of the other beings approached that being and said, ‘Come, good being, we shall go to gather rice.’ ‘There’s no need, good being! I gathered rice for four days all at once.’ So that being, following their example, gathered rice for eight days all at once, thinking: ‘This seems fine.’ 

But\marginnote{17.21} when they started to store up rice to eat, the rice grains became wrapped in powder and husk, it didn’t grow back after reaping, the cutting showed, and the rice stood in clumps. 

\section*{9. Dividing the Fields }

Then\marginnote{18.1} those beings gathered together and bemoaned, ‘Oh, how wicked things have appeared among beings! For we used to be mind-made, feeding on rapture, self-luminous, moving through the sky, steadily glorious, and we remained like that for a very long time. After a very long period had passed, solid nectar curdled in the water. But due to bad, unskillful things among us, the savory nectar vanished, the ground-sprouts vanished, the bursting pods vanished, and now the rice grains have become wrapped in powder and husk, it doesn’t grow back after reaping, the cutting shows, and the rice stands in clumps. We’d better divide up the rice and set boundaries.’ So that’s what they did. 

Now,\marginnote{19.1} one of those beings was reckless. While guarding their own share they took another’s share without it being given, and ate it. 

They\marginnote{19.2} grabbed the one who had done this and said, ‘You have done a bad thing, good being, in that while guarding your own share you took another’s share without it being given, and ate it. Do not do such a thing again.’ 

‘Yes,\marginnote{19.5} sirs,’ replied that being. But for a second time, and a third time they did the same thing, and were told not to continue. And then they struck that being, some with fists, others with stones, and still others with rods. From that day on stealing was found, and blaming and lying and the taking up of rods. 

\section*{10. The Elected King }

Then\marginnote{20.1} those beings gathered together and bemoaned, ‘Oh, how wicked things have appeared among beings, in that stealing is found, and blaming and lying and the taking up of rods! Why don’t we elect one being who would rightly accuse those who deserve it, blame those who deserve it, and expel those who deserve it? We shall pay them with a share of rice.’ 

Then\marginnote{20.5} those beings approached the being among them who was most attractive, good-looking, lovely, and illustrious, and said, ‘Come, good being, rightly accuse those who deserve it, blame those who deserve it, and banish those who deserve it. We shall pay you with a share of rice.’ ‘Yes, sirs,’ replied that being. They acted accordingly, and were paid with a share of rice. 

‘Elected\marginnote{21.1} by the people’, \textsanskrit{Vāseṭṭha}, is the meaning of ‘elected one’, the first term to be specifically invented for them. 

‘Lord\marginnote{21.2} of the fields’ is the meaning of ‘aristocrat’, the second term to be specifically invented. 

‘They\marginnote{21.3} please others with principle’ is the meaning of ‘king’, the third term to be specifically invented. 

And\marginnote{21.4} that, \textsanskrit{Vāseṭṭha}, is how the ancient primordial terms for the circle of aristocrats were created; for those very beings, not others; for those like them, not unlike; legitimately, not illegitimately. For principle, \textsanskrit{Vāseṭṭha}, is the first thing for people in both this life and the next. 

\section*{11. The Circle of Brahmins }

Then\marginnote{22.1} some of those same beings thought, ‘Oh, how wicked things have appeared among beings, in that stealing is found, and blaming and lying and the taking up of rods and banishment! Why don’t we set aside bad, unskillful things?’ So that’s what they did. 

‘They\marginnote{22.5} set aside bad, unskillful things’ is the meaning of ‘brahmin’, the first term to be specifically invented for them. 

They\marginnote{22.6} built leaf huts in a wilderness region where they meditated pure and bright, without lighting cooking fires or digging the soil. They came down in the morning for breakfast and in the evening for supper to the village, town, or royal capital seeking a meal. When they had obtained food they continued to meditate in the leaf huts. 

When\marginnote{22.8} people noticed this they said, ‘These beings build leaf huts in a wilderness region where they meditate pure and bright, without lighting cooking fires or digging the soil. They come down in the morning for breakfast and in the evening for supper to the village, town, or royal capital seeking a meal. When they have obtained food they continue to meditate in the leaf huts.’ 

‘They\marginnote{22.11} meditate’ is the meaning of ‘meditator’, the second term to be specifically invented for them. 

But\marginnote{23.1} some of those beings were unable to keep up with their meditation in the leaf huts in the wilderness. They came down to the neighborhood of a village or town where they dwelt compiling texts. 

When\marginnote{23.2} people noticed this they said, ‘These beings were unable to keep up with their meditation in the leaf huts in the wilderness. They came down to the neighborhood of a village or town where they dwelt compiling texts. Now they don’t meditate.’ 

‘Now\marginnote{23.4} they don’t meditate’ is the meaning of ‘reciter’, the third term to be specifically invented for them. What was reckoned as lesser at that time, these days is reckoned as primary. 

And\marginnote{23.6} that, \textsanskrit{Vāseṭṭha}, is how the ancient primordial terms for the circle of brahmins were created; for those very beings, not others; for those like them, not unlike; legitimately, not illegitimately. For principle, \textsanskrit{Vāseṭṭha}, is the first thing for people in both this life and the next. 

\section*{12. The Circle of Merchants }

Some\marginnote{24.1} of those same beings, taking up an active sex life, applied themselves to various jobs. 

‘Having\marginnote{24.2} taken up an active sex life, they apply themselves to various jobs’ is the meaning of ‘merchant’, the term specifically invented for them. 

And\marginnote{24.3} that, \textsanskrit{Vāseṭṭha}, is how the ancient primordial term for the circle of merchants was created; for those very beings, not others; for those like them, not unlike; legitimately, not illegitimately. For principle, \textsanskrit{Vāseṭṭha}, is the first thing for people in both this life and the next. 

\section*{13. The Circle of Workers }

The\marginnote{25.1} remaining beings lived by hunting and menial tasks. 

‘They\marginnote{25.2} live by hunting and menial tasks’ is the meaning of ‘worker’, the term specifically invented for them. 

And\marginnote{25.3} that, \textsanskrit{Vāseṭṭha}, is how the ancient primordial term for the circle of workers was created; for those very beings, not others; for those like them, not unlike; legitimately, not illegitimately. For principle, \textsanskrit{Vāseṭṭha}, is the first thing for people in both this life and the next. 

There\marginnote{26.1} came a time when an aristocrat, brahmin, merchant, or worker, deprecating their own vocation, went forth from the lay life to homelessness, thinking, ‘I will be an ascetic.’ 

And\marginnote{26.7} that, \textsanskrit{Vāseṭṭha}, is how these four circles were created; for those very beings, not others; for those like them, not unlike; legitimately, not illegitimately. For principle, \textsanskrit{Vāseṭṭha}, is the first thing for people in both this life and the next. 

\section*{14. On Bad Conduct }

An\marginnote{27.1} aristocrat, brahmin, merchant, worker, or ascetic may do bad things by way of body, speech, and mind. They have wrong view, and they act out of that wrong view. And because of that, when their body breaks up, after death, they’re reborn in a place of loss, a bad place, the underworld, hell. 

An\marginnote{28.1} aristocrat, brahmin, merchant, worker, or ascetic may do good things by way of body, speech, and mind. They have right view, and they act out of that right view. And because of that, when their body breaks up, after death, they’re reborn in a good place, a heavenly realm. 

\section*{15. The Qualities That Lead to Awakening }

An\marginnote{30.1} aristocrat, brahmin, merchant, worker, or ascetic who is restrained in body, speech, and mind, and develops the seven qualities that lead to awakening, becomes extinguished in this very life. 

Any\marginnote{31.1} mendicant from these four castes who is perfected—with defilements ended, who has completed the spiritual journey, done what had to be done, laid down the burden, achieved their own true goal, utterly ended the fetters of rebirth, and is rightly freed through enlightenment—is said to be the foremost by virtue of principle, not without principle. For principle, \textsanskrit{Vāseṭṭha}, is the first thing for people in both this life and the next. 

\textsanskrit{Brahmā}\marginnote{32.1} \textsanskrit{Sanaṅkumāra} also spoke this verse: 

\begin{verse}%
‘The\marginnote{32.2} aristocrat is first among people \\
who take clan as the standard. \\
But one accomplished in knowledge and conduct \\
is first among gods and humans.’ 

%
\end{verse}

That\marginnote{32.6} verse was well sung by \textsanskrit{Brahmā} \textsanskrit{Sanaṅkumāra}, not poorly sung; well spoken, not poorly spoken; beneficial, not harmful, and I agree with it. I also say: 

\begin{verse}%
The\marginnote{32.8} aristocrat is first among people \\
who take clan as the standard. \\
But one accomplished in knowledge and conduct \\
is first among gods and humans.” 

%
\end{verse}

That\marginnote{32.12} is what the Buddha said. Satisfied, \textsanskrit{Vāseṭṭha} and \textsanskrit{Bhāradvāja} were happy with what the Buddha said. 

%
\chapter*{{\suttatitleacronym DN 28}{\suttatitletranslation Inspiring Confidence }{\suttatitleroot Sampasādanīyasutta}}
\addcontentsline{toc}{chapter}{\tocacronym{DN 28} \toctranslation{Inspiring Confidence } \tocroot{Sampasādanīyasutta}}
\markboth{Inspiring Confidence }{Sampasādanīyasutta}
\extramarks{DN 28}{DN 28}

\section*{1. \textsanskrit{Sāriputta}’s Lion’s Roar }

\scevam{So\marginnote{1.1} I have heard. }At one time the Buddha was staying near \textsanskrit{Nālandā} in \textsanskrit{Pāvārika}’s mango grove. Then \textsanskrit{Sāriputta} went up to the Buddha, bowed, sat down to one side, and said to him: 

“Sir,\marginnote{1.4} I have such confidence in the Buddha that I believe there’s no other ascetic or brahmin—whether past, future, or present—whose direct knowledge is superior to the Buddha when it comes to awakening.” 

“That’s\marginnote{1.5} a grand and dramatic statement, \textsanskrit{Sāriputta}. You’ve roared a definitive, categorical lion’s roar, saying: ‘I have such confidence in the Buddha that I believe there’s no other ascetic or brahmin—whether past, future, or present—whose direct knowledge is superior to the Buddha when it comes to awakening.’ 

What\marginnote{1.8} about all the perfected ones, the fully awakened Buddhas who lived in the past? Have you comprehended their minds to know that those Buddhas had such ethics, or such qualities, or such wisdom, or such meditation, or such freedom?” 

“No,\marginnote{1.10} sir.” 

“And\marginnote{1.11} what about all the perfected ones, the fully awakened Buddhas who will live in the future? Have you comprehended their minds to know that those Buddhas will have such ethics, or such qualities, or such wisdom, or such meditation, or such freedom?” 

“No,\marginnote{1.13} sir.” 

“And\marginnote{1.14} what about me, the perfected one, the fully awakened Buddha at present? Have you comprehended my mind to know that I have such ethics, or such qualities, or such wisdom, or such meditation, or such freedom?” 

“No,\marginnote{1.16} sir.” 

“Well\marginnote{1.17} then, \textsanskrit{Sāriputta}, given that you don’t comprehend the minds of Buddhas past, future, or present, what exactly are you doing, making such a grand and dramatic statement, roaring such a definitive, categorical lion’s roar?” 

“Sir,\marginnote{2.1} though I don’t comprehend the minds of Buddhas past, future, and present, still I understand this by inference from the teaching. Suppose there were a king’s frontier citadel with fortified embankments, ramparts, and arches, and a single gate. And it has a gatekeeper who is astute, competent, and clever. He keeps strangers out and lets known people in. As he walks around the patrol path, he doesn’t see a hole or cleft in the wall, not even one big enough for a cat to slip out. They’d think, ‘Whatever sizable creatures enter or leave the citadel, all of them do so via this gate.’ 

In\marginnote{2.8} the same way, I understand this by inference from the teaching: ‘All the perfected ones, fully awakened Buddhas—whether past, future, or present—give up the five hindrances, corruptions of the heart that weaken wisdom. Their mind is firmly established in the four kinds of mindfulness meditation. They correctly develop the seven awakening factors. And they wake up to the supreme perfect awakening.’ 

Sir,\marginnote{2.12} once I approached the Buddha to listen to the teaching. He explained Dhamma with its higher and higher stages, with its better and better stages, with its dark and bright sides. When I directly knew a certain principle of those teachings, in accordance with how I was taught, I came to a conclusion about the teachings. I had confidence in the Teacher: ‘The Blessed One is a fully awakened Buddha. The teaching is well explained. The \textsanskrit{Saṅgha} is practicing well.’ 

\subsection*{1.1. Teaching Skillful Qualities }

And\marginnote{3.1} moreover, sir, how the Buddha teaches skillful qualities is unsurpassable. This consists of such skillful qualities as the four kinds of mindfulness meditation, the four right efforts, the four bases of psychic power, the five faculties, the five powers, the seven awakening factors, and the noble eightfold path. By these a mendicant realizes the undefiled freedom of heart and freedom by wisdom in this very life. And they live having realized it with their own insight due to the ending of defilements. This is unsurpassable when it comes to skillful qualities. The Buddha understands this without exception. There is nothing to be understood beyond this whereby another ascetic or brahmin might be superior in direct knowledge to the Buddha when it comes to skillful qualities. 

\subsection*{1.2. Describing the Sense Fields }

And\marginnote{4.1} moreover, sir, how the Buddha teaches the description of the sense fields is unsurpassable. There are these six interior and exterior sense fields. The eye and sights, the ear and sounds, the nose and smells, the tongue and tastes, the body and touches, and the mind and thoughts. This is unsurpassable when it comes to describing the sense fields. The Buddha understands this without exception. There is nothing to be understood beyond this whereby another ascetic or brahmin might be superior in direct knowledge to the Buddha when it comes to describing the sense fields. 

\subsection*{1.3. The Conception of the Embryo }

And\marginnote{5.1} moreover, sir, how the Buddha teaches the conception of the embryo is unsurpassable. There are these four kinds of conception. 

Firstly,\marginnote{5.3} someone is unaware when conceived in their mother’s womb, unaware as they remain there, and unaware as they emerge. This is the first kind of conception. 

Furthermore,\marginnote{5.4} someone is aware when conceived in their mother’s womb, but unaware as they remain there, and unaware as they emerge. This is the second kind of conception. 

Furthermore,\marginnote{5.5} someone is aware when conceived in their mother’s womb, aware as they remain there, but unaware as they emerge. This is the third kind of conception. 

Furthermore,\marginnote{5.6} someone is aware when conceived in their mother’s womb, aware as they remain there, and aware as they emerge. This is the fourth kind of conception. 

This\marginnote{5.7} is unsurpassable when it comes to the conception of the embryo. 

\subsection*{1.4. Ways of Revealing }

And\marginnote{6.1} moreover, sir, how the Buddha teaches the different ways of revealing is unsurpassable. There are these four ways of revealing. 

Firstly,\marginnote{6.3} someone reveals by means of a sign, ‘This is what you’re thinking, such is your thought, and thus is your state of mind.’ And even if they reveal this many times, it turns out exactly so, not otherwise. This is the first way of revealing. 

Furthermore,\marginnote{6.7} someone reveals after hearing it from humans or non-humans or deities, ‘This is what you’re thinking, such is your thought, and thus is your state of mind.’ And even if they reveal this many times, it turns out exactly so, not otherwise. This is the second way of revealing. 

Furthermore,\marginnote{6.11} someone reveals by hearing the sound of thought spreading as someone thinks and considers, ‘This is what you’re thinking, such is your thought, and thus is your state of mind.’ And even if they reveal this many times, it turns out exactly so, not otherwise. This is the third way of revealing. 

Furthermore,\marginnote{6.15} someone comprehends the mind of a person who has attained the immersion that’s free of placing the mind and keeping it connected. They understand, ‘Judging by the way this person’s intentions are directed, immediately after this mind state, they’ll think this thought.’ And even if they reveal this many times, it turns out exactly so, not otherwise. This is the fourth way of revealing. 

This\marginnote{6.19} is unsurpassable when it comes to the ways of revealing. 

\subsection*{1.5. Attainments of Vision }

And\marginnote{7.1} moreover, sir, how the Buddha teaches the attainments of vision is unsurpassable. There are these four attainments of vision. 

Firstly,\marginnote{7.3} some ascetic or brahmin—by dint of keen, resolute, committed, and diligent effort, and right focus—experiences an immersion of the heart of such a kind that they examine their own body up from the soles of the feet and down from the tips of the hairs, wrapped in skin and full of many kinds of filth. ‘In this body there is head hair, body hair, nails, teeth, skin, flesh, sinews, bones, bone marrow, kidneys, heart, liver, diaphragm, spleen, lungs, intestines, mesentery, undigested food, feces, bile, phlegm, pus, blood, sweat, fat, tears, grease, saliva, snot, synovial fluid, urine.’ This is the first attainment of vision. 

Furthermore,\marginnote{7.6} some ascetic or brahmin attains that and goes beyond it. They examine a person’s bones with skin, flesh, and blood. This is the second attainment of vision. 

Furthermore,\marginnote{7.10} some ascetic or brahmin attains that and goes beyond it. They understand a person’s stream of consciousness, unbroken on both sides, established in both this world and the next. This is the third attainment of vision. 

Furthermore,\marginnote{7.15} some ascetic or brahmin attains that and goes beyond it. They understand a person’s stream of consciousness, unbroken on both sides, not established in either this world or the next. This is the fourth attainment of vision. 

This\marginnote{7.20} is unsurpassable when it comes to attainments of vision. 

\subsection*{1.6. Descriptions of Individuals }

And\marginnote{8.1} moreover, sir, how the Buddha teaches the description of individuals is unsurpassable. There are these seven individuals. One freed both ways, one freed by wisdom, a personal witness, one attained to view, one freed by faith, a follower of the teachings, a follower by faith. This is unsurpassable when it comes to the description of individuals. 

\subsection*{1.7. Kinds of Striving }

And\marginnote{9.1} moreover, sir, how the Buddha teaches the kinds of striving is unsurpassable. There are these seven awakening factors: the awakening factors of mindfulness, investigation of principles, energy, rapture, tranquility, immersion, and equanimity. This is unsurpassable when it comes to the kinds of striving. 

\subsection*{1.8. Ways of Practice }

And\marginnote{10.1} moreover, sir, how the Buddha teaches the ways of practice is unsurpassable. 

\begin{enumerate}%
\item Painful practice with slow insight, %
\item painful practice with swift insight, %
\item pleasant practice with slow insight, and %
\item pleasant practice with swift insight. %
\end{enumerate}

Of\marginnote{10.6} these, the painful practice with slow insight is said to be inferior both ways: because it’s painful and because it’s slow. The painful practice with swift insight is said to be inferior because it’s painful. The pleasant practice with slow insight is said to be inferior because it’s slow. But the pleasant practice with swift insight is said to be superior both ways: because it’s pleasant and because it’s swift. 

This\marginnote{10.10} is unsurpassable when it comes to the ways of practice. 

\subsection*{1.9. Behavior in Speech }

And\marginnote{11.1} moreover, sir, how the Buddha teaches behavior in speech is unsurpassable. It’s when someone doesn’t use speech that’s connected with lying, or divisive, or backbiting, or aggressively trying to win. They speak only wise counsel, valuable and timely. This is unsurpassable when it comes to behavior in speech. 

And\marginnote{12.1} moreover, sir, how the Buddha teaches a person’s ethical behavior is unsurpassable. It’s when someone is honest and faithful. They don’t use deceit, flattery, hinting, or belittling, and they don’t use material possessions to chase after other material possessions. They guard the sense doors and eat in moderation. They’re fair, dedicated to wakefulness, tireless, energetic, and meditative. They have good memory, eloquence, range, retention, and thoughtfulness. They’re not greedy for sensual pleasures. They are mindful and alert. This is unsurpassable when it comes to a person’s ethical behavior. 

\subsection*{1.10. Responsiveness to Instruction }

And\marginnote{13.1} moreover, sir, how the Buddha teaches the different degrees of responsiveness to instruction is unsurpassable. There are these four degrees of responsiveness to instruction. 

The\marginnote{13.3} Buddha knows by investigating inside another individual: ‘By practicing as instructed this individual will, with the ending of three fetters, become a stream-enterer, not liable to be reborn in the underworld, bound for awakening.’ The Buddha knows by investigating inside another individual: ‘By practicing as instructed this individual will, with the ending of three fetters, and the weakening of greed, hate, and delusion, become a once-returner. They will come back to this world once only, then make an end of suffering.’ The Buddha knows by investigating inside another individual: ‘By practicing as instructed this individual will, with the ending of the five lower fetters, be reborn spontaneously. They will be extinguished there, and are not liable to return from that world.’ The Buddha knows by investigating inside another individual: ‘By practicing as instructed this individual will realize the undefiled freedom of heart and freedom by wisdom in this very life, and live having realized it with their own insight due to the ending of defilements.’ 

This\marginnote{13.11} is unsurpassable when it comes to the different degrees of responsiveness to instruction. 

\subsection*{1.11. The Knowledge and Freedom of Others }

And\marginnote{14.1} moreover, sir, how the Buddha teaches the knowledge and freedom of other individuals is unsurpassable. The Buddha knows by investigating inside another individual: ‘With the ending of three fetters this individual will become a stream-enterer, not liable to be reborn in the underworld, bound for awakening.’ The Buddha knows by investigating inside another individual: ‘With the ending of three fetters, and the weakening of greed, hate, and delusion, this individual will become a once-returner. They will come back to this world once only, then make an end of suffering.’ The Buddha knows by investigating inside another individual: ‘With the ending of the five lower fetters, this individual will be reborn spontaneously. They will be extinguished there, and are not liable to return from that world.’ The Buddha knows by investigating inside another individual: ‘This individual will realize the undefiled freedom of heart and freedom by wisdom in this very life, and live having realized it with their own insight due to the ending of defilements.’ This is unsurpassable when it comes to the knowledge and freedom of other individuals. 

\subsection*{1.12. Eternalism }

And\marginnote{15.1} moreover, sir, how the Buddha teaches eternalist doctrines is unsurpassable. There are these three eternalist doctrines. 

Firstly,\marginnote{15.3} some ascetic or brahmin—by dint of keen, resolute, committed, and diligent effort, and right focus—experiences an immersion of the heart of such a kind that they recollect many hundreds of thousands of past lives, with features and details. They say, ‘I know that in the past the cosmos expanded or contracted. I don’t know whether in the future the cosmos will expand or contract. The self and the cosmos are eternal, barren, steady as a mountain peak, standing firm like a pillar. They remain the same for all eternity, while these sentient beings wander and transmigrate and pass away and rearise.’ This is the first eternalist doctrine. 

Furthermore,\marginnote{15.10} some ascetic or brahmin—by dint of keen, resolute, committed, and diligent effort, and right focus—experiences an immersion of the heart of such a kind that they recollect their past lives for as many as ten eons of the expansion and contraction of the cosmos, with features and details. They say, ‘I know that in the past the cosmos expanded or contracted. I don’t know whether in the future the cosmos will expand or contract. The self and the cosmos are eternal, barren, steady as a mountain peak, standing firm like a pillar. They remain the same for all eternity, while these sentient beings wander and transmigrate and pass away and rearise.’ This is the second eternalist doctrine. 

Furthermore,\marginnote{15.18} some ascetic or brahmin—by dint of keen, resolute, committed, and diligent effort, and right focus—experiences an immersion of the heart of such a kind that they recollect their past lives for as many as forty eons of the expansion and contraction of the cosmos, with features and details. They say, ‘I know that in the past the cosmos expanded or contracted. I don’t know whether in the future the cosmos will expand or contract. The self and the cosmos are eternal, barren, steady as a mountain peak, standing firm like a pillar. They remain the same for all eternity, while these sentient beings wander and transmigrate and pass away and rearise.’ This is the third eternalist doctrine. 

This\marginnote{15.26} is unsurpassable when it comes to eternalist doctrines. 

\subsection*{1.13. Recollecting Past Lives }

And\marginnote{16.1} moreover, sir, how the Buddha teaches the knowledge of recollecting past lives is unsurpassable. It’s when some ascetic or brahmin—by dint of keen, resolute, committed, and diligent effort, and right focus—experiences an immersion of the heart of such a kind that they recollect their many kinds of past lives. That is: one, two, three, four, five, ten, twenty, thirty, forty, fifty, a hundred, a thousand, a hundred thousand rebirths; many eons of the world contracting, many eons of the world expanding, many eons of the world contracting and expanding. They remember: ‘There, I was named this, my clan was that, I looked like this, and that was my food. This was how I felt pleasure and pain, and that was how my life ended. When I passed away from that place I was reborn somewhere else. There, too, I was named this, my clan was that, I looked like this, and that was my food. This was how I felt pleasure and pain, and that was how my life ended. When I passed away from that place I was reborn here.’ And so they recollect their many kinds of past lives, with features and details. Sir, there are gods whose life span cannot be reckoned or calculated. Still, no matter what incarnation they have previously been reborn in—whether physical or formless or percipient or non-percipient or neither percipient nor non-percipient—they recollect their many kinds of past lives, with features and details. This is unsurpassable when it comes to the knowledge of recollecting past lives. 

\subsection*{1.14. Death and Rebirth }

And\marginnote{17.1} moreover, sir, how the Buddha teaches the knowledge of the death and rebirth of sentient beings is unsurpassable. It’s when some ascetic or brahmin—by dint of keen, resolute, committed, and diligent effort, and right focus—experiences an immersion of the heart of such a kind that with clairvoyance that is purified and superhuman, they see sentient beings passing away and being reborn —inferior and superior, beautiful and ugly, in a good place or a bad place. They understand how sentient beings are reborn according to their deeds: ‘These dear beings did bad things by way of body, speech, and mind. They spoke ill of the noble ones; they had wrong view; and they chose to act out of that wrong view. When their body breaks up, after death, they’re reborn in a place of loss, a bad place, the underworld, hell. These dear beings, however, did good things by way of body, speech, and mind. They never spoke ill of the noble ones; they had right view; and they chose to act out of that right view. When their body breaks up, after death, they’re reborn in a good place, a heavenly realm.’ And so, with clairvoyance that is purified and superhuman, they see sentient beings passing away and being reborn—inferior and superior, beautiful and ugly, in a good place or a bad place. They understand how sentient beings are reborn according to their deeds. This is unsurpassable when it comes to the knowledge of death and rebirth. 

\subsection*{1.15. Psychic Powers }

And\marginnote{18.1} moreover, sir, how the Buddha teaches psychic power is unsurpassable. There are these two kinds of psychic power. There are psychic powers that are accompanied by defilements and attachments, and are said to be ignoble. And there are psychic powers that are free of defilements and attachments, and are said to be noble. What are the psychic powers that are accompanied by defilements and attachments, and are said to be ignoble? It’s when some ascetic or brahmin—by dint of keen, resolute, committed, and diligent effort, and right focus—experiences an immersion of the heart of such a kind that they wield the many kinds of psychic power: multiplying themselves and becoming one again; going unimpeded through a wall, a rampart, or a mountain as if through space; diving in and out of the earth as if it were water; walking on water as if it were earth; flying cross-legged through the sky like a bird; touching and stroking with the hand the sun and moon, so mighty and powerful; controlling the body as far as the \textsanskrit{Brahmā} realm. These are the psychic powers that are accompanied by defilements and attachments, and are said to be ignoble. 

But\marginnote{18.9} what are the psychic powers that are free of defilements and attachments, and are said to be noble? It’s when, if a mendicant wishes: ‘May I meditate perceiving the unrepulsive in the repulsive,’ that’s what they do. If they wish: ‘May I meditate perceiving the repulsive in the unrepulsive,’ that’s what they do. If they wish: ‘May I meditate perceiving the unrepulsive in the repulsive and the unrepulsive,’ that’s what they do. If they wish: ‘May I meditate perceiving the repulsive in the unrepulsive and the repulsive,’ that’s what they do. If they wish: ‘May I meditate staying equanimous, mindful and aware, rejecting both the repulsive and the unrepulsive,’ that’s what they do. These are the psychic powers that are free of defilements and attachments, and are said to be noble. This is unsurpassable when it comes to psychic powers. The Buddha understands this without exception. There is nothing to be understood beyond this whereby another ascetic or brahmin might be superior in direct knowledge to the Buddha when it comes to psychic powers. 

\subsection*{1.16. The Four Absorptions }

The\marginnote{19.1} Buddha has achieved what should be achieved by a faithful gentleman by being energetic and strong, by human strength, energy, vigor, and exertion. The Buddha doesn’t indulge in sensual pleasures, which are low, crude, ordinary, ignoble, and pointless. And he doesn’t indulge in self-mortification, which is painful, ignoble, and pointless. He gets the four absorptions—blissful meditations in the present life that belong to the higher mind—when he wants, without trouble or difficulty. 

\subsection*{1.17. On Being Questioned }

Sir,\marginnote{19.5} if they were to ask me, ‘Reverend \textsanskrit{Sāriputta}, is there any other ascetic or brahmin—whether past, future, or present—whose direct knowledge is superior to the Buddha when it comes to awakening?’ I would tell them ‘No.’ 

But\marginnote{19.9} if they were to ask me, ‘Reverend \textsanskrit{Sāriputta}, is there any other ascetic or brahmin—whether past or future—whose direct knowledge is equal to the Buddha when it comes to awakening?’ I would tell them ‘Yes.’ But if they were to ask: ‘Reverend \textsanskrit{Sāriputta}, is there any other ascetic or brahmin at present whose direct knowledge is equal to the Buddha when it comes to awakening?’ I would tell them ‘No.’ 

But\marginnote{19.13} if they were to ask me, ‘But why does Venerable \textsanskrit{Sāriputta} grant this in respect of some but not others?’ I would answer them like this, ‘Reverends, I have heard and learned this in the presence of the Buddha: “The perfected ones, fully awakened Buddhas of the past and the future are equal to myself when it comes to awakening.” And I have also heard and learned this in the presence of the Buddha: “It’s impossible for two perfected ones, fully awakened Buddhas to arise in the same solar system at the same time.”’ 

Answering\marginnote{19.21} this way, I trust that I repeated what the Buddha has said, and didn’t misrepresent him with an untruth. I trust my explanation was in line with the teaching, and that there are no legitimate grounds for rebuke or criticism.” 

“Indeed,\marginnote{19.22} \textsanskrit{Sāriputta}, in answering this way you repeat what I’ve said, and don’t misrepresent me with an untruth. Your explanation is in line with the teaching, and there are no legitimate grounds for rebuke or criticism.” 

\section*{2. Incredible and Amazing }

When\marginnote{20.1} he had spoken, Venerable \textsanskrit{Udāyī} said to the Buddha, “It’s incredible, sir, it’s amazing! The Realized One has so few wishes, such contentment, such self-effacement! For even though the Realized One has such power and might, he will not make a display of himself. If the wanderers following other paths were to see even a single one of these qualities in themselves they’d carry around a banner to that effect. It’s incredible, sir, it’s amazing! The Realized One has so few wishes, such contentment, such self-effacement! For even though the Realized One has such power and might, he will not make a display of himself.” 

“See,\marginnote{20.7} \textsanskrit{Udāyī}, how the Realized One has so few wishes, such contentment, such self-effacement. For even though the Realized One has such power and might, he will not make a display of himself. If the wanderers following other paths were to see even a single one of these qualities in themselves they’d carry around a banner to that effect. See, \textsanskrit{Udāyī}, how the Realized One has so few wishes, such contentment, such self-effacement. For even though the Realized One has such power and might, he will not make a display of himself.” 

Then\marginnote{21.1} the Buddha said to Venerable \textsanskrit{Sāriputta}, “So \textsanskrit{Sāriputta}, you should frequently speak this exposition of the teaching to the monks, nuns, laymen, and laywomen. Though there will be some foolish people who have doubt or uncertainty regarding the Realized One, when they hear this exposition of the teaching they’ll give up that doubt or uncertainty.” 

That’s\marginnote{21.4} how Venerable \textsanskrit{Sāriputta} declared his confidence in the Buddha’s presence. And that’s why the name of this discussion is “Inspiring Confidence”. 

%
\chapter*{{\suttatitleacronym DN 29}{\suttatitletranslation An Impressive Discourse }{\suttatitleroot Pāsādikasutta}}
\addcontentsline{toc}{chapter}{\tocacronym{DN 29} \toctranslation{An Impressive Discourse } \tocroot{Pāsādikasutta}}
\markboth{An Impressive Discourse }{Pāsādikasutta}
\extramarks{DN 29}{DN 29}

\scevam{So\marginnote{1.1} I have heard. }At one time the Buddha was staying in the land of the Sakyans in a stilt longhouse in a mango grove belonging to the Sakyan family named \textsanskrit{Vedhañña}. 

Now\marginnote{1.3} at that time the \textsanskrit{Nigaṇṭha} \textsanskrit{Nātaputta} had recently passed away at \textsanskrit{Pāvā}. With his passing the Jain ascetics split, dividing into two factions, arguing, quarreling, and disputing, continually wounding each other with barbed words: 

“You\marginnote{1.5} don’t understand this teaching and training. I understand this teaching and training. What, you understand this teaching and training? You’re practicing wrong. I’m practicing right. I stay on topic, you don’t. You said last what you should have said first. You said first what you should have said last. What you’ve thought so much about has been disproved. Your doctrine is refuted. Go on, save your doctrine! You’re trapped; get yourself out of this—if you can!” 

You’d\marginnote{1.6} think there was nothing but slaughter going on among the Jain ascetics. And the \textsanskrit{Nigaṇṭha} \textsanskrit{Nātaputta}’s white-clothed lay disciples were disillusioned, dismayed, and disappointed in the Jain ascetics. They were equally disappointed with a teaching and training so poorly explained and poorly propounded, not emancipating, not leading to peace, proclaimed by someone who is not a fully awakened Buddha, with broken monument and without a refuge. 

And\marginnote{2.1} then, after completing the rainy season residence near \textsanskrit{Pāvā}, the novice Cunda went to see Venerable Ānanda at \textsanskrit{Sāma} village. He bowed, sat down to one side, and told him what had happened. 

Ānanda\marginnote{2.4} said to him, “Reverend Cunda, we should see the Buddha about this matter. Come, let’s go to the Buddha and tell him about this.” 

“Yes,\marginnote{2.7} sir,” replied Cunda. 

Then\marginnote{3.1} Ānanda and Cunda went to the Buddha, bowed, sat down to one side, and told him what had happened. 

\section*{1. The Teaching of the Unawakened }

“That’s\marginnote{4.1} what happens, Cunda, when a teaching and training is poorly explained and poorly propounded, not emancipating, not leading to peace, proclaimed by someone who is not a fully awakened Buddha. 

Take\marginnote{4.2} the case where a teacher is not awakened, and the teaching is poorly explained and poorly propounded, not emancipating, not leading to peace, proclaimed by someone who is not a fully awakened Buddha. A disciple in that teaching does not practice in line with the teachings, does not practice following that procedure, does not live in line with the teaching. They proceed having turned away from that teaching. You should say this to them, ‘You’re fortunate, reverend, you’re so very fortunate! For your teacher is not awakened, and their teaching is poorly explained and poorly propounded, not emancipating, not leading to peace, proclaimed by someone who is not a fully awakened Buddha. But you don’t practice in line with that teaching, you don’t practice following that procedure, you don’t live in line with the teaching. You proceed having turned away from that teaching.’ In such a case the teacher and the teaching are to blame, but the disciple deserves praise. Suppose someone was to say to such a disciple, ‘Come on, venerable, practice as taught and pointed out by your teacher.’ The one who encourages, the one who they encourage, and the one who practices accordingly all make much bad karma. Why is that? It’s because that teaching and training is poorly explained and poorly propounded, not emancipating, not leading to peace, proclaimed by someone who is not a fully awakened Buddha. 

Take\marginnote{5.1} the case where a teacher is not awakened, and the teaching is poorly explained and poorly propounded, not emancipating, not leading to peace, proclaimed by someone who is not a fully awakened Buddha. A disciple in that teaching practices in line with the teachings, practices following that procedure, lives in line with the teaching. They proceed having undertaken that teaching. You should say this to them, ‘It’s your loss, reverend, it’s your misfortune! For your teacher is not awakened, and their teaching is poorly explained and poorly propounded, not emancipating, not leading to peace, proclaimed by someone who is not a fully awakened Buddha. And you practice in line with that teaching, you practice following that procedure, you live in line with the teaching. You proceed having undertaken that teaching.’ In such a case the teacher, the teaching, and the disciple are all to blame. Suppose someone was to say to such a disciple, ‘Clearly the venerable is practicing methodically and will succeed in completing that method.’ The one who praises, the one who they praise, and the one who, being praised, rouses up even more energy all make much bad karma. Why is that? It’s because that teaching and training is poorly explained and poorly propounded, not emancipating, not leading to peace, proclaimed by someone who is not a fully awakened Buddha. 

\section*{2. The Teaching of the Awakened }

Take\marginnote{6.1} the case where a teacher is awakened, and the teaching is well explained and well propounded, emancipating, leading to peace, proclaimed by someone who is a fully awakened Buddha. A disciple in that teaching does not practice in line with the teachings, does not practice following that procedure, does not live in line with the teaching. They proceed having turned away from that teaching. You should say this to them, ‘It’s your loss, reverend, it’s your misfortune! For your teacher is awakened, and their teaching is well explained and well propounded, emancipating, leading to peace, proclaimed by someone who is a fully awakened Buddha. But you don’t practice in line with that teaching, you don’t practice following that procedure, you don’t live in line with the teaching. You proceed having turned away from that teaching.’ In such a case the teacher and the teaching deserve praise, but the disciple is to blame. Suppose someone was to say to such a disciple, ‘Come on, venerable, practice as taught and pointed out by your teacher.’ The one who encourages, the one who they encourage, and the one who practices accordingly all make much merit. Why is that? It’s because that teaching and training is well explained and well propounded, emancipating, leading to peace, proclaimed by someone who is a fully awakened Buddha. 

Take\marginnote{7.1} the case where a teacher is awakened, and the teaching is well explained and well propounded, emancipating, leading to peace, proclaimed by someone who is a fully awakened Buddha. A disciple in that teaching practices in line with the teachings, practices following that procedure, lives in line with the teaching. They proceed having undertaken that teaching. You should say this to them, ‘You’re fortunate, reverend, you’re so very fortunate! For your teacher is awakened, and their teaching is well explained and well propounded, emancipating, leading to peace, proclaimed by someone who is a fully awakened Buddha. And you practice in line with that teaching, you practice following that procedure, you live in line with the teaching. You proceed having undertaken that teaching.’ In such a case the teacher, the teaching, and the disciple all deserve praise. Suppose someone was to say to such a disciple, ‘Clearly the venerable is practicing methodically and will succeed in completing that method.’ The one who praises, the one who they praise, and the one who, being praised, rouses up even more energy all make much merit. Why is that? It’s because that teaching and training is well explained and well propounded, emancipating, leading to peace, proclaimed by someone who is a fully awakened Buddha. 

\section*{3. When Disciples Have Regrets }

Take\marginnote{8.1} the case where a teacher arises in the world who is perfected, a fully awakened Buddha. The teaching is well explained and well propounded, emancipating, leading to peace, proclaimed by someone who is fully awakened. But the disciples haven’t inquired about the meaning of that good teaching. And the spiritual practice that’s entirely full and pure has not been disclosed and revealed to them with all its collected sayings, with its demonstrable basis, well proclaimed wherever there are gods and humans. And then their teacher passes away. When such a teacher has passed away the disciples are tormented by regrets. Why is that? They think: ‘Our teacher was perfected, a fully awakened Buddha. His teaching was well explained, but we didn’t inquire about the meaning, and the spiritual practice was not fully disclosed to us. And then our teacher passed away.’ When such a teacher has passed away the disciples are tormented by regrets. 

\section*{4. When Disciples Have No Regrets }

Take\marginnote{9.1} the case where a teacher arises in the world who is perfected, a fully awakened Buddha. The teaching is well explained and well propounded, emancipating, leading to peace, proclaimed by someone who is fully awakened. The disciples have inquired about the meaning of that good teaching. And the spiritual practice that’s entirely full and pure has been disclosed and revealed to them with all its collected sayings, with its demonstrable basis, well proclaimed wherever there are gods and humans. And then their teacher passes away. When such a teacher has passed away the disciples are free of regrets. Why is that? They think: ‘Our teacher was perfected, a fully awakened Buddha. His teaching was well explained, we inquired about the meaning, and the spiritual practice was fully disclosed to us. And then our teacher passed away.’ When such a teacher has passed away the disciples are free of regrets. 

\section*{5. On the Incomplete Spiritual Path, Etc. }

Now\marginnote{10.1} suppose, Cunda, that a spiritual path possesses those factors. But the teacher is not senior, long standing, long gone forth, advanced in years, and reached the final stage of life. Then that spiritual path is incomplete in that respect. 

But\marginnote{10.3} when a spiritual path possesses those factors and the teacher is senior, then that spiritual path is complete in that respect. 

Now\marginnote{11.1} suppose that a spiritual path possesses those factors and the teacher is senior. But there are no senior monk disciples who are competent, educated, assured, have attained sanctuary, who can rightly explain the true teaching, and who can legitimately and completely refute the doctrines of others that come up, and teach with a demonstrable basis. Then that spiritual path is incomplete in that respect. 

But\marginnote{12.1} when a spiritual path possesses those factors and the teacher is senior and there are competent senior monks, then that spiritual path is complete in that respect. 

Now\marginnote{12.4} suppose that a spiritual path possesses those factors and the teacher is senior and there are competent senior monks. But there are no competent middle monks, junior monks, senior nuns, middle nuns, junior nuns, celibate white-clothed laymen, white-clothed laymen enjoying sensual pleasures, celibate white-clothed laywomen, white-clothed laywomen enjoying sensual pleasures. … There are white-clothed laywomen enjoying sensual pleasures, but the spiritual path is not successful and prosperous, extensive, popular, widespread, and well proclaimed wherever there are gods and humans … the spiritual path is successful and prosperous, extensive, popular, widespread, and well proclaimed wherever there are gods and humans, but it has not reached the peak of material possessions and fame. Then that spiritual path is incomplete in that respect. 

But\marginnote{13.1} when a spiritual path possesses those factors and the teacher is senior and there are competent senior monks, middle monks, junior monks, senior nuns, middle nuns, junior nuns, celibate laymen, laymen enjoying sensual pleasures, celibate laywomen, laywomen enjoying sensual pleasures, and the spiritual path is successful and prosperous, extensive, popular, widespread, and well proclaimed wherever there are gods and humans, and it has reached the peak of material possessions and fame, then that spiritual path is complete in that respect. 

I,\marginnote{14.1} Cunda, am a teacher who has arisen in the world today, perfected and fully awakened. The teaching is well explained and well propounded, emancipating, leading to peace, proclaimed by someone who is fully awakened. My disciples have inquired about the meaning of that good teaching. And the spiritual practice that’s entirely full and pure has been disclosed and revealed to them with all its collected sayings, with its demonstrable basis, well proclaimed wherever there are gods and humans. I am a teacher today who is senior, long standing, long gone forth, advanced in years, and have reached the final stage of life. 

I\marginnote{15.1} have today disciples who are competent senior monks, middle monks, junior monks, senior nuns, middle nuns, junior nuns, celibate laymen, laymen enjoying sensual pleasures, celibate laywomen, and laywomen enjoying sensual pleasures. Today my spiritual path is successful and prosperous, extensive, popular, widespread, and well proclaimed wherever there are gods and humans. 

Of\marginnote{16.1} all the teachers in the world today, Cunda, I don’t see even a single one who has reached the peak of material possessions and fame like me. Of all the spiritual communities and groups in the world today, Cunda, I don’t see even a single one who has reached the pinnacle of material possessions and fame like the mendicant \textsanskrit{Saṅgha}. And if there’s any spiritual path of which it may be rightly said that it’s endowed with all good qualities, complete in all good qualities, neither too little nor too much, well explained, whole, full, and well propounded, it’s of this spiritual path that this should be said. 

Uddaka,\marginnote{16.8} son of \textsanskrit{Rāma}, used to say: ‘Seeing, one does not see.’ But seeing what does one not see? You can see the blade of a well-sharpened razor, but not the edge. Thus it is said: ‘Seeing, one does not see.’ But that saying of Uddaka’s is low, crude, ordinary, ignoble, and pointless, as it’s only concerning a razor. If there’s anything of which it may be rightly said: ‘Seeing, one does not see,’ it’s of this that it should be said. Seeing what does one not see? One sees this: a spiritual path endowed with all good qualities, complete in all good qualities, neither too little nor too much, well explained, whole, full, and well propounded. One does not see this: anything that, were it to be removed, would make it purer. One does not see this: anything that, were it to be added, would make it more complete. Thus it is rightly said: ‘Seeing, one does not see.’ 

\section*{6. Teachings Should be Recited in Concert }

So,\marginnote{17.1} Cunda, you should all come together and recite in concert, without disputing, those things I have taught you from my direct knowledge, comparing meaning with meaning and phrasing with phrasing, so that this spiritual path may last for a long time. That would be for the welfare and happiness of the people, out of compassion for the world, for the benefit, welfare, and happiness of gods and humans. And what are those things I have taught from my direct knowledge? They are the four kinds of mindfulness meditation, the four right efforts, the four bases of psychic power, the five faculties, the five powers, the seven awakening factors, and the noble eightfold path. These are the things I have taught from my own direct knowledge. 

\section*{7. Reaching Agreement }

Suppose\marginnote{18.1} one of those spiritual companions who is training in harmony and mutual appreciation, without disputing, were to recite the teaching in the \textsanskrit{Saṅgha}. Now, you might think, ‘This venerable misconstrues the meaning and mistakes the phrasing.’ You should neither approve nor dismiss them, but say, ‘Reverend, if this is the meaning, the phrasing may either be this or that: which is more fitting? And if this is the phrasing, the meaning may be either this or that: which is more fitting?’ Suppose they reply, ‘This phrasing fits the meaning better than that. And this meaning fits the phrasing better than that.’ Without flattering or rebuking them, you should carefully convince them by examining that meaning and that phrasing. 

Suppose\marginnote{19.1} another spiritual companion were to recite the teaching in the \textsanskrit{Saṅgha}. Now, you might think, ‘This venerable misconstrues the meaning but gets the phrasing right.’ You should neither approve nor dismiss them, but say, ‘Reverend, if this is the phrasing, the meaning may be either this or that: which is more fitting?’ Suppose they reply, ‘This meaning fits the phrasing better than that.’ Without flattering or rebuking, you should carefully convince them by examining that meaning. 

Suppose\marginnote{20.1} another spiritual companion were to recite the teaching in the \textsanskrit{Saṅgha}. Now, you might think, ‘This venerable construes the meaning correctly but mistakes the phrasing.’ You should neither approve nor dismiss them, but say, ‘Reverend, if this is the meaning, the phrasing may be either this or that: which is more fitting?’ Suppose they reply, ‘This phrasing fits the meaning better than that.’ Without flattering or rebuking, you should carefully convince them by examining that phrasing. 

Suppose\marginnote{21.1} another spiritual companion were to recite the teaching in the \textsanskrit{Saṅgha}. Now, you might think, ‘This venerable construes the meaning correctly and gets the phrasing right.’ Saying ‘Good!’ you should applaud and cheer that mendicant’s statement, and then say to them, ‘We are fortunate, reverend, so very fortunate to see a venerable such as yourself, so well-versed in the meaning and the phrasing, as one of our spiritual companions!’ 

\section*{8. The Reasons for Allowing Requisites }

Cunda,\marginnote{22.1} I do not teach you solely for restraining defilements that affect the present life. Nor do I teach solely for protecting against defilements that affect lives to come. I teach both for restraining defilements that affect the present life and protecting against defilements that affect lives to come. 

And\marginnote{22.5} that’s why I have allowed robes for you that suffice only for the sake of warding off cold and heat; for warding off the touch of flies, mosquitoes, wind, sun, and reptiles; and for covering up the private parts. I have allowed almsfood for you that suffices only to sustain this body, avoid harm, and support spiritual practice; so that you will put an end to old discomfort and not give rise to new discomfort, and will keep on living blamelessly and at ease. I have allowed lodgings for you that suffice only for the sake of warding off cold and heat; for warding off the touch of flies, mosquitoes, wind, sun, and reptiles; to shelter from harsh weather and to enjoy retreat. I have allowed medicines and supplies for the sick for you that suffice only for the sake of warding off the pains of illness and to promote good health. 

\section*{9. Indulgence in Pleasure }

It’s\marginnote{23.1} possible that wanderers who follow other paths might say, ‘The ascetics who follow the Sakyan live indulging in pleasure.’ You should say to them, ‘What is that indulgence in pleasure? For there are many different kinds of indulgence in pleasure.’ 

These\marginnote{23.6} four kinds of indulgence in pleasure, Cunda, are low, crude, ordinary, ignoble, and pointless. They don’t lead to disillusionment, dispassion, cessation, peace, insight, awakening, and extinguishment. What four? 

It’s\marginnote{23.8} when some fool makes themselves happy and pleased by killing living creatures. This is the first kind of indulgence in pleasure. 

Furthermore,\marginnote{23.10} someone makes themselves happy and pleased by theft. This is the second kind of indulgence in pleasure. 

Furthermore,\marginnote{23.12} someone makes themselves happy and pleased by lying. This is the third kind of indulgence in pleasure. 

Furthermore,\marginnote{23.14} someone amuses themselves, supplied and provided with the five kinds of sensual stimulation. This is the fourth kind of indulgence in pleasure. 

These\marginnote{23.16} are the four kinds of indulgence in pleasure that are low, crude, ordinary, ignoble, and pointless. They don’t lead to disillusionment, dispassion, cessation, peace, insight, awakening, and extinguishment. 

It’s\marginnote{24.1} possible that wanderers who follow other paths might say, ‘The ascetics who follow the Sakyan live indulging in pleasure in these four ways.’ They should be told, ‘Not so!’ It isn’t right to say that about you; it misrepresents you with an untruth. 

These\marginnote{24.5} four kinds of indulgence in pleasure, when developed and cultivated, lead solely to disillusionment, dispassion, cessation, peace, insight, awakening, and extinguishment. What four? 

It’s\marginnote{24.7} when a mendicant, quite secluded from sensual pleasures, secluded from unskillful qualities, enters and remains in the first absorption, which has the rapture and bliss born of seclusion, while placing the mind and keeping it connected. This is the first kind of indulgence in pleasure. 

Furthermore,\marginnote{24.9} as the placing of the mind and keeping it connected are stilled, a mendicant enters and remains in the second absorption. It has the rapture and bliss born of immersion, with internal clarity and confidence, and unified mind, without placing the mind and keeping it connected. This is the second kind of indulgence in pleasure. 

Furthermore,\marginnote{24.11} with the fading away of rapture, a mendicant enters and remains in the third absorption. They meditate with equanimity, mindful and aware, personally experiencing the bliss of which the noble ones declare, ‘Equanimous and mindful, one meditates in bliss.’ This is the third kind of indulgence in pleasure. 

Furthermore,\marginnote{24.13} giving up pleasure and pain, and ending former happiness and sadness, a mendicant enters and remains in the fourth absorption. It is without pleasure or pain, with pure equanimity and mindfulness. This is the fourth kind of indulgence in pleasure. 

These\marginnote{24.15} are the four kinds of indulgence in pleasure which, when developed and cultivated, lead solely to disillusionment, dispassion, cessation, peace, insight, awakening, and extinguishment. 

It’s\marginnote{24.16} possible that wanderers who follow other paths might say, ‘The ascetics who follow the Sakyan live indulging in pleasure in these four ways.’ They should be told, ‘Exactly so!’ It’s right to say that about you; it doesn’t misrepresent you with an untruth. 

\section*{10. The Benefits of Indulgence in Pleasure }

It’s\marginnote{25.1} possible that wanderers who follow other paths might say, ‘How many fruits and benefits may be expected by those who live indulging in pleasure in these four ways?’ You should say to them, ‘Four benefits may be expected by those who live indulging in pleasure in these four ways. What four? 

Firstly,\marginnote{25.6} with the ending of three fetters a mendicant becomes a stream-enterer, not liable to be reborn in the underworld, bound for awakening. This is the first fruit and benefit. 

Furthermore,\marginnote{25.8} a mendicant—with the ending of three fetters, and the weakening of greed, hate, and delusion—becomes a once-returner. They come back to this world once only, then make an end of suffering. This is the second fruit and benefit. 

Furthermore,\marginnote{25.10} with the ending of the five lower fetters, a mendicant is reborn spontaneously and will become extinguished there, not liable to return from that world. This is the third fruit and benefit. 

Furthermore,\marginnote{25.12} a mendicant realizes the undefiled freedom of heart and freedom by wisdom in this very life, and lives having realized it with their own insight due to the ending of defilements. This is the fourth fruit and benefit. 

These\marginnote{25.14} four benefits may be expected by those who live indulging in pleasure in these four ways.’ 

\section*{11. Things Impossible for the Perfected }

It’s\marginnote{26.1} possible that wanderers who follow other paths might say, ‘The ascetics who follow the Sakyan are inconsistent.’ You should say to them, ‘Reverends, these things have been taught and pointed out for his disciples by the Blessed One, who knows and sees, the perfected one, the fully awakened Buddha, not to be transgressed so long as life lasts. Suppose there was a boundary pillar or an iron pillar with deep foundations, firmly embedded, imperturbable and unshakable. In the same way, these things have been taught and pointed out for his disciples by the Blessed One, who knows and sees, the perfected one, the fully awakened Buddha, not to be transgressed so long as life lasts. 

A\marginnote{26.7} mendicant who is perfected—with defilements ended, who has completed the spiritual journey, done what had to be done, laid down the burden, achieved their own true goal, utterly ended the fetters of rebirth, and is rightly freed through enlightenment—can’t transgress in nine respects. A mendicant with defilements ended can’t deliberately take the life of a living creature, take something with the intention to steal, have sex, tell a deliberate lie, or store up goods for their own enjoyment like they did as a lay person. And they can’t make decisions prejudiced by favoritism, hostility, stupidity, or cowardice. A mendicant who is perfected can’t transgress in these nine respects.’ 

\section*{12. Questions and Answers }

It’s\marginnote{27.1} possible that wanderers who follow other paths might say, ‘The ascetic Gotama demonstrates boundless knowledge and vision of the past, but not of the future. What’s up with that?’ Those wanderers, like incompetent fools, seem to imagine that one kind of knowledge and vision can be demonstrated by means of another kind of knowledge and vision. 

Regarding\marginnote{27.4} the past, the Realized One has knowledge stemming from memory. He recollects as far as he wants. 

Regarding\marginnote{27.6} the future, the Realized One has the knowledge born of awakening: ‘This is my last rebirth. Now there are no more future lives.’ 

If\marginnote{28.1} a question about the past is untrue, false, and pointless, the Realized One does not reply. If a question about the past is true and substantive, but pointless, he does not reply. If a question about the past is true, substantive, and beneficial, he knows the right time to reply. And the Realized One replies to questions about the future or the present in the same way. 

And\marginnote{28.8} so the Realized One has speech that’s well-timed, true, meaningful, in line with the teaching and training. That’s why he’s called the ‘Realized One’. 

In\marginnote{29.1} this world—with its gods, \textsanskrit{Māras}, and \textsanskrit{Brahmās}, this population with its ascetics and brahmins, its gods and humans—whatever is seen, heard, thought, known, sought, and explored by the mind, all that has been understood by the Realized One. That’s why he’s called the ‘Realized One’. 

From\marginnote{29.3} the night when the Realized One understands the supreme perfect awakening until the night he becomes fully extinguished—through the element of extinguishment with nothing left over—everything he speaks, says, and expresses is real, not otherwise. That’s why he’s called the ‘Realized One’. 

The\marginnote{29.5} Realized One does as he says, and says as he does. Since this is so, that’s why he’s called the ‘Realized One’. In this world—with its gods, \textsanskrit{Māras} and \textsanskrit{Brahmās}, this population with its ascetics and brahmins, gods and humans—the Realized One is the undefeated, the champion, the universal seer, the wielder of power. 

\section*{13. The Undeclared Points }

It’s\marginnote{30.1} possible that wanderers who follow other paths might say, ‘Is this your view: “A Realized One exists after death. This is the only truth, other ideas are silly”?’ You should say to them, ‘Reverend, this has not been declared by the Buddha.’ 

The\marginnote{30.6} wanderers might say, ‘Then is this your view: “A Realized One doesn’t exist after death. This is the only truth, other ideas are silly”?’ You should say to them, ‘This too has not been declared by the Buddha.’ 

The\marginnote{30.11} wanderers might say, ‘Then is this your view: “A Realized One both exists and doesn’t exist after death. This is the only truth, other ideas are silly”?’ You should say to them, ‘This too has not been declared by the Buddha.’ 

The\marginnote{30.16} wanderers might say, ‘Then is this your view: “A Realized One neither exists nor doesn’t exist after death. This is the only truth, other ideas are silly”?’ You should say to them, ‘This too has not been declared by the Buddha.’ 

The\marginnote{31.1} wanderers might say, ‘But why has this not been declared by the ascetic Gotama?’ You should say to them, ‘Because it’s not beneficial or relevant to the fundamentals of the spiritual life. It doesn’t lead to disillusionment, dispassion, cessation, peace, insight, awakening, and extinguishment. That’s why it hasn’t been declared by the Buddha.’ 

\section*{14. The Declared Points }

It’s\marginnote{32.1} possible that wanderers who follow other paths might say, ‘But what has been declared by the ascetic Gotama?’ You should say to them, ‘What has been declared by the Buddha is this: “This is suffering”—“This is the origin of suffering”—“This is the cessation of suffering”—“This is the practice that leads to the cessation of suffering.”’ 

The\marginnote{33.1} wanderers might say, ‘But why has this been declared by the ascetic Gotama?’ You should say to them, ‘Because it’s beneficial and relevant to the fundamentals of the spiritual life. It leads to disillusionment, dispassion, cessation, peace, insight, awakening, and extinguishment. That’s why it has been declared by the Buddha.’ 

\section*{15. Views of the Past }

Cunda,\marginnote{34.1} I have explained to you as they should be explained the views that some rely on regarding the past. Shall I explain them to you in the wrong way? I have explained to you as they should be explained the views that some rely on regarding the future. Shall I explain them to you in the wrong way? 

What\marginnote{34.5} are the views that some rely on regarding the past? There are some ascetics and brahmins who have this doctrine and view: ‘The self and the cosmos are eternal. This is the only truth, other ideas are silly.’ There are some ascetics and brahmins who have this doctrine and view: ‘The self and the cosmos are not eternal, or both eternal and not eternal, or neither eternal nor not eternal. The self and the cosmos are made by oneself, or made by another, or made by both oneself and another, or they have arisen by chance, not made by oneself or another. Pleasure and pain are eternal, or not eternal, or both eternal and not eternal, or neither eternal nor not eternal. Pleasure and pain are made by oneself, or made by another, or made by both oneself and another, or they have arisen by chance, not made by oneself or another. This is the only truth, other ideas are silly.’ 

Regarding\marginnote{35.1} this, I go up to the ascetics and brahmins whose view is that the self and the cosmos are eternal, and say, ‘Reverends, is this what you say, “The self and the cosmos are eternal”?’ But when they say, ‘Yes! This is the only truth, other ideas are silly,’ I don’t acknowledge that. Why is that? Because there are beings who have different opinions on this topic. I don’t see any such expositions that are equal to my own, still less superior. Rather, I am the one who is superior when it comes to the higher exposition. 

Regarding\marginnote{36.1} this, I go up to the ascetics and brahmins who assert all the other views as described above. And in each case, I don’t acknowledge that. Why is that? Because there are beings who have different opinions on this topic. I don’t see any such expositions that are equal to my own, still less superior. Rather, I am the one who is superior when it comes to the higher exposition. 

These\marginnote{36.27} are the views that some rely on regarding the past. 

\section*{16. Views of the Future }

What\marginnote{37.1} are the views that some rely on regarding the future? There are some ascetics and brahmins who have this doctrine and view: ‘The self is physical and well after death, or it is non-physical, or both physical and non-physical, or neither physical nor non-physical, or percipient, or non-percipient, or neither percipient nor non-percipient, or the self is annihilated and destroyed when the body breaks up, and doesn’t exist after death. This is the only truth, other ideas are silly.’ 

Regarding\marginnote{38.1} this, I go up to the ascetics and brahmins whose view is that the self is physical and well after death, and say, ‘Reverends, is this what you say, “The self is physical and well after death”?’ But when they say, ‘Yes! This is the only truth, other ideas are silly,’ I don’t acknowledge that. Why is that? Because there are beings who have different opinions on this topic. I don’t see any such expositions that are equal to my own, still less superior. Rather, I am the one who is superior when it comes to the higher exposition. 

Regarding\marginnote{39.1} this, I go up to the ascetics and brahmins who assert all the other views as described above. And in each case, I don’t acknowledge that. Why is that? Because there are beings who have different opinions on this topic. I don’t see any such expositions that are equal to my own, still less superior. Rather, I am the one who is superior when it comes to the higher exposition. 

These\marginnote{39.19} are the views that some rely on regarding the future, which I have explained to you as they should be explained. Shall I explain them to you in the wrong way? 

I\marginnote{40.1} have taught and pointed out the four kinds of mindfulness meditation for giving up and going beyond all these views of the past and the future. What four? It’s when a mendicant meditates by observing an aspect of the body—keen, aware, and mindful, rid of desire and aversion for the world. They meditate observing an aspect of feelings … mind … principles—keen, aware, and mindful, rid of desire and aversion for the world. These are the four kinds of mindfulness meditation that I have taught for giving up and going beyond all these views of the past and the future.” 

Now\marginnote{41.1} at that time Venerable \textsanskrit{Upavāṇa} was standing behind the Buddha fanning him. He said to the Buddha, “It’s incredible, sir, it’s amazing! This exposition of the teaching is impressive, sir, it is very impressive. Sir, what is the name of this exposition of the teaching?” 

“Well,\marginnote{41.6} \textsanskrit{Upavāṇa}, you may remember this exposition of the teaching as ‘The Impressive Discourse’.” 

That\marginnote{41.7} is what the Buddha said. Satisfied, Venerable \textsanskrit{Upavāṇa} was happy with what the Buddha said. 

%
\chapter*{{\suttatitleacronym DN 30}{\suttatitletranslation The Marks of a Great Man }{\suttatitleroot Lakkhaṇasutta}}
\addcontentsline{toc}{chapter}{\tocacronym{DN 30} \toctranslation{The Marks of a Great Man } \tocroot{Lakkhaṇasutta}}
\markboth{The Marks of a Great Man }{Lakkhaṇasutta}
\extramarks{DN 30}{DN 30}

\scevam{So\marginnote{1.1.1} I have heard. }At one time the Buddha was staying near \textsanskrit{Sāvatthī} in Jeta’s Grove, \textsanskrit{Anāthapiṇḍika}’s monastery. There the Buddha addressed the mendicants, “Mendicants!” 

“Venerable\marginnote{1.1.5} sir,” they replied. The Buddha said this: 

“There\marginnote{1.1.7} are thirty-two marks of a great man. A great man who possesses these has only two possible destinies, no other. If he stays at home he becomes a king, a wheel-turning monarch, a just and principled king. His dominion extends to all four sides, he achieves stability in the country, and he possesses the seven treasures. He has the following seven treasures: the wheel, the elephant, the horse, the jewel, the woman, the treasurer, and the counselor as the seventh treasure. He has over a thousand sons who are valiant and heroic, crushing the armies of his enemies. After conquering this land girt by sea, he reigns by principle, without rod or sword. But if he goes forth from the lay life to homelessness, he becomes a perfected one, a fully awakened Buddha, who draws back the veil from the world. 

And\marginnote{1.2.1} what are the thirty-two marks? 

He\marginnote{1.2.4} has well-planted feet. 

On\marginnote{1.2.5} the soles of his feet there are thousand-spoked wheels, with rims and hubs, complete in every detail. 

He\marginnote{1.2.6} has projecting heels. 

He\marginnote{1.2.7} has long fingers. 

His\marginnote{1.2.8} hands and feet are tender. 

His\marginnote{1.2.9} hands and feet cling gracefully. 

His\marginnote{1.2.10} feet are arched. 

His\marginnote{1.2.11} calves are like those of an antelope. 

When\marginnote{1.2.12} standing upright and not bending over, the palms of both hands touch the knees. 

His\marginnote{1.2.13} private parts are covered in a foreskin. 

He\marginnote{1.2.14} is gold colored; his skin has a golden sheen. 

He\marginnote{1.2.15} has delicate skin, so delicate that dust and dirt don’t stick to his body. 

His\marginnote{1.2.16} hairs grow one per pore. 

His\marginnote{1.2.17} hairs stand up; they’re blue-black and curl clockwise. 

His\marginnote{1.2.18} body is as straight as \textsanskrit{Brahmā}’s. 

He\marginnote{1.2.19} has bulging muscles in seven places. 

His\marginnote{1.2.20} chest is like that of a lion. 

The\marginnote{1.2.21} gap between the shoulder-blades is filled in. 

He\marginnote{1.2.22} has the proportional circumference of a banyan tree: the span of his arms equals the height of his body. 

His\marginnote{1.2.23} torso is cylindrical. 

He\marginnote{1.2.24} has an excellent sense of taste. 

His\marginnote{1.2.25} jaw is like that of a lion. 

He\marginnote{1.2.26} has forty teeth. 

His\marginnote{1.2.27} teeth are even. 

His\marginnote{1.2.28} teeth have no gaps. 

His\marginnote{1.2.29} teeth are perfectly white. 

He\marginnote{1.2.30} has a large tongue. 

He\marginnote{1.2.31} has the voice of \textsanskrit{Brahmā}, like a cuckoo’s call. 

His\marginnote{1.2.32} eyes are deep blue. 

He\marginnote{1.2.33} has eyelashes like a cow’s. 

Between\marginnote{1.2.34} his eyebrows there grows a tuft, soft and white like cotton-wool. 

His\marginnote{1.2.35} head is shaped like a turban. 

These\marginnote{1.3.1} are the thirty-two marks of a great man. A great man who possesses these has only two possible destinies, no other. 

Seers\marginnote{1.3.4} who are outsiders remember these marks, but they do not know the specific deeds performed in the past to obtain each mark. 

\section*{1. Well-Planted Feet }

In\marginnote{1.4.1} some past lives, past existences, past abodes the Realized One was reborn as a human being. He firmly undertook and persisted in skillful behaviors such as good conduct by way of body, speech, and mind, giving and sharing, taking precepts, observing the sabbath, paying due respect to mother and father, ascetics and brahmins, honoring the elders in the family, and various other things pertaining to skillful behaviors. Due to performing, accumulating, heaping up, and amassing those deeds, when his body broke up, after death, he was reborn in a good place, a heavenly realm. There he surpassed the other gods in ten respects: divine life span, beauty, happiness, glory, sovereignty, sights, sounds, smells, tastes, and touches. When he passed away from there and came back to this state of existence he obtained this mark of a great man: he has well-planted feet. He places his foot on the ground evenly, raises it evenly, and touches the ground evenly with the whole sole of his foot. 

Possessing\marginnote{1.5.1} this mark, if he stays at home he becomes a wheel-turning monarch. He has the following seven treasures: the wheel, the elephant, the horse, the jewel, the woman, the treasurer, and the counselor as the seventh treasure. He has over a thousand sons who are valiant and heroic, crushing the armies of his enemies. After conquering this land girt by sea—free of harassment by bandits, successful and prosperous, safe, blessed, and untroubled—he reigns by principle, without rod or sword. And what does he obtain as king? He can’t be stopped by any human foe or enemy. That’s what he obtains as king. But if he goes forth from the lay life to homelessness, he becomes a perfected one, a fully awakened Buddha, who draws back the veil from the world. And what does he obtain as Buddha? He can’t be stopped by any foe or enemy whether internal or external; nor by greed, hate, or delusion; nor by any ascetic or brahmin or god or \textsanskrit{Māra} or \textsanskrit{Brahmā} or by anyone in the world. That’s what he obtains as Buddha.” The Buddha spoke this matter. 

On\marginnote{1.6.1} this it is said: 

\begin{verse}%
“Truth,\marginnote{1.6.2} principle, self-control, and restraint; \\
purity, precepts, and observing the sabbath; \\
giving, harmlessness, delighting in non-violence—\\
firmly undertaking these things, he lived accordingly. 

By\marginnote{1.6.6} means of these deeds he went to heaven, \\
where he enjoyed happiness and merriment. \\
After passing away from there to here, \\
he steps evenly on this rich earth. 

The\marginnote{1.6.10} gathered soothsayers predicted \\
that there is no stopping one of such even tread, \\
as householder or renunciate. \\
That’s the meaning shown by this mark. 

While\marginnote{1.6.14} living at home he cannot be stopped, \\
he defeats his foes, and cannot be beaten. \\
Due to the fruit of that deed, \\
he cannot be stopped by any human. 

But\marginnote{1.6.18} if he chooses the life gone forth, \\
seeing clearly, loving renunciation, \\
not even the best can hope to stop him; \\
this is the nature of the supreme person.” 

%
\end{verse}

\section*{2. Wheels on the Feet }

“Mendicants,\marginnote{1.7.1} in some past lives the Realized One was reborn as a human being. He brought happiness to many people, eliminating threats, terror, and danger, providing just protection and security, and giving gifts with all the trimmings. Due to performing those deeds he was reborn in a heavenly realm. When he came back to this state of existence he obtained this mark: on the soles of his feet there are thousand-spoked wheels, with rims and hubs, complete in every detail and well divided inside. 

Possessing\marginnote{1.8.1} this mark, if he stays at home he becomes a wheel-turning monarch. And what does he obtain as king? He has a large following of brahmins and householders, people of town and country, treasury officials, military officers, guardsmen, ministers, counselors, rulers, tax beneficiaries, and princes. That’s what he obtains as king. But if he goes forth from the lay life to homelessness, he becomes a fully awakened Buddha. And what does he obtain as Buddha? He has a large following of monks, nuns, laymen, laywomen, gods, humans, demons, dragons, and fairies. That’s what he obtains as Buddha.” The Buddha spoke this matter. 

On\marginnote{1.9.1} this it is said: 

\begin{verse}%
“In\marginnote{1.9.2} olden days, in past lives, \\
he brought happiness to many people, \\
ridding them of fear, terror, and danger, \\
eagerly guarding and protecting them. 

By\marginnote{1.9.6} means of these deeds he went to heaven, \\
where he enjoyed happiness and merriment. \\
After passing away from there to here, \\
wheels on his two feet are found, 

all\marginnote{1.9.10} rimmed around and thousand-spoked. \\
The gathered soothsayers predicted, \\
seeing the prince with the hundred-fold mark of merits, \\
that he’d have a following, subduing foes, 

which\marginnote{1.9.14} is why he has wheels all rimmed around. \\
If he doesn’t choose the life gone forth, \\
he’ll roll the wheel and rule the land. \\
The aristocrats will be his vassals, 

flocking\marginnote{1.9.18} to his glory. \\
But if he chooses the life gone forth, \\
seeing clearly, loving renunciation, \\
the gods, humans, demons, Sakka, and monsters; 

fairies\marginnote{1.9.22} and dragons, birds and beasts, \\
will flock to his glory, \\
the supreme, honored by gods and humans.” 

%
\end{verse}

\section*{3–5. Projecting Heels, Etc. }

“Mendicants,\marginnote{1.10.1} in some past lives the Realized One was reborn as a human being. He gave up killing living creatures, renouncing the rod and the sword. He was scrupulous and kind, living full of compassion for all living beings. Due to performing those deeds he was reborn in a heavenly realm. When he came back to this state of existence he obtained these three marks: he has projecting heels, long fingers, and his body is as straight as \textsanskrit{Brahmā}’s. 

Possessing\marginnote{1.11.1} these marks, if he stays at home he becomes a wheel-turning monarch. And what does he obtain as king? He’s long-lived, preserving his life for a long time. No human foe or enemy is able to take his life before his time. That’s what he obtains as king. And what does he obtain as Buddha? He’s long-lived, preserving his life for a long time. No foes or enemies—nor any ascetic or brahmin or god or \textsanskrit{Māra} or \textsanskrit{Brahmā} or anyone in the world—is able to take his life before his time. That’s what he obtains as Buddha.” The Buddha spoke this matter. 

On\marginnote{1.12.1} this it is said: 

\begin{verse}%
“Realizing\marginnote{1.12.2} for himself the horrors of death, \\
he refrained from killing other creatures. \\
By that good conduct he went to heaven, \\
where he enjoyed the fruit of deeds well done. 

Passing\marginnote{1.12.6} away, on his return to here, \\
he obtained these three marks: \\
his projecting heels are full and long, \\
and he’s straight, beautiful, and well-formed, like \textsanskrit{Brahmā}. 

Fair\marginnote{1.12.10} of limb, youthful, of good posture and breeding, \\
his fingers are soft and tender and long. \\
By these three marks of an excellent man, \\
they indicated that the prince’s life would be long: 

‘As\marginnote{1.12.14} a householder he will live long; \\
longer still if he goes forth, due to \\
mastery in the development of psychic power. \\
Thus this is the sign of long life.’” 

%
\end{verse}

\section*{6. Seven Bulges }

“Mendicants,\marginnote{1.13.1} in some past lives the Realized One was reborn as a human being. He was a donor of fine and tasty foods and drinks of all kinds, delicious and scrumptious. Due to performing those deeds he was reborn in a heavenly realm. When he came back to this state of existence he obtained this mark: he has bulging muscles in seven places. He has bulges on both hands, both feet, both shoulders, and his chest. 

Possessing\marginnote{1.14.1} this mark, if he stays at home he becomes a wheel-turning monarch. And what does he obtain as king? He gets fine and tasty foods and drinks of all kinds, delicious and scrumptious. That’s what he obtains as king. And what does he obtain as Buddha? He gets fine and tasty foods and drinks of all kinds, delicious and scrumptious. That’s what he obtains as Buddha.” The Buddha spoke this matter. 

On\marginnote{1.15.1} this it is said: 

\begin{verse}%
“He\marginnote{1.15.2} used to give the very best of flavors—\\
scrumptious foods of every kind. \\
Because of that good deed, \\
he rejoiced long in Nandana heaven. 

On\marginnote{1.15.6} returning to here, he got seven bulging muscles \\
and tender hands and feet are found. \\
The soothsayers expert in signs declared: \\
‘He’ll get tasty foods of all sorts 

as\marginnote{1.15.10} a householder, that’s what that means. \\
But even if he goes forth he’ll get the same, \\
supreme in gaining tasty foods of all sorts, \\
cutting all bonds of the lay life.’” 

%
\end{verse}

\section*{7–8. Tender and Clinging Hands }

“Mendicants,\marginnote{1.16.1} in some past lives the Realized One was reborn as a human being. He brought people together using the four ways of being inclusive: giving, kindly words, taking care, and equality. Due to performing those deeds he was reborn in a heavenly realm. When he came back to this state of existence he obtained these two marks: his hands and feet are tender, and they cling gracefully. 

Possessing\marginnote{1.17.1} these marks, if he stays at home he becomes a wheel-turning monarch. And what does he obtain as king? His retinue is inclusive, cohesive, and well-managed. This includes brahmins and householders, people of town and country, treasury officials, military officers, guardsmen, ministers, counselors, rulers, tax beneficiaries, and princes. That’s what he obtains as king. And what does he obtain as Buddha? His retinue is inclusive, cohesive, and well-managed. This includes monks, nuns, laymen, laywomen, gods, humans, demons, dragons, and fairies. That’s what he obtains as Buddha.” The Buddha spoke this matter. 

On\marginnote{1.18.1} this it is said: 

\begin{verse}%
“By\marginnote{1.18.2} giving and helping others, \\
kindly speech, and equal treatment, \\
such action and conduct as brought people together, \\
he went to heaven due to his esteemed virtue. 

Passing\marginnote{1.18.6} away, on his return to here, \\
the young baby prince obtained \\
hands and feet so tender and clinging, \\
lovely, graceful, and good-looking. 

His\marginnote{1.18.10} retinue is loyal and manageable, \\
staying agreeably all over this broad land. \\
Speaking kindly, seeking happiness, \\
he practices the good qualities he’s adopted. 

But\marginnote{1.18.14} if he gives up all sensual enjoyments, \\
as victor he speaks Dhamma to the people. \\
Devoted, they respond to his words; \\
after listening, they practice in line with the teaching.” 

%
\end{verse}

\section*{9–10. Arched Feet and Upright Hair }

“Mendicants,\marginnote{1.19.1} in some past lives the Realized One was reborn as a human being. His speech was meaningful and principled. He educated many people, bringing welfare and happiness, offering the teaching. Due to performing those deeds he was reborn in a heavenly realm. When he came back to this state of existence he obtained these two marks: his feet are arched and his hairs stand up. 

Possessing\marginnote{1.20.1} these marks, if he stays at home he becomes a wheel-turning monarch. And what does he obtain as king? He is the foremost, best, chief, highest, and finest of those who enjoy sensual pleasures. That’s what he obtains as king. And what does he obtain as Buddha? He is the foremost, best, chief, highest, and finest of all sentient beings. That’s what he obtains as Buddha.” The Buddha spoke this matter. 

On\marginnote{1.21.1} this it is said: 

\begin{verse}%
“His\marginnote{1.21.2} word was meaningful and principled, \\
moving the people with his explanations. \\
He brought welfare and happiness to creatures, \\
unstintingly offering up teaching. 

Because\marginnote{1.21.6} of that good deed, \\
he advanced to heaven and there rejoiced. \\
On return to here two marks are found, \\
of excellence and supremacy. 

His\marginnote{1.21.10} hairs stand upright, \\
and his ankles stand out well. \\
Swollen with flesh and blood, and wrapped in skin, \\
they make it pretty above the feet. 

If\marginnote{1.21.14} such a one lives in the home, \\
he turns into the best of those who enjoy sensual pleasures. \\
There’ll be none better than him; \\
he’ll proceed having mastered all India. 

But\marginnote{1.21.18} going forth the peerless renunciate \\
turns into the best of all creatures. \\
There’ll be none better than him, \\
he’ll live having mastered the whole world.” 

%
\end{verse}

\section*{11. Antelope Calves }

“Mendicants,\marginnote{1.22.1} in some past lives the Realized One was reborn as a human being. He was a thorough teacher of a profession, a branch of knowledge, conduct, or action, thinking: ‘How might they quickly learn and practice, without getting exhausted?’ Due to performing those deeds he was reborn in a heavenly realm. When he came back to this state of existence he obtained this mark: his calves are like those of an antelope. 

Possessing\marginnote{1.23.1} this mark, if he stays at home he becomes a wheel-turning monarch. And what does he obtain as king? He quickly obtains the things worthy of a king, the factors, supports, and things befitting a king. That’s what he obtains as king. And what does he obtain as Buddha? He quickly obtains the things worthy of an ascetic, the factors, supports, and things befitting an ascetic. That’s what he obtains as Buddha.” The Buddha spoke this matter. 

On\marginnote{1.24.1} this it is said: 

\begin{verse}%
“In\marginnote{1.24.2} professions, knowledge, conduct, and deeds, \\
he thought of how they might swiftly learn. \\
Things that harm no-one at all, \\
he quickly taught so they would not get tired. 

Having\marginnote{1.24.6} done that skillful deed whose outcome is happiness, \\
he gains prominent and elegant calves. \\
Well-formed in graceful spirals, \\
he’s covered in fine rising hairs. 

They\marginnote{1.24.10} say that person has antelope calves, \\
and that this is the mark of swift success. \\
If he desires the things of the household life, \\
not going forth, they’ll quickly be his. 

But\marginnote{1.24.14} if he chooses the life gone forth, \\
seeing clearly, loving renunciation, \\
the peerless renunciate will quickly find \\
what is fitting and suitable.” 

%
\end{verse}

\section*{12. Delicate Skin }

“Mendicants,\marginnote{1.25.1} in some past lives the Realized One was reborn as a human being. He approached an ascetic or brahmin and asked: ‘Sirs, what is skillful? What is unskillful? What is blameworthy? What is blameless? What should be cultivated? What should not be cultivated? Doing what leads to my lasting harm and suffering? Doing what leads to my lasting welfare and happiness?’ Due to performing those deeds he was reborn in a heavenly realm. When he came back to this state of existence he obtained this mark: he has delicate skin, so delicate that dust and dirt don’t stick to his body. 

Possessing\marginnote{1.26.1} this mark, if he stays at home he becomes a wheel-turning monarch. And what does he obtain as king? He has great wisdom. Of those who enjoy sensual pleasures, none is his equal or better in wisdom. That’s what he obtains as king. And what does he obtain as Buddha? He has great wisdom, widespread wisdom, laughing wisdom, swift wisdom, sharp wisdom, and penetrating wisdom. No sentient being is his equal or better in wisdom. That’s what he obtains as Buddha.” The Buddha spoke this matter. 

On\marginnote{1.27.1} this it is said: 

\begin{verse}%
“In\marginnote{1.27.2} olden days, in past lives, \\
eager to understand, he asked questions. \\
Keen to learn, he waited on renunciates, \\
heeding their explanation with pure intent. 

Due\marginnote{1.27.6} to that deed of acquiring wisdom, \\
as a human being his skin is delicate. \\
At his birth the soothsayers expert in signs prophesied: \\
‘He’ll discern delicate matters.’ 

If\marginnote{1.27.10} he doesn’t choose the life gone forth, \\
he’ll roll the wheel and rule the land. \\
Among those with material possessions who have been educated, \\
none equal or better than him is found. 

But\marginnote{1.27.14} if he chooses the life gone forth, \\
seeing clearly, loving renunciation, \\
gaining wisdom that’s supreme and eminent, \\
the one of superb, vast intelligence attains awakening.” 

%
\end{verse}

\section*{13. Golden Skin }

“Mendicants,\marginnote{1.28.1} in some past lives the Realized One was reborn as a human being. He wasn’t irritable or bad-tempered. Even when heavily criticized he didn’t lose his temper, become annoyed, hostile, and hard-hearted, or display annoyance, hate, and bitterness. He donated soft and fine mats and blankets, and fine cloths of linen, cotton, silk, and wool. Due to performing those deeds he was reborn in a heavenly realm. When he came back to this state of existence he obtained this mark: he is gold colored; his skin has a golden sheen. 

Possessing\marginnote{1.29.1} this mark, if he stays at home he becomes a wheel-turning monarch. And what does he obtain as king? He obtains soft and fine mats and blankets, and fine cloths of linen, cotton, silk, and wool. That’s what he obtains as king. And what does he obtain as Buddha? He obtains soft and fine mats and blankets, and fine cloths of linen, cotton, silk, and wool. That’s what he obtains as Buddha.” The Buddha spoke this matter. 

On\marginnote{1.30.1} this it is said: 

\begin{verse}%
“Dedicated\marginnote{1.30.2} to good will, he gave gifts. \\
In an earlier life he poured forth cloth \\
fine and soft to touch, \\
like a god pouring rain on this broad earth. 

So\marginnote{1.30.6} doing he passed from here to heaven, \\
where he enjoyed the fruits of deeds well done. \\
Here he wins a figure of gold, \\
like Indra, the finest of gods. 

If\marginnote{1.30.10} that man stays in the house, not wishing to go forth, \\
he conquers and rules this vast, broad earth. \\
He obtains abundant excellent cloth, \\
so fine and soft to touch. 

He\marginnote{1.30.14} receives robes, cloth, and the finest garments \\
if he chooses the life gone forth. \\
For he still partakes of past deed’s fruit; \\
what’s been done is never lost.” 

%
\end{verse}

\section*{14. Privates in Foreskin }

“Mendicants,\marginnote{1.31.1} in some past lives the Realized One was reborn as a human being. He reunited long-lost and long-separated relatives, friends, loved ones, and companions. He reunited mother with child and child with mother; father with child and child with father; brother with brother, brother with sister, sister with brother, and sister with sister, bringing them together with rejoicing. Due to performing those deeds he was reborn in a heavenly realm. When he came back to this state of existence he obtained this mark: his private parts are covered in a foreskin. 

Possessing\marginnote{1.32.1} this mark, if he stays at home he becomes a wheel-turning monarch. And what does he obtain as king? He has many sons, over a thousand sons who are valiant and heroic, crushing the armies of his enemies. That’s what he obtains as king. And what does he obtain as Buddha? He has many sons, many thousands of sons who are valiant and heroic, crushing the armies of his enemies. That’s what he obtains as Buddha.” The Buddha spoke this matter. 

On\marginnote{1.33.1} this it is said: 

\begin{verse}%
“In\marginnote{1.33.2} olden days, in past lives, \\
he reunited long-lost \\
and long-separated friends and family, \\
bringing them together with joy. 

By\marginnote{1.33.6} means of these deeds he went to heaven, \\
where he enjoyed happiness and merriment. \\
After passing away from there to here, \\
his private parts are covered in a foreskin. 

Such\marginnote{1.33.10} a one has many sons, \\
over a thousand descendants, \\
valiant and heroic, devastating foes, \\
a layman’s joy, speaking kindly. 

But\marginnote{1.33.14} if he lives the renunciate life \\
he has even more sons following his word. \\
As householder or renunciate, \\
that’s the meaning shown by this mark.” 

%
\end{verse}

\scendsection{The first recitation section is finished. }

\section*{15–16. Equal Proportions and Touching the Knees }

“Mendicants,\marginnote{2.1.1} in some past lives the Realized One was reborn as a human being. He regarded the gathered population equally. He knew what they had in common and what was their own. He knew each person, and he knew the distinctions between people. In each case, he made appropriate distinctions between people: ‘This one deserves that; that one deserves this.’ Due to performing those deeds he was reborn in a heavenly realm. When he came back to this state of existence he obtained these two marks: he has the proportional circumference of a banyan tree; and when standing upright and not bending over, the palms of both hands touch the knees. 

Possessing\marginnote{2.2.1} these marks, if he stays at home he becomes a wheel-turning monarch. And what does he obtain as king? He is rich, affluent, and wealthy, with lots of gold and silver, lots of property and assets, lots of money and grain, and a full treasury and storehouses. That’s what he obtains as king. And what does he obtain as Buddha? He is rich, affluent, and wealthy. He has these kinds of wealth: the wealth of faith, ethical conduct, conscience, prudence, learning, generosity, and wisdom. That’s what he obtains as Buddha.” The Buddha spoke this matter. 

On\marginnote{2.3.1} this it is said: 

\begin{verse}%
“Observing\marginnote{2.3.2} the many people in a community, \\
he weighed, evaluated, and judged each case: \\
‘This one deserves that.’ \\
That’s how he used to draw distinctions between people. 

Now\marginnote{2.3.6} standing without bending \\
he can touch his knees with both hands. \\
With the remaining ripening of the fruit of good deeds, \\
his circumference was that of a great tree. 

Learned\marginnote{2.3.10} experts in the many different \\
signs and marks prophesied: \\
‘The young prince will obtain \\
many different things that householders deserve. 

Here\marginnote{2.3.14} there are many suitable pleasures \\
for the ruler of the land to enjoy as householder. \\
But if he gives up all sensual enjoyments, \\
he will gain the supreme, highest peak of wealth.’” 

%
\end{verse}

\section*{17–19. A Lion’s Chest, Etc. }

“Mendicants,\marginnote{2.4.1} in some past lives the Realized One was reborn as a human being. He desired the good, the welfare, the comfort, and sanctuary of the people, thinking: ‘How might they flourish in faith, ethics, learning, generosity, teachings, and wisdom; in wealth and grain, fields and land, birds and beasts, children and partners; in bondservants, workers, and staff; in family, friends, and kin?’ Due to performing those deeds he was reborn in a heavenly realm. When he came back to this state of existence he obtained these three marks: his chest is like that of a lion; the gap between the shoulder-blades is filled in; and his torso is cylindrical. 

Possessing\marginnote{2.5.1} these marks, if he stays at home he becomes a wheel-turning monarch. And what does he obtain as king? He’s not liable to decline. He doesn’t decline in wealth and grain, fields and land, birds and beasts, children and partners; in bondservants, workers, and staff; in family, friends, and kin. He doesn’t decline in any of his accomplishments. That’s what he obtains as king. And what does he obtain as Buddha? He’s not liable to decline. He doesn’t decline in faith, ethics, learning, generosity, and wisdom. He doesn’t decline in any of his accomplishments. That’s what he obtains as Buddha.” The Buddha spoke this matter. 

On\marginnote{2.6.1} this it is said: 

\begin{verse}%
“His\marginnote{2.6.2} wish was this: ‘How may others not decline \\
in faith, ethics, learning, and intelligence, \\
in generosity, teachings, and much good else, \\
in coin and corn, fields and lands, 

in\marginnote{2.6.6} children, partners, and livestock, \\
in family, friends, and kin, \\
in health, and both beauty and happiness?’ \\
And so he ever desired their success. 

His\marginnote{2.6.10} chest was full like that of a lion, \\
his shoulder-gap filled in, and torso cylindrical. \\
Due to the well-done deeds of the past, \\
he had that portent of non-decline. 

Even\marginnote{2.6.14} as layman he grows in corn and coin, \\
in wives, children, and livestock. \\
But once gone forth, owning nothing, he attains \\
the supreme awakening which may never decline.” 

%
\end{verse}

\section*{20. Excellent Sense of Taste }

“Mendicants,\marginnote{2.7.1} in some past lives the Realized One was reborn as a human being. He would never hurt any sentient being with fists, stones, rods, or swords. Due to performing those deeds he was reborn in a heavenly realm. When he came back to this state of existence he obtained this mark: he has an excellent sense of taste. Taste-buds are produced in the throat for the tongue-tip and dispersed evenly. 

Possessing\marginnote{2.8.1} this mark, if he stays at home he becomes a wheel-turning monarch. And what does he obtain as king? He is rarely ill or unwell. His stomach digests well, being neither too hot nor too cold. That’s what he obtains as king. And what does he obtain as Buddha? He is rarely ill or unwell. His stomach digests well, being neither too hot nor too cold, but just right, and fit for meditation. That’s what he obtains as Buddha.” The Buddha spoke this matter. 

On\marginnote{2.9.1} this it is said: 

\begin{verse}%
“Not\marginnote{2.9.2} with fist or rod or stone, \\
or sword or beating to death, \\
or by bondage or threats \\
did he ever harm anyone. 

For\marginnote{2.9.6} that very reason he rejoiced in heaven after passing away, \\
finding happiness as a fruit of happy deeds. \\
With taste-buds well formed and even, \\
on his return here he has an excellent sense of taste. 

That’s\marginnote{2.9.10} why the clever visionaries said: \\
‘This man will have much happiness \\
as householder or renunciate. \\
That’s the meaning shown by this mark.’” 

%
\end{verse}

\section*{21–22. Deep Blue Eyes }

“Mendicants,\marginnote{2.10.1} in some past lives the Realized One was reborn as a human being. When looking at others he didn’t glare, look askance, or avert his eyes. Being straightforward, he reached out to others with straightforward intentions, looking at people with kindly eyes. Due to performing those deeds he was reborn in a heavenly realm. When he came back to this state of existence he obtained these two marks: his eyes are deep blue, and he has eyelashes like a cow’s. 

Possessing\marginnote{2.11.1} these marks, if he stays at home he becomes a wheel-turning monarch. And what does he obtain as king? The people look on him with kindly eyes. He is dear and beloved to the brahmins and householders, people of town and country, treasury officials, military officers, guardsmen, ministers, counselors, rulers, tax beneficiaries, and princes. That’s what he obtains as king. And what does he obtain as Buddha? The people look on him with kindly eyes. He is dear and beloved to the monks, nuns, laymen, laywomen, gods, humans, demons, dragons, and fairies. That’s what he obtains as Buddha.” The Buddha spoke this matter. 

On\marginnote{2.12.1} this it is said: 

\begin{verse}%
“With\marginnote{2.12.2} not a glare or glance askance, \\
nor averting of the eyes; \\
straightforward, he reached out straightforwardly, \\
looking at people with kindly eyes. 

In\marginnote{2.12.6} good rebirths he enjoyed the fruit \\
and result, rejoicing there. \\
But here he has a cow’s eyelashes, \\
and eyes deep blue so fair to see. 

Many\marginnote{2.12.10} soothsayers, men clever \\
and learned in prognostic texts, \\
expert in cow-like lashes, indicated he’d \\
be looked upon with kindly eyes. 

Even\marginnote{2.12.14} as a householder he’d be regarded kindly, \\
beloved of the people. \\
But if he becomes an ascetic, not lay, \\
as destroyer of sorrow he’ll be loved by many.” 

%
\end{verse}

\section*{23. Head Like a Turban }

“Mendicants,\marginnote{2.13.1} in some past lives the Realized One was reborn as a human being. He was the leader and forerunner of people in skillful behaviors such as good conduct by way of body, speech, and mind, giving and sharing, taking precepts, observing the sabbath, paying due respect to mother and father, ascetics and brahmins, honoring the elders in the family, and various other things pertaining to skillful behaviors. Due to performing those deeds he was reborn in a heavenly realm. When he came back to this state of existence he obtained this mark: his head is shaped like a turban. 

Possessing\marginnote{2.14.1} this mark, if he stays at home he becomes a wheel-turning monarch. And what does he obtain as king? He has a large following of brahmins and householders, people of town and country, treasury officials, military officers, guardsmen, ministers, counselors, rulers, tax beneficiaries, and princes. That’s what he obtains as king. And what does he obtain as Buddha? He has a large following of monks, nuns, laymen, laywomen, gods, humans, demons, dragons, and fairies. That’s what he obtains as Buddha.” The Buddha spoke this matter. 

On\marginnote{2.15.1} this it is said: 

\begin{verse}%
“Among\marginnote{2.15.2} people of good conduct, he was the leader, \\
devoted to a life of principle among the principled. \\
The people followed him, \\
and he experienced the fruit of good deeds in heaven. 

Having\marginnote{2.15.6} experienced that fruit, \\
he acquires a head shaped like a turban. \\
The experts in omens and signs prophesied: \\
‘He will be leader of the people. 

Among\marginnote{2.15.10} people then, as before, \\
they will bring presents for him. \\
If he becomes an aristocrat, ruler of the land, \\
he’ll gain the service of the people. 

But\marginnote{2.15.14} if that man goes forth, \\
he’ll be sophisticated, proficient in the teachings. \\
Devoted to the virtues of his instruction, \\
the people will become his followers.’” 

%
\end{verse}

\section*{24–25. One Hair Per Pore, and a Tuft }

“Mendicants,\marginnote{2.16.1} in some past lives the Realized One was reborn as a human being. He refrained from lying. He spoke the truth and stuck to the truth. He was honest and trustworthy, and didn’t trick the world with his words. Due to performing those deeds he was reborn in a heavenly realm. When he came back to this state of existence he obtained these two marks: his hairs grow one per pore, and between his eyebrows there grows a tuft, soft and white like cotton-wool. 

Possessing\marginnote{2.17.1} these marks, if he stays at home he becomes a wheel-turning monarch. And what does he obtain as king? He has many close adherents among the brahmins and householders, people of town and country, treasury officials, military officers, guardsmen, ministers, counselors, rulers, tax beneficiaries, and princes. That’s what he obtains as king. And what does he obtain as Buddha? He has many close adherents among the monks, nuns, laymen, laywomen, gods, humans, demons, dragons, and fairies. That’s what he obtains as Buddha.” The Buddha spoke this matter. 

On\marginnote{2.18.1} this it is said: 

\begin{verse}%
“In\marginnote{2.18.2} past lives he was true to his promise, \\
with no forked tongue, he shunned lies. \\
He never broke his word to anyone, \\
but spoke what was true, real, and factual. 

A\marginnote{2.18.6} tuft so very white like cotton-wool \\
grew prettily between his eyebrows. \\
And never two, but only one, \\
hair grew in each of his pores. 

Many\marginnote{2.18.10} soothsayers learned in marks \\
and expert in signs gathered and prophesied: \\
‘One like this, with tuft and hair so well-formed, \\
will have many as his close adherents. 

Even\marginnote{2.18.14} as householder many people will follow him, \\
due to the power of deeds in the past. \\
But once gone forth, owning nothing, \\
as Buddha the people will follow him.’” 

%
\end{verse}

\section*{26–27. Forty Gapless Teeth }

“Mendicants,\marginnote{2.19.1} in some past lives the Realized One was reborn as a human being. He refrained from divisive speech. He didn’t repeat in one place what he heard in another so as to divide people against each other. Instead, he reconciled those who were divided, supporting unity, delighting in harmony, loving harmony, speaking words that promote harmony. Due to performing those deeds he was reborn in a heavenly realm. When he came back to this state of existence he obtained these two marks: he has forty teeth, and his teeth have no gaps. 

Possessing\marginnote{2.20.1} these marks, if he stays at home he becomes a wheel-turning monarch. And what does he obtain as king? His retinue cannot be divided. This includes brahmins and householders, people of town and country, treasury officials, military officers, guardsmen, ministers, counselors, rulers, tax beneficiaries, and princes. That’s what he obtains as king. And what does he obtain as Buddha? His retinue cannot be divided. This includes monks, nuns, laymen, laywomen, gods, humans, demons, dragons, and fairies. That’s what he obtains as Buddha.” The Buddha spoke this matter. 

On\marginnote{2.21.1} this it is said: 

\begin{verse}%
“He\marginnote{2.21.2} spoke no words divisive causing friends to split, \\
creating disputes that foster division, \\
acting improperly by fostering quarrels, \\
creating division among friends. 

He\marginnote{2.21.6} spoke kind words to foster harmony, \\
uniting those who are divided. \\
He eliminated quarrels among the people, \\
rejoicing together with the united. 

In\marginnote{2.21.10} good rebirths he enjoyed the fruit \\
and result, rejoicing there. \\
Here his teeth are gapless, close together, \\
forty standing upright in his mouth. 

If\marginnote{2.21.14} he becomes an aristocrat, ruler of the land, \\
his assembly will be indivisible. \\
And as an ascetic, stainless, immaculate, \\
his assembly will follow him, unshakable.” 

%
\end{verse}

\section*{28–29. A Large Tongue and the Voice of \textsanskrit{Brahmā} }

“Mendicants,\marginnote{2.22.1} in some past lives the Realized One was reborn as a human being. He refrained from harsh speech. He spoke in a way that’s mellow, pleasing to the ear, lovely, going to the heart, polite, likable and agreeable to the people. Due to performing those deeds he was reborn in a heavenly realm. When he came back to this state of existence he obtained these two marks: he has a large tongue, and the voice of \textsanskrit{Brahmā}, like a cuckoo’s call. 

Possessing\marginnote{2.23.1} these marks, if he stays at home he becomes a wheel-turning monarch. And what does he obtain as king? He has a persuasive voice. His words are persuasive to brahmins and householders, people of town and country, treasury officials, military officers, guardsmen, ministers, counselors, rulers, tax beneficiaries, and princes. That’s what he obtains as king. And what does he obtain as Buddha? He has a persuasive voice. His words are persuasive to monks, nuns, laymen, laywomen, gods, humans, demons, dragons, and fairies. That’s what he obtains as Buddha.” The Buddha spoke this matter. 

On\marginnote{2.24.1} this it is said: 

\begin{verse}%
“He\marginnote{2.24.2} never spoke a loud harsh word, \\
insulting, quarrelsome, \\
causing harm, rude, crushing the people. \\
His speech was sweet, helpful, and kind. 

He\marginnote{2.24.6} uttered words dear to the mind, \\
going to the heart, pleasing to the ear. \\
He enjoyed the fruit of his good verbal conduct, \\
experiencing the fruit of good deeds in heaven. 

Having\marginnote{2.24.10} experienced that fruit, \\
on his return to here he acquired the voice of \textsanskrit{Brahmā}. \\
His tongue was long and wide, \\
and his speech was persuasive. 

Even\marginnote{2.24.14} as householder his speech brings prosperity. \\
But if that man goes forth, \\
speaking often to the people, \\
they’ll be persuaded by his fair words.” 

%
\end{verse}

\section*{30. A Lion-Like Jaw }

“Mendicants,\marginnote{2.25.1} in some past lives the Realized One was reborn as a human being. He refrained from talking nonsense. His words were timely, true, and meaningful, in line with the teaching and training. He said things at the right time which were valuable, reasonable, succinct, and beneficial. Due to performing those deeds he was reborn in a heavenly realm. When he came back to this state of existence he obtained this mark: his jaw is like that of a lion. 

Possessing\marginnote{2.26.1} this mark, if he stays at home he becomes a wheel-turning monarch. And what does he obtain as king? He can’t be destroyed by any human foe or enemy. That’s what he obtains as king. And what does he obtain as Buddha? He can’t be destroyed by any foe or enemy whether internal or external; nor by greed, hate, or delusion; nor by any ascetic or brahmin or god or \textsanskrit{Māra} or \textsanskrit{Brahmā} or by anyone in the world. That’s what he obtains as Buddha.” The Buddha spoke this matter. 

On\marginnote{2.27.1} this it is said: 

\begin{verse}%
“Neither\marginnote{2.27.2} nonsensical nor silly, \\
his way of speaking was never loose. \\
He eliminated what was useless, \\
and spoke for the welfare and happiness of the people. 

So\marginnote{2.27.6} doing he passed from here to be reborn in heaven, \\
where he enjoyed the fruit of deeds well done. \\
Passing away, on his return to here, \\
he gained a jaw like the finest of beasts. 

He\marginnote{2.27.10} became a king so very hard to defeat, \\
a mighty lord and ruler of men. \\
He was equal to the best in the city of the Three and Thirty, \\
like Indra, the finest of gods. 

One\marginnote{2.27.14} such as that is not easily beaten by fairies, \\
demons, spirits, monsters, or gods. \\
If he becomes of such a kind, \\
he illuminates the quarters and in-between.” 

%
\end{verse}

\section*{31–32. Even and White Teeth }

“Mendicants,\marginnote{2.28.1} in some past lives the Realized One was reborn as a human being. He gave up wrong livelihood and earned a living by right livelihood. He refrained from falsifying weights, metals, or measures; bribery, fraud, cheating, and duplicity; mutilation, murder, abduction, banditry, plunder, and violence. Due to performing, accumulating, heaping up, and amassing those deeds, when his body broke up, after death, he was reborn in a good place, a heavenly realm. There he surpassed the other gods in ten respects: divine life span, beauty, happiness, glory, sovereignty, sights, sounds, smells, tastes, and touches. When he came back to this state of existence he obtained these two marks: his teeth are even and perfectly white. 

Possessing\marginnote{2.29.1} these marks, if he stays at home he becomes a king, a wheel-turning monarch, a just and principled king. His dominion extends to all four sides, he achieves stability in the country, and he possesses the seven treasures. He has the following seven treasures: the wheel, the elephant, the horse, the jewel, the woman, the treasurer, and the counselor as the seventh treasure. He has over a thousand sons who are valiant and heroic, crushing the armies of his enemies. After conquering this land girt by sea—free of harassment by bandits, successful and prosperous, safe, blessed, and untroubled—he reigns by principle, without rod or sword. And what does he obtain as king? His retinue is pure. This includes brahmins and householders, people of town and country, treasury officials, military officers, guardsmen, ministers, counselors, rulers, tax beneficiaries, and princes. That’s what he obtains as king. 

But\marginnote{2.30.1} if he goes forth from the lay life to homelessness, he becomes a perfected one, a fully awakened Buddha, who draws back the veil from the world. And what does he obtain as Buddha? His retinue is pure. This includes monks, nuns, laymen, laywomen, gods, humans, demons, dragons, and fairies. That’s what he obtains as Buddha.” The Buddha spoke this matter. 

On\marginnote{2.31.1} this it is said: 

\begin{verse}%
“He\marginnote{2.31.2} abandoned wrong livelihood, and created \\
a way of life that’s fair, pure, and just. \\
He eliminated what was useless, \\
and lived for the welfare and happiness of the people. 

Having\marginnote{2.31.6} done what’s praised by the clever, the wise, and the good, \\
that man experienced the fruit in heaven. \\
Equal to the best in the heaven of Three and Thirty, \\
he enjoyed himself with pleasure and play. 

From\marginnote{2.31.10} there he passed back to a human life. \\
With the remaining ripening of the fruit of good deeds, \\
he obtained teeth that are even, \\
gleaming, bright, and white. 

Many\marginnote{2.31.14} soothsayers regarded as wise men \\
gathered and predicted of him: \\
‘With twice-born teeth so even, so white, so clean and bright \\
his retinue will be so pure. 

As\marginnote{2.31.18} king, his people will also be pure, \\
when he rules having conquered this earth so broad. \\
They won’t harm the country, \\
but will live for the welfare and happiness of the people. 

But\marginnote{2.31.22} if he goes forth he’ll be an ascetic free of ill, \\
his passions quelled, the veil drawn back. \\
Rid of stress and weariness, \\
he sees this world and the next. 

Those\marginnote{2.31.26} who do his bidding, both lay and renunciate, \\
shake off wickedness, impure and blameworthy. \\
He’s surrounded by pure people, who dispel \\
stains, callousness, sin, and corruptions.’” 

%
\end{verse}

That\marginnote{2.31.30} is what the Buddha said. Satisfied, the mendicants were happy with what the Buddha said. 

%
\chapter*{{\suttatitleacronym DN 31}{\suttatitletranslation Advice to Sigālaka }{\suttatitleroot Siṅgālasutta}}
\addcontentsline{toc}{chapter}{\tocacronym{DN 31} \toctranslation{Advice to Sigālaka } \tocroot{Siṅgālasutta}}
\markboth{Advice to Sigālaka }{Siṅgālasutta}
\extramarks{DN 31}{DN 31}

\scevam{So\marginnote{1.1} I have heard. }At one time the Buddha was staying near \textsanskrit{Rājagaha}, in the Bamboo Grove, the squirrels’ feeding ground. Now at that time the householder’s son \textsanskrit{Sigālaka} rose early and left \textsanskrit{Rājagaha}. With his clothes and hair all wet, he raised his joined palms to revere the quarters—east, south, west, north, below, and above. 

Then\marginnote{2.1} the Buddha robed up in the morning and, taking his bowl and robe, entered \textsanskrit{Rājagaha} for alms. He saw \textsanskrit{Sigālaka} revering the quarters and said to him, “Householder’s son, why are you revering the quarters in this way?” 

“Sir,\marginnote{2.7} on his deathbed my father said to me: ‘My dear, please revere the quarters.’ Honoring, respecting, and venerating my father’s words, I rose early and left \textsanskrit{Rājagaha} and, with my clothes and hair all wet, raised my joined palms to revere the quarters—east, south, west, north, below, and above.” 

\section*{1. The Six Directions }

“Householder’s\marginnote{2.12} son, that’s not how the six directions should be revered in the training of the Noble One.” 

“But\marginnote{2.13} sir, how should the six directions be revered in the training of the Noble One? Sir, please teach me this.” 

“Well\marginnote{2.15} then, householder’s son, listen and pay close attention, I will speak.” 

“Yes,\marginnote{2.16} sir,” replied \textsanskrit{Sigālaka}. The Buddha said this: 

“Householder’s\marginnote{3.1} son, a noble disciple gives up four corrupt deeds, doesn’t do bad deeds on four grounds, and avoids six drains on wealth. When they’ve left these fourteen bad things behind they have the six directions covered. They’re practicing to win in both worlds, and they succeed in this world and the next. When their body breaks up, after death, they’re reborn in a good place, a heavenly realm. 

\section*{2. Four Corrupt Deeds }

What\marginnote{3.5} four corrupt deeds have they given up? Killing living creatures, stealing, sexual misconduct, and lying: these are corrupt deeds. These are the four corrupt deeds they’ve given up.” 

That\marginnote{3.8} is what the Buddha said. Then the Holy One, the Teacher, went on to say: 

\begin{verse}%
“Killing,\marginnote{4.1} stealing, \\
telling lies, \\
and committing adultery: \\
astute people don’t praise these things.” 

%
\end{verse}

\section*{3. Four Grounds }

“On\marginnote{5.1} what four grounds do they not do bad deeds? One does bad deeds prejudiced by favoritism, hostility, stupidity, and cowardice. When a noble disciple is not prejudiced by favoritism, hostility, stupidity, and cowardice, they don’t do bad deeds on these four grounds.” 

That\marginnote{5.5} is what the Buddha said. Then the Holy One, the Teacher, went on to say: 

\begin{verse}%
“If\marginnote{6.1} you act against the teaching \\
out of favoritism, hostility, cowardice, or stupidity, \\
your fame shrinks, \\
like the moon in the waning fortnight. 

If\marginnote{6.5} you don’t act against the teaching \\
out of favoritism, hostility, cowardice, and stupidity, \\
your fame swells, \\
like the moon in the waxing fortnight.” 

%
\end{verse}

\section*{4. Six Drains on Wealth }

“What\marginnote{7.1} six drains on wealth do they avoid? Habitually engaging in the following things is a drain on wealth: drinking alcohol; roaming the streets at night; frequenting festivals; gambling; bad friends; laziness. 

\section*{5. Six Drawbacks of Drinking }

There\marginnote{8.1} are these six drawbacks of habitually drinking alcohol. Immediate loss of wealth, promotion of quarrels, susceptibility to illness, disrepute, indecent exposure; and weakened wisdom is the sixth thing. These are the six drawbacks of habitually drinking alcohol. 

\section*{6. Six Drawbacks of Roaming the Streets at Night }

There\marginnote{9.1} are these six drawbacks of roaming the streets at night. Yourself, your partners and children, and your property are all left unguarded. You’re suspected of bad deeds. Untrue rumors spread about you. You’re at the forefront of many things that entail suffering. These are the six drawbacks of roaming the streets at night. 

\section*{7. Six Drawbacks of Festivals }

There\marginnote{10.1} are these six drawbacks of frequenting festivals. You’re always thinking: ‘Where’s the dancing? Where’s the singing? Where’s the music? Where are the stories? Where’s the applause? Where are the kettledrums?’ These are the six drawbacks of frequenting festivals. 

\section*{8. Six Drawbacks of Gambling }

There\marginnote{11.1} are these six drawbacks of habitually gambling. Victory breeds enmity. The loser mourns their money. There is immediate loss of wealth. A gambler’s word carries no weight in public assembly. Friends and colleagues treat them with contempt. And no-one wants to marry a gambler, for they think: ‘This individual is a gambler—they’re not able to support a partner.’ These are the six drawbacks of habitually gambling. 

\section*{9. Six Drawbacks of Bad Friends }

There\marginnote{12.1} are these six drawbacks of bad friends. You become friends and companions with those who are addicts, carousers, drunkards, frauds, swindlers, and thugs. These are the six drawbacks of bad friends. 

\section*{10. Six Drawbacks of Laziness }

There\marginnote{13.1} are these six drawbacks of habitual laziness. You don’t get your work done because you think: ‘It’s too cold! It’s too hot. It’s too late! It’s too early! I’m too hungry! I’m too full!’ By dwelling on so many excuses for not working, you don’t make any more money, and the money you already have runs out. These are the six drawbacks of habitual laziness.” 

That\marginnote{13.5} is what the Buddha said. Then the Holy One, the Teacher, went on to say: 

\begin{verse}%
“Some\marginnote{14.1} are just drinking buddies, \\
some call you their dear, dear friend, \\
but a true friend is one \\
who stands by you in need. 

Sleeping\marginnote{14.5} late, adultery, \\
making enemies, harmfulness, \\
bad friends, and avarice: \\
these six grounds ruin a person. 

With\marginnote{14.9} bad friends, bad companions, \\
bad behavior and alms-resort, \\
a man falls to ruin \\
in both this world and the next. 

Dice,\marginnote{14.13} women, drink, song and dance; \\
sleeping by day and roaming at night; \\
bad friends, and avarice: \\
these six grounds ruin a person. 

They\marginnote{14.17} play dice and drink liquor, \\
and consort with women loved by others. \\
Associating with the worse, not the better, \\
they diminish like the waning moon. 

A\marginnote{14.21} drunkard, broke, and destitute, \\
thirsty, drinking in the bar, \\
drowning in debt, \\
will quickly lose their way. 

When\marginnote{14.25} you’re in the habit of sleeping late, \\
seeing night as time to rise, \\
and always getting drunk, \\
you can’t keep up the household life. 

‘Too\marginnote{14.29} cold, too hot, \\
too late,’ they say. \\
When the young neglect their work like this, \\
riches pass them by. 

But\marginnote{14.33} one who considers heat and cold \\
as nothing more than blades of grass—\\
he does his duties as a man, \\
and happiness never fails.” 

%
\end{verse}

\section*{11. Fake Friends }

“Householder’s\marginnote{15.1} son, you should recognize these four enemies disguised as friends: the taker, the talker, the flatterer, the spender. 

You\marginnote{16.1} can recognize a fake friend who’s all take on four grounds. 

\begin{verse}%
Your\marginnote{16.2} possessions end up theirs. \\
Giving little, they expect a lot. \\
They do their duty out of fear. \\
They associate for their own advantage. 

%
\end{verse}

You\marginnote{16.6} can recognize a fake friend who’s all take on these four grounds. 

You\marginnote{17.1} can recognize a fake friend who’s all talk on four grounds. They’re hospitable in the past. They’re hospitable in the future. They’re full of meaningless pleasantries. When something needs doing in the present they point to their own misfortune. You can recognize a fake friend who’s all talk on these four grounds. 

You\marginnote{18.1} can recognize a fake friend who’s a flatterer on four grounds. They support you equally in doing bad and doing good. They praise you to your face, and put you down behind your back. You can recognize a fake friend who’s a flatterer on these four grounds. 

You\marginnote{19.1} can recognize a fake friend who’s a spender on four grounds. They accompany you when drinking, roaming the streets at night, frequenting festivals, and gambling. You can recognize a fake friend who’s a spender on these four grounds.” 

That\marginnote{19.4} is what the Buddha said. Then the Holy One, the Teacher, went on to say: 

\begin{verse}%
“One\marginnote{20.1} friend is all take, \\
another all talk; \\
one’s just a flatterer, \\
and one’s a friend who spends. 

An\marginnote{20.5} astute person understands \\
these four enemies for what they are \\
and keeps them at a distance, \\
as they’d shun a risky road.” 

%
\end{verse}

\section*{12. Good-Hearted Friends }

“Householder’s\marginnote{21.1} son, you should recognize these four good-hearted friends: the helper, the friend in good times and bad, the counselor, and the one who’s compassionate. 

You\marginnote{22.1} can recognize a good-hearted friend who’s a helper on four grounds. They guard you when you’re negligent. They guard your property when you’re negligent. They keep you safe in times of danger. When something needs doing, they supply you with twice the money you need. You can recognize a good-hearted friend who’s a helper on these four grounds. 

You\marginnote{23.1} can recognize a good-hearted friend who’s the same in good times and bad on four grounds. They tell you secrets. They keep your secrets. They don’t abandon you in times of trouble. They’d even give their life for your welfare. You can recognize a good-hearted friend who’s the same in good times and bad on these four grounds. 

You\marginnote{24.1} can recognize a good-hearted friend who’s a counselor on four grounds. They keep you from doing bad. They support you in doing good. They teach you what you do not know. They explain the path to heaven. You can recognize a good-hearted friend who’s a counselor on these four grounds. 

You\marginnote{25.1} can recognize a good-hearted friend who’s compassionate on four grounds. They don’t delight in your misfortune. They delight in your good fortune. They keep others from criticizing you. They encourage praise of you. You can recognize a good-hearted friend who’s compassionate on these four grounds.” 

The\marginnote{25.4} Buddha spoke this matter. Then the Holy One, the Teacher, went on to say: 

\begin{verse}%
“A\marginnote{26.1} friend who’s a helper, \\
one the same in both pleasure and pain, \\
a friend of good counsel, \\
and one of compassion; 

an\marginnote{26.5} astute person understands \\
these four friends for what they are \\
and carefully looks after them, \\
like a mother the child at her breast. \\
The astute and virtuous \\
shine like a burning flame. 

They\marginnote{26.11} pick up riches as bees \\
roaming round pick up pollen. \\
And their riches proceed to grow, \\
like an ant-hill piling up. 

In\marginnote{26.15} gathering wealth like this, \\
a householder does enough for their family. \\
And they’d hold on to friends \\
by dividing their wealth in four. 

One\marginnote{26.19} portion is to enjoy. \\
Two parts invest in work. \\
And the fourth should be kept \\
for times of trouble.” 

%
\end{verse}

\section*{13. Covering the Six Directions }

“And\marginnote{27.1} how, householder’s son, does a noble disciple cover the six directions? These six directions should be recognized: parents as the east, teachers as the south, partner and children as the west, friends and colleagues as the north, bondservants and workers as beneath, and ascetics and brahmins as above. 

A\marginnote{28.1} child should serve their parents as the eastern quarter in five ways, thinking: ‘I will support those who supported me. I’ll do my duty for them. I’ll maintain the family traditions. I’ll take care of the inheritance. When they have passed away, I’ll make an offering on their behalf.’ Parents served by the children in these five ways show compassion to them in five ways. They keep them from doing bad. They support them in doing good. They train them in a profession. They connect them with a suitable partner. They transfer the inheritance in due time. Parents served by their children in these five ways show compassion to them in these five ways. And that’s how the eastern quarter is covered, kept safe and free of peril. 

A\marginnote{29.1} student should serve their teacher as the southern quarter in five ways: by rising for them, by serving them, by listening well, by looking after them, and by carefully learning their profession. Teachers served by their students in these five ways show compassion to them in five ways. They make sure they’re well trained and well educated. They clearly explain all the knowledge of the profession. They introduce them to their friends and colleagues. They provide protection in every region. Teachers served by their students in these five ways show compassion to them in these five ways. And that’s how the southern quarter is covered, kept safe and free of peril. 

A\marginnote{30.1} husband should serve his wife as the western quarter in five ways: by treating her with honor, by not looking down on her, by not being unfaithful, by relinquishing authority to her, and by presenting her with adornments. A wife served by her husband in these five ways shows compassion to him in five ways. She’s well-organized in her work. She manages the domestic help. She’s not unfaithful. She preserves his earnings. She’s deft and tireless in all her duties. A wife served by her husband in these five ways shows compassion to him in these five ways. And that’s how the western quarter is covered, kept safe and free of peril. 

A\marginnote{31.1} gentleman should serve his friends and colleagues as the northern quarter in five ways: giving, kindly words, taking care, equality, and not using tricky words. Friends and colleagues served by a gentleman in these five ways show compassion to him in five ways. They guard him when they’re negligent. They guard his property when they’re negligent. They keep him safe in times of danger. They don’t abandon him in times of trouble. They honor his descendants. Friends and colleagues served by a gentleman in these five ways show compassion to him in these five ways. And that’s how the northern quarter is covered, kept safe and free of peril. 

A\marginnote{32.1} master should serve their bondservants and workers as the lower quarter in five ways: by organizing work according to ability, by paying food and wages, by nursing them when sick, by sharing special treats, and by giving time off work. Bondservants and workers served by a master in these five ways show compassion to him in five ways. They get up first, and go to bed last. They don’t steal. They do their work well. And they promote a good reputation. Bondservants and workers served by a master in these five ways show compassion to him in these five ways. And that’s how the lower quarter is covered, kept safe and free of peril. 

A\marginnote{33.1} gentleman should serve ascetics and brahmins as the upper quarter in five ways: by loving deeds of body, speech, and mind, by not turning them away at the gate, and by providing them with material needs. Ascetics and brahmins served by a gentleman in these five ways show compassion to him in six ways. They keep him from doing bad. They support him in doing good. They think of him with kindly thoughts. They teach him what he does not know. They clarify what he’s already learned. They explain the path to heaven. Ascetics and brahmins served by a gentleman in these five ways show compassion to him in these six ways. And that’s how the upper quarter is covered, kept safe and free of peril.” 

The\marginnote{33.7} Buddha spoke this matter. Then the Holy One, the Teacher, went on to say: 

\begin{verse}%
“Parents\marginnote{34.1} are the east, \\
teachers the south, \\
wives and child the west, \\
friends and colleagues the north, 

servants\marginnote{34.5} and workers below, \\
and ascetics and brahmins above. \\
By honoring these quarters \\
a householder does enough for their family. 

The\marginnote{34.9} astute and the virtuous, \\
the gentle and the articulate, \\
the humble and the kind: \\
they’re the kind who win glory. 

The\marginnote{34.13} diligent, not lazy, \\
those not disturbed by troubles, \\
those consistent in conduct, the intelligent: \\
they’re the kind who win glory. 

The\marginnote{34.17} inclusive, the makers of friends, \\
the bountiful, those rid of stinginess, \\
those who lead, train, and persuade: \\
they’re the kind who win glory. 

Giving\marginnote{34.21} and kindly words, \\
taking care here, \\
and treating equally in worldly conditions, \\
as befits them in each case; \\
these ways of being inclusive in the world \\
are like a moving chariot’s linchpin. 

If\marginnote{34.27} there were no such ways of being inclusive, \\
neither mother nor father \\
would be respected and honored \\
for what they’ve done for their children. 

But\marginnote{34.31} since these ways of being inclusive do exist, \\
the astute do regard them well, \\
so they achieve greatness \\
and are praised.” 

%
\end{verse}

When\marginnote{35.1} this was said, \textsanskrit{Sigālaka} the householder’s son said to the Buddha, “Excellent, sir! Excellent! As if he were righting the overturned, or revealing the hidden, or pointing out the path to the lost, or lighting a lamp in the dark so people with good eyes can see what’s there, the Buddha has made the teaching clear in many ways. I go for refuge to the Buddha, to the teaching, and to the mendicant \textsanskrit{Saṅgha}. From this day forth, may the Buddha remember me as a lay follower who has gone for refuge for life.” 

%
\chapter*{{\suttatitleacronym DN 32}{\suttatitletranslation The Āṭānāṭiya Protection }{\suttatitleroot Āṭānāṭiyasutta}}
\addcontentsline{toc}{chapter}{\tocacronym{DN 32} \toctranslation{The Āṭānāṭiya Protection } \tocroot{Āṭānāṭiyasutta}}
\markboth{The Āṭānāṭiya Protection }{Āṭānāṭiyasutta}
\extramarks{DN 32}{DN 32}

\section*{1. The First Recitation Section }

\scevam{So\marginnote{1.1} I have heard. }At one time the Buddha was staying near \textsanskrit{Rājagaha}, on the Vulture’s Peak Mountain. Then, late at night, the Four Great Kings—with large armies of spirits, fairies, goblins, and dragons—set guards, troops, and wards at the four quarters and then, lighting up the entire Vulture’s Peak with their beauty, went up to the Buddha, bowed, and sat down to one side. Before sitting down to one side, some spirits bowed, some exchanged greetings and polite conversation, some held up their joined palms toward the Buddha, some announced their name and clan, while some kept silent. 

Seated\marginnote{2.1} to one side, the Great King \textsanskrit{Vessavaṇa} said to the Buddha, “Sir, some high spirits have confidence in the Buddha, some do not. Some middling spirits have confidence in the Buddha, some do not. Some low spirits have confidence in the Buddha, some do not. But mostly the spirits don’t have confidence in the Buddha. Why is that? Because the Buddha teaches them to refrain from killing living creatures, stealing, lying, sexual misconduct, and drinking alcohol. But mostly they don’t refrain from such things. They don’t like that or approve of it. 

Sir,\marginnote{2.13} there are disciples of the Buddha who frequent remote lodgings in the wilderness and the forest that are quiet and still, far from the madding crowd, remote from human settlements, and fit for retreat. There dwell high spirits who have no confidence in the Buddha’s dispensation. To give them confidence, may the Buddha please learn the \textsanskrit{Āṭānāṭiya} protection for the guarding, protection, safety, and comfort of the monks, nuns, laymen, and laywomen.” The Buddha consented in silence. 

Then,\marginnote{3.1} knowing that the Buddha had consented, on that occasion Great King \textsanskrit{Vessavaṇa} recited the \textsanskrit{Āṭānāṭiya} protection. 

\begin{verse}%
“Hail\marginnote{3.2} \textsanskrit{Vipassī}, \\
the glorious seer! \\
Hail \textsanskrit{Sikhī}, \\
compassionate for all beings! 

Hail\marginnote{3.6} \textsanskrit{Vessabhū}, \\
cleansed and austere! \\
Hail Kakusandha, \\
crusher of \textsanskrit{Māra}’s army! 

Hail\marginnote{3.10} \textsanskrit{Koṇāgamana}, \\
the brahmin who has lived the life! \\
Hail Kassapa, \\
everywhere free! 

Hail\marginnote{3.14} \textsanskrit{Aṅgīrasa}, \\
the glorious Sakyan! \\
He taught this Dhamma \\
that dispels all suffering. 

Those\marginnote{3.18} in the world who are extinguished, \\
truly discerning, \\
not backbiters; such people \\
being great of heart and rid of naivety, 

revere\marginnote{3.22} that Gotama; \\
he who is helpful to gods and humans, \\
accomplished in knowledge and conduct, \\
great of heart and rid of naivety. 

Where\marginnote{4.1} rises the sun—\\
Aditi’s child, the great circle, \\
who in his rising \\
dispels the night, \\
and of whom, when sun has risen, \\
it’s said to be the day—

there\marginnote{4.7} is a deep lake \\
an ocean, where water flows. \\
So they know that in that place \\
there is an ocean where waters flow. 

From\marginnote{4.11} here that is the eastern quarter, \\
so the people say. \\
That quarter is warded \\
by a great king, glorious, 

the\marginnote{4.15} lord of the fairies; \\
his name is \textsanskrit{Dhataraṭṭha}. \\
He delights in song and dance, \\
honored by the fairies. 

And\marginnote{4.19} he has many mighty sons \\
all of one name, so I’ve heard. \\
Eighty, and ten, and one—\\
all of them named Indra. 

After\marginnote{4.23} seeing the Awakened One, \\
the Buddha, Kinsman of the Sun, \\
they revere him from afar, \\
the one great of heart and rid of naivety. 

Homage\marginnote{4.27} to you, O thoroughbred! \\
Homage to you, supreme among men! \\
You’ve seen us with clarity and kindness. \\
The non-humans bow to you. \\
We’ve been asked many a time, \\
‘Do you bow to Gotama the victor?’ 

And\marginnote{4.33} so we ought to declare: \\
‘We bow to Gotama the victor, \\
accomplished in knowledge and conduct! \\
We bow to Gotama the awakened!’ 

It’s\marginnote{5.1} where the departed go, they say, \\
who are dividers and backbiters, \\
killers and hunters, \\
bandits and frauds. 

From\marginnote{5.5} here that is the southern quarter, \\
so the people say. \\
That quarter is warded \\
by a great king, glorious, 

the\marginnote{5.9} lord of the goblins; \\
his name is \textsanskrit{Virūḷha}. \\
He delights in song and dance, \\
honored by the goblins. 

And\marginnote{5.13} he has many mighty sons \\
all of one name, so I’ve heard. \\
Eighty, and ten, and one—\\
all of them named Indra. 

After\marginnote{5.17} seeing the Awakened One, \\
the Buddha, Kinsman of the Sun, \\
they revere him from afar, \\
the one great of heart and rid of naivety. 

Homage\marginnote{5.21} to you, O thoroughbred! \\
Homage to you, supreme among men! \\
You’ve seen us with clarity and kindness. \\
The non-humans bow to you. \\
We’ve been asked many a time, \\
‘Do you bow to Gotama the victor?’ 

And\marginnote{5.27} so we ought to declare: \\
‘We bow to Gotama the victor, \\
accomplished in knowledge and conduct! \\
We bow to Gotama the awakened!’ 

Where\marginnote{6.1} sets the sun—\\
Aditi’s child, the great circle, \\
who in his setting \\
closes the day, \\
and of whom, when sun has set, \\
it’s said to be the night—

there\marginnote{6.7} is a deep lake \\
an ocean, where water flows. \\
So they know that in that place \\
there is an ocean where waters flow. 

From\marginnote{6.11} here that is the western quarter, \\
so the people say. \\
That quarter is warded \\
by a great king, glorious, 

the\marginnote{6.15} lord of the dragons; \\
his name is \textsanskrit{Virūpakkha}. \\
He delights in song and dance, \\
honored by the dragons. 

And\marginnote{6.19} he has many mighty sons \\
all of one name, so I’ve heard. \\
Eighty, and ten, and one—\\
all of them named Indra. 

After\marginnote{6.23} seeing the Awakened One, \\
the Buddha, Kinsman of the Sun, \\
they revere him from afar, \\
the one great of heart and rid of naivety. 

Homage\marginnote{6.27} to you, O thoroughbred! \\
Homage to you, supreme among men! \\
You’ve seen us with clarity and kindness. \\
The non-humans bow to you. \\
We’ve been asked many a time, \\
‘Do you bow to Gotama the victor?’ 

And\marginnote{6.33} so we ought to declare: \\
‘We bow to Gotama the victor, \\
accomplished in knowledge and conduct! \\
We bow to Gotama the awakened!’ 

Where\marginnote{7.1} lovely Uttarakuru is, \\
and the beautiful Mount Meru, \\
humans born there \\
are unselfish, not possessive. 

They\marginnote{7.5} do not sow the seed, \\
nor do they draw the plough. \\
The rice eaten by people \\
ripens in untilled soil, 

free\marginnote{7.9} of powder or husk, pure, \\
fragrant, with only the rice-grain. \\
They eat that food \\
after cooking it in a ‘parrot’s beak’. 

Having\marginnote{7.13} prepared a cow with hooves uncloven, \\
they’re drawn about from place to place. \\
Having prepared a beast with hooves uncloven, \\
they’re drawn about from place to place. 

Having\marginnote{7.17} prepared a woman-drawn carriage, \\
they’re drawn about from place to place. \\
Having prepared a man-drawn carriage, \\
they’re drawn about from place to place. 

Having\marginnote{7.21} prepared a girl-drawn carriage, \\
they’re drawn about from place to place. \\
Having prepared a boy-drawn carriage, \\
they’re drawn about from place to place. 

Having\marginnote{7.25} ascended their vehicle, \\
that king’s servants \\
tour about in every quarter, 

provided\marginnote{7.28} with vehicles, \\
elephant, horse, and divine. \\
And there are mansions and palanquins \\
for that great and glorious king. 

And\marginnote{7.32} he has cities, too, \\
well-built in the sky: \\
\textsanskrit{Āṭānāṭā}, \textsanskrit{Kusināṭā}, \textsanskrit{Parakusināṭā}, \\
\textsanskrit{Nāṭasuriyā}, and \textsanskrit{Parakusiṭanāṭā}. 

To\marginnote{7.36} the north is \textsanskrit{Kapīvanta}, \\
and Jonogha lies beyond. \\
And there’s Navanavutiya, Ambara-ambaravatiya, \\
and the royal capital named \textsanskrit{Āḷakamandā}. 

The\marginnote{7.40} Great King Kuvera, dear sir, \\
has a capital named \textsanskrit{Visāṇā}, \\
which is why the great king \\
is called ‘\textsanskrit{Vessavaṇa}’. 

These\marginnote{7.44} each individually inform the King: \\
\textsanskrit{Tatolā}, \textsanskrit{Tattalā}, \textsanskrit{Tatotalā}, \\
Ojasi, Tejasi, Tatojasi, \\
\textsanskrit{Sūra}, \textsanskrit{Rājā}, \textsanskrit{Ariṭṭha}, and Nemi. 

There\marginnote{7.48} is a lake there too named \textsanskrit{Dharaṇī}, \\
from whence the clouds rain down, \\
and the rains disperse. \\
There is a hall there too named \textsanskrit{Bhagalavatī}, 

where\marginnote{7.52} the spirits frequent. \\
There the trees are ever in fruit, \\
with many different flocks of birds. \\
Peacocks and herons call out there, \\
and the sweet cuckoos too. 

One\marginnote{7.57} bird cries out ‘Live, live!’, \\
another ‘Lift up your heart!’ \\
There are cocks and kookaburras, \\
and in the wood the woodpeckers. 

The\marginnote{7.61} parrots and mynah cry out there, \\
and the ‘little stick-boy’ birds. \\
Kuvera’s pond of rushes \\
is lovely all the time. 

From\marginnote{7.65} here that is the northern quarter, \\
so the people say. \\
That quarter is warded \\
by a great king, glorious, 

the\marginnote{7.69} lord of spirits; \\
his name is ‘Kuvera’. \\
He delights in song and dance, \\
honored by the spirits. 

And\marginnote{7.73} he has many mighty sons \\
all of one name, so I’ve heard. \\
Eighty, and ten, and one—\\
all of them named Indra. 

After\marginnote{7.77} seeing the Awakened One, \\
the Buddha, Kinsman of the Sun, \\
they revere him from afar, \\
the one great of heart and rid of naivety. 

Homage\marginnote{7.81} to you, O thoroughbred! \\
Homage to you, supreme among men! \\
You’ve seen us with clarity and kindness. \\
The non-humans bow to you. \\
We’ve been asked many a time, \\
‘Do you bow to Gotama the victor?’ 

And\marginnote{7.87} so we ought to declare: \\
‘We bow to Gotama the victor, \\
accomplished in knowledge and conduct! \\
We bow to Gotama the awakened!’” 

%
\end{verse}

This,\marginnote{8.1} dear sir, is the \textsanskrit{Āṭānāṭiya} protection for the guarding, protection, safety, and comfort of the monks, nuns, laymen, and laywomen. 

The\marginnote{8.2} monks, nuns, laymen, and laywomen should learn this \textsanskrit{Āṭānāṭiya} protection well and completely memorize it. If anyone who does so is approached while walking, standing, sitting, or lying down by any non-human being with malicious intent—including males, females, boys, girls, ministers, counselors, and servants among the spirits, fairies, goblins, and dragons—that non-human will receive no homage or respect in any village or town. And they will receive no ground or dwelling in my capital of \textsanskrit{Ālakamandā}. Nor will they get to go to the conference of the spirits. In addition, the non-humans would not give or take them in marriage. They’d heap personal abuse on them, drop an empty bowl on their head, and even split their head into seven pieces! 

For\marginnote{9.1} there are, dear sir, non-humans who are fierce, cruel, and violent. They don’t obey the Great Kings or their men or their men’s men. They’re said to be rebelling against the Great Kings. They’re just like the bandits in the king of Magadha’s realm who don’t obey the king, his men, or his men’s men, and are said to be rebelling against the king. 

If\marginnote{9.6} any non-human being with malicious intent—including males, females, boys, girls, ministers, counselors, and servants among the spirits, fairies, goblins, and dragons—approaches a monk, nun, layman, or laywoman while walking, standing, sitting, or lying down, one ought to yell, cry, and scream to the spirits, great spirits, generals, great generals: ‘This spirit’s got me! This spirit’s entered me! This spirit’s annoying me! This spirit’s harassing me! This spirit’s hurting me! This spirit’s harming me! This spirit won’t let me go!’ 

To\marginnote{10.1} what spirits, great spirits, generals, great generals? 

\begin{verse}%
‘Indra,\marginnote{10.2} Soma, and \textsanskrit{Varuṇa}, \\
\textsanskrit{Bhāradvāja}, \textsanskrit{Pajāpati}, \\
Candana and \textsanskrit{Kāmaseṭṭha}, \\
\textsanskrit{Kinnughaṇḍu} and \textsanskrit{Nighaṇḍu}, 

\textsanskrit{Panāda}\marginnote{10.6} and \textsanskrit{Opamañña}, \\
and \textsanskrit{Mātali}, the god’s charioteer. \\
Cittasena the fairy, \\
and the kings Nala and Janesabha, 

\textsanskrit{Sātāgira},\marginnote{10.10} Hemavata, \\
\textsanskrit{Puṇṇaka}, Karatiya, and \textsanskrit{Guḷa}; \\
Sivaka and Mucalinda, \\
\textsanskrit{Vessāmitta}, Yugandhara, 

\textsanskrit{Gopāla},\marginnote{10.14} Supparodha, \\
Hiri, Netti, and Mandiya; \\
\textsanskrit{Pañcālacaṇḍa}, \textsanskrit{Āḷavaka}, \\
Pajjunna, Sumana, Sumukha, \\
Dadhimukha, \textsanskrit{Maṇi}, \textsanskrit{Māṇivara}, \textsanskrit{Dīgha}, \\
together with \textsanskrit{Serīsaka}.’ 

%
\end{verse}

This,\marginnote{11.1} dear sir, is the \textsanskrit{Āṭānāṭiya} protection for the guarding, protection, safety, and comfort of the monks, nuns, laymen, and laywomen. Well, now, dear sir, I must go. I have many duties, and much to do.” 

“Please,\marginnote{11.3} Great Kings, go at your convenience.” 

Then\marginnote{11.4} the Four Great Kings got up from their seats, bowed, and respectfully circled the Buddha, keeping him on their right side, before vanishing right there. And before the other spirits present vanished, some bowed and respectfully circled the Buddha, keeping him on their right side, some exchanged greetings and polite conversation, some held up their joined palms toward the Buddha, some announced their name and clan, while some kept silent. 

\scendsection{The first recitation section is finished. }

\section*{2. The Second Recitation Section }

Then,\marginnote{12.1} when the night had passed, the Buddha told the mendicants all that had happened, repeating all the verses spoken. Then he added: 

“Mendicants,\marginnote{13.1} learn the \textsanskrit{Āṭānāṭiya} protection! Memorize the \textsanskrit{Āṭānāṭiya} protection! Remember the \textsanskrit{Āṭānāṭiya} protection! The \textsanskrit{Āṭānāṭiya} protection is beneficial, and is for the guarding, protection, safety, and comfort of the monks, nuns, laymen, and laywomen.” 

That\marginnote{13.5} is what the Buddha said. Satisfied, the mendicants were happy with what the Buddha said. 

%
\chapter*{{\suttatitleacronym DN 33}{\suttatitletranslation Reciting in Concert }{\suttatitleroot Saṅgītisutta}}
\addcontentsline{toc}{chapter}{\tocacronym{DN 33} \toctranslation{Reciting in Concert } \tocroot{Saṅgītisutta}}
\markboth{Reciting in Concert }{Saṅgītisutta}
\extramarks{DN 33}{DN 33}

\scevam{So\marginnote{1.1.1} I have heard. }At one time the Buddha was wandering in the land of the Mallas together with a large \textsanskrit{Saṅgha} of five hundred mendicants when he arrived at a Mallian town named \textsanskrit{Pāvā}. There he stayed in Cunda the smith’s mango grove. 

Now\marginnote{1.2.1} at that time a new town hall named \textsanskrit{Ubbhaṭaka} had recently been constructed for the Mallas of \textsanskrit{Pāvā}. It had not yet been occupied by an ascetic or brahmin or any person at all. The Mallas of \textsanskrit{Pāvā} also heard that the Buddha had arrived and was staying in Cunda’s mango grove. Then they went up to the Buddha, bowed, sat down to one side, and said to him, “Sir, a new town hall named \textsanskrit{Ubbhaṭaka} has recently been constructed for the Mallas of \textsanskrit{Pāvā}. It has not yet been occupied by an ascetic or brahmin or any person at all. May the Buddha be the first to use it, and only then will the Mallas of \textsanskrit{Pāvā} use it. That would be for the lasting welfare and happiness of the Mallas of \textsanskrit{Pāvā}.” The Buddha consented in silence. 

Then,\marginnote{1.3.1} knowing that the Buddha had consented, the Mallas got up from their seat, bowed, and respectfully circled the Buddha, keeping him on their right. Then they went to the new town hall, where they spread carpets all over, prepared seats, set up a water jar, and placed a lamp. Then they went back to the Buddha, bowed, stood to one side, and told him of their preparations, saying, “Please, sir, come at your convenience.” 

Then\marginnote{1.4.1} the Buddha robed up and, taking his bowl and robe, went to the new town hall together with the \textsanskrit{Saṅgha} of mendicants. Having washed his feet he entered the town hall and sat against the central column facing east. The \textsanskrit{Saṅgha} of mendicants also washed their feet, entered the town hall, and sat against the west wall facing east, with the Buddha right in front of them. The Mallas of \textsanskrit{Pāvā} also washed their feet, entered the town hall, and sat against the east wall facing west, with the Buddha right in front of them. 

The\marginnote{1.4.4} Buddha spent most of the night educating, encouraging, firing up, and inspiring the Mallas with a Dhamma talk. Then he dismissed them, “The night is getting late, \textsanskrit{Vāseṭṭhas}. Please go at your convenience.” 

“Yes,\marginnote{1.4.7} sir,” replied the Mallas. They got up from their seat, bowed, and respectfully circled the Buddha, keeping him on their right, before leaving. 

Soon\marginnote{1.5.1} after they left, the Buddha looked around the \textsanskrit{Saṅgha} of mendicants, who were so very silent. He addressed Venerable \textsanskrit{Sāriputta}, “\textsanskrit{Sāriputta}, the \textsanskrit{Saṅgha} of mendicants is rid of dullness and drowsiness. Give them some Dhamma talk as you feel inspired. My back is sore, I’ll stretch it.” 

“Yes,\marginnote{1.5.6} sir,” \textsanskrit{Sāriputta} replied. 

And\marginnote{1.5.7} then the Buddha spread out his outer robe folded in four and laid down in the lion’s posture—on the right side, placing one foot on top of the other—mindful and aware, and focused on the time of getting up. 

Now\marginnote{1.6.1} at that time the \textsanskrit{Nigaṇṭha} \textsanskrit{Nātaputta} had recently passed away at \textsanskrit{Pāvā}. With his passing the Jain ascetics split, dividing into two factions, arguing, quarreling, and disputing, continually wounding each other with barbed words: “You don’t understand this teaching and training. I understand this teaching and training. What, you understand this teaching and training? You’re practicing wrong. I’m practicing right. I stay on topic, you don’t. You said last what you should have said first. You said first what you should have said last. What you’ve thought so much about has been disproved. Your doctrine is refuted. Go on, save your doctrine! You’re trapped; get yourself out of this—if you can!” 

You’d\marginnote{1.6.4} think there was nothing but slaughter going on among the Jain ascetics. And the \textsanskrit{Nigaṇṭha} \textsanskrit{Nātaputta}’s white-clothed lay disciples were disillusioned, dismayed, and disappointed in the Jain ascetics. They were equally disappointed with a teaching and training so poorly explained and poorly propounded, not emancipating, not leading to peace, proclaimed by someone who is not a fully awakened Buddha, with broken monument and without a refuge. 

Then\marginnote{1.7.1} \textsanskrit{Sāriputta} told the mendicants about these things. He went on to say, “That’s what happens, reverends, when a teaching and training is poorly explained and poorly propounded, not emancipating, not leading to peace, proclaimed by someone who is not a fully awakened Buddha. But this teaching is well explained and well propounded to us by the Blessed One, emancipating, leading to peace, proclaimed by someone who is a fully awakened Buddha. You should all recite this in concert, without disputing, so that this spiritual path may last for a long time. That would be for the welfare and happiness of the people, out of compassion for the world, for the benefit, welfare, and happiness of gods and humans. 

And\marginnote{1.7.7} what is that teaching? 

\section*{1. Ones }

There\marginnote{1.7.10} are teachings grouped by one that have been rightly explained by the Blessed One, who knows and sees, the perfected one, the fully awakened Buddha. You should all recite these in concert, without disputing, so that this spiritual path may last for a long time. That would be for the welfare and happiness of the people, out of compassion for the world, for the benefit, welfare, and happiness of gods and humans. What are the teachings grouped by one? 

‘All\marginnote{1.8.2} sentient beings are sustained by food.’ 

‘All\marginnote{1.8.3} sentient beings are sustained by conditions.’ 

These\marginnote{1.8.4} are the teachings grouped by one that have been rightly explained by the Blessed One, who knows and sees, the perfected one, the fully awakened Buddha. You should all recite these in concert, without disputing, so that this spiritual path may last for a long time. That would be for the welfare and happiness of the people, out of compassion for the world, for the benefit, welfare, and happiness of gods and humans. 

\section*{2. Twos }

There\marginnote{1.9.1} are teachings grouped by two that have been rightly explained by the Buddha. You should all recite these in concert. What are the teachings grouped by two? 

Name\marginnote{1.9.4} and form. 

Ignorance\marginnote{1.9.5} and craving for continued existence. 

Views\marginnote{1.9.6} favoring continued existence and views favoring ending existence. 

Lack\marginnote{1.9.7} of conscience and prudence. 

Conscience\marginnote{1.9.8} and prudence. 

Being\marginnote{1.9.9} hard to admonish and having bad friends. 

Being\marginnote{1.9.10} easy to admonish and having good friends. 

Skill\marginnote{1.9.11} in offenses and skill in rehabilitation from offenses. 

Skill\marginnote{1.9.12} in meditative attainments and skill in emerging from those attainments. 

Skill\marginnote{1.9.13} in the elements and skill in attention. 

Skill\marginnote{1.9.14} in the sense fields and skill in dependent origination. 

Skill\marginnote{1.9.15} in what is possible and skill in what is impossible. 

Integrity\marginnote{1.9.16} and scrupulousness. 

Patience\marginnote{1.9.17} and gentleness. 

Friendliness\marginnote{1.9.18} and hospitality. 

Harmlessness\marginnote{1.9.19} and purity. 

Lack\marginnote{1.9.20} of mindfulness and lack of situational awareness. 

Mindfulness\marginnote{1.9.21} and situational awareness. 

Not\marginnote{1.9.22} guarding the sense doors and eating too much. 

Guarding\marginnote{1.9.23} the sense doors and moderation in eating. 

The\marginnote{1.9.24} power of reflection and the power of development. 

The\marginnote{1.9.25} power of mindfulness and the power of immersion. 

Serenity\marginnote{1.9.26} and discernment. 

The\marginnote{1.9.27} foundation of serenity and the foundation of exertion. 

Exertion,\marginnote{1.9.28} and not being distracted. 

Failure\marginnote{1.9.29} in ethics and failure in view. 

Accomplishment\marginnote{1.9.30} in ethics and accomplishment in view. 

Purification\marginnote{1.9.31} of ethics and purification of view. 

Purification\marginnote{1.9.32} of view and making an effort in line with that view. 

Inspiration,\marginnote{1.9.33} and making a suitable effort when inspired by inspiring places. 

To\marginnote{1.9.34} never be content with skillful qualities, and to never stop trying. 

Knowledge\marginnote{1.9.35} and freedom. 

Knowledge\marginnote{1.9.36} of ending and knowledge of non-arising. 

These\marginnote{1.9.37} are the teachings grouped by two that have been rightly explained by the Buddha. You should all recite these in concert. 

\section*{3. Threes }

There\marginnote{1.10.1} are teachings grouped by three that have been rightly explained by the Buddha. You should all recite these in concert. What are the teachings grouped by three? 

Three\marginnote{1.10.4} unskillful roots: greed, hate, and delusion. 

Three\marginnote{1.10.6} skillful roots: non-greed, non-hate, and non-delusion. 

Three\marginnote{1.10.8} ways of performing bad conduct: by body, speech, and mind. 

Three\marginnote{1.10.10} ways of performing good conduct: by body, speech, and mind. 

Three\marginnote{1.10.12} unskillful thoughts: sensuality, malice, and cruelty. 

Three\marginnote{1.10.14} skillful thoughts: renunciation, good will, and harmlessness. 

Three\marginnote{1.10.16} unskillful intentions: sensuality, malice, and cruelty. 

Three\marginnote{1.10.18} skillful intentions: renunciation, good will, and harmlessness. 

Three\marginnote{1.10.20} unskillful perceptions: sensuality, malice, and cruelty. 

Three\marginnote{1.10.22} skillful perceptions: renunciation, good will, and harmlessness. 

Three\marginnote{1.10.24} unskillful elements: sensuality, malice, and cruelty. 

Three\marginnote{1.10.26} skillful elements: renunciation, good will, and harmlessness. 

Another\marginnote{1.10.28} three elements: sensuality, form, and formlessness. 

Another\marginnote{1.10.30} three elements: form, formlessness, and cessation. 

Another\marginnote{1.10.32} three elements: lower, middle, and higher. 

Three\marginnote{1.10.34} cravings: for sensual pleasures, to continue existence, and to end existence. 

Another\marginnote{1.10.36} three cravings: sensuality, form, and formlessness. 

Another\marginnote{1.10.38} three cravings: form, formlessness, and cessation. 

Three\marginnote{1.10.40} fetters: identity view, doubt, and misapprehension of precepts and observances. 

Three\marginnote{1.10.42} defilements: sensuality, desire for continued existence, and ignorance. 

Three\marginnote{1.10.44} realms of existence: sensual, form, and formless. 

Three\marginnote{1.10.46} searches: for sensual pleasures, for continued existence, and for a spiritual path. 

Three\marginnote{1.10.48} kinds of discrimination: ‘I’m better’, ‘I’m equal’, and ‘I’m worse’. 

Three\marginnote{1.10.50} periods: past, future, and present. 

Three\marginnote{1.10.52} extremes: identity, the origin of identity, and the cessation of identity. 

Three\marginnote{1.10.54} feelings: pleasure, pain, and neutral. 

Three\marginnote{1.10.56} forms of suffering: the suffering inherent in painful feeling, the suffering inherent in conditions, and the suffering inherent in perishing. 

Three\marginnote{1.10.58} heaps: inevitability regarding the wrong way, inevitability regarding the right way, and lack of inevitability. 

Three\marginnote{1.10.60} darknesses: one is doubtful, uncertain, undecided, and lacking confidence about the past, future, and present. 

Three\marginnote{1.10.62} things a Realized One need not hide. The Realized One’s behavior by way of body, speech, and mind is pure. He has no misconduct in these three ways that need be hidden, thinking: ‘Don’t let others find this out about me!’ 

Three\marginnote{1.10.69} possessions: greed, hate, and delusion. 

Three\marginnote{1.10.71} fires: greed, hate, and delusion. 

Another\marginnote{1.10.73} three fires: a fire for those worthy of offerings dedicated to the gods, a fire for householders, and a fire for those worthy of a religious donation. 

A\marginnote{1.10.75} threefold classification of the physical: visible and resistant, invisible and resistant, and invisible and non-resistant. 

Three\marginnote{1.10.77} choices: good choices, bad choices, and imperturbable choices. 

Three\marginnote{1.10.79} individuals: a trainee, an adept, and one who is neither a trainee nor an adept. 

Three\marginnote{1.10.81} seniors: a senior by birth, a senior in the teaching, and a senior by convention. 

Three\marginnote{1.10.83} grounds for making merit: giving, ethical conduct, and meditation. 

Three\marginnote{1.10.85} grounds for accusations: what is seen, heard, and suspected. 

Three\marginnote{1.10.87} kinds of sensual rebirth. There are sentient beings who desire what is present. They fall under the sway of presently arisen sensual pleasures. Namely, humans, some gods, and some beings in the underworld. This is the first kind of sensual rebirth. There are sentient beings who desire to create. Having repeatedly created, they fall under the sway of sensual pleasures. Namely, the Gods Who Love to Create. This is the second kind of sensual rebirth. There are sentient beings who desire what is created by others. They fall under the sway of sensual pleasures created by others. Namely, the Gods Who Control the Creations of Others. This is the third kind of sensual rebirth. 

Three\marginnote{1.10.94} kinds of pleasant rebirth. There are sentient beings who, having repeatedly given rise to it, dwell in pleasure. Namely, the gods of \textsanskrit{Brahmā}’s Host. This is the first pleasant rebirth. There are sentient beings who are drenched, steeped, filled, and soaked with pleasure. Every so often they feel inspired to exclaim: ‘Oh, what bliss! Oh, what bliss!’ Namely, the gods of streaming radiance. This is the second pleasant rebirth. There are sentient beings who are drenched, steeped, filled, and soaked with pleasure. Since they’re truly content, they experience pleasure. Namely, the gods replete with glory. This is the third pleasant rebirth. 

Three\marginnote{1.10.104} kinds of wisdom: the wisdom of a trainee, the wisdom of an adept, and the wisdom of one who is neither a trainee nor an adept. 

Another\marginnote{1.10.106} three kinds of wisdom: wisdom produced by thought, learning, and meditation. 

Three\marginnote{1.10.108} weapons: learning, seclusion, and wisdom. 

Three\marginnote{1.10.110} faculties: the faculty of understanding that one’s enlightenment is imminent, the faculty of enlightenment, and the faculty of one who is enlightened. 

Three\marginnote{1.10.112} eyes: the eye of the flesh, the eye of clairvoyance, and the eye of wisdom. 

Three\marginnote{1.10.114} trainings: in higher ethics, higher mind, and higher wisdom. 

Three\marginnote{1.10.116} kinds of development: the development of physical endurance, the development of the mind, and the development of wisdom. 

Three\marginnote{1.10.118} unsurpassable things: unsurpassable seeing, practice, and freedom. 

Three\marginnote{1.10.120} kinds of immersion. Immersion with placing the mind and keeping it connected. Immersion without placing the mind, but just keeping it connected. Immersion without placing the mind or keeping it connected. 

Another\marginnote{1.10.122} three kinds of immersion: emptiness, signless, and undirected. 

Three\marginnote{1.10.124} purities: purity of body, speech, and mind. 

Three\marginnote{1.10.126} kinds of sagacity: sagacity of body, speech, and mind. 

Three\marginnote{1.10.128} skills: skill in progress, skill in regress, and skill in means. 

Three\marginnote{1.10.130} vanities: the vanity of health, the vanity of youth, and the vanity of life. 

Three\marginnote{1.10.132} ways of putting something in charge: putting oneself, the world, or the teaching in charge. 

Three\marginnote{1.10.134} topics of discussion. You might discuss the past: ‘That is how it was in the past.’ You might discuss the future: ‘That is how it will be in the future.’ Or you might discuss the present: ‘This is how it is at present.’ 

Three\marginnote{1.10.141} knowledges: recollection of past lives, knowledge of the death and rebirth of sentient beings, and knowledge of the ending of defilements. 

Three\marginnote{1.10.143} meditative abidings: the meditation of the gods, the meditation of \textsanskrit{Brahmā}, and the meditation of the noble ones. 

Three\marginnote{1.10.145} demonstrations: a demonstration of psychic power, a demonstration of revealing, and an instructional demonstration. 

These\marginnote{1.10.147} are the teachings grouped by three that have been rightly explained by the Buddha. You should all recite these in concert. 

\section*{4. Fours }

There\marginnote{1.11.1} are teachings grouped by four that have been rightly explained by the Buddha. You should all recite these in concert. What are the teachings grouped by four? 

Four\marginnote{1.11.4} kinds of mindfulness meditation. It’s when a mendicant meditates by observing an aspect of the body—keen, aware, and mindful, rid of desire and aversion for the world. They meditate observing an aspect of feelings … mind … principles—keen, aware, and mindful, rid of desire and aversion for the world. 

Four\marginnote{1.11.9} right efforts. A mendicant generates enthusiasm, tries, makes an effort, exerts the mind, and strives so that bad, unskillful qualities don’t arise. They generate enthusiasm, try, make an effort, exert the mind, and strive so that bad, unskillful qualities that have arisen are given up. They generate enthusiasm, try, make an effort, exert the mind, and strive so that skillful qualities arise. They generate enthusiasm, try, make an effort, exert the mind, and strive so that skillful qualities that have arisen remain, are not lost, but increase, mature, and are completed by development. 

Four\marginnote{1.11.14} bases of psychic power. A mendicant develops the basis of psychic power that has immersion due to enthusiasm, and active effort. They develop the basis of psychic power that has immersion due to mental development, and active effort. They develop the basis of psychic power that has immersion due to energy, and active effort. They develop the basis of psychic power that has immersion due to inquiry, and active effort. 

Four\marginnote{1.11.19} absorptions. A mendicant, quite secluded from sensual pleasures, secluded from unskillful qualities, enters and remains in the first absorption, which has the rapture and bliss born of seclusion, while placing the mind and keeping it connected. As the placing of the mind and keeping it connected are stilled, they enter and remain in the second absorption, which has the rapture and bliss born of immersion, with internal clarity and confidence, and unified mind, without placing the mind and keeping it connected. And with the fading away of rapture, they enter and remain in the third absorption, where they meditate with equanimity, mindful and aware, personally experiencing the bliss of which the noble ones declare, ‘Equanimous and mindful, one meditates in bliss.’ Giving up pleasure and pain, and ending former happiness and sadness, they enter and remain in the fourth absorption, without pleasure or pain, with pure equanimity and mindfulness. 

Four\marginnote{1.11.24} ways of developing immersion further. There is a way of developing immersion further that leads to blissful meditation in the present life. There is a way of developing immersion further that leads to gaining knowledge and vision. There is a way of developing immersion further that leads to mindfulness and awareness. There is a way of developing immersion further that leads to the ending of defilements. 

And\marginnote{1.11.29} what is the way of developing immersion further that leads to blissful meditation in the present life? It’s when a mendicant, quite secluded from sensual pleasures, secluded from unskillful qualities, enters and remains in the first absorption … second absorption … fourth absorption. This is the way of developing immersion further that leads to blissful meditation in the present life. 

And\marginnote{1.11.34} what is the way of developing immersion further that leads to gaining knowledge and vision? A mendicant focuses on the perception of light, concentrating on the perception of day regardless of whether it’s night or day. And so, with an open and unenveloped heart, they develop a mind that’s full of radiance. This is the way of developing immersion further that leads to gaining knowledge and vision. 

And\marginnote{1.11.38} what is the way of developing immersion further that leads to mindfulness and awareness? A mendicant knows feelings as they arise, as they remain, and as they go away. They know perceptions as they arise, as they remain, and as they go away. They know thoughts as they arise, as they remain, and as they go away. This is the way of developing immersion further that leads to mindfulness and awareness. 

And\marginnote{1.11.43} what is the way of developing immersion further that leads to the ending of defilements? A mendicant meditates observing rise and fall in the five grasping aggregates. ‘Such is form, such is the origin of form, such is the ending of form. Such are feelings … perceptions … choices … consciousness, such is the origin of consciousness, such is the ending of consciousness.’ This is the way of developing immersion further that leads to the ending of defilements. 

Four\marginnote{1.11.51} immeasurables. A mendicant meditates spreading a heart full of love to one direction, and to the second, and to the third, and to the fourth. In the same way above, below, across, everywhere, all around, they spread a heart full of love to the whole world—abundant, expansive, limitless, free of enmity and ill will. They meditate spreading a heart full of compassion … rejoicing … equanimity to one direction, and to the second, and to the third, and to the fourth. In the same way above, below, across, everywhere, all around, they spread a heart full of equanimity to the whole world—abundant, expansive, limitless, free of enmity and ill will. 

Four\marginnote{1.11.56} formless states. A mendicant, going totally beyond perceptions of form, with the ending of perceptions of impingement, not focusing on perceptions of diversity, aware that ‘space is infinite’, enters and remains in the dimension of infinite space. Going totally beyond the dimension of infinite space, aware that ‘consciousness is infinite’, they enter and remain in the dimension of infinite consciousness. Going totally beyond the dimension of infinite consciousness, aware that ‘there is nothing at all’, they enter and remain in the dimension of nothingness. Going totally beyond the dimension of nothingness, they enter and remain in the dimension of neither perception nor non-perception. 

Four\marginnote{1.11.61} supports. After appraisal, a mendicant uses some things, endures some things, avoids some things, and gets rid of some things. 

Four\marginnote{1.11.63} noble traditions. A mendicant is content with any kind of robe, and praises such contentment. They don’t try to get hold of a robe in an improper way. They don’t get upset if they don’t get a robe. And if they do get a robe, they use it untied, uninfatuated, unattached, seeing the drawback, and understanding the escape. And on account of that they don’t glorify themselves or put others down. A mendicant who is deft, tireless, aware, and mindful in this is said to stand in the ancient, primordial noble tradition. 

Furthermore,\marginnote{1.11.66} a mendicant is content with any kind of almsfood … 

Furthermore,\marginnote{1.11.68} a mendicant is content with any kind of lodgings … 

Furthermore,\marginnote{1.11.70} a mendicant enjoys giving up and loves to give up. They enjoy meditation and love to meditate. But they don’t glorify themselves or put down others on account of their love for giving up and meditation. A mendicant who is deft, tireless, aware, and mindful in this is said to stand in the ancient, primordial noble tradition. 

Four\marginnote{1.11.72} efforts. The efforts to restrain, to give up, to develop, and to preserve. And what is the effort to restrain? When a mendicant sees a sight with their eyes, they don’t get caught up in the features and details. If the faculty of sight were left unrestrained, bad unskillful qualities of desire and aversion would become overwhelming. For this reason, they practice restraint, protecting the faculty of sight, and achieving its restraint. When they hear a sound with their ears … When they smell an odor with their nose … When they taste a flavor with their tongue … When they feel a touch with their body … When they know a thought with their mind, they don’t get caught up in the features and details. If the faculty of mind were left unrestrained, bad unskillful qualities of desire and aversion would become overwhelming. For this reason, they practice restraint, protecting the faculty of mind, and achieving its restraint. This is called the effort to restrain. 

And\marginnote{1.11.84} what is the effort to give up? It’s when a mendicant doesn’t tolerate a sensual, malicious, or cruel thought that’s arisen, but gives it up, gets rid of it, eliminates it, and obliterates it. They don’t tolerate any bad, unskillful qualities that have arisen, but give them up, get rid of them, eliminate them, and obliterate them. This is called the effort to give up. 

And\marginnote{1.11.90} what is the effort to develop? It’s when a mendicant develops the awakening factors of mindfulness, investigation of principles, energy, rapture, tranquility, immersion, and equanimity, which rely on seclusion, fading away, and cessation, and ripen as letting go. This is called the effort to develop. 

And\marginnote{1.11.99} what is the effort to preserve? It’s when a mendicant preserves a meditation subject that’s a fine foundation of immersion: the perception of a skeleton, a worm-infested corpse, a livid corpse, a split open corpse, or a bloated corpse. This is called the effort to preserve. 

Four\marginnote{1.11.102} knowledges: knowledge of the present phenomena, inferential knowledge, knowledge of others’ minds, and conventional knowledge. 

Another\marginnote{1.11.104} four knowledges: knowing about suffering, the origin of suffering, the cessation of suffering, and the practice that leads to the cessation of suffering. 

Four\marginnote{1.11.106} factors of stream-entry: associating with good people, listening to the true teaching, proper attention, and practicing in line with the teaching. 

Four\marginnote{1.11.108} factors of a stream-enterer. A noble disciple has experiential confidence in the Buddha: ‘That Blessed One is perfected, a fully awakened Buddha, accomplished in knowledge and conduct, holy, knower of the world, supreme guide for those who wish to train, teacher of gods and humans, awakened, blessed.’ They have experiential confidence in the teaching: ‘The teaching is well explained by the Buddha—visible in this very life, immediately effective, inviting inspection, relevant, so that sensible people can know it for themselves.’ They have experiential confidence in the \textsanskrit{Saṅgha}: ‘The \textsanskrit{Saṅgha} of the Buddha’s disciples is practicing the way that’s good, direct, methodical, and proper. It consists of the four pairs, the eight individuals. This is the \textsanskrit{Saṅgha} of the Buddha’s disciples that is worthy of offerings dedicated to the gods, worthy of hospitality, worthy of a religious donation, worthy of greeting with joined palms, and is the supreme field of merit for the world.’ And a noble disciple’s ethical conduct is loved by the noble ones, unbroken, impeccable, spotless, and unmarred, liberating, praised by sensible people, not mistaken, and leading to immersion. 

Four\marginnote{1.11.116} fruits of the ascetic life: stream-entry, once-return, non-return, and perfection. 

Four\marginnote{1.11.118} elements: earth, water, fire, and air. 

Four\marginnote{1.11.120} foods: solid food, whether coarse or fine; contact is the second, mental intention the third, and consciousness the fourth. 

Four\marginnote{1.11.122} bases for consciousness to remain. As long as consciousness remains, it remains involved with form, supported by form, founded on form. And with a sprinkle of relishing, it grows, increases, and matures. Or consciousness remains involved with feeling … Or consciousness remains involved with perception … Or as long as consciousness remains, it remains involved with choices, supported by choices, grounded on choices. And with a sprinkle of relishing, it grows, increases, and matures. 

Four\marginnote{1.11.127} prejudices: making decisions prejudiced by favoritism, hostility, stupidity, and cowardice. 

Four\marginnote{1.11.129} things that give rise to craving. Craving arises in a mendicant for the sake of robes, almsfood, lodgings, or rebirth in this or that state. 

Four\marginnote{1.11.134} ways of practice: painful practice with slow insight, painful practice with swift insight, pleasant practice with slow insight, and pleasant practice with swift insight. 

Another\marginnote{1.11.136} four ways of practice: impatient practice, patient practice, taming practice, and calming practice. 

Four\marginnote{1.11.138} footprints of the Dhamma: contentment, good will, right mindfulness, and right immersion. 

Four\marginnote{1.11.140} ways of taking up practices. There is a way of taking up practices that is painful now and results in future pain. There is a way of taking up practices that is painful now but results in future pleasure. There is a way of taking up practices that is pleasant now but results in future pain. There is a way of taking up practices that is pleasant now and results in future pleasure. 

Four\marginnote{1.11.145} spectrums of the teaching: ethics, immersion, wisdom, and freedom. 

Four\marginnote{1.11.147} powers: energy, mindfulness, immersion, and wisdom. 

Four\marginnote{1.11.149} foundations: the foundations of wisdom, truth, generosity, and peace. 

Four\marginnote{1.11.151} ways of answering questions. There is a question that should be answered definitively. There is a question that should be answered analytically. There is a question that should be answered with a counter-question. There is a question that should be set aside. 

Four\marginnote{1.11.153} deeds. There are deeds that are dark with dark result. There are deeds that are bright with bright result. There are deeds that are dark and bright with dark and bright result. There are neither dark nor bright deeds with neither dark nor bright results, which lead to the ending of deeds. 

Four\marginnote{1.11.158} things to be realized. Past lives are to be realized with recollection. The passing away and rebirth of sentient beings is to be realized with vision. The eight liberations are to be realized with direct meditative experience. The ending of defilements is to be realized with wisdom. 

Four\marginnote{1.11.163} floods: sensuality, desire for rebirth, views, and ignorance. 

Four\marginnote{1.11.165} bonds: sensuality, desire for rebirth, views, and ignorance. 

Four\marginnote{1.11.167} detachments: detachment from the bonds of sensuality, desire for rebirth, views, and ignorance. 

Four\marginnote{1.11.169} ties: the personal ties to covetousness, ill will, misapprehension of precepts and observances, and the insistence that this is the only truth. 

Four\marginnote{1.11.171} kinds of grasping: grasping at sensual pleasures, views, precepts and observances, and theories of a self. 

Four\marginnote{1.11.173} kinds of reproduction: reproduction for creatures born from an egg, from a womb, from moisture, or spontaneously. 

Four\marginnote{1.11.175} kinds of conception. Someone is unaware when conceived in their mother’s womb, unaware as they remain there, and unaware as they emerge. This is the first kind of conception. Furthermore, someone is aware when conceived in their mother’s womb, but unaware as they remain there, and unaware as they emerge. This is the second kind of conception. Furthermore, someone is aware when conceived in their mother’s womb, aware as they remain there, but unaware as they emerge. This is the third kind of conception. Furthermore, someone is aware when conceived in their mother’s womb, aware as they remain there, and aware as they emerge. This is the fourth kind of conception. 

Four\marginnote{1.11.180} kinds of reincarnation. There is a reincarnation where only one’s own intention is effective, not that of others. There is a reincarnation where only the intention of others is effective, not one’s own. There is a reincarnation where both one’s own and others’ intentions are effective. There is a reincarnation where neither one’s own nor others’ intentions are effective. 

Four\marginnote{1.11.185} ways of purifying a religious donation. There’s a religious donation that’s purified by the giver, not the recipient. There’s a religious donation that’s purified by the recipient, not the giver. There’s a religious donation that’s purified by neither the giver nor the recipient. There’s a religious donation that’s purified by both the giver and the recipient. 

Four\marginnote{1.11.190} ways of being inclusive: giving, kindly words, taking care, and equality. 

Four\marginnote{1.11.192} ignoble expressions: speech that’s false, divisive, harsh, or nonsensical. 

Four\marginnote{1.11.194} noble expressions: refraining from speech that’s false, divisive, harsh, or nonsensical. 

Another\marginnote{1.11.196} four ignoble expressions: saying you’ve seen, heard, thought, or known something, but you haven’t. 

Another\marginnote{1.11.198} four noble expressions: saying you haven’t seen, heard, thought, or known something, and you haven’t. 

Another\marginnote{1.11.200} four ignoble expressions: saying you haven’t seen, heard, thought, or known something, and you have. 

Another\marginnote{1.11.202} four noble expressions: saying you’ve seen, heard, thought, or known something, and you have. 

Four\marginnote{1.11.204} persons. One person mortifies themselves, committed to the practice of mortifying themselves. One person mortifies others, committed to the practice of mortifying others. One person mortifies themselves and others, committed to the practice of mortifying themselves and others. One person doesn’t mortify either themselves or others, committed to the practice of not mortifying themselves or others. They live without wishes in the present life, extinguished, cooled, experiencing bliss, having become holy in themselves. 

Another\marginnote{1.11.210} four persons. One person practices to benefit themselves, but not others. One person practices to benefit others, but not themselves. One person practices to benefit neither themselves nor others. One person practices to benefit both themselves and others. 

Another\marginnote{1.11.215} four persons: the dark bound for darkness, the dark bound for light, the light bound for darkness, and the light bound for light. 

Another\marginnote{1.11.217} four persons: the confirmed ascetic, the white lotus ascetic, the pink lotus ascetic, and the exquisite ascetic of ascetics. 

These\marginnote{1.11.219} are the teachings grouped by four that have been rightly explained by the Buddha. You should all recite these in concert. 

\scendsection{The first recitation section is finished. }

\section*{5. Fives }

There\marginnote{2.1.1} are teachings grouped by five that have been rightly explained by the Buddha. You should all recite these in concert. What are the teachings grouped by five? 

Five\marginnote{2.1.4} aggregates: form, feeling, perception, choices, and consciousness. 

Five\marginnote{2.1.6} grasping aggregates: form, feeling, perception, choices, and consciousness. 

Five\marginnote{2.1.8} kinds of sensual stimulation. Sights known by the eye that are likable, desirable, agreeable, pleasant, sensual, and arousing. Sounds known by the ear … Smells known by the nose … Tastes known by the tongue … Touches known by the body that are likable, desirable, agreeable, pleasant, sensual, and arousing. 

Five\marginnote{2.1.14} destinations: hell, the animal realm, the ghost realm, humanity, and the gods. 

Five\marginnote{2.1.16} kinds of stinginess: stinginess with dwellings, families, material possessions, praise, and the teachings. 

Five\marginnote{2.1.18} hindrances: sensual desire, ill will, dullness and drowsiness, restlessness and remorse, and doubt. 

Five\marginnote{2.1.20} lower fetters: identity view, doubt, misapprehension of precepts and observances, sensual desire, and ill will. 

Five\marginnote{2.1.22} higher fetters: desire for rebirth in the realm of luminous form, desire for rebirth in the formless realm, conceit, restlessness, and ignorance. 

Five\marginnote{2.1.24} precepts: refraining from killing living creatures, stealing, sexual misconduct, lying, and drinking alcohol, which is a basis for negligence. 

Five\marginnote{2.1.26} things that can’t be done. A mendicant with defilements ended can’t deliberately take the life of a living creature, take something with the intention to steal, have sex, tell a deliberate lie, or store up goods for their own enjoyment like they did as a lay person. 

Five\marginnote{2.1.28} losses: loss of relatives, wealth, health, ethics, and view. It is not because of loss of relatives, wealth, or health that sentient beings, when their body breaks up, after death, are reborn in a place of loss, a bad place, the underworld, hell. It is because of loss of ethics or view that sentient beings, when their body breaks up, after death, are reborn in a place of loss, a bad place, the underworld, hell. 

Five\marginnote{2.1.32} endowments: endowment with relatives, wealth, health, ethics, and view. It is not because of endowment with family, wealth, or health that sentient beings, when their body breaks up, after death, are reborn in a good place, a heavenly realm. It is because of endowment with ethics or view that sentient beings, when their body breaks up, after death, are reborn in a good place, a heavenly realm. 

Five\marginnote{2.1.36} drawbacks for an unethical person because of their failure in ethics. Firstly, an unethical person loses substantial wealth on account of negligence. This is the first drawback. Furthermore, an unethical person gets a bad reputation. This is the second drawback. Furthermore, an unethical person enters any kind of assembly timid and embarrassed, whether it’s an assembly of aristocrats, brahmins, householders, or ascetics. This is the third drawback. Furthermore, an unethical person feels lost when they die. This is the fourth drawback. Furthermore, an unethical person, when their body breaks up, after death, is reborn in a place of loss, a bad place, the underworld, hell. This is the fifth drawback. 

Five\marginnote{2.1.47} benefits for an ethical person because of their accomplishment in ethics. Firstly, an ethical person gains substantial wealth on account of diligence. This is the first benefit. Furthermore, an ethical person gets a good reputation. This is the second benefit. Furthermore, an ethical person enters any kind of assembly bold and self-assured, whether it’s an assembly of aristocrats, brahmins, householders, or ascetics. This is the third benefit. Furthermore, an ethical person dies not feeling lost. This is the fourth benefit. Furthermore, when an ethical person’s body breaks up, after death, they’re reborn in a good place, a heavenly realm. This is the fifth benefit. 

A\marginnote{2.1.58} mendicant who wants to accuse another should first establish five things in themselves. I will speak at the right time, not at the wrong time. I will speak truthfully, not falsely. I will speak gently, not harshly. I will speak beneficially, not harmfully. I will speak lovingly, not from secret hate. A mendicant who wants to accuse another should first establish these five things in themselves. 

Five\marginnote{2.1.65} factors that support meditation. A mendicant has faith in the Realized One’s awakening: ‘That Blessed One is perfected, a fully awakened Buddha, accomplished in knowledge and conduct, holy, knower of the world, supreme guide for those who wish to train, teacher of gods and humans, awakened, blessed.’ They are rarely ill or unwell. Their stomach digests well, being neither too hot nor too cold, but just right, and fit for meditation. They’re not devious or deceitful. They reveal themselves honestly to the Teacher or sensible spiritual companions. They live with energy roused up for giving up unskillful qualities and embracing skillful qualities. They’re strong, staunchly vigorous, not slacking off when it comes to developing skillful qualities. They’re wise. They have the wisdom of arising and passing away which is noble, penetrative, and leads to the complete ending of suffering. 

Five\marginnote{2.1.72} pure abodes: Aviha, Atappa, the Gods Fair to See, the Fair Seeing Gods, and \textsanskrit{Akaniṭṭha}. 

Five\marginnote{2.1.74} non-returners: one who is extinguished between one life and the next, one who is extinguished upon landing, one who is extinguished without extra effort, one who is extinguished with extra effort, and one who heads upstream, going to the \textsanskrit{Akaniṭṭha} realm. 

Five\marginnote{2.1.76} kinds of emotional barrenness. Firstly, a mendicant has doubts about the Teacher. They’re uncertain, undecided, and lacking confidence. This being so, their mind doesn’t incline toward keenness, commitment, persistence, and striving. This is the first kind of emotional barrenness. Furthermore, a mendicant has doubts about the teaching … the \textsanskrit{Saṅgha} … the training … A mendicant is angry and upset with their spiritual companions, resentful and closed off. This being so, their mind doesn’t incline toward keenness, commitment, persistence, and striving. This is the fifth kind of emotional barrenness. 

Five\marginnote{2.1.84} emotional shackles. Firstly, a mendicant isn’t free of greed, desire, fondness, thirst, passion, and craving for sensual pleasures. This being so, their mind doesn’t incline toward keenness, commitment, persistence, and striving. This is the first emotional shackle. Furthermore, a mendicant isn’t free of greed for the body … They’re not free of greed for form … They eat as much as they like until their bellies are full, then indulge in the pleasures of sleeping, lying down, and drowsing … They lead the spiritual life hoping to be reborn in one of the orders of gods, thinking: ‘By this precept or observance or mortification or spiritual life, may I become one of the gods!’ This being so, their mind doesn’t incline toward keenness, commitment, persistence, and striving. This is the fifth emotional shackle. 

Five\marginnote{2.1.95} faculties: eye, ear, nose, tongue, and body. 

Another\marginnote{2.1.97} five faculties: pleasure, pain, happiness, sadness, and equanimity. 

Another\marginnote{2.1.99} five faculties: faith, energy, mindfulness, immersion, and wisdom. 

Five\marginnote{2.1.101} elements of escape. Take a case where a mendicant focuses on sensual pleasures, but their mind isn’t eager, confident, settled, and decided about them. But when they focus on renunciation, their mind is eager, confident, settled, and decided about it. Their mind is in a good state, well developed, well risen, well freed, and well detached from sensual pleasures. They’re freed from the distressing and feverish defilements that arise because of sensual pleasures, so they don’t experience that kind of feeling. This is how the escape from sensual pleasures is explained. 

Take\marginnote{2.1.107} another case where a mendicant focuses on ill will, but their mind isn’t eager … But when they focus on good will, their mind is eager … Their mind is in a good state … well detached from ill will. They’re freed from the distressing and feverish defilements that arise because of ill will, so they don’t experience that kind of feeling. This is how the escape from ill will is explained. 

Take\marginnote{2.1.112} another case where a mendicant focuses on harming, but their mind isn’t eager … But when they focus on compassion, their mind is eager … Their mind is in a good state … well detached from harming. They’re freed from the distressing and feverish defilements that arise because of harming, so they don’t experience that kind of feeling. This is how the escape from harming is explained. 

Take\marginnote{2.1.117} another case where a mendicant focuses on form, but their mind isn’t eager … But when they focus on the formless, their mind is eager … Their mind is in a good state … well detached from forms. They’re freed from the distressing and feverish defilements that arise because of form, so they don’t experience that kind of feeling. This is how the escape from forms is explained. 

Take\marginnote{2.1.122} a case where a mendicant focuses on identity, but their mind isn’t eager, confident, settled, and decided about it. But when they focus on the ending of identity, their mind is eager, confident, settled, and decided about it. Their mind is in a good state, well developed, well risen, well freed, and well detached from identity. They’re freed from the distressing and feverish defilements that arise because of identity, so they don’t experience that kind of feeling. This is how the escape from identity is explained. 

Five\marginnote{2.1.127} opportunities for freedom. Firstly, the Teacher or a respected spiritual companion teaches Dhamma to a mendicant. That mendicant feels inspired by the meaning and the teaching in that Dhamma, no matter how the Teacher or a respected spiritual companion teaches it. Feeling inspired, joy springs up. Being joyful, rapture springs up. When the mind is full of rapture, the body becomes tranquil. When the body is tranquil, one feels bliss. And when blissful, the mind becomes immersed. This is the first opportunity for freedom. 

Furthermore,\marginnote{2.1.132} it may be that neither the Teacher nor a respected spiritual companion teaches Dhamma to a mendicant. But the mendicant teaches Dhamma in detail to others as they learned and memorized it. … Or the mendicant recites the teaching in detail as they learned and memorized it. … Or the mendicant thinks about and considers the teaching in their heart, examining it with the mind as they learned and memorized it. … Or a meditation subject as a foundation of immersion is properly grasped, attended, borne in mind, and comprehended with wisdom. That mendicant feels inspired by the meaning and the teaching in that Dhamma, no matter how a meditation subject as a foundation of immersion is properly grasped, attended, borne in mind, and comprehended with wisdom. Feeling inspired, joy springs up. Being joyful, rapture springs up. When the mind is full of rapture, the body becomes tranquil. When the body is tranquil, one feels bliss. And when blissful, the mind becomes immersed. This is the fifth opportunity for freedom. 

Five\marginnote{2.1.139} perceptions that ripen in freedom: the perception of impermanence, the perception of suffering in impermanence, the perception of not-self in suffering, the perception of giving up, and the perception of fading away. 

These\marginnote{2.1.141} are the teachings grouped by five that have been rightly explained by the Buddha. You should all recite these in concert. 

\section*{6. Sixes }

There\marginnote{2.2.1} are teachings grouped by six that have been rightly explained by the Buddha. You should all recite these in concert. What are the teachings grouped by six? 

Six\marginnote{2.2.4} interior sense fields: eye, ear, nose, tongue, body, and mind. 

Six\marginnote{2.2.6} exterior sense fields: sights, sounds, smells, tastes, touches, and thoughts. 

Six\marginnote{2.2.8} classes of consciousness: eye, ear, nose, tongue, body, and mind consciousness. 

Six\marginnote{2.2.10} classes of contact: contact through the eye, ear, nose, tongue, body, and mind. 

Six\marginnote{2.2.12} classes of feeling: feeling born of contact through the eye, ear, nose, tongue, body, and mind. 

Six\marginnote{2.2.14} classes of perception: perceptions of sights, sounds, smells, tastes, touches, and thoughts. 

Six\marginnote{2.2.16} bodies of intention: intention regarding sights, sounds, smells, tastes, touches, and thoughts. 

Six\marginnote{2.2.18} classes of craving: craving for sights, sounds, smells, tastes, touches, and thoughts. 

Six\marginnote{2.2.20} kinds of disrespect. A mendicant lacks respect and reverence for the Teacher, the teaching, and the \textsanskrit{Saṅgha}, the training, diligence, and hospitality. 

Six\marginnote{2.2.22} kinds of respect. A mendicant has respect and reverence for the Teacher, the teaching, and the \textsanskrit{Saṅgha}, the training, diligence, and hospitality. 

Six\marginnote{2.2.24} preoccupations with happiness. Seeing a sight with the eye, one is preoccupied with a sight that’s a basis for happiness. Hearing a sound with the ear … Smelling an odor with the nose … Tasting a flavor with the tongue … 

Feeling\marginnote{2.2.29} a touch with the body … Knowing a thought with the mind, one is preoccupied with a thought that’s a basis for happiness. 

Six\marginnote{2.2.31} preoccupations with sadness. Seeing a sight with the eye, one is preoccupied with a sight that’s a basis for sadness. … Knowing a thought with the mind, one is preoccupied with a thought that’s a basis for sadness. 

Six\marginnote{2.2.34} preoccupations with equanimity. Seeing a sight with the eye, one is preoccupied with a sight that’s a basis for equanimity. … Knowing a thought with the mind, one is preoccupied with a thought that’s a basis for equanimity. 

Six\marginnote{2.2.37} warm-hearted qualities. Firstly, a mendicant consistently treats their spiritual companions with bodily kindness, both in public and in private. This warm-hearted quality makes for fondness and respect, conducing to inclusion, harmony, and unity, without quarreling. 

Furthermore,\marginnote{2.2.40} a mendicant consistently treats their spiritual companions with verbal kindness, both in public and in private. This too is a warm-hearted quality. 

Furthermore,\marginnote{2.2.42} a mendicant consistently treats their spiritual companions with mental kindness, both in public and in private. This too is a warm-hearted quality. 

Furthermore,\marginnote{2.2.44} a mendicant shares without reservation any material possessions they have gained by legitimate means, even the food placed in the alms-bowl, using them in common with their ethical spiritual companions. This too is a warm-hearted quality. 

Furthermore,\marginnote{2.2.46} a mendicant lives according to the precepts shared with their spiritual companions, both in public and in private. Those precepts are unbroken, impeccable, spotless, and unmarred, liberating, praised by sensible people, not mistaken, and leading to immersion. This too is a warm-hearted quality. 

They\marginnote{2.2.48} live according to the view shared with their spiritual companions, both in public and in private. That view is noble and emancipating, and brings one who practices it to the complete ending of suffering. This warm-hearted quality too makes for fondness and respect, conducing to inclusion, harmony, and unity, without quarreling. 

Six\marginnote{2.2.50} roots of arguments. Firstly, a mendicant is irritable and hostile. Such a mendicant lacks respect and reverence for the Teacher, the teaching, and the \textsanskrit{Saṅgha}, and they don’t fulfill the training. They create a dispute in the \textsanskrit{Saṅgha}, which is for the hurt and unhappiness of the people, for the harm, hurt, and suffering of gods and humans. If you see such a root of arguments in yourselves or others, you should try to give up this bad thing. If you don’t see it, you should practice so that it doesn’t come up in the future. That’s how to give up this bad root of arguments, so it doesn’t come up in the future. 

Furthermore,\marginnote{2.2.57} a mendicant is offensive and contemptuous … They’re jealous and stingy … They’re devious and deceitful … They have wicked desires and wrong view … They’re attached to their own views, holding them tight, and refusing to let go. If you see such a root of arguments in yourselves or others, you should try to give up this bad thing. If you don’t see it, you should practice so that it doesn’t come up in the future. That’s how to give up this bad root of arguments, so it doesn’t come up in the future. 

Six\marginnote{2.2.67} elements: earth, water, fire, air, space, and consciousness. 

Six\marginnote{2.2.69} elements of escape. Take a mendicant who says: ‘I’ve developed the heart’s release by love. I’ve cultivated it, made it my vehicle and my basis, kept it up, consolidated it, and properly implemented it. Yet somehow ill will still occupies my mind.’ They should be told, ‘Not so, venerable! Don’t say that. Don’t misrepresent the Buddha, for misrepresentation of the Buddha is not good. And the Buddha would not say that. It’s impossible, reverend, it cannot happen that the heart’s release by love has been developed and properly implemented, yet somehow ill will still occupies the mind. For it is the heart’s release by love that is the escape from ill will.’ 

Take\marginnote{2.2.77} another mendicant who says: ‘I’ve developed the heart’s release by compassion. I’ve cultivated it, made it my vehicle and my basis, kept it up, consolidated it, and properly implemented it. Yet somehow the thought of harming still occupies my mind.’ They should be told, ‘Not so, venerable! … For it is the heart’s release by compassion that is the escape from thoughts of harming.’ 

Take\marginnote{2.2.84} another mendicant who says: ‘I’ve developed the heart’s release by rejoicing. I’ve cultivated it, made it my vehicle and my basis, kept it up, consolidated it, and properly implemented it. Yet somehow discontent still occupies my mind.’ They should be told, ‘Not so, venerable! … For it is the heart’s release by rejoicing that is the escape from discontent.’ 

Take\marginnote{2.2.91} another mendicant who says: ‘I’ve developed the heart’s release by equanimity. I’ve cultivated it, made it my vehicle and my basis, kept it up, consolidated it, and properly implemented it. Yet somehow desire still occupies my mind.’ They should be told, ‘Not so, venerable! … For it is the heart’s release by equanimity that is the escape from desire.’ 

Take\marginnote{2.2.98} another mendicant who says: ‘I’ve developed the signless release of the heart. I’ve cultivated it, made it my vehicle and my basis, kept it up, consolidated it, and properly implemented it. Yet somehow my consciousness still follows after signs.’ They should be told, ‘Not so, venerable! … For it is the signless release of the heart that is the escape from all signs.’ 

Take\marginnote{2.2.105} another mendicant who says: ‘I’m rid of the conceit “I am”. And I don’t regard anything as “I am this”. Yet somehow the dart of doubt and indecision still occupies my mind.’ They should be told, ‘Not so, venerable! Don’t say that. Don’t misrepresent the Buddha, for misrepresentation of the Buddha is not good. And the Buddha would not say that. It’s impossible, reverend, it cannot happen that the conceit “I am” has been done away with, and nothing is regarded as “I am this”, yet somehow the dart of doubt and indecision still occupy the mind. For it is the uprooting of the conceit “I am” that is the escape from the dart of doubt and indecision.’ 

Six\marginnote{2.2.112} unsurpassable things: the unsurpassable seeing, listening, acquisition, training, service, and recollection. 

Six\marginnote{2.2.114} topics for recollection: the recollection of the Buddha, the teaching, the \textsanskrit{Saṅgha}, ethics, generosity, and the deities. 

Six\marginnote{2.2.116} consistent responses. A mendicant, seeing a sight with their eyes, is neither happy nor sad. They remain equanimous, mindful and aware. Hearing a sound with their ears … Smelling an odor with their nose … Tasting a flavor with their tongue … Feeling a touch with their body … Knowing a thought with their mind, they’re neither happy nor sad. They remain equanimous, mindful and aware. 

Six\marginnote{2.2.120} classes of rebirth. Someone born into a dark class gives rise to a dark result. Someone born into a dark class gives rise to a bright result. Someone born into a dark class gives rise to extinguishment, which is neither dark nor bright. Someone born into a bright class gives rise to a bright result. Someone born into a bright class gives rise to a dark result. Someone born into a bright class gives rise to extinguishment, which is neither dark nor bright. 

Six\marginnote{2.2.127} perceptions that help penetration: the perception of impermanence, the perception of suffering in impermanence, the perception of not-self in suffering, the perception of giving up, the perception of fading away, and the perception of cessation. 

These\marginnote{2.2.129} are the teachings grouped by six that have been rightly explained by the Buddha. You should all recite these in concert. 

\section*{7. Sevens }

There\marginnote{2.3.1} are teachings grouped by seven that have been rightly explained by the Buddha. You should all recite these in concert. What are the teachings grouped by seven? 

Seven\marginnote{2.3.4} kinds of noble wealth: the wealth of faith, ethical conduct, conscience, prudence, learning, generosity, and wisdom. 

Seven\marginnote{2.3.6} awakening factors: mindfulness, investigation of principles, energy, rapture, tranquility, immersion, and equanimity. 

Seven\marginnote{2.3.8} prerequisites for immersion: right view, right thought, right speech, right action, right livelihood, right effort, and right mindfulness. 

Seven\marginnote{2.3.10} bad qualities: a mendicant is faithless, shameless, imprudent, uneducated, lazy, unmindful, and witless. 

Seven\marginnote{2.3.12} good qualities: a mendicant is faithful, conscientious, prudent, learned, energetic, mindful, and wise. 

Seven\marginnote{2.3.14} aspects of the teachings of the good persons: a mendicant knows the teachings, knows the meaning, knows themselves, knows moderation, knows the right time, knows assemblies, and knows people. 

Seven\marginnote{2.3.16} qualifications for graduation. A mendicant has a keen enthusiasm to undertake the training … to examine the teachings … to get rid of desires … for retreat … to rouse up energy … for mindfulness and alertness … to penetrate theoretically. And they don’t lose these desires in the future. 

Seven\marginnote{2.3.24} perceptions: the perception of impermanence, the perception of not-self, the perception of ugliness, the perception of drawbacks, the perception of giving up, the perception of fading away, and the perception of cessation. 

Seven\marginnote{2.3.26} powers: faith, energy, conscience, prudence, mindfulness, immersion, and wisdom. 

Seven\marginnote{2.3.28} planes of consciousness. There are sentient beings that are diverse in body and diverse in perception, such as human beings, some gods, and some beings in the underworld. This is the first plane of consciousness. 

There\marginnote{2.3.31} are sentient beings that are diverse in body and unified in perception, such as the gods reborn in \textsanskrit{Brahmā}’s Host through the first absorption. This is the second plane of consciousness. 

There\marginnote{2.3.33} are sentient beings that are unified in body and diverse in perception, such as the gods of streaming radiance. This is the third plane of consciousness. 

There\marginnote{2.3.35} are sentient beings that are unified in body and unified in perception, such as the gods replete with glory. This is the fourth plane of consciousness. 

There\marginnote{2.3.37} are sentient beings that have gone totally beyond perceptions of form. With the ending of perceptions of impingement, not focusing on perceptions of diversity, aware that ‘space is infinite’, they have been reborn in the dimension of infinite space. This is the fifth plane of consciousness. 

There\marginnote{2.3.39} are sentient beings that have gone totally beyond the dimension of infinite space. Aware that ‘consciousness is infinite’, they have been reborn in the dimension of infinite consciousness. This is the sixth plane of consciousness. 

There\marginnote{2.3.41} are sentient beings that have gone totally beyond the dimension of infinite consciousness. Aware that ‘there is nothing at all’, they have been reborn in the dimension of nothingness. This is the seventh plane of consciousness. 

Seven\marginnote{2.3.43} persons worthy of a religious donation: one freed both ways, one freed by wisdom, a personal witness, one attained to view, one freed by faith, a follower of the teachings, and a follower by faith. 

Seven\marginnote{2.3.45} underlying tendencies: sensual desire, repulsion, views, doubt, conceit, desire to be reborn, and ignorance. 

Seven\marginnote{2.3.47} fetters: attraction, repulsion, views, doubt, conceit, desire to be reborn, and ignorance. 

Seven\marginnote{2.3.49} principles for the settlement of any disciplinary issues that might arise. Removal in the presence of those concerned is applicable. Removal by accurate recollection is applicable. Removal due to recovery from madness is applicable. The acknowledgement of the offense is applicable. The decision of a majority is applicable. A verdict of aggravated misconduct is applicable. Covering over with grass is applicable. 

These\marginnote{2.3.51} are the teachings grouped by seven that have been rightly explained by the Buddha. You should all recite these in concert. 

\scendsection{The second recitation section is finished. }

\section*{8. Eights }

There\marginnote{3.1.1} are teachings grouped by eight that have been rightly explained by the Buddha. You should all recite these in concert. What are the teachings grouped by eight? 

Eight\marginnote{3.1.4} wrong ways: wrong view, wrong thought, wrong speech, wrong action, wrong livelihood, wrong effort, wrong mindfulness, and wrong immersion. 

Eight\marginnote{3.1.6} right ways: right view, right thought, right speech, right action, right livelihood, right effort, right mindfulness, and right immersion. 

Eight\marginnote{3.1.8} persons worthy of a religious donation. The stream-enterer and the one practicing to realize the fruit of stream-entry. The once-returner and the one practicing to realize the fruit of once-return. The non-returner and the one practicing to realize the fruit of non-return. The perfected one, and the one practicing for perfection. 

Eight\marginnote{3.1.10} grounds for laziness. Firstly, a mendicant has some work to do. They think: ‘I have some work to do. But while doing it my body will get tired. I’d better have a lie down.’ They lie down, and don’t rouse energy for attaining the unattained, achieving the unachieved, and realizing the unrealized. This is the first ground for laziness. 

Furthermore,\marginnote{3.1.16} a mendicant has done some work. They think: ‘I’ve done some work. But while working my body got tired. I’d better have a lie down.’ They lie down, and don’t rouse energy… This is the second ground for laziness. 

Furthermore,\marginnote{3.1.21} a mendicant has to go on a journey. They think: ‘I have to go on a journey. But while walking my body will get tired. I’d better have a lie down.’ They lie down, and don’t rouse energy… This is the third ground for laziness. 

Furthermore,\marginnote{3.1.26} a mendicant has gone on a journey. They think: ‘I’ve gone on a journey. But while walking my body got tired. I’d better have a lie down.’ They lie down, and don’t rouse energy… This is the fourth ground for laziness. 

Furthermore,\marginnote{3.1.31} a mendicant has wandered for alms, but they didn’t get to fill up on as much food as they like, rough or fine. They think: ‘I’ve wandered for alms, but I didn’t get to fill up on as much food as I like, rough or fine. My body is tired and unfit for work. I’d better have a lie down.’ They lie down, and don’t rouse energy… This is the fifth ground for laziness. 

Furthermore,\marginnote{3.1.36} a mendicant has wandered for alms, and they got to fill up on as much food as they like, rough or fine. They think: ‘I’ve wandered for alms, and I got to fill up on as much food as I like, rough or fine. My body is heavy and unfit for work, like I’ve just eaten a load of beans. I’d better have a lie down.’ They lie down, and don’t rouse energy… This is the sixth ground for laziness. 

Furthermore,\marginnote{3.1.41} a mendicant feels a little sick. They think: ‘I feel a little sick. Lying down would be good for me. I’d better have a lie down.’ They lie down, and don’t rouse energy… This is the seventh ground for laziness. 

Furthermore,\marginnote{3.1.47} a mendicant has recently recovered from illness. They think: ‘I’ve recently recovered from illness. My body is weak and unfit for work. I’d better have a lie down.’ They lie down, and don’t rouse energy for attaining the unattained, achieving the unachieved, and realizing the unrealized. This is the eighth ground for laziness. 

Eight\marginnote{3.1.52} grounds for arousing energy. Firstly, a mendicant has some work to do. They think: ‘I have some work to do. While working it’s not easy to focus on the instructions of the Buddhas. I’d better preemptively rouse up energy for attaining the unattained, achieving the unachieved, and realizing the unrealized.’ They rouse energy for attaining the unattained, achieving the unachieved, and realizing the unrealized. This is the first ground for arousing energy. 

Furthermore,\marginnote{3.1.58} a mendicant has done some work. They think: ‘I’ve done some work. While I was working I wasn’t able to focus on the instructions of the Buddhas. I’d better preemptively rouse up energy.’ They rouse up energy… This is the second ground for arousing energy. 

Furthermore,\marginnote{3.1.63} a mendicant has to go on a journey. They think: ‘I have to go on a journey. While walking it’s not easy to focus on the instructions of the Buddhas. I’d better preemptively rouse up energy.’ They rouse up energy… This is the third ground for arousing energy. 

Furthermore,\marginnote{3.1.69} a mendicant has gone on a journey. They think: ‘I’ve gone on a journey. While I was walking I wasn’t able to focus on the instructions of the Buddhas. I’d better preemptively rouse up energy.’ They rouse up energy… This is the fourth ground for arousing energy. 

Furthermore,\marginnote{3.1.74} a mendicant has wandered for alms, but they didn’t get to fill up on as much food as they like, rough or fine. They think: ‘I’ve wandered for alms, but I didn’t get to fill up on as much food as I like, rough or fine. My body is light and fit for work. I’d better preemptively rouse up energy.’ They rouse up energy… This is the fifth ground for arousing energy. 

Furthermore,\marginnote{3.1.79} a mendicant has wandered for alms, and they got to fill up on as much food as they like, rough or fine. They think: ‘I’ve wandered for alms, and I got to fill up on as much food as I like, rough or fine. My body is strong and fit for work. I’d better preemptively rouse up energy.’ They rouse up energy… This is the sixth ground for arousing energy. 

Furthermore,\marginnote{3.1.84} a mendicant feels a little sick. They think: ‘I feel a little sick. It’s possible this illness will worsen. I’d better preemptively rouse up energy.’ They rouse up energy… This is the seventh ground for arousing energy. 

Furthermore,\marginnote{3.1.89} a mendicant has recently recovered from illness. They think: ‘I’ve recently recovered from illness. It’s possible the illness will come back. I’d better preemptively rouse up energy for attaining the unattained, achieving the unachieved, and realizing the unrealized.’ They rouse energy for attaining the unattained, achieving the unachieved, and realizing the unrealized. This is the eighth ground for arousing energy. 

Eight\marginnote{3.1.94} reasons to give. A person might give a gift after insulting the recipient. Or they give out of fear. Or they give thinking, ‘They gave to me.’ Or they give thinking, ‘They’ll give to me.’ Or they give thinking, ‘It’s good to give.’ Or they give thinking, ‘I cook, they don’t. It wouldn’t be right for me to not give to them.’ Or they give thinking, ‘By giving this gift I’ll get a good reputation.’ Or they give thinking, ‘This is an adornment and requisite for the mind.’ 

Eight\marginnote{3.1.97} rebirths by giving. First, someone gives to ascetics or brahmins such things as food, drink, clothing, vehicles; garlands, fragrance, and makeup; and bed, house, and lighting. Whatever they give, they expect something back. They see an affluent aristocrat or brahmin or householder amusing themselves, supplied and provided with the five kinds of sensual stimulation. They think: ‘If only, when my body breaks up, after death, I would be reborn in the company of well-to-do aristocrats or brahmins or householders!’ They settle on that thought, concentrate on it and develop it. As they’ve settled for less and not developed further, their thought leads to rebirth there. But I say that this is only for those of ethical conduct, not for the unethical. The heart’s wish of an ethical person succeeds because of their purity. 

Next,\marginnote{3.1.106} someone gives to ascetics or brahmins. Whatever they give, they expect something back. And they’ve heard: ‘The Gods of the Four Great Kings are long-lived, beautiful, and very happy.’ They think: ‘If only, when my body breaks up, after death, I would be reborn in the company of the Gods of the Four Great Kings!’ They settle on that thought, concentrate on it and develop it. As they’ve settled for less and not developed further, their thought leads to rebirth there. But I say that this is only for those of ethical conduct, not for the unethical. The heart’s wish of an ethical person succeeds because of their purity. 

Next,\marginnote{3.1.115} someone gives to ascetics or brahmins. Whatever they give, they expect something back. And they’ve heard: ‘The Gods of the Thirty-Three … the Gods of Yama … the Joyful Gods … the Gods Who Love to Create … the Gods Who Control the Creations of Others are long-lived, beautiful, and very happy.’ They think: ‘If only, when my body breaks up, after death, I would be reborn in the company of the Gods Who Control the Creations of Others!’ They settle on that thought, concentrate on it and develop it. As they’ve settled for less and not developed further, their thought leads to rebirth there. But I say that this is only for those of ethical conduct, not for the unethical. The heart’s wish of an ethical person succeeds because of their purity. 

Next,\marginnote{3.1.128} someone gives to ascetics or brahmins such things as food, drink, clothing, vehicles; garlands, fragrance, and makeup; and bed, house, and lighting. Whatever they give, they expect something back. And they’ve heard: ‘The Gods of \textsanskrit{Brahmā}’s Host are long-lived, beautiful, and very happy.’ They think: ‘If only, when my body breaks up, after death, I would be reborn in the company of the Gods of \textsanskrit{Brahmā}’s Host!’ They settle on that thought, concentrate on it and develop it. As they’ve settled for less and not developed further, their thought leads to rebirth there. But I say that this is only for those of ethical conduct, not for the unethical. And for those free of desire, not those with desire. The heart’s wish of an ethical person succeeds because of their freedom from desire. 

Eight\marginnote{3.1.138} assemblies: the assemblies of aristocrats, brahmins, householders, and ascetics. An assembly of the gods under the Four Great Kings. An assembly of the gods under the Thirty-Three. An assembly of \textsanskrit{Māras}. An assembly of \textsanskrit{Brahmās}. 

Eight\marginnote{3.1.140} worldly conditions: gain and loss, fame and disgrace, blame and praise, pleasure and pain. 

Eight\marginnote{3.1.142} dimensions of mastery. Perceiving form internally, someone sees visions externally, limited, both pretty and ugly. Mastering them, they perceive: ‘I know and see.’ This is the first dimension of mastery. 

Perceiving\marginnote{3.1.145} form internally, someone sees visions externally, limitless, both pretty and ugly. Mastering them, they perceive: ‘I know and see.’ This is the second dimension of mastery. 

Not\marginnote{3.1.148} perceiving form internally, someone sees visions externally, limited, both pretty and ugly. Mastering them, they perceive: ‘I know and see.’ This is the third dimension of mastery. 

Not\marginnote{3.1.150} perceiving form internally, someone sees visions externally, limitless, both pretty and ugly. Mastering them, they perceive: ‘I know and see.’ This is the fourth dimension of mastery. 

Not\marginnote{3.1.152} perceiving form internally, someone sees visions externally that are blue, with blue color, blue hue, and blue tint. They’re like a flax flower that’s blue, with blue color, blue hue, and blue tint. Or a cloth from \textsanskrit{Bāraṇasī} that’s smoothed on both sides, blue, with blue color, blue hue, and blue tint. Mastering them, they perceive: ‘I know and see.’ This is the fifth dimension of mastery. 

Not\marginnote{3.1.156} perceiving form internally, someone sees visions externally that are yellow, with yellow color, yellow hue, and yellow tint. They’re like a champak flower that’s yellow, with yellow color, yellow hue, and yellow tint. Or a cloth from \textsanskrit{Bāraṇasī} that’s smoothed on both sides, yellow, with yellow color, yellow hue, and yellow tint. Mastering them, they perceive: ‘I know and see.’ This is the sixth dimension of mastery. 

Not\marginnote{3.1.160} perceiving form internally, someone sees visions externally that are red, with red color, red hue, and red tint. They’re like a scarlet mallow flower that’s red, with red color, red hue, and red tint. Or a cloth from \textsanskrit{Bāraṇasī} that’s smoothed on both sides, red, with red color, red hue, and red tint. Mastering them, they perceive: ‘I know and see.’ This is the seventh dimension of mastery. 

Not\marginnote{3.1.164} perceiving form internally, someone sees visions externally that are white, with white color, white hue, and white tint. They’re like the morning star that’s white, with white color, white hue, and white tint. Or a cloth from \textsanskrit{Bāraṇasī} that’s smoothed on both sides, white, with white color, white hue, and white tint. Mastering them, they perceive: ‘I know and see.’ This is the eighth dimension of mastery. 

Eight\marginnote{3.1.168} liberations. Having physical form, they see visions. This is the first liberation. 

Not\marginnote{3.1.171} perceiving physical form internally, they see visions externally. This is the second liberation. 

They’re\marginnote{3.1.173} focused only on beauty. This is the third liberation. 

Going\marginnote{3.1.175} totally beyond perceptions of form, with the ending of perceptions of impingement, not focusing on perceptions of diversity, aware that ‘space is infinite’, they enter and remain in the dimension of infinite space. This is the fourth liberation. 

Going\marginnote{3.1.177} totally beyond the dimension of infinite space, aware that ‘consciousness is infinite’, they enter and remain in the dimension of infinite consciousness. This is the fifth liberation. 

Going\marginnote{3.1.179} totally beyond the dimension of infinite consciousness, aware that ‘there is nothing at all’, they enter and remain in the dimension of nothingness. This is the sixth liberation. 

Going\marginnote{3.1.181} totally beyond the dimension of nothingness, they enter and remain in the dimension of neither perception nor non-perception. This is the seventh liberation. 

Going\marginnote{3.1.183} totally beyond the dimension of neither perception nor non-perception, they enter and remain in the cessation of perception and feeling. This is the eighth liberation. 

These\marginnote{3.1.185} are the teachings grouped by eight that have been rightly explained by the Buddha. You should all recite these in concert. 

\section*{9. Nines }

There\marginnote{3.2.1} are teachings grouped by nine that have been rightly explained by the Buddha. You should all recite these in concert. What are the teachings grouped by nine? 

Nine\marginnote{3.2.4} grounds for resentment. Thinking: ‘They did wrong to me,’ you harbor resentment. Thinking: ‘They are doing wrong to me’ … ‘They will do wrong to me’ … ‘They did wrong by someone I love’ … ‘They are doing wrong by someone I love’ … ‘They will do wrong by someone I love’ … ‘They helped someone I dislike’ … ‘They are helping someone I dislike’ … Thinking: ‘They will help someone I dislike,’ you harbor resentment. 

Nine\marginnote{3.2.14} methods to get rid of resentment. Thinking: ‘They did wrong to me, but what can I possibly do?’ you get rid of resentment. Thinking: ‘They are doing wrong to me …’ … ‘They will do wrong to me …’ … ‘They did wrong by someone I love …’ … ‘They are doing wrong by someone I love …’ … ‘They will do wrong by someone I love …’ … ‘They helped someone I dislike …’ … ‘They are helping someone I dislike …’ … Thinking: ‘They will help someone I dislike, but what can I possibly do?’ you get rid of resentment. 

Nine\marginnote{3.2.24} abodes of sentient beings. There are sentient beings that are diverse in body and diverse in perception, such as human beings, some gods, and some beings in the underworld. This is the first abode of sentient beings. 

There\marginnote{3.2.27} are sentient beings that are diverse in body and unified in perception, such as the gods reborn in \textsanskrit{Brahmā}’s Host through the first absorption. This is the second abode of sentient beings. 

There\marginnote{3.2.29} are sentient beings that are unified in body and diverse in perception, such as the gods of streaming radiance. This is the third abode of sentient beings. 

There\marginnote{3.2.31} are sentient beings that are unified in body and unified in perception, such as the gods replete with glory. This is the fourth abode of sentient beings. 

There\marginnote{3.2.33} are sentient beings that are non-percipient and do not experience anything, such as the gods who are non-percipient beings. This is the fifth abode of sentient beings. 

There\marginnote{3.2.35} are sentient beings that have gone totally beyond perceptions of form. With the ending of perceptions of impingement, not focusing on perceptions of diversity, aware that ‘space is infinite’, they have been reborn in the dimension of infinite space. This is the sixth abode of sentient beings. 

There\marginnote{3.2.37} are sentient beings that have gone totally beyond the dimension of infinite space. Aware that ‘consciousness is infinite’, they have been reborn in the dimension of infinite consciousness. This is the seventh abode of sentient beings. 

There\marginnote{3.2.39} are sentient beings that have gone totally beyond the dimension of infinite consciousness. Aware that ‘there is nothing at all’, they have been reborn in the dimension of nothingness. This is the eighth abode of sentient beings. 

There\marginnote{3.2.41} are sentient beings that have gone totally beyond the dimension of nothingness. They have been reborn in the dimension of neither perception nor non-perception. This is the ninth abode of sentient beings. 

Nine\marginnote{3.2.43} lost opportunities for spiritual practice. Firstly, a Realized One has arisen in the world. He teaches the Dhamma leading to peace, extinguishment, awakening, as proclaimed by the Holy One. But a person has been reborn in hell. This is the first lost opportunity for spiritual practice. 

Furthermore,\marginnote{3.2.47} a Realized One has arisen in the world. But a person has been reborn in the animal realm. This is the second lost opportunity for spiritual practice. 

Furthermore,\marginnote{3.2.50} a Realized One has arisen in the world. But a person has been reborn in the ghost realm. This is the third lost opportunity for spiritual practice. 

Furthermore,\marginnote{3.2.53} a Realized One has arisen in the world. But a person has been reborn among the demons. This is the fourth lost opportunity for spiritual practice. 

Furthermore,\marginnote{3.2.56} a Realized One has arisen in the world. But a person has been reborn in one of the long-lived orders of gods. This is the fifth lost opportunity for spiritual practice. 

Furthermore,\marginnote{3.2.59} a Realized One has arisen in the world. But a person has been reborn in the borderlands, among strange barbarian tribes, where monks, nuns, laymen, and laywomen do not go. This is the sixth lost opportunity for spiritual practice. 

Furthermore,\marginnote{3.2.62} a Realized One has arisen in the world. And a person is reborn in a central country. But they have wrong view and distorted perspective: ‘There’s no meaning in giving, sacrifice, or offerings. There’s no fruit or result of good and bad deeds. There’s no afterlife. There are no duties to mother and father. No beings are reborn spontaneously. And there’s no ascetic or brahmin who is well attained and practiced, and who describes the afterlife after realizing it with their own insight.’ This is the seventh lost opportunity for spiritual practice. 

Furthermore,\marginnote{3.2.66} a Realized One has arisen in the world. And a person is reborn in a central country. But they’re witless, dull, stupid, and unable to distinguish what is well said from what is poorly said. This is the eighth lost opportunity for spiritual practice. 

Furthermore,\marginnote{3.2.69} a Realized One has arisen in the world. But he doesn’t teach the Dhamma leading to peace, extinguishment, awakening, as proclaimed by the Holy One. And a person is reborn in a central country. And they’re wise, bright, clever, and able to distinguish what is well said from what is poorly said. This is the ninth lost opportunity for spiritual practice. 

Nine\marginnote{3.2.72} progressive meditations. A mendicant, quite secluded from sensual pleasures, secluded from unskillful qualities, enters and remains in the first absorption, which has the rapture and bliss born of seclusion, while placing the mind and keeping it connected. As the placing of the mind and keeping it connected are stilled, they enter and remain in the second absorption, which has the rapture and bliss born of immersion, with internal clarity and confidence, and unified mind, without placing the mind and keeping it connected. And with the fading away of rapture, they enter and remain in the third absorption, where they meditate with equanimity, mindful and aware, personally experiencing the bliss of which the noble ones declare, ‘Equanimous and mindful, one meditates in bliss.’ Giving up pleasure and pain, and ending former happiness and sadness, they enter and remain in the fourth absorption, without pleasure or pain, with pure equanimity and mindfulness. Going totally beyond perceptions of form, with the ending of perceptions of impingement, not focusing on perceptions of diversity, aware that ‘space is infinite’, they enter and remain in the dimension of infinite space. Going totally beyond the dimension of infinite space, aware that ‘consciousness is infinite’, they enter and remain in the dimension of infinite consciousness. Going totally beyond the dimension of infinite consciousness, aware that ‘there is nothing at all’, they enter and remain in the dimension of nothingness. Going totally beyond the dimension of nothingness, they enter and remain in the dimension of neither perception nor non-perception. Going totally beyond the dimension of neither perception nor non-perception, they enter and remain in the cessation of perception and feeling. 

Nine\marginnote{3.2.82} progressive cessations. For someone who has attained the first absorption, sensual perceptions have ceased. For someone who has attained the second absorption, the placing of the mind and keeping it connected have ceased. For someone who has attained the third absorption, rapture has ceased. For someone who has attained the fourth absorption, breathing has ceased. For someone who has attained the dimension of infinite space, the perception of form has ceased. For someone who has attained the dimension of infinite consciousness, the perception of the dimension of infinite space has ceased. For someone who has attained the dimension of nothingness, the perception of the dimension of infinite consciousness has ceased. For someone who has attained the dimension of neither perception nor non-perception, the perception of the dimension of nothingness has ceased. For someone who has attained the cessation of perception and feeling, perception and feeling have ceased. 

These\marginnote{3.2.92} are the teachings grouped by nine that have been rightly explained by the Buddha. You should all recite these in concert. 

\section*{10. Tens }

There\marginnote{3.3.1} are teachings grouped by ten that have been rightly explained by the Buddha. You should all recite these in concert. What are the teachings grouped by ten? 

Ten\marginnote{3.3.4} qualities that serve as protector. Firstly, a mendicant is ethical, restrained in the monastic code, conducting themselves well and seeking alms in suitable places. Seeing danger in the slightest fault, they keep the rules they’ve undertaken. This is a quality that serves as protector. 

Furthermore,\marginnote{3.3.8} a mendicant is very learned, remembering and keeping what they’ve learned. These teachings are good in the beginning, good in the middle, and good in the end, meaningful and well-phrased, describing a spiritual practice that’s entirely full and pure. They are very learned in such teachings, remembering them, reinforcing them by recitation, mentally scrutinizing them, and comprehending them theoretically. This too is a quality that serves as protector. 

Furthermore,\marginnote{3.3.11} a mendicant has good friends, companions, and associates. This too is a quality that serves as protector. 

Furthermore,\marginnote{3.3.14} a mendicant is easy to admonish, having qualities that make them easy to admonish. They’re patient, and take instruction respectfully. This too is a quality that serves as protector. 

Furthermore,\marginnote{3.3.17} a mendicant is deft and tireless in a diverse spectrum of duties for their spiritual companions, understanding how to go about things in order to complete and organize the work. This too is a quality that serves as protector. 

Furthermore,\marginnote{3.3.20} a mendicant loves the teachings and is a delight to converse with, being full of joy in the teaching and training. This too is a quality that serves as protector. 

Furthermore,\marginnote{3.3.23} a mendicant is content with any kind of robes, almsfood, lodgings, and medicines and supplies for the sick. This too is a quality that serves as protector. 

Furthermore,\marginnote{3.3.26} a mendicant lives with energy roused up for giving up unskillful qualities and embracing skillful qualities. They are strong, staunchly vigorous, not slacking off when it comes to developing skillful qualities. This too is a quality that serves as protector. 

Furthermore,\marginnote{3.3.29} a mendicant is mindful. They have utmost mindfulness and alertness, and can remember and recall what was said and done long ago. This too is a quality that serves as protector. 

Furthermore,\marginnote{3.3.32} a mendicant is wise. They have the wisdom of arising and passing away which is noble, penetrative, and leads to the complete ending of suffering. This too is a quality that serves as protector. 

Ten\marginnote{3.3.35} universal dimensions of meditation. Someone perceives the meditation on universal earth above, below, across, non-dual and limitless. They perceive the meditation on universal water … the meditation on universal fire … the meditation on universal air … the meditation on universal blue … the meditation on universal yellow … the meditation on universal red … the meditation on universal white … the meditation on universal space … They perceive the meditation on universal consciousness above, below, across, non-dual and limitless. 

Ten\marginnote{3.3.46} ways of doing unskillful deeds: killing living creatures, stealing, and sexual misconduct; speech that’s false, divisive, harsh, or nonsensical; covetousness, ill will, and wrong view. 

Ten\marginnote{3.3.48} ways of doing skillful deeds: refraining from killing living creatures, stealing, and sexual misconduct; refraining from speech that’s false, divisive, harsh, or nonsensical; contentment, good will, and right view. 

Ten\marginnote{3.3.50} noble abodes. A mendicant has given up five factors, possesses six factors, has a single guard, has four supports, has eliminated idiosyncratic interpretations of the truth, has totally given up searching, has unsullied intentions, has stilled the physical process, and is well freed in mind and well freed by wisdom. 

And\marginnote{3.3.52} how has a mendicant given up five factors? It’s when a mendicant has given up sensual desire, ill will, dullness and drowsiness, restlessness and remorse, and doubt. That’s how a mendicant has given up five factors. 

And\marginnote{3.3.55} how does a mendicant possess six factors? A mendicant, seeing a sight with their eyes, is neither happy nor sad. They remain equanimous, mindful and aware. Hearing a sound with their ears … Smelling an odor with their nose … Tasting a flavor with their tongue … Feeling a touch with their body … Knowing a thought with their mind, they’re neither happy nor sad. They remain equanimous, mindful and aware. That’s how a mendicant possesses six factors. 

And\marginnote{3.3.60} how does a mendicant have a single guard? It’s when a mendicant’s heart is guarded by mindfulness. That’s how a mendicant has a single guard. 

And\marginnote{3.3.63} how does a mendicant have four supports? After appraisal, a mendicant uses some things, endures some things, avoids some things, and gets rid of some things. That’s how a mendicant has four supports. 

And\marginnote{3.3.66} how has a mendicant eliminated idiosyncratic interpretations of the truth? Different ascetics and brahmins have different idiosyncratic interpretations of the truth. A mendicant has dispelled, eliminated, thrown out, rejected, let go of, given up, and relinquished all these. That’s how a mendicant has eliminated idiosyncratic interpretations of the truth. 

And\marginnote{3.3.69} how has a mendicant totally given up searching? It’s when they’ve given up searching for sensual pleasures, for continued existence, and for a spiritual path. That’s how a mendicant has totally given up searching. 

And\marginnote{3.3.72} how does a mendicant have unsullied intentions? It’s when they’ve given up sensual, malicious, and cruel intentions. That’s how a mendicant has unsullied intentions. 

And\marginnote{3.3.75} how has a mendicant stilled the physical process? It’s when, giving up pleasure and pain, and ending former happiness and sadness, they enter and remain in the fourth absorption, without pleasure or pain, with pure equanimity and mindfulness. That’s how a mendicant has stilled the physical process. 

And\marginnote{3.3.78} how is a mendicant well freed in mind? It’s when a mendicant’s mind is freed from greed, hate, and delusion. That’s how a mendicant is well freed in mind. 

And\marginnote{3.3.81} how is a mendicant well freed by wisdom? It’s when a mendicant understands: ‘I’ve given up greed, hate, and delusion, cut them off at the root, made them like a palm stump, obliterated them, so they’re unable to arise in the future.’ That’s how a mendicant’s mind is well freed by wisdom. 

Ten\marginnote{3.3.86} qualities of an adept: an adept’s right view, right thought, right speech, right action, right livelihood, right effort, right mindfulness, right immersion, right knowledge, and right freedom. 

Reverends,\marginnote{3.3.88} these are the teachings grouped by ten that have been rightly explained by the Buddha. You should all recite these in concert, without disputing, so that this spiritual path may last for a long time. That would be for the welfare and happiness of the people, out of compassion for the world, for the benefit, welfare, and happiness of gods and humans.” 

Then\marginnote{3.4.1} the Buddha got up and said to Venerable \textsanskrit{Sāriputta}, “Good, good, \textsanskrit{Sāriputta}! It’s good that you’ve taught this exposition of the reciting in concert.” 

That\marginnote{3.4.4} is what Venerable \textsanskrit{Sāriputta} said, and the teacher approved. Satisfied, the mendicants were happy with what \textsanskrit{Sāriputta} said. 

%
\chapter*{{\suttatitleacronym DN 34}{\suttatitletranslation Up to Ten }{\suttatitleroot Dasuttarasutta}}
\addcontentsline{toc}{chapter}{\tocacronym{DN 34} \toctranslation{Up to Ten } \tocroot{Dasuttarasutta}}
\markboth{Up to Ten }{Dasuttarasutta}
\extramarks{DN 34}{DN 34}

\scevam{So\marginnote{1.1.1} I have heard. }At one time the Buddha was staying near \textsanskrit{Campā} on the banks of the \textsanskrit{Gaggarā} Lotus Pond together with a large \textsanskrit{Saṅgha} of five hundred mendicants. There \textsanskrit{Sāriputta} addressed the mendicants: “Reverends, mendicants!” 

“Reverend,”\marginnote{1.1.5} they replied. \textsanskrit{Sāriputta} said this: 

\begin{verse}%
“I\marginnote{1.1.7} will relate the teachings \\
up to ten for attaining extinguishment, \\
for making an end of suffering, \\
the release from all ties. 

%
\end{verse}

\section*{1. Groups of One }

Reverends,\marginnote{1.2.1} one thing is helpful, one thing should be developed, one thing should be completely understood, one thing should be given up, one thing makes things worse, one thing leads to distinction, one thing is hard to comprehend, one thing should be produced, one thing should be directly known, one thing should be realized. 

What\marginnote{1.2.2} one thing is helpful? Diligence in skillful qualities. 

What\marginnote{1.2.5} one thing should be developed? Mindfulness of the body that is full of pleasure. 

What\marginnote{1.2.8} one thing should be completely understood? Contact, which is accompanied by defilements and is prone to being grasped. 

What\marginnote{1.2.11} one thing should be given up? The conceit ‘I am’. 

What\marginnote{1.2.14} one thing makes things worse? Improper attention. 

What\marginnote{1.2.17} one thing leads to distinction? Proper attention. 

What\marginnote{1.2.20} one thing is hard to comprehend? The heart’s immersion of immediate result. 

What\marginnote{1.2.23} one thing should be produced? Unshakable knowledge. 

What\marginnote{1.2.26} one thing should be directly known? All sentient beings are sustained by food. 

What\marginnote{1.2.29} one thing should be realized? The unshakable release of the heart. 

So\marginnote{1.2.32} these ten things that are true, real, and accurate, not unreal, not otherwise were rightly awakened to by the Realized One. 

\section*{2. Groups of Two }

Two\marginnote{1.3.1} things are helpful, two things should be developed, two things should be completely understood, two things should be given up, two things make things worse, two things lead to distinction, two things are hard to comprehend, two things should be produced, two things should be directly known, two things should be realized. 

What\marginnote{1.3.2} two things are helpful? Mindfulness and situational awareness. 

What\marginnote{1.3.5} two things should be developed? Serenity and discernment. 

What\marginnote{1.3.8} two things should be completely understood? Name and form. 

What\marginnote{1.3.11} two things should be given up? Ignorance and craving for continued existence. 

What\marginnote{1.3.14} two things make things worse? Being hard to admonish and having bad friends. 

What\marginnote{1.3.17} two things lead to distinction? Being easy to admonish and having good friends. 

What\marginnote{1.3.20} two things are hard to comprehend? What are the causes and conditions for the corruption of sentient beings, and what are the causes and conditions for the purification of sentient beings. 

What\marginnote{1.3.23} two things should be produced? Two knowledges: knowledge of ending, and knowledge of non-arising. 

What\marginnote{1.3.27} two things should be directly known? Two elements: the conditioned element and the unconditioned element. 

What\marginnote{1.3.31} two things should be realized? Knowledge and freedom. 

So\marginnote{1.3.34} these twenty things that are true, real, and accurate, not unreal, not otherwise were rightly awakened to by the Realized One. 

\section*{3. Groups of Three }

Three\marginnote{1.4.1} things are helpful, etc. 

What\marginnote{1.4.2} three things are helpful? Associating with good people, listening to the true teaching, and practicing in line with the teaching. 

What\marginnote{1.4.5} three things should be developed? Three kinds of immersion. Immersion with placing the mind and keeping it connected. Immersion without placing the mind, but just keeping it connected. Immersion without placing the mind or keeping it connected. 

What\marginnote{1.4.9} three things should be completely understood? Three feelings: pleasant, painful, and neutral. 

What\marginnote{1.4.13} three things should be given up? Three cravings: craving for sensual pleasures, craving for continued existence, and craving to end existence. 

What\marginnote{1.4.17} three things make things worse? Three unskillful roots: greed, hate, and delusion. 

What\marginnote{1.4.21} three things lead to distinction? Three skillful roots: non-greed, non-hate, and non-delusion. 

What\marginnote{1.4.25} three things are hard to comprehend? Three elements of escape. Renunciation is the escape from sensual pleasures. The formless is the escape from form. Cessation is the escape from whatever is created, conditioned, and dependently originated. 

What\marginnote{1.4.29} three things should be produced? Three knowledges: regarding the past, future, and present. 

What\marginnote{1.4.33} three things should be directly known? Three elements: sensuality, form, and formlessness. 

What\marginnote{1.4.37} three things should be realized? Three knowledges: recollection of past lives, knowledge of the death and rebirth of sentient beings, and knowledge of the ending of defilements. 

So\marginnote{1.4.41} these thirty things that are true, real, and accurate, not unreal, not otherwise were rightly awakened to by the Realized One. 

\section*{4. Groups of Four }

Four\marginnote{1.5.1} things are helpful, etc. 

What\marginnote{1.5.2} four things are helpful? Four situations: living in a suitable region, relying on good people, being rightly resolved in oneself, and past merit. 

What\marginnote{1.5.6} four things should be developed? The four kinds of mindfulness meditation. A mendicant meditates by observing an aspect of the body—keen, aware, and mindful, rid of desire and aversion for the world. They meditate observing an aspect of feelings … mind … principles—keen, aware, and mindful, rid of desire and aversion for the world. 

What\marginnote{1.5.13} four things should be completely understood? Four foods: solid food, whether coarse or fine; contact is the second, mental intention the third, and consciousness the fourth. 

What\marginnote{1.5.17} four things should be given up? Four floods: sensuality, desire for rebirth, views, and ignorance. 

What\marginnote{1.5.21} four things make things worse? Four bonds: sensuality, desire for rebirth, views, and ignorance. 

What\marginnote{1.5.25} four things lead to distinction? Four kinds of detachment: detachment from the bonds of sensuality, desire for rebirth, views, and ignorance. 

What\marginnote{1.5.29} four things are hard to comprehend? Four kinds of immersion: immersion liable to decline, stable immersion, immersion that leads to distinction, and immersion that leads to penetration. 

What\marginnote{1.5.33} four things should be produced? Four knowledges: knowledge of the present phenomena, inferential knowledge, knowledge of others’ minds, and conventional knowledge. 

What\marginnote{1.5.37} four things should be directly known? The four noble truths: suffering, the origin of suffering, the cessation of suffering, and the practice that leads to the cessation of suffering. 

What\marginnote{1.5.41} four things should be realized? Four fruits of the ascetic life: stream-entry, once-return, non-return, and perfection. 

So\marginnote{1.5.45} these forty things that are true, real, and accurate, not unreal, not otherwise were rightly awakened to by the Realized One. 

\section*{5. Groups of Five }

Five\marginnote{1.6.1} things are helpful, etc. 

What\marginnote{1.6.2} five things are helpful? Five factors that support meditation. A mendicant has faith in the Realized One’s awakening: ‘That Blessed One is perfected, a fully awakened Buddha, accomplished in knowledge and conduct, holy, knower of the world, supreme guide for those who wish to train, teacher of gods and humans, awakened, blessed.’ They are rarely ill or unwell. Their stomach digests well, being neither too hot nor too cold, but just right, and fit for meditation. They’re not devious or deceitful. They reveal themselves honestly to the Teacher or sensible spiritual companions. They live with energy roused up for giving up unskillful qualities and embracing skillful qualities. They’re strong, staunchly vigorous, not slacking off when it comes to developing skillful qualities. They’re wise. They have the wisdom of arising and passing away which is noble, penetrative, and leads to the complete ending of suffering. 

What\marginnote{1.6.11} five things should be developed? Right immersion with five factors: pervaded with rapture, pervaded with pleasure, pervaded with mind, pervaded with light, and the foundation for reviewing. 

What\marginnote{1.6.15} five things should be completely understood? Five grasping aggregates: form, feeling, perception, choices, and consciousness. 

What\marginnote{1.6.19} five things should be given up? Five hindrances: sensual desire, ill will, dullness and drowsiness, restlessness and remorse, and doubt. 

What\marginnote{1.6.23} five things make things worse? Five kinds of emotional barrenness. Firstly, a mendicant has doubts about the Teacher. They’re uncertain, undecided, and lacking confidence. This being so, their mind doesn’t incline toward keenness, commitment, persistence, and striving. This is the first kind of emotional barrenness. Furthermore, a mendicant has doubts about the teaching … the \textsanskrit{Saṅgha} … the training … A mendicant is angry and upset with their spiritual companions, resentful and closed off. This being so, their mind doesn’t incline toward keenness, commitment, persistence, and striving. This is the fifth kind of emotional barrenness. 

What\marginnote{1.6.36} five things lead to distinction? Five faculties: faith, energy, mindfulness, immersion, and wisdom. 

What\marginnote{1.6.40} five things are hard to comprehend? Five elements of escape. A mendicant focuses on sensual pleasures, but their mind isn’t eager, confident, settled, and decided about them. But when they focus on renunciation, their mind is eager, confident, settled, and decided about it. Their mind is in a good state, well developed, well risen, well freed, and well detached from sensual pleasures. They’re freed from the distressing and feverish defilements that arise because of sensual pleasures, so they don’t experience that kind of feeling. This is how the escape from sensual pleasures is explained. 

Take\marginnote{1.6.47} another case where a mendicant focuses on ill will, but their mind isn’t eager … But when they focus on good will, their mind is eager … Their mind is in a good state … well detached from ill will. They’re freed from the distressing and feverish defilements that arise because of ill will, so they don’t experience that kind of feeling. This is how the escape from ill will is explained. 

Take\marginnote{1.6.52} another case where a mendicant focuses on harming, but their mind isn’t eager … But when they focus on compassion, their mind is eager … Their mind is in a good state … well detached from harming. They’re freed from the distressing and feverish defilements that arise because of harming, so they don’t experience that kind of feeling. This is how the escape from harming is explained. 

Take\marginnote{1.6.57} another case where a mendicant focuses on form, but their mind isn’t eager … But when they focus on the formless, their mind is eager … Their mind is in a good state … well detached from forms. They’re freed from the distressing and feverish defilements that arise because of form, so they don’t experience that kind of feeling. This is how the escape from forms is explained. 

Take\marginnote{1.6.62} a case where a mendicant focuses on identity, but their mind isn’t eager, confident, settled, and decided about it. But when they focus on the ending of identity, their mind is eager, confident, settled, and decided about it. Their mind is in a good state, well developed, well risen, well freed, and well detached from identity. They’re freed from the distressing and feverish defilements that arise because of identity, so they don’t experience that kind of feeling. This is how the escape from identity is explained. 

What\marginnote{1.6.68} five things should be produced? Right immersion with five knowledges. The following knowledges arise for you personally: ‘This immersion is blissful now, and results in bliss in the future.’ ‘This immersion is noble and spiritual.’ ‘This immersion is not cultivated by sinners.’ ‘This immersion is peaceful and sublime and tranquil and unified, not held in place by forceful suppression.’ ‘I mindfully enter into and emerge from this immersion.’ 

What\marginnote{1.6.76} five things should be directly known? Five opportunities for freedom. Firstly, the Teacher or a respected spiritual companion teaches Dhamma to a mendicant. That mendicant feels inspired by the meaning and the teaching in that Dhamma, no matter how the Teacher or a respected spiritual companion teaches it. Feeling inspired, joy springs up. Being joyful, rapture springs up. When the mind is full of rapture, the body becomes tranquil. When the body is tranquil, one feels bliss. And when blissful, the mind becomes immersed. This is the first opportunity for freedom. 

Furthermore,\marginnote{1.6.82} it may be that neither the Teacher nor a respected spiritual companion teaches Dhamma to a mendicant. But the mendicant teaches Dhamma in detail to others as they learned and memorized it. That mendicant feels inspired by the meaning and the teaching in that Dhamma, no matter how they teach it in detail to others as they learned and memorized it. Feeling inspired, joy springs up. Being joyful, rapture springs up. When the mind is full of rapture, the body becomes tranquil. When the body is tranquil, one feels bliss. And when blissful, the mind becomes immersed. This is the second opportunity for freedom. 

Furthermore,\marginnote{1.6.86} it may be that neither the Teacher nor … the mendicant teaches Dhamma. But the mendicant recites the teaching in detail as they learned and memorized it. That mendicant feels inspired by the meaning and the teaching in that Dhamma, no matter how they recite it in detail as they learned and memorized it. Feeling inspired, joy springs up. Being joyful, rapture springs up. When the mind is full of rapture, the body becomes tranquil. When the body is tranquil, one feels bliss. And when blissful, the mind becomes immersed. This is the third opportunity for freedom. 

Furthermore,\marginnote{1.6.90} it may be that neither the Teacher nor … the mendicant teaches Dhamma … nor does the mendicant recite the teaching. But the mendicant thinks about and considers the teaching in their heart, examining it with the mind as they learned and memorized it. That mendicant feels inspired by the meaning and the teaching in that Dhamma, no matter how they think about and consider it in their heart, examining it with the mind as they learned and memorized it. Feeling inspired, joy springs up. Being joyful, rapture springs up. When the mind is full of rapture, the body becomes tranquil. When the body is tranquil, one feels bliss. And when blissful, the mind becomes immersed. This is the fourth opportunity for freedom. 

Furthermore,\marginnote{1.6.95} it may be that neither the Teacher nor … the mendicant teaches Dhamma … nor does the mendicant recite the teaching … or think about it. But a meditation subject as a foundation of immersion is properly grasped, attended, borne in mind, and comprehended with wisdom. That mendicant feels inspired by the meaning and the teaching in that Dhamma, no matter how a meditation subject as a foundation of immersion is properly grasped, attended, borne in mind, and comprehended with wisdom. Feeling inspired, joy springs up. Being joyful, rapture springs up. When the mind is full of rapture, the body becomes tranquil. When the body is tranquil, one feels bliss. And when blissful, the mind becomes immersed. This is the fifth opportunity for freedom. 

What\marginnote{1.6.101} five things should be realized? Five spectrums of the teaching: ethics, immersion, wisdom, freedom, and knowledge and vision of freedom. 

So\marginnote{1.6.105} these fifty things that are true, real, and accurate, not unreal, not otherwise were rightly awakened to by the Realized One. 

\section*{6. Groups of Six }

Six\marginnote{1.7.1} things are helpful, etc. 

What\marginnote{1.7.2} six things are helpful? Six warm-hearted qualities. Firstly, a mendicant consistently treats their spiritual companions with bodily kindness, both in public and in private. This warm-hearted quality makes for fondness and respect, conducing to inclusion, harmony, and unity, without quarreling. 

Furthermore,\marginnote{1.7.6} a mendicant consistently treats their spiritual companions with verbal kindness. 

Furthermore,\marginnote{1.7.7} a mendicant consistently treats their spiritual companions with mental kindness. 

Furthermore,\marginnote{1.7.8} a mendicant shares without reservation any material possessions they have gained by legitimate means, even the food placed in the alms-bowl, using them in common with their ethical spiritual companions. 

Furthermore,\marginnote{1.7.9} a mendicant lives according to the precepts shared with their spiritual companions, both in public and in private. Those precepts are unbroken, impeccable, spotless, and unmarred, liberating, praised by sensible people, not mistaken, and leading to immersion. 

Furthermore,\marginnote{1.7.10} a mendicant lives according to the view shared with their spiritual companions, both in public and in private. That view is noble and emancipating, and leads one who practices it to the complete ending of suffering. This warm-hearted quality makes for fondness and respect, conducing to inclusion, harmony, and unity, without quarreling. 

What\marginnote{1.7.13} six things should be developed? Six topics for recollection: the recollection of the Buddha, the teaching, the \textsanskrit{Saṅgha}, ethics, generosity, and the deities. 

What\marginnote{1.7.17} six things should be completely understood? Six interior sense fields: eye, ear, nose, tongue, body, and mind. 

What\marginnote{1.7.21} six things should be given up? Six classes of craving: craving for sights, sounds, smells, tastes, touches, and thoughts. 

What\marginnote{1.7.25} six things make things worse? Six kinds of disrespect. A mendicant lacks respect and reverence for the Teacher, the teaching, and the \textsanskrit{Saṅgha}, the training, diligence, and hospitality. 

What\marginnote{1.7.29} six things lead to distinction? Six kinds of respect. A mendicant has respect and reverence for the Teacher, the teaching, and the \textsanskrit{Saṅgha}, the training, diligence, and hospitality. 

What\marginnote{1.7.33} six things are hard to comprehend? Six elements of escape. Take a mendicant who says: ‘I’ve developed the heart’s release by love. I’ve cultivated it, made it my vehicle and my basis, kept it up, consolidated it, and properly implemented it. Yet somehow ill will still occupies my mind.’ They should be told, ‘Not so, venerable! Don’t say that. Don’t misrepresent the Buddha, for misrepresentation of the Buddha is not good. And the Buddha would not say that. It’s impossible, reverend, it cannot happen that the heart’s release by love has been developed and properly implemented, yet somehow ill will still occupies the mind. For it is the heart’s release by love that is the escape from ill will.’ 

Take\marginnote{1.7.42} another mendicant who says: ‘I’ve developed the heart’s release by compassion. I’ve cultivated it, made it my vehicle and my basis, kept it up, consolidated it, and properly implemented it. Yet somehow the thought of harming still occupies my mind.’ They should be told, ‘Not so, venerable! … For it is the heart’s release by compassion that is the escape from thoughts of harming.’ 

Take\marginnote{1.7.47} another mendicant who says: ‘I’ve developed the heart’s release by rejoicing. … Yet somehow discontent still occupies my mind.’ They should be told, ‘Not so, venerable! … For it is the heart’s release by rejoicing that is the escape from discontent.’ 

Take\marginnote{1.7.52} another mendicant who says: ‘I’ve developed the heart’s release by equanimity. … Yet somehow desire still occupies my mind.’ They should be told, ‘Not so, venerable! … For it is the heart’s release by equanimity that is the escape from desire.’ 

Take\marginnote{1.7.57} another mendicant who says: ‘I’ve developed the signless release of the heart. … Yet somehow my consciousness still follows after signs.’ They should be told, ‘Not so, venerable! … For it is the signless release of the heart that is the escape from all signs.’ 

Take\marginnote{1.7.62} another mendicant who says: ‘I’m rid of the conceit “I am”. And I don’t regard anything as “I am this”. Yet somehow the dart of doubt and indecision still occupies my mind.’ They should be told, ‘Not so, venerable! Don’t say that. Don’t misrepresent the Buddha, for misrepresentation of the Buddha is not good. And the Buddha would not say that. It’s impossible, reverend, it cannot happen that the conceit “I am” has been done away with, and nothing is regarded as “I am this”, yet somehow the dart of doubt and indecision still occupy the mind. For it is the uprooting of the conceit “I am” that is the escape from the dart of doubt and indecision.’ 

What\marginnote{1.7.70} six things should be produced? Six consistent responses. A mendicant, seeing a sight with their eyes, is neither happy nor sad. They remain equanimous, mindful and aware. Hearing a sound with their ears … Smelling an odor with their nose … Tasting a flavor with their tongue … 

Feeling\marginnote{1.7.76} a touch with their body … Knowing a thought with their mind, they’re neither happy nor sad. They remain equanimous, mindful and aware. 

What\marginnote{1.7.79} six things should be directly known? Six unsurpassable things: the unsurpassable seeing, listening, acquisition, training, service, and recollection. 

What\marginnote{1.7.83} six things should be realized? Six direct knowledges. A mendicant wields the many kinds of psychic power: multiplying themselves and becoming one again; appearing and disappearing; going unimpeded through a wall, a rampart, or a mountain as if through space; diving in and out of the earth as if it were water; walking on water as if it were earth; flying cross-legged through the sky like a bird; touching and stroking with the hand the sun and moon, so mighty and powerful; controlling the body as far as the \textsanskrit{Brahmā} realm. 

With\marginnote{1.7.86} clairaudience that is purified and superhuman, they hear both kinds of sounds, human and divine, whether near or far. 

They\marginnote{1.7.87} understand the minds of other beings and individuals, having comprehended them with their own mind. 

They\marginnote{1.7.90} recollect many kinds of past lives, with features and details. 

With\marginnote{1.7.91} clairvoyance that is purified and superhuman, they see sentient beings passing away and being reborn—inferior and superior, beautiful and ugly, in a good place or a bad place. They understand how sentient beings are reborn according to their deeds. 

They\marginnote{1.7.92} realize the undefiled freedom of heart and freedom by wisdom in this very life. And they live having realized it with their own insight due to the ending of defilements. 

So\marginnote{1.7.94} these sixty things that are true, real, and accurate, not unreal, not otherwise were rightly awakened to by the Realized One. 

\section*{7. Groups of Seven }

Seven\marginnote{1.8.1} things are helpful, etc. 

What\marginnote{1.8.2} seven things are helpful? Seven kinds of wealth of noble ones: the wealth of faith, ethical conduct, conscience, prudence, learning, generosity, and wisdom. 

What\marginnote{1.8.6} seven things should be developed? Seven awakening factors: mindfulness, investigation of principles, energy, rapture, tranquility, immersion, and equanimity. 

What\marginnote{1.8.10} seven things should be completely understood? Seven planes of consciousness. There are sentient beings that are diverse in body and diverse in perception, such as human beings, some gods, and some beings in the underworld. This is the first plane of consciousness. 

There\marginnote{1.8.14} are sentient beings that are diverse in body and unified in perception, such as the gods reborn in \textsanskrit{Brahmā}’s Host through the first absorption. This is the second plane of consciousness. 

There\marginnote{1.8.16} are sentient beings that are unified in body and diverse in perception, such as the gods of streaming radiance. This is the third plane of consciousness. 

There\marginnote{1.8.18} are sentient beings that are unified in body and unified in perception, such as the gods replete with glory. This is the fourth plane of consciousness. 

There\marginnote{1.8.20} are sentient beings that have gone totally beyond perceptions of form. With the ending of perceptions of impingement, not focusing on perceptions of diversity, aware that ‘space is infinite’, they have been reborn in the dimension of infinite space. This is the fifth plane of consciousness. 

There\marginnote{1.8.22} are sentient beings that have gone totally beyond the dimension of infinite space. Aware that ‘consciousness is infinite’, they have been reborn in the dimension of infinite consciousness. This is the sixth plane of consciousness. 

There\marginnote{1.8.24} are sentient beings that have gone totally beyond the dimension of infinite consciousness. Aware that ‘there is nothing at all’, they have been reborn in the dimension of nothingness. This is the seventh plane of consciousness. 

What\marginnote{1.8.27} seven things should be given up? Seven underlying tendencies: sensual desire, repulsion, views, doubt, conceit, desire to be reborn, and ignorance. 

What\marginnote{1.8.31} seven things make things worse? Seven bad qualities: a mendicant is faithless, shameless, imprudent, uneducated, lazy, unmindful, and witless. 

What\marginnote{1.8.35} seven things lead to distinction? Seven good qualities: a mendicant is faithful, conscientious, prudent, learned, energetic, mindful, and wise. 

What\marginnote{1.8.39} seven things are hard to comprehend? Seven aspects of the teachings of the good persons: a mendicant knows the teachings, knows the meaning, knows themselves, knows moderation, knows the right time, knows assemblies, and knows people. 

What\marginnote{1.8.43} seven things should be produced? Seven perceptions: the perception of impermanence, the perception of not-self, the perception of ugliness, the perception of drawbacks, the perception of giving up, the perception of fading away, and the perception of cessation. 

What\marginnote{1.8.47} seven things should be directly known? Seven qualifications for graduation. A mendicant has a keen enthusiasm to undertake the training … to examine the teachings … to get rid of desires … for retreat … to rouse up energy … for mindfulness and alertness … to penetrate theoretically. And they don’t lose these desires in the future. 

What\marginnote{1.8.57} seven things should be realized? Seven powers of one who has ended the defilements. Firstly, a mendicant with defilements ended has clearly seen with right wisdom all conditions as truly impermanent. This is a power that a mendicant who has ended the defilements relies on to claim: ‘My defilements have ended.’ 

Furthermore,\marginnote{1.8.61} a mendicant with defilements ended has clearly seen with right wisdom that sensual pleasures are truly like a pit of glowing coals. … 

Furthermore,\marginnote{1.8.63} the mind of a mendicant with defilements ended slants, slopes, and inclines to seclusion. They’re withdrawn, loving renunciation, and they’ve totally done with defiling influences. … 

Furthermore,\marginnote{1.8.65} a mendicant with defilements ended has well developed the four kinds of mindfulness meditation. … 

Furthermore,\marginnote{1.8.67} a mendicant with defilements ended has well developed the five faculties. … 

Furthermore,\marginnote{1.8.69} a mendicant with defilements ended has well developed the seven awakening factors. … 

Furthermore,\marginnote{1.8.71} a mendicant with defilements ended has well developed the noble eightfold path. … This is a power that a mendicant who has ended the defilements relies on to claim: ‘My defilements have ended.’ 

So\marginnote{1.8.75} these seventy things that are true, real, and accurate, not unreal, not otherwise were rightly awakened to by the Realized One. 

\scendsection{The first recitation section is finished. }

\section*{8. Groups of Eight }

Eight\marginnote{2.1.1} things are helpful, etc. 

What\marginnote{2.1.2} eight things are helpful? There are eight causes and reasons that lead to acquiring the wisdom fundamental to the spiritual life, and to its increase, growth, development, and fulfillment once it has been acquired. What eight? It’s when a mendicant lives relying on the Teacher or a spiritual companion in a teacher’s role. And they set up a keen sense of conscience and prudence for them, with warmth and respect. This is the first cause. 

When\marginnote{2.1.7} a mendicant lives relying on the Teacher or a spiritual companion in a teacher’s role—with a keen sense of conscience and prudence for them, with warmth and respect—from time to time they go and ask them questions: ‘Why, sir, does it say this? What does that mean?’ Those venerables clarify what is unclear, reveal what is obscure, and dispel doubt regarding the many doubtful matters. This is the second cause. 

After\marginnote{2.1.12} hearing that teaching they perfect withdrawal of both body and mind. This is the third cause. 

Furthermore,\marginnote{2.1.14} a mendicant is ethical, restrained in the monastic code, conducting themselves well and seeking alms in suitable places. Seeing danger in the slightest fault, they keep the rules they’ve undertaken. This is the fourth cause. 

Furthermore,\marginnote{2.1.16} a mendicant is very learned, remembering and keeping what they’ve learned. These teachings are good in the beginning, good in the middle, and good in the end, meaningful and well-phrased, describing a spiritual practice that’s entirely full and pure. They are very learned in such teachings, remembering them, reinforcing them by recitation, mentally scrutinizing them, and comprehending them theoretically. This is the fifth cause. 

Furthermore,\marginnote{2.1.18} a mendicant lives with energy roused up for giving up unskillful qualities and embracing skillful qualities. They are strong, staunchly vigorous, not slacking off when it comes to developing skillful qualities. This is the sixth cause. 

Furthermore,\marginnote{2.1.20} a mendicant is mindful. They have utmost mindfulness and alertness, and can remember and recall what was said and done long ago. This is the seventh cause. 

Furthermore,\marginnote{2.1.22} a mendicant meditates observing rise and fall in the five grasping aggregates. ‘Such is form, such is the origin of form, such is the ending of form. Such is feeling, such is the origin of feeling, such is the ending of feeling. Such is perception, such is the origin of perception, such is the ending of perception. Such are choices, such is the origin of choices, such is the ending of choices. Such is consciousness, such is the origin of consciousness, such is the ending of consciousness.’ This is the eighth cause. 

What\marginnote{2.1.30} eight things should be developed? The noble eightfold path, that is: right view, right thought, right speech, right action, right livelihood, right effort, right mindfulness, and right immersion. 

What\marginnote{2.1.34} eight things should be completely understood? Eight worldly conditions: gain and loss, fame and disgrace, blame and praise, pleasure and pain. 

What\marginnote{2.1.38} eight things should be given up? Eight wrong ways: wrong view, wrong thought, wrong speech, wrong action, wrong livelihood, wrong effort, wrong mindfulness, and wrong immersion. 

What\marginnote{2.1.42} eight things make things worse? Eight grounds for laziness. Firstly, a mendicant has some work to do. They think: ‘I have some work to do. But while doing it my body will get tired. I’d better have a lie down.’ They lie down, and don’t rouse energy for attaining the unattained, achieving the unachieved, and realizing the unrealized. This is the first ground for laziness. 

Furthermore,\marginnote{2.1.49} a mendicant has done some work. They think: ‘I’ve done some work. But while working my body got tired. I’d better have a lie down.’ They lie down, and don’t rouse energy… This is the second ground for laziness. 

Furthermore,\marginnote{2.1.54} a mendicant has to go on a journey. They think: ‘I have to go on a journey. But while walking my body will get tired. I’d better have a lie down.’ They lie down, and don’t rouse energy… This is the third ground for laziness. 

Furthermore,\marginnote{2.1.59} a mendicant has gone on a journey. They think: ‘I’ve gone on a journey. But while walking my body got tired. I’d better have a lie down.’ They lie down, and don’t rouse energy… This is the fourth ground for laziness. 

Furthermore,\marginnote{2.1.64} a mendicant has wandered for alms, but they didn’t get to fill up on as much food as they like, rough or fine. They think: ‘I’ve wandered for alms, but I didn’t get to fill up on as much food as I like, rough or fine. My body is tired and unfit for work. I’d better have a lie down.’… This is the fifth ground for laziness. 

Furthermore,\marginnote{2.1.68} a mendicant has wandered for alms, and they got to fill up on as much food as they like, rough or fine. They think: ‘I’ve wandered for alms, and I got to fill up on as much food as I like, rough or fine. My body is heavy, unfit for work, like I’ve just eaten a load of beans. I’d better have a lie down.’… They lie down, and don’t rouse energy… This is the sixth ground for laziness. 

Furthermore,\marginnote{2.1.73} a mendicant feels a little sick. They think: ‘I feel a little sick. Lying down would be good for me. I’d better have a lie down.’ They lie down, and don’t rouse energy… This is the seventh ground for laziness. 

Furthermore,\marginnote{2.1.77} a mendicant has recently recovered from illness. They think: ‘I’ve recently recovered from illness. My body is weak and unfit for work. I’d better have a lie down.’ They lie down, and don’t rouse energy… This is the eighth ground for laziness. 

What\marginnote{2.1.84} eight things lead to distinction? Eight grounds for arousing energy. Firstly, a mendicant has some work to do. They think: ‘I have some work to do. While working it’s not easy to focus on the instructions of the Buddhas. I’d better preemptively rouse up energy for attaining the unattained, achieving the unachieved, and realizing the unrealized.’ They rouse energy for attaining the unattained, achieving the unachieved, and realizing the unrealized. This is the first ground for arousing energy. 

Furthermore,\marginnote{2.1.90} a mendicant has done some work. They think: ‘I’ve done some work. While I was working I wasn’t able to focus on the instructions of the Buddhas. I’d better preemptively rouse up energy.’… This is the second ground for arousing energy. 

Furthermore,\marginnote{2.1.94} a mendicant has to go on a journey. They think: ‘I have to go on a journey. While walking it’s not easy to focus on the instructions of the Buddhas. I’d better preemptively rouse up energy.’… This is the third ground for arousing energy. 

Furthermore,\marginnote{2.1.98} a mendicant has gone on a journey. They think: ‘I’ve gone on a journey. While I was walking I wasn’t able to focus on the instructions of the Buddhas. I’d better preemptively rouse up energy.’… This is the fourth ground for arousing energy. 

Furthermore,\marginnote{2.1.102} a mendicant has wandered for alms, but they didn’t get to fill up on as much food as they like, rough or fine. They think: ‘I’ve wandered for alms, but I didn’t get to fill up on as much food as I like, rough or fine. My body is light and fit for work. I’d better preemptively rouse up energy.’… This is the fifth ground for arousing energy. 

Furthermore,\marginnote{2.1.106} a mendicant has wandered for alms, and they got to fill up on as much food as they like, rough or fine. They think: ‘I’ve wandered for alms, and I got to fill up on as much food as I like, rough or fine. My body is strong and fit for work. I’d better preemptively rouse up energy.’… This is the sixth ground for arousing energy. 

Furthermore,\marginnote{2.1.111} a mendicant feels a little sick. They think: ‘I feel a little sick. It’s possible this illness will worsen. I’d better preemptively rouse up energy.’… This is the seventh ground for arousing energy. 

Furthermore,\marginnote{2.1.115} a mendicant has recently recovered from illness. They think: ‘I’ve recently recovered from illness. It’s possible the illness will come back. I’d better preemptively rouse up energy for attaining the unattained, achieving the unachieved, and realizing the unrealized.’ They rouse energy for attaining the unattained, achieving the unachieved, and realizing the unrealized. This is the eighth ground for arousing energy. 

What\marginnote{2.1.121} eight things are hard to comprehend? Eight lost opportunities for spiritual practice. Firstly, a Realized One has arisen in the world. He teaches the Dhamma leading to peace, extinguishment, awakening, as proclaimed by the Holy One. But a person has been reborn in hell. This is the first lost opportunity for spiritual practice. 

Furthermore,\marginnote{2.1.126} a Realized One has arisen in the world. But a person has been reborn in the animal realm. This is the second lost opportunity for spiritual practice. 

Furthermore,\marginnote{2.1.129} a Realized One has arisen in the world. But a person has been reborn in the ghost realm. This is the third lost opportunity for spiritual practice. 

Furthermore,\marginnote{2.1.132} a Realized One has arisen in the world. But person has been reborn in one of the long-lived orders of gods. This is the fourth lost opportunity for spiritual practice. 

Furthermore,\marginnote{2.1.135} a Realized One has arisen in the world. But a person has been reborn in the borderlands, among strange barbarian tribes, where monks, nuns, laymen, and laywomen do not go. This is the fifth lost opportunity for spiritual practice. 

Furthermore,\marginnote{2.1.138} a Realized One has arisen in the world. And a person is reborn in a central country. But they have wrong view and distorted perspective: ‘There’s no meaning in giving, sacrifice, or offerings. There’s no fruit or result of good and bad deeds. There’s no afterlife. There are no duties to mother and father. No beings are reborn spontaneously. And there’s no ascetic or brahmin who is well attained and practiced, and who describes the afterlife after realizing it with their own insight.’ This is the sixth lost opportunity for spiritual practice. 

Furthermore,\marginnote{2.1.142} a Realized One has arisen in the world. And a person is reborn in a central country. But they’re witless, dull, stupid, and unable to distinguish what is well said from what is poorly said. This is the seventh lost opportunity for spiritual practice. 

Furthermore,\marginnote{2.1.145} a Realized One has arisen in the world. But he doesn’t teach the Dhamma leading to peace, extinguishment, awakening, as announced by the Holy One. And a person is reborn in a central country. And they’re wise, bright, clever, and able to distinguish what is well said from what is poorly said. This is the eighth lost opportunity for spiritual practice. 

What\marginnote{2.1.149} eight things should be produced? Eight thoughts of a great man. ‘This teaching is for those of few wishes, not those of many wishes. It’s for the contented, not those who lack contentment. It’s for the secluded, not those who enjoy company. It’s for the energetic, not the lazy. It’s for the mindful, not the unmindful. It’s for those with immersion, not those without immersion. It’s for the wise, not the witless. This teaching is for those who don’t enjoy proliferating, not for those who enjoy proliferating.’ 

What\marginnote{2.1.159} eight things should be directly known? Eight dimensions of mastery. Perceiving form internally, someone sees visions externally, limited, both pretty and ugly. Mastering them, they perceive: ‘I know and see.’ This is the first dimension of mastery. 

Perceiving\marginnote{2.1.164} form internally, someone sees visions externally, limitless, both pretty and ugly. Mastering them, they perceive: ‘I know and see.’ This is the second dimension of mastery. 

Not\marginnote{2.1.167} perceiving form internally, someone sees visions externally, limited, both pretty and ugly. Mastering them, they perceive: ‘I know and see.’ This is the third dimension of mastery. 

Not\marginnote{2.1.170} perceiving form internally, someone sees visions externally, limitless, both pretty and ugly. Mastering them, they perceive: ‘I know and see.’ This is the fourth dimension of mastery. 

Not\marginnote{2.1.173} perceiving form internally, someone sees visions externally that are blue, with blue color, blue hue, and blue tint. They’re like a flax flower that’s blue, with blue color, blue hue, and blue tint. Or a cloth from \textsanskrit{Bāraṇasī} that’s smoothed on both sides, blue, with blue color, blue hue, and blue tint. Mastering them, they perceive: ‘I know and see.’ This is the fifth dimension of mastery. 

Not\marginnote{2.1.177} perceiving form internally, someone sees visions externally that are yellow, with yellow color, yellow hue, and yellow tint. They’re like a champak flower that’s yellow, with yellow color, yellow hue, and yellow tint. Or a cloth from \textsanskrit{Bāraṇasī} that’s smoothed on both sides, yellow, with yellow color, yellow hue, and yellow tint. Mastering them, they perceive: ‘I know and see.’ This is the sixth dimension of mastery. 

Not\marginnote{2.1.181} perceiving form internally, someone sees visions externally that are red, with red color, red hue, and red tint. They’re like a scarlet mallow flower that’s red, with red color, red hue, and red tint. Or a cloth from \textsanskrit{Bāraṇasī} that’s smoothed on both sides, red, with red color, red hue, and red tint. Mastering them, they perceive: ‘I know and see.’ This is the seventh dimension of mastery. 

Not\marginnote{2.1.185} perceiving form internally, someone sees visions externally that are white, with white color, white hue, and white tint. They’re like the morning star that’s white, with white color, white hue, and white tint. Or a cloth from \textsanskrit{Bāraṇasī} that’s smoothed on both sides, white, with white color, white hue, and white tint. Mastering them, they perceive: ‘I know and see.’ This is the eighth dimension of mastery. 

What\marginnote{2.1.190} eight things should be realized? Eight liberations. Having physical form, they see visions. This is the first liberation. 

Not\marginnote{2.1.194} perceiving physical form internally, someone see visions externally. This is the second liberation. 

They’re\marginnote{2.1.196} focused only on beauty. This is the third liberation. 

Going\marginnote{2.1.198} totally beyond perceptions of form, with the ending of perceptions of impingement, not focusing on perceptions of diversity, aware that ‘space is infinite’, they enter and remain in the dimension of infinite space. This is the fourth liberation. 

Going\marginnote{2.1.200} totally beyond the dimension of infinite space, aware that ‘consciousness is infinite’, they enter and remain in the dimension of infinite consciousness. This is the fifth liberation. 

Going\marginnote{2.1.202} totally beyond the dimension of infinite consciousness, aware that ‘there is nothing at all’, they enter and remain in the dimension of nothingness. This is the sixth liberation. 

Going\marginnote{2.1.204} totally beyond the dimension of nothingness, they enter and remain in the dimension of neither perception nor non-perception. This is the seventh liberation. 

Going\marginnote{2.1.206} totally beyond the dimension of neither perception nor non-perception, they enter and remain in the cessation of perception and feeling. This is the eighth liberation. 

So\marginnote{2.1.209} these eighty things that are true, real, and accurate, not unreal, not otherwise were rightly awakened to by the Realized One. 

\section*{9. Groups of Nine }

Nine\marginnote{2.2.1} things are helpful, etc. 

What\marginnote{2.2.2} nine things are helpful? Nine things rooted in proper attention. When you attend properly, joy springs up. When you’re joyful, rapture springs up. When the mind is full of rapture, the body becomes tranquil. When the body is tranquil, you feel bliss. And when you’re blissful, the mind becomes immersed. When your mind is immersed, you truly know and see. When you truly know and see, you grow disillusioned. Being disillusioned, desire fades away. When desire fades away you’re freed. 

What\marginnote{2.2.5} nine things should be developed? Nine factors of trying to be pure. The factors of trying to be pure in ethics, mind, view, overcoming doubt, knowledge and vision of the variety of paths, knowledge and vision of the practice, knowledge and vision, wisdom, and freedom. 

What\marginnote{2.2.9} nine things should be completely understood? Nine abodes of sentient beings. There are sentient beings that are diverse in body and diverse in perception, such as human beings, some gods, and some beings in the underworld. This is the first abode of sentient beings. 

There\marginnote{2.2.13} are sentient beings that are diverse in body and unified in perception, such as the gods reborn in \textsanskrit{Brahmā}’s Host through the first absorption. This is the second abode of sentient beings. 

There\marginnote{2.2.15} are sentient beings that are unified in body and diverse in perception, such as the gods of streaming radiance. This is the third abode of sentient beings. 

There\marginnote{2.2.17} are sentient beings that are unified in body and unified in perception, such as the gods replete with glory. This is the fourth abode of sentient beings. 

There\marginnote{2.2.19} are sentient beings that are non-percipient and do not experience anything, such as the gods who are non-percipient beings. This is the fifth abode of sentient beings. 

There\marginnote{2.2.21} are sentient beings that have gone totally beyond perceptions of form. With the ending of perceptions of impingement, not focusing on perceptions of diversity, aware that ‘space is infinite’, they have been reborn in the dimension of infinite space. This is the sixth abode of sentient beings. 

There\marginnote{2.2.23} are sentient beings that have gone totally beyond the dimension of infinite space. Aware that ‘consciousness is infinite’, they have been reborn in the dimension of infinite consciousness. This is the seventh abode of sentient beings. 

There\marginnote{2.2.25} are sentient beings that have gone totally beyond the dimension of infinite consciousness. Aware that ‘there is nothing at all’, they have been reborn in the dimension of nothingness. This is the eighth abode of sentient beings. 

There\marginnote{2.2.27} are sentient beings that have gone totally beyond the dimension of nothingness. They have been reborn in the dimension of neither perception nor non-perception. This is the ninth abode of sentient beings. 

What\marginnote{2.2.30} nine things should be given up? Nine things rooted in craving. Craving is a cause of seeking. Seeking is a cause of gaining material possessions. Gaining material possessions is a cause of assessing. Assessing is a cause of desire and lust. Desire and lust is a cause of attachment. Attachment is a cause of ownership. Ownership is a cause of stinginess. Stinginess is a cause of safeguarding. Owing to safeguarding, many bad, unskillful things come to be: taking up the rod and the sword, quarrels, arguments, disputes, accusations, divisive speech, and lies. 

What\marginnote{2.2.34} nine things make things worse? Nine grounds for resentment. Thinking: ‘They did wrong to me,’ you harbor resentment. Thinking: ‘They are doing wrong to me’ … ‘They will do wrong to me’ … ‘They did wrong by someone I love’ … ‘They are doing wrong by someone I love’ … ‘They will do wrong by someone I love’ … ‘They helped someone I dislike’ … ‘They are helping someone I dislike’ … Thinking: ‘They will help someone I dislike,’ you harbor resentment. 

What\marginnote{2.2.46} nine things lead to distinction? Nine methods to get rid of resentment. Thinking: ‘They did wrong to me, but what can I possibly do?’ you get rid of resentment. Thinking: ‘They are doing wrong to me …’ … ‘They will do wrong to me …’ … ‘They did wrong by someone I love …’ … ‘They are doing wrong by someone I love …’ … ‘They will do wrong by someone I love …’ … ‘They helped someone I dislike …’ … ‘They are helping someone I dislike …’ … Thinking: ‘They will help someone I dislike, but what can I possibly do?’ you get rid of resentment. 

What\marginnote{2.2.58} nine things are hard to comprehend? Nine kinds of diversity. Diversity of elements gives rise to diversity of contacts. Diversity of contacts gives rise to diversity of feelings. Diversity of feelings gives rise to diversity of perceptions. Diversity of perceptions gives rise to diversity of thoughts. Diversity of thoughts gives rise to diversity of desires. Diversity of desires gives rise to diversity of passions. Diversity of passions gives rise to diversity of searches. Diversity of searches gives rise to diversity of gains. 

What\marginnote{2.2.62} nine things should be produced? Nine perceptions: the perceptions of ugliness, death, repulsiveness in food, dissatisfaction with the whole world, impermanence, suffering in impermanence, not-self in suffering, giving up, and fading away. 

What\marginnote{2.2.66} nine things should be directly known? Nine progressive meditations. A mendicant, quite secluded from sensual pleasures, secluded from unskillful qualities, enters and remains in the first absorption … second absorption … third absorption … fourth absorption. Going totally beyond perceptions of form, with the ending of perceptions of impingement, not focusing on perceptions of diversity, aware that ‘space is infinite’, they enter and remain in the dimension of infinite space. Going totally beyond the dimension of infinite space, aware that ‘consciousness is infinite’, they enter and remain in the dimension of infinite consciousness. Going totally beyond the dimension of infinite consciousness, aware that ‘there is nothing at all’, they enter and remain in the dimension of nothingness. Going totally beyond the dimension of nothingness, they enter and remain in the dimension of neither perception nor non-perception. Going totally beyond the dimension of neither perception nor non-perception, they enter and remain in the cessation of perception and feeling. 

What\marginnote{2.2.78} nine things should be realized? Nine progressive cessations. For someone who has attained the first absorption, sensual perceptions have ceased. For someone who has attained the second absorption, the placing of the mind and keeping it connected have ceased. For someone who has attained the third absorption, rapture has ceased. For someone who has attained the fourth absorption, breathing has ceased. For someone who has attained the dimension of infinite space, the perception of form has ceased. For someone who has attained the dimension of infinite consciousness, the perception of the dimension of infinite space has ceased. For someone who has attained the dimension of nothingness, the perception of the dimension of infinite consciousness has ceased. For someone who has attained the dimension of neither perception nor non-perception, the perception of the dimension of nothingness has ceased. For someone who has attained the cessation of perception and feeling, perception and feeling have ceased. 

So\marginnote{2.2.90} these ninety things that are true, real, and accurate, not unreal, not otherwise were rightly awakened to by the Realized One. 

\section*{10. Groups of Ten }

Ten\marginnote{2.3.1} things are helpful, ten things should be developed, ten things should be completely understood, ten things should be given up, ten things make things worse, ten things lead to distinction, ten things are hard to comprehend, ten things should be produced, ten things should be directly known, ten things should be realized. 

What\marginnote{2.3.2} ten things are helpful? Ten qualities that serve as protector. First, a mendicant is ethical, restrained in the monastic code, conducting themselves well and seeking alms in suitable places. Seeing danger in the slightest fault, they keep the rules they’ve undertaken. This is a quality that serves as protector. 

Furthermore,\marginnote{2.3.7} a mendicant is learned. This too is a quality that serves as protector. 

Furthermore,\marginnote{2.3.10} a mendicant has good friends, companions, and associates. This too is a quality that serves as protector. 

Furthermore,\marginnote{2.3.13} a mendicant is easy to admonish, having qualities that make them easy to admonish. They’re patient, and take instruction respectfully. This too is a quality that serves as protector. 

Furthermore,\marginnote{2.3.16} a mendicant is deft and tireless in a diverse spectrum of duties for their spiritual companions, understanding how to go about things in order to complete and organize the work. This too is a quality that serves as protector. 

Furthermore,\marginnote{2.3.19} a mendicant loves the teachings and is a delight to converse with, being full of joy in the teaching and training. This too is a quality that serves as protector. 

Furthermore,\marginnote{2.3.22} a mendicant is content with any kind of robes, almsfood, lodgings, and medicines and supplies for the sick. This too is a quality that serves as protector. 

Furthermore,\marginnote{2.3.25} a mendicant is energetic. This too is a quality that serves as protector. 

Furthermore,\marginnote{2.3.28} a mendicant is mindful. They have utmost mindfulness and alertness, and can remember and recall what was said and done long ago. This too is a quality that serves as protector. 

Furthermore,\marginnote{2.3.31} a mendicant is wise. They have the wisdom of arising and passing away which is noble, penetrative, and leads to the complete ending of suffering. This too is a quality that serves as protector. 

What\marginnote{2.3.35} ten things should be developed? Ten universal dimensions of meditation. Someone perceives the meditation on universal earth above, below, across, non-dual and limitless. They perceive the meditation on universal water … the meditation on universal fire … the meditation on universal air … the meditation on universal blue … the meditation on universal yellow … the meditation on universal red … the meditation on universal white … the meditation on universal space … They perceive the meditation on universal consciousness above, below, across, non-dual and limitless. 

What\marginnote{2.3.48} ten things should be completely understood? Ten sense fields: eye and sights, ear and sounds, nose and smells, tongue and tastes, body and touches. 

What\marginnote{2.3.52} ten things should be given up? Ten wrong ways: wrong view, wrong thought, wrong speech, wrong action, wrong livelihood, wrong effort, wrong mindfulness, wrong immersion, wrong knowledge, and wrong freedom. 

What\marginnote{2.3.56} ten things make things worse? Ten ways of doing unskillful deeds: killing living creatures, stealing, and sexual misconduct; speech that’s false, divisive, harsh, or nonsensical; covetousness, ill will, and wrong view. 

What\marginnote{2.3.60} ten things lead to distinction? Ten ways of doing skillful deeds: refraining from killing living creatures, stealing, and sexual misconduct; avoiding speech that’s false, divisive, harsh, or nonsensical; contentment, good will, and right view. 

What\marginnote{2.3.64} ten things are hard to comprehend? Ten noble abodes. A mendicant has given up five factors, possesses six factors, has a single guard, has four supports, has eliminated idiosyncratic interpretations of the truth, has totally given up searching, has unsullied intentions, has stilled the physical process, and is well freed in mind and well freed by wisdom. 

And\marginnote{2.3.67} how has a mendicant given up five factors? It’s when a mendicant has given up sensual desire, ill will, dullness and drowsiness, restlessness and remorse, and doubt. That’s how a mendicant has given up five factors. 

And\marginnote{2.3.70} how does a mendicant possess six factors? A mendicant, seeing a sight with their eyes, is neither happy nor sad. They remain equanimous, mindful and aware. Hearing a sound with their ears … Smelling an odor with their nose … Tasting a flavor with their tongue … 

Feeling\marginnote{2.3.75} a touch with their body … Knowing a thought with their mind, they’re neither happy nor sad. They remain equanimous, mindful and aware. That’s how a mendicant possesses six factors. 

And\marginnote{2.3.78} how does a mendicant have a single guard? It’s when a mendicant’s heart is guarded by mindfulness. That’s how a mendicant has a single guard. 

And\marginnote{2.3.81} how does a mendicant have four supports? After appraisal, a mendicant uses some things, endures some things, avoids some things, and gets rid of some things. That’s how a mendicant has four supports. 

And\marginnote{2.3.84} how has a mendicant eliminated idiosyncratic interpretations of the truth? Different ascetics and brahmins have different idiosyncratic interpretations of the truth. A mendicant has dispelled, eliminated, thrown out, rejected, let go of, given up, and relinquished all these. That’s how a mendicant has eliminated idiosyncratic interpretations of the truth. 

And\marginnote{2.3.87} how has a mendicant totally given up searching? It’s when they’ve given up searching for sensual pleasures, for continued existence, and for a spiritual path. That’s how a mendicant has totally given up searching. 

And\marginnote{2.3.90} how does a mendicant have unsullied intentions? It’s when they’ve given up sensual, malicious, and cruel intentions. That’s how a mendicant has unsullied intentions. 

And\marginnote{2.3.93} how has a mendicant stilled the physical process? Giving up pleasure and pain, and ending former happiness and sadness, they enter and remain in the fourth absorption, without pleasure or pain, with pure equanimity and mindfulness. That’s how a mendicant has stilled the physical process. 

And\marginnote{2.3.96} how is a mendicant well freed in mind? It’s when a mendicant’s mind is freed from greed, hate, and delusion. That’s how a mendicant is well freed in mind. 

And\marginnote{2.3.99} how is a mendicant well freed by wisdom? It’s when a mendicant understands: ‘I’ve given up greed, hate, and delusion, cut them off at the root, made them like a palm stump, obliterated them, so they’re unable to arise in the future.’ That’s how a mendicant’s mind is well freed by wisdom. 

What\marginnote{2.3.107} ten things should be produced? Ten perceptions: the perceptions of ugliness, death, repulsiveness in food, dissatisfaction with the whole world, impermanence, suffering in impermanence, not-self in suffering, giving up, fading away, and cessation. 

What\marginnote{2.3.111} ten things should be directly known? Ten grounds for wearing away. For one of right view, wrong view is worn away. And the many bad, unskillful qualities that arise because of wrong view are worn away. For one of right intention, wrong intention is worn away. … For one of right speech, wrong speech is worn away. … For one of right action, wrong action is worn away. … For one of right livelihood, wrong livelihood is worn away. … For one of right effort, wrong effort is worn away. … For one of right mindfulness, wrong mindfulness is worn away. … For one of right immersion, wrong immersion is worn away. … For one of right knowledge, wrong knowledge is worn away. … For one of right freedom, wrong freedom is worn away. And the many bad, unskillful qualities that arise because of wrong freedom are worn away. 

What\marginnote{2.3.126} ten things should be realized? Ten qualities of an adept: an adept’s right view, right thought, right speech, right action, right livelihood, right effort, right mindfulness, right immersion, right knowledge, and right freedom. 

So\marginnote{2.3.130} these hundred things that are true, real, and accurate, not unreal, not otherwise were rightly awakened to by the Realized One.” This is what Venerable \textsanskrit{Sāriputta} said. Satisfied, the mendicants were happy with what \textsanskrit{Sāriputta} said. 

\scendbook{The Long Discourses are completed. }

%
\backmatter%
\chapter*{Colophon}
\addcontentsline{toc}{chapter}{Colophon}
\markboth{Colophon}{Colophon}

\section*{The Translator}

Bhikkhu Sujato was born as Anthony Aidan Best on 4/11/1966 in Perth, Western Australia. He grew up in the pleasant suburbs of Mt Lawley and Attadale alongside his sister Nicola, who was the good child. His mother, Margaret Lorraine Huntsman née Pinder, said “he’ll either be a priest or a poet”, while his father, Anthony Thomas Best, advised him to “never do anything for money”. He attended Aquinas College, a Catholic school, where he decided to become an atheist. At the University of WA he studied philosophy, aiming to learn what he wanted to do with his life. Finding that what he wanted to do was play guitar, he dropped out. His main band was named Martha’s Vineyard, which achieved modest success in the indie circuit. Then it broke up, because everyone thought they personally were reason for the success, which, oddly enough, turns out not to have been the case. 

A seemingly random encounter with a roadside joey took him to Thailand, where he entered his first meditation retreat at Wat Ram Poeng, Chieng Mai in 1992. He decided to devote himself to the Buddha’s path, and took full ordination in Wat Pa Nanachat in 1994, where his teachers were Ajahn Pasanno and Ajahn Jayasaro. In 1997 he returned to Perth to study with Ajahn Brahm at Bodhinyana Monastery. 

He spent several years practicing in seclusion in Malaysia and Thailand before establishing Santi Forest Monastery in Bundanoon, NSW, in 2003. There he was instrumental in supporting the establishment of the Theravada bhikkhuni order in Australia and advocating for women’s rights. He continues to teach in Australia and globally, with a special concern for the moral implications of climate change and other forms of environmental destruction. He has published a series of books of original and groundbreaking research on early Buddhism. 

In 2005 he founded SuttaCentral together with Rod Bucknell and John Kelly. In 2015, seeing the need for a complete, accurate, plain English translation of the Pali texts, he undertook the task, spending nearly three years in isolation on the isle of Qi Mei off the coast of the nation of Taiwan. He completed the four main \textsanskrit{Nikāyas} in 2018, and the early books of the Khuddaka \textsanskrit{Nikāya} were complete by 2021. All this work is dedicated to the public domain and is entirely free of copyright encumbrance. 

In 2019 he returned to Sydney where, together with Bhikkhu Akaliko, he established Lokanta Vihara (The Monastery at the End of the World). 

\section*{Creation Process}

Primary source was the digital \textsanskrit{Mahāsaṅgīti} edition of the Pali \textsanskrit{Tipiṭaka}. Translated from the Pali, with reference to several English translations, especially those of Bhikkhu Bodhi. Older translations by Maurice Walshe and T.W. and C.A.F. Rhys Davids were also consulted.

\section*{The Translation}

This translation was part of a project to translate the four Pali \textsanskrit{Nikāyas} with the following aims: plain, approachable English; consistent terminology; accurate rendition of the Pali; free of copyright. It was made during 2016–2018 while Bhikkhu Sujato was staying in Qimei, Taiwan.

\section*{About SuttaCentral}

SuttaCentral publishes early Buddhist texts. Since 2005 we have provided root texts in Pali, Chinese, Sanskrit, Tibetan, and other languages, parallels between these texts, and translations in many modern languages. We build on the work of generations of scholars, and offer our contribution freely.

SuttaCentral is driven by volunteer contributions, and in addition we employ professional developers. We offer a sponsorship program for high quality translations from the original languages. Financial support for SuttaCentral is handled by the SuttaCentral Development Trust, a charitable trust registered in Australia.

\section*{About Bilara}

“Bilara” means “cat” in Pali, and it is the name of our Computer Assisted Translation (CAT) software. Bilara is a web app that enables translators to translate early Buddhist texts into their own language. These translations are published on SuttaCentral with the root text and translation side by side.

\section*{About SuttaCentral Editions}

The SuttaCentral Editions project makes high quality books from selected Bilara translations. These are published in formats including HTML, EPUB, PDF, and print.

If you want to print any of our Editions, please let us know and we will help prepare a file to your specifications.

%
\end{document}