\documentclass[12pt,openany]{book}%
\usepackage{lastpage}%
%
\usepackage[inner=1in, outer=1in, top=.7in, bottom=1in, papersize={6in,9in}, headheight=13pt]{geometry}
\usepackage{polyglossia}
\usepackage[12pt]{moresize}
\usepackage{soul}%
\usepackage{microtype}
\usepackage{tocbasic}
\usepackage{realscripts}
\usepackage{epigraph}%
\usepackage{setspace}%
\usepackage{sectsty}
\usepackage{fontspec}
\usepackage{marginnote}
\usepackage[bottom]{footmisc}
\usepackage{enumitem}
\usepackage{fancyhdr}
\usepackage{extramarks}
\usepackage{graphicx}
\usepackage{verse}
\usepackage{relsize}
\usepackage{etoolbox}
\usepackage[a-3u]{pdfx}

\hypersetup{
colorlinks=true,
urlcolor=black,
linkcolor=black,
citecolor=black
}

% use a small amount of tracking on small caps
\SetTracking[ spacing = {25*,166, } ]{ encoding = *, shape = sc }{ 25 }

% add a blank page
\newcommand{\blankpage}{
\newpage
\thispagestyle{empty}
\mbox{}
\newpage
}

% define languages
\setdefaultlanguage[]{english}
\setotherlanguage[script=Latin]{sanskrit}

%\usepackage{pagegrid}
%\pagegridsetup{top-left, step=.25in}

% define fonts
% use if arno sanskrit is unavailable
%\setmainfont{Gentium Plus}
%\newfontfamily\Semiboldsubheadfont[]{Gentium Plus}
%\newfontfamily\Semiboldnormalfont[]{Gentium Plus}
%\newfontfamily\Lightfont[]{Gentium Plus}
%\newfontfamily\Marginalfont[]{Gentium Plus}
%\newfontfamily\Allsmallcapsfont[RawFeature=+c2sc]{Gentium Plus}
%\newfontfamily\Noligaturefont[Renderer=Basic]{Gentium Plus}
%\newfontfamily\Noligaturecaptionfont[Renderer=Basic]{Gentium Plus}
%\newfontfamily\Fleuronfont[Ornament=1]{Gentium Plus}

% use if arno sanskrit is available. display is applied to \chapter and \part, subhead to \section and \subsection. When specifying semibold, the italic must be defined.
\setmainfont[Numbers=OldStyle]{Arno Pro}
\newfontfamily\Semibolddisplayfont[BoldItalicFont = Arno Pro Semibold Italic Display]{Arno Pro Semibold Display} %
\newfontfamily\Semiboldsubheadfont[BoldItalicFont = Arno Pro Semibold Italic Subhead]{Arno Pro Semibold Subhead}
\newfontfamily\Semiboldnormalfont[BoldItalicFont = Arno Pro Semibold Italic]{Arno Pro Semibold}
\newfontfamily\Marginalfont[RawFeature=+subs]{Arno Pro Regular}
\newfontfamily\Allsmallcapsfont[RawFeature=+c2sc]{Arno Pro}
\newfontfamily\Noligaturefont[Renderer=Basic]{Arno Pro}
\newfontfamily\Noligaturecaptionfont[Renderer=Basic]{Arno Pro Caption}

% chinese fonts
\newfontfamily\cjk{Noto Serif TC}
\newcommand*{\langlzh}[1]{\cjk{#1}\normalfont}%

% logo
\newfontfamily\Logofont{sclogo.ttf}
\newcommand*{\sclogo}[1]{\large\Logofont{#1}}

% use subscript numerals for margin notes
\renewcommand*{\marginfont}{\Marginalfont}

% ensure margin notes have consistent vertical alignment
\renewcommand*{\marginnotevadjust}{-.17em}

% use compact lists
\setitemize{noitemsep,leftmargin=1em}
\setenumerate{noitemsep,leftmargin=1em}
\setdescription{noitemsep, style=unboxed, leftmargin=0em}

% style ToC
\DeclareTOCStyleEntries[
  raggedentrytext,
  linefill=\hfill,
  pagenumberwidth=.5in,
  pagenumberformat=\normalfont,
  entryformat=\normalfont
]{tocline}{chapter,section}


  \setlength\topsep{0pt}%
  \setlength\parskip{0pt}%

% define new \centerpars command for use in ToC. This ensures centering, proper wrapping, and no page break after
\def\startcenter{%
  \par
  \begingroup
  \leftskip=0pt plus 1fil
  \rightskip=\leftskip
  \parindent=0pt
  \parfillskip=0pt
}
\def\stopcenter{%
  \par
  \endgroup
}
\long\def\centerpars#1{\startcenter#1\stopcenter}

% redefine part, so that it adds a toc entry without page number
\let\oldcontentsline\contentsline
\newcommand{\nopagecontentsline}[3]{\oldcontentsline{#1}{#2}{}}

    \makeatletter
\renewcommand*\l@part[2]{%
  \ifnum \c@tocdepth >-2\relax
    \addpenalty{-\@highpenalty}%
    \addvspace{0em \@plus\p@}%
    \setlength\@tempdima{3em}%
    \begingroup
      \parindent \z@ \rightskip \@pnumwidth
      \parfillskip -\@pnumwidth
      {\leavevmode
       \setstretch{.85}\large\scshape\centerpars{#1}\vspace*{-1em}\llap{#2}}\par
       \nobreak
         \global\@nobreaktrue
         \everypar{\global\@nobreakfalse\everypar{}}%
    \endgroup
  \fi}
\makeatother

\makeatletter
\def\@pnumwidth{2em}
\makeatother

% define new sectioning command, which is only used in volumes where the pannasa is found in some parts but not others, especially in an and sn

\newcommand*{\pannasa}[1]{\clearpage\thispagestyle{empty}\begin{center}\vspace*{14em}\setstretch{.85}\huge\itshape\scshape\MakeLowercase{#1}\end{center}}

    \makeatletter
\newcommand*\l@pannasa[2]{%
  \ifnum \c@tocdepth >-2\relax
    \addpenalty{-\@highpenalty}%
    \addvspace{.5em \@plus\p@}%
    \setlength\@tempdima{3em}%
    \begingroup
      \parindent \z@ \rightskip \@pnumwidth
      \parfillskip -\@pnumwidth
      {\leavevmode
       \setstretch{.85}\large\itshape\scshape\lowercase{\centerpars{#1}}\vspace*{-1em}\llap{#2}}\par
       \nobreak
         \global\@nobreaktrue
         \everypar{\global\@nobreakfalse\everypar{}}%
    \endgroup
  \fi}
\makeatother

% don't put page number on first page of toc (relies on etoolbox)
\patchcmd{\chapter}{plain}{empty}{}{}

% global line height
\setstretch{1.05}

% allow linebreak after em-dash
\catcode`\—=13
\protected\def—{\unskip\textemdash\allowbreak}

% style headings with secsty. chapter and section are defined per-edition
\partfont{\setstretch{.85}\normalfont\centering\textsc}
\subsectionfont{\setstretch{.85}\Semiboldsubheadfont}%
\subsubsectionfont{\setstretch{.85}\Semiboldnormalfont}

% style elements of suttatitle
\newcommand*{\suttatitleacronym}[1]{\smaller[2]{#1}\vspace*{.3em}}
\newcommand*{\suttatitletranslation}[1]{\linebreak{#1}}
\newcommand*{\suttatitleroot}[1]{\linebreak\smaller[2]\itshape{#1}}

\DeclareTOCStyleEntries[
  indent=3.3em,
  dynindent,
  beforeskip=.2em plus -2pt minus -1pt,
]{tocline}{section}

\DeclareTOCStyleEntries[
  indent=0em,
  dynindent,
  beforeskip=.4em plus -2pt minus -1pt,
]{tocline}{chapter}

\newcommand*{\tocacronym}[1]{\hspace*{-3.3em}{#1}\quad}
\newcommand*{\toctranslation}[1]{#1}
\newcommand*{\tocroot}[1]{(\textit{#1})}
\newcommand*{\tocchapterline}[1]{\bfseries\itshape{#1}}


% redefine paragraph and subparagraph headings to not be inline
\makeatletter
% Change the style of paragraph headings %
\renewcommand\paragraph{\@startsection{paragraph}{4}{\z@}%
            {-2.5ex\@plus -1ex \@minus -.25ex}%
            {1.25ex \@plus .25ex}%
            {\noindent\Semiboldnormalfont\normalsize}}

% Change the style of subparagraph headings %
\renewcommand\subparagraph{\@startsection{subparagraph}{5}{\z@}%
            {-2.5ex\@plus -1ex \@minus -.25ex}%
            {1.25ex \@plus .25ex}%
            {\noindent\Semiboldnormalfont\small}}
\makeatother

% use etoolbox to suppress page numbers on \part
\patchcmd{\part}{\thispagestyle{plain}}{\thispagestyle{empty}}
  {}{\errmessage{Cannot patch \string\part}}

% and to reduce margins on quotation
\patchcmd{\quotation}{\rightmargin}{\leftmargin 1.2em \rightmargin}{}{}
\AtBeginEnvironment{quotation}{\small}

% titlepage
\newcommand*{\titlepageTranslationTitle}[1]{{\begin{center}\begin{large}{#1}\end{large}\end{center}}}
\newcommand*{\titlepageCreatorName}[1]{{\begin{center}\begin{normalsize}{#1}\end{normalsize}\end{center}}}

% halftitlepage
\newcommand*{\halftitlepageTranslationTitle}[1]{\setstretch{2.5}{\begin{Huge}\uppercase{\so{#1}}\end{Huge}}}
\newcommand*{\halftitlepageTranslationSubtitle}[1]{\setstretch{1.2}{\begin{large}{#1}\end{large}}}
\newcommand*{\halftitlepageFleuron}[1]{{\begin{large}\Fleuronfont{{#1}}\end{large}}}
\newcommand*{\halftitlepageByline}[1]{{\begin{normalsize}\textit{{#1}}\end{normalsize}}}
\newcommand*{\halftitlepageCreatorName}[1]{{\begin{LARGE}{\textsc{#1}}\end{LARGE}}}
\newcommand*{\halftitlepageVolumeNumber}[1]{{\begin{normalsize}{\Allsmallcapsfont{\textsc{#1}}}\end{normalsize}}}
\newcommand*{\halftitlepageVolumeAcronym}[1]{{\begin{normalsize}{#1}\end{normalsize}}}
\newcommand*{\halftitlepageVolumeTranslationTitle}[1]{{\begin{Large}{\textsc{#1}}\end{Large}}}
\newcommand*{\halftitlepageVolumeRootTitle}[1]{{\begin{normalsize}{\Allsmallcapsfont{\textsc{\itshape #1}}}\end{normalsize}}}
\newcommand*{\halftitlepagePublisher}[1]{{\begin{large}{\Noligaturecaptionfont\textsc{#1}}\end{large}}}

% epigraph
\renewcommand{\epigraphflush}{center}
\renewcommand*{\epigraphwidth}{.85\textwidth}
\newcommand*{\epigraphTranslatedTitle}[1]{\vspace*{.5em}\footnotesize\textsc{#1}\\}%
\newcommand*{\epigraphRootTitle}[1]{\footnotesize\textit{#1}\\}%
\newcommand*{\epigraphReference}[1]{\footnotesize{#1}}%

% custom commands for html styling classes
\newcommand*{\scnamo}[1]{\begin{center}\textit{#1}\end{center}}
\newcommand*{\scendsection}[1]{\begin{center}\textit{#1}\end{center}}
\newcommand*{\scendsutta}[1]{\begin{center}\textit{#1}\end{center}}
\newcommand*{\scendbook}[1]{\begin{center}\uppercase{#1}\end{center}}
\newcommand*{\scendkanda}[1]{\begin{center}\textbf{#1}\end{center}}
\newcommand*{\scend}[1]{\begin{center}\textit{#1}\end{center}}
\newcommand*{\scuddanaintro}[1]{\textit{#1}}
\newcommand*{\scendvagga}[1]{\begin{center}\textbf{#1}\end{center}}
\newcommand*{\scrule}[1]{\textbf{#1}}
\newcommand*{\scadd}[1]{\textit{#1}}
\newcommand*{\scevam}[1]{\textsc{#1}}
\newcommand*{\scspeaker}[1]{\hspace{2em}\textit{#1}}
\newcommand*{\scbyline}[1]{\begin{flushright}\textit{#1}\end{flushright}\bigskip}

% custom command for thematic break = hr
\newcommand*{\thematicbreak}{\begin{center}\rule[.5ex]{6em}{.4pt}\begin{normalsize}\quad\Fleuronfont{•}\quad\end{normalsize}\rule[.5ex]{6em}{.4pt}\end{center}}

% manage and style page header and footer. "fancy" has header and footer, "plain" has footer only

\pagestyle{fancy}
\fancyhf{}
\fancyfoot[RE,LO]{\thepage}
\fancyfoot[LE,RO]{\footnotesize\lastleftxmark}
\fancyhead[CE]{\setstretch{.85}\Noligaturefont\MakeLowercase{\textsc{\firstrightmark}}}
\fancyhead[CO]{\setstretch{.85}\Noligaturefont\MakeLowercase{\textsc{\firstleftmark}}}
\renewcommand{\headrulewidth}{0pt}
\fancypagestyle{plain}{ %
\fancyhf{} % remove everything
\fancyfoot[RE,LO]{\thepage}
\fancyfoot[LE,RO]{\footnotesize\lastleftxmark}
\renewcommand{\headrulewidth}{0pt}
\renewcommand{\footrulewidth}{0pt}}

% style footnotes
\setlength{\skip\footins}{1em}

\makeatletter
\newcommand{\@makefntextcustom}[1]{%
    \parindent 0em%
    \thefootnote.\enskip #1%
}
\renewcommand{\@makefntext}[1]{\@makefntextcustom{#1}}
\makeatother

% hang quotes (requires microtype)
\microtypesetup{
  protrusion = true,
  expansion  = true,
  tracking   = true,
  factor     = 1000,
  patch      = all,
  final
}

% Custom protrusion rules to allow hanging punctuation
\SetProtrusion
{ encoding = *}
{
% char   right left
  {-} = {    , 500 },
  % Double Quotes
  \textquotedblleft
      = {1000,     },
  \textquotedblright
      = {    , 1000},
  \quotedblbase
      = {1000,     },
  % Single Quotes
  \textquoteleft
      = {1000,     },
  \textquoteright
      = {    , 1000},
  \quotesinglbase
      = {1000,     }
}

% make latex use actual font em for parindent, not Computer Modern Roman
\AtBeginDocument{\setlength{\parindent}{1em}}%
%

% Default values; a bit sloppier than normal
\tolerance 1414
\hbadness 1414
\emergencystretch 1.5em
\hfuzz 0.3pt
\clubpenalty = 10000
\widowpenalty = 10000
\displaywidowpenalty = 10000
\hfuzz \vfuzz
 \raggedbottom%

\title{Sayings of the Dhamma}
\author{Bhikkhu Sujato}
\date{}%
% define a different fleuron for each edition
\newfontfamily\Fleuronfont[Ornament=6]{Arno Pro}

% Define heading styles per edition for chapter and section. Suttatitle can be either of these, depending on the volume. 

\let\oldfrontmatter\frontmatter
\renewcommand{\frontmatter}{%
\chapterfont{\setstretch{.85}\normalfont\centering}%
\sectionfont{\setstretch{.85}\Semiboldsubheadfont}%
\oldfrontmatter}

\let\oldmainmatter\mainmatter
\renewcommand{\mainmatter}{%
\chapterfont{\setstretch{.85}\normalfont\centering\Large}%
\sectionfont{\setstretch{.85}\Semiboldsubheadfont}%
\oldmainmatter}

\let\oldbackmatter\backmatter
\renewcommand{\backmatter}{%
\chapterfont{\setstretch{.85}\normalfont\centering}%
\sectionfont{\setstretch{.85}\Semiboldsubheadfont}%
\oldbackmatter}
%
%
\begin{document}%
\normalsize%
\frontmatter%
\setlength{\parindent}{0cm}

\pagestyle{empty}

\maketitle

\blankpage%
\begin{center}

\vspace*{2.2em}

\halftitlepageTranslationTitle{Sayings of the Dhamma}

\vspace*{1em}

\halftitlepageTranslationSubtitle{A meaningful translation of the Dhammapada}

\vspace*{2em}

\halftitlepageFleuron{•}

\vspace*{2em}

\halftitlepageByline{translated and introduced by}

\vspace*{.5em}

\halftitlepageCreatorName{Bhikkhu Sujato}

\vspace*{4em}

\halftitlepageVolumeNumber{}

\smallskip

\halftitlepageVolumeAcronym{Dhp}

\smallskip

\halftitlepageVolumeTranslationTitle{}

\smallskip

\halftitlepageVolumeRootTitle{}

\vspace*{\fill}

\sclogo{0}
 \halftitlepagePublisher{SuttaCentral}

\end{center}

\newpage
%
\setstretch{1.05}

\begin{footnotesize}

\textit{Sayings of the Dhamma} is a translation of the Dhammapada by Bhikkhu Sujato.

\medskip

Creative Commons Zero (CC0)

To the extent possible under law, Bhikkhu Sujato has waived all copyright and related or neighboring rights to \textit{Sayings of the Dhamma}.

\medskip

This work is published from Australia.

\begin{center}
\textit{This translation is an expression of an ancient spiritual text that has been passed down by the Buddhist tradition for the benefit of all sentient beings. It is dedicated to the public domain via Creative Commons Zero (CC0). You are encouraged to copy, reproduce, adapt, alter, or otherwise make use of this translation. The translator respectfully requests that any use be in accordance with the values and principles of the Buddhist community.}
\end{center}

\medskip

\begin{description}
    \item[Web publication date] 2021
    \item[This edition] 2022-11-03 09:25:25
    \item[Publication type] paperback
    \item[Edition] ed5
    \item[Number of volumes] 1
    \item[Publication ISBN] 978-1-76132-071-2
    \item[Publication URL] https://suttacentral.net/editions/dhp/en/sujato
    \item[Source URL] https://github.com/suttacentral/bilara-data/tree/published/translation/en/sujato/sutta/kn/dhp
    \item[Publication number] scpub7
\end{description}

\medskip

Published by SuttaCentral

\medskip

\textit{SuttaCentral,\\
c/o Alwis \& Alwis Pty Ltd\\
Kaurna Country,\\
Suite 12,\\
198 Greenhill Road,\\
Eastwood,\\
SA 5063,\\
Australia}

\end{footnotesize}

\newpage

\setlength{\parindent}{1.5em}%%
\newpage

\vspace*{\fill}

\begin{center}
\epigraph{Much though they may recite scripture,\\
if a negligent person does not apply them,\\
then, like a cowherd who counts the cattle of others,\\
they miss out on the blessings of the ascetic life.}
{
\epigraphTranslatedTitle{}
\epigraphRootTitle{}
\epigraphReference{Dhammapada 19}
}
\end{center}

\vspace*{2in}

\vspace*{\fill}

\blankpage%

\setlength{\parindent}{1em}
%
\tableofcontents
\newpage
\pagestyle{fancy}
%
\chapter*{The SuttaCentral Editions Series}
\addcontentsline{toc}{chapter}{The SuttaCentral Editions Series}
\markboth{The SuttaCentral Editions Series}{The SuttaCentral Editions Series}

Since 2005 SuttaCentral has provided access to the texts, translations, and parallels of early Buddhist texts. In 2018 we started creating and publishing our own translations of these seminal spiritual classics. The “Editions” series now makes selected translations available as books in various forms, including print, PDF, and EPUB.

Editions are selected from our most complete, well-crafted, and reliable translations. They aim to bring these texts to a wider audience in forms that reward mindful reading. Care is taken with every detail of the production, and we aim to meet or exceed professional best standards in every way. These are the core scriptures underlying the entire Buddhist tradition, and we believe that they deserve to be preserved and made available in highest quality without compromise.

SuttaCentral is a charitable organization. Our work is accomplished by volunteers and through the generosity of our donors. Everything we create is offered to all of humanity free of any copyright or licensing restrictions. 

%
\chapter*{Preface}
\addcontentsline{toc}{chapter}{Preface}
\markboth{Preface}{Preface}

In 2009 Brian Ashen, the then-president of the Federation of Australia Buddhist Councils, toured the Federal Parliament of Australia in Canberra. The tour was shown the Despatch Box, which contained sacred scriptures when required for oath-taking. This box, which sits in a place of honor before the Prime Minister, contained a Bible and a Quran.

Brian raised this with the FABC and we agreed to propose a Buddhist scripture to be placed alongside the scriptures of Christianity and Islam. Unfortunately, the \textsanskrit{Tipiṭaka} is large, even more so if we consider the canons of all schools. So we needed to suggest a single text that would well represent all the Buddhist traditions.

I proposed the Dhammapada. Here is an edited excerpt from my proposal.

\begin{quotation}%
The Dhammapada is one of the ancient texts, spoken, as far as we can tell, largely by the historical Buddha, and organized and edited by the Sangha of old. We cannot know, of course, that all of the verses were spoken as we have them by the Buddha himself, and indeed several of them share things in common with Jain or Brahmanical verses. Nevertheless, as a historical scholar I feel that the teachings found there are very likely to represent the actual teachings of Siddhattha.

The Dhammapada is not a sectarian document. It is true that the best-known version, which I have proposed for inclusion, stems from the Theravada school, but this is just an accident of history. This particular version happened to have been passed down through the Sri Lankan lineage. But many other versions have come down to us.

There are no significant doctrinal differences between these versions. They merely choose slightly different readings, some different verses, and change the order. It would be a nice gesture to non-sectarianism to include one of these texts as well as or instead of the Pali, but I am not aware of any suitable translations.

The teachings found in the Dhammapada are those common to all schools. They are particularly relevant for lay instruction, and are frequently used in that way in Buddhist communities. But perhaps even more significant, the Dhammapadas are often associated by scholars with “Ashokan Buddhism”. That is to say, that the emphasis on a practical application of Dhamma to a good life as found in the Dhammapada, and especially the emphasis on non-violence, relates very closely to the teachings found in the Ashokan edicts. This means that they are particularly suitable for a leader who seeks moral and spiritual guidance in the practicalities of life.

Can we imagine, what would a politician do if she happened, on a difficult night in Parliament, to seek some solace from the religious texts found there? She opens the box, is delighted to see a Buddhist text, and, having heard that Buddhism is a rational religion of ethics and meditation, opens a random page. What does she find?

Hatred is never appeased by hatred,\\

Hatred is only ever appeased by love:\\

This is an ancient law.

%
\end{quotation}

Happily, my proposal was accepted. We approached the Pali Text Society, who kindly donated a hardcover edition of K.R. Norman’s excellent analytical translation. And on the 15th September 2009, I was proud to be part of the Buddhist delegation that met with the Speaker of the House, who accepted our copy of the Dhammapada and placed it in the Despatch Box, where it remains to this day.

%
\chapter*{Sayings of the Dhamma: a path of love and wisdom}
\addcontentsline{toc}{chapter}{Sayings of the Dhamma: a path of love and wisdom}
\markboth{Sayings of the Dhamma: a path of love and wisdom}{Sayings of the Dhamma: a path of love and wisdom}

\scbyline{Bhikkhu Sujato, 2022}

The Dhammapada is, in terms of sequence, the second collection in the Pali Khuddhaka \textsanskrit{Nikāya}; but in terms of fame and popularity it is, without any competitor, the first. It consists of 423 verses arranged in thematic chapters. Its powerful, engaging, and evocative verses have ensured its popularity from ancient times until now.

The Dhammapada is closely allied to the \textsanskrit{Udāna}, and I refer you to my essay there for the relation between these texts. Many of the verses of the Dhammapada can be found elsewhere in the Pali Canon and are also widely shared across traditions. They are not restricted to Buddhist texts either, for they may also be found occasionally in the law books of Manu, in the \textsanskrit{Mahābhārata}, in Jaina sutras, and in the Sanskrit collection of fables, the \textsanskrit{Pañcatantra}. The special quality of the Dhammapada lies not in any doctrinal innovations, but in the appealing and meaningful selection and arrangement of verses by topic.

There are at least twelve versions of the Dhammapada, far more than any comparable ancient Buddhist text. They exist in Pali, Sanskrit, \textsanskrit{Prākrit}, \textsanskrit{Gandhārī}, Tibetan, and no less than four Chinese translations. Study of this large and linguistically-diverse mass of texts reveals much of the manner in which Buddhist texts were compiled and later translated. All differences aside, however, the texts share not just a name, but an overall structure and style, and many individual verses and lines of verse.

While the Dhammapada deservedly has a reputation for its practical and accessible nature, it is by no means a watered-down version of the Dhamma. It contains some of the most enigmatic and profound teachings of the Pali canon, and like all early Buddhist teachings, challenges our desires and assumptions to the core. It grants the reader, the practitioner, the audience, the foremost place in realizing its truths, acknowledging that its words alone are not enough. And as such, it reveals the deep love and compassion that lie at the heart of the Buddha’s teaching, his profound conviction that freedom is possible, and that we have what it takes.

\section*{What Others Have Said}

Many words have been written in eulogy of the Dhammapada’s qualities, and I can do no better than quote them. In the Preface to his translation, Ven. Buddharakkhita says:

\begin{quotation}%
The contents of the verses, however, transcend the limited and particular circumstances of their origin, reaching out through the ages to various types of people in all the diverse situations of life. For the simple and unsophisticated the Dhammapada is a sympathetic counselor; for the intellectually overburdened its clear and direct teachings inspire humility and reflection; for the earnest seeker it is a perennial source of inspiration and practical instruction.

%
\end{quotation}

And Bhikkhu Bodhi, in his introduction to the same translation, says:

\begin{quotation}%
It is an ever-fecund source of themes for sermons and discussions, a guidebook for resolving the countless problems of everyday life, a primer for the instruction of novices in the monasteries. Even the experienced contemplative, withdrawn to forest hermitage or mountainside cave for a life of meditation, can be expected to count a copy of the book among his few material possessions. Yet the admiration the Dhammapada has elicited has not been confined to avowed followers of Buddhism. Wherever it has become known its moral earnestness, realistic understanding of human life, aphoristic wisdom and stirring message of a way to freedom from suffering have won for it the devotion and veneration of those responsive to the good and the true.

%
\end{quotation}

He proposes a four-fold scheme for understanding the aims of the diverse teachings found in the Dhammapada.

\begin{enumerate}%
\item Happiness in this life.%
\item Happiness in the next life.%
\item The path to freedom from suffering.%
\item Celebrations of freedom.%
\end{enumerate}

These are not incompatible purposes, but rather build on each other. He goes on to give a detailed analysis of the doctrinal content of the verses seen through this framework.

While some commentators have tended to tame and blandify the teachings of the Dhammapada, not so Albert Edmunds, whose 1902 translation \textit{Hymns of the Faith} was one of the earliest into English, and whose introduction remains perhaps the most dramatic:

\begin{quotation}%
If ever an immortal classic was produced upon the continent of Asia, it was this. … No trite ephemeral songs are here, but red-hot lava from the abysses of the human soul … These old refrains from a life beyond time and sense, as it was wrought out by generations of earnest thinkers, have been fire in many a muse. They burned in the brains of the Chinese pilgrims, who braved the blasts of the Mongolian desert, climbed the cliffs of the Himalayas, swung by rope-bridges across the Indus where it rages through its gloomiest gorge, and faced the bandit and beast, to peregrinate the Holy Land of their religion and tread in the footsteps of their Master.

%
\end{quotation}

His description of the travails endured by the ancient Chinese pilgrims in search of the Dhamma is in no way exaggerated, and it serves as a timely reminder to us, in our age of lazy access to the world’s information, that some forms of wisdom are truly rare and priceless, and worth putting in effort.

The renowned meditation teacher Daw Mya Tin, known as Mother Sayamagyi, when introducing her 1984 translation on behalf of the Burma Pitaka Association under the title \textit{The Dhammapada: Verses and Stories} notes the prevalence of the Dhammapada in Burmese Buddhism:

\begin{quotation}%
Through these verses, the Buddha exhorts one to achieve that greatest of all conquests, the conquest of self; to escape from the evils of passion, hatred and ignorance; and to strive hard to attain freedom from craving and freedom from the round of rebirths. Each verse contains a truth (dhamma), an exhortation, a piece of advice. … In Burma, translations have been made into Burmese, mostly in prose, some with paraphrases, explanations and abridgements of stories relating to the verses. In recent years, some books on Dhammapada with both Burmese and English translations, together with Pali verses, have also been published.

%
\end{quotation}

\textsanskrit{Ṭhānissaro} Bhikkhu applies the sophisticated aesthetic theories of later Indian philosophy to analyze the literary qualities of the Dhammapada, arguing that it aims “to instruct in the highest ends of life while simultaneously giving delight.”

Ānandajoti Bhikkhu has studied the Dhammapada literature extensively, in both Pali and other Indic languages. In the introduction to his translation he says:

\begin{quotation}%
The timeless ethical teachings contained in these verses are still considered relevant to people’s lives, and they are a good guide to living well, and show how to reap the rewards of good living. … The verses and stories are well known in traditional \textsanskrit{Theravāda} Buddhist cultures, and most born and brought up in those societies will be able to recite many of the verses, and relate the stories that go with them, even from a young age.

%
\end{quotation}

As a jewel of Indian literature, the Dhammapada has been widely translated and commented on by Indian pundits. The celebrated Hindu scholar and second president of India, Sarvepalli Radhakrishnan, in the Preface to his revised edition of 1950, approaches the Dhammapada from a deeply humanistic perspective:

\begin{quotation}%
The effort to build one world requires a closer understanding among the peoples of the world and their cultures. This translation of the Dhammapada, the most popular and influential book of Buddhist canonical literature, is offered as a small contribution to world understanding. The central thesis of the book—that human conduct, righteous behavior, reflection, and meditation are more important than vain speculations about the transcendent—has an appeal to the modern mind. Its teaching—to repress the instincts entirely its to generate neuroses; to give them full rein is also to end up in neuroses—is supported by modern psychology. Books so rich in significance as the Dhammapada require to be understood by each generation in relation to its own problems.

%
\end{quotation}

Eknath Easwaran, a Hindu yogi and scholar is less circumspect in his approach. While quoting widely from brahmanical scriptures in his introduction, he does not hesitate to claim that, “if everything else were lost, we would need nothing more than the Dhammapada to follow the way of the Buddha.” It’s a debatable claim; but what is not debatable is that, were we to lose all other Dhamma, that would include all the many places the Buddha criticized brahmanism, its rituals, texts, beliefs, and practices.

This is the downside of the “context collapse” in the Dhammapada: it is one thing to enjoy the Dhamma in the form of delightful bon mots; it is quite another to reduce it to nothing more than that. The Dhammapada serves well as an introduction to the Buddha’s teaching and as an inspiring reminder for experienced practitioners, but it is no replacement for the detailed and careful presentations of the Buddha’s path found in the prose suttas.

As if to illustrate this point, Easwaran goes on to say that the Dhammapada is a guide to “nothing less than the highest goal life can offer: Self-realization.” He apparently does not notice that “self-realization” is nowhere mentioned in the Dhammapada, nor is it a goal of Buddhism. The goal of the Dhammapada is the same as that of all Buddhism: freedom from suffering. A careful study of the prose Suttas might have helped him to draw the Buddha’s message from the text, instead of reading his own ideas into it.

Despite his evident preconceptions, Easwaran is sincere in his approach. But not all those who comment on the Dhammapada do so from a place of learning or wisdom. A notorious cult leader like Osho cannot help but reveal his nature in the way he introduces his commentary.

\begin{quotation}%
My beloved Bodhisattvas … You are bodhisattvas because of your longing to be conscious … And THE DHAMMAPADA, the teaching of Gautama the Buddha, can only be taught to the bodhisattvas. It cannot be taught to the ordinary, mediocre humanity, because it cannot be understood by them.

%
\end{quotation}

The Buddha never spoke in this way, aiming to divide and separate, creating an egoistic in-group with special access to the truth. These are the ways of a con artist or a cult leader, and they show how readily and how swiftly the Dhamma may be turned into an instrument of manipulation, a tool in the hands of a grifter.

I am conscious that this selection of translators includes mostly men, reflecting the bias of the field as a whole, so it’s important to note that women have also made major contributions. Caroline A.F. Rhys Davids, then president of the PTS,  translated the Dhammapada as \textit{Verses on Dhamma} in volume 1 of \textit{The Minor Anthologies of the Pali Canon} in 1931. In 1997, Anne Bancroft together with Thomas Byrom published a translation through Element Books. And in Spanish, the erudite Argentinian philosopher Carmen Dragonetti pubished \textit{La \textsanskrit{enseñanza} de Buda} 2002, which went on to become one of the most popular renderings.

For a popular edition that is reliable and accessible, Valerie J Roebuck, an accomplished scholar of Sanskrit and Pali, as well as an experienced meditator, published a verse translation through Penguin in 2010 under the title \textit{The Dhammapada}. A review by Elizabeth Harris described it as “a gem … energetic and direct … I do not know a version of this text that is so comprehensive and informative, both for the general reader and the scholar” (\textit{Religions of South Asia} 6.1, 2012).

The Dhammapada has also stimulated a wide variety of creative responses. These began with the commentary itself, which paints a vivid if sometimes unlikely picture of the circumstances of the verses. Illustrated editions sometimes render these stories, or else pair the verses with more evocative images. There are at least two musical settings of the Dhammapada, and many individual verses have been set to song.

A forthcoming novel, \textit{The Lyrebird’s Cry} by Samantha Sirimanne Hyde, begins each chapter with a quotation from the Dhammapada, in a manner that deliberately evokes the traditional manner of sermon-giving. The story tells of a “sensitive” young Sri Lankan man living in Sydney who is forced into an arranged marriage with a “good girl” from Colombo—despite the inconvenient fact that he is gay. It highlights the heartlessness that can so often underlie a pious adherence to the maxims of a sacred text.

\section*{On Translations of the Dhammapada}

There are countless modern translations of the Dhammapada, and more than any other Pali text it is available and widely read: in massive illustrated coffee-table books, in cute inspirational booklets, in audio or on the web, or quoted on throw-pillows or coffee-mugs. There seems little need for another translation; indeed, for many decades now it has been a convention when introducing a new translation of the Dhammapada to apologize for its existence.

Yet if the proliferation of Dhammapada translations gives you the idea that it is a simple text that anyone can translate, consider the following. In the introduction to his translation of the Dhammapada for the Pali Text Society (PTS), Professor K.R. Norman—the greatest modern linguist of ancient Indic languages—said this in reference to the editor of the \textsanskrit{Gandhārī} Dhammapada:

\begin{quotation}%
John Brough is reported as saying, when asked if he would produce a new translation of the Dhammapada for the PTS, that he could not, because it was “too difficult”. I regret to say that I must agree with him. My notes reveal how often I was quite unsure about the meaning of a verse.

%
\end{quotation}

Now, notwithstanding the fact that academics have a stricter standard for confidence than most people, the fact remains that the Dhammapada is by no means an easy or beginners text. Given that the greatest linguists of the field quail before the challenge of translating the Dhammapada, one might wonder at the degree of expertise brought to the task by the countless “translators” who have expressed no such qualms.

I am being coy here, so let me be plain. The vast majority of so-called “translations” of the Dhammapada are made by people unqualified to do so. They merely rehash old versions, leaving out what they find disagreeable, and rephrasing things to sound “poetic”—by which they mean inoffensive and unchallenging. Where the Buddha spoke with specificity, they gesture vaguely to universality. In the process the translations become a more reliable guide to the ideological priors of the “translators” than they are to anything that the Buddha taught. Such, sad to say, are most of the popular Dhammapadas that you might purchase through major publishing houses, or learn from various gurus or teachers.

Any new translation must be, in part, a dialogue with older versions, which exist both as texts on a page and as echoes in memory. And when writing, it is not just the translation that matters, but its reception: translators are in dialogue with both other translators and with readers. Sometimes we draw from what they have done, sometimes, we look at things from a fresh angle, and sometimes we try to correct old errors.

In my translation, I try to remain as close as possible to the meaning of the text, while believing that readability does not compromise accuracy. On the contrary, it is through natural and idiomatic diction that the meaning is most reliably conveyed. I aim for transparency in translation; it is the Buddha’s words, not mine, that matter. And I am someone who finds beauty in things that are raw and natural, so I don’t sand down rough edges.

All these qualities you might find in other translations, but in one thing my translation is unique: consistency with the rest of the Suttas. Since I undertook the Dhammapada as part of my overall translation project, I have tried as best I can to ensure that renderings make sense in different contexts. That doesn’t mean they must be identical everywhere, but it does mean that where context is lacking in the Dhammapada itself, renderings can often be informed by their occurrence elsewhere. The very first lines of the Dhammapada are a good example of this, for they echo a short prose passage in the \textsanskrit{Aṅguttara} \textsanskrit{Nikāya} (see below).

\section*{The Commentary}

According to the traditions, each of the verses of the Dhammapada was spoken by the Buddha in response to a specific circumstance. In the Pali tradition, these background stories are preserved in the commentary edited and compiled by Buddhaghosa perhaps 800 years after the Buddha, based on much older texts. The stories are of mixed provenance. Many of them are obviously of a late origin. But the tradition of framing verses in a narrative context dates from the earliest times, and there is no reason to doubt that at least some of the stories preserve genuine historical details.

I’ll just make two observations regarding the commentary from my experience as a teacher. First, many Theravadins, hearing these stories many times since childhood, assume that they are “Suttas”, with no concept of the fact that they stem from centuries after the Buddha’s life. At the very least, we should be able to distinguish between Sutta and commentary. And second, when I taught a class, firstly just the verses, and then the verses with stories, the students universally said they preferred the verses without the stories. So the idea that the stories make the verses more meaningful or accessible doesn’t necessarily hold up in practice.

None of this, of course, is to question the inestimable value that the commentary holds for any translator. Like all Pali verse, the Dhammapada abounds in tricky idioms and difficult syntax, and the commentary stands by like a good friend ready to help the lonely and beleaguered scholar in time of need. No serious scholar would discount the value of the commentaries in making our modern understanding of the Pali texts possible.

\section*{The Title}

Both elements of the word \textit{dhammapada} can convey different meanings, and as a result translators have come up with a bewildering variety of renderings. \textit{Dhamma} means “teachings, principles, the good, virtue, phenomena, justice” etc.,  and \textit{pada} means “foot, footprint, track, step, word, passage, line of verse, state”. In such cases, the sense of words cannot be simply derived from combining the elements; rather, let us look at how it is used in the Pali canon itself.

The title Dhammapada does not feature among the nine sections of the early teachings (\textit{\textsanskrit{navaṅgadhamma}}). The word \textit{dhammapada} however does appear in the early texts, in two primary meanings.

In AN 4.30 the Buddha speaks to a group of wanderers, naming three leaders as \textsanskrit{Annabhāra}, Varadhara, and \textsanskrit{Sakuludāyī}. He describes them as “very well known”, although they are, as it turns out, only referred to a couple of time elsewhere in the canon (MN 77, AN 4.185). Here he declares that there are four \textit{dhammapadas} that are ancient and uncontested. He names them as contentment, good will, right mindfulness, and right immersion in \textit{\textsanskrit{samādhi}}. He argues that a spiritual practitioner must respect these four, and that one who does not can be legitimately criticized.

In this context, then, I have translated \textit{dhammapada} as “basic principle”. Clearly it has no direct connection with the book named Dhammapada, although one might detect a distant kinship, given that the Dhammapada too consists of teachings that are, by and large, “basic principles” that speak to people across boundaries of religion and sect. In DN 33:1.11.138 we find the same four \textit{dhammapadas} listed in summary, and they recur in later texts such as Pe 1.1:411.1, Ne 37:390.3 and Pe 2:144.1.

A quite different sense is found at SN 9.10, where a mendicant is admonished by a deity for no longer reciting the Dhamma as they did in the past. In the verses, the deity uses the term \textit{dhammapada} which here must mean something like “passages of the teaching”.  The same sense applies at SN 10.6. A similar sense is found at MN 12:62.10, where it is said that the Buddha would never run out of ways of explaining the Dhamma, here said to be \textit{\textsanskrit{dhammapadabyañjanaṁ}}, “words and phrases of the teachings”. At Snp 1.5:6.1 we find \textit{dhammapade sudesite} which seems to have a similar meaning. The same phrase occurs in the Dhammapada itself (Dhp 44, Dhp 45). At SN 1.33:19.2 the \textit{dhammapada} is said to excel even generosity; here it seems to mean the “way of the teaching”.

Turning now to later texts, the same meaning is found in Ja 424, where the gift of \textit{dhammapada} excels the highest of worldly gifts. In Ja 532, on the other hand, \textit{dhammapada} occurs in the midst of a discussion of the debt owed to parents and appears to mean “the path of duty”.

Finally the Dhammapada itself is referred to by name twice in the \textsanskrit{Milindapañha} (Mil 7.3.8:1.5, Mil 7.7.3:1.6). The verses quoted do in fact appear in the Pali Dhammapada (Dhp 327, Dhp 32; the latter also appears at AN 4.37:8.1). So we know that the Pali Dhammapada in its current form must have existed no later than the creation of the \textsanskrit{Milindapañha}. We don’t know the exact date of that, but it must have been after the time of King Menander (2nd century BCE).

Thus in the early texts we find the senses “basic principles” and “statements of the teaching”. The first is rather restricted and seems to apply only in the case of the four stated principles, which themselves are a statement of common ground between religions, rather than a presentation of the Buddha’s path. It seems, then, that the second meaning applies in this case. “Sayings of the Teaching” is an apt title for a work that gathers pithy verses from various places.

Commentators ancient and modern have drawn attention to a variety of more meaningful implications than the rather staid “Sayings of the Teaching”. Since a \textit{pada} is a footprint and the \textit{dhamma} is the truth, it might mean “tracks of truth”—the traces that the Buddha’s insight into reality have left in the world. Or, since a series of footprints is a path, and the \textit{dhamma} is the “good”, it could be the “path to virtue”. As a translator, I need to focus on the primary literary sense that is justified by the text, but as a teacher and practitioner, I also appreciate the way that wordplay can enrich the nuances and implications of a simple title.

\section*{Formation of the Dhammapada}

In his introduction, K.R. Norman suggests that, while the Dhammapada clearly borrows from elsewhere in the canon, it may also be true that the canonical texts generally borrowed from a store of relatively free-floating verses that predate the canon as we know it. Some such verses may even predate the Buddha.

The Buddha himself is recorded as quoting from pre-Buddhist verses on occasion, and it is true that, while the verses of the Dhammapada are in harmony with Buddhist teachings, many of them do not mention specifically Buddhist ideas and would be equally at home in any of the ancient Indian religions. The same may be said, it is worth noting, of the prose Suttas. There are countless Suttas that teach ethics or meditation or even philosophy in ways that do not assume a basis in basic Buddhist doctrines. This reflects the fact that the Buddha spoke to a diverse audience that often included non-Buddhists. We’ve already seen that the four \textit{dhammapadas} were taught specially to emphasize the common ground between religions, and it may be that this idea influenced the selection of verses for the Dhammapada.

It seems likely to me that the genre of Dhammapada literature is associated with the popularization and spread of Buddhism in India, and especially with its adoption as the mainstream religion of the great emperor Ashoka. The flavor of the Dhammapada resonates closely with the tenor of Ashoka’s edicts, with its emphasis on practical teachings that are universally applicable, and a special interest in harmony and non-violence. I suspect that the Dhammapada collections were created, or at least popularized and expanded, in the Ashokan era, drawing on existing verses, and forming a handy and accessible way to bring the Dhamma to a much broader audience. In other words, its modern usage as an attractive access point to the Dhamma for Buddhists and non-Buddhists alike is precisely the reason why it was created in the first place.

\section*{The Teachings of the Dhammapada}

Rather than a general overview of the teachings of the Dhammapada, which has been well undertaken by many previously, I will introduce the meaning of the text through the close study of a few verses in the opening chapters. In this way I’d like to suggest that, while the text may be read as an inspiring source of spiritual quotations, or as a verse summary of the Buddha’s doctrines, it may also be read as a carefully composed work of spiritual literature, one that repays careful attention to details, and which contains the keys to its own interpretation.

The Dhammapada announces its primary theme in its opening verses. As so often, lines that appear clear and simple in Pali turn out to be surprisingly difficult to catch exactly in English. A classic rendering, endlessly requoted, is:

\begin{verse}%
Mind is the forerunner of all things.

%
\end{verse}

It’s curiously difficult to locate the originator of this phrasing. Ven. Buddhadatta in 1922 had “Mind is the forerunner of all mental states”, Ven. \textsanskrit{Nārada} (1946) has “Mind is the forerunner of (all evil) states”, while Caroline A.F. Rhys Davids has “Things are forerun by mind” (1931). The indefatigable Bodhipaksa—whose site “Fake Buddha Quotes” is essential—has traced this phrasing to an essay by Ven. \textsanskrit{Vajirañāṇa} called \textit{The Importance of Thought in Buddhism} (Maha-bodhi vol. 49, May/June 1941). Like so many after him, he presents this translation without naming his source, so we do not know whether the rendering was his or if it was already common parlance. Ven. \textsanskrit{Vajirañāṇa} was famed not only for his erudition, but for his skill in presenting Dhamma in an accessible and relevant way for a modern audience. He was, in fact, the inventor of the modern Dhamma talk, and pioneered the practice of giving a focussed and thorough exposition of a specific verse or topic in a limited time. Such talks would frequently begin by quoting a verse from the Dhammapada. So, while I have not been able to identify a Dhammapada translation by Ven. \textsanskrit{Vajirañāṇa}, it is entirely possible that he developed his own renderings while giving teachings, that he referred to these in his writings, and that they made their way into the Buddhist zeitgeist through his many students who became teachers in their own right.

But let us leave aside questions of authorship and focus on the text. The “all” here does not appear in the text, but is justified by the closely related passage at AN 1.56:1.1, where “all” unskilful qualities are said to have mind as the forerunner. The tricky terms here, however, are \textit{mano} (“mind”) and \textit{dhamma}. As we have seen with the title of the collection, the openness of the text has invited a range of renderings. But it is possible to narrow down the sense from a careful reading of the text in light of the full range of early teachings.

The verses are about cause and effect. By acting badly, suffering will come, while by acting well, happiness will follow. That much is clear.

The pair of terms \textit{mano} and \textit{dhamma} are found together in the standard exposition of the sixth kind of consciousness, mind consciousness. There, \textit{mano} is the basis of \textit{mano}-consciousness in the same way that the “eye” (etc.) is the basis of “eye-consciousness” (etc.). Here \textit{mano} is usually rendered as “mind”. In this context, \textit{dhamma} means the phenomena of which the mind is aware, and is typically rendered as “phenomena”, “thought”, or “mind object” (though I dislike that rendering).

It’s not clear, however, that this sense pertains here, for we are not speaking of the process of consciousness, but the creation of kamma. Of the many words for “mind” in Pali, \textit{mano} often conveys the specific sense of “intentionality”. \textit{Mano} is the active dimension of mind, the exercise of choice in performing morally potent deeds. And surely that must be the sense required here.

\textit{Dhamma} must then refer to the experiences of pleasure and pain that are formed by the deed. The passage at AN 1.56:1.1 makes it clear that \textit{mano} is not apart from the \textit{dhammas}, but is one of them (\textit{\textsanskrit{tesaṁ} \textsanskrit{dhammānaṁ}}). So a rendering like “at the forefront” would be better than “precedes”.

Even though the context makes it clear that ethical intention is the subject, the opening line invites an “idealist” interpretation so long as \textit{mano} is rendered with “mind”. The Buddha, however, is emphasizing the creative power of mind in the world, rather than postulating that the entire world is nothing more than a projection of the mind. So I opted to emphasize the aspect of intention, while clarifying that \textit{dhamma} refers to a person’s experiences rather than to all “things” in general.

\begin{verse}%
Intention shapes experiences.

%
\end{verse}

That’s a lot of work to establish just one line, and you will be delighted to know that I won’t be discussing every line in so much detail. But what is interesting is how this line functions as a meta-comment on the text itself. How you experience the Dhammapada depends on what you bring to it. It is not an objective reality to which one must become subject, but a living provocation. This is why the Dhamma cannot be forced on anyone, and why someone encountering Dhamma with a “fault-finding mind” (\textit{uparambhacitta}) will never understand it.

And while it’s true that \textit{mano} conveys the primary sense of “intentionality”, it’s also true that the sense is broader than merely “volition”: it implies a whole-hearted commitment to understanding, a unity of intellect and emotion and sensibility. Wisdom arises from a peaceful and clear mind, from critical inquiry when it is supported by faith. But it will come, though slower perhaps than we would like, and only if we are patient and humble enough to let it reveal itself to us.

The text immediately proceeds, as if impelled by the opening lines, to illustrate the point in a dazzling series of verses, each pair of which draw out a particular example of just how the mind creates suffering or happiness.

The second series of verses, which is really two pairs on the same theme, is almost as famous as the opening, and justly so.

\begin{verse}%
For never is hatred\\

    settled by hate,\\

    it’s only settled by love

%
\end{verse}

The last verse in this series (Dhp 6) invites two quite different renderings, depending on whether \textit{\textsanskrit{yamāmase}} is read with the root \textit{yam} as “restrained” or, per the commentary, as a reference to Yama the god of the dead. The latter leads to such renderings as Buddharakkhita’s:

\begin{verse}%
There are those who do not realize\\

that one day we all must die.

%
\end{verse}

Although enjoying the support of the commentary, it really is a double-stretch: \textit{yama} is not really used in this way elsewhere; and to introduce death here is a dramatic shift.

While the verb \textit{\textsanskrit{yamāmase}}  seems to be only found in this context and in this unusual form (3rd plural middle imperative), it’s a common Pali idiom to say that one should be “restrained” (\textit{\textsanskrit{saṁyama}}) regarding harming living creatures. We even find this in the very same verb form: \textit{\textsanskrit{pāṇesu} ca \textsanskrit{saṁyamāmase}} (SN 10.6:3.1). This option doesn’t lack commentarial support, either, for an alternate explanation speaks of not amplifying conflicts that have arisen in the Sangha.

This reading has been adopted by linguistically-minded translators such as K.R. Norman and Ven. Ānandajoti, and I follow suit.

\begin{verse}%
Others don’t understand\\

    that here we need to be restrained.

%
\end{verse}

This is a useful detail to bear in mind when comparing different translations. Older translations, especially those that cleave more closely to the traditional explanation, tend to use the sense of “death” here, while modern translations prefer “restraint”.

The final pair of verses undercut the authority of Buddhist texts themselves, arguing that one who does not practice is like “a cowherd who counts the cattle of others”, while even one of little learning may realize the truth. Here the Dhammapada is making a meta-comment on how to read the Dhammapada, drawing out the implication in the first verses.

That the unified character of the first chapter is no accident is borne out by a consideration of the second chapter, on heedfulness or diligence (\textit{\textsanskrit{appamāda}}). Here we open with an echo of the pairs of the first chapter, contrasting the heedless with the heedful.

The opening lines are, once again, not as easy to translate as they might appear, and they offer us another litmus test to understand the perspective of different translators.

Commonly we find something like:

\begin{verse}%
Heedfulness is the path to the deathless.

%
\end{verse}

Now, that the “deathless” (\textit{amata}) refers to \textsanskrit{Nibbāna} is not in dispute. \textsanskrit{Nibbāna} is “deathless” because it is free from the cycle of transmigration through birth, old age, and death.

The tricky part is \textit{pada}. The commentary glosses it with \textit{\textsanskrit{upāyo} maggo} “the way, the path” and this is followed by many translators. While \textit{pada} doesn’t literally mean “path”, it is used in the sense of “footprint”, hence “tracks”, hence a path to follow. The elephant’s footprint is sometimes used as an example of following such tracks.

The problem is that in canonical Pali, while this exact phrase doesn’t appear elsewhere, the “deathless” \textit{pada}, like the \textit{pada} of \textsanskrit{Nibbāna}, is not “followed” but “reached” (Tha-ap 415:11.4, Tha-ap 395:24.4, Tha-ap 340:17.4) or “understood” (Bv 1:68.4). It must, then, refer to the “state” of the deathless, not the path to it.

This is doctrinally challenging, since heedfulness is a practice, and normally the texts are quite scrupulous to distinguish the practice from the fruit. The “path” is said to be the best of conditioned things (AN 5.32:4.1) because it leads to \textsanskrit{Nibbāna}, not because it \emph{is} \textsanskrit{Nibbāna}, which is the only “unconditioned” reality.

Despite this, however, the second half of the same verse makes it quite clear that this unlikely sense is, in fact, exactly what is meant:

\begin{verse}%
The heedful do not die

%
\end{verse}

That this, and by extension the whole verse, are spoken with a metaphorical force is clarified by the inclusion of \textit{\textsanskrit{yathā}} in the last line:

\begin{verse}%
while the heedless are like the dead.

%
\end{verse}

This gives us an idea of the subtle shifts in the text as explained by the commentary and back-read into the texts by translators. It’s a normalizing reading, smoothing the craggy text so that it is more easily reconciled with the systematic doctrines of the prose. The original is, to my mind, more powerful and dramatic precisely because it says something unexpected. What exactly can it mean to say that heedfulness is the state of deathlessness?

The commentary, of course, is no stranger to metaphor, and is quite happy to draw out metaphors where needed. Yet anyone familiar with traditional religious communities will recognize the way that playful and metaphorical scriptures are flattened and reduced by the dead hand of literalism, stripped of wit and nuance, driven by the fear that someone might not understand things correctly.

Once again, I see a meta-purpose in the arrangement of the text. The verse that invokes heedfulness is itself easily misread by the heedless. Heedfulness is more than just the path to the Deathless, it is itself a state of life, of active and vital response to the moment, of a continual reassessment and questioning of assumptions. The chapter deliberately opens with a verse that wakes the reader, even, and perhaps especially, one who is already versed in Buddhist doctrine.

As with the first chapter, a series of striking verses draw out the theme from various angles, but the force of the opening verses is revisited in the closing. One who loves diligence cannot fall back from the path, but is in the very presence of \textsanskrit{Nibbāna}. Here again the line between metaphor and reality is deliberately blurred by the text, as if exceeding the limits of words.

I’ll leave my reading of the text there. Hopefully this is enough to show that the poetic strength of the text is not diminished by a close reading, but rather, that it allows hidden nuances and unexpected implications to reveal themselves. There is more to poetry than a wording that sounds nice, and more to teaching Dhamma than restating standard doctrines.

\section*{A Brief Textual History}

The first 255 verses of the Dhammapada were translated by Daniel John Gogerly and published in the journal “The Friend” in Colombo in 1840. It was among the first translations of Pali into English. This is how he rendered the first verse:

\begin{verse}%
Mind precedes action. The motive is chief: actions precede from mind. If any one speak or act from a corrupt mind, suffering will follow the action, as the wheel follows the lifted foot of the ox.

%
\end{verse}

The 1855 edition edited by Viggo Fausbøll and published as \textit{Dhammapadam} was perhaps the first of all canonical Pali texts published in book form and in Roman characters. (It had been preceded by editions of the \textsanskrit{Mahāvaṁsa}—the great chronicle of Sri Lanka—by Eugène Burnouf in 1826 and George Turnour in 1837.) He also supplied excerpts from the commentary, and textual apparatus and literal translation both in Latin. For his text, Fausbøll relied primarily on three manuscripts in Sinhalese characters held at the Great Royal Library of Copenhagen. He introduced verse numbers, which were adopted by later editions and are still in use today. At such an early date, the means of Romanizing Pali had not yet been standardized, but the text remains clear and readable.

This was updated in 1900 and republished via Luzac \& Co. under the title, \textit{The Dhammapada, Being a Collection of Moral Verses in Pali}. Fausbøll notes several editions since his 1855 edition, in Sinhalese, Thai, and Burmese characters, as well as several new translations in various languages.

As an aside, Fausbøll remarks that the Thai characters are difficult to read, and argues that the Roman characters will become universal, advancing the curious opinion that English likewise will be the universal language “for it is a well known fact that in the beginning the Lord took all languages, boiled them in a pot, and forthwith extracted the English language as the essence of them all.” I felt I had to mention this, because it is rare in studies of Pali manuscripts to find evidence of a sense of humor!

As regards the Thai characters, it is noteworthy that, while printed Thai has become a perfectly readable script, it is nonetheless the case that the \textsanskrit{Mahāsaṅgīti} edition, which is used by SuttaCentral, was published by a Thai consortium purely in Roman characters. This was because they had become frustrated with the mispronunciation of Pali in Thailand, caused by the fact that the same letter sometimes has a different value in Thai and Pali. The same is true of Pali written in other local scripts, and while Pali scholars are well aware of the issues, it is still the case that not only are Pali words often mispronounced, there are entire movements of Buddhism based on a misspelling of words due to ignorance of such basic details. Having said which, the use of Roman characters are by no means a sure way of guarding against mispronunciation or misunderstanding. Pali may be perfectly well represented by many scripts, and the only real guard against misunderstanding is, as the Dhammapada itself teaches us, heedfulness.

The first complete and rigorous translation into English was that by Max Müller through Clarendon Press in 1870, revised in 1881 and 1898. Müller was one of the founding fathers of Indology, although by his own admission Pali and Buddhism were not his primary focus. His work greatly influenced later translators, and in addition contained an extensive discussion of historical matters.

It was not until 1914 that the Pali Text Society published their own edition, which was edited by \textsanskrit{Sūriyagoda} \textsanskrit{Sumaṅgala} Thera based on printed editions in Burmese, Thai, and Singalese characters, as well as “two reliable manuscripts” in his possession. The edition carefully notes variant readings and cross-references, and became the standard edition for international Pali studies until replaced by the 1994 PTS edition by O. von Hinüber and K. R. Norman.

Finally I should mention the excellent edition of the Dhammapada by Ānandajoti Bhikkhu, originally in 2002 and last updated in 2016. This is primarily a revision and correction of the Buddha Jayanthi text, but takes into consideration the PTS and other editions, as well as an extensive comparative study with the Patna Dharmapada.

Thus far a cursory and incomplete survey of Pali editions has taken us, and I have barely scratched the surface of the translations, which number over 70 in English alone. I will simply note here that when looking for assistance in unraveling the knotty problems of the text I turned first of all to the work of K.R. Norman and Ven. Ānandajoti. I also referred from time to time to the translations of Ven. Buddharakkhita and Ven. \textsanskrit{Ṭhānissaro}, the latter of whom sometimes catches aesthetic nuances that a linguist might miss.

%
\chapter*{Acknowledgements}
\addcontentsline{toc}{chapter}{Acknowledgements}
\markboth{Acknowledgements}{Acknowledgements}

I remember with gratitude all those from whom I have learned the Dhamma, especially Ajahn Brahm and Bhikkhu Bodhi, the two monks who more than anyone else showed me the depth, meaning, and practical value of the Suttas.

Special thanks to Dustin and Keiko Cheah and family, who sponsored my stay in Qi Mei while I made this translation.

Thanks also for Blake Walshe, who provided essential software support for my translation work.

Throughout the process of translation, I have frequently sought feedback and suggestions from the community on the SuttaCentral community on our forum, “Discuss and Discover”. I want to thank all those who have made suggestions and contributed to my understanding, as well as to the moderators who have made the forum possible. A special thanks is due to \textsanskrit{Sabbamittā}, a true friend of all, who has tirelessly and precisely checked my work.

Finally my everlasting thanks to all those people, far too many to mention, who have supported SuttaCentral, and those who have supported my life as a monastic. None of this would be possible without you.

%
\mainmatter%
\pagestyle{fancy}%
\addtocontents{toc}{\let\protect\contentsline\protect\nopagecontentsline}
\part*{Sayings of the Dhamma}
\addcontentsline{toc}{part}{Sayings of the Dhamma}
\markboth{}{}
\addtocontents{toc}{\let\protect\contentsline\protect\oldcontentsline}

%
\chapter*{{\suttatitleacronym Dhp 1–20}{\suttatitletranslation 1. Pairs }{\suttatitleroot Yamakavagga}}
\addcontentsline{toc}{chapter}{\tocacronym{Dhp 1–20} \toctranslation{1. Pairs } \tocroot{Yamakavagga}}
\markboth{1. Pairs }{Yamakavagga}
\extramarks{Dhp 1–20}{Dhp 1–20}

\begin{verse}%
Intention\marginnote{1} shapes experiences; \\
intention is first, they’re made by intention. \\
If with corrupt intent \\
you speak or act, \\
suffering follows you, \\
like a wheel, the ox’s foot. 

%
\end{verse}

\begin{verse}%
Intention\marginnote{2} shapes experiences; \\
intention is first, they’re made by intention. \\
If with pure intent \\
you speak or act, \\
happiness follows you \\
like a shadow that never leaves. 

%
\end{verse}

\begin{verse}%
“They\marginnote{3} abused me, they hit me! \\
They beat me, they robbed me!” \\
For those who bear such a grudge, \\
hatred never ends. 

%
\end{verse}

\begin{verse}%
“They\marginnote{4} abused me, they hit me! \\
They beat me, they robbed me!” \\
For those who bear no such grudge, \\
hatred has an end. 

%
\end{verse}

\begin{verse}%
For\marginnote{5} never is hatred \\
settled by hate, \\
it’s only settled by love: \\
this is an eternal truth. 

%
\end{verse}

\begin{verse}%
Others\marginnote{6} don’t understand \\
that here we need to be restrained. \\
But those who do understand this, \\
being clever, settle their conflicts. 

%
\end{verse}

\begin{verse}%
Those\marginnote{7} who contemplate the beautiful, \\
their faculties unrestrained, \\
immoderate in eating, \\
lazy, lacking energy: \\
\textsanskrit{Māra} strikes them down \\
like the wind, a feeble tree. 

%
\end{verse}

\begin{verse}%
Those\marginnote{8} who contemplate the ugly, \\
their faculties well-restrained, \\
eating in moderation, \\
faithful and energetic: \\
\textsanskrit{Māra} cannot strike them down, \\
like the wind, a rocky mountain. 

%
\end{verse}

\begin{verse}%
One\marginnote{9} who, not free of stains themselves, \\
would wear the robe stained in ocher, \\
bereft of self-control and of truth: \\
they are not worthy of the ocher robe. 

%
\end{verse}

\begin{verse}%
One\marginnote{10} who’s purged all their stains, \\
steady in ethics, \\
possessed of self-control and of truth, \\
they are truly worthy of the ocher robe. 

%
\end{verse}

\begin{verse}%
Thinking\marginnote{11} the inessential is essential, \\
seeing the essential as inessential; \\
they don’t realize the essential, \\
for wrong thoughts are their pasture. 

%
\end{verse}

\begin{verse}%
Having\marginnote{12} known the essential as essential, \\
and the inessential as inessential; \\
they realize the essential, \\
for right thoughts are their pasture. 

%
\end{verse}

\begin{verse}%
Just\marginnote{13} as rain seeps into \\
a poorly roofed house, \\
lust seeps into \\
an undeveloped mind. 

%
\end{verse}

\begin{verse}%
Just\marginnote{14} as rain doesn’t seep into \\
a well roofed house, \\
lust doesn’t seep into \\
a well developed mind. 

%
\end{verse}

\begin{verse}%
Here\marginnote{15} they grieve, hereafter they grieve, \\
an evildoer grieves in both places. \\
They grieve and fret, \\
seeing their own corrupt deeds. 

%
\end{verse}

\begin{verse}%
Here\marginnote{16} they rejoice, hereafter they rejoice, \\
one who does good rejoices in both places. \\
They rejoice and celebrate, \\
seeing their own pure deeds. 

%
\end{verse}

\begin{verse}%
Here\marginnote{17} they’re tormented, hereafter they’re tormented, \\
an evildoer is tormented in both places. \\
They’re tormented thinking of bad things they’ve done; \\
when gone to a bad place, they’re tormented all the more. 

%
\end{verse}

\begin{verse}%
Here\marginnote{18} they delight, hereafter they delight, \\
one who does good delights in both places. \\
They delight thinking of good things they’ve done; \\
when gone to a good place, they delight all the more. 

%
\end{verse}

\begin{verse}%
Much\marginnote{19} though they may recite scripture, \\
if a negligent person does not apply them, \\
then, like a cowherd who counts the cattle of others, \\
they miss out on the blessings of the ascetic life. 

%
\end{verse}

\begin{verse}%
Little\marginnote{20} though they may recite scripture, \\
if they live in line with the teachings, \\
having given up greed, hate, and delusion, \\
with deep understanding and heart well-freed, \\
not grasping to this world or the next, \\
they share in the blessings of the ascetic life. 

%
\end{verse}

%
\chapter*{{\suttatitleacronym Dhp 21–32}{\suttatitletranslation 2. Diligence }{\suttatitleroot Appamādavagga}}
\addcontentsline{toc}{chapter}{\tocacronym{Dhp 21–32} \toctranslation{2. Diligence } \tocroot{Appamādavagga}}
\markboth{2. Diligence }{Appamādavagga}
\extramarks{Dhp 21–32}{Dhp 21–32}

\begin{verse}%
Heedfulness\marginnote{21} is the deathless state; \\
heedlessness is the state of death. \\
The heedful do not die, \\
while the heedless are like the dead. 

%
\end{verse}

\begin{verse}%
Understanding\marginnote{22} this distinction \\
when it comes to heedfulness, \\
the astute rejoice in heedfulness, \\
happy in the noble ones’ domain. 

%
\end{verse}

\begin{verse}%
They\marginnote{23} who regularly meditate, \\
always staunchly vigorous; \\
those wise ones realize quenching, \\
the supreme sanctuary. 

%
\end{verse}

\begin{verse}%
For\marginnote{24} the hard-working and mindful, \\
pure of deed and attentive, \\
restrained, living righteously, and diligent, \\
their reputation only grows. 

%
\end{verse}

\begin{verse}%
By\marginnote{25} hard work and diligence, \\
by restraint and by self-control, \\
a smart person would build an island \\
that the floods cannot overflow. 

%
\end{verse}

\begin{verse}%
Fools\marginnote{26} and half-wits \\
devote themselves to negligence. \\
But the wise protect diligence \\
as their best treasure. 

%
\end{verse}

\begin{verse}%
Don’t\marginnote{27} devote yourself to negligence, \\
or delight in sexual intimacy. \\
For if you’re diligent and meditate, \\
you’ll attain abundant happiness. 

%
\end{verse}

\begin{verse}%
When\marginnote{28} the astute dispel negligence \\
by means of diligence, \\
ascending the palace of wisdom, \\
sorrowless, they behold this generation of sorrow, \\
as a wise man on a mountain-top \\
beholds the fools below. 

%
\end{verse}

\begin{verse}%
Heedful\marginnote{29} among the heedless, \\
wide awake while others sleep—\\
a true sage leaves them behind, \\
like a swift horse passing a feeble. 

%
\end{verse}

\begin{verse}%
\textsanskrit{Maghavā}\marginnote{30} became chief of the gods \\
by means of diligence. \\
People praise diligence, \\
while negligence is always deplored. 

%
\end{verse}

\begin{verse}%
A\marginnote{31} mendicant who loves to be diligent, \\
seeing fear in negligence—\\
advances like fire, \\
burning up fetters big and small. 

%
\end{verse}

\begin{verse}%
A\marginnote{32} mendicant who loves to be diligent, \\
seeing fear in negligence—\\
such a one can’t decline, \\
and has drawn near to extinguishment. 

%
\end{verse}

%
\chapter*{{\suttatitleacronym Dhp 33–43}{\suttatitletranslation 3. The Mind }{\suttatitleroot Cittavagga}}
\addcontentsline{toc}{chapter}{\tocacronym{Dhp 33–43} \toctranslation{3. The Mind } \tocroot{Cittavagga}}
\markboth{3. The Mind }{Cittavagga}
\extramarks{Dhp 33–43}{Dhp 33–43}

\begin{verse}%
The\marginnote{33} mind quivers and shakes, \\
hard to guard, hard to curb. \\
The discerning straighten it out, \\
like a fletcher straightens an arrow. 

%
\end{verse}

\begin{verse}%
Like\marginnote{34} a fish pulled from the sea \\
and cast upon the shore, \\
this mind flounders about, \\
trying to throw off \textsanskrit{Māra}’s sway. 

%
\end{verse}

\begin{verse}%
Hard\marginnote{35} to hold back, flighty, \\
alighting where it will; \\
it’s good to tame the mind; \\
a tamed mind leads to bliss. 

%
\end{verse}

\begin{verse}%
So\marginnote{36} hard to see, so subtle, \\
alighting where it will; \\
the discerning protect the mind, \\
a guarded mind leads to bliss. 

%
\end{verse}

\begin{verse}%
The\marginnote{37} mind travels far, wandering alone; \\
incorporeal, it hides in a cave. \\
Those who will restrain the mind \\
are freed from \textsanskrit{Māra}’s bonds. 

%
\end{verse}

\begin{verse}%
Those\marginnote{38} of unsteady mind, \\
who don’t understand the true teaching, \\
and whose confidence wavers, \\
do not perfect their wisdom. 

%
\end{verse}

\begin{verse}%
One\marginnote{39} whose mind is uncorrupted, \\
whose heart is undamaged, \\
who’s given up right and wrong, \\
alert, has nothing to fear. 

%
\end{verse}

\begin{verse}%
Knowing\marginnote{40} this body breaks like a pot, \\
and fortifying the mind like a citadel, \\
attack \textsanskrit{Māra} with the sword of wisdom, \\
guard your conquest, and never settle. 

%
\end{verse}

\begin{verse}%
All\marginnote{41} too soon this body \\
will lie upon the earth, \\
bereft of consciousness, \\
tossed aside like a useless log. 

%
\end{verse}

\begin{verse}%
A\marginnote{42} wrongly directed mind \\
would do you more harm \\
than a hater to the hated, \\
or an enemy to their foe. 

%
\end{verse}

\begin{verse}%
A\marginnote{43} rightly directed mind \\
would do you more good \\
than your mother or father \\
or any other relative. 

%
\end{verse}

%
\chapter*{{\suttatitleacronym Dhp 44–59}{\suttatitletranslation 4. Flowers }{\suttatitleroot Pupphavagga}}
\addcontentsline{toc}{chapter}{\tocacronym{Dhp 44–59} \toctranslation{4. Flowers } \tocroot{Pupphavagga}}
\markboth{4. Flowers }{Pupphavagga}
\extramarks{Dhp 44–59}{Dhp 44–59}

\begin{verse}%
Who\marginnote{44} shall explore this land, \\
and the Yama realm with its gods? \\
Who shall examine the well-taught word of truth, \\
as an expert examines a flower? 

%
\end{verse}

\begin{verse}%
A\marginnote{45} trainee shall explore this land, \\
and the Yama realm with its gods. \\
A trainee shall examine the well-taught word of truth, \\
as an expert examines a flower. 

%
\end{verse}

\begin{verse}%
Knowing\marginnote{46} this body’s like foam, \\
realizing it’s all just a mirage, \\
and cutting off \textsanskrit{Māra}’s blossoming, \\
vanish from the King of Death. 

%
\end{verse}

\begin{verse}%
As\marginnote{47} a mighty flood sweeps off a sleeping village, \\
death steals away a man \\
even as he gathers flowers, \\
his mind caught up in them. 

%
\end{verse}

\begin{verse}%
The\marginnote{48} terminator gains control of the man \\
who has not had his fill of pleasures, \\
even as he gathers flowers, \\
his mind caught up in them. 

%
\end{verse}

\begin{verse}%
A\marginnote{49} bee takes the nectar \\
and moves on, doing no damage \\
to the flower’s beauty and fragrance; \\
and that’s how a sage should walk in the village. 

%
\end{verse}

\begin{verse}%
Don’t\marginnote{50} find fault with others, \\
with what they’ve done or left undone. \\
You should only watch yourself, \\
what you’ve done or left undone. 

%
\end{verse}

\begin{verse}%
Just\marginnote{51} like a glorious flower \\
that’s colorful but lacks fragrance; \\
eloquent speech is fruitless \\
for one who does not act on it. 

%
\end{verse}

\begin{verse}%
Just\marginnote{52} like a glorious flower \\
that’s both colorful and fragrant, \\
eloquent speech is fruitful \\
for one who acts on it. 

%
\end{verse}

\begin{verse}%
Just\marginnote{53} as one would create many garlands \\
from a heap of flowers, \\
when a person has come to be born, \\
they should do many skillful things. 

%
\end{verse}

\begin{verse}%
The\marginnote{54} fragrance of flowers doesn’t spread upwind, \\
nor sandalwood, pinwheel, or jasmine; \\
but the fragrance of the good spreads upwind; \\
a good person’s virtue spreads in every direction. 

%
\end{verse}

\begin{verse}%
Among\marginnote{55} all the fragrances—\\
sandalwood or pinwheel \\
or lotus or jasmine—\\
the fragrance of virtue is supreme. 

%
\end{verse}

\begin{verse}%
Faint\marginnote{56} is the fragrance \\
of sandal or pinwheel; \\
but the fragrance of the virtuous \\
floats to the highest gods. 

%
\end{verse}

\begin{verse}%
For\marginnote{57} those accomplished in ethics, \\
meditating diligently, \\
freed through the highest knowledge, \\
\textsanskrit{Māra} cannot find their path. 

%
\end{verse}

\begin{verse}%
From\marginnote{58} a forsaken heap \\
discarded on the highway, \\
a lotus might blossom, \\
fragrant and delightful. 

%
\end{verse}

\begin{verse}%
So\marginnote{59} too, among the forsaken, \\
a disciple of the perfect Buddha \\
outshines with their wisdom \\
the blind ordinary folk. 

%
\end{verse}

%
\chapter*{{\suttatitleacronym Dhp 60–75}{\suttatitletranslation 5. The Fool }{\suttatitleroot Bālavagga}}
\addcontentsline{toc}{chapter}{\tocacronym{Dhp 60–75} \toctranslation{5. The Fool } \tocroot{Bālavagga}}
\markboth{5. The Fool }{Bālavagga}
\extramarks{Dhp 60–75}{Dhp 60–75}

\begin{verse}%
Long\marginnote{60} is the night for the wakeful; \\
long is the league for the weary; \\
long transmigrate the fools \\
who don’t understand the true teaching. 

%
\end{verse}

\begin{verse}%
If\marginnote{61} while wandering you find no partner \\
equal or better than yourself, \\
then firmly resolve to wander alone—\\
there’s no fellowship with fools. 

%
\end{verse}

\begin{verse}%
“Sons\marginnote{62} are mine, wealth is mine”—\\
thus the fool frets. \\
But you can’t even call your self your own, \\
let alone your sons or wealth. 

%
\end{verse}

\begin{verse}%
The\marginnote{63} fool who thinks they’re a fool \\
is wise at least to that extent. \\
But the true fool is said to be one \\
who imagines that they are wise. 

%
\end{verse}

\begin{verse}%
Though\marginnote{64} a fool attends to the wise \\
even for the rest of their life, \\
they still don’t experience the teaching, \\
like a spoon the taste of the soup. 

%
\end{verse}

\begin{verse}%
If\marginnote{65} a clever person attends to the wise \\
even just for an hour or so, \\
they swiftly experience the teaching, \\
like a tongue the taste of the soup. 

%
\end{verse}

\begin{verse}%
Witless\marginnote{66} fools behave \\
like their own worst enemies, \\
doing wicked deeds \\
that ripen as bitter fruit. 

%
\end{verse}

\begin{verse}%
It’s\marginnote{67} not good to do a deed \\
that plagues you later on, \\
for which you weep and wail, \\
as its effect stays with you. 

%
\end{verse}

\begin{verse}%
It\marginnote{68} is good to do a deed \\
that doesn’t plague you later on, \\
that gladdens and cheers, \\
as its effect stays with you. 

%
\end{verse}

\begin{verse}%
The\marginnote{69} fool imagines that evil is sweet, \\
so long as it has not yet ripened. \\
But as soon as that evil ripens, \\
they fall into suffering. 

%
\end{verse}

\begin{verse}%
Month\marginnote{70} after month a fool may eat \\
food from a grass-blade’s tip; \\
but they’ll never be worth a sixteenth part \\
of one who has fathomed the teaching. 

%
\end{verse}

\begin{verse}%
For\marginnote{71} a wicked deed that has been done \\
does not spoil quickly like milk. \\
Smoldering, it follows the fool, \\
like a fire smothered over with ash. 

%
\end{verse}

\begin{verse}%
Whatever\marginnote{72} fame a fool may get, \\
it only gives rise to harm. \\
Whatever good features they have it ruins, \\
and blows their head into bits. 

%
\end{verse}

\begin{verse}%
They’d\marginnote{73} seek the esteem that they lack, \\
and status among the mendicants; \\
authority over monasteries, \\
and honor among other families. 

%
\end{verse}

\begin{verse}%
“Let\marginnote{74} both layfolk and renunciants think \\
the work was done by me alone. \\
In anything at all that’s to be done, \\
let them fall under my sway alone.” \\
So thinks the fool, \\
their greed and pride only growing. 

%
\end{verse}

\begin{verse}%
For\marginnote{75} the means to profit and the path to quenching \\
are two quite different things. \\
A mendicant disciple of the Buddha, \\
understanding what this really means, \\
would never delight in honors, \\
but rather would foster seclusion. 

%
\end{verse}

%
\chapter*{{\suttatitleacronym Dhp 76–89}{\suttatitletranslation 6. The Astute }{\suttatitleroot Paṇḍitavagga}}
\addcontentsline{toc}{chapter}{\tocacronym{Dhp 76–89} \toctranslation{6. The Astute } \tocroot{Paṇḍitavagga}}
\markboth{6. The Astute }{Paṇḍitavagga}
\extramarks{Dhp 76–89}{Dhp 76–89}

\begin{verse}%
Regard\marginnote{76} one who sees your faults \\
as a guide to a hidden treasure. \\
Stay close to one so wise and astute \\
who corrects you when you need it. \\
Sticking close to such an impartial person, \\
things get better, not worse. 

%
\end{verse}

\begin{verse}%
Advise\marginnote{77} and instruct; \\
curb wickedness: \\
for you shall be loved by the good, \\
and disliked by the bad. 

%
\end{verse}

\begin{verse}%
Don’t\marginnote{78} mix with bad friends, \\
nor with the worst of men. \\
Mix with spiritual friends, \\
and with the best of men. 

%
\end{verse}

\begin{verse}%
Through\marginnote{79} joy in the teaching you sleep at ease, \\
with clear and confident heart. \\
An astute person always delights in the teaching \\
proclaimed by the Noble One. 

%
\end{verse}

\begin{verse}%
While\marginnote{80} irrigators guide water, \\
fletchers straighten arrows, \\
and carpenters carve timber, \\
the astute tame themselves. 

%
\end{verse}

\begin{verse}%
As\marginnote{81} the wind cannot stir \\
a solid mass of rock, \\
so too blame and praise \\
do not affect the wise. 

%
\end{verse}

\begin{verse}%
Like\marginnote{82} a deep lake, \\
clear and unclouded, \\
so clear are the astute \\
when they hear the teachings. 

%
\end{verse}

\begin{verse}%
Good\marginnote{83} people give up everything, \\
they don’t cajole for the things they desire. \\
Though touched by sadness or happiness, \\
the astute appear neither depressed nor elated. 

%
\end{verse}

\begin{verse}%
Never\marginnote{84} wish for success by unjust means, \\
for your own sake or that of another, \\
desiring children, wealth, or nation; \\
rather, be virtuous, wise, and just. 

%
\end{verse}

\begin{verse}%
Few\marginnote{85} are those among humans \\
who cross to the far shore. \\
The rest just run around \\
on the near shore. 

%
\end{verse}

\begin{verse}%
When\marginnote{86} the teaching is well explained, \\
those who practice accordingly \\
will cross over \\
Death’s domain so hard to pass. 

%
\end{verse}

\begin{verse}%
Rid\marginnote{87} of dark qualities, \\
an astute person should develop the bright. \\
Leaving home behind \\
for the seclusion so hard to enjoy, 

%
\end{verse}

\begin{verse}%
find\marginnote{88} delight there, \\
having left behind sensual pleasures. \\
With no possessions, an astute person \\
would cleanse themselves of mental corruptions. 

%
\end{verse}

\begin{verse}%
Those\marginnote{89} whose minds are rightly developed \\
in the awakening factors; \\
who, letting go of attachments, \\
delight in not grasping: \\
with defilements ended, brilliant, \\
they in this world are quenched. 

%
\end{verse}

%
\chapter*{{\suttatitleacronym Dhp 90–99}{\suttatitletranslation 7. The Perfected Ones }{\suttatitleroot Arahantavagga}}
\addcontentsline{toc}{chapter}{\tocacronym{Dhp 90–99} \toctranslation{7. The Perfected Ones } \tocroot{Arahantavagga}}
\markboth{7. The Perfected Ones }{Arahantavagga}
\extramarks{Dhp 90–99}{Dhp 90–99}

\begin{verse}%
At\marginnote{90} journey’s end, rid of sorrow; \\
everywhere free, \\
all ties given up, \\
no fever is found in them. 

%
\end{verse}

\begin{verse}%
The\marginnote{91} mindful apply themselves; \\
they delight in no abode. \\
Like a swan from the marsh that’s gone, \\
they leave behind home after home. 

%
\end{verse}

\begin{verse}%
Those\marginnote{92} with nothing stored up, \\
who have understood their food, \\
whose domain is the liberation \\
of the signless and the empty: \\
their path is hard to trace, \\
like birds in the sky. 

%
\end{verse}

\begin{verse}%
One\marginnote{93} whose defilements have ended; \\
who’s not attached to food; \\
whose domain is the liberation \\
of the signless and the empty: \\
their track is hard to trace, \\
like birds in the sky. 

%
\end{verse}

\begin{verse}%
Whose\marginnote{94} faculties have become serene, \\
like horses tamed by a charioteer, \\
who has abandoned conceit and defilements; \\
the poised one is envied by even the gods. 

%
\end{verse}

\begin{verse}%
Undisturbed\marginnote{95} like the earth, \\
true to their vows, steady as a post, \\
like a lake clear of mud; \\
such a one does not transmigrate. 

%
\end{verse}

\begin{verse}%
Their\marginnote{96} mind is peaceful, \\
peaceful are their speech and deeds. \\
Such a one is at peace, \\
rightly freed through enlightenment. 

%
\end{verse}

\begin{verse}%
Lacking\marginnote{97} faith, a house-breaker, \\
one who acknowledges nothing, \\
purged of hope, they’ve wasted their chance: \\
that is indeed the supreme person! 

%
\end{verse}

\begin{verse}%
Whether\marginnote{98} in village or wilderness, \\
in a valley or the uplands, \\
wherever the perfected ones live \\
is a delightful place. 

%
\end{verse}

\begin{verse}%
Delightful\marginnote{99} are the wildernesses \\
where no people delight. \\
Those free of greed will delight there, \\
not those who seek sensual pleasures. 

%
\end{verse}

%
\chapter*{{\suttatitleacronym Dhp 100–115}{\suttatitletranslation 8. The Thousands }{\suttatitleroot Sahassavagga}}
\addcontentsline{toc}{chapter}{\tocacronym{Dhp 100–115} \toctranslation{8. The Thousands } \tocroot{Sahassavagga}}
\markboth{8. The Thousands }{Sahassavagga}
\extramarks{Dhp 100–115}{Dhp 100–115}

\begin{verse}%
Better\marginnote{100} than a thousand \\
meaningless sayings \\
is a single meaningful saying, \\
hearing which brings you peace. 

%
\end{verse}

\begin{verse}%
Better\marginnote{101} than a thousand \\
meaningless verses \\
is a single meaningful verse, \\
hearing which brings you peace. 

%
\end{verse}

\begin{verse}%
Better\marginnote{102} than reciting \\
a hundred meaningless verses \\
is a single saying of Dhamma, \\
hearing which brings you peace. 

%
\end{verse}

\begin{verse}%
The\marginnote{103} supreme conqueror is \\
not he who conquers a million men in battle, \\
but he who conquers a single man: \\
himself. 

%
\end{verse}

\begin{verse}%
It\marginnote{104} is surely better to conquer oneself \\
than all those other folk. \\
When a person has tamed themselves, \\
always living restrained, 

%
\end{verse}

\begin{verse}%
no\marginnote{105} god nor fairy, \\
nor \textsanskrit{Māra} nor \textsanskrit{Brahmā}, \\
can undo the victory \\
of such a one. 

%
\end{verse}

\begin{verse}%
Rather\marginnote{106} than a thousand-fold sacrifice, \\
every month for a hundred years, \\
it’s better to honor for a single moment \\
one who has developed themselves. \\
That offering is better \\
than the hundred year sacrifice. 

%
\end{verse}

\begin{verse}%
Rather\marginnote{107} than serve the sacred flame \\
in the forest for a hundred years, \\
it’s better to honor for a single moment \\
one who has developed themselves. \\
That offering is better \\
than the hundred year sacrifice. 

%
\end{verse}

\begin{verse}%
Whatever\marginnote{108} sacrifice or offering in the world \\
a seeker of merit may make for a year, \\
none of it is worth a quarter \\
of bowing to the upright. 

%
\end{verse}

\begin{verse}%
For\marginnote{109} one in the habit of bowing, \\
always honoring the elders, \\
four blessings grow: \\
lifespan, beauty, happiness, and strength. 

%
\end{verse}

\begin{verse}%
Better\marginnote{110} to live a single day \\
ethical and absorbed in meditation \\
than to live a hundred years \\
unethical and lacking immersion. 

%
\end{verse}

\begin{verse}%
Better\marginnote{111} to live a single day \\
wise and absorbed in meditation \\
than to live a hundred years \\
witless and lacking immersion. 

%
\end{verse}

\begin{verse}%
Better\marginnote{112} to live a single day \\
energetic and strong, \\
than to live a hundred years \\
lazy and lacking energy. 

%
\end{verse}

\begin{verse}%
Better\marginnote{113} to live a single day \\
seeing rise and fall \\
than to live a hundred years \\
blind to rise and fall. 

%
\end{verse}

\begin{verse}%
Better\marginnote{114} to live a single day \\
seeing the deathless state \\
than to live a hundred years \\
blind to the deathless state. 

%
\end{verse}

\begin{verse}%
Better\marginnote{115} to live a single day \\
seeing the supreme teaching \\
than to live a hundred years \\
blind to the supreme teaching. 

%
\end{verse}

%
\chapter*{{\suttatitleacronym Dhp 116–128}{\suttatitletranslation 9. Wickedness }{\suttatitleroot Pāpavagga}}
\addcontentsline{toc}{chapter}{\tocacronym{Dhp 116–128} \toctranslation{9. Wickedness } \tocroot{Pāpavagga}}
\markboth{9. Wickedness }{Pāpavagga}
\extramarks{Dhp 116–128}{Dhp 116–128}

\begin{verse}%
Rush\marginnote{116} to do good, \\
shield your mind from evil; \\
for when you’re slow to do good, \\
your thoughts delight in wickedness. 

%
\end{verse}

\begin{verse}%
If\marginnote{117} you do something bad, \\
don’t do it again and again, \\
don’t set your heart on it, \\
for piling up evil is suffering. 

%
\end{verse}

\begin{verse}%
If\marginnote{118} you do something good, \\
do it again and again, \\
set your heart on it, \\
for piling up goodness is joyful. 

%
\end{verse}

\begin{verse}%
Even\marginnote{119} the wicked see good things, \\
so long as their wickedness has not ripened. \\
But as soon as that wickedness ripens, \\
then the wicked see wicked things. 

%
\end{verse}

\begin{verse}%
Even\marginnote{120} the good see wicked things, \\
so long as their goodness has not ripened. \\
But as soon as that goodness ripens, \\
then the good see good things. 

%
\end{verse}

\begin{verse}%
Think\marginnote{121} not lightly of evil, \\
that it won’t come back to you. \\
The pot is filled with water \\
falling drop by drop; \\
the fool is filled with wickedness \\
piled up bit by bit. 

%
\end{verse}

\begin{verse}%
Think\marginnote{122} not lightly of goodness, \\
that it won’t come back to you. \\
The pot is filled with water \\
falling drop by drop; \\
the sage is filled with goodness \\
piled up bit by bit. 

%
\end{verse}

\begin{verse}%
Avoid\marginnote{123} wickedness, \\
as a merchant with rich cargo and small escort \\
would avoid a dangerous road, \\
or one who loves life would avoid drinking poison. 

%
\end{verse}

\begin{verse}%
You\marginnote{124} can carry poison in your hand \\
if it has no wound, \\
for poison does not infect without a wound; \\
nothing bad happens unless you do bad. 

%
\end{verse}

\begin{verse}%
Whoever\marginnote{125} wrongs a man who has done no wrong, \\
a pure man who has not a blemish, \\
the evil backfires on the fool, \\
like fine dust thrown upwind. 

%
\end{verse}

\begin{verse}%
Some\marginnote{126} are born in a womb; \\
evil-doers go to hell; \\
the virtuous go to heaven; \\
the undefiled become fully extinguished. 

%
\end{verse}

\begin{verse}%
Not\marginnote{127} in the sky, nor mid-ocean, \\
nor hiding in a mountain cleft; \\
you’ll find no place in the world \\
to escape your wicked deeds. 

%
\end{verse}

\begin{verse}%
Not\marginnote{128} in the sky, nor mid-ocean, \\
nor hiding in a mountain cleft; \\
you’ll find no place in the world \\
where you won’t be vanquished by death. 

%
\end{verse}

%
\chapter*{{\suttatitleacronym Dhp 129–145}{\suttatitletranslation 10. The Rod }{\suttatitleroot Daṇḍavagga}}
\addcontentsline{toc}{chapter}{\tocacronym{Dhp 129–145} \toctranslation{10. The Rod } \tocroot{Daṇḍavagga}}
\markboth{10. The Rod }{Daṇḍavagga}
\extramarks{Dhp 129–145}{Dhp 129–145}

\begin{verse}%
All\marginnote{129} tremble at the rod, \\
all fear death. \\
Treating others like oneself, \\
neither kill nor incite to kill. 

%
\end{verse}

\begin{verse}%
All\marginnote{130} tremble at the rod, \\
all love life. \\
Treating others like oneself, \\
neither kill nor incite to kill. 

%
\end{verse}

\begin{verse}%
Creatures\marginnote{131} love happiness, \\
so if you harm them with a stick \\
in search of your own happiness, \\
after death you won’t find happiness. 

%
\end{verse}

\begin{verse}%
Creatures\marginnote{132} love happiness, \\
so if you don’t hurt them with a stick \\
in search of your own happiness, \\
after death you will find happiness. 

%
\end{verse}

\begin{verse}%
Don’t\marginnote{133} speak harshly, \\
they may speak harshly back. \\
For aggressive speech is painful, \\
and the rod may spring back on you. 

%
\end{verse}

\begin{verse}%
If\marginnote{134} you still yourself \\
like a broken gong, \\
you’re quenched \\
and conflict-free. 

%
\end{verse}

\begin{verse}%
As\marginnote{135} a cowherd drives the cows \\
to pasture with the rod, \\
so too old age and death \\
drive life from living beings. 

%
\end{verse}

\begin{verse}%
The\marginnote{136} fool does not understand \\
the evil that they do. \\
But because of those deeds, that dullard \\
is tormented as if burnt by fire. 

%
\end{verse}

\begin{verse}%
One\marginnote{137} who violently attacks \\
the peaceful and the innocent \\
swiftly falls \\
to one of ten bad states: 

%
\end{verse}

\begin{verse}%
harsh\marginnote{138} pain; loss; \\
the breakup of the body; \\
serious illness; \\
mental distress; 

%
\end{verse}

\begin{verse}%
hazards\marginnote{139} from rulers; \\
vicious slander; \\
loss of kin; \\
destruction of wealth; 

%
\end{verse}

\begin{verse}%
or\marginnote{140} else their home \\
is consumed by fire. \\
When their body breaks up, that witless person \\
is reborn in hell. 

%
\end{verse}

\begin{verse}%
Not\marginnote{141} nakedness, nor matted hair, nor mud, \\
nor fasting, nor lying on bare ground, \\
nor wearing dust and dirt, or squatting on the heels, \\
will cleanse a mortal not free of doubt. 

%
\end{verse}

\begin{verse}%
Dressed-up\marginnote{142} they may be, but if they live well—\\
peaceful, tamed, committed to the spiritual path, \\
having laid aside violence towards all creatures—\\
they are a brahmin, an ascetic, a mendicant. 

%
\end{verse}

\begin{verse}%
Can\marginnote{143} a person constrained by conscience \\
be found in the world? \\
Who shies away from blame, \\
like a fine horse from the whip? 

%
\end{verse}

\begin{verse}%
Like\marginnote{144} a fine horse under the whip, \\
be keen and full of urgency. \\
With faith, ethics, and energy, \\
immersion, and investigation of principles, \\
accomplished in knowledge and conduct, mindful, \\
give up this vast suffering. 

%
\end{verse}

\begin{verse}%
While\marginnote{145} irrigators guide water, \\
fletchers shape arrows, \\
and carpenters carve timber—\\
those true to their vows tame themselves. 

%
\end{verse}

%
\chapter*{{\suttatitleacronym Dhp 146–156}{\suttatitletranslation 11. Old Age }{\suttatitleroot Jarāvagga}}
\addcontentsline{toc}{chapter}{\tocacronym{Dhp 146–156} \toctranslation{11. Old Age } \tocroot{Jarāvagga}}
\markboth{11. Old Age }{Jarāvagga}
\extramarks{Dhp 146–156}{Dhp 146–156}

\begin{verse}%
What\marginnote{146} is joy, what is laughter, \\
when the flames are ever burning? \\
Shrouded by darkness, \\
would you not seek a light? 

%
\end{verse}

\begin{verse}%
See\marginnote{147} this fancy puppet, \\
a body built of sores, \\
diseased, obsessed over, \\
in which nothing lasts at all. 

%
\end{verse}

\begin{verse}%
This\marginnote{148} body is decrepit and frail, \\
a nest of disease. \\
This foul carcass falls apart, \\
for life ends only in death. 

%
\end{verse}

\begin{verse}%
These\marginnote{149} dove-grey bones \\
are tossed away like \\
dried gourds in the autumn—\\
what joy is there in such a sight? 

%
\end{verse}

\begin{verse}%
In\marginnote{150} this city built of bones, \\
plastered with flesh and blood, \\
old age and death are stashed away, \\
along with conceit and contempt. 

%
\end{verse}

\begin{verse}%
Fancy\marginnote{151} chariots of kings wear out, \\
and even this body gets old. \\
But the teaching of the good never gets old; \\
so the true and the good proclaim. 

%
\end{verse}

\begin{verse}%
A\marginnote{152} person of little learning \\
ages like an ox—\\
their flesh grows, \\
but not their wisdom. 

%
\end{verse}

\begin{verse}%
Transmigrating\marginnote{153} through countless rebirths, \\
I’ve journeyed without reward, \\
searching for the house-builder; \\
painful is birth again and again. 

%
\end{verse}

\begin{verse}%
I’ve\marginnote{154} seen you, house-builder! \\
You won’t build a house again! \\
Your rafters are all broken, \\
your roof-peak is demolished. \\
My mind, set on demolition, \\
has reached the end of craving. 

%
\end{verse}

\begin{verse}%
When\marginnote{155} young they spurned the spiritual path \\
and failed to earn any wealth. \\
Now they languish like old cranes \\
in a pond bereft of fish. 

%
\end{verse}

\begin{verse}%
When\marginnote{156} young they spurned the spiritual path \\
and failed to earn any wealth. \\
Now they lie like spent arrows, \\
bemoaning over things past. 

%
\end{verse}

%
\chapter*{{\suttatitleacronym Dhp 157–166}{\suttatitletranslation 12. The Self }{\suttatitleroot Attavagga}}
\addcontentsline{toc}{chapter}{\tocacronym{Dhp 157–166} \toctranslation{12. The Self } \tocroot{Attavagga}}
\markboth{12. The Self }{Attavagga}
\extramarks{Dhp 157–166}{Dhp 157–166}

\begin{verse}%
If\marginnote{157} you’d only love yourself, \\
you’d look after yourself right well. \\
In one of the night’s three watches, \\
an astute person would remain alert. 

%
\end{verse}

\begin{verse}%
The\marginnote{158} astute would avoid being corrupted \\
by grounding themselves first of all \\
in what is suitable, \\
and only then instructing others. 

%
\end{verse}

\begin{verse}%
If\marginnote{159} one so acts \\
as one instructs, \\
the well-tamed would tame others, \\
for the self is hard to tame, they say. 

%
\end{verse}

\begin{verse}%
Self\marginnote{160} is indeed the lord of self, \\
for who else would be one’s lord? \\
When one’s self is well-tamed, \\
one gains a lord that’s rare indeed. 

%
\end{verse}

\begin{verse}%
For\marginnote{161} the evil that one does, \\
born and produced in oneself, \\
grinds down a fool, \\
as diamond grinds a lesser gem. 

%
\end{verse}

\begin{verse}%
One\marginnote{162} choked by immorality, \\
as a sal tree by a creeper, \\
does to themselves \\
what a foe only wishes. 

%
\end{verse}

\begin{verse}%
It’s\marginnote{163} easy to do bad things \\
harmful to oneself, \\
but good things that are helpful \\
are the hardest things to do. 

%
\end{verse}

\begin{verse}%
On\marginnote{164} account of wicked views—\\
scorning the guidance \\
of the perfected ones, \\
the noble ones living righteously—\\
the idiot begets their own demise, \\
like the bamboo bearing fruit. 

%
\end{verse}

\begin{verse}%
For\marginnote{165} it is by oneself that evil’s done, \\
one is corrupted by oneself. \\
It’s by oneself that evil’s not done, \\
one is purified by oneself. \\
Purity and impurity are personal matters, \\
no one can purify another. 

%
\end{verse}

\begin{verse}%
Never\marginnote{166} neglect your own good \\
for the sake of another, however great. \\
Knowing well what’s good for you, \\
be intent upon your true goal. 

%
\end{verse}

%
\chapter*{{\suttatitleacronym Dhp 167–178}{\suttatitletranslation 13. The World }{\suttatitleroot Lokavagga}}
\addcontentsline{toc}{chapter}{\tocacronym{Dhp 167–178} \toctranslation{13. The World } \tocroot{Lokavagga}}
\markboth{13. The World }{Lokavagga}
\extramarks{Dhp 167–178}{Dhp 167–178}

\begin{verse}%
Don’t\marginnote{167} resort to lowly things, \\
don’t abide in negligence, \\
don’t resort to wrong views, \\
don’t perpetuate the world. 

%
\end{verse}

\begin{verse}%
Get\marginnote{168} up, don’t be heedless, \\
live by principle, with good conduct. \\
For one of good conduct sleeps at ease, \\
in this world and the next. 

%
\end{verse}

\begin{verse}%
Live\marginnote{169} by principle, with good conduct, \\
don’t conduct yourself badly. \\
For one of good conduct sleeps at ease, \\
in this world and the next. 

%
\end{verse}

\begin{verse}%
Look\marginnote{170} upon the world \\
as a bubble \\
or a mirage, \\
then the King of Death won’t see you. 

%
\end{verse}

\begin{verse}%
Come,\marginnote{171} see this world decked out \\
like a fancy royal chariot. \\
Here fools flounder, \\
but the discerning are not chained. 

%
\end{verse}

\begin{verse}%
He\marginnote{172} who once was heedless, \\
but turned to heedfulness, \\
lights up the world \\
like the moon freed from clouds. 

%
\end{verse}

\begin{verse}%
Someone\marginnote{173} whose bad deed \\
is supplanted by the good, \\
lights up the world, \\
like the moon freed from clouds. 

%
\end{verse}

\begin{verse}%
Blind\marginnote{174} is the world, \\
few are those who clearly see. \\
Only a handful go to heaven, \\
like a bird freed from a net. 

%
\end{verse}

\begin{verse}%
Swans\marginnote{175} fly by the sun’s path, \\
psychic sages fly through space. \\
The wise leave the world, \\
having vanquished \textsanskrit{Māra} and his mount. 

%
\end{verse}

\begin{verse}%
When\marginnote{176} a person, spurning the hereafter, \\
transgresses in just one thing—\\
lying—\\
there is no evil they would not do. 

%
\end{verse}

\begin{verse}%
The\marginnote{177} miserly don’t ascend to heaven, \\
it takes a fool to not praise giving. \\
The wise celebrate giving, \\
and so find happiness in the hereafter. 

%
\end{verse}

\begin{verse}%
The\marginnote{178} fruit of stream-entry is better \\
than being the one king of the earth, \\
than going to heaven, \\
than lordship over all the world. 

%
\end{verse}

%
\chapter*{{\suttatitleacronym Dhp 179–196}{\suttatitletranslation 14. The Buddhas }{\suttatitleroot Buddhavagga}}
\addcontentsline{toc}{chapter}{\tocacronym{Dhp 179–196} \toctranslation{14. The Buddhas } \tocroot{Buddhavagga}}
\markboth{14. The Buddhas }{Buddhavagga}
\extramarks{Dhp 179–196}{Dhp 179–196}

\begin{verse}%
He\marginnote{179} whose victory may not be undone, \\
a victory unrivaled in all the world; \\
by what track would you trace that Buddha, \\
who leaves no track in his infinite range? 

%
\end{verse}

\begin{verse}%
Of\marginnote{180} craving, the weaver, the clinger, he has none: \\
so where can he be traced? \\
By what track would you trace that Buddha, \\
who leaves no track in his infinite range? 

%
\end{verse}

\begin{verse}%
The\marginnote{181} wise intent on absorption, \\
who love the peace of renunciation, \\
the Buddhas, ever mindful, \\
are envied by even the gods. 

%
\end{verse}

\begin{verse}%
It’s\marginnote{182} hard to gain a human birth; \\
the life of mortals is hard; \\
it’s hard to hear the true teaching; \\
the arising of Buddhas is hard. 

%
\end{verse}

\begin{verse}%
Not\marginnote{183} to do any evil; \\
to embrace the good; \\
to purify one’s mind: \\
this is the instruction of the Buddhas. 

%
\end{verse}

\begin{verse}%
Patient\marginnote{184} acceptance is the ultimate austerity. \\
Extinguishment is the ultimate, say the Buddhas. \\
No true renunciate injures another, \\
nor does an ascetic hurt another. 

%
\end{verse}

\begin{verse}%
Not\marginnote{185} speaking ill nor doing harm; \\
restraint in the monastic code; \\
moderation in eating; \\
staying in remote lodgings; \\
commitment to the higher mind—\\
this is the instruction of the Buddhas. 

%
\end{verse}

\begin{verse}%
Even\marginnote{186} if it were raining money, \\
you’d not be sated in sensual pleasures. \\
An astute person understands that sensual pleasures \\
offer little gratification and much suffering. 

%
\end{verse}

\begin{verse}%
Thus\marginnote{187} they find no delight \\
even in celestial pleasures. \\
A disciple of the fully awakened Buddha \\
delights in the ending of craving. 

%
\end{verse}

\begin{verse}%
So\marginnote{188} many go for refuge \\
to mountains and forest groves, \\
to tree-shrines in tended parks; \\
those people are driven by fear. 

%
\end{verse}

\begin{verse}%
But\marginnote{189} such refuge is no sanctuary, \\
it is no supreme refuge. \\
By going to that refuge, \\
you’re not released from suffering. 

%
\end{verse}

\begin{verse}%
One\marginnote{190} gone for refuge to the Buddha, \\
to his teaching and to the \textsanskrit{Saṅgha}, \\
sees the four noble truths \\
with right understanding: 

%
\end{verse}

\begin{verse}%
suffering,\marginnote{191} suffering’s origin, \\
suffering’s transcendence, \\
and the noble eightfold path \\
that leads to the stilling of suffering. 

%
\end{verse}

\begin{verse}%
Such\marginnote{192} refuge is a sanctuary, \\
it is the supreme refuge. \\
By going to that refuge, \\
you’re released from all suffering. 

%
\end{verse}

\begin{verse}%
It’s\marginnote{193} hard to find a thoroughbred man \\
they’re not born just anywhere. \\
A family where that sage is born \\
prospers in happiness. 

%
\end{verse}

\begin{verse}%
Happy,\marginnote{194} the arising of Buddhas! \\
Happy, the teaching of Dhamma! \\
Happy is the harmony of the \textsanskrit{Saṅgha}, \\
and the striving of the harmonious is happy. 

%
\end{verse}

\begin{verse}%
When\marginnote{195} a person venerates the worthy—\\
the Buddha or his disciple, \\
who have transcended proliferation, \\
and have left behind grief and lamentation, 

%
\end{verse}

\begin{verse}%
quenched,\marginnote{196} fearing nothing from any quarter—\\
the merit of one venerating such as these, \\
cannot be calculated by anyone, \\
saying it is just this much. 

%
\end{verse}

%
\chapter*{{\suttatitleacronym Dhp 197–208}{\suttatitletranslation 15. Happiness }{\suttatitleroot Sukhavagga}}
\addcontentsline{toc}{chapter}{\tocacronym{Dhp 197–208} \toctranslation{15. Happiness } \tocroot{Sukhavagga}}
\markboth{15. Happiness }{Sukhavagga}
\extramarks{Dhp 197–208}{Dhp 197–208}

\begin{verse}%
Let\marginnote{197} us live so very happily, \\
loving among the hostile. \\
Among hostile people, \\
let us live with love. 

%
\end{verse}

\begin{verse}%
Let\marginnote{198} us live so very happily, \\
healthy among the ailing. \\
Among ailing people \\
let us live healthily. 

%
\end{verse}

\begin{verse}%
Let\marginnote{199} us live so very happily, \\
content among the greedy. \\
Among greedy people, \\
let us live content. 

%
\end{verse}

\begin{verse}%
Let\marginnote{200} us live so very happily, \\
we who have nothing. \\
We shall feed on rapture, \\
like the gods of streaming radiance. 

%
\end{verse}

\begin{verse}%
Victory\marginnote{201} breeds enmity; \\
the defeated sleep badly. \\
The peaceful sleep at ease, \\
having left victory and defeat behind. 

%
\end{verse}

\begin{verse}%
There\marginnote{202} is no fire like greed, \\
no crime like hate, \\
no suffering like the aggregates, \\
no bliss beyond peace. 

%
\end{verse}

\begin{verse}%
Hunger\marginnote{203} is the worst illness, \\
conditions are the worst suffering, \\
For one who truly knows this, \\
extinguishment is the ultimate happiness. 

%
\end{verse}

\begin{verse}%
Health\marginnote{204} is the ultimate blessing; \\
contentment, the ultimate wealth; \\
trust is the ultimate family; \\
extinguishment, the ultimate happiness. 

%
\end{verse}

\begin{verse}%
Having\marginnote{205} drunk the nectar of seclusion \\
and the nectar of peace, \\
free of stress, free of evil, \\
one drinks the joyous nectar of Dhamma. 

%
\end{verse}

\begin{verse}%
It’s\marginnote{206} good to see the noble ones, \\
staying with them is always good. \\
Were you not to see fools, \\
you’d always be happy. 

%
\end{verse}

\begin{verse}%
For\marginnote{207} one who consorts with fools \\
grieves long. \\
Painful is dwelling with fools, \\
like being stuck with your enemy. \\
Happy is dwelling with a sage, \\
like meeting with your kin. 

%
\end{verse}

\begin{verse}%
Therefore:\marginnote{208} \\
A sage, wise and learned, \\
a mammoth of virtue, true to their vows, noble: \\
follow a good and intelligent person such as this, \\
as the moon tracks the path of the stars. 

%
\end{verse}

%
\chapter*{{\suttatitleacronym Dhp 209–220}{\suttatitletranslation 16. The Beloved }{\suttatitleroot Piyavagga}}
\addcontentsline{toc}{chapter}{\tocacronym{Dhp 209–220} \toctranslation{16. The Beloved } \tocroot{Piyavagga}}
\markboth{16. The Beloved }{Piyavagga}
\extramarks{Dhp 209–220}{Dhp 209–220}

\begin{verse}%
Applying\marginnote{209} yourself where you ought not, \\
neglecting what you should be doing, \\
forgetting your goal, you cling to what you hold dear, \\
jealous of those devoted to their own goal. 

%
\end{verse}

\begin{verse}%
Don’t\marginnote{210} ever get too close \\
to those you like or dislike. \\
For not seeing the liked is suffering, \\
and so is seeing the disliked. 

%
\end{verse}

\begin{verse}%
Therefore\marginnote{211} don’t hold anything dear, \\
for it’s bad to lose those you love. \\
No ties are found in they who \\
hold nothing loved or loathed. 

%
\end{verse}

\begin{verse}%
Sorrow\marginnote{212} springs from what we hold dear, \\
fear springs from what we hold dear; \\
one free from holding anything dear \\
has no sorrow, let alone fear. 

%
\end{verse}

\begin{verse}%
Sorrow\marginnote{213} springs from attachment, \\
fear springs from attachment; \\
one free from attachment \\
has no sorrow, let alone fear. 

%
\end{verse}

\begin{verse}%
Sorrow\marginnote{214} springs from relishing, \\
fear springs from relishing; \\
one free from relishing \\
has no sorrow, let alone fear. 

%
\end{verse}

\begin{verse}%
Sorrow\marginnote{215} springs from desire, \\
fear springs from desire; \\
one free from desire \\
has no sorrow, let alone fear. 

%
\end{verse}

\begin{verse}%
Sorrow\marginnote{216} springs from craving, \\
fear springs from craving; \\
one free from craving \\
has no sorrow, let alone fear. 

%
\end{verse}

\begin{verse}%
One\marginnote{217} accomplished in virtue and vision, \\
firm in principle, and truthful, \\
doing oneself what ought be done: \\
that’s who the people love. 

%
\end{verse}

\begin{verse}%
One\marginnote{218} eager to realize the ineffable \\
would be filled with awareness. \\
Their mind not bound to pleasures of sense, \\
they’re said to be heading upstream. 

%
\end{verse}

\begin{verse}%
When\marginnote{219} a man returns safely \\
after a long time spent abroad, \\
family, friends, and loved ones \\
celebrate his return. 

%
\end{verse}

\begin{verse}%
Just\marginnote{220} so, when one who has done good \\
goes from this world to the next, \\
their good deeds receive them there, \\
as family welcomes home one they love. 

%
\end{verse}

%
\chapter*{{\suttatitleacronym Dhp 221–234}{\suttatitletranslation 17. Anger }{\suttatitleroot Kodhavagga}}
\addcontentsline{toc}{chapter}{\tocacronym{Dhp 221–234} \toctranslation{17. Anger } \tocroot{Kodhavagga}}
\markboth{17. Anger }{Kodhavagga}
\extramarks{Dhp 221–234}{Dhp 221–234}

\begin{verse}%
Give\marginnote{221} up anger, get rid of conceit, \\
and escape every fetter. \\
Sufferings don’t befall one who has nothing, \\
not clinging to name and form. 

%
\end{verse}

\begin{verse}%
When\marginnote{222} anger surges like a lurching chariot, \\
keep it in check. \\
That’s what I call a charioteer; \\
others just hold the reins. 

%
\end{verse}

\begin{verse}%
Defeat\marginnote{223} anger with kindness, \\
villainy with virtue, \\
stinginess with giving, \\
and lies with truth. 

%
\end{verse}

\begin{verse}%
Speak\marginnote{224} the truth, do not be angry, \\
and give when asked, if only a little. \\
By these three means, \\
you may enter the presence of the gods. 

%
\end{verse}

\begin{verse}%
Those\marginnote{225} harmless sages, \\
always restrained in body, \\
go to the imperishable state, \\
where there is no sorrow. 

%
\end{verse}

\begin{verse}%
Always\marginnote{226} wakeful, \\
practicing night and day, \\
focused only on quenching, \\
their defilements come to an end. 

%
\end{verse}

\begin{verse}%
It’s\marginnote{227} always been like this, \\
it’s not just today. \\
They blame you when you’re silent, \\
they blame you when you speak a lot, \\
and even when you speak just right: \\
no-one in the world escapes blame. 

%
\end{verse}

\begin{verse}%
There\marginnote{228} never was, nor will be, \\
nor is there today, \\
someone who is wholly praised \\
or wholly blamed. 

%
\end{verse}

\begin{verse}%
If,\marginnote{229} after watching them day in day out, \\
discerning people praise \\
that sage of impeccable conduct, \\
endowed with ethics and wisdom; 

%
\end{verse}

\begin{verse}%
like\marginnote{230} a pendant of river gold, \\
who is worthy to criticize them? \\
Even the gods praise them, \\
and by \textsanskrit{Brahmā}, too, they’re praised. 

%
\end{verse}

\begin{verse}%
Guard\marginnote{231} against ill-tempered deeds, \\
be restrained in body. \\
Giving up bad bodily conduct, \\
conduct yourself well in body. 

%
\end{verse}

\begin{verse}%
Guard\marginnote{232} against ill-tempered words, \\
be restrained in speech. \\
Giving up bad verbal conduct, \\
conduct yourself well in speech. 

%
\end{verse}

\begin{verse}%
Guard\marginnote{233} against ill-tempered thoughts, \\
be restrained in mind. \\
Giving up bad mental conduct, \\
conduct yourself well in mind. 

%
\end{verse}

\begin{verse}%
A\marginnote{234} sage is restrained in body \\
restrained also in speech, \\
in thought, too, they are restrained: \\
they are restrained in every way. 

%
\end{verse}

%
\chapter*{{\suttatitleacronym Dhp 235–255}{\suttatitletranslation 18. Stains }{\suttatitleroot Malavagga}}
\addcontentsline{toc}{chapter}{\tocacronym{Dhp 235–255} \toctranslation{18. Stains } \tocroot{Malavagga}}
\markboth{18. Stains }{Malavagga}
\extramarks{Dhp 235–255}{Dhp 235–255}

\begin{verse}%
Today\marginnote{235} you’re like a withered leaf, \\
Yama’s men await you. \\
You stand at the departure gates, \\
yet you have no supplies for the road. 

%
\end{verse}

\begin{verse}%
Make\marginnote{236} an island of yourself! \\
Swiftly strive, learn to be wise! \\
Purged of stains, flawless, \\
you’ll go to the divine realm of the noble ones. 

%
\end{verse}

\begin{verse}%
You’ve\marginnote{237} journeyed the stages of life, \\
and now you set out to meet Yama. \\
Along the way there’s nowhere to stay, \\
yet you have no supplies for the road. 

%
\end{verse}

\begin{verse}%
Make\marginnote{238} an island of yourself! \\
Swiftly strive, learn to be wise! \\
Purged of stains, flawless, \\
you’ll not come again to rebirth and old age. 

%
\end{verse}

\begin{verse}%
A\marginnote{239} smart person would purge \\
their own stains gradually, \\
bit by bit, moment by moment, \\
like a smith smelting silver. 

%
\end{verse}

\begin{verse}%
It\marginnote{240} is the rust born on the iron \\
that eats away the place it arose. \\
And so it is their own deeds \\
that lead the overly-ascetic to a bad place. 

%
\end{verse}

\begin{verse}%
Not\marginnote{241} reciting is the stain of hymns. \\
The stain of houses is neglect. \\
Laziness is the stain of beauty. \\
A guard’s stain is negligence. 

%
\end{verse}

\begin{verse}%
Misconduct\marginnote{242} is a woman’s stain. \\
A giver’s stain is stinginess. \\
Bad qualities are a stain \\
in this world and the next. 

%
\end{verse}

\begin{verse}%
But\marginnote{243} a worse stain than these \\
is ignorance, the worst stain of all. \\
Having given up that stain, \\
be without stains, mendicants! 

%
\end{verse}

\begin{verse}%
Life\marginnote{244} is easy for the shameless. \\
With all the rude courage of a crow, \\
they live pushy, \\
rude, and corrupt. 

%
\end{verse}

\begin{verse}%
Life\marginnote{245} is hard for the conscientious, \\
always seeking purity, \\
neither clinging nor rude, \\
pure of livelihood and discerning. 

%
\end{verse}

\begin{verse}%
Take\marginnote{246} anyone in this world \\
who kills living creatures, \\
speaks falsely, steals, \\
commits adultery, 

%
\end{verse}

\begin{verse}%
and\marginnote{247} indulges in drinking \\
alcohol and liquor. \\
Right here they dig up \\
the root of their own self. 

%
\end{verse}

\begin{verse}%
Know\marginnote{248} this, good sir: \\
they are unrestrained and wicked. \\
Don’t let greed and hate \\
subject you to pain for long. 

%
\end{verse}

\begin{verse}%
The\marginnote{249} people give according to their faith, \\
according to their confidence. \\
If you get upset over that, \\
over other’s food and drink, \\
you’ll not, by day or by night, \\
become immersed in \textsanskrit{samādhi}. 

%
\end{verse}

\begin{verse}%
Those\marginnote{250} who have cut that out, \\
dug it up at the root, eradicated it, \\
they will, by day or by night, \\
become immersed in \textsanskrit{samādhi}. 

%
\end{verse}

\begin{verse}%
There\marginnote{251} is no fire like greed, \\
no crime like hate, \\
no net like delusion, \\
no river like craving. 

%
\end{verse}

\begin{verse}%
It’s\marginnote{252} easy to see the faults of others, \\
hard to see one’s own. \\
For the faults of others \\
are tossed high like chaff, \\
while one’s own are hidden, \\
as a cheat hides a bad hand. 

%
\end{verse}

\begin{verse}%
When\marginnote{253} you look for the flaws of others, \\
always finding fault, \\
your defilements only grow, \\
you’re far from ending defilements. 

%
\end{verse}

\begin{verse}%
In\marginnote{254} the sky there is no track, \\
there’s no true ascetic outside here. \\
People enjoy proliferation, \\
the Realized Ones are free of proliferation. 

%
\end{verse}

\begin{verse}%
In\marginnote{255} the sky there is no track, \\
there’s no true ascetic outside here. \\
No conditions last forever, \\
the Awakened Ones are not shaken. 

%
\end{verse}

%
\chapter*{{\suttatitleacronym Dhp 256–272}{\suttatitletranslation 19. The Just }{\suttatitleroot Dhammaṭṭhavagga}}
\addcontentsline{toc}{chapter}{\tocacronym{Dhp 256–272} \toctranslation{19. The Just } \tocroot{Dhammaṭṭhavagga}}
\markboth{19. The Just }{Dhammaṭṭhavagga}
\extramarks{Dhp 256–272}{Dhp 256–272}

\begin{verse}%
You\marginnote{256} don’t become just \\
by passing hasty judgement. \\
An astute person evaluates both \\
what is pertinent and what is irrelevant. 

%
\end{verse}

\begin{verse}%
A\marginnote{257} wise one judges others without haste, \\
justly and impartially; \\
that guardian of the law \\
is said to be just. 

%
\end{verse}

\begin{verse}%
You’re\marginnote{258} not an astute scholar \\
just because you speak a lot. \\
One who is secure, free of enmity and fear, \\
is said to be astute. 

%
\end{verse}

\begin{verse}%
You’re\marginnote{259} not one who has memorized the teaching \\
just because you recite a lot. \\
Someone who directly sees the teaching \\
after hearing only a little \\
is truly one who has memorized the teaching, \\
for they can never forget it. 

%
\end{verse}

\begin{verse}%
You\marginnote{260} don’t become a senior \\
by getting some grey hairs; \\
for one ripe only in age, \\
is said to have aged in vain. 

%
\end{verse}

\begin{verse}%
One\marginnote{261} who has truth and principle, \\
harmlessness, restraint, and self-control, \\
that wise one, purged of stains, \\
is said to be a senior. 

%
\end{verse}

\begin{verse}%
Not\marginnote{262} by mere eloquence, \\
or a beautiful complexion \\
does a person appear holy, \\
if they’re jealous, stingy, and devious. 

%
\end{verse}

\begin{verse}%
But\marginnote{263} if they’ve cut that out, \\
dug it up at the root, eradicated it, \\
that wise one, purged of vice, \\
is said to be holy. 

%
\end{verse}

\begin{verse}%
A\marginnote{264} liar and breaker of vows is no ascetic \\
just because they shave their head. \\
How on earth can one be an ascetic \\
who’s full of desire and greed? 

%
\end{verse}

\begin{verse}%
One\marginnote{265} who stops all wicked deeds, \\
great and small, \\
because of stopping wicked deeds \\
is said to be an ascetic. 

%
\end{verse}

\begin{verse}%
You\marginnote{266} don’t become a mendicant \\
just by begging from others. \\
One who has undertaken domestic duties \\
has not yet become a mendicant. 

%
\end{verse}

\begin{verse}%
But\marginnote{267} one living a spiritual life, \\
who has banished both merit and evil, \\
who wanders having assessed the world, \\
is said to be a mendicant. 

%
\end{verse}

\begin{verse}%
You\marginnote{268} don’t become a sage by silence, \\
while still confused and ignorant. \\
The astute one holds up the scales, \\
taking only the best, 

%
\end{verse}

\begin{verse}%
and\marginnote{269} rejecting the bad; \\
a sage becomes a sage by measuring. \\
One who measures good and bad in the world, \\
is thereby said to be a sage. 

%
\end{verse}

\begin{verse}%
You\marginnote{270} don’t become a noble one \\
by harming living beings. \\
One harmless towards all living beings \\
is said to be a noble one. 

%
\end{verse}

\begin{verse}%
Not\marginnote{271} by precepts and observances, \\
nor by much learning, \\
nor by meditative immersion, \\
nor by living in seclusion, 

%
\end{verse}

\begin{verse}%
do\marginnote{272} I experience the bliss of renunciation \\
not frequented by ordinary people. \\
A mendicant cannot rest confident \\
without attaining the end of defilements. 

%
\end{verse}

%
\chapter*{{\suttatitleacronym Dhp 273–289}{\suttatitletranslation 20. The Path }{\suttatitleroot Maggavagga}}
\addcontentsline{toc}{chapter}{\tocacronym{Dhp 273–289} \toctranslation{20. The Path } \tocroot{Maggavagga}}
\markboth{20. The Path }{Maggavagga}
\extramarks{Dhp 273–289}{Dhp 273–289}

\begin{verse}%
Of\marginnote{273} paths, the eightfold is the best; \\
of truths, the four statements; \\
dispassion is the best of things, \\
and the Seer is the best of humans. 

%
\end{verse}

\begin{verse}%
\emph{This}\marginnote{274} is the path, there is no other \\
for the purification of vision. \\
You all must practice this, \\
it is the way to baffle \textsanskrit{Māra}. 

%
\end{verse}

\begin{verse}%
When\marginnote{275} you all are practicing this, \\
you will make an end of suffering. \\
I have explained the path to you \\
for extracting the thorn with wisdom. 

%
\end{verse}

\begin{verse}%
You\marginnote{276} yourselves must do the work, \\
the Realized Ones just show the way. \\
Meditators practicing absorption \\
are released from \textsanskrit{Māra}’s bonds. 

%
\end{verse}

\begin{verse}%
All\marginnote{277} conditions are impermanent—\\
when this is seen with wisdom, \\
one grows disillusioned with suffering: \\
this is the path to purity. 

%
\end{verse}

\begin{verse}%
All\marginnote{278} conditions are suffering—\\
when this is seen with wisdom, \\
one grows disillusioned with suffering: \\
this is the path to purity. 

%
\end{verse}

\begin{verse}%
All\marginnote{279} things are not-self—\\
when this is seen with wisdom, \\
one grows disillusioned with suffering: \\
this is the path to purity. 

%
\end{verse}

\begin{verse}%
They\marginnote{280} don’t get going when it’s time to start; \\
they’re young and strong, but given to sloth. \\
Their mind depressed in sunken thought, \\
lazy and slothful, they can’t discern the path. 

%
\end{verse}

\begin{verse}%
Guarded\marginnote{281} in speech, restrained in mind, \\
doing no unskillful bodily deed. \\
Purify these three ways of performing deeds, \\
and win the path known to hermits. 

%
\end{verse}

\begin{verse}%
From\marginnote{282} meditation springs wisdom, \\
without meditation, wisdom ends. \\
Knowing these two paths—\\
of progress and decline—\\
you should conduct yourself \\
so that wisdom grows. 

%
\end{verse}

\begin{verse}%
Cut\marginnote{283} down the jungle, not just a tree; \\
from the jungle springs fear. \\
Having cut down jungle and vine, \\
be free of jungles, mendicants! 

%
\end{verse}

\begin{verse}%
So\marginnote{284} long as the vine, no matter how small, \\
that ties a man to women is not cut, \\
his mind remains trapped, \\
like a calf suckling its mother. 

%
\end{verse}

\begin{verse}%
Cut\marginnote{285} out fondness for oneself, \\
like plucking an autumn lotus. \\
Foster only the path to peace, \\
the quenching the Holy One taught. 

%
\end{verse}

\begin{verse}%
“Here\marginnote{286} I will stay for the rains; \\
here for winter, here the summer”; \\
thus the fool thinks, \\
not realizing the danger. 

%
\end{verse}

\begin{verse}%
As\marginnote{287} a mighty flood sweeps away a sleeping village, \\
death steals away a man \\
who dotes on children and cattle, \\
his mind caught up in them. 

%
\end{verse}

\begin{verse}%
Children\marginnote{288} provide you no shelter, \\
nor does father, nor relatives. \\
When you’re seized by the terminator, \\
there’s no shelter in family. 

%
\end{verse}

\begin{verse}%
Knowing\marginnote{289} the reason for this, \\
astute, and ethically restrained, \\
one would quickly clear the path \\
that leads to extinguishment. 

%
\end{verse}

%
\chapter*{{\suttatitleacronym Dhp 290–305}{\suttatitletranslation 21. Miscellaneous }{\suttatitleroot Pakiṇṇakavagga}}
\addcontentsline{toc}{chapter}{\tocacronym{Dhp 290–305} \toctranslation{21. Miscellaneous } \tocroot{Pakiṇṇakavagga}}
\markboth{21. Miscellaneous }{Pakiṇṇakavagga}
\extramarks{Dhp 290–305}{Dhp 290–305}

\begin{verse}%
If\marginnote{290} by giving up material happiness \\
one sees abundant happiness, \\
a wise one would give up material happiness, \\
seeing the abundant happiness. 

%
\end{verse}

\begin{verse}%
Some\marginnote{291} seek their own happiness \\
by imposing suffering on others. \\
Living intimate with enmity, \\
they’re not freed from enmity. 

%
\end{verse}

\begin{verse}%
They\marginnote{292} disregard what should be done, \\
and do what should not be done. \\
For the insolent and the negligent, \\
their defilements only grow. 

%
\end{verse}

\begin{verse}%
Those\marginnote{293} that have properly undertaken \\
constant mindfulness of the body, \\
don’t cultivate what should not be done, \\
but always do what should be done. \\
Mindful and aware, \\
their defilements come to an end. 

%
\end{verse}

\begin{verse}%
Having\marginnote{294} slain mother and father, \\
and two aristocratic kings, \\
and having wiped out the kingdom with its subjects, \\
the brahmin walks on without worry. 

%
\end{verse}

\begin{verse}%
Having\marginnote{295} slain mother and father, \\
and two aristocratic kings, \\
and a tiger as the fifth, \\
the brahmin walks on without worry. 

%
\end{verse}

\begin{verse}%
The\marginnote{296} disciples of Gotama \\
always wake up refreshed, \\
who day and night \\
constantly recollect the Buddha. 

%
\end{verse}

\begin{verse}%
The\marginnote{297} disciples of Gotama \\
always wake up refreshed, \\
who day and night \\
constantly recollect the teaching. 

%
\end{verse}

\begin{verse}%
The\marginnote{298} disciples of Gotama \\
always wake up refreshed, \\
who day and night \\
constantly recollect the \textsanskrit{Saṅgha}. 

%
\end{verse}

\begin{verse}%
The\marginnote{299} disciples of Gotama \\
always wake up refreshed, \\
who day and night \\
are constantly mindful of the body. 

%
\end{verse}

\begin{verse}%
The\marginnote{300} disciples of Gotama \\
always wake up refreshed, \\
whose minds day and night \\
delight in harmlessness. 

%
\end{verse}

\begin{verse}%
The\marginnote{301} disciples of Gotama \\
always wake up refreshed, \\
whose minds day and night \\
delight in meditation. 

%
\end{verse}

\begin{verse}%
Going\marginnote{302} forth is hard, it’s hard to be happy; \\
life at home is hard too, and painful, \\
it’s painful to stay when you’ve nothing in common. \\
A traveler is a prey to pain, \\
so don’t be a traveler, \\
don’t be prey to pain. 

%
\end{verse}

\begin{verse}%
One\marginnote{303} who is faithful, accomplished in ethics, \\
blessed with fame and wealth, \\
is honored in whatever place \\
they frequent. 

%
\end{verse}

\begin{verse}%
The\marginnote{304} good shine from afar, \\
like the Himalayan peaks, \\
but the wicked are not seen, \\
like arrows scattered in the night. 

%
\end{verse}

\begin{verse}%
Sitting\marginnote{305} alone, sleeping alone, \\
tirelessly wandering alone; \\
one who tames themselves alone \\
would delight within a forest. 

%
\end{verse}

%
\chapter*{{\suttatitleacronym Dhp 306–319}{\suttatitletranslation 22. Hell }{\suttatitleroot Nirayavagga}}
\addcontentsline{toc}{chapter}{\tocacronym{Dhp 306–319} \toctranslation{22. Hell } \tocroot{Nirayavagga}}
\markboth{22. Hell }{Nirayavagga}
\extramarks{Dhp 306–319}{Dhp 306–319}

\begin{verse}%
A\marginnote{306} liar goes to hell, \\
as does one who denies what they did. \\
Both are equal in the hereafter, \\
those men of base deeds. 

%
\end{verse}

\begin{verse}%
Many\marginnote{307} who wrap their necks in ocher robes \\
are unrestrained and wicked. \\
Being wicked, they are reborn in hell \\
due to their bad deeds. 

%
\end{verse}

\begin{verse}%
It’d\marginnote{308} be better for the immoral and unrestrained \\
to eat an iron ball, \\
scorching, like a burning flame, \\
than to eat the nation’s alms. 

%
\end{verse}

\begin{verse}%
Four\marginnote{309} things befall a heedless man \\
who sleeps with another’s wife: \\
bad karma, poor sleep, \\
ill-repute, and rebirth in hell. 

%
\end{verse}

\begin{verse}%
He\marginnote{310} accrues bad karma and is reborn in a bad place, \\
all so a frightened couple may snatch a moment’s pleasure, \\
for which rulers impose a heavy punishment. \\
That’s why a man should not sleep with another’s wife. 

%
\end{verse}

\begin{verse}%
When\marginnote{311} kusa grass is wrongly grasped \\
it only cuts the hand. \\
So too, the ascetic life, when wrongly taken, \\
drags you to hell. 

%
\end{verse}

\begin{verse}%
Any\marginnote{312} lax act, \\
any corrupt observance, \\
or suspicious spiritual life, \\
is not very fruitful. 

%
\end{verse}

\begin{verse}%
If\marginnote{313} one is to do what should be done, \\
one should staunchly strive. \\
For the life gone forth when laxly led \\
just stirs up dust all the more. 

%
\end{verse}

\begin{verse}%
A\marginnote{314} bad deed is better left undone, \\
for it will plague you later on. \\
A good deed is better done, \\
one that does not plague you. 

%
\end{verse}

\begin{verse}%
As\marginnote{315} a frontier city \\
is guarded inside and out, \\
so you should ward yourselves—\\
don’t let the moment pass you by. \\
For if you miss your moment \\
you’ll grieve when sent to hell. 

%
\end{verse}

\begin{verse}%
Unashamed\marginnote{316} of what is shameful, \\
ashamed of what is not shameful; \\
beings who uphold wrong view \\
go to a bad place. 

%
\end{verse}

\begin{verse}%
Seeing\marginnote{317} danger where there is none, \\
and blind to the actual danger, \\
beings who uphold wrong view \\
go to a bad place. 

%
\end{verse}

\begin{verse}%
Seeing\marginnote{318} fault where there is none, \\
and blind to the actual fault, \\
beings who uphold wrong view \\
go to a bad place. 

%
\end{verse}

\begin{verse}%
Knowing\marginnote{319} a fault as a fault \\
and the faultless as faultless, \\
beings who uphold right view \\
go to a good place. 

%
\end{verse}

%
\chapter*{{\suttatitleacronym Dhp 320–333}{\suttatitletranslation 23. Elephants }{\suttatitleroot Nāgavagga}}
\addcontentsline{toc}{chapter}{\tocacronym{Dhp 320–333} \toctranslation{23. Elephants } \tocroot{Nāgavagga}}
\markboth{23. Elephants }{Nāgavagga}
\extramarks{Dhp 320–333}{Dhp 320–333}

\begin{verse}%
Like\marginnote{320} an elephant struck \\
with arrows in battle, \\
I shall put up with abuse, \\
for so many folk are badly behaved. 

%
\end{verse}

\begin{verse}%
The\marginnote{321} well-tamed beast is the one led to the crowd; \\
the tamed elephant’s the one the king mounts; \\
the tamed person who endures abuse \\
is the best of human beings. 

%
\end{verse}

\begin{verse}%
Those\marginnote{322} who have tamed themselves are better \\
than fine tamed mules, \\
thoroughbreds from Sindh, \\
or giant tuskers. 

%
\end{verse}

\begin{verse}%
For\marginnote{323} not on those mounts \\
would you go to the untrodden place, \\
whereas, with the help of one whose self is well tamed, \\
you go there, tamed by the tamed. 

%
\end{verse}

\begin{verse}%
The\marginnote{324} tusker named \textsanskrit{Dhanapāla} \\
is musky in rut, hard to control. \\
Bound, he eats not a bite, \\
for he misses the elephant forest. 

%
\end{verse}

\begin{verse}%
One\marginnote{325} who gets drowsy from overeating, \\
fond of sleep, rolling round the bed \\
like a great hog stuffed with grain: \\
that idiot is reborn again and again. 

%
\end{verse}

\begin{verse}%
In\marginnote{326} the past my mind wandered \\
how it wished, where it liked, as it pleased. \\
Now I’ll carefully guide it, \\
as a trainer with a hook guides a rutting elephant. 

%
\end{verse}

\begin{verse}%
Delight\marginnote{327} in diligence! \\
Take good care of your mind! \\
Pull yourself out of this pit, \\
like an elephant sunk in a bog. 

%
\end{verse}

\begin{verse}%
If\marginnote{328} you find an alert companion, \\
a wise and virtuous friend, \\
then, overcoming all adversities, \\
wander with them, joyful and mindful. 

%
\end{verse}

\begin{verse}%
If\marginnote{329} you find no alert companion, \\
no wise and virtuous friend, \\
then, like a king who flees his conquered realm, \\
wander alone like a tusker in the wilds. 

%
\end{verse}

\begin{verse}%
It’s\marginnote{330} better to wander alone, \\
there’s no fellowship with fools. \\
Wander alone and do no wrong, \\
at ease like a tusker in the wilds. 

%
\end{verse}

\begin{verse}%
A\marginnote{331} friend in need is a blessing; \\
it’s a blessing to be content with whatever; \\
good deeds are a blessing at the end of life, \\
and giving up all suffering is a blessing. 

%
\end{verse}

\begin{verse}%
In\marginnote{332} this world it’s a blessing to serve \\
one’s mother and one’s father. \\
And it’s a blessing also to serve \\
ascetics and brahmins. 

%
\end{verse}

\begin{verse}%
It’s\marginnote{333} a blessing to keep precepts until you grow old; \\
a blessing to be grounded in faith; \\
the getting of wisdom’s a blessing; \\
and it’s a blessing to avoid doing wrong. 

%
\end{verse}

%
\chapter*{{\suttatitleacronym Dhp 334–359}{\suttatitletranslation 24. Craving }{\suttatitleroot Taṇhāvagga}}
\addcontentsline{toc}{chapter}{\tocacronym{Dhp 334–359} \toctranslation{24. Craving } \tocroot{Taṇhāvagga}}
\markboth{24. Craving }{Taṇhāvagga}
\extramarks{Dhp 334–359}{Dhp 334–359}

\begin{verse}%
When\marginnote{334} a person lives heedlessly, \\
craving grows in them like a parasitic creeper. \\
They jump from life to life, like a monkey \\
greedy for fruit in a forest grove. 

%
\end{verse}

\begin{verse}%
Whoever\marginnote{335} is beaten by this wretched craving, \\
this attachment to the world, \\
their sorrow grows, \\
like grass in the rain. 

%
\end{verse}

\begin{verse}%
But\marginnote{336} whoever prevails over this wretched craving, \\
so hard to get over in the world, \\
their sorrows fall from them, \\
like a drop from a lotus-leaf. 

%
\end{verse}

\begin{verse}%
I\marginnote{337} say this to you, good people, \\
all those who have gathered here: \\
dig up the root of craving, \\
as you’d dig up grass in search of roots. \\
Don’t let \textsanskrit{Māra} break you again and again, \\
like a stream breaking a reed. 

%
\end{verse}

\begin{verse}%
A\marginnote{338} tree grows back even when cut down, \\
so long as its roots are healthy; \\
suffering springs up again and again, \\
so long as the tendency to craving is not pulled out. 

%
\end{verse}

\begin{verse}%
A\marginnote{339} person of low views \\
in whom the thirty-six streams \\
that flow to pleasure are mighty, \\
is swept away by lustful thoughts. 

%
\end{verse}

\begin{verse}%
The\marginnote{340} streams flow everywhere; \\
a weed springs up and remains. \\
Seeing this weed that has been born, \\
cut the root with wisdom. 

%
\end{verse}

\begin{verse}%
A\marginnote{341} persons’s joys \\
flow from senses and cravings. \\
Seekers of happiness, bent on pleasure, \\
continue to be reborn and grow old. 

%
\end{verse}

\begin{verse}%
People\marginnote{342} governed by thirst, \\
crawl about like a trapped rabbit. \\
Bound and fettered, for a long time \\
they return to pain time and again. 

%
\end{verse}

\begin{verse}%
People\marginnote{343} governed by thirst, \\
crawl about like a trapped rabbit. \\
That’s why one who longs for dispassion \\
should dispel thirst. 

%
\end{verse}

\begin{verse}%
Rejecting\marginnote{344} the household jungle, they set out for the real jungle, \\
then they run right back to the jungle they left behind. \\
Just look at this person! \\
Freed, they run to bondage. 

%
\end{verse}

\begin{verse}%
The\marginnote{345} wise say that shackle is not strong \\
that’s made of iron, wood, or knots. \\
But obsession with jeweled earrings, \\
concern for your partners and children: 

%
\end{verse}

\begin{verse}%
this,\marginnote{346} say the wise, is a strong shackle \\
dragging the indulgent down, hard to escape. \\
Having cut this one too they go forth, \\
unconcerned, having given up sensual pleasures. 

%
\end{verse}

\begin{verse}%
Besotted\marginnote{347} by lust they fall into the stream, \\
like a spider caught in the web she wove. \\
The wise proceed, having cut this one too, \\
unconcerned, having given up all suffering. 

%
\end{verse}

\begin{verse}%
Let\marginnote{348} go of the past, let go of the future, \\
let go of the present, having gone beyond rebirth. \\
With your heart freed in every respect, \\
you’ll not come again to rebirth and old age. 

%
\end{verse}

\begin{verse}%
For\marginnote{349} a person crushed by thoughts, \\
very lustful, focusing on beauty, \\
their craving grows and grows, \\
tying them with a stout bond. 

%
\end{verse}

\begin{verse}%
But\marginnote{350} one who loves to calm their thoughts, \\
developing perception of ugliness, ever mindful, \\
will surely eliminate that craving, \\
cutting off the bonds of \textsanskrit{Māra}. 

%
\end{verse}

\begin{verse}%
One\marginnote{351} who is confident, unafraid, \\
rid of craving, free of blemish, \\
having struck down the arrows flying to future lives, \\
this bag of bones is their last. 

%
\end{verse}

\begin{verse}%
Rid\marginnote{352} of craving, free of grasping, \\
expert in the interpretation of terms, \\
knowing the correct \\
structure and sequence of syllables, \\
they are said to be one who bears their final body, \\
one of great wisdom, a great person. 

%
\end{verse}

\begin{verse}%
I\marginnote{353} am the champion, the knower of all, \\
unsullied in the midst of all things. \\
I’ve given up all, freed in the ending of craving. \\
Since I know for myself, whose follower should I be? 

%
\end{verse}

\begin{verse}%
The\marginnote{354} gift of the teaching beats all other gifts; \\
the taste of the teaching beats all other tastes; \\
the joy of the teaching beats all other joys; \\
one who has ended craving beats all suffering. 

%
\end{verse}

\begin{verse}%
Riches\marginnote{355} ruin an idiot, \\
but not a seeker of the far shore. \\
From craving for wealth, an idiot \\
ruins themselves and others. 

%
\end{verse}

\begin{verse}%
Weeds\marginnote{356} are the bane of crops, \\
but greed is these folk’s bane. \\
That’s why a gift to one rid of greed \\
is so very fruitful. 

%
\end{verse}

\begin{verse}%
Weeds\marginnote{357} are the bane of crops, \\
but hate is these folk’s bane. \\
That’s why a gift to one rid of hate \\
is so very fruitful. 

%
\end{verse}

\begin{verse}%
Weeds\marginnote{358} are the bane of crops, \\
but delusion is these folk’s bane. \\
That’s why a gift to one rid of delusion \\
is so very fruitful. 

%
\end{verse}

\begin{verse}%
Weeds\marginnote{359} are the bane of crops, \\
but desire is these folk’s bane. \\
That’s why a gift to one rid of desire \\
is so very fruitful. 

%
\end{verse}

%
\chapter*{{\suttatitleacronym Dhp 360–382}{\suttatitletranslation 25. Mendicants }{\suttatitleroot Bhikkhuvagga}}
\addcontentsline{toc}{chapter}{\tocacronym{Dhp 360–382} \toctranslation{25. Mendicants } \tocroot{Bhikkhuvagga}}
\markboth{25. Mendicants }{Bhikkhuvagga}
\extramarks{Dhp 360–382}{Dhp 360–382}

\begin{verse}%
Restraint\marginnote{360} of the eye is good; \\
good is restraint of the ear; \\
restraint of the nose is good; \\
good is restraint of the tongue. 

%
\end{verse}

\begin{verse}%
Restraint\marginnote{361} of the body is good; \\
good is restraint of speech; \\
restraint of mind is good; \\
everywhere, restraint is good. \\
The mendicant restrained everywhere \\
is released from suffering. 

%
\end{verse}

\begin{verse}%
One\marginnote{362} restrained in hand and foot, \\
and in speech, the supreme restraint; \\
happy inside, serene, \\
solitary, content, I call a mendicant. 

%
\end{verse}

\begin{verse}%
When\marginnote{363} a mendicant of restrained mouth, \\
thoughtful in counsel, and stable, \\
explains the text and its meaning, \\
their words are sweet. 

%
\end{verse}

\begin{verse}%
Delighting\marginnote{364} in the teaching, enjoying the teaching, \\
contemplating the teaching, \\
a mendicant who recollects the teaching \\
doesn’t decline in the true teaching. 

%
\end{verse}

\begin{verse}%
A\marginnote{365} well-off mendicant ought not look down \\
on others, nor should they be envious. \\
A mendicant who envies others \\
does not achieve immersion. 

%
\end{verse}

\begin{verse}%
If\marginnote{366} a mendicant is poor in offerings, \\
the well-to-do ought not look down on them. \\
For the gods indeed praise them, \\
who are tireless and pure of livelihood. 

%
\end{verse}

\begin{verse}%
One\marginnote{367} who has no sense of ownership \\
in the whole realm of name and form, \\
who does not grieve for that which is not, \\
is said to be a mendicant. 

%
\end{verse}

\begin{verse}%
A\marginnote{368} mendicant who meditates on love, \\
devoted to the Buddha’s teaching, \\
would realize the peaceful state, \\
the blissful stilling of conditions. 

%
\end{verse}

\begin{verse}%
Bail\marginnote{369} out this boat, mendicant! \\
When bailed out it will float lightly. \\
Having cut off desire and hate, \\
you shall reach quenching. 

%
\end{verse}

\begin{verse}%
Five\marginnote{370} to cut, five to drop, \\
and five more to develop. \\
A mendicant who escapes five chains \\
is said to have crossed the flood. 

%
\end{verse}

\begin{verse}%
Practice\marginnote{371} absorption, don’t be negligent! \\
Don’t let the mind delight in the senses! \\
Don’t heedlessly swallow a hot iron ball! \\
And when it burns, don’t cry, “Oh, the pain!” 

%
\end{verse}

\begin{verse}%
No\marginnote{372} absorption for one without wisdom, \\
no wisdom for one without absorption. \\
But one with absorption and wisdom—\\
they have truly drawn near to extinguishment. 

%
\end{verse}

\begin{verse}%
A\marginnote{373} mendicant who enters an empty hut \\
with mind at peace \\
finds a superhuman delight \\
as they rightly discern the Dhamma. 

%
\end{verse}

\begin{verse}%
Whenever\marginnote{374} they are mindful \\
of the rise and fall of the aggregates, \\
they feel rapture and joy: \\
that is the deathless for one who knows. 

%
\end{verse}

\begin{verse}%
This\marginnote{375} is the very start of the path \\
for a wise mendicant: \\
guarding the senses, contentment, \\
and restraint in the monastic code. 

%
\end{verse}

\begin{verse}%
Mix\marginnote{376} with spiritual friends, \\
who are tireless and pure of livelihood. \\
Share what you have with others, \\
being skillful in your conduct. \\
And when you’re full of joy, \\
you’ll make an end to suffering. 

%
\end{verse}

\begin{verse}%
As\marginnote{377} a jasmine sheds \\
its withered flowers, \\
O mendicants, \\
cast off greed and hate. 

%
\end{verse}

\begin{verse}%
Calm\marginnote{378} in body, calm in speech, \\
peaceful and serene; \\
a mendicant who’s spat out the world’s bait \\
is said to be one at peace. 

%
\end{verse}

\begin{verse}%
Urge\marginnote{379} yourself on, \\
reflect on yourself. \\
A mendicant self-controlled and mindful \\
will always dwell in happiness. 

%
\end{verse}

\begin{verse}%
Self\marginnote{380} is indeed the lord of self, \\
for who else would be one’s lord? \\
Self is indeed the home of self, \\
so restrain yourself, \\
as a merchant his thoroughbred steed. 

%
\end{verse}

\begin{verse}%
A\marginnote{381} monk full of joy \\
trusting in the Buddha’s teaching, \\
would realize the peaceful state, \\
the blissful stilling of conditions. 

%
\end{verse}

\begin{verse}%
A\marginnote{382} young mendicant \\
devoted to the Buddha’s teaching, \\
lights up the world, \\
like the moon freed from a cloud. 

%
\end{verse}

%
\chapter*{{\suttatitleacronym Dhp 383–423}{\suttatitletranslation 26. Brahmins }{\suttatitleroot Brāhmaṇavagga}}
\addcontentsline{toc}{chapter}{\tocacronym{Dhp 383–423} \toctranslation{26. Brahmins } \tocroot{Brāhmaṇavagga}}
\markboth{26. Brahmins }{Brāhmaṇavagga}
\extramarks{Dhp 383–423}{Dhp 383–423}

\begin{verse}%
Strive\marginnote{383} and cut the stream! \\
Dispel sensual pleasures, brahmin. \\
Knowing the ending of conditions, \\
know the uncreated, brahmin. 

%
\end{verse}

\begin{verse}%
When\marginnote{384} a brahmin \\
has gone beyond two things, \\
then they consciously \\
make an end of all fetters. 

%
\end{verse}

\begin{verse}%
One\marginnote{385} for whom there is no crossing over \\
or crossing back, or crossing over and back; \\
stress-free, detached, \\
that’s who I call a brahmin. 

%
\end{verse}

\begin{verse}%
Absorbed,\marginnote{386} rid of hopes, \\
their task completed, without defilements, \\
arrived at the highest goal: \\
that’s who I call a brahmin. 

%
\end{verse}

\begin{verse}%
The\marginnote{387} sun blazes by day, \\
the moon glows at night, \\
the aristocrat shines in armor, \\
and the brahmin shines in absorption. \\
But all day and all night, \\
the Buddha shines with glory. 

%
\end{verse}

\begin{verse}%
A\marginnote{388} brahmin’s so-called because they have banished evil, \\
an ascetic’s so-called since they live a serene life. \\
One who has renounced all stains \\
is said to be a “renunciant”. 

%
\end{verse}

\begin{verse}%
One\marginnote{389} should never strike a brahmin, \\
nor should a brahmin retaliate. \\
Woe to the one who hurts a brahmin, \\
and woe for the one who retaliates. 

%
\end{verse}

\begin{verse}%
Nothing\marginnote{390} is better for a brahmin \\
than to hold their mind back from attachment. \\
As cruelty in the mind gradually subsides, \\
suffering also subsides. 

%
\end{verse}

\begin{verse}%
Who\marginnote{391} does nothing wrong \\
by body, speech or mind, \\
restrained in these three respects, \\
that’s who I call a brahmin. 

%
\end{verse}

\begin{verse}%
You\marginnote{392} should graciously honor \\
the one from whom you learn the Dhamma \\
taught by the awakened Buddha, \\
as a brahmin honors the sacred flame. 

%
\end{verse}

\begin{verse}%
Not\marginnote{393} by matted hair or family, \\
or birth is one a brahmin. \\
Those who have truth and principle: \\
they are pure, they are brahmins. 

%
\end{verse}

\begin{verse}%
Why\marginnote{394} the matted hair, you fool, \\
and why the skin of deer? \\
The tangle is inside you, \\
yet you polish up your outsides. 

%
\end{verse}

\begin{verse}%
A\marginnote{395} person who wears robes of rags, \\
lean, their limbs showing veins, \\
meditating alone in the forest, \\
that’s who I call a brahmin. 

%
\end{verse}

\begin{verse}%
I\marginnote{396} don’t call someone a brahmin \\
after the mother or womb they came from. \\
If they still have attachments, \\
they’re just someone who says “sir”. \\
Having nothing, taking nothing: \\
that’s who I call a brahmin. 

%
\end{verse}

\begin{verse}%
Having\marginnote{397} cut off all fetters \\
they have no anxiety. \\
They’ve got over clinging, and are detached: \\
that’s who I call a brahmin. 

%
\end{verse}

\begin{verse}%
They’ve\marginnote{398} cut the strap and harness, \\
the reins and bridle too, \\
with cross-bar lifted, they’re awakened: \\
that’s who I call a brahmin. 

%
\end{verse}

\begin{verse}%
Abuse,\marginnote{399} killing, caging: \\
they endure these without anger. \\
Patience is their powerful army: \\
that’s who I call a brahmin. 

%
\end{verse}

\begin{verse}%
Not\marginnote{400} irritable or stuck up, \\
dutiful in precepts and observances, \\
tamed, bearing their final body: \\
that’s who I call a brahmin. 

%
\end{verse}

\begin{verse}%
Like\marginnote{401} water from a lotus leaf, \\
like a mustard seed off a pin-point, \\
sensual pleasures slip off them: \\
that’s who I call a brahmin. 

%
\end{verse}

\begin{verse}%
They\marginnote{402} understand for themselves \\
the end of suffering in this life; \\
with burden put down, detached: \\
that’s who I call a brahmin. 

%
\end{verse}

\begin{verse}%
Deep\marginnote{403} in wisdom, intelligent, \\
expert in the variety of paths; \\
arrived at the highest goal: \\
that’s who I call a brahmin. 

%
\end{verse}

\begin{verse}%
Socializing\marginnote{404} with neither \\
householders nor the homeless. \\
A migrant with no shelter, few in wishes: \\
that’s who I call a brahmin. 

%
\end{verse}

\begin{verse}%
They’ve\marginnote{405} laid aside violence \\
against creatures firm and frail; \\
not killing or making others kill: \\
that’s who I call a brahmin. 

%
\end{verse}

\begin{verse}%
Not\marginnote{406} fighting among those who fight, \\
extinguished among those who are armed, \\
not taking among those who take: \\
that’s who I call a brahmin. 

%
\end{verse}

\begin{verse}%
They’ve\marginnote{407} discarded greed and hate, \\
along with conceit and contempt, \\
like a mustard seed off the point of a pin: \\
that’s who I call a brahmin. 

%
\end{verse}

\begin{verse}%
The\marginnote{408} words they utter \\
are sweet, informative, and true, \\
and don’t offend anyone: \\
that’s who I call a brahmin. 

%
\end{verse}

\begin{verse}%
They\marginnote{409} don’t steal anything in the world, \\
long or short, \\
fine or coarse, beautiful or ugly: \\
that’s who I call a brahmin. 

%
\end{verse}

\begin{verse}%
They\marginnote{410} have no hope \\
in this world or the next. \\
with no need for hope, detached: \\
that’s who I call a brahmin. 

%
\end{verse}

\begin{verse}%
They\marginnote{411} have no clinging, \\
knowledge has freed them of indecision, \\
they’ve plunged right into the deathless: \\
that’s who I call a brahmin. 

%
\end{verse}

\begin{verse}%
They’ve\marginnote{412} escaped clinging \\
to both good and bad deeds; \\
sorrowless, stainless, pure: \\
that’s who I call a brahmin. 

%
\end{verse}

\begin{verse}%
Pure\marginnote{413} as the spotless moon, \\
clear and undisturbed, \\
they’ve ended delight and future lives: \\
that’s who I call a brahmin. 

%
\end{verse}

\begin{verse}%
They’ve\marginnote{414} got past this grueling swamp \\
of delusion, transmigration. \\
Meditating in stillness, free of indecision, \\
they have crossed over to the far shore. \\
They’re extinguished by not grasping: \\
that’s who I call a brahmin. 

%
\end{verse}

\begin{verse}%
They’ve\marginnote{415} given up sensual stimulations, \\
and have gone forth from lay life; \\
they’ve ended rebirth in the sensual realm: \\
that’s who I call a brahmin. 

%
\end{verse}

\begin{verse}%
They’ve\marginnote{416} given up craving, \\
and have gone forth from lay life; \\
they’ve ended craving to be reborn: \\
that’s who I call a brahmin. 

%
\end{verse}

\begin{verse}%
They’ve\marginnote{416} given up craving, \\
and have gone forth from lay life; \\
they’ve ended craving to be reborn: \\
that’s who I call a brahmin. 

%
\end{verse}

\begin{verse}%
They’ve\marginnote{417} given up human bonds, \\
and gone beyond heavenly bonds; \\
detached from all attachments: \\
that’s who I call a brahmin. 

%
\end{verse}

\begin{verse}%
Giving\marginnote{418} up discontent and desire, \\
they’re cooled and free of attachments; \\
a hero, master of the whole world: \\
that’s who I call a brahmin. 

%
\end{verse}

\begin{verse}%
They\marginnote{419} know the passing away \\
and rebirth of all beings; \\
unattached, holy, awakened: \\
that’s who I call a brahmin. 

%
\end{verse}

\begin{verse}%
Gods,\marginnote{420} fairies, and humans \\
don’t know their destiny; \\
the perfected ones with defilements ended: \\
that’s who I call a brahmin. 

%
\end{verse}

\begin{verse}%
They\marginnote{421} have nothing before or after, \\
or even in between. \\
Having nothing, taking nothing: \\
that’s who I call a brahmin. 

%
\end{verse}

\begin{verse}%
Leader\marginnote{422} of the herd, excellent hero, \\
great hermit and victor; \\
unstirred, washed, awakened: \\
that’s who I call a brahmin. 

%
\end{verse}

\begin{verse}%
They\marginnote{423} know their past lives, \\
seeing heaven and places of loss, \\
and have attained the end of rebirth; \\
that sage who has perfect insight, \\
at the summit of spiritual perfection: \\
that’s who I call a brahmin. 

%
\end{verse}

\begin{verse}%
The\marginnote{423} Sayings of the Dhamma is completed. 

%
\end{verse}

\scendbook{The Sayings of the Dhamma is completed. }

%
\backmatter%
\chapter*{Colophon}
\addcontentsline{toc}{chapter}{Colophon}
\markboth{Colophon}{Colophon}

\section*{The Translator}

Bhikkhu Sujato was born as Anthony Aidan Best on 4/11/1966 in Perth, Western Australia. He grew up in the pleasant suburbs of Mt Lawley and Attadale alongside his sister Nicola, who was the good child. His mother, Margaret Lorraine Huntsman née Pinder, said “he’ll either be a priest or a poet”, while his father, Anthony Thomas Best, advised him to “never do anything for money”. He attended Aquinas College, a Catholic school, where he decided to become an atheist. At the University of WA he studied philosophy, aiming to learn what he wanted to do with his life. Finding that what he wanted to do was play guitar, he dropped out. His main band was named Martha’s Vineyard, which achieved modest success in the indie circuit. Then it broke up, because everyone thought they personally were reason for the success, which, oddly enough, turns out not to have been the case. 

A seemingly random encounter with a roadside joey took him to Thailand, where he entered his first meditation retreat at Wat Ram Poeng, Chieng Mai in 1992. He decided to devote himself to the Buddha’s path, and took full ordination in Wat Pa Nanachat in 1994, where his teachers were Ajahn Pasanno and Ajahn Jayasaro. In 1997 he returned to Perth to study with Ajahn Brahm at Bodhinyana Monastery. 

He spent several years practicing in seclusion in Malaysia and Thailand before establishing Santi Forest Monastery in Bundanoon, NSW, in 2003. There he was instrumental in supporting the establishment of the Theravada bhikkhuni order in Australia and advocating for women’s rights. He continues to teach in Australia and globally, with a special concern for the moral implications of climate change and other forms of environmental destruction. He has published a series of books of original and groundbreaking research on early Buddhism. 

In 2005 he founded SuttaCentral together with Rod Bucknell and John Kelly. In 2015, seeing the need for a complete, accurate, plain English translation of the Pali texts, he undertook the task, spending nearly three years in isolation on the isle of Qi Mei off the coast of the nation of Taiwan. He completed the four main \textsanskrit{Nikāyas} in 2018, and the early books of the Khuddaka \textsanskrit{Nikāya} were complete by 2021. All this work is dedicated to the public domain and is entirely free of copyright encumbrance. 

In 2019 he returned to Sydney where, together with Bhikkhu Akaliko, he established Lokanta Vihara (The Monastery at the End of the World). 

\section*{Creation Process}

Translated from the Pali. Primary source was the \textsanskrit{Mahāsaṅgīti} edition, with reference to several English translations, especially those of K.R. Norman and Venerable Buddharakkhita.

\section*{The Translation}

This translation aims to make a clear, readable, and accurate rendering of the Dhammapada. Unlike most Dhammapadas in English, this is a new translation from the source Pali text. The aim was to make the sense as transparent as possible.

\section*{About SuttaCentral}

SuttaCentral publishes early Buddhist texts. Since 2005 we have provided root texts in Pali, Chinese, Sanskrit, Tibetan, and other languages, parallels between these texts, and translations in many modern languages. We build on the work of generations of scholars, and offer our contribution freely.

SuttaCentral is driven by volunteer contributions, and in addition we employ professional developers. We offer a sponsorship program for high quality translations from the original languages. Financial support for SuttaCentral is handled by the SuttaCentral Development Trust, a charitable trust registered in Australia.

\section*{About Bilara}

“Bilara” means “cat” in Pali, and it is the name of our Computer Assisted Translation (CAT) software. Bilara is a web app that enables translators to translate early Buddhist texts into their own language. These translations are published on SuttaCentral with the root text and translation side by side.

\section*{About SuttaCentral Editions}

The SuttaCentral Editions project makes high quality books from selected Bilara translations. These are published in formats including HTML, EPUB, PDF, and print.

If you want to print any of our Editions, please let us know and we will help prepare a file to your specifications.

%
\end{document}