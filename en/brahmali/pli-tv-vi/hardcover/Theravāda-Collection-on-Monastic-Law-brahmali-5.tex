\documentclass[12pt,openany]{book}%
\usepackage{lastpage}%
%
\usepackage{ragged2e}
\usepackage{verse}
\usepackage[a-3u]{pdfx}
\usepackage[inner=1in, outer=1in, top=.7in, bottom=1in, papersize={6in,9in}, headheight=13pt]{geometry}
\usepackage{polyglossia}
\usepackage[12pt]{moresize}
\usepackage{soul}%
\usepackage{microtype}
\usepackage{tocbasic}
\usepackage{realscripts}
\usepackage{epigraph}%
\usepackage{setspace}%
\usepackage{sectsty}
\usepackage{fontspec}
\usepackage{marginnote}
\usepackage[bottom]{footmisc}
\usepackage{enumitem}
\usepackage{fancyhdr}
\usepackage{emptypage}
\usepackage{extramarks}
\usepackage{graphicx}
\usepackage{relsize}
\usepackage{etoolbox}

% improve ragged right headings by suppressing hyphenation and orphans. spaceskip plus and minus adjust interword spacing; increase rightskip stretch to make it want to push a word on the first line(s) to the next line; reduce parfillskip stretch to make line length more equal . spacefillskip and xspacefillskip can be deleted to use defaults.
\protected\def\BalancedRagged{
\leftskip     0pt
\rightskip    0pt plus 10em
\spaceskip=1\fontdimen2\font plus .5\fontdimen3\font minus 1.5\fontdimen4\font
\xspaceskip=1\fontdimen2\font plus 1\fontdimen3\font minus 1\fontdimen4\font
\parfillskip  0pt plus 15em
\relax
}

\hypersetup{
colorlinks=true,
urlcolor=black,
linkcolor=black,
citecolor=black,
allcolors=black
}

% use a small amount of tracking on small caps
\SetTracking[ spacing = {25*,166, } ]{ encoding = *, shape = sc }{ 25 }

% add a blank page
\newcommand{\blankpage}{
\newpage
\thispagestyle{empty}
\mbox{}
\newpage
}

% define languages
\setdefaultlanguage[]{english}
\setotherlanguage[script=Latin]{sanskrit}

%\usepackage{pagegrid}
%\pagegridsetup{top-left, step=.25in}

% define fonts
% use if arno sanskrit is unavailable
%\setmainfont{Gentium Plus}
%\newfontfamily\Marginalfont[]{Gentium Plus}
%\newfontfamily\Allsmallcapsfont[RawFeature=+c2sc]{Gentium Plus}
%\newfontfamily\Noligaturefont[Renderer=Basic]{Gentium Plus}
%\newfontfamily\Noligaturecaptionfont[Renderer=Basic]{Gentium Plus}
%\newfontfamily\Fleuronfont[Ornament=1]{Gentium Plus}

% use if arno sanskrit is available. display is applied to \chapter and \part, subhead to \section and \subsection.
\setmainfont[
  FontFace={sb}{n}{Font = {Arno Pro Semibold}},
  FontFace={sb}{it}{Font = {Arno  Pro Semibold Italic}}
]{Arno Pro}

% create commands for using semibold
\DeclareRobustCommand{\sbseries}{\fontseries{sb}\selectfont}
\DeclareTextFontCommand{\textsb}{\sbseries}

\newfontfamily\Marginalfont[RawFeature=+subs]{Arno Pro Regular}
\newfontfamily\Allsmallcapsfont[RawFeature=+c2sc]{Arno Pro}
\newfontfamily\Noligaturefont[Renderer=Basic]{Arno Pro}
\newfontfamily\Noligaturecaptionfont[Renderer=Basic]{Arno Pro Caption}

% chinese fonts
\newfontfamily\cjk{Noto Serif TC}
\newcommand*{\langlzh}[1]{\cjk{#1}\normalfont}%

% logo
\newfontfamily\Logofont{sclogo.ttf}
\newcommand*{\sclogo}[1]{\large\Logofont{#1}}

% use subscript numerals for margin notes
\renewcommand*{\marginfont}{\Marginalfont}

% ensure margin notes have consistent vertical alignment
\renewcommand*{\marginnotevadjust}{-.17em}

% use compact lists
\setitemize{noitemsep,leftmargin=1em}
\setenumerate{noitemsep,leftmargin=1em}
\setdescription{noitemsep, style=unboxed, leftmargin=1em}

% style ToC
\DeclareTOCStyleEntries[
  raggedentrytext,
  linefill=\hfill,
  pagenumberwidth=.5in,
  pagenumberformat=\normalfont,
  entryformat=\normalfont
]{tocline}{chapter,section}


  \setlength\topsep{0pt}%
  \setlength\parskip{0pt}%

% define new \centerpars command for use in ToC. This ensures centering, proper wrapping, and no page break after
\def\startcenter{%
  \par
  \begingroup
  \leftskip=0pt plus 1fil
  \rightskip=\leftskip
  \parindent=0pt
  \parfillskip=0pt
}
\def\stopcenter{%
  \par
  \endgroup
}
\long\def\centerpars#1{\startcenter#1\stopcenter}

% redefine part, so that it adds a toc entry without page number
\let\oldcontentsline\contentsline
\newcommand{\nopagecontentsline}[3]{\oldcontentsline{#1}{#2}{}}

    \makeatletter
\renewcommand*\l@part[2]{%
  \ifnum \c@tocdepth >-2\relax
    \addpenalty{-\@highpenalty}%
    \addvspace{0em \@plus\p@}%
    \setlength\@tempdima{3em}%
    \begingroup
      \parindent \z@ \rightskip \@pnumwidth
      \parfillskip -\@pnumwidth
      {\leavevmode
       \setstretch{.85}\large\scshape\centerpars{#1}\vspace*{-1em}\llap{#2}}\par
       \nobreak
         \global\@nobreaktrue
         \everypar{\global\@nobreakfalse\everypar{}}%
    \endgroup
  \fi}
\makeatother

\makeatletter
\def\@pnumwidth{2em}
\makeatother

% define new sectioning command, which is only used in volumes where the pannasa is found in some parts but not others, especially in an and sn

\newcommand*{\pannasa}[1]{\clearpage\thispagestyle{empty}\begin{center}\vspace*{14em}\setstretch{.85}\huge\itshape\scshape\MakeLowercase{#1}\end{center}}

    \makeatletter
\newcommand*\l@pannasa[2]{%
  \ifnum \c@tocdepth >-2\relax
    \addpenalty{-\@highpenalty}%
    \addvspace{.5em \@plus\p@}%
    \setlength\@tempdima{3em}%
    \begingroup
      \parindent \z@ \rightskip \@pnumwidth
      \parfillskip -\@pnumwidth
      {\leavevmode
       \setstretch{.85}\large\itshape\scshape\lowercase{\centerpars{#1}}\vspace*{-1em}\llap{#2}}\par
       \nobreak
         \global\@nobreaktrue
         \everypar{\global\@nobreakfalse\everypar{}}%
    \endgroup
  \fi}
\makeatother

% don't put page number on first page of toc (relies on etoolbox)
\patchcmd{\chapter}{plain}{empty}{}{}

% global line height
\setstretch{1.05}

% allow linebreak after em-dash
\catcode`\—=13
\protected\def—{\unskip\textemdash\allowbreak}

% style headings with secsty. chapter and section are defined per-edition
\partfont{\setstretch{.85}\normalfont\centering\textsc}
\subsectionfont{\setstretch{.95}\normalfont\BalancedRagged}%
\subsubsectionfont{\setstretch{1}\normalfont\itshape\BalancedRagged}

% style elements of suttatitle
\newcommand*{\suttatitleacronym}[1]{\smaller[2]{#1}\vspace*{.3em}}
\newcommand*{\suttatitletranslation}[1]{\linebreak{#1}}
\newcommand*{\suttatitleroot}[1]{\linebreak\smaller[2]\itshape{#1}}

\DeclareTOCStyleEntries[
  indent=3.3em,
  dynindent,
  beforeskip=.2em plus -2pt minus -1pt,
]{tocline}{section}

\DeclareTOCStyleEntries[
  indent=0em,
  dynindent,
  beforeskip=.4em plus -2pt minus -1pt,
]{tocline}{chapter}

\newcommand*{\tocacronym}[1]{\hspace*{-3.3em}{#1}\quad}
\newcommand*{\toctranslation}[1]{#1}
\newcommand*{\tocroot}[1]{(\textit{#1})}
\newcommand*{\tocchapterline}[1]{\bfseries\itshape{#1}}


% redefine paragraph and subparagraph headings to not be inline
\makeatletter
% Change the style of paragraph headings %
\renewcommand\paragraph{\@startsection{paragraph}{4}{\z@}%
            {-2.5ex\@plus -1ex \@minus -.25ex}%
            {1.25ex \@plus .25ex}%
            {\noindent\normalfont\itshape\small}}

% Change the style of subparagraph headings %
\renewcommand\subparagraph{\@startsection{subparagraph}{5}{\z@}%
            {-2.5ex\@plus -1ex \@minus -.25ex}%
            {1.25ex \@plus .25ex}%
            {\noindent\normalfont\itshape\footnotesize}}
\makeatother

% use etoolbox to suppress page numbers on \part
\patchcmd{\part}{\thispagestyle{plain}}{\thispagestyle{empty}}
  {}{\errmessage{Cannot patch \string\part}}

% and to reduce margins on quotation
\patchcmd{\quotation}{\rightmargin}{\leftmargin 1.2em \rightmargin}{}{}
\AtBeginEnvironment{quotation}{\small}

% titlepage
\newcommand*{\titlepageTranslationTitle}[1]{{\begin{center}\begin{large}{#1}\end{large}\end{center}}}
\newcommand*{\titlepageCreatorName}[1]{{\begin{center}\begin{normalsize}{#1}\end{normalsize}\end{center}}}

% halftitlepage
\newcommand*{\halftitlepageTranslationTitle}[1]{\setstretch{2.5}{\begin{Huge}\uppercase{\so{#1}}\end{Huge}}}
\newcommand*{\halftitlepageTranslationSubtitle}[1]{\setstretch{1.2}{\begin{large}{#1}\end{large}}}
\newcommand*{\halftitlepageFleuron}[1]{{\begin{large}\Fleuronfont{{#1}}\end{large}}}
\newcommand*{\halftitlepageByline}[1]{{\begin{normalsize}\textit{{#1}}\end{normalsize}}}
\newcommand*{\halftitlepageCreatorName}[1]{{\begin{LARGE}{\textsc{#1}}\end{LARGE}}}
\newcommand*{\halftitlepageVolumeNumber}[1]{{\begin{normalsize}{\Allsmallcapsfont{\textsc{#1}}}\end{normalsize}}}
\newcommand*{\halftitlepageVolumeAcronym}[1]{{\begin{normalsize}{#1}\end{normalsize}}}
\newcommand*{\halftitlepageVolumeTranslationTitle}[1]{{\begin{Large}{\textsc{#1}}\end{Large}}}
\newcommand*{\halftitlepageVolumeRootTitle}[1]{{\begin{normalsize}{\Allsmallcapsfont{\textsc{\itshape #1}}}\end{normalsize}}}
\newcommand*{\halftitlepagePublisher}[1]{{\begin{large}{\Noligaturecaptionfont\textsc{#1}}\end{large}}}

% epigraph
\renewcommand{\epigraphflush}{center}
\renewcommand*{\epigraphwidth}{.85\textwidth}
\newcommand*{\epigraphTranslatedTitle}[1]{\vspace*{.5em}\footnotesize\textsc{#1}\\}%
\newcommand*{\epigraphRootTitle}[1]{\footnotesize\textit{#1}\\}%
\newcommand*{\epigraphReference}[1]{\footnotesize{#1}}%

% map
\newsavebox\IBox

% custom commands for html styling classes
\newcommand*{\scnamo}[1]{\begin{Center}\textit{#1}\end{Center}\bigskip}
\newcommand*{\scendsection}[1]{\begin{Center}\begin{small}\textit{#1}\end{small}\end{Center}\addvspace{1em}}
\newcommand*{\scendsutta}[1]{\begin{Center}\textit{#1}\end{Center}\addvspace{1em}}
\newcommand*{\scendbook}[1]{\bigskip\begin{Center}\uppercase{#1}\end{Center}\addvspace{1em}}
\newcommand*{\scendkanda}[1]{\begin{Center}\textbf{#1}\end{Center}\addvspace{1em}} % use for ending vinaya rule sections and also samyuttas %
\newcommand*{\scend}[1]{\begin{Center}\begin{small}\textit{#1}\end{small}\end{Center}\addvspace{1em}}
\newcommand*{\scendvagga}[1]{\begin{Center}\textbf{#1}\end{Center}\addvspace{1em}}
\newcommand*{\scrule}[1]{\textsb{#1}}
\newcommand*{\scadd}[1]{\textit{#1}}
\newcommand*{\scevam}[1]{\textsc{#1}}
\newcommand*{\scspeaker}[1]{\hspace{2em}\textit{#1}}
\newcommand*{\scbyline}[1]{\begin{flushright}\textit{#1}\end{flushright}\bigskip}
\newcommand*{\scexpansioninstructions}[1]{\begin{small}\textit{#1}\end{small}}
\newcommand*{\scuddanaintro}[1]{\medskip\noindent\begin{footnotesize}\textit{#1}\end{footnotesize}\smallskip}

\newenvironment{scuddana}{%
\setlength{\stanzaskip}{.5\baselineskip}%
  \vspace{-1em}\begin{verse}\begin{footnotesize}%
}{%
\end{footnotesize}\end{verse}
}%

% custom command for thematic break = hr
\newcommand*{\thematicbreak}{\begin{center}\rule[.5ex]{6em}{.4pt}\begin{normalsize}\quad\Fleuronfont{•}\quad\end{normalsize}\rule[.5ex]{6em}{.4pt}\end{center}}

% manage and style page header and footer. "fancy" has header and footer, "plain" has footer only

\pagestyle{fancy}
\fancyhf{}
\fancyfoot[RE,LO]{\thepage}
\fancyfoot[LE,RO]{\footnotesize\lastleftxmark}
\fancyhead[CE]{\setstretch{.85}\Noligaturefont\MakeLowercase{\textsc{\firstrightmark}}}
\fancyhead[CO]{\setstretch{.85}\Noligaturefont\MakeLowercase{\textsc{\firstleftmark}}}
\renewcommand{\headrulewidth}{0pt}
\fancypagestyle{plain}{ %
\fancyhf{} % remove everything
\fancyfoot[RE,LO]{\thepage}
\fancyfoot[LE,RO]{\footnotesize\lastleftxmark}
\renewcommand{\headrulewidth}{0pt}
\renewcommand{\footrulewidth}{0pt}}
\fancypagestyle{plainer}{ %
\fancyhf{} % remove everything
\fancyfoot[RE,LO]{\thepage}
\renewcommand{\headrulewidth}{0pt}
\renewcommand{\footrulewidth}{0pt}}

% style footnotes
\setlength{\skip\footins}{1em}

\makeatletter
\newcommand{\@makefntextcustom}[1]{%
    \parindent 0em%
    \thefootnote.\enskip #1%
}
\renewcommand{\@makefntext}[1]{\@makefntextcustom{#1}}
\makeatother

% hang quotes (requires microtype)
\microtypesetup{
  protrusion = true,
  expansion  = true,
  tracking   = true,
  factor     = 1000,
  patch      = all,
  final
}

% Custom protrusion rules to allow hanging punctuation
\SetProtrusion
{ encoding = *}
{
% char   right left
  {-} = {    , 500 },
  % Double Quotes
  \textquotedblleft
      = {1000,     },
  \textquotedblright
      = {    , 1000},
  \quotedblbase
      = {1000,     },
  % Single Quotes
  \textquoteleft
      = {1000,     },
  \textquoteright
      = {    , 1000},
  \quotesinglbase
      = {1000,     }
}

% make latex use actual font em for parindent, not Computer Modern Roman
\AtBeginDocument{\setlength{\parindent}{1em}}%
%

% Default values; a bit sloppier than normal
\tolerance 1414
\hbadness 1414
\emergencystretch 1.5em
\hfuzz 0.3pt
\clubpenalty = 10000
\widowpenalty = 10000
\displaywidowpenalty = 10000
\hfuzz \vfuzz
 \raggedbottom%

\title{Theravāda Collection on Monastic Law}
\author{Bhikkhu Brahmali}
\date{}%
% define a different fleuron for each edition
\newfontfamily\Fleuronfont[Ornament=9]{Arno Pro}

% Define heading styles per edition for chapter and section. Suttatitle can be either of these, depending on the volume. 

\let\oldfrontmatter\frontmatter
\renewcommand{\frontmatter}{%
\chapterfont{\setstretch{.85}\normalfont\centering}%
\sectionfont{\setstretch{.85}\normalfont\BalancedRagged}%
\oldfrontmatter}

\let\oldmainmatter\mainmatter
\renewcommand{\mainmatter}{%
\chapterfont{\setstretch{.85}\normalfont\centering}%
\sectionfont{\setstretch{.85}\normalfont\BalancedRagged}%
\oldmainmatter}

\let\oldbackmatter\backmatter
\renewcommand{\backmatter}{%
\chapterfont{\setstretch{.85}\normalfont\centering}%
\sectionfont{\setstretch{.85}\normalfont\BalancedRagged}%
\pagestyle{plainer}%
\oldbackmatter}

% for reasons, flat texts align too far in the margin in ToC, this fixes it. 
\renewcommand*{\tocacronym}[1]{\hspace*{0em}{#1}\quad}%
%
\begin{document}%
\normalsize%
\frontmatter%
\setlength{\parindent}{0cm}

\pagestyle{empty}

\maketitle

\blankpage%
\begin{center}

\vspace*{2.2em}

\halftitlepageTranslationTitle{Theravāda Collection on Monastic Law}

\vspace*{1em}

\halftitlepageTranslationSubtitle{A translation of the Pali Vinaya Piṭaka into English}

\vspace*{2em}

\halftitlepageFleuron{•}

\vspace*{2em}

\halftitlepageByline{translated and introduced by}

\vspace*{.5em}

\halftitlepageCreatorName{Bhikkhu Brahmali}

\vspace*{4em}

\halftitlepageVolumeNumber{Volume 5}

\smallskip

\halftitlepageVolumeAcronym{Kd 11–22}

\smallskip

\halftitlepageVolumeTranslationTitle{The Lesser Division}

\smallskip

\halftitlepageVolumeRootTitle{Cūḷavagga}

\vspace*{\fill}

\sclogo{0}
 \halftitlepagePublisher{SuttaCentral}

\end{center}

\newpage
%
\setstretch{1.05}

\begin{footnotesize}

\textit{Theravāda Collection on Monastic Law} is a translation of the Theravāda Vinayapiṭaka by Bhikkhu Brahmali.

\medskip

Creative Commons Zero (CC0)

To the extent possible under law, Bhikkhu Brahmali has waived all copyright and related or neighboring rights to \textit{Theravāda Collection on Monastic Law}.

\medskip

This work is published from Australia.

\begin{center}
\textit{This translation is an expression of an ancient spiritual text that has been passed down by the Buddhist tradition for the benefit of all sentient beings. It is dedicated to the public domain via Creative Commons Zero (CC0). You are encouraged to copy, reproduce, adapt, alter, or otherwise make use of this translation. The translator respectfully requests that any use be in accordance with the values and principles of the Buddhist community.}
\end{center}

\medskip

\begin{description}
    \item[Web publication date] 2021
    \item[This edition] 2025-01-13 01:01:43
    \item[Publication type] hardcover
    \item[Edition] ed3
    \item[Number of volumes] 6
    \item[Publication ISBN] 978-1-76132-006-4
    \item[Volume ISBN] 978-1-76132-011-8
    \item[Publication URL] \href{https://suttacentral.net/editions/pli-tv-vi/en/brahmali}{https://suttacentral.net/editions/pli-tv-vi/en/brahmali}
    \item[Source URL] \href{https://github.com/suttacentral/bilara-data/tree/published/translation/en/brahmali/vinaya}{https://github.com/suttacentral/bilara-data/tree/published/translation/en/brahmali/vinaya}
    \item[Publication number] scpub8
\end{description}

\medskip

Map of Jambudīpa is by Jonas David Mitja Lang, and is released by him under Creative Commons Zero (CC0).

\medskip

Published by SuttaCentral

\medskip

\textit{SuttaCentral,\\
c/o Alwis \& Alwis Pty Ltd\\
Kaurna Country,\\
Suite 12,\\
198 Greenhill Road,\\
Eastwood,\\
SA 5063,\\
Australia}

\end{footnotesize}

\newpage

\setlength{\parindent}{1em}%%
\tableofcontents
\newpage
\pagestyle{fancy}
%
\chapter*{Introduction to the Khandhakas, “The Chapters”, part II, Kd 11–22}
\addcontentsline{toc}{chapter}{Introduction to the Khandhakas, “The Chapters”, part II, Kd 11–22}
\markboth{Introduction to the Khandhakas, “The Chapters”, part II, Kd 11–22}{Introduction to the Khandhakas, “The Chapters”, part II, Kd 11–22}

\scbyline{Bhikkhu Brahmali, 2024}

The present volume is the fifth of six, the total of which constitutes a complete translation of the Vinaya \textsanskrit{Piṭaka}, the Monastic Law. This volume consists of the second part of the Khandhakas, also known as the Cullavagga, “the Small Division”, comprising the last 12 of altogether 22 chapters. The first 10 chapters, which make up volume 4, are collectively called the \textsanskrit{Mahāvagga}, “the Great Division”. In the present introduction, I will survey the contents of volume 5 and make observations of points of particular interest. For a general introduction to the Monastic Law, see volume 1. For a general introduction to the Khandhakas, see volume 4.

The Cullavagga as a collection is similar to the \textsanskrit{Mahāvagga}, but there is at least one noteworthy difference between the two. Where the \textsanskrit{Mahāvagga} focuses mostly on the main regulations and ceremonies of the Sangha, the Cullavagga is more concerned with lesser regulations and the working out of details. We see this tendency especially in Kd 12–14 and Kd 18–19. This gives the impression that, apart from the \textsanskrit{Parivāra}, the Cullavagga is, overall, slightly later than the rest of the Vinaya \textsanskrit{Piṭaka}. This impression is strengthened by the fact that the last two chapters, Kd 21–22, are concerned with the time after the Buddha’s demise. That the Cullavagga is late fits with my suggestion in the introduction to volume 4 that new material was normally added at the end of the evolving Khandhakas.\footnote{Or when appropriate, it was added at the end of existing chapters. }

I have argued in the introduction to volume 4 that the Buddha’s biography forms the framework for the Khandhakas as a whole. This biography, although incomplete in the Khandhakas as we have them, comes to an end in Kd 17. This means that the material after Kd 17, that is, the last five chapters of the Khandhakas, have an appendix-like quality to them. I will comment further on this as I look at the individual chapters.

The fact that Kd 17 rounds off the Buddha biography in the Khandhakas makes it worthy of special consideration. The stories of \textsanskrit{Ajātasattu} becoming the king of Magadha (\href{https://suttacentral.net/pli-tv-kd17/en/brahmali\#3.4.1}{Kd~17:3.4.1}) and of Devadatta’s schism (\href{https://suttacentral.net/pli-tv-kd17/en/brahmali\#4.1.1}{Kd~17:4.1.1}) are connected with the events in \href{https://suttacentral.net/dn2/en/sujato\#99.6}{DN~2}, where King \textsanskrit{Ajātasattu} is consumed with remorse for having killed his father. It is here that he approaches the Buddha, seemingly for the first time, having understood that Devadatta was not worthy of special respect. DN 16, which appears to dovetail with DN 2, completes the Buddha’s biography until the time of his death. The narrative of DN 16 fits nicely between the events of Kd 17 and the story of the two Councils, \textit{\textsanskrit{saṅgītis}}, told in Kd 21 and Kd 22. It is almost as if DN 16 belongs to this part of the Vinaya \textsanskrit{Piṭaka}. When we discuss Kd 21 below, we shall see that this suggestion is more than mere speculation.

We shall now move onto discussing the individual \textit{khandhakas} of this collection. As an initial overview, here are the twelve chapters of the Cullavagga:

\begin{enumerate}%
\item The Chapter on Legal Procedures, Kamma-kkhandhaka (\href{https://suttacentral.net/pli-tv-kd11/en/brahmali}{Kd~11})%
\item The Chapter on Those on Probation, \textsanskrit{Pārivāsika}-kkhandhaka (\href{https://suttacentral.net/pli-tv-kd12/en/brahmali}{Kd~12})%
\item The Gathering up Chapter, Samuccaya-kkhandhaka (\href{https://suttacentral.net/pli-tv-kd13/en/brahmali}{Kd~13})%
\item The Chapter on the Settling of Legal Issues, Samatha-kkhandhaka (\href{https://suttacentral.net/pli-tv-kd14/en/brahmali}{Kd~14})%
\item The Chapter on Minor Topics, Khuddakavatthu-kkhandhaka (\href{https://suttacentral.net/pli-tv-kd15/en/brahmali}{Kd~15})%
\item The Chapter on Resting Places, \textsanskrit{Senāsana}-kkhandhaka (\href{https://suttacentral.net/pli-tv-kd16/en/brahmali}{Kd~16})%
\item The Chapter on Schism in the Sangha, \textsanskrit{Saṅghabhedaka}-kkhandhaka (\href{https://suttacentral.net/pli-tv-kd17/en/brahmali}{Kd~17})%
\item The Chapter on Proper Conduct, Vatta-kkhandhaka (\href{https://suttacentral.net/pli-tv-kd18/en/brahmali}{Kd~18})%
\item The Chapter on the Cancellation of the Monastic Code, \textsanskrit{Pātimokkhaṭṭhapana}-kkhandhaka (\href{https://suttacentral.net/pli-tv-kd19/en/brahmali}{Kd~19})%
\item The Chapter on Nuns, Bhikkhuni-kkhandhaka (\href{https://suttacentral.net/pli-tv-kd20/en/brahmali}{Kd~20})%
\item The Chapter on the Group of Five Hundred, \textsanskrit{Pañcasatika}-kkhandhaka (\href{https://suttacentral.net/pli-tv-kd21/en/brahmali}{Kd~21})%
\item The Chapter on the Group of Seven Hundred, Sattasatika-kkhandhaka (\href{https://suttacentral.net/pli-tv-kd22/en/brahmali}{Kd~22}).%
\end{enumerate}

\section*{The Chapter on Legal Procedures, Kamma-kkhandhaka, Kd 11}

Kd 11 lays down regulations for a set of seven legal procedures that function as a mild punishment and whose purpose it is to make a misbehaving monastic change their course. They are responses that are available to the Sangha to be used at its discretion.

The seven are as follows:

\begin{enumerate}%
\item \textit{\textsanskrit{Tajjanīyakamma},} “the legal procedure of condemnation”, is used to censure a monastic who is quarrelsome and a creator of conflict in the Sangha (\href{https://suttacentral.net/pli-tv-kd11/en/brahmali\#1.1.2}{Kd~11:1.1.2}).%
\item \textit{Niyassakamma,} “the legal procedure of demotion”, is imposed on a monastic who is ignorant, often commits offenses, and socializes improperly with householders. The legal procedure instructs such a monastic to live with formal support (\href{https://suttacentral.net/pli-tv-kd11/en/brahmali\#9.1.1}{Kd~11:9.1.1}).%
\item \textit{\textsanskrit{Pabbājanīyakamma}}, “the legal procedure of banishment”, is used to ban a monastic from a specified location because of their corrupting effect, causing people to lose faith in the real Dhamma. This legal procedure is the preliminary step to \href{https://suttacentral.net/pli-tv-bu-vb-ss13/en/brahmali\#1.7.1}{Bu~Ss~13:1.7.1} (\href{https://suttacentral.net/pli-tv-kd11/en/brahmali\#13.1.1}{Kd~11:13.1.1}).%
\item \textit{\textsanskrit{Paṭisāraṇīyakamma}}, “the legal procedure of reconciliation”, is imposed on a monastic who abuses or insults lay people. The procedure instructs them to ask forgiveness of the lay people concerned (\href{https://suttacentral.net/pli-tv-kd11/en/brahmali\#18.1.1}{Kd~11:18.1.1}).%
\item \textit{\textsanskrit{Āpattiyā} adassane \textsanskrit{ukkhepanīyakamma}}, “the legal procedure of ejection for not recognizing an offense”, is imposed on a monastic who is unwilling to accept that their wrong behavior is an offense. This and the two following procedures have the effect of ejecting the misbehaving monastic from the Sangha (\href{https://suttacentral.net/pli-tv-kd11/en/brahmali\#25.1.1}{Kd~11:25.1.1}).%
\item \textit{\textsanskrit{Āpattiyā} \textsanskrit{appaṭikamme} \textsanskrit{ukkhepanīyakamma}}, “the legal procedure of ejection for not making amends for an offense”, is imposed on a monastic who is unwilling to follow the required procedure for the clearing of an offense (\href{https://suttacentral.net/pli-tv-kd11/en/brahmali\#31.1.1}{Kd~11:31.1.1}).%
\item \textit{\textsanskrit{Pāpikāya} \textsanskrit{diṭṭhiyā} \textsanskrit{appaṭinissagge} \textsanskrit{ukkhepanīyakamma}}, “the legal procedure of ejection for not giving up a bad view”, is imposed on a monastic who is unwilling to let go of a bad view (\href{https://suttacentral.net/pli-tv-kd11/en/brahmali\#32.1.1}{Kd~11:32.1.1}).%
\end{enumerate}

When any of these procedures has been imposed on a monastic, they are obliged to follow a set of eighteen observances, which in sum amount to a loss of status. In addition, they should not commit any offense similar to or worse than the offense that led to the legal procedure. The full list of eighteen is at \href{https://suttacentral.net/pli-tv-kd11/en/brahmali\#5.1.3}{Kd~11:5.1.3}. If the monastic in question complies with the required conduct to the Sangha’s satisfaction, the Sangha may lift the procedure, returning the monastic to their normal status. The Canonical text does not say how long this period of compliance must last, but according to the commentary it is five or ten days.\footnote{Sp 4.8: \textit{\textsanskrit{Kittakaṁ} \textsanskrit{kālaṁ} \textsanskrit{vattaṁ} \textsanskrit{pūretabbanti}? Dasa \textsanskrit{vā} \textsanskrit{pañca} \textsanskrit{vā} \textsanskrit{divasāni}}, “How long should the conduct be fulfilled? For ten or five days.” }

The last three procedures, collectively known as \textit{\textsanskrit{ukkhepanīyakamma}}, “legal procedures of ejection”, are more serious than the others.\footnote{See respectively \href{https://suttacentral.net/pli-tv-kd11/en/brahmali\#25.1.1}{Kd~11:25.1.1}, \href{https://suttacentral.net/pli-tv-kd11/en/brahmali\#31.1.1}{Kd~11:31.1.1}, and \href{https://suttacentral.net/pli-tv-kd11/en/brahmali\#32.1.1}{Kd~11:32.1.1}. } They bar a monastic from normal association with the Sangha, such as taking part in the observance-day ceremony or legal procedures, effectively creating a temporary schism.\footnote{See \href{https://suttacentral.net/pli-tv-kd10/en/brahmali\#1.6.1}{Kd~10:1.6.1}–1.8.16 for details on how this may happen. } If the legal procedure causes the monastic to mend their ways, all is well. If, however, they do not, then such procedures can lead to a proper schism in the Sangha. For this reason, such procedures should only be done in exceptional circumstances and only if a schism is unlikely to happen.\footnote{See for instance \href{https://suttacentral.net/pli-tv-kd10/en/brahmali\#1.6.4}{Kd~10:1.6.4}, etc. }

Because of the severity of the legal procedures of ejection, the ejected monastic is required to keep 43 observances rather than 18. Among these 43, there are further observances to do with loss of status. There is also a dual prohibition from living apart from other monastics and from living under the same roof as other monastics. In other words, while they should not associate closely with other monastics, they should live near enough to show that their behavior has changed for the better. The full list is at \href{https://suttacentral.net/pli-tv-kd11/en/brahmali\#27.1.3}{Kd~11:27.1.3}.

\section*{The Chapter on Those on Probation, \textsanskrit{Pārivāsika}-kkhandhaka, Kd 12}

Kd 12 sets out the conduct to be observed by a monastic who is undertaking the process of rehabilitation for a \textit{\textsanskrit{saṅghādisesa}} offense. There are five stages in this process:

\begin{enumerate}%
\item \textit{\textsanskrit{Parivāsa}}, “probation” (\href{https://suttacentral.net/pli-tv-kd12/en/brahmali\#1.1.1}{Kd~12:1.1.1})%
\item \textit{\textsanskrit{Mūlāya} \textsanskrit{paṭikassanārahā}}, “deserving to be sent back to the beginning” (\href{https://suttacentral.net/pli-tv-kd12/en/brahmali\#4.1.1}{Kd~12:4.1.1})%
\item \textit{\textsanskrit{Mānattārahā}}, “deserving the trial period” (\href{https://suttacentral.net/pli-tv-kd12/en/brahmali\#5.1.1}{Kd~12:5.1.1})%
\item \textit{\textsanskrit{Mānattacārikā},} “undertaking the trial period” (\href{https://suttacentral.net/pli-tv-kd12/en/brahmali\#6.1.1}{Kd~12:6.1.1})%
\item \textit{\textsanskrit{Abbhānārahā}}, “deserving rehabilitation” (\href{https://suttacentral.net/pli-tv-kd12/en/brahmali\#9.1.1}{Kd~12:9.1.1}).%
\end{enumerate}

The first and third of these are only undertaken by monks, whereas the remaining three are observed both by monks and nuns. The period of “probation” is the same as the period a monk has concealed an offense. If a monk or nun commits another \textit{\textsanskrit{saṅghādisesa}} offense during the process of rehabilitation, they must restart the process from the beginning, that is, they “deserve to be sent back to the beginning”. When a monk has finished his period of probation, he “deserves the trial period”. All monastics who have committed a \textit{\textsanskrit{saṅghādisesa}} offense, whether it is concealed or not, must “undertake the trial period”. For monks it lasts for six days, whereas for nuns it is half a month. When the trial period is complete, they “deserve rehabilitation”.

During each phase of the rehabilitation process, the offending monastic must undertake the 94 observances set out at \href{https://suttacentral.net/pli-tv-kd12/en/brahmali\#1.2.3}{Kd~12:1.2.3}. As in Kd 11, the effect of these observances is to lower the status of the offender. The eighteen observances mentioned at \href{https://suttacentral.net/pli-tv-kd11/en/brahmali\#5.1.3}{Kd~11:5.1.3} are found here too. Moreover, many are similar to the 43 observances to be followed by one who has been ejected (\href{https://suttacentral.net/pli-tv-kd11/en/brahmali\#27.1.3}{Kd~11:27.1.3}). All the remaining rules, bar one, are elaborations on or slight expansions of the 43. According to the final rule, which only applies to someone undertaking the probation or the trial period, the offender must announce their status to all monastics in the monastery where they are staying (\href{https://suttacentral.net/pli-tv-kd12/en/brahmali\#1.2.27}{Kd~12:1.2.27}). Finally, a monk or nun undertaking the trial period can only travel if accompanied by a Sangha of four or more monastics (\href{https://suttacentral.net/pli-tv-kd12/en/brahmali\#6.1.62}{Kd~12:6.1.62}).

\section*{The Gathering up Chapter, Samuccaya-kkhandhaka, Kd 13}

\href{https://suttacentral.net/pli-tv-kd13/en/brahmali\#1.1.1}{Kd~13} gives a detailed description of the \textit{\textsanskrit{saṅghādisesa}} rehabilitation process. The basic process is straightforward, but it can get quite complex if the offender commits further \textit{\textsanskrit{saṅghādisesa}} offences during the process. Kd 13 also deals with the case of a monk not remembering the number of offenses he has committed or the number of days he has concealed an offense (\href{https://suttacentral.net/pli-tv-kd13/en/brahmali\#26.1.1}{Kd~13:26.1.1}–26.4.12).

The summary verses include the following rather striking four lines:

\begin{verse}%
The teachers of analytical statements,\\

Who are the inspiration of Sri Lanka,\\

The residents of the \textsanskrit{Mahāvihāra} monastery—\\

These were their words for the longevity of the true Teachings.

%
\end{verse}

The \textsanskrit{Mahāvihāra} Monastery was the main monastery of the usually dominant \textsanskrit{Mahāvihāra} sect in Sri Lanka. This verse seems to suggest that this entire \textit{khandhaka} was authored on the island. Yet we know from Frauwallner’s study that all early schools had an equivalent chapter.\footnote{Frauwallner, pp. 109–110. } Nevertheless, we can reasonably conclude that the Sangha in Sri Lanka would have had a major hand in forming Kd 13. By extension, we can infer that it is likely it would also have been involved in shaping other parts of the Khandhakas, although perhaps to a lesser extent.

\section*{The Chapter on the Settling of Legal Issues, Samatha-kkhandhaka, Kd 14}

This chapter is essentially an expansion and analysis of the seven principles for the settling of legal issues, the \textit{\textsanskrit{adhikaraṇasamathas}}, which are found at the end of the two \textsanskrit{Pātimokkhas}. Interestingly, this is the only \textit{khandhaka} without summary verses.

\href{https://suttacentral.net/pli-tv-kd14/en/brahmali\#1.1.1}{Kd~14} begins by providing origin stories and permutation series for the seven, reflecting the structure of the majority of \textsanskrit{Pātimokkha} rules as found in the Sutta-\textsanskrit{vibhaṅga}.\footnote{The other core aspects of the \textsanskrit{Vibhaṅga}, that is, the word analysis and the non-offense clause, are not relevant to the \textit{\textsanskrit{adhikaraṇasamathas}}. } There is little new in these origin stories. Five of the seven are no more than pro forma stories, two of which feature the group of six monks.\footnote{These are the stories to the first, fourth, fifth, sixth, and seventh of the seven principles. The only original content is the name of a monk, \textsanskrit{Upavāḷa}, in the origin story to the sixth principle on “further penalty”. } Of the remaining two, which relate to As 2 and As 3, “resolution through recollection” and “resolution because of past insanity”, the first has the same origin story as \href{https://suttacentral.net/pli-tv-bu-vb-ss8/en/brahmali\#1.1.1}{Bu~Ss~8}. Only the origin story to As 3 is properly unique to this chapter. It involves the monk Gagga who is cleared of his offences due to past insanity (\href{https://suttacentral.net/pli-tv-kd14/en/brahmali\#5.1.1}{Kd~14:5.1.1}).

As to the permutation series, they are mostly concerned with the legitimate and illegitimate application of the seven principles. They are closely related to the rules on \textit{\textsanskrit{saṅghakamma}} as we find them in \href{https://suttacentral.net/pli-tv-kd9/en/brahmali}{Kd~9}.

The middle part of \href{https://suttacentral.net/pli-tv-kd14/en/brahmali\#14.1.1}{Kd~14} is a dry Abhidhamma-style analysis of the four kinds of legal issues.\footnote{That is, legal issues arising from disputes, legal issues arising from accusations, legal issues arising from offenses, and legal issues arising from business. } The distinct Abhidhamma flavor shows up in several ways. First, topics are introduced through a series of questions beginning with “What (there)”, \textit{tattha katama}.\footnote{Starting at \href{https://suttacentral.net/pli-tv-kd14/en/brahmali\#14.2.2}{Kd~14:14.2.2}. } This is the same procedure and wording as we find in several Abhidhamma texts, especially the \textsanskrit{Vibhaṅga}, and to a lesser extent in late Canonical texts such as the \textsanskrit{Paṭisambhidāmagga} and the \textsanskrit{Nettippakaraṇa}.\footnote{It occurs a total of 1,209 times in the \textsanskrit{Vibhaṅga}, and 126 times in the \textsanskrit{Nettippakaraṇa} and 27 times in the \textsanskrit{Paṭisambhidāmagga}. } There is nothing quite like it anywhere in the four main \textsanskrit{Nikāyas} or the rest of the Vinaya \textsanskrit{Piṭaka}, apart from the \textsanskrit{Parivāra}. Second, we are introduced to the triad “wholesome, unwholesome, or indeterminate”, \textit{kusala, akusala,} or \textit{\textsanskrit{abyākataṁ}},\footnote{At \href{https://suttacentral.net/pli-tv-kd14/en/brahmali\#14.3.1}{Kd~14:14.3.1}–14.11.14. } which, with the exception of the \textsanskrit{Parivāra}, is only found in the Abhidhamma and Abhidhamma-style texts.\footnote{To be precise, in the \textsanskrit{Mahāniddesa}, the \textsanskrit{Cūḷaniddesa}, the \textsanskrit{Paṭisambhidāmagga}, and the \textsanskrit{Peṭakopadesa}. It is also found in the \textsanskrit{Milindapañha}, which is normally not considered part of the \textsanskrit{Tipiṭaka}. } Lastly, the exposition here is especially dry and theoretical, even by the standards of other abstract \textit{khandhakas}, such as Kd 9. I conclude that this section must have been added quite late in the evolution of the Khandhakas.

The last part of \href{https://suttacentral.net/pli-tv-kd14/en/brahmali\#14.16.1}{Kd~14} sets out the process by which the seven principles should be applied to resolve the four kinds of legal issues. This section too has a certain Abhidhamma feel, and is likely late. The structure is repetitive in a way the Suttas are not, replicating the same long paragraph verbatim fourteen times.\footnote{E.g. at \href{https://suttacentral.net/pli-tv-kd14/en/brahmali\#14.16.11}{Kd~14:14.16.11}: “It’s been resolved face-to-face. Face-to-face with what? Face-to-face with the Sangha, the Teaching, the Monastic Law, and the persons concerned. This is the meaning of face-to-face with the Sangha: the monks who should be present have arrived, consent has been brought for those who are eligible to give their consent, and no one present objects to the decision. This is the meaning of face-to-face with the Teaching and the Monastic Law: the Teaching, the Monastic Law, the Teacher’s instruction—that by which that legal issue is resolved. This is the meaning of face-to-face with the persons concerned: both sides—those who are disputing and those they’re disputing with—are present. When a legal issue has been resolved like this, if any of the participants reopen it, they incur an offense entailing confession for the reopening. If anyone who gave their consent criticizes the resolution, they incur an offense entailing confession.” } Then there is the list of synonyms, which again goes beyond what is normally found in the Suttas.\footnote{E.g. at \href{https://suttacentral.net/pli-tv-kd14/en/brahmali\#14.27.44}{Kd~14:14.27.44}: “The doing of, the performing of, the participation in, the consent to, the agreement to, the non-objection to …”. } Finally we have the frequent use of the words \textit{\textsanskrit{siyā}} and \textit{tattha}, “might it be” and “therein”, to define and differentiate terms and concepts, much as they tend to be used in the Abhidhamma.\footnote{\textit{\textsanskrit{Siyā}} is used 2,300 times in this sense in the \textsanskrit{Vibhaṅga}, while \textit{tattha} is found over 9,000 times in the Abhidhamma as a whole. }

As to its content, this section includes details on how a majority decision is achieved, different ways of conducting a vote, and how to press someone who is initially unwilling to admit to their offenses.\footnote{Respectively at \href{https://suttacentral.net/pli-tv-kd14/en/brahmali\#14.24.1}{Kd~14:14.24.1}–14.24.21, \href{https://suttacentral.net/pli-tv-kd14/en/brahmali\#14.25.25}{Kd~14:14.25.25}–14.26.27, and \href{https://suttacentral.net/pli-tv-kd14/en/brahmali\#14.29.1}{Kd~14:14.29.1}–14.29.27. } These are detailed explanations that must have evolved over time in response to uncertainties about how the seven principles were to be used.

I now wish to return to the question of the lack of summary verses and the related question of the development of this chapter. In the introduction to volume 4, I make the point that the \textit{\textsanskrit{adhikaraṇasamathas}}, as they are now found in the \textsanskrit{Pātimokkha}, are missing an analysis, a \textit{\textsanskrit{vibhaṅga}}. I argue there that such an analysis must have existed at some point, and that in the course of history it was moved elsewhere, most likely to the current chapter. Initially this may have looked similar to the analysis of the seven principles as we find it in \href{https://suttacentral.net/mn104/en/sujato\#13.1}{MN~104}, followed by a gradual expansion until it reached the current size of Kd 14, counting about 30 pages in the PTS edition.

Let us take a brief look at how this expansion may have happened and get some idea of the volume of text that was added. The description of the seven principles in MN 104 is comparatively concise, extending over approximately three pages in the PTS version.\footnote{The Pali word count at MN 104 is about 650 compared to almost 10,000 in Kd 14. } If we divide these three pages equally over the seven, we are left with a \textit{\textsanskrit{vibhaṅga}} for each that is somewhere in length between a short \textit{\textsanskrit{pācittiya}} rule and a \textit{sekhiya}. This is reasonable as a starting point from which the \textit{\textsanskrit{vibhaṅga}} developed further. To fit with the other \textsanskrit{Pātimokkha} rules the \textit{\textsanskrit{adhikaraṇasamathas}} would have required origin stories, thus expanding these rules significantly. If we assume that these origin stories are the same as what we now have in Kd 14, and we add them to the short explanations at MN 104, we increase the length of the \textit{\textsanskrit{vibhaṅga}} roughly by a factor of five.\footnote{From 650 to almost 3,250 words. } In addition to origin stories, the seven principles needed other explanatory material, such as permutation series, setting out their proper and improper use. This material adds another 40 percent to the length of the text.\footnote{Reaching a total of 4,650 words in the Pali. This comprises the first section of the chapter, \href{https://suttacentral.net/pli-tv-kd14/en/brahmali\#1.1.1}{Kd~14:1.1.1}–13.4.1. } On top of this, the last part of \href{https://suttacentral.net/pli-tv-Kd%C2%A014/en/brahmali\#14.16.1}{Kd~14}, which, as we have seen, sets out the practical application of the seven, adds yet another 75% to the total length.\footnote{That is, it adds another 3,560 words to the total. }

At some point in this process, it was decided that the amount of material was too much for the \textsanskrit{Vibhaṅga} and a new \textit{khandhaka} was created. New material continued to accrue to Kd 14, eventually resulting in the chapter as we have it today. We are left with seven principles for settling legal issues in the \textsanskrit{Pātimokkha} that are no more than bare bones, yet a large \textit{khandhaka} setting out these principles in great detail.

We are now in a position to discuss the anomalous lack of summary verses in Kd 14. I mentioned in the introduction to volume 4 that this is unlikely to be because Kd 14 is particularly late compared to other \textit{khandhakas}, as can be inferred from the fact that it has parallels in the other schools.\footnote{Frauwallner, pp. 113–116. } We may find an explanation, however, if we consider the origin of Kd 14 as part of the Sutta-\textsanskrit{vibhaṅga}.

In the Sutta-\textsanskrit{vibhaṅga} there are two kinds of summary verses, \textit{\textsanskrit{uddānas}}, one for the case studies and one at the end of each chapter. The case studies look at specific actions by monastics that may or may not be an offense, and then adjudicate whether in fact it is. Since the \textit{\textsanskrit{adhikaraṇasamathas}} are not rules but principles to be followed, there is no adjudication of potentially broken rules. And since case studies do not apply, there is no corresponding \textit{\textsanskrit{uddāna}} either. As to the end-of-chapter \textit{\textsanskrit{uddānas}}, the seven principles did not require this. In the early version of the seven as set out at MN 104, we find them instead listed at the beginning,.

We can conclude, then, that when the seven principles were moved from the \textsanskrit{Vibhaṅga} to the Khandhakas, there was probably no \textit{\textsanskrit{uddāna}} to follow along. And since there was no \textit{\textsanskrit{uddāna}} when Kd 14 was created, there was no precedent for an \textit{\textsanskrit{uddāna}} to this chapter.

\section*{The Chapter on Minor Topics, Khuddakavatthu-kkhandhaka, Kd 15}

Kd 15 has no over-arching theme. It is mostly a collection of odds and ends that do not fit naturally anywhere else, hence its name. Still, Kd 15 does include a number of interesting rules and regulations, some of which are at the core of the monastic life, as well as a famous protection chant.

The chapter starts with a number of rules on bathing, personal beautification, and entertainment (\href{https://suttacentral.net/pli-tv-kd15/en/brahmali\#1.1.1}{Kd~15:1.1.1}–3.2.3). The overall message of these rules, and similar rules elsewhere, is that luxuries and indulgent behavior are not appropriate for monastics. Such universal principles are sometimes better guides to proper conduct than specific rules, which often lose their relevance in the course of time.

A bit further on we come to a curious rule that fruit is allowable if it has been “damaged by fire, a knife, or a fingernail, or it’s seedless, or the seeds have been removed” (\href{https://suttacentral.net/pli-tv-kd15/en/brahmali\#5.2.9}{Kd~15:5.2.9}). The purpose of this rule is to avoid damaging seeds, for which the last two of the five are obvious solutions. The meaning of the first three, however, is less clear. The typical modern interpretation is that they refer to rituals, whereby perforating the skin of a fruit with a knife or a nail is sufficient to make it allowable for monastic consumption. Yet this is at odds with teachings that deny the efficacy of ritual purity, such as the verses at \href{https://suttacentral.net/mn7/en/sujato\#20.1}{MN~7} that dismiss the act of ritual bathing. It is possible, therefore, that this rule should be interpreted to mean that the seeds need to be properly damaged, for instance by cooking.

Kd 15 continues with a set of protective verses known as the Khandha-paritta (\href{https://suttacentral.net/pli-tv-kd15/en/brahmali\#6.1.1}{Kd~15:6.1.1}). The chant speaks of the well-wishing and spreading of love toward all beings, but especially snakes. Whether such a chant has any protective effect in its own right is debatable. Just prior to the chant, the Buddha is quoted as saying that one should actually suffuse the snakes with love. The chant should therefore probably be regarded as an encouragement to develop this sublime state of mind. This matches what we find in the \textsanskrit{Mettānisaṁsa} Sutta at \href{https://suttacentral.net/an11.15/en/sujato}{AN~11.15}. One is protected from danger only if one develops love to a high level.

We then have the memorable story of \textsanskrit{Piṇḍola} \textsanskrit{Bhāradvāja} who displays his psychic powers to win a wooden bowl as a prize, which the Buddha compares to a woman exposing her genitals for a small sum of money (\href{https://suttacentral.net/pli-tv-kd15/en/brahmali\#8.1.1}{Kd~15:8.1.1}). The Buddha forbids monastics from displaying such powers. This is followed by a large number of rules on bowls and related requisites, and then rules related to the sewing of robes and more.\footnote{The rules on bowls and related requisites are at \href{https://suttacentral.net/pli-tv-kd15/en/brahmali\#8.2.26}{Kd~15:8.2.26}–10.3.7, while the rules on the sewing of robes are at \href{https://suttacentral.net/pli-tv-kd15/en/brahmali\#11.1.1}{Kd~15:11.1.1}–11.7.11. } Next we have rules on the construction of a variety of buildings, including walking paths (both outdoors and indoors), saunas, and wells (\href{https://suttacentral.net/pli-tv-kd15/en/brahmali\#14.1.1}{Kd~15:14.1.1}–17.2.16). The first two of these were allowed for health reasons. It is interesting to observe that covered or indoor walking paths were part of monastery infrastructure from such an early period. Toward to the end of the chapter, we find regulations on the building of restrooms (\href{https://suttacentral.net/pli-tv-kd15/en/brahmali\#35.1.1}{Kd~15:35.1.1}).

More rules follow, before we come to the allowance to overturn the bowl, one of the few sanctions that monastics may impose on lay followers (\href{https://suttacentral.net/pli-tv-kd15/en/brahmali\#20.3.3}{Kd~15:20.3.3}). This allowance may be used against a lay person who is acting to harm monastics or Buddhism more generally. Once the overturning of the bowl has been effected through \textit{\textsanskrit{saṅghakamma}}, the lay person in question may not interact with the Sangha. In particular, the monastics will not receive alms from them, which is the practice from which this allowance gets its name.

The remainder of the chapter consists of a variety of minor rules, of which I will mention a few that may be of particular interest. We have rules against monks growing sideburns and goatees (\href{https://suttacentral.net/pli-tv-kd15/en/brahmali\#27.4.12}{Kd~15:27.4.12}), a remarkable testimony to the stability of certain aspects of human culture. We have a well-known rule against adding Vedic-style verses to the word of the Buddha (\href{https://suttacentral.net/pli-tv-kd15/en/brahmali\#33.1.13}{Kd~15:33.1.13}). The Buddha says that this should not be done. The exact meaning of the overall passage is disputed, with many learned papers written in support of various views.\footnote{For instance, Norman, 1992, and Levman, 2008–2009. See also discussion by Bhikkhu Sujato at https://discourse.suttacentral.net/t/sakaya-niruttiya-with-my-own-interpretation/. } What seems clear, however, is that one should not try to artificially elevate the Dhamma by giving it a fancy form. According to yet another rule, monastics should neither study nor teach cosmological theories or worldly subjects (\href{https://suttacentral.net/pli-tv-kd15/en/brahmali\#33.2.1}{Kd~15:33.2.1}–33.2.28), a timely reminder to modern monastics who have unlimited access to information via the internet. Monastics are not allowed to bless someone who sneezes, for, says the Buddha, what can a blessing do!\footnote{Here the blessing consists of saying, “May you live long!” See \href{https://suttacentral.net/pli-tv-kd15/en/brahmali\#33.3.10}{Kd~15:33.3.10}. } These texts may be ancient, but sometimes they have a remarkably modern flavor.

\section*{The Chapter on Resting Places, \textsanskrit{Senāsana}-kkhandhaka, Kd 16}

The main focus of Kd 16 is \textit{\textsanskrit{senāsana}}, a word that encompasses everything from dwellings to furniture to simple sleeping places.\footnote{See Appendix I: Technical Terms for a discussion of \textit{\textsanskrit{senāsana}}. } The chapter also includes the inspiring story of how \textsanskrit{Anāthapiṇḍika} became a follower of the Buddha and a host of minor rules.

Kd 16 begins with the Buddha giving an allowance for dwellings, followed by details on how they are to be built and a section on allowable furniture (\href{https://suttacentral.net/pli-tv-kd16/en/brahmali\#1.1.1}{Kd~16:1.1.1}–3.5.14). Next comes an allowance to build assembly halls (\href{https://suttacentral.net/pli-tv-kd16/en/brahmali\#3.6.1}{Kd~16:3.6.1}). We can discern the gradual emergence of Buddhist monastic institutions.

The text continues with the remarkable story of \textsanskrit{Anāthapiṇḍika} (\href{https://suttacentral.net/pli-tv-kd16/en/brahmali\#4.1.1}{Kd~16:4.1.1}–4.10.17). When, on a visit to \textsanskrit{Rājagaha}, he hears the word “Buddha”, he is so excited he can hardly sleep that night. He gets up before dawn, leaves the town, but is paralyzed with fear as he is engulfed in darkness. Nevertheless, he makes his way to the \textsanskrit{Sītavana}, the Cool Grove, where the Buddha is staying. The Buddha welcomes him, gives him a Dhamma talk, and \textsanskrit{Anāthapiṇḍika} becomes a stream-enterer.

\textsanskrit{Anāthapiṇḍika} then heads back to \textsanskrit{Sāvatthī} to set up a monastery for the Sangha. The most suitable property is owned by Prince Jeta, a son of King Pasenadi, who is unwilling to sell. Yet everything has a price. When \textsanskrit{Anāthapiṇḍika} covers the whole land in gold coins, Jeta relents and even decides to make small donation of his own. \textsanskrit{Anāthapiṇḍika} then builds a monastery with all facilities.

When the Buddha eventually arrives in \textsanskrit{Sāvatthī}, he tells \textsanskrit{Anāthapiṇḍika} to dedicate the monastery to the Sangha as a whole, both present and future. This becomes the standard and ideal way of giving to the Sangha. The monastery becomes known as the Jeta grove, \textsanskrit{Anāthapiṇḍika}’s monastery. It is the monastery where the Buddha ends up spending most of his time, and was on all accounts the main center of Buddhism while the Buddha was still alive.

Embedded in the story of \textsanskrit{Anāthapiṇḍika} is a discussion of seniority and how this affects the distribution of requisites. Although more senior members of the Sangha should be treated with respect and be given the best food and seat, anything belonging to the Sangha, which would include dwellings, should \emph{not} be reserved according to seniority (\href{https://suttacentral.net/pli-tv-kd16/en/brahmali\#6.4.7}{Kd~16:6.4.7}). Again, this highlights the non-hierarchical and democratic organization of the Sangha. The Buddha illustrates how the monastics should cooperate by telling the story of the Tittira \textsanskrit{Jātaka}, number 37 of that collection (\href{https://suttacentral.net/pli-tv-kd16/en/brahmali\#6.3.2.1}{Kd~16:6.3.2.1}).

Kd 16 continues the laying down of various offices of the Sangha, the precedent for which is set at \href{https://suttacentral.net/pli-tv-kd8/en/brahmali\#5.1.5}{Kd~8:5.1.5}. These include a work manager, an allocator of dwellings, a meal designator, and more.\footnote{See \href{https://suttacentral.net/pli-tv-kd16/en/brahmali\#5.2.8}{Kd~16:5.2.8}–5.3.11, \href{https://suttacentral.net/pli-tv-kd16/en/brahmali\#11.2.1}{Kd~16:11.2.1}–11.2.16, and \href{https://suttacentral.net/pli-tv-kd16/en/brahmali\#21.1.9.1}{Kd~16:21.1.9.1}–21.3.37. } Then there is the important rule that valuable belongings of the Sangha cannot be given away, even to individual monastics (\href{https://suttacentral.net/pli-tv-kd16/en/brahmali\#15.2.1}{Kd~16:15.2.1}). In addition to this, there are a large number of assorted rules.

\section*{The Chapter on Schism in the Sangha, \textsanskrit{Saṅghabhedaka}-kkhandhaka, Kd 17}

Most of Kd 17 tells the story of Devadatta and how he tried, and eventually succeeded, in creating a schism in the Sangha. Kd 17 is a complement to Kd 10. Where Kd 10 tells of the potential for schism when a monk is ejected from the Sangha, Kd 17 tells of an actual schism when a group of monks, led by Devadatta, go their own way and form a separate community.

Kd 17 begins with the account of the going forth of a number of young men from the Sakyan clan, the Buddha’s extended family. Among them is Ānanda, Anuruddha, Bhaddiya, and Devadatta. There is an entertaining section on Anuruddha, who initially refuses to go forth because he thinks monastic life is too hard. But when he hears about the endless work of the household life—which he hitherto has been shielded from!—he decides that going forth is preferrable after all (\href{https://suttacentral.net/pli-tv-kd17/en/brahmali\#1.1.3}{Kd~17:1.1.3}–1.4.30). With such a backstory, it is all the more remarkable that he was such a successful monastic.

Then there is Bhaddiya, previously a king, who soon after his ordination reaches full awakening (\href{https://suttacentral.net/pli-tv-kd17/en/brahmali\#1.5.1}{Kd~17:1.5.1}–1.6.11). The other monks become concerned for his well-being when they see him sitting in the forest repeatedly exclaiming, “Oh, what bliss!” Is he thinking of the pleasures of the palace? Might he be about to disrobe? It turns out he is reflecting on the superiority of monastic life, even to the life of a king. Or perhaps, \emph{especially} to the life of a king!

The story continues with Devadatta’s meeting with Prince \textsanskrit{Ajātasattu} and his deterioration in good qualities when \textsanskrit{Ajātasattu} becomes his supporter (\href{https://suttacentral.net/pli-tv-kd17/en/brahmali\#2.1.4}{Kd~17:2.1.4}–2.1.24). It is the archetypal story that shows the dangers in gain, honor, and praise. Devadatta soon asks the Buddha to hand over the Sangha to him, which the Buddha refuses (\href{https://suttacentral.net/pli-tv-kd17/en/brahmali\#3.1.1}{Kd~17:3.1.1}). The Buddha then lays down a legal procedure, the so-called \textit{\textsanskrit{pakāsanīyakamma}}, by which the Sangha can make an official announcement to their lay followers, in this case about Devadatta’s corruption. The announcement is duly made (\href{https://suttacentral.net/pli-tv-kd17/en/brahmali\#3.2.1}{Kd~17:3.2.1}–3.3.30).

The story turns to Prince \textsanskrit{Ajātasattu}’s initially unsuccessful attempt at killing his own father, King \textsanskrit{Bimbisāra} (\href{https://suttacentral.net/pli-tv-kd17/en/brahmali\#3.4.1}{Kd~17:3.4.1}–3.5.36). When the king finds out that his son wants to rule the kingdom, he simply hands it over to him. Once again, we see King \textsanskrit{Bimbisāra} as the ideal king, for a further discussion of which see the introduction to volume 4.

Soon afterwards, Devadatta tries in vain to kill the Buddha through a series of different plots (\href{https://suttacentral.net/pli-tv-kd17/en/brahmali\#3.6.1}{Kd~17:3.6.1}–3.13.4). First, he convinces King \textsanskrit{Ajātasattu} to send assassins who, instead of killing the Buddha, end up bowing to him. Devadatta’s next scheme is to roll a rock off a hill, trying to hit the Buddha who is walking below. The rock misses, but a splinter hits the Buddha’s foot, causing the Buddha to bleed. Thus, Devadatta has committed one of the five actions with immediate results, a so-called \textit{\textsanskrit{ānantarika}-kamma}, condemning him to be reborn in hell in the next life.\footnote{Or \textit{\textsanskrit{ānantariya}-kammas}. They are as follows: killing one’s mother or father, killing an \textit{arahant}, causing a Buddha to bleed, and creating a schism in the Sangha. They are mentioned as a group at \href{https://suttacentral.net/an6/en/brahmali\#93}{AN~6.93}, but only explained in detail in the Abhidhamma at \href{https://suttacentral.net/vb17/en/brahmali\#941}{Vb~17:941}. } Finally, Devadatta tries to kill the Buddha by releasing \textsanskrit{Nāḷāgiri}, a fierce elephant, down a street, with the Buddha coming in the opposite direction. In this famous story, the Buddha tames \textsanskrit{Nāḷāgiri} through his power of loving kindness.

After failing to kill the Buddha, Devadatta decides instead to split the Sangha through a schism. He asks the Buddha to lay down certain ascetic practices for the monks, which the Buddha declines. Devadatta then builds up a following, both among monks and lay people, on the grounds that he is more ascetic than the Buddha. Finally, he asks his monks to vote on his proposal. When they vote in favor, the schism is formalized (\href{https://suttacentral.net/pli-tv-kd17/en/brahmali\#3.14.1}{Kd~17:3.14.1}–4.1.13).

The chapter ends with a short technical discussion of what constitutes schism (\href{https://suttacentral.net/pli-tv-kd17/en/brahmali\#5.1.1}{Kd~17:5.1.1}–5.3.22), followed by a section setting out its karmic consequences (\href{https://suttacentral.net/pli-tv-kd17/en/brahmali\#5.4.1}{Kd~17:5.4.1}–5.6.10).

\section*{The Chapter on Proper Conduct, Vatta-kkhandhaka, Kd 18}

Kd 18 is a compilation of monastic etiquette for a variety of circumstances. This gives the chapter an appendix-like feeling, which fits with its position near the end of the Khandhakas.

The chapter comprises fourteen sections on proper conduct:

\begin{enumerate}%
\item for newly-arrived monastics (\href{https://suttacentral.net/pli-tv-kd18/en/brahmali\#1.1.1}{Kd~18:1.1.1})%
\item for resident monastics (\href{https://suttacentral.net/pli-tv-kd18/en/brahmali\#2.1.1}{Kd~18:2.1.1})%
\item for departing monastics (\href{https://suttacentral.net/pli-tv-kd18/en/brahmali\#3.1.1}{Kd~18:3.1.1})%
\item in connection with the expression of appreciation (\href{https://suttacentral.net/pli-tv-kd18/en/brahmali\#4.1.1}{Kd~18:4.1.1})%
\item in relation to dining halls (\href{https://suttacentral.net/pli-tv-kd18/en/brahmali\#4.2.1}{Kd~18:4.2.1})%
\item for alms collectors (\href{https://suttacentral.net/pli-tv-kd18/en/brahmali\#5.1.1}{Kd~18:5.1.1})%
\item for those staying in the wilderness (\href{https://suttacentral.net/pli-tv-kd18/en/brahmali\#6.1.1}{Kd~18:6.1.1})%
\item in relation to dwellings (\href{https://suttacentral.net/pli-tv-kd18/en/brahmali\#7.1.1}{Kd~18:7.1.1})%
\item in relation to saunas (\href{https://suttacentral.net/pli-tv-kd18/en/brahmali\#8.1.1}{Kd~18:8.1.1})%
\item in relation to restrooms (\href{https://suttacentral.net/pli-tv-kd18/en/brahmali\#9.1.1}{Kd~18:9.1.1})%
\item toward a preceptor (\href{https://suttacentral.net/pli-tv-kd18/en/brahmali\#11.1.1}{Kd~18:11.1.1})%
\item toward a student (\href{https://suttacentral.net/pli-tv-kd18/en/brahmali\#12.1.1}{Kd~18:12.1.1})%
\item toward a teacher (\href{https://suttacentral.net/pli-tv-kd18/en/brahmali\#12.11.13.1}{Kd~18:12.11.13.1})%
\item toward a pupil (\href{https://suttacentral.net/pli-tv-kd18/en/brahmali\#12.11.143.1}{Kd~18:12.11.143.1}).%
\end{enumerate}

Most of the content concerns mundane good behavior that is not worth commenting on, but a few points may be noted. Section four shows that the expression of appreciation, the \textit{\textsanskrit{anumodanā}}, is mandatory and goes back to the time of early Buddhism (\href{https://suttacentral.net/pli-tv-kd18/en/brahmali\#4.1.7}{Kd~18:4.1.7}). At present the \textit{\textsanskrit{anumodanā}} is normally no more than a perfunctory chant of a standard set of verses. At the time of the Buddha, however, it was usually an inspiring set of verses or perhaps a short talk given after the meal. We see many examples of such \textit{\textsanskrit{anumodanās}} throughout the Suttas.\footnote{For instance at \href{https://suttacentral.net/dn16/en/sujato\#1.31.1}{DN~16:1.31.1}, \href{https://suttacentral.net/mn5/en/sujato\#16.2}{MN~5:16.2}, \href{https://suttacentral.net/mn91/en/sujato\#17.2}{MN~91:17.2}, and \href{https://suttacentral.net/mn92/en/sujato\#25.6}{MN~92:25.6}. }

In section five, on the etiquette in relation to dining halls, we find a large number of rules that are equivalent to the \textit{sekhiya} rules of the \textsanskrit{Pātimokkha}, specifically \href{https://suttacentral.net/pli-tv-bu-vb-sk1/en/brahmali\#1.16.1}{Sk~1} and \href{https://suttacentral.net/pli-tv-bu-vb-sk3/en/brahmali\#1.3.1}{Sk~3}–56 (\href{https://suttacentral.net/pli-tv-kd18/en/brahmali\#4.3.3}{Kd~18:4.3.3}–4.5.25). We have already discussed this matter in the introduction to volume 2, concluding that these rules existed in the Khandhakas first and were moved to the \textit{sekhiya} section of the \textsanskrit{Pātimokkha} at a later time.

Section seven compels monastics to learn the constellations and the so-called lunar mansions, which delineate the path of the moon through the sky (\href{https://suttacentral.net/pli-tv-kd18/en/brahmali\#6.1.12}{Kd~18:6.1.12}). This knowledge was required so that the monks and nuns could keep track of the months and the seasons. They were also supposed to know the geographical region, the \textit{\textsanskrit{disā}}, within which they were staying (\href{https://suttacentral.net/pli-tv-kd18/en/brahmali\#6.1.4}{Kd~18:6.1.4}).

According to section ten, the restrooms should be used according to the order of arrival, not according to seniority (\href{https://suttacentral.net/pli-tv-kd18/en/brahmali\#10.1.7}{Kd~18:10.1.7}). Prior to this regulation, monks had waited until they fainted! What a relief to get this rule.

The last four sections are duplicated in Kd 1. They were discussed in the introduction to volume 4.

\section*{The Chapter on the Cancellation of the Monastic Code, \textsanskrit{Pātimokkhaṭṭhapana}-kkhandhaka, Kd 19}

Kd 19 concerns the relatively obscure topic of the cancellation of the \textsanskrit{Pātimokkha}, which may be done against a monastic who has not confessed their offenses. The effect of the cancellation is to bar the monastic concerned from hearing the \textsanskrit{Pātimokkha} recitation.

The phrase \textit{\textsanskrit{pātimokkhaṁ} \textsanskrit{ṭhapeti}} or the compound \textit{\textsanskrit{pātimokkhaṭṭhapana}} is only found in this \textit{khandhaka} and the \textsanskrit{Parivāra}. The presumably closely related phrase \textit{\textsanskrit{uposathaṁ} \textsanskrit{ṭhapeti}} , “canceling the observance day”, is encountered in Kd 11 and 12 in a long list of things not to be done by anyone penalized for severe misconduct.\footnote{See for instance \href{https://suttacentral.net/pli-tv-kd11/en/brahmali\#5.1.13}{Kd~11:5.1.13} and \href{https://suttacentral.net/pli-tv-kd12/en/brahmali\#1.2.11}{Kd~12:1.2.11}. It is also found in a similar context in Kd 20 at \href{https://suttacentral.net/pli-tv-kd20/en/brahmali\#20.1.1}{Kd~20:20.1.1}. } In both of these \textit{khandhakas} the phrase is part of an elaborate sequence of rules, suggesting a late inclusion in the Vinaya. Remarkably, neither phrase is found in Kd 2, which specifically deals with the \textit{uposatha} and the recitation of the \textsanskrit{Pātimokkha}. I therefore surmise that Kd 19 is a relatively late addition to the Khandhakas, which may explain its position near the end of the collection.\footnote{It is also possible that Kd 19 was originally part of Kd 2, but was then separated out because Kd 2 became too long. }

\href{https://suttacentral.net/pli-tv-kd19/en/brahmali\#1.1.1}{Kd~19} begins with a short account of the Buddha postponing the recitation of the \textsanskrit{Pātimokkha} because of an impure monk sitting in the gathering. When the monk refuses to leave, \textsanskrit{Mahāmoggallāna} grabs him by the arms and takes him outside the enclosure. This sets the scene for the Buddha to lay down the current regulation (\href{https://suttacentral.net/pli-tv-kd19/en/brahmali\#2.1.7}{Kd~19:2.1.7}). Before doing so, however, he gives a teaching on how the Dhamma has eight qualities like the ocean (\href{https://suttacentral.net/pli-tv-kd19/en/brahmali\#1.3.1}{Kd~19:1.3.1}–1.4.32). Only one of the eight qualities is directly related to the matter at hand, which may indicate that this teaching was inserted to pad out Kd 19 when it was created.

Most of the remainder of Kd 19 lays down a variety of circumstances in which the cancellation of the \textsanskrit{Pātimokkha} is either legitimate or illegitimate. Towards the end of the chapter, the text sets out the qualities a monastic should establish before raising an issue in the Sangha or before accusing someone else (\href{https://suttacentral.net/pli-tv-kd19/en/brahmali\#4.1.1}{Kd~19:4.1.1}–5.2.7). This is a reminder that the Vinaya should always be practiced with the Dhamma as a backdrop. When the Dhamma is the priority, there is a chance any Vinaya issues may be resolved in harmony.

\section*{The Chapter on Nuns, Bhikkhuni-kkhandhaka, Kd 20}

Kd 20 deals exclusively with rules for nuns. It finds a natural place after the first nineteen chapters, which contain regulations that are common to both Sanghas, but before the text moves on to the more historical material of Kd 21–22.

Kd 20 begins with the story of the founding of the nuns’ Sangha (\href{https://suttacentral.net/pli-tv-kd20/en/brahmali\#1.1.1}{Kd~20:1.1.1}–1.5.23). While the Buddha is visiting his extended family in Kapilavatthu, his foster mother, \textsanskrit{Mahāpajāpati} \textsanskrit{Gotamī}, asks him for the going forth. The Buddha declines, but she persists. The Buddha eventually agrees to her request on the condition that she accepts eight important principles, the so-called \textit{garudhammas}. \textsanskrit{Mahāpajāpati} agrees, making the \textsanskrit{Bhikkhunī}-sangha a reality.

Much has been written about the \textit{garudhammas} and to what extent they discriminate against women. This is not the place to comment in detail on that discussion, yet a few observations seem called for. Of the eight principles, five are also found as \textsanskrit{Pātimokkha} rules and one as a subrule in the \textsanskrit{Vibhaṅga}.\footnote{That is, \textit{garudhammas} 2–7, found respectively at \href{https://suttacentral.net/pli-tv-bi-vb-pc56/en/brahmali\#1.16.1}{Bi~Pc~56}, \href{https://suttacentral.net/pli-tv-bi-vb-pc59/en/brahmali\#1.11.1}{Bi~Pc~59}, \href{https://suttacentral.net/pli-tv-bi-vb-pc57/en/brahmali\#1.15.1}{Bi~Pc~57}, \href{https://suttacentral.net/pli-tv-bi-vb-ss13/en/brahmali\#3.16}{Bi~Ss~13}, \href{https://suttacentral.net/pli-tv-bi-vb-pc63/en/brahmali\#1.41.1}{Bi~Pc~63}, and \href{https://suttacentral.net/pli-tv-bi-vb-pc52/en/brahmali\#1.29.1}{Bi~Pc~52}. } And so, in spite of their name, “important principles”, they are actually classed among the minor rules. The two remaining \textit{garudhammas}, numbers 1 and 8, are not even \textit{\textsanskrit{pācittiyas}}, and so must be regarded as even less important.\footnote{This is for the Theravada school. The \textsanskrit{Mahīśāsaka} and Dharmaguptaka Vinayas both have \textit{garudhamma} 1 as a \textit{\textsanskrit{pācittiya}} offense, but none of the other schools do. \textit{Garudhamma} 8 is not found as a \textit{\textsanskrit{pācittiya}} in any of the schools. See Bhikkhu Sujato, “Bhikkhuni Vinaya Studies”, pp. 49 and 67. } In my view they should be treated like the \textit{sekhiya} rules, that is, they are only offences if they are disregarded out of disrespect. If one has a good reason not to follow them, there is no obligation.

This matters, because \textit{garudhammas} 1 and 8 are generally considered the most discriminatory among the eight.\footnote{\textit{Garudhamma} 1: “A nun who has been fully ordained for a hundred years should bow down to a monk who was given the full ordination on that very day, and she should stand up for him, raise her joined palms to him, and do acts of respect toward him. This principle is to be honored and respected all one’s life, and is not to be breached.” \textit{Garudhamma} 8: “From today onwards, nuns may not correct monks, but monks may correct nuns. This principle too is to be honored and respected all one’s life, and is not to be breached.” } Given that they are rules of etiquette that are contrary to the norms of most modern societies, \textit{\textsanskrit{bhikkhunīs}} have solid grounds for not keeping them. Indeed, if we want Buddhism to remain relevant and thrive, we must, when we can, interpret the Vinaya in line with compassion and human decency.\footnote{Of course, the interpretation has to be reasonable. It has to fall within the natural flexibility of the rules. }

Once \textsanskrit{Mahāpajāpati} becomes a \textit{\textsanskrit{bhikkhunī}}, she wonders what to do with the other women in her company who also want to go forth (\href{https://suttacentral.net/pli-tv-kd20/en/brahmali\#2.1.1}{Kd~20:2.1.1}). This is when the Buddha gives the allowance for monks to ordain \textit{\textsanskrit{bhikkhunīs}}, an allowance that is never rescinded. It is reasonable to think that the same allowance can be made use of in the present day.

It is here that we begin to see that there are exceptions to the independence of the \textsanskrit{Bhikkhunī}-sangha. Monks were involved in the ordination of nuns from the beginning, and this continued to be the case even as the ordination procedure evolved. There are two further important exceptions to the nuns’ independence. First, the nuns are to request and receive a half-monthly instruction, an \textit{\textsanskrit{ovāda}}, from the monks. Second, at the annual invitation ceremony, a nun is to invite correction both from the Sangha of nuns and from the Sangha of monks. I will comment on these issues as we look at the rest of this chapter.

\textsanskrit{Mahāpajāpati} next asks the Buddha how the nuns should practice the \textsanskrit{Pātimokkha} rules they have in common with the monks (\href{https://suttacentral.net/pli-tv-kd20/en/brahmali\#4.1.1}{Kd~20:4.1.1}). The Buddha replies that they should practice them in the same way as the monks do. The significance of this is that the analysis of the Bhikkhu-\textsanskrit{pātimokkha}, the \textsanskrit{Mahā}-\textsanskrit{vibhaṅga}, is to be used when applicable also by the \textit{\textsanskrit{bhikkhunīs}}. Arguably, it also means that the Khandhakas apply to the \textit{\textsanskrit{bhikkhunīs}}.\footnote{Except where the \textit{\textsanskrit{bhikkhunīs}} have their own rules—either here in the \textsanskrit{Bhikkhunī}-kkhandhaka or in the \textsanskrit{Bhikkhunī}-\textsanskrit{vibhaṅga}—that supersede what is in the Khandhakas. } This is significant because many of the rules in the Khandhakas are required for the nuns to have a functioning Sangha.

The newly-formed nuns’ Sangha needed to learn the formalities of the monastic life, such as the recitation of the \textsanskrit{Pātimokkha}, the confession of offenses, the carrying out of legal procedures, and more. The Buddha tells the monks to teach the nuns (\href{https://suttacentral.net/pli-tv-kd20/en/brahmali\#6.1.1}{Kd~20:6.1.1}–7.1.12). This gives us an idea of the proper relationship between monks and nuns: monks should act as teachers when necessary, but the nuns’ community should live independently once they have the required understanding.

Next we have a number of minor regulations concerning the half-monthly instruction (\href{https://suttacentral.net/pli-tv-kd20/en/brahmali\#9.2.6}{Kd~20:9.2.6}–9.5.57), followed by rules prohibiting various kinds of indulgent behavior and luxurious habits. There is a rule that prohibits a nun from taking a fetus in her bowl, highlighting the perennial issue of abortion and that nuns should avoid getting entangled in the private affairs of lay people (\href{https://suttacentral.net/pli-tv-kd20/en/brahmali\#13.1.1}{Kd~20:13.1.1}–13.1.28). There are several rules about whether and how requisites can be shared between the Sangha of nuns and the Sangha of monks. The upshot is that individual monastics may share their own belongings, but the belongings of one Sangha may not be given to the other (\href{https://suttacentral.net/pli-tv-kd20/en/brahmali\#15.1.1}{Kd~20:15.1.1}–16.1.5). This goes to show, once again, that the two Sanghas are strictly autonomous and that Sangha property is never to be given away. There are many further rules which I will not comment on here.

As the \textsanskrit{Bhikkhunī}-sangha grew, the ordination ceremony needed to be upgraded, the final version of which is found in the third and last section of this chapter at \href{https://suttacentral.net/pli-tv-kd20/en/brahmali\#17.1.1}{Kd~20:17.1.1}–17.8.14. It is essentially an evolved version of the monks’ ordination procedure. The dual nature of the new procedure—whereby a \textit{\textsanskrit{bhikkhunī}} needs to be ordained first among the nuns and then with the monks—is the most important difference. There are also a number of additional questions for female ordination candidates, most of which relate to fertility. Infertile women could easily end up as outcasts in a society where women were expected to bear children. One possible refuge for such women would be to ordain as \textit{\textsanskrit{bhikkhunīs}}. The purpose of these questions, then, would be to ensure that the Sangha did not end up as a sanctuary for women who had few other options.

The invitation ceremony also needed an upgrade (\href{https://suttacentral.net/pli-tv-kd20/en/brahmali\#19.1.1}{Kd~20:19.1.1}–19.3.8). As I have mentioned, the nuns were expected to invite correction not just from the \textsanskrit{Bhikkhunī}-sangha, but also from the Bhikkhu-sangha. It is not clear why the nuns were put in this inferior position, but one likely reason is that the monks’ Sangha was older. No doubt, the social expectation in ancient India also played a part. A good strategy for overcoming this gender imbalance is to make the ceremony mutual, in that the monks invite correction from the nuns in return.

As Kd 20 approaches its end, we come to a regulation that was laid down specifically to help nuns, that is, ordination by messenger (\href{https://suttacentral.net/pli-tv-kd20/en/brahmali\#22.1.1}{Kd~20:22.1.1}–22.3.46). Ancient India was a dangerous place for women, and sometimes it was simply too hazardous for them to travel to the nearest Bhikkhu-sangha to receive the second half of their dual ordination. The Buddha lays down that the ordination can happen in the presence of a messenger who will inform the nun in question that her full ordination has been performed. This is one of the few exceptions to the principle that the subject of a legal procedure must be present at the proceedings. We see that the Buddha was occasionally willing to make concessions even to core principles of the Vinaya to make ordination possible for women.

Kd 20 ends with a number of miscellaneous rules. Among these, we find the rule that nuns may not stay in the wilderness (\href{https://suttacentral.net/pli-tv-kd20/en/brahmali\#23.1.4}{Kd~20:23.1.4}). This was apparently laid down as a safety measure, yet the nuns are stuck with a rule that may not be required in many contemporary societies. Once again, I would argue that these lesser rules, for which there is only an offense of wrong conduct, are cultural in nature, and thus not binding on \textit{\textsanskrit{bhikkhunīs}} who live under different social and cultural conditions. And although \textit{\textsanskrit{bhikkhunīs}} are bound by the limitations set by \href{https://suttacentral.net/pli-tv-bi-vb-ss3/en/brahmali\#4.14.1}{Bi~Ss~3}, this gives them more scope for solitude than they would otherwise have.

Then we have the rule that a nun cannot disrobe by verbally renouncing the training (\href{https://suttacentral.net/pli-tv-kd20/en/brahmali\#26.1.4}{Kd~20:26.1.4}). This is in contrast to the monks, who are able to disrobe in this way. Nuns disrobe by the act of removing their robes. The next rule bars a \textit{\textsanskrit{bhikkhunī}} from reordaining if she has earlier gone over to another religion while still wearing her robes. The commentary extends this prohibition against reordaining to include nuns who disrobe.\footnote{Sp 4.434: \textit{Yadeva \textsanskrit{sā} \textsanskrit{vibbhantāti} \textsanskrit{yasmā} \textsanskrit{sā} \textsanskrit{vibbhantā} attano \textsanskrit{ruciyā} \textsanskrit{khantiyā} \textsanskrit{odātāni} \textsanskrit{vatthāni} \textsanskrit{nivatthā}, \textsanskrit{tasmāyeva} \textsanskrit{sā} \textsanskrit{abhikkhunī}, na \textsanskrit{sikkhāpaccakkhānenāti} dasseti. \textsanskrit{Sā} puna \textsanskrit{upasampadaṁ} na labhati}, “\textit{Yadeva \textsanskrit{sā} \textsanskrit{vibbhantā}}: it is shown that she is not a \textit{\textsanskrit{bhikkhunī}} because she has disrobed due to her own will and preference and is dressed in white clothes, not by renouncing the training. She does not obtain the full ordination again.” } There is nothing, however, in the Canonical text to justify this, quite the contrary. The fact that the text mentions the prohibition only for a nun going over to another religion would seem to imply that it does not pertain to disrobing.

We then come to the surprising rule that nuns are allowed to have men shave their heads, cut their nails, and treat their sores (\href{https://suttacentral.net/pli-tv-kd20/en/brahmali\#27.1.1}{Kd~20:27.1.1}). This might seem to fall afoul of \href{https://suttacentral.net/pli-tv-bi-vb-pj5/en/brahmali\#1.54.1}{Bi~Pj~5}, by which a \textit{\textsanskrit{bhikkhunī}} incurs a \textit{\textsanskrit{pārājika}} offense for lustful contact with a lustful man. The reason it does not conflict with Bi Pj 1 is presumably that a nun is quite capable of knowing whether or not she has lust. This is relevant to \href{https://suttacentral.net/pli-tv-bu-vb-ss2/en/brahmali\#1.2.15.1}{Bu~Ss~2}, which imposes a \textit{\textsanskrit{saṅghādisesa}} offense for a monk touching a woman. It follows from what we have seen here that Bu Ss 2, too, is only an offense when the monk knows he is motivated by lust, which is contrary to how this rule is sometimes practiced.\footnote{In some monasteries it is assumed that even a single mind moment of lust is sufficient for a monk to fall into this offense. Given such an assumption, a monk must always go through the \textit{\textsanskrit{saṅghādisesa}} procedure if he touches a woman for whatever reason, because it is impossible for him to be sure he did not have lust for an infinitesimally short period of time. }

\section*{The Chapter on the Group of Five Hundred, \textsanskrit{Pañcasatika}-kkhandhaka, Kd 21}

The narrative of \href{https://suttacentral.net/pli-tv-kd21.1.1.1)%20begins%20soon%20after%20the%20Buddha%E2%80%99s%20passing%20away%20and,%20as%20I%20have%20suggested,%20it%20forms%20a%20seamless%20whole%20with%20the%20Mah%C4%81parinibb%C4%81na%20Sutta,%20DN%C2%A016.%20Up%20to%20this%20point,%20we%20have%20been%20concerned%20with%20rules%20and%20regulations%20that%20were%20at%20least%20ostensibly%20laid%20down%20by%20the%20Buddha.%20Now%20that%20we%20move%20into%20the%20post-%3Ci%20translate='no'%20lang='pli'%3Eparinibb%C4%81na%3C/i%3E%20period,%20the%20time%20for%20laying%20down%20new%20rules%20has%20come%20to%20an%20end.%20The%20monastic%20community%20would%20have%20been%20guided%20by%20the%20Buddha%E2%80%99s%20injunction,%20found%20among%20other%20places%20at%20[DN%C2%A016/en/brahmali\#1.6.13](dn16:1.6.13}{Kd~21}, not to lay down new rules after his death. As a result, Kd 21 is mostly focused on how to preserve the legacy of the Buddha for future generations. Kd 21 along with Kd 22 may thus be considered as true appendices to the Khandhakas.

Kd 21 starts with the story of a monk Subhadda who claims that the death of the Buddha is good news, in that the monks can now do what they like (\href{https://suttacentral.net/pli-tv-kd21/en/brahmali\#1.1.22}{Kd~21:1.1.22}). \textsanskrit{Mahākassapa} sees the obvious danger in such an attitude and suggests to the community that they hold a communal recitation, a Council, to confirm the teachings of the Great Master. The Sangha agrees and the recitation is duly held at \textsanskrit{Rājagaha}, one of the largest towns in ancient India, which had the capacity to support a large gathering of monks. Moreover, they could rely on the patronage of King \textsanskrit{Ajātasattu}.\footnote{Sp 1.0: \textit{Te dutiyadivase \textsanskrit{gantvā} \textsanskrit{rājadvāre} \textsanskrit{aṭṭhaṁsu}. \textsanskrit{Ajātasattu} \textsanskrit{rājā} \textsanskrit{āgantvā} \textsanskrit{vanditvā} “\textsanskrit{kiṁ}, bhante, \textsanskrit{āgatatthā}”ti \textsanskrit{attanā} \textsanskrit{kattabbakiccaṁ} \textsanskrit{paṭipucchi}}, “On the second day they went and stood at the door of the king. King \textsanskrit{Ajātasattu} came and paid his respects, saying, ‘Venerables, why have you come?’ And he asked what he could do.” }

\textsanskrit{Mahākassapa} presides over the recitation, first asking \textsanskrit{Upāli} to recite the Vinaya and then Ānanda to recite the Suttas (\href{https://suttacentral.net/pli-tv-kd21/en/brahmali\#1.7.1}{Kd~21:1.7.1}–1.8.19). It is noteworthy, once again, that Vinaya in this context is specified as the two \textsanskrit{Vibhaṅgas}, the analyses of the Monastic Codes, while the Suttas are given as the five \textsanskrit{Nikāyas}. We can be certain that if the fifth \textsanskrit{Nikāya} was recited at all, it would only have been a small fraction of the material that is now included in this collection. There is no mention of an Abhidhamma.\footnote{The Abhidhamma is not just missing from the Pali account. Frauwallner, p. 151: “In the first place we can say that the Abhidharma was missing. It is not mentioned in the accounts of the \textsanskrit{Mahīśāsaka} and of the Pali school. Even with the \textsanskrit{Mahāsāṁghika} it is missing in the account proper and is merely mentioned in passing at the end, before the list of teachers.” }

The narrative moves on to discuss what constitutes the minor training rules, and whether these can be discarded (\href{https://suttacentral.net/pli-tv-kd21/en/brahmali\#1.9.1}{Kd~21:1.9.1}). Before his passing, according to DN 16, the Buddha is reported to have said that the minor training rules could be abolished (\href{https://suttacentral.net/dn16/en/sujato\#6.3.1}{DN~16:6.3.1}). Yet according to the same Sutta, he had said that the monastics should not abolish the rules he had laid down (\href{https://suttacentral.net/dn16/en/sujato\#1.6.13}{DN~16:1.6.13}). It might be natural to conclude from this apparent contradiction that the former account must be a mistake, because the latter account fits better with the general tenor of the Canonical material, with its emphasis on memorizing and preserving the Dhamma and Vinaya for future generations.\footnote{For instance at \href{https://suttacentral.net/dn16/en/sujato\#6.1.5}{DN~16:6.1.5}. } There is no obvious way, however, of explaining how the idea of abolishing the minor rules would have made its way into DN 16 if it were not regarded as a genuine statement by the Buddha. With the institution of group recitation, as seen at the first Council, it is unlikely that later generations would have been able to get away with adding a statement that so blatantly contradicted what the Buddha had laid down. It seems we have to conclude that the contradiction is real and stems from the earliest period, presumably from the Buddha himself.

Perhaps this is not as surprising as it may seem. The world is complex, and we should expect that apparently contradictory ideas may occasionally stem from the same person. On reflection, it is not even clear that the two declarations of the Buddha in DN 16 are contradictory. When the Buddha says that the monastics should practice the rules as he has laid them down, the Buddha may have included the allowance to abolish the minor rules in this injunction. In other words, they should practice the rules as laid down by the Buddha, inclusive of any exemptions that he had made. It may well be that the two statements are not, in fact, contradictory.

What are the consequences of this for the practice of the monastic rules? Regardless of what the Buddha may have meant, it remains the case that the first Council decided that the rules should be practiced as they had been laid down (\href{https://suttacentral.net/pli-tv-kd21/en/brahmali\#1.9.20}{Kd~21:1.9.20}). It is reasonable to see this as binding also in the present day. Yet it is also clear that this discussion concerned the rules of the \textsanskrit{Pātimokkha}, the \textit{\textsanskrit{sikkhāpada}}. If the minor rules of the \textsanskrit{Pātimokkha}—which would include the \textit{\textsanskrit{pācittiyas}}, the largest category—could potentially be discarded, one may reasonably assume that this argument holds true to an even greater degree for the many minor non-\textsanskrit{Pātimokkha} rules scattered throughout the Khandhakas. In effect, it would make sense to regard them as similar to the \textit{sekhiyas}, that is, as rules that should not be discarded out of disrespect, yet, equally, as non-binding if the cultural context changes. By looking at the Vinaya in this way, it becomes a much more reasonable and adaptable document, which makes it more relevant and acceptable to the contemporary culture of any particular period and place.

That this is a reasonable interpretation is strengthened by the immediately following episode in Kd 21. The senior monks accuse Ānanda of having committed a series of \textit{dukkatas,} offenses of wrong conduct, because of certain actions he performed during the final days of the Buddha’s life (\href{https://suttacentral.net/pli-tv-kd21/en/brahmali\#1.10.1}{Kd~21:1.10.1}–1.10.23). None of these, however, are offenses as laid down elsewhere in the Vinaya \textsanskrit{Piṭaka}. It seems, then, that the elder monks use the word \textit{dukkata} to mean bad conduct in a general sense, not in a strict sense as a rule laid down by the Buddha. By extension, we can assume that the Canonical \textit{dukkatas} can be regarded in the same way. They are not to be seen as a fixed set of offenses, but more like an evolving group that is used according to time and place as the situation may demand. As such, it makes sense to treat them with flexibility, and not as binding in the way of the \textsanskrit{Pātimokkha} rules.

The narrative of Kd 21 continues with the incident of the monk \textsanskrit{Pūraṇa} who, as mentioned, refuses to take the Council as authoritative (\href{https://suttacentral.net/pli-tv-kd21/en/brahmali\#1.11.1}{Kd~21:1.11.1}). He prefers to remember the Suttas as he himself has heard them from the Buddha. That this statement, which is clearly detrimental to the authority of the Council, has nevertheless been included in the Kd 21, shows the fidelity of the Sangha to the received tradition. Little statements like this give us reasons to believe that the editors of the Canonical texts were more concerned with preserving an authentic record of what they saw as historical events than they were with the authentication of a specific set of Suttas and Vinaya.

The last part of Kd 21 concerns the imposition of the supreme penalty, the \textit{\textsanskrit{brahmadaṇḍa}}, on the monk Channa.\footnote{The supreme penalty is described as follows: “Whatever Channa says, the monks shouldn’t correct him, instruct him, or teach him.” (\href{https://suttacentral.net/pli-tv-kd21/en/brahmali\#1.12.8}{Kd~21:1.12.8}) } Before he dies, the Buddha tells Ānanda that the Sangha should penalize Channa in this way for his difficult behavior (\href{https://suttacentral.net/dn16/en/sujato\#6.4.1}{DN~16:6.4.1}). The Sangha tasks Ānanda with the job. On his way to see Channa, Ānanda meets King Udena and his harem. When the harem offers 500 robes to Ānanda, the king is upset, confronting Ānanda with his receipt of so many robes. Ānanda tells the king that he will share the robes with his fellow monks. He then says that the old robes will be made into bedspreads, the old bedspreads into mattress covers, the old mattress covers into floor covers, the old floor covers into doormats, the old doormats into dustcloths, and the old dustcloths will finally be mixed with mud and used to smear the floors. The king is mightily impressed with this frugality, deciding on the spot to offer Ānanda another 500 robes (\href{https://suttacentral.net/pli-tv-kd21/en/brahmali\#1.13.1}{Kd~21:1.13.1}–1.14.33).

Ānanda eventually reaches Ghosita’s Monastery where he meets Channa (\href{https://suttacentral.net/pli-tv-kd21/en/brahmali\#1.15.1}{Kd~21:1.15.1}). When Channa hears of the penalty, he faints! He then mends his ways, practices diligently, and becomes a perfected one, an \textit{arahant}, at which point the supreme penalty is automatically lifted. Even the Vinaya \textsanskrit{Piṭaka} has a few happy endings!

\section*{The Chapter on the Group of Seven Hundred, Sattasatika-kkhandhaka, Kd 22}

The last chapter of the Khandhakas, Kd 22, takes place one hundred years after the Buddha’s passing. By this time, differences have started to appear in the Sangha as to the practice of the monastic rules. This chapter shows how such disagreements should be dealt with.

The narrative at \href{https://suttacentral.net/pli-tv-kd22/en/brahmali\#1.1.1}{Kd~22} starts with the story of the Vajjian monks who have come to accept ten practices that are contrary to the regulations of the Vinaya, most significantly the use of money. When the monk Yasa points out that their practices are illegitimate, they try to eject him from the Sangha through a legal procedure (\href{https://suttacentral.net/pli-tv-kd22/en/brahmali\#1.7.6}{Kd~22:1.7.6}).

Yasa escapes and sets out to gather supporters, among them some highly learned and respected monks (\href{https://suttacentral.net/pli-tv-kd22/en/brahmali\#1.7.11.1}{Kd~22:1.7.11.1}–2.6.19). This process is described in quite a bit of detail, almost as if the events are told in real time. It is likely that this story was added to the Vinaya soon afterwards while the details were still fresh in mind, possibly as part of the second Council, which took place once the Vinaya issues had been sorted out.\footnote{Frauwallner, p. 67: “It must have been composed shortly before or after the second council.” }

After a prolonged process, a large Sangha meets, comprising monks from both sides of the argument (\href{https://suttacentral.net/pli-tv-kd22/en/brahmali\#2.7.1}{Kd~22:2.7.1}). Because the issues are quite complex, the large meeting is unable to come to a conclusion, and so a committee is appointed (\href{https://suttacentral.net/pli-tv-kd22/en/brahmali\#2.7.4}{Kd~22:2.7.4}). In doing so, the Sangha is specifically making use of an allowance laid down at \href{https://suttacentral.net/pli-tv-kd14/en/brahmali\#14.19.1}{Kd~14:14.19.1}. The committee comes together and decides that the ten practices of the Vajjian monks are illegitimate (\href{https://suttacentral.net/pli-tv-kd22/en/brahmali\#2.8.124}{Kd~22:2.8.124}).

The last line of Kd 22 states that this Council was a communal recitation of the Monastic Law (\href{https://suttacentral.net/pli-tv-kd22/en/brahmali\#2.9.1}{Kd~22:2.9.1}). From this we can surmise that, once agreement had been achieved, the Sangha came together and confirmed their common understanding of the scriptures. Thus ends the Khandhakas as a collection.

To the best our knowledge, this was the last time the Sangha made a decision that was effective for all its members. As shown by Frauwallner, all schools of Buddhism record this meeting.\footnote{Frauwallner, p. 129. } They all agree about the general outcome, although some of the details differ.

After the second Council, the spread of Buddhism continues across India and beyond, especially during the Ashokan period, which lies only a few decades into the future.\footnote{This is assuming the modern consensus that the Buddha died about 400 BCE. } Soon it would be impossible to reach agreements that could be disseminated to the entire Sangha. Subgroups would start to practice in their own ways, dependent on geographic location, their own interpretations of the scriptures, and according to the leadership of their community. This reality would gradually become more pronounced as Buddhism spread further and further afield, until we reach the situation we have today, characterized by a wide diversity in practices and understandings of the Canonical scriptures.

In the present day, consensus in the Sangha is further away than ever. Even among small subgroups, consensus is often not achievable. To minimize the potential for disharmony, we should listen to informed voices and respected teachers. We should try to find common ground where we can. In the end, however, every monastic must take personal responsibility in following the word of the Buddha to the best of their ability.

%
\chapter*{Abbreviations}
\addcontentsline{toc}{chapter}{Abbreviations}
\markboth{Abbreviations}{Abbreviations}

\begin{description}%
\item[AN] \textsanskrit{Aṅguttara} Nikāya (references are to Nipāta and \textit{sutta} numbers)%
\item[AN-a] \textsanskrit{Aṅguttara} Nikāya \textsanskrit{aṭṭhakathā}, the commentary on the \textsanskrit{Aṅguttara} Nikāya%
\item[As] \textit{\textsanskrit{adhikaraṇasamathadhamma}}%
\item[Ay] \textit{aniyata}%
\item[Bi] \textit{\textsanskrit{bhikkhunī}}%
\item[Bu] \textit{bhikkhu}%
\item[CPD] Critical Pali Dictionary%
\item[DN] \textsanskrit{Dīgha} \textsanskrit{Nikāya} (references are to \textit{sutta} numbers)%
\item[DN-a] \textsanskrit{Dīgha} \textsanskrit{Nikāya} \textsanskrit{aṭṭhakathā}, the commentary on the \textsanskrit{Dīgha} \textsanskrit{Nikāya}%
\item[DOP] Dictionary of Pali%
\item[f, ff] and the following page, pages%
\item[Iti] Itivuttaka (references are to verse numbers)%
\item[Ja] \textsanskrit{Jātaka} and \textsanskrit{Jātaka} \textsanskrit{aṭṭhakathā}%
\item[Kd] Khandhaka%
\item[Khuddas-\textsanskrit{pṭ}] \textsanskrit{Khuddasikkhā}-\textsanskrit{purāṇaṭīkā} (references are to paragraph numbers)%
\item[Khuddas-\textsanskrit{nṭ}] \textsanskrit{Khuddasikkhā}-\textsanskrit{abhinavaṭīkā} (references are to paragraph numbers)%
\item[Kkh] \textsanskrit{Kaṅkha}̄\textsanskrit{vitaraṇi}̄%
\item[Kkh-\textsanskrit{pṭ}] \textsanskrit{Kaṅkhāvitaraṇīpurāṇa}-\textsanskrit{ṭīkā}%
\item[MN] Majjhima \textsanskrit{Nikāya} (references are to \textit{sutta} numbers)%
\item[MN-a] Majjhima \textsanskrit{Nikāya} \textsanskrit{aṭṭhakathā}, the commentary on the Majjhima \textsanskrit{Nikāya}%
\item[MS] \textsanskrit{Mahāsaṅgīti} \textsanskrit{Tipiṭaka} (the version of the \textsanskrit{Tipiṭaka} found on SuttaCentral)%
\item[N\&E] “Nature and the Environment in Early Buddhism”, Bhante Dhammika%
\item[Nidd-a] \textsanskrit{Mahāniddesa} \textsanskrit{aṭṭhakathā} (references are to VRI edition paragraph numbers)%
\item[NP] \textit{nissaggiya \textsanskrit{pācittiya}}%
\item[p., pp.] page, pages%
\item[Pc] \textit{\textsanskrit{pācittiya}}%
\item[Pd] \textit{\textsanskrit{pāṭidesanīya}}%
\item[PED] Pali English Dictionary%
\item[Pj] \textit{\textsanskrit{pārājika}}%
\item[PTS] Pali Text Society%
\item[Pvr] \textsanskrit{Parivāra}%
\item[SAF] “South Asian Flora as reflected in the twelfth-century Pali lexicon \textsanskrit{Abhidhānapadīpikā}”, J. Liyanaratne%
\item[SED] Sanskrit English Dictionary%
\item[Sk] \textit{sekhiya}%
\item[SN] \textsanskrit{Saṁyutta} \textsanskrit{Nikāya} (references are to \textsanskrit{Saṁyutta} and \textit{sutta} numbers)%
\item[SN-a] \textsanskrit{Saṁyutta} \textsanskrit{Nikāya} \textsanskrit{aṭṭhakathā}, the commentary on the \textsanskrit{Saṁyutta} \textsanskrit{Nikāya} (references are to volume number and paragraph numbers of the VRI version)%
\item[Sp] Samantapāsādikā, the commentary on the Vinaya \textsanskrit{Piṭaka} (references are to volume and paragraph numbers of the VRI version)%
\item[Sp‑ṭ] Sāratthadīpanī-ṭīkā (references follow the division into five volumes of the Canonical text and then add the paragraph number of the VRI version of the sub-commentary)%
\item[Sp‑yoj] \textsanskrit{Pācityādiyojanā} (volume numbers match those of Sp of the online VRI version, which, given that Sp‑yoj starts with the \textit{bhikkhu \textsanskrit{pācittiyas}}, means that Sp‑yoj is divided into four volumes, starting at volume 2; paragraph numbers are those of the VRI version)%
\item[SRT] Siamrath \textsanskrit{Tipiṭaka}, official edition of the \textsanskrit{Tipiṭaka} published in Thailand%
\item[Ss] \textit{\textsanskrit{saṅghādisesa}}%
\item[sv.] \textit{sub voce}, see under%
\item[\textsanskrit{Thīg}] \textsanskrit{Therīgāthā}%
\item[Ud-a] \textsanskrit{Udāna} \textsanskrit{aṭṭhakathā}, the commentary on the \textsanskrit{Udāna} (references are to \textit{sutta} number)%
\item[Vb] \textsanskrit{Vibhaṅga}, the second book of the Abhidhamma \textsanskrit{Piṭaka}%
\item[Vin-\textsanskrit{ālaṅ}-\textsanskrit{ṭ}] \textsanskrit{Vinayālaṅkāra}-\textsanskrit{ṭīkā} (references are to chapter number and paragraph numbers of the VRI version)%
\item[Vin-vn-\textsanskrit{ṭ}] \textsanskrit{Vinayavinicchayaṭīkā} (references are to paragraph numbers of the VRI version)%
\item[Vjb] \textsanskrit{Vajirabuddhiṭīkā} (references are to volume and paragraph numbers of the VRI version)%
\item[Vmv] \textsanskrit{Vimativinodanī}-\textsanskrit{ṭīkā} (references are to volume and paragraph numbers of the VRI version)%
\item[VRI] Vipassana Research Institute, the publisher of the online version of the Sixth Council edition of the Pali Canon at https://www.tipitaka.org%
\item[Vv-a] \textsanskrit{Vimānavatthu} \textsanskrit{aṭṭhakathā}, the commentary on the \textsanskrit{Vimānavatthu} (references are to paragraph numbers of the VRI edition).%
\end{description}

%
\mainmatter%
\pagestyle{fancy}%
\addtocontents{toc}{\let\protect\contentsline\protect\nopagecontentsline}
\part*{The Lesser Division}
\addcontentsline{toc}{part}{The Lesser Division}
\markboth{}{}
\addtocontents{toc}{\let\protect\contentsline\protect\oldcontentsline}

%
\chapter*{{\suttatitleacronym Kd 11}{\suttatitletranslation The chapter on legal procedures }{\suttatitleroot Kammakkhandhaka}}
\addcontentsline{toc}{chapter}{\tocacronym{Kd 11} \toctranslation{The chapter on legal procedures } \tocroot{Kammakkhandhaka}}
\markboth{The chapter on legal procedures }{Kammakkhandhaka}
\extramarks{Kd 11}{Kd 11}

\section*{1. The legal procedure of condemnation }

\scnamo{Homage to the Buddha, the Perfected One, the fully Awakened One }

At\marginnote{1.1.2} one time the Buddha was staying at \textsanskrit{Sāvatthī} in the Jeta Grove, \textsanskrit{Anāthapiṇḍika}’s Monastery. At that time the monks \textsanskrit{Paṇḍuka} and Lohitaka were quarrelsome, argumentative, and creators of legal issues in the Sangha. They went to other monks who were also quarrelsome, argumentative, and creators of legal issues in the Sangha, and said to them, “Don’t let him beat you. Argue back forcefully. You’re wiser, more competent, more learned, and more capable than he. Don’t be afraid of him. We’ll take your side.” Because of that, new quarrels started and existing quarrels became worse. 

The\marginnote{1.2.1} monks of few desires complained and criticized them, “How can the monks \textsanskrit{Paṇḍuka} and Lohitaka act like this?” 

They\marginnote{1.2.9} told the Buddha. Soon afterwards the Buddha had the Sangha gathered and questioned the monks: 

“Is\marginnote{1.2.11} it true, monks, that the monks \textsanskrit{Paṇḍuka} and Lohitaka are acting like this?” 

“It’s\marginnote{1.2.18} true, sir.” 

The\marginnote{1.2.19} Buddha rebuked them, “It’s not suitable for those foolish men, it’s not proper, it’s not worthy of a monastic, it’s not allowable, it’s not to be done. How can they act like this, causing new quarrels to start and existing quarrels to become worse? This will affect people’s confidence, and cause some to lose it.” 

The\marginnote{1.3.1} Buddha then spoke in many ways in dispraise of being difficult to support and maintain, in dispraise of great desires, discontent, socializing, and laziness; but he spoke in many ways in praise of being easy to support and maintain, of fewness of wishes, contentment, self-effacement, ascetic practices, serenity, reduction in things, and being energetic. After giving a teaching on what is right and proper, he addressed the monks: 

“Well\marginnote{1.3.2} then, monks, the Sangha should do a legal procedure of condemnation against the monks \textsanskrit{Paṇḍuka} and Lohitaka. And this is how it should be done. First you should accuse the monks \textsanskrit{Paṇḍuka} and Lohitaka. They should then be reminded of what they’ve done, before they’re charged with an offense. A competent and capable monk should then inform the Sangha: 

‘Please,\marginnote{1.4.3} venerables, I ask the Sangha to listen. These monks \textsanskrit{Paṇḍuka} and Lohitaka are quarrelsome, argumentative, and creators of legal issues in the Sangha. They go to other monks who are also quarrelsome, argumentative, and creators of legal issues in the Sangha, and they say to them, “Don’t let him beat you. Argue back forcefully. You’re wiser, more competent, more learned, and more capable than he. Don’t be afraid of him. We’ll take your side.” Because of this, new quarrels start and existing quarrels become worse. If the Sangha is ready, it should do a legal procedure of condemnation against the monks \textsanskrit{Paṇḍuka} and Lohitaka. This is the motion. 

Please,\marginnote{1.4.13} venerables, I ask the Sangha to listen. These monks \textsanskrit{Paṇḍuka} and Lohitaka are quarrelsome, argumentative, and creators of legal issues in the Sangha. They go to other monks who are also quarrelsome, argumentative, and creators of legal issues in the Sangha, and they say to them, “Don’t let him beat you. Argue back forcefully. You’re wiser, more competent, more learned, and more capable than he. Don’t be afraid of him. We’ll take your side.” Because of this, new quarrels start and existing quarrels become worse. The Sangha does a legal procedure of condemnation against the monks \textsanskrit{Paṇḍuka} and Lohitaka. Any monk who approves of doing this legal procedure should remain silent. Any monk who doesn’t approve should speak up. 

For\marginnote{1.4.26} the second time, I speak on this matter. Please, venerables, I ask the Sangha to listen. These monks \textsanskrit{Paṇḍuka} and Lohitaka are quarrelsome, argumentative, and creators of legal issues in the Sangha. They go to other monks who are also quarrelsome, argumentative, and creators of legal issues in the Sangha, and they say to them, “Don’t let him beat you. Argue back forcefully. You’re wiser, more competent, more learned, and more capable than he. Don’t be afraid of him. We’ll take your side.” Because of this, new quarrels start and existing quarrels become worse. The Sangha does a legal procedure of condemnation against the monks \textsanskrit{Paṇḍuka} and Lohitaka. Any monk who approves of doing this legal procedure should remain silent. Any monk who doesn’t approve should speak up. 

For\marginnote{1.4.40} the third time, I speak on this matter. Please, venerables, I ask the Sangha to listen. These monks \textsanskrit{Paṇḍuka} and Lohitaka are quarrelsome, argumentative, and creators of legal issues in the Sangha. They go to other monks who are also quarrelsome, argumentative, and creators of legal issues in the Sangha, and they say to them, “Don’t let him beat you. Argue back forcefully. You’re wiser, more competent, more learned, and more capable than he. Don’t be afraid of him. We’ll take your side.” Because of this, new quarrels start and existing quarrels become worse. The Sangha does a legal procedure of condemnation against the monks \textsanskrit{Paṇḍuka} and Lohitaka. Any monk who approves of doing this legal procedure should remain silent. Any monk who doesn’t approve should speak up. 

The\marginnote{1.4.54} Sangha has done a legal procedure of condemnation against the monks \textsanskrit{Paṇḍuka} and Lohitaka. The Sangha approves and is therefore silent. I’ll remember it thus.’” 

\subsection*{The group of twelve on illegitimate legal procedures }

“When\marginnote{2.1.1} a legal procedure of condemnation has three qualities, it’s illegitimate, contrary to the Monastic Law, and improperly disposed of: it’s done in the absence of the accused, it’s done without questioning the accused, it’s done without the admission of the accused. 

When\marginnote{2.1.4} a procedure of condemnation has another three qualities, it’s also illegitimate, contrary to the Monastic Law, and improperly disposed of: it’s done against one who hasn’t committed any offense, it’s done against one who’s committed an offense that isn’t clearable by confession, it’s done against one who’s confessed their offense.\footnote{Sp 4.4: \textit{\textsanskrit{Adesanāgāminiyāti} \textsanskrit{pārājikāpattiyā} \textsanskrit{vā} \textsanskrit{saṅghādisesāpattiyā} \textsanskrit{vā}}, “‘Not clearable by confession’ means an offense entailing expulsion or an offense entailing suspension.” } 

When\marginnote{2.1.7} a procedure of condemnation has another three qualities, it’s also illegitimate, contrary to the Monastic Law, and improperly disposed of: it’s done without having accused the person of their offense, it’s done without having reminded the person of their offense, it’s done without having charged the person with their offense. 

“When\marginnote{2.1.10} a procedure of condemnation has another three qualities, it’s also illegitimate, contrary to the Monastic Law, and improperly disposed of: it’s done in the absence of the accused, it’s done illegitimately, it’s done by an incomplete assembly. 

When\marginnote{2.1.13} a procedure of condemnation has another three qualities, it’s also illegitimate, contrary to the Monastic Law, and improperly disposed of: it’s done without questioning the accused, it’s done illegitimately, it’s done by an incomplete assembly. 

When\marginnote{2.1.16} a procedure of condemnation has another three qualities, it’s also illegitimate, contrary to the Monastic Law, and improperly disposed of: it’s done without the admission of the accused, it’s done illegitimately, it’s done by an incomplete assembly. 

“When\marginnote{2.1.19} a procedure of condemnation has another three qualities, it’s also illegitimate, contrary to the Monastic Law, and improperly disposed of: it’s done against one who hasn’t committed any offense, it’s done illegitimately, it’s done by an incomplete assembly. 

When\marginnote{2.1.22} a procedure of condemnation has another three qualities, it’s also illegitimate, contrary to the Monastic Law, and improperly disposed of: it’s done against one who’s committed an offense that isn’t clearable by confession, it’s done illegitimately, it’s done by an incomplete assembly. 

When\marginnote{2.1.25} a procedure of condemnation has another three qualities, it’s also illegitimate, contrary to the Monastic Law, and improperly disposed of: it’s done against one who’s confessed their offense, it’s done illegitimately, it’s done by an incomplete assembly. 

“When\marginnote{2.1.28} a procedure of condemnation has another three qualities, it’s also illegitimate, contrary to the Monastic Law, and improperly disposed of: it’s done without having accused the person of their offense, it’s done illegitimately, it’s done by an incomplete assembly. 

When\marginnote{2.1.31} a procedure of condemnation has another three qualities, it’s also illegitimate, contrary to the Monastic Law, and improperly disposed of: it’s done without having reminded the person of their offense, it’s done illegitimately, it’s done by an incomplete assembly. 

When\marginnote{2.1.34} a procedure of condemnation has another three qualities, it’s also illegitimate, contrary to the Monastic Law, and improperly disposed of: it’s done without having charged the person with their offense, it’s done illegitimately, it’s done by an incomplete assembly.” 

\scend{The group of twelve on illegitimate legal procedures is finished. }

\subsection*{The group of twelve on legitimate legal procedures }

“When\marginnote{3.1.1} a legal procedure of condemnation has three qualities, it’s legitimate, in accordance with the Monastic Law, and properly disposed of: it’s done in the presence of the accused, it’s done with the questioning of the accused, it’s done with the admission of the accused. 

When\marginnote{3.1.4} a procedure of condemnation has another three qualities, it’s also legitimate, in accordance with the Monastic Law, and properly disposed of: it’s done against one who’s committed an offense, it’s done against one who’s committed an offense that’s clearable by confession, it’s done against one who hasn’t confessed their offense. 

When\marginnote{3.1.7} a procedure of condemnation has another three qualities, it’s also legitimate, in accordance with the Monastic Law, and properly disposed of: it’s done after accusing the person of their offense, it’s done after reminding the person of their offense, it’s done after charging the person with their offense. 

“When\marginnote{3.1.10} a procedure of condemnation has another three qualities, it’s also legitimate, in accordance with the Monastic Law, and properly disposed of: it’s done in the presence of the accused, it’s done legitimately, it’s done by a unanimous assembly. 

When\marginnote{3.1.13} a procedure of condemnation has another three qualities, it’s also legitimate, in accordance with the Monastic Law, and properly disposed of: it’s done with the questioning of the accused, it’s done legitimately, it’s done by a unanimous assembly. 

When\marginnote{3.1.16} a procedure of condemnation has another three qualities, it’s also legitimate, in accordance with the Monastic Law, and properly disposed of: it’s done with the admission of the accused, it’s done legitimately, it’s done by a unanimous assembly. 

“When\marginnote{3.1.19} a procedure of condemnation has another three qualities, it’s also legitimate, in accordance with the Monastic Law, and properly disposed of: it’s done against one who’s committed an offense, it’s done legitimately, it’s done by a unanimous assembly. 

When\marginnote{3.1.22} a procedure of condemnation has another three qualities, it’s also legitimate, in accordance with the Monastic Law, and properly disposed of: it’s done against one who’s committed an offense that’s clearable by confession, it’s done legitimately, it’s done by a unanimous assembly. 

When\marginnote{3.1.25} a procedure of condemnation has another three qualities, it’s also legitimate, in accordance with the Monastic Law, and properly disposed of: it’s done against one who hasn’t confessed their offense, it’s done legitimately, it’s done by a unanimous assembly. 

“When\marginnote{3.1.28} a procedure of condemnation has another three qualities, it’s also legitimate, in accordance with the Monastic Law, and properly disposed of: it’s done after accusing the person of their offense, it’s done legitimately, it’s done by a unanimous assembly. 

When\marginnote{3.1.31} a procedure of condemnation has another three qualities, it’s also legitimate, in accordance with the Monastic Law, and properly disposed of: it’s done after reminding the person of their offense, it’s done legitimately, it’s done by a unanimous assembly. 

When\marginnote{3.1.34} a procedure of condemnation has another three qualities, it’s also legitimate, in accordance with the Monastic Law, and properly disposed of: it’s done after charging the person with their offense, it’s done legitimately, it’s done by a unanimous assembly.” 

\scend{The group of twelve on legitimate legal procedures is finished. }

\subsection*{The group of six on wishing }

“When\marginnote{4.1.1} a monk has three qualities, the Sangha may, if it wishes, do a legal procedure of condemnation against him: he’s quarrelsome, argumentative, and a creator of legal issues in the Sangha; he’s ignorant and incompetent, often committing offenses, and lacking in boundaries;\footnote{According to CPD, apparently quoting the commentary (“Bu”), \textit{\textsanskrit{anapadāna}} means “‘who is unable to discern (what is an offense)’, or ‘not setting a good example’.” It is not clear, however, why \textit{\textsanskrit{apadāna}} should be rendered as “discern”. Sp 3.407: \textit{\textsanskrit{Apadānaṁ} vuccati paricchedo; \textsanskrit{āpattiparicchedavirahitoti} attho}, “Limit is called \textit{\textsanskrit{apadāna}}; the meaning is ‘without limit to offenses’.” Sp-\textsanskrit{ṭ} 3.395: \textit{Natthi etassa \textsanskrit{apadānaṁ} \textsanskrit{avakhaṇḍanaṁ} \textsanskrit{āpattipariyantoti} \textsanskrit{anapadāno}}, “\textit{\textsanskrit{Anapadāno}}: he has no \textit{\textsanskrit{apadāna}}, no cutting off, no limit with offenses.” } he’s constantly and improperly socializing with householders.\footnote{Sp 4.6: \textit{Ananulomikehi \textsanskrit{gihisaṁsaggehīti} \textsanskrit{pabbajitānaṁ} ananucchavikehi \textsanskrit{sahasokitādīhi} \textsanskrit{gihisaṁsaggehi}}, “\textit{Ananulomikehi \textsanskrit{gihisaṁsaggehi}}: with unsuitable socializing for those gone forth, that is, sorrowing with (them), etc.” The “sorrowing with, etc.,” presumably refers to a passage found at \href{https://suttacentral.net/sn35.241/en/brahmali\#4.1}{SN 35.241}. The impression given in this Sutta is that of fairly intense socializing. } 

When\marginnote{4.1.6} a monk has another three qualities, the Sangha may, if it wishes, do a procedure of condemnation against him: he has failed in the higher morality; he has failed in conduct; he has failed in view. 

When\marginnote{4.1.9} a monk has another three qualities, the Sangha may, if it wishes, do a procedure of condemnation against him: he disparages the Buddha; he disparages the Teaching; he disparages the Sangha. 

The\marginnote{4.2.1} Sangha may, if it wishes, do a procedure of condemnation against three kinds of monks: those who are quarrelsome, argumentative, and creators of legal issues in the Sangha; those who are ignorant and incompetent, often committing offenses, and lacking in boundaries; those who are constantly and improperly socializing with householders. 

The\marginnote{4.2.6} Sangha may, if it wishes, do a procedure of condemnation against three other kinds of monks: those who’ve failed in the higher morality; those who’ve failed in conduct; those who’ve failed in view. 

The\marginnote{4.2.9} Sangha may, if it wishes, do a procedure of condemnation against three other kinds of monks: those who disparage the Buddha; those who disparage the Teaching; those who disparage the Sangha.” 

\scend{The group of six on wishing is finished. }

\subsection*{The eighteen kinds of conduct }

“A\marginnote{5.1.1} monk who’s had a legal procedure of condemnation done against himself should conduct himself properly. This is the proper conduct: 

\begin{enumerate}%
\item He shouldn’t give the full ordination. %
\item He shouldn’t give formal support.\footnote{For an explanation of the rendering “formal support” for \textit{nissaya}, see Appendix of Technical Terms. } %
\item He shouldn’t have a novice monk attend on him. %
\item He shouldn’t accept being appointed as an instructor of the nuns. %
\item Even if appointed, he shouldn’t instruct the nuns. %
\item He shouldn’t commit the same offense as the offense for which the Sangha did the procedure of condemnation against him. %
\item He shouldn’t commit an offense similar to the offense for which the Sangha did the procedure of condemnation against him. %
\item He shouldn’t commit an offense worse than the offense for which the Sangha did the procedure of condemnation against him. %
\item He shouldn’t criticize the procedure. %
\item He shouldn’t criticize those who did the procedure. %
\item He shouldn’t cancel the observance-day ceremony of a regular monk. %
\item He shouldn’t cancel the invitation ceremony of a regular monk. %
\item He shouldn’t direct a regular monk.\footnote{Sp 4.76: \textit{Na \textsanskrit{savacanīyaṁ} \textsanskrit{kātabbanti} \textsanskrit{palibodhatthāya} \textsanskrit{vā} \textsanskrit{pakkosanatthāya} \textsanskrit{vā} \textsanskrit{savacanīyaṁ} na \textsanskrit{kātabbaṁ}, \textsanskrit{palibodhatthāya} hi karonto “\textsanskrit{ahaṁ} \textsanskrit{āyasmantaṁ} \textsanskrit{imasmiṁ} \textsanskrit{vatthusmiṁ} \textsanskrit{savacanīyaṁ} karomi, \textsanskrit{imamhā} \textsanskrit{āvāsā} ekapadampi \textsanskrit{mā} \textsanskrit{pakkāmi}, \textsanskrit{yāva} na \textsanskrit{taṁ} \textsanskrit{adhikaraṇaṁ} \textsanskrit{vūpasantaṁ} \textsanskrit{hotī}”ti \textsanskrit{evaṁ} karoti. \textsanskrit{Pakkosanatthāya} karonto “\textsanskrit{ahaṁ} te \textsanskrit{savacanīyaṁ} karomi, ehi \textsanskrit{mayā} \textsanskrit{saddhiṁ} \textsanskrit{vinayadharānaṁ} \textsanskrit{sammukhībhāvaṁ} \textsanskrit{gacchāmā}”ti \textsanskrit{evaṁ} karoti; tadubhayampi na \textsanskrit{kātabbaṁ}}, “\textit{Na \textsanskrit{savacanīyaṁ} \textsanskrit{kātabba}}: \textit{\textsanskrit{savacanīya}} is not to be done for the purpose of (creating) an obstacle or for the purpose of summoning. Acting for the purpose of (creating) an obstacle is done like this: ‘I am doing \textit{\textsanskrit{savacanīya}} against the venerable in regard to this offense: do not depart from this monastery even with one foot so long as this legal issue has not been resolved.’ Acting for the purpose of summoning is done like this: ‘I am doing \textit{\textsanskrit{savacanīya}} against you: come with me and let us go to the presence of a master of the Monastic Law.’ Neither of these is to be done.” Sp-\textsanskrit{ṭ} 4.76: \textit{\textsanskrit{Savacanīyanti} \textsanskrit{sadosaṁ}}, “\textit{\textsanskrit{Savacanīyan}}: with flaw.” Vmv 4.76: \textit{\textsanskrit{Savacanīyanti} ettha “sadosa”nti \textsanskrit{atthaṁ} vadati. Attano vacane pavattanakammanti evamettha attho \textsanskrit{daṭṭhabbo}, “\textsanskrit{mā} \textsanskrit{pakkamāhī}”ti \textsanskrit{vā} “ehi \textsanskrit{vinayadharānaṁ} \textsanskrit{sammukhībhāva}”nti \textsanskrit{vā} \textsanskrit{evaṁ} attano \textsanskrit{āṇāya} \textsanskrit{pavattanakakammaṁ} na \textsanskrit{kātabbanti} \textsanskrit{adhippāyo}. \textsanskrit{Evañhi} kenaci \textsanskrit{savacanīye} kate \textsanskrit{anādarena} \textsanskrit{atikkamituṁ} na \textsanskrit{vaṭṭati}, buddhassa \textsanskrit{saṅghassa} \textsanskrit{āṇā} \textsanskrit{atikkantā} \textsanskrit{nāma} hoti}, “\textit{\textsanskrit{Savacanīyan}}: here he says the meaning is ‘with flaw’. Here the meaning is to be understood as bringing about an action when speaking oneself: ‘Don’t leave,’ ‘Come to the presence of master of the Monastic Law’. The intention is one is not to do the bringing about of an action in this way because of a command from oneself.” } %
\item He shouldn’t give instructions to a regular monk.\footnote{\textit{\textsanskrit{Anuvāda}} is not listed in CPD, and based on the commentarial interpretation, both DOP and PED have the wrong definition. Sp 4.76: \textit{Na \textsanskrit{anuvādoti} \textsanskrit{vihāre} \textsanskrit{jeṭṭhakaṭṭhānaṁ} na \textsanskrit{kātabbaṁ}. \textsanskrit{Pātimokkhuddesakena} \textsanskrit{vā} dhammajjhesakena \textsanskrit{vā} na \textsanskrit{bhavitabbaṁ}. \textsanskrit{Nāpi} terasasu \textsanskrit{sammutīsu} \textsanskrit{ekasammutivasenāpi} \textsanskrit{issariyakammaṁ} \textsanskrit{kātabbaṁ}}, “\textit{Na \textsanskrit{anuvādo}}: in a monastery, one is not to be put in a senior position. One should not be the reciter of the Monastic Code or the one who requests someone to speak on the Dhamma. One is not to be made an authority even on account of one agreement among the thirteen agreements.” The thirteen agreements are the various offices of storeman, etc., that Sangha members may hold, see Sp-\textsanskrit{ṭ} 1.69. Vjb 4.428: \textit{\textsanskrit{Anuvādanti} \textsanskrit{issariyaṭṭhānaṁ}}, “\textit{\textsanskrit{Anuvādo}}: a position of authority.” Sp-yoj 4.76: \textit{\textsanskrit{Anuvādoti} ettha \textsanskrit{anusāsanavasena} \textsanskrit{aññe} \textsanskrit{vadatīti}}, “\textit{\textsanskrit{Anuvādo}}: here one speaks to others on account of instructing (them).” The impression given here is that \textit{\textsanskrit{anuvāda}} means instructing others, especially  from a position of authority. } %
\item He shouldn’t get permission from a regular monk to correct him.\footnote{This literally means “an opportunity is not to be caused to be granted”. The idiomatic meaning is “he gets permission” or usually “he gets permission from someone to correct them”. For the sake of clarity, I use both of these renderings depending on the context. } %
\item He shouldn’t accuse a regular monk of an offense. %
\item He shouldn’t remind a regular monk of an offense. %
\item He shouldn’t associate inappropriately with other monks.”\footnote{Sp 4.76: \textit{Na \textsanskrit{bhikkhūhi} sampayojetabbanti \textsanskrit{aññamaññaṁ} \textsanskrit{yojetvā} kalaho na \textsanskrit{kāretabbo}}, “\textit{Na \textsanskrit{bhikkhūhi} \textsanskrit{sampayojetabbaṁ}} means they should not quarrel when they engage with each other.” SRT reads \textit{na \textsanskrit{bhikkhū} \textsanskrit{bhikkhūhi} sampayojetabbanti}, which may be preferrable. It could then be rendered as “He should not create conflict between monks.” Whether the verb \textit{sampayojeti} means to associate with or to cause conflict is not clear. } %
\end{enumerate}

\scend{The eighteen kinds of conduct in regard to the legal procedure of condemnation are finished. }

\subsection*{The group of eighteen on not to be lifted }

When\marginnote{6.1.1} the Sangha had done a legal procedure of condemnation against the monks \textsanskrit{Paṇḍuka} and Lohitaka, they conducted themselves properly and suitably, and deserved to be released. They then went to the monks\footnote{The meaning of the first of these phrases, \textit{\textsanskrit{sammā} vattati}, is straightforward, but the last two, \textit{\textsanskrit{lomaṁ} \textsanskrit{pāteti}} and \textit{\textsanskrit{netthāraṁ} vattati}, are more difficult. Commenting on Bu Ss 13, Sp 1.435 explains the negative version of these phrases thus: \textit{Na \textsanskrit{lomaṁ} \textsanskrit{pātentīti} \textsanskrit{anulomapaṭipadaṁ} \textsanskrit{appaṭipajjanatāya} na \textsanskrit{pannalomā} honti. Na \textsanskrit{netthāraṁ} \textsanskrit{vattantīti} attano \textsanskrit{nittharaṇamaggaṁ} na \textsanskrit{paṭipajjanti}}, “\textit{Na \textsanskrit{lomaṁ} \textsanskrit{pātenti}}: because of their non-practicing in conformity with the path, their bodily hairs are not flat. \textit{Na \textsanskrit{netthāraṁ} vattanti}: they are not practicing the path for their own getting out (of the offense).” My rendering attempts to capture the meaning in a non-literal way. } and told them about this, adding, “What should we do now?” The monks told the Buddha … 

“Well\marginnote{6.1.6} then, lift that legal procedure of condemnation against the monks \textsanskrit{Paṇḍuka} and Lohitaka. 

When\marginnote{6.2.1} a monk has five qualities, a legal procedure of condemnation against him shouldn’t be lifted: he gives the full ordination; he gives formal support; he has a novice monk attend on him; he accepts being appointed as an instructor of the nuns; he instructs the nuns, whether appointed or not. 

When\marginnote{6.2.4} a monk has another five qualities, a procedure of condemnation against him shouldn’t be lifted: he commits the same offense for which the Sangha did the procedure of condemnation against him; he commits an offense similar to the one for which the Sangha did the procedure of condemnation against him; he commits an offense worse than the one for which the Sangha did the procedure of condemnation against him; he criticizes the procedure; he criticizes those who did the procedure. 

When\marginnote{6.2.8} a monk has eight qualities, a procedure of condemnation against him shouldn’t be lifted: he cancels the observance-day ceremony of a regular monk; he cancels the invitation ceremony of a regular monk; he directs a regular monk; he gives instructions to a regular monk; he gets permission from a regular monk to correct him; he accuses a regular monk of an offense; he reminds a regular monk of an offense; he associates inappropriately with other monks.” 

\scend{The group of eighteen on not to be lifted is finished. }

\subsection*{The group of eighteen on to be lifted }

“When\marginnote{7.1.1} a monk has five qualities, a legal procedure of condemnation against him should be lifted: he doesn’t give the full ordination; he doesn’t give formal support; he doesn’t have a novice monk attend on him; he doesn’t accept being appointed as an instructor of the nuns; he doesn’t instruct the nuns, whether appointed or not. 

When\marginnote{7.1.4} a monk has another five qualities, a procedure of condemnation against him should be lifted: he doesn’t commit the same offense for which the Sangha did the procedure of condemnation against him; he doesn’t commit an offense similar to the one for which the Sangha did the procedure of condemnation against him; he doesn’t commit an offense worse than the one for which the Sangha did the procedure of condemnation against him; he doesn’t criticize the procedure; he doesn’t criticize those who did the procedure. 

When\marginnote{7.1.8} a monk has eight qualities, a procedure of condemnation against him should be lifted: he doesn’t cancel the observance-day ceremony of a regular monk; he doesn’t cancel the invitation ceremony of a regular monk; he doesn’t direct a regular monk; he doesn’t give instructions to a regular monk; he doesn’t get permission from a regular monk to correct him; he doesn’t accuse a regular monk of an offense; he doesn’t remind a regular monk of an offense; he doesn’t associate inappropriately with other monks.” 

\scend{The group of eighteen on to be lifted is finished. }

“And\marginnote{8.1.1} this is how it should be lifted. After approaching the Sangha, the monks \textsanskrit{Paṇḍuka} and Lohitaka should arrange their upper robes over one shoulder, pay respect at the feet of the senior monks, squat on their heels, raise their joined palms, and say, ‘Venerables, the Sangha has done a legal procedure of condemnation against us. We’re now conducting ourselves properly and suitably, and deserve to be released. We ask for that legal procedure to be lifted.’ And they should ask a second and a third time. A competent and capable monk should then inform the Sangha: 

‘Please,\marginnote{8.2.1} venerables, I ask the Sangha to listen. The Sangha has done a legal procedure of condemnation against these monks \textsanskrit{Paṇḍuka} and Lohitaka. They’re now conducting themselves properly and suitably, and deserve to be released. They’re asking for that legal procedure to be lifted. If the Sangha is ready, it should lift that legal procedure of condemnation against the monks \textsanskrit{Paṇḍuka} and Lohitaka. This is the motion. 

Please,\marginnote{8.2.5} venerables, I ask the Sangha to listen. The Sangha has done a legal procedure of condemnation against these monks \textsanskrit{Paṇḍuka} and Lohitaka. They’re now conducting themselves properly and suitably, and deserve to be released. They’re asking for that legal procedure to be lifted. The Sangha lifts that legal procedure of condemnation against the monks \textsanskrit{Paṇḍuka} and Lohitaka. Any monk who approves of lifting that legal procedure should remain silent. Any monk who doesn’t approve should speak up. 

For\marginnote{8.2.10} the second time, I speak on this matter. Please, venerables, I ask the Sangha to listen. The Sangha has done a legal procedure of condemnation against these monks \textsanskrit{Paṇḍuka} and Lohitaka. They’re now conducting themselves properly and suitably, and deserve to be released. They’re asking for that legal procedure to be lifted. The Sangha lifts that legal procedure of condemnation against the monks \textsanskrit{Paṇḍuka} and Lohitaka. Any monk who approves of lifting that legal procedure should remain silent. Any monk who doesn’t approve should speak up. 

For\marginnote{8.2.16} the third time, I speak on this matter. Please, venerables, I ask the Sangha to listen. The Sangha has done a legal procedure of condemnation against these monks \textsanskrit{Paṇḍuka} and Lohitaka. They’re now conducting themselves properly and suitably, and deserve to be released. They’re asking for that legal procedure to be lifted. The Sangha lifts that legal procedure of condemnation against the monks \textsanskrit{Paṇḍuka} and Lohitaka. Any monk who approves of lifting that legal procedure should remain silent. Any monk who doesn’t approve should speak up. 

The\marginnote{8.2.22} Sangha has lifted that legal procedure of condemnation against the monks \textsanskrit{Paṇḍuka} and Lohitaka. The Sangha approves and is therefore silent. I’ll remember it thus.’” 

\scend{The first section on the legal procedure of condemnation is finished. }

\section*{2. The legal procedure of demotion }

At\marginnote{9.1.1} that time Venerable Seyyasaka was ignorant and incompetent, often committing offenses, and lacking in boundaries. And he was constantly and improperly socializing with householders. In addition, the monks regularly gave him probation, sent him back to the beginning, gave him trial periods, and rehabilitated him.\footnote{Sp 3.11: \textit{apissu \textsanskrit{bhikkhū} \textsanskrit{pakatāti} apissu \textsanskrit{bhikkhū} \textsanskrit{niccaṁ} \textsanskrit{byāvaṭā} honti}, “\textit{Apissu \textsanskrit{bhikkhū} \textsanskrit{pakatā}}: in addition, the monks were regularly busy.” } 

The\marginnote{9.1.4} monks of few desires complained and criticized him, “How can Venerable Seyyasaka go on like this?” They told the Buddha. 

Soon\marginnote{9.1.9} afterwards the Buddha had the Sangha gathered and questioned the monks: 

“Is\marginnote{9.1.10} it true, monks, that the monk Seyyasaka goes on like this?” 

“It’s\marginnote{9.1.13} true, sir.” 

The\marginnote{9.1.14} Buddha rebuked him, “It’s not suitable for that foolish man, it’s not proper, it’s not worthy of a monastic, it’s not allowable, it’s not to be done. How can he be ignorant and incompetent, often committing offenses, and lacking in boundaries? How can he constantly and improperly socialize with householders? And how can it be that the monks regularly give him probation, send him back to the beginning, give him trial periods, and rehabilitate him? This will affect people’s confidence …” After rebuking him … the Buddha gave a teaching and addressed the monks: 

“Well\marginnote{9.1.22} then, the Sangha should do a legal procedure of demotion against the monk Seyyasaka,\footnote{For an explanation of the rendering “demotion” for \textit{niyassa}, see Appendix of Technical Terms. } instructing him to live with formal support. And this is how it should be done. First you should accuse the monk Seyyasaka. He should then be reminded of what he has done, before he’s charged with an offense. A competent and capable monk should then inform the Sangha: 

‘Please,\marginnote{9.2.3} venerables, I ask the Sangha to listen. This monk Seyyasaka is ignorant and incompetent, often committing offenses, and lacking in boundaries. He’s constantly and improperly socializing with householders. In addition, the monks regularly give him probation, send him back to the beginning, give him the trial period, and rehabilitate him. If the Sangha is ready, it should do a legal procedure of demotion against the monk Seyyasaka, instructing him to live with formal support. This is the motion. 

Please,\marginnote{9.2.10} venerables, I ask the Sangha to listen. This monk Seyyasaka is ignorant and incompetent, often committing offenses, and lacking in boundaries. He’s constantly and improperly socializing with householders. In addition, the monks regularly give him probation, send him back to the beginning, give him the trial period, and rehabilitate him. The Sangha does a legal procedure of demotion against the monk Seyyasaka, instructing him to live with formal support. Any monk who approves of doing this legal procedure should remain silent. Any monk who doesn’t approve should speak up. 

For\marginnote{9.2.19} the second time, I speak on this matter. … For the third time, I speak on this matter. Please, venerables, I ask the Sangha to listen. This monk Seyyasaka is ignorant and incompetent, often committing offenses, and lacking in boundaries. He’s constantly and improperly socializing with householders. In addition, the monks regularly give him probation, send him back to the beginning, give him the trial period, and rehabilitate him. The Sangha does a legal procedure of demotion against the monk Seyyasaka, instructing him to live with formal support. Any monk who approves of doing this legal procedure should remain silent. Any monk who doesn’t approve should speak up. 

The\marginnote{9.2.30} Sangha has done a legal procedure of demotion against the monk Seyyasaka, instructing him to live with formal support. The Sangha approves and is therefore silent. I’ll remember it thus.’” 

\subsection*{The group of twelve on illegitimate legal procedures }

“When\marginnote{10.1.1} a legal procedure of demotion has three qualities, it’s illegitimate, contrary to the Monastic Law, and improperly disposed of: it’s done in the absence of the accused, it’s done without questioning the accused, it’s done without the admission of the accused. 

When\marginnote{10.1.4} a procedure of demotion has another three qualities, it’s also illegitimate, contrary to the Monastic Law, and improperly disposed of: it’s done against one who hasn’t committed any offense, it’s done against one who’s committed an offense that isn’t clearable by confession, it’s done against one who’s confessed their offense. 

When\marginnote{10.1.7} a procedure of demotion has another three qualities, it’s also illegitimate, contrary to the Monastic Law, and improperly disposed of: it’s done without having accused the person of their offense, it’s done without having reminded the person of their offense, it’s done without having charged the person with their offense. 

“When\marginnote{10.1.10} a procedure of demotion has another three qualities, it’s also illegitimate, contrary to the Monastic Law, and improperly disposed of: it’s done in the absence of the accused, it’s done illegitimately, it’s done by an incomplete assembly. 

When\marginnote{10.1.13} a procedure of demotion has another three qualities, it’s also illegitimate, contrary to the Monastic Law, and improperly disposed of: it’s done without questioning the accused, it’s done illegitimately, it’s done by an incomplete assembly. 

When\marginnote{10.1.16} a procedure of demotion has another three qualities, it’s also illegitimate, contrary to the Monastic Law, and improperly disposed of: it’s done without the admission of the accused, it’s done illegitimately, it’s done by an incomplete assembly. 

“When\marginnote{10.1.19} a procedure of demotion has another three qualities, it’s also illegitimate, contrary to the Monastic Law, and improperly disposed of: it’s done against one who hasn’t committed any offense, it’s done illegitimately, it’s done by an incomplete assembly. 

When\marginnote{10.1.22} a procedure of demotion has another three qualities, it’s also illegitimate, contrary to the Monastic Law, and improperly disposed of: it’s done against one who’s committed an offense that isn’t clearable by confession, it’s done illegitimately, it’s done by an incomplete assembly. 

When\marginnote{10.1.25} a procedure of demotion has another three qualities, it’s also illegitimate, contrary to the Monastic Law, and improperly disposed of: it’s done against one who’s confessed their offense, it’s done illegitimately, it’s done by an incomplete assembly. 

“When\marginnote{10.1.28} a procedure of demotion has another three qualities, it’s also illegitimate, contrary to the Monastic Law, and improperly disposed of: it’s done without having accused the person of their offense, it’s done illegitimately, it’s done by an incomplete assembly. 

When\marginnote{10.1.31} a procedure of demotion has another three qualities, it’s also illegitimate, contrary to the Monastic Law, and improperly disposed of: it’s done without having reminded the person of their offense, it’s done illegitimately, it’s done by an incomplete assembly. 

When\marginnote{10.1.34} a procedure of demotion has another three qualities, it’s also illegitimate, contrary to the Monastic Law, and improperly disposed of: it’s done without having charged the person with their offense, it’s done illegitimately, it’s done by an incomplete assembly.” 

\scend{The group of twelve on illegitimate legal procedures is finished. }

\subsection*{The group of twelve on legitimate legal procedures }

“When\marginnote{10.1.38.1} a legal procedure of demotion has three qualities, it’s legitimate, in accordance with the Monastic Law, and properly disposed of: it’s done in the presence of the accused, it’s done with the questioning of the accused, it’s done with the admission of the accused. 

When\marginnote{10.1.41} a procedure of demotion has another three qualities, it’s also legitimate, in accordance with the Monastic Law, and properly disposed of: it’s done against one who’s committed an offense, it’s done against one who’s committed an offense that’s clearable by confession, it’s done against one who hasn’t confessed their offense. 

When\marginnote{10.1.44} a procedure of demotion has another three qualities, it’s also legitimate, in accordance with the Monastic Law, and properly disposed of: it’s done after having accused the person of their offense, it’s done after having reminded the person of their offense, it’s done after having charged the person with their offense. 

“When\marginnote{10.1.47} a procedure of demotion has another three qualities, it’s also legitimate, in accordance with the Monastic Law, and properly disposed of: it’s done in the presence of the accused, it’s done legitimately, it’s done by a unanimous assembly. 

When\marginnote{10.1.50} a procedure of demotion has another three qualities, it’s also legitimate, in accordance with the Monastic Law, and properly disposed of: it’s done with the questioning of the accused, it’s done legitimately, it’s done by a unanimous assembly. 

When\marginnote{10.1.53} a procedure of demotion has another three qualities, it’s also legitimate, in accordance with the Monastic Law, and properly disposed of: it’s done with the admission of the accused, it’s done legitimately, it’s done by a unanimous assembly. 

“When\marginnote{10.1.56} a procedure of demotion has another three qualities, it’s also legitimate, in accordance with the Monastic Law, and properly disposed of: it’s done against one who’s committed an offense, it’s done legitimately, it’s done by a unanimous assembly. 

When\marginnote{10.1.59} a procedure of demotion has another three qualities, it’s also legitimate, in accordance with the Monastic Law, and properly disposed of: it’s done against one who’s committed an offense that’s clearable by confession, it’s done legitimately, it’s done by a unanimous assembly. 

When\marginnote{10.1.62} a procedure of demotion has another three qualities, it’s also legitimate, in accordance with the Monastic Law, and properly disposed of: it’s done against one who hasn’t confessed their offense, it’s done legitimately, it’s done by a unanimous assembly. 

“When\marginnote{10.1.65} a procedure of demotion has another three qualities, it’s also legitimate, in accordance with the Monastic Law, and properly disposed of: it’s done after accusing the person of their offense, it’s done legitimately, it’s done by a unanimous assembly. 

When\marginnote{10.1.68} a procedure of demotion has another three qualities, it’s also legitimate, in accordance with the Monastic Law, and properly disposed of: it’s done after reminding the person of their offense, it’s done legitimately, it’s done by a unanimous assembly. 

When\marginnote{10.1.71} a procedure of demotion has another three qualities, it’s also legitimate, in accordance with the Monastic Law, and properly disposed of: it’s done after charging the person with their offense, it’s done legitimately, it’s done by a unanimous assembly.” 

\scend{The group of twelve on legitimate legal procedures is finished. }

\subsection*{The group of six on wishing }

“When\marginnote{10.1.75.1} a monk has three qualities, the Sangha may, if it wishes, do a legal procedure of demotion against him: he’s quarrelsome, argumentative, and a creator of legal issues in the Sangha; he’s ignorant and incompetent, often committing offenses, and lacking in boundaries; he’s constantly and improperly socializing with householders. 

When\marginnote{10.1.80} a monk has another three qualities, the Sangha may, if it wishes, do a procedure of demotion against him: he has failed in the higher morality; he has failed in conduct; he has failed in view. 

When\marginnote{10.1.83} a monk has another three qualities, the Sangha may, if it wishes, do a procedure of demotion against him: he disparages the Buddha; he disparages the Teaching; he disparages the Sangha. 

The\marginnote{10.1.86} Sangha may, if it wishes, do a procedure of demotion against three kinds of monks: those who are quarrelsome, argumentative, and creators of legal issues in the Sangha; those who are ignorant and incompetent, often committing offenses, and lacking in boundaries; those who are constantly and improperly socializing with householders. 

The\marginnote{10.1.91} Sangha may, if it wishes, do a procedure of demotion against three other kinds of monks: those who’ve failed in the higher morality; those who’ve failed in conduct; those who’ve failed in view. 

The\marginnote{10.1.94} Sangha may, if it wishes, do a procedure of demotion against three other kinds of monks: those who disparage the Buddha; those who disparage the Teaching; those who disparage the Sangha.” 

\scend{The group of six on wishing is finished. }

\subsection*{The eighteen kinds of conduct }

“A\marginnote{10.1.98.1} monk who’s had a legal procedure of demotion done against himself should conduct himself properly. This is the proper conduct: 

\begin{enumerate}%
\item He shouldn’t give the full ordination. %
\item He shouldn’t give formal support. %
\item He shouldn’t have a novice monk attend on him. %
\item He shouldn’t accept being appointed as an instructor of the nuns. %
\item Even if appointed, he shouldn’t instruct the nuns. %
\item He shouldn’t commit the same offense as the offense for which the Sangha did the procedure of demotion against him. %
\item He shouldn’t commit an offense similar to the offense for which the Sangha did the procedure of demotion against him. %
\item He shouldn’t commit an offense worse than the offense for which the Sangha did the procedure of demotion against him. %
\item He shouldn’t criticize the procedure. %
\item He shouldn’t criticize those who did the procedure. %
\item He shouldn’t cancel the observance-day ceremony of a regular monk. %
\item He shouldn’t cancel the invitation ceremony of a regular monk. %
\item He shouldn’t direct a regular monk. %
\item He shouldn’t give instructions to a regular monk. %
\item He shouldn’t get permission from a regular monk to correct him. %
\item He shouldn’t accuse a regular monk of an offense. %
\item He shouldn’t remind a regular monk of an offense. %
\item He shouldn’t associate inappropriately with other monks.” %
\end{enumerate}

\scend{The eighteen kinds of conduct in regard to the legal procedure of demotion are finished. }

Soon\marginnote{11.1.1} the Sangha did a legal procedure of demotion against the monk Seyyasaka, instructing him to live with formal support. Then, by associating with good friends, by having them recite, and by questioning them, he became learned and a master of the tradition; he became an expert on the Teaching, the Monastic Law, and the Key Terms; he became knowledgeable and competent, had a sense of conscience, and was afraid of wrongdoing and fond of the training.\footnote{For an explanation of the rendering “training” for \textit{vinaya}, see Appendix of Technical Terms. } And he conducted himself properly and suitably, and deserved to be released. He then went to the monks and told them about this, adding, “What should I do now?” The monks told the Buddha. He had the monks gathered and said, “Well then, lift that legal procedure of demotion against the monk Seyyasaka.” 

\subsection*{The group of eighteen on not to be lifted }

“When\marginnote{11.2.1} a monk has five qualities, a legal procedure of demotion against him shouldn’t be lifted: he gives the full ordination; he gives formal support; he has a novice monk attend on him; he accepts being appointed as an instructor of the nuns; he instructs the nuns, whether appointed or not. 

When\marginnote{11.2.4} a monk has another five qualities, a procedure of demotion against him shouldn’t be lifted: he commits the same offense for which the Sangha did the procedure of demotion against him; he commits an offense similar to the one for which the Sangha did the procedure of demotion against him; he commits an offense worse than the one for which the Sangha did the procedure of demotion against him; he criticizes the procedure; he criticizes those who did the procedure. 

When\marginnote{11.2.8} a monk has eight qualities, a legal procedure of demotion against him shouldn’t be lifted: he cancels the observance-day ceremony of a regular monk; he cancels the invitation ceremony of a regular monk; he directs a regular monk; he gives instructions to a regular monk; he gets permission from a regular monk to correct him; he accuses a regular monk of an offense; he reminds a regular monk of an offense; he associates inappropriately with other monks.” 

\scend{The group of eighteen on not to be lifted is finished. }

\subsection*{The group of eighteen on to be lifted }

“When\marginnote{11.2.12.1} a monk has five qualities, a legal procedure of demotion against him should be lifted: he doesn’t give the full ordination; he doesn’t give formal support; he doesn’t have a novice monk attend on him; he doesn’t accept being appointed as an instructor of the nuns; he doesn’t instruct the nuns, whether appointed or not. 

When\marginnote{11.2.15} a monk has another five qualities, a procedure of demotion against him should be lifted: he doesn’t commit the same offense for which the Sangha did the procedure of demotion against him; he doesn’t commit an offense similar to the one for which the Sangha did the procedure of demotion against him; he doesn’t commit an offense worse than the one for which the Sangha did the procedure of demotion against him; he doesn’t criticize the procedure; he doesn’t criticize those who did the procedure. 

When\marginnote{11.2.19} a monk has eight qualities, a procedure of demotion against him should be lifted: he doesn’t cancel the observance-day ceremony of a regular monk; he doesn’t cancel the invitation ceremony of a regular monk; he doesn’t direct a regular monk; he doesn’t give instructions to a regular monk; he doesn’t get permission from a regular monk to correct him; he doesn’t accuse a regular monk of an offense; he doesn’t remind a regular monk of an offense; he doesn’t associate inappropriately with other monks.” 

\scend{The group of eighteen on to be lifted is finished. }

“And\marginnote{12.1.1} this is how it should be lifted. After approaching the Sangha, the monk Seyyasaka should arrange his upper robe over one shoulder, pay respect at the feet of the senior monks, squat on his heels, raise his joined palms, and say, ‘Venerables, the Sangha has done a legal procedure of demotion against me. I’m now conducting myself properly and suitably, and deserve to be released. I ask for that legal procedure to be lifted.’ And he should ask a second and a third time. A competent and capable monk should then inform the Sangha: 

‘Please,\marginnote{12.2.2} venerables, I ask the Sangha to listen. The Sangha has done a legal procedure of demotion against the monk Seyyasaka. He’s now conducting himself properly and suitably, and deserves to be released. He’s asking for that legal procedure to be lifted. If the Sangha is ready, it should lift that legal procedure of demotion against the monk Seyyasaka. This is the motion. 

Please,\marginnote{12.2.6} venerables, I ask the Sangha to listen. The Sangha has done a legal procedure of demotion against the monk Seyyasaka. He’s now conducting himself properly and suitably, and deserves to be released. He’s asking for that legal procedure to be lifted. The Sangha lifts that legal procedure of demotion against the monk Seyyasaka. Any monk who approves of lifting that legal procedure should remain silent. Any monk who doesn’t approve should speak up. 

For\marginnote{12.2.11} the second time, I speak on this matter. Please, venerables, I ask the Sangha to listen. The Sangha has done a legal procedure of demotion against the monk Seyyasaka. He’s now conducting himself properly and suitably, and deserves to be released. He’s asking for that legal procedure to be lifted. The Sangha lifts that legal procedure of demotion against the monk Seyyasaka. Any monk who approves of lifting that legal procedure should remain silent. Any monk who doesn’t approve should speak up. 

For\marginnote{12.2.17} the third time, I speak on this matter. Please, venerables, I ask the Sangha to listen. The Sangha has done a legal procedure of demotion against the monk Seyyasaka. He’s now conducting himself properly and suitably, and deserves to be released. He’s asking for that legal procedure to be lifted. The Sangha lifts that legal procedure of demotion against the monk Seyyasaka. Any monk who approves of lifting that legal procedure should remain silent. Any monk who doesn’t approve should speak up. 

The\marginnote{12.2.23} Sangha has lifted that legal procedure of demotion against the monk Seyyasaka. The Sangha approves and is therefore silent. I’ll remember it thus.’” 

\scend{The second section on the legal procedure of demotion is finished. }

\section*{3. The legal procedure of banishment }

At\marginnote{13.1.1} one time the bad and shameless monks Assaji and Punabbasuka were staying at \textsanskrit{Kīṭāgiri}. They were misbehaving in many ways. 

They\marginnote{13.1.3} planted flowering trees, watered and plucked them, and then tied the flowers together. They made the flowers into garlands, garlands with stalks on one side and garlands with stalks on both sides. They made flower arrangements, wreaths, ornaments for the head, ornaments for the ears, and ornaments for the chest. And they had others do the same. They then took these things, or sent them, to the women, the daughters, the girls, the daughters-in-law, and the female slaves of respectable families. 

They\marginnote{13.1.5} ate from the same plates as these women and drank from the same vessels. They sat on the same seats as them, and they lay down on the same beds, on the same sheets, under the same covers, both on the same sheets and under the same covers. They ate at the wrong time, drank alcohol, and wore garlands, perfumes, and cosmetics. They danced, sang, played instruments, and performed. While the women were dancing, singing, playing instruments, and performing, so would they. 

They\marginnote{13.2.1} played various games: eight-row checkers, ten-row checkers, imaginary checkers, hopscotch, pick-up-sticks, board games, tip-cat, painting with the hand, dice, leaf flutes, toy plows, somersaults, pinwheels, toy measures, toy carriages, toy bows, guessing from syllables, thought guessing, mimicking deformities. 

They\marginnote{13.2.2} trained in elephant riding, in horsemanship, in carriage riding, in archery, in swordsmanship. And they ran in front of elephants, in front of horses, and in front of carriages, and they ran backward and forward. They whistled, clapped their hands, wrestled, and boxed. They spread their outer robe on a stage and said to the dancing girls, “Dance here, Sister,” and they made gestures of approval. And they misbehaved in a variety of ways. 

Just\marginnote{13.3.1} then a monk who had completed the rains residence in \textsanskrit{Kāsi} was on his way to visit the Buddha at \textsanskrit{Sāvatthī} when he arrived at \textsanskrit{Kīṭāgiri}. In the morning he robed up, took his bowl and robe, and entered \textsanskrit{Kīṭāgiri} to collect almsfood. He was pleasing in his conduct: in going out and coming back, in looking ahead and looking aside, in bending and stretching his arms. His eyes were lowered, and he was perfect in deportment. When people saw him, they said, “Who’s this, acting like a moron and always frowning? Who’s gonna give almsfood to him? Almsfood should be given to our Venerables Assaji and Punabbasuka, for they’re gentle, congenial, pleasant to speak with, greeting one with a smile, welcoming, friendly, open, the first to speak.” 

A\marginnote{13.3.8} certain lay follower saw that monk walking for alms in \textsanskrit{Kīṭāgiri}. He approached him, bowed, and said, “Venerable, have you received any almsfood?” 

“No,\marginnote{13.3.11} I haven’t.” 

“Come,\marginnote{13.3.12} let’s go to my house.” 

He\marginnote{13.4.1} took that monk to his house and gave him a meal. He then said, “Where are you going, venerable?” 

“I’m\marginnote{13.4.3} going to \textsanskrit{Sāvatthī} to see the Buddha.” 

“Well\marginnote{13.4.4} then, would you please pay respect at the feet of the Buddha in my name and say, ‘Sir, the monastery at \textsanskrit{Kīṭāgiri} has been corrupted. The bad and shameless monks Assaji and Punabbasuka are staying there. And they’re misbehaving in many ways. They plant flowering trees, water them … 

And\marginnote{13.4.10} they misbehave in a variety of ways. Those who previously had faith and confidence have now lost it, and there’s no longer any support for the Sangha. The good monks have left and the bad monks are staying on. Sir, please send monks to stay at the monastery in \textsanskrit{Kīṭāgiri}.’” 

The\marginnote{13.5.1} monk agreed, got up, and set out for \textsanskrit{Sāvatthī}. When he eventually arrived, he went to the Buddha in \textsanskrit{Anāthapiṇḍika}’s Monastery. He bowed to the Buddha and sat down. Since it is the custom for Buddhas to greet newly-arrived monks, the Buddha said to him, “I hope you are keeping well, monk, I hope you’re getting by?  I hope you’re not tired from traveling?  And where have you come from?” 

“I’m\marginnote{13.5.8} keeping well, sir, I’m getting by. I’m not tired from traveling.” He then told the Buddha all that had happened at \textsanskrit{Kīṭāgiri}, adding, “That’s where I’ve come from, sir.” 

Soon\marginnote{13.6.1} afterwards the Buddha had the Sangha gathered and questioned the monks: “Is it true, monks, that the bad and shameless monks Assaji and Punabbasuka are staying at \textsanskrit{Kīṭāgiri} and misbehaving in this way? And is it true that those people who previously had faith and confidence have now lost it, that there’s no longer any support for the Sangha, and that the good monks have left and the bad monks are staying on?” 

“It’s\marginnote{13.6.9} true, sir.” 

The\marginnote{13.6.10} Buddha rebuked them, “It’s not suitable … How can those foolish men misbehave in this way? This will affect people’s confidence …” After rebuking them … he gave a teaching and addressed \textsanskrit{Sāriputta} and \textsanskrit{Moggallāna}: “Go to \textsanskrit{Kīṭāgiri} and do a legal procedure of banishing the monks Assaji and Punabbasuka. They’re your students.” 

“Sir,\marginnote{13.6.32} how can we do a procedure of banishing these monks from \textsanskrit{Kīṭāgiri}? They’re temperamental and harsh.” 

“Well\marginnote{13.6.33} then, take many monks.” 

“Alright.”\marginnote{13.6.34} 

“And\marginnote{13.7.1} this is how it should be done. First you should accuse the monks Assaji and Punabbasuka. They should then be reminded of what they’ve done, before they’re charged with an offense. A competent and capable monk should then inform the Sangha: 

‘Please,\marginnote{13.7.3} venerables, I ask the Sangha to listen. These monks, Assaji and Punabbasuka, are corrupters of families and badly behaved. Their bad behavior is seen and heard about, and the families corrupted by them are seen and heard about. If the Sangha is ready, it should do a legal procedure of banishing them, prohibiting the monks Assaji and Punabbasuka from staying at \textsanskrit{Kīṭāgiri}. This is the motion. 

Please,\marginnote{13.7.10} venerables, I ask the Sangha to listen. These monks, Assaji and Punabbasuka, are corrupters of families and badly behaved. Their bad behavior is seen and heard about, and the families corrupted by them are seen and heard about. The Sangha does a legal procedure of banishing them, prohibiting the monks Assaji and Punabbasuka from staying at \textsanskrit{Kīṭāgiri}. Any monk who approves of doing this legal procedure should remain silent. Any monk who doesn’t approve should speak up. 

For\marginnote{13.7.18} the second time, I speak on this matter. … For the third time, I speak on this matter. Please, venerables, I ask the Sangha to listen. These monks, Assaji and Punabbasuka, are corrupters of families and badly behaved. Their bad behavior is seen and heard about, and the families corrupted by them are seen and heard about. The Sangha does a legal procedure of banishing them, prohibiting the monks Assaji and Punabbasuka from staying at \textsanskrit{Kīṭāgiri}. Any monk who approves of doing this legal procedure should remain silent. Any monk who doesn’t approve should speak up. 

The\marginnote{13.7.28} Sangha has done the legal procedure of banishing the monks Assaji and Punabbasuka, prohibiting them from staying at \textsanskrit{Kīṭāgiri}. The Sangha approves and is therefore silent. I’ll remember it thus.’” 

\subsection*{The group of twelve on illegitimate legal procedures }

“When\marginnote{14.1.1} a legal procedure of banishment has three qualities, it’s illegitimate, contrary to the Monastic Law, and improperly disposed of: it’s done in the absence of the accused, it’s done without questioning the accused, it’s done without the admission of the accused. 

When\marginnote{14.1.4} a procedure of banishment has another three qualities, it’s also illegitimate, contrary to the Monastic Law, and improperly disposed of: it’s done against one who hasn’t committed any offense, it’s done against one who’s committed an offense that isn’t clearable by confession, it’s done against one who’s confessed their offense. 

When\marginnote{14.1.6} a procedure of banishment has another three qualities, it’s also illegitimate, contrary to the Monastic Law, and improperly disposed of: it’s done without having accused the person of their offense, it’s done without having reminded the person of their offense, it’s done without having charged the person with their offense. 

“When\marginnote{14.1.7} a procedure of banishment has another three qualities, it’s also illegitimate, contrary to the Monastic Law, and improperly disposed of: it’s done in the absence of the accused, it’s done illegitimately, it’s done by an incomplete assembly. 

When\marginnote{14.1.8} a procedure of banishment has another three qualities, it’s also illegitimate, contrary to the Monastic Law, and improperly disposed of: it’s done without questioning the accused, it’s done illegitimately, it’s done by an incomplete assembly. 

When\marginnote{14.1.9} a procedure of banishment has another three qualities, it’s also illegitimate, contrary to the Monastic Law, and improperly disposed of: it’s done without the admission of the accused, it’s done illegitimately, it’s done by an incomplete assembly. 

“When\marginnote{14.1.10} a procedure of banishment has another three qualities, it’s also illegitimate, contrary to the Monastic Law, and improperly disposed of: it’s done against one who hasn’t committed any offense, it’s done illegitimately, it’s done by an incomplete assembly. 

When\marginnote{14.1.11} a procedure of banishment has another three qualities, it’s also illegitimate, contrary to the Monastic Law, and improperly disposed of: it’s done against one who’s committed an offense that isn’t clearable by confession, it’s done illegitimately, it’s done by an incomplete assembly. 

When\marginnote{14.1.12} a procedure of banishment has another three qualities, it’s also illegitimate, contrary to the Monastic Law, and improperly disposed of: it’s done against one who’s confessed their offense, it’s done illegitimately, it’s done by an incomplete assembly. 

“When\marginnote{14.1.13} a procedure of banishment has another three qualities, it’s also illegitimate, contrary to the Monastic Law, and improperly disposed of: it’s done without having accused the person of their offense, it’s done illegitimately, it’s done by an incomplete assembly. 

When\marginnote{14.1.14} a procedure of banishment has another three qualities, it’s also illegitimate, contrary to the Monastic Law, and improperly disposed of: it’s done without having reminded the person of their offense, it’s done illegitimately, it’s done by an incomplete assembly. 

When\marginnote{14.1.15} a procedure of banishment has another three qualities, it’s also illegitimate, contrary to the Monastic Law, and improperly disposed of: it’s done without having charged the person with their offense, it’s done illegitimately, it’s done by an incomplete assembly.” 

\scend{The group of twelve on illegitimate legal procedures is finished. }

\subsection*{The group of twelve on legitimate legal procedures }

“When\marginnote{14.1.18.1} a legal procedure of banishment has three qualities, it’s legitimate, in accordance with the Monastic Law, and properly disposed of: it’s done in the presence of the accused, it’s done with the questioning of the accused, it’s done with the admission of the accused. 

When\marginnote{14.1.21} a procedure of banishment has another three qualities, it’s also legitimate, in accordance with the Monastic Law, and properly disposed of: it’s done against one who’s committed an offense, it’s done against one who’s committed an offense that’s clearable by confession, it’s done against one who hasn’t confessed their offense. 

When\marginnote{14.1.23} a procedure of banishment has another three qualities, it’s also legitimate, in accordance with the Monastic Law, and properly disposed of: it’s done after having accused the person of their offense, it’s done after having reminded the person of their offense, it’s done after having charged the person with their offense. 

“When\marginnote{14.1.24} a procedure of banishment has another three qualities, it’s also legitimate, in accordance with the Monastic Law, and properly disposed of: it’s done in the presence of the accused, it’s done legitimately, it’s done by a unanimous assembly. 

When\marginnote{14.1.25} a procedure of banishment has another three qualities, it’s also legitimate, in accordance with the Monastic Law, and properly disposed of: it’s done with the questioning of the accused, it’s done legitimately, it’s done by a unanimous assembly. 

When\marginnote{14.1.26} a procedure of banishment has another three qualities, it’s also legitimate, in accordance with the Monastic Law, and properly disposed of: it’s done with the admission of the accused, it’s done legitimately, it’s done by a unanimous assembly. 

“When\marginnote{14.1.27} a procedure of banishment has another three qualities, it’s also legitimate, in accordance with the Monastic Law, and properly disposed of: it’s done against one who’s committed an offense, it’s done legitimately, it’s done by a unanimous assembly. 

When\marginnote{14.1.28} a procedure of banishment has another three qualities, it’s also legitimate, in accordance with the Monastic Law, and properly disposed of: it’s done against one who’s committed an offense that’s clearable by confession, it’s done legitimately, it’s done by a unanimous assembly. 

When\marginnote{14.1.29} a procedure of banishment has another three qualities, it’s also legitimate, in accordance with the Monastic Law, and properly disposed of: it’s done against one who hasn’t confessed their offense, it’s done legitimately, it’s done by a unanimous assembly. 

“When\marginnote{14.1.30} a procedure of banishment has another three qualities, it’s also legitimate, in accordance with the Monastic Law, and properly disposed of: it’s done after accusing the person of their offense, it’s done legitimately, it’s done by a unanimous assembly. 

When\marginnote{14.1.31} a procedure of banishment has another three qualities, it’s also legitimate, in accordance with the Monastic Law, and properly disposed of: it’s done after reminding the person of their offense, it’s done legitimately, it’s done by a unanimous assembly. 

When\marginnote{14.1.32} a procedure of banishment has another three qualities, it’s also legitimate, in accordance with the Monastic Law, and properly disposed of: it’s done after charging the person with their offense, it’s done legitimately, it’s done by a unanimous assembly.” 

\scend{The group of twelve on legitimate legal procedures is finished. }

\subsection*{The group of fourteen on wishing }

“When\marginnote{14.1.35.1} a monk has three qualities, the Sangha may, if it wishes, do a legal procedure of banishing him: he’s quarrelsome, argumentative, and a creator of legal issues in the Sangha; he’s ignorant and incompetent, often committing offenses, and lacking in boundaries; he’s constantly and improperly socializing with householders. 

When\marginnote{14.1.40} a monk has another three qualities, the Sangha may, if it wishes, do a procedure of banishing him: he has failed in the higher morality; he has failed in conduct; he has failed in view. 

When\marginnote{14.1.43} a monk has another three qualities, the Sangha may, if it wishes, do a procedure of banishing him: he disparages the Buddha; he disparages the Teaching; he disparages the Sangha. 

When\marginnote{14.1.46} a monk has another three qualities, the Sangha may, if it wishes, do a procedure of banishing him: his bodily conduct is frivolous; his verbal conduct is frivolous; his bodily and verbal conduct are frivolous. 

When\marginnote{14.1.49} a monk has another three qualities, the Sangha may, if it wishes, do a procedure of banishing him: he’s improperly behaved by body; he’s improperly behaved by speech; he’s improperly behaved by body and speech. 

When\marginnote{14.1.52} a monk has another three qualities, the Sangha may, if it wishes, do a procedure of banishing him: his bodily conduct is harmful; his verbal conduct is harmful; his bodily and verbal conduct are harmful. 

When\marginnote{14.1.55} a monk has another three qualities, the Sangha may, if it wishes, do a procedure of banishing him: he has wrong livelihood by body; he has wrong livelihood by speech; he has wrong livelihood by body and speech. 

“The\marginnote{14.2.1} Sangha may, if it wishes, do a procedure of banishment against three kinds of monks: those who are quarrelsome, argumentative, and creators of legal issues in the Sangha; those who are ignorant and incompetent, often committing offenses, and lacking in boundaries; those who are constantly and improperly socializing with householders. 

The\marginnote{14.2.6} Sangha may, if it wishes, do a procedure of banishment against three other kinds of monks: those who’ve failed in the higher morality; those who’ve failed in conduct; those who’ve failed in view. 

The\marginnote{14.2.9} Sangha may, if it wishes, do a procedure of banishment against three other kinds of monks: those who disparage the Buddha; those who disparage the Teaching; those who disparage the Sangha. 

The\marginnote{14.2.12} Sangha may, if it wishes, do a procedure of banishment against three other kinds of monks: those who are frivolous in bodily conduct; those who are frivolous in verbal conduct; those who are frivolous in bodily and verbal conduct. 

The\marginnote{14.2.15} Sangha may, if it wishes, do a procedure of banishment against three other kinds of monks: those who are improperly behaved by body; those who are improperly behaved by speech; those who are improperly behaved by body and speech. 

The\marginnote{14.2.18} Sangha may, if it wishes, do a procedure of banishment against three other kinds of monks: those whose bodily conduct is harmful; those whose verbal conduct is harmful; those whose bodily and verbal conduct are harmful. 

The\marginnote{14.2.21} Sangha may, if it wishes, do a procedure of banishment against three other kinds of monks: those who have wrong livelihood by body; those who have wrong livelihood by speech; those who have wrong livelihood by body and speech.” 

\scend{The group of fourteen on wishing is finished. }

\subsection*{The eighteen kinds of conduct }

“A\marginnote{15.1.1} monk who’s had a legal procedure of banishment done against himself should conduct himself properly. This is the proper conduct: 

\begin{enumerate}%
\item He shouldn’t give the full ordination. %
\item He shouldn’t give formal support. %
\item He shouldn’t have a novice monk attend on him. %
\item He shouldn’t accept being appointed as an instructor of the nuns. %
\item Even if appointed, he shouldn’t instruct the nuns. %
\item He shouldn’t commit the same offense as the offense for which the Sangha did the procedure of banishing him. %
\item He shouldn’t commit an offense similar to the offense for which the Sangha did the procedure of banishing him. %
\item He shouldn’t commit an offense worse than the offense for which the Sangha did the procedure of banishing him. %
\item He shouldn’t criticize the procedure. %
\item He shouldn’t criticize those who did the procedure. %
\item He shouldn’t cancel the observance-day ceremony of a regular monk. %
\item He shouldn’t cancel the invitation ceremony of a regular monk. %
\item He shouldn’t direct a regular monk. %
\item He shouldn’t give instructions to a regular monk. %
\item He shouldn’t get permission from a regular monk to correct him. %
\item He shouldn’t accuse a regular monk of an offense. %
\item He shouldn’t remind a regular monk of an offense. %
\item He shouldn’t associate inappropriately with other monks.” %
\end{enumerate}

\scend{The eighteen kinds of conduct in regard to the legal procedure of banishment are finished. }

Soon\marginnote{16.1.1} afterwards a sangha of monks, headed by \textsanskrit{Sāriputta} and \textsanskrit{Moggallāna}, went to \textsanskrit{Kīṭāgiri} and did the legal procedure of banishing the monks Assaji and Punabbasuka, prohibiting them from staying at \textsanskrit{Kīṭāgiri}. Even so, they did not conduct themselves properly or suitably so as to deserve to be released, nor did they ask the monks for forgiveness. Instead they abused and reviled them, and they slandered them as acting from favoritism, ill will, confusion, and fear. And they left and they disrobed.\footnote{For an explanation of the rendering “disrobe” for \textit{vibbhamati}, see Appendix of Technical Terms. } The monks of few desires complained and criticized them, “How can these monks act like this when the Sangha has done a legal procedure of banishing them?” 

They\marginnote{16.1.12} told the Buddha. Soon afterwards the Buddha had the Sangha gathered and questioned the monks: “Is it true, monks, that the monks Assaji and Punabbasuka are acting like this?” 

“It’s\marginnote{16.1.18} true, sir.” 

The\marginnote{16.1.19} Buddha rebuked them … “It’s not suitable … How can those foolish men act like this? This will affect people’s confidence …” After rebuking them … he gave a teaching and addressed the monks: 

“Well\marginnote{16.1.28} then, don’t lift that legal procedure of banishment.”\footnote{MS reads, \textit{tena hi, bhikkhave, \textsanskrit{saṅgho} \textsanskrit{pabbājanīyakammaṁ} \textsanskrit{paṭippassambhetu}}, “Well then, monks, lift that procedure of banishment”, which must be an error. I here follow the text as found in the PTS edition, \textit{tena hi, bhikkhave, \textsanskrit{saṅgho} \textsanskrit{pabbājanīyakammaṁ} na \textsanskrit{paṭippassambhetu}}. } 

\subsection*{The group of eighteen on not to be lifted }

“When\marginnote{16.2.1} a monk has five qualities, a legal procedure of banishing him shouldn’t be lifted: he gives the full ordination; he gives formal support; he has a novice monk attend on him; he accepts being appointed as an instructor of the nuns; he instructs the nuns, whether appointed or not. 

When\marginnote{16.2.4} a monk has another five qualities, a procedure of banishing him shouldn’t be lifted: he commits the same offense for which the Sangha did the procedure of banishing him; he commits an offense similar to the one for which the Sangha did the procedure of banishing him; he commits an offense worse than the one for which the Sangha did the procedure of banishing him; he criticizes the procedure; he criticizes those who did the procedure. 

When\marginnote{16.2.8} a monk has eight qualities, a procedure of banishing him shouldn’t be lifted: he cancels the observance-day ceremony of a regular monk; he cancels the invitation ceremony of a regular monk; he directs a regular monk; he gives instructions to a regular monk; he gets permission from a regular monk to correct him; he accuses a regular monk of an offense; he reminds a regular monk of an offense; he associates inappropriately with other monks.” 

\scend{The group of eighteen on not to be lifted is finished. }

\subsection*{The group of eighteen on to be lifted }

“When\marginnote{16.2.12.1} a monk has five qualities, a legal procedure of banishing him should be lifted: he doesn’t give the full ordination; he doesn’t give formal support; he doesn’t have a novice monk attend on him; he doesn’t accept being appointed as an instructor of the nuns; he doesn’t instruct the nuns, whether appointed or not. 

When\marginnote{16.2.15} a monk has another five qualities, a procedure of banishing him should be lifted: he doesn’t commit the same offense for which the Sangha did the procedure of banishing him; he doesn’t commit an offense similar to the one for which the Sangha did the procedure of banishing him; he doesn’t commit an offense worse than the one for which the Sangha did the procedure of banishing him; he doesn’t criticize the procedure; he doesn’t criticize those who did the procedure. 

When\marginnote{16.2.19} a monk has eight qualities, a procedure of banishing him should be lifted: he doesn’t cancel the observance-day ceremony of a regular monk; he doesn’t cancel the invitation ceremony of a regular monk; he doesn’t direct a regular monk; he doesn’t give instructions to a regular monk; he doesn’t get permission from a regular monk to correct him; he doesn’t accuse a regular monk of an offense; he doesn’t remind a regular monk of an offense; he doesn’t associate inappropriately with other monks.” 

\scend{The group of eighteen on to be lifted is finished. }

“And\marginnote{17.1.1} this is how it should be lifted. After approaching the Sangha, the monk who’s had a legal procedure of banishment done against himself should arrange his upper robe over one shoulder, pay respect at the feet of the senior monks, squat on his heels, raise his joined palms, and say, ‘Venerables, the Sangha has done a legal procedure of banishing me. I’m now conducting myself properly and suitably, and deserve to be released. I ask for that legal procedure to be lifted.’ And he should ask a second and a third time. A competent and capable monk should then inform the Sangha: 

‘Please,\marginnote{17.2.1} venerables, I ask the Sangha to listen. The Sangha has done a legal procedure of banishing monk so-and-so. He’s now conducting himself properly and suitably, and deserves to be released. He’s asking for that legal procedure to be lifted. If the Sangha is ready, it should lift that legal procedure of banishing him. This is the motion. 

Please,\marginnote{17.2.5} venerables, I ask the Sangha to listen. The Sangha has done a legal procedure of banishing monk so-and-so. He’s now conducting himself properly and suitably, and deserves to be released. He’s asking for that legal procedure to be lifted. The Sangha lifts that legal procedure of banishing him. Any monk who approves of lifting that legal procedure should remain silent. Any monk who doesn’t approve should speak up. 

For\marginnote{17.2.10} the second time, I speak on this matter. … For the third time, I speak on this matter. Please, venerables, I ask the Sangha to listen. The Sangha has done a legal procedure of banishing monk so-and-so. He’s now conducting himself properly and suitably, and deserves to be released. He’s asking for that legal procedure to be lifted. The Sangha lifts that legal procedure of banishing him. Any monk who approves of lifting that legal procedure should remain silent. Any monk who doesn’t approve should speak up. 

The\marginnote{17.2.17} Sangha has lifted that legal procedure of banishing monk so-and-so. The Sangha approves and is therefore silent. I’ll remember it thus.’” 

\scend{The second section on the legal procedure of banishment is finished. }

\section*{4. The legal procedure of reconciliation }

At\marginnote{18.1.1} one time Venerable Sudhamma was the staying at the householder Citta’s monastery at \textsanskrit{Macchikāsaṇḍa}. He was in charge of the building work and received a regular supply of food. Whenever Citta wanted to invite the Sangha, a group of monks, or an individual monk, he would not do so without getting permission from Sudhamma. 

On\marginnote{18.1.3} one occasion a number of senior monks—Venerable \textsanskrit{Sāriputta}, Venerable \textsanskrit{Mahāmoggallāna}, Venerable \textsanskrit{Mahākaccāna}, Venerable \textsanskrit{Mahākoṭṭhika}, Venerable \textsanskrit{Mahākappina}, Venerable \textsanskrit{Mahācunda}, Venerable Anuruddha, Venerable Revata, Venerable \textsanskrit{Upāli}, Venerable Ānanda, Venerable \textsanskrit{Rāhula}—were wandering in \textsanskrit{Kāsi}, when they arrived at \textsanskrit{Macchikāsaṇḍa}. 

When\marginnote{18.1.4} Citta heard that they had arrived, he went to them, bowed, and sat down. After \textsanskrit{Sāriputta} had instructed, inspired, and gladdened Citta with a teaching, Citta said, “Venerables, please accept a meal for newly-arrived monks from me tomorrow.” They accepted by remaining silent. 

When\marginnote{18.2.1} he knew that they had accepted, Citta got up from his seat, bowed down, circumambulated them with his right side toward them, and went to Sudhamma. He bowed to him and said, “Venerable, please accept tomorrow’s meal from me together with the senior monks.” 

Sudhamma\marginnote{18.2.4} thought, “Previously when Citta wanted to invite the Sangha, a group of monks, or an individual monk, he wouldn’t do so without getting my permission. But now he has. He’s been corrupted, this Citta. He’s uninterested and has no affection for me.” He said to Citta, “There’s no need. I won’t accept.” Citta asked him a second and a third time, but got the same reply. He thought, “What difference does it make to me whether Sudhamma accepts or not?” He then bowed, circumambulated Sudhamma with his right side toward him, and left. 

The\marginnote{18.3.1} following morning Citta prepared various kinds of fine foods for the senior monks. Sudhamma thought, “Why don’t I go and see what Citta has prepared for the senior monks?” He then robed up, took his bowl and robe, and went to Citta’s house where he sat down on the prepared seat. Citta approached Sudhamma, bowed, and sat down. And Sudhamma said to him, “You have prepared many kinds of food. But there’s one that’s missing: sesame cookies.”\footnote{According to the commentary this is an insult. Citta had a relative who was a baker, a low class occupation. By asking for sesame cookies, Sudhamma was apparently trying to remind Citta of his low class relative, thereby insulting him. } 

“When\marginnote{18.3.8} there are so many jewels in the word of the Buddha, sir, you speak of sesame cookies. In the past there were some traders from the south who went to an eastern country to trade. From there they brought back a hen. That hen mated with a crow and because of that she had a chick. When that chick wanted to caw like a crow, it cried, ‘caw-ca-doodle-doo.’ And when that chick wanted to crow like a rooster, it cried, ‘cock-a-doodle-caw.’ In the same way, when there are so many jewels in the word of the Buddha, you speak of sesame cookies.” 

“Householder,\marginnote{18.4.1} you’re abusing and insulting me. I will leave your monastery.” 

“I’m\marginnote{18.4.3} not abusing and insulting you. Please stay at \textsanskrit{Macchikāsaṇḍa}. The mango grove is delightful. I’ll do my best to provide you with robe-cloth, almsfood, a dwelling, and medicinal supplies.” 

Sudhamma\marginnote{18.4.7} repeated what he had said a second time and Citta responded as before. When Sudhamma repeated it a third time, Citta said, “Where will you go?” 

“I’ll\marginnote{18.4.12} go to \textsanskrit{Sāvatthī} to visit the Buddha.” 

“Well\marginnote{18.4.13} then, please tell the Buddha of our entire conversation. And I would not be surprised if you returned to \textsanskrit{Macchikāsaṇḍa}.” 

Sudhamma\marginnote{18.5.1} put his dwelling in order, took his bowl and robe, and left for \textsanskrit{Sāvatthī}. When he eventually arrived, he went to the Buddha in \textsanskrit{Anāthapiṇḍika}’s Monastery. He bowed to the Buddha, sat down, and told him about the conversation he had had with Citta. 

The\marginnote{18.5.4} Buddha rebuked him, “It’s not suitable, foolish man, it’s not proper, it’s not worthy of a monastic, it’s not allowable, it’s not to be done. How can you demean and insult Citta, who has faith and confidence, who’s a donor, benefactor, and supporter of the Sangha? This will affect people’s confidence …” After rebuking him … he gave a teaching and addressed the monks: 

“Well\marginnote{18.5.10} then, do a legal procedure of reconciliation against the monk Sudhamma, instructing him to ask Citta for forgiveness. And it should be done like this. First you should accuse the monk Sudhamma. He should then be reminded of what he has done, before he’s charged with an offense. A competent and capable monk should then inform the Sangha: 

‘Please,\marginnote{18.6.3} venerables, I ask the Sangha to listen. This monk Sudhamma has demeaned and insulted the householder Citta, who has faith and confidence, who’s a donor, benefactor, and supporter of the Sangha. If the Sangha is ready, the Sangha should do a legal procedure of reconciliation against the monk Sudhamma, instructing him to ask Citta for forgiveness. This is the motion. 

Please,\marginnote{18.6.8} venerables, I ask the Sangha to listen. This monk Sudhamma has demeaned and insulted the householder Citta, who has faith and confidence, who’s a donor, benefactor, and supporter of the Sangha. The Sangha does a legal procedure of reconciliation against the monk Sudhamma, instructing him to ask Citta for forgiveness. Any monk who approves of doing this legal procedure should remain silent. Any monk who doesn’t approve should speak up. 

For\marginnote{18.6.14} the second time, I speak on this matter. … For the third time, I speak on this matter. Please, venerables, I ask the Sangha to listen. This monk Sudhamma has demeaned and insulted the householder Citta, who has faith and confidence, who’s a donor, benefactor, and supporter of the Sangha. The Sangha does a legal procedure of reconciliation against the monk Sudhamma, instructing him to ask Citta for forgiveness. Any monk who approves of doing this legal procedure should remain silent. Any monk who doesn’t approve should speak up. 

The\marginnote{18.6.22} Sangha has done the legal procedure of reconciliation against the monk Sudhamma, instructing him to ask Citta for forgiveness. The Sangha approves and is therefore silent. I’ll remember it thus.’” 

\subsection*{The group of twelve on illegitimate legal procedures }

“When\marginnote{19.1.1} a legal procedure of reconciliation has three qualities, it’s illegitimate, contrary to the Monastic Law, and improperly disposed of: it’s done in the absence of the accused, it’s done without questioning the accused, it’s done without the admission of the accused. 

When\marginnote{19.1.4} a procedure of reconciliation has another three qualities, it’s also illegitimate, contrary to the Monastic Law, and improperly disposed of: it’s done against one who hasn’t committed any offense, it’s done against one who’s committed an offense that isn’t clearable by confession, it’s done against one who’s confessed their offense. 

When\marginnote{19.1.6} a procedure of reconciliation has another three qualities, it’s also illegitimate, contrary to the Monastic Law, and improperly disposed of: it’s done without having accused the person of their offense, it’s done without having reminded the person of their offense, it’s done without having charged the person with their offense. 

“When\marginnote{19.1.7} a procedure of reconciliation has another three qualities, it’s also illegitimate, contrary to the Monastic Law, and improperly disposed of: it’s done in the absence of the accused, it’s done illegitimately, it’s done by an incomplete assembly. 

When\marginnote{19.1.8} a procedure of reconciliation has another three qualities, it’s also illegitimate, contrary to the Monastic Law, and improperly disposed of: it’s done without questioning the accused, it’s done illegitimately, it’s done by an incomplete assembly. 

When\marginnote{19.1.9} a procedure of reconciliation has another three qualities, it’s also illegitimate, contrary to the Monastic Law, and improperly disposed of: it’s done without the admission of the accused, it’s done illegitimately, it’s done by an incomplete assembly. 

“When\marginnote{19.1.10} a procedure of reconciliation has another three qualities, it’s also illegitimate, contrary to the Monastic Law, and improperly disposed of: it’s done against one who hasn’t committed any offense, it’s done illegitimately, it’s done by an incomplete assembly. 

When\marginnote{19.1.11} a procedure of reconciliation has another three qualities, it’s also illegitimate, contrary to the Monastic Law, and improperly disposed of: it’s done against one who’s committed an offense that isn’t clearable by confession, it’s done illegitimately, it’s done by an incomplete assembly. 

When\marginnote{19.1.12} a procedure of reconciliation has another three qualities, it’s also illegitimate, contrary to the Monastic Law, and improperly disposed of: it’s done against one who’s confessed their offense, it’s done illegitimately, it’s done by an incomplete assembly. 

“When\marginnote{19.1.13} a procedure of reconciliation has another three qualities, it’s also illegitimate, contrary to the Monastic Law, and improperly disposed of: it’s done without having accused the person of their offense, it’s done illegitimately, it’s done by an incomplete assembly. 

When\marginnote{19.1.14} a procedure of reconciliation has another three qualities, it’s also illegitimate, contrary to the Monastic Law, and improperly disposed of: it’s done without having reminded the person of their offense, it’s done illegitimately, it’s done by an incomplete assembly. 

When\marginnote{19.1.15} a procedure of reconciliation has another three qualities, it’s also illegitimate, contrary to the Monastic Law, and improperly disposed of: it’s done without having charged the person with their offense, it’s done illegitimately, it’s done by an incomplete assembly.” 

\scend{The group of twelve on illegitimate legal procedures of reconciliation is finished. }

\subsection*{The group of twelve on legitimate legal procedures }

“When\marginnote{19.1.18.1} a legal procedure of reconciliation has three qualities, it’s legitimate, in accordance with the Monastic Law, and properly disposed of: it’s done in the presence of the accused, it’s done with the questioning of the accused, it’s done with the admission of the accused. 

When\marginnote{19.1.21} a procedure of reconciliation has another three qualities, it’s also legitimate, in accordance with the Monastic Law, and properly disposed of: it’s done against one who’s committed an offense, it’s done against one who’s committed an offense that’s clearable by confession, it’s done against one who hasn’t confessed their offense. 

When\marginnote{19.1.23} a procedure of reconciliation has another three qualities, it’s also legitimate, in accordance with the Monastic Law, and properly disposed of: it’s done after having accused the person of their offense, it’s done after having reminded the person of their offense, it’s done after having charged the person with their offense. 

“When\marginnote{19.1.24} a procedure of reconciliation has another three qualities, it’s also legitimate, in accordance with the Monastic Law, and properly disposed of: it’s done in the presence of the accused, it’s done legitimately, it’s done by a unanimous assembly. 

When\marginnote{19.1.25} a procedure of reconciliation has another three qualities, it’s also legitimate, in accordance with the Monastic Law, and properly disposed of: it’s done with the questioning of the accused, it’s done legitimately, it’s done by a unanimous assembly. 

When\marginnote{19.1.26} a procedure of reconciliation has another three qualities, it’s also legitimate, in accordance with the Monastic Law, and properly disposed of: it’s done with the admission of the accused, it’s done legitimately, it’s done by a unanimous assembly. 

“When\marginnote{19.1.27} a procedure of reconciliation has another three qualities, it’s also legitimate, in accordance with the Monastic Law, and properly disposed of: it’s done against one who’s committed an offense, it’s done legitimately, it’s done by a unanimous assembly. 

When\marginnote{19.1.28} a procedure of reconciliation has another three qualities, it’s also legitimate, in accordance with the Monastic Law, and properly disposed of: it’s done against one who’s committed an offense that’s clearable by confession, it’s done legitimately, it’s done by a unanimous assembly. 

When\marginnote{19.1.29} a procedure of reconciliation has another three qualities, it’s also legitimate, in accordance with the Monastic Law, and properly disposed of: it’s done against one who hasn’t confessed their offense, it’s done legitimately, it’s done by a unanimous assembly. 

“When\marginnote{19.1.30} a procedure of reconciliation has another three qualities, it’s also legitimate, in accordance with the Monastic Law, and properly disposed of: it’s done after accusing the person of their offense, it’s done legitimately, it’s done by a unanimous assembly. 

When\marginnote{19.1.31} a procedure of reconciliation has another three qualities, it’s also legitimate, in accordance with the Monastic Law, and properly disposed of: it’s done after reminding the person of their offense, it’s done legitimately, it’s done by a unanimous assembly. 

When\marginnote{19.1.32} a procedure of reconciliation has another three qualities, it’s also legitimate, in accordance with the Monastic Law, and properly disposed of: it’s done after charging the person with their offense, it’s done legitimately, it’s done by a unanimous assembly.” 

\scend{The group of twelve on legitimate legal procedures of reconciliation is finished. }

\subsection*{The group of four on wishing }

“When\marginnote{20.1.1} a monk has five qualities, the Sangha may, if it wishes, do a legal procedure of reconciliation against him: he’s trying to stop householders from getting things; he’s trying to harm householders; he’s trying to get householders to lose their place of residence; he abuses and reviles householders; he causes division between householders. 

When\marginnote{20.1.4} a monk has another five qualities, the Sangha may, if it wishes, do a procedure of reconciliation against him: he disparages the Buddha to householders; he disparages the Teaching to householders; he disparages the Sangha to householders; he demeans and insults householders; he doesn’t fulfill legitimate promises to householders. 

The\marginnote{20.1.7} Sangha may, if it wishes, do a procedure of reconciliation against five kinds of monks: those who are trying to stop householders from getting things; those who are trying to harm householders; those who are trying to get householders to lose their place of residence; those who abuse and revile householders; those who cause division between householders. 

The\marginnote{20.1.10} Sangha may, if it wishes, do a procedure of reconciliation against another five kinds of monks: those who disparage the Buddha to householders; those who disparage the Teaching to householders; those who disparage the Sangha to householders; those who demean and insult householders; those who don’t fulfill legitimate promises to householders.” 

\scend{The group of four on wishing is finished. }

\subsection*{The eighteen kinds of conduct }

“A\marginnote{21.1.1} monk who’s had a legal procedure of reconciliation done against himself should conduct himself properly. This is the proper conduct: 

\begin{enumerate}%
\item He shouldn’t give the full ordination. %
\item He shouldn’t give formal support. %
\item He shouldn’t have a novice monk attend on him. %
\item He shouldn’t accept being appointed as an instructor of the nuns. %
\item Even if appointed, he shouldn’t instruct the nuns. %
\item He shouldn’t commit the same offense as the offense for which the Sangha did the procedure of reconciliation against him. %
\item He shouldn’t commit an offense similar to the offense for which the Sangha did the procedure of reconciliation against him. %
\item He shouldn’t commit an offense worse than the offense for which the Sangha did the procedure of reconciliation against him. %
\item He shouldn’t criticize the procedure. %
\item He shouldn’t criticize those who did the procedure. %
\item He shouldn’t cancel the observance-day ceremony of a regular monk. %
\item He shouldn’t cancel the invitation ceremony of a regular monk. %
\item He shouldn’t direct a regular monk. %
\item He shouldn’t give instructions to a regular monk. %
\item He shouldn’t get permission from a regular monk to correct him. %
\item He shouldn’t accuse a regular monk of an offense. %
\item He shouldn’t remind a regular monk of an offense. %
\item He shouldn’t associate inappropriately with other monks.” %
\end{enumerate}

\scend{The eighteen kinds of conduct in regard to the legal procedure of reconciliation are finished. }

Soon\marginnote{22.1.1} afterwards the Sangha did a legal procedure of reconciliation against the monk Sudhamma, instructing him to ask Citta for forgiveness. He went to \textsanskrit{Macchikāsaṇḍa}, but feeling humiliated, he was unable to ask Citta for forgiveness. He returned to \textsanskrit{Sāvatthī}. The monks asked him if he had asked Citta for forgiveness, and he told them what had happened. The monks told the Buddha. He had the monks gathered and said: 

“Well\marginnote{22.2.1} then, the Sangha should give a companion messenger to Sudhamma to ask Citta for forgiveness. And this is how the messenger should be given. First you should ask a monk, and then a competent and capable monk should inform the Sangha: 

‘Please,\marginnote{22.2.5} venerables, I ask the Sangha to listen. If the Sangha is ready, it should give monk so-and-so to Sudhamma as a companion messenger to ask Citta for forgiveness. This is the motion. 

Please,\marginnote{22.2.8} venerables, I ask the Sangha to listen. The Sangha gives monk so-and-so to Sudhamma as a companion messenger to ask Citta for forgiveness. Any monk who approves of this should remain silent. Any monk who doesn’t approve should speak up. 

The\marginnote{22.2.12} Sangha has given monk so-and-so to Sudhamma as a companion messenger to ask Citta for forgiveness. The Sangha approves and is therefore silent. I’ll remember it thus.’ 

Sudhamma\marginnote{22.3.1} should now go to \textsanskrit{Macchikāsaṇḍa} with that monk as a companion messenger to ask Citta for forgiveness, saying, ‘Please forgive me, householder; I wish to reconcile with you.’ If he forgives, all is well. If not, the companion messenger should say, ‘Please forgive this monk, householder; he wishes to reconcile with you.’ If he forgives, all is well. If not, the companion messenger should say, ‘Please forgive this monk, householder; I wish to reconcile with you.’ If he forgives, all is well. If not, the companion messenger should say, ‘Please forgive this monk, householder; I ask in the name of the Sangha.’ If he forgives, all is well. If not, then within sight and hearing of Citta, the monk Sudhamma should arrange his upper robe over one shoulder, squat on his heels, raise his joined palms, and confess that offense.” 

Soon\marginnote{23.1.1} afterwards Sudhamma went to \textsanskrit{Macchikāsaṇḍa} with a monk as a companion messenger, and he asked Citta for forgiveness. And he conducted himself properly and suitably, and deserved to be released. He then went to the monks and told them about this, adding, “What should I do now?” The monks told the Buddha. He had the monks gathered and said, 

“Well\marginnote{23.1.6} then, lift that legal procedure of reconciliation against Sudhamma.” 

\subsection*{The group of eighteen on not to be lifted }

“When\marginnote{23.2.1} a monk has five qualities, a legal procedure of reconciliation against him shouldn’t be lifted: he gives the full ordination; he gives formal support; he has a novice monk attend on him; he accepts being appointed as an instructor of the nuns; he instructs the nuns, whether appointed or not. 

When\marginnote{23.2.4} a monk has another five qualities, a procedure of reconciliation against him shouldn’t be lifted: he commits the same offense for which the Sangha did the procedure of reconciliation against him; he commits an offense similar to the one for which the Sangha did the procedure of reconciliation against him; he commits an offense worse than the one for which the Sangha did the procedure of reconciliation against him; he criticizes the procedure; he criticizes those who did the procedure. 

When\marginnote{23.2.8} a monk has eight qualities, a procedure of reconciliation against him shouldn’t be lifted: he cancels the observance-day ceremony of a regular monk; he cancels the invitation ceremony of a regular monk; he directs a regular monk; he gives instructions to a regular monk; he gets permission from a regular monk to correct him; he accuses a regular monk of an offense; he reminds a regular monk of an offense; he associates inappropriately with other monks.” 

\scend{The group of eighteen on not to be lifted is finished. }

\subsection*{The group of eighteen on to be lifted }

“When\marginnote{23.2.12.1} a monk has five qualities, a legal procedure of reconciliation against him should be lifted: he doesn’t give the full ordination; he doesn’t give formal support; he doesn’t have a novice monk attend on him; he doesn’t accept being appointed as an instructor of the nuns; he doesn’t instruct the nuns, whether appointed or not. 

When\marginnote{23.2.15} a monk has another five qualities, a procedure of reconciliation against him should be lifted: he doesn’t commit the same offense for which the Sangha did the procedure of reconciliation against him; he doesn’t commit an offense similar to the one for which the Sangha did the procedure of reconciliation against him; he doesn’t commit an offense worse than the one for which the Sangha did the procedure of reconciliation against him; he doesn’t criticize the procedure; he doesn’t criticize those who did the procedure. 

When\marginnote{23.2.19} a monk has eight qualities, a procedure of reconciliation against him should be lifted: he doesn’t cancel the observance-day ceremony of a regular monk; he doesn’t cancel the invitation ceremony of a regular monk; he doesn’t direct a regular monk; he doesn’t give instructions to a regular monk; he doesn’t get permission from a regular monk to correct him; he doesn’t accuse a regular monk of an offense; he doesn’t remind a regular monk of an offense; he doesn’t associate inappropriately with other monks.” 

\scend{The group of eighteen on to be lifted is finished. }

“And\marginnote{24.1.1} this is how it should be lifted. The monk Sudhamma should approach the Sangha, arrange his upper robe over one shoulder, pay respect at the feet of the senior monks, squat on his heels, raise his joined palms, and say, ‘Venerables, the Sangha has done a legal procedure of reconciliation against me. I’m now conducting myself properly and suitably, and deserve to be released. I ask for that legal procedure to be lifted.’ And he should ask a second and a third time. A competent and capable monk should then inform the Sangha: 

‘Please,\marginnote{24.1.7} venerables, I ask the Sangha to listen. The Sangha has done a legal procedure of reconciliation against the monk Sudhamma. He’s now conducting himself properly and suitably, and deserves to be released. He’s asking for that legal procedure to be lifted. If the Sangha is ready, it should lift that legal procedure of reconciliation against him. This is the motion. 

Please,\marginnote{24.1.11} venerables, I ask the Sangha to listen. The Sangha has done a legal procedure of reconciliation against the monk Sudhamma. He’s now conducting himself properly and suitably, and deserves to be released. He’s asking for that legal procedure to be lifted. The Sangha lifts that legal procedure of reconciliation against him. Any monk who approves of lifting that legal procedure should remain silent. Any monk who doesn’t approve should speak up. 

For\marginnote{24.1.16} the second time, I speak on this matter. … For the third time, I speak on this matter. Please, venerables, I ask the Sangha to listen. The Sangha has done a legal procedure of reconciliation against the monk Sudhamma. He’s now conducting himself properly and suitably, and deserves to be released. He’s asking for that legal procedure to be lifted. The Sangha lifts that legal procedure of reconciliation against him. Any monk who approves of lifting that legal procedure should remain silent. Any monk who doesn’t approve should speak up. 

The\marginnote{24.1.23} Sangha has lifted that legal procedure of reconciliation against the monk Sudhamma. The Sangha approves and is therefore silent. I’ll remember it thus.’” 

\scend{The fourth section on the legal procedure of reconciliation is finished. }

\section*{5. The legal procedure of ejection for not recognizing an offense }

At\marginnote{25.1.1} one time the Buddha was staying at \textsanskrit{Kosambī} in Ghosita’s Monastery. At that time Venerable Channa had committed an offense, but refused to recognize it. The monks of few desires complained and criticized him, “How can Venerable Channa commit an offense, but then refuse to recognize it?” They told the Buddha. 

Soon\marginnote{25.1.6} afterwards the Buddha had the Sangha gathered and questioned the monks: “Is it true, monks, that Channa is acting like this?” 

“It’s\marginnote{25.1.8} true, sir.” 

The\marginnote{25.1.9} Buddha rebuked him, “It’s not suitable … How can Channa commit an offense, but then refuse to recognize it? This will affect people’s confidence …” After rebuking him … he gave a teaching and addressed the monks: 

“Well\marginnote{25.1.15} then, the Sangha should do a legal procedure of ejecting the monk Channa for not recognizing an offense, prohibiting him from living with the Sangha. And this is how it should be done. First you should accuse the monk Channa. He should then be reminded of what he has done, before he’s charged with an offense. A competent and capable monk should then inform the Sangha: 

‘Please,\marginnote{25.2.3} venerables, I ask the Sangha to listen. This monk Channa has committed an offense, but refuses to recognize it. If the Sangha is ready, it should do a legal procedure of ejecting Channa for not recognizing an offense, prohibiting him from living with the Sangha. This is the motion. 

Please,\marginnote{25.2.8} venerables, I ask the Sangha to listen. This monk Channa has committed an offense, but refuses to recognize it. The Sangha does a legal procedure of ejecting Channa for not recognizing an offense, prohibiting him from living with the Sangha. Any monk who approves of doing this legal procedure should remain silent. Any monk who doesn’t approve should speak up. 

For\marginnote{25.2.15} the second time, I speak on this matter. … For the third time, I speak on this matter. Please, venerables, I ask the Sangha to listen. This monk Channa has committed an offense, but refuses to recognize it. The Sangha does a legal procedure of ejecting Channa for not recognizing an offense, prohibiting him from living with the Sangha. Any monk who approves of doing this legal procedure should remain silent. Any monk who doesn’t approve should speak up. 

The\marginnote{25.2.24} Sangha has done the legal procedure of ejecting Channa for not recognizing an offense, prohibiting him from living with the Sangha. The Sangha approves and is therefore silent. I’ll remember it thus.’ 

Monks,\marginnote{25.2.27} you should proclaim from monastery to monastery that the Sangha has done a legal procedure of ejecting Channa for not recognizing an offense, prohibiting him from living with the Sangha.” 

\subsection*{The group of twelve on illegitimate legal procedures }

“When\marginnote{26.1.1} a legal procedure of ejection for not recognizing an offense has three qualities, it’s illegitimate, contrary to the Monastic Law, and improperly disposed of: it’s done in the absence of the accused, it’s done without questioning the accused, it’s done without the admission of the accused. 

When\marginnote{26.1.4} a procedure of ejection for not recognizing an offense has another three qualities, it’s also illegitimate, contrary to the Monastic Law, and improperly disposed of: it’s done against one who hasn’t committed any offense, it’s done against one who’s committed an offense that isn’t clearable by confession, it’s done against one who’s confessed their offense. 

When\marginnote{26.1.6} a procedure of ejection for not recognizing an offense has another three qualities, it’s also illegitimate, contrary to the Monastic Law, and improperly disposed of: it’s done without having accused the person of their offense, it’s done without having reminded the person of their offense, it’s done without having charged the person with their offense. 

“When\marginnote{26.1.7} a procedure of ejection for not recognizing an offense has another three qualities, it’s also illegitimate, contrary to the Monastic Law, and improperly disposed of: it’s done in the absence of the accused, it’s done illegitimately, it’s done by an incomplete assembly. 

When\marginnote{26.1.8} a procedure of ejection for not recognizing an offense has another three qualities, it’s also illegitimate, contrary to the Monastic Law, and improperly disposed of: it’s done without questioning the accused, it’s done illegitimately, it’s done by an incomplete assembly. 

When\marginnote{26.1.9} a procedure of ejection for not recognizing an offense has another three qualities, it’s also illegitimate, contrary to the Monastic Law, and improperly disposed of: it’s done without the admission of the accused, it’s done illegitimately, it’s done by an incomplete assembly. 

“When\marginnote{26.1.10} a procedure of ejection for not recognizing an offense has another three qualities, it’s also illegitimate, contrary to the Monastic Law, and improperly disposed of: it’s done against one who hasn’t committed any offense, it’s done illegitimately, it’s done by an incomplete assembly. 

When\marginnote{26.1.11} a procedure of ejection for not recognizing an offense has another three qualities, it’s also illegitimate, contrary to the Monastic Law, and improperly disposed of: it’s done against one who’s committed an offense that isn’t clearable by confession, it’s done illegitimately, it’s done by an incomplete assembly. 

When\marginnote{26.1.12} a procedure of ejection for not recognizing an offense has another three qualities, it’s also illegitimate, contrary to the Monastic Law, and improperly disposed of: it’s done against one who’s confessed their offense, it’s done illegitimately, it’s done by an incomplete assembly. 

“When\marginnote{26.1.13} a procedure of ejection for not recognizing an offense has another three qualities, it’s also illegitimate, contrary to the Monastic Law, and improperly disposed of: it’s done without having accused the person of their offense, it’s done illegitimately, it’s done by an incomplete assembly. 

When\marginnote{26.1.14} a procedure of ejection for not recognizing an offense has another three qualities, it’s also illegitimate, contrary to the Monastic Law, and improperly disposed of: it’s done without having reminded the person of their offense, it’s done illegitimately, it’s done by an incomplete assembly. 

When\marginnote{26.1.15} a procedure of ejection for not recognizing an offense has another three qualities, it’s also illegitimate, contrary to the Monastic Law, and improperly disposed of: it’s done without having charged the person with their offense, it’s done illegitimately, it’s done by an incomplete assembly.” 

\scend{The group of twelve on illegitimate legal procedures of ejection for not recognizing an offense is finished. }

\subsection*{The group of twelve on legitimate legal procedures }

“When\marginnote{26.1.18.1} a legal procedure of ejection for not recognizing an offense has three qualities, it’s legitimate, in accordance with the Monastic Law, and properly disposed of: it’s done in the presence of the accused, it’s done with the questioning of the accused, it’s done with the admission of the accused. 

When\marginnote{26.1.21} a procedure of ejection for not recognizing an offense has another three qualities, it’s also legitimate, in accordance with the Monastic Law, and properly disposed of: it’s done against one who’s committed an offense, it’s done against one who’s committed an offense that’s clearable by confession, it’s done against one who hasn’t confessed their offense. 

When\marginnote{26.1.23} a procedure of ejection for not recognizing an offense has another three qualities, it’s also legitimate, in accordance with the Monastic Law, and properly disposed of: it’s done after having accused the person of their offense, it’s done after having reminded the person of their offense, it’s done after having charged the person with their offense. 

“When\marginnote{26.1.24} a procedure of ejection for not recognizing an offense has another three qualities, it’s also legitimate, in accordance with the Monastic Law, and properly disposed of: it’s done in the presence of the accused, it’s done legitimately, it’s done by a unanimous assembly. 

When\marginnote{26.1.25} a procedure of ejection for not recognizing an offense has another three qualities, it’s also legitimate, in accordance with the Monastic Law, and properly disposed of: it’s done with the questioning of the accused, it’s done legitimately, it’s done by a unanimous assembly. 

When\marginnote{26.1.26} a procedure of ejection for not recognizing an offense has another three qualities, it’s also legitimate, in accordance with the Monastic Law, and properly disposed of: it’s done with the admission of the accused, it’s done legitimately, it’s done by a unanimous assembly. 

“When\marginnote{26.1.27} a procedure of ejection for not recognizing an offense has another three qualities, it’s also legitimate, in accordance with the Monastic Law, and properly disposed of: it’s done against one who’s committed an offense, it’s done legitimately, it’s done by a unanimous assembly. 

When\marginnote{26.1.28} a procedure of ejection for not recognizing an offense has another three qualities, it’s also legitimate, in accordance with the Monastic Law, and properly disposed of: it’s done against one who’s committed an offense that’s clearable by confession, it’s done legitimately, it’s done by a unanimous assembly. 

When\marginnote{26.1.29} a procedure of ejection for not recognizing an offense has another three qualities, it’s also legitimate, in accordance with the Monastic Law, and properly disposed of: it’s done against one who hasn’t confessed their offense, it’s done legitimately, it’s done by a unanimous assembly. 

“When\marginnote{26.1.30} a procedure of ejection for not recognizing an offense has another three qualities, it’s also legitimate, in accordance with the Monastic Law, and properly disposed of: it’s done after accusing the person of their offense, it’s done legitimately, it’s done by a unanimous assembly. 

When\marginnote{26.1.31} a procedure of ejection for not recognizing an offense has another three qualities, it’s also legitimate, in accordance with the Monastic Law, and properly disposed of: it’s done after reminding the person of their offense, it’s done legitimately, it’s done by a unanimous assembly. 

When\marginnote{26.1.32} a procedure of ejection for not recognizing an offense has another three qualities, it’s also legitimate, in accordance with the Monastic Law, and properly disposed of: it’s done after charging the person with their offense, it’s done legitimately, it’s done by a unanimous assembly.” 

\scend{The group of twelve on legitimate legal procedures of ejection for not recognizing an offense is finished. }

\subsection*{The group of six on wishing }

“When\marginnote{26.1.35.1} a monk has three qualities, the Sangha may, if it wishes, do a legal procedure of ejecting him for not recognizing an offense: he’s quarrelsome, argumentative, and a creator of legal issues in the Sangha; he’s ignorant and incompetent, often committing offenses, and lacking in boundaries; he’s constantly and improperly socializing with householders. 

When\marginnote{26.1.40} a monk has another three qualities, the Sangha may, if it wishes, do a procedure of ejecting him for not recognizing an offense: he has failed in the higher morality; he has failed in conduct; he has failed in view. 

When\marginnote{26.1.43} a monk has another three qualities, the Sangha may, if it wishes, do a procedure of ejecting him for not recognizing an offense: he disparages the Buddha; he disparages the Teaching; he disparages the Sangha. 

The\marginnote{26.1.46} Sangha may, if it wishes, do a procedure of ejection for not recognizing an offense against three kinds of monks: those who are quarrelsome, argumentative, and creators of legal issues in the Sangha; those who are ignorant and incompetent, often committing offenses, and lacking in boundaries; those who are constantly and improperly socializing with householders. 

The\marginnote{26.1.51} Sangha may, if it wishes, do a procedure of ejection for not recognizing an offense against three other kinds of monks: those who’ve failed in the higher morality; those who’ve failed in conduct; those who’ve failed in view. 

The\marginnote{26.1.54} Sangha may, if it wishes, do a procedure of ejection for not recognizing an offense against three other kinds of monks: those who disparage the Buddha; those who disparage the Teaching; those who disparage the Sangha.” 

\scend{The group of six on wishing in regard to a procedure of ejection for not recognizing an offense is finished. }

\subsection*{The forty-three kinds of conduct }

“A\marginnote{27.1.1} monk who’s had a legal procedure of ejection for not recognizing an offense done against himself should conduct himself properly. This is the proper conduct: 

\begin{enumerate}%
\item He shouldn’t give the full ordination. %
\item He shouldn’t give formal support. %
\item He shouldn’t have a novice monk attend on him. %
\item He shouldn’t accept being appointed as an instructor of the nuns. %
\item Even if appointed, he shouldn’t instruct the nuns. %
\item He shouldn’t commit the same offense as the offense for which the Sangha did the procedure of ejecting him for not recognizing an offense. %
\item He shouldn’t commit an offense similar to the offense for which the Sangha did the procedure of ejecting him for not recognizing an offense. %
\item He shouldn’t commit an offense worse than the offense for which the Sangha did the procedure of ejecting him for not recognizing an offense. %
\item He shouldn’t criticize the procedure. %
\item He shouldn’t criticize those who did the procedure. %
\item He shouldn’t consent to a regular monk bowing down to him. %
\item He shouldn’t consent to a regular monk standing up for him. %
\item He shouldn’t consent to a regular monk raising his joined palms to him. %
\item He shouldn’t consent to a regular monk doing acts of respect toward him. %
\item He shouldn’t consent to a regular monk offering him a seat. %
\item He shouldn’t consent to a regular monk offering him a bed. %
\item He shouldn’t consent to a regular monk offering him water for washing his feet and a foot stool. %
\item He shouldn’t consent to a regular monk offering him a foot scraper. %
\item He shouldn’t consent to a regular monk receiving his bowl and robe. %
\item He shouldn’t consent to a regular monk massaging his back when bathing. %
\item He shouldn’t charge a regular monk with failure in morality. %
\item He shouldn’t charge a regular monk with failure in conduct. %
\item He shouldn’t charge a regular monk with failure in view. %
\item He shouldn’t charge a regular monk with failure in livelihood. %
\item He shouldn’t cause division between monks. %
\item He shouldn’t wear lay clothes. %
\item He shouldn’t wear the robes of the monastics of other religions. %
\item He shouldn’t associate with the monastics of other religions. %
\item He should associate with monks. %
\item He should train in the monks’ training. %
\item He shouldn’t stay in the same room in a monastery as a regular monk. %
\item He shouldn’t stay in the same room in a non-monastery as a regular monk. %
\item He shouldn’t stay in the same room in a monastery or a non-monastery as a regular monk. %
\item He should get up from his seat when he sees a regular monk. %
\item He shouldn’t dismiss a regular monk, whether indoors or outdoors. %
\item He shouldn’t cancel the observance-day ceremony of a regular monk. %
\item He shouldn’t cancel the invitation ceremony of a regular monk. %
\item He shouldn’t direct a regular monk. %
\item He shouldn’t give instructions to a regular monk. %
\item He shouldn’t get permission from a regular monk to correct him. %
\item He shouldn’t accuse a regular monk of an offense. %
\item He shouldn’t remind a regular monk of an offense. %
\item He shouldn’t associate inappropriately with other monks.” %
\end{enumerate}

\scend{The forty-three kinds of conduct in regard to the legal procedure of ejection for not recognizing an offense are finished. }

Soon\marginnote{28.1.1} afterwards the Sangha did a legal procedure of ejecting the monk Channa for not recognizing an offense, prohibiting him from living with the Sangha. He then left that monastery and went to another one. The monks there did not bow down to him, stand up for him, raise their joined palms to him, or do acts of respect toward him. They did not honor, respect, or esteem him. Because of this, he left that monastery too and went to yet another one. There too the monks did not bow down to him, stand up for him, raise their joined palms to him, or do acts of respect toward him. They did not honor, respect, or esteem him. Because of this, he left that monastery too and went to yet another one. There too the monks did not bow down to him, stand up for him, raise their joined palms to him, or do acts of respect toward him. They did not honor, respect, or esteem him. Because of this, he left that monastery too and returned to \textsanskrit{Kosambī}. 

He\marginnote{28.1.10} then conducted himself properly and suitably, and deserved to be released. He went to the monks and told them about this, adding, “What should I do now?” The monks told the Buddha. He had the monks gathered and said, “Well then, lift that legal procedure of ejecting the monk Channa for not recognizing an offense.” 

\subsection*{The group of forty-three on not to be lifted }

“When\marginnote{28.2.1} a monk has five qualities, a legal procedure of ejecting him for not recognizing an offense shouldn’t be lifted: he gives the full ordination; he gives formal support; he has a novice monk attend on him; he accepts being appointed as an instructor of the nuns; he instructs the nuns, whether appointed or not. 

When\marginnote{28.2.4} a monk has another five qualities, a procedure of ejecting him for not recognizing an offense shouldn’t be lifted: he commits the same offense for which the Sangha did the procedure of ejecting him for not recognizing an offense; he commits an offense similar to the one for which the Sangha did the procedure of ejecting him for not recognizing an offense; he commits an offense worse than the one for which the Sangha did the procedure of ejecting him for not recognizing an offense; he criticizes the procedure; he criticizes those who did the procedure. 

When\marginnote{28.2.8} a monk has another five qualities, a procedure of ejecting him for not recognizing an offense shouldn’t be lifted: he consents to a regular monk bowing down to him; he consents to a regular monk standing up for him; he consents to a regular monk raising his joined palms to him; he consents to a regular monk doing acts of respect toward him; he consents to a regular monk offering him a seat. 

When\marginnote{28.2.11} a monk has another five qualities, a procedure of ejecting him for not recognizing an offense shouldn’t be lifted: he consents to a regular monk offering him a bed; he consents to a regular monk offering him water for washing his feet and a foot stool; he consents to a regular monk offering him a foot scraper; he consents to a regular monk receiving his bowl and robe; he consents to a regular monk massaging his back when bathing. 

When\marginnote{28.2.14} a monk has another five qualities, a procedure of ejecting him for not recognizing an offense shouldn’t be lifted: he charges a regular monk with failure in morality; he charges a regular monk with failure in conduct; he charges a regular monk with failure in view; he charges a regular monk with failure in livelihood; he causes division between monks. 

When\marginnote{28.2.17} a monk has another five qualities, a procedure of ejecting him for not recognizing an offense shouldn’t be lifted: he wears lay clothes; he wears the robes of the monastics of other religions; he associates with the monastics of other religions; he doesn’t associate with monks; he doesn’t train in the monks’ training. 

When\marginnote{28.2.21} a monk has another five qualities, a procedure of ejecting him for not recognizing an offense shouldn’t be lifted: he stays in the same room in a monastery as a regular monk; he stays in the same room in a non-monastery as a regular monk; he stays in the same room in a monastery or a non-monastery as a regular monk; he doesn’t get up from his seat when he sees a regular monk; he dismisses a regular monk, whether indoors or outdoors. 

When\marginnote{28.2.26} a monk has eight qualities, a procedure of ejecting him for not recognizing an offense shouldn’t be lifted: he cancels the observance-day ceremony of a regular monk; he cancels the invitation ceremony of a regular monk; he directs a regular monk; he gives instructions to a regular monk; he gets permission from a regular monk to correct him; he accuses a regular monk of an offense; he reminds a regular monk of an offense; he associates inappropriately with other monks.” 

\scend{The group of forty-three on not to be lifted in regard to the legal procedure of ejection for not recognizing an offense is finished. }

\subsection*{The group of forty-three on to be lifted }

“When\marginnote{29.1.1} a monk has five qualities, a legal procedure of ejecting him for not recognizing an offense should be lifted: he doesn’t give the full ordination; he doesn’t give formal support; he doesn’t have a novice monk attend on him; he doesn’t accept being appointed as an instructor of the nuns; he doesn’t instruct the nuns, whether appointed or not. 

When\marginnote{29.1.4} a monk has another five qualities, a procedure of ejecting him for not recognizing an offense should be lifted: he doesn’t commit the same offense for which the Sangha did the procedure of ejecting him for not recognizing an offense; he doesn’t commit an offense similar to the one for which the Sangha did the procedure of ejecting him for not recognizing an offense; he doesn’t commit an offense worse than the one for which the Sangha did the procedure of ejecting him for not recognizing an offense; he doesn’t criticize the procedure; he doesn’t criticize those who did the procedure. 

When\marginnote{29.1.8} a monk has another five qualities, a procedure of ejecting him for not recognizing an offense should be lifted: he doesn’t consent to a regular monk bowing down to him; he doesn’t consent to a regular monk standing up for him; he doesn’t consent to a regular monk raising his joined palms to him; he doesn’t consent to a regular monk doing acts of respect toward him; he doesn’t consent to a regular monk offering him a seat. 

When\marginnote{29.1.11} a monk has another five qualities, a procedure of ejecting him for not recognizing an offense should be lifted: he doesn’t consent to a regular monk offering him a bed; he doesn’t consent to a regular monk offering him water for washing his feet and a foot stool; he doesn’t consent to a regular monk offering him a foot scraper; he doesn’t consent to a regular monk receiving his bowl and robe; he doesn’t consent to a regular monk massaging his back when bathing. 

When\marginnote{29.1.14} a monk has another five qualities, a procedure of ejecting him for not recognizing an offense should be lifted: he doesn’t charge a regular monk with failure in morality; he doesn’t charge a regular monk with failure in conduct; he doesn’t charge a regular monk with failure in view; he doesn’t charge a regular monk with failure in livelihood; he doesn’t cause division between monks. 

When\marginnote{29.1.17} a monk has another five qualities, a procedure of ejecting him for not recognizing an offense should be lifted: he doesn’t wear lay clothes; he doesn’t wear the robes of the monastics of other religions; he doesn’t associate with the monastics of other religions; he associates with monks; he trains in the monks’ training. 

When\marginnote{29.1.20} a monk has another five qualities, a procedure of ejecting him for not recognizing an offense should be lifted: he doesn’t stay in the same room in a monastery as a regular monk; he doesn’t stay in the same room in a non-monastery as a regular monk; he doesn’t stay in the same room in a monastery or a non-monastery as a regular monk; he gets up from his seat when he sees a regular monk; he doesn’t dismiss a regular monk, whether indoors or outdoors. 

When\marginnote{29.1.23} a monk has eight qualities, a procedure of ejecting him for not recognizing an offense should be lifted: he doesn’t cancel the observance-day ceremony of a regular monk; he doesn’t cancel the invitation ceremony of a regular monk; he doesn’t direct a regular monk; he doesn’t give instructions to a regular monk; he doesn’t get permission from a regular monk to correct him; he doesn’t accuse a regular monk of an offense; he doesn’t remind a regular monk of an offense; he doesn’t associate inappropriately with other monks.” 

\scend{The group of forty-three on to be lifted in regard to the legal procedure of ejection for not recognizing an offense is finished. }

“And\marginnote{30.1.1} this is how it should be lifted. The monk Channa should approach the Sangha, arrange his upper robe over one shoulder, pay respect at the feet of the senior monks, squat on his heels, raise his joined palms, and say, ‘Venerables, the Sangha has done a legal procedure of ejecting me for not recognizing an offense. I’m now conducting myself properly and suitably, and deserve to be released. I ask for that legal procedure to be lifted.’ And he should ask a second and a third time. A competent and capable monk should then inform the Sangha: 

‘Please,\marginnote{30.1.7} venerables, I ask the Sangha to listen. The Sangha has done a legal procedure of ejecting the monk Channa for not recognizing an offense. He’s now conducting himself properly and suitably, and deserves to be released. He’s asking for that legal procedure to be lifted. If the Sangha is ready, it should lift that legal procedure of ejecting him for not recognizing an offense. This is the motion. 

Please,\marginnote{30.1.11} venerables, I ask the Sangha to listen. The Sangha has done a legal procedure of ejecting the monk Channa for not recognizing an offense. He’s now conducting himself properly and suitably, and deserves to be released. He’s asking for that legal procedure to be lifted. The Sangha lifts that legal procedure of ejecting him for not recognizing an offense. Any monk who approves of lifting that legal procedure should remain silent. Any monk who doesn’t approve should speak up. 

For\marginnote{30.1.16} the second time, I speak on this matter. … For the third time, I speak on this matter. Please, venerables, I ask the Sangha to listen. The Sangha has done a legal procedure of ejecting the monk Channa for not recognizing an offense. He’s now conducting himself properly and suitably, and deserves to be released. He’s asking for that legal procedure to be lifted. The Sangha lifts that legal procedure of ejecting him for not recognizing an offense. Any monk who approves of lifting that legal procedure should remain silent. Any monk who doesn’t approve should speak up. 

The\marginnote{30.1.23} Sangha has lifted that legal procedure of ejecting the monk Channa for not recognizing an offense. The Sangha approves and is therefore silent. I’ll remember it thus.’” 

\scend{The fifth section on the legal procedure of ejection for not recognizing an offense is finished. }

\section*{6. The legal procedure of ejection for not making amends for an offense }

At\marginnote{31.1.1} one time the Buddha was staying at \textsanskrit{Kosambī} in Ghosita’s Monastery. At this time Venerable Channa had committed an offense, but refused to make amends for it. The monks of few desires complained and criticized him, “How can Venerable Channa commit an offense, but refuse to make amends for it?” They told the Buddha. 

Soon\marginnote{31.1.6} afterwards the Buddha had the Sangha gathered and questioned the monks: “Is it true, monks, that Channa is acting like this?” 

“It’s\marginnote{31.1.8} true, sir.” 

The\marginnote{31.1.9} Buddha rebuked him, “It’s not suitable … How can Channa commit an offense, but refuse to make amends for it? This will affect people’s confidence …” After rebuking him … he gave a teaching and addressed the monks: 

“Well\marginnote{31.1.15} then, the Sangha should do a legal procedure of ejecting the monk Channa for not making amends for an offense, prohibiting him from living with the Sangha. And this is how it should be done. First you should accuse the monk Channa. He should then be reminded of what he has done, before he’s charged with an offense. A competent and capable monk should then inform the Sangha: 

‘Please,\marginnote{31.1.19} venerables, I ask the Sangha to listen. This monk Channa has committed an offense, but refuses to make amends for it. If the Sangha is ready, it should do a legal procedure of ejecting Channa for not making amends for an offense, prohibiting him from living with the Sangha. This is the motion. 

Please,\marginnote{31.1.24} venerables, I ask the Sangha to listen. This monk Channa has committed an offense, but refuses to make amends for it. The Sangha does a legal procedure of ejecting Channa for not making amends for an offense, prohibiting him from living with the Sangha. Any monk who approves of doing this legal procedure should remain silent. Any monk who doesn’t approve should speak up. 

For\marginnote{31.1.31} the second time, I speak on this matter. … For the third time, I speak on this matter. Please, venerables, I ask the Sangha to listen. This monk Channa has committed an offense, but refuses to make amends for it. The Sangha is doing a legal procedure of ejecting Channa for not making amends for an offense, prohibiting him from living with the Sangha. Any monk who approves of doing this legal procedure should remain silent. Any monk who doesn’t approve should speak up. 

The\marginnote{31.1.40} Sangha has done the legal procedure of ejecting Channa for not making amends for an offense, prohibiting him from living with the Sangha. The Sangha approves and is therefore silent. I’ll remember it thus.’ 

Monks,\marginnote{31.1.43} you should proclaim from monastery to monastery that the Sangha has done a legal procedure of ejecting Channa for not making amends for an offense, prohibiting him from living with the Sangha.” 

\subsection*{The group of twelve on illegitimate legal procedures }

“When\marginnote{31.1.46.1} a legal procedure of ejection for not making amends for an offense has three qualities, it’s illegitimate, contrary to the Monastic Law, and improperly disposed of: it’s done in the absence of the accused, it’s done without questioning the accused, it’s done without the admission of the accused. 

When\marginnote{31.1.49} a procedure of ejection for not making amends for an offense has another three qualities, it’s also illegitimate, contrary to the Monastic Law, and improperly disposed of: it’s done against one who hasn’t committed any offense, it’s done against one who’s committed an offense that isn’t clearable by confession, it’s done against one who’s confessed their offense. 

When\marginnote{31.1.51} a procedure of ejection for not making amends for an offense has another three qualities, it’s also illegitimate, contrary to the Monastic Law, and improperly disposed of: it’s done without having accused the person of their offense, it’s done without having reminded the person of their offense, it’s done without having charged the person with their offense. 

“When\marginnote{31.1.52} a procedure of ejection for not making amends for an offense has another three qualities, it’s also illegitimate, contrary to the Monastic Law, and improperly disposed of: it’s done in the absence of the accused, it’s done illegitimately, it’s done by an incomplete assembly. 

When\marginnote{31.1.53} a procedure of ejection for not making amends for an offense has another three qualities, it’s also illegitimate, contrary to the Monastic Law, and improperly disposed of: it’s done without questioning the accused, it’s done illegitimately, it’s done by an incomplete assembly. 

When\marginnote{31.1.54} a procedure of ejection for not making amends for an offense has another three qualities, it’s also illegitimate, contrary to the Monastic Law, and improperly disposed of: it’s done without the admission of the accused, it’s done illegitimately, it’s done by an incomplete assembly. 

“When\marginnote{31.1.55} a procedure of ejection for not making amends for an offense has another three qualities, it’s also illegitimate, contrary to the Monastic Law, and improperly disposed of: it’s done against one who hasn’t committed any offense, it’s done illegitimately, it’s done by an incomplete assembly. 

When\marginnote{31.1.56} a procedure of ejection for not making amends for an offense has another three qualities, it’s also illegitimate, contrary to the Monastic Law, and improperly disposed of: it’s done against one who’s committed an offense that isn’t clearable by confession, it’s done illegitimately, it’s done by an incomplete assembly. 

When\marginnote{31.1.57} a procedure of ejection for not making amends for an offense has another three qualities, it’s also illegitimate, contrary to the Monastic Law, and improperly disposed of: it’s done against one who’s confessed their offense, it’s done illegitimately, it’s done by an incomplete assembly. 

“When\marginnote{31.1.58} a procedure of ejection for not making amends for an offense has another three qualities, it’s also illegitimate, contrary to the Monastic Law, and improperly disposed of: it’s done without having accused the person of their offense, it’s done illegitimately, it’s done by an incomplete assembly. 

When\marginnote{31.1.59} a procedure of ejection for not making amends for an offense has another three qualities, it’s also illegitimate, contrary to the Monastic Law, and improperly disposed of: it’s done without having reminded the person of their offense, it’s done illegitimately, it’s done by an incomplete assembly. 

When\marginnote{31.1.60} a procedure of ejection for not making amends for an offense has another three qualities, it’s also illegitimate, contrary to the Monastic Law, and improperly disposed of: it’s done without having charged the person with their offense, it’s done illegitimately, it’s done by an incomplete assembly.” 

\scend{The group of twelve on illegitimate legal procedures of ejection for not making amends for an offense is finished. }

\subsection*{The group of twelve on legitimate legal procedures }

“When\marginnote{31.1.63.1} a legal procedure of ejection for not making amends for an offense has three qualities, it’s legitimate, in accordance with the Monastic Law, and properly disposed of: it’s done in the presence of the accused, it’s done with the questioning of the accused, it’s done with the admission of the accused. 

When\marginnote{31.1.66} a procedure of ejection for not making amends for an offense has another three qualities, it’s also legitimate, in accordance with the Monastic Law, and properly disposed of: it’s done against one who’s committed an offense, it’s done against one who’s committed an offense that’s clearable by confession, it’s done against one who hasn’t confessed their offense. 

When\marginnote{31.1.68} a procedure of ejection for not making amends for an offense has another three qualities, it’s also legitimate, in accordance with the Monastic Law, and properly disposed of: it’s done after having accused the person of their offense, it’s done after having reminded the person of their offense, it’s done after having charged the person with their offense. 

“When\marginnote{31.1.69} a procedure of ejection for not making amends for an offense has another three qualities, it’s also legitimate, in accordance with the Monastic Law, and properly disposed of: it’s done in the presence of the accused, it’s done legitimately, it’s done by a unanimous assembly. 

When\marginnote{31.1.70} a procedure of ejection for not making amends for an offense has another three qualities, it’s also legitimate, in accordance with the Monastic Law, and properly disposed of: it’s done with the questioning of the accused, it’s done legitimately, it’s done by a unanimous assembly. 

When\marginnote{31.1.71} a procedure of ejection for not making amends for an offense has another three qualities, it’s also legitimate, in accordance with the Monastic Law, and properly disposed of: it’s done with the admission of the accused, it’s done legitimately, it’s done by a unanimous assembly. 

“When\marginnote{31.1.72} a procedure of ejection for not making amends for an offense has another three qualities, it’s also legitimate, in accordance with the Monastic Law, and properly disposed of: it’s done against one who’s committed an offense, it’s done legitimately, it’s done by a unanimous assembly. 

When\marginnote{31.1.73} a procedure of ejection for not making amends for an offense has another three qualities, it’s also legitimate, in accordance with the Monastic Law, and properly disposed of: it’s done against one who’s committed an offense that’s clearable by confession, it’s done legitimately, it’s done by a unanimous assembly. 

When\marginnote{31.1.74} a procedure of ejection for not making amends for an offense has another three qualities, it’s also legitimate, in accordance with the Monastic Law, and properly disposed of: it’s done against one who hasn’t confessed their offense, it’s done legitimately, it’s done by a unanimous assembly. 

“When\marginnote{31.1.75} a procedure of ejection for not making amends for an offense has another three qualities, it’s also legitimate, in accordance with the Monastic Law, and properly disposed of: it’s done after accusing the person of their offense, it’s done legitimately, it’s done by a unanimous assembly. 

When\marginnote{31.1.76} a procedure of ejection for not making amends for an offense has another three qualities, it’s also legitimate, in accordance with the Monastic Law, and properly disposed of: it’s done after reminding the person of their offense, it’s done legitimately, it’s done by a unanimous assembly. 

When\marginnote{31.1.77} a procedure of ejection for not making amends for an offense has another three qualities, it’s also legitimate, in accordance with the Monastic Law, and properly disposed of: it’s done after charging the person with their offense, it’s done legitimately, it’s done by a unanimous assembly.” 

\scend{The group of twelve on legitimate legal procedures of ejection for not making amends for an offense is finished. }

\subsection*{The group of six on wishing }

“When\marginnote{31.1.80.1} a monk has three qualities, the Sangha may, if it wishes, do a legal procedure of ejecting him for not making amends for an offense: he’s quarrelsome, argumentative, and a creator of legal issues in the Sangha; he’s ignorant and incompetent, often committing offenses, and lacking in boundaries; he’s constantly and improperly socializing with householders. 

When\marginnote{31.1.85} a monk has another three qualities, the Sangha may, if it wishes, do a procedure of ejecting him for not making amends for an offense: he has failed in the higher morality; he has failed in conduct; he has failed in view. 

When\marginnote{31.1.88} a monk has another three qualities, the Sangha may, if it wishes, do a procedure of ejecting him for not making amends for an offense: he disparages the Buddha; he disparages the Teaching; he disparages the Sangha. 

The\marginnote{31.1.91} Sangha may, if it wishes, do a procedure of ejection for not making amends for an offense against three kinds of monks: those who are quarrelsome, argumentative, and creators of legal issues in the Sangha; those who are ignorant and incompetent, often committing offenses, and lacking in boundaries; those who are constantly and improperly socializing with householders. 

The\marginnote{31.1.96} Sangha may, if it wishes, do a procedure of ejection for not making amends for an offense against three other kinds of monks: those who’ve failed in the higher morality; those who’ve failed in conduct; those who’ve failed in view. 

The\marginnote{31.1.99} Sangha may, if it wishes, do a procedure of ejection for not making amends for an offense against three other kinds of monks: those who disparage the Buddha; those who disparage the Teaching; those who disparage the Sangha.” 

\scend{The group of six on wishing in regard to the legal procedure of ejection for not making amends for an offense is finished. }

\subsection*{The forty-three kinds of conduct }

“A\marginnote{31.1.103.1} monk who’s had a legal procedure of ejection for not making amends for an offense done against himself should conduct himself properly. This is the proper conduct: 

\begin{enumerate}%
\item He shouldn’t give the full ordination. %
\item He shouldn’t give formal support. %
\item He shouldn’t have a novice monk attend on him. %
\item He shouldn’t accept being appointed as an instructor of the nuns. %
\item Even if appointed, he shouldn’t instruct the nuns. %
\item He shouldn’t commit the same offense as the offense for which the Sangha did the procedure of ejecting him for not making amends for an offense. %
\item He shouldn’t commit an offense similar to the offense for which the Sangha did the procedure of ejecting him for not making amends for an offense. %
\item He shouldn’t commit an offense worse than the offense for which the Sangha did the procedure of ejecting him for not making amends for an offense. %
\item He shouldn’t criticize the procedure. %
\item He shouldn’t criticize those who did the procedure. %
\item He shouldn’t consent to a regular monk bowing down to him. %
\item He shouldn’t consent to a regular monk standing up for him. %
\item He shouldn’t consent to a regular monk raising his joined palms to him. %
\item He shouldn’t consent to a regular monk doing acts of respect toward him. %
\item He shouldn’t consent to a regular monk offering him a seat. %
\item He shouldn’t consent to a regular monk offering him a bed. %
\item He shouldn’t consent to a regular monk offering him water for washing his feet and a foot stool. %
\item He shouldn’t consent to a regular monk offering him a foot scraper. %
\item He shouldn’t consent to a regular monk receiving his bowl and robe. %
\item He shouldn’t consent to a regular monk massaging his back when bathing. %
\item He shouldn’t charge a regular monk with failure in morality. %
\item He shouldn’t charge a regular monk with failure in conduct. %
\item He shouldn’t charge a regular monk with failure in view. %
\item He shouldn’t charge a regular monk with failure in livelihood. %
\item He shouldn’t cause division between monks. %
\item He shouldn’t wear lay clothes. %
\item He shouldn’t wear the robes of the monastics of other religions. %
\item He shouldn’t associate with the monastics of other religions. %
\item He should associate with monks. %
\item He should train in the monks’ training. %
\item He shouldn’t stay in the same room in a monastery as a regular monk. %
\item He shouldn’t stay in the same room in a non-monastery as a regular monk. %
\item He shouldn’t stay in the same room in a monastery or a non-monastery as a regular monk. %
\item He should get up from his seat when he sees a regular monk. %
\item He shouldn’t dismiss a regular monk, whether indoors or outdoors. %
\item He shouldn’t cancel the observance-day ceremony of a regular monk. %
\item He shouldn’t cancel the invitation ceremony of a regular monk. %
\item He shouldn’t direct a regular monk. %
\item He shouldn’t give instructions to a regular monk. %
\item He shouldn’t get permission from a regular monk to correct him. %
\item He shouldn’t accuse a regular monk of an offense. %
\item He shouldn’t remind a regular monk of an offense. %
\item He shouldn’t associate inappropriately with other monks.” %
\end{enumerate}

\scend{The forty-three kinds of conduct in regard to the legal procedure of ejection for not making amends for an offense are finished. }

Soon\marginnote{31.1.149} afterwards the Sangha did a legal procedure of ejecting the monk Channa for not making amends for an offense, prohibiting him from living with the Sangha. He then left that monastery and went to another one. The monks there did not bow down to him, stand up for him, raise their joined palms to him, or do acts of respect toward him. They did not honor, respect, or esteem him. Because of this, he left that monastery too and went to yet another one. There too the monks did not bow down to him, stand up for him, raise their joined palms to him, or do acts of respect toward him. They did not honor, respect, or esteem him. Because of this, he left that monastery too and went to yet another one. There too the monks did not bow down to him, stand up for him, raise their joined palms to him, or do acts of respect toward him. They did not honor, respect, or esteem him. Because of this, he left that monastery too and returned to \textsanskrit{Kosambī}. 

He\marginnote{31.1.158} then conducted himself properly and suitably, and deserved to be released. He went to the monks and told them about this, adding, “What should I do now?” The monks told the Buddha. … 

“Well\marginnote{31.1.162} then, lift that legal procedure of ejecting the monk Channa for not making amends for an offense.” 

\subsection*{The group of forty-three on not to be lifted }

“When\marginnote{31.1.163.1} a monk has five qualities, a legal procedure of ejecting him for not making amends for an offense shouldn’t be lifted: he gives the full ordination; he gives formal support; he has a novice monk attend on him; he accepts being appointed as an instructor of the nuns; he instructs the nuns, whether appointed or not. 

When\marginnote{31.1.166} a monk has another five qualities, a procedure of ejecting him for not making amends for an offense shouldn’t be lifted: he commits the same offense for which the Sangha did the procedure of ejecting him for not making amends for an offense; he commits an offense similar to the one for which the Sangha did the procedure of ejecting him for not making amends for an offense; he commits an offense worse than the one for which the Sangha did the procedure of ejecting him for not making amends for an offense; he criticizes the procedure; he criticizes those who did the procedure. … he consents to a regular monk bowing down to him; he consents to a regular monk standing up for him; he consents to a regular monk raising his joined palms to him; he consents to a regular monk doing acts of respect toward him; he consents to a regular monk offering him a seat. … he consents to a regular monk offering him a bed; he consents to a regular monk offering him water for washing his feet and a foot stool; he consents to a regular monk offering him a foot scraper; he consents to a regular monk receiving his bowl and robe; he consents to a regular monk massaging his back when bathing. … he charges a regular monk with failure in morality; he charges a regular monk with failure in conduct; he charges a regular monk with failure in view; he charges a regular monk with failure in livelihood; he causes division between monks. … he wears lay clothes; he wears the robes of the monastics of other religions; he associates with the monastics of other religions; he doesn’t associate with monks; he doesn’t train in the monks’ training. … he stays in the same room in a monastery as a regular monk; he stays in the same room in a non-monastery as a regular monk; he stays in the same room in a monastery or a non-monastery as a regular monk; he doesn’t get up from his seat when he sees a regular monk; he dismisses a regular monk, whether indoors or outdoors. 

When\marginnote{31.1.175} a monk has eight qualities, a procedure of ejecting him for not making amends for an offense shouldn’t be lifted: he cancels the observance-day ceremony of a regular monk; he cancels the invitation ceremony of a regular monk; he directs a regular monk; he gives instructions to a regular monk; he gets permission from a regular monk to correct him; he accuses a regular monk of an offense; he reminds a regular monk of an offense; he associates inappropriately with other monks.” 

\scend{The group of forty-three on not to be lifted in regard to the legal procedure of ejection for not making amends for an offense is finished. }

\subsection*{The group of forty-three on to be lifted }

“When\marginnote{31.1.179.1} a monk has five qualities, a legal procedure of ejecting him for not making amends for an offense should be lifted: he doesn’t give the full ordination; he doesn’t give formal support; he doesn’t have a novice monk attend on him; he doesn’t accept being appointed as an instructor of the nuns; he doesn’t instruct the nuns, whether appointed or not. 

When\marginnote{31.1.182} a monk has another five qualities, a procedure of ejecting him for not making amends for an offense should be lifted: he doesn’t commit the same offense for which the Sangha did the procedure of ejecting him for not making amends for an offense; he doesn’t commit an offense similar to the one for which the Sangha did the procedure of ejecting him for not making amends for an offense; he doesn’t commit an offense worse than the one for which the Sangha did the procedure of ejecting him for not making amends for an offense; he doesn’t criticize the procedure; he doesn’t criticize those who did the procedure. … he doesn’t consent to a regular monk bowing down to him; he doesn’t consent to a regular monk standing up for him; he doesn’t consent to a regular monk raising his joined palms to him; he doesn’t consent to a regular monk doing acts of respect toward him; he doesn’t consent to a regular monk offering him a seat. … he doesn’t consent to a regular monk offering him a bed; he doesn’t consent to a regular monk offering him water for washing his feet and a foot stool; he doesn’t consent to a regular monk offering him a foot scraper; he doesn’t consent to a regular monk receiving his bowl and robe; he doesn’t consent to a regular monk massaging his back when bathing. … he doesn’t charge a regular monk with failure in morality; he doesn’t charge a regular monk with failure in conduct; he doesn’t charge a regular monk with failure in view; he doesn’t charge a regular monk with failure in livelihood; he doesn’t cause division between monks. … he doesn’t wear lay clothes; he doesn’t wear the robes of the monastics of other religions; he doesn’t associate with the monastics of other religions; he associates with monks; he trains in the monks’ training. … he doesn’t stay in the same room in a monastery as a regular monk; he doesn’t stay in the same room in a non-monastery as a regular monk; he doesn’t stay in the same room in a monastery or a non-monastery as a regular monk; he gets up from his seat when he sees a regular monk; he doesn’t dismiss a regular monk, whether indoors or outdoors. 

When\marginnote{31.1.191} a monk has eight qualities, a procedure of ejecting him for not making amends for an offense should be lifted: he doesn’t cancel the observance-day ceremony of a regular monk; he doesn’t cancel the invitation ceremony of a regular monk; he doesn’t direct a regular monk; he doesn’t give instructions to a regular monk; he doesn’t get permission from a regular monk to correct him; he doesn’t accuse a regular monk of an offense; he doesn’t remind a regular monk of an offense; he doesn’t associate inappropriately with other monks.” 

\scend{The group of forty-three on to be lifted in regard to the legal procedure of ejection for not making amends for an offense is finished. }

“And\marginnote{31.1.195} this is how it should be lifted. The monk Channa should approach the Sangha, arrange his upper robe over one shoulder, pay respect at the feet of the senior monks, squat on his heels, raise his joined palms, and say, ‘Venerables, the Sangha has done a legal procedure of ejecting me for not making amends for an offense. I’m now conducting myself properly and suitably, and deserve to be released. I ask for that legal procedure to be lifted.’ And he should ask a second and a third time. A competent and capable monk should then inform the Sangha: 

‘Please,\marginnote{31.1.201} venerables, I ask the Sangha to listen. The Sangha has done a legal procedure of ejecting the monk Channa for not making amends for an offense. He’s now conducting himself properly and suitably, and deserves to be released. He’s asking for that legal procedure to be lifted. If the Sangha is ready, it should lift that legal procedure of ejecting him for not making amends for an offense. This is the motion. 

Please,\marginnote{31.1.205} venerables, I ask the Sangha to listen. The Sangha has done a legal procedure of ejecting the monk Channa for not making amends for an offense. He’s now conducting himself properly and suitably, and deserves to be released. He’s asking for that legal procedure to be lifted. The Sangha lifts that legal procedure of ejecting him for not making amends for an offense. Any monk who approves of lifting that legal procedure should remain silent. Any monk who doesn’t approve should speak up. 

For\marginnote{31.1.210} the second time, I speak on this matter. … For the third time, I speak on this matter. Please, venerables, I ask the Sangha to listen. The Sangha has done a legal procedure of ejecting the monk Channa for not making amends for an offense. He’s now conducting himself properly and suitably, and deserves to be released. He’s asking for that legal procedure to be lifted. The Sangha lifts that legal procedure of ejecting him for not making amends for an offense. Any monk who approves of lifting that legal procedure should remain silent. Any monk who doesn’t approve should speak up. 

The\marginnote{31.1.217} Sangha has lifted that legal procedure of ejecting the monk Channa for not making amends for an offense. The Sangha approves and is therefore silent. I’ll remember it thus.’” 

\scend{The sixth section on the legal procedure of ejection for not making amends for an offense is finished. }

\section*{7. The legal procedure of ejection for not giving up a bad view }

At\marginnote{32.1.1} one time the Buddha was staying at \textsanskrit{Sāvatthī} in the Jeta Grove, \textsanskrit{Anāthapiṇḍika}’s Monastery. At that time the monk \textsanskrit{Ariṭṭha}, an ex-vulture-killer, had the following bad and erroneous view: “As I understand the Teaching of the Buddha, the things he calls obstacles are not able to obstruct one who indulges in them.” 

A\marginnote{32.1.4} number of monks heard that \textsanskrit{Ariṭṭha} had that view. 

They\marginnote{32.1.6} went to him and asked, “Is it true, \textsanskrit{Ariṭṭha}, that you have such a view?” 

“Yes,\marginnote{32.1.10} indeed. As I understand the Buddha’s Teaching, the things he calls obstacles are not able to obstruct one who indulges in them.” 

“No,\marginnote{32.2.1} \textsanskrit{Ariṭṭha}, don’t misrepresent the Buddha, for it’s not good to misrepresent him. The Buddha would never say such a thing. The Buddha has given many discourses about the obstacles being obstructive and how they obstruct one who indulges in them. The Buddha has said that the enjoyment provided by worldly pleasures is small, whereas the suffering and trouble with them are huge, and so their drawbacks are greater. The Buddha has said that worldly pleasures are similar to a skeleton, similar to a piece of meat, similar to a grass torch, similar to a pit of coals, similar to a dream, similar to borrowed goods, similar to fruits on a tree, similar to a knife and chopping block, similar to swords and stakes, and similar to a snake’s head. The suffering and trouble with them are huge, and so their drawbacks are greater.” 

But\marginnote{32.2.18} even though the monks corrected \textsanskrit{Ariṭṭha} in this way, he stubbornly held on to that bad and erroneous view, and continued to insist on it. And since they were unable to make him give up that view, they went to the Buddha and told him what had happened. Soon afterwards the Buddha had the Sangha gathered and questioned \textsanskrit{Ariṭṭha}: 

“Is\marginnote{32.3.3} it true, \textsanskrit{Ariṭṭha}, that you have a bad and erroneous view such as this: ‘As I understand the Buddha’s Teaching, the things he calls obstacles are not able to obstruct one who indulges in them’?” 

“Yes\marginnote{32.3.5} indeed, sir.” 

“Foolish\marginnote{32.3.6} man, who do you think I’ve taught like this? Haven’t I given many discourses about the obstacles being obstructive and how they obstruct one who indulges in them? I’ve said that the enjoyment provided by worldly pleasures is small, whereas the suffering and trouble with them are huge, and so their drawbacks are greater. I’ve said that worldly pleasures are similar to a skeleton, similar to a piece of meat, similar to a grass torch, similar to a pit of coals, similar to a dream, similar to borrowed goods, similar to fruits on a tree, similar to a knife and chopping block, similar to swords and stakes, and similar to a snake’s head. The suffering and trouble with them are huge, and so their drawbacks are greater. And yet by misunderstanding, you have misrepresented me, hurt yourself, and made much demerit. This will be for your long-lasting harm and suffering. And this will affect people’s confidence …” After rebuking him … he gave a teaching and addressed the monks: 

“Well\marginnote{32.3.25} then, the Sangha should do a legal procedure of ejecting \textsanskrit{Ariṭṭha} for not giving up a bad view, prohibiting him from living with the Sangha. And this is how it should be done. First you should accuse the monk \textsanskrit{Ariṭṭha}. He should then be reminded of what he has done, before he’s charged with an offense. A competent and capable monk should then inform the Sangha: 

‘Please,\marginnote{32.4.3} venerables, I ask the Sangha to listen. The monk \textsanskrit{Ariṭṭha}, an ex-vulture-killer, has the following bad and erroneous view: “As I understand the Teaching of the Buddha, the things he calls obstacles are not able to obstruct one who indulges in them.” He’s not giving up that view. If the Sangha is ready, it should do a legal procedure of ejecting \textsanskrit{Ariṭṭha} for not giving up a bad view, prohibiting him from living with the Sangha. This is the motion. 

Please,\marginnote{32.4.10} venerables, I ask the Sangha to listen. The monk \textsanskrit{Ariṭṭha}, an ex-vulture-killer, has the following bad and erroneous view: “As I understand the Teaching of the Buddha, the things he calls obstacles are not able to obstruct one who indulges in them.” He’s not giving up that view. The Sangha does a legal procedure of ejecting \textsanskrit{Ariṭṭha} for not giving up a bad view, prohibiting him from living with the Sangha. Any monk who approves of doing this legal procedure should remain silent. Any monk who doesn’t approve should speak up. 

For\marginnote{32.4.19} the second time, I speak on this matter. … For the third time, I speak on this matter. Please, venerables, I ask the Sangha to listen. The monk \textsanskrit{Ariṭṭha}, an ex-vulture-killer, has the following bad and erroneous view: “As I understand the Teaching of the Buddha, the things he calls obstacles are not able to obstruct one who indulges in them.” He’s not giving up that view. The Sangha does a legal procedure of ejecting \textsanskrit{Ariṭṭha} for not giving up a bad view, prohibiting him from living with the Sangha. Any monk who approves of doing this legal procedure should remain silent. Any monk who doesn’t approve should speak up. 

The\marginnote{32.4.30} Sangha has done the legal procedure of ejecting \textsanskrit{Ariṭṭha} for not giving up a bad view, prohibiting him from living with the Sangha. The Sangha approves and is therefore silent. I’ll remember it thus.’ 

Monks,\marginnote{32.4.33} you should proclaim from monastery to monastery that the Sangha has done a legal procedure of ejecting \textsanskrit{Ariṭṭha} for not giving up a bad view, prohibiting him from living with the Sangha.” 

\subsection*{The group of twelve on illegitimate legal procedures }

“When\marginnote{33.1.1} a legal procedure of ejection for not giving up a bad view has three qualities, it’s illegitimate, contrary to the Monastic Law, and improperly disposed of: it’s done in the absence of the accused, it’s done without questioning the accused, it’s done without the admission of the accused. 

When\marginnote{33.1.4} a procedure of ejection for not giving up a bad view has another three qualities, it’s also illegitimate, contrary to the Monastic Law, and improperly disposed of: it’s done against one who hasn’t committed any offense, it’s done against one who’s committed an offense that isn’t clearable by confession, it’s done against one who’s confessed their offense. 

When\marginnote{33.1.6} a procedure of ejection for not giving up a bad view has another three qualities, it’s also illegitimate, contrary to the Monastic Law, and improperly disposed of: it’s done without having accused the person of their offense, it’s done without having reminded the person of their offense, it’s done without having charged the person with their offense. 

“When\marginnote{33.1.7} a procedure of ejection for not giving up a bad view has another three qualities, it’s also illegitimate, contrary to the Monastic Law, and improperly disposed of: it’s done in the absence of the accused, it’s done illegitimately, it’s done by an incomplete assembly. 

When\marginnote{33.1.8} a procedure of ejection for not giving up a bad view has another three qualities, it’s also illegitimate, contrary to the Monastic Law, and improperly disposed of: it’s done without questioning the accused, it’s done illegitimately, it’s done by an incomplete assembly. 

When\marginnote{33.1.9} a procedure of ejection for not giving up a bad view has another three qualities, it’s also illegitimate, contrary to the Monastic Law, and improperly disposed of: it’s done without the admission of the accused, it’s done illegitimately, it’s done by an incomplete assembly. 

“When\marginnote{33.1.10} a procedure of ejection for not giving up a bad view has another three qualities, it’s also illegitimate, contrary to the Monastic Law, and improperly disposed of: it’s done against one who hasn’t committed any offense, it’s done illegitimately, it’s done by an incomplete assembly. 

When\marginnote{33.1.11} a procedure of ejection for not giving up a bad view has another three qualities, it’s also illegitimate, contrary to the Monastic Law, and improperly disposed of: it’s done against one who’s committed an offense that isn’t clearable by confession, it’s done illegitimately, it’s done by an incomplete assembly. 

When\marginnote{33.1.12} a procedure of ejection for not giving up a bad view has another three qualities, it’s also illegitimate, contrary to the Monastic Law, and improperly disposed of: it’s done against one who’s confessed their offense, it’s done illegitimately, it’s done by an incomplete assembly. 

“When\marginnote{33.1.13} a procedure of ejection for not giving up a bad view has another three qualities, it’s also illegitimate, contrary to the Monastic Law, and improperly disposed of: it’s done without having accused the person of their offense, it’s done illegitimately, it’s done by an incomplete assembly. 

When\marginnote{33.1.14} a procedure of ejection for not giving up a bad view has another three qualities, it’s also illegitimate, contrary to the Monastic Law, and improperly disposed of: it’s done without having reminded the person of their offense, it’s done illegitimately, it’s done by an incomplete assembly. 

When\marginnote{33.1.15} a procedure of ejection for not giving up a bad view has another three qualities, it’s also illegitimate, contrary to the Monastic Law, and improperly disposed of: it’s done without having charged the person with their offense, it’s done illegitimately, it’s done by an incomplete assembly.” 

\scend{The group of twelve on illegitimate legal procedures of ejection for not giving up a bad view is finished. }

\subsection*{The group of twelve on legitimate legal procedures }

“When\marginnote{33.1.18.1} a legal procedure of ejection for not giving up a bad view has three qualities, it’s legitimate, in accordance with the Monastic Law, and properly disposed of: it’s done in the presence of the accused, it’s done with the questioning of the accused, it’s done with the admission of the accused. 

When\marginnote{33.1.21} a procedure of ejection for not giving up a bad view has another three qualities, it’s also legitimate, in accordance with the Monastic Law, and properly disposed of: it’s done against one who’s committed an offense, it’s done against one who’s committed an offense that’s clearable by confession, it’s done against one who hasn’t confessed their offense. 

When\marginnote{33.1.23} a procedure of ejection for not giving up a bad view has another three qualities, it’s also legitimate, in accordance with the Monastic Law, and properly disposed of: it’s done after having accused the person of their offense, it’s done after having reminded the person of their offense, it’s done after having charged the person with their offense. 

“When\marginnote{33.1.24} a procedure of ejection for not giving up a bad view has another three qualities, it’s also legitimate, in accordance with the Monastic Law, and properly disposed of: it’s done in the presence of the accused, it’s done legitimately, it’s done by a unanimous assembly. 

When\marginnote{33.1.25} a procedure of ejection for not giving up a bad view has another three qualities, it’s also legitimate, in accordance with the Monastic Law, and properly disposed of: it’s done with the questioning of the accused, it’s done legitimately, it’s done by a unanimous assembly. 

When\marginnote{33.1.26} a procedure of ejection for not giving up a bad view has another three qualities, it’s also legitimate, in accordance with the Monastic Law, and properly disposed of: it’s done with the admission of the accused, it’s done legitimately, it’s done by a unanimous assembly. 

“When\marginnote{33.1.27} a procedure of ejection for not giving up a bad view has another three qualities, it’s also legitimate, in accordance with the Monastic Law, and properly disposed of: it’s done against one who’s committed an offense, it’s done legitimately, it’s done by a unanimous assembly. 

When\marginnote{33.1.28} a procedure of ejection for not giving up a bad view has another three qualities, it’s also legitimate, in accordance with the Monastic Law, and properly disposed of: it’s done against one who’s committed an offense that’s clearable by confession, it’s done legitimately, it’s done by a unanimous assembly. 

When\marginnote{33.1.29} a procedure of ejection for not giving up a bad view has another three qualities, it’s also legitimate, in accordance with the Monastic Law, and properly disposed of: it’s done against one who hasn’t confessed their offense, it’s done legitimately, it’s done by a unanimous assembly. 

“When\marginnote{33.1.30} a procedure of ejection for not giving up a bad view has another three qualities, it’s also legitimate, in accordance with the Monastic Law, and properly disposed of: it’s done after accusing the person of their offense, it’s done legitimately, it’s done by a unanimous assembly. 

When\marginnote{33.1.31} a procedure of ejection for not giving up a bad view has another three qualities, it’s also legitimate, in accordance with the Monastic Law, and properly disposed of: it’s done after reminding the person of their offense, it’s done legitimately, it’s done by a unanimous assembly. 

When\marginnote{33.1.32} a procedure of ejection for not giving up a bad view has another three qualities, it’s also legitimate, in accordance with the Monastic Law, and properly disposed of: it’s done after charging the person with their offense, it’s done legitimately, it’s done by a unanimous assembly.” 

\scend{The group of twelve on legitimate legal procedures of ejection for not giving up a bad view is finished. }

\subsection*{The group of six on wishing }

“When\marginnote{33.1.35.1} a monk has three qualities, the Sangha may, if it wishes, do a legal procedure of ejecting him for not giving up a bad view: he’s quarrelsome, argumentative, and a creator of legal issues in the Sangha; he’s ignorant and incompetent, often committing offenses, and lacking in boundaries; he’s constantly and improperly socializing with householders. 

When\marginnote{33.1.40} a monk has another three qualities, the Sangha may, if it wishes, do a procedure of ejecting him for not giving up a bad view: he has failed in the higher morality; he has failed in conduct; he has failed in view. 

When\marginnote{33.1.43} a monk has another three qualities, the Sangha may, if it wishes, do a procedure of ejecting him for not giving up a bad view: he disparages the Buddha; he disparages the Teaching; he disparages the Sangha. 

The\marginnote{33.1.46} Sangha may, if it wishes, do a procedure of ejection for not giving up a bad view against three kinds of monks: those who are quarrelsome, argumentative, and creators of legal issues in the Sangha; those who are ignorant and incompetent, often committing offenses, and lacking in boundaries; those who are constantly and improperly socializing with householders. 

The\marginnote{33.1.51} Sangha may, if it wishes, do a procedure of ejection for not giving up a bad view against three other kinds of monks: those who’ve failed in the higher morality; those who’ve failed in conduct; those who’ve failed in view. 

The\marginnote{33.1.54} Sangha may, if it wishes, do a procedure of ejection for not giving up a bad view against three other kinds of monks: those who disparage the Buddha; those who disparage the Teaching; those who disparage the Sangha.” 

\scend{The group of six on wishing in regard to the legal procedure of ejection for not giving up a bad view is finished. }

\subsection*{The forty-three kinds of conduct }

“A\marginnote{33.1.58.1} monk who’s had a legal procedure of ejection for not giving up a bad view done against himself should conduct himself properly. This is the proper conduct: 

\begin{enumerate}%
\item He shouldn’t give the full ordination. %
\item He shouldn’t give formal support. %
\item He shouldn’t have a novice monk attend on him. %
\item He shouldn’t accept being appointed as an instructor of the nuns. %
\item Even if appointed, he shouldn’t instruct the nuns. %
\item He shouldn’t commit the same offense for which the Sangha did the procedure of ejecting him for not giving up a bad view. %
\item He shouldn’t commit an offense similar to the offense for which the Sangha did the procedure of ejecting him for not giving up a bad view. %
\item He shouldn’t commit an offense worse than the offense for which the Sangha did the procedure of ejecting him for not giving up a bad view. %
\item He shouldn’t criticize the procedure. %
\item He shouldn’t criticize those who did the procedure. %
\item He shouldn’t consent to a regular monk bowing down to him. %
\item He shouldn’t consent to a regular monk standing up for him. %
\item He shouldn’t consent to a regular monk raising his joined palms to him. %
\item He shouldn’t consent to a regular monk doing appropriate duties for him. %
\item He shouldn’t consent to a regular monk offering him a seat. %
\item He shouldn’t consent to a regular monk offering him a bed. %
\item He shouldn’t consent to a regular monk offering him water for washing his feet and a foot-stool. %
\item He shouldn’t consent to a regular monk offering him a foot-scraper. %
\item He shouldn’t consent to a regular monk receiving his bowl and robe. %
\item He shouldn’t consent to a regular monk massaging his back when bathing. %
\item He shouldn’t charge a regular monk with failure in morality. %
\item He shouldn’t charge a regular monk with failure in conduct. %
\item He shouldn’t charge a regular monk with failure in view. %
\item He shouldn’t charge a regular monk with failure in livelihood. %
\item He shouldn’t cause division between monks. %
\item He shouldn’t wear lay clothes. %
\item He shouldn’t wear the robes of the monastics of other religions. %
\item He shouldn’t associate with the monastics of other religions. %
\item He should associate with monks. %
\item He should train in the monks’ training. %
\item He shouldn’t stay in the same room in a monastery as a regular monk. %
\item He shouldn’t stay in the same room in a non-monastery as a regular monk. %
\item He shouldn’t stay in the same room in a monastery or a non-monastery as a regular monk. %
\item He should get up from his seat when he sees a regular monk. %
\item He shouldn’t dismiss a regular monk, whether indoors or outdoors. %
\item He shouldn’t cancel the observance-day ceremony of a regular monk. %
\item He shouldn’t cancel the invitation ceremony of a regular monk. %
\item He shouldn’t direct a regular monk. %
\item He shouldn’t give instructions to a regular monk. %
\item He shouldn’t get permission from a regular monk to correct him. %
\item He shouldn’t accuse a regular monk of an offense. %
\item He shouldn’t remind a regular monk of an offense. %
\item He shouldn’t associate inappropriately with other monks.” %
\end{enumerate}

\scend{The forty-three kinds of conduct in regard to the legal procedure of ejection for not giving up a bad view are finished. }

Soon\marginnote{34.1.1} afterwards the Sangha did a legal procedure of ejecting \textsanskrit{Ariṭṭha} for not giving up a bad view, prohibiting him from living with the Sangha. He then disrobed. The monks of few desires complained and criticized him, “How could the monk \textsanskrit{Ariṭṭha} disrobe after the Sangha had done a legal procedure of ejecting him for not giving up a bad view?” The monks told the Buddha. 

Soon\marginnote{34.1.7} afterwards the Buddha had the Sangha gathered and questioned the monks: 

“Is\marginnote{34.1.8} it true, monks, that the monk \textsanskrit{Ariṭṭha} disrobed after the Sangha had done a procedure of ejecting him for not giving up a bad view?” 

“It’s\marginnote{34.1.9} true, sir.” 

The\marginnote{34.1.10} Buddha rebuked him … “It’s not suitable … How can that foolish man disrobe after the Sangha has done a procedure of ejecting him for not giving up a bad view? This will affect people’s confidence …” After rebuking him … he gave a teaching and addressed the monks: 

“Well\marginnote{34.1.16} then, the Sangha should lift that legal procedure of ejection for not giving up a bad view.”\footnote{All versions of the Pali available to me have this reading, which presumably is wrong. The expected reading is that the \textit{\textsanskrit{saṅghakamma}} should \emph{not} be lifted. \textsanskrit{Ariṭṭha} has returned to lay life. To lift the \textit{\textsanskrit{saṅghakamma}} done against him would only be possible if he were still a monk. Moreover, even if he were a monk, he would have to give up his view before the \textit{\textsanskrit{saṅghakamma}} could be lifted. Perhaps this should be understood as a general teaching, not directly related to \textsanskrit{Ariṭṭha}. } 

\subsection*{The group of forty-three on not to be lifted }

“When\marginnote{34.2.1} a monk has five qualities, a legal procedure of ejecting him for not giving up a bad view shouldn’t be lifted: he gives the full ordination; he gives formal support; he has a novice monk attend on him; he accepts being appointed as an instructor of the nuns; he instructs the nuns, whether appointed or not. 

When\marginnote{34.2.4} a monk has another five qualities, a procedure of ejecting him for not giving up a bad view shouldn’t be lifted: he commits the same offense for which the Sangha did the procedure of ejecting him for not giving up a bad view; he commits an offense similar to the one for which the Sangha did the procedure of ejecting him for not giving up a bad view; he commits an offense worse than the one for which the Sangha did the procedure of ejecting him for not giving up a bad view; he criticizes the procedure; he criticizes those who did the procedure. … he consents to a regular monk bowing down to him; he consents to a regular monk standing up for him; he consents to a regular monk raising his joined palms to him; he consents to a regular monk doing appropriate duties for him; he consents to a regular monk offering him a seat. … he consents to a regular monk offering him a bed; he consents to a regular monk offering him water for washing his feet and a foot stool; he consents to a regular monk offering him a foot scraper; he consents to a regular monk receiving his bowl and robe; he consents to a regular monk massaging his back when bathing. … he charges a regular monk with failure in morality; he charges a regular monk with failure in conduct; he charges a regular monk with failure in view; he charges a regular monk with failure in livelihood; he causes division between monks. … he wears lay clothes; he wears the robes of the monastics of other religions; he associates with the monastics of other religions; he doesn’t associate with monks; he doesn’t train in the monks’ training. … he stays in the same room in a monastery as a regular monk; he stays in the same room in a non-monastery as a regular monk; he stays in the same room in a monastery or a non-monastery as a regular monk; he doesn’t get up from his seat when he sees a regular monk; he dismisses a regular monk, whether indoors or outdoors. 

When\marginnote{34.2.9} a monk has eight qualities, a procedure of ejecting him for not giving up a bad view shouldn’t be lifted: he cancels the observance-day ceremony of a regular monk; he cancels the invitation ceremony of a regular monk; he directs a regular monk; he gives instructions to a regular monk; he gets permission from a regular monk to correct him; he accuses a regular monk of an offense; he reminds a regular monk of an offense; he associates inappropriately with other monks.” 

\scend{The group of forty-three on not to be lifted in regard to the legal procedure of ejection for not giving up a bad view is finished. }

\subsection*{The group of forty-three on to be lifted }

“When\marginnote{34.2.13.1} a monk has five qualities, a legal procedure of ejecting him for not giving up a bad view should be lifted: he doesn’t give the full ordination; he doesn’t give formal support; he doesn’t have a novice monk attend on him; he doesn’t accept being appointed as an instructor of the nuns; he doesn’t instruct the nuns, whether appointed or not. 

When\marginnote{34.2.16} a monk has another five qualities, a procedure of ejecting him for not giving up a bad view should be lifted: he doesn’t commit the same offense as the offense for which the Sangha did the procedure of ejecting him for not giving up a bad view; he doesn’t commit an offense similar to the one for which the Sangha did the procedure of ejecting him for not giving up a bad view; he doesn’t commit an offense worse than the one for which the Sangha did the procedure of ejecting him for not giving up a bad view; he doesn’t criticize the procedure; he doesn’t criticize those who did the procedure. … he doesn’t consent to a regular monk bowing down to him; he doesn’t consent to a regular monk standing up for him; he doesn’t consent to a regular monk raising his joined palms to him; he doesn’t consent to a regular monk doing appropriate duties for him; he doesn’t consent to a regular monk offering him a seat. … he doesn’t consent to a regular monk offering him a bed; he doesn’t consent to a regular monk offering him water for washing his feet and a foot stool; he doesn’t consent to a regular monk offering him a foot scraper; he doesn’t consent to a regular monk receiving his bowl and robe; he doesn’t consent to a regular monk massaging his back when bathing. … he doesn’t charge a regular monk with failure in morality; he doesn’t charge a regular monk with failure in conduct; he doesn’t charge a regular monk with failure in view; he doesn’t charge a regular monk with failure in livelihood; he doesn’t cause division between monks. … he doesn’t wear lay clothes; he doesn’t wear the robes of the monastics of other religions; he doesn’t associate with the monastics of other religions; he associates with monks; he trains in the monks’ training. … he doesn’t stay in the same room in a monastery as a regular monk; he doesn’t stay in the same room in a non-monastery as a regular monk; he doesn’t stay in the same room in a monastery or a non-monastery as a regular monk; he gets up from his seat when he sees a regular monk; he doesn’t dismiss a regular monk, whether indoors or outdoors. 

When\marginnote{34.2.21} a monk has eight qualities, a procedure of ejecting him for not giving up a bad view should be lifted: he doesn’t cancel the observance-day ceremony of a regular monk; he doesn’t cancel the invitation ceremony of a regular monk; he doesn’t direct a regular monk; he doesn’t give instructions to a regular monk; he doesn’t get permission from a regular monk to correct him; he doesn’t accuse a regular monk of an offense; he doesn’t remind a regular monk of an offense; he doesn’t associate inappropriately with other monks.” 

\scend{The group of forty-three on to be lifted in regard to the legal procedure of ejection for not giving up a bad view is finished. }

“And,\marginnote{35.1.1} monks, this is how it should be lifted. The monk who’s been ejected by the Sangha for not giving up a bad view should approach the Sangha, arrange his upper robe over one shoulder, pay respect at the feet of the senior monks, squat on his heels, raise his joined palms, and say, ‘Venerables, the Sangha has done a legal procedure of ejecting me for not giving up a bad view. I’m now conducting myself properly and suitably, and deserve to be released. I ask for that legal procedure to be lifted.’ And he should ask a second and a third time. A competent and capable monk should then inform the Sangha: 

‘Please,\marginnote{35.1.7} venerables, I ask the Sangha to listen. The Sangha has done a legal procedure of ejecting monk so-and-so for not giving up a bad view. He’s now conducting himself properly and suitably, and deserves to be released. He’s asking for that legal procedure to be lifted. If the Sangha is ready, it should lift that legal procedure of ejecting him for not giving up a bad view. This is the motion. 

Please,\marginnote{35.1.11} venerables, I ask the Sangha to listen. The Sangha has done a legal procedure of ejecting monk so-and-so for not giving up a bad view. He’s now conducting himself properly and suitably, and deserves to be released. He’s asking for that legal procedure to be lifted. The Sangha lifts that legal procedure of ejecting him for not giving up a bad view. Any monk who approves of lifting that legal procedure should remain silent. Any monk who doesn’t approve should speak up. 

For\marginnote{35.1.16} the second time, I speak on this matter. … For the third time, I speak on this matter. Please, venerables, I ask the Sangha to listen. The Sangha has done a legal procedure of ejecting monk so-and-so for not giving up a bad view. He’s now conducting himself properly and suitably, and deserves to be released. He’s asking for that legal procedure to be lifted. The Sangha lifts that legal procedure of ejecting him for not giving up a bad view. Any monk who approves of lifting that legal procedure should remain silent. Any monk who doesn’t approve should speak up. 

The\marginnote{35.1.23} Sangha has lifted that legal procedure of ejecting monk so-and-so for not giving up a bad view. The Sangha approves and is therefore silent. I’ll remember it thus.’” 

\scend{The seventh section on the legal procedure of ejection for not giving up a bad view is finished. }

\scendsutta{The first chapter on legal procedures is finished. In this chapter there are seven topics. }

\scuddanaintro{This is the summary: }

\begin{scuddana}%
“The\marginnote{35.1.28} monks \textsanskrit{Paṇḍu} and Lohitaka, \\
Themselves quarrelsome; \\
They went to those of the same kind, \\
And encouraged quarrels. 

They\marginnote{35.1.32} gave rise to new ones, \\
And worsened the existing ones; \\
The good monks of few desires, \\
Criticized. The One who Shows,\footnote{Reading \textit{padassaka} with the PTS edition. } 

The\marginnote{35.1.36} Buddha, standing in the True Dhamma, \\
Independent, the Supreme Person; \\
The Victor: at \textsanskrit{Sāvatthī} he ordered \\
The procedure of condemnation. 

In\marginnote{35.1.40} the absence, without questioning, \\
Without admission, and done against \\
One without offense, not clearable by confession, \\
Done against one who has confessed. 

Not\marginnote{35.1.44} having accused, not having reminded, \\
And done without having charged; \\
In the absence, with illegitimate, \\
And also done with an incomplete assembly. 

Without\marginnote{35.1.48} questioning, with illegitimate, \\
Again done with an incomplete assembly. \\
Without admission, with illegitimate, \\
And also done with an incomplete assembly. 

One\marginnote{35.1.52} without offense, with illegitimate, \\
And also done with an incomplete assembly. \\
Not clearable by confession, \\
And illegitimately, with an incomplete assembly. 

Against\marginnote{35.1.56} one who has confessed, with illegitimate, \\
And so also with incomplete assembly; \\
Not having accused, with illegitimate, \\
And so also with incomplete assembly; 

Not\marginnote{35.1.60} having reminded, with illegitimate, \\
And so also with incomplete assembly; \\
Not having charged, with illegitimate, \\
And so also with incomplete assembly. 

Just\marginnote{35.1.64} as the method of the dark section, \\
One should understand the bright section; \\
And the Sangha wishing, \\
Might do condemnation against one: 

Quarrelsome,\marginnote{35.1.68} ignorant, socializing, \\
In the higher morality, in the higher conduct; \\
For those failed in view, \\
The Sangha might do condemnation. 

And\marginnote{35.1.72} the Buddha, the Teaching, the Sangha, \\
Whoever dispraises them; \\
And against three kinds of monks, \\
The Sangha might carry out condemnation: 

The\marginnote{35.1.76} quarrelsome, \\
The ignorant, the one attached to socializing; \\
In the higher morality, in the higher conduct; \\
Just so about view. 

And\marginnote{35.1.80} the Buddha, the Teaching, the Sangha, \\
Whoever dispraises them; \\
Who has had a legal procedure of condemnation done against himself, \\
Should conduct himself properly thus: 

Full\marginnote{35.1.84} ordination, formal support, \\
A novice monk attending on; \\
Instruction, even if appointed, \\
He should not do. Against the one condemnation was done, 

He\marginnote{35.1.88} should not commit that offense, \\
Or one similar, or one beyond; \\
And the procedure, and also the doers, \\
He should not criticize those. 

The\marginnote{35.1.92} observance, the invitation, \\
He should not cancel for a regular monk; \\
Directing, instructing, \\
Permission, and with accusing. 

Reminding,\marginnote{35.1.96} and associating, \\
He should not do such things; \\
Full ordination, formal support, \\
A novice monk attending on. 

Instruction,\marginnote{35.1.100} even if appointed, \\
With five factors, it should not lift; \\
And should not commit that offense, \\
Or one similar, or one beyond. 

And\marginnote{35.1.104} the procedure, and also the doers, \\
Criticizing, it should not lift; \\
The observance, the invitation, \\
And directing, instructing. 

Permission,\marginnote{35.1.108} and accusing, \\
Reminding, associating; \\
Whoever is engaged in these eight factors, \\
The condemnation should not be lifted. 

Just\marginnote{35.1.112} as with the method of the dark section, \\
One should understand the bright section; \\
Ignorant, with many offenses, \\
And socializing, Seyyaso. 

The\marginnote{35.1.116} procedure of demotion was ordered, \\
By the Fully Awakened One, the Great Sage; \\
Two monks at \textsanskrit{Kīṭāgiri}, \\
Assaji and Punabbasuka. 

And\marginnote{35.1.120} many kinds of misconduct, \\
They did without restraint; \\
To be banished, the Fully Awakened One, \\
Procedure, at \textsanskrit{Sāvatthī}, the Victor; \\
Sudhamma at \textsanskrit{Macchikāsaṇḍa}, \\
Was staying with Citta. 

He\marginnote{35.1.126} abused as low status, \\
Sudhamma, the lay follower Citta; \\
The procedure of reconciliation, \\
The Buddha ordered. 

The\marginnote{35.1.130} monk Channa at \textsanskrit{Kosambī}, \\
Did not wish to see his offense; \\
To eject for not recognizing, \\
The Supreme Victor ordered. 

Channa\marginnote{35.1.134} that same offense, \\
Did not wish to make amends; \\
Ejection for not making amends, \\
The Leader ordered. 

The\marginnote{35.1.138} bad view of \textsanskrit{Ariṭṭha}, \\
Was attached to because of ignorance; \\
For not giving up a bad view, \\
Ejection was spoken of by the Victor. 

The\marginnote{35.1.142} procedure of demotion, banishment, \\
Just so reconciliation; \\
Not recognizing, not making amends, \\
And not giving up a view. 

Frivolous,\marginnote{35.1.146} misconduct, he hurts, \\
And just wrong livelihood; \\
For the procedure of banishment, \\
There are these extra lines. 

Stopping\marginnote{35.1.150} gain, disparaging, two, five, \\
They are called two sets of five; \\
For the procedure of reconciliation, \\
There are these extra lines. 

Condemnation,\marginnote{35.1.154} and demotion, \\
Are two procedures of the same kind; \\
Banishment, and reconciled, \\
There are extra lines. 

Three\marginnote{35.1.158} procedures of ejection, \\
Are analyzed in the same way; \\
As with the method of condemnation, \\
One should understand the rest of the procedures.” 

%
\end{scuddana}

\scendsutta{The chapter on legal procedures is finished. }

%
\chapter*{{\suttatitleacronym Kd 12}{\suttatitletranslation The chapter on those on probation }{\suttatitleroot Pārivāsikakkhandhaka}}
\addcontentsline{toc}{chapter}{\tocacronym{Kd 12} \toctranslation{The chapter on those on probation } \tocroot{Pārivāsikakkhandhaka}}
\markboth{The chapter on those on probation }{Pārivāsikakkhandhaka}
\extramarks{Kd 12}{Kd 12}

\section*{1. The proper conduct for those on probation }

At\marginnote{1.1.1} one time the Buddha was staying at \textsanskrit{Sāvatthī} in the Jeta Grove, \textsanskrit{Anāthapiṇḍika}’s Monastery. At that time monks on probation consented to regular monks bowing down to them, standing up for them, raising their joined palms to them, doing acts of respect toward them, offering them a seat, offering them a bed, offering them water for washing their feet and a foot stool, offering them a foot scraper, receiving their bowls and robes, and massaging their backs when bathing. The monks of few desires complained and criticized them, “How can monks on probation consent to these things?” They told the Buddha. 

Soon\marginnote{1.1.6} afterwards the Buddha had the Sangha gathered and questioned the monks: 

“Is\marginnote{1.1.7} it true, monks, that the monks on probation are consenting to these things?” 

“It’s\marginnote{1.1.8} true, sir.” 

The\marginnote{1.1.9} Buddha rebuked them … “It’s not suitable … How can monks on probation consent to these things? This will affect people’s confidence …” After rebuking them … he gave a teaching and addressed the monks: 

“A\marginnote{1.1.15} monk on probation shouldn’t consent to: 

\begin{itemize}%
\item Regular monks bowing down to him %
\item Regular monks standing up for him %
\item Regular monks raising their joined palms to him %
\item Regular monks doing acts of respect toward him %
\item Regular monks offering him a seat %
\item Regular monks offering him a bed %
\item Regular monks offering him water for washing his feet and a foot stool %
\item Regular monks offering him a foot scraper %
\item Regular monks receiving his bowl and robe %
\item Regular monks massaging his back when he’s bathing. %
\end{itemize}

\scrule{If he consents to any of these, he commits an offense of wrong conduct. }

Monks\marginnote{1.1.27} on probation should do the following with one another according to seniority: bow down, stand up, raise their joined palms, do acts of respect, offer a seat, offer a bed, offer water for washing the feet and a foot stool, offer a foot scraper, receive bowl and robe, and massage one another’s backs when bathing.\footnote{For the rendering “should” for \textit{\textsanskrit{anujānāmi}}, see Appendix of Technical Terms. } 

Monks\marginnote{1.1.28} on probation should do five things with regular monks according to seniority: the observance-day ceremony, the invitation ceremony, distributing rainy-season robes, meal invitations, and meals.\footnote{Sp 3.275: \textit{\textsanskrit{Oṇojananti} \textsanskrit{vissajjanaṁ} vuccati}, “\textit{\textsanskrit{Oṇojana}} means an offering.” } 

And\marginnote{1.1.30} now I will lay down the proper conduct for a monk on probation. This is the proper conduct: 

\begin{enumerate}%
\item He shouldn’t give the full ordination. %
\item He shouldn’t give formal support. %
\item He shouldn’t have a novice monk attend on him. %
\item He shouldn’t accept being appointed as an instructor of the nuns. %
\item Even if appointed, he shouldn’t instruct the nuns. %
\item He shouldn’t commit the same offense as the offense for which the Sangha gave him probation, nor one that’s similar or worse. %
\item He shouldn’t criticize the legal procedure. %
\item He shouldn’t criticize those who did the procedure. %
\item He shouldn’t cancel the observance-day ceremony of a regular monk. %
\item He shouldn’t cancel the invitation of a regular monk. %
\item He shouldn’t direct a regular monk.\footnote{Sp 4.76: \textit{Na \textsanskrit{savacanīyaṁ} \textsanskrit{kātabbanti} \textsanskrit{palibodhatthāya} \textsanskrit{vā} \textsanskrit{pakkosanatthāya} \textsanskrit{vā} \textsanskrit{savacanīyaṁ} na \textsanskrit{kātabbaṁ}, \textsanskrit{palibodhatthāya} hi karonto “\textsanskrit{ahaṁ} \textsanskrit{āyasmantaṁ} \textsanskrit{imasmiṁ} \textsanskrit{vatthusmiṁ} \textsanskrit{savacanīyaṁ} karomi, \textsanskrit{imamhā} \textsanskrit{āvāsā} ekapadampi \textsanskrit{mā} \textsanskrit{pakkāmi}, \textsanskrit{yāva} na \textsanskrit{taṁ} \textsanskrit{adhikaraṇaṁ} \textsanskrit{vūpasantaṁ} \textsanskrit{hotī}”ti \textsanskrit{evaṁ} karoti. \textsanskrit{Pakkosanatthāya} karonto “\textsanskrit{ahaṁ} te \textsanskrit{savacanīyaṁ} karomi, ehi \textsanskrit{mayā} \textsanskrit{saddhiṁ} \textsanskrit{vinayadharānaṁ} \textsanskrit{sammukhībhāvaṁ} \textsanskrit{gacchāmā}”ti \textsanskrit{evaṁ} karoti; tadubhayampi na \textsanskrit{kātabbaṁ}}, “\textit{Na \textsanskrit{savacanīyaṁ} \textsanskrit{kātabba}}: \textit{\textsanskrit{savacanīya}} is not to be done for the purpose of (creating) an obstacle or for the purpose of summoning. Acting for the purpose of (creating) an obstacle is done like this: ‘I am doing \textit{\textsanskrit{savacanīya}} against the venerable in regard to this offense: do not depart from this monastery even with one foot so long as this legal issue has not been resolved.’ Acting for the purpose of summoning is done like this: ‘I am doing \textit{\textsanskrit{savacanīya}} against you: come with me and let us go to the presence of a master of the Monastic Law.’ Neither of these is to be done.” Sp-\textsanskrit{ṭ} 4.76: \textit{\textsanskrit{Savacanīyanti} \textsanskrit{sadosaṁ}}, “\textit{\textsanskrit{Savacanīyan}}: with flaw.”  Vmv 4.76: \textit{\textsanskrit{Savacanīyanti} ettha “sadosa”nti \textsanskrit{atthaṁ} vadati. Attano vacane pavattanakammanti evamettha attho \textsanskrit{daṭṭhabbo}, “\textsanskrit{mā} \textsanskrit{pakkamāhī}”ti \textsanskrit{vā} “ehi \textsanskrit{vinayadharānaṁ} \textsanskrit{sammukhībhāva}”nti \textsanskrit{vā} \textsanskrit{evaṁ} attano \textsanskrit{āṇāya} \textsanskrit{pavattanakakammaṁ} na \textsanskrit{kātabbanti} \textsanskrit{adhippāyo}. \textsanskrit{Evañhi} kenaci \textsanskrit{savacanīye} kate \textsanskrit{anādarena} \textsanskrit{atikkamituṁ} na \textsanskrit{vaṭṭati}, buddhassa \textsanskrit{saṅghassa} \textsanskrit{āṇā} \textsanskrit{atikkantā} \textsanskrit{nāma} hoti}, “\textit{\textsanskrit{Savacanīyan}}: here he says the meaning is ‘with flaw’. Here the meaning is to be understood as bringing about an action when speaking oneself: ‘Don’t leave,’ ‘Come to the presence of master of the Monastic Law’. The intention is one is not to do the bringing about of an action in this way because of a command from oneself.” } %
\item He shouldn’t give instructions to a regular monk.\footnote{\textit{\textsanskrit{Anuvāda}} is not listed in CPD, and based on the commentarial interpretation, both DOP and PED have the wrong definition. Sp 4.76: \textit{Na \textsanskrit{anuvādoti} \textsanskrit{vihāre} \textsanskrit{jeṭṭhakaṭṭhānaṁ} na \textsanskrit{kātabbaṁ}. \textsanskrit{Pātimokkhuddesakena} \textsanskrit{vā} dhammajjhesakena \textsanskrit{vā} na \textsanskrit{bhavitabbaṁ}. \textsanskrit{Nāpi} terasasu \textsanskrit{sammutīsu} \textsanskrit{ekasammutivasenāpi} \textsanskrit{issariyakammaṁ} \textsanskrit{kātabbaṁ}}, “\textit{Na \textsanskrit{anuvādo}}: in a monastery, one is not to be put in a senior position. One should not be the reciter of the Monastic Code or the one who requests someone to speak on the Dhamma. One is not to be made an authority even on account of one agreement among the thirteen agreements.” The thirteen agreements are the various offices of storeman, etc., that Sangha members may hold, see Sp-\textsanskrit{ṭ} 1.69. Vjb 4.428: \textit{\textsanskrit{Anuvādanti} \textsanskrit{issariyaṭṭhānaṁ}}, “\textit{\textsanskrit{Anuvādo}}: a position of authority.” Sp-yoj 4.76: \textit{\textsanskrit{Anuvādoti} ettha \textsanskrit{anusāsanavasena} \textsanskrit{aññe} \textsanskrit{vadatīti}}, “\textit{\textsanskrit{Anuvādo}}: here one speaks to others on account of instructing (them).” The impression given here is that \textit{\textsanskrit{anuvāda}} means instructing others, especially from a position of authority. } %
\item He shouldn’t ask a regular monk for permission to accuse him of an offense. %
\item He shouldn’t accuse a regular monk of an offense. %
\item He shouldn’t remind a regular monk of an offense. %
\item He shouldn’t associate inappropriately with other monks. %
\item He shouldn’t walk in front of a regular monk. %
\item He shouldn’t sit in front of a regular monk. %
\item He should be given the last seat, the last bed, and the last dwelling of the Sangha, and he should consent to that. %
\item He shouldn’t attend on a regular monk when going to families. %
\item He shouldn’t be attended on by a regular monk when going to families. %
\item He shouldn’t undertake the practice of staying in the wilderness. %
\item He shouldn’t undertake the practice of eating only almsfood. %
\item He shouldn’t have someone bring back almsfood for him because he doesn’t want others to know about his status. %
\item He should inform about his status when he’s newly arrived in a monastery.  He should inform new arrivals of his status.  He should inform about his status on the observance day.  He should inform about his status on the invitation day. If he’s sick, he should inform about his status by messenger. %
\item He shouldn’t go from a monastery with monks to one without monks, except together with a regular monk or if there are dangers. He shouldn’t go from a monastery with monks to a non-monastery without monks, except together with a regular monk or if there are dangers. He shouldn’t go from a monastery with monks to a monastery or non-monastery without monks, except together with a regular monk or if there are dangers. %
\item He shouldn’t go from a non-monastery with monks to a monastery without monks, except together with a regular monk or if there are dangers. He shouldn’t go from a non-monastery with monks to a non-monastery without monks, except together with a regular monk or if there are dangers. He shouldn’t go from a non-monastery with monks to a monastery or non-monastery without monks, except together with a regular monk or if there are dangers. %
\item He shouldn’t go from a monastery or non-monastery with monks to a monastery without monks, except together with a regular monk or if there are dangers. He shouldn’t go from a monastery or non-monastery with monks to a non-monastery without monks, except together with a regular monk or if there are dangers. He shouldn’t go from a monastery or non-monastery with monks to a monastery or non-monastery without monks, except together with a regular monk or if there are dangers. %
\item He shouldn’t go from a monastery with monks to one with monks who belong to a different Buddhist sect, except together with a regular monk or if there are dangers. He shouldn’t go from a monastery with monks to a non-monastery with monks who belong to a different Buddhist sect, except together with a regular monk or if there are dangers. He shouldn’t go from a monastery with monks to a monastery or non-monastery with monks who belong to a different Buddhist sect, except together with a regular monk or if there are dangers. %
\item He shouldn’t go from a non-monastery with monks to a monastery with monks who belong to a different Buddhist sect, except together with a regular monk or if there are dangers. He shouldn’t go from a non-monastery with monks to a non-monastery with monks who belong to a different Buddhist sect, except together with a regular monk or if there are dangers. He shouldn’t go from a non-monastery with monks to a monastery or non-monastery with monks who belong to a different Buddhist sect, except together with a regular monk or if there are dangers. %
\item He shouldn’t go from a monastery or non-monastery with monks to a monastery with monks who belong to a different Buddhist sect, except together with a regular monk or if there are dangers. He shouldn’t go from a monastery or non-monastery with monks to a non-monastery with monks who belong to a different Buddhist sect, except together with a regular monk or if there are dangers. He shouldn’t go from a monastery or non-monastery with monks to a monastery or non-monastery with monks who belong to a different Buddhist sect, except together with a regular monk or if there are dangers. %
\item He may go from a monastery with monks to one with monks who belong to the same Buddhist sect if he knows he’ll be able to arrive on the same day. He may go from a monastery with monks to a non-monastery with monks who belong to the same Buddhist sect if he knows he’ll be able to arrive on the same day. He may go from a monastery with monks to a monastery or non-monastery with monks who belong to the same Buddhist sect if he knows he’ll be able to arrive on the same day. %
\item He may go from a non-monastery with monks to a monastery with monks who belong to the same Buddhist sect if he knows he’ll be able to arrive on the same day. He may go from a non-monastery with monks to a non-monastery with monks who belong to the same Buddhist sect if he knows he’ll be able to arrive on the same day. He may go from a non-monastery with monks to a monastery or non-monastery with monks who belong to the same Buddhist sect if he knows he’ll be able to arrive on the same day. %
\item He may go from a monastery or non-monastery with monks to a monastery with monks who belong to the same Buddhist sect if he knows he’ll be able to arrive on the same day. He may go from a monastery or non-monastery with monks to a non-monastery with monks who belong to the same Buddhist sect if he knows he’ll be able to arrive on the same day. He may go from a monastery or non-monastery with monks to a monastery or non-monastery with monks who belong to the same Buddhist sect if he knows he’ll be able to arrive on the same day. %
\item He shouldn’t, in a monastery, stay in the same room as a regular monk. %
\item He shouldn’t, in a non-monastery, stay in the same room as a regular monk. %
\item He shouldn’t, in a monastery or a non-monastery, stay in the same room as a regular monk. %
\item If he sees a regular monk, he should get up from his seat. %
\item He should offer a seat to a regular monk. %
\item He shouldn’t sit on the same seat as a regular monk. %
\item He shouldn’t sit on a higher seat than a regular monk. %
\item He shouldn’t sit on a seat when a regular monk is sitting on the ground. %
\item He shouldn’t do walking meditation on the same walking path as a regular monk. %
\item He shouldn’t do walking meditation on a higher walking path than a regular monk. %
\item He shouldn’t do walking meditation on a walking path when a regular monk is walking on the ground. %
\item He shouldn’t, in a monastery, stay in the same room as a more senior monk on probation. … %
\item[55.] He shouldn’t, in a monastery, stay in the same room as a monk deserving to be sent back to the beginning. … %
\item[64.] He shouldn’t, in a monastery, stay in the same room as a monk deserving the trial period. … %
\item[73.] He shouldn’t, in a monastery, stay in the same room as a monk undertaking the trial period. … %
\item[82.] He shouldn’t, in a monastery, stay in the same room as a monk deserving rehabilitation. %
\item[83.] He shouldn’t, in a non-monastery, stay in the same room as a monk deserving rehabilitation. %
\item[84.] He shouldn’t, in a monastery or a non-monastery, stay in the same room as a monk deserving rehabilitation. %
\item[85.] He shouldn’t sit on the same seat as a monk deserving rehabilitation. %
\item[86.] He shouldn’t sit on a higher seat than a monk deserving rehabilitation. %
\item[87.] He shouldn’t sit on a seat when a monk deserving rehabilitation is sitting on the ground. %
\item[88.] He shouldn’t do walking meditation on the same walking path as a monk deserving rehabilitation. %
\item[89.] He shouldn’t do walking meditation on a higher walking path than a monk deserving rehabilitation. %
\item[90.] He shouldn’t do walking meditation on a walking path when a monk deserving rehabilitation is walking on the ground. %
\item[91.] If, as the fourth member of a group, he gives probation, %
\item[92.] sends back to the beginning, %
\item[93.] or gives the trial period, %
\item[94.] or as the twentieth member of a group, he rehabilitates, it’s invalid and not to be done.” %
\end{enumerate}

\scend{The ninety-four kinds of proper conduct for one on probation are finished. }

\subsection*{Further regulations for probation}

Soon\marginnote{2.1.1} afterwards Venerable \textsanskrit{Upāli} went to the Buddha, bowed, sat down, and said, “How many things are there, sir, that stop a monk on probation from counting a particular day toward his probationary period?” 

\scrule{“There are three such things, \textsanskrit{Upāli}: he stays in the same room as a regular monk; he stays apart from other monks; he doesn’t inform other monks of his status.” }

On\marginnote{3.1.1} one occasion a large sangha of monks had gathered at \textsanskrit{Sāvatthī}. The monks on probation were unable to fulfill their probationary duties. They told the Buddha. 

\scrule{“I allow you to set aside the probation. }

And\marginnote{3.1.5} it should be done like this. The monk on probation should approach a monk, arrange his upper robe over one shoulder, squat on his heels, raise his joined palms, and say, ‘I set aside the probation,’ or ‘I set aside the proper conduct.’” 

Soon\marginnote{3.2.1} afterwards the monks at \textsanskrit{Sāvatthī} left for various destinations. The monks on probation were once again able to fulfill their probationary duties. They told the Buddha. 

\scrule{“I allow you to take up the probation. }

And\marginnote{3.2.5} it should be done like this. The monk on probation should approach a monk, arrange his upper robe over one shoulder, squat on his heels, raise his joined palms, and say, ‘I take up the probation,’ or ‘I take up the proper conduct.’” 

\scend{The proper conduct for those on probation is finished. }

\section*{2. The proper conduct for those deserving to be sent back to the beginning }

At\marginnote{4.1.1} this time monks deserving to be sent back to the beginning consented to regular monks bowing down to them, standing up for them, raising their joined palms to them, doing acts of respect toward them, offering them a seat, offering them a bed, offering them water for washing their feet and a foot stool, offering them a foot scraper, receiving their bowls and robes, and massaging their backs when bathing. The monks of few desires complained and criticized them, “How can monks deserving to be sent back to the beginning consent to these things?” The monks told the Buddha. 

Soon\marginnote{4.1.5} afterwards the Buddha had the Sangha gathered and questioned the monks: 

“Is\marginnote{4.1.6} it true, monks, that monks deserving to be sent back to the beginning consent to these things?” 

“It’s\marginnote{4.1.7} true, sir.” 

The\marginnote{4.1.8} Buddha rebuked them … “It’s not suitable … How can monks deserving to be sent back to the beginning consent to these things? This will affect people’s confidence …” After rebuking them … he gave a teaching and addressed the monks: 

“A\marginnote{4.1.14} monk deserving to be sent back to the beginning shouldn’t consent to: 

\begin{itemize}%
\item Regular monks bowing down to him %
\item Regular monks standing up for him %
\item Regular monks raising their joined palms to him %
\item Regular monks doing acts of respect toward him %
\item Regular monks offering him a seat %
\item Regular monks offering him a bed %
\item Regular monks offering him water for washing his feet and a foot stool %
\item Regular monks offering him a foot scraper %
\item Regular monks receiving his bowl and robe %
\item Regular monks massaging his back when he’s bathing. %
\end{itemize}

\scrule{If he consents to any of these, he commits an offense of wrong conduct. }

Monks\marginnote{4.1.26} deserving to be sent back to the beginning should do the following with one another according to seniority: bow down, stand up, raise their joined palms, do acts of respect, offer a seat, offer a bed, offer water for washing the feet and a foot stool, offer a foot scraper, receive bowl and robe, and massage one another’s backs when bathing. 

Monks\marginnote{4.1.27} deserving to be sent back to the beginning should do five things with regular monks according to seniority: the observance-day ceremony, the invitation ceremony, distributing rainy-season robes, meal invitations, and meals. 

And\marginnote{4.1.29} now I will lay down the proper conduct for a monk deserving to be sent back to the beginning. This is the proper conduct: 

\begin{enumerate}%
\item He shouldn’t give the full ordination. %
\item He shouldn’t give formal support. %
\item He shouldn’t have a novice monk attend on him. %
\item He shouldn’t accept being appointed as an instructor of the nuns. %
\item Even if appointed, he shouldn’t instruct the nuns. %
\item He shouldn’t commit the same offense as the offense for which he deserves to be sent back to the beginning by the Sangha, nor one that is similar or worse. %
\item He shouldn’t criticize the legal procedure. %
\item He shouldn’t criticize those who did the procedure. %
\item He shouldn’t cancel the observance-day ceremony of a regular monk. %
\item He shouldn’t cancel the invitation of a regular monk. %
\item He shouldn’t direct a regular monk. %
\item He shouldn’t give instructions to a regular monk. %
\item He shouldn’t ask a regular monk for permission to accuse him of an offense. %
\item He shouldn’t accuse a regular monk of an offense. %
\item He shouldn’t remind a regular monk of an offense. %
\item He shouldn’t associate inappropriately with other monks. %
\item He shouldn’t walk in front of a regular monk. %
\item He shouldn’t sit in front of a regular monk. %
\item He should be given the last seat, the last bed, and the last dwelling of the Sangha, and he should consent to that. %
\item He shouldn’t attend on a regular monk when going to families. %
\item He shouldn’t be attended on by a regular monk when going to families. %
\item He shouldn’t undertake the practice of staying in the wilderness. %
\item He shouldn’t undertake the practice of eating only almsfood. %
\item He shouldn’t have someone bring back almsfood for him because he doesn’t want others to know about his status. %
\item He shouldn’t go from a monastery with monks to one without monks, except together with a regular monk or if there are dangers. He shouldn’t go from a monastery with monks to a non-monastery without monks, except together with a regular monk or if there are dangers. He shouldn’t go from a monastery with monks to a monastery or non-monastery without monks, except together with a regular monk or if there are dangers. %
\item He shouldn’t go from a non-monastery with monks to a monastery without monks, except together with a regular monk or if there are dangers. He shouldn’t go from a non-monastery with monks to a non-monastery without monks, except together with a regular monk or if there are dangers. He shouldn’t go from a non-monastery with monks to a monastery or non-monastery without monks, except together with a regular monk or if there are dangers. %
\item He shouldn’t go from a monastery or non-monastery with monks to a monastery without monks, except together with a regular monk or if there are dangers. He shouldn’t go from a monastery or non-monastery with monks to a non-monastery without monks, except together with a regular monk or if there are dangers. He shouldn’t go from a monastery or non-monastery with monks to a monastery or non-monastery without monks, except together with a regular monk or if there are dangers. %
\item He shouldn’t go from a monastery with monks to one with monks who belong to a different Buddhist sect, except together with a regular monk or if there are dangers. He shouldn’t go from a monastery with monks to a non-monastery with monks who belong to a different Buddhist sect, except together with a regular monk or if there are dangers. He shouldn’t go from a monastery with monks to a monastery or non-monastery with monks who belong to a different Buddhist sect, except together with a regular monk or if there are dangers. %
\item He shouldn’t go from a non-monastery with monks to a monastery with monks who belong to a different Buddhist sect, except together with a regular monk or if there are dangers. He shouldn’t go from a non-monastery with monks to a non-monastery with monks who belong to a different Buddhist sect, except together with a regular monk or if there are dangers. He shouldn’t go from a non-monastery with monks to a monastery or non-monastery with monks who belong to a different Buddhist sect, except together with a regular monk or if there are dangers. %
\item He shouldn’t go from a monastery or non-monastery with monks to a monastery with monks who belong to a different Buddhist sect, except together with a regular monk or if there are dangers. He shouldn’t go from a monastery or non-monastery with monks to a non-monastery with monks who belong to a different Buddhist sect, except together with a regular monk or if there are dangers. He shouldn’t go from a monastery or non-monastery with monks to a monastery or non-monastery with monks who belong to a different Buddhist sect, except together with a regular monk or if there are dangers. %
\item He may go from a monastery with monks to one with monks who belong to the same Buddhist sect if he knows he’ll be able to arrive on the same day. He may go from a monastery with monks to a non-monastery with monks who belong to the same Buddhist sect if he knows he’ll be able to arrive on the same day. He may go from a monastery with monks to a monastery or non-monastery with monks who belong to the same Buddhist sect if he knows he’ll be able to arrive on the same day. %
\item He may go from a non-monastery with monks to a monastery with monks who belong to the same Buddhist sect if he knows he’ll be able to arrive on the same day. He may go from a non-monastery with monks to a non-monastery with monks who belong to the same Buddhist sect if he knows he’ll be able to arrive on the same day. %
\item He may go from a non-monastery with monks to a monastery or non-monastery with monks who belong to the same Buddhist sect if he knows he’ll be able to arrive on the same day. %
\item He may go from a monastery or non-monastery with monks to a monastery with monks who belong to the same Buddhist sect if he knows he’ll be able to arrive on the same day. He may go from a monastery or non-monastery with monks to a non-monastery with monks who belong to the same Buddhist sect if he knows he’ll be able to arrive on the same day. He may go from a monastery or non-monastery with monks to a monastery or non-monastery with monks who belong to the same Buddhist sect if he knows he’ll be able to arrive on the same day. %
\item He shouldn’t, in a monastery, stay in the same room as a regular monk. %
\item He shouldn’t, in a non-monastery, stay in the same room as a regular monk. %
\item He shouldn’t, in a monastery or a non-monastery, stay in the same room as a regular monk. %
\item If he sees a regular monk, he should get up from his seat. %
\item He should offer a seat to a regular monk. %
\item He shouldn’t sit on the same seat as a regular monk. %
\item He shouldn’t sit on a higher seat than a regular monk. %
\item He shouldn’t sit on a seat when a regular monk is sitting on the ground %
\item He shouldn’t do walking meditation on the same walking path as a regular monk. %
\item He shouldn’t do walking meditation on a higher walking path than a regular monk. %
\item He shouldn’t do walking meditation on a walking path when a regular monk is walking on the ground. %
\item He shouldn’t, in a monastery, stay in the same room as a monk on probation. … %
\item[54.] He shouldn’t, in a monastery, stay in the same room as a more senior monk deserving to be sent back to the beginning. … %
\item[63.] He shouldn’t, in a monastery, stay in the same room as a monk deserving the trial period. … %
\item[72.] He shouldn’t, in a monastery, stay in the same room as a monk undertaking the trial period. … %
\item[81.] He shouldn’t, in a monastery, stay in the same room as a monk deserving rehabilitation. %
\item[82.] He shouldn’t, in a non-monastery, stay in the same room as a monk deserving rehabilitation. %
\item[83.] He shouldn’t, in a monastery or a non-monastery, stay in the same room as a monk deserving rehabilitation. %
\item[84.] He shouldn’t sit on the same seat as a monk deserving rehabilitation. %
\item[85.] He shouldn’t sit on a higher seat than a monk deserving rehabilitation. %
\item[86.] He shouldn’t sit on a seat when a monk deserving rehabilitation is sitting on the ground. %
\item[87.] He shouldn’t do walking meditation on the same walking path as a monk deserving rehabilitation. %
\item[88.] He shouldn’t do walking meditation on a higher walking path than a monk deserving rehabilitation. %
\item[89.] He shouldn’t do walking meditation on a walking path when a monk deserving rehabilitation is walking on the ground. %
\item[90.] If, as the fourth member of a group, he gives probation, %
\item[91.] sends back to the beginning, %
\item[92.] or gives the trial period, %
\item[93.] or as the twentieth member of a group, he rehabilitates, it’s invalid and not to be done.” %
\end{enumerate}

\scend{The proper conduct for those deserving to be sent back to the beginning is finished. }

\section*{3. The proper conduct for those deserving the trial period }

At\marginnote{5.1.1} this time monks deserving the trial period consented to regular monks bowing down to them, standing up for them, raising their joined palms to them, doing acts of respect toward them, offering them a seat, offering them a bed, offering them water for washing their feet and a foot stool, offering them a foot scraper, receiving their bowls and robes, and massaging their backs when bathing. The monks of few desires complained and criticized them, “How can monks deserving the trial period consent to these things?” The monks told the Buddha. Soon afterwards he had the Sangha of monks gathered and questioned them: 

“Is\marginnote{5.1.5} it true, monks, that monks deserving the trial period consent to these things?” 

“It’s\marginnote{5.1.6} true, sir.” 

The\marginnote{5.1.7} Buddha rebuked them … “It’s not suitable … How can monks deserving the trial period consent to these things? This will affect people’s confidence …” After rebuking them … he gave a teaching and addressed the monks: 

“A\marginnote{5.1.13} monk deserving the trial period shouldn’t consent to: 

\begin{itemize}%
\item Regular monks bowing down to him %
\item Regular monks standing up for him %
\item Regular monks raising their joined palms to him %
\item Regular monks doing acts of respect toward him %
\item Regular monks offering him a seat %
\item Regular monks offering him a bed %
\item Regular monks offering him water for washing his feet and a foot-stool %
\item Regular monks offering him a foot-scraper %
\item Regular monks receiving his bowl and robe %
\item Regular monks massaging his back when he’s bathing. %
\end{itemize}

\scrule{If he consents to any of these, he commits an offense of wrong conduct. }

Monks\marginnote{5.1.19} deserving the trial period should do the following with one another according to seniority: bow down, stand up, raise their joined palms, do acts of respect, offer a seat, offer a bed, offer water for washing the feet and a foot stool, offer a foot scraper, receive bowl and robe, and massage one another’s backs when bathing. 

Monks\marginnote{5.1.20} deserving the trial period should do five things with regular monks according to seniority: the observance-day ceremony, the invitation ceremony, distributing rainy-season robes, meal invitations, and meals. 

And\marginnote{5.1.22} now I will lay down the proper conduct for a monk deserving the trial period. This is the proper conduct: 

\begin{enumerate}%
\item He shouldn’t give the full ordination. %
\item He shouldn’t give formal support. %
\item He shouldn’t have a novice monk attend on him. %
\item He shouldn’t accept being appointed as an instructor of the nuns. %
\item Even if appointed, he shouldn’t instruct the nuns. %
\item He shouldn’t commit the same offense as the offense for which he deserves the trial period by the Sangha, nor one that is similar or worse. %
\item He shouldn’t criticize the legal procedure. %
\item He shouldn’t criticize those who did the procedure. %
\item He shouldn’t cancel the observance-day ceremony of a regular monk. %
\item He shouldn’t cancel the invitation of a regular monk. %
\item He shouldn’t direct a regular monk. %
\item He shouldn’t give instructions to a regular monk. %
\item He shouldn’t ask a regular monk for permission to accuse him of an offense. %
\item He shouldn’t accuse a regular monk of an offense. %
\item He shouldn’t remind a regular monk of an offense. %
\item He shouldn’t associate inappropriately with other monks. %
\item He shouldn’t walk in front of a regular monk. %
\item He shouldn’t sit in front of a regular monk. %
\item He should be given the last seat, the last bed, and the last dwelling of the Sangha, and he should consent to that. %
\item He shouldn’t attend on a regular monk when going to families. %
\item He shouldn’t be attended on by a regular monk when going to families. %
\item He shouldn’t undertake the practice of staying in the wilderness. %
\item He shouldn’t undertake the practice of eating only almsfood. %
\item He shouldn’t have someone bring back almsfood for him because he doesn’t want others to know about his status. %
\item He shouldn’t go from a monastery with monks to one without monks, except together with a regular monk or if there are dangers. He shouldn’t go from a monastery with monks to a non-monastery without monks, except together with a regular monk or if there are dangers. He shouldn’t go from a monastery with monks to a monastery or non-monastery without monks, except together with a regular monk or if there are dangers. %
\item He shouldn’t go from a non-monastery with monks to a monastery without monks, except together with a regular monk or if there are dangers. He shouldn’t go from a non-monastery with monks to a non-monastery without monks, except together with a regular monk or if there are dangers. He shouldn’t go from a non-monastery with monks to a monastery or non-monastery without monks, except together with a regular monk or if there are dangers. %
\item He shouldn’t go from a monastery or non-monastery with monks to a monastery without monks, except together with a regular monk or if there are dangers. He shouldn’t go from a monastery or non-monastery with monks to a non-monastery without monks, except together with a regular monk or if there are dangers. He shouldn’t go from a monastery or non-monastery with monks to a monastery or non-monastery without monks, except together with a regular monk or if there are dangers. %
\item He shouldn’t go from a monastery with monks to one with monks who belong to a different Buddhist sect, except together with a regular monk or if there are dangers. He shouldn’t go from a monastery with monks to a non-monastery with monks who belong to a different Buddhist sect, except together with a regular monk or if there are dangers. He shouldn’t go from a monastery with monks to a monastery or non-monastery with monks who belong to a different Buddhist sect, except together with a regular monk or if there are dangers. %
\item He shouldn’t go from a non-monastery with monks to a monastery with monks who belong to a different Buddhist sect, except together with a regular monk or if there are dangers. He shouldn’t go from a non-monastery with monks to a non-monastery with monks who belong to a different Buddhist sect, except together with a regular monk or if there are dangers. He shouldn’t go from a non-monastery with monks to a monastery or non-monastery with monks who belong to a different Buddhist sect, except together with a regular monk or if there are dangers. %
\item He shouldn’t go from a monastery or non-monastery with monks to a monastery with monks who belong to a different Buddhist sect, except together with a regular monk or if there are dangers. He shouldn’t go from a monastery or non-monastery with monks to a non-monastery with monks who belong to a different Buddhist sect, except together with a regular monk or if there are dangers. He shouldn’t go from a monastery or non-monastery with monks to a monastery or non-monastery with monks who belong to a different Buddhist sect, except together with a regular monk or if there are dangers. %
\item He may go from a monastery with monks to one with monks who belong to the same Buddhist sect if he knows he’ll be able to arrive on the same day. He may go from a monastery with monks to a non-monastery with monks who belong to the same Buddhist sect if he knows he’ll be able to arrive on the same day.  He may go from a monastery with monks to a monastery or non-monastery with monks who belong to the same Buddhist sect if he knows he’ll be able to arrive on the same day. %
\item He may go from a non-monastery with monks to a monastery with monks who belong to the same Buddhist sect if he knows he’ll be able to arrive on the same day. He may go from a non-monastery with monks to a non-monastery with monks who belong to the same Buddhist sect if he knows he’ll be able to arrive on the same day. He may go from a non-monastery with monks to a monastery or non-monastery with monks who belong to the same Buddhist sect if he knows he’ll be able to arrive on the same day. %
\item He may go from a monastery or non-monastery with monks to a monastery with monks who belong to the same Buddhist sect if he knows he’ll be able to arrive on the same day. He may go from a monastery or non-monastery with monks to a non-monastery with monks who belong to the same Buddhist sect if he knows he’ll be able to arrive on the same day. He may go from a monastery or non-monastery with monks to a monastery or non-monastery with monks who belong to the same Buddhist sect if he knows he’ll be able to arrive on the same day. %
\item He shouldn’t, in a monastery, stay in the same room as a regular monk. %
\item He shouldn’t, in a non-monastery, stay in the same room as a regular monk. %
\item He shouldn’t, in a monastery or a non-monastery, stay in the same room as a regular monk. %
\item If he sees a regular monk, he should get up from his seat. %
\item He should offer a seat to a regular monk. %
\item He shouldn’t sit on the same seat as a regular monk. %
\item He shouldn’t sit on a higher seat than a regular monk. %
\item He shouldn’t sit on a seat when a regular monk is sitting on the ground. %
\item He shouldn’t do walking meditation on the same walking path as a regular monk. %
\item He shouldn’t do walking meditation on a higher walking path than a regular monk. %
\item He shouldn’t do walking meditation on a walking path when a regular monk is walking on the ground. %
\item He shouldn’t, in a monastery, stay in the same room as a monk on probation. … %
\item[54.] He shouldn’t, in a monastery, stay in the same room as a monk deserving to be sent back to the beginning. … %
\item[63.] He shouldn’t, in a monastery, stay in the same room as a more senior monk deserving the trial period. … %
\item[72.] He shouldn’t, in a monastery, stay in the same room as a monk undertaking the trial period. … %
\item[81.] He shouldn’t, in a monastery, stay in the same room as a monk deserving rehabilitation. %
\item[82.] He shouldn’t, in a non-monastery, stay in the same room as a monk deserving rehabilitation. %
\item[83.] He shouldn’t, in a monastery or a non-monastery, stay in the same room as a monk deserving rehabilitation. %
\item[84.] He shouldn’t sit on the same seat as a monk deserving rehabilitation. %
\item[85.] He shouldn’t sit on a higher seat than a monk deserving rehabilitation. %
\item[86.] He shouldn’t sit on a seat when a monk deserving rehabilitation is sitting on the ground. %
\item[87.] He shouldn’t do walking meditation on the same walking path as a monk deserving rehabilitation. %
\item[88.] He shouldn’t do walking meditation on a higher walking path than a monk deserving rehabilitation. %
\item[89.] He shouldn’t do walking meditation on a walking path when a monk deserving rehabilitation is walking on the ground. %
\item[90.] If, as the fourth member of a group, he gives probation, %
\item[91.] sends back to the beginning, %
\item[92.] or gives the trial period, %
\item[93.] or as the twentieth member of a group, he rehabilitates, it’s invalid and not to be done.” %
\end{enumerate}

\scend{The proper conduct for those deserving the trial period is finished. }

\section*{4. The proper conduct for those undertaking the trial period }

At\marginnote{6.1.1} this time monks undertaking the trial period consented to regular monks bowing down to them, standing up for them, raising their joined palms to them, doing acts of respect toward them, offering them a seat, offering them a bed, offering them water for washing their feet and a foot stool, offering them a foot scraper, receiving their bowls and robes, and massaging their backs when bathing. The monks of few desires complained and criticized them, “How can monks undertaking the trial period consent to these things?” The monks told the Buddha. 

Soon\marginnote{6.1.5} afterwards the Buddha had the Sangha gathered and questioned the monks: 

“Is\marginnote{6.1.6} it true, monks, that monks undertaking the trial period consent to these things?” 

“It’s\marginnote{6.1.7} true, sir.” 

The\marginnote{6.1.8} Buddha rebuked them … “It’s not suitable … How can monks undertaking the trial period consent to these things? This will affect people’s confidence …” After rebuking them … he gave a teaching and addressed the monks: 

“A\marginnote{6.1.14} monk undertaking the trial period shouldn’t consent to: 

\begin{itemize}%
\item Regular monks bowing down to him %
\item Regular monks standing up for him %
\item Regular monks raising their joined palms to him %
\item Regular monks doing acts of respect toward him %
\item Regular monks offering him a seat %
\item Regular monks offering him a bed %
\item Regular monks offering him water for washing his feet and a foot stool %
\item Regular monks offering him a foot scraper %
\item Regular monks receiving his bowl and robe %
\item Regular monks massaging his back when he’s bathing. %
\end{itemize}

\scrule{If he consents to any of these, he commits an offense of wrong conduct. }

Monks\marginnote{6.1.26} undertaking the trial period should do the following with one another according to seniority: bow down, stand up, raise their joined palms, do acts of respect, offer a seat, offer a bed, offer water for washing the feet and a foot stool, offer a foot scraper, receive bowl and robe, and massage one another’s backs when bathing. 

Monks\marginnote{6.1.27} undertaking the trial period should do five things with regular monks according to seniority: the observance-day ceremony, the invitation ceremony, distributing rainy-season robes, meal invitations, and meals. 

And\marginnote{6.1.29} now I will lay down the proper conduct for a monk undertaking the trial period. This is the proper conduct: 

\begin{enumerate}%
\item He shouldn’t give the full ordination. %
\item He shouldn’t give formal support. %
\item He shouldn’t have a novice monk attend on him. %
\item He shouldn’t accept being appointed as an instructor of the nuns. %
\item Even if appointed, he shouldn’t instruct the nuns. %
\item He shouldn’t commit the same offense as the offense for which the Sangha gave him the trial period, nor one that is similar or worse. %
\item He shouldn’t criticize the legal procedure. %
\item He shouldn’t criticize those who did the procedure. %
\item He shouldn’t cancel the observance-day ceremony of a regular monk. %
\item He shouldn’t cancel the invitation of a regular monk. %
\item He shouldn’t direct a regular monk. %
\item He shouldn’t give instructions to a regular monk. %
\item He shouldn’t ask a regular monk for permission to accuse him of an offense. %
\item He shouldn’t accuse a regular monk of an offense. %
\item He shouldn’t remind a regular monk of an offense. %
\item He shouldn’t associate inappropriately with other monks. %
\item He shouldn’t walk in front of a regular monk. %
\item He shouldn’t sit in front of a regular monk. %
\item He should be given the last seat, the last bed, and the last dwelling of the Sangha, and he should consent to that. %
\item He shouldn’t attend on a regular monk when going to families. %
\item He shouldn’t be attended on by a regular monk when going to families. %
\item He shouldn’t undertake the practice of staying in the wilderness. %
\item He shouldn’t undertake the practice of eating only almsfood. %
\item He shouldn’t have someone bring back almsfood for him because he doesn’t want others to know about his status. %
\item He should inform about his status when he’s newly arrived in a monastery.  He should inform new arrivals of his status.  He should inform about his status on the observance day.  He should inform about his status on the invitation day.  He should inform about his status on a daily basis. If he’s sick, he should inform about his status by messenger. %
\item He shouldn’t go from a monastery with monks to one without monks, except together with a sangha or if there are dangers. He shouldn’t go from a monastery with monks to a non-monastery without monks, except together with a sangha or if there are dangers. He shouldn’t go from a monastery with monks to a monastery or non-monastery without monks, except together with a sangha or if there are dangers. %
\item He shouldn’t go from a non-monastery with monks to a monastery without monks, except together with a sangha or if there are dangers. He shouldn’t go from a non-monastery with monks to a non-monastery without monks, except together with a sangha or if there are dangers. He shouldn’t go from a non-monastery with monks to a monastery or non-monastery without monks, except together with a sangha or if there are dangers. %
\item He shouldn’t go from a monastery or non-monastery with monks to a monastery without monks, except together with a sangha or if there are dangers. He shouldn’t go from a monastery or non-monastery with monks to a non-monastery without monks, except together with a sangha or if there are dangers. He shouldn’t go from a monastery or non-monastery with monks to a monastery or non-monastery without monks, except together with a sangha or if there are dangers. %
\item He shouldn’t go from a monastery with monks to one with monks who belong to a different Buddhist sect, except together with a sangha or if there are dangers. He shouldn’t go from a monastery with monks to a non-monastery with monks who belong to a different Buddhist sect, except together with a sangha or if there are dangers. He shouldn’t go from a monastery with monks to a monastery or non-monastery with monks who belong to a different Buddhist sect, except together with a sangha or if there are dangers. %
\item He shouldn’t go from a non-monastery with monks to a monastery with monks who belong to a different Buddhist sect, except together with a sangha or if there are dangers.  He shouldn’t go from a non-monastery with monks to a non-monastery with monks who belong to a different Buddhist sect, except together with a sangha or if there are dangers. He shouldn’t go from a non-monastery with monks to a monastery or non-monastery with monks who belong to a different Buddhist sect, except together with a sangha or if there are dangers. %
\item He shouldn’t go from a monastery or non-monastery with monks to a monastery with monks who belong to a different Buddhist sect, except together with a sangha or if there are dangers. He shouldn’t go from a monastery or non-monastery with monks to a non-monastery with monks who belong to a different Buddhist sect, except together with a sangha or if there are dangers. He shouldn’t go from a monastery or non-monastery with monks to a monastery or non-monastery with monks who belong to a different Buddhist sect, except together with a sangha or if there are dangers. %
\item He may go from a monastery with monks to one with monks who belong to the same Buddhist sect if he knows he’ll be able to arrive on the same day. He may go from a monastery with monks to a non-monastery with monks who belong to the same Buddhist sect if he knows he’ll be able to arrive on the same day. He may go from a monastery with monks to a monastery or non-monastery with monks who belong to the same Buddhist sect if he knows he’ll be able to arrive on the same day. %
\item He may go from a non-monastery with monks to a monastery with monks who belong to the same Buddhist sect if he knows he’ll be able to arrive on the same day. He may go from a non-monastery with monks to a non-monastery with monks who belong to the same Buddhist sect if he knows he’ll be able to arrive on the same day. He may go from a non-monastery with monks to a monastery or non-monastery with monks who belong to the same Buddhist sect if he knows he’ll be able to arrive on the same day. %
\item He may go from a monastery or non-monastery with monks to a monastery with monks who belong to the same Buddhist sect if he knows he’ll be able to arrive on the same day. He may go from a monastery or non-monastery with monks to a non-monastery with monks who belong to the same Buddhist sect if he knows he’ll be able to arrive on the same day. He may go from a monastery or non-monastery with monks to a monastery or non-monastery with monks who belong to the same Buddhist sect if he knows he’ll be able to arrive on the same day. %
\item He shouldn’t, in a monastery, stay in the same room as a regular monk. %
\item He shouldn’t, in a non-monastery, stay in the same room as a regular monk. %
\item He shouldn’t, in a monastery or a non-monastery, stay in the same room as a regular monk. %
\item If he sees a regular monk, he should get up from his seat. %
\item He should offer a seat to a regular monk. %
\item He shouldn’t sit on the same seat as a regular monk. %
\item He shouldn’t sit on a higher seat than a regular monk. %
\item He shouldn’t sit on a seat when a regular monk is sitting on the ground. %
\item He shouldn’t do walking meditation on the same walking path as a regular monk. %
\item He shouldn’t do walking meditation on a higher walking path than a regular monk. %
\item He shouldn’t do walking meditation on a walking path when a regular monk is walking on the ground. %
\item He shouldn’t, in a monastery, stay in the same room as a monk on probation. … %
\item[55.] He shouldn’t, in a monastery, stay in the same room as a monk deserving to be sent back to the beginning. … %
\item[64.] He shouldn’t, in a monastery, stay in the same room as a monk deserving the trial period. … %
\item[73.] He shouldn’t, in a monastery, stay in the same room as a more senior monk undertaking the trial period. … %
\item[82.] He shouldn’t, in a monastery, stay in the same room as a monk deserving rehabilitation. %
\item[83.] He shouldn’t, in a non-monastery, stay in the same room as a monk deserving rehabilitation. %
\item[84.] He shouldn’t, in a monastery or a non-monastery, stay in the same room as a monk deserving rehabilitation. %
\item[85.] He shouldn’t sit on the same seat as a monk deserving rehabilitation. %
\item[86.] He shouldn’t sit on a higher seat than a monk deserving rehabilitation. %
\item[87.] He shouldn’t sit on a seat when a monk deserving rehabilitation is sitting on the ground. %
\item[88.] He shouldn’t do walking meditation on the same walking path as a monk deserving rehabilitation. %
\item[89.] He shouldn’t do walking meditation on a higher walking path than a monk deserving rehabilitation. %
\item[90.] He shouldn’t do walking meditation on a walking path when a monk deserving rehabilitation is walking on the ground. %
\item[91.] If, as the fourth member of a group, he gives probation, %
\item[92.] sends back to the beginning, %
\item[93.] or gives the trial period, %
\item[94.] or as the twentieth member of a group, he rehabilitates, it’s invalid and not to be done.” %
\end{enumerate}

\subsection*{Further regulations for the trial period}

Soon\marginnote{7.1.1} afterwards Venerable \textsanskrit{Upāli} went to the Buddha, bowed, sat down, and said, “How many things are there, sir, that stop a monk undertaking the trial period from counting a particular day toward his trial period?” 

\scrule{“There are four such things, \textsanskrit{Upāli}: he stays in the same room as a regular monk; he stays apart from other monks; he doesn’t inform other monks of his status; he travels without a group.” }

On\marginnote{8.1.1} one occasion a large sangha of monks had gathered at \textsanskrit{Sāvatthī}. The monks undertaking the trial period were unable to fulfill their duties. They told the Buddha. 

\scrule{“I allow you to set aside the trial period. }

And\marginnote{8.1.5} it should be done like this. The monk undertaking the trial period should approach a monk, arrange his upper robe over one shoulder, squat on his heels, raise his joined palms, and say, ‘I set aside the trial period,’ or ‘I set aside the proper conduct.’” 

Soon\marginnote{8.1.11} afterwards the monks at \textsanskrit{Sāvatthī} left for various destinations. The monks undertaking the trial period were again able to fulfill their duties. They told the Buddha. 

\scrule{“I allow you to take up the trial period. }

And\marginnote{8.1.15} it should be done like this. The monk undertaking the trial period should approach a monk, arrange his upper robe over one shoulder, squat on his heels, raise his joined palms, and say, ‘I take up the trial period,’ or ‘I take up the proper conduct.’” 

\scend{The proper conduct for those undertaking the trial period is finished. }

\section*{5. The proper conduct for those deserving rehabilitation. }

At\marginnote{9.1.1} this time monks deserving rehabilitation consented to regular monks bowing down to them, standing up for them, raising their joined palms to them, doing acts of respect toward them, offering them a seat, offering them a bed, offering them water for washing their feet and a foot stool, offering them a foot scraper, receiving their bowls and robes, and massaging their backs when bathing.  The monks of few desires complained and criticized them, “How can monks deserving rehabilitation consent to these things?” The monks told the Buddha. 

Soon\marginnote{9.1.6} afterwards the Buddha had the Sangha gathered and questioned the monks: 

“Is\marginnote{9.1.7} it true, monks, that monks deserving rehabilitation consent to these things?” 

“It’s\marginnote{9.1.8} true, sir.” 

The\marginnote{9.1.9} Buddha rebuked them … “It’s not suitable … How can monks deserving rehabilitation consent to these things? This will affect people’s confidence …” After rebuking them … he gave a teaching and addressed the monks: 

“A\marginnote{9.1.15} monk deserving rehabilitation shouldn’t consent to: 

\begin{itemize}%
\item Regular monks bowing down to him %
\item Regular monks standing up for him %
\item Regular monks raising their joined palms to him %
\item Regular monks doing acts of respect toward him %
\item Regular monks offering him a seat %
\item Regular monks offering him a bed %
\item Regular monks offering him water for washing his feet and a foot-stool %
\item Regular monks offering him a foot-scraper %
\item Regular monks receiving his bowl and robe %
\item Regular monks massaging his back when he’s bathing. %
\end{itemize}

\scrule{If he consents to any of these, he commits an offense of wrong conduct. }

Monks\marginnote{9.1.20} deserving rehabilitation should do the following with one another according to seniority: bow down, stand up, raise their joined palms, do acts of respect, offer a seat, offer a bed, offer water for washing the feet and a foot stool, offer a foot scraper, receive bowl and robe, and massage one another’s backs when bathing. 

Monks\marginnote{9.1.21} deserving rehabilitation should do five things with regular monks according to seniority: the observance-day ceremony, the invitation ceremony, distributing rainy-season robes, meal invitations, and meals. 

And\marginnote{9.1.23} now I will lay down the proper conduct for a monk deserving rehabilitation. This is the proper conduct: 

\begin{enumerate}%
\item He shouldn’t give the full ordination. %
\item He shouldn’t give formal support. %
\item He shouldn’t have a novice monk attend on him. %
\item He shouldn’t accept being appointed as an instructor of the nuns. %
\item Even if appointed, he shouldn’t instruct the nuns. %
\item He shouldn’t commit the same offense as the offense for which he deserves to be rehabilitated by the Sangha, nor one that is similar or worse. %
\item He shouldn’t criticize the legal procedure. %
\item He shouldn’t criticize those who did the procedure. %
\item He shouldn’t cancel the observance-day ceremony of a regular monk. %
\item He shouldn’t cancel the invitation of a regular monk. %
\item He shouldn’t direct a regular monk. %
\item He shouldn’t give instructions to a regular monk. %
\item He shouldn’t ask a regular monk for permission to accuse him of an offense. %
\item He shouldn’t accuse a regular monk of an offense. %
\item He shouldn’t remind a regular monk of an offense. %
\item He shouldn’t associate inappropriately with other monks. %
\item He shouldn’t walk in front of a regular monk. %
\item He shouldn’t sit in front of a regular monk. %
\item He should be given the last seat, the last bed, and the last dwelling of the Sangha, and he should consent to that. %
\item He shouldn’t attend on a regular monk when going to families %
\item He shouldn’t be attended on by a regular monk when going to families %
\item He shouldn’t undertake the practice of staying in the wilderness %
\item He shouldn’t undertake the practice of eating only almsfood. %
\item He shouldn’t have someone bring back almsfood for him because he doesn’t want others to know about his status. %
\item He shouldn’t go from a monastery with monks to one without monks, except together with a regular monk or if there are dangers. He shouldn’t go from a monastery with monks to a non-monastery without monks, except together with a regular monk or if there are dangers. He shouldn’t go from a monastery with monks to a monastery or non-monastery without monks, except together with a regular monk or if there are dangers. %
\item He shouldn’t go from a non-monastery with monks to a monastery without monks, except together with a regular monk or if there are dangers. He shouldn’t go from a non-monastery with monks to a non-monastery without monks, except together with a regular monk or if there are dangers.  He shouldn’t go from a non-monastery with monks to a monastery or non-monastery without monks, except together with a regular monk or if there are dangers. %
\item He shouldn’t go from a monastery or non-monastery with monks to a monastery without monks, except together with a regular monk or if there are dangers.  He shouldn’t go from a monastery or non-monastery with monks to a non-monastery without monks, except together with a regular monk or if there are dangers.  He shouldn’t go from a monastery or non-monastery with monks to a monastery or non-monastery without monks, except together with a regular monk or if there are dangers. %
\item He shouldn’t go from a monastery with monks to one with monks who belong to a different Buddhist sect, except together with a regular monk or if there are dangers.  He shouldn’t go from a monastery with monks to a non-monastery with monks who belong to a different Buddhist sect, except together with a regular monk or if there are dangers.  He shouldn’t go from a monastery with monks to a monastery or non-monastery with monks who belong to a different Buddhist sect, except together with a regular monk or if there are dangers. %
\item He shouldn’t go from a non-monastery with monks to a monastery with monks who belong to a different Buddhist sect, except together with a regular monk or if there are dangers.  He shouldn’t go from a non-monastery with monks to a non-monastery with monks who belong to a different Buddhist sect, except together with a regular monk or if there are dangers.  He shouldn’t go from a non-monastery with monks to a monastery or non-monastery with monks who belong to a different Buddhist sect, except together with a regular monk or if there are dangers. %
\item He shouldn’t go from a monastery or non-monastery with monks to a monastery with monks who belong to a different Buddhist sect, except together with a regular monk or if there are dangers.  He shouldn’t go from a monastery or non-monastery with monks to a non-monastery with monks who belong to a different Buddhist sect, except together with a regular monk or if there are dangers.  He shouldn’t go from a monastery or non-monastery with monks to a monastery or non-monastery with monks who belong to a different Buddhist sect, except together with a regular monk or if there are dangers. %
\item He may go from a monastery with monks to one with monks who belong to the same Buddhist sect if he knows he’ll be able to arrive on the same day. He may go from a monastery with monks to a non-monastery with monks who belong to the same Buddhist sect if he knows he’ll be able to arrive on the same day. He may go from a monastery with monks to a monastery or non-monastery with monks who belong to the same Buddhist sect if he knows he’ll be able to arrive on the same day. %
\item He may go from a non-monastery with monks to a monastery with monks who belong to the same Buddhist sect if he knows he’ll be able to arrive on the same day. He may go from a non-monastery with monks to a non-monastery with monks who belong to the same Buddhist sect if he knows he’ll be able to arrive on the same day.  He may go from a non-monastery with monks to a monastery or non-monastery with monks who belong to the same Buddhist sect if he knows he’ll be able to arrive on the same day. %
\item He may go from a monastery or non-monastery with monks to a monastery with monks who belong to the same Buddhist sect if he knows he’ll be able to arrive on the same day. He may go from a monastery or non-monastery with monks to a non-monastery with monks who belong to the same Buddhist sect if he knows he’ll be able to arrive on the same day. He may go from a monastery or non-monastery with monks to a monastery or non-monastery with monks who belong to the same Buddhist sect if he knows he’ll be able to arrive on the same day. %
\item He shouldn’t, in a monastery, stay in the same room as a regular monk. %
\item He shouldn’t, in a non-monastery, stay in the same room as a regular monk. %
\item He shouldn’t, in a monastery or a non-monastery, stay in the same room as a regular monk. %
\item If he sees a regular monk, he should get up from his seat. %
\item He should offer a seat to a regular monk. %
\item He shouldn’t sit on the same seat as a regular monk. %
\item He shouldn’t sit on a higher seat than a regular monk. %
\item He shouldn’t sit on a seat when a regular monk is sitting on the ground. %
\item He shouldn’t do walking meditation on the same walking path as a regular monk. %
\item He shouldn’t do walking meditation on a higher walking path than a regular monk. %
\item He shouldn’t do walking meditation on a walking path when a regular monk is walking on the ground. %
\item He shouldn’t, in a monastery, stay in the same room as a monk on probation. … %
\item[66.] He shouldn’t, in a monastery, stay in the same room as a monk deserving to be sent back to the beginning. … %
\item[75.] He shouldn’t, in a monastery, stay in the same room as a monk deserving the trial period. … %
\item[84.] He shouldn’t, in a monastery, stay in the same room as a monk undertaking the trial period. … %
\item[96.] He shouldn’t, in a monastery, stay in the same room as a more senior monk deserving rehabilitation. %
\item He shouldn’t, in a non-monastery, stay in the same room as a more senior monk deserving rehabilitation. %
\item He shouldn’t, in a monastery or a non-monastery, stay in the same room as a more senior monk deserving rehabilitation. %
\item He shouldn’t sit on the same seat as a more senior monk deserving rehabilitation. %
\item He shouldn’t sit on a higher seat than a more senior monk deserving rehabilitation. %
\item He shouldn’t sit on a seat when a more senior monk deserving rehabilitation is sitting on the ground. %
\item He shouldn’t do walking meditation on the same walking path as a more senior monk deserving rehabilitation. %
\item He shouldn’t do walking meditation on a higher walking path than a more senior monk deserving rehabilitation. %
\item He shouldn’t do walking meditation on a walking path when a more senior monk deserving rehabilitation is walking on the ground.  %
\item If, as the fourth member of a group, he gives probation, %
\item sends back to the beginning, %
\item or gives the trial period, %
\item or as the twentieth member of a group, he rehabilitates, it’s invalid and not to be done.” %
\end{enumerate}

\scend{The proper conduct for those deserving rehabilitation is finished. }

\scendsutta{The second chapter on those on probation is finished. }

\scend{In this chapter there are five topics. }

\scuddanaintro{This is the summary: }

\begin{scuddana}%
“Those\marginnote{9.1.78} on probation consented to \\
The regular monks \\
Bowing down to them, standing up, \\
And raising their joined palms, acting respectfully. 

Offering\marginnote{9.1.82} a seat, a bed, \\
Foot, stool, scraper; \\
Bowl, massaging when bathing, \\
And the good monks criticized them. 

An\marginnote{9.1.86} offense of wrong conduct for one who consents, \\
Mutual, five things according to seniority; \\
The observance day, the invitation ceremony, \\
Rainy-season robes, invitations, meals. 

And\marginnote{9.1.90} proper conduct there, \\
Walking in front of a regular one; \\
And whatever is the last, \\
And just so attending on. 

Wilderness,\marginnote{9.1.94} alms, bringing back, \\
About new arrivals, on the observance day; \\
On the invitation day, by messenger, \\
And he may go to a place with monks. 

And\marginnote{9.1.98} in the same room, getting up, \\
And just so he should invite; \\
On a seat, on a lower, on a walking path, \\
On the ground, and with walking path. 

With\marginnote{9.1.102} one who is more senior, invalid, \\
And stop the counting, fulfilling; \\
Setting aside, taking up, \\
Just the duties for one on probation. 

To\marginnote{9.1.106} the beginning, deserving the trial, \\
So those undertaking the trial; \\
And also the method for deserving rehabilitation, \\
Again putting together from the method. 

Three\marginnote{9.1.110} for those on probation, \\
Four for those undergoing the trial; \\
Are not the same in regard to stopping the count, \\
And daily for the trial period; \\
Two procedures are such, the remaining \\
Three procedures are the same.” 

%
\end{scuddana}

\scendsutta{The chapter on those on probation is finished. }

%
\chapter*{{\suttatitleacronym Kd 13}{\suttatitletranslation The gathering up chapter }{\suttatitleroot Samuccayakkhandhaka}}
\addcontentsline{toc}{chapter}{\tocacronym{Kd 13} \toctranslation{The gathering up chapter } \tocroot{Samuccayakkhandhaka}}
\markboth{The gathering up chapter }{Samuccayakkhandhaka}
\extramarks{Kd 13}{Kd 13}

\section*{1. Emission of semen }

At\marginnote{1.1.1} one time the Buddha was staying at \textsanskrit{Sāvatthī} in the Jeta Grove, \textsanskrit{Anāthapiṇḍika}’s Monastery.\footnote{This chapter concerns the processes for the clearing of offenses entailing suspension. Vmv 4.134: \textit{\textsanskrit{Saṅghādisesānaṁ} \textsanskrit{parivāsadānādisabbavinicchayassa} \textsanskrit{samuccayattā} panesa samuccayakkhandhakoti vuttoti veditabbo}, “It is to be understood that because of the collection of all the explanations of the giving of probation, etc., for offenses entailing suspension, it is said ‘\textit{samuccayakkhandhaka}’.” } At that time Venerable \textsanskrit{Udāyī} had committed one unconcealed offense of intentional emission of semen. He told the monks about this, adding, “What should I do now?” They told the Buddha. 

“Well\marginnote{1.1.7} then, the Sangha should give the monk \textsanskrit{Udāyī} the trial period of six days for one unconcealed offense of intentional emission of semen. And it should be given like this. 

\subsection*{1.1 Trial periods for those with unconcealed offenses }

“The\marginnote{1.2.2.1} monk \textsanskrit{Udāyī} should approach the Sangha, arrange his upper robe over one shoulder, pay respect at the feet of the senior monks, squat on his heels, raise his joined palms, and say: 

‘Venerables,\marginnote{1.2.3} I’ve committed one unconcealed offense of intentional emission of semen. I ask the Sangha for the trial period of six days for that offense. 

I’ve\marginnote{1.2.5} committed one unconcealed offense of intentional emission of semen. For the second time, I ask the Sangha for the trial period of six days for that offense. 

I’ve\marginnote{1.2.7} committed one unconcealed offense of intentional emission of semen. For the third time, I ask the Sangha for the trial period of six days for that offense.’ 

A\marginnote{1.3.1} competent and capable monk should then inform the Sangha: 

‘Please,\marginnote{1.3.2} venerables, I ask the Sangha to listen. The monk \textsanskrit{Udāyī} has committed one unconcealed offense of intentional emission of semen. He’s asking the Sangha for the trial period of six days for that offense. If the Sangha is ready, it should give him that trial period. This is the motion. 

Please,\marginnote{1.3.7} venerables, I ask the Sangha to listen. The monk \textsanskrit{Udāyī} has committed one unconcealed offense of intentional emission of semen. He’s asking the Sangha for the trial period of six days for that offense. The Sangha gives him that trial period. Any monk who approves of giving him that trial period of six days should remain silent. Any monk who doesn’t approve should speak up. 

For\marginnote{1.3.13} the second time, I speak on this matter. Please, venerables, I ask the Sangha to listen. The monk \textsanskrit{Udāyī} has committed one unconcealed offense of intentional emission of semen. He’s asking the Sangha for the trial period of six days for that offense. The Sangha gives him that trial period. Any monk who approves of giving him that trial period of six days should remain silent. Any monk who doesn’t approve should speak up. 

For\marginnote{1.3.20} the third time, I speak on this matter. Please, venerables, I ask the Sangha to listen. The monk \textsanskrit{Udāyī} has committed one unconcealed offense of intentional emission of semen. He’s asking the Sangha for the trial period of six days for that offense. The Sangha gives him the trial period. Any monk who approves of giving him that trial period of six days should remain silent. Any monk who doesn’t approve should speak up. 

The\marginnote{1.3.27} Sangha has given the monk \textsanskrit{Udāyī} the trial period of six days for one unconcealed offense of intentional emission of semen. The Sangha approves and is therefore silent. I’ll remember it thus.’” 

\subsection*{Rehabilitation for those with unconcealed offenses }

When\marginnote{2.1.1} he had completed that trial period, he told the monks, “I had committed one unconcealed offense of intentional emission of semen. I asked the Sangha to give me the trial period of six days for that offense, which it did. I’ve now completed it. What should I do next?” They told the Buddha. 

“Well\marginnote{2.1.8} then, the Sangha should rehabilitate the monk \textsanskrit{Udāyī}. And it should be done like this. The monk \textsanskrit{Udāyī} should approach the Sangha, arrange his upper robe over one shoulder, pay respect at the feet of the senior monks, squat on his heels, raise his joined palms, and say: 

‘Venerables,\marginnote{2.2.3} I had committed one unconcealed offense of intentional emission of semen. I asked the Sangha to give me the trial period of six days for that offense, which it did. I’ve now completed that trial period and ask the Sangha for rehabilitation. 

I\marginnote{2.2.6} had committed one unconcealed offense of intentional emission of semen. I asked the Sangha to give me the trial period of six days for that offense, which it did. I’ve now completed that trial period, and for the second time, I ask the Sangha for rehabilitation. 

I\marginnote{2.2.10} had committed one unconcealed offense of intentional emission of semen. I asked the Sangha to give me the trial period of six days for that offense, which it did. I’ve now completed that trial period and, for the third time, I ask the Sangha for rehabilitation.’ 

A\marginnote{2.3.1} competent and capable monk should then inform the Sangha: 

‘Please,\marginnote{2.3.2} venerables, I ask the Sangha to listen. The monk \textsanskrit{Udāyī} had committed one unconcealed offense of intentional emission of semen. He asked the Sangha to give him the trial period of six days for that offense, which it did. He’s now completed that trial period and is asking the Sangha for rehabilitation. If the Sangha is ready, it should rehabilitate him. This is the motion. 

Please,\marginnote{2.3.9} venerables, I ask the Sangha to listen. The monk \textsanskrit{Udāyī} had committed one unconcealed offense of intentional emission of semen. He asked the Sangha to give him the trial period of six days for that offense, which it did. He’s now completed that trial period and is asking the Sangha for rehabilitation. The Sangha rehabilitates him. Any monk who approves of rehabilitating him should remain silent. Any monk who doesn’t approve should speak up. 

For\marginnote{2.3.17} the second time, I speak on this matter. Please, venerables, I ask the Sangha to listen. The monk \textsanskrit{Udāyī} had committed one unconcealed offense of intentional emission of semen. He asked the Sangha to give him the trial period of six days for that offense, which it did. He’s now completed that trial period and is asking the Sangha for rehabilitation. The Sangha rehabilitates him. Any monk who approves of rehabilitating him should remain silent. Any monk who doesn’t approve should speak up. 

For\marginnote{2.3.26} the third time, I speak on this matter. Please, venerables, I ask the Sangha to listen. The monk \textsanskrit{Udāyī} had committed one unconcealed offense of intentional emission of semen. He asked the Sangha to give him the trial period of six days for that offense, which it did. He’s now completed that trial period and is asking the Sangha for rehabilitation. The Sangha rehabilitates him. Any monk who approves of rehabilitating him should remain silent. Any monk who doesn’t approve should speak up. 

The\marginnote{2.3.35} Sangha has rehabilitated the monk \textsanskrit{Udāyī}. The Sangha approves and is therefore silent. I’ll remember it thus.’” 

\subsection*{Probation for those with offenses concealed for one day }

On\marginnote{3.1.1} one occasion Venerable \textsanskrit{Udāyī} had committed one offense of intentional emission of semen, concealed for one day. He told the monks about this, adding, “What should I do now?” They told the Buddha. 

“Well\marginnote{3.1.6} then, the Sangha should give the monk \textsanskrit{Udāyī} probation for one day for one offense of intentional emission of semen, concealed for one day. And it should be given like this. The monk \textsanskrit{Udāyī} should approach the Sangha, arrange his upper robe over one shoulder, pay respect at the feet of the senior monks, squat on his heels, raise his joined palms, and say: 

‘Venerables,\marginnote{3.2.3} I’ve committed one offense of intentional emission of semen, concealed for one day. I ask the Sangha for probation for one day for that offense.’ And he should ask a second and a third time. 

A\marginnote{3.3.1} competent and capable monk should then inform the Sangha: 

‘Please,\marginnote{3.3.2} venerables, I ask the Sangha to listen. The monk \textsanskrit{Udāyī} has committed one offense of intentional emission of semen, concealed for one day. He’s asking the Sangha for probation for one day for that offense. If the Sangha is ready, it should give him that probation. This is the motion. 

Please,\marginnote{3.3.7} venerables, I ask the Sangha to listen. The monk \textsanskrit{Udāyī} has committed one offense of intentional emission of semen, concealed for one day. He’s asking the Sangha for probation for one day for that offense. The Sangha gives him that probation. Any monk who approves of giving him that probation should remain silent. Any monk who doesn’t approve should speak up. 

For\marginnote{3.3.13} the second time, I speak on this matter. … For the third time, I speak on this matter. … 

The\marginnote{3.3.15} Sangha has given the monk \textsanskrit{Udāyī} probation for one day for one offense of intentional emission of semen, concealed for one day. The Sangha approves and is therefore silent. I’ll remember it thus.’” 

\subsection*{Trial periods for those with offenses concealed for one day }

When\marginnote{4.1.1} he had completed that probation, he told the monks, “I had committed one offense of intentional emission of semen, concealed for one day. I asked the Sangha to give me probation for one day for that offense, which it did. I’ve now completed it. What should I do next?” They told the Buddha. 

“Well\marginnote{4.1.8} then, the Sangha should give the monk \textsanskrit{Udāyī} the trial period of six days for one offense of intentional emission of semen, concealed for one day. And it should be given like this. The monk \textsanskrit{Udāyī} should approach the Sangha, arrange his upper robe over one shoulder, pay respect at the feet of the senior monks, squat on his heels, raise his joined palms, and say: 

‘Venerables,\marginnote{4.2.4} I had committed one offense of intentional emission of semen, concealed for one day. I asked the Sangha to give me probation for one day for that offense, which it did. I’ve now completed that probation and ask the Sangha for the trial period of six days.’ And he should ask a second and a third time. 

A\marginnote{4.3.1} competent and capable monk should then inform the Sangha: 

‘Please,\marginnote{4.3.2} venerables, I ask the Sangha to listen. The monk \textsanskrit{Udāyī} had committed one offense of intentional emission of semen, concealed for one day. He asked the Sangha to give him probation for one day for that offense, which it did. He’s now completed that probation and is asking the Sangha for the trial period of six days. If the Sangha is ready, it should give him that trial period. This is the motion. 

Please,\marginnote{4.3.9} venerables, I ask the Sangha to listen. The monk \textsanskrit{Udāyī} had committed one offense of intentional emission of semen, concealed for one day. He asked the Sangha to give him probation for one day for that offense, which it did. He’s now completed that probation and is asking the Sangha for the trial period of six days. The Sangha gives him that trial period. Any monk who approves of giving him the trial period of six days should remain silent. Any monk who doesn’t approve should speak up. 

For\marginnote{4.3.17} the second time, I speak on this matter. … For the third time, I speak on this matter. … 

The\marginnote{4.3.19} Sangha has given the monk \textsanskrit{Udāyī} the trial period of six days for one offense of intentional emission of semen, concealed for one day. The Sangha approves and is therefore silent. I’ll remember it thus.’” 

\subsection*{Rehabilitation for those with offenses concealed for one day }

When\marginnote{5.1.1} he had completed that trial period, he told the monks, “I had committed one offense of intentional emission of semen, concealed for one day. I asked the Sangha to give me probation for one day for that offense, which it did. When I had completed that probation, I asked the Sangha to give me the trial period of six days, which it did. I’ve now completed it. What should I do next?” They told the Buddha. 

“Well\marginnote{5.1.10} then, the Sangha should rehabilitate the monk \textsanskrit{Udāyī}. And it should be done like this. The monk \textsanskrit{Udāyī} should approach the Sangha, put his upper robe over one shoulder, pay respect at the feet of the senior monks, squat on his heels, raise his joined palms, and say: 

‘I\marginnote{5.2.4} had committed one offense of intentional emission of semen, concealed for one day. I asked the Sangha to give me probation for one day for that offense, which it did. When I had completed that probation, I asked the Sangha to give me the trial period of six days, which it did. I’ve now completed the trial period and ask the Sangha for rehabilitation.’ And he should ask a second and a third time. 

A\marginnote{5.3.1} competent and capable monk should then inform the Sangha: 

‘Please,\marginnote{5.3.2} venerables, I ask the Sangha to listen. The monk \textsanskrit{Udāyī} had committed one offense of intentional emission of semen, concealed for one day. He asked the Sangha to give him probation for one day for that offense, which it did. When he had completed that probation, he asked the Sangha to give him the trial period of six days, which it did. He’s now completed that trial period and is asking the Sangha for rehabilitation. If the Sangha is ready, it should rehabilitate him. This is the motion. 

Please,\marginnote{5.3.11} venerables, I ask the Sangha to listen. The monk \textsanskrit{Udāyī} had committed one offense of intentional emission of semen, concealed for one day. He asked the Sangha to give him probation for one day for that offense, which it did. When he had completed that probation, he asked the Sangha to give him the trial period of six days, which it did. He’s now completed that trial period and is asking the Sangha for rehabilitation. The Sangha rehabilitates him. Any monk who approves of rehabilitating him should remain silent. Any monk who doesn’t approve should speak up. 

For\marginnote{5.3.21} the second time, I speak on this matter. … For the third time, I speak on this matter. … 

The\marginnote{5.3.23} Sangha has rehabilitated the monk \textsanskrit{Udāyī}. The Sangha approves and is therefore silent. I’ll remember it thus.’” 

\subsection*{Probation for those with offenses concealed for five days }

On\marginnote{6.1.1} one occasion Venerable \textsanskrit{Udāyī} had committed one offense of intentional emission of semen, concealed for two days. … concealed for three days. … concealed for four days. … concealed for five days. He told the monks about this, adding, “What should I do now?” They told the Buddha. 

“Well\marginnote{6.1.9} then, the Sangha should give the monk \textsanskrit{Udāyī} probation for five days for one offense of intentional emission of semen, concealed for five days. And it should be given like this. The monk \textsanskrit{Udāyī} should approach the Sangha, arrange his upper robe over one shoulder, pay respect at the feet of the senior monks, squat on his heels, raise his joined palms, and say: 

‘Venerables,\marginnote{6.1.12} I’ve committed one offense of intentional emission of semen, concealed for five days. I ask the Sangha for probation for five days for that offense.’ And he should ask a second and a third time. 

A\marginnote{6.1.16} competent and capable monk should then inform the Sangha: 

‘Please,\marginnote{6.1.17} venerables, I ask the Sangha to listen. The monk \textsanskrit{Udāyī} has committed one offense of intentional emission of semen, concealed for five days. He’s asking the Sangha for probation for five days for that offense. If the Sangha is ready, it should give him that probation. This is the motion. 

Please,\marginnote{6.1.22} venerables, I ask the Sangha to listen. The monk \textsanskrit{Udāyī} has committed one offense of intentional emission of semen, concealed for five days. He’s asking the Sangha for probation for five days for that offense. The Sangha gives him that probation. Any monk who approves of giving him that probation should remain silent. Any monk who doesn’t approve should speak up. 

For\marginnote{6.1.28} the second time, I speak on this matter. … For the third time, I speak on this matter. … 

The\marginnote{6.1.30} Sangha has given the monk \textsanskrit{Udāyī} probation for five days for one offense of intentional emission of semen, concealed for five days. The Sangha approves and is therefore silent. I’ll remember it thus.’” 

\subsection*{Sending back to the beginning of those on probation }

While\marginnote{7.1.1} on probation, he committed one unconcealed offense of intentional emission of semen. He told the monks, “I had committed one offense of intentional emission of semen, concealed for five days. I asked the Sangha to give me probation for five days for that offense, which it did. While on probation, I committed one unconcealed offense of intentional emission of semen. What should I do now?” They told the Buddha. 

“Well\marginnote{7.1.9} then, the Sangha should send the monk \textsanskrit{Udāyī} back to the beginning for one unconcealed offense of intentional emission of semen, committed while on probation. And it should be done like this. The monk \textsanskrit{Udāyī} should approach the Sangha, arrange his upper robe over one shoulder, pay respect at the feet of the senior monks, squat on his heels, raise his joined palms, and say: 

‘Venerables,\marginnote{7.2.4} I had committed one offense of intentional emission of semen, concealed for five days. I asked the Sangha to give me probation for five days for that offense, which it did. While on probation, I committed one unconcealed offense of intentional emission of semen. I ask the Sangha to send me back to the beginning for that offense.’ And he should ask a second and a third time. 

A\marginnote{7.3.1} competent and capable monk should then inform the Sangha: 

‘Please,\marginnote{7.3.2} venerables, I ask the Sangha to listen. The monk \textsanskrit{Udāyī} had committed one offense of intentional emission of semen, concealed for five days. He asked the Sangha to give him probation for five days for that offense, which it did. While on probation, he committed one unconcealed offense of intentional emission of semen. He’s now asking the Sangha to send him back to the beginning for that offense. If the Sangha is ready, it should send him back to the beginning. This is the motion. 

Please,\marginnote{7.3.10} venerables, I ask the Sangha to listen. The monk \textsanskrit{Udāyī} had committed one offense of intentional emission of semen, concealed for five days. He asked the Sangha to give him probation for five days for that offense, which it did. While on probation, he committed one unconcealed offense of intentional emission of semen. He’s now asking the Sangha to send him back to the beginning for that offense. The Sangha sends him back to the beginning. Any monk who approves of sending him back to the beginning should remain silent. Any monk who doesn’t approve should speak up. 

For\marginnote{7.3.19} the second time, I speak on this matter. … For the third time, I speak on this matter. … 

The\marginnote{7.3.21} Sangha has sent the monk \textsanskrit{Udāyī} back to the beginning for one unconcealed offense of intentional emission of semen, committed while on probation. The Sangha approves and is therefore silent. I’ll remember it thus.’” 

\subsection*{Sending back to the beginning of those deserving the trial period }

When\marginnote{8.1.1} he had completed that probation, while deserving the trial period, he committed one unconcealed offense of intentional emission of semen. He told the monks, “I had committed one offense of intentional emission of semen, concealed for five days. I asked the Sangha to give me probation for five days for that offense, which it did. While on probation, I committed one unconcealed offense of intentional emission of semen. I asked the Sangha to send me back to the beginning for that offense, which it did. When I had completed the probation, while deserving the trial period, I committed one unconcealed offense of intentional emission of semen. What should I do now?” They told the Buddha. 

“Well\marginnote{8.1.12} then, the Sangha should send the monk \textsanskrit{Udāyī} back to the beginning for one unconcealed offense of intentional emission of semen, committed while deserving the trial period. And it should be done like this. The monk \textsanskrit{Udāyī} should approach the Sangha, arrange his upper robe over one shoulder, pay respect at the feet of the senior monks, squat on his heels, raise his joined palms, and say: 

‘Venerables,\marginnote{8.2.4} I had committed one offense of intentional emission of semen, concealed for five days. … When I had completed the probation, while deserving the trial period, I committed one unconcealed offense of intentional emission of semen. I ask the Sangha to send me back to the beginning for that offense.’ And he should ask a second and a third time. 

A\marginnote{8.3.1} competent and capable monk should then inform the Sangha: 

‘Please,\marginnote{8.3.2} venerables, I ask the Sangha to listen. The monk \textsanskrit{Udāyī} had committed one offense of intentional emission of semen, concealed for five days. … When he had completed that probation, while deserving the trial period, he committed one unconcealed offense of intentional emission of semen. He’s asking the Sangha to send him back to the beginning for that offense. If the Sangha is ready, it should send him back to the beginning. This is the motion. 

Please,\marginnote{8.3.8} venerables, I ask the Sangha to listen. The monk \textsanskrit{Udāyī} had committed one offense of intentional emission of semen, concealed for five days. … When he had completed that probation, while deserving the trial period, he committed one unconcealed offense of intentional emission of semen. He’s asking the Sangha to send him back to the beginning for that offense. The Sangha sends him back to the beginning. Any monk who approves of sending him back to the beginning should remain silent. Any monk who doesn’t approve should speak up. 

For\marginnote{8.3.15} the second time, I speak on this matter. … For the third time, I speak on this matter. … 

The\marginnote{8.3.17} Sangha has sent the monk \textsanskrit{Udāyī} back to the beginning for one unconcealed offense of intentional emission of semen, committed while deserving the trial period. The Sangha approves and is therefore silent. I’ll remember it thus.’” 

\subsection*{Trial period for the three offenses }

When\marginnote{9.1.1} he had completed that probation, he told the monks, “I had committed one offense of intentional emission of semen, concealed for five days … I’ve now completed the probation. 

What\marginnote{9.1.4} should I do next?” They told the Buddha. 

“Well\marginnote{9.1.6} then, the Sangha should give the monk \textsanskrit{Udāyī} the trial period of six days for the three offenses. And it should be given like this. The monk \textsanskrit{Udāyī} should approach the Sangha, arrange his upper robe over one shoulder, pay respect at the feet of the senior monks, squat on his heels, raise his joined palms, and say: 

‘Venerables,\marginnote{9.2.4} I had committed one offense of intentional emission of semen, concealed for five days. I asked the Sangha to give me probation for five days for that offense, which it did. … I’ve now completed that probation and ask the Sangha for the trial period of six days for the three offenses.’ And he should ask a second and a third time. 

A\marginnote{9.3.1} competent and capable monk should then inform the Sangha: 

‘Please,\marginnote{9.3.2} venerables, I ask the Sangha to listen. The monk \textsanskrit{Udāyī} had committed one offense of intentional emission of semen, concealed for five days. … He’s now completed that probation and is asking the Sangha for the trial period of six days for the three offenses. If the Sangha is ready, it should give him that trial period. This is the motion. 

Please,\marginnote{9.3.7} venerables, I ask the Sangha to listen. The monk \textsanskrit{Udāyī} had committed one offense of intentional emission of semen, concealed for five days. … He’s now completed that probation and is asking the Sangha for the trial period of six days for the three offenses. The Sangha gives him that trial period. Any monk who approves of giving him the trial period of six days should remain silent. Any monk who doesn’t approve should speak up. 

For\marginnote{9.3.13} the second time, I speak on this matter. … For the third time, I speak on this matter. … 

The\marginnote{9.3.15} Sangha has given the monk \textsanskrit{Udāyī} the trial period of six days for the three offenses. The Sangha approves and is therefore silent. I’ll remember it thus.’” 

\subsection*{Sending back to the beginning of those undertaking the trial period }

While\marginnote{10.1.1} he was undertaking the trial period, he committed one unconcealed offense of intentional emission of semen. He told the monks, “I had committed one offense of intentional emission of semen, concealed for five days. … While undertaking the trial period, I committed one unconcealed offense of intentional emission of semen. What should I do now?” They told the Buddha. 

“Well\marginnote{10.1.7} then, the Sangha should send the monk \textsanskrit{Udāyī} back to the beginning for one unconcealed offense of intentional emission of semen, committed while undertaking the trial period. It should then give him the trial period of six days. And he should be sent back to the beginning like this. The monk \textsanskrit{Udāyī} should approach the Sangha, arrange his upper robe over one shoulder, pay respect at the feet of the senior monks, squat on his heels, raise his joined palms, and say: 

‘Venerables,\marginnote{10.1.11} I had committed one offense of intentional emission of semen, concealed for five days. … While undertaking the trial period, I committed one unconcealed offense of intentional emission of semen. I ask the Sangha to send me back to the beginning for that offense.’ And he should ask a second and a third time. 

A\marginnote{10.1.16} competent and capable monk should then inform the Sangha: 

‘Please,\marginnote{10.1.17} venerables, I ask the Sangha to listen. The monk \textsanskrit{Udāyī} … is asking the Sangha to send him back to the beginning for that offense, committed while undertaking the trial period. If the Sangha is ready, it should send him back to the beginning. This is the motion. … The Sangha has sent the monk \textsanskrit{Udāyī} back to the beginning for one unconcealed offense of intentional emission of semen, committed while undertaking the trial period. The Sangha approves and is therefore silent. I’ll remember it thus.’ 

And\marginnote{10.1.24} he should be given the trial period of six days like this. The monk \textsanskrit{Udāyī} should approach the Sangha, arrange his upper robe over one shoulder, pay respect at the feet of the senior monks, squat on his heels, raise his joined palms, and say: 

‘Venerables,\marginnote{10.1.27} I had committed one offense of intentional emission of semen, concealed for five days. … While undertaking the trial period, I committed one unconcealed offense of intentional emission of semen. I asked the Sangha to send me back to the beginning for that offense, which it did. I now ask the Sangha for the trial period of six days for that offense.’ And he should ask a second and a third time. 

A\marginnote{10.1.34} competent and capable monk should then inform the Sangha: 

‘Please,\marginnote{10.1.35} venerables, I ask the Sangha to listen. The monk \textsanskrit{Udāyī} … is asking the Sangha for the trial period of six days for that offense, committed while undertaking the trial period. If the Sangha is ready, it should give him that trial period. This is the motion. … 

The\marginnote{10.1.40} Sangha has given the monk \textsanskrit{Udāyī} the trial period of six days for one unconcealed offense of intentional emission of semen, committed while undertaking the trial period. The Sangha approves and is therefore silent. I’ll remember it thus.’” 

\subsection*{Sending back to the beginning of those deserving rehabilitation }

When\marginnote{11.1.1} he had completed that trial period, while deserving rehabilitation, he committed one unconcealed offense of intentional emission of semen. He told the monks, “I had committed one offense of intentional emission of semen, concealed for five days. … When I had completed the trial period, while deserving rehabilitation, I committed one unconcealed offense of intentional emission of semen. What should I do now?” They told the Buddha. 

“Well\marginnote{11.1.7} then, the Sangha should send the monk \textsanskrit{Udāyī} back to the beginning for one unconcealed offense of intentional emission of semen, committed while deserving rehabilitation. It should then give him the trial period of six days. And he should be sent back to the beginning like this. … And he should be given the trial period of six days like this. … 

The\marginnote{11.1.10} Sangha has given the monk \textsanskrit{Udāyī} the trial period of six days for one unconcealed offense of intentional emission of semen, committed while deserving rehabilitation. The Sangha approves and is therefore silent. I’ll remember it thus.’” 

\subsection*{Rehabilitation of those sent back to the beginning }

When\marginnote{12.1.1} he had completed that trial period, he told the monks, “I had committed one offense of intentional emission of semen, concealed for five days. … I’ve now completed the trial period. What should I do next?” They told the Buddha. 

“Well\marginnote{12.1.6} then, the Sangha should rehabilitate the monk \textsanskrit{Udāyī}. And it should be done like this. The monk \textsanskrit{Udāyī} should approach the Sangha, arrange his upper robe over one shoulder, pay respect at the feet of the senior monks, squat on his heels, raise his joined palms, and say: 

‘Venerables,\marginnote{12.2.4} I had committed one offense of intentional emission of semen, concealed for five days. I asked the Sangha to give me probation for five days for that offense, which it did. While on probation, I committed one unconcealed offense of intentional emission of semen. I asked the Sangha to send me back to the beginning for that offense, which it did. When I had completed the probation, while deserving the trial period, I committed one unconcealed offense of intentional emission of semen. I asked the Sangha to send me back to the beginning for that offense, which it did. When I had completed the probation, I asked the Sangha to give me the trial period of six days, which it did. While undertaking the trial period, I committed one unconcealed offense of intentional emission of semen. I asked the Sangha to send me back to the beginning for that offense, which it did. I then asked the Sangha to give me the trial period for that offense, which it did. When I had completed that trial period, while deserving rehabilitation, I committed one unconcealed offense of intentional emission of semen. I asked the Sangha to send me back to the beginning for that offense, which it did. I then asked the Sangha to give me the trial period for that offense, which it did. I’ve now completed the trial period and ask the Sangha for rehabilitation.’ And he should ask a second and a third time. 

A\marginnote{12.3.1} competent and capable monk should then inform the Sangha: 

‘Please,\marginnote{12.3.2} venerables, I ask the Sangha to listen. The monk \textsanskrit{Udāyī} had committed one offense of intentional emission of semen, concealed for five days. He asked the Sangha to give him probation for five days for that offense, which it did. While on probation, he committed one unconcealed offense of intentional emission of semen. He asked the Sangha to send him back to the beginning for that offense, which it did. When he had completed that probation, while deserving the trial period, he committed one unconcealed offense of intentional emission of semen. He asked the Sangha to send him back to the beginning for that offense, which it did. When he had completed that probation, he asked the Sangha to give him the trial period of six days for the three offenses, which it did. While he was undertaking the trial period, he committed one unconcealed offense of intentional emission of semen. He asked the Sangha to send him back to the beginning for that offense, which it did. He then asked the Sangha to give him the trial period for that offense, which it did. When he had completed that trial period, while deserving rehabilitation, he committed one unconcealed offense of intentional emission of semen. He asked the Sangha to send him back to the beginning for that offense, which it did. He then asked the Sangha to give him the trial period for that offense, which it did. He’s now completed that trial period and is asking the Sangha for rehabilitation. If the Sangha is ready, it should rehabilitate him. This is the motion. 

‘Please,\marginnote{12.3.27} venerables, I ask the Sangha to listen. The monk \textsanskrit{Udāyī} had committed one offense of intentional emission of semen, concealed for five days. … He’s now completed that trial period and is asking the Sangha for rehabilitation. The Sangha rehabilitates him. Any monk who approves of rehabilitating him should remain silent. Any monk who doesn’t approve should speak up. 

For\marginnote{12.3.33} the second time, I speak on this matter. … For the third time, I speak on this matter. … 

The\marginnote{12.3.35} Sangha has rehabilitated the monk \textsanskrit{Udāyī}. The Sangha approves and is therefore silent. I’ll remember it thus.’” 

\subsection*{Probation for those with offenses concealed for a half-month }

On\marginnote{13.1.1} one occasion Venerable \textsanskrit{Udāyī} had committed one offense of intentional emission of semen, concealed for a half-month. He told the monks about this, adding, “What should I do now?” They told the Buddha. 

“Well\marginnote{13.1.6} then, the Sangha should give the monk \textsanskrit{Udāyī} probation for a half-month for one offense of intentional emission of semen, concealed for a half-month. And it should be given like this. The monk \textsanskrit{Udāyī} should approach the Sangha, arrange his upper robe over one shoulder, pay respect at the feet of the senior monks, squat on his heels, raise his joined palms, and say: 

‘Venerables,\marginnote{13.1.10} I’ve committed one offense of intentional emission of semen, concealed for a half-month. I ask the Sangha for probation for a half-month for that offense.’ And he should ask a second and a third time. 

A\marginnote{13.1.14} competent and capable monk should then inform the Sangha: 

‘Please,\marginnote{13.1.15} venerables, I ask the Sangha to listen. The monk \textsanskrit{Udāyī} has committed one offense of intentional emission of semen, concealed for a half-month. He’s asking the Sangha for probation for a half-month for that offense. If the Sangha is ready, it should give him that probation. This is the motion. 

Please,\marginnote{13.1.20} venerables, I ask the Sangha to listen. The monk \textsanskrit{Udāyī} has committed one offense of intentional emission of semen, concealed for a half-month. He’s asking the Sangha for probation for a half-month for that offense. The Sangha gives him that probation. Any monk who approves of giving him that probation should remain silent. Any monk who doesn’t approve should speak up. 

For\marginnote{13.1.26} the second time, I speak on this matter. … For the third time, I speak on this matter. … 

The\marginnote{13.1.28} Sangha has given the monk \textsanskrit{Udāyī} probation for a half-month for one offense of intentional emission of semen, concealed for a half-month. The Sangha approves and is therefore silent. I’ll remember it thus.’” 

\subsection*{Sending back to the beginning of those on probation for a half-month }

While\marginnote{14.1.1} on probation, he committed one offense of intentional emission of semen, concealed for five days. He told the monks, “I had committed one offense of intentional emission of semen, concealed for a half-month. I asked the Sangha to give me probation for a half-month for that offense, which it did. While on probation, I committed one offense of intentional emission of semen, concealed for five days. What should I do now?” They told the Buddha. 

“Well\marginnote{14.1.9} then, the Sangha should send the monk \textsanskrit{Udāyī} back to the beginning for one offense of intentional emission of semen, concealed for five days and committed while on probation, and it should then give him probation simultaneous with the probation for the previous offense. And he should be sent back to the beginning like this. The monk \textsanskrit{Udāyī} should approach the Sangha, arrange his upper robe over one shoulder, pay respect at the feet of the senior monks, squat on his heels, raise his joined palms, and say: 

‘Venerables,\marginnote{14.2.4} I had committed one offense of intentional emission of semen, concealed for a half-month. I asked the Sangha to give me probation for a half-month for that offense, which it did. While on probation, I committed one offense of intentional emission of semen, concealed for five days. I ask the Sangha to send me back to the beginning for that offense.’ And he should ask a second and a third time. 

A\marginnote{14.2.11} competent and capable monk should then inform the Sangha: 

‘Please,\marginnote{14.2.12} venerables, I ask the Sangha to listen. The monk \textsanskrit{Udāyī} had committed one offense of intentional emission of semen, concealed for a half-month. He asked the Sangha to give him probation for a half-month for that offense, which it did. While on probation, he committed one offense of intentional emission of semen, concealed for five days. He’s now asking the Sangha to send him back to the beginning for that offense. If the Sangha is ready, it should send him back to the beginning. This is the motion. 

Please,\marginnote{14.2.20} venerables, I ask the Sangha to listen. The monk \textsanskrit{Udāyī} had committed one offense of intentional emission of semen, concealed for a half-month. He asked the Sangha to give him probation for a half-month for that offense, which it did. While on probation, he committed one offense of intentional emission of semen, concealed for five days. He’s now asking the Sangha to send him back to the beginning for that offense. The Sangha sends him back to the beginning. Any monk who approves of sending him back to the beginning should remain silent. Any monk who doesn’t approve should speak up. 

For\marginnote{14.2.29} the second time, I speak on this matter. … For the third time, I speak on this matter. … 

The\marginnote{14.2.31} Sangha has sent the monk \textsanskrit{Udāyī} back to the beginning for one offense of intentional emission of semen, concealed for five days and committed while on probation. The Sangha approves and is therefore silent. I’ll remember it thus.’” 

\subsection*{Simultaneous probations }

“And\marginnote{14.3.1} he should be given probation simultaneous with probation for the previous offense like this. The monk \textsanskrit{Udāyī} should approach the Sangha, arrange his upper robe over one shoulder, pay respect at the feet of the senior monks, squat on his heels, raise his joined palms, and say: 

‘I\marginnote{14.3.4} had committed one offense of intentional emission of semen, concealed for a half-month. I asked the Sangha to give me probation for a half-month for that offense, which it did. While on probation, I committed one offense of intentional emission of semen, concealed for five days. I asked the Sangha to send me back to the beginning for that offense, which it did. I now ask the Sangha for probation for that offense, simultaneous with the probation for the previous offense.’ And he should ask a second and a third time. 

A\marginnote{14.3.13} competent and capable monk should then inform the Sangha: 

‘Please,\marginnote{14.3.14} venerables, I ask the Sangha to listen. The monk \textsanskrit{Udāyī} had committed one offense of intentional emission of semen, concealed for a half-month. He asked the Sangha to give him probation for a half-month for that offense, which it did. While on probation, he committed one offense of intentional emission of semen, concealed for five days. He asked the Sangha to send him back to the beginning for that offense, which it did. He’s now asking the Sangha for probation for that offense, simultaneous with the probation for the previous offense. If the Sangha is ready, it should give him that simultaneous probation. This is the motion. … 

‘Please,\marginnote{14.3.24} venerables, I ask the Sangha to listen. The monk \textsanskrit{Udāyī} had committed one offense of intentional emission of semen, concealed for a half-month. He asked the Sangha to give him probation for a half-month for that offense, which it did. While on probation, he committed one offense of intentional emission of semen, concealed for five days. He asked the Sangha to send him back to the beginning for that offense, which it did. He’s now asking the Sangha for probation for that offense, simultaneous with the probation for the previous offense. The Sangha gives him that simultaneous probation. Any monk who approves of giving him that simultaneous probation should remain silent. Any monk who doesn’t approve should speak up. 

For\marginnote{14.3.35} the second time, I speak on this matter. … For the third time, I speak on this matter. … 

The\marginnote{14.3.37} Sangha has given the monk \textsanskrit{Udāyī} probation for one offense of intentional emission of semen—concealed for five days and committed while on probation—simultaneous with the probation for the previous offense. The Sangha approves and is therefore silent. I’ll remember it thus.’” 

\subsection*{Sending back to the beginning of those deserving the trial period, etc. }

When\marginnote{15.1.1} he had completed that probation, while deserving the trial period, he committed one offense of intentional emission of semen, concealed for five days. He told the monks, “I had committed one offense of intentional emission of semen, concealed for a half-month. … When I had completed the probation, while deserving the trial period, I committed one offense of intentional emission of semen, concealed for five days. What should I do now?” They told the Buddha. 

“Well\marginnote{15.1.7} then, the Sangha should send the monk \textsanskrit{Udāyī} back to the beginning for one offense of intentional emission of semen—concealed for five days and committed while deserving the trial period—and it should then give him probation simultaneous with probation for the previous offense. And he should be sent back to the beginning like this. … And he should be given probation simultaneous with probation for the previous offense like this. … 

The\marginnote{15.1.11} Sangha has given the monk \textsanskrit{Udāyī} probation for one offense of intentional emission of semen—concealed for five days and committed while deserving the trial period—simultaneous with the probation for the previous offense. The Sangha approves and is therefore silent. I’ll remember it thus.’” 

\subsection*{Trial period for the three offenses }

When\marginnote{16.1.1} he had completed that probation, he told the monks, “I had committed one offense of intentional emission of semen, concealed for a half-month … I’ve now completed the probation. What should I do next?” They told the Buddha. 

“Well\marginnote{16.1.6} then, the Sangha should give the monk \textsanskrit{Udāyī} the trial period of six days for the three offenses. And it should be given like this. The monk \textsanskrit{Udāyī} should approach the Sangha, arrange his upper robe over one shoulder, pay respect at the feet of the senior monks, squat on his heels, raise his joined palms, and say: 

‘Venerables,\marginnote{16.1.10} I had committed one offense of intentional emission of semen, concealed for a half-month. … I’ve now completed that probation and ask the Sangha for the trial period of six days for the three offenses.’ And he should ask a second and a third time. 

A\marginnote{16.1.14} competent and capable monk should then inform the Sangha: 

‘Please,\marginnote{16.1.15} venerables, I ask the Sangha to listen. The monk \textsanskrit{Udāyī} had committed one offense of intentional emission of semen, concealed for a half-month. … He’s now completed that probation and is asking the Sangha for the trial period of six days for the three offenses. If the Sangha is ready, it should give him that trial period. This is the motion. 

Please,\marginnote{16.1.20} venerables, I ask the Sangha to listen. The monk \textsanskrit{Udāyī} had committed one offense of intentional emission of semen, concealed for a half-month. … He’s now completed that probation and is asking the Sangha for the trial period of six days for the three offenses. The Sangha gives him that trial period. Any monk who approves of giving him that trial period of six days should remain silent. Any monk who doesn’t approve should speak up. 

For\marginnote{16.1.26} the second time, I speak on this matter. … For the third time, I speak on this matter. … 

The\marginnote{16.1.28} Sangha has given the monk \textsanskrit{Udāyī} the trial period of six days for the three offenses. The Sangha approves and is therefore silent. I’ll remember it thus.’” 

\subsection*{Sending back to the beginning of those undertaking the trial period, etc. }

While\marginnote{17.1.1} he was undertaking the trial period, he committed one offense of intentional emission of semen, concealed for five days. He told the monks, “I had committed one offense of intentional emission of semen, concealed for a half-month. … While undertaking the trial period, I committed one offense of intentional emission of semen, concealed for five days. What should I do now?” They told the Buddha. 

“Well\marginnote{17.1.7} then, the Sangha should send the monk \textsanskrit{Udāyī} back to the beginning for one offense of intentional emission of semen—concealed for five days and committed while undertaking the trial period—and it should then give him probation for that offense simultaneous with probation for the previous offense, and it should then give him the trial period of six days. And he should be sent back to the beginning like this. … And he should be given probation simultaneous with probation for the previous offense like this. … And he should be given the trial period of six days like this. … 

The\marginnote{17.1.12} Sangha has given the monk \textsanskrit{Udāyī} the trial period of six days for one offense of intentional emission of semen, concealed for five days and committed while undertaking the trial period. The Sangha approves and is therefore silent. I’ll remember it thus.’” 

\subsection*{Sending back to the beginning of those deserving rehabilitation, etc. }

When\marginnote{18.1.1} he had completed that trial period, while deserving rehabilitation, he committed one offense of intentional emission of semen, concealed for five days. He told the monks, “I had committed one offense of intentional emission of semen, concealed for a half-month. … When I had completed the trial period, while deserving rehabilitation, I committed one offense of intentional emission of semen, concealed for five days. What should I do now?” They told the Buddha. 

“Well\marginnote{18.1.7} then, the Sangha should send the monk \textsanskrit{Udāyī} back to the beginning for one offense of intentional emission of semen—concealed for five days and committed while deserving rehabilitation—and it should then give him probation for that offense simultaneous with probation for the previous offense, and it should then give him the trial period of six days. And he should be sent back to the beginning like this. … And he should be given probation simultaneous with probation for the previous offense like this. … And he should be given the trial period of six days like this. … 

The\marginnote{18.1.12} Sangha has given the monk \textsanskrit{Udāyī} the trial period of six days for one offense of intentional emission of semen, concealed for five days and committed while deserving rehabilitation. The Sangha approves and is therefore silent. I’ll remember it thus.’” 

\subsection*{Rehabilitation of those with offenses concealed for a half-month }

When\marginnote{19.1.1} he had completed that trial period, he told the monks, “I had committed one offense of intentional emission of semen, concealed for a half-month. … I’ve now completed the trial period. What should I do next?” They told the Buddha. 

“Well\marginnote{19.1.6} then, the Sangha should rehabilitate the monk \textsanskrit{Udāyī}. And it should be done like this. The monk \textsanskrit{Udāyī} should approach the Sangha, arrange his upper robe over one shoulder, pay respect at the feet of the senior monks, squat on his heels, raise his joined palms, and say: 

‘Venerables,\marginnote{19.1.9} I had committed one offense of intentional emission of semen, concealed for a half-month. I asked the Sangha to give me probation for a half-month for that offense, which it did. While on probation, I committed one offense of intentional emission of semen, concealed for five days. I asked the Sangha to send me back to the beginning for that offense, which it did. I then asked the Sangha to give me probation for that offense, simultaneous with the probation for the previous offense, which it did. When I had completed that probation, while deserving the trial period, I committed one offense of intentional emission of semen, concealed for five days. I asked the Sangha to send me back to the beginning for that offense, which it did. I then asked the Sangha to give me probation for that offense, simultaneous with the probation for the previous offense, which it did. When I had completed that probation, I asked the Sangha to give me the trial period of six days, which it did. While undertaking the trial period, I committed one offense of intentional emission of semen, concealed for five days. I asked the Sangha to send me back to the beginning for that offense, which it did. I then asked the Sangha to give me probation for that offense, simultaneous with the probation for the previous offense, which it did. When I had completed that probation, I asked the Sangha to give me the trial period of six days, which it did. When I had completed that trial period, while deserving rehabilitation, I committed one offense of intentional emission of semen, concealed for five days. I asked the Sangha to send me back to the beginning for that offense, which it did. I then asked the Sangha to give me probation for that offense, simultaneous with the probation for the previous offense, which it did. When I had completed that probation, I asked the Sangha to give me the trial period of six days, which it did. I’ve now completed the trial period and ask the Sangha for rehabilitation.’ And he should ask a second 

and\marginnote{19.1.40} a third time. A competent and capable monk should then inform the Sangha: 

‘Please,\marginnote{19.1.42} venerables, I ask the Sangha to listen. The monk \textsanskrit{Udāyī} had committed one offense of intentional emission of semen, concealed for a half-month. He asked the Sangha to give him probation for a half-month for that offense, which it did. While on probation, he committed one offense of intentional emission of semen, concealed for five days. He asked the Sangha to send him back to the beginning for that offense, which it did. He then asked the Sangha to give him probation for that offense, simultaneous with the probation for the previous offense, which it did. When he had completed that probation, while deserving the trial period, he committed one offense of intentional emission of semen, concealed for five days. He asked the Sangha to send him back to the beginning for that offense, which it did. He then asked the Sangha to give him probation for that offense, simultaneous with the probation for the previous offense, which it did. When he had completed that probation, he asked the Sangha to give him the trial period of six days for the three offenses, which it did. While he was undertaking the trial period, he committed one offense of intentional emission of semen, concealed for five days. He asked the Sangha to send him back to the beginning for that offense, which it did. He then asked the Sangha to give him probation for that offense, simultaneous with the probation for the previous offense, which it did. He then asked the Sangha to give him the trial period of six days for that offense, which it did. When he had completed that trial period, while deserving rehabilitation, he committed one offense of intentional emission of semen, concealed for five days. He asked the Sangha to send him back to the beginning for that offense, which it did. He then asked the Sangha to give him probation for that offense, simultaneous with the probation for the previous offense, which it did. When he had completed that probation, he asked the Sangha to give him the trial period of six days, which it did. He’s now completed that trial period and is asking the Sangha for rehabilitation. If the Sangha is ready, it should rehabilitate him. This is the motion. 

‘Please,\marginnote{19.1.75} venerables, I ask the Sangha to listen. The monk \textsanskrit{Udāyī} had committed one offense of intentional emission of semen, concealed for a half-month. … He’s now completed that trial period and is asking the Sangha for rehabilitation. The Sangha rehabilitates him. Any monk who approves of rehabilitating him should remain silent. Any monk who doesn’t approve should speak up. 

For\marginnote{19.1.81} the second time, I speak on this matter. … For the third time, I speak on this matter. … 

The\marginnote{19.1.83} Sangha has rehabilitated the monk \textsanskrit{Udāyī}. The Sangha approves and is therefore silent. I’ll remember it thus.’” 

\scend{The section on emission of semen is finished. }

\section*{2. Probation }

\subsection*{Simultaneous probations according to the longest duration }

At\marginnote{20.1.1} one time a monk had committed a number of offenses entailing suspension: one concealed for one day, one for two days, one for three days, one for four days, one for five days, one for six days, one for seven days, one for eight days, one for nine days, and one for ten days. He told the monks about this, adding, “What should I do now?” They told the Buddha. 

“Well\marginnote{20.1.9} then, the Sangha should give that monk simultaneous probation according to the longest duration, that is, for the offense that was concealed for ten days. And it should be given like this. That monk should approach the Sangha, arrange his upper robe over one shoulder, pay respect at the feet of the senior monks, squat on his heels, raise his joined palms, and say: 

‘Venerables,\marginnote{20.2.4} I’ve committed a number of offenses entailing suspension: one concealed for one day … one for ten days. I ask the Sangha for simultaneous probation according to the longest duration, that is, for the offense that was concealed for ten days.’ And he should ask a second and a third time. 

A\marginnote{20.2.10} competent and capable monk should then inform the Sangha: 

‘Please,\marginnote{20.2.11} venerables, I ask the Sangha to listen. The monk so-and-so has committed a number of offenses entailing suspension: one concealed for one day … one for ten days. He’s asking the Sangha for simultaneous probation according to the longest duration, that is, for the offense that was concealed for ten days. If the Sangha is ready, it should give him that simultaneous probation. This is the motion. 

Please,\marginnote{20.2.18} venerables, I ask the Sangha to listen. The monk so-and-so has committed a number of offenses entailing suspension: one concealed for one day … one for ten days. He’s asking the Sangha for simultaneous probation according to the longest duration, that is, for the offense that was concealed for ten days. The Sangha gives him that simultaneous probation. Any monk who approves of giving him that simultaneous probation should remain silent. Any monk who doesn’t approve should speak up. 

For\marginnote{20.2.27} the second time, I speak on this matter. … For the third time, I speak on this matter. … 

The\marginnote{20.2.29} Sangha has given monk so-and-so simultaneous probation according to the longest duration, that is, for the offense that was concealed for ten days. The Sangha approves and is therefore silent. I’ll remember it thus.’” 

\subsection*{Simultaneous probations according to the longest duration, that is, for all the offenses that were concealed the longest }

At\marginnote{21.1.1} one time a monk had committed a number of offenses entailing suspension: one concealed for one day, two for two days, three for three days, four for four days, five for five days, six for six days, seven for seven days, eight for eight days, nine for nine days, ten for ten days. He told the monks about this, adding, “What should I do now?” They told the Buddha. 

“Well\marginnote{21.1.8} then, the Sangha should give that monk simultaneous probation according to the longest duration, that is, for all the offenses that were concealed the longest. And it should be given like this. That monk should approach the Sangha, arrange his upper robe over one shoulder, pay respect at the feet of the senior monks, squat on his heels, raise his joined palms, and say: 

‘Venerables,\marginnote{21.1.12} I’ve committed a number of offenses entailing suspension: one concealed for one day … ten for ten days. I ask the Sangha for simultaneous probation according to the longest duration, that is, for all the offenses that were concealed the longest.’ And he should ask a second and a third time. 

A\marginnote{21.1.17} competent and capable monk should then inform the Sangha: 

‘Please,\marginnote{21.1.18} venerables, I ask the Sangha to listen. The monk so-and-so has committed a number of offenses entailing suspension: one concealed for one day … ten for ten days. He’s asking the Sangha for simultaneous probation according to the longest duration, that is, for all the offenses that were concealed the longest. If the Sangha is ready, it should give him that simultaneous probation. This is the motion. 

Please,\marginnote{21.1.24} venerables, I ask the Sangha to listen. The monk so-and-so has committed a number of offenses entailing suspension: one concealed for one day … ten for ten days. He’s asking the Sangha for simultaneous probation according to the longest duration, that is, for all the offenses that were concealed the longest. The Sangha gives him that simultaneous probation. Any monk who approves of giving him that simultaneous probation should remain silent. Any monk who doesn’t approve should speak up. 

For\marginnote{21.1.32} the second time, I speak on this matter. … For the third time, I speak on this matter. … 

The\marginnote{21.1.34} Sangha has given monk so-and-so simultaneous probation according to the longest duration, that is, for all the offenses that were concealed the longest. The Sangha approves and is therefore silent. I’ll remember it thus.’” 

\subsection*{Probation for two months }

At\marginnote{22.1.1} one time a monk had committed two offenses entailing suspension, both concealed for two months. He thought, “I’ve committed two offenses entailing suspension, both concealed for two months. Let me ask the Sangha for probation for two months for one of those offenses.” He asked and got it. While on probation, he was overcome with guilt. He considered what he had done and thought, “Let me ask the Sangha for probation for two months for the other offense too.” 

He\marginnote{22.2.1} told the monks everything that had happened, adding, “What should I do now?” They told the Buddha. 

“Well\marginnote{22.3.1} then, the Sangha should give that monk probation for two months for that offense. And it should be given like this. That monk should approach the Sangha, arrange his upper robe over one shoulder, pay respect at the feet of the senior monks, squat on his heels, raise his joined palms, and say: 

‘Venerables,\marginnote{22.3.4} I had committed two offenses entailing suspension, both concealed for two months. I thought, “I’ve committed two offenses entailing suspension, both concealed for two months. Let me ask the Sangha for probation for two months for one of those offenses.” I asked and got it. While on probation, I was overcome with guilt. I considered what I had done and thought, “Let me ask the Sangha for probation for two months for the other offense too.” And so I ask the Sangha for probation for two months for the other offense, concealed for two months.’ And he should ask a second and a third time. 

A\marginnote{22.4.1} competent and capable monk should then inform the Sangha: 

‘Please,\marginnote{22.4.2} venerables, I ask the Sangha to listen. The monk so-and-so had committed two offenses entailing suspension, both concealed for two months. He thought, “I’ve committed two offenses entailing suspension, both concealed for two months. Let me ask the Sangha for probation for two months for one of those offenses.” He asked and got it. While on probation, he was overcome with guilt. He considered what he had done and thought, “Let me ask the Sangha for probation for two months for the other offense too.” He’s now asking the Sangha for probation for two months for that other offense. If the Sangha is ready, it should give him that probation. This is the motion. 

Please,\marginnote{22.4.21} venerables, I ask the Sangha to listen. The monk so-and-so had committed two offenses entailing suspension, both concealed for two months. He thought, “I’ve committed two offenses entailing suspension, both concealed for two months. Let me ask the Sangha for probation for two months for one of those offenses.” He asked and got it. While on probation, he was overcome with guilt. He considered what he had done and thought, “Let me ask the Sangha for probation for two months for the other offense too.” He’s now asking the Sangha for probation for two months for that other offense. The Sangha gives him that probation. Any monk who approves of giving him that probation should remain silent. Any monk who doesn’t approve should speak up. 

For\marginnote{22.4.41} the second time, I speak on this matter. … For the third time, I speak on this matter. … 

The\marginnote{22.4.43} Sangha has given monk so-and-so probation for two months for the other offense, concealed for two months. The Sangha approves and is therefore silent. I’ll remember it thus.’ 

Starting\marginnote{22.4.45} right there, that monk must stay on probation for two months.” 

\subsection*{Processes for staying on probation for two months }

“It\marginnote{23.1.1} may be that a monk has committed two offenses entailing suspension, both concealed for two months. He thinks, ‘I’ve committed two offenses entailing suspension, both concealed for two months. Let me ask the Sangha for probation for two months for one of those offenses.’ He asks and gets it. While on probation, he’s overcome with guilt. He considers all this and thinks, ‘Let me ask the Sangha for probation for two months for the other offense too.’ He asks and gets it. Starting right there, that monk must stay on probation for two months. 

It\marginnote{23.2.1} may be that a monk has committed two offenses entailing suspension, both concealed for two months. He’s aware of one, but not the other. He asks the Sangha for probation for two months for the offense he’s aware of, which he gets. While on probation, he finds out about the other offense. He considers all this and thinks, ‘Let me ask the Sangha for probation for two months for the other offense too.’ He asks and gets it. Starting right there, that monk must stay on probation for two months. 

It\marginnote{23.3.1} may be that a monk has committed two offenses entailing suspension, both concealed for two months. He remembers one, but not the other. He asks the Sangha for probation for two months for the offense he remembers, which he gets. While on probation, he remembers the other offense. He considers all this and thinks, ‘Let me ask the Sangha for probation for two months for the other offense too.’ He asks and gets it. Starting right there, that monk must stay on probation for two months. 

It\marginnote{23.4.1} may be that a monk has committed two offenses entailing suspension, both concealed for two months. He’s sure of one, but unsure of the other. He asks the Sangha for probation for two months for the offense he’s sure of, which he gets. While on probation, he becomes sure of the other offense too. He considers all this and thinks, ‘Let me ask the Sangha for probation for two months for the other offense too.’ He asks and gets it. Starting right there, that monk must stay on probation for two months. 

“It\marginnote{23.5.1} may be that a monk has committed two offenses entailing suspension, both concealed for two months. He’s aware he has concealed one offense, but not the other.\footnote{I understand \textit{\textsanskrit{jāna}} as an adjective qualifying \textit{\textsanskrit{paṭicchannā}}, in other words, that he knows about the concealing. } He asks the Sangha for probation for two months for both offenses, which he gets. While he’s on probation, another monk arrives. He’s learned, a master of the tradition; he’s an expert on the Teaching, the Monastic Law, and the Key Terms; he’s knowledgeable and competent, has a sense of conscience, and is afraid of wrongdoing and fond of the training. He says, ‘What has this monk committed? Why is he on probation?’ They tell him everything, and he says, ‘The giving of probation for the offense he’s aware of having concealed is legitimate, legal, and has effect. The giving of probation for the offense he’s not aware of having concealed is illegitimate, illegal, and has no effect. For one offense he only deserves the trial period.’ 

It\marginnote{23.6.1} may be that a monk has committed two offenses entailing suspension, both concealed for two months. He remembers concealing one, but not the other. He asks the Sangha for probation for two months for both offenses, which he gets. While he’s on probation, another monk arrives. He’s learned, a master of the tradition; he’s an expert on the Teaching, the Monastic Law, and the Key Terms; he’s knowledgeable and competent, has a sense of conscience, and is afraid of wrongdoing and fond of the training. He says, ‘What has this monk committed? Why is he on probation?’ They tell him everything, and he says, ‘The giving of probation for the offense he remembers concealing is legitimate, legal, and has effect. The giving of probation for the offense he doesn’t remember concealing is illegitimate, illegal, and has no effect. For one offense he only deserves the trial period.’ 

It\marginnote{23.6.22} may be that a monk has committed two offenses entailing suspension, both concealed for two months. He’s sure of having concealed one, but unsure of the other. He asks the Sangha for probation for two months for both offenses, which he gets. While he’s on probation, another monk arrives. He’s learned, a master of the tradition; he’s an expert on the Teaching, the Monastic Law, and the Key Terms; he’s knowledgeable and competent, has a sense of conscience, and is afraid of wrongdoing and fond of the training. He says, ‘What has this monk committed? Why is he on probation?’ They tell him everything, and he says, ‘The giving of probation for the offense he’s sure of having concealed is legitimate, legal, and has effect. The giving of probation for the offense he’s unsure of having concealed is illegitimate, illegal, and has no effect. For one offense he only deserves the trial period.’” 

At\marginnote{24.1.1} one time a monk committed two offenses entailing suspension, both concealed for two months. He thought, “I’ve committed two offenses entailing suspension, both concealed for two months. Let me ask the Sangha for probation for one month for those offenses.” He asked and got it. While on probation, he was overcome with guilt. He considered all this and thought, “Let me ask the Sangha for probation for an additional month for those two offenses.” 

He\marginnote{24.2.1} told the monks everything, adding, “What should I do now?” They told the Buddha. 

“Well\marginnote{24.3.1} then, the Sangha should give that monk probation for an additional month for those two offenses concealed for two months. And it should be given like this. That monk should approach the Sangha, arrange his upper robe over one shoulder, pay respect at the feet of the senior monks, squat on his heels, raise his joined palms, and say: 

‘Venerables,\marginnote{24.3.5} I had committed two offenses entailing suspension, both concealed for two months. I thought, “I’ve committed two offenses entailing suspension, both concealed for two months. Let me ask the Sangha for probation for one month for those offenses.” I asked and got it. While on probation, I was overcome with guilt. I considered all this and thought, “Let me ask the Sangha for probation for an additional month for those two offenses.” I now ask the Sangha for probation for an additional month for those two offenses.’ And he should ask a second and a third time. 

A\marginnote{24.3.23} competent and capable monk should then inform the Sangha: 

‘Please,\marginnote{24.3.24} venerables, I ask the Sangha to listen. The monk so-and-so had committed two offenses entailing suspension, both concealed for two months. He thought, “I’ve committed two offenses entailing suspension, both concealed for two months. Let me ask the Sangha for probation for one month for those offenses.” He asked and got it. While on probation, he was overcome with guilt. He considered all this and thought, “Let me ask the Sangha for probation for an additional month for those two offenses.” He’s now asking the Sangha for probation for an additional month for those two offenses. If the Sangha is ready, it should give him that probation. This is the motion. 

Please,\marginnote{24.3.43} venerables, I ask the Sangha to listen. The monk so-and-so had committed two offenses entailing suspension, both concealed for two months. He thought, “I’ve committed two offenses entailing suspension, both concealed for two months. Let me ask the Sangha for probation for one month for those offenses.” He asked and got it. While on probation, he was overcome with guilt. He considered all this and thought, “Let me ask the Sangha for probation for an additional month for those two offenses.” He’s now asking the Sangha for probation for an additional month for those two offenses. The Sangha gives him that probation. Any monk who approves of giving him that probation should remain silent. Any monk who doesn’t approve should speak up. 

For\marginnote{24.3.63} the second time, I speak on this matter. … For the third time, I speak on this matter. … 

The\marginnote{24.3.65} Sangha has given monk so-and-so probation for an additional month for those two offenses, concealed for two months. The Sangha approves and is therefore silent. I’ll remember it thus.’ 

Counting\marginnote{24.3.67} the previous month, that monk must stay on probation for two months. 

“It\marginnote{25.1.1} may be that a monk has committed two offenses entailing suspension, both concealed for two months. He thinks, ‘I’ve committed two offenses entailing suspension, both concealed for two months. Let me ask the Sangha for probation for one month for those offenses.’ He asks and gets it. While on probation, he’s overcome with guilt. He considers all this and thinks, ‘Let me ask the Sangha for probation for an additional month for those two offenses.’ He asks and gets it. Counting the previous month, that monk must stay on probation for two months. 

It\marginnote{25.2.1} may be that a monk has committed two offenses entailing suspension, both concealed for two months. He’s aware of one month, but not the other. He asks the Sangha for one month probation for the month he’s aware of, which he gets. While on probation, he finds out about the other month. He considers all of this and thinks, ‘Let me ask the Sangha for probation for an additional month for those two offenses.’ He asks and gets it. Counting the previous month, that monk must stay on probation for two months. 

It\marginnote{25.2.16} may be that a monk has committed two offenses entailing suspension, both concealed for two months. He remembers one month, but not the other. He asks the Sangha for one month probation for the month he remembers, which he gets. While on probation, he remembers the other month. He considers all of this and thinks, ‘Let me ask the Sangha for probation for an additional month for those two offenses.’ He asks and gets it. Counting the previous month, that monk must stay on probation for two months. 

It\marginnote{25.2.31} may be that a monk has committed two offenses entailing suspension, both concealed for two months. He’s sure of one month, but unsure of the other. He asks the Sangha for one month probation for the month he’s sure of, which he gets. While on probation, he becomes sure of the other month. He considers all this and thinks, ‘Let me ask the Sangha for probation for an additional month for those two offenses.’ He asks and gets it. Counting the previous month, that monk must stay on probation for two months. 

“It\marginnote{25.3.1} may be that a monk has committed two offenses entailing suspension, both concealed for two months. He’s aware of having concealed for one month, but not for the other. He asks the Sangha for probation for two months for both offenses, which he gets. While he’s on probation, another monk arrives. He’s learned, a master of the tradition; he’s an expert on the Teaching, the Monastic Law, and the Key Terms; he’s knowledgeable and competent, has a sense of conscience, and is afraid of wrongdoing and fond of the training. He says, ‘What has this monk committed? Why is he on probation?’ They tell him everything, and he says, ‘The giving of probation for the month he’s aware of having concealed is legitimate, legal, and has effect. The giving of probation for the month he’s not aware of having concealed is illegitimate, illegal, and has no effect. For one month he only deserves the trial period.’ 

It\marginnote{25.3.21} may be that a monk has committed two offenses entailing suspension, both concealed for two months. He remembers concealing for one month, but not for the other. He asks the Sangha for probation for two months for both offenses, which he gets. While he’s on probation, another monk arrives. He’s learned, a master of the tradition; he’s an expert on the Teaching, the Monastic Law, and the Key Terms; he’s knowledgeable and competent, has a sense of conscience, and is afraid of wrongdoing and fond of the training. He says, ‘What has this monk committed? Why is he on probation?’ They tell him everything, and he says, ‘The giving of probation for the month he remembers concealing is legitimate, legal, and has effect. The giving of probation for the month he doesn’t remember concealing is illegitimate, illegal, and has no effect. For one month he only deserves the trial period.’ 

It\marginnote{25.3.41} may be that a monk has committed two offenses entailing suspension, both concealed for two months. He’s sure of having concealed for one month, but unsure of the other. He asks the Sangha for probation for two months for both offenses, which he gets. While he’s on probation, another monk arrives. He’s learned, a master of the tradition; he’s an expert on the Teaching, the Monastic Law, and the Key Terms; he’s knowledgeable and competent, has a sense of conscience, and is afraid of wrongdoing and fond of the training. He says, ‘What has this monk committed? Why is he on probation?’ They tell him everything, and he says, ‘The giving of probation for the month he’s sure of having concealed is legitimate, legal, and has effect. The giving of probation for the month he’s unsure of having concealed is illegitimate, illegal, and has no effect. For one month he only deserves the trial period.’” 

\subsection*{Purifying probation }

At\marginnote{26.1.1} one time a monk had committed a number of offenses entailing suspension. He did not know the number of offenses or the number of days; he did not remember the number of offenses or the number of days; he was unsure of the number of offenses and the number of days. He told the monks about this, adding, “What should I do now?” They told the Buddha. 

“Well\marginnote{26.1.13} then, the Sangha should give that monk a purifying probation for those offenses. And it should be given like this. That monk should approach the Sangha, arrange his upper robe over one shoulder, pay respect at the feet of the senior monks, squat on his heels, raise his joined palms, and say: 

‘Venerables,\marginnote{26.2.4} I’ve committed a number of offenses entailing suspension. I don’t know the number of offenses or the number of days; I don’t remember the number of offenses or the number of days; I’m unsure of the number of offenses and the number of days. I ask the Sangha for a purifying probation for those offenses.’ And he should ask a second and a third time. 

A\marginnote{26.2.11} competent and capable monk should then inform the Sangha: 

‘Please,\marginnote{26.2.12} venerables, I ask the Sangha to listen. The monk so-and-so has committed a number of offenses entailing suspension. He doesn’t know the number of offenses or the number of days; he doesn’t remember the number of offenses or the number of days; he’s unsure of the number of offenses and the number of days. He’s asking the Sangha for a purifying probation for those offenses. If the Sangha is ready, it should give him a that purifying probation. This is the motion. 

Please,\marginnote{26.2.20} venerables, I ask the Sangha to listen. The monk so-and-so has committed a number of offenses entailing suspension. He doesn’t know the number of offenses or the number of days; he doesn’t remember the number of offenses or the number of days; he’s unsure of the number of offenses and the number of days. He’s asking the Sangha for purifying probation for those offenses. The Sangha gives him that purifying probation. Any monk who approves of giving him that purifying probation should remain silent. Any monk who doesn’t approve should speak up. 

For\marginnote{26.2.29} the second time, I speak on this matter. … For the third time, I speak on this matter. … 

The\marginnote{26.2.31} Sangha has given monk so-and-so a purifying probation for those offenses. The Sangha approves and is therefore silent. I’ll remember it thus.’ 

\subsection*{When to give purifying probation}

“When\marginnote{26.3.1} should purifying probation be given? It should be given: when one doesn’t know the number of offenses, nor the number of days; when one doesn’t remember the number of offenses, nor the number of days; when one is unsure of the number of offenses and the number of days. 

It\marginnote{26.3.6} should be given: when one knows the number of offenses, but not the number of days; when one remembers the number of offenses, but not the number of days; when one is sure of the number of offenses, but not the number of days. 

It\marginnote{26.3.10} should be given: when one knows some of the offenses but not others, and not the number of days; when one remembers some of the offenses but not others, and not the number of days; when one is sure of some of the offenses but not others, and not the number of days. 

It\marginnote{26.3.14} should be given: when one doesn’t know the number of offenses, but one knows some of the days but not others; when one doesn’t remember the number of offenses, but one remembers some of the days but not others; when one is unsure of the number of offenses, but one is sure of some of the days but not others. 

It\marginnote{26.3.18} should be given: when one knows the number of offenses, and one knows some of the days but not others; when one remembers the number of offenses, and one remembers some of the days but not others; when one is sure of the number of offenses, and one is sure of some of the days but not others. 

It\marginnote{26.3.22} should be given: when one knows some of the offenses but not others, and one knows some of the days but not others; when one remembers some of the offenses but not others, and one remembers some of the days but not others; when one is sure of some of the offenses but not others, and one is sure of some of the days but not others.” 

\subsection*{When to give regular probation}

“When\marginnote{26.4.1} should probation be given? It should be given: when one knows the number of offenses and the number of days; when one remembers the number of offenses and the number of days; when one is sure of the number of offenses and the number of days. 

It\marginnote{26.4.6} should be given: when one doesn’t know the number of offenses, but one knows the number of days; when one doesn’t remember the number of offenses, but one remembers the number of days;  when one is unsure of the number of offenses, but sure of the number of days. 

It\marginnote{26.4.10} should be given: when one knows some of the offenses but not others, and one knows the number of days; when one remembers some of the offenses but not others, and one remembers the number of days; when one is sure of some of the offenses but not others, and one is sure of the number of days.” 

\scend{The section on probation is finished. }

\section*{3. The group of forty }

At\marginnote{27.1.1} one time a monk on probation disrobed.\footnote{For an explanation of the rendering “disrobe” for \textit{vibbhamati}, see Appendix of Technical Terms. } He then came back and asked the monks for the full ordination. They told the Buddha. 

“It\marginnote{27.1.4} may be, monks, that a monk on probation disrobes. For one who’s disrobed, the probation is suspended. If he’s given the full ordination again, he continues the previous probationary process right away. The probation that was given is valid. The probation he’s already undertaken is valid. And the remainder is to be undertaken. 

It\marginnote{27.1.8} may be that a monk on probation becomes a novice monk. For a novice monk, the probation is suspended. If he’s given the full ordination again, he continues the previous probationary process right away. The probation that was given is valid. The probation he’s already undertaken is valid. And the remainder is to be undertaken. 

It\marginnote{27.1.12} may be that a monk on probation goes insane. For one who’s insane, the probation is suspended. If he regains his sanity, he continues the previous probationary process right away. The probation that was given is valid. The probation he’s already undertaken is valid. And the remainder is to be undertaken. 

It\marginnote{27.1.16} may be that a monk on probation becomes deranged. For one who’s deranged, the probation is suspended. If he regains his sanity, he continues the previous probationary process right away. The probation that was given is valid. The probation he’s already undertaken is valid. And the remainder is to be undertaken. 

It\marginnote{27.1.20} may be that a monk on probation is overwhelmed by pain. For one who’s overwhelmed by pain, the probation is suspended. If he recovers, he continues the previous probationary process right away. The probation that was given is valid. The probation he’s already undertaken is valid. And the remainder is to be undertaken. 

It\marginnote{27.1.24} may be that a monk on probation is ejected for not recognizing an offense. For one who’s ejected, the probation is suspended. If he’s readmitted, he continues the previous probationary process right away. The probation that was given is valid. The probation he’s already undertaken is valid. And the remainder is to be undertaken. 

It\marginnote{27.1.28} may be that a monk on probation is ejected for not making amends for an offense. For one who’s ejected, the probation is suspended. If he’s readmitted, he continues the previous probationary process right away. The probation that was given is valid. The probation he’s already undertaken is valid. And the remainder is to be undertaken. 

It\marginnote{27.1.32} may be that a monk on probation is ejected for not giving up a bad view. For one who’s ejected, the probation is suspended. If he’s readmitted, he continues the previous probationary process right away. The probation that was given is valid. The probation he’s already undertaken is valid. And the remainder is to be undertaken.” 

“It\marginnote{27.2.1} may be that a monk deserving to be sent back to the beginning disrobes. For one who’s disrobed, the sending back to the beginning is suspended. If he’s given the full ordination again, he continues the previous probationary process right away. The probation that was given is valid. The probation he’s already undertaken is valid. And he’s to be sent back to the beginning. 

It\marginnote{27.2.6} may be that a monk deserving to be sent back to the beginning becomes a novice monk, goes insane, becomes deranged, is overwhelmed by pain, is ejected for not recognizing an offense, is ejected for not making amends for an offense, or is ejected for not giving up a bad view. For one who’s ejected, the sending back to the beginning is suspended. If he’s readmitted, he continues the previous probationary process right away. The probation that was given is valid. The probation he’s already undertaken is valid. And he’s to be sent back to the beginning.” 

“It\marginnote{27.3.1} may be that a monk deserving the trial period disrobes. For one who’s disrobed, the giving of the trial period is suspended. If he’s given the full ordination again, he continues the previous probationary process right away. The probation that was given is valid. The probation he’s undertaken is valid. And he’s to be given the trial period. 

It\marginnote{27.3.6} may be that a monk deserving the trial period becomes a novice monk, goes insane, becomes deranged, is overwhelmed by pain, is ejected for not recognizing an offense, is ejected for not making amends for an offense, or is ejected for not giving up a bad view. For one who’s ejected, the giving of the trial period is suspended. If he’s readmitted, he continues the previous probationary process right away. The probation that was given is valid. The probation he’s undertaken is valid. And he’s to be given the trial period.” 

“It\marginnote{27.4.1} may be that a monk undertaking the trial period disrobes. For one who’s disrobed, the trial period is suspended. If he’s given the full ordination again, he continues the previous probationary process right away. The probation that was given is valid. The probation he’s undertaken is valid. The trial period that was given is valid. The trial period he’s already undertaken is valid. And the remainder is to be undertaken. 

It\marginnote{27.4.6} may be that a monk undertaking the trial period becomes a novice monk, goes insane, becomes deranged, is overwhelmed by pain, is ejected for not recognizing an offense, is ejected for not making amends for an offense, or is ejected for not giving up a bad view. For one who’s ejected, the trial period is suspended. If he’s readmitted, he continues the previous probationary process right away. The probation that was given is valid. The probation he’s undertaken is valid. The trial period that was given is valid. The trial period he’s undertaken is valid. And the remainder is to be undertaken.” 

“It\marginnote{27.5.1} may be that a monk deserving rehabilitation disrobes. For one who’s disrobed, the rehabilitation is suspended. If he’s given the full ordination again, he continues the previous probationary process right away. The probation that was given is valid. The probation he’s undertaken is valid. The trial period that was given is valid. The trial period he’s undertaken is valid. And he’s to be rehabilitated. 

It\marginnote{27.5.7} may be that a monk deserving rehabilitation becomes a novice monk, goes insane, becomes deranged, is overwhelmed by pain, is ejected for not recognizing an offense, is ejected for not making amends for an offense, or is ejected for not giving up a bad view. For one who’s ejected, the rehabilitation is suspended. If he’s readmitted, he continues the previous probationary process right away. The probation that was given is valid. The probation he’s undertaken is valid. The trial period that was given is valid. The trial period he’s undertaken is valid. And he’s to be rehabilitated.” 

\scend{The group of forty is finished. }

\section*{4. The group of thirty-six }

“It\marginnote{28.1.1} may be that a monk on probation commits a number of offenses entailing suspension, unconcealed and specified.\footnote{“Specified” renders \textit{\textsanskrit{parimāṇā}}, literally, “limited”. Sp 4.165: \textit{Tattha “\textsanskrit{antarā} \textsanskrit{sambahulā} \textsanskrit{āpattiyo} \textsanskrit{āpajjati} \textsanskrit{parimāṇā} \textsanskrit{appaṭicchannāyo}”\textsanskrit{tiādīsu} \textsanskrit{āpattiparicchedavasena} \textsanskrit{parimāṇāyo} ceva \textsanskrit{appaṭicchannāyo} \textsanskrit{cāti} attho}, “Here the meaning of \textit{\textsanskrit{antarā} \textsanskrit{sambahulā} \textsanskrit{āpattiyo} \textsanskrit{āpajjati} \textsanskrit{parimāṇā} \textsanskrit{appaṭicchannāyo}}, etc., is that they are specified and unconcealed on account of the offenses being determined.” } He’s to be sent back to the beginning. 

It\marginnote{28.1.3} may be that a monk on probation commits a number of offenses entailing suspension, concealed and specified. He’s to be sent back to the beginning. He should then be given probation according to the length of the concealment of those offenses and simultaneously with the probation for the previous offense. 

It\marginnote{28.1.6} may be that a monk on probation commits a number of offenses entailing suspension, both concealed and unconcealed and specified. He’s to be sent back to the beginning. He should then be given probation according to the length of the concealment of those offenses and simultaneously with the probation for the previous offense. 

It\marginnote{28.1.9} may be that a monk on probation commits a number of offenses entailing suspension, unconcealed and unspecified … concealed and unspecified … both concealed and unconcealed and unspecified … unconcealed and both specified and unspecified … concealed and both specified and unspecified … both concealed and unconcealed and both specified and unspecified. He’s to be sent back to the beginning. He should then be given probation according to the length of the concealment of those offenses and simultaneously with the probation for the previous offense. 

It\marginnote{28.2.1} may be that a monk who deserves the trial period … who’s undertaking the trial period … (to be expanded as for probation) …  who deserves rehabilitation commits a number of offenses entailing suspension, unconcealed and specified … concealed and specified … both concealed and unconcealed and specified … unconcealed and unspecified … concealed and unspecified … both concealed and unconcealed and unspecified … unconcealed and both specified and unspecified … concealed and both specified and unspecified … both concealed and unconcealed and both specified and unspecified. He’s to be sent back to the beginning. He should then be given probation according to the length of the concealment of those offenses and simultaneously with the probation for the previous offense.” 

\scend{The group of thirty-six is finished. }

\section*{5. The group of one hundred on the trial period }

“It\marginnote{29.1.1} may be that a monk commits a number of unconcealed offenses entailing suspension and then disrobes. Being reordained, he doesn’t conceal them. He should be given the trial period. 

It\marginnote{29.1.4} may be that a monk commits a number of unconcealed offenses entailing suspension and then disrobes. Being reordained, he conceals them. He should be given probation according to the length of the subsequent concealment of those offenses, and he should then be given the trial period.\footnote{Sp 4.166: \textit{\textsanskrit{Pacchimasmiṁ} \textsanskrit{āpattikkhandheti} ekova so \textsanskrit{āpattikkhandho}, \textsanskrit{pacchā} \textsanskrit{chāditattā} pana “\textsanskrit{pacchimasmiṁ} \textsanskrit{āpattikkhandhe}”ti \textsanskrit{vuttaṁ}. Purimasminti \textsanskrit{etthāpi} eseva nayo}, “\textit{\textsanskrit{Pacchimasmiṁ} \textsanskrit{āpattikkhandhe}}: there is just one group of offenses. But since it is concealed subsequently, it is called \textit{\textsanskrit{pacchimasmiṁ} \textsanskrit{āpattikkhandhe}}.” The word “group” is redundant on translation. } 

It\marginnote{29.1.7} may be that a monk commits a number of concealed offenses entailing suspension and then disrobes. Being reordained, he doesn’t conceal them. He should be given probation according to the length of the earlier concealment of those offenses, and he should then be given the trial period. 

It\marginnote{29.1.10} may be that a monk commits a number of concealed offenses entailing suspension and then disrobes. Being reordained, he conceals them. He should be given probation according to the length of the earlier and subsequent concealment of those offenses, and he should then be given the trial period. 

“It\marginnote{29.2.1} may be that a monk commits a number of offenses entailing suspension, both concealed and unconcealed, and then disrobes. Being reordained, he doesn’t conceal any of them. He should be given probation according to the length of the earlier concealment of those offenses, and he should then be given the trial period. 

It\marginnote{29.2.6} may be that a monk commits a number of offenses entailing suspension, both concealed and unconcealed, and then disrobes. Being reordained, he doesn’t conceal those offenses he previously concealed, but conceals those offenses he previously didn’t conceal. He should be given probation according to the length of the earlier and subsequent concealment of those offenses, and he should then be given the trial period. 

It\marginnote{29.2.11} may be that a monk commits a number of offenses entailing suspension, both concealed and unconcealed, and then disrobes. Being reordained, he conceals those offenses he previously concealed, but doesn’t conceal those offenses he previously didn’t conceal. He should be given probation according to the length of the earlier and subsequent concealment of those offenses, and he should then be given the trial period. 

It\marginnote{29.2.16} may be that a monk commits a number of offenses entailing suspension, both concealed and unconcealed, and then disrobes. Being reordained, he conceals all of them. He should be given probation according to the length of the earlier and subsequent concealment of those offenses, and he should then be given the trial period. 

“It\marginnote{29.3.1} may be that a monk commits a number of offenses entailing suspension. He’s aware of some of them, but not others. He conceals the offenses he’s aware of, but not those he’s not aware of. He then disrobes. Being reordained and having found out about all of them, he conceals none of them. He should be given probation according to the length of the earlier concealment of those offenses, and he should then be given the trial period. 

It\marginnote{29.3.8} may be that a monk commits a number of offenses entailing suspension. He’s aware of some of them, but not others. He conceals the offenses he’s aware of, but not those he’s not aware of. He then disrobes. Being reordained and having found out about all of them, he doesn’t conceal those offenses he was previously aware of, but conceals those he wasn’t aware of. He should be given probation according to the length of the earlier and subsequent concealment of those offenses, and he should then be given the trial period. 

It\marginnote{29.3.15} may be that a monk commits a number of offenses entailing suspension. He’s aware of some of them, but not others. He conceals the offenses he’s aware of, but not those he’s not aware of. He then disrobes. Being reordained and having found out about all of them, he conceals those offenses he was previously aware of, but not those he wasn’t aware of. He should be given probation according to the length of the earlier and subsequent concealment of those offenses, and he should then be given the trial period. 

It\marginnote{29.3.22} may be that a monk commits a number of offenses entailing suspension. He’s aware of some of them, but not others. He conceals the offenses he’s aware of, but not those he’s not aware of. He then disrobes. Being reordained and having found out about all of them, he conceals all of them. He should be given probation according to the length of the earlier and subsequent concealment of those offenses, and he should then be given the trial period. 

“It\marginnote{29.4.1} may be that a monk commits a number of offenses entailing suspension. He remembers some of them, but not others. He conceals the offenses he remembers, but not those he doesn’t remember. He then disrobes. Being reordained and having remembered all of them, he conceals none of them. He should be given probation according to the length of the earlier concealment of those offenses, and he should then be given the trial period. 

It\marginnote{29.4.8} may be that a monk commits a number of offenses entailing suspension. He remembers some of them, but not others. He conceals the offenses he remembers, but not those he doesn’t remember. He then disrobes. Being reordained and having remembered all of them, he doesn’t conceal those offenses he previously remembered, but conceals those he didn’t remember. He should be given probation according to the length of the earlier and subsequent concealment of those offenses, and he should then be given the trial period. 

It\marginnote{29.4.15} may be that a monk commits a number of offenses entailing suspension. He remembers some of them, but not others. He conceals the offenses he remembers, but not those he doesn’t remember. He then disrobes. Being reordained and having remembered all of them, he conceals those offenses he previously remembered, but not those he didn’t remember. He should be given probation according to the length of the earlier and subsequent concealment of those offenses, and he should then be given the trial period. 

It\marginnote{29.4.22} may be that a monk commits a number of offenses entailing suspension. He remembers some of them, but not others. He conceals the offenses he remembers, but not those he doesn’t remember. He then disrobes. Being reordained and having remembered all of them, he conceals all of them. He should be given probation according to the length of the earlier and subsequent concealment of those offenses, and he should then be given the trial period. 

“It\marginnote{29.5.1} may be that a monk commits a number of offenses entailing suspension. He’s sure of some of them, but unsure of others. He conceals the offenses he’s sure of, but not those he’s unsure of. He then disrobes. Being reordained and having become sure of all of them, he conceals none of them. He should be given probation according to the length of the earlier concealment of those offenses, and he should then be given the trial period. 

It\marginnote{29.5.8} may be that a monk commits a number of offenses entailing suspension. He’s sure of some of them, but unsure of others. He conceals the offenses he’s sure of, but not those he’s unsure of. He then disrobes. Being reordained and having become sure of all of them, he doesn’t conceal those offenses he was previously sure of, but conceals those he was unsure of. He should be given probation according to the length of the earlier and subsequent concealment of those offenses, and he should then be given the trial period. 

It\marginnote{29.5.15} may be that a monk commits a number of offenses entailing suspension. He’s sure of some of them, but unsure of others. He conceals the offenses he’s sure of, but not those he’s unsure of. He then disrobes. Being reordained and having become sure of all of them, he conceals those offenses he was previously sure of, but not those he was unsure of. He should be given probation according to the length of the earlier and subsequent concealment of those offenses, and he should then be given the trial period. 

It\marginnote{29.5.22} may be that a monk commits a number of offenses entailing suspension. He’s sure of some of them, but unsure of others. He conceals the offenses he’s sure of, but not those he’s unsure of. He then disrobes. Being reordained and having become sure of all of them, he conceals all of them. He should be given probation according to the length of the earlier and subsequent concealment of those offenses, and he should then be given the trial period. 

“It\marginnote{30.1.1} may be that a monk commits a number of unconcealed offenses entailing suspension and then becomes a novice monk … goes insane … becomes deranged … (to be expanded as above) …  is overwhelmed by pain … both concealed and unconcealed … he’s aware of some of them, but not others … he remembers some of them, but not others … he’s sure of some of them, but unsure of others. He conceals the offenses he’s sure of, but not those he’s unsure of. He then becomes overwhelmed by pain. Having recovered and having become sure of all of them, he conceals none of them. … having become sure of all of them, he doesn’t conceal those offenses he was previously sure of, but conceals those he was unsure of. … having become sure of all of them, he conceals those offenses he was previously sure of, but not those he was unsure of. … having become sure of all of them, he conceals all of them. He should be given probation according to the length of the earlier and subsequent concealment of those offenses, and he should then be given the trial period.” 

\scend{The group of one hundred on the trial period is finished. }

\section*{6. The group of four hundred on simultaneous probation with sending back to the beginning }

“It\marginnote{31.1.1} may be that a monk on probation commits a number of unconcealed offenses entailing suspension and then disrobes. Being reordained, he doesn’t conceal those offenses. He’s to be sent back to the beginning. 

It\marginnote{31.1.4} may be that a monk on probation commits a number of unconcealed offenses entailing suspension and then disrobes. Being reordained, he conceals those offenses. He’s to be sent back to the beginning. He’s then to be given probation according to the length of the concealment of those offenses and simultaneously with the probation for the previous offenses. 

It\marginnote{31.1.8} may be that a monk on probation commits a number of concealed offenses entailing suspension and then disrobes. Being reordained, he doesn’t conceal those offenses. He’s to be sent back to the beginning. He’s then to be given probation according to the length of the concealment of those offenses and simultaneously with the probation for the previous offenses. 

It\marginnote{31.1.12} may be that a monk on probation commits a number of concealed offenses entailing suspension and then disrobes. Being reordained, he conceals those offenses. He’s to be sent back to the beginning. He’s then to be given probation according to the length of the concealment of those offenses and simultaneously with the probation for the previous offenses. 

“It\marginnote{31.2.1} may be that a monk on probation commits a number of offenses entailing suspension, both concealed and unconcealed, and then disrobes. Being reordained, he conceals none of those offenses. He’s to be sent back to the beginning. He’s then to be given probation according to the length of the concealment of those offenses and simultaneously with the probation for the previous offenses. 

It\marginnote{31.2.7} may be that a monk on probation commits a number of offenses entailing suspension, both concealed and unconcealed, and then disrobes. Being reordained, he doesn’t conceal those offenses he previously concealed, but conceals those he previously didn’t conceal. He’s to be sent back to the beginning. He’s then to be given probation according to the length of the concealment of those offenses and simultaneously with the probation for the previous offenses. 

It\marginnote{31.2.13} may be that a monk on probation commits a number of offenses entailing suspension, both concealed and unconcealed, and then disrobes. Being reordained, he conceals those offenses he previously concealed, but doesn’t conceal those he previously didn’t conceal. He’s to be sent back to the beginning. He’s then to be given probation according to the length of the concealment of those offenses and simultaneously with the probation for the previous offenses. 

It\marginnote{31.2.19} may be that a monk on probation commits a number of offenses entailing suspension, both concealed and unconcealed, and then disrobes. Being reordained, he conceals all of those offenses. He’s to be sent back to the beginning. He’s then to be given probation according to the length of the concealment of those offenses and simultaneously with the probation for the previous offenses. 

“It\marginnote{31.3.1} may be that a monk on probation commits a number of offenses entailing suspension. He’s aware of some of them, but not others. He conceals the offenses he’s aware of, but not those he isn’t aware of. He then disrobes. Being reordained and having found out about all of them, he conceals none of them. He’s to be sent back to the beginning. He’s then to be given probation according to the length of the concealment of those offenses and simultaneously with the probation for the previous offenses. 

It\marginnote{31.3.9} may be that a monk on probation commits a number of offenses entailing suspension. He’s aware of some of them, but not others. He conceals the offenses he’s aware of, but not those he isn’t aware of. He then disrobes. Being reordained and having found out about all of them, he doesn’t conceal those offenses he was previously aware of, but conceals those he wasn’t aware of. He’s to be sent back to the beginning. He’s then to be given probation according to the length of the concealment of those offenses and simultaneously with the probation for the previous offenses. 

It\marginnote{31.3.17} may be that a monk on probation commits a number of offenses entailing suspension. He’s aware of some of them, but not others. He conceals the offenses he’s aware of, but not those he isn’t aware of. He then disrobes. Being reordained and having found out about all of them, he conceals those offenses he was previously aware of, but not those he wasn’t aware of. He’s to be sent back to the beginning. He’s then to be given probation according to the length of the concealment of those offenses and simultaneously with the probation for the previous offenses. 

It\marginnote{31.3.25} may be that a monk on probation commits a number of offenses entailing suspension. He’s aware of some of them, but not others. He conceals the offenses he’s aware of, but not those he isn’t aware of. He then disrobes. Being reordained and having found out about all of them, he conceals all of them. He’s to be sent back to the beginning. He’s then to be given probation according to the length of the concealment of those offenses and simultaneously with the probation for the previous offenses. 

“It\marginnote{31.3.32} may be that a monk on probation commits a number of offenses entailing suspension. He remembers some of them, but not others. He conceals the offenses he remembers, but not those he doesn’t remember. He then disrobes. Being reordained and having remembered all of them, he conceals none of them. He’s to be sent back to the beginning. He’s then to be given probation according to the length of the concealment of those offenses and simultaneously with the probation for the previous offenses. 

It\marginnote{31.3.40} may be that a monk on probation commits a number of offenses entailing suspension. He remembers some of them, but not others. He conceals the offenses he remembers, but not those he doesn’t remember. He then disrobes. Being reordained and having remembered all the offenses, he doesn’t conceal those offenses he previously remembered, but conceals those he didn’t remember. He’s to be sent back to the beginning. He’s then to be given probation according to the length of the concealment of those offenses and simultaneously with the probation for the previous offenses. 

It\marginnote{31.3.48} may be that a monk on probation commits a number of offenses entailing suspension. He remembers some of them, but not others. He conceals the offenses he remembers, but not those he doesn’t remember. He then disrobes. Being reordained and having remembered all of them, he conceals those offenses he previously remembered, but not those he didn’t remember. He’s to be sent back to the beginning. He’s then to be given probation according to the length of the concealment of those offenses and simultaneously with the probation for the previous offenses. 

It\marginnote{31.3.56} may be that a monk on probation commits a number of offenses entailing suspension. He remembers some of them, but not others. He conceals the offenses he remembers, but not those he doesn’t remember. He then disrobes. Being reordained and having remembered all of them, he conceals all of them. He’s to be sent back to the beginning. He’s then to be given probation according to the length of the concealment of those offenses and simultaneously with the probation for the previous offenses. 

“It\marginnote{31.3.64} may be that a monk on probation commits a number of offenses entailing suspension. He’s sure of some of them, but unsure of others. He conceals the offenses he’s sure of, but not those he’s unsure of. He then disrobes. Being reordained and having become sure of all of them, he conceals none of them. He’s to be sent back to the beginning. He’s then to be given probation according to the length of the concealment of those offenses and simultaneously with the probation for the previous offenses. 

It\marginnote{31.3.72} may be that a monk on probation commits a number of offenses entailing suspension. He’s sure of some of them, but unsure of others. He conceals the offenses he’s sure of, but not those he’s unsure of. He then disrobes. Being reordained and having become sure of all of them, he doesn’t conceal those offenses he was previously sure of, but conceals those he was unsure of. He’s to be sent back to the beginning. He’s then to be given probation according to the length of the concealment of those offenses and simultaneously with the probation for the previous offenses. 

It\marginnote{31.3.80} may be that a monk on probation commits a number of offenses entailing suspension. He’s sure of some of them, but unsure of others. He conceals the offenses he’s sure of, but not those he’s unsure of. He then disrobes. Being reordained and having become sure of all of them, he conceals those offenses he was previously sure of, but not those he was unsure of. He’s to be sent back to the beginning. He’s then to be given probation according to the length of the concealment of those offenses and simultaneously with the probation for the previous offenses. 

It\marginnote{31.3.88} may be that a monk on probation commits a number of offenses entailing suspension. He’s sure of some of them, but unsure of others. He conceals the offenses he’s sure of, but not those he’s unsure of. He then disrobes. Being reordained and having become sure of all of them, he conceals all of them. He’s to be sent back to the beginning. He’s then to be given probation according to the length of the concealment of those offenses and simultaneously with the probation for the previous offenses. 

“It\marginnote{31.3.96} may be that a monk on probation commits a number of unconcealed offenses entailing suspension and then becomes a novice monk … goes insane … becomes deranged … is overwhelmed by pain … both concealed and unconcealed …  (to be expanded as above) …  he’s aware of some of the offenses, but not others … he remembers some of the offenses, but not others … he’s sure of some of the offenses, but unsure of others. He conceals the offenses he’s sure of, but not those he’s unsure of. He then becomes overwhelmed by pain. Having recovered and having become sure of all of them, he conceals none of them. … having become sure of all of them, he doesn’t conceal those offenses he was previously sure of, but conceals those he was unsure of. … having become sure of all of them, he conceals those offenses he was previously sure of, but not those he was unsure of. … having become sure of all of them, he conceals all of them. He’s to be sent back to the beginning. He’s then to be given probation according to the length of the concealment of those offenses and simultaneously with the probation for the previous offenses. 

“It\marginnote{32.1.1} may be that a monk who deserves the trial period … who’s undertaking the trial period … who deserves rehabilitation commits a number of unconcealed offenses entailing suspension and then disrobes. … 

(‘who\marginnote{32.1.4} deserves the trial period’, ‘who’s undertaking the trial period’, and ‘who deserves rehabilitation’ are to be expanded as for ‘on probation’) … 

It\marginnote{32.1.5} may be that a monk who deserves rehabilitation commits a number of unconcealed offenses entailing suspension and then becomes a novice monk … goes insane … becomes deranged … is overwhelmed by pain … both concealed and unconcealed … he’s aware of some of the offenses, but not others … he remembers some of the offenses, but not others … he’s sure of some of the offenses, but unsure of others. He conceals the offenses he’s sure of, but not those he’s unsure of. He then becomes overwhelmed by pain. Having recovered and having become sure of all the offenses, he conceals none of them. … having become sure of all of them, he doesn’t conceal those offenses he was previously sure of, but conceals those he was unsure of. … having become sure of all of them, he conceals those offenses he was previously sure of, but not those he was unsure of. … having become sure of all of them, he conceals all of them. He’s to be sent back to the beginning. He’s then to be given probation according to the length of the concealment of those offenses and simultaneously with the probation for the previous offenses.” 

\scend{The group of four hundred on simultaneous probation with being sent back to the beginning is finished. }

\section*{7. The group of eight sections on “specified”, etc. }

“It\marginnote{33.1.1} may be that a monk commits a number of offenses entailing suspension, unconcealed and specified … unconcealed and unspecified … unconcealed and having the same name … unconcealed and having different names … unconcealed and of the same kind … unconcealed and of different kinds … unconcealed and of the same sort … unconcealed and of different sorts and then disrobes. …\footnote{“Of the same sort” renders \textit{\textsanskrit{vavatthitā}}, while “of various sorts” renders \textit{\textsanskrit{sambhinnā}}. Sp 4.180: \textit{\textsanskrit{Vavatthitā} \textsanskrit{sambhinnāti} \textsanskrit{sabhāgavisabhāgānamevetaṁ} \textsanskrit{pariyāyavacanaṁ}}, “‘Of the same sort’ and ‘of different sorts’ are synonymous with ‘of the same kind’ and ‘of different kinds’.” } (to be expanded as above) …” 

\scend{The group of eight sections on “specified”, etc., is finished. }

\section*{8. The group of eleven sections on two monks }

Two\marginnote{34.1.1} monks have each committed an offense entailing suspension, and they regard it as such. One conceals his offense, but not the other. The one who conceals it is to confess an offense of wrong conduct. He should be given probation according to the length of that concealment, and both should then be given the trial period. 

Two\marginnote{34.1.6} monks have each committed an offense entailing suspension, but they are unsure of it. One conceals his offense, but not the other. The one who conceals it is to confess an offense of wrong conduct. He should be given probation according to the length of that concealment, and both should then be given the trial period. 

Two\marginnote{34.1.11} monks have each committed an offense entailing suspension, but they regard it as mixed with other offenses.\footnote{\textit{\textsanskrit{Missakadiṭṭhino}}, literally, “They see (it) as mixed.” Sp 4.181: \textit{Tattha missakanti \textsanskrit{thullaccayādīhi} \textsanskrit{missakaṁ}}, “Therein ‘mixed’ means mixed with serious offenses, etc.” } One conceals his offense, but not the other. The one who conceals it is to confess an offense of wrong conduct. He should be given probation according to the length of that concealment, and both should then be given the trial period. 

Two\marginnote{34.1.16} monks have each committed an offense entailing suspension mixed with other offenses,\footnote{The Pali just says \textit{missaka}, “mixed”, without specifying the offenses. However, since both probation and the trial period are mentioned further down, it is clear that an offense entailing suspension is included. } but they regard them as an offense entailing suspension. One conceals his offenses, but not the other. The one who conceals them is to confess an offense of wrong conduct. He should be given probation according to the length of that concealment, and both should then be given the trial period. 

Two\marginnote{34.1.21} monks have each committed an offense entailing suspension mixed with other offenses, and they regard them as such. One conceals his offenses, but not the other. The one who conceals them is to confess an offense of wrong conduct. He should be given probation according to the length of that concealment, and both should then be given the trial period. 

Two\marginnote{34.1.26} monks have each committed a light offense,\footnote{Sp 4.181: \textit{Suddhakanti \textsanskrit{saṅghādisesaṁ} \textsanskrit{vinā} \textsanskrit{lahukāpattikkhandhameva}}, “\textit{Suddhaka}: just belonging to the group of light offenses without an offense entailing suspension.” } but they regard it as an offense entailing suspension. One conceals his offense, but not the other. The one who conceals it is to confess an offense of wrong conduct. Both are then to be dealt with according to the rule. 

Two\marginnote{34.1.31} monks have each committed a light offense, and they regard it as such. One conceals his offense, but not the other. The one who conceals it is to confess an offense of wrong conduct. Both are then to be dealt with according to the rule. 

Two\marginnote{34.2.1} monks have each committed an offense entailing suspension, and they regard it as such. One thinks, “I’ll inform,” while the other thinks, “I won’t inform.” He then conceals it during the first part of the night, during the second part of the night, and during the third part of the night. If he’s still concealing it at dawn, he has committed an offense and is to confess an offense of wrong conduct. He should be given probation according to the length of that concealment, and both should then be given the trial period. 

Two\marginnote{34.2.8} monks have each committed an offense entailing suspension, and they regard it as such. They go, thinking, “We’ll inform.” On their way, one of them decides to conceal it, thinking, “I won’t inform.”\footnote{Sp-\textsanskrit{ṭ} 4.181: \textit{Makkhadhammo \textsanskrit{nāma} \textsanskrit{chādetukāmatā}}, “\textit{Makkhadhammo}: desiring to cover over.” } He then conceals it during the first part of the night, during the second part of the night, and during the third part of the night. If he’s still concealing it at dawn, he has committed an offense and is to confess an offense of wrong conduct. He should be given probation according to the length of that concealment, and both should then be given the trial period. 

Two\marginnote{34.2.16} monks have each committed an offense entailing suspension, and they regard it as such. They then go insane.\footnote{Although the Pali uses the present tense, the insanity must be subsequent to the committing of the offense. Otherwise there would be no offense. } When they regain their sanity, one conceals his offense, but not the other. The one who conceals it is to confess an offense of wrong conduct. He should be given probation according to the length of that concealment, and both should then be given the trial period. 

Two\marginnote{34.2.22} monks have each committed an offense entailing suspension. During the recitation of the Monastic Code, they say, “Just now did we find out that this rule too has come down in the Monastic Code, is included in the Monastic Code, and comes up for recitation every half-month.” They regard it as an offense entailing suspension. One conceals his offense, but not the other. The one who conceals it is to confess an offense of wrong conduct. He should be given probation according to the length of that concealment, and both should then be given the trial period. 

\scend{The group of eleven sections on two monks is finished. }

\section*{9. The group of nine on unpurified with sending back to the beginning }

“It\marginnote{35.1.1} may be, monks, that a monk has committed a number of offenses entailing suspension, both specified and unspecified, both having the same name and having different names, both of the same kind and of different kinds, both of the same sort and of different sorts. He asks the Sangha for simultaneous probation for those offenses, which he gets. While on probation, he commits a number of offenses entailing suspension, unconcealed and specified. He asks the Sangha to send him back to the beginning for those offenses, which it does. The legal procedure is legitimate, irreversible, and fit to stand. The Sangha gives him the simultaneous probation legitimately,\footnote{It is not clear from the Pali whether the sending back to the beginning and the giving of simultaneous probation are supposed to happen in one and the same legal procedure or in two different ones. It would seem, however, that either way would be acceptable, so long as the procedure is phrased appropriately. For the rendering “irreversible” for \textit{akuppa}, see \textit{kuppa} in Appendix of Technical Terms. } but then gives him the trial period and the rehabilitation illegitimately. He’s not purified of those offenses. 

It\marginnote{35.1.9} may be that a monk has committed a number of offenses entailing suspension, both specified and unspecified, both having the same name and having different names, both of the same kind and of different kinds, both of the same sort and of different sorts. He asks the Sangha for simultaneous probation for those offenses, which he gets. While on probation, he commits a number of offenses entailing suspension, concealed and specified. He asks the Sangha to send him back to the beginning for those offenses, which it does. The legal procedure is legitimate, irreversible, and fit to stand. The Sangha gives him the simultaneous probation legitimately, but then gives him the trial period and the rehabilitation illegitimately. He’s not purified of those offenses. 

It\marginnote{35.1.17} may be that a monk has committed a number of offenses entailing suspension, both specified and unspecified, both having the same name and having different names, both of the same kind and of different kinds, both of the same sort and of different sorts. He asks the Sangha for simultaneous probation for those offenses, which he gets. While on probation, he commits a number of offenses entailing suspension, both concealed and unconcealed and specified. He asks the Sangha to send him back to the beginning for those offenses, which it does. The legal procedure is legitimate, irreversible, and fit to stand. The Sangha gives him the simultaneous probation legitimately, but then gives him the trial period and the rehabilitation illegitimately. He’s not purified of those offenses. 

“It\marginnote{35.2.1} may be that a monk has committed a number of offenses entailing suspension, both specified and unspecified, both having the same name and having different names, both of the same kind and of different kinds, both of the same sort and of different sorts. He asks the Sangha for simultaneous probation for those offenses, which he gets. While on probation, he commits a number of offenses entailing suspension, unconcealed and unspecified … concealed and unspecified … both concealed and unconcealed and unspecified … unconcealed and both specified and unspecified. He asks the Sangha to send him back to the beginning for those offenses, which it does. The legal procedure is legitimate, irreversible, and fit to stand. The Sangha gives him the simultaneous probation legitimately, but then gives him the trial period and the rehabilitation illegitimately. He’s not purified of those offenses. 

It\marginnote{35.2.12} may be that a monk has committed a number of offenses entailing suspension, both specified and unspecified, both having the same name and having different names, both of the same kind and of different kinds, both of the same sort and of different sorts. He asks the Sangha for simultaneous probation for those offenses, which he gets. While on probation, he commits a number of offenses entailing suspension, concealed and both specified and unspecified. He asks the Sangha to send him back to the beginning for those offenses, which it does. The legal procedure is legitimate, irreversible, and fit to stand. The Sangha gives him the simultaneous probation legitimately, but then gives him the trial period and the rehabilitation illegitimately. He’s not purified of those offenses. 

It\marginnote{35.2.20} may be that a monk has committed a number of offenses entailing suspension, both specified and unspecified, both having the same name and having different names, both of the same kind and of different kinds, both of the same sort and of different sorts. He asks the Sangha for simultaneous probation for those offenses, which he gets. While on probation, he commits a number of offenses entailing suspension, both concealed and unconcealed and both specified and unspecified. He asks the Sangha to send him back to the beginning for those offenses, which it does. The legal procedure is legitimate, irreversible, and fit to stand. The Sangha gives him the simultaneous probation legitimately, but then gives him the trial period and the rehabilitation illegitimately. He’s not purified of those offenses.” 

\scend{The group of nine on unpurified with sending back to the beginning is finished. }

\section*{10. The second group of nine }

“It\marginnote{36.1.1} may be that a monk has committed a number of offenses entailing suspension, both specified and unspecified, both having the same name and having different names, both of the same kind and of different kinds, both of the same sort and of different sorts. He asks the Sangha for simultaneous probation for those offenses, which he gets. While on probation, he commits a number of offenses entailing suspension, unconcealed and specified. He asks the Sangha to send him back to the beginning for those offenses, which it does. But the legal procedure is illegitimate, reversible, and unfit to stand. The Sangha gives him the simultaneous probation illegitimately, but then gives him the trial period and the rehabilitation legitimately. He’s not purified of those offenses. 

It\marginnote{36.1.9} may be that a monk has committed a number of offenses entailing suspension, both specified and unspecified, both having the same name and having different names, both of the same kind and of different kinds, both of the same sort and of different sorts. He asks the Sangha for simultaneous probation for those offenses, which he gets. While on probation, he commits a number of offenses entailing suspension, concealed and specified. He asks the Sangha to send him back to the beginning for those offenses, which it does. But the legal procedure is illegitimate, reversible, and unfit to stand. The Sangha gives him the simultaneous probation illegitimately, but then gives him the trial period and the rehabilitation legitimately. He’s not purified of those offenses. 

It\marginnote{36.1.18} may be that a monk has committed a number of offenses entailing suspension, both specified and unspecified, both having the same name and having different names, both of the same kind and of different kinds, both of the same sort and of different sorts. He asks the Sangha for simultaneous probation for those offenses, which he gets. While on probation, he commits a number of offenses entailing suspension, both concealed and unconcealed and specified. He asks the Sangha to send him back to the beginning for those offenses, which it does. But the legal procedure is illegitimate, reversible, and unfit to stand. The Sangha gives him the simultaneous probation illegitimately, but then gives him the trial period and the rehabilitation legitimately. He’s not purified of those offenses. 

“It\marginnote{36.1.27} may be that a monk has committed a number of offenses entailing suspension, both specified and unspecified, both having the same name and having different names, both of the same kind and of different kinds, both of the same sort and of different sorts. He asks the Sangha for simultaneous probation for those offenses, which he gets. While on probation, he commits a number of offenses entailing suspension, unconcealed and unspecified … concealed and unspecified … both concealed and unconcealed and unspecified … unconcealed and both specified and unspecified. He asks the Sangha to send him back to the beginning for those offenses, which it does. But the legal procedure is illegitimate, reversible, and unfit to stand. The Sangha gives him the simultaneous probation illegitimately, but then gives him the trial period and the rehabilitation legitimately. He’s not purified of those offenses. 

It\marginnote{36.1.38} may be that a monk has committed a number of offenses entailing suspension, both specified and unspecified, both having the same name and having different names, both of the same kind and of different kinds, both of the same sort and of different sorts. He asks the Sangha for simultaneous probation for those offenses, which he gets. While on probation, he commits a number of offenses entailing suspension, concealed and both specified and unspecified. He asks the Sangha to send him back to the beginning for those offenses, which it does. But the legal procedure is illegitimate, reversible, and unfit to stand. The Sangha gives him the simultaneous probation illegitimately, but then gives him the trial period and the rehabilitation legitimately. He’s not purified of those offenses. 

It\marginnote{36.1.46} may be that a monk has committed a number of offenses entailing suspension, both specified and unspecified, both having the same name and having different names, both of the same kind and of different kinds, both of the same sort and of different sorts. He asks the Sangha for simultaneous probation for those offenses, which he gets. While on probation, he commits a number of offenses entailing suspension, both concealed and unconcealed and both specified and unspecified. He asks the Sangha to send him back to the beginning for those offenses, which it does. But the legal procedure is illegitimate, reversible, and unfit to stand. The Sangha gives him the simultaneous probation illegitimately, but then gives him the trial period and the rehabilitation legitimately. He’s not purified of those offenses.” 

\scend{The second group of nine is finished. }

\section*{11. The third group of nine }

“It\marginnote{36.2.1} may be that a monk has committed a number of offenses entailing suspension, both specified and unspecified, both having the same name and having different names, both of the same kind and of different kinds, both of the same sort and of different sorts. He asks the Sangha for simultaneous probation for those offenses, which he gets. While on probation, he commits a number of offenses entailing suspension, unconcealed and specified. He asks the Sangha to send him back to the beginning for those offenses, which it does. But the legal procedure is illegitimate, reversible, and unfit to stand. The Sangha gives him the simultaneous probation illegitimately. Thinking that he’s on probation, he commits a number of offenses entailing suspension, unconcealed and specified. At this point, he remembers offenses committed while on probation for the former offenses, and he remembers offenses committed while on probation for the further offenses. He considers all this and thinks, ‘Let me ask the Sangha to send me back to the beginning for all those offenses. The procedure must be legitimate, irreversible, and fit to stand. The simultaneous probation, the trial period, and the rehabilitation must all be legitimate.’ He asks the Sangha to be sent back to the beginning in this way, which it does. The legal procedure is legitimate, irreversible, and fit to stand. The Sangha gives him the simultaneous probation, the trial period, and the rehabilitation legitimately. He’s purified of those offenses. 

It\marginnote{36.2.27} may be that a monk has committed a number of offenses entailing suspension, both specified and unspecified, both having the same name and having different names, both of the same kind and of different kinds, both of the same sort and of different sorts. He asks the Sangha for simultaneous probation for those offenses, which he gets. While on probation, he commits a number of offenses entailing suspension, concealed and specified. He asks the Sangha to send him back to the beginning for those offenses, which it does. But the legal procedure is illegitimate, reversible, and unfit to stand. The Sangha gives him the simultaneous probation illegitimately. Thinking that he’s on probation, he commits a number of offenses entailing suspension, concealed and specified. At this point he remembers offenses committed while on probation for the former offenses, and he remembers offenses committed while on probation for the further offenses. He considers all this and thinks, ‘Let me ask the Sangha to send me back to the beginning for all those offenses. The procedure must be legitimate, irreversible, and fit to stand. The simultaneous probation, the trial period, and the rehabilitation must all be legitimate.’ He asks the Sangha to be sent back to the beginning in this way, which it does. The legal procedure is legitimate, irreversible, and fit to stand. The Sangha gives him the simultaneous probation, the trial period, and the rehabilitation legitimately. He’s purified of those offenses. 

It\marginnote{36.2.50} may be that a monk has committed a number of offenses entailing suspension, both specified and unspecified, both having the same name and having different names, both of the same kind and of different kinds, both of the same sort and of different sorts. He asks the Sangha for simultaneous probation for those offenses, which he gets. While on probation, he commits a number of offenses entailing suspension, both concealed and unconcealed and specified. He asks the Sangha to send him back to the beginning for those offenses, which it does. But the legal procedure is illegitimate, reversible, and unfit to stand. The Sangha gives him the simultaneous probation illegitimately. Thinking that he’s on probation, he commits a number of offenses entailing suspension, both concealed and unconcealed and specified. At this point he remembers offenses committed while on probation for the former offenses, and he remembers offenses committed while on probation for the further offenses. He considers all this and thinks, ‘Let me ask the Sangha to send me back to the beginning for all those offenses. The procedure must be legitimate, irreversible, and fit to stand. The simultaneous probation, the trial period, and the rehabilitation must all be legitimate.’ He asks the Sangha to be sent back to the beginning in this way, which it does. The legal procedure is legitimate, irreversible, and fit to stand. The Sangha gives him the simultaneous probation legitimately, the trial period, and the rehabilitation legitimately. He’s purified of those offenses. 

“It\marginnote{36.3.1} may be that a monk has committed a number of offenses entailing suspension, both specified and unspecified, both having the same name and having different names, both of the same kind and of different kinds, both of the same sort and of different sorts. He asks the Sangha for simultaneous probation for those offenses, which he gets. While on probation, he commits a number of offenses entailing suspension, unconcealed and unspecified … concealed and unspecified … both concealed and unconcealed and unspecified … unconcealed and both specified and unspecified. He asks the Sangha to send him back to the beginning for those offenses, which it does. But the legal procedure is illegitimate, reversible, and unfit to stand. The Sangha gives him the simultaneous probation illegitimately. Thinking that he’s on probation … which it does. The legal procedure is legitimate, irreversible, and fit to stand. The Sangha gives him the simultaneous probation, the trial period, and the rehabilitation legitimately. He’s purified of those offenses. 

It\marginnote{36.4.1} may be that a monk has committed a number of offenses entailing suspension, both specified and unspecified, both having the same name and having different names, both of the same kind and of different kinds, both of the same sort and of different sorts. He asks the Sangha for simultaneous probation for those offenses, which he gets. While on probation, he commits a number of offenses entailing suspension, concealed and both specified and unspecified. He asks the Sangha to send him back to the beginning for those offenses, which it does. But the legal procedure is illegitimate, reversible, and unfit to stand. The Sangha gives him the simultaneous probation illegitimately. Thinking that he’s on probation, he commits a number of offenses entailing suspension, concealed and both specified and unspecified. At this point he remembers offenses committed while on probation for the former offenses, and he remembers offenses committed while on probation for the further offenses. He considers all this and thinks, ‘Let me ask the Sangha to send me back to the beginning for all those offenses. The procedure must be legitimate, irreversible, and fit to stand. The simultaneous probation, the trial period, and the rehabilitation must all be legitimate.’ He asks the Sangha to be sent back to the beginning in this way, which it does. The legal procedure is legitimate, irreversible, and fit to stand. The Sangha gives him the simultaneous probation, the trial period, and the rehabilitation legitimately. He’s purified of those offenses. 

It\marginnote{36.4.24} may be that a monk has committed a number of offenses entailing suspension, both specified and unspecified, both having the same name and having different names, both of the same kind and of different kinds, both of the same sort and of different sorts. He asks the Sangha for simultaneous probation for those offenses, which he gets. While on probation, he commits a number of offenses entailing suspension, both concealed and unconcealed and both specified and unspecified. He asks the Sangha to send him back to the beginning for those offenses, which it does. But the legal procedure is illegitimate, reversible, and unfit to stand. The Sangha gives him the simultaneous probation illegitimately. Thinking that he’s on probation, he commits a number of offenses entailing suspension, both concealed and unconcealed and both specified and unspecified. At this point he remembers offenses committed while on probation for the former offenses, and he remembers offenses committed while on probation for the further offenses. He considers all this and thinks, ‘Let me ask the Sangha to send me back to the beginning for all those offenses. The procedure must be legitimate, irreversible, and fit to stand. The simultaneous probation, the trial period, and the rehabilitation must all be legitimate.’ He asks the Sangha to be sent back to the beginning in this way, which it does. The legal procedure is legitimate, irreversible, and fit to stand. The Sangha gives him the simultaneous probation, the trial period, and the rehabilitation legitimately. He’s purified of those offenses.” 

\scend{The third group of nine is finished. }

\scendsutta{The third chapter on gathering is finished. }

\scuddanaintro{This is the summary: }

\begin{scuddana}%
“Unconcealed,\marginnote{36.4.50} one day, \\
Two days, three days, and four days; \\
Five days, a half-month, for ten, \\
Offense, said the Great Sage. 

And\marginnote{36.4.54} purifying, disrobing, \\
Specified, two monks;\footnote{It is not clear what \textit{\textsanskrit{mukhaṁ}} might refer to. I therefore read \textit{\textsanskrit{parimāṇā}} with SRT over \textit{\textsanskrit{parimāṇamukhaṁ}} here. } \\
There both perceive accordingly, \\
And just the same for unsure. 

And\marginnote{36.4.58} both see it as mixed, \\
They see it as not light;\footnote{Reading \textit{asuddhake va \textsanskrit{diṭṭhino}} with the SRT. } \\
And both see it as light. 

And\marginnote{36.4.61} just so one conceals, \\
And then with the thought of concealing; \\
And one who is insane, confession, \\
To the beginning, eighteen as to purity. 

The\marginnote{36.4.65} teachers of analytical statements, \\
Who are the inspiration of Sri Lanka, \\
The residents of the \textsanskrit{Mahāvihāra} monastery—\\
These were their words for the longevity of the true Teachings.” 

%
\end{scuddana}

\scendsutta{The gathering up chapter is finished. }

%
\chapter*{{\suttatitleacronym Kd 14}{\suttatitletranslation The chapter on the settling of legal issues }{\suttatitleroot Samathakkhandhaka}}
\addcontentsline{toc}{chapter}{\tocacronym{Kd 14} \toctranslation{The chapter on the settling of legal issues } \tocroot{Samathakkhandhaka}}
\markboth{The chapter on the settling of legal issues }{Samathakkhandhaka}
\extramarks{Kd 14}{Kd 14}

\section*{1. Resolution face-to-face }

At\marginnote{1.1.1} one time the Buddha was staying at \textsanskrit{Sāvatthī} in the Jeta Grove, \textsanskrit{Anāthapiṇḍika}’s Monastery. At that time the monks from the group of six did legal procedures—condemnation, demotion, banishment, reconciliation, and ejection—against monks who were absent.\footnote{For an explanation of the rendering “demotion” for \textit{niyassa}, see Appendix of Technical Terms. } The monks of few desires complained and criticized them, “How can the monks from the group of six do this?” They told the Buddha. Soon afterwards he had the Sangha of monks gathered and questioned them: 

“Is\marginnote{1.1.8} it true, monks, that the monks from the group of six are doing this?” 

“It’s\marginnote{1.1.10} true, sir.” 

The\marginnote{1.1.11} Buddha rebuked them, “It’s not suitable for those foolish men, it’s not proper, it’s not worthy of a monastic, it’s not allowable, it’s not to be done. How can they do this? This will affect people’s confidence …” After rebuking them … the Buddha gave a teaching and addressed the monks: 

\scrule{“You shouldn’t do legal procedures—condemnation, demotion, banishment, reconciliation, or ejection—against monks who are absent. If you do, you commit an offense of wrong conduct. }

An\marginnote{1.1.21} individual who speaks contrary to the Teaching; several people who speak contrary to the Teaching; a sangha that speaks contrary to the Teaching. An individual who speaks in accordance with the Teaching; several people who speak in accordance with the Teaching; a sangha that speaks in accordance with the Teaching.” 

\subsection*{The group of nine on the dark side }

“An\marginnote{2.1.1} individual who speaks contrary to the Teaching persuades an individual who speaks in accordance with the Teaching—convinces him, makes him see, makes him consider, shows him, teaches him: ‘This is the Teaching, this is the Monastic Law, this is the Teacher’s instruction; learn this, accept this.’ If a legal issue is resolved like this, it’s resolved illegitimately by a face-to-face-like resolution. 

An\marginnote{2.1.4} individual who speaks contrary to the Teaching persuades several people who speak in accordance with the Teaching—convinces them, makes them see, makes them consider, shows them, teaches them: ‘This is the Teaching, this is the Monastic Law, this is the Teacher’s instruction; learn this, accept this.’ If a legal issue is resolved like this, it’s resolved illegitimately by a face-to-face-like resolution. 

An\marginnote{2.1.7} individual who speaks contrary to the Teaching persuades a sangha that speaks in accordance with the Teaching—convinces it, makes it see, makes it consider, shows it, teaches it: ‘This is the Teaching, this is the Monastic Law, this is the Teacher’s instruction; learn this, accept this.’ If a legal issue is resolved like this, it’s resolved illegitimately by a face-to-face-like resolution. 

Several\marginnote{2.1.10} people who speak contrary to the Teaching persuade an individual who speaks in accordance with the Teaching—convince him, make him see, make him consider, show him, teach him: ‘This is the Teaching, this is the Monastic Law, this is the Teacher’s instruction; learn this, accept this.’ If a legal issue is resolved like this, it’s resolved illegitimately by a face-to-face-like resolution. 

Several\marginnote{2.1.13} people who speak contrary to the Teaching persuade several people who speak in accordance with the Teaching—convince them, make them see, make them consider, show them, teach them: ‘This is the Teaching, this is the Monastic Law, this is the Teacher’s instruction; learn this, accept this.’ If a legal issue is resolved like this, it’s resolved illegitimately by a face-to-face-like resolution. 

Several\marginnote{2.1.16} people who speak contrary to the Teaching persuade a sangha that speaks in accordance with the Teaching—convince it, make it see, make it consider, show it, teach it: ‘This is the Teaching, this is the Monastic Law, this is the Teacher’s instruction; learn this, accept this.’ If a legal issue is resolved like this, it’s resolved illegitimately by a face-to-face-like resolution. 

A\marginnote{2.1.19} sangha that speaks contrary to the Teaching persuades an individual who speaks in accordance with the Teaching—convinces him, makes him see, makes him consider, shows him, teaches him: ‘This is the Teaching, this is the Monastic Law, this is the Teacher’s instruction; learn this, accept this.’ If a legal issue is resolved like this, it’s resolved illegitimately by a face-to-face-like resolution. 

A\marginnote{2.1.22} sangha that speaks contrary to the Teaching persuades several people who speak in accordance with the Teaching—convinces them, makes them see, makes them consider, shows them, teaches them: ‘This is the Teaching, this is the Monastic Law, this is the Teacher’s instruction; learn this, accept this.’ If a legal issue is resolved like this, it’s resolved illegitimately by a face-to-face-like resolution. 

A\marginnote{2.1.25} sangha that speaks contrary to the Teaching persuades a sangha that speaks in accordance with the Teaching—convinces it, makes it see, makes it consider, shows it, teaches it: ‘This is the Teaching, this is the Monastic Law, this is the Teacher’s instruction; learn this, accept this.’ If a legal issue is resolved like this, it’s resolved illegitimately by a face-to-face-like resolution.” 

\scend{The group of nine on the dark side is finished. }

\subsection*{The group of nine on the bright side }

“An\marginnote{3.1.1} individual who speaks in accordance with the Teaching persuades an individual who speaks contrary to the Teaching—convinces him, makes him see, makes him consider, shows him, teaches him: ‘This is the Teaching, this is the Monastic Law, this is the Teacher’s instruction; learn this, accept this.’ If a legal issue is resolved like this, it’s resolved legitimately by face-to-face resolution. 

An\marginnote{3.1.4} individual who speaks in accordance with the Teaching persuades several people who speak contrary to the Teaching—convinces them, makes them see, makes them consider, shows them, teaches them: ‘This is the Teaching, this is the Monastic Law, this is the Teacher’s instruction; learn this, accept this.’ If a legal issue is resolved like this, it’s resolved legitimately by face-to-face resolution. 

An\marginnote{3.1.7} individual who speaks in accordance with the Teaching persuades a sangha that speaks contrary to the Teaching—convinces it, makes it see, makes it consider, shows it, teaches it: ‘This is the Teaching, this is the Monastic Law, this is the Teacher’s instruction; learn this, accept this.’ If a legal issue is resolved like this, it’s resolved legitimately by face-to-face resolution. 

Several\marginnote{3.1.10} people who speak in accordance with the Teaching persuade an individual who speaks contrary to the Teaching—convince him, make him see, make him consider, show him, teach him: ‘This is the Teaching, this is the Monastic Law, this is the Teacher’s instruction; learn this, accept this.’ If a legal issue is resolved like this, it’s resolved legitimately by face-to-face resolution. 

Several\marginnote{3.1.13} people who speak in accordance with the Teaching persuade several people who speak contrary to the Teaching—convince them, make them see, make them consider, show them, teach them: ‘This is the Teaching, this is the Monastic Law, this is the Teacher’s instruction; learn this, accept this.’ If a legal issue is resolved like this, it’s resolved legitimately by face-to-face resolution. 

Several\marginnote{3.1.16} people who speak in accordance with the Teaching persuade a sangha that speaks contrary to the Teaching—convince it, make it see, make it consider, show it, teach it: ‘This is the Teaching, this is the Monastic Law, this is the Teacher’s instruction; learn this, accept this.’ If a legal issue is resolved like this, it’s resolved legitimately by face-to-face resolution. 

A\marginnote{3.1.19} sangha that speaks in accordance with the Teaching persuades an individual who speaks contrary to the Teaching—convinces him, makes him see, makes him consider, shows him, teaches him: ‘This is the Teaching, this is the Monastic Law, this is the Teacher’s instruction; learn this, accept this.’ If a legal issue is resolved like this, it’s resolved legitimately by face-to-face resolution. 

A\marginnote{3.1.22} sangha that speaks in accordance with the Teaching persuades several people who speak contrary to the Teaching—convinces them, makes them see, makes them consider, shows them, teaches them: ‘This is the Teaching, this is the Monastic Law, this is the Teacher’s instruction; learn this, accept this.’ If a legal issue is resolved like this, it’s resolved legitimately by face-to-face resolution. 

A\marginnote{3.1.25} sangha that speaks in accordance with the Teaching persuades a sangha that speaks contrary to the Teaching—convinces it, makes it see, makes it consider, shows it, teaches it: ‘This is the Teaching, this is the Monastic Law, this is the Teacher’s instruction; learn this, accept this.’ If a legal issue is resolved like this, it’s resolved legitimately by face-to-face resolution.” 

\scend{The group of nine on the bright side is finished. }

\section*{2. Resolution through recollection }

At\marginnote{4.1.1} one time when the Buddha was staying at \textsanskrit{Rājagaha} in the Bamboo Grove, Venerable Dabba the Mallian realized perfection at the age of seven. He had achieved all there is to achieve by a disciple and had nothing further to do. Then, while reflecting in private, he thought, “How can I be of service to the Sangha? Why don’t I assign the dwellings and designate the meals?” In the evening Dabba came out of seclusion and went to the Buddha. He bowed, sat down, and said, “Sir, while I was reflecting in private, it occurred to me that I’ve achieved all there is to achieve by a disciple, and I was wondering how I could be of service to the Sangha. I thought, ‘Why don’t I assign the dwellings and designate the meals?’” 

“Good,\marginnote{4.2.11} good, Dabba, please do so.” 

“Yes,\marginnote{4.2.13} sir.” 

Soon\marginnote{4.3.1} afterwards the Buddha gave a teaching and addressed the monks: “Monks, the Sangha should appoint Dabba the Mallian as the assigner of dwellings and the designator of meals. And he should be appointed like this. First Dabba should be asked. A competent and capable monk should then inform the Sangha: 

‘Please,\marginnote{4.3.6} venerables, I ask the Sangha to listen. If the Sangha is ready, it should appoint Venerable Dabba the Mallian as the assigner of dwellings and the designator of meals. This is the motion. 

Please,\marginnote{4.3.9} venerables, I ask the Sangha to listen. The Sangha appoints Venerable Dabba the Mallian as the assigner of dwellings and the designator of meals. Any monk who approves of appointing Venerable Dabba as the assigner of dwellings and the designator of meals should remain silent. Any monk who doesn’t approve should speak up. 

The\marginnote{4.3.13} Sangha has appointed Venerable Dabba the Mallian as the assigner of dwellings and the designator of meals. The Sangha approves and is therefore silent. I’ll remember it thus.’” 

Dabba\marginnote{4.4.1} assigned dwellings to the monks according to their character. He assigned dwellings in the same place to those monks who were experts on the discourse, thinking, “They’ll recite the discourses to one another.” And he did likewise for the experts on Monastic Law, thinking, “They’ll discuss the Monastic Law;” for the expounders of the Teaching, thinking, “They’ll discuss the Teaching;” for the meditators, thinking, “They won’t disturb one another;” and for the gossips and the bodybuilders, thinking, “In this way even these venerables will be happy.” 

When\marginnote{4.4.12} monks arrived at night, he entered the fire element and assigned dwellings with the help of that light. Monks even arrived late on purpose, hoping to see the marvel of Dabba’s supernormal powers. They would approach Dabba and say, “Venerable Dabba, please assign us a dwelling.” 

“Where\marginnote{4.4.18} would you like to stay?” 

They\marginnote{4.4.19} would intentionally suggest somewhere far away: “On the Vulture Peak,” “At Robbers’ Cliff,” “On Black Rock on the slope of Mount Isigili,” “In the \textsanskrit{Sattapaṇṇi} Cave on the slope of Mount \textsanskrit{Vebhāra},” “In Cool Grove on the hill at the Snake’s Pool,” “At Gotamaka Gorge,” “At Tinduka Gorge,” “At Tapoda Gorge,” “In Tapoda Park,” “In \textsanskrit{Jīvaka}’s Mango Grove,” “In the deer park at Maddakucchi.” 

Dabba\marginnote{4.4.31} then entered the fire element, and with his finger glowing, he walked in front of those monks. They followed behind him with the help of that light. And he would assign them dwellings: “This is the bed, this the bench, this the mattress, this the pillow, this the place for defecating, this the place for urinating, this the water for drinking, this the water for washing, this the walking stick; these are the Sangha’s agreements concerning the right time to enter and the right time to leave.”\footnote{See \href{https://suttacentral.net/pli-tv-kd18/en/brahmali\#1.2.22}{Kd 18:1.2.22} for the correct interpretation of \textit{\textsanskrit{idaṁ} \textsanskrit{saṅghassa} \textsanskrit{katikasaṇṭhānaṁ}, \textsanskrit{imaṁ} \textsanskrit{kālaṁ} \textsanskrit{pavisitabbaṁ}, \textsanskrit{imaṁ} \textsanskrit{kālaṁ} nikkhamitabbanti}. } Dabba then returned to the Bamboo Grove. At that time the monks Mettiya and \textsanskrit{Bhūmajaka} were only newly ordained. They had little merit, getting inferior dwellings and meals. The people of \textsanskrit{Rājagaha} were keen on giving specially prepared almsfood to the senior monks—ghee, oil, and special curries—but to the monks Mettiya and \textsanskrit{Bhūmajaka} they gave ordinary food of porridge and broken rice. 

When\marginnote{4.5.7} they had eaten their meal and returned from almsround, they asked the senior monks, “What did you get at the dining hall?” 

Some\marginnote{4.5.9} said, “We got ghee, oil, and special curries.” 

But\marginnote{4.5.11} the monks Mettiya and \textsanskrit{Bhūmajaka} said, “We didn’t get anything except ordinary food of porridge and broken rice.” 

At\marginnote{4.6.1} that time there was a householder who gave a regular meal of fine food to four monks. He made his offering in the dining hall together with his wives and children. Some of them offered rice, some bean curry, some oil, and some special curries. 

On\marginnote{4.6.4} one occasion the meal to be given by this householder on the following day had been designated to the monks Mettiya and \textsanskrit{Bhūmajaka}. Just then that householder went to the monastery on some business. He approached Dabba, bowed, and sat down. Dabba instructed, inspired, and gladdened him with a teaching. After the talk, he asked Dabba, “Sir, who’s been designated to receive tomorrow’s meal in our house?” 

“Mettiya\marginnote{4.6.10} and \textsanskrit{Bhūmajaka}.” 

He\marginnote{4.6.11} was disappointed, and thought, “Why should bad monks eat in our house?” After returning to his house, he told a female slave, “For those who are coming for tomorrow’s meal, prepare seats in the gatehouse and serve them broken rice and porridge.” 

“Yes,\marginnote{4.6.14} sir.” 

The\marginnote{4.7.1} monks Mettiya and \textsanskrit{Bhūmajaka} said to each other, “Yesterday we were designated a meal from that householder who offers fine food. Tomorrow he’ll serve us together with his wives and children. Some of them will offer us rice, some bean curry, some oil, and some special curries.” And because they were excited, they did not sleep properly that night. 

The\marginnote{4.7.6} following morning they robed up, took their bowls and robes, and went to the house of that householder. When the female slave saw them coming, she prepared seats in the gatehouse and said to them, “Please sit, venerables.” 

They\marginnote{4.7.10} thought, “The meal can’t be ready, since we’re given seats in the gatehouse.” She then brought them broken rice and porridge and said, “Eat, sirs.” 

“But,\marginnote{4.7.14} Sister, we’ve come for the regular meal.” 

“I\marginnote{4.7.15} know. But yesterday I was told by the head of the household to serve you like this. Please eat.” 

They\marginnote{4.7.19} said to each other, “Yesterday this householder came to the monastery and spoke with Dabba. Dabba must be responsible for this split between the householder and us.” And because they were dejected, they did not eat as much as they had intended. When they had eaten their meal and returned from almsround, they put their bowls and robes away, and squatted on their heels outside the monastery gatehouse, using their upper robes as a back-and-knee strap. They were silent and humiliated, with shoulders drooping and heads down, glum and speechless. 

Just\marginnote{4.8.1} then the nun \textsanskrit{Mettiyā} came to them and said, “My respectful greetings to you, venerables.” But they did not respond. A second time and a third time she said the same thing, but they still did not respond. 

“Have\marginnote{4.8.8} I done something wrong? Why don’t you respond?” 

“It’s\marginnote{4.8.10} because we’ve been badly treated by Dabba the Mallian, and you’re not taking an interest.” 

“But\marginnote{4.8.11} what can I do?” 

“If\marginnote{4.8.12} you like, you could make the Buddha expel Dabba.” 

“And\marginnote{4.8.13} how can I do that?” 

“Go\marginnote{4.8.15} to the Buddha and say, ‘Sir, this isn’t proper or appropriate. There’s fear, distress, and oppression in this district, where none of these should exist. It’s windy where it should be calm. It’s as if water is burning. Venerable Dabba the Mallian has raped me.’” 

Saying,\marginnote{4.8.20} “Alright, venerables,” she went to the Buddha, bowed, and repeated what she had been told to say. 

Soon\marginnote{4.9.1} afterwards the Buddha had the Sangha gathered and questioned Dabba: “Dabba, do you remember doing as the nun \textsanskrit{Mettiyā} says?” 

“Sir,\marginnote{4.9.3} you know what I’m like.” 

A\marginnote{4.9.4} second and a third time the Buddha asked the same question and got the same response. He then said, “Dabba, the Dabbas don’t give such evasive answers. If it was done by you, say so; if it wasn’t, then say that.” 

“Since\marginnote{4.9.13} I was born, sir, I don’t recall having sexual intercourse even in a dream, let alone when awake.” 

The\marginnote{4.9.14} Buddha addressed the monks: “Well then, monks, expel the nun \textsanskrit{Mettiyā},\footnote{For an explanation of the rendering “expel” for \textit{\textsanskrit{nāsetha}}, see Appendix of Technical Terms. } and call these monks to account.” The Buddha then got up from his seat and entered his dwelling. 

When\marginnote{4.9.18} the monks had expelled the nun \textsanskrit{Mettiyā}, the monks Mettiya and \textsanskrit{Bhūmajaka} said to them, “Don’t expel the nun \textsanskrit{Mettiyā}; she hasn’t done anything wrong. She was urged on by us. We were angry and displeased, and trying to make Dabba give up the monastic life.” 

“But\marginnote{4.9.22} did you groundlessly charge Venerable Dabba with failure in morality?” 

“Yes.”\marginnote{4.9.23} 

The\marginnote{4.9.24} monks of few desires … complained and criticized them, “How could the monks Mettiya and \textsanskrit{Bhūmajaka} groundlessly charge Venerable Dabba with failure in morality?” 

They\marginnote{4.9.26} then told the Buddha. … “Is it true, monks, that you did this?” 

“It’s\marginnote{4.9.28} true, sir.” … 

After\marginnote{4.9.29} rebuking them … he gave a teaching and addressed the monks: 

“Well\marginnote{4.10.1} then, because of his great clarity of memory, grant resolution through recollection to Dabba the Mallian. And it should be granted like this. Dabba should approach the Sangha, arrange his upper robe over one shoulder, pay respect at the feet of the senior monks, squat on his heels, raise his joined palms, and say: 

‘Venerables,\marginnote{4.10.4} these monks Mettiya and \textsanskrit{Bhūmajaka} are groundlessly charging me with failure in morality. Because of my great clarity of memory, I ask the Sangha for resolution through recollection. 

Venerables,\marginnote{4.10.6} these monks Mettiya and \textsanskrit{Bhūmajaka} are groundlessly charging me with failure in morality. Because of my great clarity of memory, for the second time, I ask the Sangha for resolution through recollection. 

Venerables,\marginnote{4.10.8} these monks Mettiya and \textsanskrit{Bhūmajaka} are groundlessly charging me with failure in morality. Because of my great clarity of memory, for the third time, I ask the Sangha for resolution through recollection.’ 

A\marginnote{4.10.10} competent and capable monk should then inform the Sangha: 

‘Please,\marginnote{4.10.11} venerables, I ask the Sangha to listen. These monks Mettiya and \textsanskrit{Bhūmajaka} are groundlessly charging Venerable Dabba the Mallian with failure in morality. Because of his great clarity of memory, Dabba is asking the Sangha for resolution through recollection. If the Sangha is ready, it should grant Dabba resolution through recollection. This is the motion. 

Please,\marginnote{4.10.16} venerables, I ask the Sangha to listen. These monks Mettiya and \textsanskrit{Bhūmajaka} are groundlessly charging Venerable Dabba the Mallian with failure in morality. Because of his great clarity of memory, Dabba is asking the Sangha for resolution through recollection. The Sangha grants Dabba resolution through recollection. Any monk who approves of granting Dabba resolution through recollection should remain silent. Any monk who doesn’t approve should speak up. 

For\marginnote{4.10.22} the second time, I speak on this matter. … For the third time, I speak on this matter. Please, venerables, I ask the Sangha to listen. These monks Mettiya and \textsanskrit{Bhūmajaka} are groundlessly charging Venerable Dabba the Mallian with failure in morality. Because of his great clarity of memory, Dabba is asking the Sangha for resolution through recollection. The Sangha grants Dabba resolution through recollection. Any monk who approves of granting Dabba resolution through recollection should remain silent. Any monk who doesn’t approve should speak up. 

Because\marginnote{4.10.30} of his great clarity of memory, the Sangha has granted Dabba the Mallian resolution through recollection. The Sangha approves and is therefore silent. I’ll remember it thus.’ 

There\marginnote{4.11.1} are five factors for the legitimate granting of resolution through recollection: the monk is pure and free of offenses; he has been accused; he asks for resolution through recollection; the Sangha grants him resolution through recollection; the legal procedure is legitimate and done by a unanimous assembly.”\footnote{Sp 4.195: \textit{Tattha ca \textsanskrit{anuvadantīti} codenti}, “And there \textit{anuvadanti} means they accuse.” } 

\section*{3. Resolution because of past insanity }

At\marginnote{5.1.1} one time the monk Gagga was insane and suffering from psychosis.\footnote{\textit{\textsanskrit{Cittavipariyāsakata}}, literally, “(his) mind was made distorted”. } Because of this, he did and said many things unworthy of a monastic. The monks accused him of an offense, saying, “Venerable, do you remember committing such-and-such an offense?” 

“I\marginnote{5.1.6} was insane and suffering from psychosis. Because of that, I did and said many things unworthy of a monastic. I don’t remember it. I did it because I was insane.” 

But\marginnote{5.1.10} they kept on accusing him in the same way. The monks of few desires … complained and criticized them, “How can these monks keep on accusing Gagga when he says he was insane?” 

They\marginnote{5.1.22} told the Buddha. … He said, “Is it true, monks, that these monks are doing this?” 

“It’s\marginnote{5.1.24} true, sir.” … 

After\marginnote{5.1.25} rebuking them … he gave a teaching and addressed the monks: 

“Well\marginnote{5.1.27} then, since he’s no longer insane, grant the monk Gagga resolution because of past insanity. And it should be granted like this. The monk Gagga should approach the Sangha, arrange his upper robe over one shoulder, pay respect at the feet of the senior monks, squat on his heels, raise his joined palms, and say: 

‘Venerables,\marginnote{5.2.3} I’ve been insane and suffering from psychosis. Because of that, I did and said many things unworthy of a monastic. The monks accused me of an offense, saying, “Venerable, do you remember committing such-and-such an offense?” I replied, “I was insane and suffering from psychosis. Because of that, I did and said many things unworthy of a monastic. I don’t remember it. I did it because I was insane.” But they kept on accusing me in the same way. Because I’m no longer insane, I ask the Sangha for resolution because of past insanity.’ 

And\marginnote{5.2.15} he should ask a second time, and a third time: 

‘Venerables,\marginnote{5.2.17} I’ve been insane and suffering from psychosis. Because of that, I did and said many things unworthy of a monastic. The monks accused me of an offense, saying, “Venerable, do you remember committing such-and-such an offense?” I replied, “I was insane and suffering from psychosis. Because of that, I did and said many things unworthy of a monastic. I don’t remember it. I did it because I was insane.” But they kept on accusing me in the same way. Because I’m no longer insane, for the third time, I ask the Sangha for resolution because of past insanity.’ 

A\marginnote{5.2.28} competent and capable monk should then inform the Sangha: 

‘Please,\marginnote{5.2.29} venerables, I ask the Sangha to listen. The monk Gagga has been insane and suffering from psychosis. Because of that, he did and said many things unworthy of a monastic. The monks accused him of an offense, saying, “Venerable, do you remember committing such-and-such an offense?” He replied, “I was insane and suffering from psychosis. Because of that, I did and said many things unworthy of a monastic. I don’t remember it. I did it because I was insane.” But they kept on accusing him in the same way. Because he’s no longer insane, he’s asking the Sangha for resolution because of past insanity. If the Sangha is ready, it should grant the monk Gagga resolution because of past insanity. This is the motion. 

Please,\marginnote{5.2.43} venerables, I ask the Sangha to listen. The monk Gagga has been insane and suffering from psychosis. Because of that, he did and said many things unworthy of a monastic. The monks accused him of an offense, saying, “Venerable, do you remember committing such-and-such an offense?” He replied, “I was insane and suffering from psychosis. Because of that, I did and said many things unworthy of a monastic. I don’t remember it. I did it because I was insane.” But they kept on accusing him in the same way. Because he’s no longer insane, he’s asking the Sangha for resolution because of past insanity. The Sangha grants the monk Gagga resolution because of past insanity. Any monk who approves of granting the monk Gagga resolution because of past insanity should remain silent. Any monk who doesn’t approve should speak up. 

For\marginnote{5.2.58} the second time, I speak on this matter. … For the third time, I speak on this matter. … 

Since\marginnote{5.2.60} he’s no longer insane, the Sangha has granted the monk Gagga resolution because of past insanity. The Sangha approves and is therefore silent. I’ll remember it thus.’ 

There\marginnote{6.1.1} are three illegitimate and three legitimate grantings of resolution because of past insanity. What are the three illegitimate grantings of resolution because of past insanity? 

It\marginnote{6.1.3} may be that a monk has committed an offense. The Sangha, several monks, or a single monk accuses him, saying,\footnote{For an explanation of the rendering “several” for \textit{sambahula}, see Appendix of Technical Terms. } ‘Venerable, do you remember committing such-and-such an offense?’ Although he remembers, he says he doesn’t. If the Sangha grants him resolution because of past insanity, then that granting is illegitimate. 

It\marginnote{6.1.10} may be that a monk has committed an offense. The Sangha, several monks, or a single monk accuses him, saying, ‘Venerable, do you remember committing such-and-such an offense?’ Although he remembers, he says, ‘I remember as if in a dream.’ If the Sangha grants him resolution because of past insanity, then that granting is illegitimate. 

It\marginnote{6.1.17} may be that a monk has committed an offense. The Sangha, several monks, or a single monk accuses him, saying, ‘Venerable, do you remember committing such-and-such an offense?’ Although he’s sane, he acts insane, saying, ‘I do this, and so do you. This is allowable for me, and also for you.’ If the Sangha grants him resolution because of past insanity, then that granting is illegitimate. 

And\marginnote{6.2.2} what are the three legitimate grantings of resolution because of past insanity? 

It\marginnote{6.2.3} may be that a monk has been insane and suffering from psychosis. Because of that, he did and said many things unworthy of a monastic. The Sangha, several monks, or a single monk accuses him, saying, ‘Venerable, do you remember committing such-and-such an offense?’ Not remembering, he says he doesn’t. If the Sangha grants him resolution because of past insanity, then that granting is legitimate. 

It\marginnote{6.2.11} may be that a monk has been insane and suffering from psychosis. Because of that, he did and said many things unworthy of a monastic. The Sangha, several monks, or a single monk accuses him, saying, ‘Venerable, do you remember committing such-and-such an offense?’ Not remembering, he says, ‘I remember as if in a dream.’ If the Sangha grants him resolution because of past insanity, then that granting is legitimate. 

It\marginnote{6.2.19} may be that a monk has been insane and suffering from psychosis. Because of that, he did and said many things unworthy of a monastic. The Sangha, several monks, or a single monk accuses him, saying, ‘Venerable, do you remember committing such-and-such an offense?’ Being insane, he acts insane, saying, ‘I do this, and so do you. This is allowable for me, and also for you.’ If the Sangha grants him resolution because of past insanity, then that granting is legitimate.” 

\section*{4. Acting according to what has been admitted }

At\marginnote{7.1.1} one time the monks from the group of six did legal procedures—condemnation, demotion, banishment, reconciliation, and ejection—against other monks without their admission. The monks of few desires complained and criticized them, “How can the monks from the group of six do this?” They told the Buddha. … 

“Is\marginnote{7.1.7} it true, monks, that the monks from the group of six are doing this?” “It’s true, sir.” … After rebuking them … the Buddha gave a teaching and addressed the monks: 

\scrule{“You shouldn’t do legal procedures—condemnation, demotion, banishment, reconciliation, or ejection—against monks without their admission. If you do, you commit an offense of wrong conduct. }

And\marginnote{8.1.1} how’s acting according to what’s been admitted illegitimate? 

It\marginnote{8.1.3} may be that a monk has committed an offense entailing expulsion. The Sangha, several monks, or a single monk accuses him, saying, ‘Venerable, you’ve committed an offense entailing expulsion.’ He says, ‘I haven’t committed an offense entailing expulsion, but one entailing suspension.’ If the Sangha makes him act according to an offense entailing suspension, then that acting according to what’s been admitted is illegitimate. 

It\marginnote{8.1.10} may be that a monk has committed an offense entailing expulsion. The Sangha, several monks, or a single monk accuses him, saying, ‘Venerable, you’ve committed an offense entailing expulsion.’ He says, ‘I haven’t committed an offense entailing expulsion, but a serious offense … but an offense entailing confession … but an offense entailing acknowledgment … but an offense of wrong conduct … but an offense of wrong speech.’ If the Sangha makes him act according to an offense of wrong speech, then that acting according to what’s been admitted is illegitimate. 

It\marginnote{8.1.21} may be that a monk has committed an offense entailing suspension … a serious offense … an offense entailing confession … an offense entailing acknowledgment … an offense of wrong conduct … an offense of wrong speech. The Sangha, several monks, or a single monk accuses him, saying, ‘Venerable, you’ve committed an offense of wrong speech.’ He says, ‘I haven’t committed an offense of wrong speech, but an offense entailing expulsion.’ If the Sangha makes him act according to an offense entailing expulsion, then that acting according to what’s been admitted is illegitimate. 

It\marginnote{8.1.33} may be that a monk has committed an offense of wrong speech. The Sangha, several monks, or a single monk accuses him, saying, ‘Venerable, you’ve committed an offense of wrong speech.’ He says, ‘I haven’t committed an offense of wrong speech, but an offense entailing suspension … but a serious offense … but an offense entailing confession … but an offense entailing acknowledgment … but an offense of wrong conduct.’ If the Sangha makes him act according to an offense of wrong conduct, then that acting according to what’s been admitted is illegitimate. 

And\marginnote{8.2.2} how’s acting according to what’s been admitted legitimate? 

It\marginnote{8.2.3} may be that a monk has committed an offense entailing expulsion. The Sangha, several monks, or a single monk accuses him, saying ‘Venerable, you’ve committed an offense entailing expulsion.’ He says, ‘Yes, I’ve committed an offense entailing expulsion.’ If the Sangha makes him act according to an offense entailing expulsion, then that acting according to what’s been admitted is legitimate. 

It\marginnote{8.2.10} may be that a monk has committed an offense entailing suspension … a serious offense … an offense entailing confession … an offense entailing acknowledgment … an offense of wrong conduct … an offense of wrong speech. The Sangha, several monks, or a single monk accuses him, saying, ‘Venerable, you’ve committed an offense of wrong speech.’ He says, ‘Yes, I’ve committed an offense of wrong speech.’ If the Sangha makes him act according to an offense wrong speech, then that acting according to what’s been admitted is legitimate.” 

\section*{5. Majority decision }

At\marginnote{9.1.1} one time the monks were arguing and disputing in the midst of the Sangha, attacking one another verbally, and they were unable to resolve that legal issue.\footnote{“That legal issue” refers back to the arguing and disputing just mentioned. } They told the Buddha. 

\scrule{“I allow you to resolve such legal issues by majority decision. }

You\marginnote{9.1.4} should appoint a monk who has five qualities as the manager of the vote: one who isn’t biased by favoritism, ill will, confusion, or fear, and who knows who has and who hasn’t voted. 

And\marginnote{9.1.6} he should be appointed like this. First a monk should be asked, and then a competent and capable monk should inform the Sangha: 

‘Please,\marginnote{9.1.8} venerables, I ask the Sangha to listen. If the Sangha is ready, it should appoint monk so-and-so as the manager of the vote. This is the motion. 

Please,\marginnote{9.1.11} venerables, I ask the Sangha to listen. The Sangha appoints monk so-and-so as the manager of the vote. Any monk who approves of appointing monk so-and-so as the manager of the vote should remain silent. Any monk who doesn’t approve should speak up. 

The\marginnote{9.1.15} Sangha has appointed monk so-and-so as the manager of the vote. The Sangha approves and is therefore silent. I’ll remember it thus.’ 

There\marginnote{10.1.1} are ten reasons why a vote is illegitimate: it’s only a minor legal issue; the full process for settling it hasn’t run its course; they haven’t tried to remember offenses and remind about offenses; the manager knows that those who speak contrary to the Teaching are in the majority; the manager expects that those who speak contrary to the Teaching will be in the majority; the manager knows that the Sangha will split; the manager expects that the Sangha will split; they vote illegitimately; they vote with an incomplete assembly; they don’t vote according to their own views.\footnote{Sp 4.204: \textit{Na ca gatigatanti dve tayo \textsanskrit{āvāse} na \textsanskrit{gataṁ}, tattha tattheva \textsanskrit{vā} \textsanskrit{dvattikkhattuṁ} \textsanskrit{avinicchitaṁ}}, “\textit{Na ca gatigatan}: they have not gone to two or three monasteries, or they have not investigated two or three times in this or that place.” Sp 4.204: \textit{Na ca \textsanskrit{saritasāritanti} \textsanskrit{dvattikkhattuṁ} tehi \textsanskrit{bhikkhūhi} \textsanskrit{sayaṁ} \textsanskrit{saritaṁ} \textsanskrit{vā} \textsanskrit{aññehi} \textsanskrit{sāritaṁ} \textsanskrit{vā} na hoti}, “\textit{Na ca \textsanskrit{saritasāritan}}: Those monks have not themselves remembered or had others remember two or three times.” In regard to the verb \textit{\textsanskrit{jānāti}}, “knows”, it is not immediately clear who the agent is. Sp 4.204: \textit{\textsanskrit{Jānātīti} \textsanskrit{salākaṁ} \textsanskrit{gāhento} \textsanskrit{jānāti} “\textsanskrit{adhammavādī} \textsanskrit{bahutarā}”ti}, “\textit{\textsanskrit{Jānāti}}: the one distributing the ballots (i.e. the manager) knows that those speaking contrary to the Teaching are in the majority.” } 

And\marginnote{10.2.1} there are ten reasons why a vote is legitimate: it’s not a minor legal issue; the full process for settling it has run its course; they’ve tried to remember offenses and remind about offenses; the manager knows that those who speak in accordance with the Teaching are in the majority; the manager expects that those who speak in accordance with the Teaching will be in the majority; the manager knows that the Sangha won’t split; the manager expects that the Sangha won’t split; they vote legitimately; they vote with a complete assembly; they vote according to their own views.” 

\section*{6. Further penalty }

On\marginnote{11.1.1} one occasion, when the monk \textsanskrit{Upavāḷa} was being examined in the midst of the Sangha about an offense, he asserted things after denying them, denied things after asserting them, evaded the issue, and lied. The monks of few desires complained and criticized him, “How can the monk \textsanskrit{Upavāḷa} act like this?” 

They\marginnote{11.1.4} told the Buddha. … “Is it true, monks, that the monk \textsanskrit{Upavāḷa} is acting like this?” 

“It’s\marginnote{11.1.6} true, sir.” … 

After\marginnote{11.1.7} rebuking him … the Buddha gave a teaching and addressed the monks: 

“Well\marginnote{11.1.9} then, the Sangha should do a legal procedure of further penalty against the monk \textsanskrit{Upavāḷa}. And it should be done like this. First you should accuse the monk \textsanskrit{Upavāḷa}. He should then be reminded of what he has done, before he’s charged with an offense. A competent and capable monk should then inform the Sangha: 

‘Please,\marginnote{11.2.3} venerables, I ask the Sangha to listen. The monk \textsanskrit{Upavāḷa}, while being examined in the midst of the Sangha about an offense, asserted things after denying them, denied things after asserting them, evaded the issue, and lied. If the Sangha is ready, it should do a legal procedure of further penalty against him. This is the motion. 

Please,\marginnote{11.2.7} venerables, I ask the Sangha to listen. The monk \textsanskrit{Upavāḷa}, while being examined in the midst of the Sangha about an offense, asserted things after denying them, denied things after asserting them, evaded the issue, and lied. The Sangha does a legal procedure of further penalty against him. Any monk who approves of doing a legal procedure of further penalty against him should remain silent. Any monk who doesn’t approve should speak up. 

For\marginnote{11.2.12} the second time, I speak on this matter. … For the third time, I speak on this matter. … 

The\marginnote{11.2.14} Sangha has done a legal procedure of further penalty against the monk \textsanskrit{Upavāḷa}. The Sangha approves and is therefore silent. I’ll remember it thus.’ 

There\marginnote{12.1.1} are these five factors of a legitimate legal procedure of further penalty: the subject of the procedure is impure; he’s shameless; he has been accused; the procedure is legitimate; the procedure is done by a unanimous assembly.” 

\subsection*{The group of twelve on illegitimate legal procedures }

“When\marginnote{12.2.1} a legal procedure of further penalty has three qualities, it’s illegitimate, contrary to the Monastic Law, and not properly resolved: it’s done in the absence of the accused, it’s done without questioning the accused, it’s done without the admission of the accused. 

When\marginnote{12.2.2.2} a procedure of further penalty has another three qualities, it’s also illegitimate, contrary to the Monastic Law, and not properly resolved: it’s done against one who hasn’t committed any offense, it’s done against one who’s committed an offense that isn’t clearable by confession, it’s done against one who’s confessed their offense. 

When\marginnote{12.2.2.3} a procedure of further penalty has another three qualities, it’s also illegitimate, contrary to the Monastic Law, and not properly resolved: it’s done without having accused the person of their offense, it’s done without having reminded the person of their offense, it’s done without having charged the person with their offense. 

When\marginnote{12.2.2.4} a procedure of further penalty has another three qualities, it’s also illegitimate, contrary to the Monastic Law, and not properly resolved: it’s done in the absence of the accused, it’s done illegitimately, it’s done by an incomplete assembly. 

When\marginnote{12.2.2.5} a procedure of further penalty has another three qualities, it’s also illegitimate, contrary to the Monastic Law, and not properly resolved: it’s done without questioning the accused, it’s done illegitimately, it’s done by an incomplete assembly. 

When\marginnote{12.2.2.6} a procedure of further penalty has another three qualities, it’s also illegitimate, contrary to the Monastic Law, and not properly resolved: it’s done without the admission of the accused, it’s done illegitimately, it’s done by an incomplete assembly. 

When\marginnote{12.2.2.7} a procedure of further penalty has another three qualities, it’s also illegitimate, contrary to the Monastic Law, and not properly resolved: it’s done against one who hasn’t committed any offense, it’s done illegitimately, it’s done by an incomplete assembly. 

When\marginnote{12.2.2.8} a procedure of further penalty has another three qualities, it’s also illegitimate, contrary to the Monastic Law, and not properly resolved: it’s done against one who’s committed an offense that isn’t clearable by confession, it’s done illegitimately, it’s done by an incomplete assembly. 

When\marginnote{12.2.2.9} a procedure of further penalty has another three qualities, it’s also illegitimate, contrary to the Monastic Law, and not properly resolved: it’s done against one who’s confessed their offense, it’s done illegitimately, it’s done by an incomplete assembly. 

When\marginnote{12.2.2.10} a procedure of further penalty has another three qualities, it’s also illegitimate, contrary to the Monastic Law, and not properly resolved: it’s done without having accused the person of their offense, it’s done illegitimately, it’s done by an incomplete assembly. 

When\marginnote{12.2.2.11} a procedure of further penalty has another three qualities, it’s also illegitimate, contrary to the Monastic Law, and not properly resolved: it’s done without having reminded the person of their offense, it’s done illegitimately, it’s done by an incomplete assembly. 

When\marginnote{12.2.3} a procedure of further penalty has another three qualities, it’s also illegitimate, contrary to the Monastic Law, and not properly resolved: it’s done without having charged the person with their offense, it’s done illegitimately, it’s done by an incomplete assembly.” 

\subsection*{The group of twelve on legitimate legal procedures }

“When\marginnote{12.3.1} a legal procedure of further penalty has three qualities, it’s legitimate, in accordance with the Monastic Law, and properly disposed of: it’s done in the presence of the accused, it’s done with the questioning of the accused, it’s done with the admission of the accused. 

When\marginnote{12.3.2.2} a procedure of further penalty has another three qualities, it’s also legitimate, in accordance with the Monastic Law, and properly resolved: it’s done against one who’s committed an offense, it’s done against one who’s committed an offense that’s clearable by confession, it’s done against one who hasn’t confessed their offense. 

When\marginnote{12.3.2.3} a procedure of further penalty has another three qualities, it’s also legitimate, in accordance with the Monastic Law, and properly resolved: it’s done after having accused the person of their offense, it’s done after having reminded the person of their offense, it’s done after having charged the person with their offense. 

When\marginnote{12.3.2.4} a procedure of further penalty has another three qualities, it’s also legitimate, in accordance with the Monastic Law, and properly resolved: it’s done in the presence of the accused, it’s done legitimately, it’s done by a unanimous assembly. 

When\marginnote{12.3.2.5} a procedure of further penalty has another three qualities, it’s also legitimate, in accordance with the Monastic Law, and properly resolved: it’s done with the questioning of the accused, it’s done legitimately, it’s done by a unanimous assembly. 

When\marginnote{12.3.2.6} a procedure of further penalty has another three qualities, it’s also legitimate, in accordance with the Monastic Law, and properly resolved: it’s done with the admission of the accused, it’s done legitimately, it’s done by a unanimous assembly. 

When\marginnote{12.3.2.7} a procedure of further penalty has another three qualities, it’s also legitimate, in accordance with the Monastic Law, and properly resolved: it’s done against one who’s committed an offense, it’s done legitimately, it’s done by a unanimous assembly. 

When\marginnote{12.3.2.8} a procedure of further penalty has another three qualities, it’s also legitimate, in accordance with the Monastic Law, and properly resolved: it’s done against one who’s committed an offense that’s clearable by confession, it’s done legitimately, it’s done by a unanimous assembly. 

When\marginnote{12.3.2.9} a procedure of further penalty has another three qualities, it’s also legitimate, in accordance with the Monastic Law, and properly resolved: it’s done against one who hasn’t confessed their offense, it’s done legitimately, it’s done by a unanimous assembly. 

When\marginnote{12.3.2.10} a procedure of further penalty has another three qualities, it’s also legitimate, in accordance with the Monastic Law, and properly resolved: it’s done after having accused the person of their offense, it’s done legitimately, it’s done by a unanimous assembly. 

When\marginnote{12.3.2.11} a procedure of further penalty has another three qualities, it’s also legitimate, in accordance with the Monastic Law, and properly resolved: it’s done after having reminded the person of their offense, it’s done legitimately, it’s done by a unanimous assembly. 

When\marginnote{12.3.3} a procedure of further penalty has another three qualities, it’s also legitimate, in accordance with the Monastic Law, and properly resolved: it’s done after having charged the person with their offense, it’s done legitimately, it’s done by a unanimous assembly.” 

\subsection*{The group of six on wishing }

“When\marginnote{12.4.1} a monk has three qualities, the Sangha may, if it wishes, do a legal procedure of further penalty against him: he’s quarrelsome, argumentative, and a creator of legal issues in the Sangha; he’s ignorant, incompetent, often committing offenses, and lacking in boundaries; he’s constantly and improperly socializing with householders. 

When\marginnote{12.4.6} a monk has another three qualities, the Sangha may, if it wishes, do a procedure of further penalty against him: he has failed in the higher morality; he has failed in conduct; he has failed in view. 

When\marginnote{12.4.9} a monk has another three qualities, the Sangha may, if it wishes, do a procedure of further penalty against him: he disparages the Buddha; he disparages the Teaching; he disparages the Sangha. 

The\marginnote{12.4.12} Sangha may, if it wishes, do a procedure of further penalty against three kinds of monks: those who are quarrelsome, argumentative, and creators of legal issues in the Sangha; those who are ignorant, incompetent, often committing offenses, and lacking in boundaries; those who are constantly and improperly socializing with householders. 

The\marginnote{12.4.18} Sangha may, if it wishes, do a procedure of further penalty against three other kinds of monks: those who’ve failed in the higher morality; those who’ve failed in the higher conduct; those who’ve failed in view. 

The\marginnote{12.4.21} Sangha may, if it wishes, do a procedure of further penalty against three other kinds of monks: those who disparage the Buddha; those who disparage the Teaching; those who disparage the Sangha.” 

\subsection*{The eighteen kinds of conduct }

“A\marginnote{12.5.1} monk who’s had a legal procedure of further penalty done against himself should conduct himself properly. This is the proper conduct: 

\begin{enumerate}%
\item He shouldn’t give the full ordination. %
\item He shouldn’t give formal support. %
\item He shouldn’t have a novice monk attend on him. %
\item He shouldn’t accept being appointed as an instructor of the nuns. %
\item Even if appointed, he shouldn’t instruct the nuns. %
\item He shouldn’t commit the same offense as the offense for which the Sangha did the procedure of further penalty against him. %
\item He shouldn’t commit an offense similar to the offense for which the Sangha did the procedure of further penalty against him. %
\item He shouldn’t commit an offense worse than the offense for which the Sangha did the procedure of further penalty against him. %
\item He shouldn’t criticize the procedure. %
\item He shouldn’t criticize those who did the procedure. %
\item He shouldn’t cancel the observance-day procedure of a regular monk. %
\item He shouldn’t cancel the invitation of a regular monk. %
\item He shouldn’t direct a regular monk. %
\item He shouldn’t give instructions to a regular monk. %
\item He shouldn’t ask a regular monk for permission to accuse him of an offense. %
\item He shouldn’t accuse a regular monk of an offense. %
\item He shouldn’t remind a regular monk of an offense. %
\item He shouldn’t associate inappropriately with other monks.” %
\end{enumerate}

The\marginnote{12.6.1} Sangha then did a legal procedure of further penalty against the monk \textsanskrit{Upavāḷa}. 

\section*{7. Covering over as if with grass }

At\marginnote{13.1.1} one time, while the monks were arguing and disputing, they did and said many things unworthy of monastics. They considered this and thought, “If we deal with one another for these offenses, this legal issue might lead to harshness, nastiness, and schism. So what should we do now?” They told the Buddha. 

\scrule{“It may be, monks, that monks who are arguing and disputing do and say many things unworthy of monastics. If they consider this and think, ‘If we deal with one another for these offenses, this legal issue might lead to harshness, nastiness, and schism,’ then I allow you to resolve that legal issue by covering over as if with grass. }

And\marginnote{13.2.1} it should be resolved like this. Everyone should gather in one place. A competent and capable monk should then inform the Sangha: 

‘Please,\marginnote{13.2.3} venerables, I ask the Sangha to listen. While we were arguing and disputing, we did and said many things unworthy of monastics. If we deal with one another for these offenses, this legal issue might lead to harshness, nastiness, and schism. If the Sangha is ready, it should resolve this legal issue by covering over as if with grass, except for heavy offenses and offenses connected with householders.’ 

The\marginnote{13.2.7} monks belonging to one side should then be informed by a competent and capable monk belonging to their own side: 

‘Please,\marginnote{13.2.8} venerables, I ask you to listen. While we were arguing and disputing, we did and said many things unworthy of monastics. If we deal with one another for these offenses, this legal issue might lead to harshness, nastiness, and schism. If the venerables are ready, then for your benefit and for my own, I’ll confess in the midst of the Sangha both your and my own offenses by covering over as if with grass, except for heavy offenses and offenses connected with householders.’ 

And\marginnote{13.2.12} the monks belonging to the other side should be informed by a competent and capable monk belonging to their own side: 

‘Please,\marginnote{13.2.13} venerables, I ask you to listen. While we were arguing and disputing, we did and said many things unworthy of monastics. If we deal with one another for these offenses, this legal issue might lead to harshness, nastiness, and schism. If the venerables are ready, then for your benefit and for my own, I’ll confess in the midst of the Sangha both your and my own offenses by covering over as if with grass, except for heavy offenses and offenses connected with householders.’ 

A\marginnote{13.3.1} competent and capable monk belonging to one side should then inform the Sangha:\footnote{I here follow the PTS reading, \textit{\textsanskrit{ekatopakkhikānaṁ} \textsanskrit{bhikkhūnaṁ}}, as opposed to the MS reading, \textit{\textsanskrit{athāparesaṁ} \textsanskrit{ekatopakkhikānaṁ} \textsanskrit{bhikkhūnaṁ}}. With the MS reading there is no distinction between the two groups of monks. } 

‘Please,\marginnote{13.3.2} venerables, I ask the Sangha to listen. While we were arguing and disputing, we did and said many things unworthy of monastics. If we deal with one another for these offenses, this legal issue might lead to harshness, nastiness, and schism. If the Sangha is ready, then for the benefit of these venerables and myself, I’ll confess in the midst of the Sangha both their and my own offenses by covering over as if with grass, except for heavy offenses and offenses connected with householders. This is the motion. 

Please,\marginnote{13.3.7} venerables, I ask the Sangha to listen. While we were arguing and disputing, we did and said many things unworthy of monastics. If we deal with one another for these offenses, this legal issue might lead to harshness, nastiness, and schism. For the benefit of these venerables and myself, I confess in the midst of the Sangha both their and my own offenses by covering over as if with grass, except for heavy offenses and offenses connected with householders. Any monk who approves of confessing our offenses in the midst of the Sangha by covering over as if with grass should remain silent. Any monk who doesn’t approve should speak up. 

We\marginnote{13.3.13} have confessed our offenses in the midst of the Sangha by covering over as if with grass, except for heavy offenses and offenses connected with householders. The Sangha approves and is therefore silent. I’ll remember it thus.’ 

And\marginnote{13.3.15} a competent and capable monk belonging to the other side should inform the Sangha: 

‘Please,\marginnote{13.3.16} venerables, I ask the Sangha to listen. While we were arguing and disputing, we did and said many things unworthy of monastics. If we deal with one another for these offenses, this legal issue might lead to harshness, nastiness, and schism. If the Sangha is ready, then for the benefit of these venerables and myself, I’ll confess in the midst of the Sangha both their and my own offenses by covering over as if with grass, except for heavy offenses and offenses connected with householders. This is the motion. 

Please,\marginnote{13.3.21} venerables, I ask the Sangha to listen. While we were arguing and disputing, we did and said many things unworthy of monastics. If we deal with one another for these offenses, this legal issue might lead to harshness, nastiness, and schism. For the benefit of these venerables and myself, I confess in the midst of the Sangha both their and my own offenses by covering over as if with grass, except for heavy offenses and offenses connected with householders. Any monk who approves of confessing our offenses in the midst of the Sangha by covering over as if with grass should remain silent. Any monk who doesn’t approve should speak up. 

We\marginnote{13.3.27} have confessed our offenses in the midst of the Sangha by covering over as if with grass, except for heavy offenses and offenses connected with householders. The Sangha approves and is therefore silent. I’ll remember it thus.’ 

In\marginnote{13.4.1} this way those monks are cleared of those offenses, except for heavy offenses and offenses connected with householders, and except for those monks who voice their disapproval and those who are absent.”\footnote{“Voice their disapproval”  renders \textit{\textsanskrit{diṭṭhāvikamma}}, literally, “disclosure of view”. This cannot mean an objection made in the midst of the Sangha, for that would invalidate the procedure. Sp 4.214: \textit{Ye pana “na \textsanskrit{metaṁ} \textsanskrit{khamatī}”ti \textsanskrit{aññamaññaṁ} \textsanskrit{diṭṭhāvikammaṁ} karonti … te \textsanskrit{āpattīhi} na \textsanskrit{vuṭṭhahanti}}, “But those who disclose their view to each other, saying, ‘I do not approve of this’ … they are not cleared of offenses.” } 

\section*{8. Legal issues }

On\marginnote{14.1.1} one occasion monks were disputing with monks, monks with nuns, and nuns with monks. Standing with the nuns, the monk Channa disputed with the monks, making others side with the nuns.\footnote{\textit{\textsanskrit{Bhikkhunīnaṁ} anupakhajja}, literally, “intruding on the nuns”. Sp 4.215: \textit{\textsanskrit{Bhikkhunīnaṁ} \textsanskrit{anupakhajjāti} \textsanskrit{bhikkhunīnaṁ} anto \textsanskrit{pavisitvā}}, “Intruding on the nuns means having entered among the nuns.” The contextual point is that he is siding with the nuns. } The monks of few desires … complained and criticized him, “How can the monk Channa act like this?” They told the Buddha. … “Is it true, monks, that the monk Channa is acting like this?” 

“It’s\marginnote{14.1.6} true, sir.” … 

After\marginnote{14.1.7} rebuking him … the Buddha gave a teaching and addressed the monks: 

\subsection*{Definitions}

“Monks,\marginnote{14.2.1} there are four kinds of legal issues: legal issues arising from disputes; legal issues arising from accusations; legal issues arising from offenses; legal issues arising from business.” 

“What’s\marginnote{14.2.2} a legal issue arising from a dispute? It may be that the monks are disputing, saying, ‘This is the Teaching’, ‘This is contrary to the Teaching’, ‘This is the Monastic Law’, ‘This is contrary to the Monastic Law’, ‘This was spoken by the Buddha’, ‘This wasn’t spoken by the Buddha’, ‘This was practiced by the Buddha’, ‘This wasn’t practiced by the Buddha’, ‘This was laid down by the Buddha’, ‘This wasn’t laid down by the Buddha’, ‘This is an offense’, ‘This isn’t an offense’, ‘This is a light offense’, ‘This is a heavy offense’, ‘This is a curable offense’, ‘This is an incurable offense’, ‘This is a grave offense’, or ‘This is a minor offense.’ In regard to this, whatever there is of quarreling, arguing, conflict, disputing, variety in opinion, difference in opinion, heated speech, or strife—this is called a legal issue arising from a dispute. 

What’s\marginnote{14.2.7} a legal issue arising from an accusation? It may be that the monks accuse a monk of failure in morality, failure in conduct, failure in view, or failure in livelihood. In regard to this, whatever there is of accusations, accusing, allegations, blame, taking sides because of friendship, taking part in the accusation, or supporting the accusation—this is called a legal issue arising from an accusation. 

What’s\marginnote{14.2.11} a legal issue arising from an offense? There are legal issues arising from offenses because of the five classes of offenses; there are legal issues arising from offenses because of the seven classes of offenses—this is called a legal issue arising from an offense. 

What’s\marginnote{14.2.14} a legal issue arising from business? Whatever is the duty or the business of the Sangha—a legal procedure consisting of getting permission, a legal procedure consisting of one motion, a legal procedure consisting of one motion and one announcement, a legal procedure consisting of one motion and three announcements—this is called a legal issue arising from business.” 

\subsection*{The roots of legal issues arising from disputes}

“What’s\marginnote{14.3.1} the root of legal issues arising from disputes? There are six roots of disputes that in turn are the root of legal issues arising from disputes. There are also three unwholesome and three wholesome roots of legal issues arising from disputes. 

What\marginnote{14.3.4} are the six roots of disputes that in turn are the root of legal issues arising from disputes? It may be that a monk is angry and resentful. One who’s angry and resentful is disrespectful and undeferential toward the Teacher, the Teaching, and the Sangha, and he doesn’t fulfill the training. Such a person creates disputes in the Sangha. Disputes are unbeneficial and a cause of unhappiness for humanity; they’re harmful, detrimental, and a cause of suffering for gods and humans. Monks, when you see such a root of disputes either in yourself or others, you should make an effort to get rid of it. If you don’t see such a root either in yourself or others, you should practice so that it doesn’t emerge in the future. In this way, that bad root of disputes is abandoned and doesn’t emerge in the future. 

Or\marginnote{14.3.15} it may be that a monk is denigrating and domineering, envious and stingy, treacherous and deceitful, one who has bad desires and wrong views, or one who obstinately grasps his own views and only gives them up with difficulty. Such a monk is disrespectful and undeferential toward the Teacher, the Teaching, and the Sangha, and he doesn’t fulfill the training. Such a person creates disputes in the Sangha. Disputes are unbeneficial and a cause of unhappiness for humanity; they’re harmful, detrimental, and a cause of suffering for gods and humans. Monks, when you see such a root of disputes either in yourself or others, you should make an effort to get rid of it. If you don’t see such a root either in yourself or others, you should practice so that it doesn’t emerge in the future. In this way that bad root of disputes is abandoned and doesn’t emerge in the future. 

What\marginnote{14.4.1} are the three unwholesome roots of legal issues arising from disputes? It may be that monks dispute with a mind of greed, ill will, or delusion, saying, ‘This is the Teaching’, ‘This is contrary to the Teaching’, ‘This is the Monastic Law’, ‘This is contrary to the Monastic Law’, ‘This was spoken by the Buddha’, ‘This wasn’t spoken by the Buddha’, ‘This was practiced by the Buddha’, ‘This wasn’t practiced by the Buddha’, ‘This was laid down by the Buddha’, ‘This wasn’t laid down by the Buddha’, ‘This is an offense’, ‘This isn’t an offense’, ‘This is a light offense’, ‘This is a heavy offense’, ‘This is a curable offense’, ‘This is an incurable offense’, ‘This is a grave offense’, or ‘This is a minor offense.’ 

What\marginnote{14.4.5} are the three wholesome roots of legal issues arising from disputes? It may be that monks dispute with a mind free from greed, ill will, and delusion, saying, ‘This is the Teaching’, ‘This is contrary to the Teaching’ … ‘This is a grave offense’, or ‘This is a minor offense.’” 

\subsection*{The roots of legal issues arising from accusations}

“What’s\marginnote{14.5.1} the root of legal issues arising from accusations? There are six roots of accusations that in turn are the root of legal issues arising from accusations. There are also three unwholesome and three wholesome roots of legal issues arising from accusations. The body, too, is a root of legal issues arising from accusations, as is speech. 

What\marginnote{14.5.4} are the six roots of accusations that in turn are the root of legal issues arising from accusations? It may be that a monk is angry and resentful. One who is angry and resentful is disrespectful and undeferential toward the Teacher, the Teaching, and the Sangha, and he doesn’t fulfill the training. Such a person makes accusations in the Sangha. Accusations are unbeneficial and a cause of unhappiness for humanity; they’re harmful, detrimental, and a cause of suffering for gods and humans. Monks, when you see such a root of accusations either in yourself or others, you should make an effort to get rid of it. If you don’t see such a root either in yourself or others, you should practice so that it doesn’t emerge in the future. 

Or\marginnote{14.5.15} it may be that a monk is denigrating and domineering, envious and stingy, treacherous and deceitful, one who has bad desires and wrong views, or one who obstinately grasps his own views and only gives them up with difficulty. Such a monk is disrespectful and undeferential toward the Teacher, the Teaching, and the Sangha, and he doesn’t fulfill the training. Such a person makes accusations in the Sangha. Accusations are unbeneficial and a cause of unhappiness for humanity; they’re harmful, detrimental, and a cause of suffering for gods and humans. Monks, when you see such a root of accusations either in yourself or others, you should make an effort to get rid of it. If you don’t see such a root either in yourself or others, you should practice so that it doesn’t emerge in the future. 

What\marginnote{14.5.27} are the three unwholesome roots of accusations? It may be that monks, because of greed, ill will, or delusion, accuse a monk of failure in morality, conduct, view, or livelihood. 

What\marginnote{14.5.31} are the three wholesome roots of accusations? It may be that monks, because of non-greed, non-ill will, and non-delusion, accuse a monk of failure in morality, conduct, view, or livelihood. 

How’s\marginnote{14.5.35} the body a root of legal issues arising from accusations? It may be that someone is ugly, unsightly, a dwarf, sickly, blind in one eye, crooked-limbed, lame, or paralyzed, and they blame him for that. 

How’s\marginnote{14.5.38} speech a root of legal issues arising from accusations? It may be that someone is difficult to correct, or he stutters or dribbles while speaking, and they blame him for that.” 

\subsection*{The roots of legal issues arising from offenses and business}

“What’s\marginnote{14.6.1} the root of legal issues arising from offenses? There are six originations of offenses that in turn are the root of legal issues arising from offenses. There are offenses that originate from the body, but not from speech or the mind. There are offenses that originate from speech, but not from the body or the mind. There are offenses that originate from the body and speech, but not from the mind. There are offenses that originate from the body and the mind, but not from speech. There are offenses that originate from speech and the mind, but not from the body. There are offenses that originate from the body, speech, and the mind. 

What’s\marginnote{14.7.1} the root of legal issues arising from business? There’s one root of legal issues arising from business: the Sangha.” 

\subsection*{Ethical qualities of legal issues arising from disputes}

“Is\marginnote{14.8.1} a legal issue arising from a dispute wholesome, unwholesome, or indeterminate?\footnote{According to the sub-commentary, the Pali should be understood as a question. Sp-\textsanskrit{ṭ} 4.220: \textit{\textsanskrit{Vivādādhikaraṇaṁ} \textsanskrit{kusalaṁ} \textsanskrit{akusalaṁ} \textsanskrit{abyākatanti} \textsanskrit{vivādādhikaraṇaṁ} \textsanskrit{kiṁ} \textsanskrit{kusalaṁ} \textsanskrit{akusalaṁ} \textsanskrit{udāhu} \textsanskrit{abyākatanti} pucchati. \textsanskrit{Vivādādhikaraṇaṁ} \textsanskrit{siyā} \textsanskrit{kusalantiādi} \textsanskrit{vissajjanaṁ}. Esa nayo sesesupi}, “\textit{\textsanskrit{Vivādādhikaraṇaṁ} \textsanskrit{kusalaṁ} \textsanskrit{akusalaṁ} \textsanskrit{abyākatan}}: he asks, ‘Is a legal issue arising from a dispute wholesome, unwholesome, or indeterminate?’ ‘A legal issue arising from a dispute may be wholesome,’ etc., is the response. This method also applies to the rest (below).” } A legal issue arising from a dispute may be wholesome, unwholesome, or indeterminate. What’s a wholesome legal issue arising from a dispute? It may be that monks dispute with a wholesome mind, saying, ‘This is the Teaching’, ‘This is contrary to the Teaching’ … ‘This is a grave offense’, or ‘This is a minor offense.’ In regard to this, whatever there is of quarreling, arguing, conflict, disputing, variety in opinion, difference in opinion, heated speech, or strife—this is called a wholesome legal issue arising from a dispute. 

What’s\marginnote{14.8.9} an unwholesome legal issue arising from a dispute? It may be that monks dispute with an unwholesome mind, saying, ‘This is the Teaching’, ‘This is contrary to the Teaching’ … ‘This is a grave offense’, or ‘This is a minor offense.’ In regard to this, whatever there is of quarreling, arguing, conflict, disputing, variety in opinion, difference in opinion, heated speech, or strife—this is called an unwholesome legal issue arising from a dispute. 

What’s\marginnote{14.8.15} an indeterminate legal issue arising from a dispute? It may be that monks dispute with an indeterminate mind, saying, ‘This is the Teaching’, ‘This is contrary to the Teaching’ … ‘This is a grave offense’, or ‘This is a minor offense.’ In regard to this, whatever there is of quarreling, arguing, conflict, disputing, variety in opinion, difference in opinion, heated speech, or strife—this is called an indeterminate legal issue arising from a dispute.” 

\subsection*{Ethical qualities of legal issues arising from accusations}

“Is\marginnote{14.9.1} a legal issue arising from an accusation wholesome, unwholesome, or indeterminate? A legal issue arising from an accusation may be wholesome, unwholesome, or indeterminate. What’s a wholesome legal issue arising from an accusation? It may be that monks with wholesome minds accuse a monk of failure in morality, conduct, view, or livelihood. In regard to this, whatever there is of accusations, accusing, allegations, blame, taking sides because of friendship, taking part in the accusation, or supporting the accusation—this is called a wholesome legal issue arising from an accusation. 

What’s\marginnote{14.9.8} an unwholesome legal issue arising from an accusation? It may be that monks with unwholesome minds accuse a monk of failure in morality,  conduct, view, or livelihood. In regard to this, whatever there is of accusations, accusing, allegations, blame, taking sides because of friendship, taking part in the accusation, or supporting the accusation—this is called an unwholesome legal issue arising from an accusation. 

What’s\marginnote{14.9.13} an indeterminate legal issue arising from an accusation? It may be that monks with indeterminate minds accuse a monk of failure in morality, conduct, view, or livelihood. In regard to this, whatever there is of accusations, accusing, allegations, blame, taking sides because of friendship, taking part in the accusation, or supporting the accusation—this is called an indeterminate legal issue arising from an accusation.” 

“Is\marginnote{14.9.18} a legal issue arising from an offense wholesome, unwholesome, or indeterminate? 

\subsection*{Ethical qualities of legal issues arising from offenses}

A\marginnote{14.10.1} legal issue arising from an offense may be unwholesome or indeterminate. There are no wholesome legal issues arising from an offense. What’s an unwholesome legal issue arising from an offense? When one transgresses, knowing, perceiving, having intended, having decided—this is called an unwholesome legal issue arising from an offense. 

What’s\marginnote{14.10.6} an indeterminate legal issue arising from an offense? When one transgresses, not knowing, not perceiving, not having intended, not having decided—this is called an indeterminate legal issue arising from an offense.” 

\subsection*{Ethical qualities of legal issues arising from business}

“Is\marginnote{14.11.1} a legal issue arising from business wholesome, unwholesome, or indeterminate? A legal issue arising from business may be wholesome, unwholesome, or indeterminate. What’s a wholesome legal issue arising from business? When the Sangha does a legal procedure with a wholesome mind—whether a procedure consisting of getting permission, a procedure consisting of one motion, a procedure consisting of one motion and one announcement, or a procedure consisting of one motion and three announcements—this is called a wholesome legal issue arising from business. 

What’s\marginnote{14.11.7} an unwholesome legal issue arising from business? When the Sangha does a legal procedure with an unwholesome mind—whether a procedure consisting of getting permission, a procedure consisting of one motion, a procedure consisting of one motion and one announcement, or a procedure consisting of one motion and three announcements—this is called an unwholesome legal issue arising from business. 

What’s\marginnote{14.11.11} an indeterminate legal issue arising from business? When the Sangha does a legal procedure with an indeterminate mind—whether a procedure consisting of getting permission, a procedure consisting of one motion, a procedure consisting of one motion and one announcement, or a procedure consisting of one motion and three announcements—this is called an indeterminate legal issue arising from business.” 

\subsection*{Relationship between disputes and legal issues}

“Are\marginnote{14.12.1} there disputes that are also legal issues arising from a dispute? Are there disputes that aren’t also legal issues? Are there legal issues that aren’t also disputes? Are there legal issues that are also disputes? 

There\marginnote{14.12.2} are disputes that are also legal issues arising from a dispute. There are disputes that aren’t also legal issues. There are legal issues that aren’t also disputes. There are legal issues that are also disputes. 

How’s\marginnote{14.12.3} there a dispute that’s also a legal issue arising from a dispute? It may be that monks are disputing, saying, ‘This is the Teaching’, ‘This is contrary to the Teaching’ … ‘This is a grave offense’, or ‘This is a minor offense.’ In regard to this, whatever there is of quarreling, arguing, conflict, disputing, variety in opinion, difference in opinion, heated speech, or strife—this is a dispute that’s also a legal issue arising from a dispute. 

How’s\marginnote{14.12.9} there a dispute that isn’t also a legal issue? A mother disputes with her offspring; an offspring with their mother; a father with his offspring; an offspring with their father; a brother with his brother; a brother with his sister; a sister with her brother; a friend with their friend—this is a dispute that isn’t also a legal issue. 

How’s\marginnote{14.12.12} there a legal issue that isn’t also a dispute? A legal issue arising from an accusation, a legal issue arising from an offense, a legal issue arising from business—this is a legal issue that isn’t also a dispute. 

How’s\marginnote{14.12.15} there a legal issue that’s also a dispute? A legal issue arising from a dispute is a legal issue and also a dispute.” 

\subsection*{Relationship between accusations and legal issues}

“Are\marginnote{14.13.1} there accusations that are also legal issues arising from accusations? Are there accusations that aren’t also legal issues? Are there legal issues that aren’t also accusations? Are there legal issues that are also accusations? 

There\marginnote{14.13.2} are accusations that are also legal issues arising from accusations. There are accusations that aren’t also legal issues. There are legal issues that aren’t also accusations. There are legal issues that are also accusations. 

How’s\marginnote{14.13.3} there an accusation that’s also a legal issue arising from an accusation? It may be that monks accuse a monk of failure in morality, conduct, view, or livelihood. In regard to this, whatever there is of accusations, accusing, allegations, blame, taking sides because of friendship, taking part in the accusation, or supporting the accusation—this is an accusation that’s also a legal issue arising from an accusation. 

How’s\marginnote{14.13.8} there an accusation that isn’t also a legal issue? A mother accuses her offspring; an offspring their mother; a father his offspring; an offspring their father; a brother his brother; a brother his sister; a sister her brother; a friend their friend—this is an accusation that isn’t also a legal issue. 

How’s\marginnote{14.13.11} there a legal issue that isn’t also an accusation? A legal issue arising from an offense, a legal issue arising from business, a legal issue arising from a dispute—this is a legal issue that isn’t also an accusation. 

How’s\marginnote{14.13.14} there a legal issue that’s also an accusation? A legal issue arising from an accusation is a legal issue and also an accusation.” 

\subsection*{Relationship between offenses and legal issues}

“Are\marginnote{14.14.1} there offenses that are also legal issues arising from an offense? Are there offenses that aren’t also legal issues? Are there legal issues that aren’t also offenses? Are there legal issues that are also offenses? 

There\marginnote{14.14.2} are offenses that are also legal issues arising from an offense. There are offenses that aren’t also legal issues. There are legal issues that aren’t also offenses. There are legal issues that are also offenses. 

How’s\marginnote{14.14.3} there an offense that’s also a legal issue arising from an offense? There’s a legal issue arising from an offense because of the five classes of offenses, and there’s a legal issue arising from an offense because of the seven classes of offenses—this is an offense that’s also a legal issue arising from an offense. 

How’s\marginnote{14.14.6} there an offense that isn’t also a legal issue? The attainment of stream-entry—this is an attainment/offense that isn’t also a legal issue.\footnote{The Pali word \textit{\textsanskrit{āpatti}} means both “attainment” and “offense”. } 

How’s\marginnote{14.14.9} there a legal issue that isn’t also an offense? A legal issue arising from business, a legal issue arising from a dispute, a legal issue arising from an accusation—this is a legal issue that isn’t also an offense. 

How’s\marginnote{14.14.12} there a legal issue that’s also an offense? A legal issue arising from an offense is a legal issue and also an offense.” 

\subsection*{Relationship between business and legal issues}

“Is\marginnote{14.15.1} there business that’s also a legal issue arising from business? Is there business that isn’t also a legal issue? Are there legal issues that aren’t also business? Are there legal issues that are also business? 

There’s\marginnote{14.15.2} business that’s also a legal issue arising from business. There’s business that’s not a legal issue. There are legal issues that aren’t also business. There are legal issues that are also business. 

How’s\marginnote{14.15.3} there business that’s also a legal issue arising from business? Whatever is the duty or the business of the Sangha—a legal procedure consisting of getting permission, a legal procedure consisting of one motion, a legal procedure consisting of one motion and one announcement, a legal procedure consisting of one motion and three announcements—this is business that’s also a legal issue arising from business. 

How’s\marginnote{14.15.6} there business that isn’t also a legal issue? The duty to teacher, the duty to a preceptor, the duty to a co-student, the duty to a co-pupil—this is business that isn’t also a legal issue. 

How’s\marginnote{14.15.9} there a legal issue that isn’t also business? A legal issue arising from a dispute, a legal issue arising from an accusation, a legal issue arising from an offense—this is a legal issue that isn’t also business. 

How’s\marginnote{14.15.12} there a legal issue that’s also business? A legal issue arising from business is a legal issue and also business.” 

\section*{9. The resolution and settling of legal issues }

\subsection*{Resolution face-to-face }

“There\marginnote{14.16.1} are two principles for settling a legal issue arising from a dispute: resolution face-to-face and majority decision. Is it possible that a legal issue arising from a dispute should be settled not by majority decision, but by resolution face-to-face? It is. How? It may be, monks, that monks are disputing, saying, ‘This is the Teaching’, ‘This is contrary to the Teaching’, ‘This is the Monastic Law’, ‘This is contrary to the Monastic Law’, ‘This was spoken by the Buddha’, ‘This wasn’t spoken by the Buddha’, ‘This was practiced by the Buddha’, ‘This wasn’t practiced by the Buddha’, ‘This was laid down by the Buddha’, ‘This wasn’t laid down by the Buddha’, ‘This is an offense’, ‘This isn’t an offense’, ‘This is a light offense’, ‘This is a heavy offense’, ‘This is a curable offense’, ‘This is an incurable offense’, ‘This is a grave offense’, or ‘This is a minor offense.’ 

If\marginnote{14.16.9} those monks are able to resolve that legal issue, this is called the resolution of a legal issue. It’s been resolved face-to-face. Face-to-face with what? Face-to-face with the Sangha, the Teaching, the Monastic Law, and the persons concerned. This is the meaning of face-to-face with the Sangha: the monks who should be present have arrived, consent has been brought for those who are eligible to give their consent, and no-one present objects to the decision. This is the meaning of face-to-face with the Teaching and the Monastic Law: the Teaching, the Monastic Law, the Teacher’s instruction—that by which that legal issue is resolved. This is the meaning of face-to-face with the persons concerned: both sides—those who are disputing and those they’re disputing with—are present. When a legal issue has been resolved like this, if any of the participants reopen it, they incur an offense entailing confession for the reopening.\footnote{See \href{https://suttacentral.net/pli-tv-bu-vb-pc63/en/brahmali\#1.12.1}{Bu Pc 63:1.12.1}. } If anyone who gave their consent criticizes the resolution, they incur an offense entailing confession.\footnote{See \href{https://suttacentral.net/pli-tv-bu-vb-pc79/en/brahmali\#1.22.1}{Bu Pc 79:1.22.1}. } 

If\marginnote{14.17.1} those monks are unable to resolve that legal issue in that monastery, they should go to another monastery that has a number of monks. If they’re able to resolve that legal issue while on their way, this is called the resolution of a legal issue. It’s been resolved face-to-face. Face-to-face with what? Face-to-face with the Sangha, the Teaching, the Monastic Law, and the persons concerned. … When a legal issue has been resolved like this, if any of the participants reopen it, they incur an offense entailing confession for the reopening. If anyone who gave their consent criticizes the resolution, they incur an offense entailing confession. 

If\marginnote{14.18.1} those monks are unable to resolve that legal issue while on their way, they should go to that other monastery and say to the resident monks, ‘This legal issue has come about in such-and-such a way. Please resolve it, venerables, according to the Teaching, the Monastic Law, and the Teacher’s instruction, so that this legal issue may be properly disposed of.’ 

If\marginnote{14.18.4} the resident monks are senior to the newly-arrived monks, the resident monks should say, ‘Now, venerables, please go to one side for a moment while we discuss this matter.’ If the newly-arrived monks are senior to the resident monks, the resident monks should say, ‘Well then, venerables, please wait right here for a moment while we discuss this matter.’ 

If,\marginnote{14.18.8} while discussing that matter, the resident monks think, ‘We’re unable to resolve this legal issue according to the Teaching, the Monastic Law, and the Teacher’s instruction,’ then they shouldn’t take on that legal issue. But if they think, ‘We’re able to resolve this legal issue according to the Teaching, the Monastic Law, and the Teacher’s instruction,’ then they should say to the newly-arrived monks, ‘If you’ll tell us how this legal issue came about, we’ll be able to dispose of it properly according to the Teaching, the Monastic Law, and the Teacher’s instruction. And so we’ll take it on. If, however, you won’t tell us, we won’t be able to properly dispose of it. And so we won’t take it on.’ Having properly examined it like this, the resident monks should take it on. 

The\marginnote{14.18.17} newly-arrived monks should say this to the resident monks, ‘We’ll tell you how this legal issue came about. If you’re able to dispose of it properly by such and such means—according to the Teaching, the Monastic Law, and the Teacher’s instruction—\footnote{Sp-\textsanskrit{ṭ} 4.230: \textit{\textsanskrit{Antarenāti} \textsanskrit{kāraṇena}}, “\textit{Antarena}: by means of”. } then we’ll hand it over to you. If you’re unable to properly dispose of it  by such and such means, then we won’t hand it over to you. We’ll take responsibility for it ourselves.’ Having properly examined it like this, the newly-arrived monks should hand it over to the resident monks. 

If\marginnote{14.18.24} those monks are able to resolve that legal issue, this is called the resolution of a legal issue. It’s been resolved face-to-face. Face-to-face with what? Face-to-face with the Sangha, the Teaching, the Monastic Law, and the persons concerned. … When a legal issue has been resolved like this, if any of the participants reopen it, they incur an offense entailing confession for the reopening. If anyone who gave their consent criticizes the resolution, they incur an offense entailing confession.” 

\subsection*{Resolution by committee }

“If,\marginnote{14.19.1} monks, while they’re discussing that legal issue, there’s endless talk, but not a single statement that’s clear, then they should resolve that legal issue by means of a committee. 

A\marginnote{14.19.2} monk who has ten qualities may be appointed to that committee: 

\begin{enumerate}%
\item One who’s virtuous and restrained by the Monastic Code. His conduct is good, he associates with the right people, and he sees danger in minor faults. He undertakes and trains in the training rules. %
\item One who has learned much, and who retains and accumulates what he has learned. Those teachings that are good in the beginning, good in the middle, and good in the end, that have a true goal and are well articulated, and that set out the perfectly complete and pure spiritual life—he has learned many such teachings, retained them in mind, recited them verbally, mentally investigated them, and penetrated them well by view. %
\item One who has properly learned both Monastic Codes in detail; who has analyzed them well, thoroughly mastered them, and investigated them well, both in terms of the rules and their detailed exposition. %
\item One who is firmly committed to the Monastic Law. %
\item One who is capable of making both sides relax, of persuading them, of convincing them, of making them see, of reconciling them. %
\item One who is knowledgeable about the arising and resolution of legal issues. %
\item One who understands legal issues. %
\item One who understands the arising of legal issues. %
\item One who understands the ending of legal issues. %
\item One who understands the way to the ending of legal issues. %
\end{enumerate}

And\marginnote{14.20.1} he should be appointed like this. First a monk should be asked, and then a competent and capable monk should inform the Sangha: 

‘Please,\marginnote{14.20.3} venerables, I ask the Sangha to listen. While we were discussing this legal issue, there was endless talk, but not a single statement that was clear. If the Sangha is ready, it should appoint monk so-and-so and monk so-and-so to a committee to resolve this legal issue. This is the motion. 

Please,\marginnote{14.20.7} venerables, I ask the Sangha to listen. While we were discussing this legal issue, there was endless talk, but not a single statement that was clear. The Sangha appoints monk so-and-so and monk so-and-so to a committee to resolve this legal issue. Any monk who approves of appointing monk so-and-so and monk so-and-so to a committee should remain silent. Any monk who doesn’t approve should speak up. 

The\marginnote{14.20.12} Sangha has appointed monk so-and-so and monk so-and-so to a committee to resolve this legal issue. The Sangha approves and is therefore silent. I’ll remember it thus.’ 

If\marginnote{14.21.1} those monks are able to resolve that legal issue by committee, this is called the resolution of a legal issue. It’s been resolved face-to-face. Face-to-face with what? Face-to-face with the Sangha, the Teaching, the Monastic Law, and the persons concerned. This is the meaning of face-to-face with the Sangha: the monks who should be present have arrived, consent has been brought for those who are eligible to give their consent, and no-one present objects to the decision. This is the meaning of face-to-face with the Teaching and the Monastic Law: the Teaching, the Monastic Law, the Teacher’s instruction—that by which that legal issue is resolved. This is the meaning of face-to-face with the persons concerned: both sides—those who are disputing and those they’re disputing with—are present. When a legal issue has been resolved like this, if any of the participants reopen it, they incur an offense entailing confession for the reopening. 

\subsection*{Dealing with obstructive monks}

While\marginnote{14.22.1} those monks are discussing that legal issue, there may be a monk there who’s an expounder of the Teaching, but who doesn’t know the Monastic Code or its analysis. Not understanding the meaning, he obstructs it by obscuring it with the wording. A competent and capable monk should then inform the Sangha: 

‘Please,\marginnote{14.22.2} venerables, I ask you to listen. The monk so-and-so is an expounder of the Teaching, but doesn’t know the Monastic Code or its analysis. Not understanding the meaning, he obstructs it by obscuring it with the wording. If the venerables are ready, we should ask monk so-and-so to leave, with the remainder of us resolving this legal issue.’ 

If,\marginnote{14.22.7} after that monk has left, those monks are able to resolve that legal issue, this is called the resolution of a legal issue. It’s been resolved face-to-face. Face-to-face with what? Face-to-face with the Sangha, the Teaching, the Monastic Law, and the persons concerned. … When a legal issue has been resolved like this, if any of the participants reopen it, they incur an offense entailing confession for the reopening. 

While\marginnote{14.23.1} those monks are discussing that legal issue, there may be a monk there who’s an expounder of the Teaching. He knows the Monastic Code, but not its analysis. Not understanding the meaning, he obstructs it by obscuring it with the wording. A competent and capable monk should then inform the Sangha: 

‘Please,\marginnote{14.23.2} venerables, I ask you to listen. The monk so-and-so is an expounder of the Teaching. He knows the Monastic Code, but not its analysis. Not understanding the meaning, he obstructs it by obscuring it with the wording. If the venerables are ready, we should ask monk so-and-so to leave, with the remainder of us resolving this legal issue.’ 

If,\marginnote{14.23.7} after that monk has left, those monks are able to resolve that legal issue, this is called the resolution of a legal issue. It’s been resolved face-to-face. Face-to-face with what? Face-to-face with the Sangha, the Teaching, the Monastic Law, and the persons concerned. … When a legal issue has been resolved like this, if any of the participants reopen it, they incur an offense entailing confession for the reopening.” 

\subsection*{Resolution by majority decision }

\scrule{“If, monks, those monks are unable to resolve that legal issue by committee, they should hand it over to the Sangha, saying, ‘Venerables, we’re unable to resolve this legal issue by committee. Can the Sangha please resolve it.’ I allow such a legal issue to be resolved by majority decision. }

A\marginnote{14.24.4} monk who has five qualities should be appointed as the manager of the vote: one who isn’t biased by favoritism, ill will, confusion, or fear, and who knows who has and who hasn’t voted.\footnote{“Who knows who has and who hasn’t voted” renders \textit{\textsanskrit{gahitāgahitañca} \textsanskrit{jāneyya}}, literally, “He should know taken and not taken.” The taking refers to the taking of ballots, \textit{\textsanskrit{salāka}}. } 

And\marginnote{14.24.6} he should be appointed like this. First a monk should be asked, and then a competent and capable monk should inform the Sangha: 

‘Please,\marginnote{14.24.8} venerables, I ask the Sangha to listen. If the Sangha is ready, it should appoint monk so-and-so as the manager of the vote. This is the motion. 

Please,\marginnote{14.24.11} venerables, I ask the Sangha to listen. The Sangha appoints monk so-and-so as the manager of the vote. Any monk who approves of appointing monk so-and-so as the manager of the vote should remain silent. Any monk who doesn’t approve should speak up. 

The\marginnote{14.24.15} Sangha has appointed monk so-and-so as the manager of the vote. The Sangha approves and is therefore silent. I’ll remember it thus.’ 

That\marginnote{14.24.17} monk should then distribute the ballots. If the majority of monks vote in accordance with the Teaching, then that legal issue has been resolved. This is called the resolution of a legal issue. It’s been resolved face-to-face and by majority decision. Face-to-face with what? Face-to-face with the Sangha, the Teaching, the Monastic Law, and the persons concerned. This is the meaning of face-to-face with the Sangha: the monks who should be present have arrived, consent has been brought for those who are eligible to give their consent, and no-one present objects to the decision. This is the meaning of face-to-face with the Teaching and the Monastic Law: the Teaching, the Monastic Law, the Teacher’s instruction—that by which that legal issue is resolved. This is the meaning of face-to-face with the persons concerned: both sides—those who are disputing and those they’re disputing with—are present. This is the meaning of by majority decision: the doing of, the performing of, the participation in, the consent to, the agreement to, the non-objection to that legal procedure of majority decision. When a legal issue has been resolved like this, if any of the participants reopen it, they incur an offense entailing confession for the reopening. If anyone who gave their consent criticizes the resolution, they incur an offense entailing confession.” 

\subsection*{The three kinds of voting }

At\marginnote{14.25.1} one time at \textsanskrit{Sāvatthī} a legal issue had come about in this way.\footnote{That is, it had come about in accordance with the immediately preceding discussion. } But there were monks who were dissatisfied with how the Sangha at \textsanskrit{Sāvatthī} had resolved it. They heard that in a certain monastery there was a number of senior monks who were learned and masters of the tradition; who were experts on the Teaching, the Monastic Law, and the Key Terms; who were knowledgeable and competent, had a sense of conscience, and were afraid of wrongdoing and fond of the training. They thought, “If these senior monks resolve this legal issue—according to the Teaching, the Monastic Law, and the Teacher’s instruction—then it will be properly disposed of.” They then went to that monastery and said to those senior monks, “This legal issue has come about like this. Venerables, please resolve it—according to the Teaching, the Monastic Law, and the Teacher’s instruction—so that it’ll be properly disposed of.” Those senior monks thought, “This legal issue was properly disposed of by the Sangha at \textsanskrit{Sāvatthī},” and they resolved it in the same way. 

Those\marginnote{14.25.12} monks were dissatisfied with how the Sangha at \textsanskrit{Sāvatthī} had resolved that legal issue and also with how that number of senior monks had resolved it. They then heard that in a certain monastery there were three senior monks … two senior monks … one senior monk who was learned and a master of the tradition; who was an expert on the Teaching, the Monastic Law, and the Key Terms; who was knowledgeable and competent, had a sense of conscience, and was afraid of wrongdoing and fond of the training. They thought, “If this senior monk resolves this legal issue—according to the Teaching, the Monastic Law, and the Teacher’s instruction—then it will be properly disposed of.” They then went to that monastery and said to that senior monk, “This legal issue has come about in this way. Venerable, please resolve it—according to the Teaching, the Monastic Law, and the Teacher’s instruction—so that it will be properly disposed of.” That senior monk thought, “This legal issue was properly disposed of by the Sangha at \textsanskrit{Sāvatthī}, and likewise by that number of senior monks, by those three senior monks, and by those two senior monks,” and he disposed of it in the same way. 

Since\marginnote{14.25.24} those monks were dissatisfied with how all of these had disposed of that legal issue, they went to the Buddha and told him what had happened. … The Buddha said: 

\scrule{“Monks, this legal issue has been settled, laid to rest, and properly disposed of. To persuade those monks, I allow three kinds of voting: a secret ballot, whispering in the ear, and an open vote. }

What’s\marginnote{14.26.3} a secret ballot? The manager of the vote should make ballots of two different colors, go up to the monks one by one and say, ‘This is the ballot for those who hold this view, and this is the ballot for those who hold that view. Take the one you like.’ When anyone has taken a ballot, they should be told, ‘Don’t show it to anyone.’ If the manager knows that those who speak contrary to the Teaching are in the majority, then the vote is invalid and to be postponed. If he knows that those who speak in accordance with the Teaching are in the majority, then the vote is valid and to be announced. 

What’s\marginnote{14.26.14} voting by whispering in the ear? The manager of the vote should inform the monks one by one by whispering in the ear, ‘This is the ballot for those who hold this view, and this is the ballot for those who hold that view. Take the one you like.’ When someone has taken a ballot, they should be told, ‘Don’t tell anyone.’ If the manager knows that those who speak contrary to the Teaching are in the majority, then the vote is invalid and to be postponed. If he knows that those who speak in accordance with the Teaching are in the majority, then the vote is valid and to be announced. 

What’s\marginnote{14.26.25} an open vote? If he knows that those who speak in accordance with the Teaching are in the majority, the ballots should be distributed openly.” 

\subsection*{Resolution through recollection }

“There\marginnote{14.27.1} are four principles for settling a legal issue arising from an accusation: resolution face-to-face, resolution through recollection, resolution because of past insanity, and by further penalty. Is it possible that a legal issue arising from an accusation should be settled not by resolution because of past insanity or by further penalty, but by resolution face-to-face and by resolution through recollection? It is. How? It may be that monks are groundlessly charging a monk with failure in morality. If that monk has great clarity of memory, he’s to be granted resolution through recollection. 

And\marginnote{14.27.12} it should be granted like this. That monk should approach the Sangha, arrange his upper robe over one shoulder, pay respect at the feet of the senior monks, squat on his heels, raise his joined palms, and say: 

‘Venerables,\marginnote{14.27.15} monks are groundlessly charging me with failure in morality. Because of my great clarity of memory, I ask the Sangha for resolution through recollection.’ And he should ask a second and a third time. 

A\marginnote{14.27.19} competent and capable monk should then inform the Sangha: 

‘Please,\marginnote{14.27.20} venerables, I ask the Sangha to listen. Monks are groundlessly charging monk so-and-so with failure in morality. Because of his great clarity of memory, he’s asking the Sangha for resolution through recollection. If the Sangha is ready, it should grant him resolution through recollection. This is the motion. 

Please,\marginnote{14.27.25} venerables, I ask the Sangha to listen. Monks are groundlessly charging monk so-and-so with failure in morality. Because of his great clarity of memory, he’s asking the Sangha for resolution through recollection. The Sangha grants him resolution through recollection. Any monk who approves of granting him resolution through recollection should remain silent. Any monk who doesn’t approve should speak up. 

For\marginnote{14.27.31} the second time, I speak on this matter. … For the third time, I speak on this matter. … 

Because\marginnote{14.27.33} of his great clarity of memory, the Sangha has granted monk so-and-so resolution through recollection. The Sangha approves and is therefore silent. I’ll remember it thus.’ 

This\marginnote{14.27.35} is called the resolution of a legal issue. It’s been resolved by resolution face-to-face and by resolution through recollection. Face-to-face with what? Face-to-face with the Sangha, the Teaching, the Monastic Law, and the persons concerned. This is the meaning of face-to-face with the Sangha: the monks who should be present have arrived, consent has been brought for those who are eligible to give their consent, and no-one present objects to the decision. This is the meaning of face-to-face with the Teaching and the Monastic Law: the Teaching, the Monastic Law, the Teacher’s instruction—that by which that legal issue is resolved. This is the meaning of face-to-face with the persons concerned: both those who are accusing and those who have been accused are present. This is the meaning of resolution through recollection: the doing of, the performing of, the participation in, the consent to, the agreement to, the non-objection to that legal procedure of resolution through recollection. When a legal issue has been resolved like this, if any of the participants reopen it, they incur an offense entailing confession for the reopening. If anyone who gave their consent criticizes the resolution, they incur an offense entailing confession.” 

\subsection*{Resolution because of past insanity }

“Is\marginnote{14.28.1} it possible that a legal issue arising from an accusation should be settled not by resolution through recollection or by further penalty, but by resolution face-to-face and by resolution because of past insanity? It is. How? It may be that a monk is insane and suffering from psychosis. Because of that, he does and says many things unworthy of a monastic. Monks accuse him of an offense, saying, ‘Venerable, do you remember committing such-and-such an offense?’ He replies, ‘I was insane and suffering from psychosis. Because of that, I did and said many things unworthy of a monastic. I don’t remember it. I did it because I was insane.’ But they still accuse him in the same way. If he’s no longer insane, that monk should be granted resolution because of past insanity. 

And\marginnote{14.28.19} it should be granted like this. That monk should approach the Sangha, arrange his upper robe over one shoulder, pay respect at the feet of the senior monks, squat on his heels, raise his joined palms, and say: 

‘Venerables,\marginnote{14.28.22} I’ve been insane and suffering from psychosis. Because of that, I did and said many things unworthy of a monastic. Monks accused me of an offense, saying, “Venerable, do you remember committing such-and-such an offense?” I replied, “I was insane and suffering from psychosis. Because of that, I did and said many things unworthy of a monastic. I don’t remember it. I did it because I was insane.” But they still accused me in the same way. Because I’m no longer insane, I ask the Sangha for resolution because of past insanity.’ And he should ask a second and a third time. 

A\marginnote{14.28.36} competent and capable monk should then inform the Sangha: 

‘Please,\marginnote{14.28.37} venerables, I ask the Sangha to listen. The monk so-and-so has been insane and suffering from psychosis. Because of that, he did and said many things unworthy of a monastic. Monks accused him of an offense, saying, “Venerable, do you remember committing such-and-such an offense?” He replied, “I was insane and suffering from psychosis. Because of that, I did and said many things unworthy of a monastic. I don’t remember it. I did it because I was insane.” But they still accused him in the same way. Because he’s no longer insane, he’s asking the Sangha for resolution because of past insanity. If the Sangha is ready, it should grant monk so-and-so resolution because of past insanity. This is the motion. 

Please,\marginnote{14.28.52} venerables, I ask the Sangha to listen. The monk so-and-so has been insane and suffering from psychosis. Because of that, he did and said many things unworthy of a monastic. Monks accused him of an offense, saying, “Venerable, do you remember committing such-and-such an offense?” He replied, “I was insane and suffering from psychosis. Because of that, I did and said many things unworthy of a monastic. I don’t remember it. I did it because I was insane.” But they still accused him in the same way. Because he’s no longer insane, he’s asking the Sangha for resolution because of past insanity. The Sangha grants monk so-and-so resolution because of past insanity. Any monk who approves of granting monk so-and-so resolution because of past insanity should remain silent. Any monk who doesn’t approve should speak up. 

For\marginnote{14.28.68} the second time, I speak on this matter. … For the third time, I speak on this matter. … 

Because\marginnote{14.28.70} he’s no longer insane, the Sangha has granted monk so-and-so resolution because of past insanity. The Sangha approves and is therefore silent. I’ll remember it thus.’ 

This\marginnote{14.28.72} is called the resolution of a legal issue. It’s been resolved by resolution face-to-face and by resolution because of past insanity. Face-to-face with what? Face-to-face with the Sangha, the Teaching, the Monastic Law, and the persons concerned. This is the meaning of face-to-face with the Sangha: the monks who should be present have arrived, consent has been brought for those who are eligible to give their consent, and no-one present objects to the decision. This is the meaning of face-to-face with the Teaching and the Monastic Law: the Teaching, the Monastic Law, the Teacher’s instruction—that by which that legal issue is resolved. This is the meaning of face-to-face with the persons concerned: both those who are accusing and those who have been accused are present. This is the meaning of resolution because of past insanity: the doing of, the performing of, the participation in, the consent to, the agreement to, the non-objection to that legal procedure of resolution because of past insanity. When a legal issue has been resolved like this, if any of the participants reopen it, they incur an offense entailing confession for the reopening. If anyone who gave their consent criticizes the resolution, they incur an offense entailing confession.” 

\subsection*{Resolution by further penalty }

“Is\marginnote{14.29.1} it possible that a legal issue arising from an accusation should be settled not by resolution through recollection or by resolution because of past insanity, but by resolution face-to-face and by further penalty? It is. How? It may be that a monk accuses a monk of an offense in the midst of the Sangha: ‘Venerable, do you remember committing such-and-such a heavy offense entailing expulsion or bordering on expulsion?’ He replies, ‘I don’t.’ As he tries to free himself, the accusing monk presses him further: ‘Come on, venerable, try again to  remember whether you’ve committed such-and-such a heavy offense.’ He replies, ‘I don’t remember committing such an offense, but I do remember committing such-and-such a minor offense.’ As he tries to free himself, the accusing monk presses him further: ‘Come on, venerable, try harder to  remember whether you’ve committed such-and-such a heavy offense.’ He replies, ‘I’ve admitted that I’ve committed this minor offense without being asked. So when asked about a heavy offense, why wouldn’t I admit it?’ The accusing monk says, ‘But you didn’t admit that you had committed this minor offense without being asked. So when asked about a heavy offense, why would you admit it? Come on, venerable, try harder to  remember whether you’ve committed such-and-such a heavy offense.’ He then says, ‘I remember committing such-and-such a heavy offense entailing expulsion or bordering on expulsion. When I said that I didn’t remember, I spoke playfully, I spoke too fast.’ They should do a legal procedure of further penalty against that monk. And it should be done like this. 

A\marginnote{14.29.31} competent and capable monk should inform the Sangha: 

‘Please,\marginnote{14.29.32} venerables, I ask the Sangha to listen. The monk so-and-so, while being examined in the midst of the Sangha about a heavy offense, asserts things after denying them, denies things after asserting them, evades the issue, and lies. If the Sangha is ready, it should do a legal procedure of further penalty against monk so-and-so. This is the motion. 

Please,\marginnote{14.29.36} venerables, I ask the Sangha to listen. The monk so-and-so, while being examined in the midst of the Sangha about a heavy offense, asserts things after denying them, denies things after asserting them, evades the issue, and lies. The Sangha does a legal procedure of further penalty against monk so-and-so. Any monk who approves of doing a legal procedure of further penalty against monk so-and-so should remain silent. Any monk who doesn’t approve should speak up. 

For\marginnote{14.29.41} the second time, I speak on this matter. … For the third time, I speak on this matter. … 

The\marginnote{14.29.43} Sangha has done a legal procedure of further penalty against monk so-and-so. The Sangha approves and is therefore silent. I’ll remember it thus.’ 

This\marginnote{14.29.45} is called the resolution of a legal issue. It’s been resolved face-to-face and by further penalty. Face-to-face with what? Face-to-face with the Sangha, the Teaching, the Monastic Law, and the persons concerned. This is the meaning of face-to-face with the Sangha: the monks who should be present have arrived, consent has been brought for those who are eligible to give their consent, and no-one present objects to the decision. This is the meaning of face-to-face with the Teaching and the Monastic Law: the Teaching, the Monastic Law, the Teacher’s instruction—that by which that legal issue is resolved. This is the meaning of face-to-face with the persons concerned: both those who are accusing and those who have been accused are present. This is the meaning of ‘by further penalty’: the doing of, the performing of, the participation in, the consent to, the agreement to, the non-objection to that legal procedure of further penalty. When a legal issue has been resolved like this, if any of the participants reopen it, they incur an offense entailing confession for the reopening. If anyone who gave their consent criticizes the resolution, they incur an offense entailing confession.” 

\subsection*{Acting according to what has been admitted }

“There\marginnote{14.30.1} are three principles for settling a legal issue arising from an offense: resolution face-to-face, acting according to what’s been admitted, and covering over as if with grass. Is it possible that a legal issue arising from an accusation should be settled not by covering over as if with grass, but by resolution face-to-face and by acting according to what’s been admitted? It is. How? It may be that a monk has committed a light offense. That monk should approach a single monk, arrange his upper robe over one shoulder, squat on his heels, raise his joined palms, and say: 

‘I’ve\marginnote{14.30.11} committed such-and-such an offense. I confess it.’ —‘Do you recognize that offense?’ —‘Yes, I recognize it.’ —‘You should restrain yourself in the future.’ 

This\marginnote{14.30.17} is called the resolution of a legal issue. It’s been resolved face-to-face and by acting according to what’s been admitted. Face-to-face with what? Face-to-face with the Teaching, the Monastic Law, and the persons concerned. This is the meaning of face-to-face with the Teaching and the Monastic Law: the Teaching, the Monastic Law, the Teacher’s instruction—that by which that legal issue is resolved. This is the meaning of face-to-face with the persons concerned: both the one who confesses and the one he confesses to are present. This is the meaning of acting according to what’s been admitted: the doing of, the performing of, the participation in, the consent to, the agreement to, the non-objection to that legal procedure of acting according to what’s been admitted. When a legal issue has been resolved like this, if the receiver of the confession reopens it, he incurs an offense entailing confession for the reopening. 

If\marginnote{14.31.1} this is what happens, all’s well. If not, that monk should approach several monks, arrange his upper robe over one shoulder, pay respect at the feet of the senior monks, squat on his heels, raise his joined palms, and say, ‘Venerables, I’ve committed such-and-such an offense. I confess it.’ A competent and capable monk should then inform those monks: 

‘Please,\marginnote{14.31.6} venerables, I ask you to listen. The monk so-and-so remembers an offense—he reveals it, makes it plain, and confesses it. If the venerables are ready, I’ll receive his confession.’ And he should say: 

‘Do\marginnote{14.31.10} you recognize that offense?’ —‘Yes, I recognize it.’ —‘You should restrain yourself in the future.’ 

This\marginnote{14.31.13} is called the resolution of a legal issue. It’s been resolved face-to-face and by acting according to what’s been admitted. Face-to-face with what? Face-to-face with the Teaching, the Monastic Law, and the persons concerned. This is the meaning of face-to-face with the Teaching and the Monastic Law: the Teaching, the Monastic Law, the Teacher’s instruction—that by which that legal issue is resolved. This is the meaning of face-to-face with the persons concerned: both the one who confesses and the one he confesses to are present. This is the meaning of acting according to what’s been admitted: the doing of, the performing of, the participation in, the consent to, the agreement to, the non-objection to that legal procedure of acting according to what’s been admitted. When a legal issue has been resolved like this, if the receiver of the confession reopens it, he incurs an offense entailing confession for the reopening. 

If\marginnote{14.32.1} this is what happens, all’s well. If not, that monk should approach the Sangha, arrange his upper robe over one shoulder, pay respect at the feet of the senior monks, squat on his heels, raise his joined palms, and say, ‘Venerables, I’ve committed such-and-such an offense. I confess it.’ A competent and capable monk should then inform the Sangha: 

‘Please,\marginnote{14.32.6} venerables, I ask the Sangha to listen. The monk so-and-so remembers an offense—he reveals it, makes it plain, and confesses it. If the Sangha is ready, I’ll receive his confession.’ And he should say: 

‘Do\marginnote{14.32.10} you recognize that offense?’ —‘Yes, I recognize it.’ —‘You should restrain yourself in the future.’ 

This\marginnote{14.32.13} is called the resolution of a legal issue. It’s been resolved face-to-face and by acting according to what’s been admitted. Face-to-face with what? Face-to-face with the Sangha, the Teaching, the Monastic Law, and the persons concerned. This is the meaning of face-to-face with the Sangha: the monks who should be present have arrived, consent has been brought for those who are eligible to give their consent, and no-one present objects to the decision. This is the meaning of face-to-face with the Teaching and the Monastic Law: the Teaching, the Monastic Law, the Teacher’s instruction—that by which that legal issue is resolved. This is the meaning of face-to-face with the persons concerned: both the one who confesses and the one he confesses to are present. This is the meaning of acting according to what’s been admitted: the doing of, the performing of, the participation in, the consent to, the agreement to, the non-objection to that legal procedure of acting according to what’s been admitted. When a legal issue has been resolved like this, if the receiver of the confession reopens it, he incurs an offense entailing confession for the reopening. If anyone who gave their consent criticizes the resolution, they incur an offense entailing confession.” 

\subsection*{Covering over as if with grass }

“Is\marginnote{14.33.1} it possible that a legal issue arising from an offense should be settled not by acting according to what’s been admitted, but by resolution face-to-face and by covering over as if with grass? It is. How? 

\scrule{It may be that monks who are arguing and disputing do and say many things unworthy of monastics. If they consider this and think, ‘If we deal with one another for these offenses, this legal issue might lead to harshness, nastiness, and schism,’ then I allow you to resolve this legal issue by covering over as if with grass. }

And\marginnote{14.33.11} it should be resolved like this. Everyone should gather in one place. A competent and capable monk should then inform the Sangha: 

‘Please,\marginnote{14.33.13} venerables, I ask the Sangha to listen. While we were arguing and disputing, we did and said many things unworthy of monastics. If we deal with one another for these offenses, this legal issue might lead to harshness, nastiness, and schism. If the Sangha is ready, it should resolve this legal issue by covering over as if with grass, except for heavy offenses and offenses connected with householders.’ 

The\marginnote{14.33.17} monks belonging to one side should be informed by a competent and capable monk belonging to their own side: 

‘Please,\marginnote{14.33.18} venerables, I ask you to listen. While we were arguing and disputing, we did and said many things unworthy of monastics. If we deal with one another for these offenses, this legal issue might lead to harshness, nastiness, and schism. If the venerables are ready, then for your benefit and for my own, I’ll confess in the midst of the Sangha both your and my own offenses by covering over as if with grass, except for heavy offenses and offenses connected with householders.’ 

The\marginnote{14.33.22} monks belonging to the other side should be informed by a competent and capable monk belonging to their own side: 

‘Please,\marginnote{14.33.23} venerables, I ask you to listen. While we were arguing and disputing, we did and said many things unworthy of monastics. If we deal with one another for these offenses, this legal issue might lead to harshness, nastiness, and schism. If the venerables are ready, then for your benefit and for my own, I’ll confess in the midst of the Sangha both your and my own offenses by covering over as if with grass, except for heavy offenses and offenses connected with householders.’ 

A\marginnote{14.33.27} competent and capable monk belonging to one side should inform the Sangha:\footnote{I here follow the PTS reading, \textit{\textsanskrit{ekatopakkhikānaṁ} \textsanskrit{bhikkhūnaṁ}}, as opposed to MS, \textit{\textsanskrit{athāparesaṁ} \textsanskrit{ekatopakkhikānaṁ} \textsanskrit{bhikkhūnaṁ}}. With the MS reading there is no distinction between the two groups of monks. The PTS reading is supported by the reading in OPM. } 

‘Please,\marginnote{14.33.28} venerables, I ask the Sangha to listen. While we were arguing and disputing, we did and said many things unworthy of monastics. If we deal with one another for these offenses, this legal issue might lead to harshness, nastiness, and schism. If the Sangha is ready, then for the benefit of these venerables and myself, I’ll confess in the midst of the Sangha both their and my own offenses by covering over as if with grass, except for heavy offenses and offenses connected with householders. This is the motion. 

‘Please,\marginnote{14.33.33} venerables, I ask the Sangha to listen. While we were arguing and disputing, we did and said many things unworthy of monastics. If we deal with one another for these offenses, this legal issue might lead to harshness, nastiness, and schism. For the benefit of these venerables and myself, I confess both their and my own offenses in the midst of the Sangha by covering over as if with grass, except for heavy offenses and offenses connected with householders. Any monk who approves of confessing our offenses in the midst of the Sangha by covering over as if with grass should remain silent. Any monk who doesn’t approve should speak up. 

We\marginnote{14.33.39} have confessed our offenses in the midst of the Sangha by covering over as if with grass, except for heavy offenses and offenses connected with householders. The Sangha approves and is therefore silent. I’ll remember it thus.’ 

A\marginnote{14.33.41.1} competent and capable monk belonging to the other side should inform the Sangha: 

‘Please,\marginnote{14.33.41.2} Venerables, I ask the Sangha to listen. While we were arguing and disputing, we did and said many things unworthy of monastics. If we deal with one another for these offenses, this legal issue might lead to harshness, nastiness, and schism. If it seems appropriate to the Sangha, then for the benefit of these venerables and myself, I’ll confess in the midst of the Sangha both their and my own offenses by covering over as if with grass, except for heavy offenses and offenses connected with householders. This is the motion. 

‘Please,\marginnote{14.33.41.3} Venerables, I ask the Sangha to listen. While we were arguing and disputing, we did and said many things unworthy of monastics. If we deal with one another for these offenses, this legal issue might lead to harshness, nastiness, and schism. For the benefit of these venerables and myself, I confess in the midst of the Sangha both their and my own offenses by covering over, as if with grass, except for heavy offenses and offenses connected with householders. Any monk who approves of confessing our offenses in the midst of the Sangha by covering over as if with grass should remain silent. Any monk who doesn’t approve should speak up. 

We\marginnote{14.33.41.4} have confessed our offenses in the midst of the Sangha by covering over as if with grass, except for heavy offenses and offenses connected with householders. The Sangha approves and is therefore silent.  I’ll remember it thus.’ 

This\marginnote{14.33.43} is called the resolution of a legal issue. It’s been resolved face-to-face and by covering over as if with grass. Face-to-face with what? Face-to-face with the Sangha, the Teaching, the Monastic Law, and the persons concerned. 

This\marginnote{14.33.48} is the meaning of face-to-face with the Sangha: the monks who should be present have arrived, consent has been brought for those who are eligible to give their consent, and no-one present objects to the decision. 

This\marginnote{14.33.51} is the meaning of face-to-face with the Teaching and the Monastic Law: the Teaching, the Monastic Law, the Teacher’s instruction—that by which that legal issue is resolved. 

This\marginnote{14.33.54} is the meaning of face-to-face with the persons concerned: both the one who confesses and the one he confesses to are present.\footnote{Here “the one he confesses to” should presumably be seen as the Sangha. } 

This\marginnote{14.33.57} is the meaning of covering over as if with grass: the doing of, the performing of, the participation in, the consent to, the agreement to, the non-objection to that legal procedure of covering over as if with grass. When a legal issue has been resolved like this, if a receiver of the confession reopens it, he incurs an offense entailing confession for the reopening.\footnote{Again, “a receiver” should presumably be understood as any member of the Sangha who is taking part in the legal procedure. } If anyone who gave their consent criticizes the resolution, they incur an offense entailing confession. 

There’s\marginnote{14.34.1} one way of settling a legal issue arising from business: by resolution face-to-face.” 

\scendsutta{The fourth chapter on the settling of legal issues is finished. }

%
\chapter*{{\suttatitleacronym Kd 15}{\suttatitletranslation The chapter on minor topics }{\suttatitleroot Khuddakavatthukkhandhaka}}
\addcontentsline{toc}{chapter}{\tocacronym{Kd 15} \toctranslation{The chapter on minor topics } \tocroot{Khuddakavatthukkhandhaka}}
\markboth{The chapter on minor topics }{Khuddakavatthukkhandhaka}
\extramarks{Kd 15}{Kd 15}

\section*{Bathing }

At\marginnote{1.1.1} one time the Buddha was staying at \textsanskrit{Rājagaha} in the Bamboo Grove, the squirrel sanctuary. At that time the monks from the group of six rubbed their bodies—thighs, arms, chest, and back—against trees while bathing. People complained and criticized them, “How can the Sakyan monastics do this? They’re just like boxers and city slickers who beautify their bodies!”\footnote{“Boxers” renders \textit{\textsanskrit{mallamuṭṭhikā}} and “city slickers who beautify their bodies” is for \textit{\textsanskrit{gāmamoddavā}}. In relation to \textit{\textsanskrit{mallamuṭṭhikā}} Sp 4.243 simply says \textit{\textsanskrit{mallamuṭṭhikāti} \textsanskrit{muṭṭhikamallā}}, which is not very helpful. \textit{\textsanskrit{Muṭṭhikamallā}} is then explained at Sp-yoj 4.243: \textit{\textsanskrit{Muṭṭhikamallāti} \textsanskrit{muṭṭhikena} mathanti \textsanskrit{aññamaññaṁ} \textsanskrit{hiṁsantīti}}, “\textit{\textsanskrit{Muṭṭhikamallā}}: hitting with a fist, they hurt each other.” As regards \textit{\textsanskrit{gāmamoddavā}}, Sp 4.243 says: \textit{\textsanskrit{Gāmamuddavāti} \textsanskrit{chavirāgamaṇḍanānuyuttā} \textsanskrit{nāgarikamanussā}. \textsanskrit{Gāmamoddavātipi} \textsanskrit{pāṭho}; esevattho}, “\textit{\textsanskrit{Gāmamuddavā}} means townspeople who are devoted to beautifying their skin with color; \textit{\textsanskrit{gāmamoddavā}} is another reading, with the same meaning.” See OPM, pp. 13-20 and 27, for a detailed discussion of these two phrases. The discussion in OPM concludes that “the word formation of \textit{\textsanskrit{gāmaphoḍava}} and the first member of this compound remain obscure.” } The monks heard the complaints of those people and they told the Buddha. Soon afterwards the Buddha had the Sangha gathered and questioned the monks: 

“Is\marginnote{1.1.8} it true, monks, that the monks from the group of six are doing this?” 

“It’s\marginnote{1.1.9} true, sir.” 

The\marginnote{1.1.10} Buddha rebuked them, “It’s not suitable for those foolish men, it’s not proper, it’s not worthy of a monastic, it’s not allowable, it’s not to be done. How can they do this? This will affect people’s confidence …” After rebuking them … the Buddha gave a teaching and addressed the monks: 

\scrule{“You shouldn’t rub your body against a tree while bathing. If you do, you commit an offense of wrong conduct.” }

At\marginnote{1.2.1} that time the monks from the group of six rubbed their bodies—thighs, arms, chest, and back—against posts while bathing. People complained and criticized them, “How can the Sakyan monastics do this? They’re just like boxers and city slickers who beautify their bodies!” The monks heard the complaints of those people and they told the Buddha. … “It’s true, sir.” … After rebuking them … the Buddha gave a teaching and addressed the monks: 

\scrule{“You shouldn’t rub your body against a post while bathing. If you do, you commit an offense of wrong conduct.” }

At\marginnote{1.2.11} that time the monks from the group of six rubbed their bodies—thighs, arms, chest, and back—against walls while bathing. People complained and criticized them, “How can the Sakyan monastics do this? They’re just like boxers and city slickers who beautify their bodies!” … 

\scrule{“You shouldn’t rub your body against a wall while bathing. If you do, you commit an offense of wrong conduct.” }

At\marginnote{1.3.1} that time the monks from the group of six rubbed their bodies—thighs, arms, chest, and back—against a rubbing board while bathing.\footnote{Sp 4.243: \textit{\textsanskrit{Aṭṭāne} \textsanskrit{nhāyantīti} ettha \textsanskrit{aṭṭānaṁ} \textsanskrit{nāma} \textsanskrit{rukkhaṁ} \textsanskrit{phalakaṁ} viya \textsanskrit{tacchetvā} \textsanskrit{aṭṭhapadākārena} \textsanskrit{rājiyo} \textsanskrit{chinditvā} \textsanskrit{nhānatitthe} \textsanskrit{nikhaṇanti}, tattha \textsanskrit{cuṇṇāni} \textsanskrit{ākiritvā} \textsanskrit{manussā} \textsanskrit{kāyaṁ} \textsanskrit{ghaṁsanti}}, “\textit{\textsanskrit{Aṭṭāne} \textsanskrit{nahāyanti}}: here an \textit{\textsanskrit{aṭṭāna}} means: having carved a tree to become like a plank, having cut lines in a cross-wise pattern, having dug it in at a bathing ford, people sprinkle powder there and rub their bodies.” } People complained and criticized them, “How can the Sakyan monastics do this? They’re just like householders who indulge in worldly pleasures!” 

\scrule{“You shouldn’t rub your body against a rubbing board while bathing. If you do, you commit an offense of wrong conduct.” }

At\marginnote{1.3.11} that time the monks from the group of six bathed with a wooden rubbing-hand.\footnote{Sp 4.243: \textit{\textsanskrit{Gandhabbahatthakenāti} \textsanskrit{nhānatitthe} \textsanskrit{ṭhapitena} \textsanskrit{dārumayahatthena}, tena kira \textsanskrit{cuṇṇāni} \textsanskrit{gahetvā} \textsanskrit{manussā} \textsanskrit{sarīraṁ} \textsanskrit{ghaṁsanti}}, “\textit{Gandhabbahatthakena}: having taken powders with a hand made of wood, which is kept at a bathing ford, people rub their bodies.” } People complained and criticized them, “How can the Sakyan monastics do this? They’re just like householders who indulge in worldly pleasures!” The monks heard the complaints of those people and they told the Buddha. 

\scrule{“You shouldn’t bathe with a wooden rubbing-hand. If you do, you commit an offense of wrong conduct.” }

At\marginnote{1.3.18} that time the monks from the group of six bathed with a string of cinnabar beads.\footnote{Sp 4.243: \textit{\textsanskrit{Kuruvindakasuttiyāti} \textsanskrit{kuruvindakapāsāṇacuṇṇāni} \textsanskrit{lākhāya} \textsanskrit{bandhitvā} \textsanskrit{kataguḷikakalāpako} vuccati, \textsanskrit{taṁ} ubhosu antesu \textsanskrit{gahetvā} \textsanskrit{sarīraṁ} \textsanskrit{ghaṁsanti}}, “\textit{\textsanskrit{Kuruvindakasuttiyā}}: having bound powder from a cinnabar-stone with resin, it is said a string of beads is made; having grasped it at both ends, they rub their bodies.” } People complained and criticized them, “How can the Sakyan monastics do this? They’re just like householders who indulge in worldly pleasures!” 

\scrule{“You shouldn’t bathe with a string of cinnabar beads. If you do, you commit an offense of wrong conduct.” }

At\marginnote{1.4.1} that time the monks from the group of six massaged one another.\footnote{\textit{Viggayha \textsanskrit{parikammaṁ} \textsanskrit{kārāpenti}}, literally, “Having stretched out, they caused a massage to be done.” Sp 4.243: \textit{Viggayha \textsanskrit{parikammaṁ} \textsanskrit{kārāpentīti} \textsanskrit{aññamaññaṁ} \textsanskrit{sarīrena} \textsanskrit{sarīraṁ} \textsanskrit{ghaṁsanti}}, “\textit{Viggayha \textsanskrit{parikammaṁ} \textsanskrit{kārāpenti}}: they rubbed each other, body with body.” } People complained and criticized them, “How can the Sakyan monastics do this? They’re just like householders who indulge in worldly pleasures!” 

\scrule{“You shouldn’t massage one another. If you do, you commit an offense of wrong conduct.” }

At\marginnote{1.4.7} that time the monks from the group of six bathed with an ornamented scrubber.\footnote{Sp 4.243: \textit{\textsanskrit{Mallakaṁ} \textsanskrit{nāma} makaradantake \textsanskrit{chinditvā} \textsanskrit{mallakamūlasaṇṭhānena} \textsanskrit{kataṁ} mallakanti vuccati}, “\textit{Mallaka}: having cut a shark-teeth pattern at the position of the base of the \textit{mallaka}, it is called a \textit{mallaka}.” It is not clear what this refers to, apart from the fact that it is decorated. I follow the suggestion in PED. } People complained and criticized them, “How can the Sakyan monastics do this? They’re just like householders who indulge in worldly pleasures!” … They told the Buddha. 

\scrule{“You shouldn’t bathe with an ornamented scrubber. If you do, you commit an offense of wrong conduct.” }

Soon\marginnote{1.4.13} afterwards a certain monk had an itchy skin disease. He was not comfortable without a scrubber. 

\scrule{“I allow a plain scrubber for those who are sick.” }

At\marginnote{1.5.1} that time a monk who was weak from old age was unable to rub his own body while bathing. 

\scrule{“I allow gloves of cloth.”\footnote{Sp 4.244: \textit{\textsanskrit{Ukkāsikanti} \textsanskrit{vatthavaṭṭiṁ}}, “\textit{\textsanskrit{Ukkāsika}} means a sheath of cloth.” } }

Being\marginnote{1.5.4} afraid of wrongdoing, the monks did not give back massages. 

\scrule{“I allow a massage with the flat of the hand.” }

\section*{Personal beautification }

At\marginnote{2.1.1} that time the monks from the group of six wore earrings,\footnote{Sp 4.245: \textit{\textsanskrit{Vallikāti} \textsanskrit{kaṇṇato} \textsanskrit{nikkhantamuttolambakādīnaṁ} \textsanskrit{etaṁ} \textsanskrit{adhivacanaṁ}}, “\textit{\textsanskrit{Vallikā}}: this is a term for a pearl, etc., hanging and protruding from the ear.” } ornamental hanging strings,\footnote{Sp 4.245: \textit{\textsanskrit{pāmaṅganti} \textsanskrit{yaṅkiñci} \textsanskrit{palambakasuttaṁ}}, “\textit{\textsanskrit{Pāmaṅga}}: whatever is a hanging thread.” } necklaces, ornamental girdles, bangles,\footnote{Sp 4.245: \textit{\textsanskrit{Ovaṭṭikanti} \textsanskrit{valayaṁ}}, “An \textit{\textsanskrit{ovaṭṭika}} is a bangle.” This is further explained at Vmv 4.245: \textit{Valayanti \textsanskrit{hatthapādavalayaṁ}}, “A bangle: a bangle for the hands or the feet.” } armlets,\footnote{Sp 4.245: \textit{\textsanskrit{Kāyūrādīni} \textsanskrit{pākaṭāneva}, \textsanskrit{akkhakānaṁ} \textsanskrit{heṭṭhā} \textsanskrit{bāhābharaṇaṁ} \textsanskrit{yaṅkiñci} \textsanskrit{ābharaṇaṁ} na \textsanskrit{vaṭṭati}}, “Just ordinary \textit{\textsanskrit{kāyūra}}s, etc.; any arm-ornaments below the collar-bone, are not allowable.” } bracelets,\footnote{\textit{\textsanskrit{Hatthābharaṇa}}, literally, “a hand ornament”, is not further explained in the commentaries. } and rings. People complained and criticized them … “… They’re just like householders who indulge in worldly pleasures!” The monks heard the complaints of those people. They told the Buddha. 

“Is\marginnote{2.1.13} it true, monks, that the monks from the group of six are wearing these things?” “It’s true, sir.” … After rebuking them … the Buddha gave a teaching and addressed the monks: 

\scrule{“You shouldn’t wear earrings, an ornamental hanging string, a necklace, an ornamental girdle, a bangle, an armlet, a bracelet, or a ring. If you do, you commit an offense of wrong conduct.” }

At\marginnote{2.2.1} that time the monks from the group of six grew their hair long. People complained and criticized them, “They’re just like householders who indulge in worldly pleasures!” 

\scrule{“You shouldn’t grow your hair long. If you do, you commit an offense of wrong conduct. I allow you to grow it to a length of 3.5 centimeters or for two months at the most.”\footnote{That is, two fingerbreadths. For a discussion of the \textit{\textsanskrit{aṅgula}}, see \textit{sugata} in Appendix of Technical Terms. } }

At\marginnote{2.3.1} that time the monks from the group of six brushed their hair, and they combed it, combed it with their hands, smoothed it with beeswax, and smoothed it with water and oil. People complained and criticized them, “They’re just like householders who indulge in worldly pleasures!” 

\scrule{“You shouldn’t brush your hair, or comb it, comb it with your hands, smooth it with beeswax, or smooth it with water and oil. If you do, you commit an offense of wrong conduct.” }

At\marginnote{2.4.1} that time the monks from the group of six looked at their faces in mirrors and in bowls of water. People complained and criticized them, “They’re just like householders who indulge in worldly pleasures!” 

\scrule{“You shouldn’t look at your face in a mirror or in a bowl of water. If you do, you commit an offense of wrong conduct.” }

Soon\marginnote{2.4.7} afterwards a certain monk had a sore on his face. He asked the monks, “What kind of sore is it?” They replied, “It’s this kind of sore.” He did not trust them. 

\scrule{“I allow you to look at your face in a mirror or in a bowl of water if you have a disease.” }

At\marginnote{2.5.1} that time the monks from the group of six used facial ointments, applied facial creams, powdered their face, applied rouge to their face, wore cosmetics on the body, wore cosmetics on the face, and wore cosmetics on the body and face.\footnote{“Applied facial creams” renders \textit{\textsanskrit{mukhaṁ} ummaddenti}. The verb \textit{ummaddeti} normally just means “rubs” or “massages”, but here the contexts required the application of some kind of cosmetic or cream. Sp 3.247: \textit{\textsanskrit{Ummaddentīti} \textsanskrit{nānāummaddanehi} ummaddenti}, “\textit{Ummaddenti}: they rub with various creams.” } People complained and criticized them, “They’re just like householders who indulge in worldly pleasures!” 

\scrule{“You shouldn’t use facial ointments, apply facial creams, powder your face, apply rouge to your face, wear cosmetics on the body, wear cosmetics on the face, or wear cosmetics on the body and face. If you do, you commit an offense of wrong conduct.” }

Soon\marginnote{2.5.9} afterwards a certain monk had an eye disease. 

\scrule{“I allow facial ointments for those who are sick.” }

\section*{Entertainment, etc. }

On\marginnote{2.6.1} one occasion in \textsanskrit{Rājagaha} there was a hilltop fair, and the monks from the group of six went to see it. People complained and criticized them, “How can the Sakyan monastics go and see dancing, singing, and music? They’re just like householders who indulge in worldly pleasures!” They told the Buddha. 

\scrule{“You shouldn’t go and see dancing, singing, and music. If you do, you commit an offense of wrong conduct.” }

At\marginnote{3.1.1} that time the monks from the group of six were singing the Teaching with a drawn-out voice. People complained and criticized them, “These Sakyan monastics sing with a drawn-out voice just like we do.” The monks heard the complaints of those people, and the monks of few desires complained and criticized them, “How can the monks from the group of six sing like this?” They told the Buddha. 

“Is\marginnote{3.1.8} it true, monks, that the monks from the group of six are singing like this?” “It’s true, sir.” … the Buddha gave a teaching and addressed the monks: 

“There\marginnote{3.1.11} are these five drawbacks to singing the Teaching with a drawn-out voice:\footnote{This is parallel to \href{https://suttacentral.net/an5.209/en/brahmali\#0.3}{AN 5.209:0.3}. } one delights in the sound; others delight in the sound; householders criticize it; for one who takes pleasure in performing with the voice, the concentration is disrupted; later generations follow one’s example.\footnote{“Performing with the voice” renders \textit{sarakutti}. According to DOP, \textit{kutti} means “arranging, dressing, preparation; action; forming; contriving”. Sp 4.249: \textit{Sarakuttinti \textsanskrit{sarakiriyaṁ}}, “\textit{\textsanskrit{Sarakuttiṁ}} means activity/performance with the voice”. } 

\scrule{You shouldn’t sing the Teaching with a drawn-out voice. If you do, you commit an offense of wrong conduct.” }

Being\marginnote{3.2.1} afraid of wrongdoing, the monks did not chant.\footnote{“Chant” renders \textit{\textsanskrit{sarabhañña}}. PED says: “Intoning, a particular mode of reciting.” Sp 4.249: \textit{\textsanskrit{Sarabhaññanti} sarena \textsanskrit{bhaṇanaṁ}}, “\textit{\textsanskrit{Sarabhañña}} means reciting with intonation.” \textit{Sara} can mean either “voice” or “intonation”. But since “voice” would be redundant in the present context (“reciting with a voice”), intonation seems to be the likely meaning. } They told the Buddha. 

\scrule{“I allow chanting.” }

At\marginnote{4.1.1} that time the monks from the group of six wore fleecy woolen robes with fleece on the outside. People complained and criticized them, “They’re just like householders who indulge in worldly pleasures!” 

\scrule{“You shouldn’t wear a fleecy woolen robe with fleece on the outside. If you do, you commit an offense of wrong conduct.” }

\section*{Fruit }

On\marginnote{5.1.1} one occasion when the mango trees in King \textsanskrit{Bimbisāra}’s park were bearing fruit, the king allowed the monks to eat as many mangoes as they wished. The monks from the group of six plucked and ate them all, even the unripe ones.\footnote{Vmv 4.250: \textit{\textsanskrit{Pāḷiyaṁ} \textsanskrit{taruṇaññeva} ambanti \textsanskrit{taruṇaṁ} \textsanskrit{asañjātabījaṁ} eva \textsanskrit{ambaphalaṁ}}, “In the Canonical text, \textit{\textsanskrit{taruṇaññeva} \textsanskrit{ambanṁ}} means: a mango fruit even with undeveloped seed.” } Just then the king needed mangoes. He told his people, “Go to the park and bring back mangoes.” Saying, “Yes, sir,” they went to the park and said to the park keeper, “The king needs mangoes. Please get some.” 

“There\marginnote{5.1.10} aren’t any. The monks plucked and ate them all, including the unripe ones.” 

They\marginnote{5.1.12} reported it to the king. He said, “It’s good that the venerables have eaten the mangoes. Still, the Buddha has praised moderation.” 

People\marginnote{5.1.14} complained and criticized them, “How can the Sakyan monastics eat the king’s mangoes without moderation?” The monks heard the complaints of those people and then told the Buddha. 

\scrule{“You shouldn’t eat mangoes. If you do, you commit an offense of wrong conduct.” }

Soon\marginnote{5.2.1} afterwards a certain association was offering a meal to the Sangha. They had prepared mango curry. Being afraid of wrongdoing, the monks did not accept it. 

\scrule{“Accept, monks, and eat it. I allow pieces of mango.” }

Soon\marginnote{5.2.6} afterwards a certain association was offering a meal to the Sangha. They were unable to prepare mango pieces and so gave whole mangoes in the dining hall. Being afraid of wrongdoing, the monks did not accept them. 

\scrule{“Accept, monks, and eat it. I allow you to eat fruit that’s allowable for monastics for any of five reasons: it’s been damaged by fire, a knife, or a fingernail, or it’s seedless, or the seeds have been removed.” }

\section*{Protection }

On\marginnote{6.1.1} one occasion a monk had been bitten by a snake and died. They told the Buddha. 

“That\marginnote{6.1.3} monk hadn’t spread good will to the four royal snake clans.\footnote{The following is parallel to \href{https://suttacentral.net/an4.67/en/brahmali\#1.1}{AN 4.67:1.1}. } Had he done so, he wouldn’t have died. What are the four clans? The \textsanskrit{Virūpakkhas}, the \textsanskrit{Erāpathas}, the \textsanskrit{Chabyāputtas}, and the \textsanskrit{Kaṇhāgotamas}. To protect yourselves, monks, you should spread good will to these four royal snake clans.\footnote{We see here that it is the spreading of good will that matters, not the chant as such. This accords with the \textsanskrit{Mettānisaṁsa} Sutta at \href{https://suttacentral.net/an11.15/en/brahmali}{AN 11.15}, which states that one is protected from various things only if one develops love to a high level. For the rendering “should” for \textit{\textsanskrit{anujānāmi}}, see Appendix of Technical Terms. } And it should be done like this: 

\begin{verse}%
I\marginnote{6.1.11} have good will toward the \textsanskrit{Virūpakkhas}, \\
Toward the \textsanskrit{Erāpathas} I have good will; \\
I have good will toward the \textsanskrit{Chabyāputtas}, \\
And toward the \textsanskrit{Kaṇhāgotamas}. 

I\marginnote{6.1.15} have good will toward the legless, \\
Toward the two-legged I have good will; \\
I have good will toward the four-legged, \\
And toward the many-legged. 

May\marginnote{6.1.19} the legless not hurt me, \\
May the two-legged not hurt me. \\
May the four-legged not hurt me, \\
May the many-legged not hurt me. 

All\marginnote{6.1.23} beings, all creatures, \\
All living beings everywhere, \\
May they all have good fortune, \\
May none meet with anything bad. 

The\marginnote{6.1.27} Buddha is unlimited, \\
The Teaching is unlimited, \\
The Sangha is unlimited. \\
But creeping animals are limited: 

Snakes,\marginnote{6.1.31} scorpions, centipedes, \\
Spiders, lizards, and mice. \\
I’m now protected and guarded; \\
May the creatures turn back. \\
I pay homage to the Buddha, \\
To the seven fully awakened Buddhas.” 

%
\end{verse}

On\marginnote{7.1.1} one occasion a monk who was plagued by lust cut off his own penis. They told the Buddha. He said, “This fool has cut off one thing, when he should’ve cut off something else. 

\scrule{You shouldn’t cut off your own penis. If you do, you commit a serious offense.” }

\section*{Bowls }

On\marginnote{8.1.1} one occasion a wealthy merchant of \textsanskrit{Rājagaha} had obtained a valuable block of sandalwood. He thought, “Why don’t I have a bowl carved from this block of sandal? I’d use the chips myself, but give the bowl away as a gift.” He then had a bowl carved, put it in a carrying net, hung it at the end of a succession of vertical bamboo poles, and announced,\footnote{“Carrying net” renders \textit{\textsanskrit{sikkā}}. Vin-vn-\textsanskrit{ṭ} 91: \textit{\textsanskrit{Sikkāyāti} \textsanskrit{olambikādhāre}}, “\textit{\textsanskrit{Sikkā}} means a holder for hanging.” CPD, under \textit{\textsanskrit{uḍḍeti}}, says, “To put in a sling or carrying net”, evidently referring to the \textit{\textsanskrit{sikkā}}. } “I’ll give this bowl to any perfected monastic or brahmin who brings it down by supernormal power.” 

\textsanskrit{Pūraṇa}\marginnote{8.1.7} Kassapa went to that merchant and said, “I’m perfected and have supernormal powers. Give me the bowl.” 

“If\marginnote{8.1.9} you’re perfected and have supernormal powers, then bring it down and it’s yours.” 

The\marginnote{8.1.10} same thing happened with Makkhali \textsanskrit{Gosāla}, Ajita Kesakambala, Pakudha \textsanskrit{Kaccāyana}, \textsanskrit{Sañcaya} \textsanskrit{Belaṭṭhaputta}, and the Jain ascetic from \textsanskrit{Ñātika}. 

Soon\marginnote{8.1.17} afterwards, after robing up in the morning, Venerable \textsanskrit{Mahāmoggallāna} and Venerable \textsanskrit{Piṇḍola} \textsanskrit{Bhāradvāja} took their bowls and robes and entered \textsanskrit{Rājagaha} for alms. \textsanskrit{Piṇḍola} \textsanskrit{Bhāradvāja} said to \textsanskrit{Mahāmoggallāna}, “Venerable, you’re perfected and have supernormal powers. If you go and bring down that bowl, it’s yours.” 

“Venerable,\marginnote{8.1.22} you too are perfected and have supernormal powers. If you bring it down, it’s yours.” 

\textsanskrit{Piṇḍola}\marginnote{8.1.25} \textsanskrit{Bhāradvāja} then rose into the air, took hold of that bowl, and circled around \textsanskrit{Rājagaha} three times. 

Just\marginnote{8.1.26} then that merchant, together with his wives and children, was standing in his own house, raising his joined palms in homage, thinking, “May Venerable \textsanskrit{Piṇḍola} \textsanskrit{Bhāradvāja} land right here at our house.” And \textsanskrit{Piṇḍola} \textsanskrit{Bhāradvāja} did just that. The merchant then took the bowl from his hands, filled it with expensive fresh foods, and gave it back to \textsanskrit{Piṇḍola} \textsanskrit{Bhāradvāja},\footnote{For the rendering “fresh food” for \textit{\textsanskrit{khādanīya}}, see Appendix of Technical Terms. } who then left for the monastery. 

People\marginnote{8.2.1} heard that \textsanskrit{Piṇḍola} \textsanskrit{Bhāradvāja} had taken down the merchant’s bowl, and making a great uproar, they followed right behind him. Hearing all the noise, the Buddha asked Venerable Ānanda what it was, and Ānanda told him what had happened. 

Soon\marginnote{8.2.12} afterwards the Buddha had the Sangha of monks gathered and questioned \textsanskrit{Piṇḍola} \textsanskrit{Bhāradvāja}: “Is it true, \textsanskrit{Bhāradvāja}, that you brought down that merchant’s bowl?” 

“It’s\marginnote{8.2.14} true, sir.” 

The\marginnote{8.2.15} Buddha rebuked him, “It’s not suitable, \textsanskrit{Bhāradvāja}, it’s not proper, it’s not worthy of a monastic, it’s not allowable, it’s not to be done. How could you show a superhuman ability, a wonder of supernormal power, to householders for the sake of a miserable wooden bowl? It’s just like a woman showing her private parts for a miserable \textit{\textsanskrit{māsaka}} coin. This will affect people’s confidence …” After rebuking him … the Buddha gave a teaching and addressed the monks: 

\scrule{“You shouldn’t show a superhuman ability, a wonder of supernormal power, to householders. If you do, you commit an offense of wrong conduct. }

Now\marginnote{8.2.25} destroy that wooden bowl and turn it into splinters. Give these to the monks to use as scent in ointments. 

\scrule{And you shouldn’t use a wooden bowl. If you do, you commit an offense of wrong conduct.” }

At\marginnote{9.1.1} that time the monks from the group of six used luxurious bowls made with gold and silver.\footnote{It is not clear whether \textit{\textsanskrit{sovaṇṇamayaṁ} \textsanskrit{rūpiyamayaṁ}} means “made of gold or silver” or “made with gold or silver”. It seems unlikely, however, that a bowl made partly of gold/silver would be acceptable, since it would still be considered luxurious. Moreover, just below the text mentions bowls made of/with crystal and gems (\textit{\textsanskrit{maṇimayo} … \textsanskrit{veḷuriyamayo}}). It is hard to imagine that a bowl could be made of any of these in its entirety. It seems more likely that they were used as decorations. In either case “made with” seems preferable to “made of”. With smaller requisites, however, “made of” gold and silver may be the preferable rendering. The commentaries are silent. } People complained and criticized them, “They’re just like householders who indulge in worldly pleasures!” 

\scrule{“You shouldn’t use almsbowls made with gold, silver, gems, beryl,\footnote{“Beryl” renders \textit{\textsanskrit{veḷuriya}}. The commentary at Sp-\textsanskrit{ṭ} 1.281 says: \textit{\textsanskrit{Veḷuriyoti} \textsanskrit{vaṁsavaṇṇamaṇi}}, “The bamboo-colored gem is called \textit{\textsanskrit{veḷuriya}}.” According to PED \textit{\textsanskrit{veḷuriya}} is lapis lazuli, which cannot be correct because lapis lazuli is blue. } crystal, bronze, glass, tin, lead, or copper. If you do, you commit an offense of wrong conduct. I allow two kinds of almsbowls: iron bowls and ceramic bowls.” }

At\marginnote{9.2.1} that time the bottoms of the bowls became scratched. 

\scrule{“I allow a circular bowl rest.” }

Soon\marginnote{9.2.4} afterwards the monks from the group of six used luxurious bowl rests made with gold and silver. People complained and criticized them, “They’re just like householders who indulge in worldly pleasures!” 

\scrule{“You shouldn’t use luxurious bowl rests. If you do, you commit an offense of wrong conduct. I allow two kinds of bowl rests: bowl-rests made of tin and a bowl-rests made of lead.” }

There\marginnote{9.2.12} were thick bowl rests on which the bowls did not sit properly.\footnote{Sp-\textsanskrit{ṭ} 4.253: \textit{Na \textsanskrit{acchupiyantīti} na \textsanskrit{suphassitāni} honti}, “\textit{Na acchupiyanti}: not touching properly.” } 

\scrule{“I allow you to carve them out.” }

There\marginnote{9.2.15} were marks left from the carving. 

\scrule{“I allow you to cut a shark-teeth pattern.”\footnote{“A shark-teeth pattern” renders \textit{makaradantaka}, literally, “like the teeth of a \textit{makara}”. In later Buddhism the \textit{makara} is the name of a mythological marine animal, but what it refers to in this context is not clear. According to Sp-yoj 4.243 \textit{makara} is the name of a certain species of fish: \textit{Makaradantaketi \textsanskrit{makaranāmakassa} macchassa dantasadise dante}, “Teeth like the teeth of a fish called \textit{makara}.” Vin-vn-\textsanskrit{ṭ} 3048: \textit{Makaradantakanti \textsanskrit{girikūṭākāraṁ}}, “\textit{Makaradantaka} means making (a design) like the peak of a hill.” PED suggests “the tooth of a swordfish”, but apparently swordfish do not have teeth. Given that the \textit{makara} were fearsome creatures and that their teeth looked like the peak of a hill, presumably meaning that their teeth were pointed, “shark teeth” seems like a reasonable guess. } }

Soon\marginnote{9.2.18} the monks from the group of six used colorful bowl rests, decorated like walls.\footnote{\textit{Bhittikamma}, literally, “wall-work”. Sp 4.297: \textit{Bhittikammanti \textsanskrit{nānāvaṇṇehi} \textsanskrit{vibhittirājikaraṇaṁ}}, “\textit{Bhittikamma} means making separate lines by means of many colors.” } As they were walking about, they showed them off in the streets. People complained and criticized them, “They’re just like householders who indulge in worldly pleasures!” 

\scrule{“You shouldn’t use colorful bowl rests, decorated like a wall. If you do, you commit an offense of wrong conduct. I allow ordinary bowl rests.” }

At\marginnote{9.3.1} this time there were monks who put away their bowls while still wet. The bowls were stained. 

\scrule{“You shouldn’t put away your bowl while still wet. If you do, you commit an offense of wrong conduct. You should sun your bowl and then put it away.” }

Soon\marginnote{9.3.7} afterwards there were monks who sunned their bowls while still wet. The bowls became smelly. 

\scrule{“You shouldn’t sun your bowl while still wet. If you do, you commit an offense of wrong conduct. You should dry the bowl and then sun it before you put it away.” }

Monks\marginnote{9.3.13} left their bowls in the heat of the sun. The bowls became discolored. 

\scrule{“You shouldn’t leave your bowl in the heat of the sun. If you do, you commit an offense of wrong conduct. You should sun it in the heat for a short time and then put it away.” }

On\marginnote{9.4.1} one occasion a number of almsbowls had been put down outside without support. A whirlwind rolled the bowls around and as a result they broke.\footnote{Sp 4.254: \textit{\textsanskrit{Āvaṭṭitvāti} \textsanskrit{aññamaññaṁ} \textsanskrit{paharitvā}}, “\textit{\textsanskrit{Āvaṭṭitva}}: (the bowls) struck one another.” } 

\scrule{“I allow a bowl rack.”\footnote{Sp 4.254: \textit{\textsanskrit{Pattādhārakanti} ettha “\textsanskrit{dantavallivettādīhi} kate \textsanskrit{bhūmiādhārake} tayo, \textsanskrit{dāruādhārake} dve patte \textsanskrit{uparūpari} \textsanskrit{ṭhapetuṁ} \textsanskrit{vaṭṭatī}”ti \textsanskrit{kurundiyaṁ} \textsanskrit{vuttaṁ}. \textsanskrit{Mahāaṭṭhakathāyaṁ} pana \textsanskrit{vuttaṁ} – “\textsanskrit{bhūmiādhārake} \textsanskrit{tiṇṇaṁ} \textsanskrit{pattānaṁ} \textsanskrit{anokāso}, dve \textsanskrit{ṭhapetuṁ} \textsanskrit{vaṭṭati}”}, “\textit{\textsanskrit{Pattādhāraka}}: in the \textsanskrit{Kurundī} it is said: in a tiered rack made of ivory, creepers, cane, etc., three bowls may be placed on top of one another; in a rack made of wood, two bowls. But in the \textsanskrit{Mahāaṭṭhakathā} it is said: in a tiered rack, there is no occasion for three bowls, but placing two is allowed.” } }

On\marginnote{9.4.5} one occasion there were monks who had put their bowls on the edge of a bench. They fell down and broke. 

\scrule{“You shouldn’t put your almsbowl on the edge of a bench.\footnote{Sp 4.254: \textit{\textsanskrit{Miḍḍhanteti} \textsanskrit{ālindakamiḍḍhikādīnaṁ} ante}, “\textit{\textsanskrit{Miḍḍhante}} means on the edge of a porch bench, etc.” See CPD, sv. \textit{\textsanskrit{ālindakamiḍḍhika}}. } If you do, you commit an offense of wrong conduct.” }

On\marginnote{9.4.10} one occasion there were monks who had put their bowls on the edge of a ledge. They fell down and broke. 

\scrule{“You shouldn’t put your almsbowl on the edge of a ledge.\footnote{Sp 4.254: \textit{\textsanskrit{Paribhaṇḍanteti} \textsanskrit{bāhirapasse} \textsanskrit{katāya} \textsanskrit{tanukamiḍḍhikāya} ante}, “\textit{\textsanskrit{Paribhaṇḍante}} means on the edge of a small bench made on an external slope.” Vmv 4.254: \textit{\textsanskrit{Tanukamiḍḍhikāyāti} \textsanskrit{vedikāya}}, “On a small bench means on a ledge (or railing).” } If you do, you commit an offense of wrong conduct.” }

At\marginnote{9.4.15} that time the monks put their bowls upside down on the ground. The edges of the bowls were scratched. 

\scrule{“I allow a spread of grass.” }

The\marginnote{9.4.19} grass was eaten by termites. 

\scrule{“I allow a cloth.” }

The\marginnote{9.4.22} cloth was eaten by termites. 

\scrule{“I allow a platform for bowls.”\footnote{Sp 4.254: \textit{\textsanskrit{Pattamāḷakaṁ} \textsanskrit{iṭṭhakāhi} \textsanskrit{vā} \textsanskrit{dārūhi} \textsanskrit{vā} \textsanskrit{kātuṁ} \textsanskrit{vaṭṭati}}, “\textit{\textsanskrit{Pattamāḷaka}}: made of bricks or wood is allowed.” Sp-yoj 4.254: \textit{\textsanskrit{Pattamāḷakanti} pattassa \textsanskrit{ṭhapanatthāya} \textsanskrit{kataṁ} \textsanskrit{aṭṭaṁ}}, “\textit{\textsanskrit{Pattamāḷaka}}: a platform made for the purpose of placing the bowl.” } }

The\marginnote{9.4.25} bowls fell off the platform and broke.\footnote{I here follow the PTS reading, \textit{\textsanskrit{pattamāḷakā}}, rather than the MS reading of \textit{\textsanskrit{pattamāḷako}}. According to the PTS reading the bowl fell from the platform, whereas on the MS reading the platform itself fell. } 

\scrule{“I allow a storage container for almsbowls.”\footnote{Sp 4.254: \textit{\textsanskrit{Pattakuṇḍolikāti} \textsanskrit{mahāmukhakuṇḍasaṇṭhānā} \textsanskrit{bhaṇḍakukkhalikā} vuccati}, “A pot for goods in the appearance of a pot with a large mouth is called a \textit{\textsanskrit{pattakuṇḍolika}}.” } }

The\marginnote{9.4.28} bowls were scratched in the storage containers. 

\scrule{“I allow a bowl bag.” }

There\marginnote{9.4.31} were no shoulder straps. 

\scrule{“I allow a shoulder strap and a string for fastening.” }

At\marginnote{9.5.1} that time there were monks who hung their bowls from wall pegs.\footnote{\textit{Bhittikhilepi \textsanskrit{nāgadantakepi}}, literally, “wall pegs and elephant tusks”. These are different kinds of pegs and I have not tried to differentiate between them. } The bowls fell down and broke. 

\scrule{“You shouldn’t hang up your almsbowl. If you do, you commit an offense of wrong conduct.” }

At\marginnote{9.5.6} that time there were monks who placed their bowls on beds. Sitting down absentmindedly, they crushed the bowls, breaking them. 

\scrule{“You shouldn’t place your bowl on a bed. If you do, you commit an offense of wrong conduct.” }

At\marginnote{9.5.10} that time monks placed their bowls on benches. Sitting down absentmindedly, they crushed the bowls, breaking them. 

\scrule{“You shouldn’t place your bowl on a bench. If you do, you commit an offense of wrong conduct.” }

At\marginnote{9.5.14} that time there were monks who put their bowls in their laps. When they got up absentmindedly, the bowls fell down and broke. 

\scrule{“You shouldn’t put your bowl in your lap. If you do, you commit an offense of wrong conduct.” }

At\marginnote{9.5.19} one time there were monks who put their bowls on a sunshade. A whirlwind lifted the sunshade, and the bowls fell down and broke. 

\scrule{“You shouldn’t put your bowl on a sunshade. If you do, you commit an offense of wrong conduct.” }

At\marginnote{9.5.24} that time there were monks who opened doors with a bowl in their hand. The doors swung back and the bowls broke. 

\scrule{“You shouldn’t open a door with an almsbowl in your hand. If you do, you commit an offense of wrong conduct.” }

At\marginnote{10.1.1} that time there were monks who walked for alms with gourds as bowls. People complained and criticized them, “They’re just like the monastics of other religions.” 

\scrule{“You shouldn’t walk for alms with a gourd as a bowl.\footnote{Sp 4.255: \textit{\textsanskrit{Tumbakaṭāhanti} \textsanskrit{lābukaṭāhaṁ} vuccati}, “A gourd as a bowl is called a \textit{\textsanskrit{tumbakaṭāha}}.” } If you do, you commit an offense of wrong conduct.” }

At\marginnote{10.1.7} that time there were monks who walked for alms with waterpots as bowls. People complained and criticized them, “They’re just like the monastics of other religions.” 

\scrule{“You shouldn’t walk for alms with a waterpot as a bowl. If you do, you commit an offense of wrong conduct.” }

At\marginnote{10.2.1} that time a certain monk who only used discarded things used a skull as a bowl. A woman who saw this was terrified, screaming, “Oh my God, a demon!” People complained and criticized him, “How can the Sakyan monastics use skulls as bowls? They’re just like goblins.” 

\scrule{“You shouldn’t use a skull as a bowl. If you do, you commit an offense of wrong conduct. And you shouldn’t use only discarded things. If you do, you commit an offense of wrong conduct.” }

At\marginnote{10.3.1} that time there were monks who carried away chewed food remnants, bones, and dirty mouth-rinsing water in their bowls. People complained and criticized them, “These Sakyan monastics use the vessel they’re eating from as a trash can.” 

\scrule{“You shouldn’t carry away chewed food remnants, bones, and dirty mouth-rinsing water in your almsbowl. If you do, you commit an offense of wrong conduct. I allow trash cans.” }

\section*{Robe making }

At\marginnote{11.1.1} that time the monks tore cloth to pieces by hand and then sewed robes. The robes were ugly. They told the Buddha. 

\scrule{“I allow a knife and a felt sheath.” }

Soon\marginnote{11.1.5} afterwards the Sangha was offered a knife with a handle. 

\scrule{“I allow a knife with a handle.” }

At\marginnote{11.1.8} this time the monks from the group of six used luxurious knife handles made with gold and silver. People complained and criticized them, “They’re just like householders who indulge in worldly pleasures!” 

\scrule{“You shouldn’t use luxurious knife handles. If you do, you commit an offense of wrong conduct. I allow knife handles made of bone, ivory, horn, reed, bamboo, wood, resin, fruit, metal, and shell.”\footnote{“Made of fruit” renders \textit{phalamaya}. In its discussion of ointment boxes, \textsanskrit{Khuddasikkhā}-\textsanskrit{purāṇaṭīkā} 185: \textit{\textsanskrit{Āmalakakakkādīhi} \textsanskrit{katā} \textsanskrit{phalamayā}}, “Made of fruit means made from ground emblic myrobalan, etc.” } }

At\marginnote{11.2.1} that time the monks used chicken feathers and pieces of bamboo to sew robes. The robes were badly sewn. 

\scrule{“I allow needles.” }

The\marginnote{11.2.5} needles rusted. 

\scrule{“I allow a cylinder for the needles.” }

The\marginnote{11.2.7} needles still rusted. 

\scrule{“I allow you to fill them with yeast.”\footnote{Sp 4.256: \textit{\textsanskrit{Kiṇṇena} \textsanskrit{pūretunti} \textsanskrit{kiṇṇacuṇṇena} \textsanskrit{pūretuṁ}}, “To fill it with yeast means to fill it with yeast powder.’” } }

The\marginnote{11.2.9} needles still rusted. 

\scrule{“I allow you to fill them with flour.” }

The\marginnote{11.2.11} needles still rusted. 

\scrule{“I allow stone powder.”\footnote{Sp 4.256: \textit{Saritakanti \textsanskrit{pāsāṇacuṇṇaṁ} vuccati; tena \textsanskrit{pūretuṁ} \textsanskrit{anujānāmīti} attho}, “Stone powder is called \textit{saritaka}. The meaning is ‘I allow you to fill with that.’” } }

The\marginnote{11.2.13} needles still rusted. 

\scrule{“I allow you to mix it with beeswax.”\footnote{Sp 4.256: \textit{Madhusitthakena \textsanskrit{sāretunti} madhusitthakena \textsanskrit{makkhetuṁ}}, “\textit{Madhusitthakena \textsanskrit{sāretuṁ}} means to mix with beeswax.” } }

The\marginnote{11.2.15} stone powder broke apart.\footnote{Sp 4.256: \textit{\textsanskrit{Saritakaṁ} \textsanskrit{paribhijjatīti} \textsanskrit{taṁ} \textsanskrit{makkhitamadhusitthakaṁ} bhijjati}, “\textit{\textsanskrit{Saritakaṁ} paribhijjati}: that mixture with beeswax broke apart.” } 

\scrule{“I allow a case.”\footnote{Sp 4.256: \textit{\textsanskrit{Saritasipāṭikanti} \textsanskrit{madhusitthakapilotikaṁ}; \textsanskrit{satthakosakasipāṭiyā} pana \textsanskrit{saritasipāṭikāya} \textsanskrit{anulomāti} \textsanskrit{kurundiyaṁ} \textsanskrit{vuttaṁ}}, “\textit{\textsanskrit{Saritasipāṭika}}: a piece of cloth with beeswax; but in the \textsanskrit{Kurundī} it is said that a knife-sheath-\textit{\textsanskrit{sipāṭiyā}} is in conformity with a \textit{\textsanskrit{saritakasipāṭika}}.” At \href{https://suttacentral.net/pli-tv-kd15/en/brahmali\#27.3.7}{Kd 15:27.3.7} \textit{\textsanskrit{sipāṭika}} refers to a case (for a razor). I translate accordingly. } }

At\marginnote{11.3.1} that time the monks erected posts here and there, bound them together, and sewed robes. The corners of the robes were deformed. They told the Buddha. 

\scrule{“I allow a frame and a string. You should tie down the cloth to the frame as required, before sewing the robe.”\footnote{The word rendered as “frame” is \textit{kathina}. See discussion of \textit{kathina} in Appendix of Technical Terms. } }

They\marginnote{11.3.5} laid the frame on uneven ground. The frame broke. 

\scrule{“You shouldn’t lay the frame on uneven ground. If you do, you commit an offense of wrong conduct.” }

They\marginnote{11.3.9} laid the frame on the ground. The frame became dirty. 

\scrule{“I allow a spread of grass.” }

The\marginnote{11.3.12} edges of the frame deteriorated. 

\scrule{“I allow you to add an edge lengthwise and crosswise.” }

The\marginnote{11.3.14} frame was not the right size.\footnote{Sp 4.256: \textit{\textsanskrit{Kathinaṁ} \textsanskrit{nappahotīti} \textsanskrit{dīghassa} bhikkhuno \textsanskrit{pamāṇena} \textsanskrit{kataṁ} \textsanskrit{kathinaṁ}; tattha rassassa bhikkhuno \textsanskrit{cīvaraṁ} \textsanskrit{patthariyamānaṁ} nappahoti, antoyeva hoti; \textsanskrit{daṇḍake} na \textsanskrit{pāpuṇātīti} attho}, “The frame was not the right size: the frame had been made to fit the size of a tall monk and it was not the right size for spreading out the robe-cloth for a short monk—it fell within; the meaning is the sticks (of the frame) are the wrong size.” } 

\scrule{“I allow an inner frame, folding a straw mat to fit the frame, spacers, strings for tying together, and strings for tying down. After tying it together, you should sew the robe.”\footnote{“An inner frame” renders \textit{\textsanskrit{daṇḍakathina}}. Sp 4.256: \textit{\textsanskrit{Daṇḍakathinanti} tassa majjhe itarassa bhikkhuno \textsanskrit{pamāṇena} \textsanskrit{aññaṁ} \textsanskrit{nisseṇiṁ} \textsanskrit{bandhituṁ} \textsanskrit{anujānāmīti} attho}, “\textit{\textsanskrit{Daṇḍakathina}}: the meaning is ‘I allow you to bind another frame to fit the size of another monk in the middle of that (large frame).’” Sp 4.256: \textit{Bidalakanti \textsanskrit{daṇḍakathinappamāṇena} \textsanskrit{kaṭasārakassa} pariyante \textsanskrit{paṭisaṁharitvā} \textsanskrit{duguṇakaraṇaṁ}}, “\textit{Bidalaka}: making a double layer by folding the ends of a straw mat to fit the inner frame.” Sp 4.256: \textit{\textsanskrit{Salākanti} \textsanskrit{dupaṭṭacīvarassa} antare \textsanskrit{pavesanasalākaṁ}}, “\textit{\textsanskrit{Salāka}}: a \textit{\textsanskrit{salāka}} for inserting between a double-layer robe-cloth.” Sp 4.256: \textit{Vinandhanarajjunti \textsanskrit{mahānisseṇiyā} \textsanskrit{saddhiṁ} \textsanskrit{khuddakaṁ} \textsanskrit{nisseṇiṁ} \textsanskrit{vinandhituṁ} \textsanskrit{rajjuṁ}}, “\textit{Vinandhanarajju}: a string to tie the small frame to the large frame.” Sp 4.256: \textit{Vinandhanasuttanti \textsanskrit{khuddakanisseṇiyā} \textsanskrit{cīvaraṁ} \textsanskrit{vinandhituṁ} \textsanskrit{suttakaṁ}}, “\textit{Vinandhanasutta}: a string to tie the robe-cloth to the small frame.” } }

The\marginnote{11.3.16} seams were unevenly spaced. 

\scrule{“I allow a ruler.”\footnote{Sp 4.256: \textit{\textsanskrit{Kaḷimbhakanti} \textsanskrit{pamāṇasaññākaraṇaṁ} \textsanskrit{yaṅkiñci} \textsanskrit{tālapaṇṇādiṁ}}, “\textit{\textsanskrit{Kaḷimbhaka}}: whatever enables one to perceive the distance, (such as) a palmleaf, etc.” } }

The\marginnote{11.3.18} seams were not straight. 

\scrule{“I allow you to make a guide line.”\footnote{Sp 4.256: \textit{Moghasuttakanti \textsanskrit{vaḍḍhakīnaṁ} \textsanskrit{dārūsu} \textsanskrit{kāḷasuttena} viya haliddisuttena \textsanskrit{saññākaraṇaṁ}}, “\textit{Moghasuttaka}: by means of a thread with turmeric, like the black thread on wood used by carpenters; what enables one to perceive.” } }

There\marginnote{11.4.1} were monks who stepped on the frame with dirty feet. The frame became dirty. 

\scrule{“You shouldn’t step on the frame with dirty feet. If you do, you commit an offense of wrong conduct.” }

There\marginnote{11.4.6} were monks who stepped on the frame with wet feet. The frame became dirty. 

\scrule{“You shouldn’t step on the frame with wet feet. If you do, you commit an offense of wrong conduct.” }

There\marginnote{11.4.11} were monks who stepped on the frame while wearing sandals. The frame became dirty. 

\scrule{“You shouldn’t step on the frame while wearing sandals. If you do, you commit an offense of wrong conduct.” }

When\marginnote{11.5.1} sewing robes, the monks used their bare fingers. They pricked their fingers. 

\scrule{“I allow thimbles.” }

Soon\marginnote{11.5.5} the monks from the group of six used luxurious thimbles made with gold and silver. People complained and criticized them, “They’re just like householders who indulge in worldly pleasures!” 

\scrule{“You shouldn’t use luxurious thimbles. If you do, you commit an offense of wrong conduct. I allow a thimble made of bone, ivory, horn, reed, bamboo, wood, resin, fruit, metal, and shell.” }

Needles,\marginnote{11.5.13} knives, and thimbles got lost. 

\scrule{“I allow a small bowl.” }

The\marginnote{11.5.16} small bowls became crowded. 

\scrule{“I allow a bag for thimbles.” }

There\marginnote{11.5.19} was no shoulder strap. 

\scrule{“I allow a shoulder strap and a string for fastening.” }

When\marginnote{11.6.1} sewing robes outside, the monks were troubled by the cold and heat. They told the Buddha. 

\scrule{“I allow sewing sheds and roof covers.”\footnote{For an explanation of rendering \textit{\textsanskrit{sālā}} as “shed”, see Appendix of Technical Terms. } }

They\marginnote{11.6.4} built the sewing shed on a low base. It was flooded. 

\scrule{“I allow you to raise the base.” }

The\marginnote{11.6.7} mound collapsed. 

\scrule{“I allow you to construct three kinds of raised foundations: raised foundations of brick, stone, and wood.” }

It\marginnote{11.6.10} was difficult to get up to the sewing sheds. 

\scrule{“I allow three kinds of stairs: stairs of brick, stone, and wood.” }

People\marginnote{11.6.13} fell down while climbing the stairs. 

\scrule{“I allow rails.” }

Grass\marginnote{11.6.15} and dust fell into the sewing sheds. 

\scrule{“I allow you to firm up the structure and then plaster it inside and outside, including:\footnote{“To firm up the structure” renders \textit{\textsanskrit{ogumphetvā}}. Sp 4.257: \textit{\textsanskrit{Ogumphetvā} \textsanskrit{ullittāvalittaṁ} \textsanskrit{kātunti} \textsanskrit{chadanaṁ} \textsanskrit{odhunitvā} \textsanskrit{ghanadaṇḍakaṁ} \textsanskrit{katvā} anto ceva bahi ca \textsanskrit{mattikāya} limpitunti attho}, “\textit{\textsanskrit{Ogumphetvā} \textsanskrit{ullittāvalittaṁ} \textsanskrit{kātuṁ}}: having shook out the roof cover and added rods to firm up (the structure), to smear with clay inside and outside—this is the meaning.” At \href{https://suttacentral.net/pli-tv-kd5/en/brahmali\#11.1.5}{Kd 5:11.1.5} the same verb, in the form \textit{ogumphiyanti}, is used to show how dwellings are “held together” by straps of leather. This makes it certain that \textit{\textsanskrit{ogumphetvā}} in the present context must refer to the “firming up” rather than the “shaking out”. } treating it with white color, black color, and red ocher; making garland patterns, creeper patterns, shark-teeth patterns, and the fivefold pattern; putting up bamboo robe racks and clotheslines.”\footnote{“The fivefold pattern” renders \textit{\textsanskrit{pañcapaṭika}}. Vmv 4.299: \textit{\textsanskrit{Pāḷiyaṁ} \textsanskrit{pañcapaṭikanti} \textsanskrit{jātiādipañcappakāravaṇṇamaṭṭhaṁ}}, “\textit{\textsanskrit{Pañcapaṭika}} in the canonical text means the five-fold appearance, starting with jasmine.” The meaning is not clear. } }

At\marginnote{11.7.1} that time, when they had finished sewing a robe, there were monks who abandoned the frame right there and left. Rats and termites ate it. 

\scrule{“You should fold up the frame.” }

The\marginnote{11.7.4} frame broke. 

\scrule{“You should fold it up with a rod for support.” }

The\marginnote{11.7.6} frame unfolded. 

\scrule{“You should tie it up with a rope.” }

At\marginnote{11.7.8} that time there were monks who leaned the frame against a wall or a pillar and left. It fell down and broke. 

\scrule{“You should hang it from a wall peg.” }

\section*{Various allowable requisites }

When\marginnote{12.1.1} the Buddha had stayed at \textsanskrit{Rājagaha} for as long as he liked, he set out wandering toward \textsanskrit{Vesālī}. And so did the monks, having put their needles, knives, and medicines in their almsbowls. They told the Buddha. 

\scrule{“I allow a medicine bag.” }

There\marginnote{12.1.5} was no shoulder strap. 

\scrule{“I allow a shoulder strap and a string for fastening it.” }

On\marginnote{12.1.7} one occasion a certain monk bound his sandals to his belt and entered the village for alms. A lay follower who bowed to him hit his head against those sandals. The monk was embarrassed. When he had returned to the monastery, he told the monks what had happened, who in turn told the Buddha. 

\scrule{“I allow a bag for sandals.” }

There\marginnote{12.1.13} was no shoulder strap. 

\scrule{“I allow a shoulder strap and a string for fastening it.” }

While\marginnote{13.1.1} they were traveling, there was only unallowable water,\footnote{Vmv 4.258: \textit{\textsanskrit{Udakaṁ} akappiyanti \textsanskrit{sappāṇakaṁ}}, “Unallowable water: it contained living beings.” } but no water filter. 

\scrule{“I allow a water filter.” }

There\marginnote{13.1.5} was no suitable cloth. 

\scrule{“I allow a filter with a handle.” }

There\marginnote{13.1.7} was still no suitable cloth. 

\scrule{“I allow a water strainer.”\footnote{It is not clear how the “water strainer”, \textit{\textsanskrit{dhammakaraṇa}}, is different from the “water filter”, \textit{\textsanskrit{arissāvana}}. The commentaries are silent. } }

On\marginnote{13.2.1} one occasion, there were two monks traveling through the Kosalan country. One monk misbehaved, and the second monk said to him, “Don’t do that. It’s not allowable.” Because of that, the first monk became resentful. 

Soon\marginnote{13.2.7} afterwards the second monk was very thirsty. He asked the resentful monk to borrow his water filter. He refused and the second monk died from thirst. When the resentful monk arrived at the monastery, he told the monks what had happened. 

“So\marginnote{13.2.12} you refused to lend your water filter when asked?” 

“Yes.”\marginnote{13.2.13} 

The\marginnote{13.2.14} monks of few desires complained and criticized him, “How could a monk do such a thing?” They told the Buddha. Soon afterwards the Buddha had the monks gathered and questioned that monk: 

“Is\marginnote{13.2.18} it true, monk, that you did this?” 

“It’s\marginnote{13.2.19} true, sir.” 

The\marginnote{13.2.20} Buddha rebuked him, “It’s not suitable, foolish man, it’s not proper, it’s not worthy of a monastic, it’s not allowable, it’s not to be done. How could you refuse to lend your water filter when asked? This will affect people’s confidence …” After rebuking him … the Buddha gave a teaching and addressed the monks: 

\scrule{“If you’re traveling with a monk and he asks to borrow your water filter, you should lend it. If you don’t, you commit an offense of wrong conduct. On the other hand, you shouldn’t travel without a water filter. If you do, you commit an offense of wrong conduct. If there’s no water filter or strainer, you should determine a corner of your robe:\footnote{For an explanation of rendering \textit{\textsanskrit{saṅghāṭi}} as “robe”, see Appendix of Technical Terms. } ‘I’ll drink after filtering with this.’” }

The\marginnote{13.3.1} Buddha eventually arrived at \textsanskrit{Vesālī} where he stayed in the hall with the peaked roof in the Great Wood. At this time the monks were doing building work, but there were not enough water filters.\footnote{Sp-\textsanskrit{ṭ} 4.259: \textit{Na \textsanskrit{sammatīti} nappahoti}, “\textit{Na sammati}: not enough.” } They told the Buddha. 

\scrule{“I allow a water filter of cloth fitted to a wooden framework.”\footnote{Sp 4.259: \textit{\textsanskrit{Daṇḍaparissāvananti} \textsanskrit{rajakānaṁ} \textsanskrit{khāraparissāvanaṁ} viya \textsanskrit{catūsu} \textsanskrit{pādesu} \textsanskrit{baddhanisseṇikāya} \textsanskrit{sāṭakaṁ} \textsanskrit{bandhitvā} \textsanskrit{majjhedaṇḍake} \textsanskrit{udakaṁ} \textsanskrit{āsiñcitabbaṁ}, \textsanskrit{taṁ} ubhopi \textsanskrit{koṭṭhāse} \textsanskrit{pūretvā} parissavati}, “\textit{\textsanskrit{Daṇḍaparissāvana}}: like the filters for caustic substances used by dyers, having bound a cloth to a framework fastened on four legs, the water is filtered by pouring on the middle rod, having filled on both sides.” } }

There\marginnote{13.3.7} were still not enough filters. 

\scrule{“I allow you to filter by spreading cloth on water.”\footnote{Sp 4.259: \textit{\textsanskrit{Ottharakaṁ} \textsanskrit{nāma} \textsanskrit{yaṁ} udake \textsanskrit{ottharitvā} \textsanskrit{ghaṭakena} \textsanskrit{udakaṁ} \textsanskrit{gaṇhanti}, \textsanskrit{tañhi} \textsanskrit{catūsu} \textsanskrit{daṇḍakesu} \textsanskrit{vatthaṁ} \textsanskrit{bandhitvā} udake \textsanskrit{cattāro} \textsanskrit{khāṇuke} \textsanskrit{nikhaṇitvā} tesu \textsanskrit{bandhitvā} sabbapariyante udakato \textsanskrit{mocetvā} majjhe \textsanskrit{ottharitvā} \textsanskrit{ghaṭena} \textsanskrit{udakaṁ} \textsanskrit{gaṇhanti}}, “\textit{Ottharaka}: having spread it on water, they collect water with a jar. That is, having bound a cloth onto four rods, having implanted four stakes in the water, having bound them together, keeping it out of the water on all sides, having spread it out (in the water) in the middle, they collect water with a jar.” } }

At\marginnote{13.3.10} this time the monks were troubled by mosquitoes. 

\scrule{“I allow a mosquito tent.”\footnote{\textit{\textsanskrit{Makasakuṭika}}, literally, “a mosquito hut”. } }

\section*{Buildings }

\subsection*{Walking paths }

At\marginnote{14.1.1} this time in \textsanskrit{Vesālī} people had arranged a succession of fine meals. After eating the fine food, the monks were often sick, their bodies being full of impurities. 

Just\marginnote{14.1.3} then \textsanskrit{Jīvaka} \textsanskrit{Komārabhacca} was in \textsanskrit{Vesālī} on some business, and he saw those monks. He went to the Buddha, bowed, sat down, and said, “At present, sir, there are monks who are often sick, their bodies being full of impurities. Please allow walking-meditation paths and saunas. In this way the monks will rarely get sick.” The Buddha then instructed, inspired, and gladdened him with a teaching, after which \textsanskrit{Jīvaka} got up from his seat, bowed, circumambulated the Buddha with his right side toward him, and left. Soon afterwards the Buddha gave a teaching and addressed the monks: 

\scrule{“I allow walking-meditation paths and saunas.”\footnote{For an explanation of rendering \textit{\textsanskrit{jantāghara}} as “sauna”, see Appendix of Technical Terms. } }

Monks\marginnote{14.2.1} did walking meditation on uneven walking paths. As a result their feet hurt. 

\scrule{“I allow you to even them out.” }

They\marginnote{14.2.5} built walking-meditation paths on a low base. They were flooded. 

\scrule{“I allow you to raise the base.” }

The\marginnote{14.2.8} mound collapsed. 

\scrule{“I allow you to construct three kinds of raised foundations: raised foundations of brick, stone, and wood.” }

It\marginnote{14.2.11} was difficult to get up on the walking-meditation paths. 

\scrule{“I allow three kinds of stairs: stairs of brick, stone, and wood.” }

People\marginnote{14.2.14} fell down while climbing the stairs. 

\scrule{“I allow rails.” }

Monks\marginnote{14.2.16} fell off while doing walking meditation. 

\scrule{“I allow railings.” }

Doing\marginnote{14.2.19} walking meditation outside, monks were troubled by the cold and the heat. They told the Buddha. 

\scrule{“I allow indoor walking-meditation paths.”\footnote{For a discussion of \textit{\textsanskrit{sālā}}, see Appendix of Technical Terms. } }

Grass\marginnote{14.2.22} and dust fell into the indoor walking-meditation paths. 

\scrule{“I allow you to firm up the structure and then to plaster it inside and outside, including: treating with white color, black color, and red ocher; making garland patterns, creeper patterns, shark-teeth patterns, and the fivefold pattern; putting up bamboo robe racks and clotheslines.” }

\subsection*{Saunas }

They\marginnote{14.3.1} built saunas on a low base. They were flooded. 

\scrule{“I allow you to raise the base.” }

The\marginnote{14.3.3} mound collapsed. 

\scrule{“I allow you to construct three kinds of raised foundations: raised foundations of brick, stone, and wood.” }

It\marginnote{14.3.6} was difficult to get up to the saunas. 

\scrule{“I allow three kinds of stairs: stairs of brick, stone, and wood.” }

People\marginnote{14.3.9} fell down while climbing the stairs. 

\scrule{“I allow rails.” }

The\marginnote{14.3.11} saunas didn’t have doors. 

\scrule{“I allow doors, door frames, lower hinges, upper hinges, door jambs, bolt sockets, bolts, latches, keyholes, door-pulling holes, and door-pulling ropes.”\footnote{“A lower hinge” renders \textit{udukkhalika}, while “an upper hinge” is for \textit{\textsanskrit{uttarapāsaka}}. Sp 1.77: … \textit{yena kenaci \textsanskrit{kavāṭaṁ} \textsanskrit{katvā} \textsanskrit{heṭṭhā} udukkhale upari \textsanskrit{uttarapāsake} ca \textsanskrit{pavesetvā} \textsanskrit{kataṁ} \textsanskrit{parivattakadvārameva} \textsanskrit{saṁvaritabbaṁ} … atha \textsanskrit{dvārassa} \textsanskrit{udukkhalaṁ} \textsanskrit{vā} \textsanskrit{uttarapāsako} \textsanskrit{vā} bhinno \textsanskrit{vā} hoti \textsanskrit{aṭṭhapito} \textsanskrit{vā}, \textsanskrit{saṁvarituṁ} na sakkoti}, “Having made a door by whatever (material), having entered it into the \textit{udukkhala} below and into the \textit{\textsanskrit{uttarapāsaka}} above, it is made a revolving door to be closed … but when the \textit{udukkhala} or the \textit{\textsanskrit{uttarapāsaka}} is broken or not mounted, then one cannot close the door.” From this it seems that the \textit{udukkhala} and \textit{\textsanskrit{uttarapāsaka}}, together with the two corresponding “projecting pivots” on the door, are the functional equivalents of hinges. \textit{Udukkhalika} and \textit{udukkhala} refer to the same thing, the former being used in the canonical text, whereas the latter is found in the summary verses. For further details see CPD, sv. \textit{\textsanskrit{uttarapāsaka}}. “Door jambs” renders \textit{\textsanskrit{aggaḷavaṭṭi}}. Sp 4.260: \textit{\textsanskrit{Aggaḷavaṭṭi} \textsanskrit{nāma} \textsanskrit{dvārabāhāya} \textsanskrit{samappamāṇoyeva} \textsanskrit{aggaḷatthambho} vuccati, yattha \textsanskrit{tīṇi} \textsanskrit{cattāri} \textsanskrit{chiddāni} \textsanskrit{katvā} \textsanskrit{sūciyo} denti}, “\textit{\textsanskrit{Aggaḷavaṭṭi}}: it is called a door post, which is the same length as the door frame. It is where three or four holes are made for inserting bolts.” Whenever the Canonical text lists the parts of a door and door frame, the \textit{\textsanskrit{aggaḷavaṭṭi}} always has the same position, being grouped together with the parts for the locking mechanism, such as latches and bolts. Given the commentarial explanation, it is natural to think that it was a special post added to the door frame, or perhaps replacing the door frame, for the purpose of receiving bolts. “Bolt socket” renders \textit{\textsanskrit{kapisīsaka}}. Sp 4.260: \textit{\textsanskrit{Kapisīsakaṁ} \textsanskrit{nāma} \textsanskrit{dvārabāhaṁ} \textsanskrit{vijjhitvā} tattha pavesito \textsanskrit{aggaḷapāsako} vuccati}, “The bolt-receiving socket which is inserted after piercing the door post is called a \textit{\textsanskrit{kapisīsaka}}.” “Bolt” renders \textit{\textsanskrit{sūcika}}. Sp 4.260: \textit{\textsanskrit{Sūcikāti} tattha majjhe \textsanskrit{chiddaṁ} \textsanskrit{katvā} \textsanskrit{pavesitā}}, “It is entered, having made a hole in the middle there.” Vmv 4.260: \textit{Tattha majjheti \textsanskrit{aggaḷapāsakassa} majjhe}, “In the middle there: in the middle of the door jamb.” “Latch” renders \textit{\textsanskrit{ghaṭika}}. Sp-\textsanskrit{ṭ} 4.255: \textit{\textsanskrit{Ghaṭikanti} upari \textsanskrit{yojitaṁ} \textsanskrit{aggaḷaṁ}}, “\textit{\textsanskrit{Ghaṭika}}: it connected the door at the top.” From the origin story of \href{https://suttacentral.net/pli-tv-bu-vb-ss2/en/brahmali\#1.1.11}{Bu Ss 2:1.1.11} it seems that the \textit{\textsanskrit{ghaṭika}} was a device that could be opened with a key: \textit{\textsanskrit{Avāpuraṇaṁ} \textsanskrit{ādāya} \textsanskrit{ghaṭikaṁ} \textsanskrit{ugghāṭetvā} \textsanskrit{kavāṭaṁ} \textsanskrit{paṇāmetvā} \textsanskrit{vihāraṁ} \textsanskrit{pāvisi}}, “(\textsanskrit{Udāyī}) took the key, lifted the latch, opened the door, and entered the dwelling.” It follows from this that the \textit{\textsanskrit{ghaṭika}} is unlikely to be a bolt, but probably a kind of bar, like a latch, that would require lifting for the door to open. The lifting would be done with \textit{\textsanskrit{tāḷa}}, a key-like device. “A door-pulling hole” renders \textit{\textsanskrit{āviñchanachidda}}. Vmv 4.296: \textit{\textsanskrit{Āviñchanachiddanti} yattha \textsanskrit{aṅguliṁ} \textsanskrit{vā} \textsanskrit{rajjusaṅkhalikādiṁ} \textsanskrit{vā} \textsanskrit{pavesetvā} \textsanskrit{kavāṭaṁ} \textsanskrit{ākaḍḍhantā} \textsanskrit{dvārabāhaṁ} \textsanskrit{phusāpenti}}, “\textit{\textsanskrit{Āviñchanachidda}}: where, having entered the finger or a rope or a chain, etc., they pull the door and make it touch the door post.” “A door-pulling rope” renders \textit{\textsanskrit{āviñchanarajju}}. Sp 4.296: \textit{\textsanskrit{Āviñchanarajjunti} \textsanskrit{kavāṭeyeva} \textsanskrit{chiddaṁ} \textsanskrit{katvā} tattha \textsanskrit{pavesetvā} yena rajjukena \textsanskrit{kaḍḍhantā} \textsanskrit{dvāraṁ} \textsanskrit{phusāpenti}}, “Having made a hole in the door, having entered (the rope) there, the rope with which they close and make the door touch (the post).” For a discussion of the \textit{\textsanskrit{aggaḷa}} as “door”, see Appendix of Technical Terms. } }

The\marginnote{14.3.13} base of the sauna walls deteriorated. 

\scrule{“I allow you to make encircling trenches.”\footnote{Sp 4.260: \textit{\textsanskrit{Maṇḍalikaṁ} \textsanskrit{kātunti} \textsanskrit{nīcavatthukaṁ} \textsanskrit{cinituṁ}}, “To make a \textit{\textsanskrit{maṇḍalika}} means to construct a low site.” Sp-\textsanskrit{ṭ} 4.260: \textit{\textsanskrit{Nīcavatthukaṁ} cinitunti \textsanskrit{bahikuṭṭassa} samantato \textsanskrit{nīcavatthukaṁ} \textsanskrit{katvā} \textsanskrit{cinituṁ}}, “\textit{\textsanskrit{Nīcavatthukaṁ} \textsanskrit{cinituṁ}} means to construct after making the ground low on all sides outside the wall.” } }

The\marginnote{14.3.15} saunas didn’t have flues. 

\scrule{“I allow flues.” }

At\marginnote{14.3.17} that time the monks built a fireplace in the middle of a small sauna. There was no access around the fireplace. 

\scrule{“In a small sauna, you should make the fireplace to one side, but in a large one in the middle.” }

The\marginnote{14.3.21} fire in the sauna scorched their faces. 

\scrule{“I allow clay for the face.” }

They\marginnote{14.3.23} moistened the clay in their hands. 

\scrule{“I allow a trough for the clay.” }

The\marginnote{14.3.25} clay was smelly. 

\scrule{“I allow you to add scent.” }

The\marginnote{14.3.27} fire in the sauna scorched their bodies. 

\scrule{“I allow you to bring water.” }

They\marginnote{14.3.29} brought the water in basins and bowls. 

\scrule{“I allow a place for the water and a water scoop.” }

Because\marginnote{14.3.31} the sauna had a grass roof, they did not sweat. 

\scrule{“I allow you to firm up the structure and then to plaster it inside and outside.” }

The\marginnote{14.3.33} sauna was muddy. 

\scrule{“I allow three kinds of floors: floors of brick, stone, and wood.” }

It\marginnote{14.3.36} was still muddy. 

\scrule{“You should wash it.” }

The\marginnote{14.3.38} water remained. 

\scrule{“I allow water drains.” }

The\marginnote{14.3.40} monks sat on the ground and their limbs became itchy. 

\scrule{“I allow sauna benches.” }

At\marginnote{14.3.43} that time the saunas were unenclosed. 

\scrule{“I allow three kinds of encircling walls: walls of brick, stone, and wood.” }

There\marginnote{14.4.1} were no gatehouses.\footnote{For the rendering “gatehouse” for \textit{\textsanskrit{koṭṭhaka}}, see Appendix of Technical Terms. } 

\scrule{“I allow gatehouses.” }

They\marginnote{14.4.3} built the gatehouses on a low base. They were flooded. 

\scrule{“I allow you to raise the base.” }

The\marginnote{14.4.5} mound collapsed. 

\scrule{“I allow you to construct three kinds of raised foundations: raised foundations of brick, stone, and wood.” }

It\marginnote{14.4.8} was difficult to get up to the gatehouses. 

\scrule{“I allow three kinds of stairs: stairs of brick, stone, and wood.” }

People\marginnote{14.4.11} fell down while climbing the stairs. 

\scrule{“I allow rails.” }

The\marginnote{14.4.13} gatehouses didn’t have doors. 

\scrule{“I allow doors, door frames, lower hinges, upper hinges, door jambs, bolt sockets, bolts, latches, keyholes, door-pulling holes, and door-pulling ropes.” }

Grass\marginnote{14.4.15} and dust fell into the gatehouses. 

\scrule{“I allow you to firm up the structure and then plaster it inside and outside, including: treating with white color, black color, and red ocher; making garland patterns, creeper patterns, shark-teeth patterns, and the fivefold pattern.” }

The\marginnote{14.5.1} yards were muddy.\footnote{For an explanation of rendering \textit{\textsanskrit{pariveṇa}} as “yard”, see Appendix of Technical Terms. } 

\scrule{“I allow a you to cover them with gravel.” }

They\marginnote{14.5.3} were unable to do it.\footnote{This is an unusual use of the verb \textit{\textsanskrit{pariyāpuṇati}}, which normally means “to learn”. I follow the suggestion in DOP. } 

\scrule{“I allow you to lay paving stones.”\footnote{Or “slabs of stone”, \textit{padarasila}. See \href{https://suttacentral.net/pli-tv-bu-vb-pc18/en/brahmali\#2.3.5}{Bu Pc 18:2.3.5} where \textit{padara} means “floor boards”. } }

The\marginnote{14.5.5} water remained. 

\scrule{“I allow water drains.” }

At\marginnote{15.1.1} that time naked monks bowed down to other naked monks, had other naked monks bow down to them, provided assistance to other naked monks, had other naked monks provide assistance to them, gave to other naked monks, received, ate fresh foods, ate cooked foods, ate other foods, and drank. They told the Buddha. 

\scrule{“One who is naked shouldn’t bow down to one who is naked, shouldn’t bow down to anyone, shouldn’t have a naked monk bow down to him, shouldn’t have anyone bow down to him, shouldn’t provide assistance to a naked monk, shouldn’t have a naked monk provide assistance to him, shouldn’t give to a naked monk, shouldn’t receive, shouldn’t eat fresh foods, shouldn’t eat cooked food, shouldn’t eat anything, and shouldn’t drink. If you do, you commit an offense of wrong conduct.” }

At\marginnote{16.1.1} that time the monks put their robes on the ground in the sauna. The robes became dirty. They told the Buddha. 

\scrule{“I allow bamboo robe racks and clotheslines.” }

It\marginnote{16.1.5} rained and the robes became wet. 

\scrule{“I allow sauna sheds.”\footnote{“Sauna shed” renders \textit{\textsanskrit{jantāgharasālā}}. It seems from the above description of a sauna that it was a building, including a roof. A \textit{\textsanskrit{jantāgharasālā}} is then presumably another building (from the description below it seems to be a separate building from the sauna) where you get undressed and hang up your robes, etc. For an explanation of rendering \textit{\textsanskrit{sālā}} as “shed”, see Appendix of Technical Terms. } }

They\marginnote{16.1.7} built the sauna sheds on a low base. They were flooded. 

\scrule{“I allow you to raise the base.” }

The\marginnote{16.1.9} mound collapsed. 

\scrule{“I allow you to construct three kinds of raised foundations: raised foundations of brick, stone, and wood.” }

It\marginnote{16.1.11.1} was difficult to get up to the sauna sheds. 

\scrule{“I allow three kinds of stairs: stairs of brick, stone, and wood.” }

People\marginnote{16.1.12} fell down while climbing the stairs. 

\scrule{“I allow rails.” }

Grass\marginnote{16.1.14} and dust fell into the sauna sheds. 

\scrule{“I allow you to firm up the structure and then to plaster it inside and outside, including: treating with white color, black color, and red ocher; making garland patterns, creeper patterns, shark-teeth patterns, and the fivefold pattern; putting up bamboo robe racks and clotheslines.” }

Being\marginnote{16.2.1} afraid of wrongdoing, the monks did not provide assistance to one another either in the sauna or in the water. 

\scrule{“I allow you to regard three things as a ‘covering’: a sauna, water, and a cloth.” }

\subsection*{Wells }

On\marginnote{16.2.5.1} one occasion there was no water in the sauna. They told the Buddha. 

\scrule{“I allow a well.” }

The\marginnote{16.2.8} edge of the well collapsed. 

\scrule{“I allow you to construct three kinds of foundations: foundations of brick, stone, and wood.” }

The\marginnote{16.2.11} well was situated at a low point. It was flooded. 

\scrule{“I allow you to raise the base.” }

The\marginnote{16.2.13.1} mound collapsed. 

\scrule{“I allow you to construct three kinds of raised foundations: raised foundations of brick, stone, and wood.” }

It\marginnote{16.2.14.1} was difficult to get up to the well. 

\scrule{“I allow three kinds of stairs: stairs of brick, stone, and wood.” }

People\marginnote{16.2.15} fell down while climbing the stairs. 

\scrule{“I allow rails.” }

At\marginnote{16.2.17} that time the monks used creepers and belts to haul water. 

\scrule{“I allow a water-hauling rope.” }

It\marginnote{16.2.20} hurt their hands. 

\scrule{“I allow a well-sweep, a pulley, and well-wheels.”\footnote{“Well-sweep”, “pulley”, and “well-wheels” respectively render \textit{tula}, \textit{\textsanskrit{karakaṭaka}}, and \textit{\textsanskrit{cakkavaṭṭaka}}. For all three words see CPD, sv. \textit{\textsanskrit{karakaṭaka}}. } }

Many\marginnote{16.2.22} vessels broke. 

\scrule{“I allow three kinds of buckets: buckets made of iron, wood, and hide.” }

Hauling\marginnote{16.2.25} water outside, the monks were troubled by the cold and the heat. They told the Buddha. 

\scrule{“I allow well houses.”\footnote{For an explanation of rendering \textit{\textsanskrit{sālā}} as “house”, see Appendix of Technical Terms. } }

Grass\marginnote{16.2.28} and dust fell into the well houses. 

\scrule{“I allow you to firm up the structure and then plaster it inside and outside, including: treating with white color, black color, and red ocher; making garland patterns, creeper patterns, shark-teeth patterns, and the fivefold pattern; putting up bamboo robe racks and clotheslines.” }

The\marginnote{16.2.31} wells were not covered. Grass, dust, and dirt fell into them. 

\scrule{“I allow covers.” }

There\marginnote{16.2.33} were no vessels for the water. 

\scrule{“I allow water troughs and waterpots.” }

\subsection*{Other structures }

At\marginnote{17.1.1} that time the monks bathed here and there in the monastery. The monastery became muddy. They told the Buddha. 

\scrule{“I allow a waste-water disposal area.”\footnote{“Waste-water disposal area” renders \textit{candanika}, which is not defined in the Vinaya commentaries. Elsewhere, however, we find the following definition, e.g. at MN-a 1.25: \textit{Candanikanti \textsanskrit{ucchiṭṭhodakagabbhamalādīnaṁ} \textsanskrit{chaḍḍanaṭṭhānaṁ}}, “\textit{Candanika} means a place for the discarding of used water, afterbirth, etc.” We also know that it is not a cesspit, which is known as a \textit{\textsanskrit{vaccakūpa}}, see below. } }

The\marginnote{17.1.5} area was unenclosed.\footnote{Sp 4.262: \textit{\textsanskrit{Pākaṭā} \textsanskrit{hotīti} \textsanskrit{aparikkhittā} hoti}, “\textit{\textsanskrit{Pākaṭā} hoti} means it is unenclosed.” } The monks were embarrassed to bathe there. 

\scrule{“I allow three kinds of encircling walls: walls of brick, stone, and wood.” }

The\marginnote{17.1.9} area became muddy. 

\scrule{“I allow three kinds of deckings: deckings of brick, stone, and wood.” }

The\marginnote{17.1.12} water remained. 

\scrule{“I allow water drains.” }

The\marginnote{17.1.14} monks were cold. 

\scrule{“I allow a water wiper and a towel to dry yourselves.” }

On\marginnote{17.2.1} one occasion a lay follower wanted to build a lotus bathing tank for the benefit of the Sangha.\footnote{For an explanation of the rendering “lotus bathing tank” for \textit{\textsanskrit{pokkharaṇī}}, see Appendix of Technical Terms. } They told the Buddha. 

\scrule{“I allow lotus bathing tanks.” }

The\marginnote{17.2.4} edges of the tank collapsed. 

\scrule{“I allow you to construct three kinds of foundations: foundations of brick, stone, and wood.” }

It\marginnote{17.2.7} was difficult to get up to the tank. 

\scrule{“I allow three kinds of stairs: stairs of brick, stone, and wood.” }

People\marginnote{17.2.10} fell down while climbing the stairs. 

\scrule{“I allow rails.” }

The\marginnote{17.2.12} water in the tanks became stagnant. 

\scrule{“I allow a channel and a drain.” }

On\marginnote{17.2.14} one occasion a certain monk wanted to build a sauna with a pointed roof for the benefit of the Sangha. 

\scrule{“I allow saunas with pointed roofs.”\footnote{“Saunas with pointed roofs” renders \textit{\textsanskrit{nillekhaṁ} \textsanskrit{jantāgharaṁ}}. Sp 4.263: \textit{\textsanskrit{Nillekhajantāgharaṁ} \textsanskrit{nāma} \textsanskrit{āviddhapakkhapāsakaṁ} vuccati, \textsanskrit{gopānasīnaṁ} upari \textsanskrit{maṇḍale} \textsanskrit{pakkhapāsake} \textsanskrit{ṭhapetvā} \textsanskrit{katakūṭacchadanassetaṁ} \textsanskrit{nāmaṁ}}, “One with a side loop all around is called a \textit{\textsanskrit{nillekhajantāghara}}. Having fixed on a side loop circle above the rafters, this is a name for a roof made with a peak.” The exact meaning is not clear to me. } }

\section*{Various regulations on proper conduct and allowable requisites }

At\marginnote{18.1.1} one time the monks from the group of six did not have sitting mats for a period of four months. They told the Buddha. 

\scrule{“You shouldn’t be without a sitting mat for a period of four months.\footnote{For an explanation of the rendering “sitting mat” for \textit{\textsanskrit{nisīdana}}, see Appendix of Technical Terms. } If you are, you commit an offense of wrong conduct.” }

At\marginnote{18.1.5} that time the monks from the group of six slept in beds covered in flowers. When people walking about the dwellings saw this, they complained and criticized them, “They’re just like householders who indulge in worldly pleasures!” 

\scrule{“You shouldn’t sleep in a bed covered in flowers. If you do, you commit an offense of wrong conduct.” }

Soon,\marginnote{18.1.11} people brought scents and garlands to the monastery. Being afraid of wrongdoing, the monks did not accept. 

\scrule{“I allow you to accept scent to make the five-finger mark on your door and to accept flowers to place to one side in your dwelling.” }

On\marginnote{19.1.1} one occasion the Sangha was offered a piece of felt. 

\scrule{“I allow felt.” }

The\marginnote{19.1.4} monks thought, “Should it be determined or assigned to another?” 

\scrule{“It should neither be determined nor assigned to another.”\footnote{For an explanation of the idea of \textit{\textsanskrit{vikappanā}}, see Appendix of Technical Terms. } }

The\marginnote{19.1.7} monks from the group of six ate food on a stand with a heating device.\footnote{Sp 4.264: \textit{\textsanskrit{Āsittakūpadhānaṁ} \textsanskrit{nāma} tambalohena \textsanskrit{vā} rajatena \textsanskrit{vā} \textsanskrit{katāya} \textsanskrit{peḷāya} \textsanskrit{etaṁ} \textsanskrit{adhivacanaṁ}, \textsanskrit{paṭikkhittattā} pana \textsanskrit{dārumayāpi} na \textsanskrit{vaṭṭati}}, “\textit{\textsanskrit{Āsittakūpadhāna}}: this is a term for a container made of copper or silver. But a prohibited one made of wood is also not allowable.” Vmv 4.264: \textit{\textsanskrit{Peḷāyāti} \textsanskrit{aṭṭhaṁsasoḷasaṁsādiākārena} \textsanskrit{katāya} \textsanskrit{bhājanākārāya} \textsanskrit{peḷāya}. Yattha \textsanskrit{uṇhapāyāsādiṁ} \textsanskrit{pakkhipitvā} upari \textsanskrit{bhojanapātiṁ} \textsanskrit{ṭhapenti} bhattassa \textsanskrit{uṇhabhāvāvigamanatthaṁ}, \textsanskrit{tādisassa} \textsanskrit{bhājanākārassa} \textsanskrit{ādhārassetaṁ} \textsanskrit{adhivacanaṁ}. Teneva \textsanskrit{pāḷiyaṁ} \textsanskrit{āsittakūpadhānanti} \textsanskrit{vuttaṁ}}, “\textsanskrit{Pelāya}: a container having the form of a vessel made with eight or sixteen edges, etc., where, for the purpose of keeping the food hot, having filled with hot milk-rice, etc., they place a bowl with food on top. This is a term for the stand for such a vessel. For this reason it is said \textit{\textsanskrit{āsittakūpadhāna}} in the Canonical text.” } People complained and criticized them, “They’re just like householders who indulge in worldly pleasures!” 

\scrule{“You shouldn’t eat food on a stand with a heating device. If you do, you commit an offense of wrong conduct.” }

On\marginnote{19.1.13} one occasion a certain sick monk was unable to hold his bowl with his hands while eating. 

\scrule{“I allow a stand.”\footnote{Sp-\textsanskrit{ṭ} 4.264: \textit{Pubbe \textsanskrit{pattasaṅgopanatthaṁ} \textsanskrit{ādhārako} \textsanskrit{anuññāto}, \textsanskrit{idāni} \textsanskrit{bhuñjanatthaṁ}}, “Previously a stand for the purpose of protecting the bowls was allowed; now (it is allowed) for the purpose of eating.” The stand referred to here was allowed above at \href{https://suttacentral.net/pli-tv-kd15/en/brahmali\#9.4.4}{Kd 15:9.4.4} (there rendered as “rack”). } }

At\marginnote{19.2.1} that time the monks from the group of six ate from the same vessel and drank from the same vessel, and they lay down on the same bed, on the same sheet, under the same cover, and both on the same sheet and under the same cover.\footnote{“The same sheet and under the same cover” renders \textit{\textsanskrit{ekattharaṇapāvuraṇa}}. Sp 2.937: \textit{\textsanskrit{Ekattharaṇapāvuraṇā}; \textsanskrit{saṁhārimānaṁ} \textsanskrit{pāvārattharaṇakaṭasārakādīnaṁ} \textsanskrit{ekaṁ} \textsanskrit{antaṁ} \textsanskrit{attharitvā} \textsanskrit{ekaṁ} \textsanskrit{pārupitvā} \textsanskrit{tuvaṭṭentīnametaṁ} \textsanskrit{adhivacanaṁ}}, “\textit{\textsanskrit{Ekattharaṇapāvuraṇa}}: this is an expression for: a mobile blanket, a mat, a straw mat, etc., having spread one of these to one side and having covered with another, one lies down.” } People complained and criticized them, “They’re just like householders who indulge in worldly pleasures!” 

\scrule{“You shouldn’t eat from the same vessel, drink from the same vessel, lie down on the same bed, lie down on the same sheet, lie down under the same cover, or lie down both on the same sheet and under the same cover. If you do, you commit an offense of wrong conduct.” }

\section*{Overturning the bowl }

At\marginnote{20.1.1} one time \textsanskrit{Vaḍḍha} the \textsanskrit{Licchavī} was a friend of the monks Mettiya and \textsanskrit{Bhūmajaka}. On one occasion he went to them and said, “Respectful greetings, venerables.” They did not respond. A second time and a third time he said the same thing, but they still did not respond. 

“Have\marginnote{20.1.11} I done something wrong? Why don’t you respond?” 

“It’s\marginnote{20.1.12} because we’ve been treated badly by Dabba the Mallian, and you’re not taking an interest.” 

“But\marginnote{20.1.13} what can I do?” 

“If\marginnote{20.1.14} you like, you could make the Buddha expel Dabba.” 

“And\marginnote{20.1.15} how can I do that?” 

“Go\marginnote{20.1.16} to the Buddha and say, ‘Sir, this is not proper or appropriate. There’s fear, distress, and oppression in this district, where none of these should exist. It’s windy where it should be calm. It’s as if water is burning. Venerable Dabba the Mallian has raped my wife.’” 

Saying,\marginnote{20.2.1} “Alright, venerables,” he went to the Buddha, bowed, sat down, and repeated what he had been told to say. 

Soon\marginnote{20.2.8} afterwards the Buddha had the Sangha gathered and questioned Dabba: “Dabba, do you remember doing as \textsanskrit{Vaḍḍha} says?” 

“Sir,\marginnote{20.2.10} you know what I’m like.” 

A\marginnote{20.2.11} second and a third time the Buddha asked the same question and got the same response. He then said, “Dabba, the Dabbas don’t give such evasive answers. If it was done by you, say so; if it wasn’t, then say that.” 

“Since\marginnote{20.2.18} I was born, sir, I don’t recall having sexual intercourse even in a dream, let alone when awake.” 

The\marginnote{20.3.1} Buddha addressed the monks: “Well then, monks, the Sangha should overturn the almsbowl against \textsanskrit{Vaḍḍha} the \textsanskrit{Licchavī}, prohibiting him from interacting with the Sangha. 

When\marginnote{20.3.3} a lay follower has eight qualities, you should overturn your bowl against him: he’s trying to stop monks from getting material support; he’s trying to harm monks; he’s trying to get monks to lose their place of residence; he abuses and reviles monks; he causes division between monks; he disparages the Buddha; he disparages the Teaching; he disparages the Sangha. 

And\marginnote{20.4.1} the overturning of the bowl is to be done like this. A competent and capable monk should inform the Sangha: 

‘Please,\marginnote{20.4.3} venerables, I ask the Sangha to listen. \textsanskrit{Vaḍḍha} the \textsanskrit{Licchavī} is groundlessly charging Venerable Dabba the Mallian with failure in morality. If the Sangha is ready, it should overturn the bowl against \textsanskrit{Vaḍḍha} the \textsanskrit{Licchavī}, prohibiting him from interacting with the Sangha. This is the motion. 

Please,\marginnote{20.4.7} venerables, I ask the Sangha to listen. \textsanskrit{Vaḍḍha} the \textsanskrit{Licchavī} is groundlessly charging Venerable Dabba the Mallian with failure in morality. The Sangha overturns the bowl against \textsanskrit{Vaḍḍha} the \textsanskrit{Licchavī}, prohibiting him from interacting with the Sangha. Any monk who approves of overturning the bowl against \textsanskrit{Vaḍḍha} the \textsanskrit{Licchavī} should remain silent. Any monk who doesn’t approve should speak up. 

The\marginnote{20.4.12} Sangha has overturned the bowl against \textsanskrit{Vaḍḍha} the \textsanskrit{Licchavī}, prohibiting him from interacting with the Sangha. The Sangha approves and is therefore silent. I’ll remember it thus.’” 

\section*{Turning the bowl upright }

After\marginnote{20.5.1} robing up the following morning, Venerable Ānanda took his bowl and robe, went to the house of \textsanskrit{Vaḍḍha} the \textsanskrit{Licchavī}, and told him, “\textsanskrit{Vaḍḍha}, the Sangha has overturned the bowl against you. You’re prohibited from interacting with the Sangha.” And \textsanskrit{Vaḍḍha} fainted right there. But \textsanskrit{Vaḍḍha}’s friends and relatives said to him, “Don’t be sad, \textsanskrit{Vaḍḍha}. We’ll reconcile you with the Buddha and the Sangha of monks.” 

Soon\marginnote{20.5.10} afterwards \textsanskrit{Vaḍḍha}, together with his wives and children, together with his friends and relatives, with wet clothes and wet hair, went to the Buddha. He bowed down at the Buddha’s feet and said, “Sir, I’ve made a mistake. I’ve been foolish, confused, and unskillful. Please forgive me so that I may restrain myself in the future.” 

“You\marginnote{20.5.13} have certainly made a mistake. You’ve been foolish, confused, and unskillful. But since you acknowledge your mistake and make proper amends, I forgive you. For this is called growth in the training of the noble ones: acknowledging a mistake, making proper amends, and undertaking restraint in the future.”\footnote{For an explanation of the rendering “training” for \textit{vinaya}, see Appendix of Technical Terms. } 

The\marginnote{20.6.1} Buddha then addressed the monks: “Well then, the Sangha should turn the almsbowl upright for \textsanskrit{Vaḍḍha} the \textsanskrit{Licchavī}, allowing him to interact with the Sangha. 

When\marginnote{20.6.3} a lay follower has eight qualities, you should turn your bowl upright for him: he’s not trying to stop monks from getting material support; he’s not trying to harm monks; he’s not trying to get monks to lose their place of residence; he doesn’t abuse or revile monks; he doesn’t cause division between monks; he doesn’t disparage the Buddha; he doesn’t disparage the Teaching; he doesn’t disparage the Sangha. 

And\marginnote{20.7.1} the turning of the bowl upright is to be done like this. \textsanskrit{Vaḍḍha} the \textsanskrit{Licchavī} should approach the Sangha, arrange his upper robe over one shoulder, pay respect at the feet of the monks, squat on his heels, raise his joined palms, and say: 

‘Venerables,\marginnote{20.7.3} the Sangha has overturned the bowl against me, prohibiting me from interacting with the Sangha. I’m now conducting myself properly and suitably so as to deserve to be released. I ask the Sangha to turn the bowl upright for me.’\footnote{The meaning of the first of these phrases, \textit{\textsanskrit{sammā} vattati}, is straightforward, but the last two, \textit{\textsanskrit{lomaṁ} \textsanskrit{pāteti}} and \textit{\textsanskrit{netthāraṁ} vattati}, are more difficult. Commenting on Bu Ss 13, Sp 1.435 explains: \textit{Na \textsanskrit{lomaṁ} \textsanskrit{pātentīti} \textsanskrit{anulomapaṭipadaṁ} \textsanskrit{appaṭipajjanatāya} na \textsanskrit{pannalomā} honti. Na \textsanskrit{netthāraṁ} \textsanskrit{vattantīti} attano \textsanskrit{nittharaṇamaggaṁ} na \textsanskrit{paṭipajjanti}}, “\textit{Na \textsanskrit{lomaṁ} \textsanskrit{pātenti}}: because of their non-practicing in conformity with the path, their bodily hairs are not flat. \textit{Na \textsanskrit{netthāraṁ} vattanti}: they are not practicing the path for their own getting out (of the offense).” My rendering attempts to capture the meaning in a non-literal way. } And he should ask a second and a third time. A competent and capable monk should then inform the Sangha: 

‘Please,\marginnote{20.7.8} venerables, I ask the Sangha to listen. The Sangha has overturned the bowl against \textsanskrit{Vaḍḍha} the \textsanskrit{Licchavī}, prohibiting him from interacting with the Sangha. He’s now conducting himself properly and suitably so as to deserve to be released, and is asking the Sangha to turn the bowl upright for him. If the Sangha is ready, it should turn the bowl upright for \textsanskrit{Vaḍḍha} the \textsanskrit{Licchavī}, allowing him to interact with the Sangha. This is the motion. 

Please,\marginnote{20.7.13} venerables, I ask the Sangha to listen. The Sangha has overturned the bowl against \textsanskrit{Vaḍḍha} the \textsanskrit{Licchavī}, prohibiting him from interacting with the Sangha. He’s now conducting himself properly and suitably so as to deserve to be released, and is asking the Sangha to turn the bowl upright for him. The Sangha turns the bowl upright for \textsanskrit{Vaḍḍha} the \textsanskrit{Licchavī}, allowing him to interact with the Sangha. Any monk who approves of turning the bowl upright for \textsanskrit{Vaḍḍha} the \textsanskrit{Licchavī} should remain silent. Any monk who doesn’t approve should speak up. 

The\marginnote{20.7.19} Sangha has turned the bowl upright for \textsanskrit{Vaḍḍha} the \textsanskrit{Licchavī}, allowing him to interact with the Sangha. The Sangha approves and is therefore silent. I’ll remember it thus.’” 

\section*{Stepping on cloth }

When\marginnote{21.1.1} the Buddha had stayed at \textsanskrit{Vesālī} for as long as he liked, he set out wandering toward the country of \textsanskrit{Bhaggā}. When he eventually arrived, he stayed at \textsanskrit{Susumāragira} in the \textsanskrit{Bhesakaḷā} Grove, the deer park. 

At\marginnote{21.1.4} this time Prince Bodhi had recently built the Kokanada stilt house. It had not yet been inhabited by any monastic or brahmin, or anyone else. 

The\marginnote{21.1.5} prince said to the young brahmin \textsanskrit{Sañcikāputta}, “My dear \textsanskrit{Sañcikāputta}, please go to the Buddha, bow down in my name with your head at his feet, and ask if he’s healthy, strong, and living at ease. And then say, ‘Sir, please accept tomorrow’s meal from Prince Bodhi together with the Sangha of monks.’” 

Saying,\marginnote{21.1.11} “Yes, sir,” \textsanskrit{Sañcikāputta} went to the Buddha and exchanged pleasantries with him. He then sat down and told the Buddha all he had been asked to say, concluding with the invitation for the meal on the following day. The Buddha consented by remaining silent. Knowing that the Buddha had consented, \textsanskrit{Sañcikāputta} got up from his seat, returned to the prince, and told him what had happened. 

The\marginnote{21.2.7} next morning Prince Bodhi had various kinds of fine foods prepared, and had the entire Kokanada stilt house covered with white cloth, all the way to the bottom step of the staircase. He then said to \textsanskrit{Sañcikāputta}, “Go to the Buddha and tell him the meal is ready.” And \textsanskrit{Sañcikāputta} did as instructed. 

Soon\marginnote{21.2.12} afterwards, the Buddha robed up in the morning, took his bowl and robe, and went the prince’s house. The prince was standing outside the gatehouse, waiting for the Buddha. When he saw the Buddha coming, he went out to meet him, bowed down to him, and then returned to the Kokanada stilt house with the Buddha in front. 

But\marginnote{21.2.16} the Buddha stopped at the bottom stair of that staircase. The prince said, “Sir, please step on the cloth. It will be for my long-term benefit and happiness.” The Buddha remained silent. A second time the prince repeated his request, but the Buddha still remained silent. When the prince made his request for the third time, the Buddha looked at Venerable Ānanda. And Ānanda said to the prince, “Please fold up the cloth. The Buddha doesn’t step on cloth coverings. He has compassion for later generations.” 

The\marginnote{21.3.1} prince then had the cloth folded up and had a seat prepared up in the stilt house. The Buddha ascended the house and sat down on the prepared seat together with the Sangha of monks. The prince personally served the various kinds of fine foods to the Sangha of monks headed by the Buddha. When the Buddha had finished his meal and had washed his hands and bowl, the prince sat down to one side. The Buddha instructed, inspired, and gladdened him with a teaching, after which he got up from his seat and left. 

Soon\marginnote{21.3.5} afterwards the Buddha gave a teaching and addressed the monks: 

\scrule{“You shouldn’t step on a cloth covering. If you do, you commit an offense of wrong conduct.” }

On\marginnote{21.4.1} one occasion a woman who was unable to conceive invited the monks, prepared a cloth, and said, “Venerables, please step on the cloth.” But being afraid of wrongdoing, they refused. “Please step on the cloth as a blessing.” They still refused. That woman complained and criticized them, “How can the venerables not step on a cloth as a blessing when asked?” The monks heard the complaints of that woman, and they told the Buddha what had happened. 

\scrule{“Householders want blessings. I allow you, when asked, to step on a cloth covering as a blessing for householders.” }

Being\marginnote{21.4.12} afraid of wrongdoing, the monks did not step on a towel after washing their feet. 

\scrule{“I allow you to step on a towel after washing your feet.”\footnote{Sp 4.268: \textit{\textsanskrit{Dhotapādakaṁ} \textsanskrit{nāma} \textsanskrit{pādadhovanaṭṭhāne} dhotehi \textsanskrit{pādehi} \textsanskrit{akkamanatthāya} \textsanskrit{paccattharaṇaṁ} \textsanskrit{atthataṁ} hoti, \textsanskrit{taṁ} \textsanskrit{akkamituṁ} \textsanskrit{vaṭṭati}}, “\textit{\textsanskrit{Dhotapādaka}}: a mat that has been spread out for the purpose of stepping on with the washed feet at the place of washing the feet. To step on that is allowable.” } }

\scend{The second section for recitation is finished. }

\section*{More regulations on proper conduct and allowable requisites. }

When\marginnote{22.1.1} the Buddha had stayed in the country of \textsanskrit{Bhaggā} for as long as he liked, he set out wandering toward \textsanskrit{Sāvatthī}. When he eventually arrived, he stayed in the Jeta Grove, \textsanskrit{Anāthapiṇḍika}’s Monastery. 

Soon\marginnote{22.1.4} \textsanskrit{Visākhā} \textsanskrit{Migāramātā} went to the Buddha, taking a waterpot, a ceramic foot scrubber, and a broom. She bowed to the Buddha, sat down,\footnote{That the foot scrubber, the \textit{kataka}, is ceramic is clear from \href{https://suttacentral.net/pli-tv-kd15/en/brahmali\#37.1.8}{Kd 15:37.1.8} below. } and said, “Sir, for my long-term benefit and happiness, please accept this waterpot, foot scrubber, and broom.” The Buddha accepted the waterpot and the broom, but not the ceramic foot scrubber. He then instructed, inspired, and gladdened her with a teaching, after which she got up from her seat, bowed down, circumambulated him with her right side toward him, and left. 

Soon\marginnote{22.1.11} afterwards the Buddha gave a teaching and addressed the monks: 

\scrule{“I allow waterpots and brooms. But you shouldn’t use a ceramic foot scrubber. If you do, you commit an offense of wrong conduct. I allow three kinds of foot scrubbers: stones, pebbles, and pumice.”\footnote{“Stones” renders \textit{sakkhara}. Sp 4.269: \textit{\textsanskrit{Sakkharāti} \textsanskrit{pāsāṇo} vuccati}, “A rock is called \textit{sakkhara}.” “Pumice” renders \textit{\textsanskrit{samuddapheṇaka}}, literally, “ocean foam”. Sp 4.269: \textit{\textsanskrit{Pāsāṇapheṇakopi} \textsanskrit{vaṭṭati}}, “Also rock-foam is allowable”. } }

\textsanskrit{Visākhā}\marginnote{22.2.1} again went to the Buddha, now taking a standard fan and a palm-leaf fan. She bowed, sat down, and said, “Sir, for my long-term benefit and happiness, please accept this standard fan and this palm-leaf fan.” The Buddha accepted both. 

He\marginnote{22.2.5} then instructed, inspired, and gladdened her with a teaching, after which she got up from her seat, bowed, circumambulated him with her right side toward him, and left. Soon afterwards the Buddha gave a teaching and addressed the monks: 

\scrule{“I allow standard fans and a palm-leaf fans.”\footnote{“A fan” renders \textit{\textsanskrit{vidhūpana}}. Sp 4.269: \textit{\textsanskrit{Vidhūpananti} \textsanskrit{vījanī} vuccati}, “A fan is called a \textit{\textsanskrit{vidhūpana}}.” To distinguish it from the “palm-leaf fan”, I render it as “standard fan”. } }

On\marginnote{23.1.1} one occasion the Sangha was offered a mosquito whisk. 

\scrule{“I allow mosquito whisks.” }

The\marginnote{23.1.4} Sangha was offered a yak-tail whisk. 

\scrule{“You shouldn’t use a yak-tail whisk. If you do, you commit an offense of wrong conduct. I allow three kinds of fans: those made of bark, vetiver grass, and peacocks’ tail feathers.” }

On\marginnote{23.2.1} one occasion the Sangha was offered a sunshade. 

\scrule{“I allow sunshades.” }

Soon\marginnote{23.2.4} afterwards the monks from the group of six walked about holding sunshades. Then, as a certain Buddhist lay follower and a number of \textsanskrit{Ājīvaka} disciples were going to the park, the \textsanskrit{Ājīvakas} saw those monks in the distance with their sunshades. They said to that lay follower, “These venerables of yours are coming. They’re holding sunshades, just like accountants and government officials.” 

“These\marginnote{23.2.9} aren’t monks. They’re wanderers.” And they made a bet on whether they were monks or not. 

When\marginnote{23.2.11} the monks came close, that lay follower recognized them. And he complained and criticized them, “How can the venerables walk about holding sunshades?” The monks heard the complaints of that lay follower and they told the Buddha. 

“Is\marginnote{23.2.15} it true, monks, that the monks from the group of six are doing this?” “It’s true, sir.” … After rebuking them … the Buddha gave a teaching and addressed the monks: 

\scrule{“You shouldn’t use a sunshade. If you do, you commit an offense of wrong conduct.” }

Soon\marginnote{23.3.1} afterwards a certain sick monk was not comfortable without a sunshade. 

\scrule{“I allow sick monks to use sunshades.” }

When\marginnote{23.4.1} they heard that the Buddha had allowed sunshades for the sick, but not for the healthy, and being afraid of wrongdoing, the monks did not use sunshades in the monastery or in the vicinity of the monastery. 

\scrule{“I allow you to use a sunshade in a monastery and in the vicinity of a monastery, even if you’re healthy.” }

\section*{Carrying nets and staffs, etc.}

On\marginnote{24.1.1} one occasion a monk put his almsbowl in a carrying net, hung it from a staff, and passed through the gateway to a certain village at an unusual hour.\footnote{“Carrying net” renders \textit{\textsanskrit{sikkā}}. Vin-vn-\textsanskrit{ṭ} 91: \textit{\textsanskrit{Sikkāyāti} \textsanskrit{olambikādhāre}}, “\textit{\textsanskrit{Sikkā}} means a holder for hanging.” CPD, sv. \textit{\textsanskrit{uḍḍeti}}, says, “To put in a sling or carrying net”, evidently referring to the \textit{\textsanskrit{sikkā}}. } People said, “This must be a gangster coming with his gleaming sword.” They pounced and seized him, but when they recognized him, they let him go. 

He\marginnote{24.1.4} returned to the monastery and told the monks what had happened. They said, “So you used a carrying net and a staff?” 

“Yes.”\marginnote{24.1.6} 

The\marginnote{24.1.7} monks of few desires complained and criticized him, “How can a monk use a carrying net and a staff?” They told the Buddha … “It’s true, sir.” … After rebuking him … the Buddha gave a teaching and addressed the monks: 

\scrule{“You shouldn’t use a carrying net and a staff. If you do, you commit an offense of wrong conduct.” }

On\marginnote{24.2.1} one occasion there was a sick monk who was unable to walk about without a staff. 

“I\marginnote{24.2.3} allow you to give a sick monk permission to use a staff. And it should be given like this. The sick monk should approach the Sangha, arrange his upper robe over one shoulder, pay respect at the feet of the senior monks, squat on his heels, raise his joined palms, and say: 

‘Venerables,\marginnote{24.2.6} I’m sick. I’m unable to walk about without a staff. I ask the Sangha for permission to use a staff.’ And he should ask a second and a third time. A competent and capable monk should then inform the Sangha: 

‘Please,\marginnote{24.2.12} venerables, I ask the Sangha to listen. Monk so-and-so is sick. He’s unable to walk about without a staff. He’s asking the Sangha for permission to use a staff. If the Sangha is ready, it should give monk so-and-so permission to use a staff. This is the motion. 

Please,\marginnote{24.2.17} venerables, I ask the Sangha to listen. Monk so-and-so is sick. He’s unable to walk about without a staff. He’s asking the Sangha for permission to use a staff. The Sangha gives monk so-and-so permission to use a staff. Any monk who approves of this should remain silent. Any monk who doesn’t approve should speak up. 

The\marginnote{24.2.23} Sangha has given monk so-and-so permission to use a staff. The Sangha approves and is therefore silent. I’ll remember it thus.’” 

On\marginnote{24.3.1} one occasion there was a sick monk who was unable to carry his almsbowl without a carrying net. They told the Buddha. 

“I\marginnote{24.3.3} allow you to give a sick monk permission to use a carrying net. And it should be given like this. The sick monk should approach the Sangha, arrange his upper robe over one shoulder, pay respect at the feet of the senior monks, squat on his heels, raise his joined palms, and say: 

‘Venerables,\marginnote{24.3.6} I’m sick. I’m unable to carry my bowl without a carrying net. I ask the Sangha for permission to use a carrying net.’ And he should ask a second and a third time. A competent and capable monk should then inform the Sangha: 

‘Please,\marginnote{24.3.12} venerables, I ask the Sangha to listen. Monk so-and-so is sick. He’s unable to carry his bowl without a carrying net. He’s asking the Sangha for permission to use a carrying net. If the Sangha is ready, it should give monk so-and-so permission to use a carrying net. This is the motion. 

Please,\marginnote{24.3.17} venerables, I ask the Sangha to listen. Monk so-and-so is sick. He’s unable to carry his bowl without a carrying net. He’s asking the Sangha for permission to use a carrying net. The Sangha gives monk so-and-so permission to use a carrying net. Any monk who approves of this should remain silent. Any monk who doesn’t approve should speak up. 

The\marginnote{24.3.23} Sangha has given monk so-and-so permission to use a carrying net. The Sangha approves and is therefore silent. I’ll remember it thus.’” 

On\marginnote{24.3.25} one occasion there was a sick monk who was unable to walk about without a staff or to carry his almsbowl without a carrying net. They told the Buddha. 

“I\marginnote{24.3.27} allow you to give a sick monk permission to use a staff and a carrying net. And it should be given like this. The sick monk should approach the Sangha, arrange his upper robe over one shoulder, pay respect at the feet of the senior monks, squat on his heels, raise his joined palms, and say: 

‘Venerables,\marginnote{24.3.30} I’m sick. I’m unable to walk about without a staff or to carry my bowl without a carrying net. I ask the Sangha to give me permission to use a staff and a carrying net.’ And he should ask a second and a third time. A competent and capable monk should then inform the Sangha: 

‘Please,\marginnote{24.3.37} venerables, I ask the Sangha to listen. Monk so-and-so is sick. He’s unable to walk about without a staff or to carry his bowl without a carrying net. He’s asking the Sangha for permission to use a staff and a carrying net. If the Sangha is ready, it should give monk so-and-so permission to use a staff and a carrying net. This is the motion. 

Please,\marginnote{24.3.42} venerables, I ask the Sangha to listen. Monk so-and-so is sick. He’s unable to walk about without a staff or to carry his bowl without a carrying net. He’s asking the Sangha for permission to use a staff and a carrying net. The Sangha gives monk so-and-so permission to use a staff and a carrying net. Any monk who approves of this should remain silent. Any monk who doesn’t approve should speak up. 

The\marginnote{24.3.48} Sangha has given monk so-and-so permission to use a staff and a carrying net. The Sangha approves and is therefore silent. I’ll remember it thus.’” 

At\marginnote{25.1.1} that time there was a monk who was a regurgitator. After regurgitating, he would swallow. The monks complained and criticized him, “This monk is eating at the wrong time.” They told the Buddha. 

\scrule{“This monk has only recently passed away as a cow. I allow a regurgitator to regurgitate. But you shouldn’t take it out of the mouth and then swallow it. If you do, you should be dealt with according to the rule.”\footnote{That is, \href{https://suttacentral.net/pli-tv-bu-vb-pc37/en/brahmali\#1.22.1}{Bu Pc 37:1.22.1}/\href{https://suttacentral.net/pli-tv-bi-vb-pc120/en/brahmali}{Bi Pc 120}. } }

Soon\marginnote{26.1.1} afterwards a certain association was offering a meal to the Sangha. Lots of rice fell on the floor in the dining hall. People complained and criticized them, “When they’re given rice, how can the Sakyan monastics not receive it with care? Each lump of rice is the result of hard work.” The monks heard the complaints of those people. They told the Buddha. 

\scrule{“I allow you to pick up and eat what falls down while being given. It has been relinquished by the donors.” }

\section*{Personal grooming }

On\marginnote{27.1.1} one occasion a certain monk with long nails was walking for alms. A woman who saw him said to him, “Come, venerable, and have sex.” 

“It’s\marginnote{27.1.4} not allowable.” 

“If\marginnote{27.1.5} you don’t, I’ll scratch my limbs with my nails and make a scene, saying that you abused me.” 

“That’s\marginnote{27.1.7} your business, sister.”\footnote{\textit{\textsanskrit{Pajānāhi} \textsanskrit{tvaṁ}, \textsanskrit{bhaginīti}}, literally, “You understand, sister.” Commenting on a different context, DN-a 2.436 says: \textit{\textsanskrit{Tvaṁ} \textsanskrit{pajānāhīti} \textsanskrit{tvaṁ} \textsanskrit{jāna}. Sace \textsanskrit{gaṇhitukāmosi}, \textsanskrit{gaṇhāhīti} \textsanskrit{vuttaṁ} hoti}, “You understand: you know. What is said is: if you wish to take it, then take it.” In other words, do as you please. } 

Yet\marginnote{27.1.8} that woman did as she had threatened. People rushed up and took hold of that monk. But when they saw the skin and blood on that woman’s nails, they realized she had done it herself, and they released the monk. He then returned to the monastery and told the monks what had happened. 

“So\marginnote{27.1.16} you grow your nails long?” 

“Yes.”\marginnote{27.1.17} 

The\marginnote{27.1.18} monks of few desires complained and criticized him, “How can a monk grow his nails long?” They told the Buddha. 

\scrule{“You shouldn’t grow your nails long. If you do, you commit an offense of wrong conduct.” }

Soon\marginnote{27.2.1} afterwards the monks were cutting their nails with their nails and teeth, or by grinding them on walls. Their fingers hurt. 

\scrule{“I allow nail clippers.” }

They\marginnote{27.2.5} cut their nails so short that they bled. Their fingers hurt. 

\scrule{“You should cut your nails so that they’re even with the tip of the flesh.”\footnote{“Even with the tip of the flesh” renders \textit{\textsanskrit{maṁsappamāṇena}}. Vin-\textsanskrit{ālaṅ}-\textsanskrit{ṭ} 34.40: \textit{\textsanskrit{Maṁsappamāṇenāti} \textsanskrit{aṅgulaggamaṁsappamāṇena}}, “\textit{\textsanskrit{Maṁsappamāṇena}}: measured by the flesh at the tip of the finger.” } }

At\marginnote{27.2.8} this time the monks from the group of six polished their finger and toe nails.\footnote{“Polished their finger and toe nails” renders \textit{\textsanskrit{vīsatimaṭṭhaṁ}}, literally, “Polished the twenty”. Sp 4.274: \textit{\textsanskrit{Vīsatimaṭṭhanti} \textsanskrit{vīsatipi} nakhe \textsanskrit{likhitamaṭṭhe} \textsanskrit{kārāpenti}}, “\textit{\textsanskrit{Vīsatimaṭṭhaṁ}}: they polished their twenty nails.” } People complained and criticized them, “They’re just like householders who indulge in worldly pleasures!” 

\scrule{“You shouldn’t polish your finger and toe nails. If you do, you commit an offense of wrong conduct. But I allow you to remove dirt.” }

At\marginnote{27.3.1} that time there were monks who had long hair. They told the Buddha. 

“Are\marginnote{27.3.3} you able to shave each other’s heads?” 

“We\marginnote{27.3.4} are.” 

The\marginnote{27.3.5} Buddha then gave a teaching and addressed the monks:\footnote{Following the PTS version. } 

\scrule{“I allow a razor, a whetstone, a razor case, felt, and all barber equipment.” }

At\marginnote{27.4.1} this time the monks from the group of six trimmed their beards, grew their beards long, grew goatees,\footnote{“Goatees” renders \textit{\textsanskrit{golomikaṁ} \textsanskrit{kārāpenti}}. Sp 4.275: \textit{Golomikanti hanukamhi \textsanskrit{dīghaṁ} \textsanskrit{katvā} \textsanskrit{ṭhapitaṁ} \textsanskrit{eḷakamassu} vuccati}, “Making it long at the jaw, a beard like that of a goat is called \textit{golomika}.” } grew sideburns,\footnote{“Sideburns” renders \textit{\textsanskrit{caturassakaṁ} \textsanskrit{kārāpenti}}. Sp 4.275: \textit{Caturassakanti \textsanskrit{catukoṇaṁ}}, “\textit{Caturassaka}: the four-cornered one.” This is further explained by Vmv 4.275: \textit{\textsanskrit{Catukoṇanti} \textsanskrit{yathā} upari \textsanskrit{nalāṭantesu} dve, \textsanskrit{heṭṭhā} hanukapasse dveti \textsanskrit{cattāro} \textsanskrit{koṇā} \textsanskrit{paññāyanti}, \textsanskrit{evaṁ} \textsanskrit{caturassaṁ} \textsanskrit{katvā} \textsanskrit{kappāpanaṁ}}, “\textit{\textsanskrit{Catukoṇa}}: when four corners are seen—two above at the end of the forehead and two below on the side of the jaw—trimming it, having made it four-cornered like this.” } grew circle beards,\footnote{Sp 4.275: \textit{Parimukhanti ure \textsanskrit{lomasaṁharaṇaṁ}}, “\textit{Parimukha}: removing the hair from the chest.” It is a stretch, however, to take \textit{parimukha} as referring to the chest, rather than the face or the mouth. Assuming that beard fashions are fairly timeless, it seems possible that this may refer to a beard around the mouth, that is, a circle beard. } sculpted their chest hair,\footnote{My rendering is based on the reading \textit{\textsanskrit{aḍḍhuraka}}. Sp 4.275: \textit{\textsanskrit{Aḍḍhadukanti} udare \textsanskrit{lomarājiṭṭhapanaṁ}}, “\textit{\textsanskrit{Aḍḍhaduka}}: arrangement of lines of hair on the stomach.” The problem with this is that it is difficult to see how this meaning bears any relationship with \textit{\textsanskrit{aḍḍhaduka}}. In fact, \textit{\textsanskrit{aḍḍhaduka}} is a curious word. On the face of it, it should mean “half a dyad”, but this hardly makes much sense in the context. A major problem with \textit{\textsanskrit{aḍḍhaduka}} is that the reading is highly uncertain. The Buddha \textsanskrit{Jayantī} \textsanskrit{Tipiṭaka} of Sri Lanka gives the alternative \textit{\textsanskrit{aḍḍhuraka}}, SRT \textit{\textsanskrit{aḍḍharuka}}, and the PTS edition \textit{\textsanskrit{aḍḍharūka}}. The first of these is particularly interesting: it almost certainly means “half-chest-ed”. It could well be that this is the basis for the commentarial gloss. The actual commentarial reading (the lemma) may be due to a later “correction”, perhaps based on a change in the Canonical text. If we start with \textit{\textsanskrit{aḍḍhuraka}} as the original form, it is relatively easy to explain the other two forms, which weighs in favor of this reading. \textit{\textsanskrit{Aḍḍharuka}} can be explained as an \textit{a/u} metathesis in \textit{\textsanskrit{aḍḍhuraka}}, which is all too common in oral tradition, especially with obscure words. Once the meaning was lost, \textit{\textsanskrit{aḍḍhaduka}} may have formed as a kind of further corruption, either by accident or through intentional “correction”. We would then see the following pattern of corruption: \textit{\textsanskrit{aḍḍhuraka}} > \textit{\textsanskrit{aḍḍharuka}} > \textit{\textsanskrit{aḍḍhaduka}}. } grew mustaches, and removed the hair from their private parts. People complained and criticized them, “They’re just like householders who indulge in worldly pleasures!” 

\scrule{“You shouldn’t trim your beards, grow your beard long, grow goatees, grow sideburns, grow circle beards, sculpt your chest hair, grow mustaches, or remove the hair from your private parts. If you do, you commit an offense of wrong conduct.” }

A\marginnote{27.4.21} certain monk had a sore on his private parts, and the medicine did not stick. 

\scrule{“I allow you to remove hair from the private parts if you have a disease.” }

At\marginnote{27.5.1} that time the monks from the group of six cut their hair with scissors. People complained and criticized them, “They’re just like householders who indulge in worldly pleasures!” 

\scrule{“You shouldn’t cut your hair with scissors. If you do, you commit an offense of wrong conduct.” }

A\marginnote{27.5.7} monk who had a sore on his head was unable to shave with a razor. 

\scrule{“I allow you to cut your hair with scissors if you have a disease.” }

At\marginnote{27.5.10} this time there were monks who grew their nasal hair long. People complained and criticized them, “They’re just like goblins!” 

\scrule{“You shouldn’t grow your nasal hair long. If you do, you commit an offense of wrong conduct.” }

The\marginnote{27.5.16} monks had their nasal hair removed with small stones and beeswax. Their noses hurt. 

\scrule{“I allow tweezers.” }

The\marginnote{27.5.20} monks from the group of six had their gray hairs removed. People complained and criticized them, “They’re just like householders who indulge in worldly pleasures!” 

\scrule{“You shouldn’t remove gray hairs. If you do, you commit an offense of wrong conduct.” }

On\marginnote{27.6.1} one occasion a monk’s ear was blocked by earwax. 

\scrule{“I allow earpicks.” }

Soon\marginnote{27.6.4} the monks from the group of six used luxurious earpicks made with gold and silver. People complained and criticized them, “They’re just like householders who indulge in worldly pleasures!” 

\scrule{“You shouldn’t use luxurious earpicks. If you do, you commit an offense of wrong conduct. I allow earpicks made of bone, ivory, horn, reed, bamboo, wood, resin, fruit, metal, or shell.”\footnote{“Made of fruit” renders \textit{phalamaya}. \textsanskrit{Khuddasikkhā}-\textsanskrit{purāṇaṭīkā} 185: \textit{\textsanskrit{Āmalakakakkādīhi} \textsanskrit{katā} \textsanskrit{phalamayā}}, “Made of fruit means made from ground emblic myrobalan, etc.” } }

\section*{More regulations on proper conduct and allowable requisites }

At\marginnote{28.1.1} one time the monks from the group of six had amassed a large number of metal and bronze goods. When people walking about the dwellings saw this, they complained and criticized them, “How can the Sakyan monastics amass a large number of metal and bronze goods? They’re just like merchants.” 

\scrule{“You shouldn’t amass a large number of metal and bronze goods. If you do, you commit an offense of wrong conduct.” }

Being\marginnote{28.2.1} afraid of wrongdoing, the monks did not accept ointment boxes, ointment sticks, earpicks, or even metal used for binding.\footnote{“Even metal used for binding” renders \textit{bandhanamatta}. Vmv 4.277: \textit{Bandhanamattanti \textsanskrit{vāsidaṇḍādīnaṁ} \textsanskrit{koṭīsu} \textsanskrit{apātanatthaṁ}, lohehi \textsanskrit{bandhanaṁ}}, “\textit{Bandhanamatta}: bindings of metal used for the purpose of the non-falling off at the top end of the handle of an adz, etc.” } 

\scrule{“I allow ointment boxes, ointment sticks, earpicks, and metal used for binding.” }

On\marginnote{28.2.4} one occasion the monks from the group of six were sitting with their upper robes as a back-and-knee strap.\footnote{This refers to using the upper robe as a support in the \textit{\textsanskrit{pallatthikā}} sitting posture. See Bhikkhu Ñā\textsanskrit{ṇatusita}, “Analysis of the Bhikkhu Pātimokkha”, p. 259 (re. \href{https://suttacentral.net/pli-tv-bu-vb-sk26/en/brahmali\#1.3.1}{Bu Sk 26:1.3.1}). } The panels of the robes were torn apart.\footnote{Sp 3.359: \textit{\textsanskrit{Pattā} \textsanskrit{lujjantīti} mahantesu pattamukhesu \textsanskrit{dinnāni} \textsanskrit{suttāni} \textsanskrit{gaḷanti}, tato \textsanskrit{pattā} lujjanti}, “\textit{\textsanskrit{Pattā} lujjanti}: concerning large openings in the panels, the sewing thread had vanished; therefore the panels were torn apart.” } 

\scrule{“You shouldn’t sit with your upper robe as a back-and-knee strap. If you do, you commit an offense of wrong conduct.” }

There\marginnote{28.2.9} was a sick monk who was not comfortable without a back-and-knee strap.\footnote{The \textit{\textsanskrit{āyoga}} is used as a support for the \textit{\textsanskrit{pallattikā}} sitting posture. } 

\scrule{“I allow back-and-knee straps.” }

The\marginnote{28.2.13} monks thought, “How are the back-and-knee straps to be made?” 

\scrule{“I allow a warp, a reed, a weft, a shuttle, and all weaving equipment.”\footnote{“Warp” renders \textit{tantaka}. Vmv 4.277: \textit{Tantakanti \textsanskrit{āyogavāyanatthaṁ} \textsanskrit{tadākārena} \textsanskrit{pasāritatantaṁ}}, “The stretched thread by means of which the purpose of weaving the back-and-knee strap (is achieved).” It is curious, however, that the loom is missing from this list of weaving equipment. It is possible that \textit{tantaka} is merely another term for \textit{tanta} “a loom”. “Reed” renders \textit{vema}. Commenting on Bu Np 26, Kkh-\textsanskrit{pṭ} says: \textit{Vemanti \textsanskrit{vāyanūpakaraṇo} eko \textsanskrit{daṇḍo}, \textsanskrit{suttaṁ} \textsanskrit{pavesetvā} yena \textsanskrit{ākoṭento} \textsanskrit{ghanabhāvaṁ} \textsanskrit{sampādenti}}, “\textit{Vema}: a weaving instrument consisting of a single rod; having entered the thread, that by which they make it firm by knocking.” Vmv 4.277 clarifies that this concerns the cross-going thread, the weft: \textit{\textsanskrit{Vāyantā} \textsanskrit{tiriyaṁ} \textsanskrit{suttaṁ} \textsanskrit{pavesetvā}} …, “Those weaving, having entered the weft …” According to SED, however, the \textit{vema} is the loom. “Weft” renders \textit{\textsanskrit{kavaṭa}}. I am unable to trace any exegetical information on this word; quite possibly the reading is corrupted. It may be that we should prefer the PTS reading of \textit{\textsanskrit{vaṭa}}, which can mean “string” or “tie”, according to SED. Since the warp seems to be covered by \textit{tantaka}, it does not seem unreasonable to suggest \textit{\textsanskrit{vaṭa}} refers to the “weft”, that is, the cross-thread. “Shuttle” renders \textit{\textsanskrit{salāka}}. The basic meaning of this word is “small stick”. In the Vinaya \textsanskrit{Piṭaka} the word \textit{\textsanskrit{salāka}} is used especially to designate the “voting tickets” that were sometimes used when making decisions in the monastic Sangha. The “shuttle” is a basic weaving equipment and its size would seem to fit well with the size of a \textit{\textsanskrit{salāka}}. However, since there is no exegetical information on this word used in this context, this is no more than a hypothesis. } }

On\marginnote{29.1.1} one occasion a monk went to the village for alms without a belt. His sarong fell off on the street. People shouted out, and he felt humiliated. When he had returned to the monastery, he told the monks what had happened. They in turn told the Buddha, who said: 

\scrule{“You shouldn’t enter an inhabited area without a belt.\footnote{For the rendering “inhabited area” for \textit{\textsanskrit{gāma}}, see Appendix of Technical Terms. } If you do, you commit an offense of wrong conduct. I allow belts.” }

Soon\marginnote{29.2.1} the monks from the group of six wore luxurious belts: belts with multiple strings, belts like the head of a water snake, belts of twisted strings of various colors, belts like ornamental ropes.\footnote{Sp 4.278: \textit{\textsanskrit{Kalābukaṁ} \textsanskrit{nāma} \textsanskrit{bahurajjukaṁ}}, “\textit{\textsanskrit{Kalābuka}}: one having many strings.” Sp 4.278: \textit{\textsanskrit{Deḍḍubhakaṁ} \textsanskrit{nāma} \textsanskrit{udakasappasīsasadisaṁ}}, “\textit{\textsanskrit{Deḍḍubhaka}}: one that is like the head of a water snake.” Sp-\textsanskrit{ṭ} 1.85: \textit{\textsanskrit{Murajañhi} \textsanskrit{nāma} \textsanskrit{nānāvaṇṇehi} suttehi \textsanskrit{murajavaṭṭisaṇṭhānaṁ} \textsanskrit{veṭhetvā} \textsanskrit{kataṁ}}, “\textit{Muraja}: one that is made by the twisting of strings of various colors in the manner of the skin of a \textit{muraja} drum.” Sp 4.275: \textit{\textsanskrit{Maddavīṇaṁ} \textsanskrit{nāma} \textsanskrit{pāmaṅgasaṇṭhānaṁ}}, “\textit{\textsanskrit{Maddavīṇa}}: it looks like an ornamental hanging string.” } People complained and criticized them, “They’re just like householders who indulge in worldly pleasures!” 

\scrule{“You shouldn’t wear luxurious belts. If you do, you commit an offense of wrong conduct. I allow two kinds of belts: belts made from strips of cloth and from pigs’ intestines.” }

The\marginnote{29.2.11} edges of the belts wore away. 

\scrule{“I allow belts of twisted strings of various colors and belts like ornamental ropes.”\footnote{Sp 4.278: \textit{\textsanskrit{Idaṁ} \textsanskrit{dasāsuyeva} \textsanskrit{anuññātaṁ}}, “This is only allowed in regard to the edges.” According to the commentary, in other words, the Buddha is not overturning the prohibition found in the previous rule. } }

The\marginnote{29.2.13} ends of the belts wore away. 

\scrule{“I allow making a loop and a knot.”\footnote{Sp 4.278: \textit{\textsanskrit{Sobhaṇaṁ} \textsanskrit{nāma} \textsanskrit{veṭhetvā} \textsanskrit{mukhavaṭṭisibbanaṁ}}, “\textit{\textsanskrit{Sobhaṇa}}: having twisted, sewing along the edge of the opening.” Sp 4.278: \textit{\textsanskrit{Guṇakaṁ} \textsanskrit{nāma} \textsanskrit{mudiṅgasaṇṭhānena} \textsanskrit{sibbanaṁ}}, “\textit{\textsanskrit{Guṇaka}}: sewing with the appearance of a \textit{\textsanskrit{mudiṅga}} drum”, which according to Sp-\textsanskrit{ṭ} 4.278 means: \textit{\textsanskrit{Mudiṅgasaṇṭhānenāti} \textsanskrit{varakasīsākārena}}, “\textit{\textsanskrit{Mudiṅgasaṇṭhānena}}: with the characteristic of the head of the \textit{varaka} bean.” It is not clear what is meant. Perhaps the point is that one end of the belt should form a loop and the other a knot, with the knot fitting into the loop. } }

The\marginnote{29.2.15} loop at the end wore away.\footnote{Sp 4.275: \textit{Pavanantoti \textsanskrit{pāsanto} vuccati}, “The loop-end is called \textit{pavananta}.” } 

\scrule{“I allow buckles.” }

Soon\marginnote{29.2.17} the monks from the group of six wore luxurious buckles made with gold and silver. People complained and criticized them, “They’re just like householders who indulge in worldly pleasures!” 

\scrule{“You shouldn’t wear luxurious buckles. If you do, you commit an offense of wrong conduct. I allow buckles made of bone, ivory, horn, reed, bamboo, wood, resin, fruit, metal, shell, and string.” }

On\marginnote{29.3.1} one occasion Venerable Ānanda robed up in light upper robes and went to the village for alms. A whirlwind lifted up his robes. When he had returned to the monastery, he told the monks what had happened. They in turn told the Buddha. 

\scrule{“I allow toggles and loops.” }

Soon\marginnote{29.3.6} the monks from the group of six wore luxurious toggles made with gold and silver. People complained and criticized them, “They’re just like householders who indulge in worldly pleasures!” 

\scrule{“You shouldn’t wear luxurious toggles. If you do, you commit an offense of wrong conduct. I allow toggles made of bone, ivory, horn, reed, bamboo, wood, resin, fruit, metal, shell, and string.” }

When\marginnote{29.3.13} the monks fastened toggles and loops to their robes, they caused the robes to wear. 

\scrule{“I allow toggle shields and loop shields.” }

They\marginnote{29.3.17} fastened the toggle shields and the loop shields on the edge of the robe. The corners of the robe separated.\footnote{Reading \textit{\textsanskrit{koṇa}} with SRT, as against \textit{\textsanskrit{koṭṭa}} of MS. } 

\scrule{“I allow you to fasten the toggle shields at the edge and the loop shields twelve or thirteen centimeters in from the edge.”\footnote{That is, seven or eight fingerbreadths. For a discussion of the \textit{\textsanskrit{aṅgula}}, see \textit{sugata} in Appendix of Technical Terms. } }

At\marginnote{29.4.1} this time the monks from the group of six wore their sarongs like householders—in the elephant-trunk style, the fish-tail style, the four-corner style, the palm-leaf style, and the hundred-fold style. People complained and criticized them, “They’re just like householders who indulge in worldly pleasures!” 

\scrule{“You shouldn’t wear your sarong like householders—in the elephant-trunk style, the fish-tail style, the four-corner style, the palm-leaf style, or the hundred-fold style. If you do, you commit an offense of wrong conduct.” }

The\marginnote{29.4.9} monks from the group of six wore their upper robes like householders.\footnote{The context provided by the previous rule on sarongs suggests that this rule concerns the style of dressing rather than the kind of robe used. } People complained and criticized them, “They’re just like householders who indulge in worldly pleasures!” 

\scrule{“You shouldn’t wear your upper robe like householders. If you do, you commit an offense of wrong conduct.” }

The\marginnote{29.5.1} monks from the group of six wore their sarongs like loin cloths. People complained and criticized them, “They’re just like the king’s porters!” 

\scrule{“You shouldn’t wear your sarong like a loin cloth. If you do, you commit an offense of wrong conduct.” }

At\marginnote{30.1.1} that time the monks from the group of six used carrying poles with loads on both ends. People complained and criticized them, “They’re just like the king’s porters!” 

\scrule{“You shouldn’t use a carrying pole with loads on both ends. If you do, you commit an offense of wrong conduct. I allow a carrying pole with a load on one end, a two-person carrying pole with a load in the middle, loads for the head, loads for the shoulder, loads for the hip, and hanging loads.” }

At\marginnote{31.1.1} that time there were monks who did not clean their teeth. As a result, they had bad breath. They told the Buddha. 

“There\marginnote{31.1.4} are these five drawbacks to not cleaning your teeth:\footnote{This is parallel to \href{https://suttacentral.net/an5.208/en/brahmali\#0.3}{AN 5.208:0.3}. } it’s bad for your eyes; you get bad breath; the taste buds aren’t cleansed; bile and phlegm cover the food; you don’t enjoy the food.\footnote{“It’s bad for your eyes” renders \textit{acakkhussa}. Sp 4.282: \textit{Acakkhussanti \textsanskrit{cakkhūnaṁ} \textsanskrit{hitaṁ} na hoti; \textsanskrit{parihāniṁ} janeti}, “\textit{Acakkhussa}: it is not beneficial for the eyes; it causes their decline.” Tooth infections can apparently affect the eyes. } 

There\marginnote{31.1.7} are these five benefits of cleaning your teeth: it’s good for your eyes; you don’t get bad breath; your taste buds are cleansed; bile and phlegm don’t cover the food; you enjoy the food. 

\scrule{I allow tooth cleaners.” }

The\marginnote{31.2.1} monks from the group of six used long tooth cleaners, which they even used to smack the novices. 

\scrule{“You shouldn’t use long tooth cleaners. If you do, you commit an offense of wrong conduct. I allow tooth cleaners that are at most thirteen centimeters long. And you shouldn’t use them to smack the novice monks.\footnote{That is, eight fingerbreadths. For a discussion of the \textit{\textsanskrit{aṅgula}}, see \textit{sugata} in Appendix of Technical Terms. } If you do, you commit an offense of wrong conduct.” }

On\marginnote{31.2.7} one occasion a monk used a tooth cleaner that was too short and it got stuck in his throat. 

\scrule{“You shouldn’t use tooth cleaners that are too short. If you do, you commit an offense of wrong conduct. You shouldn’t use tooth cleaners shorter than seven centimeters.” }

On\marginnote{32.1.1} one occasion the monks from the group of six set fire to a forest. People complained and criticized them, “They’re just like land clearers.” 

\scrule{“You shouldn’t set fire to a forest. If you do, you commit an offense of wrong conduct.” }

On\marginnote{32.1.7} one occasion the dwellings were overgrown with grass. There was a forest fire and the dwellings burned down. Being afraid of wrongdoing, the monks had not lit a counterfire for protection. 

\scrule{“When there is a forest fire, I allow you to light a counterfire for protection.” }

At\marginnote{32.2.1} that time the monks from the group of six climbed trees and then jumped between them. People complained and criticized them, “They’re just like monkeys!” 

\scrule{“You shouldn’t climb trees. If you do, you commit an offense of wrong conduct.” }

On\marginnote{32.2.7} one occasion a certain monk was walking on a road through the Kosalan country on his way to \textsanskrit{Sāvatthī} when he was blocked by an elephant. He quickly went up to a tree, but being afraid of wrongdoing, he did not climb it. Soon the elephant went away. When he arrived at \textsanskrit{Sāvatthī}, he told the monks what had happened. They in turn told the Buddha. 

\scrule{“I allow you to climb a tree to the height of a man if there’s something to be done, or as far as you need if there’s an emergency.”\footnote{For the rendering “emergency” for \textit{\textsanskrit{āpadāsu}}, see Appendix of Technical Terms. } }

\section*{Teaching, etc. }

At\marginnote{33.1.1} this time there were two monks called \textsanskrit{Yameḷa} and \textsanskrit{Kekuṭa}, brothers born into a brahmin family, who were well-spoken and had good voices. They went to the Buddha, bowed, sat down, and said, “Sir, the monks now have a variety of names and come from a variety of families, castes, and households. They corrupt the word of the Buddha by using their own expressions.\footnote{“Their own expressions” renders \textit{\textsanskrit{sakāya} \textsanskrit{niruttiyā}}. \textit{Nirutti} is found at \href{https://suttacentral.net/pli-tv-bu-vb-pj2/en/brahmali\#7.6.9}{Bu Pj 2:7.6.9}, \href{https://suttacentral.net/pli-tv-bu-vb-pj2/en/brahmali\#7.6.19}{Bu Pj 2:7.6.19}, and \href{https://suttacentral.net/pli-tv-bu-vb-pj2/en/brahmali\#7.6.30}{Bu Pj 2:7.6.30} where its contextual meaning must be “expression” or “manner of speaking” rather than “language”. If the word is used in the same way here, it follows that \textit{\textsanskrit{sakāya}} must refer to the particular expressions used by the monks. Using the Buddha’s own way of expression would not have been a problem. For a scholarly discussion of \textit{nirutti} that supports this view, see Bryan Levman, “\textsanskrit{Sakāya} \textsanskrit{niruttiyā} revisited”, BEI 26–27 (2008–2009): 33–51. } Now we could give metrical form to the word of the Buddha.” 

The\marginnote{33.1.7} Buddha rebuked them, “Foolish men, how can you suggest such a thing? This will affect people’s confidence …” After rebuking them … the Buddha gave a teaching and addressed the monks: 

\scrule{“You shouldn’t give metrical form to the word of the Buddha. If you do, you commit an offense of wrong conduct. You should learn the word of the Buddha using its own expressions.”\footnote{Sp 4.275: \textit{\textsanskrit{Sakāya} \textsanskrit{niruttiyāti} ettha \textsanskrit{sakā} nirutti \textsanskrit{nāma} \textsanskrit{sammāsambuddhena} \textsanskrit{vuttappakāro} \textsanskrit{māgadhiko} \textsanskrit{vohāro}}, “\textit{\textsanskrit{Sakāya} \textsanskrit{niruttiyā}}: here ‘own expression’ means the \textsanskrit{Māghadhan} language as spoken by the fully Awakened One.” This gloss refers to this particular usage of this expression, and it does not necessarily follow that it also glosses the same expression as used immediately above. } }

At\marginnote{33.2.1} that time the monks from the group of six were studying cosmological theory.\footnote{“Cosmological theory” renders \textit{\textsanskrit{lokāyata}}. At \href{https://suttacentral.net/sn12.48/en/brahmali\#2.2}{SN 12.48:2.2} we have: \textit{\textsanskrit{Sabbamatthī}’ti kho, \textsanskrit{brāhmaṇa}, \textsanskrit{jeṭṭhametaṁ} \textsanskrit{lokāyataṁ} … \textsanskrit{Sabbaṁ} \textsanskrit{natthī}’ti kho, \textsanskrit{brāhmaṇa}, \textsanskrit{dutiyametaṁ} \textsanskrit{lokāyataṁ} … Sabbamekatta’nti kho, \textsanskrit{brāhmaṇa}, \textsanskrit{tatiyametaṁ} \textsanskrit{lokāyataṁ} … \textsanskrit{Sabbaṁ} puthutta’nti kho, \textsanskrit{brāhmaṇa}, \textsanskrit{catutthametaṁ} \textsanskrit{lokāyataṁ}}, “‘Everything exists’ is the oldest cosmological theory … ‘Nothing exists’ is the second oldest cosmological theory … ‘Everything is one’ is the third oldest cosmological theory … ‘Everything is diversity’ is the fourth oldest cosmological theory.” } People complained and criticized them, “They’re just like householders who indulge in worldly pleasures!” The monks heard the complaints of those lay people and told the Buddha. 

“Is\marginnote{33.2.6} there any growth and fulfillment on this spiritual path, monks, for one who sees cosmological theory as the essence?”\footnote{For an explanation of the rendering “spiritual path” for \textit{dhammavinaya}, see Appendix of Technical Terms. } 

“No,\marginnote{33.2.7} sir.” 

“Would\marginnote{33.2.8} anyone who sees this spiritual path as the essence learn cosmological theory?” 

“No.”\marginnote{33.2.9} 

\scrule{“You shouldn’t learn cosmological theory. If you do, you commit an offense of wrong conduct.” }

The\marginnote{33.2.12} monks from the group of six taught cosmological theory. People complained and criticized them, “They’re just like householders who indulge in worldly pleasures!” They told the Buddha. 

\scrule{“You shouldn’t teach cosmological theory. If you do, you commit an offense of wrong conduct.” }

The\marginnote{33.2.18} monks from the group of six studied worldly subjects.\footnote{At \href{https://suttacentral.net/dn1/en/brahmali\#1.21.1}{DN 1:1.21.1} we find a list of such worldly subjects, most prominently prognostication and the performing of rituals. At \href{https://suttacentral.net/pli-tv-bi-vb-pc49/en/brahmali\#2.1.5}{Bi Pc 49:2.1.5} “worldly subjects” is defined more broadly as anything outside of the Buddha’s teachings. } People complained and criticized them, “They’re just like householders who indulge in worldly pleasures!” The monks heard the complaints of those lay people and told the Buddha. 

\scrule{“You shouldn’t study worldly subjects. If you do, you commit an offense of wrong conduct.” }

The\marginnote{33.2.24} monks from the group of six taught worldly subjects. People complained and criticized them, “They’re just like householders who indulge in worldly pleasures!” 

\scrule{“You shouldn’t teach worldly subjects. If you do, you commit an offense of wrong conduct.” }

On\marginnote{33.3.1} one occasion the Buddha sneezed while teaching a large gathering. The monks made an uproar, saying, “May you live long, venerable sir!” Because of the noise, the teaching was interrupted. The Buddha said to the monks: 

“If\marginnote{33.3.8} you say, ‘May you live long!’ to one who sneezes, will they live or die because of that?” 

“No,\marginnote{33.3.9} sir.” 

\scrule{“You shouldn’t say, ‘May you live long!’ to one who sneezes. If you do, you commit an offense of wrong conduct.” }

At\marginnote{33.3.12} that time when monks sneezed, people said, “May you live long, venerable!” Being afraid of wrongdoing, the monks did not respond. People complained and criticized them, “How can the Sakyan monastics not respond when spoken to like this?” They told the Buddha. 

\scrule{“Monks, householders want blessings. When householders say, ‘May you live long!’ I allow you to respond with similar words.” }

On\marginnote{34.1.1} one occasion when the Buddha was seated teaching a large gathering, there was a monk who had eaten garlic. Not to annoy the other monks, he was sitting at a distance. The Buddha saw him and asked the monks why he was sitting there. The monks told him and the Buddha said, “Monks, should one eat anything that would stop one from hearing a teaching such as this?” 

“No,\marginnote{34.1.14} sir.” 

\scrule{“You shouldn’t eat garlic. If you do, you commit an offense of wrong conduct.” }

On\marginnote{34.2.1} one occasion Venerable \textsanskrit{Sāriputta} had a stomachache. Venerable \textsanskrit{Mahāmoggallāna} went to him and said, “When you had a stomachache in the past, \textsanskrit{Sāriputta}, what made you better?” 

“Garlic.”\marginnote{34.2.4} They told the Buddha. 

\scrule{“I allow you to eat garlic if you’re sick.” }

\section*{Restrooms }

At\marginnote{35.1.1} that time there were monks who urinated here and there in the monastery. The monastery became filthy. 

\scrule{“You should urinate in one location.” }

The\marginnote{35.1.5} monastery became smelly. 

\scrule{“I allow urine-collection pots.” }

It\marginnote{35.1.7} was painful to sit there while urinating. 

\scrule{“I allow foot stands for urinating.”\footnote{\textit{\textsanskrit{Pāduka}}, normally rendered as “shoe”, I here translate as “foot-stands”. This seems to be required from the context. Sp 4.290: \textit{\textsanskrit{Passāvapādukanti} ettha \textsanskrit{pādukā} \textsanskrit{iṭṭhakāhipi} \textsanskrit{silāhipi} \textsanskrit{dārūhipi} \textsanskrit{kātuṁ} \textsanskrit{vaṭṭati}. \textsanskrit{Vaccapādukāyapi} eseva nayo}, “A \textit{\textsanskrit{pāduka}} for urinating: here it is allowable to make a \textit{\textsanskrit{pāduka}} of bricks, stone, or wood. The same method also applies for \textit{\textsanskrit{pāduka}} for defecating.” These fixtures seem more likely to be platforms or stands than shoes in any ordinary sense. } }

The\marginnote{35.1.9} foot stands were unenclosed. The monks were embarrassed to urinate there. 

\scrule{“I allow three kinds of encircling walls: walls of brick, stone, and wood.” }

Not\marginnote{35.1.13} being covered, the urine-collection pots were smelly. 

\scrule{“I allow lids.” }

At\marginnote{35.2.1} that time there were monks who defecated here and there in the monastery. The monastery became filthy. 

\scrule{“You should defecate in one location.” }

The\marginnote{35.2.5} monastery became smelly. 

\scrule{“I allow cesspits.” }

The\marginnote{35.2.7} edge of the cesspit collapsed. 

\scrule{“I allow you to construct three kinds of foundations: foundations of brick, stone, and wood.” }

The\marginnote{35.2.10} cesspit was situated at a low point. It was flooded. 

\scrule{“I allow you to raise the base.” }

The\marginnote{35.2.12} mound collapsed. 

\scrule{“I allow you to construct three kinds of raised foundations: raised foundations of brick, stone, and wood.” }

It\marginnote{35.2.15} was difficult to get up to the cesspit. 

\scrule{“I allow three kinds of stairs: stairs of brick, stone, and wood.” }

People\marginnote{35.2.18} fell down while climbing the stairs. 

\scrule{“I allow rails.” }

When\marginnote{35.2.20} seated on the edge to defecate, they fell down. 

\scrule{“I allow you to lay a floor with a gap in the middle for defecating.” }

It\marginnote{35.2.22} was painful to sit there while defecating. 

\scrule{“I allow foot stands for defecating.” }

They\marginnote{35.3.1} urinated outside the cesspit. 

\scrule{“I allow urinals.” }

There\marginnote{35.3.3} were no wiping sticks. 

\scrule{“I allow wiping sticks.” }

There\marginnote{35.3.5} was no container for the wiping sticks. 

\scrule{“I allow containers for wiping sticks.” }

Not\marginnote{35.3.7} being covered, the cesspit was smelly. 

\scrule{“I allow lids.” }

Because\marginnote{35.3.9} they were defecating outdoors, the monks were troubled by the cold and the heat. 

\scrule{“I allow restrooms.” }

The\marginnote{35.3.11} restrooms didn’t have doors. 

\scrule{“I allow doors, door frames, lower hinges, upper hinges, door jambs, bolt sockets, bolts, latches, keyholes, door-pulling holes, and door-pulling ropes.” }

Grass\marginnote{35.3.13} and dust fell into the restrooms. 

\scrule{“I allow you to firm up the structure and then to plaster it inside and outside, including: treating with white color, black color, and red ocher; making garland patterns, creeper patterns, shark-teeth patterns, and the fivefold pattern; putting up bamboo robe racks and clotheslines.” }

On\marginnote{35.3.16} one occasion a monk who was weak from old age fell over as he was getting up after defecating. 

\scrule{“I allow suspended ropes to hold onto.” }

The\marginnote{35.3.19} restrooms were unenclosed. 

\scrule{“I allow three kinds of encircling walls: walls of brick, stone, and wood.” }

There\marginnote{35.4.1} were no gatehouses. 

\scrule{“I allow gatehouses.” }

The\marginnote{35.4.3} gatehouses didn’t have doors. 

\scrule{“I allow doors, door frames, lower hinges, upper hinges, door jambs, bolt sockets, bolts, latches, keyholes, door-pulling holes, and door-pulling ropes.” }

Grass\marginnote{35.4.5} and dust fell into the gatehouses. 

\scrule{“I allow you to firm up the structure and then plaster it inside and outside, including: treating with white color, black color, and red ocher; making garland patterns, creeper patterns, shark-teeth patterns, and the fivefold pattern.” }

The\marginnote{35.4.8} yards were muddy. 

\scrule{“I allow you to cover them with gravel.” }

They\marginnote{35.4.10} were unable to do it. 

\scrule{“I allow you to lay paving stones.” }

The\marginnote{35.4.12} water remained. 

\scrule{“I allow water drains.” }

There\marginnote{35.4.14} were no restroom ablution pots. 

\scrule{“I allow restroom ablution pots.” }

There\marginnote{35.4.16} were no scoops for the ablution pots. 

\scrule{“I allow scoops for the ablution pots.” }

It\marginnote{35.4.18} was painful to sit there while washing. 

\scrule{“I allow ablution foot stands.” }

The\marginnote{35.4.20} foot stands were unenclosed. The monks were embarrassed to wash there. 

\scrule{“I allow three kinds of encircling walls: walls of brick, stone, and wood.” }

The\marginnote{35.4.23} restroom ablution pots were not covered. Grass, dust, and dirt fell into them. 

\scrule{“I allow lids.” }

\section*{Even more regulations on proper conduct and allowable requisites }

At\marginnote{36.1.1} one time the monks from the group of six were misbehaving in many ways. 

They\marginnote{36.1.2} planted flowering trees, watered and plucked them, and then tied the flowers together. They made the flowers into garlands, garlands with stalks on one side and garlands with stalks on both sides. They made flower arrangements, wreaths, ornaments for the head, ornaments for the ears, and ornaments for the chest. And they had others do the same. They then took these things, or sent them, to the women, the daughters, the girls, the daughters-in-law, and the female slaves of respectable families. 

They\marginnote{36.1.4} ate from the same plates as these women and drank from the same vessels. They sat on the same seats as them, and they lay down on the same beds, on the same sheets, under the same covers, and both on the same sheets and under the same covers. They ate at the wrong time, drank alcohol, and wore garlands, perfumes, and cosmetics. They danced, sang, played instruments, and performed. While the women were dancing, singing, playing instruments, and performing, so would they. 

They\marginnote{36.1.7} played various games: eight-row checkers, ten-row checkers, imaginary checkers, hopscotch, pick-up-sticks, board games, tip-cat, painting with the hand, dice, leaf flutes, toy plows, somersaults, pinwheels, toy measures, toy carriages, toy bows, guessing from syllables, thought guessing, mimicking deformities. 

They\marginnote{36.1.8} trained in elephant riding, in horsemanship, in carriage riding, in archery, in swordsmanship. And they ran in front of elephants, horses, and carriages, and they ran backward and forward. They whistled, clapped their hands, wrestled, and boxed. They spread their outer robes on a stage and said to the dancing girls, “Dance here, Sister,” and they made gestures of approval. And they misbehaved in a variety of ways. 

They\marginnote{36.1.16} told the Buddha. Soon afterwards he gave a teaching and addressed the monks: 

\scrule{“You shouldn’t engage in various kinds of misbehavior If you do, you should be dealt with according to the rule.”\footnote{That is, \href{https://suttacentral.net/pli-tv-bu-vb-ss13/en/brahmali\#1.8.10.1}{Bu Ss 13:1.8.10.1}. } }

At\marginnote{37.1.1} the time of Venerable Uruvelakassapa’s going forth, the Sangha was offered a large number of iron, wooden, and ceramic goods. The monks thought, “Which iron, wooden, and ceramic goods have been allowed by the Buddha, and which not?” They told the Buddha. The Buddha then gave a teaching and addressed the monks: 

\scrule{“I allow all iron goods except weapons; all wooden goods except high couches, luxurious couches, wooden almsbowls, and wooden shoes; and all ceramic goods except ceramic foot scrubbers and clay huts.”\footnote{Reading \textit{\textsanskrit{kumbhakārika}} as an adjective, it means “that which belongs to the potter” or “of the potter”, that is, a clay hut. Sp 4.293: \textit{\textsanskrit{Kumbhakārikañcāti} dhaniyasseva \textsanskrit{sabbamattikāmayakuṭi} vuccati}, “Dhaniya’s hut made entirely of clay is called ‘that which belongs to the potter’.” } }

\scendsutta{The fifth chapter on minor topics is finished. }

\scuddanaintro{This is the summary: }

\begin{scuddana}%
“On\marginnote{37.1.11} a tree, and on a post, and against a wall, \\
On a rubbing board, rubbing hand, and with a string; \\
Massage, scrubber, itchy, \\
And old age, ordinary hand massage. 

And\marginnote{37.1.15} earrings, hanging strings, \\
shouldn’t wear a necklace; \\
Hips, bangles, armlets, \\
Bracelets, rings. 

Long,\marginnote{37.1.19} brush, comb, hands, \\
Beeswax, water and oil; \\
Sores in a mirror or bowl of water, \\
Ointment, creams, powder. 

They\marginnote{37.1.23} applied, and body cosmetics, \\
Facial cosmetics, both; \\
Eye disease, and hilltop, \\
drawn-out voice, outside. 

Mangoes,\marginnote{37.1.27} pieces, with whole, \\
Snake, and cut off, sandal; \\
Luxurious, bowl bottoms, \\
Gold, thick, marks. 

Colorful,\marginnote{37.1.31} stained, smelly, \\
In the heat, they broke, on a bench; \\
Ledge, straw, cloth, \\
Platform, and container. 

Bag,\marginnote{37.1.35} and shoulder strap, \\
So a string for fastening; \\
From a peg, and on a bed, and on a bench, \\
In the lap, on a sunshade, opening. 

Gourd,\marginnote{37.1.39} waterpots, skull, \\
Chewed food remnants, trash can; \\
Tore, handle, gold, \\
Feather, and piece, cylinder. 

Yeast,\marginnote{37.1.43} flour, and stone powder, \\
Beeswax, case; \\
Deformed corners, tied down, uneven, \\
Ground, deteriorating, and not right. 

Ruler,\marginnote{37.1.47} and guide line, \\
Dirty, wet, sandals; \\
Finger, and thimble, \\
Small bowl, bag, strap. 

Outside,\marginnote{37.1.51} low base, \\
And also the mound, they had difficulty; \\
Fell down, grass and dust, \\
Plaster inside and outside. 

White,\marginnote{37.1.55} and black color, \\
And treating with red ocher; \\
Making a garland pattern, and a creeper pattern, \\
A shark-teeth pattern, decoration. 

Bamboo\marginnote{37.1.59} robe rack, and clothesline, \\
Did the Leader allow; \\
Abandoned and left, \\
The frame broke. 

Unfolded,\marginnote{37.1.63} and against a wall, \\
Taking their bowls they went; \\
Bag, and fastening string, \\
And bound the sandals. 

And\marginnote{37.1.67} sandal bag, \\
And shoulder strap, string; \\
Unallowable water while traveling, \\
Water filter, cloth. 

Water\marginnote{37.1.71} strainer, two monks, \\
The Sage went to \textsanskrit{Vesālī}; \\
Wooden frame, spreading there, \\
He allowed a filter. 

With\marginnote{37.1.75} mosquitoes, with fine, \\
And often sick, \textsanskrit{Jīvaka}; \\
Walking-meditation paths, sauna, \\
On uneven, low base. 

Three\marginnote{37.1.79} foundations, they had difficulty, \\
Stairs, rails, railings; \\
Outside, grass and dust, \\
Plastered inside and outside. 

White,\marginnote{37.1.83} and black color, \\
And treating with red ocher; \\
Making a garland pattern, and a creeper pattern, \\
A shark-teeth pattern, decoration. 

Bamboo,\marginnote{37.1.87} and clothesline, \\
And should raise the base; \\
Mound, and stairs, rails, \\
Door, door frame. 

Lower\marginnote{37.1.91} hinge, upper hinge, \\
Door jamb, bolt socket; \\
Bolt, latch, key hole, \\
And door-pulling, rope. 

Encircling\marginnote{37.1.95} trench, and flue, \\
And in the middle, clay for the face; \\
Trough, smelly, scorched, \\
Water place, scoop. 

And\marginnote{37.1.99} did not sweat, muddy, \\
Wash, should make a drain; \\
And bench, gatehouse, making, \\
Gravel, stones, drain. 

Naked,\marginnote{37.1.103} on the ground, raining, \\
Three coverings there; \\
Well, collapsed, low, \\
With creepers, belt. 

Well-sweep,\marginnote{37.1.107} pulley, wheel, \\
Many vessels broke; \\
Iron, wood, hide, \\
House, grass, and cover. 

Trough,\marginnote{37.1.111} disposal area, wall, \\
Muddy, and with a drain; \\
Cold, lotus bathing tank, \\
And stagnant, pointed roof. 

Four\marginnote{37.1.115} months, and they slept, \\
And piece of felt, should not determine; \\
Heating, stand, \\
They ate from one, they lay down. 

\textsanskrit{Vaḍḍha},\marginnote{37.1.119} Bodhi, he did not step on, \\
Waterpot, ceramic foot scrubber, broom; \\
Stone, and pebbles, \\
Pumice foot scrubbers. 

Standard\marginnote{37.1.123} fan, palm-leaf fan, \\
And mosquito, yak-tail; \\
Sunshade, and without, in a monastery, \\
Three with agreement on carrying net. 

Regurgitator,\marginnote{37.1.127} rice, long nails, \\
Cutting, the fingers hurt; \\
Bled, and measure, \\
Twenty, long haired. 

Razor,\marginnote{37.1.131} stone, case, \\
Felt, barber equipment; \\
They trimmed beards, grew them long, \\
Goatee, sideburns. 

Circle\marginnote{37.1.135} beard, and chest hair, \\
Mustache, would remove from private parts; \\
Disease, sore with scissors, \\
Long, and small stones. 

Gray,\marginnote{37.1.139} blocked, luxurious, \\
Metal goods, with ointment box; \\
And clasping the knees, back-and-knee strap, \\
Loom, shuttle, belt. 

Multiple\marginnote{37.1.143} string belt, water snake head belt, \\
Belts of twisted strings of various colors, belts like ornamental ropes; \\
Strips of cloth, and pigs’ intestines, \\
Edges, twisted strings of various colors, ornamental ropes; \\
End, loop, and knot, \\
Also the loop at the end wore away. 

Toggles,\marginnote{37.1.149} and luxurious, \\
Also should insert a shield at the edge; \\
Sarongs like householders, elephant trunk, \\
Fish style, four-corner style. 

Palm\marginnote{37.1.153} leaf, hundred fold, \\
Wearing upper robes like householders; \\
Loin cloths, carrying poles with loads on both ends, \\
Tooth cleaner, smacking. 

Stuck\marginnote{37.1.157} in the throat, and forest, \\
Counterfire, tree, with elephant; \\
\textsanskrit{Yameḷa}, cosmological theory, \\
They learned, they taught. 

Worldly\marginnote{37.1.161} talk, knowledge, \\
He sneezed, blessing, and he ate; \\
Stomachache, and became filthy, \\
Smelly, painful, foot stands. 

They\marginnote{37.1.165} were embarrassed, covered, smelly, \\
And they did it here and there; \\
Smelly, cesspit, collapsed, \\
Raised base, and with foundation. 

Stairs,\marginnote{37.1.169} rails, \\
On the edge, and painful, foot-stands; \\
Outside, urinal, and sticks, \\
And container, uncovered. 

Restroom,\marginnote{37.1.173} and door, \\
And just the door frame; \\
Lower hinge, upper hinge, \\
And door jamb, bolt socket. 

Bolt,\marginnote{37.1.177} latch, key hole, \\
And just a door-pulling hole; \\
Rope, plastered inside and outside, \\
And white color, black. 

Making\marginnote{37.1.181} a garland pattern, a creeper pattern, \\
A shark-teeth pattern, the fivefold pattern; \\
Bamboo robe rack, and rope, \\
Weak from old age, wall. 

And\marginnote{37.1.185} so also just a gatehouse, \\
Gravel, paving stones; \\
Remained, drain, \\
And also pot, scoop. 

Painful,\marginnote{37.1.189} embarrassed, lid, \\
And they misbehaved; \\
He allowed iron goods, \\
Except weapons. 

Except\marginnote{37.1.193} high couches and luxurious couches, \\
And wooden almsbowls, wooden shoes; \\
All wooden goods, \\
The Great Sage allowed. 

Ceramic\marginnote{37.1.197} foot scrubbers, and clay huts, \\
The Buddha having excepted; \\
Also all clay goods, \\
The Compassionate One allowed. 

The\marginnote{37.1.201} details of the topics, \\
If the same as the preceding, \\
Is also found in brief in the summary verses, \\
For the purpose of guiding those who have understood it. 

Thus\marginnote{37.1.205} there are one hundred and ten topics \\
In the chapter on minor topics in the Monastic Law. \\
Indeed, the true Teaching will be long lived, \\
And good people will be supported. 

A\marginnote{37.1.209} well-trained expert in the Monastic Law, \\
A good person intent on what’s beneficial, \\
A wise one, lighting a lamp—\\
This is a learned one worthy of homage.” 

%
\end{scuddana}

\scendsutta{The chapter on minor topics is finished. }

%
\chapter*{{\suttatitleacronym Kd 16}{\suttatitletranslation The chapter on resting places }{\suttatitleroot Senāsanakkhandhaka}}
\addcontentsline{toc}{chapter}{\tocacronym{Kd 16} \toctranslation{The chapter on resting places } \tocroot{Senāsanakkhandhaka}}
\markboth{The chapter on resting places }{Senāsanakkhandhaka}
\extramarks{Kd 16}{Kd 16}

\section*{The first section for recitation }

\subsection*{1. The allowance for dwellings }

At\marginnote{1.1.1} one time the Buddha was staying at \textsanskrit{Rājagaha} in the Bamboo Grove, the squirrel sanctuary. At this time the Buddha had not yet allowed dwellings.\footnote{In the broadest sense \textit{\textsanskrit{senāsana}} means “resting place”, which includes everything from the foot of a tree to furniture and huts. The current chapter focuses on furniture and dwellings. Here, however, the meaning is huts or dwellings. For a further discussion of \textit{\textsanskrit{senāsana}}, see Appendix of Technical Terms. } As a result, the monks stayed here and there: in the wilderness, at the foot of a tree, on a hill, in a gorge, in a hillside cave, in a charnel ground, in the forest, in the open, on a heap of straw. Early in the morning, they would emerge from those places. They were pleasing in their conduct: in going out and coming back, in looking ahead and looking aside, in bending and stretching their arms. Their eyes were lowered, and they were perfect in deportment. 

One\marginnote{1.2.1} morning a wealthy merchant of \textsanskrit{Rājagaha} was going to a park when he saw those monks. Being inspired, he approached them and said, “If I build dwellings, sirs, would you stay in them?” 

“The\marginnote{1.2.7} Buddha hasn’t allowed dwellings.” 

“Well\marginnote{1.2.8} then, please ask the Buddha and let me know his response.” 

“Yes.”\marginnote{1.2.9} 

Those\marginnote{1.2.10} monks then went to the Buddha, bowed, sat down, and said, “Sir, a wealthy merchant in \textsanskrit{Rājagaha} wants to build dwellings. What should we do?” Soon afterwards the Buddha gave a teaching and addressed the monks: 

\scrule{“I allow five kinds of shelters:\footnote{Sp-\textsanskrit{ṭ} 4.294: \textit{\textsanskrit{Pañca} \textsanskrit{leṇānīti} \textsanskrit{pañca} \textsanskrit{līyanaṭṭhānāni}. \textsanskrit{Nilīyanti} ettha \textsanskrit{bhikkhūti} \textsanskrit{leṇāni}}, “The five \textit{\textsanskrit{leṇas}}: the five places for resting. \textit{\textsanskrit{Leṇas}} are where monks hide away.” } dwellings, three kinds of stilt houses, and caves.”\footnote{For an explanation of the renderings “stilt house” and “cave” for \textit{\textsanskrit{pāsāda}} and \textit{\textsanskrit{guhā}} respectively, see Appendix of Technical Terms. Apart from \textit{\textsanskrit{vihāra}}, “dwellings”, and \textit{\textsanskrit{guhā}}, “caves”, the Pali mentions three kinds of buildings, the \textit{\textsanskrit{aḍḍhayoga}}, the \textit{\textsanskrit{pāsāda}}, and the \textit{hammiya}, all of which, according to the commentaries, are different kinds of \textit{\textsanskrit{pāsāda}}, “stilt houses”. Rather than try to differentiate between these buildings, which is unlikely to be useful from a practical perspective, I have instead grouped them together as “stilt houses”. Here is what the commentaries have to say. Sp 4.294: \textit{\textsanskrit{Aḍḍhayogoti} \textsanskrit{supaṇṇavaṅkagehaṁ}}, “An \textit{\textsanskrit{aḍḍhayoga}} is a house bent like a \textit{\textsanskrit{supaṇṇa}}.” Sp-\textsanskrit{ṭ} 4.294 clarifies: \textit{\textsanskrit{Supaṇṇavaṅkagehanti} \textsanskrit{garuḷapakkhasaṇṭhānena} \textsanskrit{katagehaṁ}}, “\textit{\textsanskrit{Supaṇṇavaṅkageha}}: a house made in the shape of the wings of a \textit{\textsanskrit{garuḷa}}.” A \textit{\textsanskrit{garuḷa}}, better known in its Sanskrit form \textit{\textsanskrit{garuḍa}}, is a mythological bird. Sp 4.294 continues: \textit{\textsanskrit{Pāsādoti} \textsanskrit{dīghapāsādo}. Hammiyanti \textsanskrit{upariākāsatale} \textsanskrit{patiṭṭhitakūṭāgāro} \textsanskrit{pāsādoyeva}}, “A \textit{\textsanskrit{pāsāda}} is a long stilt house. A \textit{hammiya} is just a \textit{\textsanskrit{pāsāda}} that has an upper room on top of its flat roof.” At Sp-\textsanskrit{ṭ} 3.74, however, we find slightly different explanations. It seems clear, however, that all three are stilt houses and that they are distinguished according to their shape and the kind of roof they possess. } }

The\marginnote{1.3.1} monks went to that merchant and said, “The Buddha has allowed dwellings. Please do as you think appropriate.” Then, on a single day, that merchant built sixty dwellings. When the dwellings were finished, he went to the Buddha, bowed, sat down, and said, “Sir, please accept tomorrow’s meal from me together with the Sangha of monks.” The Buddha consented by remaining silent. Knowing that the Buddha had consented, he got up from his seat, bowed down, circumambulated the Buddha with his right side toward him, and left. 

The\marginnote{1.4.1} next morning he had various kinds of fine foods prepared and then had the Buddha informed that the meal was ready. The Buddha robe up, took his bowl and robe, and went to the house of that merchant where he sat down on the prepared seat together with the Sangha of monks. That merchant personally served various kinds of fine foods to the Sangha of monks headed by the Buddha. When the Buddha had finished his meal and had washed his hands and bowl, the merchant sat down to one side and said, “Sir, I’ve had these sixty dwellings built in order to make merit and for the purpose of going to heaven. What should I do now?” 

“Well\marginnote{1.4.8} then, give those sixty dwellings to the Sangha as a whole, both present and future.” 

Saying,\marginnote{1.4.9} “Yes, sir,” he did just that. 

The\marginnote{1.5.1} Buddha then expressed his appreciation to the merchant with these verses: 

\begin{verse}%
“Cold\marginnote{1.5.2} and heat are kept away, \\
And so are predatory beasts, \\
And creeping animals and mosquitoes, \\
And also chill and rain. 

They\marginnote{1.5.6} keep away the wind and burning sun, \\
When those awful things arise. \\
Their purpose is to shelter and for happiness, \\
To attain absorption and to see clearly. 

Giving\marginnote{1.5.10} dwellings to the Sangha \\
Is praised as the best by the Buddha. \\
Therefore the wise man, \\
Seeing what’s beneficial for himself, 

Should\marginnote{1.5.14} build delightful dwellings \\
And have the learned stay there. \\
Food, drink, robes, and dwellings—\\
With an inspired mind, 

He\marginnote{1.5.18} should give to them, \\
The upright ones. \\
They will give him the Teaching \\
For removing all suffering; \\
And understanding this Teaching in this life, \\
He attains extinguishment, free of corruptions.” 

%
\end{verse}

The\marginnote{1.5.24} Buddha then got up from his seat and left. 

Hearing\marginnote{2.1.1} that the Buddha had allowed dwellings, people had dwellings built with care. But because the dwellings did not have doors, snakes, scorpions, and centipedes came inside. They told the Buddha. 

\scrule{“I allow doors.” }

They\marginnote{2.1.7} made a hole in the wall and bound the doors with creepers and ropes. Rats and termites ate the creepers and ropes, and the doors fell off. 

\scrule{“I allow door frames, and lower and upper hinges.”\footnote{“Door frame” renders \textit{\textsanskrit{piṭṭhasaṅghāṭa}}. The precise meaning of the two words that make up this compound is never clarified in the Pali texts, but it seems clear enough that the compound as a whole refers to the entire door frame, for instance when it is used to define the \textit{\textsanskrit{dvārakosa}} at \href{https://suttacentral.net/pli-tv-bu-vb-pc19/en/brahmali\#2.1.7}{Bu Pc 19:2.1.7}. Vmv 1.349: \textit{\textsanskrit{Piṭṭhasaṅghāṭo} \textsanskrit{nāma} \textsanskrit{dvārabāhasaṅkhāto} \textsanskrit{caturassadārusaṅghāṭo}, yattha \textsanskrit{sauttarapāsaṁ} \textsanskrit{kavāṭaṁ} \textsanskrit{apassāya} \textsanskrit{dvāraṁ} pidahanti}, “\textit{\textsanskrit{Piṭṭhasaṅghāṭo}}: a rectangular binding together of wood, called a doorcase, where the door with its hinges are supported and shut the doorway.” “Lower and upper hinges” renders \textit{udukkhalika} and \textit{\textsanskrit{uttarapāsaka}}, respectively. Sp 1.77: … \textit{yena kenaci \textsanskrit{kavāṭaṁ} \textsanskrit{katvā} \textsanskrit{heṭṭhā} udukkhale upari \textsanskrit{uttarapāsake} ca \textsanskrit{pavesetvā} \textsanskrit{kataṁ} \textsanskrit{parivattakadvārameva} \textsanskrit{saṁvaritabbaṁ} … atha \textsanskrit{dvārassa} \textsanskrit{udukkhalaṁ} \textsanskrit{vā} \textsanskrit{uttarapāsako} \textsanskrit{vā} bhinno \textsanskrit{vā} hoti \textsanskrit{aṭṭhapito} \textsanskrit{vā}, \textsanskrit{saṁvarituṁ} na sakkoti}, “Having made a door by whatever (material), having entered it into the \textit{udukkhala} below and into the \textit{\textsanskrit{uttarapāsaka}} above, it is made a revolving door to be closed … but when the \textit{udukkhala} or the \textit{\textsanskrit{uttarapāsaka}} is broken or not mounted, then one cannot close the door.” From this it seems that the \textit{udukkhala} and \textit{\textsanskrit{uttarapāsaka}}, together with the two corresponding “projecting pivots” on the door, are the functional equivalents of hinges. \textit{Udukkhalika} and \textit{udukkhala} refer to the same thing, the former being used in the canonical text, whereas the latter is found in the summary verses. For further details see CPD, sv. \textit{\textsanskrit{uttarapāsaka}}. } }

The\marginnote{2.1.12} doors did not fit the door frames.\footnote{Elsewhere we find the closely related expression \textit{\textsanskrit{phusitaggaḷa}}, “with a touching door”, to describe a well-built house, e.g. at \href{https://suttacentral.net/mn12/en/brahmali\#41.4}{MN 12:41.4}. The same description says that the house is \textit{\textsanskrit{nivāta}}, “free from drafts”. MN-a 1.154 comments: \textit{\textsanskrit{Phusitaggaḷanti} \textsanskrit{dvārabāhāhi} \textsanskrit{saddhiṁ} \textsanskrit{supihitakavāṭaṁ}}, “With a touching door: the door-panel closes well with the door frame.” The point seems to be that the door is closed all the way so that it touches the door frame. } 

\scrule{“I allow a hole in the door and a rope for pulling.”\footnote{“A hole in the door (for pulling)” and “a rope for pulling” render \textit{\textsanskrit{āviñchanachidda}} and \textit{\textsanskrit{āviñchanarajju}} respectively. Vmv 4.296: \textit{\textsanskrit{Āviñchanachiddanti} yattha \textsanskrit{aṅguliṁ} \textsanskrit{vā} \textsanskrit{rajjusaṅkhalikādiṁ} \textsanskrit{vā} \textsanskrit{pavesetvā} \textsanskrit{kavāṭaṁ} \textsanskrit{ākaḍḍhantā} \textsanskrit{dvārabāhaṁ} \textsanskrit{phusāpenti}}, “\textit{\textsanskrit{Āviñchanachidda}}: where, having entered the finger or a rope or a chain, etc., they pull the door and make it touch the door post.” Sp 4.296: \textit{\textsanskrit{Āviñchanarajjunti} \textsanskrit{kavāṭeyeva} \textsanskrit{chiddaṁ} \textsanskrit{katvā} tattha \textsanskrit{pavesetvā} yena rajjukena \textsanskrit{kaḍḍhantā} \textsanskrit{dvāraṁ} \textsanskrit{phusāpenti}}, “Having made a hole in the door, having entered (the rope) there, the rope with which they close and make the door touch (the post).” } }

The\marginnote{2.1.15} doors did not stay closed.\footnote{The commentaries are silent. Under the entry \textit{thaketi}, DOP says: “covers, covers up; closes”. The context, however, suggests that this concerns more than merely closing the door, which is already effected by the hole and the rope mentioned just above. That more than closing is meant is also apparent from the fact that the remedy is to allow the use of bolts. } 

\scrule{“I allow door jambs, bolt sockets, bolts, and latches.”\footnote{“Door jamb” renders \textit{\textsanskrit{aggaḷavaṭṭi}}. Sp 4.260: \textit{\textsanskrit{Aggaḷavaṭṭi} \textsanskrit{nāma} \textsanskrit{dvārabāhāya} \textsanskrit{samappamāṇoyeva} \textsanskrit{aggaḷatthambho} vuccati, yattha \textsanskrit{tīṇi} \textsanskrit{cattāri} \textsanskrit{chiddāni} \textsanskrit{katvā} \textsanskrit{sūciyo} denti}, “\textit{\textsanskrit{Aggaḷavaṭṭi}}: it is called a door post, which is the same length as the door frame. It is where three or four holes are made for inserting bolts.” Whenever the Canonical text lists the parts of a door and door frame, the \textit{\textsanskrit{aggaḷavaṭṭi}} always has the same position, being grouped together with the parts for the locking mechanism, such as latches and bolts. Given the commentarial explanation, it is natural to think that it was a special post added to the door frame, or perhaps replacing the door frame, for the purpose of receiving bolts. “Bolt socket” renders \textit{\textsanskrit{kapisīsaka}}. Sp 4.260: \textit{\textsanskrit{Kapisīsakaṁ} \textsanskrit{nāma} \textsanskrit{dvārabāhaṁ} \textsanskrit{vijjhitvā} tattha pavesito \textsanskrit{aggaḷapāsako} vuccati}, “The bolt-receiving socket which is inserted after piercing the door post is called a \textit{\textsanskrit{kapisīsaka}}.” “Bolt” renders \textit{\textsanskrit{sūcika}}. Sp 4.260: \textit{\textsanskrit{Sūcikāti} tattha majjhe \textsanskrit{chiddaṁ} \textsanskrit{katvā} \textsanskrit{pavesitā}}, “It is entered, having made a hole in the middle there.” “Latch” renders \textit{\textsanskrit{ghaṭika}}. Sp-\textsanskrit{ṭ} 4.255: \textit{\textsanskrit{Ghaṭikanti} upari \textsanskrit{yojitaṁ} \textsanskrit{aggaḷaṁ}}, “\textit{\textsanskrit{Ghaṭika}}: it connected the door at the top.” From the origin story to \href{https://suttacentral.net/pli-tv-bu-vb-ss2/en/brahmali\#1.1.11}{Bu Ss 2:1.1.11} it seems that the \textit{\textsanskrit{ghaṭika}} was a device that could be opened with a key: \textit{\textsanskrit{Avāpuraṇaṁ} \textsanskrit{ādāya} \textsanskrit{ghaṭikaṁ} \textsanskrit{ugghāṭetvā} \textsanskrit{kavāṭaṁ} \textsanskrit{paṇāmetvā} \textsanskrit{vihāraṁ} \textsanskrit{pāvisi}}, “(\textsanskrit{Udāyī}) took the key, lifted the latch, opened the door, and entered the dwelling.” It follows from this that the \textit{\textsanskrit{ghaṭika}} is unlikely to be a bolt, but probably a kind of bar, like a latch, that would require lifting for the door to open. The lifting would be done with \textit{\textsanskrit{tāḷa}}, a key-like device, for which see below. For a discussion of the \textit{\textsanskrit{aggaḷa}} as “door”, see Appendix of Technical Terms. } }

The\marginnote{2.1.18} monks were unable to open the doors.\footnote{It is not clear why this would be so. Perhaps the latch would fasten automatically as the door was closed, but it would then be impossible to get back inside. } 

\scrule{“I allow a keyhole and three kinds of keys: metal keys, wooden keys, and keys made of horn.” }

They\marginnote{2.1.23} lifted the latches with the keys and entered, but the dwellings were unprotected.\footnote{Again, it is not clear to me why this would be the case. } 

\scrule{“I allow bolts.”\footnote{A \textit{yantaka} seems to be any implement that is used as a bolt, whereas a \textit{\textsanskrit{sūcika}} is a bolt proper. Sp 4.296: \textit{\textsanskrit{Yantakaṁ} \textsanskrit{sūcikanti} ettha \textsanskrit{yaṁ} \textsanskrit{yaṁ} \textsanskrit{jānāti} \textsanskrit{taṁ} \textsanskrit{taṁ} \textsanskrit{yantakaṁ}, tassa \textsanskrit{vivaraṇasūcikañca} \textsanskrit{kātuṁ} \textsanskrit{vaṭṭati}}, “Here, whatever he finds, that is a \textit{yantaka}; to make a bolt of it for opening is allowed.” It is not clear why bolts, \textit{\textsanskrit{sūcika}}, are mentioned both here and above. It is possible that the text has evolved over time. } }

At\marginnote{2.2.1} that time the dwellings had roofs of straw. When the weather was cold, they were cold, and when the weather was hot, they were hot. 

\scrule{“I allow you to firm up the structure and then to plaster it inside and outside.”\footnote{“Firm up the structure” renders \textit{\textsanskrit{ogumphetvā}}. Sp 4.257: \textit{\textsanskrit{Ogumphetvā} \textsanskrit{ullittāvalittaṁ} \textsanskrit{kātunti} \textsanskrit{chadanaṁ} \textsanskrit{odhunitvā} \textsanskrit{ghanadaṇḍakaṁ} \textsanskrit{katvā} anto ceva bahi ca \textsanskrit{mattikāya} limpitunti attho}, “\textit{\textsanskrit{Ogumphetvā} \textsanskrit{ullittāvalittaṁ} \textsanskrit{kātuṁ}}: having shook out the roof cover and added rods to firm up (the structure), to smear with clay inside and outside—this is the meaning.” At \href{https://suttacentral.net/pli-tv-kd5/en/brahmali\#11.1.5}{Kd 5:11.1.5} the same verb, in the form \textit{ogumphiyanti}, is used to show how dwellings are “held together” by straps of leather. This makes it certain that \textit{\textsanskrit{ogumphetvā}} in the present context must refer to “firming up” rather than “shaking out”. } }

At\marginnote{2.2.5} that time the dwellings did not have windows. It was hard to see and the dwellings were smelly.\footnote{The usual meaning of \textit{acakkhussa} is “bad for the eyes”. According to SED, however, it can also mean “blind”; sv. \textit{acakshushka}. } They told the Buddha. 

\scrule{“I allow three kinds of windows: railing windows, lattice windows, and windows with bars.”\footnote{“Railing windows” renders \textit{\textsanskrit{vedikāvātapāna}}. Sp 4.296: \textit{\textsanskrit{Vedikāvātapānaṁ} \textsanskrit{nāma} cetiye \textsanskrit{vedikāsadisaṁ}}, “(A window) like the railing on a shrine is called \textit{\textsanskrit{vedikāvātapāna}}.” “Windows with bars” renders \textit{\textsanskrit{salākavātapāna}}. Sp 4.296: \textit{\textsanskrit{Salākavātapānaṁ} \textsanskrit{nāma} \textsanskrit{thambhakavātapānaṁ}}, “Windows with rods are called \textit{\textsanskrit{salākavātapāna}}.” } }

Squirrels\marginnote{2.2.9} and bats entered the dwellings through the gaps in the windows. 

\scrule{“I allow cloth covers.”\footnote{Sp 4.296: \textit{Cakkalikanti ettha “\textsanskrit{coḷakapādapuñchanaṁ} \textsanskrit{bandhituṁ} \textsanskrit{anujānāmī}”ti attho}, “\textit{Cakkalika}: the meaning is: ‘I allow to bind a foot-wiping cloth there.’” } }

The\marginnote{2.2.12} squirrels and bats entered in the gaps around the cloth cover. 

\scrule{“I allow shutters.”\footnote{I am not able to make a distinction between \textit{\textsanskrit{vātapānakavāṭaka}} and \textit{\textsanskrit{vātapānabhisika}} that is meaningful in English. \textit{\textsanskrit{Vātapānakavāṭaka}}, literally, means “a small door for the windows”. The \textit{\textsanskrit{vātapānabhisika}} is described at Sp 4.296: \textit{\textsanskrit{Vātapānabhisīti} \textsanskrit{vātapānappamāṇena} \textsanskrit{bhisiṁ} \textsanskrit{katvā} \textsanskrit{bandhituṁ} \textsanskrit{anujānāmīti} attho}, “\textit{\textsanskrit{Vātapānabhisi}}: the meaning is: ‘I allow, having made a small mattress the size of the window, to bind it on.’” } }

At\marginnote{2.3.1} that time the monks lay down on the ground. They became dirty, as did their robes. 

\scrule{“I allow a spread of grass.” }

The\marginnote{2.3.5} grass was eaten by rats and termites. 

\scrule{“I allow benches.”\footnote{For further discussion of this and the next item, see Appendix of Furniture. } }

The\marginnote{2.3.8} benches were painful to lie on. 

\scrule{“I allow wicker beds.” }

\subsection*{2. The allowance for beds and benches }

Soon\marginnote{2.3.11.1} afterwards the Sangha was offered various kinds of beds with legs and frames from a charnel ground.\footnote{Vin-vn-\textsanskrit{ṭ} 663: \textit{\textsanskrit{Sosānikanti} \textsanskrit{susāne} \textsanskrit{chaḍḍitaṁ}}, “\textit{\textsanskrit{Sosānika}} means what is discarded on a charnel ground.” The text gives two different kinds of beds (and benches just below) called \textit{\textsanskrit{masāraka}} and \textit{\textsanskrit{bundikābaddha}}. The only difference between the two seems to be how the rails are fastened to the legs. From this it seems reasonable to conclude that all beds with legs and a frame are allowable, as long as they adhere to the other standards of the Vinaya, such as \href{https://suttacentral.net/pli-tv-bu-vb-pc87/en/brahmali\#1.11.1}{Bu Pc 87:1.11.1}. Thus my decision to group these beds and benches together as “various kinds of beds/benches”. For further discussion, see Appendix of Furniture. } They told the Buddha. 

\scrule{“I allow the various kinds of beds with legs and frames.” }

The\marginnote{2.3.14} Sangha was offered various kinds of benches with legs and frames. 

\scrule{“I allow the various kinds of benches with legs and frames.” }

The\marginnote{2.3.23} Sangha was offered a bed with crooked legs from a charnel ground. 

\scrule{“I allow beds with crooked legs.”\footnote{For a further discussion of this and the various items below, see Appendix of Furniture. } }

The\marginnote{2.3.26} Sangha was offered a bench with crooked legs. 

\scrule{“I allow benches with crooked legs.” }

The\marginnote{2.3.29} Sangha was offered a bed with detachable legs from a charnel ground. 

\scrule{“I allow beds with detachable legs.” }

The\marginnote{2.3.32} Sangha was offered a bench with detachable legs. 

\scrule{“I allow benches with detachable legs.” }

The\marginnote{2.4.1} Sangha was offered a square bench. 

\scrule{“I allow square benches.” }

The\marginnote{2.4.4} Sangha was offered a tall square bench. 

\scrule{“I also allow tall square benches.” }

The\marginnote{2.4.7} Sangha was offered a sofa. 

\scrule{“I allow sofas.” }

The\marginnote{2.4.10} Sangha was offered a high sofa. 

\scrule{“I also allow high sofas.” }

The\marginnote{2.4.13} Sangha was offered a cane bench. 

\scrule{“I allow cane benches.” }

The\marginnote{2.4.16} Sangha was offered a small bench bound with pieces of cloth. 

\scrule{“I allow small benches bound with pieces of cloth.” }

The\marginnote{2.4.19} Sangha was offered a bench with ram-like legs. 

\scrule{“I allow benches with ram-like legs.” }

The\marginnote{2.4.22} Sangha was offered a bench with many legs. 

\scrule{“I allow benches with many legs.” }

The\marginnote{2.4.25} Sangha was offered a plank as a bench. 

\scrule{“I allow plank benches.” }

The\marginnote{2.4.28} Sangha was offered a stool. 

\scrule{“I allow stools.” }

The\marginnote{2.4.31} Sangha was offered a bench made of straw. 

\scrule{“I allow benches made of straw.” }

At\marginnote{2.5.1} that time the monks from the group of six slept on high beds. When people walking about the dwellings saw this, they complained and criticized them, “They’re just like householders who indulge in worldly pleasures!” They told the Buddha. 

\scrule{“You shouldn’t sleep on high beds. If you do, you commit an offense of wrong conduct.” }

Soon\marginnote{2.5.7} afterwards a monk was bitten by a snake while sleeping on a low bed. 

\scrule{“I allow bed supports.”\footnote{These were not legs but a loose kind of prop used to keep the bed off the ground, see \href{https://suttacentral.net/pli-tv-kd1/en/brahmali\#25.15.3}{Kd 1:25.15.3} and \href{https://suttacentral.net/pli-tv-kd1/en/brahmali\#25.16.2}{Kd 1:25.16.2}. } }

The\marginnote{2.5.10} monks from the group of six used high bed supports and then made the beds shake. When people walking about the dwellings saw this, they complained and criticized them, “They’re just like householders who indulge in worldly pleasures!” 

\scrule{“You shouldn’t use high bed supports. If you do, you commit an offense of wrong conduct. I allow bed supports that are at the most eight standard fingerbreadths long.”\footnote{I take \textit{\textsanskrit{aṭṭhaṅgulaparama}} to be equivalent to the maximum height allowed under \href{https://suttacentral.net/pli-tv-bu-vb-pc87/en/brahmali\#1.11.1}{Bu Pc 87:1.11.1}, where the \textit{\textsanskrit{aṅgula}} is specified as the \textit{\textsanskrit{sugataṅgula}}, “the standard fingerbreadth”. See \textit{sugata} in Appendix of Technical Terms. } }

The\marginnote{2.6.1} Sangha was offered string. 

\scrule{“I allow you to wrap the beds with string.”\footnote{Reading \textit{\textsanskrit{veṭhetuṁ}}. } }

The\marginnote{2.6.4} limbs of the bed took up a lot of string. 

\scrule{“I allow you to perforate the limbs and wrap with a cross weaving.”\footnote{The meaning of this is not clear. The commentaries are silent. } }

The\marginnote{2.6.7} Sangha was offered a cloth. 

\scrule{“I allow you to make a mat underlay.”\footnote{For further discussion, see Appendix of Furniture. } }

The\marginnote{2.6.10} Sangha was offered a cotton-down quilt. 

\scrule{“I allow you to remove the cotton down and make pillows. There are three kinds of cotton down: cotton down from trees, cotton down from creepers, and cotton down from grass.” }

The\marginnote{2.6.15} monks from the group of six used pillows that were half the size of the body. When people walking about the dwellings saw this, they complained and criticized them, “They’re just like householders who indulge in worldly pleasures!” 

\scrule{“You shouldn’t use pillows that are half the size of the body. If you do, you commit an offense of wrong conduct. I allow you to make pillows the size of the head.” }

On\marginnote{2.7.1} one occasion in \textsanskrit{Rājagaha} there was a hilltop fair. People prepared mattresses for the government officials:\footnote{“Government official” renders \textit{\textsanskrit{mahāmattā}}. Sp 1.92, commenting on Bu Pj 2, says: \textit{\textsanskrit{Mahāmattāti} \textsanskrit{ṭhānantarappattā} \textsanskrit{mahāamaccā}; tepi tattha tattha \textsanskrit{gāme} \textsanskrit{vā} nigame \textsanskrit{vā} \textsanskrit{nisīditvā} \textsanskrit{rājakiccaṁ} karonti}, “\textit{\textsanskrit{Mahāmattā}} means an important worker, who has attained a position; those who sit down in this or that village or town, and do the work of the king.” \textit{\textsanskrit{Mahāmatta}} is often translated as “minister” or even “great minister”, but the relatively small scale of ancient Indian society suggests a more humble translation. } mattresses stuffed with wool, cloth, bark, grass, or leaves. When the fair was over, they removed the covers and took them away. The monks saw a large quantity of wool, cloth, bark, grass, and leaves abandoned on the ground. They told the Buddha. 

\scrule{“I allow five kinds of mattresses: mattresses stuffed with wool, cloth, bark, grass, or leaves.” }

The\marginnote{2.7.9} Sangha was offered furniture cloth. 

\scrule{“I allow you to cover the mattresses.” }

The\marginnote{2.7.12} monks laid a bed mattress on a bench and a bench mattress on a bed. The mattresses split open. 

\scrule{“I allow upholstered beds and upholstered benches.” }

They\marginnote{2.7.16} laid out mattresses without underlay. They sank down.\footnote{Sp 4.297: \textit{\textsanskrit{Ullokaṁ} \textsanskrit{akaritvāti} \textsanskrit{heṭṭhā} \textsanskrit{cimilikaṁ} \textsanskrit{adatvā}}, “\textit{\textsanskrit{Ullokaṁ} \textsanskrit{akaritvā}}: not having applied an underlay beneath.” } 

\scrule{“I allow you to arrange an underlay, then to lay down a mattress, and then to cover it.” }

The\marginnote{2.7.18} covers were removed and taken away. 

\scrule{“I allow you to sprinkle them.”\footnote{Sp 4.297: \textit{Phositunti rajanena \textsanskrit{vā} \textsanskrit{haliddiyā} \textsanskrit{vā} upari \textsanskrit{phusitāni} \textsanskrit{dātuṁ}}, “To sprinkle means to apply spots of dye or turmeric on top.” Sp-\textsanskrit{ṭ} 4.297: \textit{\textsanskrit{Phusitāni} \textsanskrit{dātunti} \textsanskrit{saññākaraṇatthaṁ} \textsanskrit{bindūni} \textsanskrit{dātuṁ}}, “To apply spots means to apply marks for the purpose of recognition.” } }

They\marginnote{2.7.20} were still taken away. 

\scrule{“I allow you to make multi-colored lines.”\footnote{Sp 4.297: “\textit{Bhattikammanti \textsanskrit{bhisicchaviyā} upari \textsanskrit{bhattikammaṁ}}, “\textit{Bhattikamma} means \textit{bhattikamma} on top of the mattress covering.” Vmv 4.297: “\textit{Bhittikammanti \textsanskrit{nānāvaṇṇehi} \textsanskrit{vibhittirājikaraṇaṁ}}”, “\textit{Bhittikamma} means making separate lines by means of many colors.” According to SED, the Sanskrit \textit{bhakti} can mean “distribution, partition, separation”, “division by streaks or lines”, or “a streak, line, variegated decoration”. Based on this, it seems possible that \textit{bhatti} refers to multi-colored lines drawn on the cloth. } }

They\marginnote{2.7.22} were still taken away. 

\scrule{“I allow you to make multi-colored lines by hand.”\footnote{It is not clear what the difference is between \textit{hatthabhattikamma} and \textit{hatthabhatti}, the latter being found in the next rule. \textit{Hatthabhattikamma} is not mentioned in the commentaries, nor is it found in the PTS version of the Canon. It may be that this word has been accidentally added to MS. } }

They\marginnote{2.7.24} were still taken away. 

\scrule{“I allow multi-colored lines by hand.”\footnote{Sp 4.297: “\textit{Hatthabhattinti \textsanskrit{pañcaṅgulibhattiṁ}}, “\textit{Hatthabhatti} means \textit{bhatti} with the five fingers.” Commenting on the word \textit{hatthakamma}, “hand-work”, Vmv 4.297 says: “\textit{Hatthakammanti hatthena \textsanskrit{yaṁ} \textsanskrit{kiñci} \textsanskrit{saññākaraṇaṁ}}”, “\textit{Hatthakamma} means whatever mark is made by hand.” } }

\subsection*{3. The allowance for the color white, etc. }

At\marginnote{3.1.1} that time the monastics of other religions had white beds, black floors, and red ocher walls. Many people went to see their beds. 

\scrule{“I allow the colors white, black, and red ocher in the dwellings.” }

The\marginnote{3.1.5} white color did not adhere to the rough walls.\footnote{Vmv 4.298: \textit{\textsanskrit{Pāḷiyaṁ} na \textsanskrit{nipatatīti} na \textsanskrit{allīyati}}, “In the Canonical text, \textit{na nipatati} means not sticking.” } 

\scrule{“I allow you to apply balls of husk, smooth with a trowel, and then apply the white color.”\footnote{Sp-\textsanskrit{ṭ} 4.298: \textit{\textsanskrit{Paṭibāhetvāti} \textsanskrit{maṭṭhaṁ} \textsanskrit{katvā}}, “\textit{\textsanskrit{Paṭibāhetva}} means having polished.” Vmv 4.298: \textit{\textsanskrit{Paṭibāhetvāti} \textsanskrit{ghaṁsitvā}}, “\textit{\textsanskrit{Paṭibāhetva}} means having rubbed.” } }

The\marginnote{3.1.8} white color still did not adhere.\footnote{Sp-\textsanskrit{ṭ} 4.298: \textit{Na \textsanskrit{nibandhatīti} \textsanskrit{anibandhanīyo}, na \textsanskrit{allīyatīti} attho}, “\textit{\textsanskrit{Anibandhanīyo}}: ʻIt does not bind’; the meaning is ʻit does not stick’.” } 

\scrule{“I allow you to apply soft clay, smooth with a trowel, and then apply the white color.” }

The\marginnote{3.1.10} white color still did not adhere. 

\scrule{“I allow sap and flour paste.”\footnote{“Sap” renders \textit{\textsanskrit{ikkāsa}}. Sp 4.298: \textit{\textsanskrit{Ikkāsanti} \textsanskrit{rukkhaniyyāsaṁ} \textsanskrit{vā} \textsanskrit{silesaṁ} \textsanskrit{vā}}, “\textit{\textsanskrit{Ikkāsa}} means exudation from a tree or gum.” } }

The\marginnote{3.1.12} red ocher did not adhere to the rough walls. 

\scrule{“I allow you to apply balls of husk, smooth with a trowel, and then apply the red ocher.” }

The\marginnote{3.1.15} red ocher still did not adhere. 

\scrule{“I allow you to apply clay mixed with bran, smooth with a trowel, and then apply the red ocher.”\footnote{Sp 4.298: \textit{\textsanskrit{Kuṇḍakamattikanti} \textsanskrit{kuṇḍakamissakamattikaṁ}}, “\textit{\textsanskrit{Kuṇḍakamattika}} means clay mixed with bran.” } }

The\marginnote{3.1.17} red ocher still did not adhere. 

\scrule{“I allow mustard powder and beeswax.” }

It\marginnote{3.1.19} was too thick. 

\scrule{“I allow you to wipe it off with a cloth.” }

The\marginnote{3.1.21} black color did not adhere to the rough floors. 

\scrule{“I allow you to apply balls of husk, smooth with a trowel, and then apply the black color.” }

The\marginnote{3.1.24} black color still did not adhere. 

\scrule{“I allow you to apply excreted clay, smooth with a trowel, and then apply the black color.”\footnote{“Excreted clay” renders \textit{\textsanskrit{gaṇḍumattika}}. Sp 4.275: \textit{\textsanskrit{Gaṇḍumattikanti} \textsanskrit{gaṇḍuppādagūthamattikaṁ}}, “\textit{\textsanskrit{Gaṇḍumattika}} means the clay excretions of worms.” } }

The\marginnote{3.1.26} black color still did not adhere. 

\scrule{“I allow sap and bitter substances.” }

\subsection*{4. The prohibition against pictures }

At\marginnote{3.2.1} that time the monks from the group of six had pictures of women and men drawn in a dwelling. When people walking about the dwellings saw this, they complained and criticized them, “They’re just like householders who indulge in worldly pleasures!” They told the Buddha. 

\scrule{“You shouldn’t have pictures drawn of women and men. If you do, you commit an offense of wrong conduct. I allow you to make garland patterns, creeper patterns, shark-teeth patterns, and the fivefold pattern.”\footnote{“A shark-teeth pattern” renders \textit{makaradantaka}, literally, “like the teeth of a \textit{makara}”. In later Buddhism the \textit{makara} is the name of a mythological marine animal, but what it refers to in this context is not clear. According to Sp-yoj 4.243 \textit{makara} is the name of a certain species of fish: \textit{Makaradantaketi \textsanskrit{makaranāmakassa} macchassa dantasadise dante}, “Teeth like the teeth of a fish called \textit{makara}.” Vin-vn-\textsanskrit{ṭ} 3048, \textit{Makaradantakanti \textsanskrit{girikūṭākāraṁ}}, “\textit{Makaradantaka} means making (a design) like the peak of a hill.” PED suggests “the tooth of a swordfish”, but apparently swordfish do not have teeth. Given that the \textit{makara} were fearsome creatures and that their teeth looked like the peak of a hill, presumably meaning that their teeth were pointed, “shark teeth” seems like a reasonable guess. “The fivefold pattern” renders \textit{\textsanskrit{pañcapaṭika}}. Vmv 4.299: \textit{\textsanskrit{Pāḷiyaṁ} \textsanskrit{pañcapaṭikanti} \textsanskrit{jātiādipañcappakāravaṇṇamaṭṭhaṁ}}, “\textit{\textsanskrit{Pañcapaṭika}} in the canonical text means treated with the color of the five kinds, starting with jasmine.” The meaning is not clear. It seems unlikely to me, however, that it should refer to colors, since such have just been listed. } }

\subsection*{5. The allowance for foundations of bricks, etc. }

At\marginnote{3.3.1} that time the dwellings were built on a low base. They were flooded. 

\scrule{“I allow you to raise the base.” }

The\marginnote{3.3.4} mound collapsed. 

\scrule{“I allow you to construct three kinds of raised foundations: raised foundations of brick, stone, and wood.” }

It\marginnote{3.3.7} was difficult to get up to the dwelling. 

\scrule{“I allow three kinds of stairs: stairs of brick, stone, and wood.” }

People\marginnote{3.3.10} fell down while climbing the stairs. 

\scrule{“I allow rails.” }

At\marginnote{3.3.12} that time the dwellings were accessible to the public.\footnote{“Accessible to the public” renders \textit{\textsanskrit{āḷakamandā}}. In the Suttas, \textit{\textsanskrit{āḷakamandā}} is a name for the capital city of the \textit{devas.} In the present context, however, the meaning must be something else, albeit perhaps semantically related. Sp 4.300: \textit{\textsanskrit{Ekaṅgaṇā} \textsanskrit{manussābhikiṇṇā}}, “A single open space, thronged with people.” } The monks were embarrassed to lie down there. 

\scrule{“I allow curtains.” }

People\marginnote{3.3.16} lifted them up and looked in. 

\scrule{“I allow half walls.” }

People\marginnote{3.3.18} looked over the half walls. 

\scrule{“I allow three kinds of rooms: rectangular rooms, long rooms, and upper rooms.”\footnote{Sp 4.300: \textit{\textsanskrit{Sivikāgabbhoti} caturassagabbho}, “\textit{\textsanskrit{Sivikāgabbha}} means a rectangular room.” Sp 4.300: \textit{\textsanskrit{Nāḷikāgabbhoti} \textsanskrit{vitthārato} \textsanskrit{diguṇatiguṇāyāmo} \textsanskrit{dīghagabbho}}, “\textit{\textsanskrit{Nāḷikāgabbha}} means a long room, twice or three times as long as it is wide.” Sp 4.300: \textit{Hammiyagabbhoti \textsanskrit{ākāsatale} \textsanskrit{kūṭāgāragabbho} \textsanskrit{vā} \textsanskrit{muṇḍacchadanagabbho} \textsanskrit{vā}}, “\textit{Hammiyagabbha} means a room on the roof, either with a peaked roof or with a flat roof.” } }

On\marginnote{3.3.21} one occasion the monks made a room in the middle of a small dwelling. There was no access around the room.\footnote{Vmv 4.300: \textit{\textsanskrit{Upacāro} na \textsanskrit{hotīti} gabbhassa bahi \textsanskrit{samantā} anuparigamanassa \textsanskrit{okāso} nappahoti}, “\textit{\textsanskrit{Upacāro} na hoti} means there was insufficient space for walking all the way around outside the room.” For the rendering “access” for \textit{\textsanskrit{upacāra}}, see Appendix of Technical Terms. } 

\scrule{“In a small dwelling you should make the room on the side, but in a large dwelling in the middle.” }

At\marginnote{3.4.1} that time the base of the wall of a certain dwelling was deteriorating. 

\scrule{“I allow timber supports.” }

It\marginnote{3.4.4} rained through the wall. 

\scrule{“I allow protection screens and plaster.”\footnote{“A protection screen” and “plastering” render \textit{\textsanskrit{parittāṇakiṭika}} and \textit{uddasudha} respectively. Sp 4.300: \textit{\textsanskrit{Parittāṇakiṭikanti} \textsanskrit{vassaparittāṇatthaṁ} \textsanskrit{kiṭikaṁ}. Uddasudhanti vacchakagomayena ceva \textsanskrit{chārikāya} ca \textsanskrit{saddhiṁ} \textsanskrit{madditamattikaṁ}}, “\textit{\textsanskrit{Parittāṇakiṭika}} means a screen for the purpose of protection against the rain. \textit{Uddasudha} means clay mixed with calf-manure and ash.” } }

On\marginnote{3.4.6} one occasion a snake fell from the grass roof onto the shoulder of a certain monk. Terrified, he screamed. Other monks ran up to him and asked him why he was screaming. He told them. 

\scrule{“I allow canopies.” }

At\marginnote{3.5.1} that time the monks hung their bags from the legs of the beds and benches. Rats and termites ate them. 

\scrule{“I allow wall pegs.”\footnote{\textit{\textsanskrit{Bhittikhilaṁ} \textsanskrit{nāgadantakaṁ}}, literally, “A wall peg and an elephant tusk”. These are different kinds of pegs, and I have not tried to differentiate between them. } }

At\marginnote{3.5.5} that time the monks lay their robes on their beds and benches. The robes tore. 

\scrule{“I allow bamboo robe racks and clotheslines in the dwellings.” }

At\marginnote{3.5.9} that time the dwellings were not protected by porches.\footnote{Sp 4.300: \textit{\textsanskrit{Āḷindo} \textsanskrit{nāma} \textsanskrit{pamukhaṁ} vuccati}, “The forecourt is called an \textit{\textsanskrit{āḷinda}}.” In other words, this seems to refer to an entrance area with a roof. } 

\scrule{“I allow porches, screened doorsteps, encircling corridors, and entrance roofs.”\footnote{“Screened doorstep” renders \textit{paghana}. Sp 4.300: \textit{\textsanskrit{Paghanaṁ} \textsanskrit{nāma} \textsanskrit{yaṁ} \textsanskrit{nikkhamantā} ca \textsanskrit{pavisantā} ca \textsanskrit{pādehi} hananti, tassa \textsanskrit{vihāradvāre} ubhato \textsanskrit{kuṭṭaṁ} \textsanskrit{nīharitvā} \textsanskrit{katapadesassetaṁ} \textsanskrit{adhivacanaṁ}}, “Where those who leave and enter stomp their feet is called a \textit{paghana}. This is a word for a place made at the door to the dwelling, having removed the wall on both sides.” Sp-\textsanskrit{ṭ} 4.300 further explains this as follows: \textit{Ubhato \textsanskrit{kuṭṭaṁ} \textsanskrit{nīharitvā} \textsanskrit{katapadesassāti} \textsanskrit{yathā} \textsanskrit{antodvārasamīpe} nisinnehi \textsanskrit{ujukaṁ} bahi \textsanskrit{oloketuṁ} na \textsanskrit{sakkā} hoti, \textsanskrit{evaṁ} ubhohi passehi \textsanskrit{kuṭṭaṁ} \textsanskrit{nīharitvā} abhimukhe \textsanskrit{bhittiṁ} \textsanskrit{upaṭṭhapetvā} katapadesassa}, “The place made (…) having removed the wall on both sides means: just as one seated near the inside of the door is unable to see straight outside, so it is for the place made after removing the wall from both sides and building a wall in front.” “Encircling corridor” renders \textit{\textsanskrit{pakuṭṭa}}. Sp 4.300: \textit{\textsanskrit{Pakuṭṭanti} majjhe gabbhassa \textsanskrit{samantā} \textsanskrit{pariyāgāro} vuccati}, “A corridor on all sides of a room in the middle is called a \textit{\textsanskrit{pakuṭṭa}}.” “Entrance roof” \textit{\textsanskrit{osāraka}}. Sp 4.300: \textit{\textsanskrit{Osārakanti} \textsanskrit{anāḷindake} \textsanskrit{vihāre} \textsanskrit{vaṁsaṁ} \textsanskrit{datvā} tato \textsanskrit{daṇḍake} \textsanskrit{osāretvā} \textsanskrit{katachadanapamukhaṁ}}, “\textit{\textsanskrit{Osāraka}} means: for a dwelling without a porch, one makes a roof in front with sticks fixed to a bamboo pole.” } }

The\marginnote{3.5.12} porches were unenclosed. The monks were embarrassed to lie down there. 

\scrule{“I allow sliding screens and shutters.”\footnote{Sp 4.300: \textit{\textsanskrit{Saṁsāraṇakiṭiko} \textsanskrit{nāma} cakkalayutto \textsanskrit{kiṭiko}}, “\textit{\textsanskrit{Saṁsāraṇakiṭika}} means a screen connected to wheels.” Sp-\textsanskrit{ṭ} 4.300: \textit{Cakkalayutto \textsanskrit{kiṭikoti} \textsanskrit{kavāṭaṁ} viya \textsanskrit{vivaraṇathakanasukhatthaṁ} \textsanskrit{cakkalabandhakiṭikaṁ}}, “\textit{Cakkalayutto \textsanskrit{kiṭika}} means a screen fixed to wheels for the purpose of easy opening and closing, like a door.” Vmv 4.300: \textit{\textsanskrit{Pāḷiyaṁ} \textsanskrit{ugghāṭanakiṭikanti} \textsanskrit{āpaṇādīsu} \textsanskrit{anatthikakāle} \textsanskrit{ukkhipitvā}, upari ca \textsanskrit{bandhitvā} \textsanskrit{pacchā} \textsanskrit{otaraṇakiṭikaṁ}, \textsanskrit{kappasīsehi} \textsanskrit{vā} \textsanskrit{upatthambhanīhi} \textsanskrit{ukkhipitvā} \textsanskrit{pacchā} \textsanskrit{otaraṇakiṭikampi}}, “\textit{\textsanskrit{Ugghāṭanakiṭika}}, in the Canonical text, means: outside of business hours, having lifted it up onto a shop, etc., having fixed it at the top, then lowering the screen; or having raised it with a post and a bolt socket, then lowering the screen.” } }

\subsection*{6. The allowance for an assembly hall }

At\marginnote{3.6.1} that time the monks were taking their meals outside. They were troubled by the cold and the heat. 

\scrule{“I allow assembly halls.” }

The\marginnote{3.6.4} assembly halls were built on a low base. They were flooded. 

\scrule{“I allow you to raise the base.” }

The\marginnote{3.6.6} mound collapsed. 

\scrule{“I allow you to construct three kinds of raised foundations: raised foundations of brick, stone, and wood.” }

It\marginnote{3.6.9} was difficult to get up to the assembly halls. 

\scrule{“I allow three kinds of stairs: stairs of brick, stone, and wood.” }

People\marginnote{3.6.12} fell down while climbing the stairs. 

\scrule{“I allow rails.” }

Grass\marginnote{3.6.14} and dust fell into the assembly halls. 

\scrule{“I allow you to firm up the structure and then to plaster it inside and outside, including: treating with white color, black color, and red ocher; making garland patterns, creeper patterns, shark-teeth patterns, and the fivefold pattern; putting up bamboo robe racks and clotheslines.” }

At\marginnote{3.6.17} that time the monks spread their robes on the ground outside. The robes became dirty. 

\scrule{“I allow bamboo robe racks and clotheslines outside.” }

The\marginnote{3.7.1} drinking water became warm. 

\scrule{“I allow sheds and roof covers for the drinking water.” }

The\marginnote{3.7.3} drinking-water sheds were built on a low base. They were flooded.\footnote{“Drinking-water sheds” renders \textit{\textsanskrit{pāniyasālā}}. For further discussion of \textit{\textsanskrit{sālā}}, see Appendix of Technical Terms. } 

\scrule{“I allow you to raise the base.” }

The\marginnote{3.7.5} mound collapsed. 

\scrule{“I allow you to construct three kinds of raised foundations: raised foundations of brick, stone, and wood.” }

It\marginnote{3.7.8} was difficult to get up to the sheds. 

\scrule{“I allow three kinds of stairs: stairs of brick, stone, and wood.” }

People\marginnote{3.7.11} fell down while climbing the stairs. 

\scrule{“I allow rails.” }

Grass\marginnote{3.7.13} and dust fell into the drinking-water sheds. 

\scrule{“I allow you to firm up the structure and then to plaster it inside and outside, including: treating with white color, black color, and red ocher; making garland patterns, creeper patterns, shark-teeth patterns, and the fivefold pattern; putting up bamboo robe racks and clotheslines.” }

There\marginnote{3.7.16} were no vessels for the drinking water.\footnote{Sp 4.301: \textit{\textsanskrit{Pānīyabhājananti} \textsanskrit{pivantānaṁ} \textsanskrit{pānīyadānabhājanaṁ}}, “\textit{\textsanskrit{Pānīyabhājana}} means vessel for taking the drinking water for those who are drinking.” } 

\scrule{“I allow shells and scoops.” }

\subsection*{7. The allowance for encircling walls, etc. }

At\marginnote{3.8.1} that time the dwellings were unenclosed. 

\scrule{“I allow you to make enclosures with three kinds of encircling walls: brick walls, stone walls, and wooden walls.” }

There\marginnote{3.8.5} were no gatehouses.\footnote{For the rendering “gatehouse” for \textit{\textsanskrit{koṭṭhaka}}, see Appendix of Technical Terms. } 

\scrule{“I allow gatehouses.” }

They\marginnote{3.8.7} built the gatehouses on a low base. They were flooded. 

\scrule{“I allow you to raise the base.” }

The\marginnote{3.8.9} gatehouses didn’t have doors. 

\scrule{“I allow doors, door frames, lower hinges, upper hinges, door jambs, bolt sockets, bolts, latches, keyholes, door-pulling holes, and door-pulling ropes.” }

Grass\marginnote{3.8.11} and dust fell into the gatehouses. 

\scrule{“I allow you to firm up the structure and then plaster it inside and outside, including: treating with white color, black color, and red ocher; making garland patterns, creeper patterns, shark-teeth patterns, and the fivefold pattern.” }

The\marginnote{3.8.14} yards were muddy.\footnote{For an explanation of rendering \textit{\textsanskrit{pariveṇa}} as “yard”, see Appendix of Technical Terms. } 

\scrule{“I allow you to cover them with gravel.” }

They\marginnote{3.8.17} were unable to do it. 

\scrule{“I allow you to lay paving stones.”\footnote{Or “slabs of stone”, \textit{padarasila}. See \href{https://suttacentral.net/pli-tv-bu-vb-pc18/en/brahmali\#2.3.5}{Bu Pc 18:2.3.5} where \textit{padara} means “floor boards”. } }

The\marginnote{3.8.19} water remained. 

\scrule{“I allow water drains.” }

At\marginnote{3.9.1} that time the monks made fireplaces here and there in the yards. The yards became dirty. 

\scrule{“I allow you to build water-boiling sheds out of the way.”\footnote{For a further discussion of \textit{\textsanskrit{aggisālā}} and \textit{\textsanskrit{sālā}}, see Appendix of Technical Terms. } }

They\marginnote{3.9.5} built the water-boiling sheds on a low base. They were flooded. 

\scrule{“I allow you to raise the base.” }

The\marginnote{3.9.7} mound collapsed. 

\scrule{“I allow you to construct three kinds of raised foundations: raised foundations of brick, stone, and wood.” }

It\marginnote{3.9.10} was difficult to get up to the water-boiling sheds. 

\scrule{“I allow three kinds of stairs: stairs of brick, stone, and wood.” }

People\marginnote{3.9.13} fell down while climbing the stairs. 

\scrule{“I allow rails.” }

The\marginnote{3.9.15} water-boiling sheds didn’t have doors. 

\scrule{“I allow doors, door frames, lower hinges, upper hinges, door jambs, bolt sockets, bolts, latches, keyholes, door-pulling holes, and door-pulling ropes.” }

Grass\marginnote{3.9.17} and dust fell into the water-boiling sheds. 

\scrule{“I allow you to firm up the structure and then to plaster it inside and outside, including: treating with white color, black color, and red ocher; making garland patterns, creeper patterns, shark-teeth patterns, and the fivefold pattern; putting up bamboo robe racks and clotheslines.” }

\subsection*{8. The allowance to enclose a monastery }

At\marginnote{3.10.1} that time the monasteries were unenclosed. Goats and domesticated animals harmed the saplings. 

\scrule{“I allow you to make three kinds of enclosures: enclosures of bamboo, enclosures of thorny branches, and trenches.”\footnote{Reading \textit{\textsanskrit{kaṇṭakīvāṭa}} and \textit{parikha} with the PTS edition. } }

There\marginnote{3.10.6} were no gatehouses. Goats and domesticated animals harmed the saplings just the same. 

\scrule{“I allow gatehouses, gates of wood and thorny branches, double doors, arches, and crossbars.”\footnote{“Gates of wood and thorny branches” renders \textit{apesi}. Sp 4.303: \textit{\textsanskrit{Apesīti} \textsanskrit{dīghadārumhi} \textsanskrit{khāṇuke} \textsanskrit{pavesetvā} \textsanskrit{kaṇṭakasākhāhi} \textsanskrit{vinandhitvā} \textsanskrit{kataṁ} \textsanskrit{dvārathakanakaṁ}}, “\textit{Apesi} means a closing gate that is made by entering stakes into a long piece of wood and covering it with thorny branches.” } }

Grass\marginnote{3.10.9} and dust fell into the gatehouses. 

\scrule{“I allow you to firm up the structure and then plaster it inside and outside, including: treating with white color, black color, and red ocher; making garland patterns, creeper patterns, shark-teeth patterns, and the fivefold pattern.” }

The\marginnote{3.10.12} monasteries were muddy. 

\scrule{“I allow you to cover them with gravel.” }

They\marginnote{3.10.14} were unable to do it. 

\scrule{“I allow you to lay paving stones.”\footnote{Or “slabs of stone”, \textit{padarasila}. See \href{https://suttacentral.net/pli-tv-bu-vb-pc18/en/brahmali\#2.3.5}{Bu Pc 18:2.3.5} where \textit{padara} means “floor boards”. } }

The\marginnote{3.10.16} water remained. 

\scrule{“I allow water drains.” }

On\marginnote{3.11.1} one occasion King Seniya \textsanskrit{Bimbisāra} of Magadha wanted to build a stilt house smeared with clay plaster for the Sangha. The monks thought, “Which roofing materials has the Buddha allowed and which hasn’t he?” 

\scrule{“I allow five kinds of roofing materials: tiles, slate, plaster, grass, and leaves.” }

\scend{The first section for recitation is finished. }

\section*{The second section for recitation}

\subsection*{2.1 The account of \textsanskrit{Anāthapiṇḍika} }

At\marginnote{4.1.1} that time the householder \textsanskrit{Anāthapiṇḍika} had a brother in law in \textsanskrit{Rājagaha} who was a wealthy merchant. On one occasion when \textsanskrit{Anāthapiṇḍika} was in \textsanskrit{Rājagaha} on business, that merchant had invited the Sangha headed by the Buddha for the meal on the following day. The merchant was telling his slaves and workers to get up early, to cook rice and congee, and to prepare various kinds of curry. \textsanskrit{Anāthapiṇḍika} thought, “When I’ve arrived here previously, this householder put aside all his business to greet me. But this time he’s all over the place, telling his workers what to do. Is he preparing for a wedding or a great sacrifice, or has he invited King Seniya \textsanskrit{Bimbisāra} of Magadha and the army for a meal?” 

When\marginnote{4.2.1} the merchant was finished with instructing his workers, he went up to \textsanskrit{Anāthapiṇḍika}, greeted him, and sat down. \textsanskrit{Anāthapiṇḍika} then told him what he had observed and asked him what was happening. The merchant replied, “I’m not preparing for a wedding, nor have I invited King Seniya \textsanskrit{Bimbisāra} of Magadha and the army. I’m preparing for a great sacrifice. I’ve invited the Sangha headed by the Buddha for a meal tomorrow.” 

“Did\marginnote{4.2.10} you say, ‘Buddha’?” —“I did.” —“Did you say, ‘Buddha’?” —“I did.” —“Did you say, ‘Buddha’?” —“I did.” 

“It’s\marginnote{4.2.16} rare in the world to hear the word ‘Buddha’. Is it possible right now to go and visit that Buddha, that fully awakened and perfected one?” 

“Now\marginnote{4.2.19} is the wrong time to visit the Buddha. Tomorrow morning is a good time.” 

Because\marginnote{4.2.21} he went to bed preoccupied with the Buddha—“Tomorrow morning I will go and visit that Buddha, that fully awakened and perfected one!”—he got up three times during the night thinking it was light. 

\textsanskrit{Anāthapiṇḍika}\marginnote{4.3.1} then went to the Sivaka gate, which was opened by spirits. Then, as he was leaving town, the light disappeared and darkness descended. Paralyzed with fear and getting goosebumps all over, he wanted to turn back. But then the invisible spirit Sivaka spoke up: 

\begin{verse}%
“A\marginnote{4.3.6} hundred elephants, a hundred horses, \\
A hundred carriages drawn by mules, \\
A hundred thousand girls \\
Wearing jeweled earrings—\\
None is worth a sixteenth part \\
Of a single step forward. 

Go\marginnote{4.3.12} forward, householder, go forward! \\
Going forward is better for you than going back.” 

%
\end{verse}

The\marginnote{4.3.14} light returned, the darkness disappeared, and his fear subsided. A second and a third time the light disappeared and he was paralyzed with fear, upon which the spirit proclaimed the same verses. And on both occasions the light returned, the darkness disappeared, and his fear subsided. 

\textsanskrit{Anāthapiṇḍika}\marginnote{4.4.1} then went to the \textsanskrit{Sītavana}, the Cool Grove. Just then the Buddha was doing walking meditation outside, having gotten up early in the morning. When the Buddha saw \textsanskrit{Anāthapiṇḍika} coming, he stepped down from the walking path, sat down on the prepared seat, and said, “Come, Sudatta.” \textsanskrit{Anāthapiṇḍika} thought, “The Buddha is calling me by name!” and glad and joyful he went up to him, bowed down with his head at the Buddha’s feet, and said, “Sir, I hope you have slept well?” 

\begin{verse}%
“Indeed,\marginnote{4.4.11} he always sleeps well,\footnote{This and the following verse are also found at \href{https://suttacentral.net/sn10.8/en/brahmali\#12.1}{SN 10.8:12.1} and \href{https://suttacentral.net/an3.35/en/brahmali\#8.1}{AN 3.35:8.1}. } \\
The brahmin who’s extinguished, \\
Who’s not soiled among worldly pleasures, \\
But is cool and without ownership. 

After\marginnote{4.4.15} cutting all attachments, \\
After removing distress from the heart, \\
Calmed, he sleeps well, \\
Having attained peace of mind.” 

%
\end{verse}

The\marginnote{4.5.1} Buddha then gave him a progressive talk—on generosity, morality, and heaven; on the downside, degradation, and defilement of worldly pleasures; and he revealed the benefits of renunciation. When the Buddha knew that his mind was ready, supple, without hindrances, joyful, and confident, he revealed the teaching unique to the Buddhas: suffering, its origin, its end, and the path. And just as a clean and stainless cloth absorbs dye properly, so too, while he was sitting right there, \textsanskrit{Anāthapiṇḍika} experienced the stainless vision of the Truth: “Anything that has a beginning has an end.” 

He\marginnote{4.5.8} had seen the Truth, had reached, understood, and penetrated it. He had gone beyond doubt and uncertainty, had attained to confidence, and had become independent of others in the Teacher’s instruction. He then said to the Buddha, “Wonderful, sir, wonderful! Just as one might set upright what’s overturned, or reveal what’s hidden, or show the way to one who’s lost, or bring a lamp into the darkness so that one with eyes might see what’s there—just so has the Buddha made the Teaching clear in many ways. I go for refuge to the Buddha, the Teaching, and the Sangha of monks. Please accept me as a lay follower who’s gone for refuge for life. And please accept tomorrow’s meal from me together with the Sangha of monks.” The Buddha consented by remaining silent. 

Knowing\marginnote{4.5.17} that the Buddha had consented, \textsanskrit{Anāthapiṇḍika} got up from his seat, bowed down, circumambulated the Buddha with his right side toward him, and left. 

The\marginnote{4.6.1} merchant heard that \textsanskrit{Anāthapiṇḍika} had invited the Sangha headed by the Buddha for the meal on the following day. He said to \textsanskrit{Anāthapiṇḍika}, “You’ve invited the Sangha headed by the Buddha, yet you’ve just arrived here. I’ll pay for it.” 

“There’s\marginnote{4.6.7} no need. I have the means.” 

The\marginnote{4.6.8} householder association of \textsanskrit{Rājagaha} heard that \textsanskrit{Anāthapiṇḍika} had invited the Sangha headed by the Buddha for a meal on the following day. They said to \textsanskrit{Anāthapiṇḍika}, “You’ve invited the Sangha headed by the Buddha, yet you’ve just arrived here. We’ll pay for it.” 

“There’s\marginnote{4.6.14} no need, sirs. I have the means.” 

King\marginnote{4.6.16} Seniya \textsanskrit{Bimbisāra} of Magadha heard that \textsanskrit{Anāthapiṇḍika} had invited the Sangha headed by the Buddha for a meal on the following day. He said to \textsanskrit{Anāthapiṇḍika}, “You’ve invited the Sangha headed by the Buddha, yet you’ve just arrived here. I’ll pay for it.” 

“There’s\marginnote{4.6.22} no need, sir. I have the means.” 

The\marginnote{4.7.1} following morning, in that wealthy merchant’s house, \textsanskrit{Anāthapiṇḍika} had various kinds of fine foods prepared. He then had the Buddha informed that the meal was ready. The Buddha robed up, took his bowl and robe, and went to that merchant’s house where he sat down on the prepared seat together with the Sangha of monks. \textsanskrit{Anāthapiṇḍika} then personally served the various kinds of fine foods to the Sangha of monks headed by the Buddha. When the Buddha had finished his meal and had washed his hands and bowl, \textsanskrit{Anāthapiṇḍika} sat down to one side and said, “Sir, please spend the rainy-season residence at \textsanskrit{Sāvatthī} together with the Sangha of monks.” 

“Buddhas\marginnote{4.7.7} delight in solitude, householder.” 

“I\marginnote{4.7.8} understand, sir, I understand!” 

Then,\marginnote{4.7.9} after instructing, inspiring, and gladdening him with a teaching, the Buddha got up from his seat and left. 

After\marginnote{4.8.1} finishing his business in \textsanskrit{Rājagaha}, \textsanskrit{Anāthapiṇḍika} set out for \textsanskrit{Sāvatthī}. Now at that time \textsanskrit{Anāthapiṇḍika} had many friends and acquaintances who took his advice. On his way to \textsanskrit{Sāvatthī} he told people to establish monasteries, build dwellings, and prepare offerings, saying, “There’s a Buddha in the world! He’s been invited by me and will be traveling this way.” And that’s what they did. When he arrived at \textsanskrit{Sāvatthī}, \textsanskrit{Anāthapiṇḍika} searched all over the town for a place for the Buddha to stay, thinking, “Where might the Buddha stay that’s neither too far from habitation nor too close, that has good access roads and is easily accessible for people who seek him, that has few people during the day and is quiet at night, that’s free from chatter and offers solitude, a private resting place suitable for seclusion?” 

\textsanskrit{Anāthapiṇḍika}\marginnote{4.9.1} saw that Prince Jeta’s park had all these qualities. He then went to Prince Jeta and said, “Sir, please give me your park to set up a monastery.” 

“I\marginnote{4.9.5} wouldn’t give it away even if you covered the park with ten million gold coins.”\footnote{The Pali just says \textit{\textsanskrit{koṭi}}, “ten million”. That this refers to gold coins, \textit{\textsanskrit{hirañña}}, is clear from the events that follow. } 

“It’s\marginnote{4.9.6} a deal!” 

“No\marginnote{4.9.7} it isn’t!” 

They\marginnote{4.9.8} then asked judges to decide on the matter. They said, “Sir, since you gave a price, the park is sold.” 

\textsanskrit{Anāthapiṇḍika}\marginnote{4.9.11} then had gold coins brought out in carts and covered the Jeta Grove with ten million coins. After yet another load had been brought out, there was still a small area near the gatehouse that was not covered. \textsanskrit{Anāthapiṇḍika} told his people, “Go and get more coins. We’ll cover this area too.” 

But\marginnote{4.10.5} Prince Jeta thought, “This must be really worthwhile, as he’s giving up so much money.”\footnote{For an explanation of the rendering “money” for \textit{\textsanskrit{hirañña}}, see Appendix of Technical Terms. } And he said to \textsanskrit{Anāthapiṇḍika}, “That’s enough, householder, don’t cover that area. Let me keep it. It will be my gift.” 

\textsanskrit{Anāthapiṇḍika}\marginnote{4.10.12} thought, “Prince Jeta is a well-known person. It’s very beneficial that such well-known people gain confidence in this spiritual path.” And he granted that area to Prince Jeta. Prince Jeta then had a gatehouse built at that place. 

But\marginnote{4.10.17} \textsanskrit{Anāthapiṇḍika} had dwellings built in the Jeta Grove, and yards, gatehouses, assembly halls, water-boiling sheds, food-storage huts, restrooms, walking-meditation paths, indoor walking-meditation paths, wells, well houses, saunas, sauna sheds, ponds, and roof-covers. 

\subsection*{10. Putting in charge of building work }

When\marginnote{5.1.1} the Buddha had stayed at \textsanskrit{Rājagaha} for as long as he liked, he set out wandering toward \textsanskrit{Vesālī}. When he eventually arrived, he stayed in the hall with the peaked roof in the Great Wood. 

At\marginnote{5.1.4} that time people were doing building work out of respect. And the monks who supervised the building work were respectfully supported with robe-cloth, almsfood, dwellings, and medicinal supplies. Then a certain poor tailor thought, “This must be really worthwhile, seeing as these people do building work with such respect. Why don’t I too do building work?” He then made a mixture of mud, made bricks out of it, and built a wall. But because of his lack of skill, the wall was crooked and fell down. A second and a third time the same thing happened. He then complained and criticized the monks, “These Sakyan monastics teach and instruct only those who give them robe-cloth, almsfood, dwellings, and medicinal supplies. They only supervise their building work. But since I’m poor, nobody teaches, instructs, or supervises me.” 

The\marginnote{5.2.5} monks heard the complaints of that poor tailor. They told the Buddha, who then gave a teaching and addressed the monks: 

\scrule{“You should put a monk in charge of the building work. }

He\marginnote{5.2.9} should make an effort to complete the dwelling as quickly as possible and should repair what’s broken or damaged. 

And\marginnote{5.3.1} he should be put in charge like this. First a monk should be asked, and then a competent and capable monk should inform the Sangha: 

‘Please,\marginnote{5.3.3} venerables, I ask the Sangha to listen. If the Sangha is ready, it should put monk so-and-so in charge of the building work relating to the dwelling of householder so-and-so.\footnote{My rendering “should put (…) in charge” is for the Pali word \textit{dadeyya}, which literally means, “Should give (out)”. Vmv 4.309: \textit{\textsanskrit{Dadeyyāti} \textsanskrit{navakammaṁ} \textsanskrit{adhiṭṭhātuṁ} \textsanskrit{vihāre} \textsanskrit{issariyaṁ} \textsanskrit{dadeyyāti} attho}, “The meaning of \textit{dadeyya} is that they should give control over the dwelling to supervise the building work.” This definition seems required by the context. The expression “the dwelling of householder so-and-so” can only be understood to mean a dwelling \textit{to be built for the Sangha} by householder so-and-so. } This is the motion. 

Please,\marginnote{5.3.6} venerables, I ask the Sangha to listen. The Sangha puts monk so-and-so in charge of the building work relating to the dwelling of householder so-and-so. Any monk who approves of putting monk so-and-so in charge of the building work relating to the dwelling of householder so-and-so should remain silent. Any monk who doesn’t approve should speak up. 

The\marginnote{5.3.10} Sangha has put monk so-and-so in charge of the building work relating to the dwelling of householder so-and-so. The Sangha approves and is therefore silent. I’ll remember it thus.’” 

\subsection*{11. The instruction on the best seat, etc. }

When\marginnote{6.1.1} the Buddha had stayed at \textsanskrit{Vesālī} for as long as he liked, he set out wandering toward \textsanskrit{Sāvatthī}. On that occasion the monks who were the pupils of the monks from the group of six went ahead of the Sangha headed by the Buddha. They then took possession of dwellings and beds, thinking, “This will be for our preceptors and teachers, and also for ourselves.” 

Following\marginnote{6.1.4} behind the Sangha, Venerable \textsanskrit{Sāriputta} was unable to get a bed—the dwellings and beds having all been taken. And so he sat down at the foot of a tree. 

Getting\marginnote{6.1.5} up early in the morning, the Buddha cleared his throat. \textsanskrit{Sāriputta}, too, cleared his throat. 

“Who’s\marginnote{6.1.7} there?” 

“It’s\marginnote{6.1.8} me, sir, \textsanskrit{Sāriputta}.” 

“Why\marginnote{6.1.9} are you sitting here?” 

\textsanskrit{Sāriputta}\marginnote{6.1.10} told the Buddha what had happened. Soon afterwards the Buddha had the Sangha gathered and questioned the monks: 

“Is\marginnote{6.2.2} it true, monks, that the monks who are the pupils of the monks from the group of six did this?” 

“It’s\marginnote{6.2.4} true, sir.” 

The\marginnote{6.2.5} Buddha rebuked them … “How can they act like this? This will affect people’s confidence …” After rebuking them … the Buddha gave a teaching and addressed the monks: 

“Who,\marginnote{6.2.11} monks, deserves the best seat, the best water, and the best almsfood?” 

Some\marginnote{6.2.12} monks said, “Those who’ve gone forth from an aristocratic family deserve the best seat, water, and almsfood.” Others said, “Those who’ve gone forth from a brahmin family deserve the best seat, water, and almsfood.” Still others said, “Those who’ve gone forth from a householder family, the experts on the discourses, the experts on the Monastic Law, the expounders of the Teaching, those who obtain the first absorption, those who obtain the second absorption, those who obtain the third absorption, those who obtain the fourth absorption, the stream-enterers, the once-returners, the nonreturners, the perfected ones, those who have attained the three true insights, or those who have attained the six direct knowledges deserve the best seat, water, and almsfood.” 

The\marginnote{6.3.1} Buddha then addressed the monks: 

\paragraph*{\textsanskrit{Jātaka} }

“Once\marginnote{6.3.2.1} upon a time, monks, there was a great banyan tree on the slopes of the Himalayas.\footnote{Reading \textit{himavantapasse} with the PTS edition. This story also exists as the Tittira \textsanskrit{Jātaka}, number 37 of that collection. } Three friends lived near it: a partridge, a monkey, and an elephant. They were disrespectful, undeferential, and rude toward one another. They thought, ‘If we only knew which one of us was the oldest. We would honor, respect, and esteem him, and we would wait for his instructions.’ 

The\marginnote{6.3.8} partridge and the monkey then asked the elephant, ‘What’s your first memory?’ 

‘When\marginnote{6.3.10} I was young, I stepped over this banyan tree, keeping it between my legs, and the top shoots touched my belly. That’s my first memory.’ 

The\marginnote{6.3.12} partridge and the elephant asked the monkey, ‘What’s your first memory?’ 

‘When\marginnote{6.3.14} I was young, I sat on the ground and ate the top shoots of this banyan tree. That’s my first memory.’ 

The\marginnote{6.3.16} monkey and the elephant asked the partridge, ‘What’s your first memory?’ 

‘In\marginnote{6.3.18} such and such a spot there was a great banyan tree. I ate one of its fruits and defecated here. This banyan tree has grown from that. Well then, I must be the oldest one.’ 

The\marginnote{6.3.22} monkey and the elephant said to the partridge, ‘You’re the oldest. We will honor, respect, and esteem you, and we’ll wait for your instructions.’ 

The\marginnote{6.3.25} partridge had the monkey and the elephant take the five precepts, and he also undertook them himself. They were respectful, deferential, and courteous toward one another. And when they died, they were reborn in a happy, heavenly destination. In this way the spiritual life called \textit{tittiriya} came to be.\footnote{There seems to be a play on words here, in which \textit{tittiriya} refers both to a partridge and to a class of brahmins. } 

\begin{verse}%
Those\marginnote{6.3.28} who respect the seniors, \\
And who are learned in the Teaching, \\
They are praised while still alive, \\
And then go to a good destination. 

%
\end{verse}

“Even\marginnote{6.4.1} those animals, monks, were respectful, deferential, and courteous toward one another. Having gone forth on this well-proclaimed spiritual path, will you look good if you are disrespectful, undeferential, and rude toward one another?\footnote{Sp 3.248: \textit{Ettha tanti \textsanskrit{nipātamattaṁ}, idha kho bhikkhave \textsanskrit{sobheyyāthāti} attho}, “Here \textit{\textsanskrit{taṁ}} is a mere indeclinable. The meaning is, ‘In this case, monks, would you shine?’” } This will affect people’s confidence …” After rebuking them … the Buddha gave a teaching and addressed the monks: 

\scrule{“You should do these things according to seniority: bowing down, standing up, raising your joined palms, doing acts of respect, giving the best seat, giving the best water, and giving the best almsfood. }

\scrule{But what belongs to the Sangha shouldn’t be reserved according to seniority.\footnote{Which effectively means that such things cannot be reserved at all. } If you do, you commit an offense of wrong conduct.” }

\subsection*{12. Persons one should not pay respect to, etc. }

\scrule{“Monks, you shouldn’t pay respect to any of these ten kinds of persons: one who’s been given the full ordination after you; one who isn’t fully ordained; one who belongs to a different Buddhist sect who’s senior to you, but who speaks contrary to the Teaching; a woman; a \textit{\textsanskrit{paṇḍaka}}; one who’s on probation; one who deserves to be sent back to the beginning; one who deserves the trial period; one who’s undertaking the trial period; one who deserves rehabilitation.\footnote{For an explanation of the word \textit{\textsanskrit{paṇḍaka}}, see Appendix of Technical Terms. } }

\scrule{But you should pay respect to these three kinds of persons: one who’s been given the full ordination before you; one who belongs to a different Buddhist sect who’s senior to you and who speaks in accordance with the Teaching; and in this world with its gods, lords of death, and supreme beings, in this society with its monastics and brahmins, its gods and humans, you should pay respect to the Buddha, perfected and fully awakened.” }

\subsection*{13. The prohibition against reserving seats }

At\marginnote{7.1} that time people prepared roof covers, mats, and places to stay for the Sangha.\footnote{Vmv 4.313 defines \textit{\textsanskrit{okāsa}} as \textit{\textsanskrit{nivāsokāsa}}, “a place to stay”. } The monks who were the pupils of the monks from the group of six, thought, “The Buddha has instructed that what belongs to the Sangha shouldn’t be reserved according to seniority. But he’s given no such instruction about what has merely been prepared for the Sangha.” They then went ahead of the Sangha headed by the Buddha and took possession of the roof-covers, mats, and places to stay, thinking,\footnote{To make this sentence fit the context I read \textit{\textsanskrit{saṅghikaññeva}} as \textit{\textsanskrit{saṅghikaṁ} + na + eva}. } “This will be for our preceptors and teachers, and also for ourselves.” 

Following\marginnote{7.5} behind the Sangha, Venerable \textsanskrit{Sāriputta} was unable to find a place to stay—the roof-covers, the mats, and the places to stay having all been taken. And so he sat down at the foot of a tree. 

Getting\marginnote{7.6} up early in the morning, the Buddha cleared his throat. \textsanskrit{Sāriputta}, too, cleared his throat. 

“Who’s\marginnote{7.8} there?” 

“It’s\marginnote{7.9} me, sir, \textsanskrit{Sāriputta}.” 

“Why\marginnote{7.10} are you sitting here?” 

\textsanskrit{Sāriputta}\marginnote{7.11} told the Buddha what had happened. Soon afterwards the Buddha had the Sangha gathered and questioned the monks: “Is it true, monks, that the monks who are the pupils of the monks from the group of six did this?” 

“It’s\marginnote{7.16} true, sir.” … 

After\marginnote{7.17} rebuking them … the Buddha gave a teaching and addressed the monks: 

\scrule{“Even what has merely been prepared for the Sangha shouldn’t be reserved according to seniority.\footnote{Again, this effectively means that such things cannot be reserved at all. } If you do, you commit an offense of wrong conduct.” }

\subsection*{14. The allowance for what belongs to a householder }

At\marginnote{8.1} that time people prepared high and luxurious resting places in the dining halls in inhabited areas, that is: high couches, luxurious couches, long-fleeced woolen rugs, multi-colored woolen rugs, white woolen rugs, red woolen rugs, cotton-down quilts, woolen rugs decorated with the images of predatory animals, woolen rugs with long fleece on one side, woolen rugs with long fleece on both sides, sheets of silk embroidered with gems, silken sheets, woolen rugs like a dancer’s rug, elephant-back rugs, horse-back rugs, carriage-seat rugs, rugs made of black antelope hide, exquisite sheets made of \textit{\textsanskrit{kadalī}}-deer hide, seats with canopies, seats with red cushions at each end.\footnote{For further discussion of these renderings, see Appendix of Furniture. } Being afraid of wrongdoing, the monks did not sit on them. They told the Buddha. 

\scrule{“Apart from high couches, luxurious couches, and cotton-down quilts, I allow you to sit down on what belongs to householders, but not to lie down on it.” }

At\marginnote{8.8} that time people prepared beds and benches upholstered with cotton down in the dining halls in inhabited areas. Being afraid of wrongdoing, the monks did not sit on them. They told the Buddha. 

\scrule{“I allow you to sit down on what belongs to householders, but not to lie down on it.” }

\subsection*{15. The expression of appreciation for the Jeta Grove dwellings }

Wandering\marginnote{9.1.1} on, the Buddha eventually arrived at \textsanskrit{Sāvatthī}, where he stayed in the Jeta Grove, \textsanskrit{Anāthapiṇḍika}’s Monastery. \textsanskrit{Anāthapiṇḍika} then went to the Buddha, bowed, sat down, and said, “Sir, please accept tomorrow’s meal from me together with the Sangha of monks.” The Buddha consented by remaining silent. Knowing that the Buddha had consented, \textsanskrit{Anāthapiṇḍika} got up from his seat, bowed down, circumambulated the Buddha with his right side toward him, and left. 

The\marginnote{9.1.8} following morning \textsanskrit{Anāthapiṇḍika} had various kinds of fine foods prepared. He then had the Buddha informed that the meal was ready. The Buddha robed up, took his bowl and robe, and went to \textsanskrit{Anāthapiṇḍika}’s house where he sat down on the prepared seat together with the Sangha of monks. \textsanskrit{Anāthapiṇḍika} then personally served various kinds of fine foods to the Sangha of monks headed by the Buddha. When the Buddha had finished his meal and had washed his hands and bowl, \textsanskrit{Anāthapiṇḍika} sat down to one side and said, “Sir, what should I do in regard to the Jeta Grove?” 

“You\marginnote{9.1.14} should dedicate the Jeta Grove to the Sangha as a whole, both present and future.” 

“Yes,\marginnote{9.1.15} sir.” And he did just that. 

The\marginnote{9.2.1} Buddha then expressed his appreciation with these verses: 

\begin{verse}%
“Cold\marginnote{9.2.2} and heat are kept away, \\
And so are predatory beasts, \\
And creeping animals and mosquitoes, \\
And also chill and rain. 

They\marginnote{9.2.6} keep away the wind and burning sun, \\
When those awful things arise. \\
Their purpose is to shelter and for happiness, \\
To attain absorption and to see clearly. 

Giving\marginnote{9.2.10} dwellings to the Sangha \\
Is praised as the best by the Buddha. \\
Therefore the wise man, \\
Seeing what’s beneficial for himself, 

Should\marginnote{9.2.14} build delightful dwellings \\
And have the learned stay there. \\
Food, drink, cloth, and dwellings—\\
With an inspired mind, 

He\marginnote{9.2.18} should give to them, \\
The upright ones. \\
They will give him the Teaching \\
For removing all suffering; \\
And understanding this Teaching in this very life, \\
He attains extinguishment, free of corruptions.” 

%
\end{verse}

The\marginnote{9.2.24} Buddha then got up from his seat and left. 

\subsection*{16. Reservation of seats, etc. }

On\marginnote{10.1.1} one occasion a certain government official who was an \textsanskrit{Ājīvaka} disciple was offering a meal to the Sangha. Arriving late, Venerable Upananda the Sakyan made the nearest monk get up before he had finished his meal. There was an uproar in the dining hall.\footnote{For an explanation of the rendering “dining hall” for \textit{bhattagga}, see Appendix of Technical Terms. } That official then complained and criticized him, “How can the Sakyan monastics arrive late and make the nearest monk get up in the middle of his meal? There was an uproar in the dining hall. It’s impossible to eat as much as you like when you’re not seated.” 

The\marginnote{10.1.8} monks heard the complaints of that official, and the monks of few desires complained and criticized Upananda, “How could he act like this?” They told the Buddha what had happened. 

“Is\marginnote{10.1.13} it true, Upananda, that you acted like this?” 

“It’s\marginnote{10.1.14} true, sir.” 

The\marginnote{10.1.15} Buddha rebuked him … “Foolish man, how could you act like this? This will affect people’s confidence …” After rebuking him … the Buddha gave a teaching and addressed the monks: 

\scrule{“You shouldn’t make a monk get up who hasn’t finished his meal. If you do, you commit an offense of wrong conduct. }

\scrule{If you’re asked to get up, and you’ve already refused an invitation to eat more, you should say, “Please go and get some water.”\footnote{Sp 3.316: \textit{\textsanskrit{Pavārito} ca \textsanskrit{hotīti} \textsanskrit{yaṁ} so \textsanskrit{vuṭṭhāpeti}, \textsanskrit{ayañca} bhikkhu \textsanskrit{pavārito} ca hoti, tena vattabbo – “gaccha \textsanskrit{udakaṁ} \textsanskrit{āharāhī}”ti}, “‘Have already refused an invitation to eat more’: whom he asks to get up, this monk has already refused an invitation to eat more. He should say, ‘Go and get some water.’” In other words, it is the monk who is asked to get up who has refused the invitation to eat more. The point is that this prohibits him from continuing his meal elsewhere after getting up. According to \href{https://suttacentral.net/pli-tv-bu-vb-pc35/en/brahmali\#2.15.1}{Bu Pc 35:2.15.1}, once you have refused an invitation to eat more and risen from your seat, you cannot eat anything further. For further discussion, see \textit{\textsanskrit{pavāraṇā}} in Appendix of Technical Terms. } If the other goes, all is well. If not, you should properly swallow the mouthful and give the seat to the more senior monk. Under no circumstances should you block a more senior monk from a seat. If you do, you commit an offense of wrong conduct.” }

On\marginnote{10.2.1} one occasion the monks from the group of six asked the sick monks to get up. The sick monks said, “We’re not able to get up. We’re sick.” 

Saying,\marginnote{10.2.4} “We’ll make the venerables get up,” they took hold of them, lifted them up, and then released them when they were standing. The sick monks fainted and collapsed. 

\scrule{“You shouldn’t make the sick get up. If you do, you commit an offense of wrong conduct.” }

The\marginnote{10.2.9} monks from the group of six took possession of the best beds, saying, “We’re sick and no-one can make us get up.” 

\scrule{“You should give suitable beds to those who are sick.” }

The\marginnote{10.2.13} monks from the group of six used a pretext to reserve resting places.\footnote{Sp 4.316: \textit{\textsanskrit{Lesakappenāti} appakena \textsanskrit{sīsābādhādimattena}}, “Pretext: merely a minor headache.” } 

\scrule{“You shouldn’t use a pretext to reserve a resting place.\footnote{For an explanation of rendering \textit{\textsanskrit{senāsana}} as “resting place”, see Appendix of Technical Terms. } If you do, you commit an offense of wrong conduct.” }

At\marginnote{11.1.1} that time the monks from the group of seventeen were repairing a large dwelling nearby, intending to stay there for the rainy season. The monks from the group of six saw this and said, “These monks from the group of seventeen are repairing a dwelling. Let’s throw them out.” But some of them said, “Let’s wait until they’ve finished repairing it.” 

Soon\marginnote{11.1.9} afterwards the monks from the group of six said to those from the group of seventeen, “Leave, this dwelling is ours.” 

“Shouldn’t\marginnote{11.1.11} you have told us beforehand? We would have repaired another one.” 

“Doesn’t\marginnote{11.1.13} this dwelling belong to the Sangha?” 

“Yes\marginnote{11.1.14} it does.” 

“Well\marginnote{11.1.15} then, leave. This dwelling is ours.” 

“The\marginnote{11.1.16} dwelling is large. You can stay here and so can we.” 

But\marginnote{11.1.18} they said, “Leave, this dwelling is ours,” and they grabbed them by the necks and threw them out in anger. The monks from the group of seventeen cried. When other monks asked them why, they told them what had happened. 

The\marginnote{11.1.23} monks of few desires complained and criticized them, “How could the monks from the group of six angrily throw other monks out of a dwelling belonging to the Sangha?” 

They\marginnote{11.1.25} told the Buddha. Soon afterwards he had the Sangha gathered and questioned the monks: “Is it true, monks, that you did this?” 

“It’s\marginnote{11.1.27} true, sir.” 

The\marginnote{11.1.28} Buddha rebuked them … and after giving a teaching, he addressed the monks: 

\scrule{“You shouldn’t, in anger, throw a monk out of a dwelling belonging to the Sangha. If you do, you should be dealt with according to the rule. You should allocate the dwellings.” }

\subsection*{17. The appointment of allocators of dwellings }

The\marginnote{11.2.1} monks thought, “Who should allocate the dwellings?”\footnote{Again, see \textit{\textsanskrit{senāsana}} in Appendix of Technical Terms. } They told the Buddha, who then gave a teaching and addressed the monks: 

“You\marginnote{11.2.4} should appoint a monk who has five qualities as the allocator of dwellings: one who’s not biased by favoritism, ill will, confusion, or fear, and who knows which dwellings have and have not been allocated. And he should be appointed like this. First a monk should be asked, and then a competent and capable monk should inform the Sangha: 

‘Please,\marginnote{11.2.8} venerables, I ask the Sangha to listen. If the Sangha is ready, it should appoint monk so-and-so as allocator of dwellings. This is the motion. 

Please,\marginnote{11.2.11} venerables, I ask the Sangha to listen. The Sangha appoints monk so-and-so as allocator of dwellings. Any monk who approves of appointing monk so-and-so as allocator of dwellings should remain silent. Any monk who doesn’t approve should speak up. 

The\marginnote{11.2.15} Sangha has appointed monk so-and-so as allocator of dwellings. The Sangha approves and is therefore silent. I’ll remember it thus.’” 

The\marginnote{11.3.1} allocators of dwellings thought, “How should we allocate the dwellings?” They told the Buddha. 

\scrule{“First you should count the monks and the beds. You should then allocate one monk to each bed.”\footnote{“One monk to each bed” renders \textit{seyyaggena}. Sp 4.318: \textit{\textsanskrit{Seyyaggenāti} \textsanskrit{seyyāparicchedena}, \textsanskrit{vassūpanāyikadivase} \textsanskrit{kālaṁ} \textsanskrit{ghosetvā} \textsanskrit{ekamañcaṭṭhānaṁ} ekassa bhikkhuno \textsanskrit{gāhetuṁ} \textsanskrit{anujānāmīti} attho}, “\textit{Seyyaggena} means by dividing the beds; the meaning is I allow you, after announcing the time on the day for entering the rainy-season residence, to allocate one bed-place to one monk.” } }

When\marginnote{11.3.5} they had allocated the beds, there were beds left over. 

\scrule{“You should allocate one monk to each dwelling.” }

When\marginnote{11.3.7} they had allocated the dwellings, there were dwellings left over. 

\scrule{“You should allocate one monk to each yard.” }

When\marginnote{11.3.9} they had allocated the yards, there were yards left over. 

\scrule{“You should give out additional shares.\footnote{Sp 4.318: \textit{\textsanskrit{Anubhāganti} puna aparampi \textsanskrit{bhāgaṁ} \textsanskrit{dātuṁ}. Atimandesu hi \textsanskrit{bhikkhūsu} ekekassa bhikkhuno dve \textsanskrit{tīṇi} \textsanskrit{pariveṇāni} \textsanskrit{dātabbāni}}, “\textit{\textsanskrit{Anubhāga}} to give another part too. Two or three yards are to be given to a single monk among excessively foolish monks.” } If another monk arrives after the additional shares have been allocated, then, if you’re unwilling, you need not give him a share.” }

On\marginnote{11.3.12} one occasion the monks allocated a dwelling to one who was outside the monastery zone.\footnote{For an explanation of the rendering “monastery zone” for \textit{\textsanskrit{sīmā}}, see Appendix of Technical Terms. } 

\scrule{“You shouldn’t allocate a dwelling to one outside the monastery zone. If you do, you commit an offense of wrong conduct.” }

After\marginnote{11.3.16} accepting a dwelling, the monks reserved it at all times. 

\scrule{“After accepting a dwelling, you shouldn’t reserve it at all times. If you do, you commit an offense of wrong conduct. I allow you to reserve it for the three months of the rainy-season residence, but not at other times.” }

The\marginnote{11.4.1} monks thought, “How many times are there for the allocation of dwellings?” 

“There\marginnote{11.4.4} are three times for the allocation of dwellings: the first, the second, and when given up in between.\footnote{“When given up in between” renders \textit{\textsanskrit{antarāmuttaka}}. This is obscure. Sp-yoj 4.318: \textit{\textsanskrit{Antarā} \textsanskrit{dvīhi} \textsanskrit{vassūpanāyikadivasehi} mutte \textsanskrit{kāle} \textsanskrit{gāho} \textsanskrit{antarāmuttako}}, “When the time it is given up is in between the two days for entering the rainy season retreat, then the allocation is given up in between.” This would seem to refer to what is relinquished between the former and the latter entry to the rainy-season residence. I am not sure, however, whether this judgment makes sense, since such a dwelling might then be allocated for the second rainy-season residence, which is already covered. It seems to me, rather, that \textit{\textsanskrit{antarāmuttaka}} must refer to any dwelling that has been either vacant for the whole rainy-season residence or vacated during the rainy-season residence. } The first allocation is on the day after the full moon of July. The second allocation is one month after the full moon of July. The allocation of what is given up in between is on the day after the invitation ceremony and is for the purpose of spending the next rains residence.” 

\scend{The second section for recitation is finished. }

\section*{The third section for recitation }

\subsection*{Regulations on dwellings, furniture, etc. }

On\marginnote{12.1.1} one occasion Venerable Upananda the Sakyan had accepted a dwelling at \textsanskrit{Sāvatthī}, but then went to a certain village monastery, where he was also allocated a dwelling. The monks there thought, “This Upananda is quarrelsome and argumentative, and creates legal issues in the Sangha. If he spends the rainy season here, none of us will be at ease. Well then, let’s question him.” 

And\marginnote{12.1.7} they said to Upananda, “Haven’t you been allocated a dwelling at \textsanskrit{Sāvatthī}?” 

“Yes,\marginnote{12.1.9} I have.” 

“But\marginnote{12.1.10} if it’s only you, why do you reserve two dwellings?” 

“I’ll\marginnote{12.1.11} give up this one and take the one at \textsanskrit{Sāvatthī}.” 

The\marginnote{12.1.13} monks of few desires complained and criticized him, “How could Upananda reserve two dwellings for himself?” They told the Buddha. Soon afterwards he had the Sangha gathered and questioned Upananda: “Is it true, Upananda, that you did this?” 

“It’s\marginnote{12.1.17} true, sir.” 

The\marginnote{12.1.18} Buddha rebuked him … “Foolish man, how could you reserve two dwellings for yourself? When you accepted a dwelling there, the dwelling here was given up, and when you accepted a dwelling here, the dwelling there was given up.\footnote{My translation is a paraphrasing that accords with the explanation given in the commentary at Sp 4.319. The Canonical phrasing is succinct to the point of being incomprehensible. } You’re now excluded from both. This will affect people’s confidence …” After rebuking him … the Buddha gave a teaching and addressed the monks: 

\scrule{“A single monk shouldn’t reserve two dwellings. If you do, you commit an offense of wrong conduct.” }

At\marginnote{13.1.1} one time the Buddha was giving many talks on the Monastic Law. He spoke in praise of it and of learning it, and he repeatedly praised Venerable \textsanskrit{Upāli}. When they heard this, the monks thought, “Well then, let’s learn the Monastic Law from Venerable \textsanskrit{Upāli}.” And many monks, both senior and junior, as well as those of middle standing, learned the Monastic Law from \textsanskrit{Upāli}. 

Out\marginnote{13.1.6} of respect for the senior monks, \textsanskrit{Upāli} taught while standing. And out of respect for the Teaching, the senior monks, too, were standing. They all became tired. They told the Buddha. 

\scrule{“A junior monk who’s teaching should sit on a similar or higher seat out of respect for the Teaching. A senior monk who’s being taught should sit on a similar or lower seat out of respect for the Teaching.” }

On\marginnote{13.2.1} one occasion many monks were standing in the presence of \textsanskrit{Upāli}, honoring the recitation. They became tired. 

\scrule{“I allow those who are entitled to sit on the same seat to sit together.” }

The\marginnote{13.2.4} monks thought, “Who are entitled to sit on the same seat?” 

\scrule{“I allow those with a difference of three years or less in seniority to sit together.” }

On\marginnote{13.2.8} one occasion a number of monks who were entitled to sit on the same seat were seated on a bed. The bed broke. They were seated on the same bench, and the bench broke. 

\scrule{“I allow a maximum of three on the same bed or bench.” }

The\marginnote{13.2.11} beds and benches still broke. 

\scrule{“I allow a maximum of two on the same bed or bench.” }

At\marginnote{13.2.13} that time, being afraid of wrongdoing, monks who were not entitled to sit on the same seat did not sit together on a long seat. 

\scrule{“I allow those who aren’t entitled to sit on the same seat to sit together on a long seat, except with a \textit{\textsanskrit{paṇḍaka}}, a woman, or a hermaphrodite.”\footnote{For the rendering “hermaphrodite” for \textit{\textsanskrit{ubhatobyañjanaka}}, see Appendix of Technical Terms. } }

The\marginnote{13.2.16} monks thought, “What’s the size of the smallest long seat?” 

\scrule{“A seat for three is the smallest long seat.” }

At\marginnote{14.1.1} one time \textsanskrit{Visākhā} \textsanskrit{Migāramātā} wanted to build a stilt house for the benefit of the Sangha, including a porch and elephant globes.\footnote{“Elephant globes” renders \textit{hatthinakhaka}, literally, “elephant nails”. Sp 4.319: \textit{Hatthinakhakanti hatthikumbhe \textsanskrit{patiṭṭhitaṁ}; \textsanskrit{evaṁ} katassa \textsanskrit{kiretaṁ} \textsanskrit{nāmaṁ}}, “\textit{Hatthinakhaka} means established on elephant globes; this is a name for what is made in this way.” Elephant globes are the frontal globes on an elephant’s forehead. } The monks thought, “What stilt-house equipment has the Buddha allowed and what hasn’t he allowed?” They told the Buddha. 

\scrule{“I allow all stilt-house equipment.” }

At\marginnote{14.1.6} one time King Pasenadi of Kosala’s grandmother had just died. As a result, many unallowable goods were offered to the Sangha, that is:\footnote{All the following items seem to fall into the category seat/bed (see \href{https://suttacentral.net/pli-tv-kd16/en/brahmali\#8.7}{Kd 16:8.7} and \href{https://suttacentral.net/pli-tv-kd5/en/brahmali\#10.5.1}{Kd 5:10.5.1}), the two often not being differentiated in the Canonical texts. In other words, beds were often used as seats, and vice versa. } high couches, luxurious couches, long-fleeced woolen rugs, multi-colored woolen rugs, white woolen rugs, red woolen rugs, cotton-down quilts, woolen rugs decorated with the images of predatory animals, woolen rugs with long fleece on one side, woolen rugs with long fleece on both sides, sheets of silk embroidered with gems, silken sheets, woolen rugs like a dancer’s rug, elephant-back rugs, horse-back rugs, carriage-seat rugs, rugs made of black antelope hide, exquisite sheets made of \textit{\textsanskrit{kadalī}}-deer hide, seats with canopies, and seats with red cushions at each end.\footnote{For further discussion of these, see Appendix of Furniture. } 

\scrule{“I allow you to use a high couch after cutting its legs down to size, to use a luxurious couch after removing the images of predatory animals, to make a pillow after removing the cotton down from the cotton-down quilt, and to make floor covers of the rest.” }

\subsection*{19. What is not to be given away }

At\marginnote{15.1.1} one time in a village monastery not far from \textsanskrit{Sāvatthī} the resident monks were fed up with assigning dwellings to monks who were coming and going. They considered this and thought, “Well, let’s give all the dwellings belonging to the Sangha to one of us. We’ll then use what belongs to him.” And they did just that. 

When\marginnote{15.1.7} newly-arrived monks said to them, “Please assign us a dwelling,” they replied, “There aren’t any dwellings belonging to the Sangha. We’ve given them to one monk.” 

“So\marginnote{15.1.11} you’ve given away the dwellings belonging to the Sangha?” 

“Yes.”\marginnote{15.1.12} 

The\marginnote{15.1.13} monks of few desires complained and criticized them, “How could they give away the dwellings belonging to the Sangha?” They told the Buddha. Soon afterwards he had the Sangha gathered and questioned the monks: “Is it true, monks, that they did this?” 

“It’s\marginnote{15.1.17} true, sir.” 

The\marginnote{15.1.18} Buddha rebuked them, “How could those foolish men give away dwellings belonging to the Sangha? This will affect people’s confidence …” After rebuking them … the Buddha gave a teaching and addressed the monks: 

\scrule{“There are five things not to be given away, either by a sangha, a group, or an individual. Even if given away, they’re not actually given away. If you give any of them away, you commit a serious offense. }

What\marginnote{15.2.4} five? 

\begin{enumerate}%
\item A monastery or the site of a monastery %
\item A dwelling or the site of a dwelling %
\item A bed, bench, mattress, or pillow %
\item A metal pot, a metal jar, a metal bucket, a metal bowl, an adz, a hatchet, an ax, a spade, or a chisel %
\item A creeper, bamboo, reed, grass, clay, wooden goods, or ceramic goods.”\footnote{I have rendered \textit{\textsanskrit{muñja}}-reed and \textit{pabbaja}-reed with the single word “reed”. I am not aware that these two kinds of reed can be distinguished in English. } %
\end{enumerate}

\subsection*{20. What is not to be distributed }

When\marginnote{16.1.1} the Buddha had stayed at \textsanskrit{Sāvatthī} for as long as he liked, he set out wandering toward \textsanskrit{Kīṭāgiri} with a large sangha of five hundred monks, including \textsanskrit{Sāriputta} and \textsanskrit{Mahāmoggallāna}. The monks Assaji and Punabbasuka heard about this and said, “Well then, let’s distribute all the dwellings belonging to the Sangha. \textsanskrit{Sāriputta} and \textsanskrit{Mahāmoggallāna} are in the grip of bad desires. So let’s not assign them any dwellings.” And they distributed all the dwellings belonging to the Sangha. 

When\marginnote{16.1.7} the Buddha eventually arrived at \textsanskrit{Kīṭāgiri}, he said to a group of monks, “Go to the monks Assaji and Punabbasuka and say, ‘The Buddha is coming with a large sangha of five hundred monks, including \textsanskrit{Sāriputta} and \textsanskrit{Mahāmoggallāna}. Please assign dwellings to the Buddha, to the Sangha of monks, and to \textsanskrit{Sāriputta} and \textsanskrit{Mahāmoggallāna}.’” 

Saying,\marginnote{16.1.13} “Yes, sir,” they did just that. 

The\marginnote{16.1.16} monks Assaji and Punabbasuka replied, “There aren’t any dwellings belonging to the Sangha. We’ve shared them all out. The Buddha is welcome and he may stay wherever he likes. But \textsanskrit{Sāriputta} and \textsanskrit{Mahāmoggallāna} are in the grip of bad desires. We won’t assign them any dwellings.” 

“So\marginnote{16.2.1} you’ve distributed the dwellings belonging to the Sangha?” 

“Yes.”\marginnote{16.2.2} 

The\marginnote{16.2.3} monks of few desires complained and criticized them, “How could they distribute the dwellings belonging to the Sangha?” They told the Buddha. Soon afterwards he had the Sangha gathered and questioned the monks: “Is it true, monks, that they did this?” 

“It’s\marginnote{16.2.7} true, sir.” 

The\marginnote{16.2.8} Buddha rebuked them, “How could those foolish men distribute the dwellings belonging to the Sangha? This will affect people’s confidence …” After rebuking them … the Buddha gave a teaching and addressed the monks: 

\scrule{“There are five things not to be distributed, either by a sangha, a group, or an individual.\footnote{It is not immediately clear how “distributing” is different from “giving” in the previous section. Presumably “giving” refers to a change in ownership. Following the usage elsewhere, such as in the case of the distribution of dwellings, it seems reasonable to assume that “distributing” also here refers to a right of usage, not to a change in ownership. We can surmise from the context that “distribution” gives a stronger right to usage than “allocation”, for which see below. } Even if distributed, they’re not actually distributed. If you distribute any of them, you commit a serious offense. }

What\marginnote{16.2.16} five? 

\begin{enumerate}%
\item A monastery or the land of a monastery %
\item A dwelling or the land of a dwelling %
\item A bed, bench, mattress, or pillow %
\item A metal pot, a metal jar, a metal bucket, a metal bowl, an adz, a hatchet, an ax, a spade, or a chisel %
\item A creeper, bamboo, reed, grass, clay, wooden goods, or ceramic goods.” %
\end{enumerate}

\subsection*{21. Discussion on putting in charge of building work }

When\marginnote{17.1.1} the Buddha had stayed at \textsanskrit{Kīṭāgiri} for as long as he liked, he set out wandering toward \textsanskrit{Āḷavī}. When he eventually arrived, he stayed at \textsanskrit{Aggāḷava} Shrine. 

At\marginnote{17.1.4} that time the monks of \textsanskrit{Āḷava} put monks in charge of building work such as this: the mere filling of gaps,\footnote{Vmv 4.323: \textit{\textsanskrit{Pāḷiyaṁ} \textsanskrit{piṇḍanikkhepanamattenātiādīsu} \textsanskrit{khaṇḍaphullaṭṭhāne} \textsanskrit{mattikāpiṇḍaṭṭhapanaṁ} \textsanskrit{piṇḍanikkhepanaṁ} \textsanskrit{nāma}}, “Among \textit{\textsanskrit{piṇḍanikkhepanamattena}}, etc., in the Canonical text, the placing of bits of clay in the gaps and cracks is called \textit{\textsanskrit{piṇḍanikkhepana}}.” } the mere plastering of walls, the mere hanging of doors, the mere making of door jambs, the mere making of windows, the mere application of white coloring, the mere application of black coloring, the mere treatment with red ocher, the mere covering with a roof, the mere fastening of a roof,\footnote{Vmv 4.323: \textit{\textsanskrit{Bandhanaṁ} \textsanskrit{nāma} \textsanskrit{daṇḍavalliādīhi} chadanabandhanameva}, “Just the binding of a roof with sticks and creepers, etc., is called \textit{bandhana}.” } the mere fixing of cornices,\footnote{The meaning of \textit{\textsanskrit{bhaṇḍikāṭṭhapanamatta}} is not clear. I follow the suggestion given in DOP, sv. “\textit{kapota}”. } the mere repair of what was defective or broken, and the mere plastering of floors;\footnote{Sp 4.323: \textit{\textsanskrit{Paribhaṇḍakaraṇamattenāti} \textsanskrit{gomayaparibhaṇḍakasāvaparibhaṇḍakaraṇamattena}}, “\textit{\textsanskrit{Paribhaṇḍakaraṇamattena}} means merely doing \textit{\textsanskrit{paribhaṇḍa}} with cow dung or with a bitter substance.” Vmv 4.318: \textit{\textsanskrit{Kasāvaparibhaṇḍanti} \textsanskrit{kasāvarasehi} \textsanskrit{bhūmiparikammaṁ}}, “\textit{\textsanskrit{Paribhaṇḍa}} with a bitter substance means treating the floor with a bitter substance.” } and they put monks in charge of building work for twenty years, for thirty years, and for life; and they put monks in charge of building work for life on finished dwellings.\footnote{Sp 4.323: \textit{\textsanskrit{Dhūmakālikanti} \textsanskrit{idaṁ} \textsanskrit{yāvassa} \textsanskrit{citakadhūmo} na \textsanskrit{paññāyati}, \textsanskrit{tāva} \textsanskrit{ayaṁ} \textsanskrit{vihāro} \textsanskrit{etassevāti} \textsanskrit{evaṁ} \textsanskrit{dhūmakāle} \textsanskrit{apaloketvā} \textsanskrit{katapariyositaṁ} \textsanskrit{vihāraṁ} denti}, “\textit{\textsanskrit{Dhūmakālika}}: here, ‘As long as the smoke from the funeral pile is not seen, until then this dwelling is for this one’, in this way, looking for the time of the smoke, they give a dwelling that has been finished.” } 

The\marginnote{17.1.22} monks of few desires complained and criticized them, “How can the monks at \textsanskrit{Āḷavī} put monks in charge of such kinds of work?” They told the Buddha. … “Is it true, monks, that they do this?” “It’s true, sir.” … After rebuking them … the Buddha gave a teaching and addressed the monks: 

\scrule{“You shouldn’t put monks in charge of building work such as this: the mere filling of gaps, the mere plastering of walls, the mere hanging of doors, the mere making of door jambs, the mere making of windows, the mere application of white coloring, the mere application of black coloring, the mere treatment with red ocher, the mere covering with a roof, the mere fastening of a roof, the mere fixing of cornices, the mere repair of what is defective or broken, or the mere plastering of floors; and you shouldn’t put monks in charge of building work for twenty years, for thirty years, or for life, or put monks in charge of building work for life on finished dwellings. If you do, you commit an offense of wrong conduct. }

\scrule{I allow you to put monks in charge of building work that isn’t yet started or that’s partially complete. For a small dwelling, you should inspect the work and then put a monk in charge of the building work for five or six years. For a small stilt house, you should inspect the work and then put a monk in charge of the building work for seven or eight years. For a large dwelling or stilt house, you should inspect the work and then put a monk in charge of the building work for ten or twelve years.”\footnote{“A small stilt house” renders \textit{\textsanskrit{aḍḍhayoga}}. Vmv 3.73: \textit{\textsanskrit{Aḍḍhayogoti} \textsanskrit{ekasālo} \textsanskrit{dīghapāsādo}}, “An \textit{\textsanskrit{aḍḍhayoga}} is a long stilt house with a single room.” That the \textit{\textsanskrit{aḍḍhayoga}} is smaller than a regular \textit{\textsanskrit{pāsāda}}, which is the generic term for a “stilt house”, is also implied by the duration of the work that may be given. } }

On\marginnote{17.2.1} one occasion the monks put one person in charge of the building work on all the dwellings.\footnote{Vmv 4.323: \textit{Ekassa sabbesu \textsanskrit{vihāresu} \textsanskrit{navakammaṁ} \textsanskrit{detīti} attho}, “The meaning is they gave the building work on all the buildings to one person.” } 

\scrule{“You shouldn’t put one person in charge of the building work on all the dwellings. If you do, you commit an offense of wrong conduct.” }

On\marginnote{17.2.5} one occasion the monks put one person in charge of the building work on two dwellings. 

\scrule{“You shouldn’t put one person in charge of the building work on two dwellings. If you do, you commit an offense of wrong conduct.” }

On\marginnote{17.2.9} one occasion the monks who had taken on building work had someone else stay in that dwelling. 

\scrule{“When you have taken on building work, you shouldn’t have someone else stay in that dwelling. If you do, you commit an offense of wrong conduct.” }

At\marginnote{17.2.13} one time monks who had taken on building work reserved what belonged to the Sangha. 

\scrule{“When you have taken on building work, you shouldn’t reserve what belongs to the Sangha. If you do, you commit an offense of wrong conduct. I allow you to take one good bed.” }

On\marginnote{17.2.18} one occasion the monks put one who was outside the monastery zone in charge of building work. 

\scrule{“You shouldn’t put one who’s outside the monastery zone in charge of building work. If you do, you commit an offense of wrong conduct.” }

At\marginnote{17.2.22} one time the monks who had taken on building work reserved a dwelling at all times. 

\scrule{“When you have taken on building work, you shouldn’t reserve a dwelling at all times. If you do, you commit an offense of wrong conduct. I allow you to reserve it for the three months of the rainy-season residence, but not at other times.” }

At\marginnote{17.3.1} that time monks who had taken on building work left, disrobed, died, admitted to being novice monks, admitted to having renounced the training, admitted to having committed the worst kind of offense, admitted to being insane, admitted to being deranged, admitted to being overwhelmed by pain, admitted to having been suspended for not recognizing an offense, admitted to having been suspended for not making amends for an offense, admitted to having been suspended for not giving up a bad view, admitted to being a \textit{\textsanskrit{paṇḍaka}}, admitted to being fake monks, admitted to previously having left to join the monastics of another religion, admitted to being an animal, admitted to being a matricide, admitted to being a patricide, admitted to being a murderer of a perfected one, admitted to having raped a nun,\footnote{For an explanation of the rendering “raped” for \textit{\textsanskrit{dūsita}}, see Appendix of Technical Terms. } admitted to having caused a schism in the Sangha, admitted to having caused the Buddha to bleed, or admitted to being a hermaphrodite. They told the Buddha. 

“If\marginnote{17.3.23} a monk who’s taken on building work departs, it should be given to another, with the thought, ‘What belongs to the Sangha shouldn’t be allowed to deteriorate.’ 

If\marginnote{17.3.25} a monk who’s taken on building work disrobes, dies, admits to being a novice monk, admits to having renounced the training, admits to having committed the worst kind of offense, admits to being insane, admits to being deranged, admits to being overwhelmed by pain, admits to having been suspended for not recognizing an offense, admits to having been suspended for not making amends for an offense, admits to having been suspended for not giving up a bad view, admits to being a \textit{\textsanskrit{paṇḍaka}}, admits to being a fake monk, admits to previously having left to join the monastics of another religion, admits to being an animal, admits to being a matricide, admits to being a patricide, admits to being a murderer of a perfected one, admits to having raped a nun, admits to having caused a schism in the Sangha, admits to having caused the Buddha to bleed, or admits to being a hermaphrodite, it should be given to another, with the thought, ‘What belongs to the Sangha shouldn’t be allowed to deteriorate.’ 

If\marginnote{17.3.28} a monk who’s taken on building work departs while it’s still unfinished, it should be given to another, with the thought, ‘What belongs to the Sangha shouldn’t be allowed to deteriorate.’ 

If\marginnote{17.3.30} a monk who’s taken on building work disrobes while it’s still unfinished … or admits to being a hermaphrodite while it’s still unfinished, it should be given to another, with the thought, ‘What belongs to the Sangha shouldn’t be allowed to deteriorate.’ 

If\marginnote{17.3.33} a monk who’s taken on building work departs when it’s finished, then it’s still for him.\footnote{Sp 4.323: \textit{Pariyosite pakkamati tassevetanti puna \textsanskrit{āgantvā} vasantassa \textsanskrit{antovassaṁ} tasseva \textsanskrit{taṁ}}, “‘Departs when it is finished, then it is still his’ means: after returning, if he stays there, then within the rainy-season residence it is still his.” } 

If\marginnote{17.3.35} a monk who’s taken on building work disrobes when it’s finished; dies when it’s finished; admits, when it’s finished, to being a novice monk; admits, when it’s finished, to having renounced the training; or admits, when it’s finished, to having committed the worst kind of offense—then the Sangha is the owner. 

If\marginnote{17.3.38} a monk who’s taken on building work admits, when it’s finished, to being insane; admits, when it’s finished, to being deranged; admits, when it’s finished, to being overwhelmed by pain; admits, when it’s finished, to having been suspended for not recognizing an offense; admits, when it’s finished, to having been suspended for not making amends for an offense; or admits, when it’s finished, to having been suspended for not giving up a bad view—then it’s still for him. 

If\marginnote{17.3.40} a monk who’s taken on building work admits, when it’s finished, to being a \textit{\textsanskrit{paṇḍaka}}; admits, when it’s finished, to being a fake monk; admits, when it’s finished, to previously having left to join the monastics of another religion; admits, when it’s finished, to being an animal; admits, when it’s finished, to being a matricide; admits, when it’s finished, to being a patricide; admits, when it’s finished, to being a murderer of a perfected one; admits, when it’s finished, to having raped a nun; admits, when it’s finished, to having caused a schism in the Sangha; admits, when it’s finished, to having caused the Buddha to bleed; or admits, when it’s finished, to being a hermaphrodite—then the Sangha is the owner.” 

\subsection*{22. The prohibition against using equipment where it doesn’t belong, etc. }

At\marginnote{18.1.1} one time the monks used a certain lay follower’s equipment where it did not belong. That lay follower complained and criticized them, “How can the venerables use the equipment where it doesn’t belong?” They told the Buddha. 

\scrule{“You shouldn’t use equipment where it doesn’t belong. If you do, you commit an offense of wrong conduct.” }

Being\marginnote{18.1.7} afraid of wrongdoing, the monks did not take any equipment to the observance-day hall or to meetings. They sat down on the bare ground, their limbs and robes becoming dirty. 

\scrule{“I allow you to borrow.” }

At\marginnote{18.1.11} that time a large dwelling belonging to the Sangha was decaying. Being afraid of wrongdoing, the monks did not remove the furniture. 

\scrule{“I allow you to move it for the purpose of protection.” }

On\marginnote{19.1.1} one occasion the Sangha had been given a valuable, woolen furniture cloth.\footnote{It can be seen from \href{https://suttacentral.net/pli-tv-bi-vb-np11/en/brahmali\#1.1}{Bi NP 11:1.1} that valuable, woolen cloth, \textit{mahaggha kambala}, was considered inappropriate for monastics. } 

\scrule{“I allow you to do a beneficial trade.”\footnote{Sp 4.324: \textit{\textsanskrit{Phātikammatthāyāti} \textsanskrit{vaḍḍhikammatthāya}. \textsanskrit{Phātikammañcettha} \textsanskrit{samakaṁ} \textsanskrit{vā} \textsanskrit{atirekaṁ} \textsanskrit{vā} \textsanskrit{agghanakaṁ} \textsanskrit{mañcapīṭhādisenāsanameva} \textsanskrit{vaṭṭati}}, “\textit{\textsanskrit{Phātikammatthāya}} means for the purpose of making a profit. Here, doing a beneficial (trade) with furniture, such as a bed or bench, etc., equal or greater in value, is allowed.” } }

On\marginnote{19.1.4} one occasion the Sangha was offered a valuable furniture cloth. 

\scrule{“I allow you to do a beneficial trade.” }

On\marginnote{19.1.7} one occasion the Sangha was offered a bear skin. 

\scrule{“I allow you to make it into a doormat.” }

On\marginnote{19.1.10} one occasion the Sangha was offered a round pad.\footnote{Sp 4.324: \textit{Cakkalikanti \textsanskrit{kambalādīhi} \textsanskrit{veṭhetvā} \textsanskrit{katacakkalikaṁ}}, “\textit{Cakkalika} means a \textit{cakkalika} made by wrapping with woolen cloth, etc.” Sp-\textsanskrit{ṭ} 4.324: \textit{Cakkalikanti \textsanskrit{kambalādīhi} \textsanskrit{veṭhetvā} \textsanskrit{cakkasaṇṭhānena} \textsanskrit{pādapuñchanayoggaṁ} \textsanskrit{kataṁ}}, “\textit{Cakkalika} means after wrapping with woolen cloth, etc., having the shape of a wheel, it is made suitable as a doormat.” } 

\scrule{“I allow you to make it into a doormat.” }

On\marginnote{19.1.13} one occasion the Sangha was offered a cloth. 

\scrule{“I allow you to make it into a doormat.” }

At\marginnote{20.1.1} that time there were monks who stepped into the dwellings with dirty feet. The dwellings became dirty. 

\scrule{“You shouldn’t step into a dwelling with dirty feet. If you do, you commit an offense of wrong conduct.” }

At\marginnote{20.1.6} that time there were monks who stepped into the dwellings with wet feet. The dwellings became dirty. 

\scrule{“You shouldn’t step into a dwelling with wet feet. If you do, you commit an offense of wrong conduct.” }

At\marginnote{20.1.11} that time there were monks who stepped into the dwellings with their sandals on. The dwellings became dirty. 

\scrule{“You shouldn’t step into a dwelling with your sandals on. If you do, you commit an offense of wrong conduct.” }

At\marginnote{20.2.1} that time there were monks who spat on treated floors. The coloring was spoiled. 

\scrule{“You shouldn’t spit on treated floors. If you do, you commit an offense of wrong conduct. I allow spittoons.” }

At\marginnote{20.2.7} that time the legs of the beds and benches scratched the treated floors. The coloring was spoiled. 

\scrule{“You should wrap the legs in cloth.” }

At\marginnote{20.2.10} that time there were monks who leaned on treated walls. The coloring was spoiled. 

\scrule{“You shouldn’t lean on treated walls. If you do, you commit an offense of wrong conduct. I allow leaning boards.” }

The\marginnote{20.2.16} lower edge of the leaning boards scratched the floor and the upper edge scratched the wall. 

\scrule{“You should wrap the lower and upper edges in cloth.” }

Being\marginnote{20.2.19} afraid of wrongdoing, the monks did not lie down with washed feet.\footnote{Presumably because their feet were still wet. } 

\scrule{“You should spread a sheet and then lie down.” }

\subsection*{23. The allowance for meals for the Sangha, etc. }

When\marginnote{21.1.1} the Buddha had stayed at \textsanskrit{Āḷavī} for as long as he liked, he set out wandering toward \textsanskrit{Rājagaha}. When he eventually arrived, he stayed in the Bamboo Grove, the squirrel sanctuary. 

At\marginnote{21.1.4} that time \textsanskrit{Rājagaha} was short of food and people were unable to make meals for the whole Sangha.\footnote{“Whole” is supplied from the commentary at Sp 4.325. } Instead they wished to make meals for designated monks, invitational meals, meals for which lots are drawn, half-monthly meals, meals on the observance day, and meals on the day after the observance day. 

\scrule{“I allow meals for the Sangha, meals for designated monks, invitational meals, meals for which lots are drawn, half-monthly meals, meals on the observance days, and meals on the days after the observance day.” }

\subsection*{24. The appointment of a designator of meals }

At\marginnote{21.1.9.1} that time the monks from the group of six took the best meals for themselves and gave the inferior ones to the other monks. They told the Buddha. 

“You\marginnote{21.1.11} should appoint a monk who has five qualities as the designator of meals: he’s not biased by favoritism, ill will, confusion, or fear, and he knows what has and has not been designated. And he should be appointed like this. First a monk should be asked, and then a competent and capable monk should inform the Sangha: 

‘Please,\marginnote{21.1.15} venerables, I ask the Sangha to listen. If the Sangha is ready, it should appoint monk so-and-so as designator of meals. This is the motion. 

Please,\marginnote{21.1.18} venerables, I ask the Sangha to listen. The Sangha appoints monk so-and-so as designator of meals. Any monk who approves of appointing monk so-and-so as designator of meals should remain silent. Any monk who doesn’t approve should speak up. 

The\marginnote{21.1.22} Sangha has appointed monk so-and-so as designator of meals. The Sangha approves and is therefore silent. I’ll remember it thus.’” 

The\marginnote{21.1.24} monks who were designator of meals thought, “How should we designate the meals?” 

\scrule{“You should mark tickets, make a heap of them, and then designate the meals.”\footnote{Sp 4.325: \textit{\textsanskrit{Rukkhasāramayāya} \textsanskrit{salākāya} \textsanskrit{vā} \textsanskrit{veḷuvilīvatālapaṇṇādimayāya} \textsanskrit{paṭṭikāya} \textsanskrit{vā} “asukassa \textsanskrit{nāma} \textsanskrit{salākabhattan}”ti \textsanskrit{evaṁ} \textsanskrit{akkharāni} \textsanskrit{upanibandhitvā} \textsanskrit{pacchiyaṁ} \textsanskrit{vā} \textsanskrit{cīvarabhoge} \textsanskrit{vā} \textsanskrit{katvā} \textsanskrit{sabbā} \textsanskrit{salākāyo} \textsanskrit{opuñjitvā} \textsanskrit{punappunaṁ} \textsanskrit{heṭṭhuparivaseneva} \textsanskrit{āloḷetvā} \textsanskrit{pañcaṅgasamannāgatena} bhattuddesakena sace \textsanskrit{ṭhitikā} atthi, \textsanskrit{ṭhitikato} \textsanskrit{paṭṭhāya}; no ce atthi, \textsanskrit{therāsanato} \textsanskrit{paṭṭhāya} \textsanskrit{salākā} \textsanskrit{dātabbā}}, “Having marked with letters a (\textit{\textsanskrit{salākā}})-ticket made of heartwood or (\textit{\textsanskrit{paṭṭikā}})-ticket made of bamboo or palm leaves, (showing), ‘This is the meal drawn by lots of so-and-so’, having placed it in a basket or the fold of a robe, having collected all the tickets, having mixed them again and again from below and above, the distributor of meals who has five qualities should give tickets according to a schedule, if there is one, or according to seniority if there is not.” The point, presumably, is that the donors’ names that have been drawn are distributed according to seniority. } }

\subsection*{25. The appointment of an assigner of dwellings, etc. }

At\marginnote{21.2.1} that time there was no assigner of dwellings … no storeman … no receiver of robe-cloth …\footnote{For an explanation of the rendering “robe-cloth” for \textit{\textsanskrit{cīvara}}, see Appendix of Technical Terms. } no distributor of robe-cloth … no distributor of congee … no distributor of fruit … no distributor of fresh food. Because it was not distributed, the fresh food perished. 

“You\marginnote{21.2.10} should appoint a monk who has five qualities as the distributor of fresh food: he’s not biased by favoritism, ill will, confusion, or fear, and he knows what has and has not been distributed. And he should be appointed like this. First a monk should be asked, and then a competent and capable monk should inform the Sangha: 

‘Please,\marginnote{21.2.14} venerables, I ask the Sangha to listen. If the Sangha is ready, it should appoint monk so-and-so as the distributor of fresh food. This is the motion. 

Please,\marginnote{21.2.17} venerables, I ask the Sangha to listen. The Sangha appoints monk so-and-so as the distributor of fresh food. Any monk who approves of appointing monk so-and-so as the distributor of fresh food should remain silent. Any monk who doesn’t approve should speak up. 

The\marginnote{21.2.21} Sangha has appointed monk so-and-so as the distributor of fresh food. The Sangha approves and is therefore silent. I’ll remember it thus.’” 

\subsection*{26. The appointment of a distributor of minor requisites }

At\marginnote{21.3.1} that time there were minor requisites in the storeroom.\footnote{For an explanation of the rendering “requisites” for \textit{\textsanskrit{parikkhāra}}, see Appendix of Technical Terms. } They told the Buddha. 

“You\marginnote{21.3.3} should appoint a monk who has five qualities as the distributor of minor requisites: he’s not biased by favoritism, ill will, confusion, or fear, and he knows what has and has not been distributed. And he should be appointed like this. First a monk should be asked, and then a competent and capable monk should inform the Sangha: 

‘Please,\marginnote{21.3.7} venerables, I ask the Sangha to listen. If the Sangha is ready, it should appoint monk so-and-so as the distributor of minor requisites. This is the motion. 

Please,\marginnote{21.3.10} venerables, I ask the Sangha to listen. The Sangha appoints monk so-and-so as the distributor of minor requisites. Any monk who approves of appointing monk so-and-so as the distributor of minor requisites should remain silent. Any monk who doesn’t approve should speak up. 

The\marginnote{21.3.14} Sangha has appointed monk so-and-so as the distributor of minor requisites. The Sangha approves and is therefore silent. I’ll remember it thus.’” 

The\marginnote{21.3.16} monk who is the distributor of minor requisites should give things out one by one: needles, knives, sandals, belts, shoulder straps, water filters, and water strainers, and also robe material for long inter-panel strips, for short inter-panel strips, for large panels, for medium-sized panels, for lengthwise borders, and for crosswise borders.\footnote{The last six terms are explained in the commentaries as follows. Vin-vn-\textsanskrit{ṭ} 563: \textit{Kusinti \textsanskrit{āyāmato} ca \textsanskrit{vitthārato} ca \textsanskrit{anuvātaṁ} \textsanskrit{cīvaramajjhe} \textsanskrit{tādisameva} \textsanskrit{dīghapattañca}}, “A \textit{kusi} is a long panel that is a lengthwise or crosswise border in the middle of the robe.” Vin-vn-\textsanskrit{ṭ} 563: \textit{\textsanskrit{Aḍḍhakusinti} \textsanskrit{anuvātasadisaṁ} \textsanskrit{cīvaramajjhe} tattha tattha \textsanskrit{rassapattaṁ}}, “An \textit{\textsanskrit{aḍḍhakusi}} is a short panel like a border, here and there in the middle of the robe.” Sp 3.245: \textit{\textsanskrit{Maṇḍalanti} \textsanskrit{pañcakhaṇḍikacīvarassa} \textsanskrit{ekekasmiṁ} \textsanskrit{khaṇḍe} \textsanskrit{mahāmaṇḍalaṁ}}, “A \textit{\textsanskrit{maṇḍala}} is the large panel in each section of a robe with five sections.” The \textit{\textsanskrit{aḍḍhamaṇḍala}} is explained in connection with comments on the \textit{\textsanskrit{vivaṭṭa}}, “the middle section”. Vin-vn-\textsanskrit{ṭ} 563: \textit{\textsanskrit{Vivaṭṭanti} \textsanskrit{maṇḍalaṁ}, \textsanskrit{aḍḍhamaṇḍalañcāti} dve ekato \textsanskrit{katvā} \textsanskrit{sibbitaṁ} vemajjhe \textsanskrit{khaṇḍaṁ}}, “The \textit{\textsanskrit{vivaṭṭa}} is the section in the middle, which is made by sewing together a large panel (\textit{\textsanskrit{maṇḍala}}) and a medium-sized panel (\textit{\textsanskrit{aḍḍhamaṇḍala}}).” The \textit{\textsanskrit{vivaṭṭa}}, “section in the middle”, is one of usually five main sections of the robe; see just below. \textit{\textsanskrit{Anuvāta}} is explained in connection with making the \textit{kathina} robe in the commentary to the Kathinakhandhaka. Sp 3.308: \textit{\textsanskrit{Anuvātakaraṇamattenāti} \textsanskrit{piṭṭhianuvātāropanamattena}}, “\textit{\textsanskrit{Anuvātakaraṇamattena}} means merely by mounting a border at the back.” Which is further explained at Sp-\textsanskrit{ṭ} 3.308: \textit{\textsanskrit{Piṭṭhianuvātāropanamattenāti} \textsanskrit{dīghato} \textsanskrit{anuvātassa} \textsanskrit{āropanamattena}}, “\textit{\textsanskrit{Piṭṭhianuvātāropanamattena}} means merely by mounting a border lengthwise.” \textit{\textsanskrit{Paribhaṇḍa}} is also explained in connection with making the \textit{kathina} robe in the commentary to the Kathinakhandhaka. Sp 3.308: \textit{\textsanskrit{Paribhaṇḍakaraṇamattenāti} \textsanskrit{kucchianauvātāropanamattena}}, “\textit{\textsanskrit{Paribhaṇḍakaraṇamattena}} means merely by mounting a border at the belly.” Which is further explained at Sp-\textsanskrit{ṭ} 3.308: \textit{\textsanskrit{Kucchianuvātāropanamattenāti} puthulato \textsanskrit{anuvātassa} \textsanskrit{āropanamattena}}, “\textit{\textsanskrit{Kucchianuvātāropanamattena}} means merely by adding a border crosswise.” } 

\scrule{If the Sangha has ghee, oil, honey, or syrup, he should give it out for a single use. If it is needed again, he should give it out again. }

\subsection*{27. The appointment of a distributor of rainy-season bathing cloths, etc. }

At\marginnote{21.3.19.1} that time there was no distributor of rainy-season bathing cloths …\footnote{Commenting on Bu Pc 79, Kkh-\textsanskrit{pṭ} says: \textit{\textsanskrit{Sāṭiyaggāhāpakoti} \textsanskrit{vassikasāṭiyaggāhāpako}}, “\textit{\textsanskrit{Sāṭiyaggāhāpaka}} means a distributor of rainy-season bathing cloths.” } no distributor of almsbowls … no supervisor of monastery workers …\footnote{The typical translation of \textit{\textsanskrit{ārāmika}}, found both in the CPD and DOP, is “monastery attendant” or “monastery servant”. Yet, perhaps the only place in the Canonical texts where the meaning of the word is clear from the context, \href{https://suttacentral.net/pli-tv-bu-vb-np23/en/brahmali\#1.1.5}{Bu NP 23:1.1.5}, it means “monastery worker”. } no supervisor of novice monks. Not being supervised, the novice monks didn’t do their work. 

“You\marginnote{21.3.25} should appoint a monk who has five qualities as the supervisor of novice monks: he’s not biased by favoritism, ill will, confusion, or fear, and he knows who is and isn’t supervised. And he should be appointed like this. First a monk should be asked, and then a competent and capable monk should inform the Sangha: 

‘Please,\marginnote{21.3.29} venerables, I ask the Sangha to listen. If the Sangha is ready, it should appoint monk so-and-so as the supervisor of novice monks. This is the motion. 

Please,\marginnote{21.3.32} venerables, I ask the Sangha to listen. The Sangha appoints monk so-and-so as the supervisor of novice monks. Any monk who approves of appointing monk so-and-so as the supervisor of novice monks should remain silent. Any monk who doesn’t approve should speak up. 

The\marginnote{21.3.36} Sangha has appointed monk so-and-so as the supervisor of novice monks. The Sangha approves and is therefore silent. I’ll remember it thus.’” 

\scend{The third section for recitation is finished. }

\scendsutta{The sixth chapter on resting places is finished. }

\scuddanaintro{This is the summary: }

\begin{scuddana}%
“Dwellings\marginnote{21.3.41} by the excellent Buddha, \\
Had then not been allowed; \\
The disciples of the Victor emerged, \\
From their resting places here and there. 

The\marginnote{21.3.45} wealthy merchant having seen this, \\
Said this to the monks; \\
If I make, will you dwell, \\
They asked the Leader. 

Dwellings,\marginnote{21.3.49} and stilt houses, \\
Of three kinds, caves; \\
He allowed five kinds of shelters, \\
The merchant had dwellings made. 

People\marginnote{21.3.53} had dwellings made, \\
Without door it was unguarded; \\
Door, door frame, \\
And hinge below, above. 

Hole\marginnote{21.3.57} for pulling, rope, \\
And door jamb, bolt socket; \\
Bolt, latch, key hole, \\
Metal, wood, horn. 

\textit{Yantaka}-bolts,\marginnote{21.3.61} and just bolts, \\
Roof, plaster inside and outside; \\
Railing, lattice, and bars, \\
Cloth covers, and with a mat. 

Bench,\marginnote{21.3.65} and wicker bed, \\
Charnel ground, \textit{\textsanskrit{masāraka}}; \\
\textit{Bundi}, and having crooked legs; \\
Detachable, square bench, about a tall one. 

And\marginnote{21.3.69} sofa, cane bench, \\
Small bench with cloth, ram-like legs; \\
Many legs, plank, stools, \\
And just a bench of straw. 

High,\marginnote{21.3.73} snake, supports, \\
And supports of eight fingerbreadths; \\
String, cross weaving, cloth, \\
Cotton-down quilt, half the size of the body. 

Hilltop\marginnote{21.3.77} fair, and also mattresses, \\
And also furniture cloth; \\
Upholstered, sank down, \\
And removed and taken away. 

Multi-colored\marginnote{21.3.81} lines, and multi-colored lines by hand, \\
Was allowed by the Buddha; \\
And also in the dwellings of monastics of other religions, \\
Husk, and soft clay. 

Sap,\marginnote{21.3.85} trowel, bran, \\
Mustard seed, beeswax; \\
To wipe off when thick, \\
Rough, excreted clay. 

Sap,\marginnote{21.3.89} and picture, \\
Low, and mound, getting up; \\
They fell down, accessible to the public, \\
A half wall, again three. 

A\marginnote{21.3.93} small one, and base of a wall, \\
Rains through, scream, peg; \\
Bamboo robe rack, and line, \\
Porch, and with a screen. 

Rails,\marginnote{21.3.97} grass and dust, \\
The method should be applied in the way below; \\
Outside, it became warm, \\
Shed, and below, vessel. 

Dwelling,\marginnote{21.3.101} and just a gatehouse, \\
Yard, water-boiling shed; \\
And about a monastery, again about a gatehouse, \\
The same method should be applied below. 

Plaster,\marginnote{21.3.105} and \textsanskrit{Anāthapiṇḍika}, \\
Faith, went to Cool Grove; \\
Having seen the Truth, he invited, \\
The Leader together with the Sangha. 

On\marginnote{21.3.109} his way he told, \\
The group built a monastery; \\
Building work in \textsanskrit{Vesālī}, \\
And ahead possession was taken. 

Who\marginnote{21.3.113} deserves in the dining hall, \\
And partridge, not to be paid respect; \\
Took possession, inhabited areas, \\
Cotton down, he entered \textsanskrit{Sāvatthī}. 

He\marginnote{21.3.117} dedicated the monastery, \\
And an uproar in the dining hall; \\
The sick, and the best beds, \\
Pretext, seventeen there, 

Who,\marginnote{21.3.121} how, \\
One was allocated to each dwelling; \\
Yard, and an additional share, \\
Unwillingly a share should not be given. 

Outside\marginnote{21.3.125} the zone, and at all time, \\
Three allocations of dwellings; \\
And Upananda, he praised, \\
Standing, a similar seat. 

Those\marginnote{21.3.129} entitled to the same seats, they broke, \\
And a group of three, group of two; \\
Not entitled to the same seat, long, \\
Including a porch, to use. 

And\marginnote{21.3.133} grandmother, not far, \\
And distributed, in \textsanskrit{Kīṭāgiri}; \\
\textsanskrit{Āḷavī}, fill, with walls, \\
Door, door jamb. 

Window,\marginnote{21.3.137} white, and black, \\
Red ocher, roof, fastening; \\
Cornice, defective, plastering the floor, \\
Twenty, and thirty, for life. 

When\marginnote{21.3.141} inhabited, not started, unfinished, \\
Five or six years in a small one; \\
And seven or eight in a small stilt house, \\
Ten or twelve in a large one. 

All\marginnote{21.3.145} dwellings to one, \\
Had someone else stay, belonging to the Sangha; \\
Outside the zone, and at all times, \\
Left, and disrobed. 

And\marginnote{21.3.149} died, and novice monk, \\
Renounced the training, the worst; \\
Insane, and deranged, \\
Pain, not recognizing an offense. 

Not\marginnote{21.3.153} making amends, of a view, \\
\textit{\textsanskrit{Paṇḍakā}}, fake, monastics of another religion; \\
Animal, mother, father, \\
And perfected ones, rapists. 

Schismatics,\marginnote{21.3.157} those who cause the Buddha to bleed, \\
And also hermaphrodites; \\
Let not the belongings of the Sangha deteriorate. \\
The work should be given to another. 

And\marginnote{21.3.161} when unfinished to another, \\
When finished it’s just for him should he depart; \\
He disrobes, dies, \\
And becomes a novice. 

And\marginnote{21.3.165} renounces the training, \\
If he has committed the worst; \\
Just the Sangha is the owner, \\
Insane, deranged, pain. 

Not\marginnote{21.3.169} recognizing, not making amends, \\
He has just that view; \\
\textit{\textsanskrit{Paṇḍako}}, fake, and monastic of another religion, \\
Animal, mother, paternal. 

A\marginnote{21.3.173} killer, and also a rapist, \\
Schismatic, blood, hermaphrodite; \\
If he admits, \\
Just the Sangha is the owner. 

Took\marginnote{21.3.177} elsewhere, anxious, \\
And it decayed, woolen cloth; \\
And cloth, skin, round cloth, \\
Cloth, and they stepped. 

Wet,\marginnote{21.3.181} sandals, spitting, \\
They spoiled, and they leaned; \\
The leaning board scratched, \\
Washed, and with a sheet. 

In\marginnote{21.3.185} \textsanskrit{Rājagaha} they were unable, \\
Inferior, a designator of meals; \\
How, an assigner, \\
Appointment of a storeman. 

A\marginnote{21.3.189} receiver, and also a distributor, \\
And congee, a distributor of fruit; \\
And just a distributor of fresh food, \\
About a distributor of minor requisites. 

And\marginnote{21.3.193} also a distributor of rainy-season bathing cloths. \\
Just so a distributor of almsbowls; \\
Monastery worker, novice monk, \\
And agreement for a supervisor. 

He\marginnote{21.3.197} has conquered all and knows the world, \\
The Leader intent on what’s beneficial: \\
They’re for the sake of shelter and happiness, \\
To attain absorption and to see clearly.” 

%
\end{scuddana}

\scendsutta{The chapter on resting places is finished. }

%
\chapter*{{\suttatitleacronym Kd 17}{\suttatitletranslation The chapter on schism in the Sangha }{\suttatitleroot Saṁghabhedakakkhandhaka}}
\addcontentsline{toc}{chapter}{\tocacronym{Kd 17} \toctranslation{The chapter on schism in the Sangha } \tocroot{Saṁghabhedakakkhandhaka}}
\markboth{The chapter on schism in the Sangha }{Saṁghabhedakakkhandhaka}
\extramarks{Kd 17}{Kd 17}

\section*{The first section for recitation }

\subsection*{1. The account of the going forth of the six Sakyans }

At\marginnote{1.1.1} one time the Buddha was staying at the Mallian town of \textsanskrit{Anupiyā}. At that time a number of well-known young Sakyans had followed the Buddha in going forth. 

Then\marginnote{1.1.3} there were the brothers \textsanskrit{Mahānāma} and Anuruddha. Anuruddha had been brought up in great comfort. He had three stilt houses:\footnote{For an explanation of the rendering “stilt house” for \textit{\textsanskrit{pāsāda}}, see Appendix of Technical Terms. } one for the winter, one for the summer, and one for the rainy season. He spent the four months of the rainy season in the rainy-season house, attended on only by female musicians, never descending from that house. 

\textsanskrit{Mahānāma}\marginnote{1.1.8} thought, “A number of well-known young Sakyans have just followed the Buddha in going forth, but no-one from our household. Why doesn’t Anuruddha or I go forth?” He then went to Anuruddha and told him what he had thought. Anuruddha replied, “I’ve been brought up in great comfort. I’m not able to go forth. You go forth.” 

“Come,\marginnote{1.2.1} then, let me teach you how to run the family business. First you must plow the fields, then sow the seeds, irrigate, and drain, and then weed. Next you must cut the harvest, gather it together, and make sheaves.\footnote{Sp-\textsanskrit{ṭ} 4.330: \textit{\textsanskrit{Ujuṁ} \textsanskrit{kārāpetabbanti} \textsanskrit{puñjaṁ} \textsanskrit{kārāpetabbaṁ}}, “To be made straight means to be made into heaps.” It seems from this that the original meaning may have been \textit{\textsanskrit{ujuṁ} \textsanskrit{kārāpetabbaṁ}} rather than \textit{\textsanskrit{puñjaṁ} \textsanskrit{kārāpetabbaṁ}}. I interpret “to make straight” as organizing the harvest into sheaves. } You must then thresh it, remove the straw and husk, and then winnow it before you bring it into storage. And next year you must do the same, and the next.” 

“But\marginnote{1.2.16} when does the work stop? I can’t see any end to it. When can you enjoy yourself with worldly pleasures, free from bother?” 

“The\marginnote{1.2.21} work doesn’t stop and you won’t see any end to it. While the work was still unfinished, our fathers and grandfathers have all died.” 

“Well\marginnote{1.2.24} then, you go ahead and run the family business.\footnote{“Go ahead” renders \textit{\textsanskrit{upajānāhi}}. According to Sp 4.330, \textit{\textsanskrit{upajānāhi}} is just another word for \textit{\textsanskrit{jānāhi}}. Sp-\textsanskrit{ṭ} 4.330 adds: \textit{\textsanskrit{Jānāhīti} cettha \textsanskrit{paṭipajjāti} attho veditabbo}, “\textit{\textsanskrit{Jānāhi}}: here the meaning ‘practice’ should be understood.” } I’ll go forth into homelessness.” 

Anuruddha\marginnote{1.2.26} then went to his mother and said, “Mom, I wish to go forth into homelessness. Please allow me.” 

“Both\marginnote{1.2.30} of you, Anuruddha, my two sons, are dear and beloved to me. Even if you died, I would lose you against my wishes. So how can I allow you to go forth into homelessness while you’re still living?” 

A\marginnote{1.2.33} second time Anuruddha asked the same question and got the same reply. He then asked a third time. 

At\marginnote{1.3.1} that time the Sakyans were ruled by King Bhaddiya, a friend of Anuruddha’s. Anuruddha’s mother considered this and thought, “Bhaddiya won’t be able to go forth into homelessness.” And she said to Anuruddha, “If King Bhaddiya goes forth, you may go forth too.” 

Anuruddha\marginnote{1.3.9} then went to King Bhaddiya and said, “My going forth depends on yours.” 

“Don’t\marginnote{1.3.11} worry about whether your going forth depends on mine or not. I’m with you. Just go forth when you like.” 

“Come,\marginnote{1.3.13} let’s go forth together.” 

“I’m\marginnote{1.3.14} not able to go forth. I’m not able to do what you do. There’s nothing I can do about it. You go forth.” 

“My\marginnote{1.3.17} mother told me that I may go forth only if you go forth. And then you said, ‘Don’t worry about whether your going forth depends on mine or not. I’m with you. Just go forth when you like.’ So come, let’s go forth together.” 

At\marginnote{1.3.23} that time people spoke the truth, and so Bhaddiya said to Anuruddha, “Please wait seven years. Then we’ll go forth together.” 

“That’s\marginnote{1.3.27} too long. I’m not able to wait for seven years.” 

“Then\marginnote{1.3.29} wait six years … five years … four years … three years … two years … one year, and we’ll go forth together.” 

“That’s\marginnote{1.3.36} too long. I’m not able to wait for one year.” 

“Then\marginnote{1.3.38} wait seven months, and we’ll go forth together.” 

“That’s\marginnote{1.3.40} too long. I’m not able to wait for seven months.” 

“Then\marginnote{1.3.42} wait six months … five months … four months … three months … two months … one month … half a month, and we’ll go forth together.” 

“That’s\marginnote{1.3.50} too long. I’m not able to wait for half a month.” 

“Then\marginnote{1.3.52} wait seven days, while I hand over the rulership to my sons and brothers.” 

“Seven\marginnote{1.3.53} days isn’t long. I’ll wait.” 

Soon\marginnote{1.4.1} afterwards King Bhaddiya, Anuruddha, Ānanda, Bhagu, Kimila, and Devadatta, with the barber \textsanskrit{Upāli} as the seventh, went out to the park with the fourfold army, just as they had in the past. After going a good distance, they turned back the army. They then entered a foreign territory, removed their ornaments, bound them into a bundle with an upper robe, and said to \textsanskrit{Upāli}, “Now turn back, \textsanskrit{Upāli}. This is enough for you to live on.” As \textsanskrit{Upāli} was returning, he thought, “The Sakyans are temperamental. They may think that I’m responsible for the departure of these young men and have me executed. Now, if these young Sakyans are going forth into homelessness, why shouldn’t I?” 

Undoing\marginnote{1.4.10} the bundle, he hung the goods from a tree, saying, “This is given to whoever sees it. They may take it away.” And he returned to the young Sakyans. 

Seeing\marginnote{1.4.11} \textsanskrit{Upāli} coming, they said to him, “Why are you coming back, \textsanskrit{Upāli}?” And he told what he had done. “You have done the right thing. The Sakyans are indeed temperamental. They might well have held you responsible for our departure and have had you executed.” 

The\marginnote{1.4.23} young Sakyans then took \textsanskrit{Upāli} with them and went to the Buddha, where they bowed, sat down, and said, “Sir, we Sakyans are proud. This barber \textsanskrit{Upāli} has been serving us for a long time. Please let him go forth first. We’ll then bow down to him, rise up for him, raise our joined palms to him, and do acts of respect toward him. In this way our Sakyan pride will be humbled.” And the Buddha had \textsanskrit{Upāli} go forth first, and afterwards the young Sakyans. 

During\marginnote{1.4.31} that very rainy season Venerable Bhaddiya realized the three insights, Venerable Anuruddha obtained clairvoyance, Venerable Ānanda realized stream-entry, whereas Devadatta obtained supernormal powers, but no stage of awakening. 

Then,\marginnote{1.5.1} when Bhaddiya was in the wilderness, at the foot of a tree, or in an empty dwelling, he frequently uttered this heartfelt exclamation: “Oh, what happiness! Oh, what happiness!” A number of monks went to the Buddha, bowed, sat down, and told him what was happening, adding, “No doubt Bhaddiya is dissatisfied with the spiritual life. He’s saying this because he’s recalling his former happiness as a king.” 

The\marginnote{1.5.10} Buddha addressed a certain monk: “Go, monk, and in my name say to Bhaddiya, ‘Bhaddiya, the Teacher is calling you.’” 

Saying,\marginnote{1.5.13} “Yes, sir,” he did just that. Bhaddiya consented. He then went to the Buddha, bowed, and sat down. The Buddha said to him, “Is it true, Bhaddiya, that when you’re in the wilderness, at the foot of a tree, or in an empty dwelling, you frequently utter this heartfelt exclamation: ‘Oh, what happiness! Oh, what happiness!’?” 

“Yes,\marginnote{1.6.5} venerable sir.” 

“But\marginnote{1.6.6} why do you say this?” 

“In\marginnote{1.6.8} the past, sir, when I was a king, I was well protected within and outside the royal compound, within and outside of town, and within and outside the country. But although I was protected and guarded in this way, I was fearful, agitated, and distrustful. But now, sir, when I’m in the wilderness, at the foot of a tree, or in an empty dwelling, I’m free from fear, agitation, and distrust. I’m free from bother, relaxed, living on what’s given by others, with a mind as free as a wild animal. That’s why I say this.” 

Seeing\marginnote{1.6.13} the significance of this, on that occasion the Buddha uttered a heartfelt exclamation: 

\begin{verse}%
“They\marginnote{1.6.14} who have no anger within, \\
Gone beyond any kind of existence, \\
Happy, free from fear and sorrow—\\
Even the gods are unable to see them.”\footnote{Sp 4.332: \textit{\textsanskrit{Nānubhavantīti} na \textsanskrit{sampāpuṇanti}; tassa \textsanskrit{dassanaṁ} \textsanskrit{devānampi} dullabhanti \textsanskrit{adhippāyo}}, “\textit{\textsanskrit{Nānubhavanti}}: they do not attain; the meaning is that even for the gods seeing him is difficult.” } 

%
\end{verse}

\subsection*{2. The account of Devadatta }

When\marginnote{2.1.1} the Buddha had stayed at \textsanskrit{Anupiyā} for as long as he liked, he set out wandering toward \textsanskrit{Kosambī}. When he eventually arrived, he stayed in Ghosita’s Monastery. 

Then,\marginnote{2.1.4} while reflecting in private, Devadatta thought, “Who might I inspire to have confidence in me, so that I’d get much material support and honor?” And it occurred to him, “Prince \textsanskrit{Ajātasattu} is young and has a bright future. Why don’t I inspire him to have confidence in me? Then I’ll get much material support and honor.” 

He\marginnote{2.1.10} put his dwelling in order, took his bowl and robes, and left for \textsanskrit{Rājagaha}, where he eventually arrived. He then transformed himself into a boy wearing a snake as a belt and appeared on Prince \textsanskrit{Ajātasattu}’s lap. When \textsanskrit{Ajātasattu} became fearful and agitated, Devadatta said to him, “Are you afraid of me, prince?” 

“I\marginnote{2.1.16} am. Who are you?” 

“I’m\marginnote{2.1.18} Devadatta.” 

“If\marginnote{2.1.19} you’re Venerable Devadatta, please appear in your own form.” Devadatta abandoned the form of a boy and stood in front of \textsanskrit{Ajātasattu}, wearing his robes and carrying his bowl.\footnote{“Wearing his robes and carrying his bowl” renders \textit{\textsanskrit{saṅghāṭipattacīvaradharo}}. In the \textit{vinaya}, \textit{\textsanskrit{saṅghāṭi}} frequently refers to whatever robe is worn on the upper body without distinguishing whether it is the single layered one or the double layered one. In other instances where \textit{\textsanskrit{saṅghāṭi}} and \textit{\textsanskrit{cīvara}} are spoken of together, as they are here, the \textit{\textsanskrit{cīvara}} seems to refer to the third robe, which is often spare, whereas the \textit{\textsanskrit{saṅghāṭi}} refers to the upper robe. See for instance \href{https://suttacentral.net/pli-tv-bu-vb-ss8/en/brahmali\#1.7.21}{Bu Ss 8:1.7.21} where we find the following: \textit{Atha kho \textsanskrit{mettiyabhūmajakā} \textsanskrit{bhikkhū} \textsanskrit{pacchābhattaṁ} \textsanskrit{piṇḍapātapaṭikkantā} \textsanskrit{ārāmaṁ} \textsanskrit{gantvā} \textsanskrit{pattacīvaraṁ} \textsanskrit{paṭisāmetvā} \textsanskrit{bahārāmakoṭṭhake} \textsanskrit{saṅghāṭipallatthikāya} \textsanskrit{nisīdiṁsu}}, “When they had eaten their meal and returned from alms round, they put their bowls and robes away, and squatted on their heels outside the monastery gatehouse, using their upper robes as back-and-knee straps.” See also note on \textit{\textsanskrit{saṅghāṭi}} in Appendix of Technical Terms. } 

By\marginnote{2.1.21} means of this wonder Devadatta was able to inspire \textsanskrit{Ajātasattu} to have confidence in him. \textsanskrit{Ajātasattu} then attended on him morning and evening, with five hundred carriages and a meal offering of five hundred dishes of food. Overcome by material support, honor, and praise, Devadatta started desiring to lead the Sangha of monks. But with the appearance of that thought, his supernormal powers disappeared. 

At\marginnote{2.2.1} that time Kakudha the \textsanskrit{Koḷiyan}, the attendant to Venerable \textsanskrit{Mahāmoggallāna}, had recently died and been reborn in a mind-made body.\footnote{This is a parallel to \href{https://suttacentral.net/an5.100/en/brahmali\#0.1}{AN 5.100:0.1}. } He had acquired a body two or three times the size of the fields of a Magadhan village. Yet he harmed neither himself nor others with that body. 

Soon\marginnote{2.2.5} afterwards the god Kakudha approached \textsanskrit{Mahāmoggallāna}, bowed down, and told him about Devadatta’s desire and the disappearance of his supernormal powers. He then bowed down, circumambulated \textsanskrit{Mahāmoggallāna} with his right side toward him, and disappeared right there. 

\textsanskrit{Mahāmoggallāna}\marginnote{2.2.12} then went to the Buddha, bowed, sat down, and told him all that had happened. The Buddha said, “But, \textsanskrit{Moggallāna}, have you read Kakudha’s mind so that you know that all he says is just so and not otherwise?” 

“I\marginnote{2.2.27} have, venerable sir.” 

“Remember\marginnote{2.2.29} these words, \textsanskrit{Moggallāna}! Soon that fool will show himself as he truly is.” 

\subsection*{3. Discussion of the five kinds of teachers }

“\textsanskrit{Moggallāna},\marginnote{2.3.1} there are five kinds of teachers in the world. One kind of teacher is impure in behavior, while claiming it is pure. His disciples know about this, but think, ‘It would be unpleasant for him if we inform the householders. And it’s because of him that we’re honored with gifts of robe-cloth, almsfood, dwellings, and medicinal supplies. How, then, can we inform them?\footnote{The Pali is rather obscure. Vmv 4.334: \textit{\textsanskrit{Sammannatīti} \textsanskrit{cīvarādinā} \textsanskrit{amhākaṁ} \textsanskrit{sammānaṁ} karoti, parehi \textsanskrit{vā} \textsanskrit{ayaṁ} \textsanskrit{satthā} \textsanskrit{sammānīyatīti} attho}, “\textit{Sammannati}: the meaning is: he creates honor for us by way of robes, etc., or this teacher is being honored by others.” } He’ll be known through his own actions.’ The disciples conceal the impure behavior of such a teacher, and the teacher expects them to do so. 

Another\marginnote{2.4.1} kind of teacher is impure in livelihood, while claiming it is pure. His disciples know about this, but think, ‘It would be unpleasant for him if we inform the householders. And it’s because of him that we’re honored with gifts of robe-cloth, almsfood, dwellings, and medicinal supplies. How, then, can we inform them? He’ll be known through his own actions.’ The disciples conceal the impure livelihood of such a teacher, and the teacher expects them to do so. 

Still\marginnote{2.4.10} another kind of teacher gives impure teachings, while claiming they are pure. His disciples know about this, but think, ‘It would be unpleasant for him if we inform the householders. And it’s because of him that we’re honored with gifts of robe-cloth, almsfood, dwellings, and medicinal supplies. How, then, can we inform them? He’ll be known through his own actions.’ The disciples conceal the impure teachings of such a teacher, and the teacher expects them to do so. 

Still\marginnote{2.4.19} another kind of teacher gives impure explanations, while claiming they are pure. His disciples know about this, but think, ‘It would be unpleasant for him if we inform the householders. And it’s because of him that we’re honored with gifts of robe-cloth, almsfood, dwellings, and medicinal supplies. How, then, can we inform them? He’ll be known through his own actions.’ The disciples conceal the impure explanations of such a teacher, and the teacher expects them to do so. 

Still\marginnote{2.4.28} another kind of teacher has impure knowledge and vision, while claiming they are pure. His disciples know about this, but think, ‘It would be unpleasant for him if we inform the householders. And it’s because of him that we’re honored with gifts of robe-cloth, almsfood, dwellings, and medicinal supplies. How, then, can we inform them? He’ll be known through his own actions.’ The disciples conceal the impure knowledge and vision of such a teacher, and the teacher expects them to do so. 

But\marginnote{2.4.38} in my case, \textsanskrit{Moggallāna}, I claim my behavior is pure because it is. My disciples don’t conceal my behavior, and I don’t expect them to do so. I claim my livelihood is pure … I claim my teachings are pure … I claim my explanations are pure … I claim my knowledge and vision are pure because they are. My disciples don’t conceal my knowledge and vision, and I don’t expect them to do so.” 

When\marginnote{2.5.1} the Buddha had stayed at \textsanskrit{Kosambī} for as long as he liked, he set out wandering toward \textsanskrit{Rājagaha}. When he eventually arrived, he stayed in the Bamboo Grove, the squirrel sanctuary. 

Then\marginnote{2.5.4} a number of monks went to the Buddha, bowed, sat down, and said, “Sir, Prince \textsanskrit{Ajātasattu} attends on Devadatta morning and evening, with five hundred carriages and a meal offering of five hundred dishes of food.” 

“Monks,\marginnote{2.5.8} don’t envy Devadatta his material support, honor, and praise. So long as Prince \textsanskrit{Ajātasattu} treats him like this, Devadatta can be expected to decline in good qualities. 

Just\marginnote{2.5.10} as a fierce dog would get even fiercer if you break a gall bladder on its nose, so too, so long as Prince \textsanskrit{Ajātasattu} treats him like this, Devadatta can be expected to decline in good qualities. 

Just\marginnote{2.5.13} as a plantain banana tree produces fruit to its own destruction and ruin, so too will Devadatta’s material support, honor, and praise cause his own destruction and ruin. 

Just\marginnote{2.5.15} as a bamboo produces fruit to its own destruction and ruin, so too will Devadatta’s material support, honor, and praise cause his own destruction and ruin. 

Just\marginnote{2.5.17} as a \textit{\textsanskrit{naḷa}} reed produces fruit to its own destruction and ruin, so too will Devadatta’s material support, honor, and praise cause his own destruction and ruin. 

Just\marginnote{2.5.19} as a mule becomes pregnant to its own destruction and ruin, so too will Devadatta’s material support, honor, and praise cause his own destruction and ruin.” 

\begin{verse}%
“The\marginnote{2.5.21} fruit destroys the plantain, \\
And the bamboo and the reed. \\
Honor destroys the bad person, \\
As the fetus destroys the mule.” 

%
\end{verse}

\scend{The first section for recitation is finished. }

\section*{The second section}

\subsection*{2.1 The legal procedure of announcement }

Soon\marginnote{3.1.1} afterwards the Buddha was seated giving a teaching surrounded by a large gathering of people, including the king. Devadatta then got up from his seat, arranged his upper robe over one shoulder, raised his joined palms, and said, “Sir, you’re now old and close to the end of life. You should live free from bother and enjoy the happiness of meditation. Hand the Sangha of monks over to me. Let me lead the Sangha.” 

“Let\marginnote{3.1.6} it be, Devadatta, don’t think of leading the Sangha of monks.” 

A\marginnote{3.1.7} second time Devadatta said the same thing and got the same reply. He then said it a third time, and the Buddha replied: 

“I\marginnote{3.1.12} wouldn’t even hand the Sangha over to \textsanskrit{Sāriputta} and \textsanskrit{Mogallāna}, so why then to you, a wretched devourer of junk?”\footnote{\textit{\textsanskrit{Kheḷāsaka}}, literally, “an eater of spittle”. The spittle apparently refers to the material support, honor, and praise he had acquired in an inappropriate manner. Sp 4.336 says: \textit{\textsanskrit{Kheḷāsakassāti} ettha \textsanskrit{micchājīvena} \textsanskrit{uppannapaccayā} ariyehi \textsanskrit{vantabbā} \textsanskrit{kheḷasadisā}, \textsanskrit{tathārūpe} paccaye \textsanskrit{ayaṁ} \textsanskrit{ajjhoharatīti} \textsanskrit{katvā} \textsanskrit{kheḷāsakoti} \textsanskrit{bhagavatā} vutto}, “\textit{\textsanskrit{Kheḷāsakassa}} here means that those requisites he had acquired through wrong livelihood, which the noble ones would vomit up like spittle, such requisites he has swallowed down. \textit{\textsanskrit{Kheḷāsakassa}} was said by the Buddha in this connection.” } 

Devadatta\marginnote{3.1.13} thought, “The Buddha disparages me in front of a gathering that includes the king as a devourer of junk, while praising \textsanskrit{Sāriputta} and \textsanskrit{Mogallāna},” and he bowed down in anger, circumambulated the Buddha with his right side toward him, and left. This was the first time Devadatta had ill will toward the Buddha. 

Soon\marginnote{3.2.1} afterwards the Buddha addressed the monks: 

“Well\marginnote{3.2.2} then, the Sangha should do a legal procedure for the purpose of making an announcement about Devadatta in \textsanskrit{Rājagaha} like this: ‘Devadatta’s character has changed. Whatever Devadatta now does by body or speech has nothing to do with the Buddha, the Teaching, or the Sangha, but only with Devadatta.’ And it should be done in this way. A competent and capable monk should inform the Sangha: 

‘Please,\marginnote{3.2.7} venerables, I ask the Sangha to listen. If the Sangha is ready, it should do a legal procedure for the purpose of making an announcement about Devadatta in \textsanskrit{Rājagaha} like this: “Devadatta’s character has changed. Whatever Devadatta now does by body or speech has nothing to do with the Buddha, the Teaching, or the Sangha, but only with Devadatta.” This is the motion. 

Please,\marginnote{3.2.12} venerables, I ask the Sangha to listen. The Sangha does a legal procedure  for the purpose of making an announcement about Devadatta in \textsanskrit{Rājagaha} like this: “Devadatta’s character has changed. Whatever Devadatta now does by body or speech has nothing to do with the Buddha, the Teaching, or the Sangha, but only with Devadatta.” Any monk who approves of doing such a legal procedure should remain silent. Any monk who doesn’t approve should speak up. 

The\marginnote{3.2.20} Sangha has done a legal procedure for the purpose of making an announcement about Devadatta in \textsanskrit{Rājagaha} like this: “Devadatta’s character has changed. Whatever Devadatta now does by body or speech has nothing to do with the Buddha, the Teaching, or the Sangha, but only with Devadatta.” The Sangha approves and is therefore silent. I’ll remember it thus.’” 

The\marginnote{3.2.24} Buddha then addressed \textsanskrit{Sāriputta}: “Well then, \textsanskrit{Sāriputta}, make that announcement about Devadatta in \textsanskrit{Rājagaha}.” 

“In\marginnote{3.2.26} the past, sir, I have praised Devadatta in \textsanskrit{Rājagaha}, saying, ‘Godhiputta is powerful and mighty.’ How, then, can I now make this announcement about him?” 

“Didn’t\marginnote{3.2.29} you praise him truthfully when you said that?” 

“I\marginnote{3.2.31} did.” 

“In\marginnote{3.2.32} the same way, you should make this announcement truthfully.” 

“Yes,\marginnote{3.2.33} sir.” 

The\marginnote{3.3.1} Buddha then addressed the monks: “Well then, the Sangha should appoint \textsanskrit{Sāriputta} to make that announcement about Devadatta in \textsanskrit{Rājagaha}. And he should be appointed like this. First \textsanskrit{Sāriputta} should be asked, and then a competent and capable monk should inform the Sangha: 

‘Please,\marginnote{3.3.8} venerables, I ask the Sangha to listen. If the Sangha is ready, it should appoint Venerable \textsanskrit{Sāriputta} to make an announcement about Devadatta in \textsanskrit{Rājagaha} like this: “Devadatta’s character has changed. Whatever Devadatta now does by body or speech has nothing to do with the Buddha, the Teaching, or the Sangha, but only with Devadatta.” This is the motion. 

Please,\marginnote{3.3.13} venerables, I ask the Sangha to listen. The Sangha appoints Venerable \textsanskrit{Sāriputta} to make an announcement about Devadatta in \textsanskrit{Rājagaha} like this: “Devadatta’s character has changed. Whatever Devadatta now does by body or speech has nothing to do with the Buddha, the Teaching, or the Sangha, but only with Devadatta.” Any monk who approves of appointing Venerable \textsanskrit{Sāriputta} in this way should remain silent. Any monk who doesn’t approve should speak up. 

The\marginnote{3.3.20} Sangha has appointed Venerable \textsanskrit{Sāriputta} to make an announcement about Devadatta in \textsanskrit{Rājagaha} like  this: “Devadatta’s character has changed. Whatever Devadatta now does by body or speech has nothing to do with the Buddha, the Teaching, or the Sangha, but only with Devadatta.” The Sangha approves and is therefore silent. I’ll remember it thus.’” 

When\marginnote{3.3.24} he had been appointed, \textsanskrit{Sāriputta} entered \textsanskrit{Rājagaha} accompanied by a number of monks. He then made that announcement about Devadatta: “Devadatta’s character has changed. Whatever Devadatta now does by body or speech has nothing to do with the Buddha, the Teaching, or the Sangha, but only with Devadatta.” The foolish people there, those with little faith and confidence, said, “These Sakyan monastics are envious of Devadatta’s material support and honor.” But the wise ones, those with faith and confidence, said, “This must be a serious matter, seeing as the Buddha has had an announcement made about Devadatta in \textsanskrit{Rājagaha}.” 

\subsection*{5. The account of Prince \textsanskrit{Ajātasattu} }

Soon\marginnote{3.4.1} afterwards Devadatta went to Prince \textsanskrit{Ajātasattu} and said, “In the past, prince, people were long-lived, but now they’re short-lived. It’s possible that you might die while still a prince. So then, kill your father and become the king. And I’ll kill the Buddha and take his place.” 

\textsanskrit{Ajātasattu}\marginnote{3.4.6} thought, “Venerable Devadatta is powerful and mighty. He would know.” He then bound a dagger to his thigh, and while fearful and agitated, he hastily entered the royal compound in the middle of the day.\footnote{For the rendering “royal compound” for \textit{antepura}, see Appendix of Technical Terms. } The officials there saw \textsanskrit{Ajātasattu}’s strange behavior and seized him. When they examined him, they found the dagger tied to his thigh. They asked him what he was up to. 

“I\marginnote{3.4.13} wish to kill my father.” 

“Has\marginnote{3.4.14} anyone encouraged you?” 

“Venerable\marginnote{3.4.15} Devadatta.” 

Some\marginnote{3.4.16} officials said, “The prince should be executed, together with Devadatta and all the monks.” Others said, “The monks shouldn’t be executed. They haven’t done anything wrong. The prince should be executed, together with Devadatta.” Still others said, “Neither the prince nor Devadatta nor the monks should be executed. The king should be informed, and we should do as he says.” 

They\marginnote{3.5.1} took \textsanskrit{Ajātasattu} with them and went to King Seniya \textsanskrit{Bimbisāra} of Magadha to inform him of what had happened. The king said, “What do you all make of this?” They told him their views. 

The\marginnote{3.5.14} king then said, “What’s this got to do with the Buddha, the Teaching, and the Sangha? Didn’t the Buddha have an announcement made in \textsanskrit{Rājagaha} as a warning: ‘Devadatta’s character has changed. Whatever Devadatta now does by body or speech has nothing to do with the Buddha, the Teaching, or the Sangha, but only with Devadatta’?” He then fired those officials who had suggested to execute the prince, Devadatta, and the Sangha; he demoted those who had suggested to execute the prince and Devadatta; and he promoted those who had suggested to act according to the king’s orders. 

And\marginnote{3.5.33} he said to \textsanskrit{Ajātasattu}, “Why do you want to kill me?” 

“I\marginnote{3.5.35} want to rule, sir.” 

“If\marginnote{3.5.36} you want to rule, the kingdom is yours.” And he handed the rulership over to the prince. 

\subsection*{6. The sending of assassins }

Soon\marginnote{3.6.1} afterwards Devadatta went to \textsanskrit{Ajātasattu} and said, “Great king, please order your men to kill the ascetic Gotama.” And the king told his men, “Do as Venerable Devadatta says.” 

Devadatta\marginnote{3.6.5} then told one man, “Go to such and such a place where the ascetic Gotama is staying. Kill him and return via that path.” On that path he stationed two men, saying, “Kill the man who comes along this path and return via that path.” On that path he stationed four men, saying, “Kill the two men who come along this path and return via that path.” On that path he stationed eight men, saying, “Kill the four men who come along this path and return via that path.” On that path he stationed sixteen men, saying, “Kill the eight men who come along this path and then return.” 

Soon\marginnote{3.7.1} afterwards that one man armed himself with a bow and arrows, as well as a sword and shield, and went to the Buddha. As he got close, he became fearful and agitated, standing rigidly. The Buddha saw him and said, “Come, don’t be afraid.” 

He\marginnote{3.7.5} then placed his sword and shield to one side, put down his bow and arrows, and went up to the Buddha. He bowed down with his head at the Buddha’s feet, and said, “Sir, I’ve made a mistake. I’ve been foolish, confused, and unskillful in coming here with a malicious mind intent on murder. Please forgive me so that I may restrain myself in the future.” 

“You\marginnote{3.7.8} have certainly made a mistake. You’ve been foolish, confused, and unskillful. But since you acknowledge your mistake and make proper amends, I forgive you. For this is called growth in the training of the noble ones: acknowledging a mistake, making proper amends, and undertaking restraint for the future.” 

The\marginnote{3.7.12} Buddha then gave him a progressive talk—on generosity, morality, and heaven; on the downside, degradation, and defilement of worldly pleasures; and he revealed the benefits of renunciation. When the Buddha knew that his mind was ready, supple, without hindrances, joyful, and confident, he revealed the teaching unique to the Buddhas: suffering, its origin, its end, and the path. And just as a clean and stainless cloth absorbs dye properly, so too, while he was sitting right there, he experienced the stainless vision of the Truth: “Anything that has a beginning has an end.” 

He\marginnote{3.7.19} had seen the Truth, had reached, understood, and penetrated it. He had gone beyond doubt and uncertainty, had attained to confidence, and had become independent of others in the Teacher’s instruction. And he said to the Buddha, “Wonderful, sir, wonderful! Just as one might set upright what’s overturned, or reveal what’s hidden, or show the way to one who’s lost, or bring a lamp into the darkness so that one with eyes might see what’s there—just so has the Buddha made the Teaching clear in many ways. I go for refuge to the Buddha, the Teaching, and the Sangha of monks. Please accept me as a lay follower who’s gone for refuge for life.” 

The\marginnote{3.7.26} Buddha then said to him, “Don’t go back along this path, go along that one.” And he sent him down a different path. 

The\marginnote{3.8.1} two men thought, “Why is it taking that one man so long to arrive?” As they were walking along that path in the opposite direction, they saw the Buddha seated at the foot of a tree. They approached him, bowed, and sat down. The Buddha gave them a progressive talk … and they had become independent of others in the Teacher’s instruction. And they said to the Buddha, “Wonderful, sir … Please accept us as lay followers who have gone for refuge for life.” 

The\marginnote{3.8.9} Buddha then said to them, “Don’t go back along this path, go along that one.” And he sent them down a different path. 

The\marginnote{3.8.11} four men … the eight men … the sixteen men thought, “Why is it taking those eight men so long to arrive?” As they were walking along that path in the opposite direction, they saw the Buddha seated at the foot of a tree. They approached him, bowed, and sat down. The Buddha gave them a progressive talk … and they had become independent of others in the Teacher’s instruction. And they said to the Buddha, “Wonderful, sir … Please accept us as lay followers who have gone for refuge for life.” 

\footnote{Here\marginnote{3.8.22} I follow the PTS edition, which does not have this or the next segment. The text means, “The Buddha then said to them, ‘Don’t go back along this path, go along that one.’ And he sent them down a different path”. This seems to be an error in MS based on automatic copying from the previous paragraphs. } 

Soon\marginnote{3.9.1} afterwards that one man went to Devadatta and said, “I wasn’t able to kill him, sir. He’s powerful and mighty, that Buddha.” 

“Never\marginnote{3.9.4} mind. There’s no need for you to kill the ascetic Gotama. I’ll do it myself.” 

\subsection*{7. Causing the Buddha to bleed }

Soon\marginnote{3.9.7.1} afterwards the Buddha was doing walking meditation in the shade of the Vulture Peak. Devadatta climbed the peak and threw down a large stone, thinking, “With this I’ll kill the ascetic Gotama.” But the stone got stuck in the junction of two outcrops. A chip flew off, striking the Buddha’s foot and causing him to bleed. The Buddha looked up and said to Devadatta, “Foolish man, you’ve made much demerit. With a malicious mind intent on murder you’ve made the Buddha bleed.” Soon afterwards the Buddha addressed the monks: 

“With\marginnote{3.9.15} a malicious mind intent on murder Devadatta has made the Buddha bleed. This is his first action with consequences in his very next life.” 

When\marginnote{3.10.1} the monks heard that Devadatta was trying to murder the Buddha, they walked back and forth on all sides of the Buddha’s dwelling, trying to protect him by reciting loudly. The Buddha heard that loud sound of recitation. He asked Ānanda what it was, and Ānanda told him. The Buddha said, “Well then, Ānanda, tell those monks in my name that the Teacher calls them.” 

Saying,\marginnote{3.10.13} “Yes, sir,” he did just that. The monks consented. They then went to the Buddha, bowed, and sat down. The Buddha said to them: 

“It’s\marginnote{3.10.17} impossible, monks, for anyone to kill me through an act of violence. The Buddha won’t attain final extinguishment through an act of violence. 

Monks,\marginnote{3.10.19} there are five kinds of teachers in the world.\footnote{Partial parallel to \href{https://suttacentral.net/an5.100/en/brahmali\#6.1}{AN 5.100:6.1}. } 

One\marginnote{3.10.21} kind of teacher is impure in behavior, while claiming to be pure. His disciples know about this, but think, ‘It would be unpleasant for him if we inform the householders. And it’s because of him that we’re honored with gifts of robe-cloth, almsfood, dwellings, and medicinal supplies. How, then, can we inform them?\footnote{The Pali is rather obscure. Vmv 4.334: \textit{\textsanskrit{Sammannatīti} \textsanskrit{cīvarādinā} \textsanskrit{amhākaṁ} \textsanskrit{sammānaṁ} karoti, parehi \textsanskrit{vā} \textsanskrit{ayaṁ} \textsanskrit{satthā} \textsanskrit{sammānīyatīti} attho}, “\textit{Sammannati}: the meaning is: he creates honor for us by way of robes, etc., or this teacher is being honored by others.” } He’ll be known through his own actions.’ The disciples conceal the impure behavior of such a teacher, and the teacher expects them to do so. 

Another\marginnote{3.10.30} kind of teacher is impure in livelihood … gives impure teachings … gives impure explanations … has impure knowledge and vision, while claiming they’re pure. His disciples know about this, but think, ‘It would be unpleasant for him if we inform the householders. And it’s because of him that we’re honored with gifts of robe-cloth, almsfood, dwellings, and medicinal supplies. How, then, can we inform them? He’ll be known through his own actions.’ The disciples conceal the impure knowledge and vision of such a teacher, and the teacher expects them to do so. 

But\marginnote{3.10.43} in my case, I claim my behavior is pure because it is. My disciples don’t conceal my behavior, and I don’t expect them to do so. I claim my livelihood is pure … I claim my teachings are pure … I claim my explanations are pure … I claim my knowledge and vision are pure because they are. My disciples don’t conceal my knowledge and vision, and I don’t expect them to do so. 

It’s\marginnote{3.10.50} impossible for anyone to kill me through an act of violence. The Buddha won’t attain final extinguishment through an act of violence. Go to your dwellings, monks. I don’t need any protection.” 

\subsection*{8. The letting loose of \textsanskrit{Nāḷāgiri} }

At\marginnote{3.11.1} that time in \textsanskrit{Rājagaha} there was a fierce and man-killing elephant called \textsanskrit{Nāḷāgiri}. Just then Devadatta entered \textsanskrit{Rājagaha}, went to the elephant stables, and said to the elephant keepers, “We who are relatives of the king are capable of having people promoted and getting them a raise. So then, when the ascetic Gotama comes walking along this street, release the elephant \textsanskrit{Nāḷāgiri} down it.” 

“Yes,\marginnote{3.11.5} sir.” 

Then,\marginnote{3.11.6} one morning, the Buddha robed up, took his bowl and robe, and entered \textsanskrit{Rājagaha} for alms together with a number of monks. And the Buddha walked down that very street. When the elephant keepers saw the Buddha coming, they released \textsanskrit{Nāḷāgiri} down the same street. \textsanskrit{Nāḷāgiri} saw the Buddha coming. He blew his trunk, and with ears and tail bristling, he charged toward the Buddha. When the monks saw \textsanskrit{Nāḷāgiri} coming, they said to the Buddha, “This elephant coming down the street is the fierce, man-killer \textsanskrit{Nāḷāgiri}. Please retreat, venerable sir.” 

“Come,\marginnote{3.11.17} don’t be afraid. It’s impossible for anyone to kill the Buddha through an act of violence. The Buddha won’t attain final extinguishment through an act of violence.” 

A\marginnote{3.11.20} second time and a third time those monks said the same thing to the Buddha, each time getting the same reply. 

On\marginnote{3.12.1} that occasion people had ascended their stilt houses and even their roofs. The foolish people with little faith and confidence said, “The elephant will hurt the handsome, great ascetic.” But the wise people with faith and confidence said, “Soon the great man and the great elephant will meet in battle.” 

The\marginnote{3.12.6} Buddha then pervaded \textsanskrit{Nāḷāgiri} with a mind of love. Feeling it, \textsanskrit{Nāḷāgiri} lowered his trunk, went up to the Buddha, and stood in front of him. And while stroking \textsanskrit{Nāḷāgiri} on the forehead with his right hand, the Buddha spoke these verses: 

\begin{verse}%
“Do\marginnote{3.12.9} not, elephant, attack a great man; \\
Painful it is to attack a great man. \\
For a killer of a great man, \\
The next birth is not good. 

Don’t\marginnote{3.12.13} be intoxicated or heedless, \\
For the heedless are not happily reborn. \\
Only do those things \\
That take you to a good destination.” 

%
\end{verse}

\textsanskrit{Nāḷāgiri}\marginnote{3.12.17} sucked the dust from the Buddha’s feet with his trunk and scattered it overhead. He then walked backward while looking at the Buddha, and returned to his stall in the elephant stables. That is how tame \textsanskrit{Nāḷāgiri} had become. On that occasion people chanted this verse: 

\begin{verse}%
“Some\marginnote{3.12.21} are tamed with sticks, \\
And some with goads and whips. \\
Without stick or sword, \\
The great sage tamed the elephant.” 

%
\end{verse}

And\marginnote{3.13.1} people complained and criticized Devadatta, “How evil and indiscriminate he is, this Devadatta, in trying to kill the ascetic Gotama so powerful and mighty!”\footnote{Sp 4.342: \textit{Alakkhikoti ettha na \textsanskrit{lakkhetīti} alakkhiko; na \textsanskrit{jānātīti} attho, \textsanskrit{ahaṁ} \textsanskrit{pāpakammaṁ} \textsanskrit{karomīti} na \textsanskrit{jānāti}}, “\textit{Alakkhika}: here \textit{alakkhika} means ʻOne does not distinguish’. The meaning is ʻHe does not know’.  He does not know ʻI am making bad karma’.” } Devadatta’s material support and honor declined, whereas those of the Buddha increased. 

\subsection*{9. The account of the request for the five points }

Because\marginnote{3.13.5.1} of his loss of material support and honor, Devadatta and his followers had to ask families repeatedly to get invited to meals.\footnote{Sp 2.209: \textit{Kulesu \textsanskrit{viññāpetvā} \textsanskrit{viññāpetvā} \textsanskrit{bhuñjatīti} “\textsanskrit{mā} me \textsanskrit{gaṇo} \textsanskrit{bhijjī}”ti \textsanskrit{parisaṁ} posento “\textsanskrit{tvaṁ} ekassa bhikkhuno \textsanskrit{bhattaṁ} dehi, \textsanskrit{tvaṁ} dvinna”nti \textsanskrit{evaṁ} \textsanskrit{viññāpetvā} sapariso kulesu \textsanskrit{bhuñjati}}, “\textit{Kulesu \textsanskrit{viññāpetvā} \textsanskrit{viññāpetvā} \textsanskrit{bhuñjati}}: thinking, ʻMay my group not split,’ he looks after his group, saying, ʻYou give food to one monk; you to two.’ Having asked families in this way, he eats together with his followers.” } People complained and criticized them, “How can the Sakyan monastics repeatedly ask families to get invited to meals? Who doesn’t like nice food? Who doesn’t prefer tasty food?” 

The\marginnote{3.13.9} monks heard the complaints of those people, and the monks of few desires complained and criticized those monks, “How can Devadatta and his followers repeatedly ask families to get invited to meals?” They told the Buddha. Soon afterwards he had the Sangha gathered and questioned Devadatta: “Is it true, Devadatta, that you do this?” “It’s true, sir.” … After rebuking him … the Buddha gave a teaching and addressed the monks: 

“Well\marginnote{3.13.17} then, for monks eating among families, I’ll lay down a rule against eating in groups of more than three, for these three reasons: for the restraint of bad people; for the ease of good monks, stopping those with bad desires from creating a faction and then a schism in the Sangha; and out of compassion for families. Anyone eating in a group is to be dealt with according to the rule.”\footnote{See \href{https://suttacentral.net/pli-tv-bu-vb-pc32/en/brahmali\#8.15}{Bu Pc 32:8.15}. } 

Soon\marginnote{3.14.1} afterwards Devadatta went to \textsanskrit{Kokālika}, \textsanskrit{Kaṭamodakatissaka}, \textsanskrit{Khaṇḍadeviyāputta}, and Samuddadatta. He said to them, “Let’s cause a schism in the Sangha of the ascetic Gotama. Let’s break its authority.”\footnote{“Break its authority” renders \textit{cakkabheda}. Sp 3.410: \textit{\textsanskrit{Cakkabhedāyāti} \textsanskrit{āṇābhedāya}}, “\textit{Cakkabheda}: by breaking the authority.” Vjb 4.343: \textit{Cakkabhedanti \textsanskrit{sāsanabhedaṁ}}, “\textit{Cakkabheda}: a break in the instruction.” The break in authority is presumably both from the Buddha and the Sangha. Although the Buddha was the only authority in laying down rules, the Sangha was autonomous in its decision making. For practical purposes, it was the Sangha that Devadatta was breaking with. } 

\textsanskrit{Kokālika}\marginnote{3.14.3} said to Devadatta, “The ascetic Gotama is powerful and mighty. How can we achieve this?” 

“Well,\marginnote{3.14.6} let’s go to the ascetic Gotama and request five things: ‘In many ways, sir, you praise fewness of wishes, contentment, self-effacement, ascetic practices, being inspiring, the reduction in things, and being energetic. And there are five things that lead to just that: 

\begin{enumerate}%
\item It would be good, sir, if the monks stayed in the wilderness for life, and whoever stayed near an inhabited area would commit an offense;\footnote{For the rendering “inhabited area” for \textit{\textsanskrit{gāma}}, see Appendix of Technical Terms. } %
\item if they ate only almsfood for life, and whoever accepted an invitational meal would commit an offense; %
\item if they wore rag-robes for life, and whoever accepted robe-cloth from a householder would commit an offense; %
\item if they lived at the foot of a tree for life, and whoever took shelter would commit an offense; %
\item if they didn’t eat fish or meat for life, and whoever did would commit an offense.’ %
\end{enumerate}

The\marginnote{3.14.14} ascetic Gotama won’t allow this. We’ll then be able to win people over with these five points.” 

\textsanskrit{Kokālika}\marginnote{3.14.16} said, “It might be possible to cause a schism in the Sangha with these five points, for people have confidence in austerity.” 

Devadatta\marginnote{3.15.1} and his followers then went to the Buddha, bowed, sat down, and Devadatta made his request. The Buddha replied, “No, Devadatta. Those who wish may stay in the wilderness and those who wish may live near inhabited areas. Those who wish may eat only almsfood and those who wish may accept invitations. Those who wish may wear rag-robes and those who wish may accept robe-cloth from householders. I have allowed the foot of a tree as a resting place for eight months of the year,\footnote{For an explanation of the rendering “resting place” for \textit{\textsanskrit{senāsana}}, see Appendix of Technical Terms. } as well as fish and meat that are pure in three respects: one hasn’t seen, heard, or suspected that the animal was specifically killed to feed a monastic.” 

Devadatta\marginnote{3.15.17} thought, “The Buddha doesn’t allow the five points.” Glad and elated, he got up from his seat, bowed down, circumambulated the Buddha with his right side toward him, and left with his followers. 

Devadatta\marginnote{3.15.20} then entered \textsanskrit{Rājagaha} and won people over with the five points, saying, “The ascetic Gotama doesn’t agree to them, but we practice in accordance with them.” 

The\marginnote{3.16.1} foolish people with little faith and confidence said, “These Sakyan monastics are practicing asceticism and living with the aim of self-effacement. But the ascetic Gotama is extravagant and has chosen a life of indulgence.” But the wise people who had faith and confidence complained and criticized Devadatta, “How can Devadatta pursue schism in the Sangha of the Buddha? How can he break its authority?” 

The\marginnote{3.16.6} monks heard the criticism of those people, and the monks of few desires complained and criticized him in the same way. 

They\marginnote{3.16.9} then told the Buddha. Soon afterwards he had the Sangha gathered and questioned Devadatta: 

“Is\marginnote{3.16.10} it true, Devadatta, that you are doing this?” 

“It’s\marginnote{3.16.11} true, sir.” 

“Let\marginnote{3.16.12} it be, Devadatta, don’t cause a schism in the Sangha. Schism in the Sangha is a serious matter. Whoever causes a schism in a united Sangha does a bad act with effect for an eon. He’s boiled in hell for an eon. But whoever unites a divided Sangha generates the supreme merit. He rejoices in heaven for an eon. So let it be, Devadatta, don’t cause a schism in the Sangha. It’s a serious matter.” 

Soon\marginnote{3.17.1} afterwards, Venerable Ānanda robed up in the morning, took his bowl and robe, and entered \textsanskrit{Rājagaha} for alms. When Devadatta saw him, he went up to him, and said, “From today on, Ānanda, I’ll do the observance-day ceremony and the legal procedures of the Sangha separate from the Buddha and the Sangha of monks.”\footnote{For the rendering “observance-day ceremony” for \textit{uposatha}, see Appendix of Technical Terms. } 

When\marginnote{3.17.5} Ānanda had completed his almsround, eaten his meal, and returned, he went to the Buddha, bowed, sat down, and told him what had happened, adding, “From today, sir, Devadatta has caused a schism in the Sangha.” Understanding the significance of this, on that occasion the Buddha uttered a heartfelt exclamation: 

\begin{verse}%
“For\marginnote{3.17.13} the good, doing good is easy; \\
For the bad, doing good is hard. \\
For the bad, doing evil is easy; \\
For the noble ones, doing evil is hard.” 

%
\end{verse}

\scend{The second section for recitation is finished. }

\section*{The third section}

\subsection*{3.1 The account of schism in the Sangha }

On\marginnote{4.1.1} the observance day soon afterwards, Devadatta got up from his seat and distributed ballots, saying, “We have gone to the ascetic Gotama and asked for five things: ‘In many ways, sir, you praise fewness of wishes, contentment, self-effacement, ascetic practices, being inspiring, the reduction in things, and being energetic. And there are five things that lead to just that: 

\begin{enumerate}%
\item It would be good, sir, if the monks stayed in the wilderness for life, and whoever stayed near inhabited areas would commit an offense; %
\item if they were alms-collectors for life, and whoever accepted an invitation would commit an offense; %
\item if they were rag-robe wearers for life, and whoever accepted robe-cloth from a householder would commit an offense; %
\item if they dwelt at the foot of a tree for life, and whoever took shelter would commit an offense; %
\item if they didn’t eat fish or meat for life, and whoever did would commit an offense.’ %
\end{enumerate}

The\marginnote{4.1.9} ascetic Gotama doesn’t agree to them, but we practice in accordance with them. Any monk who approves of these five things should vote in favor.” 

On\marginnote{4.1.12} that occasion five hundred Vajjian monks from \textsanskrit{Vesālī}, newly ordained and ignorant, were present. Thinking, “This is the Teaching, this is the Monastic Law, this is the Teacher’s instruction,” they voted in favor. 

Then,\marginnote{4.1.14} after causing a schism in the Sangha, Devadatta left for \textsanskrit{Gayāsīsa} together with the five hundred monks. 

Soon\marginnote{4.1.15} afterwards \textsanskrit{Sāriputta} and \textsanskrit{Moggallāna} went to the Buddha, bowed, sat down, and said, “Sir, Devadatta has split the Sangha and left for \textsanskrit{Gayāsīsa} together with five hundred monks.” 

“You\marginnote{4.1.18} have compassion for those five hundred newly ordained monks, don’t you? Go then, \textsanskrit{Sāriputta} and \textsanskrit{Moggallāna}, before they’re afflicted with misfortune and disaster.” 

Saying,\marginnote{4.1.20} “Yes, sir,” they got up from their seats, bowed down, circumambulated the Buddha with their right sides toward him, and went to \textsanskrit{Gayāsīsa}. 

Just\marginnote{4.1.21} then a certain monk who was standing near the Buddha was crying. The Buddha asked him why. He said, “Sir, even \textsanskrit{Sāriputta} and \textsanskrit{Moggallāna}, the Buddha’s chief disciples, are going to Devadatta because they approve of his teaching.” 

“It’s\marginnote{4.1.25} impossible for \textsanskrit{Sāriputta} and \textsanskrit{Moggallāna} to approve of Devadatta’s teaching. In fact, they’ve gone to win those monks over.” 

At\marginnote{4.2.1} that time Devadatta was seated giving a teaching surrounded by a large gathering. When Devadatta saw \textsanskrit{Sāriputta} and \textsanskrit{Moggallāna} coming, he said to his monks, “See how well-taught my teaching is, as even \textsanskrit{Sāriputta} and \textsanskrit{Moggallāna}, the ascetic Gotama’s chief disciples, are coming here because they approve of it.” 

But\marginnote{4.2.6} \textsanskrit{Kokālika} said, “Don’t trust \textsanskrit{Sāriputta} and \textsanskrit{Moggallāna}. They have bad desires. They’re in the grip of bad desires.” 

“Don’t\marginnote{4.2.9} worry. Anyone who comes to approve of my teaching is welcome.” 

Devadatta\marginnote{4.2.11} invited Venerable \textsanskrit{Sāriputta} to sit on a seat half the height of his own. Saying, “There’s no need,” \textsanskrit{Sāriputta} took another seat and sat down to one side, as did \textsanskrit{Mahāmoggallāna}. After spending most of the night instructing, inspiring, and gladdening the monks with a teaching, Devadatta invited \textsanskrit{Sāriputta}, saying “The Sangha of monks is without dullness and drowsiness. Give a teaching, \textsanskrit{Sāriputta}. My back is aching. I need to stretch it.” 

“Yes.”\marginnote{4.2.18} 

Devadatta\marginnote{4.2.19} then folded his upper robe in four and lay down on his right side. Because he was tired, absentminded, and heedless, he fell asleep immediately. 

Venerable\marginnote{4.3.1} \textsanskrit{Sāriputta} then used the wonder of mind reading to instruct those monks, and Venerable \textsanskrit{Mahāmoggallāna} used the wonder of supernormal powers to the same effect. While they were being instructed like this, they experienced the stainless vision of the Truth: “Anything that has a beginning has an end.” 

And\marginnote{4.3.5} \textsanskrit{Sāriputta} addressed them: “We’re going to the Buddha. Whoever approves of the teaching of the Buddha should come along.” \textsanskrit{Sāriputta} and \textsanskrit{Moggallāna} then went to the Bamboo Grove accompanied by those five hundred monks. 

In\marginnote{4.3.9} the meantime \textsanskrit{Kokālika} woke up Devadatta, saying, “Get up, Devadatta, your monks are being led away by \textsanskrit{Sāriputta} and \textsanskrit{Moggallāna}. Didn’t I tell you not to trust \textsanskrit{Sāriputta} and \textsanskrit{Moggallāna}? Didn’t I say that they have bad desires, that they are in the grip of bad desires?” And Devadatta vomited hot blood right there. 

\textsanskrit{Sāriputta}\marginnote{4.4.1} and \textsanskrit{Moggallāna} then went to the Buddha, bowed, sat down, and said, “Sir, may we reordain the monks who supported the schism?” 

“Let\marginnote{4.4.4} it be, \textsanskrit{Sāriputta}, don’t think of reordaining the monks who supported the schism. Instead, have them confess a serious offense. And Devadatta, how did he treat you?” 

“Just\marginnote{4.4.8} as you, sir, spend most of the night instructing, inspiring, and gladdening the monks with a teaching, and then invite me, saying, ‘The Sangha of monks is without dullness and drowsiness. Give a teaching, \textsanskrit{Sāriputta}. My back is aching. I need to stretch it,’ that’s how Devadatta treated us.” 

The\marginnote{4.5.1} Buddha then addressed the monks: 

“Once\marginnote{4.5.2} upon a time there was a great lake in a wilderness area with elephants living nearby. After plunging into the lake, they pulled up lotus roots and tubers with their trunks. They gave them a good rinse to remove the mud, before chewing and swallowing them. That gave them beauty and strength. And they didn’t die or experience death-like suffering because of that. 

Then\marginnote{4.5.7} the baby elephants tried to imitate those great elephants. After plunging into the lake, they pulled up lotus roots and tubers with their trunks. But they didn’t give them a good rinse to remove the mud, and so they chewed and swallowed them while muddy. That didn’t give them any beauty or strength. And they died or experienced death-like suffering because of that. Just so, by imitating me, Devadatta will die miserably. 

\begin{verse}%
‘While\marginnote{4.5.12} the great being removes the earth,\footnote{Sp 4.346: \textit{\textsanskrit{Mahāvarāhassāti} \textsanskrit{mahānāgassa}}, “\textit{\textsanskrit{Mahāvarāhassa}} means the great being/elephant.” } \\
Eats the tuber, and is alert in the rivers, \\
He’s like a baby elephant that’s eaten mud: \\
By imitating me, he’ll die miserably.’ 

%
\end{verse}

“Monks,\marginnote{4.6.1} a monk who has eight qualities is qualified to take messages.\footnote{Parallel to \href{https://suttacentral.net/an8.16/en/brahmali\#0.3}{AN 8.16:0.3}. } He listens and communicates, he learns and remembers, he understands and gets things across, he’s skilled in what is and what isn’t relevant, he’s not argumentative. 

Because\marginnote{4.6.6} he has these eight qualities, \textsanskrit{Sāriputta} is qualified to take messages. 

\begin{verse}%
‘He\marginnote{4.6.9} doesn’t tremble when faced \\
With a gathering of fierce debaters. \\
He doesn’t neglect the words \\
Or fail to get the instruction across. 

He\marginnote{4.6.13} speaks with confidence \\
And isn’t agitated when questioned. \\
This kind of monk, indeed, \\
Is qualified to take messages.’ 

%
\end{verse}

“It’s\marginnote{4.7.1} because he’s overcome and consumed by eight bad qualities that Devadatta is irredeemably destined to an eon in hell.\footnote{Partial parallel to \href{https://suttacentral.net/an8.7/en/brahmali\#0.3}{AN 8.7:0.3}. } What eight? Material support, lack of material support, being popular, being unpopular,\footnote{Sp 2.264: \textit{Ayasoti \textsanskrit{parivāravipatti}; \textsanskrit{parammukhagarahā} \textsanskrit{vā}}, “\textit{Ayaso} means lacking an entourage or being disparaged in one’s absence.” } honor, lack of honor, bad desires, and bad friendship. 

It’s\marginnote{4.7.12} good for a monk to overcome whatever material support he’s affected by, whatever lack of material support he’s affected by, whatever popularity he’s affected by, whatever unpopularity he’s affected by, whatever honor he’s affected by, whatever lack of honor he’s affected by, whatever bad desires he’s affected by, and whatever bad friendship he’s affected by. 

For\marginnote{4.7.19} what reason should a monk overcome these things? 

If\marginnote{4.7.26} he doesn’t overcome whatever material support he’s affected by, he will experience distressful and feverish corruptions. But if he overcomes whatever material support he’s affected by, he won’t have those distressful and feverish corruptions. If he doesn’t overcome whatever lack of material support he’s affected by, whatever popularity he’s affected by, whatever unpopularity he’s affected by, whatever honor he’s affected by, whatever lack of honor he’s affected by, whatever bad desires he’s affected by, or whatever bad friendship he’s affected by, he will experience distressful and feverish corruptions. But if he overcomes whatever bad friendship he’s affected by, he won’t have those distressful and feverish corruptions. 

And\marginnote{4.7.34} so, you should overcome whatever material support you’re affected by, whatever lack of material support you’re affected by, whatever popularity you’re affected by, whatever unpopularity you’re affected by, whatever honor you’re affected by, whatever lack of honor you’re affected by, whatever bad desires you’re affected by, and whatever bad friendship you’re affected by. This is how you should train yourselves. 

And,\marginnote{4.7.49} monks, it’s because he’s overcome and consumed by three bad qualities that Devadatta is irredeemably destined to an eon in hell.\footnote{Parallel to \href{https://suttacentral.net/iti89/en/brahmali\#1.1}{Iti 89}. } What three? Bad desires; bad friendship; and after trifling successes, he has stopped short of the goal.” 

\begin{verse}%
“No-one\marginnote{4.8.1} with bad desires \\
Is ever reborn in this world. \\
In this way you may know \\
The destination of those with bad desires. 

Designated\marginnote{4.8.5} as “wise”, \\
Agreed upon as “highly developed”, \\
It was as if he was shining with fame—\\
I have heard Devadatta was like this. 

He\marginnote{4.8.9} was heedless, \\
And after hurting the Buddha, \\
He’s gone to the \textsanskrit{Avīci} hell, \\
Frightful and with four doors. 

If\marginnote{4.8.13} you hurt one free from anger, \\
One who doesn’t do anything bad, \\
You experience that evil yourself, \\
Having a malicious mind and being disrespectful. 

You\marginnote{4.8.17} might think to pollute \\
The ocean with a pot of poison, \\
But you would not be able to do so, \\
For the ocean is frightfully large.\footnote{\textit{\textsanskrit{Bhesmā}}, literally, “frightening”. The commentary to the parallel verse at \href{https://suttacentral.net/iti89/en/brahmali\#7.4}{Iti 89} says: \textit{\textsanskrit{Bhesmāti} \textsanskrit{vipulabhāvena} \textsanskrit{gambhīrabhāvena} ca \textsanskrit{bhiṁsāpento} viya, \textsanskrit{vipulagambhīroti} attho}, “‘Frightening’ means large and deep, as in frightening because of its size and depth.” } 

It’s\marginnote{4.8.21} the same with the Buddha: \\
If by speech one tries to harm him—\\
He with right conduct and a peaceful mind—\footnote{Reading \textit{sammaggata}/\textit{\textsanskrit{sammāgata}} with the Vipassana Research Institute edition at https://www.tipitaka.org and the PTS edition. } \\
That speech doesn’t affect him. 

The\marginnote{4.8.25} wise make friends with such a person, \\
And they associate with him. \\
The monk who follows his path, \\
Achieves the end of suffering.” 

%
\end{verse}

\subsection*{11. \textsanskrit{Upāli}’s questions }

On\marginnote{5.1.1} one occasion \textsanskrit{Upāli} went to the Buddha, bowed, sat down, and said, “Sir, we speak of ‘fracture in the Sangha’. But how is there a fracture in the Sangha, yet not a schism in the Sangha? And how is there both a fracture and a schism in the Sangha?” 

\begin{itemize}%
\item “If, \textsanskrit{Upāli}, there is one monk on one side and two on the other, and a fourth makes the proclamation and distributes ballots, saying, ‘This is the Teaching, this is the Monastic Law, this is the Teacher’s instruction; take this, approve of this,’\footnote{“Take this” refers to the ballots. Sp 4.351: \textit{\textsanskrit{Salākaṁ} \textsanskrit{gāhetīti} \textsanskrit{evaṁ} \textsanskrit{anussāvetvā} pana “\textsanskrit{idaṁ} \textsanskrit{gaṇhatha}, \textsanskrit{idaṁ} \textsanskrit{rocethā}”ti vadanto \textsanskrit{salākaṁ} \textsanskrit{gāheti}}, “Distributes ballots: having proclaimed this, saying, ‘Take this, approve of this,’ he distributes the ballots.” } then this is a fracture in the Sangha, but not a schism in the Sangha. %
\item If there are two monks on one side and two on the other, and a fifth makes the proclamation and distributes ballots, saying, ‘This is the Teaching, this is the Monastic Law, this is the Teacher’s instruction; take this, approve of this,’ then this is a fracture in the Sangha, but not a schism in the Sangha. %
\item If there are two monks on one side and three on the other, and a sixth makes the proclamation and distributes ballots, saying, ‘This is the Teaching, this is the Monastic Law, this is the Teacher’s instruction; take this, approve of this,’ then this is a fracture in the Sangha, but not a schism in the Sangha. %
\item If there are three monks on one side and three on the other, and a seventh makes the proclamation and distributes ballots, saying, ‘This is the Teaching, this is the Monastic Law, this is the Teacher’s instruction; take this, approve of this,’ then this is a fracture in the Sangha, but not a schism in the Sangha. %
\item If there are three monks on one side and four on the other, and an eighth makes the proclamation and distributes ballots, saying, ‘This is the Teaching, this is the Monastic Law, this is the Teacher’s instruction; take this, approve of this,’ then this is a fracture in the Sangha, but not a schism in the Sangha. %
\item But if there are four monks on one side and four on the other, and a ninth makes the proclamation and distributes ballots, saying, ‘This is the Teaching, this is the Monastic Law, this is the Teacher’s instruction; take this, approve of this,’ then this is both a fracture in the Sangha and also a schism in the Sangha. %
\item \scrule{If there are nine or more monks, then this is both a fracture in the Sangha and also a schism in the Sangha. }

%
\end{itemize}

A\marginnote{5.1.25} nun cannot cause a schism in the Sangha, even if she makes an effort to do so. A trainee nun, a novice monk, a novice nun, a male lay follower, or a female lay follower cannot cause a schism in the Sangha, even if she makes an effort to do so. 

\scrule{Only a monk of regular standing, one who belongs to the same Buddhist sect and is present within the same monastery zone, can cause a schism in the Sangha.”\footnote{For an explanation of the rendering “monastery zone” for \textit{\textsanskrit{sīmā}}, see Appendix of Technical Terms. } }

“Sir,\marginnote{5.2.1} we speak of ‘schism in the Sangha’. But how is there a schism in the Sangha?” 

\begin{enumerate}%
\item “Take the case when monks proclaim what’s contrary to the Teaching as being in accordance with it. %
\item They proclaim what’s in accordance with the Teaching as contrary to it. %
\item They proclaim what’s contrary to the Monastic Law as being in accordance with it. %
\item They proclaim what’s in accordance with the Monastic Law as contrary to it. %
\item They proclaim what hasn’t been spoken by the Buddha as spoken by him. %
\item They proclaim what’s been spoken by the Buddha as not spoken by him. %
\item They proclaim what wasn’t practiced by the Buddha as practiced by him. %
\item They proclaim what was practiced by the Buddha as not practiced by him. %
\item They proclaim what wasn’t laid down by the Buddha as laid down by him. %
\item They proclaim what was laid down by the Buddha as not laid down by him. %
\item They proclaim a non-offense as an offense. %
\item They proclaim an offense as a non-offense. %
\item They proclaim a light offense as heavy. %
\item They proclaim a heavy offense as light. %
\item They proclaim a curable offense as incurable. %
\item They proclaim an incurable offense as curable. %
\item They proclaim a grave offense as minor. %
\item They proclaim a minor offense as grave. %
\end{enumerate}

\scrule{If, based on any of these eighteen grounds, they pull away and separate, and they do the observance-day ceremony, the invitation ceremony, or legal procedures of the Sangha separately, then there is a schism in the Sangha.”\footnote{For an explanation of the rendering “invitation ceremony” for \textit{\textsanskrit{pavāraṇā}}, see Appendix of Technical Terms. } }

“Sir,\marginnote{5.3.1} we speak of ‘unity in the Sangha’. But how is there unity in the Sangha?” 

\begin{enumerate}%
\item “Take the case when monks proclaim what’s contrary to the Teaching as such. %
\item They proclaim what’s in accordance with the Teaching as such. %
\item They proclaim what’s contrary to the Monastic Law as such. %
\item They proclaim what’s in accordance with the Monastic Law as such. %
\item They proclaim what hasn’t been spoken by the Buddha as such. %
\item They proclaim what’s been spoken by the Buddha as such. %
\item They proclaim what wasn’t practiced by the Buddha as such. %
\item They proclaim what was practiced by the Buddha as such. %
\item They proclaim what wasn’t laid down by the Buddha as such. %
\item They proclaim what was laid down by the Buddha as such. %
\item They proclaim a non-offense as such. %
\item They proclaim an offense as such. %
\item They proclaim a light offense as light. %
\item They proclaim a heavy offense as heavy. %
\item They proclaim a curable offense as curable. %
\item They proclaim an incurable offense as incurable. %
\item They proclaim a grave offense as grave. %
\item They proclaim a minor offense as minor. %
\end{enumerate}

\scrule{If, based on any of these eighteen grounds, they don’t pull away or separate, and they don’t do the observance-day ceremony, the invitation ceremony, or legal procedures of the Sangha separately, then there is unity in the Sangha.” }

“But,\marginnote{5.4.1} sir, what’s the consequence of causing a schism in a united Sangha?”\footnote{Partial parallels to \href{https://suttacentral.net/an10.39/en/brahmali\#0.3}{AN 10.39:0.3} and \href{https://suttacentral.net/iti18/en/brahmali\#1.1}{Iti 18}. } 

“Anyone\marginnote{5.4.2} who causes a schism in a united Sangha does an evil act with effect for an eon. He’s boiled in hell for an eon.” 

\begin{verse}%
“Going\marginnote{5.4.3} downwards, bound for hell—\\
The schismatic stays there for an eon. \\
Delighting in division and immoral, \\
Barred from sanctuary, \\
Having divided a united Sangha, \\
He boils in hell for an eon.” 

%
\end{verse}

“But,\marginnote{5.4.9} sir, what’s the consequence of uniting a schismatic sangha?”\footnote{Partial parallels to \href{https://suttacentral.net/an10.40/en/brahmali\#0.3}{AN 10.40:0.3} and \href{https://suttacentral.net/iti19/en/brahmali\#1.1}{Iti 19}. } 

“Anyone\marginnote{5.4.10} who unites a schismatic sangha generates supreme merit. He rejoices in heaven for an eon.” 

\begin{verse}%
“Pleasant\marginnote{5.4.11} is unity in the Sangha, \\
And to help the fostering of harmony.\footnote{\textit{\textsanskrit{Samaggānañca} anuggaho}, literally, “and the helping of those who are united”. The commentary to the parallel verse at \href{https://suttacentral.net/iti19/en/brahmali\#3.2}{Iti 19:3.2} says: \textit{\textsanskrit{Samaggānañcanuggahoti} \textsanskrit{samaggānaṁ} \textsanskrit{sāmaggianumodanena} \textsanskrit{anuggaṇhanaṁ} \textsanskrit{sāmaggianurūpaṁ}, \textsanskrit{yathā} te \textsanskrit{sāmaggiṁ} na vijahanti, \textsanskrit{tathā} \textsanskrit{gahaṇaṁ} \textsanskrit{ṭhapanaṁ} \textsanskrit{anubalappadānanti} attho}, “\textit{\textsanskrit{Samaggānañcanuggaho}} means: one who rejoices in unity helps those who are united in accordance with what is suitable for unity; he takes hold of it, firms it up, and supports it, so that they do not abandon that unity.” } \\
Delighting in unity and moral, \\
Not barred from sanctuary, \\
Having united the Sangha, \\
He rejoices in heaven for an eon.” 

%
\end{verse}

“Might\marginnote{5.5.1} one who causes a schism in the Sangha be irredeemably destined to an eon in hell?” 

“He\marginnote{5.5.2} might.” 

“Might\marginnote{5.5.3} one who causes a schism in the Sangha not be irredeemably destined to an eon in hell?” 

“He\marginnote{5.5.4} might.” 

\subsection*{Schismatics destined to hell}

“What\marginnote{5.5.5.1} sort of person who causes a schism in the Sangha is irredeemably destined to an eon in hell?” 

“In\marginnote{5.5.6} this case a monk proclaims what’s contrary to the Teaching as being in accordance with it. He has the view that what he says is illegitimate and the view that the schism is illegitimate. He misrepresents his view of what’s true, his belief of what’s true, his acceptance of what’s true, or his sentiment of what’s true. He makes a proclamation and distributes ballots, saying, ‘This is the Teaching, this is the Monastic Law, this is the Teacher’s instruction; take this, approve of this.’ When such a person causes a schism in the Sangha, he’s irredeemably destined to an eon in hell. 

Again,\marginnote{5.5.10} a monk proclaims what’s contrary to the Teaching as being in accordance with it. He has the view that what he says is illegitimate, but the view that the schism is legitimate. He misrepresents his view of what’s true, his belief of what’s true, his acceptance of what’s true, or his sentiment of what’s true. He makes a proclamation and distributes ballots, saying, ‘This is the Teaching, this is the Monastic Law, this is the Teacher’s instruction; take this, approve of this.’ When such a person causes a schism in the Sangha, he too is irredeemably destined to an eon in hell. 

Again,\marginnote{5.5.14} a monk proclaims what’s contrary to the Teaching as being in accordance with it. He has the view that what he says is illegitimate, but is unsure about the schism. He misrepresents his view of what’s true, his belief of what’s true, his acceptance of what’s true, or his sentiment of what’s true. He makes a proclamation and distributes ballots, saying, ‘This is the Teaching, this is the Monastic Law, this is the Teacher’s instruction; take this, approve of this.’ When such a person causes a schism in the Sangha, he too is irredeemably destined to an eon in hell. 

Again,\marginnote{5.5.18} a monk proclaims what’s contrary to the Teaching as being in accordance with it. He has the view that what he says is legitimate, but the view that the schism is illegitimate … \footnote{This is only found in MS, and is missing in the PTS edition. It could be rendered as follows: “He has the view that what he says is legitimate and the view that the schism is legitimate.” This seems to be a mistake in MS, since the whole point of this is that the monk “misrepresents his view of what’s true, his belief of what’s true, his acceptance of what’s true, or his sentiment of what’s true”. This assessment is supported by the fact that the parallel passage at \href{https://suttacentral.net/pli-tv-pvr19/en/brahmali\#80.1}{Pvr 19:80.1} does not have this combination. See Sp 5.477, which gives the details on this passage, which is only contracted in the Canonical text. } He has the view that what he says is legitimate, but is unsure about the schism … He is unsure about what he says, but has the view that the schism is illegitimate … He is unsure about what he says, but has the view that the schism is legitimate … He is unsure about what he says and is unsure about the schism. He misrepresents his view of what’s true, his belief of what’s true, his acceptance of what’s true, or his sentiment of what’s true. He makes a proclamation and distributes ballots, saying, ‘This is the Teaching, this is the Monastic Law, this is the Teacher’s instruction; take this, approve of this.’ When such a person causes a schism in the Sangha, he too is irredeemably destined to an eon in hell. 

Again,\marginnote{5.5.27} a monk proclaims what’s in accordance with the Teaching as contrary to it, what’s contrary to the Monastic Law as being in accordance with it, what’s in accordance with the Monastic Law as contrary to it, what hasn’t been spoken by the Buddha as spoken by him, what’s been spoken by the Buddha as not spoken by him, what wasn’t practiced by the Buddha as practiced by him, what was practiced by the Buddha as not practiced by him, what wasn’t laid down by the Buddha as laid down by him, what was laid down by the Buddha as not laid down by him, a non-offense as an offense, an offense as a non-offense, a light offense as heavy, a heavy offense as light, a curable offense as incurable, an incurable offense as curable, a grave offense as minor, or a minor offense as grave. He has the view that what he says is illegitimate and the view that the schism is illegitimate. … He has the view that what he says is illegitimate, but the view that the schism is legitimate. … He has the view that what he says is illegitimate, but is unsure about the schism. … He has the view that what he says is legitimate, but the view that the schism is illegitimate. … \footnote{This too is only found in MS, and is missing in the PTS edition. As pointed out above, this must be a mistake in MS. } He has the view that what he says is legitimate, but is unsure about the schism. … He is unsure about what he says, but has the view that the schism is illegitimate. … He is unsure about what he says, but has the view that the schism is legitimate. … He is unsure about what he says and unsure about the schism. He misrepresents his view of what’s true, his belief of what’s true, his acceptance of what’s true, or his sentiment of what’s true. He makes a proclamation and distributes ballots, saying, ‘This is the Teaching, this is the Monastic Law, this is the Teacher’s instruction; take this, approve of this.’ When such a person causes a schism in the Sangha, he too is irredeemably destined to an eon in hell.” 

\subsection*{Schismatics not destined to hell}

“What\marginnote{5.6.1} sort of person who causes a schism in the Sangha isn’t irredeemably destined to an eon in hell?” 

“In\marginnote{5.6.2} this case a monk proclaims what’s contrary to the Teaching as being in accordance with it. He has the view that what he says is legitimate and the view that the schism is legitimate. He doesn’t misrepresent his view of what’s true, his belief of what’s true, his acceptance of what’s true, or his sentiment of what’s true. He makes a proclamation and distributes ballots, saying, ‘This is the Teaching, this is the Monastic Law, this is the Teacher’s instruction; take this, approve of this.’ When such a person causes a schism in the Sangha, he’s not irredeemably destined to an eon in hell. 

Again,\marginnote{5.6.6} a monk proclaims what’s in accordance with the Teaching as contrary to it … or a minor offense as grave.\footnote{The Pali of MS is incorrect. I follow the PTS edition which has \textit{\textsanskrit{aduṭṭhullaṁ} \textsanskrit{āpattiṁ} \textsanskrit{duṭṭhullā} \textsanskrit{āpattīti} \textsanskrit{dīpeti}}. } He has the view that what he says is legitimate and the view that the schism is legitimate. He doesn’t misrepresent his view of what’s true, his belief of what’s true, his acceptance of what’s true, or his sentiment of what’s true. He makes a proclamation and distributes ballots, saying, ‘This is the Teaching, this is the Monastic Law, this is the Teacher’s instruction; take this, approve of this.’ When such a person causes a schism in the Sangha, he too isn’t irredeemably destined to an eon in hell.” 

\scend{The third section for recitation is finished. }

\scendsutta{The seventh chapter on schism in the Sangha is finished. }

\scuddanaintro{This is the summary: }

\begin{scuddana}%
“At\marginnote{5.6.14} Anupiya, well-known, \\
Great comfort, did not wish; \\
Plow, sow, irrigate, drain, \\
Weed, and cut, gather. 

Sheaves,\marginnote{5.6.18} thresh, and straw, \\
Husk, winnow, storage; \\
Also the future, they never stop, \\
And fathers, grandfathers. 

Bhaddiya,\marginnote{5.6.22} and Anuruddha, \\
Ānanda, Bhagu, Kimila; \\
And Sakyan pride, \textsanskrit{Kosambī}, \\
Disappeared, and with Kakudha. 

He\marginnote{5.6.26} announced, and of the father, \\
Men, stone, \textsanskrit{Nāḷāgiri}; \\
A triad, five, serious, \\
He split, and with a serious offense; \\
Three, eight, again, three, \\
Fracture, schism, might there be.” 

%
\end{scuddana}

\scendsutta{The chapter on schism in the Sangha is finished. }

%
\chapter*{{\suttatitleacronym Kd 18}{\suttatitletranslation The chapter on proper conduct }{\suttatitleroot Vattakkhandhaka}}
\addcontentsline{toc}{chapter}{\tocacronym{Kd 18} \toctranslation{The chapter on proper conduct } \tocroot{Vattakkhandhaka}}
\markboth{The chapter on proper conduct }{Vattakkhandhaka}
\extramarks{Kd 18}{Kd 18}

\section*{1. Discussion of the proper conduct for newly-arrived monks }

At\marginnote{1.1.1} one time the Buddha was staying at \textsanskrit{Sāvatthī} in the Jeta Grove, \textsanskrit{Anāthapiṇḍika}’s Monastery. At that time newly-arrived monks entered the monastery wearing sandals, holding sunshades, with their heads covered, with their robes on their heads; and they washed their feet with drinking water, did not bow down to the resident monks who were senior to them, and did not ask about dwellings. A certain newly-arrived monk lifted the latch of an unoccupied dwelling, opened the door, and entered hastily. A snake fell from above the door frame onto his shoulders. Terrified, he screamed. The monks came running and asked him why he was screaming. And he told them what had happened. 

The\marginnote{1.1.9} monks of few desires complained and criticized them, “How can the newly-arrived monks enter the monastery wearing sandals, holding sunshades, with their heads covered, with their robe on their head; and wash their feet with drinking water, not bow down to the resident monks who are senior to them, and not ask about dwellings?” They told the Buddha. Soon afterwards he had the Sangha gathered and questioned the monks: “Is it true, monks, that the newly-arrived monks are acting like this?” 

“It’s\marginnote{1.1.13} true, sir.” 

The\marginnote{1.1.14} Buddha rebuked them … “How can the newly-arrived monks act like this? This will affect people’s confidence …” After rebuking them … the Buddha gave a teaching and addressed the monks: 

“Well\marginnote{1.1.18} then, I’ll lay down the proper conduct for newly-arrived monks. When a newly-arrived monk enters a monastery, he should remove his sandals, hold them low, knock them together, and carry them along; he should lower his sunshade, uncover his head, and put his robe over his shoulders; he should then enter the monastery carefully and without hurry. As he enters the monastery, he should look out for where the resident monks gather—\footnote{Sp 4.357: \textit{\textsanskrit{Paṭikkamantīti} sannipatanti}, “\textit{\textsanskrit{Paṭikkamanti}}: (where) they gather.” } whether in the assembly hall, under a roof cover, or at the foot of a tree—and he should go there. He should then put down his bowl and robe, find a suitable seat, and sit down. He should ask which is the water for drinking and which the water for washing. If he needs water to drink, he may take some and drink. If he needs water to wash, he may take some and wash his feet. When he washes his feet, he should pour the water with one hand and wash with the other. He shouldn’t pour the water and wash his feet with the same hand. He should ask for a sandal-wiping cloth, and then wipe them. When he wipes his sandals, he should first wipe them with a dry cloth, then with a wet one. He should wash the cloth and spread it out.\footnote{Sp 4.357: \textit{Vissajjetabbanti \textsanskrit{pattharitabbaṁ}}, “\textit{Vissajjetabba}: to be spread out.” } 

If\marginnote{1.2.18} a resident monk is senior to him, the newly-arrived monk should bow down to him. If a resident monk is junior, he should bow down to the newly-arrived monk. The newly-arrived monk should ask which dwelling he may stay in and whether it’s occupied or not. He should ask about where to go for alms and where not to go, about any families designated as “in training”, about the place for defecating and the place for urinating, about the water for drinking and the water for washing, about walking sticks, and about the Sangha’s agreements\footnote{For \textit{sekkhasammata} see \href{https://suttacentral.net/pli-tv-bu-vb-pd3/en/brahmali\#3.15}{Bu Pd 3:3.15}. } concerning the right time to enter and the right time to leave. 

If\marginnote{1.3.1} the dwelling is unoccupied, he should knock on the door, wait for a moment, then lift the latch, open the door, and look inside while standing outside. 

If\marginnote{1.3.2} the dwelling is dirty, and if the beds or benches are stacked on top of one another with furniture piled on top, he should clean it if he’s able.\footnote{For an explanation of the rendering “furniture” for \textit{\textsanskrit{senāsana}}, see Appendix of Technical Terms. } When he’s cleaning the dwelling, he should first take out the floor cover and put it aside. He should take out the bed supports and put them aside. He should take out the mattress and the pillow and put them aside. He should take out the sitting mat and the sheet and put them aside.\footnote{For an explanation of the rendering “sitting mat” for \textit{\textsanskrit{nisīdana}}, see Appendix of Technical Terms. } Holding the bed low, he should carefully take it out without scratching it or knocking it against the door or the door frame, and he should put it aside. Holding the bench low, he should carefully take it out without scratching it or knocking it against the door or the door frame, and he should put it aside. He should take out the spittoon and put it aside. He should take out the leaning board and put it aside. If the dwelling has cobwebs, he should first remove them from the ceiling cloth, and he should then wipe the windows and the corners of the room. If the walls have been treated with red ocher and they’re moldy, he should moisten a cloth, wring it out, and wipe the walls. If the floor has been treated with a black finish and it’s moldy, he should moisten a cloth, wring it out, and wipe the floor. If the floor is untreated, he should sprinkle it with water and then sweep it, trying to avoid stirring up dust. He should look out for any trash and discard it. 

He\marginnote{1.4.1} should sun the floor cover, clean it, beat it, bring it back inside, and put it back as before. He should sun the bed supports, wipe them, bring them back inside, and put them back where they were. He should sun the bed, clean it, and beat it. Holding it low, he should carefully bring it back inside without scratching it or knocking it against the door or the door frame, and he should put it back as before. He should sun the bench, clean it, and beat it. Holding it low, he should carefully bring it back inside without scratching it or knocking it against the door or the door frame, and he should put it back as before. He should sun the mattress and the pillow, clean them, beat them, bring them back inside, and put them back as before. He should sun the sitting mat and the sheet, clean them, beat them, bring them back inside, and put them back as before. He should sun the spittoon, wipe it, bring it back inside, and put it back where it was. He should sun the leaning board, wipe it, bring it back inside, and put it back where it was. He should put away the bowl and robe. When putting away the bowl, he should hold the bowl in one hand, feel under the bed or the bench with the other, and then put it away. He shouldn’t put the bowl away on the bare floor. When putting away the robe, he should hold the robe in one hand, wipe the bamboo robe rack or the clothesline with the other, and then put it away by folding the robe over it, making the ends face the wall and the fold face out. 

If\marginnote{1.5.5} dusty winds are blowing from the east, he should close the windows on the eastern side. If dusty winds are blowing from the west, he should close the windows on the western side. If dusty winds are blowing from the north, he should close the windows on the northern side. If dusty winds are blowing from the south, he should close the windows on the southern side. If the weather is cold, he should open the windows during the day and close them at night. If the weather is hot, he should close the windows during the day and open them at night. 

If\marginnote{1.5.11} the yard is dirty, he should sweep it.\footnote{For an explanation of the rendering “yard” for \textit{\textsanskrit{pariveṇa}}, see Appendix of Technical Terms. } If the gatehouse is dirty, he should sweep it.\footnote{For the rendering “gatehouse” for \textit{\textsanskrit{koṭṭhaka}}, see Appendix of Technical Terms. } If the assembly hall is dirty, he should sweep it. If the water-boiling shed is dirty, he should sweep it.\footnote{For \textit{\textsanskrit{aggisālā}} as “water-boiling shed”, see Appendix of Technical Terms. } If the restroom is dirty, he should sweep it. If there’s no water for drinking, he should get some. If there’s no water for washing, he should get some. If there’s no water in the restroom ablutions pot, he should fill it. 

This\marginnote{1.5.19} is the proper conduct for newly-arrived monks.” 

\section*{2. Discussion of the proper conduct for resident monks }

At\marginnote{2.1.1} that time, when they saw newly-arrived monks, the resident monks did not prepare seats, or put out foot stools, foot scrapers, or water for washing the feet. They did not go out to meet them to receive their bowls and robes, or ask if they wanted water to drink. They did not bow down to newly-arrived monks who were senior to them or assign dwellings to them. 

The\marginnote{2.1.2} monks of few desires complained and criticized them, “How can the resident monks act like this?” They told the Buddha. Soon afterwards he had the Sangha gathered and questioned the monks: “Is it true, monks, that the resident monks are acting like this?” “It’s true, sir.” … After rebuking them … the Buddha gave a teaching and addressed the monks: 

“Well\marginnote{2.1.9} then, I’ll lay down the proper conduct for resident monks. When a resident monk sees a newly-arrived monk who is senior to him, he should prepare a seat, and put out a foot stool, a foot scraper, and water for washing the feet. He should go out to meet him to receive his bowl and robe, and ask if he wants water to drink. If he’s able, he should wipe his sandals, first with a dry cloth and then with a wet one. He should wash the cloth and spread it out. 

If\marginnote{2.2.5} the newly-arrived monk is senior to him, the resident monk should bow down to him. He should assign him a dwelling, tell him where it is, and inform him whether it’s occupied or not. He should tell him where to go for alms and where not to go, and about any families designated as ‘in training’. He should point out the place for defecating and the place for urinating, the water for drinking and the water for washing, and the walking sticks. He should tell him about the Sangha’s agreements concerning the right time to enter and the right time to leave. 

If\marginnote{2.3.1} the newly-arrived monk is junior to him, the resident monk should remain seated while telling him where to put his bowl and robe, and which seat to sit on. He should point out the water for drinking and the water for washing, as well as a sandal-wiping cloth. 

If\marginnote{2.3.6} the newly-arrived monk is junior to the resident monk, he should bow down to him. The resident monk should tell him which dwelling he may stay in and whether it’s occupied or not. He should tell him where to go for alms and where not to go, and about any families designated as ‘in training’. He should point out the place for defecating and the place for urinating, the water for drinking and the water for washing, and the walking sticks. He should tell him about the Sangha’s agreements concerning the right time to enter and the right time to leave. 

This\marginnote{2.3.20} is the proper conduct for resident monks.” 

\section*{3. Discussion of the proper conduct for departing monks }

At\marginnote{3.1.1} that time there were monks who departed without putting the wooden and ceramic goods in order, without closing the door and the windows, and without informing anyone. The wooden and ceramic goods were lost and the dwelling was unprotected. 

The\marginnote{3.1.4} monks of few desires complained and criticized them, “How can the departing monks act like this?” They told the Buddha. Soon afterwards he had the Sangha gathered and questioned the monks: “Is it true, monks, that the departing monks are acting like this?” “It’s true, sir.” … After rebuking them … the Buddha gave a teaching and addressed the monks: 

“Well\marginnote{3.1.13} then, I’ll lay down the proper conduct for departing monks. Before a monk departs, he should put the wooden and ceramic goods in order, close the door and the windows, and inform someone. If there are no monks, he should inform a novice monk; if there are no novice monks, he should inform a monastery worker; if there are no monastery workers, he should inform a lay follower. If there are no monks, novice monks, monastery workers, or lay followers, he should place the bed on four rocks. He should then stack the beds and benches on top of one another, with the other furniture piled on top, and put away the wooden and ceramic goods. He should close the door and the windows, and then depart. 

If\marginnote{3.3.1} rain enters the dwelling, he should cover it if he’s able, or he should make an effort to have it covered. If this works out, all is well. If not, he should place the bed on four rocks in a dry spot. He should then stack the beds and benches on top of one another, with the other furniture piled on top, and put away the wooden and ceramic goods. He should close the door and the windows, and then depart. If the whole dwelling is getting wet, he should carry the furniture to the village if he’s able, or he should make an effort to have it carried to the village. If this works out, all is well. If not, he should place the bed on four rocks outside. He should then stack the beds and benches on top of one another, with the other furniture piled on top, and he should put away the wooden and ceramic goods. He should cover it all with grass and leaves and then depart, thinking, ‘Hopefully the requisites will be okay.’ 

This\marginnote{3.3.10} is the proper conduct for departing monks.” 

\section*{4. Discussion of the proper conduct in connection with the expression of appreciation }

At\marginnote{4.1.1} that time there were monks who did not express their appreciation in the dining hall.\footnote{For an explanation of the rendering “dining hall” for \textit{bhattagga}, see Appendix of Technical Terms. } People complained and criticized them, “How can the Sakyan monastics not express their appreciation in the dining hall?” The monks heard the complaints of those people and they told the Buddha. The Buddha gave a teaching and addressed the monks: 

\scrule{“You should express your appreciation in the dining hall.” }

The\marginnote{4.1.8} monks thought, “Who should give the expression of appreciation?” They told the Buddha. He gave a teaching and addressed the monks: 

\scrule{“The most senior monk should give the expression of appreciation in the dining hall.” }

Soon\marginnote{4.1.13} afterwards a certain association was offering a meal to the Sangha. Venerable \textsanskrit{Sāriputta} was the most senior monk. Because the Buddha had said the most senior monk should give the expression of appreciation, the other monks left, leaving \textsanskrit{Sāriputta} behind by himself. After giving the expression of appreciation, he left by himself. The Buddha saw him coming and asked, “Did the meal go well?” 

“The\marginnote{4.1.22} meal went well, sir, but the monks all left, leaving me behind by myself.” Soon afterwards the Buddha gave a teaching and addressed the monks: 

\scrule{“The four or five most senior monks should wait in the dining hall.” }

On\marginnote{4.1.26} one occasion a senior monk who needed to defecate was waiting in the dining hall. Being unable to hold out, he fainted and fell over. They told the Buddha. 

\scrule{“When there’s something to be done, I allow you to go after informing the monk next to you.” }

\section*{5. Discussion of the proper conduct in relation to dining halls }

At\marginnote{4.2.1} that time the monks from the group of six walked to the dining hall shabbily dressed and improper in appearance. Taking a short cut, they walked in front of the senior monks. They sat down encroaching on the senior monks and blocked the junior monks from getting a seat. And they spread out their upper robes and sat on them in inhabited areas.\footnote{For a discussion of the \textit{bhattagga}, see Appendix of Technical Terms. } 

The\marginnote{4.2.2} monks of few desires complained and criticized them, “How can the monks from the group of six act like this?” They told the Buddha. Soon afterwards he had the Sangha gathered and questioned the monks: “Is it true, monks, that the monks from the group of six are acting like this?” “It’s true, sir.” … After rebuking them … the Buddha gave a teaching and addressed the monks: 

“Well\marginnote{4.2.9} then, I’ll lay down the proper conduct in relation to dining halls. In a monastery where the time for departure gets announced, a monk should put on his sarong evenly all around, covering the navel and the knees. He should put on a belt. Putting the upper robes together, overlapping each other edge-to-edge, he should put them on and fasten the toggle. He should rinse his bowl, bring it along, and enter the village carefully and without hurry.\footnote{“Putting the upper robes together, overlapping each other edge-to-edge” renders \textit{\textsanskrit{saguṇaṁ} \textsanskrit{katvā} \textsanskrit{saṅghāṭiyo}}. Sp 3.66: \textit{\textsanskrit{Saguṇaṁ} \textsanskrit{katvāti} dve \textsanskrit{cīvarāni} ekato \textsanskrit{katvā}, \textsanskrit{tā} ekato \textsanskrit{katā} dvepi \textsanskrit{saṅghāṭiyo} \textsanskrit{dātabbā}. \textsanskrit{Sabbañhi} \textsanskrit{cīvaraṁ} \textsanskrit{saṅghaṭitattā} “\textsanskrit{saṅghāṭī}”ti vuccati}, “\textit{\textsanskrit{Saguṇaṁ} \textsanskrit{katvā}}: having made two robes into one, even those two upper robes made into one are to be given. All robes are called \textit{\textsanskrit{saṅghāṭi}} because of being pieced together.” For a further discussion of the meaning of \textit{\textsanskrit{saṅghāṭi}}, see Appendix of Technical Terms. } 

He\marginnote{4.3.2} shouldn’t take a short cut and walk in front of the senior monks. He should be well-covered while walking in inhabited areas;\footnote{This rule of conduct is the same as \href{https://suttacentral.net/pli-tv-bu-vb-sk3/en/brahmali\#0.5}{Bu Sk 3:0.5}. } he should be well-restrained while walking in inhabited areas;\footnote{= \href{https://suttacentral.net/pli-tv-bu-vb-sk5/en/brahmali\#0.5}{Bu Sk 5:0.5}. } he should lower his eyes while walking in inhabited areas;\footnote{= \href{https://suttacentral.net/pli-tv-bu-vb-sk7/en/brahmali\#0.5}{Bu Sk 7:0.5}. } he shouldn’t lift his robe while walking in inhabited areas;\footnote{= \href{https://suttacentral.net/pli-tv-bu-vb-sk9/en/brahmali\#0.5}{Bu Sk 9:0.5}. } he shouldn’t laugh loudly while walking in inhabited areas;\footnote{= \href{https://suttacentral.net/pli-tv-bu-vb-sk11/en/brahmali\#0.5}{Bu Sk 11:0.5}. } he shouldn’t be noisy while walking in inhabited areas;\footnote{= \href{https://suttacentral.net/pli-tv-bu-vb-sk13/en/brahmali\#0.5}{Bu Sk 13:0.5}. } he shouldn’t sway his body while walking in inhabited areas;\footnote{= \href{https://suttacentral.net/pli-tv-bu-vb-sk15/en/brahmali\#0.5}{Bu Sk 15:0.5}. } he shouldn’t swing his arms while walking in inhabited areas;\footnote{= \href{https://suttacentral.net/pli-tv-bu-vb-sk17/en/brahmali\#0.5}{Bu Sk 17:0.5}. } he shouldn’t sway his head while walking in inhabited areas;\footnote{= \href{https://suttacentral.net/pli-tv-bu-vb-sk19/en/brahmali\#0.5}{Bu Sk 19:0.5}. } he shouldn’t have his hands on his hips while walking in inhabited areas;\footnote{= \href{https://suttacentral.net/pli-tv-bu-vb-sk21/en/brahmali\#0.5}{Bu Sk 21:0.5}. } he shouldn’t cover his head while walking in inhabited areas;\footnote{= \href{https://suttacentral.net/pli-tv-bu-vb-sk23/en/brahmali\#0.5}{Bu Sk 23:0.5}. } he shouldn’t move about while squatting on his heels in inhabited areas.\footnote{= \href{https://suttacentral.net/pli-tv-bu-vb-sk25/en/brahmali\#0.5}{Bu Sk 25:0.5}. } 

He\marginnote{4.3.15} should be well-covered while sitting in inhabited areas;\footnote{= \href{https://suttacentral.net/pli-tv-bu-vb-sk4/en/brahmali\#0.5}{Bu Sk 4:0.5}. } he should be well-restrained while sitting in inhabited areas;\footnote{= \href{https://suttacentral.net/pli-tv-bu-vb-sk6/en/brahmali\#0.5}{Bu Sk 6:0.5}. } he should lower his eyes while sitting in inhabited areas;\footnote{= \href{https://suttacentral.net/pli-tv-bu-vb-sk8/en/brahmali\#0.5}{Bu Sk 8:0.5}. } he shouldn’t lift his robe while sitting in inhabited areas;\footnote{= \href{https://suttacentral.net/pli-tv-bu-vb-sk10/en/brahmali\#0.5}{Bu Sk 10:0.5}. }  he shouldn’t laugh loudly while sitting in inhabited areas;\footnote{= \href{https://suttacentral.net/pli-tv-bu-vb-sk12/en/brahmali\#0.5}{Bu Sk 12:0.5}. }  he shouldn’t be noisy while sitting in inhabited areas;\footnote{= \href{https://suttacentral.net/pli-tv-bu-vb-sk14/en/brahmali\#0.5}{Bu Sk 14:0.5}. } he shouldn’t sway his body while sitting in inhabited areas;\footnote{= \href{https://suttacentral.net/pli-tv-bu-vb-sk16/en/brahmali\#0.5}{Bu Sk 16:0.5}. } he shouldn’t swing his arms while sitting in inhabited areas;\footnote{= \href{https://suttacentral.net/pli-tv-bu-vb-sk18/en/brahmali\#0.5}{Bu Sk 18:0.5}. } he shouldn’t sway his head while sitting in inhabited areas;\footnote{= \href{https://suttacentral.net/pli-tv-bu-vb-sk20/en/brahmali\#0.5}{Bu Sk 20:0.5}. } he shouldn’t have his hands on his hips while sitting in inhabited areas;\footnote{= \href{https://suttacentral.net/pli-tv-bu-vb-sk22/en/brahmali\#0.5}{Bu Sk 22:0.5}. } he shouldn’t cover his head while sitting in inhabited areas;\footnote{= \href{https://suttacentral.net/pli-tv-bu-vb-sk24/en/brahmali\#0.5}{Bu Sk 24:0.5}. } he shouldn’t clasp his knees while sitting in inhabited areas.\footnote{= \href{https://suttacentral.net/pli-tv-bu-vb-sk26/en/brahmali\#0.5}{Bu Sk 26:0.5}. } He shouldn’t sit encroaching on the senior monks or block the junior monks from getting a seat. He shouldn’t spread out his upper robe and sit on it in inhabited areas. 

When\marginnote{4.4.1} given water, he should receive it while holding his bowl with both hands. Holding the bowl low, he should rinse it carefully without scratching it. If there’s someone to receive the water, he should hold his bowl low and pour the water into the container, trying to avoid splashing the person receiving the water, the monks sitting nearby, and his upper robe. If there’s no-one to receive the water, he should hold his bowl low and pour the water on the ground, trying to avoid splashing the monks sitting nearby and his upper robe. 

When\marginnote{4.4.7} given rice, he should receive it while holding his bowl with both hands, leaving room for the curry. If there’s ghee, oil, or special curry, the senior monk should say, ‘Everyone gets an equal share.’ He should receive the almsfood respectfully;\footnote{This rule of conduct is the same as \href{https://suttacentral.net/pli-tv-bu-vb-sk27/en/brahmali\#0.5}{Bu Sk 27:0.5}. } he should receive the almsfood with attention on the bowl;\footnote{= \href{https://suttacentral.net/pli-tv-bu-vb-sk28/en/brahmali\#0.5}{Bu Sk 28:0.5}. } he should receive the almsfood with the right proportion of bean curry;\footnote{= \href{https://suttacentral.net/pli-tv-bu-vb-sk29/en/brahmali\#0.5}{Bu Sk 29:0.5}. } he should receive an even level of almsfood.\footnote{= \href{https://suttacentral.net/pli-tv-bu-vb-sk30/en/brahmali\#0.5}{Bu Sk 30:0.5}. } 

The\marginnote{4.4.14} senior monk shouldn’t start eating until everyone has received rice. A monk should eat almsfood respectfully;\footnote{= \href{https://suttacentral.net/pli-tv-bu-vb-sk31/en/brahmali\#0.5}{Bu Sk 31:0.5}. } he should eat almsfood with attention on the bowl;\footnote{= \href{https://suttacentral.net/pli-tv-bu-vb-sk32/en/brahmali\#0.5}{Bu Sk 32:0.5}. } he should eat almsfood in order;\footnote{= \href{https://suttacentral.net/pli-tv-bu-vb-sk33/en/brahmali\#0.5}{Bu Sk 33:0.5}. } he should eat almsfood with the right proportion of bean curry;\footnote{= \href{https://suttacentral.net/pli-tv-bu-vb-sk34/en/brahmali\#0.5}{Bu Sk 34:0.5}. } he shouldn’t eat almsfood after making a heap;\footnote{= \href{https://suttacentral.net/pli-tv-bu-vb-sk35/en/brahmali\#0.5}{Bu Sk 35:0.5}. } he shouldn’t cover his curries with rice because he wants more;\footnote{= \href{https://suttacentral.net/pli-tv-bu-vb-sk36/en/brahmali\#0.5}{Bu Sk 36:0.5}. } when not sick, he shouldn’t request bean curry and rice for himself and then eat it;\footnote{= \href{https://suttacentral.net/pli-tv-bu-vb-sk37/en/brahmali\#0.5}{Bu Sk 37:0.5}. } he shouldn’t look at another’s almsbowl finding fault;\footnote{= \href{https://suttacentral.net/pli-tv-bu-vb-sk38/en/brahmali\#0.5}{Bu Sk 38:0.5}. } he shouldn’t make mouthfuls that are too large;\footnote{= \href{https://suttacentral.net/pli-tv-bu-vb-sk39/en/brahmali\#0.5}{Bu Sk 39:0.5}. } he should make rounded mouthfuls;\footnote{= \href{https://suttacentral.net/pli-tv-bu-vb-sk40/en/brahmali\#0.5}{Bu Sk 40:0.5}. } he shouldn’t open his mouth without bringing a mouthful to it;\footnote{= \href{https://suttacentral.net/pli-tv-bu-vb-sk41/en/brahmali\#0.5}{Bu Sk 41:0.5}. } he shouldn’t put his whole hand in his mouth while eating;\footnote{= \href{https://suttacentral.net/pli-tv-bu-vb-sk42/en/brahmali\#0.5}{Bu Sk 42:0.5}. } he shouldn’t speak with food in his mouth;\footnote{= \href{https://suttacentral.net/pli-tv-bu-vb-sk43/en/brahmali\#0.5}{Bu Sk 43:0.5}. } he shouldn’t eat from a lifted ball of food;\footnote{= \href{https://suttacentral.net/pli-tv-bu-vb-sk44/en/brahmali\#0.5}{Bu Sk 44:0.5}. } he shouldn’t eat breaking up mouthfuls;\footnote{= \href{https://suttacentral.net/pli-tv-bu-vb-sk45/en/brahmali\#0.5}{Bu Sk 45:0.5}. } he shouldn’t eat stuffing his cheeks;\footnote{= \href{https://suttacentral.net/pli-tv-bu-vb-sk46/en/brahmali\#0.5}{Bu Sk 46:0.5}. } he shouldn’t eat shaking his hand;\footnote{= \href{https://suttacentral.net/pli-tv-bu-vb-sk47/en/brahmali\#0.5}{Bu Sk 47:0.5}. } he shouldn’t eat scattering rice;\footnote{= \href{https://suttacentral.net/pli-tv-bu-vb-sk48/en/brahmali\#0.5}{Bu Sk 48:0.5}. } he shouldn’t eat sticking out his tongue;\footnote{= \href{https://suttacentral.net/pli-tv-bu-vb-sk49/en/brahmali\#0.5}{Bu Sk 49:0.5}. } he shouldn’t make a chomping sound while eating;\footnote{= \href{https://suttacentral.net/pli-tv-bu-vb-sk50/en/brahmali\#0.5}{Bu Sk 50:0.5}. } he shouldn’t slurp while eating;\footnote{= \href{https://suttacentral.net/pli-tv-bu-vb-sk51/en/brahmali\#0.5}{Bu Sk 51:0.5}. } he shouldn’t lick his hands while eating;\footnote{= \href{https://suttacentral.net/pli-tv-bu-vb-sk52/en/brahmali\#0.5}{Bu Sk 52:0.5}. } he shouldn’t lick his almsbowl while eating;\footnote{= \href{https://suttacentral.net/pli-tv-bu-vb-sk53/en/brahmali\#0.5}{Bu Sk 53:0.5}. } he shouldn’t lick his lips while eating;\footnote{= \href{https://suttacentral.net/pli-tv-bu-vb-sk54/en/brahmali\#0.5}{Bu Sk 54:0.5}. } he shouldn’t receive the drinking-water vessel with a hand soiled with food.\footnote{= \href{https://suttacentral.net/pli-tv-bu-vb-sk55/en/brahmali\#0.5}{Bu Sk 55:0.5}. } 

The\marginnote{4.6.1} senior monk shouldn’t receive water for washing until all the monks are finished eating. When given water, a monk should receive it while holding his bowl with both hands. Holding the bowl low, he should rinse it carefully without scratching it. If there’s someone to receive the water, he should hold his bowl low and pour the water into the container, trying to avoid splashing the person receiving the water, the monks sitting nearby, and his upper robe. If there’s no-one to receive the water, he should hold his bowl low and pour the water on the ground, trying to avoid splashing the monks sitting nearby and his upper robe. He shouldn’t discard bowl-washing water containing rice in inhabited areas.\footnote{= \href{https://suttacentral.net/pli-tv-bu-vb-sk56/en/brahmali\#0.5}{Bu Sk 56:0.5}. } 

When\marginnote{4.6.9} returning, the junior monks should go first and then the senior monks. A monk should be well-covered while walking in inhabited areas;\footnote{= \href{https://suttacentral.net/pli-tv-bu-vb-sk3/en/brahmali\#0.5}{Bu Sk 3:0.5}. } he should be well-restrained while walking in inhabited areas;\footnote{= \href{https://suttacentral.net/pli-tv-bu-vb-sk5/en/brahmali\#0.5}{Bu Sk 5:0.5}. } he should lower his eyes while walking in inhabited areas;\footnote{= \href{https://suttacentral.net/pli-tv-bu-vb-sk7/en/brahmali\#0.5}{Bu Sk 7:0.5}. } he shouldn’t lift his robe while walking in inhabited areas;\footnote{= \href{https://suttacentral.net/pli-tv-bu-vb-sk9/en/brahmali\#0.5}{Bu Sk 9:0.5}. } he shouldn’t laugh loudly while walking in inhabited areas;\footnote{= \href{https://suttacentral.net/pli-tv-bu-vb-sk11/en/brahmali\#0.5}{Bu Sk 11:0.5}. } he shouldn’t be noisy while walking in inhabited areas;\footnote{= \href{https://suttacentral.net/pli-tv-bu-vb-sk13/en/brahmali\#0.5}{Bu Sk 13:0.5}. } he shouldn’t sway his body while walking in inhabited areas;\footnote{= \href{https://suttacentral.net/pli-tv-bu-vb-sk15/en/brahmali\#0.5}{Bu Sk 15:0.5}. } he shouldn’t swing his arms while walking in inhabited areas;\footnote{= \href{https://suttacentral.net/pli-tv-bu-vb-sk17/en/brahmali\#0.5}{Bu Sk 17:0.5}. } he shouldn’t sway his head while walking in inhabited areas;\footnote{= \href{https://suttacentral.net/pli-tv-bu-vb-sk19/en/brahmali\#0.5}{Bu Sk 19:0.5}. } he shouldn’t have his hands on his hips while walking in inhabited areas;\footnote{= \href{https://suttacentral.net/pli-tv-bu-vb-sk21/en/brahmali\#0.5}{Bu Sk 21:0.5}. } he shouldn’t cover his head while walking in inhabited areas;\footnote{= \href{https://suttacentral.net/pli-tv-bu-vb-sk23/en/brahmali\#0.5}{Bu Sk 23:0.5}. } he shouldn’t move about while squatting on his heels in inhabited areas.\footnote{= \href{https://suttacentral.net/pli-tv-bu-vb-sk25/en/brahmali\#0.5}{Bu Sk 25:0.5}. } 

This\marginnote{4.6.22} is the proper conduct in relation to dining halls.” 

\scend{The first section for recitation is finished. }

\section*{6. Discussion of the proper conduct for alms collectors }

At\marginnote{5.1.1} that time there were alms-collecting monks who were shabbily dressed and improper in appearance. They entered and left houses without being attentive, entered and left too hastily, stood too far away or too close, and waited too long or left too soon. 

On\marginnote{5.1.2} one occasion a certain monk entered a house without being attentive. Thinking it was the main door, he entered a room where a woman was lying naked. When he saw her, he left the room. But when her husband saw her there, he thought, “My wife has been raped by this monk,” and he took hold of that monk and gave him a beating. The woman woke up from the commotion and asked her husband, “Why are you beating this monk?” 

“Didn’t\marginnote{5.1.13} he rape you?” 

“No\marginnote{5.1.14} he didn’t. He’s innocent.” And she had him release that monk. The monk then returned to the monastery and told the monks what had happened. 

The\marginnote{5.1.17} monks of few desires complained and criticized him, “How can the alms-collecting monks act like this?” They told the Buddha. Soon afterwards he had the Sangha gathered and questioned the monks: “Is it true, monks, that the alms-collecting monks are acting like this?” “It’s true, sir.” … After rebuking them … the Buddha gave a teaching and addressed the monks: 

“Well\marginnote{5.1.24} then, I’ll lay down the proper conduct for alms-collecting monks. When an alms-collecting monk is about to enter the village, he should put on his sarong evenly all around, covering the navel and the knees. He should put on a belt. Putting the upper robes together, overlapping each other edge-to-edge, he should put them on and fasten the toggle. He should rinse his bowl, bring it along, and enter the village carefully and without hurry. 

He\marginnote{5.2.3} should be well-covered while walking in an inhabited areas;\footnote{= \href{https://suttacentral.net/pli-tv-bu-vb-sk3/en/brahmali\#0.5}{Bu Sk 3:0.5}. } he should be well-restrained while walking in inhabited areas;\footnote{= \href{https://suttacentral.net/pli-tv-bu-vb-sk5/en/brahmali\#0.5}{Bu Sk 5:0.5}. } he should lower his eyes while walking in inhabited areas;\footnote{= \href{https://suttacentral.net/pli-tv-bu-vb-sk7/en/brahmali\#0.5}{Bu Sk 7:0.5}. } he shouldn’t lift his robe while walking in inhabited areas;\footnote{= \href{https://suttacentral.net/pli-tv-bu-vb-sk9/en/brahmali\#0.5}{Bu Sk 9:0.5}. } he shouldn’t laugh loudly while walking in inhabited areas;\footnote{= \href{https://suttacentral.net/pli-tv-bu-vb-sk11/en/brahmali\#0.5}{Bu Sk 11:0.5}. } he shouldn’t be noisy while walking in inhabited areas;\footnote{= \href{https://suttacentral.net/pli-tv-bu-vb-sk13/en/brahmali\#0.5}{Bu Sk 13:0.5}. } he shouldn’t sway his body while walking in inhabited areas;\footnote{= \href{https://suttacentral.net/pli-tv-bu-vb-sk15/en/brahmali\#0.5}{Bu Sk 15:0.5}. } he shouldn’t swing his arms while walking in inhabited areas;\footnote{= \href{https://suttacentral.net/pli-tv-bu-vb-sk17/en/brahmali\#0.5}{Bu Sk 17:0.5}. } he shouldn’t sway his head while walking in inhabited areas;\footnote{= \href{https://suttacentral.net/pli-tv-bu-vb-sk19/en/brahmali\#0.5}{Bu Sk 19:0.5}. } he shouldn’t have his hands on his hips while walking in inhabited areas;\footnote{= \href{https://suttacentral.net/pli-tv-bu-vb-sk21/en/brahmali\#0.5}{Bu Sk 21:0.5}. } he shouldn’t cover his head while walking in inhabited areas;\footnote{= \href{https://suttacentral.net/pli-tv-bu-vb-sk23/en/brahmali\#0.5}{Bu Sk 23:0.5}. } he shouldn’t move about while squatting on his heels in inhabited areas.\footnote{= \href{https://suttacentral.net/pli-tv-bu-vb-sk25/en/brahmali\#0.5}{Bu Sk 25:0.5}. } 

When\marginnote{5.2.15} entering a house, he should be attentive to where to enter and where to leave. He shouldn’t enter or leave too hastily, stand too far away or too close, or wait too long or leave too soon. While waiting, he should be attentive to whether they wish to give alms or not. If they put down their work, get up from their seat, take hold of a serving spoon or a vessel, or they tell him to wait, then he should assume they wish to give, and he should wait.\footnote{“Tell him to wait” renders \textit{\textsanskrit{ṭhapeti}}. Sp-\textsanskrit{ṭ} 4.366: \textit{Ṭhapeti \textsanskrit{vāti} “\textsanskrit{tiṭṭhatha}, bhante”ti \textsanskrit{vadantī} \textsanskrit{ṭhapeti} \textsanskrit{nāma}}, “\textit{Ṭhapeti \textsanskrit{vā}}: saying, ‘Wait, venerable sir’, is called \textit{\textsanskrit{ṭhapeti} \textsanskrit{vā}}.” } When they give him almsfood, he should lift his upper robe with his left hand, stretch out his bowl with his right hand, and receive the alms while holding the bowl with both hands. He shouldn’t look the donor in the face. He should be attentive to whether they wish to give curry or not. If they take hold of a serving spoon or a vessel, or they tell him to wait, then he should assume they wish to give, and he should wait. When they have given alms, he should cover the bowl with his upper robe, and leave carefully and without hurry. 

He\marginnote{5.2.33} should be well-covered while walking in inhabited areas;\footnote{= \href{https://suttacentral.net/pli-tv-bu-vb-sk3/en/brahmali\#0.5}{Bu Sk 3:0.5}. } he should be well-restrained while walking in inhabited areas;\footnote{= \href{https://suttacentral.net/pli-tv-bu-vb-sk5/en/brahmali\#0.5}{Bu Sk 5:0.5}. } he should lower his eyes while walking in inhabited areas;\footnote{= \href{https://suttacentral.net/pli-tv-bu-vb-sk7/en/brahmali\#0.5}{Bu Sk 7:0.5}. } he shouldn’t lift his robe while walking in inhabited areas;\footnote{= \href{https://suttacentral.net/pli-tv-bu-vb-sk9/en/brahmali\#0.5}{Bu Sk 9:0.5}. } he shouldn’t laugh loudly while walking in inhabited areas;\footnote{= \href{https://suttacentral.net/pli-tv-bu-vb-sk11/en/brahmali\#0.5}{Bu Sk 11:0.5}. } he shouldn’t be noisy while walking in inhabited areas;\footnote{= \href{https://suttacentral.net/pli-tv-bu-vb-sk13/en/brahmali\#0.5}{Bu Sk 13:0.5}. } he shouldn’t sway his body while walking in inhabited areas;\footnote{= \href{https://suttacentral.net/pli-tv-bu-vb-sk15/en/brahmali\#0.5}{Bu Sk 15:0.5}. } he shouldn’t swing his arms while walking in inhabited areas;\footnote{= \href{https://suttacentral.net/pli-tv-bu-vb-sk17/en/brahmali\#0.5}{Bu Sk 17:0.5}. } he shouldn’t sway his head while walking in inhabited areas;\footnote{= \href{https://suttacentral.net/pli-tv-bu-vb-sk19/en/brahmali\#0.5}{Bu Sk 19:0.5}. } he shouldn’t have his hands on his hips while walking in inhabited areas;\footnote{= \href{https://suttacentral.net/pli-tv-bu-vb-sk21/en/brahmali\#0.5}{Bu Sk 21:0.5}. } he shouldn’t cover his head while walking in inhabited areas;\footnote{= \href{https://suttacentral.net/pli-tv-bu-vb-sk23/en/brahmali\#0.5}{Bu Sk 23:0.5}. } he shouldn’t move about while squatting on his heels in inhabited areas.\footnote{= \href{https://suttacentral.net/pli-tv-bu-vb-sk25/en/brahmali\#0.5}{Bu Sk 25:0.5}. } 

Whoever\marginnote{5.3.1} returns first from almsround in the village should prepare the seats and set out a foot stool, a foot scraper, and water for washing the feet. He should wash the bowl for leftovers and put it back out, and set out water for drinking and water for washing. 

Whoever\marginnote{5.3.2} returns last from almsround may eat whatever is left over, or he should discard it where there are no cultivated plants or in water without life.\footnote{\textit{Harita} could in principle refer to all plants, but it is elsewhere defined as what is cultivated, see \href{https://suttacentral.net/pli-tv-bu-vb-pc19/en/brahmali\#2.1.14}{Bu Pc 19:2.1.14} and \href{https://suttacentral.net/pli-tv-bi-vb-pc9/en/brahmali\#2.1.14}{Bi Pc 9:2.1.14}. } He should put away the seats and also the foot stool, the foot scraper, and the water for washing the feet. He should wash the bowl for leftovers and put it away, put away the water for drinking and the water for washing, and sweep the dining hall. 

Whoever\marginnote{5.3.5} sees that the pot for drinking water, the pot for washing water, or the restroom pot is empty should fill it. If he can’t do it by himself, he should call someone over by hand signal and they should fill it together. He shouldn’t speak because of that. 

This\marginnote{5.3.7} is the proper conduct for alms collectors.” 

\section*{7. Discussion of the proper conduct for those staying in the wilderness }

At\marginnote{6.1.1} that time there was a number of monks staying in the wilderness. They did not set out water for drinking or water for washing, did not light fires or provide fire-making implements, and did not know the constellations or the regions.\footnote{Vin-\textsanskrit{ālaṅ}-\textsanskrit{ṭ} 27.190: \textit{Aggi \textsanskrit{upaṭṭhāpetabbotiādi} \textsanskrit{vāḷamigasarīsapādibāhiraparissayakāle} ca \textsanskrit{vātapittādiajjhattapaassayakāle} ca \textsanskrit{icchitabbattā}}, “\textit{Aggi \textsanskrit{upaṭṭhāpetabbo}}, etc., means what is required at a time of danger outside from wild animals, snakes, and centipedes, and at a time of refuge indoors due to wind, bile, and other illnesses.” The point seems to be that they should have access to fire. “Constellation” here refers to the constellation the moon is “stationed in” at any particular time, sometimes known as a “lunar mansion”, a “lunar house”, or a “lunar station.” As the earth orbits the sun, the moon appears to be moving from one constellation to the next. By observing the movement of the moon, one can determine what is the current month or fortnight. Regarding the translation “region” for \textit{\textsanskrit{disā}}, see below. } 

Criminals\marginnote{6.1.3} went to that place and asked the monks, “Sir, is there any water for drinking?” —“No, there isn’t.” —“Is there any water for washing?” —“No.” —“Is there any fire?” —“No.” —“Are there any fire-making implements?” —“No.” —“Which constellation is the moon in today?” —\footnote{Sp 4.367: \textit{Kenajja, bhante, yuttanti kena nakkhattena ajja cando yuttoti}, “\textit{Kenajja, bhante, yutta}: With which constellation is the moon connected today?” } “We don’t know.” —“Which region is this?” —\footnote{Here the common rendering of “cardinal direction” for \textit{\textsanskrit{disā}} does not work. Consequently, I render \textit{\textsanskrit{disā}} as “region” throughout this section. } “We don’t know.” Thinking, “They’re not monks; they’re criminals,” they beat them up and left. 

The\marginnote{6.1.20} monks told other monks what had happened and they in turn told the Buddha. Soon afterwards the Buddha gave a teaching and addressed the monks: 

“Well\marginnote{6.1.23} then, I’ll lay down the proper conduct for monks staying in the wilderness. After getting up early in the morning, a monk who is staying in the wilderness should put his bowl in its bag, hang it from his shoulder, put his robe over his shoulders, put on his sandals, put the wooden and ceramic goods in order, close the door and windows, and come down from his dwelling. 

When\marginnote{6.2.2} he’s about to enter the village, he should remove his sandals, hold them low and knock them together. He should put them in a bag, which he should hang from his shoulder. He should put on his sarong evenly all around, covering the navel and the knees, and he should put on a belt. Putting the upper robes together, overlapping each other edge-to-edge, he should put them on and fasten the toggle. He should rinse his bowl, bring it along, and enter the village carefully and without hurry. 

He\marginnote{6.2.4} should be well-covered while walking in inhabited areas; he should be well-restrained while walking in inhabited areas; he should lower his eyes while walking in inhabited areas; he shouldn’t lift his robe while walking in inhabited areas; he shouldn’t laugh loudly while walking in inhabited areas; he shouldn’t be noisy while walking in inhabited areas; he shouldn’t sway his body while walking in inhabited areas; he shouldn’t swing his arms while walking in inhabited areas; he shouldn’t sway his head while walking in inhabited areas; he shouldn’t have his hands on his hips while walking in inhabited areas; he shouldn’t cover his head while walking in inhabited areas; he shouldn’t move about while squatting on his heels in inhabited areas. 

When\marginnote{6.2.8} entering a house, he should be attentive to where to enter and where to leave. He shouldn’t enter or leave too hastily, stand too far away or too close, or wait too long or leave too soon. While waiting, he should be attentive to whether they wish to give alms or not. If they put down their work, get up from their seat, take hold of a serving spoon or a vessel, or they tell him to wait, then he should assume they wish to give, and he should wait. When they give him almsfood, he should lift his upper robe with his left hand, stretch out his bowl with his right hand, and receive the alms while holding the bowl with both hands. He shouldn’t look the donor in the face. He should be attentive to whether they wish to give curry or not. If they take hold of a serving spoon or a vessel, or they tell him to wait, then he should assume they wish to give, and he should wait. When they have given alms, he should cover the bowl with his upper robe, and return carefully and without hurry. 

He\marginnote{6.2.27} should be well-covered while walking in inhabited areas; he should be well-restrained while walking in inhabited areas; he should lower his eyes while walking in inhabited areas; he shouldn’t lift his robe while walking in inhabited areas; he shouldn’t laugh loudly while walking in inhabited areas; he shouldn’t be noisy while walking in inhabited areas; he shouldn’t sway his body while walking in inhabited areas; he shouldn’t swing his arms while walking in inhabited areas; he shouldn’t sway his head while walking in inhabited areas; he shouldn’t have his hands on his hips while walking in inhabited areas; he shouldn’t cover his head while walking in inhabited areas; he shouldn’t move about while squatting on his heels in inhabited areas. When he has left the village, he should put his bowl in its bag and hang it from his shoulder, fold up his robe and put it on his head, and put on his sandals and go. 

A\marginnote{6.3.2} monk who is staying in the wilderness should set out water for drinking and water for washing, should light a fire and provide fire-making implements, should provide a walking stick, and should learn the constellations—either all of them or a portion—and become skilled in the regions. 

This\marginnote{6.3.4} is the proper conduct for those staying in the wilderness.” 

\section*{8. Discussion of the proper conduct in regard to dwellings }

On\marginnote{7.1.1} one occasion when a number of monks were making robes outside, the monks from the group of six were beating furniture in an open space upwind from them. The robe-making monks became dusty. 

The\marginnote{7.1.4} monks of few desires complained and criticized them, “How can the monks from the group of six do this?” 

They\marginnote{7.1.7} told the Buddha. Soon afterwards he had the Sangha gathered and questioned the monks: “Is it true, monks, that the monks from the group of six did this?” “It’s true, sir.” … After rebuking them … the Buddha gave a teaching and addressed the monks: 

“Well\marginnote{7.1.12} then, I’ll lay down the proper conduct in regard to dwellings. If the dwelling he’s staying in is dirty, a monk should clean it if he’s able. When he’s cleaning the dwelling, he should first take out the bowl and robe and put them aside. He should take out the sitting mat and the sheet and put them aside. He should take out the mattress and the pillow and put them aside. Holding the bed low, he should carefully take it out without scratching it or knocking it against the door or the door frame, and he should put it aside. Holding the bench low, he should carefully take it out without scratching it or knocking it against the door or the door frame, and he should put it aside. He should take out the bed supports and put them aside. He should take out the spittoon and put it aside. He should take out the leaning board and put it aside. After taking note of its position, he should take out the floor cover and put it aside. If the dwelling has cobwebs, he should first remove them from the ceiling cloth, and he should then wipe the windows and the corners of the room. If the walls have been treated with red ocher and they’re moldy, he should moisten a cloth, wring it out, and wipe the walls. If the floor has been treated with a black finish and it’s moldy, he should moisten a cloth, wring it out, and wipe the floor. If the floor is untreated, he should sprinkle it with water and then sweep it, trying to avoid stirring up dust. He should look out for any trash and discard it. 

He\marginnote{7.2.17} shouldn’t beat the furniture near other monks, near other dwellings, near water for drinking, or near water for washing. He shouldn’t beat the furniture in an open area upwind from these things, but downwind from them. 

He\marginnote{7.3.1} should sun the floor cover, clean it, beat it, bring it back inside, and put it back as before. He should sun the bed supports, wipe them, bring them back inside, and put them back where they were. He should sun the bed, clean it, and beat it. Holding it low, he should carefully bring it back inside without scratching it or knocking it against the door or the door frame, and he should put it back as before. He should sun the bench, clean it, and beat it. Holding it low, he should carefully bring it back inside without scratching it or knocking it against the door or the door frame, and he should put it back as before. He should sun the mattress and the pillow, clean them, beat them, bring them back inside, and put them back as before. He should sun the sitting mat and the sheet, clean them, beat them, bring them back inside, and put them back as before. He should sun the spittoon, wipe it, bring it back inside, and put it back where it was. He should sun the leaning board, wipe it, bring it back inside, and put it back where it was. He should put away the bowl and robe. When putting away the bowl, he should hold the bowl in one hand, feel under the bed or the bench with the other, and then put it away. He shouldn’t put the bowl away on the bare floor. When putting away the robe, he should hold the robe in one hand, wipe the bamboo robe rack or the clothesline with the other, and then put it away by folding the robe over it, making the ends face the wall and the fold face out. 

If\marginnote{7.4.1} dusty winds are blowing from the east, he should close the windows on the eastern side. If dusty winds are blowing from the west, he should close the windows on the western side. If dusty winds are blowing from the north, he should close the windows on the northern side. If dusty winds are blowing from the south, he should close the windows on the southern side. If the weather is cold, he should open the windows during the day and close them at night. If the weather is hot, he should close the windows during the day and open them at night. 

If\marginnote{7.4.7} the yard is dirty, he should sweep it. If the gatehouse is dirty, he should sweep it. If the assembly hall is dirty, he should sweep it. If the water-boiling shed is dirty, he should sweep it. If the restroom is dirty, he should sweep it. If there’s no water for drinking, he should get some. If there’s no water for washing, he should get some. If there’s no water in the restroom ablutions pot, he should fill it. 

If\marginnote{7.4.15} he’s staying in the same dwelling as a more senior monk, he shouldn’t do any of the following without asking him for permission: recite, question, rehearse, teach, turn a lamp on or off, or open or close a window. If he’s doing walking meditation on the same walking path as a more senior monk, they should turn around according to seniority, but he shouldn’t touch the senior monk with the corner of his robe. 

This\marginnote{7.4.17} is the proper conduct in regard to dwellings.” 

\section*{9. Discussion of the proper conduct in regard to saunas }

On\marginnote{8.1.1} one occasion the monks from the group of six were kept out of the sauna by the senior monks. Then, out of disrespect, they stacked up much firewood, lit it, closed the door, and sat down against it. The monks overheated, but being unable to open the door, they fainted and collapsed. 

The\marginnote{8.1.3} monks of few desires complained and criticized them, “How can the monks from the group of six do this?” They told the Buddha. Soon afterwards he had the Sangha gathered and questioned the monks: “Is it true, monks, that the monks from the group of six did this?” “It’s true, sir.” … After rebuking them … the Buddha gave a teaching and addressed the monks: 

\scrule{“If you’re kept out of the sauna by senior monks, you shouldn’t, out of disrespect, stack up much firewood and light it. If you do, you commit an offense of wrong conduct. }

\scrule{And you shouldn’t close the door and sit down against it. If you do, you commit an offense of wrong conduct. }

Well\marginnote{8.2.1} then, I’ll lay down the proper conduct in regard to saunas.\footnote{See discussion on the \textit{\textsanskrit{jantāghara}} in Appendix of Technical Terms. } The monk who goes first to the sauna should discard any ashes that are building up. If the sauna is dirty, he should sweep it. If the area surrounding the sauna is dirty, he should sweep it.\footnote{Sp 4.371: \textit{\textsanskrit{Paribhaṇḍanti} bahijagati}, “\textit{\textsanskrit{Paribhaṇḍa}}: the floor outside.” } If the yard is dirty, he should sweep it. If the gatehouse is dirty, he should sweep it. If the sauna shed is dirty, he should sweep it. 

He\marginnote{8.2.8} should knead bath powder, moisten the clay, and fill the water trough with water. When entering the sauna, he should smear his face with clay, cover himself front and back, and then enter. He shouldn’t sit encroaching on the senior monks, or block the junior monks from getting a seat. If he’s able, he should provide assistance to the senior monks in the sauna. When leaving the sauna, he should take the sauna bench, cover himself front and back, and then leave. 

If\marginnote{8.2.14} he’s able, he should also provide assistance to the senior monks in the water. He shouldn’t bathe in front of the senior monks or upstream from them. When he’s coming out of the water after bathing, he should give way to those who are entering the water. 

If\marginnote{8.2.17} the sauna is muddy, the last monk to leave it should clean it. He should wash the clay trough, put away the sauna bench, extinguish the fire, close the door, and then leave. 

This\marginnote{8.2.19} is the proper conduct in regard to saunas.” 

\section*{10. Discussion of the proper conduct in regard to restrooms }

At\marginnote{9.1.1} that time a monk who had been born as a brahmin did not want to wash after defecating, thinking, “Who would touch this foul, stinking stuff?” As a result, a worm settled in his rectum. He told the monks, who said, “So you don’t wash after defecating?” 

“That’s\marginnote{9.1.6} right.” 

The\marginnote{9.1.7} monks of few desires complained and criticized him, “How can a monk not wash after defecating?” They told the Buddha. Soon afterwards he had the Sangha gathered and questioned the monks: “Is it true, monk, that you don’t wash after defecating?” “It’s true, sir.” … After rebuking them … the Buddha gave a teaching and addressed the monks: 

\scrule{“When there’s water available, you should wash after defecating. If you don’t, you commit an offense of wrong conduct.” }

At\marginnote{10.1.1} that time the monks used the restrooms according to seniority. Junior monks who had arrived first had to wait to defecate. Unable to hold out, they fainted and collapsed. They told the Buddha. … “Is it true, monks, that this is happening?” “It’s true, sir.” … 

\scrule{“The restrooms shouldn’t be used according to seniority. If you do, you commit an offense of wrong conduct. You should use the restroom according to the order of arrival.” }

At\marginnote{10.2.1} that time the monks from the group of six entered the restrooms too hastily, pulled up their robes before entering, groaned while defecating, cleaned their teeth while defecating, defecated outside the toilet, urinated outside the urinal, spat in the urinal, used coarse wiping sticks, threw the wiping sticks in the cesspit, left the restroom too hastily, came out with their robes still pulled up, made a chomping sound while washing, and they left water in the ablutions scoop. 

The\marginnote{10.2.2} monks of few desires complained and criticized them, “How can the monks from the group of six act like this?” They told the Buddha. … “Is it true, monks, that the monks from the group of six are acting like this?” “It’s true, sir.” … After rebuking them … the Buddha gave a teaching and addressed the monks: 

“Well\marginnote{10.2.9} then, I’ll lay down the proper conduct in regard to restrooms. When a monk goes to the restroom, he should stand outside and clear his throat. Anyone sitting inside should also clear his throat. After hanging his robe on a bamboo robe rack or a clothesline, he should enter the restroom carefully and without hurry. He shouldn’t enter the restroom too hastily; he shouldn’t pull up his robe before he has entered; he should pull up his robe when he’s standing on the foot stands for defecating; he shouldn’t groan while defecating; he shouldn’t clean his teeth while defecating; he shouldn’t defecate outside the toilet; he shouldn’t urinate outside the urinal; he shouldn’t spit in the urinal; he shouldn’t use coarse wiping sticks; he shouldn’t throw the wiping sticks in the cesspit; he should cover himself while still standing on the foot stands for defecating; he shouldn’t leave the restroom too hastily; he shouldn’t come out with his robe still pulled up; he should pull up his robe when standing on the ablutions foot stands; he shouldn’t make a chomping sound while washing; he shouldn’t leave water in the ablutions scoop; he should cover himself while still standing on the ablutions foot stands. 

If\marginnote{10.3.21} the restroom is dirty, he should clean it. If the wiping-stick container is full, he should discard the wiping sticks. If the restroom is dirty, he should sweep it. If the area surrounding the restroom is dirty, he should sweep it.\footnote{“Area surrounding” renders \textit{\textsanskrit{paribhaṇḍa}}. Sp 4.371: \textit{\textsanskrit{Paribhaṇḍanti} bahijagati}, “\textit{\textsanskrit{Paribhaṇḍa}}: the floor outside.” } If the yard is dirty, he should sweep it. If the gatehouse is dirty, he should sweep it. If there’s no water in the restroom ablutions pot, he should fill it. 

This\marginnote{10.3.28} is the proper conduct in regard to restrooms.” 

\section*{11. Discussion of the proper conduct toward a preceptor }

At\marginnote{11.1.1} that time the students were not conducting themselves properly toward their preceptors. The monks of few desires complained and criticized them, “How can the students not conduct themselves properly toward their preceptors?” They told the Buddha. Soon afterwards he had the Sangha gathered and questioned the monks: “Is it true, monks, that the students are not conducting themselves properly toward their preceptors?” “It’s true, sir.” … The Buddha rebuked them … “How can they act like this?  This will affect people’s confidence …” After rebuking them … the Buddha gave a teaching and addressed the monks: 

“Well\marginnote{11.1.12} then, I’ll lay down the proper conduct for a student toward his preceptor. A student should conduct himself properly toward his preceptor. This is the proper conduct: 

\subsection*{Meals and almsround}

Having\marginnote{11.2.3.1} gotten up at the appropriate time, the student should remove his sandals and arrange his upper robe over one shoulder. He should then give his preceptor a tooth cleaner and water for rinsing the mouth, and he should prepare a seat for him. If there’s congee, he should rinse a vessel and bring the congee to his preceptor. When he has drunk the congee, the student should give him water and receive the vessel. Holding it low, he should wash it carefully without scratching it and then put it away. When the preceptor has gotten up, the student should put away the seat. If the place is dirty, he should sweep it. 

If\marginnote{11.3.1} the preceptor wants to enter the village, the student should give him a sarong and receive the one he’s wearing in return. He should give him a belt. He should put the upper robes together, overlapping each other edge-to-edge, and then give them to him. He should rinse his preceptor’s bowl and give it to him while wet. If the preceptor wants an attendant, the student should put on his sarong evenly all around, covering the navel and the knees. He should put on a belt. Putting the upper robes together, overlapping each other edge-to-edge, he should put them on and fasten the toggle. He should rinse his bowl, bring it along, and be his preceptor’s attendant. 

He\marginnote{11.3.3} shouldn’t walk too far behind his preceptor or too close to him. He should receive the contents of his bowl. He shouldn’t interrupt his preceptor when he’s speaking. But if the preceptor’s speech is bordering on an offense, he should stop him. 

When\marginnote{11.4.3} returning, the student should go first to prepare a seat and to set out a foot stool, a foot scraper, and water for washing the feet. He should go out to meet the preceptor and receive his bowl and robe. He should give him a sarong and receive the one he’s wearing in return. If the robe is damp, he should sun it for a short while, but shouldn’t leave it in the heat. He should fold the robe, offsetting the edges by seven centimeters,\footnote{That is, four \textit{\textsanskrit{aṅgula}}. See discussion under \textit{sugata} in Appendix of Technical Terms. } so that the fold doesn’t become worn. He should place the belt in the fold. 

If\marginnote{11.4.9} there’s almsfood and his preceptor wants to eat, the student should give him water and then the almsfood. He should ask his preceptor if he wants water to drink. When the preceptor has eaten, the student should give him water and receive his bowl. Holding it low, he should wash it carefully without scratching it. He should then dry it and sun it for a short while, but shouldn’t leave it in the heat. 

The\marginnote{11.5.4} student should put away the robe and bowl. When putting away the bowl, he should hold the bowl in one hand, feel under the bed or the bench with the other, and then put it away. He shouldn’t put the bowl away on the bare floor. When putting away the robe, he should hold the robe in one hand, wipe the bamboo robe rack or the clothesline with the other, and then put it away by folding the robe over it, making the ends face the wall and the fold face out. When the preceptor has gotten up, the student should put away the seat and also the foot stool, the foot scraper, and the water for washing the feet. If the place is dirty, he should sweep it.” 

\subsection*{Bathing}

“If\marginnote{11.6.1} the preceptor wants to bathe, the student should prepare a bath. If he wants a cold bath, he should prepare that; if he wants a hot bath, he should prepare that. 

If\marginnote{11.6.4} the preceptor wants to take a sauna, the student should knead bath powder, moisten the clay, take a sauna bench, and follow behind his preceptor. After giving his preceptor the sauna bench, receiving his robe, and putting it aside, he should give him the bath powder and the clay. If he’s able, he should enter the sauna. When entering the sauna, he should smear his face with clay, cover himself front and back, and then enter. He shouldn’t sit encroaching on the senior monks, or block the junior monks from getting a seat. While in the sauna, he should provide assistance to his preceptor. When leaving the sauna, he should take the sauna bench, cover himself front and back, and then leave. 

He\marginnote{11.7.5} should also provide assistance to his preceptor in the water. When he has bathed, he should be the first to come out. He should dry himself and put on his sarong. He should then wipe the water off his preceptor’s body, and he should give him his sarong and then his upper robe. Taking the sauna bench, he should be the first to return. He should prepare a seat, and also set out a foot stool, a foot scraper, and water for washing the feet. He should ask his preceptor if he wants water to drink. If the preceptor wants him to recite, he should do so. If the preceptor wants to question him, he should be questioned.” 

\subsection*{The dwelling}

“If\marginnote{11.8.3.1} the dwelling where the preceptor is staying is dirty, the student should clean it if he’s able. When he’s cleaning the dwelling, he should first take out the bowl and robe and put them aside. He should take out the sitting mat and the sheet and put them aside. He should take out the mattress and the pillow and put them aside. Holding the bed low, he should carefully take it out without scratching it or knocking it against the door or the door frame, and he should put it aside. Holding the bench low, he should carefully take it out without scratching it or knocking it against the door or the door frame, and he should put it aside. He should take out the bed supports and put them aside. He should take out the spittoon and put it aside. He should take out the leaning board and put it aside. After taking note of its position, he should take out the floor cover and put it aside. If the dwelling has cobwebs, he should first remove them from the ceiling cloth, and he should then wipe the windows and the corners of the room. If the walls have been treated with red ocher and they’re moldy, he should moisten a cloth, wring it out, and wipe the walls. If the floor has been treated with a black finish and it’s moldy, he should moisten a cloth, wring it out, and wipe the floor. If the floor is untreated, he should sprinkle it with water and then sweep it, trying to avoid stirring up dust. He should look out for any trash and discard it. 

He\marginnote{11.10.1} should sun the floor cover, clean it, beat it, bring it back inside, and put it back as before. He should sun the bed supports, wipe them, bring them back inside, and put them back where they were. He should sun the bed, clean it, and beat it. Holding it low, he should carefully bring it back inside without scratching it or knocking it against the door or the door frame, and he should put it back as before. He should sun the bench, clean it, and beat it. Holding it low, he should carefully bring it back inside without scratching it or knocking it against the door or the door frame, and he should put it back as before. He should sun the mattress and the pillow, clean them, beat them, bring them back inside, and put them back as before. He should sun the sitting mat and the sheet, clean them, beat them, bring them back inside, and put them back as before. He should sun the spittoon, wipe it, bring it back inside, and put it back where it was. He should sun the leaning board, wipe it, bring it back inside, and put it back where it was. He should put away the bowl and robe. When putting away the bowl, he should hold the bowl in one hand, feel under the bed or the bench with the other, and then put it away. He shouldn’t put the bowl away on the bare floor. When putting away the robe, he should hold the robe in one hand, wipe the bamboo robe rack or the clothesline with the other, and then put it away by folding the robe over it, making the ends face the wall and the fold face out. 

If\marginnote{11.12.1} dusty winds are blowing from the east, he should close the windows on the eastern side. If dusty winds are blowing from the west, he should close the windows on the western side. If dusty winds are blowing from the north, he should close the windows on the northern side. If dusty winds are blowing from the south, he should close the windows on the southern side. If the weather is cold, he should open the windows during the day and close them at night. If the weather is hot, he should close the windows during the day and open them at night. 

If\marginnote{11.13.1} the yard is dirty, he should sweep it. If the gatehouse is dirty, he should sweep it. If the assembly hall is dirty, he should sweep it. If the water-boiling shed is dirty, he should sweep it. If the restroom is dirty, he should sweep it. If there’s no water for drinking, he should get some. If there’s no water for washing, he should get some. If there’s no water in the restroom ablutions pot, he should fill it.” 

\subsection*{Spiritual support, etc.}

“If\marginnote{11.14.1} the preceptor becomes discontent with the spiritual life, the student should send him away or have him sent away, or he should give him a teaching. If the preceptor becomes anxious, the student should dispel it or have it dispelled, or he should give him a teaching. If the preceptor has wrong view, the student should make him give it up or have someone else do it, or he should give him a teaching. If the preceptor has committed a heavy offense and deserves probation, the student should try to get the Sangha to give it to him. If the preceptor has committed a heavy offense and deserves to be sent back to the beginning, the student should try to get the Sangha to do it. If the preceptor has committed a heavy offense and deserves the trial period, the student should try to get the Sangha to give it to him. If the preceptor has committed a heavy offense and deserves rehabilitation, the student should try to get the Sangha to give it to him. 

If\marginnote{11.16.1} the Sangha wants to do a legal procedure against his preceptor—whether a procedure of condemnation, demotion, banishment, reconciliation, or ejection—\footnote{For an explanation of the rendering “demotion” for \textit{niyassa}, see Appendix of Technical Terms. } the student should make an effort to stop it or to reduce the penalty. But if the Sangha has already done a legal procedure against his preceptor—whether a procedure of condemnation, demotion, banishment, reconciliation, or ejection—the student should help the preceptor conduct himself properly and suitably so as to deserve to be released, and try to get the Sangha to lift that procedure.\footnote{The meaning of the first of these phrases, \textit{\textsanskrit{sammā} vattati}, is straightforward, but the last two, \textit{\textsanskrit{lomaṁ} \textsanskrit{pāteti}} and \textit{\textsanskrit{netthāraṁ} vattati}, are more difficult. Commenting on Bu Ss 13, Sp 1.435 explains: \textit{Na \textsanskrit{lomaṁ} \textsanskrit{pātentīti} \textsanskrit{anulomapaṭipadaṁ} \textsanskrit{appaṭipajjanatāya} na \textsanskrit{pannalomā} honti. Na \textsanskrit{netthāraṁ} \textsanskrit{vattantīti} attano \textsanskrit{nittharaṇamaggaṁ} na \textsanskrit{paṭipajjanti}}, “\textit{Na \textsanskrit{lomaṁ} \textsanskrit{pātenti}}: because of their non-practicing in conformity with the path, their bodily hairs are not flat. \textit{Na \textsanskrit{netthāraṁ} vattanti}: they are not practicing the path for their own getting out (of the offense).” My rendering attempts to capture the meaning in a non-literal way. } 

If\marginnote{11.17.1} the preceptor’s robe needs washing, the student should do it himself, or he should make an effort to get it done. If the preceptor needs a robe, the student should make one himself, or he should make an effort to get one made. If the preceptor needs dye, the student should make it himself, or he should make an effort to get it made. If the preceptor’s robe needs dyeing, the student should do it himself, or he should make an effort to get it done. When he’s dyeing the robe, he should carefully and repeatedly turn it over, and shouldn’t go away while it’s still dripping. 

Without\marginnote{11.18.1} asking his preceptor for permission, he shouldn’t do any of the following: give away or receive a bowl; give away or receive a robe; give away or receive a requisite; cut anyone’s hair or get it cut; provide assistance to anyone or have assistance provided by anyone; do a service for anyone or get a service done by anyone; be the attendant monk for anyone or take anyone as his attendant monk; bring back almsfood for anyone or get almsfood brought back by anyone; enter the village, go to the charnel ground, or leave for another region. If his preceptor is sick, he should nurse him for as long as he lives or wait until he has recovered. 

This\marginnote{11.18.13} is the proper conduct of a student toward his preceptor.” 

\section*{12. Discussion of the proper conduct toward a student }

At\marginnote{12.1.1} that time the preceptors were not conducting themselves properly toward their students. The monks of few desires complained and criticized them, “How can the preceptors not conduct themselves properly toward their students?” They told the Buddha. Soon afterwards he had the Sangha gathered and questioned the monks: “Is it true, monks, that the preceptors are not conducting themselves properly toward their students?” “It’s true, sir.” … After rebuking them … the Buddha gave a teaching and addressed the monks: 

“Well\marginnote{12.1.9} then, I’ll lay down the proper conduct for preceptors toward their students. A preceptor should conduct himself properly toward his student. This is the proper conduct: 

A\marginnote{12.2.3} preceptor should help and take care of his student through recitation, questioning, and instruction. If the preceptor has a bowl, but not the student, the preceptor should give it to him,\footnote{Sp 3.67: \textit{Sace \textsanskrit{upajjhāyassa} patto \textsanskrit{hotīti} sace atirekapatto hoti. Esa nayo sabbattha}, “‘If the preceptor has a bowl’ means if the preceptor has an extra bowl. This method applies to everything (below).” } or he should make an effort to get him one. If the preceptor has a robe, but not the student, the preceptor should give it to him, or he should make an effort to get him one. If the preceptor has a requisite, but not the student, the preceptor should give it to him, or he should make an effort to get him one.” 

\subsection*{Meals and almsround}

“If\marginnote{12.3.1} the student is sick, the preceptor should get up at the appropriate time and give his student a tooth cleaner and water for rinsing the mouth, and he should prepare a seat for him. If there’s congee, he should rinse a vessel and bring the congee to his student. When he has drunk the congee, the preceptor should give him water and receive the vessel. Holding it low, he should wash it carefully without scratching it and then put it away. When the student has gotten up, the preceptor should put away the seat. If the place is dirty, he should sweep it. 

If\marginnote{12.4.1} the student wants to enter the village, the preceptor should give him a sarong and receive the one he’s wearing in return. He should give him a belt. He should put the upper robes together, overlapping each other edge-to-edge, and then give them to him. He should rinse his student’s bowl and give it to him while wet. 

Before\marginnote{12.4.2} he’s due back, the preceptor should prepare a seat and set out a foot stool, a foot scraper, and water for washing the feet. He should go out to meet the student and receive his bowl and robe. He should give him a sarong and receive the one he’s wearing in return. If the robe is damp, he should sun it for a short while, but shouldn’t leave it in the heat. He should fold the robe, offsetting the edges by seven centimeters,\footnote{That is, four \textit{\textsanskrit{aṅgula}}. See discussion under \textit{sugata} in Appendix of Technical Terms. } so that the fold doesn’t become worn. He should place the belt in the fold. 

If\marginnote{12.4.8} there’s almsfood and his student wants to eat, the preceptor should give him water and then the almsfood. He should ask his student if he wants water to drink. When the student has eaten, the preceptor should give him water and receive his bowl. Holding it low, he should wash it carefully without scratching it. He should then dry it and sun it for a short while, but shouldn’t leave it in the heat. The preceptor should put away the robe and bowl. When putting away the bowl, he should hold the bowl in one hand, feel under the bed or the bench with the other, and then put it away. He shouldn’t put the bowl away on the bare floor. When putting away the robe, he should hold the robe in one hand, wipe the bamboo robe rack or the clothesline with the other, and then put it away by folding the robe over it, making the ends face the wall and the fold face out. When the student has gotten up, the preceptor should put away the seat and also the foot stool, the foot scraper, and the water for washing the feet. If the place is dirty, he should sweep it.” 

\subsection*{Bathing}

“If\marginnote{12.6.1} the student wants to bathe, the preceptor should prepare a bath. If he wants a cold bath, he should prepare that; if he wants a hot bath, he should prepare that. 

If\marginnote{12.6.4} the student wants to take a sauna, the preceptor should knead bath powder, moisten the clay, take a sauna bench, and go to the sauna. After giving his student the sauna bench, receiving his robe, and putting it aside, he should give him the bath powder and the clay. If he’s able, he should enter the sauna. When entering the sauna, he should smear his face with clay, cover himself front and back, and then enter. He shouldn’t sit encroaching on the senior monks, or block the junior monks from getting a seat. While in the sauna, he should provide assistance to his student. When leaving the sauna, he should take the sauna bench, cover himself front and back, and then leave. 

The\marginnote{12.7.4} preceptor should also provide assistance to his student in the water. When the preceptor has bathed, he should be the first to come out. He should dry himself and put on his sarong. He should then wipe the water off his student’s body, and he should give him his sarong and then his upper robe. Taking the sauna bench, he should be the first to return. He should prepare a seat, and also set out a foot stool, a foot scraper, and water for washing the feet. He should ask his student if he wants water to drink.” 

\subsection*{The dwelling}

“If\marginnote{12.8.1} the dwelling where the student is staying is dirty, the preceptor should clean it if he’s able. When he’s cleaning the dwelling, he should first take out the bowl and robe and put them aside. He should take out the sitting mat and the sheet and put them aside. He should take out the mattress and the pillow and put them aside. Holding the bed low, he should carefully take it out without scratching it or knocking it against the door or the door frame, and he should put it aside. Holding the bench low, he should carefully take it out without scratching it or knocking it against the door or the door frame, and he should put it aside. He should take out the bed supports and put them aside. He should take out the spittoon and put it aside. He should take out the leaning board and put it aside. After taking note of its position, he should take out the floor cover and put it aside. If the dwelling has cobwebs, he should first remove them from the ceiling cloth, and he should then wipe the windows and the corners of the room. If the walls have been treated with red ocher and they’re moldy, he should moisten a cloth, wring it out, and wipe the walls. If the floor has been treated with a black finish and it’s moldy, he should moisten a cloth, wring it out, and wipe the floor. If the floor is untreated, he should sprinkle it with water and then sweep it, trying to avoid stirring up dust. He should look out for any trash and discard it. 

He\marginnote{12.8.2.2} should sun the floor cover, clean it, beat it, bring it back inside, and put it back as before. He should sun the bed supports, wipe them, bring them back inside, and put them back where they were. He should sun the bed, clean it, and beat it. Holding it low, he should carefully bring it back inside without scratching it or knocking it against the door or the door frame, and he should put it back as before. He should sun the bench, clean it, and beat it. Holding it low, he should carefully bring it back inside without scratching it or knocking it against the door or the door frame, and he should put it back as before. He should sun the mattress and the pillow, clean them, beat them, bring them back inside, and put them back the way they were. He should sun the sitting mat and the sheet, clean them, beat them, bring them back inside, and put them back the way they were. He should sun the spittoon, wipe it, bring it back inside, and put it back where it was. He should sun the leaning board, wipe it, bring it back inside, and put it back where it was. He should put away the bowl and robe. When putting away the bowl, he should hold the bowl in one hand, feel under the bed or the bench with the other, and then put it away. He shouldn’t put the bowl away on the bare floor. When putting away the robe, he should hold the robe in one hand, wipe the bamboo robe rack or the clothesline with the other, and then put it away by folding the robe over it, making the ends face the wall and the fold face out. 

If\marginnote{12.8.2.3} dusty winds are blowing from the east, he should close the windows on the eastern side. If dusty winds are blowing from the west, he should close the windows on the western side. If dusty winds are blowing from the north, he should close the windows on the northern side. If dusty winds are blowing from the south, he should close the windows on the southern side. If the weather is cold, he should open the windows during the day and close them at night. If the weather is hot, he should close the windows during the day and open them at night. 

If\marginnote{12.8.2.4} the yard is dirty, he should sweep it. If the gatehouse is dirty, he should sweep it. If the assembly hall is dirty, he should sweep it. If the water-boiling shed is dirty, he should sweep it. If the restroom is dirty, he should sweep it. If there’s no water for drinking, he should get some. If there’s no water for washing, he should get some.  If there’s no water in the restroom ablutions pot, he should fill it.” 

\subsection*{Spiritual support, etc.}

“If\marginnote{12.8.4.1} the student becomes discontent with the spiritual life, the preceptor should send him away or have him sent away, or he should give him a teaching. If the student becomes anxious, the preceptor should dispel it or have it dispelled, or he should give him a teaching. If the student has wrong view, the preceptor should make him give it up or have someone else do it, or he should give him a teaching. If the student has committed a heavy offense and deserves probation, the preceptor should try to get the Sangha to give it to him. If the student has committed a heavy offense and deserves to be sent back to the beginning, the preceptor should try to get the Sangha to do it. If the student has committed a heavy offense and deserves the trial period, the preceptor should try to get the Sangha to give it to him. If the student has committed a heavy offense and deserves rehabilitation, the preceptor should try to get the Sangha to give it to him. 

If\marginnote{12.10.1} the Sangha wants to do a legal procedure against his student—whether a procedure of condemnation, demotion, banishment, reconciliation, or ejection—the preceptor should make an effort to stop it or to reduce the penalty. But if the Sangha has already done a legal procedure against his student—whether a procedure of condemnation, demotion, banishment, reconciliation, or ejection—the preceptor should help the student conduct himself properly and suitably so as to deserve to be released, and try to get the Sangha to lift that procedure. 

If\marginnote{12.11.1} the student’s robe needs washing, the preceptor should show him how to do it, or he should make an effort to get it done. If the student needs a robe, the preceptor should show him how to make one, or he should make an effort to get one made. If the student needs dye, the preceptor should show him how to make it, or he should make an effort to get it made. If the student’s robe needs dyeing, the preceptor should show him how to do it, or he should make an effort to get it done. When he’s dyeing the robe, he should carefully and repeatedly turn it over, and shouldn’t go away while it’s still dripping. If his student is sick, he should nurse him for as long as he lives or wait until he has recovered. 

This\marginnote{12.11.11} is the proper conduct of a preceptor toward his student.” 

\scend{The second section for recitation is finished. }

\section*{13. Discussion of the proper conduct toward a teacher }

At\marginnote{12.11.13.1} that time the pupils were not conducting themselves properly toward their teachers. The monks of few desires complained and criticized them, “How can the pupils not conduct themselves properly toward their teachers?” They told the Buddha. … “Is it true, monks, that the pupils are not conducting themselves properly toward their teachers?” “It’s true, sir.” … After rebuking them … the Buddha gave a teaching and addressed the monks: 

“Well\marginnote{12.11.21} then, I’ll lay down the proper conduct for a pupil toward his teacher. A pupil should conduct himself properly toward his teacher. This is the proper conduct: 

\subsection*{Meals and almsround}

Having\marginnote{12.11.24.1} gotten up at the appropriate time, the pupil should remove his sandals, and arrange his upper robe over one shoulder. He should then give his teacher a tooth cleaner and water for rinsing the mouth, and he should prepare a seat for him. If there’s congee, he should rinse a vessel and bring the congee to his teacher. When he has drunk the congee, the pupil should give him water and receive the vessel. Holding it low, he should wash it carefully without scratching it and then put it away. When the teacher has gotten up, the pupil should put away the seat. If the place is dirty, he should sweep it. 

If\marginnote{12.11.29} the teacher wants to enter the village, the pupil should give him a sarong and receive the one he’s wearing in return. He should give him a belt. He should put the upper robes together, overlapping each other edge-to-edge, and then give them to him. He should rinse his teacher’s bowl and give it to him while wet. If the teacher wants an attendant, the pupil should put on his sarong evenly all around, covering the navel and the knees. He should put on a belt. Putting the upper robes together, overlapping each other edge-to-edge, he should put them on and fasten the toggle. He should rinse his bowl, bring it along, and be his teacher’s attendant. 

He\marginnote{12.11.31} shouldn’t walk too far behind his teacher or too close to him. He should receive the contents of his bowl. He shouldn’t interrupt his teacher when he’s speaking. But if the teacher’s speech is bordering on an offense, he should stop him. 

When\marginnote{12.11.34} returning, the pupil should go first to prepare a seat and to set out a foot stool, a foot scraper, and water for washing the feet. He should go out to meet the teacher and receive his bowl and robe. He should give him a sarong and receive the one he’s wearing in return. If the robe is damp, he should sun it for a short while, but shouldn’t leave it in the heat. He should fold the robe, offsetting the edges by seven centimeters,\footnote{That is, four \textit{\textsanskrit{aṅgula}}. See discussion under \textit{sugata} in Appendix of Technical Terms. } so that the fold doesn’t become worn. He should place the belt in the fold. 

If\marginnote{12.11.40} there’s almsfood and his teacher wants to eat, the pupil should give him water and then the almsfood. He should ask his teacher if he wants water to drink. When the teacher has eaten, the pupil should give him water and receive his bowl. Holding it low, he should wash it carefully without scratching it. He should then dry it and sun it for a short while, but shouldn’t leave it in the heat. 

The\marginnote{12.11.43} pupil should put away the robe and bowl. When putting away the bowl, he should hold the bowl in one hand, feel under the bed or the bench with the other, and then put it away. He shouldn’t put the bowl away on the bare floor. When putting away the robe, he should hold the robe in one hand, wipe the bamboo robe rack or the clothesline with the other, and then put it away by folding the robe over it, making the ends face the wall and the fold face out. When the teacher has gotten up, the pupil should put away the seat and also the foot stool, the foot scraper, and the water for washing the feet. If the place is dirty, he should sweep it.” 

\subsection*{Bathing}

“If\marginnote{12.11.49.1} the teacher wants to bathe, the pupil should prepare a bath. If he wants a cold bath, he should prepare that; if he wants a hot bath, he should prepare that. 

If\marginnote{12.11.52} the teacher wants to take a sauna, the pupil should knead bath powder, moisten the clay, take a sauna bench, and follow behind his teacher. After giving his teacher the sauna bench, receiving his robe, and putting it aside, he should give him the bath powder and the clay. If he’s able, he should enter the sauna. When entering the sauna, he should smear his face with clay, cover himself front and back, and then enter. He shouldn’t sit encroaching on the senior monks, or block the junior monks from getting a seat. While in the sauna, he should provide assistance to his teacher. When leaving the sauna, he should take the sauna bench, cover himself front and back, and then leave. 

He\marginnote{12.11.59} should also provide assistance to his teacher in the water. When he has bathed, he should be the first to come out. He should dry himself and put on his sarong. He should then wipe the water off his teacher’s body, and he should give him his sarong and then his upper robe. Taking the sauna bench, he should be the first to return. He should prepare a seat, and also set out a foot stool, a foot scraper, and water for washing the feet. He should ask his teacher if he wants water to drink. If the teacher wants him to recite, he should do so. If the teacher wants to question him, he should be questioned.” 

\subsection*{The dwelling}

“If\marginnote{12.11.64.1} the dwelling where the teacher is staying is dirty, the pupil should clean it if he’s able. When he’s cleaning the dwelling, he should first take out the bowl and robe and put them aside. He should take out the sitting mat and the sheet and put them aside. He should take out the mattress and the pillow and put them aside. Holding the bed low, he should carefully take it out without scratching it or knocking it against the door or the door frame, and he should put it aside. Holding the bench low, he should carefully take it out without scratching it or knocking it against the door or the door frame, and he should put it aside. He should take out the bed supports and put them aside. He should take out the spittoon and put it aside. He should take out the leaning board and put it aside. After taking note of its position, he should take out the floor cover and put it aside. If the dwelling has cobwebs, he should first remove them from the ceiling cloth, and he should then wipe the windows and the corners of the room. If the walls have been treated with red ocher and they’re moldy, he should moisten a cloth, wring it out, and wipe the walls. If the floor has been treated with a black finish and it’s moldy, he should moisten a cloth, wring it out, and wipe the floor. If the floor is untreated, he should sprinkle it with water and then sweep it, trying to avoid stirring up dust. He should look out for any trash and discard it. 

He\marginnote{12.11.80} should sun the floor cover, clean it, beat it, bring it back inside, and put it back as before. He should sun the bed supports, wipe them, bring them back inside, and put them back where they were. He should sun the bed, clean it, and beat it. Holding it low, he should carefully bring it back inside without scratching it or knocking it against the door or the door frame, and he should put it back as before. He should sun the bench, clean it, and beat it. Holding it low, he should carefully bring it back inside without scratching it or knocking it against the door or the door frame, and he should put it back as before. He should sun the mattress and the pillow, clean them, beat them, bring them back inside, and put them back as before. He should sun the sitting mat and the sheet, clean them, beat them, bring them back inside, and put them back as before. He should sun the spittoon, wipe it, bring it back inside, and put it back where it was. He should sun the leaning board, wipe it, bring it back inside, and put it back where it was. He should put away the bowl and robe. When putting away the bowl, he should hold the bowl in one hand, feel under the bed or the bench with the other, and then put it away. He shouldn’t put the bowl away on the bare floor. When putting away the robe, he should hold the robe in one hand, wipe the bamboo robe rack or the clothesline with the other, and then put it away by folding the robe over it, making the ends face the wall and the fold face out. 

If\marginnote{12.11.92} dusty winds are blowing from the east, he should close the windows on the eastern side. If dusty winds are blowing from the west, he should close the windows on the western side. If dusty winds are blowing from the north, he should close the windows on the northern side. If dusty winds are blowing from the south, he should close the windows on the southern side. If the weather is cold, he should open the windows during the day and close them at night. If the weather is hot, he should close the windows during the day and open them at night. 

If\marginnote{12.11.98} the yard is dirty, he should sweep it. If the gatehouse is dirty, he should sweep it. If the assembly hall is dirty, he should sweep it. If the water-boiling shed is dirty, he should sweep it. If the restroom is dirty, he should sweep it. If there’s no water for drinking, he should get some. If there’s no water for washing, he should get some. If there’s no water in the restroom ablutions pot, he should fill it.” 

\subsection*{Spiritual support, etc.}

“If\marginnote{12.11.106.1} the teacher becomes discontent with the spiritual life, the pupil should send him away or have him sent away, or he should give him a teaching. If the teacher becomes anxious, the pupil should dispel it or have it dispelled, or he should give him a teaching. If the teacher has wrong view, the pupil should make him give it up or have someone else do it, or he should give him a teaching. If the teacher has committed a heavy offense and deserves probation, the pupil should try to get the Sangha to give it to him. If the teacher has committed a heavy offense and deserves to be sent back to the beginning, the pupil should try to get the Sangha to do it. If the teacher has committed a heavy offense and deserves the trial period, the pupil should try to get the Sangha to give it to him. If the teacher has committed a heavy offense and deserves rehabilitation, the pupil should try to get the Sangha to give it to him. 

If\marginnote{12.11.117} the Sangha wants to do a legal procedure against his teacher—whether a procedure of condemnation, demotion, banishment, reconciliation, or ejection—the pupil should make an effort to stop it or to reduce the penalty. But if the Sangha has already done a legal procedure against his teacher—whether a procedure of condemnation, demotion, banishment, reconciliation, or ejection—the pupil should help the teacher conduct himself properly and suitably so as to deserve to be released, and try to get the Sangha to lift that procedure. 

If\marginnote{12.11.121} the teacher’s robe needs washing, the pupil should do it himself, or he should make an effort to get it done. If the teacher needs a robe, the pupil should make one himself, or he should make an effort to get one made. If the teacher needs dye, the pupil should make it himself, or he should make an effort to get it made. If the teacher’s robe needs dyeing, the pupil should do it himself, or he should make an effort to get it done. When he’s dyeing the robe, he should carefully and repeatedly turn it over, and shouldn’t go away while it’s still dripping. 

Without\marginnote{12.11.130} asking his teacher for permission, he shouldn’t do any of the following: give away or receive a bowl; give away or receive a robe; give away or receive a requisite; cut anyone’s hair or get it cut; provide assistance to anyone or have assistance provided by anyone; do a service for anyone or get a service done by anyone; be the attendant monk for anyone or take anyone as his attendant monk; bring back almsfood for anyone or get almsfood brought back by anyone; enter the village, go to the charnel ground, or leave for another region. If his teacher is sick, he should nurse him for as long as he lives or wait until he has recovered. 

This\marginnote{12.11.142} is the proper conduct of a pupil toward his teacher.” 

\section*{14. Discussion of the proper conduct toward a pupil }

At\marginnote{12.11.143.1} that time the teachers were not conducting themselves properly toward their pupils. The monks of few desires complained and criticized them, “How can the teachers not conduct themselves properly toward their pupils?” They told the Buddha. Soon afterwards he had the Sangha gathered and questioned the monks: “Is it true, monks, that the teachers are not conducting themselves properly toward their pupils?” “It’s true, sir.” … After rebuking them … the Buddha gave a teaching and addressed the monks: 

“Well\marginnote{12.11.152} then, I’ll lay down the proper conduct for a teacher toward his pupil. A teacher should conduct himself properly toward his pupil. This is the proper conduct: 

A\marginnote{12.11.155} teacher should help and take care of his pupil through recitation, questioning, and instruction. If the teacher has a bowl, but not the pupil, the teacher should give it to him, or he should make an effort to get him one. If the teacher has a robe, but not the pupil, the teacher should give it to him, or he should make an effort to get him one. If the teacher has a requisite, but not the pupil, the teacher should give it to him, or he should make an effort to get him one.” 

\subsection*{Meals and almsround}

“If\marginnote{12.11.162.1} the pupil is sick, the teacher should get up at the appropriate time and give his pupil a tooth cleaner and water for rinsing the mouth, and he should prepare a seat for him. If there’s congee, he should rinse a vessel and bring the congee to his pupil. When he has drunk the congee, the teacher should give him water and receive the vessel. Holding it low, he should wash it carefully without scratching it and then put it away. When the pupil has gotten up, the teacher should put away the seat. If the place is dirty, he should sweep it. 

If\marginnote{12.11.167} the pupil wants to enter the village, the teacher should give him a sarong and receive the one he’s wearing in return. He should give him a belt. He should put the upper robes together, overlapping each other edge-to-edge, and then give them to him. He should rinse his pupil’s bowl and give it to him while wet. 

Before\marginnote{12.11.168} he’s due back, the teacher should prepare a seat and set out a foot stool, a foot scraper, and water for washing the feet. He should go out to meet the pupil and receive his bowl and robe. He should give him a sarong and receive the one he’s wearing in return. If the robe is damp, he should sun it for a short while, but shouldn’t leave it in the heat. He should fold the robe, offsetting the edges by seven centimeters,\footnote{That is, four \textit{\textsanskrit{aṅgula}}. See discussion under \textit{sugata} in Appendix of Technical Terms. } so that the fold doesn’t become worn. He should place the belt in the fold. 

If\marginnote{12.11.174} there’s almsfood and his pupil wants to eat, the teacher should give him water and then the almsfood. He should ask his pupil if he wants water to drink. When the pupil has eaten, the teacher should give him water and receive his bowl. Holding it low, he should wash it carefully without scratching it. He should then dry it and sun it for a short while, but shouldn’t leave it in the heat. The teacher should put away the robe and bowl. When putting away the bowl, he should hold the bowl in one hand, feel under the bed or the bench with the other, and then put it away. He shouldn’t put the bowl away on the bare floor. When putting away the robe, he should hold the robe in one hand, wipe the bamboo robe rack or the clothesline with the other, and then put it away by folding the robe over it, making the ends face the wall and the fold face out. When the pupil has gotten up, the teacher should put away the seat and also the foot stool, the foot scraper, and the water for washing the feet. If the place is dirty, he should sweep it.” 

\subsection*{Bathing}

“If\marginnote{12.11.183.1} the pupil wants to bathe, the teacher should prepare a bath. If he wants a cold bath, he should prepare that; if he wants a hot bath, he should prepare that. 

If\marginnote{12.11.186} the pupil wants to take a sauna, the teacher should knead bath powder, moisten the clay, take a sauna bench, and go to the sauna. After giving his pupil the sauna bench, receiving his robe, and putting it aside, he should give him the bath powder and the clay. If he’s able, he should enter the sauna. When entering the sauna, he should smear his face with clay, cover himself front and back, and then enter. He shouldn’t sit encroaching on the senior monks, or block the junior monks from getting a seat. While in the sauna, he should provide assistance to his pupil. When leaving the sauna, he should take the sauna bench, cover himself front and back, and then leave. 

The\marginnote{12.11.193} teacher should also provide assistance to his pupil in the water. When the teacher has bathed, he should be the first to come out. He should dry himself and put on his sarong. He should then wipe the water off his pupil’s body, and he should give him his sarong and then his upper robe. Taking the sauna bench, he should be the first to return. He should prepare a seat, and also set out a foot stool, a foot scraper, and water for washing the feet. He should ask his pupil if he wants water to drink.” 

\subsection*{The dwelling}

“If\marginnote{12.11.195.1} the dwelling where the pupil is staying is dirty, the teacher should clean it if he’s able. When he’s cleaning the dwelling, he should first take out the bowl and robe and put them aside. He should take out the sitting mat and the sheet and put them aside. He should take out the mattress and the pillow and put them aside. Holding the bed low, he should carefully take it out without scratching it or knocking it against the door or the door frame, and he should put it aside. Holding the bench low, he should carefully take it out without scratching it or knocking it against the door or the door frame, and he should put it aside. He should take out the bed supports and put them aside. He should take out the spittoon and put it aside. He should take out the leaning board and put it aside. After taking note of its position, he should take out the floor cover and put it aside. If the dwelling has cobwebs, he should first remove them from the ceiling cloth, and he should then wipe the windows and the corners of the room. If the walls have been treated with red ocher and they’re moldy, he should moisten a cloth, wring it out, and wipe the walls. If the floor has been treated with a black finish and it’s moldy, he should moisten a cloth, wring it out, and wipe the floor. If the floor is untreated, he should sprinkle it with water and then sweep it, trying to avoid stirring up dust. He should look out for any trash and discard it. 

He\marginnote{12.11.197} should sun the floor cover, clean it, beat it, bring it back inside, and put it back as before. He should sun the bed supports, wipe them, bring them back inside, and put them back as before. He should sun the bed, clean it, and beat it. Holding it low, he should carefully bring it back inside without scratching it or knocking it against the door or the door frame, and he should put it back as before. He should sun the bench, clean it, and beat it. Holding it low, he should carefully bring it back inside without scratching it or knocking it against the door or the door frame, and he should put it back as before. He should sun the mattress and the pillow, clean them, beat them, bring them back inside, and put them back as before. He should sun the sitting mat and the sheet, clean them, beat them, bring them back inside, and put them back as before. He should sun the spittoon, wipe it, bring it back inside, and put it back where it was. He should sun the leaning board, wipe it, bring it back inside, and put it back where it was. He should put away the bowl and robe. When putting away the bowl, he should hold the bowl in one hand, feel under the bed or the bench with the other, and then put it away. He shouldn’t put the bowl away on the bare floor. When putting away the robe, he should hold the robe in one hand, wipe the bamboo robe rack or the clothesline with the other, and then put it away by folding the robe over it, making the ends face the wall and the fold face out. 

If\marginnote{12.11.198} dusty winds are blowing from the east, he should close the windows on the eastern side. If dusty winds are blowing from the west, he should close the windows on the western side. If dusty winds are blowing from the north, he should close the windows on the northern side. If dusty winds are blowing from the south, he should close the windows on the southern side. If the weather is cold, he should open the windows during the day and close them at night. If the weather is hot, he should close the windows during the day and open them at night. 

If\marginnote{12.11.199} the yard is dirty, he should sweep it. If the gateway is dirty, he should sweep it. If the assembly hall is dirty, he should sweep it. If the water-boiling shed is dirty, he should sweep it. If the restroom is dirty, he should sweep it. If there’s no water for drinking, he should get some. If there’s no water for washing, he should get some.  If there’s no water in the restroom ablutions pot, he should fill it.” 

\subsection*{Spiritual support, etc.}

“If\marginnote{12.11.201.1} the pupil becomes discontent with the spiritual life, the teacher should send him away or have him sent away, or he should give him a teaching. If the pupil becomes anxious, the teacher should dispel it or have it dispelled, or he should give him a teaching. If the pupil has wrong view, the teacher should make him give it up or have someone else do it, or he should give him a teaching. If the pupil has committed a heavy offense and deserves probation, the teacher should try to get the Sangha to give it to him. If the pupil has committed a heavy offense and deserves to be sent back to the beginning, the teacher should try to get the Sangha to do it. If the pupil has committed a heavy offense and deserves the trial period, the teacher should try to get the Sangha to give it to him. If the pupil has committed a heavy offense and deserves rehabilitation, the teacher should try to get the Sangha to give it to him. 

If\marginnote{12.11.212} the Sangha wants to do a legal procedure against his pupil—whether a procedure of condemnation, demotion, banishment, reconciliation, or ejection—the teacher should make an effort to stop it or to reduce the penalty. But if the Sangha has already done a legal procedure against his pupil—whether a procedure of condemnation, demotion, banishment, reconciliation, or ejection—the teacher should help the pupil conduct himself properly and suitably so as to deserve to be released, and try to get the Sangha to lift that procedure. 

If\marginnote{12.11.216} the pupil’s robe needs washing, the teacher should show him how to do it, or he should make an effort to get it done. If the pupil needs a robe, the teacher should show him how to make one, or he should make an effort to get one made. If the pupil needs dye, the teacher should show him how to make it, or he should make an effort to get it made. If the pupil’s robe needs dyeing, the teacher should show him how to do it, or he should make an effort to get it done. When he’s dyeing the robe, he should carefully and repeatedly turn it over, and shouldn’t go away while it’s still dripping. If his pupil is sick, he should nurse him for as long as he lives or wait until he has recovered. 

This\marginnote{12.11.227} is the proper conduct of a teacher toward his pupil.” 

\scendsutta{The eighth chapter on proper conduct is finished. In this chapter there are nineteen topics and fourteen kinds of proper conduct. }

\scuddanaintro{This is the summary: }

\begin{scuddana}%
“With\marginnote{12.11.230} sandals, and sunshades, \\
Covered, head, drinking water; \\
Would not bow down, they did not ask, \\
Snake, the good monks complained. 

Removed,\marginnote{12.11.234} sunshade, and on the shoulder, \\
Without hurry, gather; \\
Put down bowl and robe, \\
And suitable, asked. 

Should\marginnote{12.11.238} pour, with washed, \\
With dry, with wet, sandals; \\
Senior, junior, should ask, \\
And occupied, where to go for alms. 

Training,\marginnote{12.11.242} excrement, drinking water, washing water, \\
Walking stick, then agreement; \\
The right time, moment, dirty, \\
Should take out the floor cover. 

Bed\marginnote{12.11.246} support, mattress, pillow, \\
Bed, and bench, spittoon; \\
Leaning board, ceiling cloth, corners, \\
Red ocher, black, untreated. 

And\marginnote{12.11.250} trash, floor cover, \\
Bed support, bed, bench; \\
Mattress, also sitting mat, \\
Spittoon, and leaning board. 

Bowl,\marginnote{12.11.254} robe, and floor, \\
Ends far, folds near; \\
From the east, and from the west, \\
From the north, then from the south. 

And\marginnote{12.11.258} day and night in the cold and heat, \\
And yard, gatehouse; \\
Assembly, and water-boiling shed, \\
And conduct in the restrooms. 

Drinking\marginnote{12.11.262} water, washing water, \\
And pot for ablutions; \\
Laid down by the Incomparable one, \\
These make up the conduct for newly-arrived monks. 

No\marginnote{12.11.266} seat, no water, \\
No going to meet, and no drinking water; \\
Would not bow down, would not assign, \\
And the good monks complained. 

Senior,\marginnote{12.11.270} and seat, water, \\
And having gone to meet, drinking water; \\
Sandals, aside, \\
And should bow down, should assign. 

Occupied,\marginnote{12.11.274} and where to go for alms, training, \\
Place, drinking water, washing water; \\
Walking stick, agreement, right time, \\
Remain seated for one who is junior. 

Should\marginnote{12.11.278} bow down, should point out, \\
The same method as above; \\
Declared by the Caravan Leader, \\
These make up the conduct for resident monks. 

Departing,\marginnote{12.11.282} and wood and ceramic, \\
Leaving open, no informing; \\
And they were lost, and unprotected, \\
And the good monks complained, 

Having\marginnote{12.11.286} put in order, having closed, \\
Having informed, he should depart; \\
A monk or a novice, \\
A monastery worker, a lay follower. 

And\marginnote{12.11.290} a pile on rocks, \\
He should put away, and he should close; \\
If he is able, effort, \\
And just so in a dry spot. 

The\marginnote{12.11.294} whole gets wet, village, \\
And just so in the open; \\
Hopefully the requisites will be okay, \\
The proper conduct for a departing monk. 

They\marginnote{12.11.298} did not express their appreciation, by the most senior, \\
Left behind, by four or five; \\
Needing to defecate, he fainted, \\
These make up the conduct for the expression of appreciation. 

The\marginnote{12.11.302} group of six were shabbily dressed, \\
And then also badly dressed; \\
And improper appearance, short cut, \\
In encroaching on the senior monks. 

And\marginnote{12.11.306} the junior monks, upper robe, \\
And the good monks complained; \\
Putting on the sarong while covering the navel and knees, \\
Belt, putting together, toggle. 

No\marginnote{12.11.310} short cut, covered, \\
Well-restrained, lowered eyes; \\
Lifted, laugh loudly, noisy, \\
And three on swaying. 

Hands\marginnote{12.11.314} on hips, covering the head, squatting on the heels, \\
Covered, well-restrained; \\
Lowered, lifted, laugh loudly, \\
Little noise, three on swaying. 

And\marginnote{12.11.318} hands on hips, covering the head, clasping the knees, \\
Encroaching, no seat; \\
Spread out, water, \\
Holding low, poured. 

Receiving,\marginnote{12.11.322} nearby, upper robe, \\
And rice, he should receive; \\
Curry, with special curry, \\
For everyone, and even level. 

Respectfully,\marginnote{12.11.326} and attention on the bowl, \\
And in order, bean curry; \\
Not from a heap, should cover, \\
Asking, finding fault. 

Large,\marginnote{12.11.330} rounded, mouth, \\
The whole hand, he should not speak; \\
Lifted, breaking up, cheeks, \\
Shaking, scattering rice. 

And\marginnote{12.11.334} sticking out his tongue, \\
Chomping, slurping; \\
Licking the hand, the bowl, the lips, \\
Soiled with food, should receive. 

Not\marginnote{12.11.338} until everyone, water, \\
Holding low, poured; \\
Receiving, nearby, upper robe, \\
Holding low, and on the ground. 

Containing\marginnote{12.11.342} rice, returning, \\
Well-covered, squatting on the heels; \\
Laid down by the King of the Teaching, \\
This is the proper conduct in regard to dining halls. 

Shabbily\marginnote{12.11.346} dressed, improper in appearance, \\
And non-attentive, hasty; \\
Far, too near, long, soon, \\
Just so the alms collector. 

He\marginnote{12.11.350} should go covered, \\
Well-restrained, lowered eyes; \\
Lifted, laugh loudly, without noise, \\
And three on swaying. 

Hands\marginnote{12.11.354} on hips, covering the head, squatting on the heels, \\
And attentive, hastily; \\
Far, too near, long, soon, \\
Seat, serving spoon. 

Or\marginnote{12.11.358} a vessel, and make wait, \\
Having lifted up, having stretched out; \\
He should receive, he should not look, \\
And just so for curry. 

The\marginnote{12.11.362} monk should cover with the upper robe, \\
He should go covered; \\
And well-restrained, lowered eyes, \\
And lifted, laugh loudly; \\
Not noisy, three swaying, \\
Hands on hips, covering the head, squatting on the heels. 

First,\marginnote{12.11.368} seat, leftovers, \\
Drinking water, washing water; \\
The last may eat if he desires, \\
Should discard, should put away. 

Should\marginnote{12.11.372} put away, should sweep, \\
Empty, hollow, he should fill; \\
Hand signal, he should not break into speech, \\
This is the conduct for the alms collector. 

Drinking\marginnote{12.11.376} water, washing water, fire, fire-making implements, \\
Constellations, regions, and criminals; \\
“None of it is,” having beaten up, \\
Bowl, on the shoulder, so robe. 

Now,\marginnote{12.11.380} hanging on the shoulder, \\
Covering navel and knees, evenly all around; \\
As the conduct for the alms collector, \\
So the method for those staying in the wilderness. 

Bowl,\marginnote{12.11.384} on the shoulder, robe, on the head, \\
And having put on, drinking water; \\
Washing water, fire, \\
And also fire-making implements, walking stick. 

Constellations,\marginnote{12.11.388} or a portion, \\
Should be skilled also in the regions; \\
Laid down by the Supreme Teacher,\footnote{Reading \textit{satthuttamena} with the PTS edition. } \\
These make up the conduct for those staying in the wilderness. 

Outside,\marginnote{12.11.392} they were covered, \\
And the good monks complained; \\
If the dwelling is dirty, \\
First the bowl and robe. 

Mattress,\marginnote{12.11.396} pillow, bed, \\
Bench, spittoon; \\
Leaning board, ceiling cloth, corners, \\
Red ocher, black, untreated. 

Trash,\marginnote{12.11.400} near monks, \\
Furniture, dwelling, drinking water; \\
Near washing water, \\
And in an open area upwind. 

Downwind,\marginnote{12.11.404} cover, \\
And supports, bed; \\
Bench, mattress, sitting mat, \\
Spittoon, and leaning board. 

Bowl,\marginnote{12.11.408} robe, and floor, \\
Ends far, folds near; \\
East, and west, \\
North, then south. 

And\marginnote{12.11.412} day and night in the cold and heat, \\
And yard, gatehouse; \\
Assembly, and water-boiling shed, \\
And restroom, drinking water. 

Restroom\marginnote{12.11.416} ablutions pot, and a senior monk, \\
Recitation, question, rehearse; \\
Teaching, should turn off a lamp, \\
Should not open, and also not close. 

Turning\marginnote{12.11.420} around according to seniority, \\
Should not touch even with a corner; \\
Laid down by the Great Hero, \\
That is the proper conduct in regard to dwellings. 

Were\marginnote{12.11.424} being kept out, door, \\
Fainted, the good monks complained; \\
He should discard the ashes, sauna, \\
And just so the area outside. 

Yard,\marginnote{12.11.428} gatehouse, sauna shed, \\
Bath powder, clay, trough; \\
Face, in front, not the senior monks, \\
Not the junior monks, if he is able. 

In\marginnote{12.11.432} front, upstream, way, \\
Muddy, clay, bench; \\
Having extinguished, and having closed, \\
These make up the conduct in regard to saunas. 

He\marginnote{12.11.436} did not wash, according to seniority, \\
And order, hastily; \\
Pulled up, groaning, tooth cleaner, \\
Feces, urine, spit. 

Coarse,\marginnote{12.11.440} cesspit, hastily, \\
And pulled up, chomping, with remainder; \\
Outside, and inside, should clear his throat, \\
Clothesline, and without hurry. 

Hastily,\marginnote{12.11.444} pulled up, when standing, \\
Groaning, tooth cleaner, and feces; \\
Urine, spit, coarse, \\
Cesspit, foot stands for defecating. 

Not\marginnote{12.11.448} too hastily, pulled up, \\
Foot stands, chomping; \\
He should not leave, he should cover, \\
Stained, and with container. 

Restroom,\marginnote{12.11.452}  area outside, \\
And yard, gatehouse; \\
And water for ablutions, \\
These make up the conduct in regard to restrooms. 

Sandals,\marginnote{12.11.456} tooth cleaner, \\
And water for rinsing the mouth, seat; \\
Congee, water, having washed, \\
Put away, dirty, and village. 

Sarong,\marginnote{12.11.460} belt, \\
Putting together, wet bowl; \\
Attendant, and the navel and the knees, \\
Evenly all around, belt. 

Putting\marginnote{12.11.464} together, rinsed, attendant, \\
Not too far, he should receive; \\
When speaking, offense, \\
Go first, seat. 

Water,\marginnote{12.11.468} stool, scraper, \\
Having gone to meet, sarong; \\
He should sun, he left it, fold, \\
In the fold, to eat, should give. 

Drinking\marginnote{12.11.472} water, water, low, \\
A short while, and he should not leave it; \\
Bowl and robe, and floor, \\
Ends far, folds near. 

He\marginnote{12.11.476} should put away, and he should put away, \\
And dirty, to bathe; \\
Cold, hot, sauna, \\
Bath powder, clay, behind. 

And\marginnote{12.11.480} bench, robe, bath powder, \\
Clay, he is able, face; \\
Front, the seniors, and the juniors, \\
And provide assistance, he should leave. 

Front,\marginnote{12.11.484} in the water, when he has bathed, \\
Having put on the sarong, preceptor; \\
And the sarong, upper robe, \\
Bench, and with a seat. 

Foot,\marginnote{12.11.488} stool, and scraper, \\
Drinking water, reciting, questioning; \\
Dirty, he should clean it well, \\
First the bowl and robe. 

Sitting\marginnote{12.11.492} mat and sheet, \\
Mattress, and pillow; \\
Bed, bench, support, \\
Spittoon, and leaning board. 

Floor,\marginnote{12.11.496} cobweb, window, \\
Red ocher, black, untreated; \\
Floor cover, supports, \\
Bed, bench, pillow. 

Sitting\marginnote{12.11.500} mat, sheet, spittoon, \\
Leaning board, bowl and robe; \\
From the east, and from the west, \\
From the north, then from the south. 

And\marginnote{12.11.504} day and night in the cold and heat, \\
And yard, gatehouse; \\
Assembly, and water-boiling shed, \\
Restroom, water for drinking, water for washing. 

Ablutions,\marginnote{12.11.508} discontent, \\
Anxious, and view, heavy; \\
Beginning, trial period, rehabilitation, \\
Condemnation, demotion. 

Banishment,\marginnote{12.11.512} reconciliation, \\
And ejection, or done; \\
He should wash, should make, and dye, \\
He should dye, turning over. 

And\marginnote{12.11.516} bowl, and also robe, \\
And requisite, cutting; \\
Provide assistance, service, \\
Attendant, alms, entering. 

Charnel\marginnote{12.11.520} ground, and regions, \\
He should nurse for as long as he lives; \\
This is for a student, \\
These make up the conduct for a preceptor. 

Instruction,\marginnote{12.11.524} teaching, recitation, \\
Questioning, and bowl, robe; \\
Requisite, and sick, \\
He should not be the attendant. 

This\marginnote{12.11.528} conduct toward preceptors, \\
Thus too toward teachers; \\
The conduct toward a student, \\
Just so toward a pupil. 

The\marginnote{12.11.532} conduct concerning those newly arrived, \\
And again concerning the residents; \\
Those departing, and those expressing appreciation, \\
About the dining hall, about the alms collector. 

The\marginnote{12.11.536} conduct for those staying in the wilderness, \\
And also concerning dwellings; \\
About the sauna, restroom, \\
Preceptors, toward a student. 

The\marginnote{12.11.540} conduct toward teachers, \\
Just so toward a pupil; \\
Nineteen topics,\footnote{It is not clear what the nineteen topics are and how they differ from the fourteen main headings mentioned just below. It is possible that the number nineteen comprises the fourteen sections plus some of the minor rules found in sections four, nine, and ten. } \\
Fourteen on proper conduct in this chapter. 

If\marginnote{12.11.544} you do not fulfill the proper conduct, \\
Then you do not fulfill your virtue; \\
Impure in virtue, weak in wisdom, \\
You do not know the unity of mind. 

A\marginnote{12.11.548} mind distracted, not unified, \\
Does not see the teaching rightly; \\
Not seeing the true teaching, \\
You are not released from suffering. 

But\marginnote{12.11.552} if you do fulfill the proper conduct, \\
Then you also fulfill your virtue; \\
Pure in virtue, possessed of wisdom, \\
You also know the unity of mind. 

A\marginnote{12.11.556} non-distracted mind, unified, \\
Sees the teaching rightly; \\
Seeing the true teaching, \\
You are released from suffering. 

So,\marginnote{12.11.560} fulfill the proper conduct, \\
You the Son of the Victor, possessed of insight; \\
The instruction of the Buddha, the best—\\
Go from that to extinguishment, in this way.” 

%
\end{scuddana}

\scendsutta{The chapter on proper conduct is finished. }

%
\chapter*{{\suttatitleacronym Kd 19}{\suttatitletranslation The chapter on the cancellation of the Monastic Code }{\suttatitleroot Pātimokkhaṭṭhapanakkhandhaka}}
\addcontentsline{toc}{chapter}{\tocacronym{Kd 19} \toctranslation{The chapter on the cancellation of the Monastic Code } \tocroot{Pātimokkhaṭṭhapanakkhandhaka}}
\markboth{The chapter on the cancellation of the Monastic Code }{Pātimokkhaṭṭhapanakkhandhaka}
\extramarks{Kd 19}{Kd 19}

\section*{1. The request for the recitation of the Monastic Code }

At\marginnote{1.1.1} one time on the observance day, when the Buddha was staying at \textsanskrit{Sāvatthī} in the Eastern Monastery, in \textsanskrit{Migāramāta}’s stilt house, he was seated surrounded by the Sangha of monks. Then, when the night was well advanced and the first part of the night was over, Venerable Ānanda got up from his seat, arranged his upper robe over one shoulder, raise his joined palms, and said to the Buddha, “Sir, the night is well advanced and the first part of the night is over. The Sangha of monks has been seated for a long time. Please recite the Monastic Code.” The Buddha did not reply. 

At\marginnote{1.1.7} the end of the middle part of the night, Venerable Ānanda asked a second time, and again received no reply. At the end of the last part of the night, when the sky was flaring up at dawn, he asked a third time. And the Buddha replied, “Ānanda, the gathering is impure.” 

Venerable\marginnote{1.2.1} \textsanskrit{Mahāmoggallāna} thought, “Who is the Buddha talking about?” Reading the minds of the the entire Sangha of monks, he saw that person—immoral, with bad qualities, impure and dubious in conduct, hiding his actions, not a monastic while claiming to be one, not abstaining from sexuality while claiming to do so, rotten inside, lustful, defiled—seated in the midst of the Sangha. He went up to him and said, “Get up, the Buddha has seen you. You don’t belong with the community of monks.” But he did not reply. 

\textsanskrit{Mahāmoggallāna}\marginnote{1.2.10} said the same thing a second and a third time, still not getting a reply. \textsanskrit{Mahāmoggallāna} then grabbed him by the arms, took him outside the gatehouse, and fastened the bolt and the latch. He then went to the Buddha and said, “Sir, I’ve taken that person outside; the gathering is pure. Please recite the Monastic Code to the monks.” 

“It’s\marginnote{1.2.22} amazing, \textsanskrit{Moggallāna}, how that fool waited until he was grabbed by the arms.” And the Buddha addressed the monks: 

\section*{2. The eight amazing qualities of the ocean }

“Monks,\marginnote{1.3.1} the antigods delight in the ocean because they see eight amazing qualities in it:\footnote{Parallel to \href{https://suttacentral.net/an8.19/en/brahmali\#2.4}{AN 8.19:2.4}. } 

The\marginnote{1.3.3} ocean slopes and inclines gradually. It doesn’t drop off all at once. 

The\marginnote{1.3.6} ocean is steady. It doesn’t go beyond the shoreline. 

The\marginnote{1.3.9} ocean doesn’t tolerate dead bodies, but quickly carries them to the shore and dumps them on dry land. 

When\marginnote{1.3.12} the great rivers—the Ganges, the \textsanskrit{Yamunā}, the \textsanskrit{Aciravatī}, the \textsanskrit{Sarabhū}, or the \textsanskrit{Mahī}—reach the ocean, they lose their former names and become known simply as the ocean. 

Whatever\marginnote{1.3.15} rivers in the world flow into the ocean and whatever rain falls into it, the ocean isn’t diminished or filled up because of that. 

The\marginnote{1.3.18} ocean has only one taste, the taste of salt. 

The\marginnote{1.3.21} ocean contains many precious things—pearls, gems, beryls, mother-of-pearls, quartz, corals, silver, gold, rubies, and cat’s eyes. 

There\marginnote{1.3.24} are great beings in the ocean—sea monsters, antigods, dragons, and fairies; creatures with bodies one thousand kilometers long, two thousand, three thousand, four thousand, and five thousand kilometers long.\footnote{Sea monsters is a combined rendering of the three terms \textit{timi}, \textit{\textsanskrit{timiṅgala}}, and \textit{\textsanskrit{timitimiṅgala}}. They refer to enormous fish of different sizes. The creatures mentioned, the \textit{\textsanskrit{attabhāvā}}—which presumably include the sea monsters—have huge bodies, respectively 100, 200, 300, 400, and 500 \textit{yojanas} long. I normally take a \textit{yojana} to be approximately 13 kilometers, but here use 10 kilometers instead, so as to give round figures. See discussion of the \textit{yojana} under \textit{sugata} in Appendix of Technical Terms. } 

\section*{3. The eight amazing qualities of this spiritual path }

“Just\marginnote{1.4.1} so, the monks delight in this spiritual path because they see eight amazing qualities in it:\footnote{For an explanation of the rendering “spiritual path” for \textit{dhammavinaya}, see Appendix of Technical Terms. } 

Just\marginnote{1.4.3} as the ocean slopes and inclines gradually, and doesn’t drop off all at once, so too, on this spiritual path, the training is gradual, the practice is gradual, and penetration to perfect insight doesn’t happen all at once. 

Just\marginnote{1.4.7} as the ocean is steady and doesn’t go beyond the shoreline, so too, on this spiritual path, my disciples don’t transgress the training rules I’ve laid down, even for the sake of life. 

Just\marginnote{1.4.11} as the ocean doesn’t tolerate dead bodies, but quickly carries them to the shore and dumps them on dry land, so too, the Sangha of monks doesn’t associate with anyone who is immoral—someone with bad qualities, impure and dubious in conduct, hiding his actions, not a monastic while claiming to be one, not abstaining from sexuality while claiming to do so, rotten inside, lustful, defiled. When the Sangha has gathered, they quickly eject him. Even if seated in the midst of the Sangha, he’s far from the Sangha and the Sangha is far from him. 

Just\marginnote{1.4.15} as when the great rivers—the Ganges, the \textsanskrit{Yamunā}, the \textsanskrit{Aciravatī}, the \textsanskrit{Sarabhū}, or the \textsanskrit{Mahī}—reach the ocean, they lose their former names and become known simply as the ocean, so too, when anyone goes forth on this spiritual path proclaimed by the Buddha—whether an aristocrat, brahmin, merchant, or worker—they lose their former name and class and become known simply as a Sakyan monastic. 

Just\marginnote{1.4.19} as the ocean doesn’t decrease or fill up because of all the rivers in the world that flow into it or the rain that falls into it, so too, even if many monks are extinguished without remainder, there’s no decrease or filling up of the element of extinguishment. 

Just\marginnote{1.4.23} as the ocean has only one taste, the taste of salt, so too, this spiritual path has only one taste, the taste of freedom. 

Just\marginnote{1.4.27} as the ocean contains many precious things—pearls, gems, beryls, mother-of-pearls, quartz, corals, silver, gold, rubies, and cat’s eyes—so too, this spiritual path contains many precious things—the four applications of mindfulness, the four right efforts, the four foundations for supernormal power, the five spiritual faculties, the five spiritual powers, the seven factors of awakening, and the noble eightfold path. 

Just\marginnote{1.4.31} as there are great beings in the ocean—sea monsters, antigods, dragons, and fairies; creatures with bodies one thousand kilometers long, two thousand, three thousand, four thousand, and five thousand kilometers long—\footnote{For a discussion of the length of the \textit{yojana}, see \textit{sugata} in Appendix of Technical Terms. } so too, there are great beings on this spiritual path—stream-enterers, those practicing for the realization of the fruit of stream-entry; once-returners, those practicing for the realization of the fruit of once-returning; non-returners, those practicing for the realization of the fruit of non-returning; perfected ones, and those practicing for the realization of the fruit of perfection.” 

Seeing\marginnote{1.4.36} the significance of this, on that occasion the Buddha uttered a heartfelt exclamation: 

\begin{verse}%
“It\marginnote{1.4.37} rains on what’s concealed,\footnote{Parallel to \href{https://suttacentral.net/ud5.5/en/brahmali\#28.1}{Ud 5.5:28.1}. } \\
Not on what’s revealed. \\
Therefore, reveal the concealed, \\
And it won’t be rained upon.” 

%
\end{verse}

\section*{4. One deserving to hear the Monastic Code }

The\marginnote{2.1.1} Buddha addressed the monks: “From now on, monks, I won’t be doing the observance-day ceremony or recite the Monastic Code. You should do it instead. It’s impossible for the Buddha to do the observance-day ceremony and recite the Monastic Code in an impure gathering. 

\scrule{And, monks, you shouldn’t listen to the Monastic Code if you have an unconfessed offense. If you do, you commit an offense of wrong conduct. }

\scrule{If anyone who has an unconfessed offense listens to the Monastic Code, I allow you to cancel his hearing of the Monastic Code.\footnote{The Pali just says \textit{tassa \textsanskrit{pātimokkhaṁ} \textsanskrit{ṭhapetuṁ}}, “(I allow you) to cancel his Monastic Code”. I have added “hearing” for clarity. } }

And\marginnote{2.1.8} it should be done like this. On the observance day, whether the fourteenth or the fifteenth, in the midst of the Sangha and in the presence of that person, you should announce: 

‘Please,\marginnote{2.1.10} venerables, I ask the Sangha to listen. The person so-and-so has an unconfessed offense. I cancel his hearing of the Monastic Code. The Monastic Code shouldn’t be recited in his presence.’ Then his hearing of the Monastic Code is canceled.” 

Soon\marginnote{3.1.1} afterwards, thinking that nobody knew about them, the monks from the group of six listened to the Monastic Code while having unconfessed offenses. The senior monks who could read the minds of others informed the monks about this. When they heard about it, the monks from the group of six tried to pre-empt the pure monks by, without reason, canceling their hearing of the Monastic Code. 

The\marginnote{3.1.13} monks of few desires complained and criticized them, “How could the monks from the group of six cancel the Monastic Code of pure monks without reason?” They told the Buddha. Soon afterwards he had the Sangha gathered and questioned the monks: 

“Is\marginnote{3.1.16} it true, monks, that the monks from the group of six did this?” “It’s true, sir.” … After rebuking them … the Buddha gave a teaching and addressed the monks: 

\scrule{“You shouldn’t, without reason, cancel the Monastic Code of pure monks who don’t have any offenses. If you do, you commit an offense of wrong conduct.” }

\section*{5. Legitimate and illegitimate canceling of the Monastic Code }

“One\marginnote{3.2.1} kind of canceling of the Monastic Code is illegitimate, one is legitimate; two kinds of cancelings of the Monastic Code are illegitimate, two are legitimate; three kinds of cancelings of the Monastic Code are illegitimate, three are legitimate; four kinds of cancelings of the Monastic Code are illegitimate, four are legitimate; five kinds of cancelings of the Monastic Code are illegitimate, five are legitimate; six kinds of cancelings of the Monastic Code are illegitimate, six are legitimate; seven kinds of cancelings of the Monastic Code are illegitimate, seven are legitimate; eight kinds of cancelings of the Monastic Code are illegitimate, eight are legitimate; nine kinds of cancelings of the Monastic Code are illegitimate, nine are legitimate; ten kinds of cancelings of the Monastic Code are illegitimate, ten are legitimate. 

What’s\marginnote{3.3.1} the one kind of canceling of the Monastic Code that’s illegitimate? One cancels the Monastic Code, without grounds, for failure in morality. 

“What’s\marginnote{3.3.4} the one kind of canceling of the Monastic Code that’s legitimate? One cancels the Monastic Code, having grounds, for failure in morality. 

“What\marginnote{3.3.7} are the two kinds of cancelings of the Monastic Code that are illegitimate? One cancels the Monastic Code, without grounds, for failure in morality or failure in conduct. 

What\marginnote{3.3.10} are the two kinds of cancelings of the Monastic Code that are legitimate? One cancels the Monastic Code, having grounds, for failure in morality or failure in conduct. 

“What\marginnote{3.3.13} are the three kinds of cancelings of the Monastic Code that are illegitimate? One cancels the Monastic Code, without grounds, for failure in morality, failure in conduct, or failure in view. 

What\marginnote{3.3.16} are the three kinds of cancelings of the Monastic Code that are legitimate? One cancels the Monastic Code, having grounds, for failure in morality, failure in conduct, or failure in view. 

“What\marginnote{3.3.19} are the four kinds of cancelings of the Monastic Code that are illegitimate? One cancels the Monastic Code, without grounds, for failure in morality, failure in conduct, failure in view, or failure in livelihood. 

What\marginnote{3.3.22} are the four kinds of cancelings of the Monastic Code that are legitimate? One cancels the Monastic Code, having grounds, for failure in morality, failure in conduct, failure in view, or failure in livelihood. 

“What\marginnote{3.3.25} are the five kinds of cancelings of the Monastic Code that are illegitimate? One cancels the Monastic Code, without grounds, for an offense entailing expulsion, an offense entailing suspension, an offense entailing confession, an offense entailing acknowledgment, or an offense of wrong conduct. 

What\marginnote{3.3.31} are the five kinds of cancelings of the Monastic Code that are legitimate? One cancels the Monastic Code, having grounds, for an offense entailing expulsion, an offense entailing suspension, an offense entailing confession, an offense entailing acknowledgment, or an offense of wrong conduct. 

“What\marginnote{3.3.37} are the six kinds of cancelings of the Monastic Code that are illegitimate? One cancels the Monastic Code, without grounds, for the failure in morality of one who hasn’t failed, the failure in morality of one who has failed,\footnote{Sp 4.387: \textit{\textsanskrit{Amūlikāya} \textsanskrit{sīlavipattiyā} \textsanskrit{pātimokkhaṁ} \textsanskrit{ṭhapeti} \textsanskrit{akatāyāti} tena puggalena \textsanskrit{sā} vipatti \textsanskrit{katā} \textsanskrit{vā} hotu \textsanskrit{akatā} \textsanskrit{vā}, \textsanskrit{pātimokkhaṭṭhapanakassa} \textsanskrit{saññāamūlikavasena} \textsanskrit{amūlikā} hoti}, “\textit{\textsanskrit{Amūlikāya} \textsanskrit{sīlavipattiyā} \textsanskrit{pātimokkhaṁ} \textsanskrit{ṭhapeti} \textsanskrit{akatāya}} means, whether that person has failed or not, it is groundless because of the perception of groundlessness of the person who has canceled the Monastic Code.” } the failure in conduct of one who hasn’t failed, the failure in conduct of one who has failed, the failure in view of one who hasn’t failed, or the failure in view of one who has failed. 

What\marginnote{3.3.42} are the six kinds of cancelings of the Monastic Code that are legitimate? One cancels the Monastic Code, having grounds, for the failure in morality of one who hasn’t failed, the failure in morality of one who has failed, the failure in conduct of one who hasn’t failed, the failure in conduct of one who has failed, the failure in view of one who hasn’t failed, or the failure in view of one who has failed. 

“What\marginnote{3.3.48} are the seven kinds of cancelings of the Monastic Code that are illegitimate? One cancels the Monastic Code, without grounds, for an offense entailing expulsion, an offense entailing suspension, a serious offense, an offense entailing confession, an offense entailing acknowledgment, an offense of wrong conduct, or an offense of wrong speech. 

What\marginnote{3.3.52} are the seven kinds of cancelings of the Monastic Code that are legitimate? One cancels the Monastic Code, having grounds, for an offense entailing expulsion, an offense entailing suspension, a serious offense, an offense entailing confession, an offense entailing acknowledgment, an offense of wrong conduct, or an offense of wrong speech. 

“What\marginnote{3.3.56} are the eight kinds of cancelings of the Monastic Code that are illegitimate? One cancels the Monastic Code, without grounds, for the failure in morality of one who hasn’t failed, the failure in morality of one who has failed, for the failure in conduct of one who hasn’t failed, for the failure in conduct of one who has failed, the failure in view of one who hasn’t failed, the failure in view of one who has failed, the failure in livelihood of one who hasn’t failed, or the failure in livelihood of one who has failed. 

What\marginnote{3.3.62} are the eight kinds of cancelings of the Monastic Code that are legitimate? One cancels the Monastic Code, having grounds, for the failure in morality of one who hasn’t failed, the failure in morality of one who has failed, the failure in conduct of one who hasn’t failed, the failure in conduct of one who has failed, the failure in view of one who hasn’t failed, the failure in view of one who has failed, the failure in livelihood of one who hasn’t failed, or the failure in livelihood of one who has failed. 

“What\marginnote{3.3.68} are the nine kinds of cancelings of the Monastic Code that are illegitimate? One cancels the Monastic Code, without grounds, for the failure in morality of one who hasn’t failed, the failure in morality of one who has failed, the failure in morality of one who both has and hasn’t failed, the failure in conduct of one who hasn’t failed, the failure in conduct of one who has failed, the failure in conduct of one who both has and hasn’t failed, the failure in view of one who hasn’t failed, the failure in view of one who has failed, or the failure in view of one who both has and hasn’t failed. 

What\marginnote{3.3.74} are the nine kinds of cancelings of the Monastic Code that are legitimate? One cancels the Monastic Code, having grounds, for the failure in morality of one who hasn’t failed, the failure in morality of one who has failed, the failure in morality of one who both has and hasn’t failed, the failure in conduct of one who hasn’t failed, the failure in conduct of one who has failed, the failure in conduct of one who both has and hasn’t failed, the failure in view of one who hasn’t failed, the failure in view of one who has failed, or the failure in view of one who both has and hasn’t failed. 

“What\marginnote{3.3.80} are the ten kinds of cancelings of the Monastic Code that are illegitimate? There’s no-one seated in that gathering who has committed an offense entailing expulsion; there’s no unfinished discussion about anyone committing an offense entailing expulsion; there’s no-one seated in that gathering who has renounced the training; there’s no unfinished discussion about anyone renouncing the training; the person has gone to a legitimate legal procedure of a complete assembly; the person doesn’t reopen a legitimate legal procedure of a complete assembly; there’s no unfinished discussion about the reopening of the legitimate legal procedure of a complete assembly;\footnote{The point seems to be that it is illegitimate to cancel the Monastic Code of someone who is behaving appropriately, including taking part in legal procedures in the appropriate manner and not reopening legitimate legal procedures. See discussion in the next note. } there’s no-one seen, heard, or suspected of failure in morality; there’s no-one seen, heard, or suspected of failure in conduct; there’s no-one seen, heard, or suspected of failure in view. 

What\marginnote{3.3.86} are the ten kinds of cancelings of the Monastic Code that are legitimate? There’s someone seated in that gathering who has committed an offense entailing expulsion; there’s an unfinished discussion about someone committing an offense entailing expulsion; there’s someone seated in that gathering who has renounced the training; there’s an unfinished discussion about someone renouncing the training; the person hasn’t gone to a legitimate legal procedure of a complete assembly; the person reopens a legitimate legal procedure of a complete assembly; there’s an unfinished discussion about the reopening of the legitimate legal procedure of a complete assembly;\footnote{Sp 4.387: \textit{\textsanskrit{Dhammikaṁ} \textsanskrit{sāmaggiṁ} na \textsanskrit{upetīti} \textsanskrit{kammaṁ} \textsanskrit{kopetukāmatāya} \textsanskrit{saṅghassa} kamme \textsanskrit{karīyamāne} neva \textsanskrit{āgacchati}, na \textsanskrit{chandaṁ} deti, \textsanskrit{sammukhībhūtova} \textsanskrit{paṭikkosati}, tena \textsanskrit{dukkaṭaṁ} \textsanskrit{āpajjati}. \textsanskrit{Iccassāpi} \textsanskrit{sāpattikasseva} \textsanskrit{pātimokkhaṁ} \textsanskrit{ṭhapitaṁ} hoti}, “\textit{\textsanskrit{Dhammikaṁ} \textsanskrit{sāmaggiṁ} na upeti} means, when a legal procedure is being carried out by the Sangha, one who wishes to invalidate the procedure does not come (to the meeting), or does not give their consent, or objects in the presence (of the assembled Sangha). They commit wrong conduct. Thus, because they have an offense, the Monastic Code gets canceled.” Vmv 4.387 adds: \textit{\textsanskrit{Dhammikaṁ} \textsanskrit{sāmagginti} \textsanskrit{dhammikaṁ} \textsanskrit{samaggakammaṁ}}, “\textit{\textsanskrit{Dhammikaṁ} \textsanskrit{sāmaggiṁ}} means a legitimate legal procedure of a complete assembly.” The point is that the person wants to invalidate a legal procedure, and so deliberately does not come to the meeting or send their consent. In such a case you are allowed to cancel their hearing of the Monastic Code. Sp 4.387: \textit{\textsanskrit{Paccādiyatīti} “puna \textsanskrit{kātabbaṁ} kamma”nti \textsanskrit{paccādiyati}, tena \textsanskrit{ukkoṭanakena} \textsanskrit{pācittiyaṁ} \textsanskrit{āpajjati}. \textsanskrit{Iccassāpi} \textsanskrit{sāpattikasseva} \textsanskrit{pātimokkhaṁ} \textsanskrit{ṭhapitaṁ} hoti}, “\textit{\textsanskrit{Paccādiyati}} means saying, ‘The legal procedure needs to be redone.’ Because of that reopening, they commit an offense entailing confession. Thus, because they have an offense, the Monastic Code gets canceled.” } there’s someone seen, heard, or suspected of failure in morality; there’s someone seen, heard, or suspected of failure in conduct; there’s someone seen, heard, or suspected of failure in view.” 

\section*{6. Legitimate canceling of the Monastic Code }

\subsection*{One who has committed an offense entailing expulsion}

“And\marginnote{3.4.1.1} how is one who has committed an offense entailing expulsion seated in that gathering? A monk sees in a monk the characteristics and signs of someone committing an offense entailing expulsion. Or a monk doesn’t see a monk committing an offense entailing expulsion, but another monk informs him that monk so-and-so has committed an offense entailing expulsion. Or a monk doesn’t see a monk committing an offense entailing expulsion, nor does another monk inform him, but the monk himself informs him that he has committed an offense entailing expulsion. Then, on the observance day, whether the fourteenth or the fifteenth, he may if he wishes, based on what he has seen, has heard, or suspects, announce in the midst of the Sangha and in the presence of that person: 

‘Please,\marginnote{3.4.9} venerables, I ask the Sangha to listen. So-and-so has committed an offense entailing expulsion. I cancel his hearing of the Monastic Code. The Monastic Code shouldn’t be recited in his presence.’ This cancellation of the Monastic Code is legitimate. 

When\marginnote{3.4.12} the Monastic Code has been canceled for a monk, it may be that the gathering breaks up because of any one of ten threats: a threat from kings, criminals, fire, flooding, people, spirits, predatory animals, snakes, or a threat to life, or threat to the monastic life. Then, if a monk wishes, he may, in that monastery or in another monastery, announce in the midst of the Sangha and in the presence of that person: 

‘Please,\marginnote{3.4.15} venerables, I ask the Sangha to listen. There’s an unfinished discussion about so-and-so committing an offense entailing expulsion. The case hasn’t been decided.\footnote{The point seems to be that, although the Monastic Code had been canceled for that monk, the further discussion on whether he had committed an offense entailing expulsion was interrupted by one of the ten dangers. } If the Sangha is ready, it should decide it.’ 

If\marginnote{3.4.18} this works out, all is well. If not, then on the observance day, whether the fourteenth or the fifteenth, in the midst of the Sangha and in the presence of that person, a monk should announce: 

‘Please,\marginnote{3.4.20} venerables, I ask the Sangha to listen. There’s an unfinished discussion about so-and-so committing an offense entailing expulsion. The case hasn’t been decided.\footnote{Given that his hearing of the Monastic Code is canceled once more, we may conclude that the canceling is only valid for a single occasion. } I cancel his hearing of the Monastic Code. The Monastic Code shouldn’t be recited in his presence.’ This cancellation of the Monastic Code is legitimate.” 

\subsection*{One who has renounced the training}

“And\marginnote{3.5.1} how is one who has renounced the training seated in that gathering? A monk sees in a monk the characteristics and signs of someone renouncing the training. Or a monk doesn’t see a monk renouncing the training, but another monk informs him that monk so-and-so has renounced the training. Or a monk doesn’t see a monk renouncing the training, nor does another monk inform him, but the monk himself informs him that he has renounced the training. Then, on the observance day, whether the fourteenth or the fifteenth, he may if he wishes, based on what he has seen, has heard, or suspects, announce in the midst of the Sangha and in the presence of that person: 

‘Please,\marginnote{3.5.9} venerables, I ask the Sangha to listen. So-and-so has renounced the training. I cancel his hearing of the Monastic Code. The Monastic Code shouldn’t be recited in his presence.’ This cancellation of the Monastic Code is legitimate. 

When\marginnote{3.5.12} the Monastic Code has been canceled for a monk, it may be that the gathering breaks up because of any one of ten threats: a threat from kings, criminals, fire, flooding, people, spirits, predatory animals, snakes, or a threat to life, or a threat to the monastic life. Then, if a monk wishes, he may, in that monastery or in another monastery, announce in the midst of the Sangha and in the presence of that person: 

‘Please,\marginnote{3.5.15} venerables, I ask the Sangha to listen. There’s an unfinished discussion about so-and-so renouncing the training. The case hasn’t been decided. If the Sangha is ready, it should decide it.’ 

If\marginnote{3.5.18} this works out, all is well. If not, then on the observance day, whether the fourteenth or the fifteenth, in the midst of the Sangha and in the presence of that person, a monk should announce: 

‘Please,\marginnote{3.5.20} venerables, I ask the Sangha to listen. There’s an unfinished discussion about so-and-so renouncing the training. The case hasn’t been decided. I cancel his hearing of the Monastic Code. The Monastic Code shouldn’t be recited in his presence.’ This cancellation of the Monastic Code is legitimate.” 

\subsection*{One who reopens a legitimate legal procedure}

“And\marginnote{3.6.1} how does he not go to a legitimate legal procedure of a complete assembly? A monk sees in a monk the characteristics and signs of someone who doesn’t go to a legitimate legal procedure of a complete assembly. Or a monk doesn’t see a monk not going to a legitimate legal procedure of a complete assembly, but another monk informs him that monk so-and-so didn’t go to a legitimate legal procedure of a complete assembly. Or a monk doesn’t see a monk not going to a legitimate legal procedure of a complete assembly, nor does another monk inform him, but the monk himself informs him that he didn’t go to a legitimate legal procedure of a complete assembly. Then, on the observance day, whether the fourteenth or the fifteenth, he may if he wishes, based on what he has seen, has heard, or suspects, announce in the midst of the Sangha and in the presence of that person: 

‘Please,\marginnote{3.6.9} venerables, I ask the Sangha to listen. So-and-so didn’t go to a legitimate legal procedure of a complete assembly. I cancel his hearing of the Monastic Code. The Monastic Code shouldn’t be recited in his presence.’ This cancellation of the Monastic Code is legitimate.” 

“And\marginnote{3.7.1} how does he reopen a legitimate legal procedure of a complete assembly? A monk sees in a monk the characteristics and signs of someone who reopens a legitimate legal procedure of a complete assembly. Or a monk doesn’t see a monk reopening a legitimate legal procedure of a complete assembly, but another monk informs him that monk so-and-so has reopened a legitimate legal procedure of a complete assembly. Or a monk doesn’t see a monk reopening a legitimate legal procedure of a complete assembly, nor does another monk inform him, but the monk himself informs him that he has reopened a legitimate legal procedure of a complete assembly. Then, on the observance day, whether the fourteenth or the fifteenth, he may if he wishes, based on what he has seen, has heard, or suspects, announce in the midst of the Sangha and in the presence of that person: 

‘Please,\marginnote{3.7.6} venerables, I ask the Sangha to listen. So-and-so has reopened a legitimate legal procedure of a complete assembly. I cancel his hearing of the Monastic Code. The Monastic Code shouldn’t be recited in his presence.’ This cancellation of the Monastic Code is legitimate. 

When\marginnote{3.7.9} the Monastic Code has been canceled for a monk, it may be that the gathering breaks up because of any one of ten threats: a threat from kings, criminals, fire, flooding, people, spirits, predatory animals, snakes, or a threat to life, or a threat to the monastic life. Then, if a monk wishes, he may, in that monastery or in another monastery, announce in the midst of the Sangha and in the presence of that person: 

‘Please,\marginnote{3.7.13} venerables, I ask the Sangha to listen. There’s an unfinished discussion about so-and-so reopening a legitimate legal procedure of a complete assembly. The case hasn’t been decided. If the Sangha is ready, it should decide it.’ 

If\marginnote{3.7.16} this works out, all is well. If not, then on the observance day, whether the fourteenth or the fifteenth, in the midst of the Sangha and in the presence of that person, a monk should announce: 

‘Please,\marginnote{3.7.18} venerables, I ask the Sangha to listen. There’s an unfinished discussion about so-and-so reopening a legitimate legal procedure of a complete assembly. The case hasn’t been decided. I cancel his hearing of the Monastic Code. The Monastic Code shouldn’t be recited in his presence.’ This cancellation of the Monastic Code is legitimate.” 

\subsection*{Failure in morality, conduct, or view}

“And\marginnote{3.8.1} how is failure in morality seen, heard, or suspected? A monk sees in a monk the characteristics and signs of someone who has been seen, heard, or suspected of failure in morality. Or a monk doesn’t see a monk who has been seen, heard, or suspected of failure in morality, but another monk informs him that monk so-and-so has been seen, heard, or suspected of failure in morality. Or a monk doesn’t see a monk who has been seen, heard, or suspected of failure in morality, nor does another monk inform him, but the monk himself informs him that he has been seen, heard, or suspected of failure in morality. Then, on the observance day, whether the fourteenth or the fifteenth, he may if he wishes, based on what he has seen, has heard, or suspects, announce in the midst of the Sangha and in the presence of that person: 

‘Please,\marginnote{3.8.9} venerables, I ask the Sangha to listen. So-and-so has been seen, heard, or suspected of failure in morality. I cancel his hearing of the Monastic Code. The Monastic Code shouldn’t be recited in his presence.’ This cancellation of the Monastic Code is legitimate. 

And\marginnote{3.9.1} how is failure in conduct seen, heard, or suspected? A monk sees in a monk the characteristics and signs of someone who has been seen, heard, or suspected of failure in conduct. Or a monk doesn’t see a monk who has been seen, heard, or suspected of failure in conduct, but another monk informs him that monk so-and-so has been seen, heard, or suspected of failure in conduct. Or a monk doesn’t see a monk who has been seen, heard, or suspected of failure in conduct, nor does another monk inform him, but the monk himself informs him that he has been seen, heard, or suspected of failure in conduct. Then, on the observance day, whether the fourteenth or the fifteenth, he may if he wishes, based on what he has seen, has heard, or suspects, announce in the midst of the Sangha and in the presence of that person: 

‘Please,\marginnote{3.9.9} venerables, I ask the Sangha to listen. So-and-so has been seen, heard, or suspected of failure in conduct. I cancel his hearing of the Monastic Code. The Monastic Code shouldn’t be recited in his presence.’ This cancellation of the Monastic Code is legitimate. 

And\marginnote{3.9.12} how is failure in view seen, heard, or suspected? A monk sees in a monk the characteristics and signs of someone who has been seen, heard, or suspected of failure in view. Or a monk doesn’t see a monk who has been seen, heard, or suspected of failure in view, but another monk informs him that monk so-and-so has been seen, heard, or suspected of failure in view. Or a monk doesn’t see a monk who has been seen, heard, or suspected of failure in view, nor does another monk inform him, but the monk himself informs him that he has been seen, heard, or suspected of failure in view. Then, on the observance day, whether the fourteenth or the fifteenth, he may if he wishes, based on what he has seen, has heard, or suspects, announce in the midst of the Sangha and in the presence of that person: 

‘Please,\marginnote{3.9.20} venerables, I ask the Sangha to listen. So-and-so has been seen, heard, or suspected of failure in view. I cancel his hearing of the Monastic Code. The Monastic Code shouldn’t be recited in his presence.’ This cancellation of the Monastic Code is legitimate. 

These\marginnote{3.9.23} are the ten legitimate cancellations of the Monastic Code.” 

\scend{The first section for recitation is finished. }

\section*{7. The qualities needed to raise an issue }

Venerable\marginnote{4.1.1} \textsanskrit{Upāli} went to the Buddha, bowed, sat down, and said, “Sir, if a monk wishes to raise an issue, what factors should be fulfilled?”\footnote{Sp 4.398: \textit{\textsanskrit{Attādānaṁ} \textsanskrit{ādātukāmenāti} ettha \textsanskrit{sāsanaṁ} \textsanskrit{sodhetukāmo} bhikkhu \textsanskrit{yaṁ} \textsanskrit{adhikaraṇaṁ} \textsanskrit{attanā} \textsanskrit{ādiyati}, \textsanskrit{taṁ} \textsanskrit{attādānanti} vuccati}, “\textit{\textsanskrit{Attādānaṁ} \textsanskrit{ādātukāmena}}: here, whatever legal issue a monk who wishes to purify the dispensation brings up on his own accord, that is called \textit{\textsanskrit{attādāna}}.” } 

“Five\marginnote{4.1.4} factors should be fulfilled: 

\begin{enumerate}%
\item He should reflect whether it’s the right time to raise it. If he knows it’s the wrong time, he shouldn’t raise it. %
\item But if he knows it’s the right time, he should reflect further whether it’s a real issue. If he knows it’s not, he shouldn’t raise it. %
\item But if he knows it is, he should reflect further whether raising it will be beneficial. If he knows it won’t, he shouldn’t raise it. %
\item But if he knows it will, he should reflect further whether the monks who are on the side of the Teaching and the Monastic Law will support him. If he knows that they won’t, he shouldn’t raise it. %
\item But if he knows that they will, he should reflect further whether raising the issue will lead to arguments and disputes, to fracture and schism in the Sangha. If he knows it will, he shouldn’t raise it. %
\end{enumerate}

But\marginnote{4.1.29} if he knows it won’t, he may raise it. In this way, when five factors are fulfilled, he won’t regret raising that issue.” 

\section*{8. The qualities to be reflected on by one who accuses another }

“Sir,\marginnote{5.1.1} how many qualities should a monk see in himself before accusing another?” 

“He\marginnote{5.1.2} should see five qualities in himself. 

He\marginnote{5.1.3} should reflect, ‘Is my bodily conduct pure and flawless? Is this quality found in me or not?’ If it’s not, there will be those who say, ‘Please train your own bodily conduct first.’ 

He\marginnote{5.1.7} should reflect, ‘Is my verbal conduct pure and flawless? Is this quality found in me or not?’ If it’s not, there will be those who say, ‘Please train your own verbal conduct first.’ 

He\marginnote{5.1.12} should reflect, ‘Do I have a mind of good will toward my fellow monastics, a mind free from anger? Is this quality found in me or not?’ If it’s not, there will be those who say, ‘Please set up a mind of good will toward your fellow monastics first.’ 

He\marginnote{5.1.16} should reflect, ‘Have I learned much and do I retain and accumulate what I’ve learned? Those teachings that are good in the beginning, good in the middle, and good in the end, that have a true goal and are well articulated, and that set out the perfectly complete and pure spiritual life—have I learned many such teachings, retained them in mind, recited them verbally, mentally investigated them, and penetrated them well by view? Is this quality found in me or not?’ If it’s not, there will be those who say, ‘Please learn the tradition first.’ 

He\marginnote{5.1.20} should reflect, ‘Have I properly learned both Monastic Codes in detail; have I analyzed them well, thoroughly mastered them, and investigated them well, both in terms of the rules and their detailed exposition? Is this quality found in me or not?’ If it’s not, then when he’s asked, ‘Where was this said by the Buddha?’ he won’t be able to reply. And there will be those who say, ‘Please learn the Monastic Law first.’ 

A\marginnote{5.1.24} monk should see these five qualities in himself before accusing another.” 

\section*{9. The qualities to be set up by one who accuses another }

“Sir,\marginnote{5.2.1} how many qualities should a monk set up in himself before accusing another?” 

“He\marginnote{5.2.2} should set up five qualities in himself: ‘I’ll speak at an appropriate time, not at an inappropriate one; I’ll speak the truth, not falsehood; I’ll speak gently, not harshly; I’ll speak what’s beneficial, not what’s unbeneficial; I’ll speak with a mind of good will, not with ill will.’” 

\section*{10. Discussion on the one who accuses and the one who is accused }

“Sir,\marginnote{5.3.1} if a monk accuses another illegitimately, in how many ways should regret be aroused in him?” 

“Regret\marginnote{5.3.2} should be aroused in him in five ways: ‘Venerable, you’re accusing at the wrong time, not at the right time, and so it’s appropriate for you to have regret. You’re accusing falsely, not truthfully, and so it’s appropriate for you to have regret. You’re accusing harshly, not gently, and so it’s appropriate for you to have regret. You’re accusing unbeneficially, not beneficially, and so it’s appropriate for you to have regret. You’re accusing with a mind of ill will, not with a mind of good will, and so it’s appropriate for you to have regret.’ 

These\marginnote{5.3.8} are the five ways regret should be aroused in a monk who accuses another illegitimately. For what reason? So that other monks won’t think of making false accusations.” 

“And\marginnote{5.4.1} if a monk has been accused illegitimately, in how many ways should non-regret be aroused in him?” 

“Non-regret\marginnote{5.4.2} should be aroused in him in five ways: ‘Venerable, you’ve been accused at the wrong time, not at the right time, and so there’s no need for you to have regret. You’ve been accused falsely, not truthfully, and so there’s no need for you to have regret. You’ve been accused harshly, not gently, and so there’s no need for you to have regret. You’ve been accused unbeneficially, not beneficially, and so there’s no need for you to have regret. You’ve been accused with a mind of ill will, not with a mind of good will, and so there’s no need for you to have regret.’” 

“And\marginnote{5.5.1} if a monk accuses another legitimately, in how many ways should non-regret be aroused in him?” 

“Non-regret\marginnote{5.5.2} should be aroused in him in five ways: ‘Venerable, you’re accusing at the right time, not at the wrong time, and so there’s no need for you to have regret. You’re accusing truthfully, not falsely, and so there’s no need for you to have regret. You’re accusing gently, not harshly, and so there’s no need for you to have regret. You’re accusing beneficially, not unbeneficially, and so there’s no need for you to have regret. You’re accusing with a mind of good will, not with a mind of ill will, and so there’s no need for you to have regret.’ 

These\marginnote{5.5.8} are the five ways non-regret should be aroused in a monk who accuses another legitimately. For what reason? So that other monks will think of making truthful accusations.” 

“And\marginnote{5.6.1} if a monk has been accused legitimately, in how many ways should regret be aroused in him?” 

“Regret\marginnote{5.6.2} should be aroused in him in five ways: ‘Venerable, you’ve been accused at the right time, not at the wrong time, and so it’s appropriate for you to have regret. You’ve been accused truthfully, not falsely, and so it’s appropriate for you to have regret. You’ve been accused gently, not harshly, and so it’s appropriate for you to have regret. You’ve been accused beneficially, not unbeneficially, and so it’s appropriate for you to have regret. You’ve been accused with a mind of good will, not with a mind of ill will, and so it’s appropriate for you to have regret.’” 

“And,\marginnote{5.7.1} sir, how many qualities should a monk attend to in himself before accusing another?” 

“He\marginnote{5.7.2} should attend to five qualities in himself: compassion, being of benefit, sympathy, the idea of clearing offenses, and the idea of prioritizing the training.” 

“And\marginnote{5.7.5} how many qualities should a monk who is accused set up?” 

“A\marginnote{5.7.6} monk who is accused should set up two qualities: truth and composure.” 

\scend{The second section for recitation is finished. }

\scendsutta{The ninth chapter on the cancellation of the Monastic Code is finished. }

\scend{In this chapter there are thirty topics. }

\scuddanaintro{This is the summary: }

\begin{scuddana}%
“On\marginnote{5.7.12} the observance day, as far as, \\
The bad monk did not leave; \\
Thrown out by \textsanskrit{Moggallāna}, \\
Amazing, in the instruction of the Victor. 

Slopes,\marginnote{5.7.16} and gradual training, \\
Steady, without transgressing; \\
Corpse, the Sangha ejects, \\
Rivers, and they lose. 

They\marginnote{5.7.20} flow, they are extinguished, \\
One taste, and freedom; \\
Many, and the spiritual path, \\
Being, the eight noble persons. 

Having\marginnote{5.7.24} made the ocean simile, \\
He taught the qualities of Buddhism; \\
On the observance day, the Monastic Code, \\
No-one knows about us. 

Preempt,\marginnote{5.7.28} they complained, \\
One, two, three, four; \\
Five, six, seven, eight, \\
And nine, and tenth. 

Morality,\marginnote{5.7.32} conduct, and view, \\
Livelihood, in four parts; \\
And offense entailing expulsion, offense entailing suspension, \\
Offense entailing confession, offense entailing acknowledgment. 

Offense\marginnote{5.7.36} of wrong conduct, in five parts, \\
And failure in morality and conduct; \\
Has not failed, and has failed, \\
In six parts according to the same method. 

And\marginnote{5.7.40} offense entailing expulsion, offense entailing suspension, \\
Serious offense, and with offense entailing confession; \\
And indeed offense entailing acknowledgment, \\
Offense of wrong conduct, offense of wrong speech. 

And\marginnote{5.7.44} failure in morality and conduct; \\
Failure in view and livelihood; \\
And the eight with failed and not failed, \\
These with the morality, conduct, and view. 

With\marginnote{5.7.48} not failed, also with failed, \\
And with both failed and not failed; \\
Thus a ninefold is spoken of, \\
According to the real method. 

Offense\marginnote{5.7.52} entailing expulsion, unfinished, \\
And just so renounced; \\
He goes to, he reopens, \\
And discussion on reopening. 

And\marginnote{5.7.56} failure in morality and conduct, \\
So with failure in view; \\
Seen, heard, suspected, \\
You should know it as tenfold. 

A\marginnote{5.7.60} monk sees a monk, \\
And another informs him; \\
He just tells him, \\
He cancels the Monastic Code. 

It\marginnote{5.7.64} breaks up because of a threat, \\
Kings, criminals, fire, and flooding; \\
People, and spirits, \\
Predatory animals, snakes, life, monastic life. 

By\marginnote{5.7.68} a certain one of the ten, \\
Or some in that; \\
And just legitimate, illegitimate, \\
You should know it according to the same procedure. 

Right\marginnote{5.7.72} time, real, beneficial, \\
I will gain, there will be; \\
Bodily and verbal, good will, \\
Learned, both. 

Appropriate\marginnote{5.7.76} time, with truth, with gentleness, \\
The eighth training rule is finished. \\
Regret, legitimately, \\
Thus speech should be abolished. 

For\marginnote{5.7.80} the legitimate accuser and the accused, \\
Regret should be abolished; \\
Compassion, benefit, sympathy, \\
Clearing, prioritizing. 

The\marginnote{5.7.84} practice of the accuser, \\
Was explained by the fully Awakened One; \\
Just truth and composure, \\
Are proper for the accused.” 

%
\end{scuddana}

\scendsutta{The chapter on the cancellation of the Monastic Code is finished. }

%
\chapter*{{\suttatitleacronym Kd 20}{\suttatitletranslation The chapter on nuns }{\suttatitleroot Bhikkhunikkhandhaka}}
\addcontentsline{toc}{chapter}{\tocacronym{Kd 20} \toctranslation{The chapter on nuns } \tocroot{Bhikkhunikkhandhaka}}
\markboth{The chapter on nuns }{Bhikkhunikkhandhaka}
\extramarks{Kd 20}{Kd 20}

\section*{1. The account of \textsanskrit{Mahāpajāpati} \textsanskrit{Gotamī} }

At\marginnote{1.1.1} one time the Buddha was staying in the Sakyan country in the Banyan Tree Monastery at Kapilavatthu. At this time \textsanskrit{Mahāpajāpati} \textsanskrit{Gotamī} went to the Buddha, bowed down to him, and said, “Sir, please allow women to go forth into homelessness on the spiritual path proclaimed by the Buddha.” 

“Let\marginnote{1.1.5} it be, Gotami, don’t pursue this idea.” 

A\marginnote{1.1.6} second time and a third time she asked the same question and got the same reply. She thought, “The Buddha doesn’t allow women to go forth,” and sad and tearful she bowed down, circumambulated him with her right side toward him, and left. 

When\marginnote{1.2.1} the Buddha had stayed at Kapilavatthu for as long as he liked, he set out wandering toward \textsanskrit{Vesālī}. When he eventually arrived, he stayed in the hall with the peaked roof in the Great Wood. 

In\marginnote{1.2.4} the meantime \textsanskrit{Mahāpajāpati} shaved her hair, put on ocher robes, and set out for \textsanskrit{Vesālī} together with a number of Sakyan women. When she eventually arrived, she went to the hall with the peaked roof in the Great Wood. She then stood outside the gatehouse, sad and tearful, covered in dust, her feet swollen. 

Venerable\marginnote{1.2.7} Ānanda saw her there and said to her, “Why are you standing outside the gatehouse like this?” 

“Because,\marginnote{1.2.10} Venerable Ānanda, the Buddha doesn’t allow women to go forth.” 

“Well\marginnote{1.2.11} then, \textsanskrit{Gotamī}, please wait here for a moment while I ask the Buddha.” 

Venerable\marginnote{1.3.1} Ānanda went to the Buddha, bowed, sat down, and said, “Sir, \textsanskrit{Mahāpajāpati} \textsanskrit{Gotamī} is standing outside the gatehouse, sad and tearful, covered in dust, her feet swollen. She says you won’t allow women to go forth. Sir, please allow women to go forth.” 

“Let\marginnote{1.3.6} it be, Ānanda, don’t pursue this idea.” 

A\marginnote{1.3.7} second time and a third time he asked the same question and got the same reply. 

Ānanda\marginnote{1.3.11} thought, “The Buddha doesn’t allow women to go forth. What if I try another approach?” He then said, “If women were allowed to go forth, would they be capable of realizing the fruit of stream-entry, the fruit of once-returning, the fruit of non-returning, and the fruit of perfection?” 

“Yes,\marginnote{1.3.16} they would.” 

“If\marginnote{1.3.17} that’s so, sir, and considering that \textsanskrit{Mahāpajāpati} has been very helpful to you—she’s your aunt who nurtured you, brought you up, and breastfed you when your own mother died—please allow women to go forth.” 

\section*{2. The eight important principles }

“Ānanda,\marginnote{1.4.1} if \textsanskrit{Mahāpajāpati} accepts these eight important principles, that will be her full ordination: 

\begin{enumerate}%
\item A nun who has been fully ordained for a hundred years should bow down to a monk who was given the full ordination on that very day, and she should stand up for him, raise her joined palms to him, and do acts of respect toward him.\footnote{“Acts of respect” renders \textit{\textsanskrit{sāmīcikammaṁ}}. Sp 2.149: \textit{\textsanskrit{sāmīcikammanti} \textsanskrit{maggasampadānabījanapānīyāpucchanādikaṁ} \textsanskrit{anucchavikavattaṁ}}, “\textit{\textsanskrit{Sāmīcikammaṁ}} means appropriate duties such as giving way, fanning, offering drinking water, etc.” } This principle is to be honored and respected all one’s life, and is not to be breached. %
\item A nun shouldn’t spend the rainy-season residence in a monastery without monks.\footnote{The Pali word behind the translation “monastery” is \textit{\textsanskrit{āvāsa}}. This refers to the area of the \textit{\textsanskrit{simā}}, the monastery zone, as established through a legal procedure. This area may be much larger than the space occupied by any actual buildings. See also \href{https://suttacentral.net/pli-tv-bi-vb-pc56/en/brahmali\#1.16.1}{Bi Pc 56:1.16.1}. } This principle too is to be honored and respected all one’s life, and is not to be breached. %
\item Every half-month a nun should seek two things from the Sangha of monks:\footnote{See also \href{https://suttacentral.net/pli-tv-bi-vb-pc59/en/brahmali\#1.11.1}{Bi Pc 59:1.11.1}. } asking it about the observance day and going to it for the instruction. This principle too is to be honored and respected all one’s life, and is not to be breached. %
\item A nun who has completed the rainy-season residence should invite correction from both Sanghas in regard to three things:\footnote{See also \href{https://suttacentral.net/pli-tv-bi-vb-pc57/en/brahmali\#1.15.1}{Bi Pc 57:1.15.1}. } what has been seen, heard, or suspected. This principle too is to be honored and respected all one’s life, and is not to be breached. %
\item A nun who has committed a heavy offense must undertake a trial period for a half-month toward both Sanghas.\footnote{Heavy offense, \textit{garudhamma}, here refers to the \textit{\textsanskrit{saṅghādisesa}} offenses, the offenses entailing suspension. See \href{https://suttacentral.net/pli-tv-bi-vb-ss13/en/brahmali\#3.16}{Bi Ss 13:3.16}. } This principle too is to be honored and respected all one’s life, and is not to be breached. %
\item A trainee nun who has trained for two years in the six rules may seek for full ordination in both Sanghas.\footnote{See especially \href{https://suttacentral.net/pli-tv-bi-vb-pc64/en/brahmali\#1.36.1}{Bi Pc 64:1.36.1}, but also \href{https://suttacentral.net/pli-tv-bi-vb-pc67/en/brahmali\#1.36.1}{Bi Pc 67:1.36.1} and \href{https://suttacentral.net/pli-tv-bi-vb-pc73/en/brahmali\#1.36.1}{Bi Pc 73:1.36.1}. } This principle too is to be honored and respected all one’s life, and is not to be breached. %
\item A nun may not in any way abuse or revile a monk.\footnote{See \href{https://suttacentral.net/pli-tv-bi-vb-pc52/en/brahmali\#1.29.1}{Bi Pc 52:1.29.1}. } This principle too is to be honored and respected all one’s life, and is not to be breached. %
\item From today onwards, nuns may not correct monks, but monks may correct nuns.\footnote{“Correct” renders \textit{vacanapatha}. For the meaning of this word see Bhikkhu Sujato, “Bhikkhuni Vinaya Studies”, pp. 73–76. } This principle too is to be honored and respected all one’s life, and is not to be breached. %
\end{enumerate}

Ānanda,\marginnote{1.4.21} if \textsanskrit{Mahāpajāpati} accepts these eight important principles, that will be her full ordination.” 

After\marginnote{1.5.1} learning these eight important principles from the Buddha, Ānanda went to \textsanskrit{Mahāpajāpati} and said, “If you accept eight important principles, Gotami, that will be your full ordination.” And he told her the principles. 

She\marginnote{1.5.22} replied, “Just as a young woman or man—someone fond of adornments, who has just washed their hair—would receive a garland of lotuses, jasmine, or sandan flowers with both hands and place it on their head,\footnote{For further information on these flowers, see Appendix of Plants. } so too, do I receive these eight important principles, not to be breached for life.” 

Ānanda\marginnote{1.6.1} then went to the Buddha, bowed, sat down, and said, “Sir, \textsanskrit{Mahāpajāpati} has accepted the eight important principles. Your aunt is now ordained.” 

“Ānanda,\marginnote{1.6.4} if women had not been allowed to go forth on this spiritual path proclaimed by the Buddha, the spiritual life would have lasted a long time—the true Teaching would have lasted a thousand years. But now that women have been allowed to go forth, the spiritual life won’t last long—the true Teaching will only last five hundred years. 

Just\marginnote{1.6.7} as families with many women and few men are easily robbed by thieves, so too, the spiritual life doesn’t last long on a spiritual path where women are allowed to go forth. 

Just\marginnote{1.6.9} as a ripe field of rice affected by whiteheads won’t last long,\footnote{“Whiteheads” renders \textit{\textsanskrit{setaṭṭ}(h)\textsanskrit{ikā}}, literally, “white bones”. Sp 4.403: \textit{\textsanskrit{Setaṭṭhikā} \textsanskrit{nāma} \textsanskrit{rogajātīti} eko \textsanskrit{pāṇako} \textsanskrit{nāḷimajjhagataṁ} \textsanskrit{kaṇḍaṁ} vijjhati, yena \textsanskrit{viddhattā} nikkhantampi \textsanskrit{sālisīsaṁ} \textsanskrit{khīraṁ} \textsanskrit{gahetuṁ} na sakkoti}, “The disease called \textit{\textsanskrit{setaṭṭhikā}} means: an insect penetrates the stem, goes to the middle of the stalk, from the penetration of which the rice grains are not able to get sap.” This seems to be a description of so-called “whiteheads”, pale panicles without rice grains, caused by stem borers. } so too, the spiritual life doesn’t last long on a spiritual path where women are allowed to go forth. 

Just\marginnote{1.6.11} as a ripe field of sugarcane attacked by red rot won’t last long,\footnote{Sp 4.403: \textit{\textsanskrit{Mañjiṭṭhikā} \textsanskrit{nāma} \textsanskrit{rogajātīti} \textsanskrit{ucchūnaṁ} \textsanskrit{antorattabhāvo}}, “\textit{\textsanskrit{Mañjiṭṭhikā} \textsanskrit{nāma} \textsanskrit{rogajāti}}: sugar cane becoming red within.” } so too, the spiritual life doesn’t last long on a spiritual path where women are allowed to go forth. 

Just\marginnote{1.6.13} as a man might, as a safeguard, surround a large pool with an embankment to stop the water from overflowing, so too, have I, as a safeguard, laid down the eight important principles, not to be breached for life.” 

\scend{The eight important principles for nuns are finished. }

\section*{3. The allowance for the full ordination of nuns }

\textsanskrit{Mahāpajāpati}\marginnote{2.1.1} went to the Buddha, bowed down, and said, “Sir, what should I do with these Sakyan women?” The Buddha then instructed, inspired, and gladdened her with a teaching, after which she bowed down, circumambulated him with her right side toward him, and left. Soon afterwards the Buddha gave a Teaching and addressed the monks: 

\scrule{“I allow monks to give the full ordination to nuns.” }

Soon\marginnote{2.2.1} afterwards those nuns said to \textsanskrit{Mahāpajāpati}, “We’re ordained, but you’re not, for the Buddha has laid down that monks should give the full ordination to nuns.” 

\textsanskrit{Mahāpajāpati}\marginnote{2.2.4} then went to Venerable Ānanda, bowed down, and told him what the Sakyan women had said. 

And\marginnote{2.2.9} Ānanda went to the Buddha, bowed, sat down, and told him what \textsanskrit{Mahāpajāpati} had said. 

The\marginnote{2.2.15} Buddha replied, “\textsanskrit{Mahāpajāpati} was ordained from the moment she accepted the eight important principles.” 

On\marginnote{3.1.1} another occasion \textsanskrit{Mahāpajāpati} went to Venerable Ānanda, bowed down, and said, “Venerable Ānanda, I wish to ask the Buddha for a favor: ‘Sir, please allow the monks and nuns to bow down to one another according to seniority, and likewise to rise up for one another, raise their joined palms to one another, and do acts of respect toward one another according to seniority.’” 

Venerable\marginnote{3.1.5} Ānanda went to the Buddha, bowed, sat down, and told the Buddha what \textsanskrit{Mahāpajāpati} had said. 

The\marginnote{3.1.10} Buddha replied, “It’s impossible, Ānanda, that I would allow bowing down to women, or rising up for them, raising one’s joined palms to them, or doing acts of respect toward them. Even the monastics of other religions with their flawed teachings don’t do these things. So how, then, could I allow them?” 

The\marginnote{3.1.13} Buddha then gave a teaching and addressed the monks: 

\scrule{“You shouldn’t bow down to a woman, or rise up for, raise your joined palms to, or do acts of respect toward a woman. If you do, you commit an offense of wrong conduct.” }

On\marginnote{4.1.1} another occasion \textsanskrit{Mahāpajāpati} went to the Buddha, bowed down, and said, “Sir, how should we practice those training rules that the nuns have in common with the monks?” 

“You\marginnote{4.1.4} should practice them in the same way as the monks do.” 

“And\marginnote{4.1.5} how should we practice those training rules that the nuns don’t have in common with the monks?” 

“You\marginnote{4.1.6} should practice them as they have been laid down.” 

On\marginnote{5.1.1} another occasion \textsanskrit{Mahāpajāpati} went to the Buddha, bowed down, and said, “Sir, please give me a teaching in brief. I’ll then stay by myself, secluded, heedful, energetic, and diligent.” 

“Those\marginnote{5.1.4} things, Gotami, that you know lead to passion, not to dispassion; to bondage, not to freedom from bondage; to an increase in things, not to a reduction in things; to great desires, not fewness of desires; to discontent, not to contentment; to socializing, not to seclusion; to laziness, not to being energetic; to being burdensome, not to being unburdensome—you should definitely regard them as not the Teaching, not the training, not the Teacher’s instruction.\footnote{For an explanation of the rendering “training” for \textit{vinaya}, see Appendix of Technical Terms. } But those things that you know lead to dispassion, not to passion; to freedom from bondage, not to bondage; to a reduction in things, not to an increase in things; to fewness of desires, not to great desires; to contentment, not to discontent; to seclusion, not to socializing; to being energetic, not to laziness; to being unburdensome, not to being unburdensome—you should definitely regard them as the Teaching, the training, the Teacher’s instruction.” 

At\marginnote{6.1.1} that time the Monastic Code was not being recited to the nuns. They told the Buddha. … “The Monastic Code should be recited to the nuns.” The nuns thought,\footnote{I follow the Pali of the PTS version. MS reads, “The monks thought”. } “Who should recite the Monastic Code to the nuns?” They told the Buddha. “The monks should recite the Monastic Code to the nuns.” 

Soon\marginnote{6.1.8} afterwards the monks went to the nuns’ dwelling place to recite the Monastic Code. People complained and criticized them, “They’re their wives! They’re their mistresses! Now they’re going to enjoy themselves together.” They told the Buddha. 

\scrule{“Monks, you shouldn’t recite the Monastic Code to the nuns. If you do, you commit an offense of wrong conduct. Nuns should recite the Monastic Code to the nuns.” }

The\marginnote{6.1.15} nuns did not know how to recite it. They told the Buddha. “The monks should tell the nuns how to recite the Monastic Code.” 

At\marginnote{6.2.1} that time the nuns did not make amends for their offenses. They told the Buddha. 

\scrule{“A nun should make amends for her offenses. If she doesn’t, she commits an offense of wrong conduct.” }

The\marginnote{6.2.5} nuns did not know how to make amends. They told the Buddha. “The monks should tell the nuns how to make amends for an offense.” The monks thought, “Who should receive the confession of offenses from the nuns?” They told the Buddha. “The monks should receive the confession of offenses from the nuns.” 

Soon\marginnote{6.2.14} afterwards, when the nuns saw a monk on a street, in a cul-de-sac, or at an intersection, they would put down their bowls, arrange their upper robes over one shoulder, squat on their heels, raise their joined palms, and make amends for their offenses. People complained and criticized them, “They’re their wives! They’re their mistresses! Having offended them at night, they now ask for forgiveness.” They told the Buddha. 

\scrule{“Monks, you shouldn’t receive confessions from the nuns. If you do, you commit an offense of wrong conduct. Nuns should receive the confession of offenses from the nuns.” }

The\marginnote{6.2.21} nuns did not know how to receive confessions. They told the Buddha. “The monks should tell the nuns how to receive confessions.” 

At\marginnote{6.3.1} that time the nuns’ legal procedures were not being done. They told the Buddha. “Monks, I allow the doing of the nuns’ legal procedures.” The monks thought, “Who should do the nuns’ procedures?” They told the Buddha. “The monks should do the nuns’ legal procedures.” 

Soon\marginnote{6.3.8} afterwards, when nuns who had had a legal procedure done against them saw a monk on a street, in a cul-de-sac, or at an intersection, they would put down their bowls, arrange their upper robes over one shoulder, squat on their heels, raise their joined palms, and ask for forgiveness, thinking, “This is the way to do it.”\footnote{“Who had had a legal procedure done against them” renders \textit{\textsanskrit{katakammā}}. The commentary is silent, but elsewhere \textit{\textsanskrit{katadaṇḍakamma}} (e.g. at \href{https://suttacentral.net/pli-tv-kd1/en/brahmali\#44.1.1}{Kd 1:44.1.1}) and \textit{kammakata} (e.g. at \href{https://suttacentral.net/pli-tv-bu-vb-ss13/en/brahmali\#2.45}{Bu Ss 13:2.45}) mean “who has had a legal procedure (of penalization) done against them”. } People complained and criticized them, “They’re their wives! They’re their mistresses! Having offended them at night, they now ask for forgiveness.” They told the Buddha. 

\scrule{“Monks, you shouldn’t do the nuns’ legal procedures. If you do, you commit an offense of wrong conduct. The nuns should do the nuns’ legal procedures.” }

The\marginnote{6.3.15} nuns did not know how to do procedures. They told the Buddha. “The monks should tell the nuns how to do legal procedures.” 

On\marginnote{7.1.1} one occasion the nuns were arguing and disputing in the midst of the Sangha, attacking one another verbally, and they were not able to resolve that legal issue. They told the Buddha. 

\scrule{“I allow the monks to resolve the nuns’ legal issues.” }

Soon\marginnote{7.1.5} afterwards the monks were resolving a legal issue for the nuns. While they were discussing that legal issue, there were nuns who deserved to have a legal procedure done against them and who had committed an offense.\footnote{Sp-\textsanskrit{ṭ} 4.410: \textit{\textsanskrit{Kammappattāyopīti} \textsanskrit{kammārahāpi}. \textsanskrit{Āpattigāminiyopīti} \textsanskrit{āpattiāpannāyopi}}, “\textit{\textsanskrit{Kammappattāyopi}}: who deserved a legal procedure. \textit{\textsanskrit{Āpattigāminiyopi}}: who had committed an offense.” } The nuns said, “Venerables, please do the procedure against those nuns and receive their confession of offenses, for the Buddha has laid down that the nuns’ legal issues should be resolved by the monks.” They told the Buddha. 

\scrule{“Monks, I allow you to determine the nature of the nuns’ legal procedure, before handing it over to the nuns to do it.\footnote{“Determine the nature of the nuns’ legal procedure” renders \textit{\textsanskrit{bhikkhunīnaṁ} \textsanskrit{kammaṁ} \textsanskrit{āropetvā}}. Sp 4.410: \textit{\textsanskrit{Anujānāmi}, bhikkhave, \textsanskrit{bhikkhūhi} \textsanskrit{bhikkhunīnaṁ} \textsanskrit{kammaṁ} \textsanskrit{ropetvā} \textsanskrit{niyyādetunti} ettha \textsanskrit{tajjanīyādīsu} “\textsanskrit{idaṁ} \textsanskrit{nāma} \textsanskrit{kammaṁ} \textsanskrit{etissā} \textsanskrit{kātabban}”ti \textsanskrit{evaṁ} \textsanskrit{ropetvā} “\textsanskrit{taṁ} \textsanskrit{dāni} tumheva \textsanskrit{karothā}”ti \textsanskrit{niyyādetabbaṁ}}, “\textit{\textsanskrit{Anujānāmi}, bhikkhave, \textsanskrit{bhikkhūhi} \textsanskrit{bhikkhunīnaṁ} \textsanskrit{kammaṁ} \textsanskrit{ropetvā} \textsanskrit{niyyādetuṁ}} here, in regard to (legal procedures of) condemnation, etc., saying, ‘This kind of legal procedure is to be done toward this one’, having fixed it thus, it is to be handed over (to the nuns), saying, ‘Now you should do it against her.’” } And I allow you to charge a nun with an offense, before handing it over to the nuns for them to receive the confession.” }

At\marginnote{8.1.1} that time a nun who was a pupil of the nun \textsanskrit{Uppalavaṇṇā} had followed the Buddha around for seven years to learn the Monastic Law. But because of her absentmindedness, she repeatedly forgot what she had learned. When she heard that the Buddha wanted to go to \textsanskrit{Sāvatthī}, she reflected on her absentmindedness and thought, “It’s hard for a woman to follow the Teacher around all her life. So, what should I do?” She told the nuns what she had thought, who in turn told the monks, who then told the Buddha. The Buddha said, 

\scrule{“I allow monks to teach the Monastic Law to the nuns.” }

\scend{The first section for recitation is finished. }

\subsection*{Regulations on the instruction}

When\marginnote{9.1.1} the Buddha had stayed at \textsanskrit{Vesālī} for as long as he liked, he set out wandering toward \textsanskrit{Sāvatthī}. When he eventually arrived, he stayed in the Jeta Grove, \textsanskrit{Anāthapiṇḍika}’s Monastery. At that time the monks from the group of six tried to attract the nuns by splashing them with muddy water. They told the Buddha. 

\scrule{“Monks, you shouldn’t splash the nuns with muddy water. If you do, you commit an offense of wrong conduct. I allow the nuns to penalize a monk who acts like this.”\footnote{It is not clear from the Pali whether it is the monks or the nuns who should penalize the monk. The discussion in the commentary at Sp 4.411, however, makes it clear that it is the nuns who impose the penalty. The commentary explains that the penalty is imposed by assembling at the nuns’ dwelling place, \textit{bhikkhunupassaye}, and those assembled then use the term \textit{ayya} when referring to the monk who acted inappropriately. Only nuns use this term when referring to monks. } }

The\marginnote{9.1.10} monks thought, “What sort of penalty should they impose?” They told the Buddha. 

\scrule{“The Sangha of nuns shouldn’t pay respect to such a monk.”\footnote{Sp 4.411: \textit{Avandiyo so bhikkhave bhikkhu \textsanskrit{bhikkhunisaṅghena} \textsanskrit{kātabboti} bhikkhunupassaye \textsanskrit{sannipatitvā} “asuko \textsanskrit{nāma} ayyo \textsanskrit{bhikkhunīnaṁ} \textsanskrit{apasādanīyaṁ} dasseti, etassa ayyassa \textsanskrit{avandiyakaraṇaṁ} \textsanskrit{ruccatī}”ti \textsanskrit{evaṁ} \textsanskrit{tikkhattuṁ} \textsanskrit{sāvetabbaṁ}. \textsanskrit{Ettāvatā} avandiyo kato hoti}, “‘The Sangha of nuns shouldn’t pay respect to such a monk’: Having assembled in the nuns’ quarters, this proclamation is to be done three times: ‘The venerable so-and-so is disagreeable toward the nuns. It is proper not to pay respects to this venerable.’ In this way not-paying respect is done.” } }

Soon\marginnote{9.1.14} afterwards the monks from the group of six tried to attract the nuns by exposing their bodies to them, by exposing their thighs to them, and by exposing their genitals to them. And they spoke indecently to the nuns and associated inappropriately with them. They told the Buddha. 

\scrule{“Monks, you shouldn’t expose your body to the nuns; you shouldn’t expose your thighs to the nuns; you shouldn’t expose your genitals to the nuns; you shouldn’t speak indecently to the nuns; and you shouldn’t associate inappropriately with the nuns. If you associate inappropriately with the nuns, you commit an offense of wrong conduct. I allow the nuns to penalize a monk who acts like this.” }

The\marginnote{9.1.21} monks thought, “What sort of penalty can they impose?” They told the Buddha. 

\scrule{“The Sangha of nuns shouldn’t pay respect to such a monk.” }

Soon\marginnote{9.2.1} afterwards the nuns from the group of six tried to attract a monk by splashing him with muddy water. They told the Buddha. 

\scrule{“A nun shouldn’t splash a monk with muddy water. If she does, she commits an offense of wrong conduct. I allow you to penalize such a nun.” }

The\marginnote{9.2.7} monks thought, “What sort of penalty can we impose?” They told the Buddha. 

\scrule{“I allow you to place restrictions on her.”\footnote{Sp 4.411: \textit{\textsanskrit{Āvaraṇanti} \textsanskrit{vihārappavesane} \textsanskrit{nivāraṇaṁ}}, “Restrictions: a hindrance on entering the monastery.” } }

She\marginnote{9.2.11} did not adhere to the restrictions. They told the Buddha. 

\scrule{“I allow you to cancel her half-monthly instruction.” }

At\marginnote{9.2.14} that time the nuns from the group of six tried to attract the monks by exposing their bodies to them, by exposing their breasts to them, by exposing their thighs to them, and by exposing their genitals to them. And they spoke indecently to the monks and associated inappropriately with them. They told the Buddha. 

\scrule{“A nun shouldn’t expose her body to the monks; she shouldn’t expose her breasts to the monks; she shouldn’t expose her thighs to the monks; she shouldn’t expose her genitals to the monks; she shouldn’t speak indecently to the monks; and she shouldn’t associate inappropriately with the monks. If she associates inappropriately with the monks, she commits an offense of wrong conduct. I allow the monks to penalize such a nun.” }

The\marginnote{9.2.21} monks thought, “What sort of penalty can we impose?” They told the Buddha. 

\scrule{“I allow you to place restrictions on her.” }

She\marginnote{9.2.25} did not adhere to the restrictions. They told the Buddha. 

\scrule{“I allow you to cancel her half-monthly instruction.” }

The\marginnote{9.3.1} monks thought, “Is it allowable or not to do the observance-day ceremony with a nun whose half-monthly instruction has been canceled?” They told the Buddha. 

\scrule{“Until that legal issue has been resolved, it’s not allowable to do the observance-day ceremony with a nun whose half-monthly instruction has been canceled.” }

On\marginnote{9.3.5} one occasion Venerable \textsanskrit{Udāyī} canceled the half-monthly instruction and then set out wandering. The nuns complained and criticized him, “How could Venerable \textsanskrit{Udāyī} do this?” They told the Buddha. 

\scrule{“Monks, you shouldn’t cancel the half-monthly instruction and then set out wandering. If you do, you commit an offense of wrong conduct.” }

At\marginnote{9.3.11} that time there were ignorant and incompetent monks who canceled the half-monthly instruction.\footnote{The word “monks” is missing from the Pali, but is found in other versions of the \textsanskrit{Tipiṭaka}, such as SRT. } They told the Buddha. 

\scrule{“A monk who’s ignorant and incompetent shouldn’t cancel the half-monthly instruction. If he does, he commits an offense of wrong conduct.” }

At\marginnote{9.3.15} that time there were monks who canceled the half-monthly instruction without reason. They told the Buddha. 

\scrule{“A monk shouldn’t cancel the half-monthly instruction without reason. If he does, he commits an offense of wrong conduct.” }

At\marginnote{9.3.19} that time there were monks who did not investigate after canceling the half-monthly instruction.\footnote{“Did not investigate” renders \textit{\textsanskrit{vinicchayaṁ} na denti}. The commentaries are silent. I take it to mean that the monks did not investigate whether the nun had changed her behavior. } They told the Buddha. 

\scrule{“You should investigate after canceling the half-monthly instruction. If you don’t, you commit an offense of wrong conduct.” }

At\marginnote{9.4.1} that time there were nuns who did not go to the half-monthly instruction. They told the Buddha. 

\scrule{“A nun should go to the half-monthly instruction. If she doesn’t, she should be dealt with according to the rule.”\footnote{See \href{https://suttacentral.net/pli-tv-bi-vb-pc58/en/brahmali\#1.14.1}{Bi Pc 58:1.14.1}. } }

At\marginnote{9.4.5} that time the entire Sangha of nuns went to the half-monthly instruction. People complained and criticized them, “They’re their wives! They’re their mistresses! Now they’re going to enjoy themselves together.” They told the Buddha. 

\scrule{“The whole Sangha of nuns shouldn’t go to the half-monthly instruction. If it does, there’s an offense of wrong conduct. Four or five nuns should go to the instruction.” }

Soon\marginnote{9.4.12} afterwards four or five nuns went to the half-monthly instruction. People complained and criticized them, “They’re their wives! They’re their mistresses! Now they’re going to enjoy themselves together.” They told the Buddha. 

\scrule{“Four or five nuns shouldn’t go to the half-monthly instruction. If they do, there’s an offense of wrong conduct. Two or three nuns should go to the instruction.\footnote{This is curious because, according to \href{https://suttacentral.net/pli-tv-bi-vb-pc58/en/brahmali\#1.14.1}{Bi Pc 58}, a nun must go to the instruction, the \textit{\textsanskrit{ovāda}}. The ensuing narrative, however, makes it clear that this is not about the instruction as such, but about asking for the day of the observance day and the right time to approach for the instruction. In other words, this is about \href{https://suttacentral.net/pli-tv-bi-vb-pc59/en/brahmali\#1.11.1}{Bi Pc 59}, not Bi Pc 58. It is possible that the Pali text here is corrupted and that the original phrasing was closer to what we find at Bi Pc 59. } }

They\marginnote{9.4.19} should go to a monk, arrange their upper robes over one shoulder, bow down at his feet, squat on their heels, raise their joined palms, and say, ‘Venerable, the Sangha of nuns bows down at the feet of the Sangha of monks and asks to come for the half-monthly instruction. Please allow the Sangha of nuns to come for the instruction.’ 

That\marginnote{9.4.22} monk should go to the reciter of the Monastic Code and tell him of the nuns’ request. The reciter should say, ‘Is there anyone who has been appointed as an instructor of the nuns?’ If there is, the reciter should say, ‘Monk so-and-so has been appointed. The Sangha of nuns should approach him.’ If there isn’t, the reciter should say, ‘Who’s suitable to instruct the nuns?’ If there is someone who is suitable and who has the eight required qualities, he should be appointed. The reciter should then say,\footnote{These are the eight qualities listed at \href{https://suttacentral.net/pli-tv-bu-vb-pc21/en/brahmali\#2.26}{Bu Pc 21:2.26}. } ‘Monk so-and-so has been appointed. The Sangha of nuns should approach him.’ If no-one is able to instruct the nuns, the reciter of the Monastic Code should say, ‘No monk has been appointed as an instructor of the nuns. The Sangha of nuns should carry on with serenity.’” 

On\marginnote{9.5.1} one occasion there were monks who did not agree to give the half-monthly instruction. They told the Buddha. 

\scrule{“You should agree to give the half-monthly instruction. If you don’t, you commit an offense of wrong conduct.” }

On\marginnote{9.5.5} one occasion the nuns went to a monk who was ignorant and said, “Venerable, please agree to give the half-monthly instruction.” 

“But\marginnote{9.5.8} I’m ignorant, sisters. How can I agree to give the instruction?” 

“Please\marginnote{9.5.10} agree to give the instruction, for the Buddha has laid down that a monk should agree to give the instruction to the nuns.” They told the Buddha. 

\scrule{“Except if you’re ignorant, you should agree to give the half-monthly instruction.” }

On\marginnote{9.5.15} one occasion the nuns went to a monk who was sick and said, “Venerable, please agree to give the half-monthly instruction.” 

“But\marginnote{9.5.18} I’m sick, sisters. How can I agree to give the instruction?” 

“Please\marginnote{9.5.20} agree to give the instruction, for the Buddha has laid down that a monk should agree to give the instruction to the nuns, except if he’s ignorant.” They told the Buddha. 

\scrule{“Except if you’re ignorant or sick, you should agree to give the half-monthly instruction.” }

On\marginnote{9.5.25} one occasion the nuns went to a monk who was about to depart and said, “Venerable, please agree to give the half-monthly instruction.” 

“But\marginnote{9.5.28} I’m about to depart, sisters. How can I agree to give the instruction?” 

“Please\marginnote{9.5.30} agree to give the instruction, for the Buddha has laid down that a monk should agree to give the instruction to the nuns, except if he’s ignorant or sick.” They told the Buddha. 

\scrule{“Except if you’re ignorant, sick, or about to depart, you should agree to give the half-monthly instruction.” }

On\marginnote{9.5.35} one occasion the nuns went to a monk who was staying in the wilderness and said, “Venerable, please agree to give the half-monthly instruction.” 

“But\marginnote{9.5.38} I’m staying in the wilderness, sisters. How can I agree to give the instruction?” 

“Please\marginnote{9.5.40} agree to give the instruction, for the Buddha has laid down that a monk should agree to give the instruction to the nuns, except if he’s ignorant, sick, or about to depart.” They told the Buddha. 

\scrule{“If you’re staying in the wilderness, you should agree to give the half-monthly instruction. You should make an appointment, saying, ‘I’ll return here.’” }

At\marginnote{9.5.46} that time there were monks who agreed to give the half-monthly instruction without informing.\footnote{Sp-\textsanskrit{ṭ} 4.415: \textit{Na \textsanskrit{ārocentīti} \textsanskrit{pātimokkhuddesakassa} na \textsanskrit{ārocenti}}, “‘Without informing’: without informing the reciter of the Monastic Code.” } They told the Buddha. 

\scrule{“When you have agreed to give the half-monthly instruction, you should inform. If you don’t, you commit an offense of wrong conduct.” }

At\marginnote{9.5.50} that time there were monks who had agreed to give the half-monthly instruction, but did not return to give it. They told the Buddha. 

\scrule{“You should return to give the half-monthly instruction. If you don’t, you commit an offense of wrong conduct.” }

On\marginnote{9.5.54} one occasion the nuns did not go to the appointment. They told the Buddha. 

\scrule{“A nun should go to the appointment. If she doesn’t, she commits an offense of wrong conduct.” }

\subsection*{Beautification and indulgence}

At\marginnote{10.1.1} that time there were nuns who wore long belts that they made into corsets.\footnote{Sp 4.416: \textit{\textsanskrit{Phāsukā} \textsanskrit{namentīti} \textsanskrit{gihidārikāyo} viya \textsanskrit{ghanapaṭṭakena} \textsanskrit{kāyabandhanena} \textsanskrit{phāsukā} \textsanskrit{namanatthāya} bandhanti}, “\textit{\textsanskrit{Phāsukā} namenti} means: using the thick strips of the waistband, they made ribs for the purpose of shaping, like lay girls.” Vin-vn-\textsanskrit{ṭ} 2954: \textit{Na \textsanskrit{phāsukā} \textsanskrit{nametabbāti} majjhimassa \textsanskrit{tanubhāvatthāya} \textsanskrit{gāmadārikā} viya \textsanskrit{phāsulikā} na \textsanskrit{nāmetabbā}}; “\textit{Na \textsanskrit{phāsukā} \textsanskrit{nametabbā}}: village-girl ribs are not to be made for the purpose of making one of middle size slender.” } People complained and criticized them, “They’re just like householders who indulge in worldly pleasures!” They told the Buddha. 

\scrule{“A nun shouldn’t wear a long belt. If she does, she commits an offense of wrong conduct. I allow a nun to wear a belt that goes once around her body, but she shouldn’t make a corset out of it. If she does, she commits an offense of wrong conduct.” }

There\marginnote{10.1.9} were nuns who made corsets out of strips of split bamboo, out of strips of leather, out of strips of fabric, out of interlaced fabric, out of rolled-up fabric, out of strips of cloth, out of interlaced cloth, out of rolled-up cloth, out of interlaced strings, and out of rolled-up strings. People complained and criticized them, “They’re just like householders who indulge in worldly pleasures!” They told the Buddha. 

\scrule{“A nun shouldn’t make a corset out of strips of split bamboo, strips of leather, strips of fabric, interlaced fabric, rolled-up fabric, strips of cloth, interlaced cloth, rolled-up cloth, interlaced strings, or rolled-up strings. If she does, she commits an offense of wrong conduct.” }

There\marginnote{10.2.1} were nuns who had their loins rubbed with bones, their loins tapped with a cow’s jaw bone, their palms tapped, the backs of their hands tapped, the soles of their feet tapped, the tops of their feet tapped, their thighs tapped, their faces tapped, and their gums tapped with a cow’s jaw bone.\footnote{“The backs of their hands” renders \textit{hatthakoccha}. Sp 4.416: \textit{Hatthakocchanti \textsanskrit{piṭṭhihatthaṁ}}, “\textit{Hatthakoccha} means the back of the hand.” According to PED \textit{\textsanskrit{piṭṭhi} } and \textit{tala} refer to opposite sides of the hand and the foot, with \textit{tala} referring to the palm/sole. } People complained and criticized them, “They’re just like householders who indulge in worldly pleasures!” They told the Buddha. 

\scrule{“A nun shouldn’t have her loins rubbed with bones, her loins tapped with a cow’s jaw bone, her palms tapped, the back of her hands tapped, the soles of her feet tapped, the top of her feet tapped, her thighs tapped, her face tapped, or her gums tapped with a cow’s jaw bone. If she does, she commits an offense of wrong conduct.” }

The\marginnote{10.3.1} nuns from the group of six used facial ointments, applied facial creams, powdered their face, applied rouge to their face, wore cosmetics on their body, wore cosmetics on their face, and wore cosmetics on their body and face.\footnote{“Applied facial creams” renders \textit{\textsanskrit{mukhaṁ} ummaddenti}. The verb \textit{ummaddeti} normally just means “rubs” or “massages”, but here the contexts required the application of some kind of cosmetic or cream. Sp 3.247: \textit{\textsanskrit{Ummaddentīti} \textsanskrit{nānāummaddanehi} ummaddenti}, “\textit{Ummaddenti}: they rub with various creams.” } People complained and criticized them, “They’re just like householders who indulge in worldly pleasures!” They told the Buddha. 

\scrule{“A nun shouldn’t use facial ointments, apply facial creams, powder her face, apply rouge to her face, wear cosmetics on her body, wear cosmetics on her face, or wear cosmetics on her body and face. If she does, she commits an offense of wrong conduct.” }

The\marginnote{10.4.1} nuns from the group of six made up their eyes,\footnote{Sp 4.417: \textit{\textsanskrit{Avaṅgaṁ} \textsanskrit{karontīti} \textsanskrit{akkhī} \textsanskrit{añjantiyo} \textsanskrit{avaṅgadese} \textsanskrit{adhomukhaṁ} \textsanskrit{lekhaṁ} karonti}, “\textit{\textsanskrit{Avaṅgaṁ} \textsanskrit{karontīti}}: they made up the eyes by making a downward line at the outer corner of the eye.” } applied facial marks, stared out the windows, exposed themselves to view,\footnote{Sp 4.417: \textit{Visesakanti \textsanskrit{gaṇḍappadese} \textsanskrit{vicitrasaṇṭhānaṁ} \textsanskrit{visesakaṁ} karonti. \textsanskrit{Olokentīti} \textsanskrit{vātapānaṁ} \textsanskrit{vivaritvā} \textsanskrit{vīthiṁ} olokenti. \textsanskrit{Sāloke} \textsanskrit{tiṭṭhantīti} \textsanskrit{dvāraṁ} \textsanskrit{vivaritvā} \textsanskrit{upaḍḍhakāyaṁ} dassentiyo \textsanskrit{tiṭṭhanti}}, “\textit{Visesaka}: they made a colored mark on the cheek. \textit{Olokenti}: having opened a window, they looked at the street. \textit{\textsanskrit{Sāloke} \textsanskrit{tiṭṭhanti}}: having opened a door, they stood showing half their body.” } organized dancing, appointed sex workers, set up bars, set up slaughterhouses, set up shops, made loans, engaged in trade, were attended on by slaves, were attended on by servants, were attended on by animals, traded in raw and cooked greens, and wore felt.\footnote{“Appointed sex workers” renders \textit{\textsanskrit{vesiṁ} \textsanskrit{vuṭṭhāpenti}}. Sp 3.327: \textit{\textsanskrit{Sālavatiṁ} \textsanskrit{kumāriṁ} \textsanskrit{gaṇikaṁ} \textsanskrit{vuṭṭhāpesīti} … \textsanskrit{gaṇikaṭṭhāne} \textsanskrit{ṭhapesunti} attho}, “Appointed the girl \textsanskrit{Sālavatī} as courtesan: … the meaning is that they placed her in the position of courtesan.” “Traded in raw and cooked greens” renders \textit{\textsanskrit{haritakapakkikaṁ} \textsanskrit{pakiṇanti}}. Sp 4.417: \textit{\textsanskrit{Haritakapakkikaṁ} \textsanskrit{pakiṇantīti} \textsanskrit{haritakañceva} \textsanskrit{pakkañca} \textsanskrit{pakiṇanti}; \textsanskrit{pakiṇṇakāpaṇaṁ} \textsanskrit{pasārentīti} \textsanskrit{vuttaṁ} hoti}, “\textit{\textsanskrit{Haritakapakkikaṁ} \textsanskrit{pakiṇanti}} means she traded in greens and what is cooked; it is said that she set up shop for trading.” Sp-yoj 4.417: \textit{“\textsanskrit{Haritakañceva} \textsanskrit{pakkañcā}”ti … Tattha haritakanti haritameva \textsanskrit{paṇṇaṁ}. \textsanskrit{Pakkantiseditaṁ} \textsanskrit{paṇṇaṁ}}, “\textit{\textsanskrit{Haritakañceva} \textsanskrit{pakkañcā}}: … In this \textit{haritaka} is just green leaves. \textit{Pakka} is heated leaves.” } People complained and criticized them, “They’re just like householders who indulge in worldly pleasures!” They told the Buddha. 

\scrule{“A nun shouldn’t make up her eyes, apply facial marks, stare out a window, expose herself to view, organize dancing, appoint a sex worker, set up a bar, set up a slaughterhouse, set up a shop, make a loan, engage in trade, be attended on by a slave, be attended on by a servant, be attended on by animals, trade in raw and cooked greens, or wear felt. If she does, she commits an offense of wrong conduct.” }

The\marginnote{10.5.1} nuns from the group of six wore entirely blue robes, entirely yellow robes, entirely red robes, entirely magenta robes, entirely black robes, entirely orange robes, entirely beige robes, robes with borders made from a single piece of cloth, robes with long borders, robes with floral borders, robes with borders decorated with fruit designs, close-fitting jackets, and Lodh-tree robes. People complained and criticized them, “They’re just like householders who indulge in worldly pleasures!” They told the Buddha. 

\scrule{“A nun shouldn’t wear entirely blue robes, entirely yellow robes, entirely red robes, entirely magenta robes, entirely black robes, entirely orange robes, entirely beige robes, robes with borders made from a single piece of cloth, robes with long borders, robes with floral borders, robes with borders decorated with fruit designs, close-fitting jackets,\footnote{For further explanations of some of these, see comments at \href{https://suttacentral.net/pli-tv-kd8/en/brahmali\#29.1.6}{Kd 8:29.1.6}, \href{https://suttacentral.net/pli-tv-kd8/en/brahmali\#29.1.7}{Kd 8:29.1.7}, and \href{https://suttacentral.net/pli-tv-kd8/en/brahmali\#29.1.13}{Kd 8:29.1.13}. } or Lodh-tree robes. If she does, she commits an offense of wrong conduct.” }

\subsection*{Various rules}

On\marginnote{11.1.1} one occasion a nun who was dying said, “When I’m dead, give my requisites to the Sangha.”\footnote{For an explanation of the rendering “requisites” for \textit{\textsanskrit{parikkhāra}}, see Appendix of Technical Terms. } The monks and the nuns argued with one another, saying it belonged to their Sangha. They told the Buddha. 

\scrule{“If a dying nun, a dying trainee nun, or a dying novice nun says, ‘When I’m dead, give my requisites to the Sangha,’ then they’re for the Sangha of nuns, not for the Sangha of monks. But if a dying monk, a dying novice monk, a dying male lay follower, a dying female lay follower, or anyone else who is dying says, ‘When I’m dead, give my requisites to the Sangha,’ then they’re for the Sangha of monks, not for the Sangha of nuns.” }

At\marginnote{12.1.1} one time a woman who was a former wrestler went forth as a nun. Seeing a weak monk on a street, she hit him with her shoulder. He fell over. The monks complained and criticized her, “How could a nun hit a monk?” They told the Buddha. 

\scrule{“A nun shouldn’t hit a monk. If she does, she commits an offense of wrong conduct. When a nun sees a monk coming, she should make way for him by stepping off the path.” }

At\marginnote{13.1.1} one time a certain woman became pregnant by a lover while her husband was away. After having an abortion, she said to the nun who was associating with her family, “Venerable, please take this fetus away in your almsbowl.” The nun did as asked, covered her bowl with her upper robe, and left. 

At\marginnote{13.1.5} that time a certain alms-collecting monk had resolved not to eat without giving the first almsfood he had received to another monk or nun. He saw that nun and said, “Sister, please accept some almsfood.” “There’s no need, venerable.” 

He\marginnote{13.1.10} repeated his request a second and a third time, but received the same reply. He then told her about his resolution and again requested her to accept some almsfood. Being pressured by that monk, the nun brought out her bowl and showed it to him, saying, “See sir, there’s a fetus in my bowl. Please don’t tell anyone.” 

But\marginnote{13.1.20} he complained and criticized her, “How could a nun take a fetus away in her bowl?” He told the monks. The monks of few desires complained and criticized her, “How could a nun take a fetus away in her bowl?” They told the Buddha. 

\scrule{“A nun shouldn’t take a fetus away in her bowl. If she does, she commits an offense of wrong conduct. When a nun sees a monk, she should bring out her almsbowl and show it to him.” }

Soon\marginnote{13.2.1} afterwards, when the nuns from the group of six saw a monk, they turned their bowls upside down and showed him the bottom. The monks complained and criticized them, “How could the nuns from the group of six do this?” They told the Buddha. 

\scrule{“When a nun sees a monk, she shouldn’t turn her almsbowl upside down and show him the bottom. If she does, she commits an offense of wrong conduct. When a nun sees a monk, she should turn her bowl upright and then show it to him. And she should offer whatever food is in her bowl to that monk.” }

On\marginnote{14.1.1} one occasion a penis had been thrown out on a street in \textsanskrit{Sāvatthī}. The nuns stared at it. People jeered at them, and the nuns felt humiliated. When they had returned to the nuns’ dwelling place, they told the nuns what had happened. The nuns of few desires complained and criticized them, “How could those nuns stare at a penis?” They told the monks, who in turn told the Buddha. He said, 

\scrule{“A nun shouldn’t stare at a penis. If she does, she commits an offense of wrong conduct.” }

\subsection*{Requisites}

On\marginnote{15.1.1} one occasion people gave requisites to the monks, who then gave them to the nuns. People complained and criticized them, “How can the venerables give away to others what has been given to them for their own use? Don’t we know how to give?” They told the Buddha. 

\scrule{“Monks, you shouldn’t give away to others what has been given to you for your own use. If you do, you commit an offense of wrong conduct.” }

On\marginnote{15.1.9} one occasion the monks had an abundance of requisites. They told the Buddha. 

\scrule{“I allow you to give to the Sangha.” }

The\marginnote{15.1.12} abundance became even greater. They told the Buddha. 

\scrule{“I allow individuals to give away what belongs to them.”\footnote{Thus the initial rule is overturned. } }

On\marginnote{15.1.15} one occasion the monks had an abundance of stored requisites. They told the Buddha. 

\scrule{“I allow what’s stored by the monks to be received and used by the nuns.”\footnote{Sp 4.421: \textit{\textsanskrit{Bhikkhūnaṁ} \textsanskrit{sannidhiṁ} \textsanskrit{bhikkhunīhi} \textsanskrit{paṭiggāhāpetvāti} hiyyo \textsanskrit{paṭiggahetvā} \textsanskrit{ṭhapitamaṁsaṁ} ajja \textsanskrit{aññasmiṁ} anupasampanne asati \textsanskrit{bhikkhūhi} \textsanskrit{paṭiggāhāpetvā} \textsanskrit{bhikkhunīhi} \textsanskrit{paribhuñjitabbaṁ}}, “\textit{\textsanskrit{Bhikkhūnaṁ} \textsanskrit{sannidhiṁ} \textsanskrit{bhikkhunīhi} \textsanskrit{paṭiggāhāpetvā}}: in regard to meat received yesterday by the monks, but set aside, when there is no unordained person available today, then the monks may have the nuns receive it, and it may then be used by the nuns.” } }

On\marginnote{15.2.1} one occasion people gave requisites to the nuns, who then gave them to the monks. People complained and criticized them, “How can the nuns give away to others what has been given to them for their own use? Don’t we know how to give?” They told the Buddha. 

\scrule{“A nun shouldn’t give away to others what has been given to her for her own use. If she does, she commits an offense of wrong conduct.” }

On\marginnote{15.2.9} one occasion the nuns had an abundance of requisites. They told the Buddha. 

\scrule{“I allow a nun to give to the Sangha.” }

The\marginnote{15.2.12} abundance became even greater. They told the Buddha. 

\scrule{“I allow individuals to give away what belongs to them.”\footnote{Again, the initial rule is overturned. } }

On\marginnote{15.2.15} one occasion the nuns had an abundance of stored requisites. They told the Buddha. 

\scrule{“I allow what’s stored by the nuns to be received and used by the monks.” }

On\marginnote{16.1.1} one occasion the monks had an abundance of furniture, but the nuns were lacking. The nuns sent a message to the monks, saying, “Venerables, please lend us some furniture.” They told the Buddha. 

\scrule{“I allow you to lend furniture to the nuns.” }

Menstruating\marginnote{16.2.1} nuns sat down and lay down on upholstered beds and benches. The furniture was stained with blood. They told the Buddha. 

\scrule{“A nun shouldn’t sit down or lay down on upholstered beds or benches. If she does, she commits an offense of wrong conduct. I allow a communal robe.”\footnote{\textit{\textsanskrit{Āvasathacīvara}} literally means “a house robe”, that is, belonging to a particular house. As can be seen from \href{https://suttacentral.net/pli-tv-bi-vb-pc47/en/brahmali\#1.12.1}{Bi Pc 47:1.12.1}, it was used in turn by the nuns. Vin-vn-\textsanskrit{ṭ} 2300: \textit{\textsanskrit{Āvasathacīvaranti} “utuniyo bhikkhuniyo \textsanskrit{paribhuñjantū}”ti \textsanskrit{dinnaṁ} \textsanskrit{cīvaraṁ}}, “\textit{\textsanskrit{Āvasathacīvara}}: a robe given for menstruating nuns to use.” } }

The\marginnote{16.2.7} communal robe became stained with blood. They told the Buddha. 

\scrule{“I allow menstruation pads.” }

The\marginnote{16.2.10} pads fell off. They told the Buddha. 

\scrule{“I allow the nuns to attach a string and then bind it to the thigh.” }

The\marginnote{16.2.13} string snapped. They told the Buddha. 

\scrule{“I allow a loin cloth and a girdle.”\footnote{“Loin cloth” renders \textit{\textsanskrit{saṁvelliya}}. Sp 4.280: \textit{\textsanskrit{Saṁvelliyaṁ} \textsanskrit{nivāsentīti} \textsanskrit{mallakammakārādayo} viya \textsanskrit{kacchaṁ} \textsanskrit{bandhitvā} \textsanskrit{nivāsenti}}, “\textit{\textsanskrit{Saṁvelliyaṁ} \textsanskrit{nivāsenti}} means wearing the sarong, having bound the lower end like a wrestler or worker, etc.” } }

Soon\marginnote{16.2.16} afterwards the nuns from the group of six wore girdles all the time. People complained and criticized them, “They’re just like householders who indulge in worldly pleasures!” They told the Buddha. 

\scrule{“A nun shouldn’t wear a girdle all the time. If she does, she commits an offense of wrong conduct. I allow a nun to wear a girdle while she’s menstruating.” }

\scend{The second section for recitation is finished. }

\subsection*{The ordination ceremony }

At\marginnote{17.1.1} that time the full ordination had been given to women who lacked genitals, who had incomplete genitals, who did not menstruate, who menstruated continuously, who always wore menstruation pads, who were incontinent, who had genital prolapse, who lacked sexual organs, who were manlike, who had fistula, who were hermaphrodites.\footnote{“Who were incontinent” renders \textit{\textsanskrit{paggharantī}}. Sp 1.285: \textit{\textsanskrit{Paggharantīti} \textsanskrit{savantī}; \textsanskrit{sadā} te \textsanskrit{muttaṁ} \textsanskrit{savatīti} \textsanskrit{vuttaṁ} hoti}, “\textit{\textsanskrit{Paggharantī}} means flowing. It is said, ‘Their urine is always flowing.’” “Who had genital prolapse” renders \textit{\textsanskrit{sikharaṇī}}. Sp 1.285: \textit{\textsanskrit{Sikharaṇīti} \textsanskrit{bahinikkhantaāṇimaṁsā}}, “\textit{\textsanskrit{Sikharaṇī}} means a piece of flesh is protruding outside.” “Who lacked sexual organs” renders \textit{\textsanskrit{itthipaṇḍakā}}. Despite leaving the uncompounded term \textit{\textsanskrit{paṇḍaka}} untranslated, I have opted to follow the commentarial explanation of \textit{\textsanskrit{itthipaṇḍaka}}. Sp 1.285: \textit{\textsanskrit{Itthipaṇḍakāti} \textsanskrit{animittāva} vuccati}, “It is just a woman without genitals who is called an \textit{\textsanskrit{itthipaṇḍakā}}.” “Manlike” renders \textit{\textsanskrit{vepurisikā}}. Sp 1.285: \textit{\textsanskrit{Vepurisikāti} \textsanskrit{samassudāṭhikā} \textsanskrit{purisarūpā} \textsanskrit{itthī}}, “\textit{\textsanskrit{Vepurisikā}} means a woman who has a beard and a mustache like a man.” “Who had fistula” renders \textit{\textsanskrit{sambhinnā}}. Sp 1.285: \textit{\textsanskrit{Sambhinnāti} \textsanskrit{sambhinnavaccamaggapassāvamaggā}}, “\textit{\textsanskrit{Sambhinnā}} means the anus and the vagina are joined.” That \textit{\textsanskrit{passāvamagga}}, “the path of urine”, can refer to the vagina is clear from \href{https://suttacentral.net/pli-tv-bu-vb-pj1/en/brahmali\#9.1.10}{Bu Pj 1:9.1.10} where it refers to an orifice for sexual intercourse. Also, it is anatomically more likely that the anus and vagina would be conjoined, rather than the anus and the urethra. } They told the Buddha. 

\scrule{“The nun who is giving the full ordination should ask about twenty-four obstacles. }

And\marginnote{17.1.5} it should be done like this: ‘Do you lack genitals? Are your genitals incomplete? Do you not menstruate? Do you menstruate continuously? Do you always wear a menstruation pad? Are you incontinent? Do you have genital prolapse? Do you lack sexual organs? Are you manlike? Do you have fistula? Are you a hermaphrodite? Do you have any of these diseases: leprosy, abscesses, mild leprosy, tuberculosis, or epilepsy?\footnote{For an explanation of \textit{\textsanskrit{kuṭṭha}}, \textit{\textsanskrit{gaṇḍa}}, and \textit{\textsanskrit{kilāsa}}, see Appendix of Technical Terms. } Are you human? Are you a woman? Are you free from slavery? Are you free from debt? Are you employed by the king? Do you have permission from your parents and husband? Are you twenty years old? Do you have a full set of bowl and robes? What’s your name? What’s the name of your mentor?’” 

Soon\marginnote{17.2.1} afterwards the monks asked the nuns about the obstacles. Those seeking the full ordination were embarrassed, humiliated, and unable to respond. They told the Buddha. 

\scrule{“Only when a woman who’s free from obstacles has been fully ordained on one side in the Sangha of nuns, should you give her the full ordination in the Sangha of monks.”\footnote{“Who is free from the obstacles” renders \textit{\textsanskrit{visuddhā}}, often translated as “pure”. In the present case, however, the contextual meaning is that she is pure of or free from the obstacles to ordination. It seems to be used synonymously with the expression \textit{\textsanskrit{parisuddhā} antaryike dhamme}. } }

The\marginnote{17.2.5} nuns asked those seeking the full ordination about the obstacles without first instructing them. They were embarrassed, humiliated, and unable to respond. They told the Buddha. 

\scrule{“The nuns should instruct first and then ask about the obstacles.” }

They\marginnote{17.2.9} instructed them right there in the midst of the Sangha. Once again those seeking the full ordination were embarrassed, humiliated, and unable to respond. They told the Buddha. 

\scrule{“The nuns should instruct them at a distance and then ask about the obstacles in the midst of the Sangha. }

And\marginnote{17.2.13} it should be done like this. First the candidate should be told to choose a preceptor. Her bowl and robes should then be pointed out to her: ‘This is your bowl, this your outer robe, this your upper robe, this your sarong, this your chest wrap, and this your bathing robe. Now please go and stand over there.’” 

They\marginnote{17.3.1} were instructed by nuns who were ignorant and incompetent. And because they were badly instructed, they were once again embarrassed, humiliated, and unable to respond. They told the Buddha. 

\scrule{“A nun who’s ignorant and incompetent shouldn’t instruct. If she does, she commits an offense of wrong conduct. A nun who’s competent and capable should instruct.” }

They\marginnote{17.4.1} instructed without having been appointed. They told the Buddha. 

\scrule{“A nun shouldn’t instruct if she hasn’t been appointed. If she does, she commits an offense of wrong conduct. I allow a nun to instruct if she’s been appointed to do so. }

And\marginnote{17.4.6} it should be done like this. One is either appointed through oneself or through another. And how is one appointed through oneself? A competent and capable nun should inform the Sangha: 

‘Please,\marginnote{17.4.10} venerables, I ask the Sangha to listen. So-and-so is seeking the full ordination with venerable so-and-so.\footnote{The Pali reads: \textit{\textsanskrit{Ayaṁ} \textsanskrit{itthannāmā} \textsanskrit{itthannāmāya} \textsanskrit{ayyāya} \textsanskrit{upasampadāpekkhā}}. Taking the genitive case here to be the agent genitive, which seems to be the most obvious reading, this would mean, “So-and-so who is seeking to be fully ordained \emph{by} venerable so-and-so.” But it is the Sangha that ordains, not individuals, and so this translation does not seem quite right. According to Vmv 3.126 this phrase should be understood by means of this example: \textit{\textsanskrit{Ayaṁ} buddharakkhito \textsanskrit{āyasmato} dhammarakkhitassa \textsanskrit{saddhivihārikabhūto} \textsanskrit{upasampadāpekkho}}, “This Buddharakkhita, who is seeking the full ordination, is the student of Venerable Dhammarakkhita.” I have followed this interpretation, and thus my translation “with venerable so-and-so”. } If the Sangha is ready, I will instruct so-and-so.’ 

And\marginnote{17.4.14} how is one appointed through another? A competent and capable nun should inform the Sangha: 

‘Please,\marginnote{17.4.16} venerables, I ask the Sangha to listen. So-and-so is seeking the full ordination with venerable so-and-so. If the Sangha is ready, so-and-so will instruct so-and-so.’ 

The\marginnote{17.5.1} appointed nun should go to the one seeking the full ordination and say this: ‘Listen, so-and-so. Now is the time for you to tell the truth. You will be asked in the midst of the Sangha about various matters. If something is true, you should say, “Yes,” and if it’s not, you should say, “No.” Don’t be embarrassed or humiliated. This is what they’ll ask you: “Do you lack genitals? Are your genitals incomplete? Do you not menstruate? Do you menstruate continuously? Do you always wear a menstruation pad? Are you incontinent? Do you have genital prolapse? Do you lack sexual organs? Are you manlike? Do you have fistula? Are you a hermaphrodite? Do you have any of these diseases: leprosy, abscesses, mild leprosy, tuberculosis, or epilepsy? Are you human? Are you a woman? Are you free from slavery? Are you free from debt? Are you employed by the king? Do you have permission from your parents and husband? Are you twenty years old? Do you have a full set of bowl and robes? What’s your name? What’s the name of your mentor?”’” 

They\marginnote{17.5.13} then returned to the Sangha together. 

\scrule{“They shouldn’t return together. }

The\marginnote{17.5.15} instructor should return first and inform the Sangha: 

‘Please,\marginnote{17.5.16} venerables, I ask the Sangha to listen. So-and-so is seeking the full ordination with venerable so-and-so. She’s been instructed by me. If the Sangha is ready, so-and-so should come.’ 

And\marginnote{17.5.20} she should be told to come. She should then arrange her upper robe over one shoulder, pay respect at the feet of the nuns, squat on her heels, and raise her joined palms. She should then ask for the full ordination: 

‘Venerables,\marginnote{17.5.22} I ask the Sangha for the full ordination. Please lift me up out of compassion. For the second time, venerables, I ask the Sangha for the full ordination. Please lift me up out of compassion. For the third time, venerables, I ask the Sangha for the full ordination. Please lift me up out of compassion.’ 

A\marginnote{17.6.1} competent and capable nun should then inform the Sangha: 

‘Please,\marginnote{17.6.2} venerables, I ask the Sangha to listen. So-and-so is seeking the full ordination with venerable so-and-so. If the Sangha is ready, I will ask so-and-so about the obstacles. 

Listen,\marginnote{17.6.5} so-and-so. Now is the time for you to tell the truth. I will ask you about various matters. If something is true, you should say, “Yes,” and if it’s not, you should say, “No.” So: Do you lack genitals? Are your genitals incomplete? Do you not menstruate? Do you menstruate continuously? Do you always wear a menstruation pad? Are you incontinent? Do you have genital prolapse? Do you lack sexual organs? Are you manlike? Do you have fistula? Are you a hermaphrodite? Do you have any of these diseases: leprosy, abscesses, mild leprosy, tuberculosis, or epilepsy? Are you human? Are you a woman? Are you free from slavery? Are you free from debt? Are you employed by the king? Do you have permission from your parents and husband? Are you twenty years old? Do you have a full set of bowl and robes?\footnote{See notes above. } What’s your name? What’s the name of your mentor?’ 

A\marginnote{17.7.1} competent and capable nun should inform the Sangha: 

‘Please,\marginnote{17.7.2} venerables, I ask the Sangha to listen. So-and-so is seeking the full ordination with venerable so-and-so. She is free from obstacles and her bowl and robes are complete. So-and-so is asking the Sangha for the full ordination with so-and-so as her mentor. If the Sangha is ready, it should give the full ordination to so-and-so with so-and-so as her mentor. This is the motion. 

Please,\marginnote{17.7.8} venerables, I ask the Sangha to listen. So-and-so is seeking the full ordination with venerable so-and-so. She is free from obstacles and her bowl and robes are complete. So-and-so is asking the Sangha for the full ordination with so-and-so as her mentor. The Sangha gives the full ordination to so-and-so with so-and-so as her mentor. Any nun who approves of giving the full ordination to so-and-so with so-and-so as her mentor should remain silent. Any nun who doesn’t approve should speak up. 

For\marginnote{17.7.15} the second time, I speak on this matter. Please, venerables, I ask the Sangha to listen. So-and-so is seeking the full ordination with venerable so-and-so. She is free from obstacles and her bowl and robes are complete. So-and-so is asking the Sangha for the full ordination with so-and-so as her mentor. The Sangha gives the full ordination to so-and-so with so-and-so as her mentor. Any nun who approves of giving the full ordination to so-and-so with so-and-so as her mentor should remain silent. Any nun who doesn’t approve should speak up. 

For\marginnote{17.7.16} the third time, I speak on this matter. Please, venerables, I ask the Sangha to listen. So-and-so is seeking the full ordination with venerable so-and-so. She is free from obstacles and her bowl and robes are complete. So-and-so is asking the Sangha for the full ordination with so-and-so as her mentor. The Sangha gives the full ordination to so-and-so with so-and-so as her mentor. Any nun who approves of giving the full ordination to so-and-so with so-and-so as her mentor should remain silent. Any nun who doesn’t approve should speak up. 

The\marginnote{17.7.24} Sangha has given the full ordination to so-and-so with so-and-so as her mentor. The Sangha approves and is therefore silent. I’ll remember it thus.’ 

Straightaway\marginnote{17.8.1} she should be taken to the Sangha of monks. She should arrange her upper robe over one shoulder, pay respect at the feet of the monks, squat on her heels, and raise her joined palms. She should then ask for the full ordination: 

‘Venerables,\marginnote{17.8.2} I’m seeking the full ordination with venerable so-and-so. I’m free from obstacles and have been fully ordained on one side in the Sangha of nuns.\footnote{“Who is free from the obstacles” renders \textit{\textsanskrit{visuddhā}}, often translated as “pure”. In the present case, however, the contextual meaning is that she is pure of or free from the obstacles to ordination. It seems to be used synonymously with the expression \textit{\textsanskrit{parisuddhā} antaryikehi dhammehi} above. } I ask the Sangha for the full ordination. Please lift me up out of compassion. 

Venerables,\marginnote{17.8.6} I’m seeking the full ordination with venerable so-and-so. I’m free from obstacles and have been fully ordained on one side in the Sangha of nuns. For the second time, I ask the Sangha for the full ordination. Please lift me up out of compassion. 

Venerables,\marginnote{17.8.10} I’m seeking the full ordination with venerable so-and-so. I’m free from obstacles and have been fully ordained on one side in the Sangha of nuns. For the third time, I ask the Sangha for the full ordination. Please lift me up out of compassion.’ 

A\marginnote{17.8.14} competent and capable monk should inform the Sangha: 

‘Please,\marginnote{17.8.15} venerables, I ask the Sangha to listen. So-and-so is seeking the full ordination with so-and-so. She is free from obstacles and has been fully ordained on one side in the Sangha of nuns. She is asking the Sangha for the full ordination with so-and-so as her mentor. If the Sangha is ready, it should give her the full ordination with so-and-so as her mentor. This is the motion. 

Please,\marginnote{17.8.21} venerables, I ask the Sangha to listen. So-and-so is seeking the full ordination with so-and-so. She is free from obstacles and has been fully ordained on one side in the Sangha of nuns. So-and-so is asking the Sangha for the full ordination with so-and-so as her mentor. The Sangha gives her the full ordination with so-and-so as her mentor. Any monk who approves of giving so-and-so the full ordination with so-and-so as her mentor should remain silent. Any monk who doesn’t approve should speak up. 

For\marginnote{17.8.28} the second time, I speak on this matter. Please, venerables, I ask the Sangha to listen. So-and-so is seeking the full ordination with so-and-so. She is free from obstacles and has been fully ordained on one side in the Sangha of nuns. So-and-so is asking the Sangha for the full ordination with so-and-so as her mentor. The Sangha gives her the full ordination with so-and-so as her mentor. Any monk who approves of giving so-and-so the full ordination with so-and-so as her mentor should remain silent. Any monk who doesn’t approve should speak up. 

For\marginnote{17.8.29} the third time, I speak on this matter. Please, venerables, I ask the Sangha to listen. So-and-so is seeking the full ordination with so-and-so. She is free from obstacles and has been fully ordained on one side in the Sangha of nuns. So-and-so is asking the Sangha for the full ordination with so-and-so as her mentor. The Sangha gives her the full ordination with so-and-so as her mentor. Any monk who approves of giving so-and-so the full ordination with so-and-so as her mentor should remain silent. Any monk who doesn’t approve should speak up. 

The\marginnote{17.8.37} Sangha has given so-and-so the full ordination with so-and-so as her mentor. The Sangha approves and is therefore silent. I’ll remember it thus.’ 

Straightaway\marginnote{17.8.39} the time should be noted and the date should be pointed out. These should be declared jointly to everyone.\footnote{\textit{\textsanskrit{Saṅgīti} \textsanskrit{ācikkhitabbā}}, literally, “a joint recitation is to be declared”, which is rather cryptic. Sp 3.128: \textit{\textsanskrit{Saṅgītīti} idameva \textsanskrit{sabbaṁ} ekato \textsanskrit{katvā} “\textsanskrit{tvaṁ} \textsanskrit{kiṁ} labhasi, \textsanskrit{kā} te \textsanskrit{chāyā}, \textsanskrit{kiṁ} \textsanskrit{utuppamāṇaṁ}, ko \textsanskrit{divasabhāgo}”ti \textsanskrit{puṭṭho} “\textsanskrit{idaṁ} \textsanskrit{nāma} \textsanskrit{labhāmi} – \textsanskrit{vassaṁ} \textsanskrit{vā} \textsanskrit{hemantaṁ} \textsanskrit{vā} \textsanskrit{gimhaṁ} \textsanskrit{vā}, \textsanskrit{ayaṁ} me \textsanskrit{chāyā}, \textsanskrit{idaṁ} \textsanskrit{utuppamāṇaṁ}, \textsanskrit{ayaṁ} \textsanskrit{divasabhāgoti} \textsanskrit{vadeyyāsī}”ti \textsanskrit{evaṁ} \textsanskrit{ācikkhitabbaṁ}}, “\textit{\textsanskrit{Saṅgīti}}: here it means: having brought everyone together, it should pointed out: ‘When you are asked, “What did you have: what was your time; what was your date?” you should reply, “I had this: it was the rainy season/the cold season/the hot season; it was this time; it was this date.”’” The point seems to be that a newly ordained nun should remember the time and date of her ordination so that she may respond to questions about it in future. In the above, the word “date” renders the combined meaning of \textit{\textsanskrit{utuppamāṇa}} and \textit{\textsanskrit{divasabhāga}}. Vmv 3.128: \textit{\textsanskrit{Chāyādikameva} \textsanskrit{sabbaṁ} \textsanskrit{saṅgahetvā} \textsanskrit{gāyitabbato} kathetabbato \textsanskrit{saṅgītīti} \textsanskrit{āha} “\textsanskrit{idamevā}”\textsanskrit{tiādi}. Tattha ekato \textsanskrit{katvā} \textsanskrit{ācikkhitabbaṁ}. \textsanskrit{Tvaṁ} \textsanskrit{kiṁ} \textsanskrit{labhasīti} \textsanskrit{tvaṁ} \textsanskrit{upasampādanakāle} \textsanskrit{kataravassaṁ}, \textsanskrit{katarautuñca} labhasi, \textsanskrit{katarasmiṁ} te \textsanskrit{upasampadā} \textsanskrit{laddhāti} attho}, “‘Here’ etc. means: having collected all—that is the time, etc.—\textit{\textsanskrit{saṅgīti}} is said because it is to be chanted, because it is to be declared. In regard to this, having brought (everyone) together, it is to be pointed out. ‘What did you have’ means: at the time of the ordination, what was your year and season; your ordination was obtained in which one?” } The nuns should be told to point out the three supports and the eight things not to be done.”\footnote{That is, the nuns who did the ordination should point this out to those who have just been ordained. Vmv 4.425: \textit{Tayo nissayeti \textsanskrit{rukkhamūlasenāsanassa} \textsanskrit{tāsaṁ} alabbhanato \textsanskrit{vuttaṁ}}, “\textit{Tayo nissaye}: this is said because they do not obtain a resting place at the foot of a tree.” } 

On\marginnote{18.1.1} one occasion the right time for eating passed while the nuns were trying to find the right seats in the dining hall.\footnote{Sp 4.426: \textit{\textsanskrit{Āsanaṁ} \textsanskrit{saṁkasāyantiyo} \textsanskrit{kālaṁ} \textsanskrit{vītināmesunti} \textsanskrit{aññaṁ} \textsanskrit{vuṭṭhāpetvā} \textsanskrit{aññaṁ} \textsanskrit{nisīdāpentiyo} \textsanskrit{bhojanakālaṁ} \textsanskrit{atikkāmesuṁ}}, “\textit{\textsanskrit{Āsanaṁ} \textsanskrit{saṁkasāyantiyo} \textsanskrit{kālaṁ} \textsanskrit{vītināmesuṁ}}: making (nuns) get up and sit down, they went beyond the time for eating.” } They told the Buddha. 

\scrule{“I allow eight nuns to be seated according to seniority, but the rest according to their time of arrival.”\footnote{Sp 4.426: \textit{\textsanskrit{Aṭṭhannaṁ} \textsanskrit{bhikkhunīnaṁ} \textsanskrit{yathāvuḍḍhanti} ettha sace pure \textsanskrit{aṭṭhasu} \textsanskrit{nisinnāsu} \textsanskrit{tāsaṁ} \textsanskrit{abbhantarimā} \textsanskrit{aññā} \textsanskrit{āgacchati}, \textsanskrit{sā} attano \textsanskrit{navakaṁ} \textsanskrit{uṭṭhāpetvā} \textsanskrit{nisīdituṁ} labhati. \textsanskrit{Yā} pana \textsanskrit{aṭṭhahipi} \textsanskrit{navakatarā}, \textsanskrit{sā} sacepi \textsanskrit{saṭṭhivassā} hoti, \textsanskrit{āgatapaṭipāṭiyāva} \textsanskrit{nisīdituṁ} labhati}, “\textit{\textsanskrit{Aṭṭhannaṁ} \textsanskrit{bhikkhunīnaṁ} \textsanskrit{yathāvuḍḍhaṁ}}: here, if another nun arrives who belongs (seniority-wise) among the first eight nuns who are seated, she may make a nun junior to her get up and then sit down. But whoever is junior to the eight, even if she has sixty years seniority, should sit down in accordance with the order of arrival.” } }

When\marginnote{18.1.4} they heard about the Buddha’s allowance, eight nuns reserved places everywhere according to seniority, the rest getting places according to their time of arrival. They told the Buddha. 

\scrule{“I allow eight nuns to be seated in the dining hall according to seniority and the rest according to their time of arrival,\footnote{For an explanation of the rendering “dining hall” for \textit{bhattagga}, see Appendix of Technical Terms. } but not anywhere else. If a nun makes a reservation anywhere apart from the dining hall, she commits an offense of wrong conduct.” }

\subsection*{Invitation}

At\marginnote{19.1.1} that time the nuns did not do the invitation ceremony.\footnote{For an explanation of the rendering “invitation ceremony” for \textit{\textsanskrit{pavāreti}}, see Appendix of Technical Terms. } They told the Buddha. 

\scrule{“A nun should do the invitation ceremony. If she doesn’t, she should be dealt with according to the rule.”\footnote{See \href{https://suttacentral.net/pli-tv-bi-vb-pc57/en/brahmali\#1.15.1}{Bi Pc 57:1.15.1}. } }

At\marginnote{19.1.5} that time the nuns did the invitation ceremony among themselves, but not with the Sangha of monks. They told the Buddha. 

\scrule{“After doing the invitation ceremony with the nuns, a nun should do the invitation ceremony with the monks. If she doesn’t, she should be dealt with according to the rule.” }

The\marginnote{19.1.9} nuns did their invitation ceremony together with the monks. They made a racket.\footnote{The point seems to be that the nuns did \textit{their own} invitation ceremony in the presence of the monks. The appropriate procedure, however, is to do it first among themselves, and only then with the monks. } They told the Buddha. 

\scrule{“The nuns shouldn’t do their invitation ceremony together with the monks. Any nun who does commits an offense of wrong conduct.” }

On\marginnote{19.1.13} one occasion when the nuns did the invitation ceremony before the meal, they did not finish until after the time for eating. They told the Buddha. 

\scrule{“I allow the nuns to do the invitation ceremony after the meal.” }

Doing\marginnote{19.1.16} the invitation ceremony after the meal, they did not finish until it was too late in the evening. They told the Buddha. 

\scrule{“I allow the nuns to do the invitation ceremony among themselves on one day and with the monks on the following day.” }

At\marginnote{19.2.1} that time the whole Sangha of nuns did the invitation ceremony. They made a racket. They told the Buddha. 

\scrule{“I allow the nuns to appoint one capable and competent nun to do the invitation ceremony with the Sangha of monks on behalf of the Sangha of nuns. She should be appointed like this. First a nun should be asked, and then a competent and capable nun should inform the Sangha: }

‘Please,\marginnote{19.2.7} venerables, I ask the Sangha to listen. If the Sangha is ready, it should appoint nun so-and-so to do the invitation ceremony with the Sangha of monks on behalf of the Sangha of nuns. This is the motion. 

Please,\marginnote{19.2.10} venerables, I ask the Sangha to listen. The Sangha appoints nun so-and-so to do the invitation ceremony with the Sangha of monks on behalf of the Sangha of nuns. Any nun who approves of appointing nun so-and-so to do the invitation ceremony with the Sangha of monks on behalf of the Sangha of nuns should remain silent. Any nun who doesn’t approve should speak up. 

The\marginnote{19.2.14} Sangha has appointed nun so-and-so to do the invitation ceremony with the Sangha of monks on behalf of the Sangha of nuns. The Sangha approves and is therefore silent. I’ll remember it thus.’ 

The\marginnote{19.3.1} appointed nun should take the Sangha of nuns to the Sangha of monks. She should then arrange her upper robe over one shoulder, squat on her heels, raise her joined palms, and say: ‘Venerables, the Sangha of nuns invites the Sangha of monks to correct it concerning what you’ve seen, heard, or suspect. Please correct the Sangha of nuns, venerables, out of compassion. If the Sangha of nuns sees a fault, it will make amends.\footnote{For an explanation of the rendering “correct” for \textit{vadati}, see Appendix of Technical Terms. } For the second time, the Sangha of nuns invites the Sangha of monks to correct it concerning what you’ve seen, heard, or suspect. Please correct the Sangha of nuns, venerables, out of compassion. If the Sangha of nuns sees a fault, it will make amends. For the third time, the Sangha of nuns invites the Sangha of monks to correct it concerning what you’ve seen, heard, or suspect. Please correct the Sangha of nuns, venerables, out of compassion. If the Sangha of nuns sees a fault, it will make amends.’” 

\subsection*{Mutual authority between monks and nuns}

At\marginnote{20.1.1} that time the nuns canceled the monks’ observance-day ceremony and their invitation ceremony; they directed them, gave them instructions, got permission from them to correct them, accused them of offenses, and reminded them of offenses.\footnote{Sp 4.76: \textit{Na \textsanskrit{savacanīyaṁ} \textsanskrit{kātabbanti} \textsanskrit{palibodhatthāya} \textsanskrit{vā} \textsanskrit{pakkosanatthāya} \textsanskrit{vā} \textsanskrit{savacanīyaṁ} na \textsanskrit{kātabbaṁ}, \textsanskrit{palibodhatthāya} hi karonto “\textsanskrit{ahaṁ} \textsanskrit{āyasmantaṁ} \textsanskrit{imasmiṁ} \textsanskrit{vatthusmiṁ} \textsanskrit{savacanīyaṁ} karomi, \textsanskrit{imamhā} \textsanskrit{āvāsā} ekapadampi \textsanskrit{mā} \textsanskrit{pakkāmi}, \textsanskrit{yāva} na \textsanskrit{taṁ} \textsanskrit{adhikaraṇaṁ} \textsanskrit{vūpasantaṁ} \textsanskrit{hotī}”ti \textsanskrit{evaṁ} karoti. \textsanskrit{Pakkosanatthāya} karonto “\textsanskrit{ahaṁ} te \textsanskrit{savacanīyaṁ} karomi, ehi \textsanskrit{mayā} \textsanskrit{saddhiṁ} \textsanskrit{vinayadharānaṁ} \textsanskrit{sammukhībhāvaṁ} \textsanskrit{gacchāmā}”ti \textsanskrit{evaṁ} karoti; tadubhayampi na \textsanskrit{kātabbaṁ}}, “\textit{Na \textsanskrit{savacanīyaṁ} \textsanskrit{kātabba}}: \textit{\textsanskrit{savacanīya}} is not to be done for the purpose of (creating) an obstacle or for the purpose of summoning. Acting for the purpose of (creating) an obstacle is done like this: ‘I am doing \textit{\textsanskrit{savacanīya}} against the venerable in regard to this offense: he must not depart from this monastery even with one foot so long as this legal issue has not been resolved.’ Acting for the purpose of summoning is done like this: ‘I am doing \textit{\textsanskrit{savacanīya}} against you: come with me and let us go to the presence of a master of the Monastic Law.’ Neither of these is to be done.” Sp-\textsanskrit{ṭ} 4.76: \textit{\textsanskrit{Savacanīyanti} \textsanskrit{sadosaṁ}}, “\textit{\textsanskrit{Savacanīyan}}: with flaw.” Vmv 4.76: \textit{\textsanskrit{Savacanīyanti} ettha “sadosa”nti \textsanskrit{atthaṁ} vadati. Attano vacane pavattanakammanti evamettha attho \textsanskrit{daṭṭhabbo}, “\textsanskrit{mā} \textsanskrit{pakkamāhī}”ti \textsanskrit{vā} “ehi \textsanskrit{vinayadharānaṁ} \textsanskrit{sammukhībhāva}”nti \textsanskrit{vā} \textsanskrit{evaṁ} attano \textsanskrit{āṇāya} \textsanskrit{pavattanakakammaṁ} na \textsanskrit{kātabbanti} \textsanskrit{adhippāyo}}, “\textit{\textsanskrit{Savacanīyan}}: here he says the meaning is ‘with flaw’. Here the meaning is to be understood as bringing about an action when speaking oneself: ‘Don’t leave,’ ‘Come to the presence of a master of the Monastic Law’. The intention is one is not to do the bringing about an action in this way because of a command from oneself.” \textit{\textsanskrit{Okāsaṁ} \textsanskrit{kārenti}} literally means “they have (the other) make an opportunity”. The idiomatic meaning is “they get permission” or usually “they get permission from someone to correct them”. For the sake of clarity, I use both of these renderings depending on the context. } They told the Buddha. 

\scrule{“A nun shouldn’t cancel the observance-day ceremony of a monk. If she does, it’s not valid, and she commits an offense of wrong conduct. }

\scrule{A nun shouldn’t cancel the invitation ceremony of a monk. If she does, it’s not valid, and she commits an offense of wrong conduct. }

\scrule{A nun shouldn’t direct a monk. If she does, it’s not valid, and she commits an offense of wrong conduct. }

\scrule{A nun shouldn’t give instructions to a monk. If she does, it’s not valid, and she commits an offense of wrong conduct. }

\scrule{A nun shouldn’t get permission from a monk to correct him. If she does, it’s not valid, and she commits an offense of wrong conduct. }

\scrule{A nun shouldn’t accuse a monk of an offense. If she does, it’s not valid, and she commits an offense of wrong conduct. }

\scrule{A nun shouldn’t remind a monk of an offense. If she does, it’s not valid, and she commits an offense of wrong conduct.” }

At\marginnote{20.1.24} that time the monks canceled the nuns’ observance-day ceremony and their invitation ceremony; they directed them, gave them instructions, got permission from them to correct them, accused them of offenses, and reminded them of offenses. They told the Buddha. 

\scrule{“A monk may cancel the observance-day ceremony of a nun. If he does, it’s valid, and there’s no offense for him. }

\scrule{A monk may cancel the invitation ceremony of a nun. If he does, it’s valid, and there’s no offense for him. }

\scrule{A monk may direct a nun. If he does, it’s valid, and there’s no offense for him. }

\scrule{A monk may give instructions to a nun. If he does, it’s valid, and there’s no offense for him. }

\scrule{A monk may get permission from a nun to correct her. If he does, it’s valid, and there’s no offense for him. }

\scrule{A monk may accuse a nun of an offense. If he does, it’s valid, and there’s no offense for him. }

\scrule{A monk may remind a nun of an offense. If he does, it’s valid, and there’s no offense for him.” }

At\marginnote{21.1.1} that time the nuns from the group of six traveled in vehicles, sometimes pulled by women with men inside, at other times pulled by men with women inside. People complained and criticized them, “You’d think they were at the Ganges festival!” They told the Buddha. 

\scrule{“A nun shouldn’t travel in a vehicle. If she does, she should be dealt with according to the rule.”\footnote{That is, \href{https://suttacentral.net/pli-tv-bi-vb-pc85/en/brahmali\#1.2.5.1}{Bi Pc 85:1.2.5.1}. } }

Soon\marginnote{21.1.8} afterwards there was a sick nun who was unable to go on foot. They told the Buddha. 

\scrule{“I allow a vehicle for one who is sick.” }

The\marginnote{21.1.11} nuns thought, “A vehicle pulled by women or by men?” They told the Buddha. 

\scrule{“I allow a rickshaw pulled either by men or by women.” }

Soon\marginnote{21.1.15} afterwards a certain nun was even more uncomfortable when jolted around in a vehicle. They told the Buddha. 

\scrule{“I allow a palanquin and a litter.” }

\subsection*{Ordination by messenger}

At\marginnote{22.1.1} that time the courtesan \textsanskrit{Aḍḍhakāsī} had gone forth with the nuns. She wanted to go to \textsanskrit{Sāvatthī} to get the full ordination in the presence of the Buddha. Some scoundrels heard about this and besieged the road. When \textsanskrit{Aḍḍhakāsī} heard about this, she sent a message to the Buddha, saying, “I want the full ordination. What should I do?” Soon afterwards the Buddha gave a teaching and addressed the monks: 

\scrule{“I allow you to give the full ordination also by messenger.” }

They\marginnote{22.2.1} ordained with a monk as messenger. They told the Buddha. 

\scrule{“You shouldn’t give the full ordination with a monk as messenger. If you do, you commit an offense of wrong conduct.” }

They\marginnote{22.2.5} ordained with a trainee nun as messenger … They ordained with a novice monk as messenger … They ordained with a novice nun as messenger … They ordained with an ignorant and incompetent nun as messenger. 

\scrule{“You shouldn’t give the full ordination with an ignorant and incompetent nun as messenger. If you do, you commit an offense of wrong conduct. You should give the full ordination with a capable and competent nun as messenger. }

That\marginnote{22.3.1} messenger nun should go to the Sangha, arrange her upper robe over one shoulder, bow down at the feet of the monks, squat on her heels, raise her joined palms, and say: 

‘Venerables,\marginnote{22.3.2} so-and-so is seeking the full ordination with venerable so-and-so. She is free from obstacles and has been fully ordained on one side in the Sangha of nuns.\footnote{These are the obstacles mentioned in the ordination ceremony as described above. } She hasn’t come because of an obstruction. She asks the Sangha for the full ordination. Please lift her up out of compassion. 

Venerables,\marginnote{22.3.7} so-and-so is seeking the full ordination with venerable so-and-so. She is free from obstacles and has been fully ordained on one side in the Sangha of nuns. She hasn’t come because of an obstruction. For the second time, she asks the Sangha for the full ordination. Please lift her up out of compassion. 

Venerables,\marginnote{22.3.12} so-and-so is seeking the full ordination with venerable so-and-so. She is free from obstacles and has been fully ordained on one side in the Sangha of nuns. She hasn’t come because of an obstruction. For the third time, she asks the Sangha for the full ordination. Please lift her up out of compassion.’ 

A\marginnote{22.3.17} competent and capable monk should inform the Sangha: 

‘Please,\marginnote{22.3.18} venerables, I ask the Sangha to listen. So-and-so is seeking the full ordination with so-and-so. She is free from obstacles and has been fully ordained on one side in the Sangha of nuns. She’s not present because of an obstruction. So-and-so is asking the Sangha for the full ordination with so-and-so as her mentor. If the Sangha is ready, it should give her the full ordination with so-and-so as her mentor. This is the motion. 

Please,\marginnote{22.3.25} venerables, I ask the Sangha to listen. So-and-so is seeking the full ordination with so-and-so. She is free from obstacles and has been fully ordained on one side in the Sangha of nuns. She’s not present because of an obstruction. So-and-so is asking the Sangha for the full ordination with so-and-so as her mentor. The Sangha gives her the full ordination with so-and-so as her mentor. Any monk who approves of giving so-and-so the full ordination with so-and-so as her mentor should remain silent. Any monk who doesn’t approve should speak up. 

For\marginnote{22.3.33} the second time, I speak on this matter. Please, venerables, I ask the Sangha to listen. So-and-so is seeking the full ordination with so-and-so. She is free from obstacles and has been fully ordained on one side in the Sangha of nuns. She’s not present because of an obstruction. So-and-so is asking the Sangha for the full ordination with so-and-so as her mentor. The Sangha gives her the full ordination with so-and-so as her mentor. Any monk who approves of giving so-and-so the full ordination with so-and-so as her mentor should remain silent. Any monk who doesn’t approve should speak up. 

For\marginnote{22.3.34} the third time, I speak on this matter. Please, venerables, I ask the Sangha to listen. So-and-so is seeking the full ordination with so-and-so. She is free from obstacles and has been fully ordained on one side in the Sangha of nuns. She’s not present because of an obstruction. So-and-so is asking the Sangha for the full ordination with so-and-so as her mentor. The Sangha gives her the full ordination with so-and-so as her mentor. Any monk who approves of giving so-and-so the full ordination with so-and-so as her mentor should remain silent. Any monk who doesn’t approve should speak up. 

The\marginnote{22.3.43} Sangha has given so-and-so the full ordination with so-and-so as her mentor. The Sangha approves and is therefore silent. I’ll remember it thus.’ 

Straightaway\marginnote{22.3.45} the time should be noted and the date should be pointed out. These should be declared jointly to everyone. The nuns should be told to point out the three supports and the eight things not to be done to the newly ordained nun.” 

\subsection*{Various regulations for nuns}

At\marginnote{23.1.1} that time nuns were staying in the wilderness. Scoundrels raped them.\footnote{The word \textit{\textsanskrit{dūseti}} has a wide application, but in this sort of context it seems to mean “rape”. Commenting on the related term \textit{(\textsanskrit{bhikkhunī}-)\textsanskrit{dūsaka}}, Sp 3.115 says: \textit{\textsanskrit{Bhikkhunidūsako} bhikkhaveti ettha yo \textsanskrit{pakatattaṁ} \textsanskrit{bhikkhuniṁ} \textsanskrit{tiṇṇaṁ} \textsanskrit{maggānaṁ} \textsanskrit{aññatarasmiṁ} \textsanskrit{dūseti}, \textsanskrit{ayaṁ} \textsanskrit{bhikkhunidūsako} \textsanskrit{nāma}}, “\textit{\textsanskrit{Bhikkhunidūsako} bhikkhave}: in this context it means whoever violates an ordinary nun through one of three orifices (vagina, anus, or mouth) is called a \textit{\textsanskrit{bhikkhunidūsaka}}.” See also discussion of \textit{\textsanskrit{dūsaka}} in Appendix of Technical Terms. } They told the Buddha. 

\scrule{“A nun shouldn’t stay in the wilderness. If she does, she commits an offense of wrong conduct.” }

On\marginnote{24.1.1} one occasion a lay follower gave a storehouse to the Sangha of nuns. They told the Buddha. 

\scrule{“I allow a storehouse.” }

The\marginnote{24.1.4} storehouse was insufficient. They told the Buddha. 

\scrule{“I allow a dwelling place.”\footnote{It’s a bit unexpected that the text jumps from outbuilding to dwelling place. Perhaps these buildings were multipurpose, and the overall number was insufficient. } }

The\marginnote{24.1.7} dwelling place was insufficient. They told the Buddha. 

\scrule{“I allow building work.”\footnote{Sp 4.431 explains: \textit{Navakammanti \textsanskrit{saṅghassatthāya} \textsanskrit{bhikkhuniyā} navakammampi \textsanskrit{kātuṁ} \textsanskrit{anujānāmīti} attho}, “‘Building work’: the meaning is ‘I allow the nuns to build for the benefit of the Sangha.’” } }

The\marginnote{24.1.10} building work was insufficient. They told the Buddha. 

\scrule{“I allow you to build for individuals.” }

At\marginnote{25.1.1} one time a pregnant woman went forth as a nun. After giving birth, she asked the nuns what to do with the baby boy. They told the Buddha. 

\scrule{“I allow you to rear him until he becomes self-reliant.”\footnote{\textit{\textsanskrit{Yāva} so \textsanskrit{dārako} \textsanskrit{viññutaṁ} \textsanskrit{pāpuṇāti}}, literally, “until the boy reaches discernment”. Sp 4.432: \textit{\textsanskrit{Yāva} so \textsanskrit{dārako} \textsanskrit{viññutaṁ} \textsanskrit{pāpuṇātīti} \textsanskrit{yāva} \textsanskrit{khādituṁ} \textsanskrit{bhuñjituṁ} \textsanskrit{nahāyituñca} \textsanskrit{maṇḍituñca} attano \textsanskrit{dhammatāya} \textsanskrit{sakkotīti} attho}, “\textit{\textsanskrit{Yāva} so \textsanskrit{dārako} \textsanskrit{viññutaṁ} \textsanskrit{pāpuṇāti}} means until he is able to eat, bathe, and groom himself.” } }

That\marginnote{25.1.7} nun thought, “I’m not allowed to stay by myself and other nuns are not allowed to stay with a male child. What should I do now?”\footnote{That she is unable to stay by herself probably refers to \href{https://suttacentral.net/pli-tv-bi-vb-ss3/en/brahmali\#4.14.1}{Bi Ss 3:4.14.1}. } They told the Buddha. 

\scrule{“The nuns should appoint a nun as her companion. }

And\marginnote{25.1.11} she should be appointed like this. First a nun should be asked, and then a capable and competent nun should inform the Sangha: 

‘Please,\marginnote{25.1.13} venerables, I ask the Sangha to listen. If the Sangha is ready, it should appoint nun so-and-so as a companion to nun so-and-so. This is the motion. 

Please,\marginnote{25.1.16} venerables, I ask the Sangha to listen. The Sangha appoints nun so-and-so as a companion to nun so-and-so. Any nun who approves of appointing nun so-and-so as a companion to nun so-and-so should remain silent. Any nun who doesn’t approve should speak up. 

The\marginnote{25.1.20} Sangha has appointed nun so-and-so as a companion to nun so-and-so. The Sangha approves and is therefore silent. I’ll remember it thus.’” 

That\marginnote{25.2.1} companion nun thought, “How should I act in regard to this boy?” They told the Buddha. 

\scrule{“Apart from staying in the same dwelling, the companion nun should act toward him as she would toward any other male.”\footnote{Sp 4.432: \textit{\textsanskrit{Ṭhapetvā} \textsanskrit{sāgāranti} \textsanskrit{sahagāraseyyamattaṁ} \textsanskrit{ṭhapetvā}}, “\textit{\textsanskrit{Ṭhapetvā} \textsanskrit{sāgāra}}: apart from a mere sleeping place in the same house.” } }

On\marginnote{25.3.1} one occasion a nun who had committed a heavy offense was undertaking the trial period. She thought, “I’m not allowed to stay by myself and other nuns are not allowed to stay with me. What should I do now?”\footnote{In other words, there is a conflict between \href{https://suttacentral.net/pli-tv-bi-vb-ss3/en/brahmali\#4.14.1}{Bi Ss 3:4.14.1}, according to which a nun cannot stay by herself, and the rules for the trial period, \textit{\textsanskrit{mānatta}}, according to which a nun cannot stay together with another nun, see \href{https://suttacentral.net/pli-tv-kd12/en/brahmali\#5.1.63}{Kd 12:5.1.63}–5.1.65. } They told the Buddha. 

\scrule{“You should appoint a nun as her companion. }

And\marginnote{25.3.6} she should be appointed like this. First a nun should be asked, and then a capable and competent nun should inform the Sangha: 

‘Please,\marginnote{25.3.8} venerables, I ask the Sangha to listen. If the Sangha is ready, it should appoint nun so-and-so as a companion to nun so-and-so. This is the motion. 

Please,\marginnote{25.3.11} venerables, I ask the Sangha to listen. The Sangha appoints nun so-and-so as a companion to nun so-and-so. Any nun who approves of appointing nun so-and-so as a companion to nun so-and-so should remain silent. Any nun who doesn’t approve should speak up. 

The\marginnote{25.3.15} Sangha has appointed nun so-and-so as a companion to nun so-and-so. The Sangha approves and is therefore silent. I’ll remember it thus.’” 

On\marginnote{26.1.1} one occasion a nun verbally renounced the training and disrobed.\footnote{For an explanation of the rendering “disrobe” for \textit{vibbhamati}, see Appendix of Technical Terms. } Later she returned and asked the nuns for the full ordination. They told the Buddha. 

\scrule{“A nun can’t verbally renounce the training.\footnote{The difference between “verbally renouncing” and “disrobing” just below is that disrobal refers to the act of literally removing one’s robes. } When she disrobes, she is no longer a nun.”\footnote{“Disrobes” renders \textit{\textsanskrit{vibbhantā}}. See Appendix of Technical Terms.  for a brief discussion of this word. Sp 4.434: \textit{\textsanskrit{Sā} puna \textsanskrit{upasampadaṁ} na labhati}, “She does not obtain the full ordination again”. Yet this commentarial statement has no basis in the Canonical texts. In fact it contrasts with the immediately following case, in the Canonical text, of a nun who returns after joining another religious community. In this case it is explicitly stated that she cannot reordain. The fact that this is not explicitly stated in the present case seems to suggest that she can reordain after disrobing. } }

On\marginnote{26.2.1} one occasion a nun joined another religious community while still wearing her robes.\footnote{I read \textit{\textsanskrit{sakāsāva}} with SRT, the Sri Lankan Buddha \textsanskrit{Jayantī} \textsanskrit{Tipiṭaka}, and the PTS edition, instead of \textit{\textsanskrit{sakāvāsa}}. } Later she returned and asked the nuns for the full ordination. They told the Buddha. 

\scrule{“If a nun goes over to another religious community while still wearing her robes and then returns, she shouldn’t be given the full ordination again.”\footnote{That is, she has gone over to the other religion without first disrobing. } }

At\marginnote{27.1.1} that time the nuns, being afraid of wrongdoing, did not consent to men bowing down to them, or to men shaving their heads, cutting their nails, or treating their sores. They told the Buddha. 

\scrule{“I allow you to consent to these things.” }

At\marginnote{27.2.1} that time the nuns were sitting cross-legged, enjoying the touch of their heels. They told the Buddha. 

\scrule{“A nun shouldn’t sit cross-legged. If she does, she commits an offense of wrong conduct.” }

On\marginnote{27.2.5} one occasion there was a sick nun who was not comfortable without sitting cross-legged. They told the Buddha. 

\scrule{“I allow a nun to sit half-cross-legged.”\footnote{That is, “cross-legged” with one leg. This posture is perhaps similar to the “polite posture”, \textit{puppeap}, used in Thailand. } }

At\marginnote{27.3.1} that time the nuns were using a restroom for defecating. And the nuns from the group of six performed abortions there. They told the Buddha. 

\scrule{“A nun shouldn’t defecate in a restroom. If she does, she commits an offense of wrong conduct. I allow the nuns to defecate in a place that’s open underneath but concealed on top.” }

At\marginnote{27.4.1} that time the nuns were bathing with bath powder. People complained and criticized them, “They’re just like householders who indulge in worldly pleasures!” They told the Buddha. 

\scrule{“A nun shouldn’t bathe with bath powder. If she does, she commits an offense of wrong conduct. I allow bran and clay.” }

At\marginnote{27.4.8} that time the nuns were bathing with scented clay. People complained and criticized them, “They’re just like householders who indulge in worldly pleasures!” They told the Buddha. 

\scrule{“A nun shouldn’t bathe with scented clay. If she does, she commits an offense of wrong conduct. I allow ordinary clay.” }

On\marginnote{27.4.15} one occasion when the nuns were bathing in a sauna, they made a racket. They told the Buddha. 

\scrule{“A nun shouldn’t bathe in a sauna. If she does, she commits an offense of wrong conduct.” }

On\marginnote{27.4.19} one occasion the nuns were bathing against the stream, enjoying the touch of the current. They told the Buddha. 

\scrule{“A nun shouldn’t bathe against the stream. If she does, she commits an offense of wrong conduct.” }

On\marginnote{27.4.23} one occasion the nuns bathed away from a ford. Scoundrels raped them. They told the Buddha. 

\scrule{“A nun shouldn’t bathe away from a ford. If she does, she commits an offense of wrong conduct.” }

On\marginnote{27.4.28} one occasion the nuns bathed at a ford for men. People complained and criticized them, “They’re just like householders who indulge in worldly pleasures!” They told the Buddha. 

\scrule{“A nun shouldn’t bathe at a ford for men. If she does, she commits an offense of wrong conduct. A nun should bathe at a ford for women.” }

\scend{The third section for recitation is finished. }

\scendsutta{The tenth chapter on nuns is finished. }

\scend{In this chapter there are one hundred topics. }

\scuddanaintro{This is the summary: }

\begin{scuddana}%
“\textsanskrit{Gotamī}\marginnote{27.4.39} asked for the going forth, \\
The Buddha did not allow it; \\
Kapilavatthu to \textsanskrit{Vesāli}, \\
Did the Leader go. 

At\marginnote{27.4.43} the gatehouse covered in dust, \\
Declared to Ānanda; \\
He asked wisely about capability, \\
About mother and upbringing. 

One\marginnote{27.4.47} hundred years and that very day, \\
Without monks, seeking; \\
Invitation, heavy offense, \\
Two years, not abusing. 

And\marginnote{27.4.51} may not, eight principles, \\
Practicing all one’s life; \\
Receiving the important principles, \\
That was her ordination. 

A\marginnote{27.4.55} thousand years, just five, \\
Thieves, whiteheads; \\
Red rot, with similes, \\
Thus the true Teaching was injured. 

Not\marginnote{27.4.59} to mention he would make a dyke, \\
Again he stabilized the true Teaching; \\
To ordain, venerable, \\
Bow down according to seniority. 

They\marginnote{27.4.63} will not, how then, \\
Common, not in common; \\
Instruction, and Monastic Code, \\
Who, dwelling place. 

And\marginnote{27.4.67} they do not know, he told, \\
And did not do, with monks; \\
To receive by monks, \\
Receiving by nuns. 

He\marginnote{27.4.71} told, procedure by monks, \\
They complained, or with nuns; \\
To tell, and arguing, \\
Having determined, and with \textsanskrit{Uppalā}. 

At\marginnote{27.4.75} \textsanskrit{Sāvatthī}, muddy water, \\
Non-respect, body, and thigh; \\
And genitals, indecent speech, \\
The group associated inappropriately. 

Non-respect,\marginnote{27.4.79} penalty, \\
So again the nuns; \\
And restriction, instruction, \\
Is it allowable, he set out wandering. 

Ignorant,\marginnote{27.4.83} reason, investigation, \\
Instruction, Sangha, with five; \\
Two or three, they did not agree, \\
Ignorant ones, sick, departing. 

Forest-dweller,\marginnote{27.4.87} without telling, \\
And they did not return; \\
Long, and split bamboo, leather, \\
Fabric, and interlaced, rolled up; \\
Cloth, and interlaced, and rolled up, \\
And interlaced string, rolled up. 

Bone,\marginnote{27.4.93} cow’s jaw bone, \\
Hand, back of hand, so foot; \\
Thigh, face, gums, \\
Ointment, applied creams, powdered. 

They\marginnote{27.4.97} applied, body cosmetics, \\
Facial cosmetics, so both; \\
Eye cosmetics, facial mark, staring, \\
With exposed to view, and with dancing. 

Sex\marginnote{27.4.101} worker, bar, slaughterhouse, \\
Shop, loan, trade; \\
Male and female slaves, male servants, \\
Female servants, they would attend on. 

Animals,\marginnote{27.4.105} greens, \\
They wore felt; \\
Blue, yellow, red, \\
Magenta, and black robes. 

Orange\marginnote{27.4.109} and beige, \\
Single piece, and just long; \\
Floral, fruit, and close-fitting jacket, \\
And lodh tree, they wore. 

Nun,\marginnote{27.4.113} of a trainee nun, \\
Of a novice nun, after death; \\
The requisites are handed back, \\
Just the nuns are the owners. 

Of\marginnote{27.4.117} a monk, of a novice monk, \\
Of a lay follower, a female lay follower; \\
And the requisites of others, \\
It should be handed back, the monks are the owners. 

Female\marginnote{27.4.121} wrestler, fetus, the bottom of the bowl, \\
Genitals, and with requisite; \\
Abundance, greater, \\
Stored up requisites. 

As\marginnote{27.4.125} above for the monks,\footnote{Reading \textit{\textsanskrit{heṭṭhaṁ}} with SRT, as opposed to \textit{\textsanskrit{bhoṭṭhaṁ}}. } \\
So it should be done for the nuns; \\
Dwelling, menstruating, \\
It was stained, and menstruation pads. 

They\marginnote{27.4.129} snapped, and all the times, \\
Also those without genitals were seen; \\
Genitals, and just blood, \\
Just so continuous blood. 

Continuous\marginnote{27.4.133} pad, incontinent, \\
Prolapse, without sexual organs; \\
And manlike, fistula, \\
And also hermaphrodite. 

Having\marginnote{27.4.137} set out without genitals etc., \\
As far as hermaphrodite; \\
This is according to the repetition above, \\
Leprosy, abscesses, and mild leprosy. 

Tuberculosis\marginnote{27.4.141} and epilepsy, are you human, \\
Are you a woman, and are you free; \\
Debtless, not employed by the king, \\
And permitted, twenty. 

And\marginnote{27.4.145} full set, what name, \\
What is the name of your mentor; \\
Of the twenty-four obstacles, \\
Having asked, ordination. 

They\marginnote{27.4.149} were embarrassed, not instructed, \\
And just so in the midst of the Sangha; \\
Choosing a preceptor, outer robe, \\
Upper robe, sarong. 

Chest\marginnote{27.4.153} wrap, and bathing robe, \\
Having pointed out, should send away; \\
Ignorant ones, not appointed, \\
She should ask, interval on asking. 

Ordained\marginnote{27.4.157} on one side, \\
Again so with the Sangha of monks; \\
Time, season, and date, \\
Jointly, three supports. 

Eight\marginnote{27.4.161} things not to be done, \\
The right time, everywhere, just eight; \\
The nuns did not invite, \\
And just so the Sangha of monks. 

Racket,\marginnote{27.4.165} before the meal, \\
And too late, racket; \\
Observance-day ceremony, invitation ceremony, \\
Direction, instruction. 

Permission,\marginnote{27.4.169} accused, they reminded, \\
Was prohibited by the Great Sage; \\
In that way a monk to a nun, \\
Was allowed by the Great Sage. 

Vehicle,\marginnote{27.4.173} sick, and pulled, \\
Jolted around in a vehicle, \textsanskrit{Aḍḍhakāsī}; \\
Monk, trainee nun, novice monk, \\
And novice nun, with an ignorant one. 

In\marginnote{27.4.177} the wilderness, with a lay follower, \\
Storehouse, dwelling place; \\
Not sufficient, building work, \\
Pregnant, by herself. 

And\marginnote{27.4.181} the same building, heavy offense, \\
And verbally renounced, joined; \\
And bowing down, and hair, \\
And nails, treating sores. 

Cross-legged,\marginnote{27.4.185} and sick, \\
Feces, with bath powder, scented; \\
In a sauna, against the stream, \\
Away from a ford, and with men. 

\textsanskrit{Mahāgotamī}\marginnote{27.4.189} asked, \\
And so did Ānanda wisely; \\
There are four assemblies, \\
Going forth in the instruction of the Victor. 

For\marginnote{27.4.193} the purpose of seeing the urgency, \\
And for the purpose of growth in the true Teaching; \\
Like medicine for the sick, \\
So it was taught by the Buddha. 

Thus\marginnote{27.4.197} trained in the true Teaching, \\
Other women too; \\
They go to where there is no death, \\
Having gone there, they do not sorrow.” 

%
\end{scuddana}

\scendsutta{The chapter on nuns is finished. }

%
\chapter*{{\suttatitleacronym Kd 21}{\suttatitletranslation The chapter on the group of five hundred }{\suttatitleroot Pañcasatikakkhandhaka}}
\addcontentsline{toc}{chapter}{\tocacronym{Kd 21} \toctranslation{The chapter on the group of five hundred } \tocroot{Pañcasatikakkhandhaka}}
\markboth{The chapter on the group of five hundred }{Pañcasatikakkhandhaka}
\extramarks{Kd 21}{Kd 21}

\section*{1. The origin story of the communal recitation }

Then\marginnote{1.1.1} Venerable \textsanskrit{Mahākassapa} addressed the monks:\footnote{The unusual opening to this Khandhaka suggest that this section has at some point been part of another longer text, most likely \href{https://suttacentral.net/dn16/en/brahmali\#6.19.1}{DN 16}, the \textsanskrit{Mahāparinibbāna} Sutta, with which it dovetails. See Erich Frauwallner, “The  Earliest Vinaya and the Beginnings of Buddhist Literature”, pp. 41–46. } “On one occasion, as I was traveling from \textsanskrit{Pāvā} to \textsanskrit{Kusināra} with a large sangha of five hundred monks, I left the road and sat down at the foot of a tree. 

Just\marginnote{1.1.4} then a follower of the \textsanskrit{Ājīvakas} was traveling toward \textsanskrit{Pāvā} on the same road, holding a coral-tree flower that he had picked up in \textsanskrit{Kusināra}. When I saw him coming, I asked him, ‘Do you know anything about our Teacher?’ 

‘I\marginnote{1.1.8} do. Today it’s seven days since the ascetic Gotama attained final extinguishment. That’s why I carry this coral-tree flower.’ 

Some\marginnote{1.1.11} of the monks there who were not yet free from desire threw up their arms and cried, collapsed on the ground, and rolled back and forth, lamenting, ‘The Buddha, the Happy One, has attained final extinguishment too soon—too soon has the eye of the world been put out!’ But the monks there who were free from desire bore it with mindfulness and full awareness, saying, ‘All phenomena are impermanent. How could it be any different?’ 

I\marginnote{1.1.15} said, ‘Please stop grieving, stop lamenting. Didn’t the Buddha warn us that we must be separated from everyone and everything dear and agreeable to us? How could that which is born, become, made up, and of a nature to fall apart, not fall apart? That’s impossible.’ 

On\marginnote{1.1.22} that occasion a monk called Subhadda, who had gone forth when old, was part of that group. He said to the monks, ‘Please stop grieving, stop lamenting. It’s good that we are free from that great ascetic. We were oppressed, always being told what’s allowable and what’s not. Now we can do what we like and not do what we don’t like.’ 

So\marginnote{1.1.30} then, let’s recite the Teaching and the Monastic Law—before what’s contrary to the Teaching shines forth and the Teaching is obstructed; before what’s contrary to the Monastic Law shines forth and the Monastic Law is obstructed; before those who speak contrary to the Teaching become strong and those who speak in accordance with it become weak; before those who speak contrary to the Monastic Law become strong and those who speak in accordance with it become weak.” 

“Well\marginnote{1.2.1} then, venerable, please select the monks.” \textsanskrit{Mahākassapa} then selected four hundred and ninety-nine perfected ones. The monks said to him, “There’s Venerable Ānanda who, although still a trainee, is incapable of acting out of desire, ill will, confusion, or fear. He has learned many teachings and much Monastic Law from the Buddha. Please invite him as well.” And he did. 

The\marginnote{1.3.1} senior monks thought, “Where should we recite the Teaching and the Monastic Law?” It occurred to them, “\textsanskrit{Rājagaha} has much almsfood and many dwellings. Let’s spend the rainy season there in order to recite the Teaching and the Monastic Law. No other monks should enter the rainy-season residence at \textsanskrit{Rājagaha}.” 

And\marginnote{1.4.1} Venerable \textsanskrit{Mahākassapa} informed the Sangha: 

“Please,\marginnote{1.4.2} venerables, I ask the Sangha to listen. If the Sangha is ready, it should appoint these five hundred monks to spend the rainy season at \textsanskrit{Rājagaha} in order to recite the Teaching and the Monastic Law. No other monks should enter the rainy-season residence at \textsanskrit{Rājagaha}. This is the motion. 

Please,\marginnote{1.4.6} venerables, I ask the Sangha to listen. The Sangha appoints these five hundred monks to spend the rainy season at \textsanskrit{Rājagaha} in order to recite the Teaching and the Monastic Law. No other monks should spend the rainy-season residence at \textsanskrit{Rājagaha}. Any monk who approves of appointing these five hundred monks to spend the rainy season at \textsanskrit{Rājagaha} in order to recite the Teaching and the Monastic Law, with no other monks spending the rainy-season residence at \textsanskrit{Rājagaha}, should remain silent. Any monk who doesn’t approve should speak up. 

The\marginnote{1.4.13} Sangha has appointed these five hundred monks to spend the rainy season at \textsanskrit{Rājagaha} in order to recite the Teaching and the Monastic Law. No other monks should enter the rainy-season residence at \textsanskrit{Rājagaha}. The Sangha approves and is therefore silent. I’ll remember it thus.” 

\subsection*{The communal recitation at \textsanskrit{Rājagaha}}

The\marginnote{1.5.1} senior monks then went to \textsanskrit{Rājagaha} to recite the Teaching and the Monastic Law. They thought, “The Buddha has praised repairing what’s defective and broken. Well then, let’s spend the first month doing repairs, and then gather for the middle month to recite the Teaching and the Monastic Law.” 

They\marginnote{1.5.6} then spent the first month doing repairs. Venerable Ānanda thought, “It wouldn’t be proper for me to go to the assembly tomorrow if I’m still a trainee.” After spending most of the night with mindfulness directed to the body, early in the morning he bent over to lie down. In the interval between his feet coming off the ground and his head hitting the pillow, his mind was freed from the corruptions through letting go. And Venerable Ānanda went to the assembly as a perfected one. 

Venerable\marginnote{1.7.1} \textsanskrit{Mahākassapa} then informed the Sangha: 

“Please,\marginnote{1.7.2} venerables, I ask the Sangha to listen. If the Sangha is ready, I will ask \textsanskrit{Upāli} about the Monastic Law.” 

Venerable\marginnote{1.7.4} \textsanskrit{Upāli} informed the Sangha: 

“Please,\marginnote{1.7.5} venerables, I ask the Sangha to listen. If the Sangha is ready, I will reply when asked by Venerable \textsanskrit{Mahākassapa} about the Monastic Law.” 

\textsanskrit{Mahākassapa}\marginnote{1.7.7} then asked \textsanskrit{Upāli}, “Where was the first offense entailing expulsion laid down?” 

“At\marginnote{1.7.9} \textsanskrit{Vesālī}.” 

“Who\marginnote{1.7.10} is it about?” 

“Sudinna\marginnote{1.7.11} the Kalandian.” 

“What\marginnote{1.7.12} is it about?” 

“Sexual\marginnote{1.7.13} intercourse.” 

\textsanskrit{Mahākassapa}\marginnote{1.7.14} also asked \textsanskrit{Upāli} about the topic of the first offense entailing expulsion, about the origin story, about the person, about the rule, about the additions to the rule, about the offense, and about the non-offenses. 

“And\marginnote{1.7.15} where was the second offense entailing expulsion laid down?” 

“At\marginnote{1.7.16} \textsanskrit{Rājagaha}.” 

“Who\marginnote{1.7.17} is it about?” 

“Dhaniya\marginnote{1.7.18} the potter.” 

“What\marginnote{1.7.19} is it about?” 

“Stealing.”\marginnote{1.7.20} 

\textsanskrit{Mahākassapa}\marginnote{1.7.21} also asked \textsanskrit{Upāli} about the topic of the second offense entailing expulsion, about the origin story, about the person, about the rule, about the additions to the rule, about the offense, and about the non-offenses. 

“And\marginnote{1.7.22} where was the third offense entailing expulsion laid down?” 

“At\marginnote{1.7.23} \textsanskrit{Vesālī}.” 

“Who\marginnote{1.7.24} is it about?” 

“A\marginnote{1.7.25} number of monks.” 

“What\marginnote{1.7.26} is it about?” 

“Human\marginnote{1.7.27} beings.” 

\textsanskrit{Mahākassapa}\marginnote{1.7.28} also asked \textsanskrit{Upāli} about the topic of the third offense entailing expulsion, about the origin story, about the person, about the rule, about the additions to the rule, about the offense, and about the non-offenses. 

“And\marginnote{1.7.29} where was the fourth offense entailing expulsion laid down?” 

“At\marginnote{1.7.30} \textsanskrit{Vesālī}.” 

“Who\marginnote{1.7.31} is it about?” 

“The\marginnote{1.7.32} monks from the banks of the \textsanskrit{Vaggumudā}.” 

“What\marginnote{1.7.33} is it about?” 

“Superhuman\marginnote{1.7.34} qualities.” 

\textsanskrit{Mahākassapa}\marginnote{1.7.35} also asked \textsanskrit{Upāli} about the topic of the fourth offense entailing expulsion, about the origin story, about the person, about the rule, about the additions to the rule, about the offense, and about the non-offenses. 

In\marginnote{1.7.36} this way he asked about the analyses of both Monastic Codes. \textsanskrit{Upāli} was able to reply to each and every question. 

Venerable\marginnote{1.8.1} \textsanskrit{Mahākassapa} then informed the Sangha: 

“Please,\marginnote{1.8.2} venerables, I ask the Sangha to listen. If the Sangha is ready, I will ask Ānanda about the Teaching.” 

Venerable\marginnote{1.8.4} Ānanda informed the Sangha: 

“Please,\marginnote{1.8.5} venerables, I ask the Sangha to listen. If the Sangha is ready, I will reply when asked by Venerable \textsanskrit{Mahākassapa} about the Teaching.” 

\textsanskrit{Mahākassapa}\marginnote{1.8.7} then asked Ānanda, “Where was ‘The Supreme Net’ spoken?”\footnote{This is the first discourse of the \textsanskrit{Dīgha} \textsanskrit{Nikāya}, the \textsanskrit{Brahmajāla} Sutta, \href{https://suttacentral.net/dn1/en/brahmali\#1.1.1}{DN 1:1.1.1}. } 

“At\marginnote{1.8.9} the royal rest-house at \textsanskrit{Ambalaṭṭhikā}, between \textsanskrit{Rājagaha} and \textsanskrit{Nāḷanda}.” 

“Who\marginnote{1.8.10} is it about?” 

“The\marginnote{1.8.11} wanderer Suppiya and the young brahmin Brahmadatta.” 

\textsanskrit{Mahākassapa}\marginnote{1.8.12} also asked Ānanda about the origin story of ‘The Supreme Net’ and about the person. 

“Where\marginnote{1.8.13} was ‘The Fruits of the Monastic Life’ spoken?”\footnote{This is the second discourse of the \textsanskrit{Dīgha} \textsanskrit{Nikāya}, the \textsanskrit{Sāmaññaphala} Sutta, \href{https://suttacentral.net/dn2/en/brahmali\#1.1.1}{DN 2:1.1.1}. } 

“In\marginnote{1.8.14} \textsanskrit{Jīvaka}’s Mango Grove at \textsanskrit{Rājagaha}.” 

“Who\marginnote{1.8.15} is it with?” 

“\textsanskrit{Ajātasattu}\marginnote{1.8.16} Vedehiputta.” 

\textsanskrit{Mahākassapa}\marginnote{1.8.17} also asked Ānanda about the origin story of ‘The Fruits of the Monastic Life’ and about the person. 

In\marginnote{1.8.18} this way he asked about the five collections. Ānanda was able to reply to each and every question. 

\section*{2. Discussion of the minor training rules }

Ānanda\marginnote{1.9.1} said to the senior monks, “At the time of his final extinguishment, the Buddha said to me, ‘After my passing away, Ānanda, if the Sangha wishes, it may abolish the minor training rules.’” 

“But,\marginnote{1.9.4} Ānanda, did you ask the Buddha what the minor training rules are?” 

“No,\marginnote{1.9.6} sirs, I didn’t.” 

Some\marginnote{1.9.8} senior monks said, “Apart from the four rules entailing expulsion, the rest are the minor training rules.” Others said, “Apart from the four rules entailing expulsion and the thirteen rules entailing suspension, the rest are the minor training rules.” Still others said, “Apart from the four rules entailing expulsion, the thirteen rules entailing suspension, and the two undetermined rules, the rest are the minor training rules.” Still others said, “Apart from the four rules entailing expulsion, the thirteen rules entailing suspension, the two undetermined rules, and the thirty rules entailing relinquishment and confession, the rest are the minor training rules.” Still others said, “Apart from the four rules entailing expulsion, the thirteen rules entailing suspension, the two undetermined rules, the thirty rules entailing relinquishment and confession, and the ninety-two rules entailing confession, the rest are the minor training rules.” Still others said, “Apart from the four rules entailing expulsion, the thirteen rules entailing suspension, the two undetermined rules, the thirty rules entailing relinquishment and confession, the ninety-two rules entailing confession, and the four rules entailing acknowledgment, the rest are the minor training rules.” 

Then\marginnote{1.9.20} Venerable \textsanskrit{Mahākassapa} informed the Sangha: 

“Please,\marginnote{1.9.21} venerables, I ask the Sangha to listen. We have training rules that relate to householders. The householders know what is allowable for us and what is not. If we abolish the minor training rules, some people will say, ‘The ascetic Gotama laid down training rules for his disciples until the time of his death.\footnote{For some reason \textit{\textsanskrit{dhūmakālika}} is not commented on in the Vipassana Research Institute’s version of Sp at https://www.tipitaka.org. The PTS version, however, comments as follows at Sp 5.1296: \textit{\textsanskrit{Dhūmakālikanti} \textsanskrit{yāva} \textsanskrit{samaṇassa} (gotamassa) \textsanskrit{parinibbānacitakadhūmo} \textsanskrit{paññāyati} \textsanskrit{tāva} \textsanskrit{kāloti}}, “\textit{\textsanskrit{Dhūmakālika}} means the time until the smoke from the final-extinguishment pyre of the ascetic Gotama.” } But they practice the training rules only as long as their teacher is alive. Since their teacher has now attained final extinguishment, they no longer practice them.’ If the Sangha is ready, it shouldn’t lay down new rules, nor get rid of existing ones, and it should undertake to practice the training rules as they are. This is the motion. 

Please,\marginnote{1.9.31} venerables, I ask the Sangha to listen. We have training rules that relate to householders. The householders know what is allowable for us and what is not. If we abolish the minor training rules, some people will say, ‘The ascetic Gotama laid down training rules for his disciples until the time of his death. But they practice the training rules only as long as their teacher is alive. Since their teacher has now attained final extinguishment, they no longer practice them.’ The Sangha doesn’t lay down new rules, nor get rid of existing ones, and it undertakes to practice the training rules as they are. Any monk who approves of not laying down new rules, nor of getting rid of existing ones, and of undertaking to practice the training rules as they are should remain silent. Any monk who doesn’t approve should speak up. 

The\marginnote{1.9.42} Sangha doesn’t lay down new rules, nor get rid of the existing ones, and it undertakes to practice the training rules as they are. The Sangha approves and is therefore silent. I’ll remember it thus.” 

The\marginnote{1.10.1} senior monks then said, “You have committed an act of wrong conduct, Ānanda, in that you didn’t ask the Buddha what the minor training rules are. Please confess that wrong conduct.” 

“It\marginnote{1.10.4} was because of lack of mindfulness that I didn’t ask. I can’t see that I have committed any wrong conduct, but I’ll confess it out of faith in the venerables.” 

“You\marginnote{1.10.6} have also committed an act of wrong conduct in that you stepped on the Buddha’s rainy-season robe while you were sewing it. Please confess that wrong conduct.” 

“I\marginnote{1.10.8} didn’t step on it because of disrespect. I can’t see that I’ve committed any wrong conduct, but I’ll confess it out of faith in the venerables.” 

“You\marginnote{1.10.10} have also committed an act of wrong conduct in that you first had women pay respect to the Buddha’s dead body. They soiled the Buddha’s body with tears. Please confess that wrong conduct.” 

“I\marginnote{1.10.12} did this so that it wouldn’t get too late for them. I can’t see that I’ve committed any wrong conduct, but I’ll confess it out of faith in the venerables.” 

“You\marginnote{1.10.14} have also committed an act of wrong conduct in that you didn’t ask the Buddha, even when he gave you a broad hint, to live on for an eon—for the benefit and happiness of humanity, out of compassion for the world, for the good, benefit, and happiness of gods and humans. Please confess that wrong conduct.” 

“I\marginnote{1.10.17} didn’t ask because my mind was possessed by the Lord of Death. I can’t see that I’ve committed any wrong conduct, but I’ll confess it out of faith in the venerables.” 

“You\marginnote{1.10.20} have also committed an act of wrong conduct in that you made an effort for women to be given the going forth on the spiritual path proclaimed by the Buddha. Please confess that wrong conduct.” 

“I\marginnote{1.10.22} made this effort because \textsanskrit{Mahāpajāpati} \textsanskrit{Gotamī} was the Buddha’s aunt who nurtured him, brought him up, and breastfed him when his own mother died. I can’t see that I’ve committed any wrong conduct, but I’ll confess it out of faith in the venerables.” 

At\marginnote{1.11.1} that time Venerable \textsanskrit{Purāṇa} was wandering in the Southern Hills with a large sangha of five hundred monks. Soon the senior monks had concluded the communal recitation of the Teaching and the Monastic Law. Then, when \textsanskrit{Purāṇa} had stayed in the Southern Hills for as long as he liked, he went to the Bamboo Grove at \textsanskrit{Rājagaha}. There he went up to the senior monks, exchanged pleasantries with them, and sat down. And they said to him, “\textsanskrit{Purāṇa}, the senior monks have recited the Teaching and the Monastic Law. Please accept that communal recitation.” 

“The\marginnote{1.11.6} Teaching and the Monastic Law have been well-recited by the senior monks. Nevertheless, I’ll remember what I myself have received from the Buddha.” 

\section*{3. Discussion of the supreme penalty }

Venerable\marginnote{1.12.1} Ānanda said to the senior monks, “At the time of his final extinguishment, the Buddha said to me, ‘After my passing away, Ānanda, the Sangha should impose the supreme penalty on the monk Channa.’” 

“Did\marginnote{1.12.4} you ask the Buddha what the supreme penalty is?” 

“I\marginnote{1.12.6} did, and he replied, ‘Whatever Channa says, the monks shouldn’t correct him, instruct him, or teach him.’” 

“Well\marginnote{1.12.10} then, Ānanda, impose the supreme penalty on Channa.” 

“But\marginnote{1.12.11} how should I do it? Channa is temperamental and harsh.” 

“Go\marginnote{1.12.12} together with many monks.” 

Saying,\marginnote{1.12.13} “Yes, venerables,” he traveled by boat upstream to \textsanskrit{Kosambī} with a large sangha of five hundred monks. After disembarking, he sat down at the foot of a tree not far from King Udena’s park. 

Just\marginnote{1.13.1} then King Udena was enjoying himself in the park together with his harem. The harem women heard that their teacher, Venerable Ānanda, was seated at the foot of a tree not far from the park. They told the king, adding, “Sir, we would like to see Venerable Ānanda.” 

“Well\marginnote{1.13.7} then, go ahead.” 

The\marginnote{1.13.8} harem women then went to Ānanda, bowed, and sat down. And Ānanda instructed, inspired, and gladdened them with a teaching, at the end of which they gave him five hundred upper robes. After rejoicing and expressing their appreciation for his teaching, they got up from their seats, bowed down, circumambulated him with their right sides toward him, and went to King Udena. 

When\marginnote{1.14.1} King Udena saw them coming, he said to them, “Did you see the ascetic Ānanda?” 

“We\marginnote{1.14.4} did.” 

“Did\marginnote{1.14.5} you give him anything?” 

“We\marginnote{1.14.6} gave him five hundred upper robes.” 

King\marginnote{1.14.7} Udena complained and criticized him, “How can the ascetic Ānanda receive so many robes? Is he starting up as cloth merchant or setting up shop?” 

King\marginnote{1.14.10} Udena then went to Ānanda, exchanged pleasantries with him, sat down, and said, “Sir Ānanda, did our harem women come here?” 

“They\marginnote{1.14.14} did.” 

“Did\marginnote{1.14.15} they give you anything?” 

“They\marginnote{1.14.16} gave me five hundred upper robes.” 

“But\marginnote{1.14.17} what will you do with five hundred robes?” 

“I’ll\marginnote{1.14.18} share them with those monks whose robes are worn out.” 

“And\marginnote{1.14.19} what will you do with the worn out robes?” 

“We’ll\marginnote{1.14.20} make them into bedspreads.” 

“And\marginnote{1.14.21} what will you do with the old bedspreads?” 

“We’ll\marginnote{1.14.22} make them into mattress covers.” 

“And\marginnote{1.14.23} what will you do with the old mattress covers?” 

“We’ll\marginnote{1.14.24} make them into floor covers.” 

“And\marginnote{1.14.25} what will you do with the old floor covers?” 

“We’ll\marginnote{1.14.26} make them into doormats.” 

“And\marginnote{1.14.27} what will you do with the old doormats?” 

“We’ll\marginnote{1.14.28} make them into dustcloths.” 

“And\marginnote{1.14.29} what will you do with the old dustcloths?” 

“We’ll\marginnote{1.14.30} cut them up, mix them with mud, and smear the floors.” 

King\marginnote{1.14.31} Udena thought, “These Sakyan monastics are clever at putting things to use; nothing is wasted,”\footnote{“Nothing is wasted” renders \textit{na \textsanskrit{kulavaṁ} gamenti}. Sp-\textsanskrit{ṭ} 4.445: \textit{Na \textsanskrit{kulavaṁ} \textsanskrit{gamentīti} \textsanskrit{niratthakavināsanaṁ} na gamenti. Kucchito lavo kulavo, \textsanskrit{anayavināsoti} \textsanskrit{vuttaṁ} hoti}, “\textit{Na \textsanskrit{kulavaṁ} gamenti} means they do not make it go to uselessness and ruin.” } and he gave another five hundred pieces of cloth to Ānanda. Together with the first offering of robes, Ānanda was given a total of one thousand robes. 

Ānanda\marginnote{1.15.1} then went to Ghosita’s Monastery where he sat down on the prepared seat. Venerable Channa went up to Ānanda, bowed, and sat down. And Ānanda said, “Channa, the Sangha has imposed the supreme penalty on you.” 

“What’s\marginnote{1.15.5} the supreme penalty?” 

“Whatever\marginnote{1.15.6} you say to the monks, the monks shouldn’t correct you, instruct you, or teach you.” 

Exclaiming,\marginnote{1.15.8} “I’m ruined!” he fainted right there. 

Being\marginnote{1.15.9} troubled, ashamed, and disgusted by the supreme penalty, Channa stayed by himself, secluded, heedful, energetic, and diligent. And in this very life, he soon realized with his own insight the supreme goal of the spiritual life for which gentlemen rightly go forth into homelessness. He understood that birth had come to an end, that the spiritual life had been fulfilled, that the job had been done, that there was no further state of existence. Venerable Channa became one of the perfected ones. 

He\marginnote{1.15.14} then went to Ānanda and said, “Venerable Ānanda, please lift the supreme penalty.” 

“The\marginnote{1.15.16} supreme penalty was lifted the moment you realized perfection.” 

At\marginnote{1.16.1} this communal recitation of the Monastic Law there were five hundred monks, neither more nor less.\footnote{“Communal recitation of the Monastic Law” renders \textit{\textsanskrit{vinayasaṅgīti}}. The meaning of this compound seems straightforward, but Sp-\textsanskrit{ṭ} 4.445 interprets it as follows: \textit{“\textsanskrit{Dhammavinayasaṅgītiyā}”ti vattabbe \textsanskrit{saṅgītiyā} \textsanskrit{vinayappadhānattā} “\textsanskrit{vinayasaṅgītiyā}”ti \textsanskrit{vuttaṁ}}, “When ‘communal recitation of the Teaching and the Monastic Law’ should be said, ‘\textit{\textsanskrit{vinayasaṅgīti}}’ is said instead, because of the striving according to the Monastic Law at the communal recitation.” This explanation seems contrived. Indeed, the second \textit{\textsanskrit{saṅgīti}} is also called a \textit{\textsanskrit{vinayasaṅgīti}}, except that in this case the meaning “Communal recitation of the Monastic Law” fits well. It seems likely to me that this compound should mean the same in the two contexts that are so similar. Might it then be that the first communal recitation was limited to reciting the Monastic Law? } This is why this communal recitation is called “The group of five hundred”. 

\scendsutta{The eleventh chapter on the group of five hundred is finished. In this chapter there are twenty-three topics. }

\scuddanaintro{This is the summary: }

\begin{scuddana}%
“When\marginnote{1.16.5} the Buddha had attained extinguishment, \\
The senior monk called Kassapa; \\
Addressed the community of monks, \\
Guarding the true Teaching. 

On\marginnote{1.16.9} the way from \textsanskrit{Pāvā}, \\
Subhadda declared; \\
We will recite the true Teaching, \\
Before what is contrary to the Teaching shines forth. 

Four\marginnote{1.16.13} hundred and ninety-nine, \\
And he also invited Ānanda; \\
Communal recitation of the Teaching and the Monastic Law, \\
Staying in the best of caves. 

He\marginnote{1.16.17} asked \textsanskrit{Upāli} about the Monastic Law, \\
And the wise Ānanda about the discourses; \\
Communal recitation of the three Collections, \\
Was done by the disciples of the Victor. 

The\marginnote{1.16.21} various minor rules, \\
Were continued as laid down; \\
He did not ask, having stepped on, \\
Had pay respect, and did not ask. 

The\marginnote{1.16.25} going forth of women, \\
Wrong conduct for me out of faith; \\
\textsanskrit{Purāṇa}, and the supreme penalty, \\
Harem with Udena. 

So\marginnote{1.16.29} many, and worn out, \\
Bedspreads, mattress; \\
Floor covers, doormats, \\
Dustcloth, mixed with mud. 

He\marginnote{1.16.33} got one thousand robes, \\
With the first, the one called Ānanda; \\
Condemned by the supreme punishment, \\
Acquired the four truths; \\
Mastered by the five hundred, \\
Therefore it was ‘the group of five hundred’.” 

%
\end{scuddana}

\scendsutta{The chapter on the group of five hundred is finished. }

%
\chapter*{{\suttatitleacronym Kd 22}{\suttatitletranslation The chapter on the group of seven hundred }{\suttatitleroot Sattasatikakkhandhaka}}
\addcontentsline{toc}{chapter}{\tocacronym{Kd 22} \toctranslation{The chapter on the group of seven hundred } \tocroot{Sattasatikakkhandhaka}}
\markboth{The chapter on the group of seven hundred }{Sattasatikakkhandhaka}
\extramarks{Kd 22}{Kd 22}

One\marginnote{1.1.1} hundred years after the Buddha had attained final extinguishment, the Vajjian monks of \textsanskrit{Vesālī} proclaimed ten practices as allowable: the salt-in-horn practice; the two-fingerbreadths practice; the next-village practice; the many-monasteries practice; the consent practice; customary practices; the unchurned practice; palm-juice drinking; sitting mats without borders; and gold, silver, and money.\footnote{The meaning of these practices is not immediately obvious. They are explained further down in the text. For an explanation of the rendering “gold, silver, and money” for \textit{\textsanskrit{jātarūparajata}}, see Appendix of Technical Terms. } 

At\marginnote{1.1.3} that time Venerable Yasa of \textsanskrit{Kākaṇḍa} was wandering in the Vajjian country, when he arrived at \textsanskrit{Vesālī}. There he stayed in the hall with the peaked roof in the Great Wood. 

Soon\marginnote{1.1.5} afterwards, on the observance day, the Vajjian monks of \textsanskrit{Vesālī} filled a bronze bowl with water and placed it in the midst of the Sangha of monks. Whenever a lay follower of \textsanskrit{Vesālī} came, they said, “Please give a \textit{\textsanskrit{kahāpaṇa}} coin to the Sangha, or half a \textit{\textsanskrit{kahāpaṇa}}, or a \textit{\textsanskrit{pāda}}, or a \textit{\textsanskrit{māsaka}}. The Sangha needs requisites.” 

But\marginnote{1.1.8} Yasa said to the lay followers, “Don’t give a \textit{\textsanskrit{kahāpaṇa}} to the Sangha, or half a \textit{\textsanskrit{kahāpaṇa}}, or a \textit{\textsanskrit{pāda}}, or a \textit{\textsanskrit{māsaka}}. Gold, silver, and money aren’t allowable for the Sakyan monastics. They neither accept nor receive gold, silver, or money. The Sakyan monastics have given up gems and gold, and live without gold, silver, and money.” But although Yasa said this, the lay followers continued to give money to the Sangha. 

The\marginnote{1.1.15} next morning the Vajjian monks distributed the money evenly.\footnote{For an explanation of the rendering “money” for \textit{\textsanskrit{hirañña}}, see Appendix of Technical Terms. } And they said to Yasa, “Here’s your share, Yasa.” 

“There’s\marginnote{1.1.18} no share for me. I don’t accept money.” 

The\marginnote{1.2.1} Vajjian monks said to one another, “Yasa is abusing and reviling the lay followers who have faith and confidence. He’s destroying their confidence. Let’s do a legal procedure of reconciliation against him.” And they did just that. 

Yasa\marginnote{1.2.4} said to them, “The Buddha has laid down that a monk who has had a legal procedure of reconciliation done against him should be given a monk as a companion messenger. Please give me a companion.” They then appointed a monk and gave him to Yasa as a companion messenger. 

Yasa\marginnote{1.2.9} entered \textsanskrit{Vesālī} with his companion and said to the lay followers, “It seems that I’m abusing and reviling the venerable lay followers who have faith and confidence, and that I’m destroying their confidence, in that I speak of what’s contrary to the Teaching as such and of what’s in accordance with the Teaching as such, and that I speak of what’s contrary to the Monastic Law as such and of what’s in accordance with the Monastic Law as such. 

\section*{Why monastics should not accept gold, silver, or money}

On\marginnote{1.3.1} one occasion the Buddha was staying at \textsanskrit{Sāvatthī} in the Jeta Grove, \textsanskrit{Anāthapiṇḍika}’s Monastery.\footnote{Parallel to \href{https://suttacentral.net/an4.50/en/brahmali\#0.3}{AN 4.50:0.3}. } There he addressed the monks: 

‘There\marginnote{1.3.3} are these four defilements of the sun and the moon that stop them from shining and radiating: clouds; snow; smoke and dust; and an eclipse by \textsanskrit{Rāhu}, the ruler of the antigods.\footnote{The Pali literally says that “\textsanskrit{Rāhu}, the rules of the antigods, is a defilement of the sun and the moon”, which is unintelligible without an understanding of ancient Indian mythology. See the Malalasekera’s Dictionary of Pali Proper Names for further details. } 

In\marginnote{1.3.9} the same way, there are these four defilements of monastics and brahmins that stop them from shining and radiating: drinking alcohol; having sexual intercourse; accepting gold, silver, or money; and making a living through wrong livelihood.’ 

Having\marginnote{1.3.22} said this, the Teacher added: 

\begin{verse}%
‘Defiled\marginnote{1.3.23} by desire and ill will, \\
Some monastics and brahmins, \\
Those hindered by delusion, \\
Delight in what seems lovely. 

Some\marginnote{1.3.27} monastics and brahmins, \\
Deluded, they drink alcohol, \\
Have sexual intercourse, \\
Accept gold, silver, or money, 

And\marginnote{1.3.31} make a living \\
Through wrong livelihood. \\
These are called defilements by the Buddha, \\
The Kinsman of the Sun. 

Those\marginnote{1.3.35} monastics and brahmins \\
Who are defiled by these \\
Do not shine and radiate; \\
They are impure, dirty, and low. 

Enveloped\marginnote{1.3.39} in darkness, \\
Slaves to craving that leads them on, \\
Filling the dreaded cemeteries, \\
They receive another life.’ 

%
\end{verse}

It’s\marginnote{1.3.43} by speaking like this, it seems, that I’m abusing and reviling the venerable lay followers who have faith and confidence, and that I’m destroying their confidence. 

At\marginnote{1.4.1} another time when the Buddha was staying at \textsanskrit{Rājagaha} in the Bamboo Grove,\footnote{Parallel to \href{https://suttacentral.net/sn42.10/en/brahmali\#0.3}{SN 42.10:0.3}. } the royal court was seated together in the royal compound, having the following conversation, ‘Gold, silver, and money are allowable for the Sakyan monastics; they accept and receive gold, silver, and money.’ 

On\marginnote{1.4.6} that occasion the chief \textsanskrit{Maṇicūlaka} was sitting in that gathering. He said, ‘No, gold, silver, and money aren’t allowable for the Sakyan monastics. They neither accept nor receive gold, silver, or money. The Sakyan monastics have given up gems and gold, and live without gold, silver, and money.’ And he was able to persuade that gathering. 

Soon\marginnote{1.4.14} afterwards \textsanskrit{Maṇicūlaka} went to the Buddha, bowed, sat down, and told him what had happened, adding, ‘Sir, have I explained in accordance with the Teaching so that I can’t be legitimately criticized or censured?’ 

‘You\marginnote{1.4.28} certainly have, for gold, silver, and money aren’t allowable for the Sakyan monastics. They neither accept nor receive gold, silver, or money. The Sakyan monastics have given up gems and gold, and live without gold, silver, and money. Whoever is allowed gold, silver, or money is also allowed the pleasures of the world. And you should know that anyone who’s allowed the pleasures of the world doesn’t have the qualities of an ascetic, the qualities of a Sakyan monastic. Still, I say that anyone who needs grass may look for it, likewise timber, a cart, or a worker. But under no circumstances should they accept or look for gold, silver, or money.’ 

It’s\marginnote{1.4.42} by speaking like this, it seems, that I’m abusing and reviling the venerable lay followers who have faith and confidence, and that I’m destroying their confidence. 

On\marginnote{1.5.1} another occasion at \textsanskrit{Rājagaha} the Buddha prohibited gold, silver, and money and laid down a training rule because of Venerable Upananda the Sakyan. It’s by speaking like this, it seems, that I’m abusing and reviling the venerable lay followers who have faith and confidence, and that I’m destroying their confidence, in that I speak of what’s contrary to the Teaching as such and of what’s in accordance with the Teaching as such, and that I speak of what’s contrary to the Monastic Law as such and of what’s in accordance with the Monastic Law as such. 

And\marginnote{1.6.1} the lay followers of \textsanskrit{Vesālī} said to Yasa, “Venerable, you’re the only Sakyan monastic; none of these others is. Please stay at \textsanskrit{Vesālī}. We’ll do our best to provide you with robe-cloth, almsfood, a dwelling, and medicinal supplies.” Having persuaded the lay followers of \textsanskrit{Vesālī}, Yasa returned to the monastery together with his companion messenger. 

Soon\marginnote{1.7.1} afterwards the Vajjian monks asked the monk who had been the companion messenger, “Did Yasa ask forgiveness of the lay followers?” 

“The\marginnote{1.7.3} lay followers have acted badly toward us. They now regard Yasa as the only Sakyan monastic, but none of us.”\footnote{See CPD, sub-point 3g, for this use of \textit{kata}. } 

The\marginnote{1.7.6} Vajjian monks said, “Yasa has informed the householders without our approval. Let’s do a legal procedure of ejection against him.” But when they gathered together to do the procedure against him, Yasa rose up into the air and landed at \textsanskrit{Kosambī}. 

\section*{Yasa gathers supporters}

Yasa\marginnote{1.7.11.1} then sent a message to the monks at \textsanskrit{Pāvā} and to the monks in \textsanskrit{Avantī} in the southern region: “Please come, venerables. Let’s take on this legal issue—before what’s contrary to the Teaching shines forth and the Teaching is obstructed; before what’s contrary to the Monastic Law shines forth and the Monastic Law is obstructed; before those who speak contrary to the Teaching become strong and those who speak in accordance with it become weak; before those who speak contrary to the Monastic Law become strong and those who speak in accordance with it become weak.” 

Yasa\marginnote{1.8.1} then traveled to Venerable \textsanskrit{Sambhūta} \textsanskrit{Sāṇavāsī} who was staying on the \textsanskrit{Ahogaṅga} mountain. He bowed, sat down, and said, “Sir, the Vajjian monks of \textsanskrit{Vesālī} proclaim these ten practices as allowable: the salt-in-horn practice; the two-fingerbreadths practice; the next-village practice; the many-monasteries practice; the consent practice; customary practices; the unchurned practice; palm-juice drinking; sitting mats without borders; and gold, silver, and money. Let’s take on this legal issue—before what’s contrary to the Teaching shines forth and the Teaching is obstructed; before what’s contrary to the Monastic Law shines forth and the Monastic Law is obstructed; before those who speak contrary to the Teaching become strong and those who speak in accordance with it become weak; before those who speak contrary to the Monastic Law become strong and those who speak in accordance with it become weak.” 

“Yes.”\marginnote{1.8.11} 

Soon\marginnote{1.8.12} afterwards, sixty monks from \textsanskrit{Pāvā}—all of them wilderness dwellers, almsfood-only eaters, rag-robe wearers, three-robe owners, and perfected—gathered on mount \textsanskrit{Ahogaṅga}. And eighty-eight monks from \textsanskrit{Avantī} in the southern region—some of them wilderness dwellers, some almsfood-only eaters, some rag-robe wearers, some three-robe owners, but all perfected—also gathered on mount \textsanskrit{Ahogaṅga}. Then, as the senior monks were consulting one another, it occurred to them, “This legal issue is going to be contentious and difficult. How can we get supporters to strengthen our side?” 

At\marginnote{1.9.4} this time Venerable Revata was staying at Soreyya. He was learned and a master of the tradition; he was an expert on the Teaching, the Monastic Law, and the Key Terms; he was knowledgeable and competent, had a sense of conscience, and was afraid of wrongdoing and fond of the training. The senior monks considered this and said, “If we get Revata to support us, we’ll be stronger.” 

When\marginnote{1.9.10} Revata heard this conversation between the senior monks by means of clairaudience, he thought, “This legal issue is going to be contentious and difficult. It wouldn’t be appropriate for me to stay away from it. But now these monks are coming, and I won’t be at ease when they crowd me in. Let me leave before they arrive.” And he went from Soreyya to \textsanskrit{Saṅkassa}. 

When\marginnote{1.9.20} the senior monks came to Soreyya and asked where Revata was. They were told that he had gone to \textsanskrit{Saṅkassa}. Revata then went from \textsanskrit{Saṅkassa} to \textsanskrit{Kaṇṇakujja}. When the senior monks came to \textsanskrit{Saṅkassa} and asked where Revata was, they were told he had gone to \textsanskrit{Kaṇṇakujja}. Revata then went from \textsanskrit{Kaṇṇakujja} to Udumbara. When the senior monks came to \textsanskrit{Kaṇṇakujja} and asked where Revata was, they were told he had gone to Udumbara. Revata then went from Udumbara to \textsanskrit{Aggaḷapura}. When the senior monks came to Udumbara and asked where Revata was, they were told he had gone to \textsanskrit{Aggaḷapura}. Revata then went from \textsanskrit{Aggaḷapura} to \textsanskrit{Sahajāti}. When the senior monks came to \textsanskrit{Aggaḷapura} and asked where Revata was, they were told he had gone to \textsanskrit{Sahajāti}. Finally the senior monks caught up with Revata at \textsanskrit{Sahajāti}. 

\section*{The ten practices explained}

\textsanskrit{Sambhūta}\marginnote{1.10.1} \textsanskrit{Sāṇavāsī} then said to Yasa, “Revata is learned and a master of the tradition; he’s an expert on the Teaching, the Monastic Law, and the Key Terms; he’s knowledgeable and competent, has a sense of conscience, and is afraid of wrongdoing and fond of the training. If we ask Revata a question, he would be capable of spending the whole night answering just that one. Now, soon he will ask a pupil monk to chant. Once the chanting is finished, go up to Revata and ask him about these ten practices.” 

“Yes,\marginnote{1.10.6} sir.” 

Soon\marginnote{1.10.8} afterwards, when the chanting was finished, Yasa went up to Revata, bowed, sat down, and said, “Sir, is the salt-in-horn practice allowable?” 

“What’s\marginnote{1.10.11} the salt-in-horn practice?” 

“Is\marginnote{1.10.12} it allowable to carry salt in a horn and then eat it whenever the food is unsalted?”\footnote{Sp-\textsanskrit{ṭ} 1.0: \textit{\textsanskrit{Siṅgena} \textsanskrit{loṇaṁ} \textsanskrit{pariharitvā} \textsanskrit{pariharitvā} \textsanskrit{aloṇakapiṇḍapātena} \textsanskrit{saddhiṁ} \textsanskrit{bhuñjituṁ} kappati, na \textsanskrit{sannidhiṁ} \textsanskrit{karotīti} \textsanskrit{adhippāyo}}, “Carrying salt in a horn here and there, is it allowable to eat it together with unsalted almsfood? The meaning is one should not store it.” } 

“No,\marginnote{1.10.14} it’s not allowable.” 

“Is\marginnote{1.10.15} the two-fingerbreadths practice allowable?”—“What’s the two fingerbreadths practice?”—“Is it allowable to eat at the wrong time, so long as the shadow of the sundial is within two fingerbreadths of midday?”—“No.” 

“Is\marginnote{1.10.19} the next-village practice allowable?”—“What’s the next-village practice?”—“When you have finished your meal and refused an invitation to eat more, is it allowable to eat non-leftover food if you intend to go to the next village?”—“No.” 

“Is\marginnote{1.10.23} the many-monasteries practice allowable?”—“What’s the many-monasteries practice?”—“When there are a number of monasteries within the same monastery zone, is it allowable for them to do the observance-day ceremony separately?”—“No.” 

“Is\marginnote{1.10.27} the consent practice allowable?”—“What’s the consent practice?”—“Is it allowable to do a legal procedure with an incomplete Sangha, with the intention of getting consent from the absent monks afterwards?”—“No.” 

“Are\marginnote{1.10.32} customary practices allowable?”—“What are customary practices?”—“Is it allowable to follow the practices of one’s preceptors or teachers?”—“Sometimes it is, sometimes it isn’t.” 

“Is\marginnote{1.10.36} the unchurned practice allowable?”—“What’s the unchurned practice?”—“When you have finished your meal and refused an invitation to eat more, is it allowable to drink that which is halfway between milk and curd, if it’s not left over?”—“No.” 

“Is\marginnote{1.10.40} palm-juice drinking allowable?”—“What’s palm juice?”—“Is it allowable to drink that which has started to ferment, but which hasn’t yet become a proper alcoholic drink?”—“No.” 

“Are\marginnote{1.10.44} sitting mats without borders allowable?”—“No.” 

“Is\marginnote{1.10.46} gold, silver, or money allowable?”—“No.” 

“The\marginnote{1.10.48} Vajjian monks of \textsanskrit{Vesālī} proclaim these ten practices. Venerable, let’s take on this legal issue—before what’s contrary to the Teaching shines forth and the Teaching is obstructed; before what’s contrary to the Monastic Law shines forth and the Monastic Law is obstructed; before those who speak contrary to the Teaching become strong and those who speak in accordance with it become weak; before those who speak contrary to the Monastic Law become strong and those who speak in accordance with it become weak.” 

Saying,\marginnote{1.10.54} “Yes,” he consented to Yasa’s request. 

\scend{The first section for recitation is finished. }

\section*{Both sides gathering supporters}

The\marginnote{2.1.1} Vajjian monks of \textsanskrit{Vesālī} heard: “It seems Yasa wants to take on this legal issue and is looking for supporters. And it seems he is gaining support.” They said, “This legal issue is going to be contentious and difficult. How can we get supporters to strengthen our side?” 

It\marginnote{2.1.6} occurred to them, “Venerable Revata is learned and a master of the tradition; he’s an expert on the Teaching, the Monastic Law, and the Key Terms; he’s knowledgeable and competent, has a sense of conscience, and is afraid of wrongdoing and fond of the training. If we get Revata to support us, we’ll be stronger.” 

They\marginnote{2.1.9} prepared many monastic requisites: a bowl, a robe, a sitting mat, a needle case, a belt, a water filter, and a water strainer. Taking those requisites, they traveled by boat upstream to \textsanskrit{Sahajāti}. After disembarking, they had a meal at the foot of a tree. 

At\marginnote{2.2.1} this time, Venerable \textsanskrit{Sāḷha} was reflecting in private: “Who speak in accordance with the Teaching—the monks from the east or the monks from \textsanskrit{Pāvā}?” Reflecting on the Teaching and the Monastic Law, it occurred to him, “The monks from the east speak contrary to the Teaching, but the monks from \textsanskrit{Pāvā} don’t.” 

Just\marginnote{2.2.6} then a god from the pure abodes read \textsanskrit{Sāḷha}’s mind. Then, just as a strong man might bend or stretch his arm, he disappeared from pure abodes and appeared in front of \textsanskrit{Sāḷha}. And he said to \textsanskrit{Sāḷha}, “You’re right, Venerable \textsanskrit{Sāḷha}. The monks from the east speak contrary to the Teaching, but the monks from \textsanskrit{Pāvā} don’t. So then, take a stand in accordance with the Teaching.” 

“I’ve\marginnote{2.2.11} always taken a stand in accordance with the Teaching. But I won’t reveal my view in case I’m appointed to deal with this legal issue.” 

The\marginnote{2.3.1} Vajjian monks then went to Revata and said, “Venerable, please accept these monastic requisites.” 

Not\marginnote{2.3.4} wanting to accept them, he replied, “There’s no need. My bowl and robes are complete.” 

At\marginnote{2.3.5} this time a monk called Uttara, who had twenty years of seniority, was Revata’s attendant. The Vajjian monks then went to him and said, “Please accept these monastic requisites.” 

Not\marginnote{2.3.9} wanting to accept them, he replied, “There’s no need. My bowl and robes are complete.” 

“But\marginnote{2.3.10} people brought monastic requisites to the Buddha. If the Buddha received them, they were pleased. If he didn’t, they brought them to Venerable Ānanda instead, saying, ‘Venerable, please accept these monastic requisites. It’ll be as if they were received by the Buddha himself.’ So please accept these monastic requisites. It’ll be as if they were received by the elder himself.” 

Because\marginnote{2.3.17} he was pressured, Uttara received a robe. And he said, “Please say what you want.” 

“Please\marginnote{2.3.19} say this to the elder, ‘Sir, please say this in the midst of the Sangha: “Buddhas appear in the eastern countries. The monks from the east speak in accordance with the Teaching, not so the monks from \textsanskrit{Pāvā}.”’” 

“Alright.”\marginnote{2.3.23} 

He\marginnote{2.3.24} then went to Revata and told him what he had been asked to say. 

Revata\marginnote{2.3.28} replied, “You’re urging me to act contrary to the Teaching,” and he dismissed Uttara. 

Soon\marginnote{2.3.29} afterwards the Vajjian monks asked Uttara, “What did he say?” 

“We’ve\marginnote{2.3.31} acted badly. Saying, ‘You’re urging me to act contrary to the Teaching’, the elder dismissed me.” 

“But\marginnote{2.3.33} aren’t you a senior monk of twenty years’ standing?” 

“Yes.\marginnote{2.3.34} Nevertheless, I live with formal support from him because I respect him.”\footnote{Sp-\textsanskrit{ṭ} 4.454: \textit{\textsanskrit{Garunissayaṁ} \textsanskrit{gaṇhāmāti} \textsanskrit{kiñcāpi} \textsanskrit{mayaṁ} \textsanskrit{mahallakā}, \textsanskrit{etaṁ} pana \textsanskrit{theraṁ} \textsanskrit{garuṁ} \textsanskrit{katvā} \textsanskrit{vasissāmāti} \textsanskrit{adhippāyo}}, “\textit{\textsanskrit{Garunissayaṁ} \textsanskrit{gaṇhāma}}: although I am a senior monk, I will dwell respecting this senior monk.” Vmv 4.454 adds: \textit{\textsanskrit{Garunissayaṁ} \textsanskrit{gaṇhāmāti} \textsanskrit{nissayamuttāpi} \textsanskrit{mayaṁ} \textsanskrit{ekaṁ} \textsanskrit{sambhāvanīyagaruṁ} \textsanskrit{nissayabhūtaṁ} \textsanskrit{gahetvāva} \textsanskrit{vasissāmāti} \textsanskrit{adhippāyo}}, “\textit{\textsanskrit{Garunissayaṁ} \textsanskrit{gaṇhāma}}: even though I no longer need support, I will live supported by one who is to be respected.” For an explanation of the rendering “formal support” for \textit{nissaya}, see Appendix of Technical Terms. } 

The\marginnote{2.4.1} Sangha then gathered to make a decision on that legal issue. Revata informed the Sangha: 

“Please,\marginnote{2.4.3} I ask the Sangha to listen. If we were to resolve this legal issue here, it might be that the monks who started the ten practices would reopen it.\footnote{Vmv 4.455 explains \textit{\textsanskrit{mūlādāyakā}} as follows: \textit{\textsanskrit{Mūlādāyakāti} \textsanskrit{paṭhamaṁ} \textsanskrit{dasavatthūnaṁ} \textsanskrit{dāyakā}}, “\textit{\textsanskrit{Mūlādāyakā}} means those who first gave the ten practices.” } If the Sangha is ready, the Sangha should resolve this legal issue in the place where it arose.” 

The\marginnote{2.4.6} senior monks then went to \textsanskrit{Vesālī} to make a decision on that legal issue. 

At\marginnote{2.4.8} that time there was a monk called \textsanskrit{Sabbakāmī} who had been ordained for one hundred and twenty years and was the most senior monk in the world. He had been a student of Venerable Ānanda and was now staying at \textsanskrit{Vesālī}. 

Revata\marginnote{2.4.9} said to \textsanskrit{Sambhūta} \textsanskrit{Sāṇavāsī}, “I’ll be staying in \textsanskrit{Sabbakāmī}’s dwelling. Please go to \textsanskrit{Sabbakāmī} at the appropriate time and ask about these ten practices.” 

“Yes,\marginnote{2.4.12} sir.” 

Revata\marginnote{2.4.13} then entered \textsanskrit{Sabbakāmī}’s dwelling. \textsanskrit{Sabbakāmī} had his resting place prepared in the room, whereas Revata had his prepared in the entryway. When Revata saw that the old monk did not lie down, he too did not lie down. And when \textsanskrit{Sabbakāmī} saw that the tired monk who had just arrived did not lie down, he too did not lie down. 

Getting\marginnote{2.5.1} up early in the morning, \textsanskrit{Sabbakāmī} said to Revata, “What’s your main meditation?” 

“It’s\marginnote{2.5.3} good will, sir.” 

“Your\marginnote{2.5.4} meditation is noble, for good will is a noble meditation.”\footnote{Sp 4.455: \textit{\textsanskrit{Kullakavihārenāti} \textsanskrit{uttānavihārena}}, “\textit{\textsanskrit{Kullakavihāra}} means a straightforward meditation.” This, however, seems like a put-down, which does not fit the context well. It seems more likely to me that \textit{kullaka} is instead related to Sanskrit \textit{kulya}, “of a good/noble family”. } 

“In\marginnote{2.5.6} the past, too, when I was a householder, I habitually practiced good will, and now it’s my main meditation. Besides, I attained perfection long ago. But what’s your main meditation, sir?” 

“It’s\marginnote{2.5.9} emptiness.” 

“Your\marginnote{2.5.10} meditation is that of a great man, for emptiness is the meditation of a great man.” 

“In\marginnote{2.5.12} the past, too, when I was a householder, I habitually practiced emptiness, and now it’s my main meditation. Besides, I attained perfection long ago.” 

At\marginnote{2.6.1} that moment the conversation between the senior monks was interrupted because \textsanskrit{Sambhūta} \textsanskrit{Sāṇavāsī} arrived. \textsanskrit{Sambhūta} \textsanskrit{Sāṇavāsī} went up to \textsanskrit{Sabbakāmī}, bowed, sat down, and said, “Sir, the Vajjian monks of \textsanskrit{Vesālī} proclaim ten practices as allowable: the salt-in-horn practice; the two-fingerbreadths practice; the next-village practice; the many-monasteries practice; the consent practice; customary practices; the unchurned practice; palm-juice drinking; sitting mats without borders; and gold, silver, and money. Now, you’ve learned much at the feet of your preceptor, both of the Teaching and the Monastic Law. When you reflect on the Teaching and the Monastic Law, who speak in accordance with the Teaching—the monks from the east or the monks from \textsanskrit{Pāvā}?” 

“You\marginnote{2.6.10} too have learned much at the feet of your preceptor, both of the Teaching and the Monastic Law. When you reflect on the Teaching and the Monastic Law, who speak in accordance with the Teaching—the monks from the east or the monks from \textsanskrit{Pāvā}?” 

“When\marginnote{2.6.14} I reflect like this, it occurs to me that the monks from the east speak contrary to the Teaching, but the monks from \textsanskrit{Pāvā} don’t. But I won’t reveal my view in case I’m appointed to deal with this legal issue.” 

“And\marginnote{2.6.17} when I reflect like this, it occurs to me too that the monks from the east speak contrary to the Teaching, but the monks from \textsanskrit{Pāvā} don’t. And I too won’t reveal my view in case I’m appointed to deal with this legal issue.” 

\section*{Choosing a committee}

The\marginnote{2.7.1} Sangha then gathered to make a decision on that legal issue. While they were discussing that legal issue, there was endless talk but not a single statement that could be understood. Revata then informed the Sangha: 

“Please,\marginnote{2.7.4} venerables, I ask the Sangha to listen. While we were discussing this legal issue, there was endless talk but not a single statement that could be understood. If the Sangha is ready, it should resolve this legal issue by means of a committee.” 

The\marginnote{2.7.7} Sangha then selected four monks from the east—Venerable \textsanskrit{Sabbakāmī}, Venerable \textsanskrit{Sāḷha}, Venerable Khujjasobhita, and Venerable \textsanskrit{Vāsabhagāmika}—and four monks from \textsanskrit{Pāvā}—Venerable Revata, Venerable \textsanskrit{Sambhūta} \textsanskrit{Sāṇavāsī}, Venerable Yasa of \textsanskrit{Kākaṇḍa}, and Venerable Sumana. Revata then informed the Sangha: 

“Please,\marginnote{2.7.13} venerables, I ask the Sangha to listen. While we were discussing this legal issue, there was endless talk but not a single statement that could be understood. If the Sangha is ready, it should appoint four monks from the east and four from \textsanskrit{Pāvā} to resolve this legal issue by means of a committee. This is the motion. 

Please,\marginnote{2.7.17} venerables, I ask the Sangha to listen. While we were discussing this legal issue, there was endless talk but not a single statement that could be understood. The Sangha appoints four monks from the east and four from \textsanskrit{Pāvā} to resolve this legal issue by means of a committee. Any monk who approves of appointing four monks from the east and four from \textsanskrit{Pāvā} to resolve this legal issue by means of a committee should remain silent. Any monk who doesn’t approve should speak up. 

The\marginnote{2.7.22} Sangha has appointed four monks from the east and four from \textsanskrit{Pāvā} to resolve this legal issue by means of a committee. The Sangha approves and is therefore silent. I’ll remember it thus.” 

At\marginnote{2.7.24} that time there was a monk called Ajita who had ten years of seniority and was the Sangha’s reciter of the Monastic Code. The Sangha appointed him to assign seats to the senior monks. 

The\marginnote{2.7.27} senior monks said, “Where should we resolve this legal issue?” It occurred to them, “There’s the \textsanskrit{Vālika} Monastery, which is delightful, quiet, and free from chatter. Let’s resolve this legal issue there.” And so they went to the \textsanskrit{Vālika} Monastery. 

\section*{The committee decides on the ten practices}

Revata\marginnote{2.8.1} then informed the Sangha: 

“Please,\marginnote{2.8.2} venerables, I ask the Sangha to listen. If the Sangha is ready, I will question Venerable \textsanskrit{Sabbakāmī} on the Monastic Law.” 

And\marginnote{2.8.4} \textsanskrit{Sabbakāmī} informed the Sangha: 

“Please,\marginnote{2.8.5} venerables, I ask the Sangha to listen. If the Sangha is ready, I will reply when asked by Revata about the Monastic Law.” 

Revata\marginnote{2.8.7} said to \textsanskrit{Sabbakāmī}, “Sir, is the salt-in-horn practice allowable?” 

“What’s\marginnote{2.8.9} the salt-in-horn practice?” 

“Is\marginnote{2.8.10} it allowable to carry salt in a horn and then eat it whenever the food is unsalted?” 

“No,\marginnote{2.8.12} it’s not allowable.” 

“Where\marginnote{2.8.13} was it prohibited?” 

“At\marginnote{2.8.14} \textsanskrit{Sāvatthī}, in the analysis of the Monastic Code.” 

“What\marginnote{2.8.15} was committed?” 

“An\marginnote{2.8.16} offense entailing confession for eating what has been stored.” 

“Please,\marginnote{2.8.17} venerables, I ask the Sangha to listen. The Sangha has decided on the first practice. This practice is contrary to the Teaching, contrary to the Monastic Law, and a departure from the Teacher’s instruction. I make a note of this first decision.”\footnote{Literally, “This is the first voting ticket that I place.” } 

“Is\marginnote{2.8.21} the two-fingerbreadths practice allowable?”—“What’s the two-fingerbreadths practice?”—“Is it allowable to eat at the wrong time, so long as the shadow of the sundial is within two fingerbreadths of midday?”—“No.”—“Where was it prohibited?”—“At \textsanskrit{Rājagaha}, in the analysis of the Monastic Code.”—“What was committed?”—“An offense entailing confession for eating at the wrong time.”\footnote{“Is it allowable to eat at the wrong time, so long as the shadow of the sundial is within two fingerbreadths of midday?” renders \textit{kappati, bhante, \textsanskrit{dvaṅgulāya} \textsanskrit{chāyāya} \textsanskrit{vītivattāya} \textsanskrit{vikāle} \textsanskrit{bhojanaṁ} \textsanskrit{bhuñjituni}}. More literally this might be rendered as follows: “When two fingerbreadths of shade have passed, is it allowable to eat at the wrong time?” I interpret this to mean within two fingerbreadths of midday. } 

“Please,\marginnote{2.8.29} venerables, I ask the Sangha to listen. The Sangha has decided on the second practice. This practice is contrary to the Teaching, contrary to the Monastic Law, and a departure from the Teacher’s instruction. I make a note of this second decision.” 

“Is\marginnote{2.8.33} the next-village practice allowable?”—“What’s the next-village practice?”—“When you have finished your meal and refused an invitation to eat more, is it allowable to eat non-leftover food if you intend to go to the next village?”—“No.”—“Where was it prohibited?”—“At \textsanskrit{Sāvatthī}, in the analysis of the Monastic Code.”—“What was committed?”—“An offense entailing confession for eating what isn’t left over.” 

“Please,\marginnote{2.8.41} venerables, I ask the Sangha to listen. The Sangha has decided on the third practice. This practice is contrary to the Teaching, contrary to the Monastic Law, and a departure from the Teacher’s instruction. I make a note of this third decision.” 

“Is\marginnote{2.8.45} the many-monasteries practice allowable?”—“What’s the many-monasteries practice?”—“When there are a number of monasteries within the same monastery zone, is it allowable to do the observance-day ceremony separately?”—“No.”—“Where was it prohibited?”—“At \textsanskrit{Rājagaha}, in what’s connected to the observance-day ceremony.”—“What was committed?”—“An act of wrong conduct for going beyond the Monastic Law.”\footnote{“What’s connected to the observance-day ceremony” renders \textit{\textsanskrit{uposathasaṁyutta}}, which is identified at Sp-\textsanskrit{ṭ} 4.457 as the Uposathakkhandhaka, The Chapter on the Observance Day. But perhaps it is better regarded as a \textit{forerunner} to the Uposathakkhandhaka. } 

“Please,\marginnote{2.8.53} venerables, I ask the Sangha to listen. The Sangha has decided on the fourth practice. This practice is contrary to the Teaching, contrary to the Monastic Law, and a departure from the Teacher’s instruction. I make a note of this fourth decision.” 

“Is\marginnote{2.8.57} the consent practice allowable?”—“What’s the consent practice?”—“Is it allowable to do a legal procedure with an incomplete Sangha, with the intention of getting consent from the absent monks afterwards?”—“No.”—“Where was it prohibited?”—“In the section on Those from \textsanskrit{Campā}, in the Monastic Law.”—“What was committed?”—“An act of wrong conduct for going beyond the Monastic Law.” 

“Please,\marginnote{2.8.66} venerables, I ask the Sangha to listen. The Sangha has decided on the fifth practice. This practice is contrary to the Teaching, contrary to the Monastic Law, and a departure from the Teacher’s instruction. I make a note of this fifth decision.” 

“Are\marginnote{2.8.70} customary practices allowable?”—“What are customary practices?”—“Is it allowable to follow the practices of one’s preceptors or teachers?”—“Sometimes it is, sometimes it isn’t.” 

“Please,\marginnote{2.8.76} venerables, I ask the Sangha to listen. The Sangha has decided on the sixth practice. This practice is contrary to the Teaching, contrary to the Monastic Law, and a departure from the Teacher’s instruction. I make a note of this sixth decision.” 

“Is\marginnote{2.8.80} the unchurned practice allowable?”—“What’s the unchurned practice?”—“When you have finished your meal and refused an invitation to eat more, is it allowable to drink that which is halfway between milk and curd, if it isn’t leftover?”—“No.”—“Where was it prohibited?”—“At \textsanskrit{Sāvatthī}, in the analysis of the Monastic Code.”—“What was committed?”—“An offense entailing confession for eating what isn’t left over.” 

“Please,\marginnote{2.8.88} venerables, I ask the Sangha to listen. The Sangha has decided on the seventh practice. This practice is contrary to the Teaching, contrary to the Monastic Law, and a departure from the Teacher’s instruction. I make a note of this seventh decision.” 

“Is\marginnote{2.8.92} palm-juice drinking allowable?”—“What’s palm juice?”—“Is it allowable to drink that which has started to ferment, but which hasn’t yet become a proper alcoholic drink?”—“No.”—“Where was it prohibited?”—“At \textsanskrit{Kosambī}, in the analysis of the Monastic Code.”—“What was committed?”—“An offense entailing confession for drinking alcohol.” 

“Please,\marginnote{2.8.100} venerables, I ask the Sangha to listen. The Sangha has decided on the eighth practice. This practice is contrary to the Teaching, contrary to the Monastic Law, and a departure from the Teacher’s instruction. I make a note of this eighth decision.” 

“Are\marginnote{2.8.104} sitting mats without borders allowable?”—“No.”—“Where was it prohibited?”—“At \textsanskrit{Sāvatthī}, in the analysis of the Monastic Code.”—“What was committed?”—“An offense entailing confession in relation to the rule concerning cutting.”\footnote{“The rule concerning cutting” renders \textit{chedanake \textsanskrit{pācittiyaṁ}}. Sp 4.457: \textit{\textsanskrit{Chedanakasikkhāpade} \textsanskrit{vuttapācittiyaṁ} \textsanskrit{āpajjatīti} attho}, “The meaning is he commits the said offense entailing confession in the training rule on cutting.” } 

“Please,\marginnote{2.8.110} venerables, I ask the Sangha to listen. The Sangha has decided on the ninth practice. This practice is contrary to the Teaching, contrary to the Monastic Law, and a departure from the Teacher’s instruction. I make a note of this ninth decision.” 

“Is\marginnote{2.8.114} gold, silver, or money allowable?”—“No.”—“Where was it prohibited?”—“At \textsanskrit{Rājagaha}, in the analysis of the Monastic Code.”—“What was committed?”—“An offense entailing confession for receiving gold, silver, or money.” 

“Please,\marginnote{2.8.120} venerables, I ask the Sangha to listen. The Sangha has decided on the tenth practice. This practice is contrary to the Teaching, contrary to the Monastic Law, and a departure from the Teacher’s instruction. I make a note of this tenth decision. 

Please,\marginnote{2.8.124} venerables, I ask the Sangha to listen. The Sangha has decided on the ten practices. These ten practices are contrary to the Teaching, contrary to the Monastic Law, and a departure from the Teacher’s instruction.” 

“The\marginnote{2.8.127} legal issue has been resolved and properly disposed of. Nevertheless, for the purpose of convincing the other monks, you should ask me about these ten practices also in the midst of the Sangha.” 

Revata\marginnote{2.8.129} then asked \textsanskrit{Sabbakāmī} about the ten practices in the midst of the Sangha. And \textsanskrit{Sabbakāmī} was able to reply to each and every question. 

At\marginnote{2.9.1} this communal recitation of the Monastic Law there were seven hundred monks, neither more nor less. This is why this communal recitation is called “The group of seven hundred”. 

\scendsutta{The twelfth chapter on the group of seven hundred is finished. In this chapter there are twenty-five topics. }

\scuddanaintro{This is the summary: }

\begin{scuddana}%
“The\marginnote{2.9.4} ten practices, having filled, \\
Legal procedure, he entered with a messenger; \\
Four, and again gold, \\
And \textsanskrit{Kosambī}, those from \textsanskrit{Pāvā}. 

The\marginnote{2.9.8} way to Soreyya, \textsanskrit{Saṅkassa}, \\
\textsanskrit{Kaṇṇakujja}, Udumbara; \\
And \textsanskrit{Sahajāti}, he asked,\footnote{Reading the \textit{m} as junction consonant. } \\
Heard, how can we. 

A\marginnote{2.9.12} bowl, went upstream with a boat, \\
In private, bringing; \\
Respect, Sangha, \textsanskrit{Vesālī}, \\
Good will, Sangha, committee.” 

%
\end{scuddana}

\scendsutta{The chapter on the group of seven hundred is finished. }

\scendbook{The Small Division is finished. }

\scendbook{The canonical text of the Small Division is finished. }

%
\backmatter%
%
\chapter*{Appendices}
\addcontentsline{toc}{chapter}{Appendices}
\markboth{Appendices}{Appendices}

\emph{Appendices for all volumes may be found at the end of the first volume, The Great Analysis, part I.}

%
\chapter*{Colophon}
\addcontentsline{toc}{chapter}{Colophon}
\markboth{Colophon}{Colophon}

\section*{The Translator}

Bhikkhu Brahmali was born Norway in 1964. He first became interested in Buddhism and meditation in his early 20s after a visit to Japan. Having completed degrees in engineering and finance, he began his monastic training as an anagarika (keeping the eight precepts) in England at Amaravati and Chithurst Buddhist Monastery.

After hearing teachings from Ajahn Brahm he decided to travel to Australia to train at Bodhinyana Monastery. Bhikkhu Brahmali has lived at Bodhinyana Monastery since 1994, and was ordained as a Bhikkhu, with Ajahn Brahm as his preceptor, in 1996. In 2015 he entered his 20th Rains Retreat as a fully ordained monastic and received the title Maha Thera (Great Elder).

Bhikkhu Brahmali’s knowledge of the Pali language and of the Suttas is excellent. Bhikkhu Bodhi, who translated most of the Pali Canon into English for Wisdom Publications, called him one of his major helpers for the 2012 translation of \emph{The Numerical Discourses of the Buddha}. He has also published two essays on Dependent Origination and a book called \emph{The Authenticity of the Early Buddhist Texts} with the Buddhist Publication Society in collaboration with Bhante Sujato.

The monastics of the Buddhist Society of WA (BSWA) often turn to him to clarify Vinaya (monastic discipline) or Sutta questions. They also greatly appreciate his Sutta and Pali classes. Furthermore he has been instrumental in most of the building and maintenance projects at Bodhinyana Monastery and at the emerging Hermit Hill property in Serpentine.

\section*{Creation Process}

Translated from the Pali. The primary source was the \textsanskrit{Mahāsaṅgīti} edition, with occasional reference to other Pali editions, especially the \textsanskrit{Chaṭṭha} \textsanskrit{Saṅgāyana} edition and the Pali Text Society edition. I cross-checked with I.B. Horner’s English translation, “The Book of the Discipline”, as well as Bhikkhu \textsanskrit{Ñāṇatusita}’s “A Translation and Analysis of the \textsanskrit{Pātimokkha}” and Ajahn \textsanskrit{Ṭhānissaro}’s “Buddhist Monastic Code”.

\section*{The Translation}

This is the first complete translation of the Vinaya \textsanskrit{Piṭaka} in English. The aim has been to produce a translation that is easy to read, clear, and accurate, and also modern in vocabulary and style.

\section*{About SuttaCentral}

SuttaCentral publishes early Buddhist texts. Since 2005 we have provided root texts in Pali, Chinese, Sanskrit, Tibetan, and other languages, parallels between these texts, and translations in many modern languages. Building on the work of generations of scholars, we offer our contribution freely.

SuttaCentral is driven by volunteer contributions, and in addition we employ professional developers. We offer a sponsorship program for high quality translations from the original languages. Financial support for SuttaCentral is handled by the SuttaCentral Development Trust, a charitable trust registered in Australia.

\section*{About Bilara}

“Bilara” means “cat” in Pali, and it is the name of our Computer Assisted Translation (CAT) software. Bilara is a web app that enables translators to translate early Buddhist texts into their own language. These translations are published on SuttaCentral with the root text and translation side by side.

\section*{About SuttaCentral Editions}

The SuttaCentral Editions project makes high quality books from selected Bilara translations. These are published in formats including HTML, EPUB, PDF, and print.

You are welcome to print any of our Editions.

%
\end{document}